\documentclass[10pt, spanish]{article}

\usepackage[none]{hyphenat}
\usepackage[utf8]{inputenc}
\usepackage[T1]{fontenc}
\usepackage[spanish,es-nodecimaldot]{babel}
\usepackage{csquotes}
\usepackage{multicol}
\usepackage{pgf}

\usepackage{geometry}
\def\margin{18mm}
\geometry{
    a4paper,
    left=\margin,
    right=\margin,
    top=\margin,
    bottom=\margin
}

% \usepackage[sorting=none]{biblatex}
\usepackage{hyperref}

\usepackage{lmodern}
\usepackage{enumitem}
%\setenumerate{label=(\alph*),leftmargin=0.6cm}
% \setitemize{label=---,leftmargin=0.6cm}
\usepackage{amssymb}
\usepackage{mathtools}

\renewcommand*{\thefootnote}{\fnsymbol{footnote}}

\usepackage{titlesec}

\titlespacing*{\section}{0pt}{.7em}{.5em}

% For diagrams
\usepackage{tikz}
\usetikzlibrary{arrows}

% Theorems
\usepackage{amsthm}
\usepackage{thmtools}
\newtheorem*{lema}{Lema}
\newtheorem*{obs}{Observación}
\newtheorem*{nota}{Nota}

\addto\captionsspanish{\renewcommand\proofname{Solución}}

\theoremstyle{definition}
\newtheorem*{defin}{Definición}
\newtheorem*{prop}{Proposición}
\newtheoremstyle{break}{}{}{}{}{\bfseries}{.}{.5em}{Ejercicio #2}
\theoremstyle{break}
\newtheorem{ej}{Ejercicio}

% No indentation
\setlength\parindent{0pt}
\setlength{\parskip}{0.5em}
\let\emptyset\varnothing

% Custom math commands
\newcommand{\N}{\mathbb{N}}
\newcommand{\Z}{\mathbb{Z}}
\newcommand{\Q}{\mathbb{Q}}
\newcommand{\R}{\mathbb{R}}

\renewcommand{\geq}{\geqslant}
\renewcommand{\leq}{\leqslant}

\DeclareMathSymbol{*}{\mathbin}{symbols}{"01}

\DeclareMathOperator{\im}{Im}
%\DeclareMathOperator{\ker}{Ker}
\DeclareMathOperator{\mcm}{mcm}
\DeclareMathOperator{\mcd}{mcd}

\DeclareMathOperator{\bijmap}{\ \rlap{\ensuremath{\rightarrowtail}}%
    {\ensuremath{\mkern2mu\twoheadrightarrow}}}

\newcommand{\Zp}{\mathbb{Z}_{(p)}}

\usepackage{graphicx}
\graphicspath{ {./plots/} }


\usepackage{color}
\definecolor{gray75}{gray}{0.75}
\definecolor{pycommentcol}{rgb}{0.3,0.3,0.3}     % gray
\definecolor{pystatecol}{rgb}{0,0,0.7}           % blue
\definecolor{pystringcol}{rgb}{0,0.6,0}          % green
\definecolor{pyinbuiltscol}{rgb}{0.55,0.15,0.55} % plum
\definecolor{pyspecialcol}{rgb}{0.8,0.45,0.12}   % orange
\definecolor{mygray}{gray}{0.3}
\newcommand*{\pyfontfamily}{\fontfamily{DejaVuSansMono-TLF}\selectfont}

\usepackage{listings}
\lstset{inputpath=./}
\usepackage{xcolor}


\usepackage{textgreek}
\newcommand\pythonstyle{\lstset{
        language=Python,
        literate=%esto es para que acepte acentos
        {á}{{\'a}}1
        {í}{{\'i}}1
        {é}{{\'e}}1
        {ý}{{\'y}}1
        {ú}{{\'u}}1
        {ó}{{\'o}}1
        {ě}{{\v{e}}}1
        {š}{{\v{s}}}1
        {č}{{\v{c}}}1
        {ř}{{\v{r}}}1
        {ž}{{\v{z}}}1
        {ď}{{\v{d}}}1
        {ť}{{\v{t}}}1
        {ñ}{{\~n}}1
        {ň}{{\v{n}}}1                
        {ů}{{\r{u}}}1
        {Á}{{\'A}}1
        {Í}{{\'I}}1
        {É}{{\'E}}1
        {Ý}{{\'Y}}1
        {Ú}{{\'U}}1
        {Ó}{{\'O}}1
        {Ě}{{\v{E}}}1
        {Š}{{\v{S}}}1
        {Č}{{\v{C}}}1
        {Ř}{{\v{R}}}1
        {Ž}{{\v{Z}}}1
        {Ď}{{\v{D}}}1
        {Ť}{{\v{T}}}1
        {Ň}{{\v{N}}}1                
        {ε}{{\textepsilon}}1                
        {±}{{$\pm$}}1                
        {Ů}{{\r{U}}}1,
        basicstyle=\pyfontfamily\scriptsize,
        commentstyle=\color{pycommentcol}\itshape,
        emph={self,cls,@classmethod,@property}, % Custom highlighting
        emphstyle=\color{pyspecialcol}\itshape, % Custom highlighting style
        morestring=[b]{"""},
        stringstyle=\color{pystringcol},
        keywordstyle=\color{pystatecol},        % statements
        % remove any inbuilt functions from keywords
        deletekeywords={print},
        % Switch to predefined class that contain many, but not all,
        % inbuilt functions and classes
        classoffset=1,
        % add any inbuilts, not statements
        morekeywords={print,None,TypeError},
        keywordstyle=\color{pyinbuiltscol},
        frame=leftline,
        numberstyle=\sffamily\tiny\color{mygray},
        stepnumber=1,
        numbers=left,
        numbersep=10pt,                      
        showstringspaces=false            
}}

\usepackage[labelformat=empty, labelfont={bf,it}, textfont=bf]{caption}%ponga solo el nombre en los codigos

\pythonstyle
\begin{document}
\selectfont{\Large\textbf{Geometría computacional: Práctica 3}\hfill Adrián Lattes  Grassi} \noindent\rule{17cm}{1pt}

\section{Introducción}

En esta práctica he utilizado los algoritmos de clasificación \textit{KMeans} y
\textit{DBSCAN} sobre un sistema $X$ de 1000 elementos con dos estados cada uno.
Para ello he utilizado las implementaciones de estos algoritmos de la librería
\texttt{scikit-learn}.

\section{Método}

El programa está dividido en funciones que aislan las distintas funcionalidades
del mismo y permiten la reutilización y variación de parámetros del código:
\begin{itemize}
\setlength\itemsep{0em}
    \item \texttt{kmeans\_silhouettes} y \texttt{dbscan\_silhouettes}: Dado un
        sistema y un conjunto de parámetros (\textit{n} para \textit{KMeans} y
        \textit{ε} para \textit{DBSCAN}) calcula las vecindades
        correspondientes, con sus valores medios de los coeficientes de
        \textit{Silhouette}.
    \item \texttt{kmeans\_elegir\_n\_clusters} y \texttt{dbscan\_elegir\_ε}:
        Devuelve el índice del valor máximo de los coeficientes de \textit{Silhouette}
        calculados anteriormente. Además también dibuja la gráfica de los
        distintos valores de los coeficientes de \textit{Silhouette} al variar el
        parámetro del algoritmo, destacando el rojo el valor máximo.
    \item \texttt{dbscan\_cluster\_centroids}:
        Calcula los centroides de las vecindades calculadas con \textit{DBSCAN},
        útiles para pintar las etiquetas en las gráficas.
    \item \texttt{plot\_clusters} y \texttt{plot\_voronoi}:
        Gráfica de las vecindades y diagrama de \textit{Voronoi}.
    \item \texttt{apartado1} y \texttt{apartado2}:
        Gestión de los plots y llamadas a las funciones anteriores.
\end{itemize}

\section{Resultados}
\subsection{Clasificación con algoritmo KMeans y predicción de nuevos estados}

El número de vecindades con mayor coeficiente de \textit{Silhouette} para el algoritmo
\textit{KMeans} de entre los valores $[2,15]$ es el 3, como se puede ver en la
figura de la izquierda. Además es interesante observar que no hay ningún otro
valor que proporcione un coeficiente de \textit{Silhouette} cercano al correspondiente a
3 vecindades.  Por otro lado este es precisamente el valor esperado, ya que los
estados del sistema han sido generados entorno a tres centros.

En la gráfica de la derecha se pueden observar las vecindades y diagrama
correspondientes a este número óptimo de vecindades. Aquí también he dibujado en
rojo los elementos $a=(0,0)$ y $b=(0,-1)$ que queremos clasificar. Con esto he
podido comparar el resultado esperado visualmente con el obtenido con el método
\texttt{kmeans.predict}.
\begin{itemize}
\setlength\itemsep{0em}
    \item El punto $a$ está visiblemente en la vecindad 2, lo que coincide con
        el resultado de  \texttt{kmeans.predict} 
    \item El punto $b$ está visiblemente en la frontera entre las regiones de
voronoi correspondientes a las vecindades 0 y 1. En este caso el método
\texttt{kmeans.predict} nos dice que el elemento pertenece a la vecindad 0, lo
que se corresponde con lo esperado, pero quizás es una información un tanto
imprecisa, ya que solo con este dato no sabemos que el punto está muy cerca de
la vecindad 1.  \end{itemize}

\vspace{-1em}

\begin{center}
    \makebox[\textwidth][c]{\scalebox{0.65}{%% Creator: Matplotlib, PGF backend
%%
%% To include the figure in your LaTeX document, write
%%   \input{<filename>.pgf}
%%
%% Make sure the required packages are loaded in your preamble
%%   \usepackage{pgf}
%%
%% Figures using additional raster images can only be included by \input if
%% they are in the same directory as the main LaTeX file. For loading figures
%% from other directories you can use the `import` package
%%   \usepackage{import}
%%
%% and then include the figures with
%%   \import{<path to file>}{<filename>.pgf}
%%
%% Matplotlib used the following preamble
%%   \usepackage{fontspec}
%%   \setmainfont{DejaVuSerif.ttf}[Path=\detokenize{/nix/store/zl80nl46sadml2lln6v1xgbhqks16lz2-python3.8-matplotlib-3.4.3/lib/python3.8/site-packages/matplotlib/mpl-data/fonts/ttf/}]
%%   \setsansfont{DejaVuSans.ttf}[Path=\detokenize{/nix/store/zl80nl46sadml2lln6v1xgbhqks16lz2-python3.8-matplotlib-3.4.3/lib/python3.8/site-packages/matplotlib/mpl-data/fonts/ttf/}]
%%   \setmonofont{DejaVuSansMono.ttf}[Path=\detokenize{/nix/store/zl80nl46sadml2lln6v1xgbhqks16lz2-python3.8-matplotlib-3.4.3/lib/python3.8/site-packages/matplotlib/mpl-data/fonts/ttf/}]
%%
\begingroup%
\makeatletter%
\begin{pgfpicture}%
\pgfpathrectangle{\pgfpointorigin}{\pgfqpoint{12.000000in}{4.310000in}}%
\pgfusepath{use as bounding box, clip}%
\begin{pgfscope}%
\pgfsetbuttcap%
\pgfsetmiterjoin%
\definecolor{currentfill}{rgb}{1.000000,1.000000,1.000000}%
\pgfsetfillcolor{currentfill}%
\pgfsetlinewidth{0.000000pt}%
\definecolor{currentstroke}{rgb}{1.000000,1.000000,1.000000}%
\pgfsetstrokecolor{currentstroke}%
\pgfsetdash{}{0pt}%
\pgfpathmoveto{\pgfqpoint{0.000000in}{0.000000in}}%
\pgfpathlineto{\pgfqpoint{12.000000in}{0.000000in}}%
\pgfpathlineto{\pgfqpoint{12.000000in}{4.310000in}}%
\pgfpathlineto{\pgfqpoint{0.000000in}{4.310000in}}%
\pgfpathclose%
\pgfusepath{fill}%
\end{pgfscope}%
\begin{pgfscope}%
\pgfsetbuttcap%
\pgfsetmiterjoin%
\definecolor{currentfill}{rgb}{1.000000,1.000000,1.000000}%
\pgfsetfillcolor{currentfill}%
\pgfsetlinewidth{0.000000pt}%
\definecolor{currentstroke}{rgb}{0.000000,0.000000,0.000000}%
\pgfsetstrokecolor{currentstroke}%
\pgfsetstrokeopacity{0.000000}%
\pgfsetdash{}{0pt}%
\pgfpathmoveto{\pgfqpoint{1.500000in}{0.474100in}}%
\pgfpathlineto{\pgfqpoint{5.727273in}{0.474100in}}%
\pgfpathlineto{\pgfqpoint{5.727273in}{3.792800in}}%
\pgfpathlineto{\pgfqpoint{1.500000in}{3.792800in}}%
\pgfpathclose%
\pgfusepath{fill}%
\end{pgfscope}%
\begin{pgfscope}%
\pgfpathrectangle{\pgfqpoint{1.500000in}{0.474100in}}{\pgfqpoint{4.227273in}{3.318700in}}%
\pgfusepath{clip}%
\pgfsetbuttcap%
\pgfsetmiterjoin%
\definecolor{currentfill}{rgb}{0.121569,0.466667,0.705882}%
\pgfsetfillcolor{currentfill}%
\pgfsetlinewidth{0.000000pt}%
\definecolor{currentstroke}{rgb}{0.000000,0.000000,0.000000}%
\pgfsetstrokecolor{currentstroke}%
\pgfsetstrokeopacity{0.000000}%
\pgfsetdash{}{0pt}%
\pgfpathmoveto{\pgfqpoint{1.692149in}{0.474100in}}%
\pgfpathlineto{\pgfqpoint{1.940974in}{0.474100in}}%
\pgfpathlineto{\pgfqpoint{1.940974in}{3.196856in}}%
\pgfpathlineto{\pgfqpoint{1.692149in}{3.196856in}}%
\pgfpathclose%
\pgfusepath{fill}%
\end{pgfscope}%
\begin{pgfscope}%
\pgfpathrectangle{\pgfqpoint{1.500000in}{0.474100in}}{\pgfqpoint{4.227273in}{3.318700in}}%
\pgfusepath{clip}%
\pgfsetbuttcap%
\pgfsetmiterjoin%
\definecolor{currentfill}{rgb}{1.000000,0.000000,0.000000}%
\pgfsetfillcolor{currentfill}%
\pgfsetlinewidth{1.003750pt}%
\definecolor{currentstroke}{rgb}{1.000000,0.000000,0.000000}%
\pgfsetstrokecolor{currentstroke}%
\pgfsetdash{}{0pt}%
\pgfpathmoveto{\pgfqpoint{1.968622in}{0.474100in}}%
\pgfpathlineto{\pgfqpoint{2.217448in}{0.474100in}}%
\pgfpathlineto{\pgfqpoint{2.217448in}{3.634767in}}%
\pgfpathlineto{\pgfqpoint{1.968622in}{3.634767in}}%
\pgfpathclose%
\pgfusepath{stroke,fill}%
\end{pgfscope}%
\begin{pgfscope}%
\pgfpathrectangle{\pgfqpoint{1.500000in}{0.474100in}}{\pgfqpoint{4.227273in}{3.318700in}}%
\pgfusepath{clip}%
\pgfsetbuttcap%
\pgfsetmiterjoin%
\definecolor{currentfill}{rgb}{0.121569,0.466667,0.705882}%
\pgfsetfillcolor{currentfill}%
\pgfsetlinewidth{0.000000pt}%
\definecolor{currentstroke}{rgb}{0.000000,0.000000,0.000000}%
\pgfsetstrokecolor{currentstroke}%
\pgfsetstrokeopacity{0.000000}%
\pgfsetdash{}{0pt}%
\pgfpathmoveto{\pgfqpoint{2.245095in}{0.474100in}}%
\pgfpathlineto{\pgfqpoint{2.493921in}{0.474100in}}%
\pgfpathlineto{\pgfqpoint{2.493921in}{3.047023in}}%
\pgfpathlineto{\pgfqpoint{2.245095in}{3.047023in}}%
\pgfpathclose%
\pgfusepath{fill}%
\end{pgfscope}%
\begin{pgfscope}%
\pgfpathrectangle{\pgfqpoint{1.500000in}{0.474100in}}{\pgfqpoint{4.227273in}{3.318700in}}%
\pgfusepath{clip}%
\pgfsetbuttcap%
\pgfsetmiterjoin%
\definecolor{currentfill}{rgb}{0.121569,0.466667,0.705882}%
\pgfsetfillcolor{currentfill}%
\pgfsetlinewidth{0.000000pt}%
\definecolor{currentstroke}{rgb}{0.000000,0.000000,0.000000}%
\pgfsetstrokecolor{currentstroke}%
\pgfsetstrokeopacity{0.000000}%
\pgfsetdash{}{0pt}%
\pgfpathmoveto{\pgfqpoint{2.521568in}{0.474100in}}%
\pgfpathlineto{\pgfqpoint{2.770394in}{0.474100in}}%
\pgfpathlineto{\pgfqpoint{2.770394in}{2.439913in}}%
\pgfpathlineto{\pgfqpoint{2.521568in}{2.439913in}}%
\pgfpathclose%
\pgfusepath{fill}%
\end{pgfscope}%
\begin{pgfscope}%
\pgfpathrectangle{\pgfqpoint{1.500000in}{0.474100in}}{\pgfqpoint{4.227273in}{3.318700in}}%
\pgfusepath{clip}%
\pgfsetbuttcap%
\pgfsetmiterjoin%
\definecolor{currentfill}{rgb}{0.121569,0.466667,0.705882}%
\pgfsetfillcolor{currentfill}%
\pgfsetlinewidth{0.000000pt}%
\definecolor{currentstroke}{rgb}{0.000000,0.000000,0.000000}%
\pgfsetstrokecolor{currentstroke}%
\pgfsetstrokeopacity{0.000000}%
\pgfsetdash{}{0pt}%
\pgfpathmoveto{\pgfqpoint{2.798041in}{0.474100in}}%
\pgfpathlineto{\pgfqpoint{3.046867in}{0.474100in}}%
\pgfpathlineto{\pgfqpoint{3.046867in}{2.249967in}}%
\pgfpathlineto{\pgfqpoint{2.798041in}{2.249967in}}%
\pgfpathclose%
\pgfusepath{fill}%
\end{pgfscope}%
\begin{pgfscope}%
\pgfpathrectangle{\pgfqpoint{1.500000in}{0.474100in}}{\pgfqpoint{4.227273in}{3.318700in}}%
\pgfusepath{clip}%
\pgfsetbuttcap%
\pgfsetmiterjoin%
\definecolor{currentfill}{rgb}{0.121569,0.466667,0.705882}%
\pgfsetfillcolor{currentfill}%
\pgfsetlinewidth{0.000000pt}%
\definecolor{currentstroke}{rgb}{0.000000,0.000000,0.000000}%
\pgfsetstrokecolor{currentstroke}%
\pgfsetstrokeopacity{0.000000}%
\pgfsetdash{}{0pt}%
\pgfpathmoveto{\pgfqpoint{3.074514in}{0.474100in}}%
\pgfpathlineto{\pgfqpoint{3.323340in}{0.474100in}}%
\pgfpathlineto{\pgfqpoint{3.323340in}{2.284600in}}%
\pgfpathlineto{\pgfqpoint{3.074514in}{2.284600in}}%
\pgfpathclose%
\pgfusepath{fill}%
\end{pgfscope}%
\begin{pgfscope}%
\pgfpathrectangle{\pgfqpoint{1.500000in}{0.474100in}}{\pgfqpoint{4.227273in}{3.318700in}}%
\pgfusepath{clip}%
\pgfsetbuttcap%
\pgfsetmiterjoin%
\definecolor{currentfill}{rgb}{0.121569,0.466667,0.705882}%
\pgfsetfillcolor{currentfill}%
\pgfsetlinewidth{0.000000pt}%
\definecolor{currentstroke}{rgb}{0.000000,0.000000,0.000000}%
\pgfsetstrokecolor{currentstroke}%
\pgfsetstrokeopacity{0.000000}%
\pgfsetdash{}{0pt}%
\pgfpathmoveto{\pgfqpoint{3.350987in}{0.474100in}}%
\pgfpathlineto{\pgfqpoint{3.599813in}{0.474100in}}%
\pgfpathlineto{\pgfqpoint{3.599813in}{2.302423in}}%
\pgfpathlineto{\pgfqpoint{3.350987in}{2.302423in}}%
\pgfpathclose%
\pgfusepath{fill}%
\end{pgfscope}%
\begin{pgfscope}%
\pgfpathrectangle{\pgfqpoint{1.500000in}{0.474100in}}{\pgfqpoint{4.227273in}{3.318700in}}%
\pgfusepath{clip}%
\pgfsetbuttcap%
\pgfsetmiterjoin%
\definecolor{currentfill}{rgb}{0.121569,0.466667,0.705882}%
\pgfsetfillcolor{currentfill}%
\pgfsetlinewidth{0.000000pt}%
\definecolor{currentstroke}{rgb}{0.000000,0.000000,0.000000}%
\pgfsetstrokecolor{currentstroke}%
\pgfsetstrokeopacity{0.000000}%
\pgfsetdash{}{0pt}%
\pgfpathmoveto{\pgfqpoint{3.627460in}{0.474100in}}%
\pgfpathlineto{\pgfqpoint{3.876286in}{0.474100in}}%
\pgfpathlineto{\pgfqpoint{3.876286in}{2.323086in}}%
\pgfpathlineto{\pgfqpoint{3.627460in}{2.323086in}}%
\pgfpathclose%
\pgfusepath{fill}%
\end{pgfscope}%
\begin{pgfscope}%
\pgfpathrectangle{\pgfqpoint{1.500000in}{0.474100in}}{\pgfqpoint{4.227273in}{3.318700in}}%
\pgfusepath{clip}%
\pgfsetbuttcap%
\pgfsetmiterjoin%
\definecolor{currentfill}{rgb}{0.121569,0.466667,0.705882}%
\pgfsetfillcolor{currentfill}%
\pgfsetlinewidth{0.000000pt}%
\definecolor{currentstroke}{rgb}{0.000000,0.000000,0.000000}%
\pgfsetstrokecolor{currentstroke}%
\pgfsetstrokeopacity{0.000000}%
\pgfsetdash{}{0pt}%
\pgfpathmoveto{\pgfqpoint{3.903933in}{0.474100in}}%
\pgfpathlineto{\pgfqpoint{4.152759in}{0.474100in}}%
\pgfpathlineto{\pgfqpoint{4.152759in}{2.338386in}}%
\pgfpathlineto{\pgfqpoint{3.903933in}{2.338386in}}%
\pgfpathclose%
\pgfusepath{fill}%
\end{pgfscope}%
\begin{pgfscope}%
\pgfpathrectangle{\pgfqpoint{1.500000in}{0.474100in}}{\pgfqpoint{4.227273in}{3.318700in}}%
\pgfusepath{clip}%
\pgfsetbuttcap%
\pgfsetmiterjoin%
\definecolor{currentfill}{rgb}{0.121569,0.466667,0.705882}%
\pgfsetfillcolor{currentfill}%
\pgfsetlinewidth{0.000000pt}%
\definecolor{currentstroke}{rgb}{0.000000,0.000000,0.000000}%
\pgfsetstrokecolor{currentstroke}%
\pgfsetstrokeopacity{0.000000}%
\pgfsetdash{}{0pt}%
\pgfpathmoveto{\pgfqpoint{4.180406in}{0.474100in}}%
\pgfpathlineto{\pgfqpoint{4.429232in}{0.474100in}}%
\pgfpathlineto{\pgfqpoint{4.429232in}{2.337728in}}%
\pgfpathlineto{\pgfqpoint{4.180406in}{2.337728in}}%
\pgfpathclose%
\pgfusepath{fill}%
\end{pgfscope}%
\begin{pgfscope}%
\pgfpathrectangle{\pgfqpoint{1.500000in}{0.474100in}}{\pgfqpoint{4.227273in}{3.318700in}}%
\pgfusepath{clip}%
\pgfsetbuttcap%
\pgfsetmiterjoin%
\definecolor{currentfill}{rgb}{0.121569,0.466667,0.705882}%
\pgfsetfillcolor{currentfill}%
\pgfsetlinewidth{0.000000pt}%
\definecolor{currentstroke}{rgb}{0.000000,0.000000,0.000000}%
\pgfsetstrokecolor{currentstroke}%
\pgfsetstrokeopacity{0.000000}%
\pgfsetdash{}{0pt}%
\pgfpathmoveto{\pgfqpoint{4.456879in}{0.474100in}}%
\pgfpathlineto{\pgfqpoint{4.705705in}{0.474100in}}%
\pgfpathlineto{\pgfqpoint{4.705705in}{2.236895in}}%
\pgfpathlineto{\pgfqpoint{4.456879in}{2.236895in}}%
\pgfpathclose%
\pgfusepath{fill}%
\end{pgfscope}%
\begin{pgfscope}%
\pgfpathrectangle{\pgfqpoint{1.500000in}{0.474100in}}{\pgfqpoint{4.227273in}{3.318700in}}%
\pgfusepath{clip}%
\pgfsetbuttcap%
\pgfsetmiterjoin%
\definecolor{currentfill}{rgb}{0.121569,0.466667,0.705882}%
\pgfsetfillcolor{currentfill}%
\pgfsetlinewidth{0.000000pt}%
\definecolor{currentstroke}{rgb}{0.000000,0.000000,0.000000}%
\pgfsetstrokecolor{currentstroke}%
\pgfsetstrokeopacity{0.000000}%
\pgfsetdash{}{0pt}%
\pgfpathmoveto{\pgfqpoint{4.733352in}{0.474100in}}%
\pgfpathlineto{\pgfqpoint{4.982178in}{0.474100in}}%
\pgfpathlineto{\pgfqpoint{4.982178in}{2.225507in}}%
\pgfpathlineto{\pgfqpoint{4.733352in}{2.225507in}}%
\pgfpathclose%
\pgfusepath{fill}%
\end{pgfscope}%
\begin{pgfscope}%
\pgfpathrectangle{\pgfqpoint{1.500000in}{0.474100in}}{\pgfqpoint{4.227273in}{3.318700in}}%
\pgfusepath{clip}%
\pgfsetbuttcap%
\pgfsetmiterjoin%
\definecolor{currentfill}{rgb}{0.121569,0.466667,0.705882}%
\pgfsetfillcolor{currentfill}%
\pgfsetlinewidth{0.000000pt}%
\definecolor{currentstroke}{rgb}{0.000000,0.000000,0.000000}%
\pgfsetstrokecolor{currentstroke}%
\pgfsetstrokeopacity{0.000000}%
\pgfsetdash{}{0pt}%
\pgfpathmoveto{\pgfqpoint{5.009825in}{0.474100in}}%
\pgfpathlineto{\pgfqpoint{5.258651in}{0.474100in}}%
\pgfpathlineto{\pgfqpoint{5.258651in}{2.269112in}}%
\pgfpathlineto{\pgfqpoint{5.009825in}{2.269112in}}%
\pgfpathclose%
\pgfusepath{fill}%
\end{pgfscope}%
\begin{pgfscope}%
\pgfpathrectangle{\pgfqpoint{1.500000in}{0.474100in}}{\pgfqpoint{4.227273in}{3.318700in}}%
\pgfusepath{clip}%
\pgfsetbuttcap%
\pgfsetmiterjoin%
\definecolor{currentfill}{rgb}{0.121569,0.466667,0.705882}%
\pgfsetfillcolor{currentfill}%
\pgfsetlinewidth{0.000000pt}%
\definecolor{currentstroke}{rgb}{0.000000,0.000000,0.000000}%
\pgfsetstrokecolor{currentstroke}%
\pgfsetstrokeopacity{0.000000}%
\pgfsetdash{}{0pt}%
\pgfpathmoveto{\pgfqpoint{5.286298in}{0.474100in}}%
\pgfpathlineto{\pgfqpoint{5.535124in}{0.474100in}}%
\pgfpathlineto{\pgfqpoint{5.535124in}{2.253400in}}%
\pgfpathlineto{\pgfqpoint{5.286298in}{2.253400in}}%
\pgfpathclose%
\pgfusepath{fill}%
\end{pgfscope}%
\begin{pgfscope}%
\pgfsetbuttcap%
\pgfsetroundjoin%
\definecolor{currentfill}{rgb}{0.000000,0.000000,0.000000}%
\pgfsetfillcolor{currentfill}%
\pgfsetlinewidth{0.803000pt}%
\definecolor{currentstroke}{rgb}{0.000000,0.000000,0.000000}%
\pgfsetstrokecolor{currentstroke}%
\pgfsetdash{}{0pt}%
\pgfsys@defobject{currentmarker}{\pgfqpoint{0.000000in}{-0.048611in}}{\pgfqpoint{0.000000in}{0.000000in}}{%
\pgfpathmoveto{\pgfqpoint{0.000000in}{0.000000in}}%
\pgfpathlineto{\pgfqpoint{0.000000in}{-0.048611in}}%
\pgfusepath{stroke,fill}%
}%
\begin{pgfscope}%
\pgfsys@transformshift{1.816562in}{0.474100in}%
\pgfsys@useobject{currentmarker}{}%
\end{pgfscope}%
\end{pgfscope}%
\begin{pgfscope}%
\definecolor{textcolor}{rgb}{0.000000,0.000000,0.000000}%
\pgfsetstrokecolor{textcolor}%
\pgfsetfillcolor{textcolor}%
\pgftext[x=1.816562in,y=0.376878in,,top]{\color{textcolor}\sffamily\fontsize{10.000000}{12.000000}\selectfont 2}%
\end{pgfscope}%
\begin{pgfscope}%
\pgfsetbuttcap%
\pgfsetroundjoin%
\definecolor{currentfill}{rgb}{0.000000,0.000000,0.000000}%
\pgfsetfillcolor{currentfill}%
\pgfsetlinewidth{0.803000pt}%
\definecolor{currentstroke}{rgb}{0.000000,0.000000,0.000000}%
\pgfsetstrokecolor{currentstroke}%
\pgfsetdash{}{0pt}%
\pgfsys@defobject{currentmarker}{\pgfqpoint{0.000000in}{-0.048611in}}{\pgfqpoint{0.000000in}{0.000000in}}{%
\pgfpathmoveto{\pgfqpoint{0.000000in}{0.000000in}}%
\pgfpathlineto{\pgfqpoint{0.000000in}{-0.048611in}}%
\pgfusepath{stroke,fill}%
}%
\begin{pgfscope}%
\pgfsys@transformshift{2.369508in}{0.474100in}%
\pgfsys@useobject{currentmarker}{}%
\end{pgfscope}%
\end{pgfscope}%
\begin{pgfscope}%
\definecolor{textcolor}{rgb}{0.000000,0.000000,0.000000}%
\pgfsetstrokecolor{textcolor}%
\pgfsetfillcolor{textcolor}%
\pgftext[x=2.369508in,y=0.376878in,,top]{\color{textcolor}\sffamily\fontsize{10.000000}{12.000000}\selectfont 4}%
\end{pgfscope}%
\begin{pgfscope}%
\pgfsetbuttcap%
\pgfsetroundjoin%
\definecolor{currentfill}{rgb}{0.000000,0.000000,0.000000}%
\pgfsetfillcolor{currentfill}%
\pgfsetlinewidth{0.803000pt}%
\definecolor{currentstroke}{rgb}{0.000000,0.000000,0.000000}%
\pgfsetstrokecolor{currentstroke}%
\pgfsetdash{}{0pt}%
\pgfsys@defobject{currentmarker}{\pgfqpoint{0.000000in}{-0.048611in}}{\pgfqpoint{0.000000in}{0.000000in}}{%
\pgfpathmoveto{\pgfqpoint{0.000000in}{0.000000in}}%
\pgfpathlineto{\pgfqpoint{0.000000in}{-0.048611in}}%
\pgfusepath{stroke,fill}%
}%
\begin{pgfscope}%
\pgfsys@transformshift{2.922454in}{0.474100in}%
\pgfsys@useobject{currentmarker}{}%
\end{pgfscope}%
\end{pgfscope}%
\begin{pgfscope}%
\definecolor{textcolor}{rgb}{0.000000,0.000000,0.000000}%
\pgfsetstrokecolor{textcolor}%
\pgfsetfillcolor{textcolor}%
\pgftext[x=2.922454in,y=0.376878in,,top]{\color{textcolor}\sffamily\fontsize{10.000000}{12.000000}\selectfont 6}%
\end{pgfscope}%
\begin{pgfscope}%
\pgfsetbuttcap%
\pgfsetroundjoin%
\definecolor{currentfill}{rgb}{0.000000,0.000000,0.000000}%
\pgfsetfillcolor{currentfill}%
\pgfsetlinewidth{0.803000pt}%
\definecolor{currentstroke}{rgb}{0.000000,0.000000,0.000000}%
\pgfsetstrokecolor{currentstroke}%
\pgfsetdash{}{0pt}%
\pgfsys@defobject{currentmarker}{\pgfqpoint{0.000000in}{-0.048611in}}{\pgfqpoint{0.000000in}{0.000000in}}{%
\pgfpathmoveto{\pgfqpoint{0.000000in}{0.000000in}}%
\pgfpathlineto{\pgfqpoint{0.000000in}{-0.048611in}}%
\pgfusepath{stroke,fill}%
}%
\begin{pgfscope}%
\pgfsys@transformshift{3.475400in}{0.474100in}%
\pgfsys@useobject{currentmarker}{}%
\end{pgfscope}%
\end{pgfscope}%
\begin{pgfscope}%
\definecolor{textcolor}{rgb}{0.000000,0.000000,0.000000}%
\pgfsetstrokecolor{textcolor}%
\pgfsetfillcolor{textcolor}%
\pgftext[x=3.475400in,y=0.376878in,,top]{\color{textcolor}\sffamily\fontsize{10.000000}{12.000000}\selectfont 8}%
\end{pgfscope}%
\begin{pgfscope}%
\pgfsetbuttcap%
\pgfsetroundjoin%
\definecolor{currentfill}{rgb}{0.000000,0.000000,0.000000}%
\pgfsetfillcolor{currentfill}%
\pgfsetlinewidth{0.803000pt}%
\definecolor{currentstroke}{rgb}{0.000000,0.000000,0.000000}%
\pgfsetstrokecolor{currentstroke}%
\pgfsetdash{}{0pt}%
\pgfsys@defobject{currentmarker}{\pgfqpoint{0.000000in}{-0.048611in}}{\pgfqpoint{0.000000in}{0.000000in}}{%
\pgfpathmoveto{\pgfqpoint{0.000000in}{0.000000in}}%
\pgfpathlineto{\pgfqpoint{0.000000in}{-0.048611in}}%
\pgfusepath{stroke,fill}%
}%
\begin{pgfscope}%
\pgfsys@transformshift{4.028346in}{0.474100in}%
\pgfsys@useobject{currentmarker}{}%
\end{pgfscope}%
\end{pgfscope}%
\begin{pgfscope}%
\definecolor{textcolor}{rgb}{0.000000,0.000000,0.000000}%
\pgfsetstrokecolor{textcolor}%
\pgfsetfillcolor{textcolor}%
\pgftext[x=4.028346in,y=0.376878in,,top]{\color{textcolor}\sffamily\fontsize{10.000000}{12.000000}\selectfont 10}%
\end{pgfscope}%
\begin{pgfscope}%
\pgfsetbuttcap%
\pgfsetroundjoin%
\definecolor{currentfill}{rgb}{0.000000,0.000000,0.000000}%
\pgfsetfillcolor{currentfill}%
\pgfsetlinewidth{0.803000pt}%
\definecolor{currentstroke}{rgb}{0.000000,0.000000,0.000000}%
\pgfsetstrokecolor{currentstroke}%
\pgfsetdash{}{0pt}%
\pgfsys@defobject{currentmarker}{\pgfqpoint{0.000000in}{-0.048611in}}{\pgfqpoint{0.000000in}{0.000000in}}{%
\pgfpathmoveto{\pgfqpoint{0.000000in}{0.000000in}}%
\pgfpathlineto{\pgfqpoint{0.000000in}{-0.048611in}}%
\pgfusepath{stroke,fill}%
}%
\begin{pgfscope}%
\pgfsys@transformshift{4.581292in}{0.474100in}%
\pgfsys@useobject{currentmarker}{}%
\end{pgfscope}%
\end{pgfscope}%
\begin{pgfscope}%
\definecolor{textcolor}{rgb}{0.000000,0.000000,0.000000}%
\pgfsetstrokecolor{textcolor}%
\pgfsetfillcolor{textcolor}%
\pgftext[x=4.581292in,y=0.376878in,,top]{\color{textcolor}\sffamily\fontsize{10.000000}{12.000000}\selectfont 12}%
\end{pgfscope}%
\begin{pgfscope}%
\pgfsetbuttcap%
\pgfsetroundjoin%
\definecolor{currentfill}{rgb}{0.000000,0.000000,0.000000}%
\pgfsetfillcolor{currentfill}%
\pgfsetlinewidth{0.803000pt}%
\definecolor{currentstroke}{rgb}{0.000000,0.000000,0.000000}%
\pgfsetstrokecolor{currentstroke}%
\pgfsetdash{}{0pt}%
\pgfsys@defobject{currentmarker}{\pgfqpoint{0.000000in}{-0.048611in}}{\pgfqpoint{0.000000in}{0.000000in}}{%
\pgfpathmoveto{\pgfqpoint{0.000000in}{0.000000in}}%
\pgfpathlineto{\pgfqpoint{0.000000in}{-0.048611in}}%
\pgfusepath{stroke,fill}%
}%
\begin{pgfscope}%
\pgfsys@transformshift{5.134238in}{0.474100in}%
\pgfsys@useobject{currentmarker}{}%
\end{pgfscope}%
\end{pgfscope}%
\begin{pgfscope}%
\definecolor{textcolor}{rgb}{0.000000,0.000000,0.000000}%
\pgfsetstrokecolor{textcolor}%
\pgfsetfillcolor{textcolor}%
\pgftext[x=5.134238in,y=0.376878in,,top]{\color{textcolor}\sffamily\fontsize{10.000000}{12.000000}\selectfont 14}%
\end{pgfscope}%
\begin{pgfscope}%
\pgfsetbuttcap%
\pgfsetroundjoin%
\definecolor{currentfill}{rgb}{0.000000,0.000000,0.000000}%
\pgfsetfillcolor{currentfill}%
\pgfsetlinewidth{0.803000pt}%
\definecolor{currentstroke}{rgb}{0.000000,0.000000,0.000000}%
\pgfsetstrokecolor{currentstroke}%
\pgfsetdash{}{0pt}%
\pgfsys@defobject{currentmarker}{\pgfqpoint{0.000000in}{-0.048611in}}{\pgfqpoint{0.000000in}{0.000000in}}{%
\pgfpathmoveto{\pgfqpoint{0.000000in}{0.000000in}}%
\pgfpathlineto{\pgfqpoint{0.000000in}{-0.048611in}}%
\pgfusepath{stroke,fill}%
}%
\begin{pgfscope}%
\pgfsys@transformshift{5.687184in}{0.474100in}%
\pgfsys@useobject{currentmarker}{}%
\end{pgfscope}%
\end{pgfscope}%
\begin{pgfscope}%
\definecolor{textcolor}{rgb}{0.000000,0.000000,0.000000}%
\pgfsetstrokecolor{textcolor}%
\pgfsetfillcolor{textcolor}%
\pgftext[x=5.687184in,y=0.376878in,,top]{\color{textcolor}\sffamily\fontsize{10.000000}{12.000000}\selectfont 16}%
\end{pgfscope}%
\begin{pgfscope}%
\definecolor{textcolor}{rgb}{0.000000,0.000000,0.000000}%
\pgfsetstrokecolor{textcolor}%
\pgfsetfillcolor{textcolor}%
\pgftext[x=3.613636in,y=0.186909in,,top]{\color{textcolor}\sffamily\fontsize{10.000000}{12.000000}\selectfont Número de vecindades}%
\end{pgfscope}%
\begin{pgfscope}%
\pgfsetbuttcap%
\pgfsetroundjoin%
\definecolor{currentfill}{rgb}{0.000000,0.000000,0.000000}%
\pgfsetfillcolor{currentfill}%
\pgfsetlinewidth{0.803000pt}%
\definecolor{currentstroke}{rgb}{0.000000,0.000000,0.000000}%
\pgfsetstrokecolor{currentstroke}%
\pgfsetdash{}{0pt}%
\pgfsys@defobject{currentmarker}{\pgfqpoint{-0.048611in}{0.000000in}}{\pgfqpoint{-0.000000in}{0.000000in}}{%
\pgfpathmoveto{\pgfqpoint{-0.000000in}{0.000000in}}%
\pgfpathlineto{\pgfqpoint{-0.048611in}{0.000000in}}%
\pgfusepath{stroke,fill}%
}%
\begin{pgfscope}%
\pgfsys@transformshift{1.500000in}{0.474100in}%
\pgfsys@useobject{currentmarker}{}%
\end{pgfscope}%
\end{pgfscope}%
\begin{pgfscope}%
\definecolor{textcolor}{rgb}{0.000000,0.000000,0.000000}%
\pgfsetstrokecolor{textcolor}%
\pgfsetfillcolor{textcolor}%
\pgftext[x=1.181898in, y=0.421338in, left, base]{\color{textcolor}\sffamily\fontsize{10.000000}{12.000000}\selectfont 0.0}%
\end{pgfscope}%
\begin{pgfscope}%
\pgfsetbuttcap%
\pgfsetroundjoin%
\definecolor{currentfill}{rgb}{0.000000,0.000000,0.000000}%
\pgfsetfillcolor{currentfill}%
\pgfsetlinewidth{0.803000pt}%
\definecolor{currentstroke}{rgb}{0.000000,0.000000,0.000000}%
\pgfsetstrokecolor{currentstroke}%
\pgfsetdash{}{0pt}%
\pgfsys@defobject{currentmarker}{\pgfqpoint{-0.048611in}{0.000000in}}{\pgfqpoint{-0.000000in}{0.000000in}}{%
\pgfpathmoveto{\pgfqpoint{-0.000000in}{0.000000in}}%
\pgfpathlineto{\pgfqpoint{-0.048611in}{0.000000in}}%
\pgfusepath{stroke,fill}%
}%
\begin{pgfscope}%
\pgfsys@transformshift{1.500000in}{1.011669in}%
\pgfsys@useobject{currentmarker}{}%
\end{pgfscope}%
\end{pgfscope}%
\begin{pgfscope}%
\definecolor{textcolor}{rgb}{0.000000,0.000000,0.000000}%
\pgfsetstrokecolor{textcolor}%
\pgfsetfillcolor{textcolor}%
\pgftext[x=1.181898in, y=0.958908in, left, base]{\color{textcolor}\sffamily\fontsize{10.000000}{12.000000}\selectfont 0.1}%
\end{pgfscope}%
\begin{pgfscope}%
\pgfsetbuttcap%
\pgfsetroundjoin%
\definecolor{currentfill}{rgb}{0.000000,0.000000,0.000000}%
\pgfsetfillcolor{currentfill}%
\pgfsetlinewidth{0.803000pt}%
\definecolor{currentstroke}{rgb}{0.000000,0.000000,0.000000}%
\pgfsetstrokecolor{currentstroke}%
\pgfsetdash{}{0pt}%
\pgfsys@defobject{currentmarker}{\pgfqpoint{-0.048611in}{0.000000in}}{\pgfqpoint{-0.000000in}{0.000000in}}{%
\pgfpathmoveto{\pgfqpoint{-0.000000in}{0.000000in}}%
\pgfpathlineto{\pgfqpoint{-0.048611in}{0.000000in}}%
\pgfusepath{stroke,fill}%
}%
\begin{pgfscope}%
\pgfsys@transformshift{1.500000in}{1.549238in}%
\pgfsys@useobject{currentmarker}{}%
\end{pgfscope}%
\end{pgfscope}%
\begin{pgfscope}%
\definecolor{textcolor}{rgb}{0.000000,0.000000,0.000000}%
\pgfsetstrokecolor{textcolor}%
\pgfsetfillcolor{textcolor}%
\pgftext[x=1.181898in, y=1.496477in, left, base]{\color{textcolor}\sffamily\fontsize{10.000000}{12.000000}\selectfont 0.2}%
\end{pgfscope}%
\begin{pgfscope}%
\pgfsetbuttcap%
\pgfsetroundjoin%
\definecolor{currentfill}{rgb}{0.000000,0.000000,0.000000}%
\pgfsetfillcolor{currentfill}%
\pgfsetlinewidth{0.803000pt}%
\definecolor{currentstroke}{rgb}{0.000000,0.000000,0.000000}%
\pgfsetstrokecolor{currentstroke}%
\pgfsetdash{}{0pt}%
\pgfsys@defobject{currentmarker}{\pgfqpoint{-0.048611in}{0.000000in}}{\pgfqpoint{-0.000000in}{0.000000in}}{%
\pgfpathmoveto{\pgfqpoint{-0.000000in}{0.000000in}}%
\pgfpathlineto{\pgfqpoint{-0.048611in}{0.000000in}}%
\pgfusepath{stroke,fill}%
}%
\begin{pgfscope}%
\pgfsys@transformshift{1.500000in}{2.086808in}%
\pgfsys@useobject{currentmarker}{}%
\end{pgfscope}%
\end{pgfscope}%
\begin{pgfscope}%
\definecolor{textcolor}{rgb}{0.000000,0.000000,0.000000}%
\pgfsetstrokecolor{textcolor}%
\pgfsetfillcolor{textcolor}%
\pgftext[x=1.181898in, y=2.034046in, left, base]{\color{textcolor}\sffamily\fontsize{10.000000}{12.000000}\selectfont 0.3}%
\end{pgfscope}%
\begin{pgfscope}%
\pgfsetbuttcap%
\pgfsetroundjoin%
\definecolor{currentfill}{rgb}{0.000000,0.000000,0.000000}%
\pgfsetfillcolor{currentfill}%
\pgfsetlinewidth{0.803000pt}%
\definecolor{currentstroke}{rgb}{0.000000,0.000000,0.000000}%
\pgfsetstrokecolor{currentstroke}%
\pgfsetdash{}{0pt}%
\pgfsys@defobject{currentmarker}{\pgfqpoint{-0.048611in}{0.000000in}}{\pgfqpoint{-0.000000in}{0.000000in}}{%
\pgfpathmoveto{\pgfqpoint{-0.000000in}{0.000000in}}%
\pgfpathlineto{\pgfqpoint{-0.048611in}{0.000000in}}%
\pgfusepath{stroke,fill}%
}%
\begin{pgfscope}%
\pgfsys@transformshift{1.500000in}{2.624377in}%
\pgfsys@useobject{currentmarker}{}%
\end{pgfscope}%
\end{pgfscope}%
\begin{pgfscope}%
\definecolor{textcolor}{rgb}{0.000000,0.000000,0.000000}%
\pgfsetstrokecolor{textcolor}%
\pgfsetfillcolor{textcolor}%
\pgftext[x=1.181898in, y=2.571615in, left, base]{\color{textcolor}\sffamily\fontsize{10.000000}{12.000000}\selectfont 0.4}%
\end{pgfscope}%
\begin{pgfscope}%
\pgfsetbuttcap%
\pgfsetroundjoin%
\definecolor{currentfill}{rgb}{0.000000,0.000000,0.000000}%
\pgfsetfillcolor{currentfill}%
\pgfsetlinewidth{0.803000pt}%
\definecolor{currentstroke}{rgb}{0.000000,0.000000,0.000000}%
\pgfsetstrokecolor{currentstroke}%
\pgfsetdash{}{0pt}%
\pgfsys@defobject{currentmarker}{\pgfqpoint{-0.048611in}{0.000000in}}{\pgfqpoint{-0.000000in}{0.000000in}}{%
\pgfpathmoveto{\pgfqpoint{-0.000000in}{0.000000in}}%
\pgfpathlineto{\pgfqpoint{-0.048611in}{0.000000in}}%
\pgfusepath{stroke,fill}%
}%
\begin{pgfscope}%
\pgfsys@transformshift{1.500000in}{3.161946in}%
\pgfsys@useobject{currentmarker}{}%
\end{pgfscope}%
\end{pgfscope}%
\begin{pgfscope}%
\definecolor{textcolor}{rgb}{0.000000,0.000000,0.000000}%
\pgfsetstrokecolor{textcolor}%
\pgfsetfillcolor{textcolor}%
\pgftext[x=1.181898in, y=3.109184in, left, base]{\color{textcolor}\sffamily\fontsize{10.000000}{12.000000}\selectfont 0.5}%
\end{pgfscope}%
\begin{pgfscope}%
\pgfsetbuttcap%
\pgfsetroundjoin%
\definecolor{currentfill}{rgb}{0.000000,0.000000,0.000000}%
\pgfsetfillcolor{currentfill}%
\pgfsetlinewidth{0.803000pt}%
\definecolor{currentstroke}{rgb}{0.000000,0.000000,0.000000}%
\pgfsetstrokecolor{currentstroke}%
\pgfsetdash{}{0pt}%
\pgfsys@defobject{currentmarker}{\pgfqpoint{-0.048611in}{0.000000in}}{\pgfqpoint{-0.000000in}{0.000000in}}{%
\pgfpathmoveto{\pgfqpoint{-0.000000in}{0.000000in}}%
\pgfpathlineto{\pgfqpoint{-0.048611in}{0.000000in}}%
\pgfusepath{stroke,fill}%
}%
\begin{pgfscope}%
\pgfsys@transformshift{1.500000in}{3.699515in}%
\pgfsys@useobject{currentmarker}{}%
\end{pgfscope}%
\end{pgfscope}%
\begin{pgfscope}%
\definecolor{textcolor}{rgb}{0.000000,0.000000,0.000000}%
\pgfsetstrokecolor{textcolor}%
\pgfsetfillcolor{textcolor}%
\pgftext[x=1.181898in, y=3.646754in, left, base]{\color{textcolor}\sffamily\fontsize{10.000000}{12.000000}\selectfont 0.6}%
\end{pgfscope}%
\begin{pgfscope}%
\definecolor{textcolor}{rgb}{0.000000,0.000000,0.000000}%
\pgfsetstrokecolor{textcolor}%
\pgfsetfillcolor{textcolor}%
\pgftext[x=1.126343in,y=2.133450in,,bottom,rotate=90.000000]{\color{textcolor}\sffamily\fontsize{10.000000}{12.000000}\selectfont Valor medio del coeficiente de Silhouette}%
\end{pgfscope}%
\begin{pgfscope}%
\pgfsetrectcap%
\pgfsetmiterjoin%
\pgfsetlinewidth{0.803000pt}%
\definecolor{currentstroke}{rgb}{0.000000,0.000000,0.000000}%
\pgfsetstrokecolor{currentstroke}%
\pgfsetdash{}{0pt}%
\pgfpathmoveto{\pgfqpoint{1.500000in}{0.474100in}}%
\pgfpathlineto{\pgfqpoint{1.500000in}{3.792800in}}%
\pgfusepath{stroke}%
\end{pgfscope}%
\begin{pgfscope}%
\pgfsetrectcap%
\pgfsetmiterjoin%
\pgfsetlinewidth{0.803000pt}%
\definecolor{currentstroke}{rgb}{0.000000,0.000000,0.000000}%
\pgfsetstrokecolor{currentstroke}%
\pgfsetdash{}{0pt}%
\pgfpathmoveto{\pgfqpoint{5.727273in}{0.474100in}}%
\pgfpathlineto{\pgfqpoint{5.727273in}{3.792800in}}%
\pgfusepath{stroke}%
\end{pgfscope}%
\begin{pgfscope}%
\pgfsetrectcap%
\pgfsetmiterjoin%
\pgfsetlinewidth{0.803000pt}%
\definecolor{currentstroke}{rgb}{0.000000,0.000000,0.000000}%
\pgfsetstrokecolor{currentstroke}%
\pgfsetdash{}{0pt}%
\pgfpathmoveto{\pgfqpoint{1.500000in}{0.474100in}}%
\pgfpathlineto{\pgfqpoint{5.727273in}{0.474100in}}%
\pgfusepath{stroke}%
\end{pgfscope}%
\begin{pgfscope}%
\pgfsetrectcap%
\pgfsetmiterjoin%
\pgfsetlinewidth{0.803000pt}%
\definecolor{currentstroke}{rgb}{0.000000,0.000000,0.000000}%
\pgfsetstrokecolor{currentstroke}%
\pgfsetdash{}{0pt}%
\pgfpathmoveto{\pgfqpoint{1.500000in}{3.792800in}}%
\pgfpathlineto{\pgfqpoint{5.727273in}{3.792800in}}%
\pgfusepath{stroke}%
\end{pgfscope}%
\begin{pgfscope}%
\pgfsetbuttcap%
\pgfsetmiterjoin%
\definecolor{currentfill}{rgb}{1.000000,1.000000,1.000000}%
\pgfsetfillcolor{currentfill}%
\pgfsetlinewidth{0.000000pt}%
\definecolor{currentstroke}{rgb}{0.000000,0.000000,0.000000}%
\pgfsetstrokecolor{currentstroke}%
\pgfsetstrokeopacity{0.000000}%
\pgfsetdash{}{0pt}%
\pgfpathmoveto{\pgfqpoint{6.572727in}{0.474100in}}%
\pgfpathlineto{\pgfqpoint{10.800000in}{0.474100in}}%
\pgfpathlineto{\pgfqpoint{10.800000in}{3.792800in}}%
\pgfpathlineto{\pgfqpoint{6.572727in}{3.792800in}}%
\pgfpathclose%
\pgfusepath{fill}%
\end{pgfscope}%
\begin{pgfscope}%
\pgfpathrectangle{\pgfqpoint{6.572727in}{0.474100in}}{\pgfqpoint{4.227273in}{3.318700in}}%
\pgfusepath{clip}%
\pgfsetbuttcap%
\pgfsetroundjoin%
\definecolor{currentfill}{rgb}{0.993248,0.906157,0.143936}%
\pgfsetfillcolor{currentfill}%
\pgfsetfillopacity{0.700000}%
\pgfsetlinewidth{0.000000pt}%
\definecolor{currentstroke}{rgb}{0.000000,0.000000,0.000000}%
\pgfsetstrokecolor{currentstroke}%
\pgfsetstrokeopacity{0.700000}%
\pgfsetdash{}{0pt}%
\pgfpathmoveto{\pgfqpoint{7.859498in}{2.337775in}}%
\pgfpathcurveto{\pgfqpoint{7.864542in}{2.337775in}}{\pgfqpoint{7.869379in}{2.339779in}}{\pgfqpoint{7.872946in}{2.343345in}}%
\pgfpathcurveto{\pgfqpoint{7.876512in}{2.346912in}}{\pgfqpoint{7.878516in}{2.351749in}}{\pgfqpoint{7.878516in}{2.356793in}}%
\pgfpathcurveto{\pgfqpoint{7.878516in}{2.361837in}}{\pgfqpoint{7.876512in}{2.366675in}}{\pgfqpoint{7.872946in}{2.370241in}}%
\pgfpathcurveto{\pgfqpoint{7.869379in}{2.373807in}}{\pgfqpoint{7.864542in}{2.375811in}}{\pgfqpoint{7.859498in}{2.375811in}}%
\pgfpathcurveto{\pgfqpoint{7.854454in}{2.375811in}}{\pgfqpoint{7.849616in}{2.373807in}}{\pgfqpoint{7.846050in}{2.370241in}}%
\pgfpathcurveto{\pgfqpoint{7.842484in}{2.366675in}}{\pgfqpoint{7.840480in}{2.361837in}}{\pgfqpoint{7.840480in}{2.356793in}}%
\pgfpathcurveto{\pgfqpoint{7.840480in}{2.351749in}}{\pgfqpoint{7.842484in}{2.346912in}}{\pgfqpoint{7.846050in}{2.343345in}}%
\pgfpathcurveto{\pgfqpoint{7.849616in}{2.339779in}}{\pgfqpoint{7.854454in}{2.337775in}}{\pgfqpoint{7.859498in}{2.337775in}}%
\pgfpathclose%
\pgfusepath{fill}%
\end{pgfscope}%
\begin{pgfscope}%
\pgfpathrectangle{\pgfqpoint{6.572727in}{0.474100in}}{\pgfqpoint{4.227273in}{3.318700in}}%
\pgfusepath{clip}%
\pgfsetbuttcap%
\pgfsetroundjoin%
\definecolor{currentfill}{rgb}{0.993248,0.906157,0.143936}%
\pgfsetfillcolor{currentfill}%
\pgfsetfillopacity{0.700000}%
\pgfsetlinewidth{0.000000pt}%
\definecolor{currentstroke}{rgb}{0.000000,0.000000,0.000000}%
\pgfsetstrokecolor{currentstroke}%
\pgfsetstrokeopacity{0.700000}%
\pgfsetdash{}{0pt}%
\pgfpathmoveto{\pgfqpoint{8.893046in}{2.948141in}}%
\pgfpathcurveto{\pgfqpoint{8.898090in}{2.948141in}}{\pgfqpoint{8.902927in}{2.950145in}}{\pgfqpoint{8.906494in}{2.953711in}}%
\pgfpathcurveto{\pgfqpoint{8.910060in}{2.957278in}}{\pgfqpoint{8.912064in}{2.962116in}}{\pgfqpoint{8.912064in}{2.967159in}}%
\pgfpathcurveto{\pgfqpoint{8.912064in}{2.972203in}}{\pgfqpoint{8.910060in}{2.977041in}}{\pgfqpoint{8.906494in}{2.980607in}}%
\pgfpathcurveto{\pgfqpoint{8.902927in}{2.984173in}}{\pgfqpoint{8.898090in}{2.986177in}}{\pgfqpoint{8.893046in}{2.986177in}}%
\pgfpathcurveto{\pgfqpoint{8.888002in}{2.986177in}}{\pgfqpoint{8.883164in}{2.984173in}}{\pgfqpoint{8.879598in}{2.980607in}}%
\pgfpathcurveto{\pgfqpoint{8.876032in}{2.977041in}}{\pgfqpoint{8.874028in}{2.972203in}}{\pgfqpoint{8.874028in}{2.967159in}}%
\pgfpathcurveto{\pgfqpoint{8.874028in}{2.962116in}}{\pgfqpoint{8.876032in}{2.957278in}}{\pgfqpoint{8.879598in}{2.953711in}}%
\pgfpathcurveto{\pgfqpoint{8.883164in}{2.950145in}}{\pgfqpoint{8.888002in}{2.948141in}}{\pgfqpoint{8.893046in}{2.948141in}}%
\pgfpathclose%
\pgfusepath{fill}%
\end{pgfscope}%
\begin{pgfscope}%
\pgfpathrectangle{\pgfqpoint{6.572727in}{0.474100in}}{\pgfqpoint{4.227273in}{3.318700in}}%
\pgfusepath{clip}%
\pgfsetbuttcap%
\pgfsetroundjoin%
\definecolor{currentfill}{rgb}{0.127568,0.566949,0.550556}%
\pgfsetfillcolor{currentfill}%
\pgfsetfillopacity{0.700000}%
\pgfsetlinewidth{0.000000pt}%
\definecolor{currentstroke}{rgb}{0.000000,0.000000,0.000000}%
\pgfsetstrokecolor{currentstroke}%
\pgfsetstrokeopacity{0.700000}%
\pgfsetdash{}{0pt}%
\pgfpathmoveto{\pgfqpoint{9.440632in}{1.964413in}}%
\pgfpathcurveto{\pgfqpoint{9.445676in}{1.964413in}}{\pgfqpoint{9.450513in}{1.966417in}}{\pgfqpoint{9.454080in}{1.969983in}}%
\pgfpathcurveto{\pgfqpoint{9.457646in}{1.973550in}}{\pgfqpoint{9.459650in}{1.978387in}}{\pgfqpoint{9.459650in}{1.983431in}}%
\pgfpathcurveto{\pgfqpoint{9.459650in}{1.988475in}}{\pgfqpoint{9.457646in}{1.993313in}}{\pgfqpoint{9.454080in}{1.996879in}}%
\pgfpathcurveto{\pgfqpoint{9.450513in}{2.000445in}}{\pgfqpoint{9.445676in}{2.002449in}}{\pgfqpoint{9.440632in}{2.002449in}}%
\pgfpathcurveto{\pgfqpoint{9.435588in}{2.002449in}}{\pgfqpoint{9.430751in}{2.000445in}}{\pgfqpoint{9.427184in}{1.996879in}}%
\pgfpathcurveto{\pgfqpoint{9.423618in}{1.993313in}}{\pgfqpoint{9.421614in}{1.988475in}}{\pgfqpoint{9.421614in}{1.983431in}}%
\pgfpathcurveto{\pgfqpoint{9.421614in}{1.978387in}}{\pgfqpoint{9.423618in}{1.973550in}}{\pgfqpoint{9.427184in}{1.969983in}}%
\pgfpathcurveto{\pgfqpoint{9.430751in}{1.966417in}}{\pgfqpoint{9.435588in}{1.964413in}}{\pgfqpoint{9.440632in}{1.964413in}}%
\pgfpathclose%
\pgfusepath{fill}%
\end{pgfscope}%
\begin{pgfscope}%
\pgfpathrectangle{\pgfqpoint{6.572727in}{0.474100in}}{\pgfqpoint{4.227273in}{3.318700in}}%
\pgfusepath{clip}%
\pgfsetbuttcap%
\pgfsetroundjoin%
\definecolor{currentfill}{rgb}{0.267004,0.004874,0.329415}%
\pgfsetfillcolor{currentfill}%
\pgfsetfillopacity{0.700000}%
\pgfsetlinewidth{0.000000pt}%
\definecolor{currentstroke}{rgb}{0.000000,0.000000,0.000000}%
\pgfsetstrokecolor{currentstroke}%
\pgfsetstrokeopacity{0.700000}%
\pgfsetdash{}{0pt}%
\pgfpathmoveto{\pgfqpoint{8.563329in}{1.016272in}}%
\pgfpathcurveto{\pgfqpoint{8.568373in}{1.016272in}}{\pgfqpoint{8.573211in}{1.018276in}}{\pgfqpoint{8.576777in}{1.021842in}}%
\pgfpathcurveto{\pgfqpoint{8.580344in}{1.025408in}}{\pgfqpoint{8.582347in}{1.030246in}}{\pgfqpoint{8.582347in}{1.035290in}}%
\pgfpathcurveto{\pgfqpoint{8.582347in}{1.040333in}}{\pgfqpoint{8.580344in}{1.045171in}}{\pgfqpoint{8.576777in}{1.048738in}}%
\pgfpathcurveto{\pgfqpoint{8.573211in}{1.052304in}}{\pgfqpoint{8.568373in}{1.054308in}}{\pgfqpoint{8.563329in}{1.054308in}}%
\pgfpathcurveto{\pgfqpoint{8.558286in}{1.054308in}}{\pgfqpoint{8.553448in}{1.052304in}}{\pgfqpoint{8.549881in}{1.048738in}}%
\pgfpathcurveto{\pgfqpoint{8.546315in}{1.045171in}}{\pgfqpoint{8.544311in}{1.040333in}}{\pgfqpoint{8.544311in}{1.035290in}}%
\pgfpathcurveto{\pgfqpoint{8.544311in}{1.030246in}}{\pgfqpoint{8.546315in}{1.025408in}}{\pgfqpoint{8.549881in}{1.021842in}}%
\pgfpathcurveto{\pgfqpoint{8.553448in}{1.018276in}}{\pgfqpoint{8.558286in}{1.016272in}}{\pgfqpoint{8.563329in}{1.016272in}}%
\pgfpathclose%
\pgfusepath{fill}%
\end{pgfscope}%
\begin{pgfscope}%
\pgfpathrectangle{\pgfqpoint{6.572727in}{0.474100in}}{\pgfqpoint{4.227273in}{3.318700in}}%
\pgfusepath{clip}%
\pgfsetbuttcap%
\pgfsetroundjoin%
\definecolor{currentfill}{rgb}{0.127568,0.566949,0.550556}%
\pgfsetfillcolor{currentfill}%
\pgfsetfillopacity{0.700000}%
\pgfsetlinewidth{0.000000pt}%
\definecolor{currentstroke}{rgb}{0.000000,0.000000,0.000000}%
\pgfsetstrokecolor{currentstroke}%
\pgfsetstrokeopacity{0.700000}%
\pgfsetdash{}{0pt}%
\pgfpathmoveto{\pgfqpoint{9.354248in}{1.005935in}}%
\pgfpathcurveto{\pgfqpoint{9.359292in}{1.005935in}}{\pgfqpoint{9.364129in}{1.007939in}}{\pgfqpoint{9.367696in}{1.011505in}}%
\pgfpathcurveto{\pgfqpoint{9.371262in}{1.015072in}}{\pgfqpoint{9.373266in}{1.019910in}}{\pgfqpoint{9.373266in}{1.024953in}}%
\pgfpathcurveto{\pgfqpoint{9.373266in}{1.029997in}}{\pgfqpoint{9.371262in}{1.034835in}}{\pgfqpoint{9.367696in}{1.038401in}}%
\pgfpathcurveto{\pgfqpoint{9.364129in}{1.041967in}}{\pgfqpoint{9.359292in}{1.043971in}}{\pgfqpoint{9.354248in}{1.043971in}}%
\pgfpathcurveto{\pgfqpoint{9.349204in}{1.043971in}}{\pgfqpoint{9.344366in}{1.041967in}}{\pgfqpoint{9.340800in}{1.038401in}}%
\pgfpathcurveto{\pgfqpoint{9.337234in}{1.034835in}}{\pgfqpoint{9.335230in}{1.029997in}}{\pgfqpoint{9.335230in}{1.024953in}}%
\pgfpathcurveto{\pgfqpoint{9.335230in}{1.019910in}}{\pgfqpoint{9.337234in}{1.015072in}}{\pgfqpoint{9.340800in}{1.011505in}}%
\pgfpathcurveto{\pgfqpoint{9.344366in}{1.007939in}}{\pgfqpoint{9.349204in}{1.005935in}}{\pgfqpoint{9.354248in}{1.005935in}}%
\pgfpathclose%
\pgfusepath{fill}%
\end{pgfscope}%
\begin{pgfscope}%
\pgfpathrectangle{\pgfqpoint{6.572727in}{0.474100in}}{\pgfqpoint{4.227273in}{3.318700in}}%
\pgfusepath{clip}%
\pgfsetbuttcap%
\pgfsetroundjoin%
\definecolor{currentfill}{rgb}{0.267004,0.004874,0.329415}%
\pgfsetfillcolor{currentfill}%
\pgfsetfillopacity{0.700000}%
\pgfsetlinewidth{0.000000pt}%
\definecolor{currentstroke}{rgb}{0.000000,0.000000,0.000000}%
\pgfsetstrokecolor{currentstroke}%
\pgfsetstrokeopacity{0.700000}%
\pgfsetdash{}{0pt}%
\pgfpathmoveto{\pgfqpoint{8.124062in}{1.710980in}}%
\pgfpathcurveto{\pgfqpoint{8.129106in}{1.710980in}}{\pgfqpoint{8.133943in}{1.712984in}}{\pgfqpoint{8.137510in}{1.716550in}}%
\pgfpathcurveto{\pgfqpoint{8.141076in}{1.720117in}}{\pgfqpoint{8.143080in}{1.724955in}}{\pgfqpoint{8.143080in}{1.729998in}}%
\pgfpathcurveto{\pgfqpoint{8.143080in}{1.735042in}}{\pgfqpoint{8.141076in}{1.739880in}}{\pgfqpoint{8.137510in}{1.743446in}}%
\pgfpathcurveto{\pgfqpoint{8.133943in}{1.747013in}}{\pgfqpoint{8.129106in}{1.749016in}}{\pgfqpoint{8.124062in}{1.749016in}}%
\pgfpathcurveto{\pgfqpoint{8.119018in}{1.749016in}}{\pgfqpoint{8.114181in}{1.747013in}}{\pgfqpoint{8.110614in}{1.743446in}}%
\pgfpathcurveto{\pgfqpoint{8.107048in}{1.739880in}}{\pgfqpoint{8.105044in}{1.735042in}}{\pgfqpoint{8.105044in}{1.729998in}}%
\pgfpathcurveto{\pgfqpoint{8.105044in}{1.724955in}}{\pgfqpoint{8.107048in}{1.720117in}}{\pgfqpoint{8.110614in}{1.716550in}}%
\pgfpathcurveto{\pgfqpoint{8.114181in}{1.712984in}}{\pgfqpoint{8.119018in}{1.710980in}}{\pgfqpoint{8.124062in}{1.710980in}}%
\pgfpathclose%
\pgfusepath{fill}%
\end{pgfscope}%
\begin{pgfscope}%
\pgfpathrectangle{\pgfqpoint{6.572727in}{0.474100in}}{\pgfqpoint{4.227273in}{3.318700in}}%
\pgfusepath{clip}%
\pgfsetbuttcap%
\pgfsetroundjoin%
\definecolor{currentfill}{rgb}{0.993248,0.906157,0.143936}%
\pgfsetfillcolor{currentfill}%
\pgfsetfillopacity{0.700000}%
\pgfsetlinewidth{0.000000pt}%
\definecolor{currentstroke}{rgb}{0.000000,0.000000,0.000000}%
\pgfsetstrokecolor{currentstroke}%
\pgfsetstrokeopacity{0.700000}%
\pgfsetdash{}{0pt}%
\pgfpathmoveto{\pgfqpoint{9.075038in}{2.799785in}}%
\pgfpathcurveto{\pgfqpoint{9.080081in}{2.799785in}}{\pgfqpoint{9.084919in}{2.801788in}}{\pgfqpoint{9.088486in}{2.805355in}}%
\pgfpathcurveto{\pgfqpoint{9.092052in}{2.808921in}}{\pgfqpoint{9.094056in}{2.813759in}}{\pgfqpoint{9.094056in}{2.818803in}}%
\pgfpathcurveto{\pgfqpoint{9.094056in}{2.823846in}}{\pgfqpoint{9.092052in}{2.828684in}}{\pgfqpoint{9.088486in}{2.832251in}}%
\pgfpathcurveto{\pgfqpoint{9.084919in}{2.835817in}}{\pgfqpoint{9.080081in}{2.837821in}}{\pgfqpoint{9.075038in}{2.837821in}}%
\pgfpathcurveto{\pgfqpoint{9.069994in}{2.837821in}}{\pgfqpoint{9.065156in}{2.835817in}}{\pgfqpoint{9.061590in}{2.832251in}}%
\pgfpathcurveto{\pgfqpoint{9.058023in}{2.828684in}}{\pgfqpoint{9.056020in}{2.823846in}}{\pgfqpoint{9.056020in}{2.818803in}}%
\pgfpathcurveto{\pgfqpoint{9.056020in}{2.813759in}}{\pgfqpoint{9.058023in}{2.808921in}}{\pgfqpoint{9.061590in}{2.805355in}}%
\pgfpathcurveto{\pgfqpoint{9.065156in}{2.801788in}}{\pgfqpoint{9.069994in}{2.799785in}}{\pgfqpoint{9.075038in}{2.799785in}}%
\pgfpathclose%
\pgfusepath{fill}%
\end{pgfscope}%
\begin{pgfscope}%
\pgfpathrectangle{\pgfqpoint{6.572727in}{0.474100in}}{\pgfqpoint{4.227273in}{3.318700in}}%
\pgfusepath{clip}%
\pgfsetbuttcap%
\pgfsetroundjoin%
\definecolor{currentfill}{rgb}{0.993248,0.906157,0.143936}%
\pgfsetfillcolor{currentfill}%
\pgfsetfillopacity{0.700000}%
\pgfsetlinewidth{0.000000pt}%
\definecolor{currentstroke}{rgb}{0.000000,0.000000,0.000000}%
\pgfsetstrokecolor{currentstroke}%
\pgfsetstrokeopacity{0.700000}%
\pgfsetdash{}{0pt}%
\pgfpathmoveto{\pgfqpoint{7.421822in}{2.948591in}}%
\pgfpathcurveto{\pgfqpoint{7.426866in}{2.948591in}}{\pgfqpoint{7.431704in}{2.950595in}}{\pgfqpoint{7.435270in}{2.954161in}}%
\pgfpathcurveto{\pgfqpoint{7.438837in}{2.957728in}}{\pgfqpoint{7.440840in}{2.962565in}}{\pgfqpoint{7.440840in}{2.967609in}}%
\pgfpathcurveto{\pgfqpoint{7.440840in}{2.972653in}}{\pgfqpoint{7.438837in}{2.977491in}}{\pgfqpoint{7.435270in}{2.981057in}}%
\pgfpathcurveto{\pgfqpoint{7.431704in}{2.984623in}}{\pgfqpoint{7.426866in}{2.986627in}}{\pgfqpoint{7.421822in}{2.986627in}}%
\pgfpathcurveto{\pgfqpoint{7.416779in}{2.986627in}}{\pgfqpoint{7.411941in}{2.984623in}}{\pgfqpoint{7.408374in}{2.981057in}}%
\pgfpathcurveto{\pgfqpoint{7.404808in}{2.977491in}}{\pgfqpoint{7.402804in}{2.972653in}}{\pgfqpoint{7.402804in}{2.967609in}}%
\pgfpathcurveto{\pgfqpoint{7.402804in}{2.962565in}}{\pgfqpoint{7.404808in}{2.957728in}}{\pgfqpoint{7.408374in}{2.954161in}}%
\pgfpathcurveto{\pgfqpoint{7.411941in}{2.950595in}}{\pgfqpoint{7.416779in}{2.948591in}}{\pgfqpoint{7.421822in}{2.948591in}}%
\pgfpathclose%
\pgfusepath{fill}%
\end{pgfscope}%
\begin{pgfscope}%
\pgfpathrectangle{\pgfqpoint{6.572727in}{0.474100in}}{\pgfqpoint{4.227273in}{3.318700in}}%
\pgfusepath{clip}%
\pgfsetbuttcap%
\pgfsetroundjoin%
\definecolor{currentfill}{rgb}{0.267004,0.004874,0.329415}%
\pgfsetfillcolor{currentfill}%
\pgfsetfillopacity{0.700000}%
\pgfsetlinewidth{0.000000pt}%
\definecolor{currentstroke}{rgb}{0.000000,0.000000,0.000000}%
\pgfsetstrokecolor{currentstroke}%
\pgfsetstrokeopacity{0.700000}%
\pgfsetdash{}{0pt}%
\pgfpathmoveto{\pgfqpoint{8.522392in}{1.851862in}}%
\pgfpathcurveto{\pgfqpoint{8.527435in}{1.851862in}}{\pgfqpoint{8.532273in}{1.853865in}}{\pgfqpoint{8.535840in}{1.857432in}}%
\pgfpathcurveto{\pgfqpoint{8.539406in}{1.860998in}}{\pgfqpoint{8.541410in}{1.865836in}}{\pgfqpoint{8.541410in}{1.870880in}}%
\pgfpathcurveto{\pgfqpoint{8.541410in}{1.875923in}}{\pgfqpoint{8.539406in}{1.880761in}}{\pgfqpoint{8.535840in}{1.884328in}}%
\pgfpathcurveto{\pgfqpoint{8.532273in}{1.887894in}}{\pgfqpoint{8.527435in}{1.889898in}}{\pgfqpoint{8.522392in}{1.889898in}}%
\pgfpathcurveto{\pgfqpoint{8.517348in}{1.889898in}}{\pgfqpoint{8.512510in}{1.887894in}}{\pgfqpoint{8.508944in}{1.884328in}}%
\pgfpathcurveto{\pgfqpoint{8.505377in}{1.880761in}}{\pgfqpoint{8.503374in}{1.875923in}}{\pgfqpoint{8.503374in}{1.870880in}}%
\pgfpathcurveto{\pgfqpoint{8.503374in}{1.865836in}}{\pgfqpoint{8.505377in}{1.860998in}}{\pgfqpoint{8.508944in}{1.857432in}}%
\pgfpathcurveto{\pgfqpoint{8.512510in}{1.853865in}}{\pgfqpoint{8.517348in}{1.851862in}}{\pgfqpoint{8.522392in}{1.851862in}}%
\pgfpathclose%
\pgfusepath{fill}%
\end{pgfscope}%
\begin{pgfscope}%
\pgfpathrectangle{\pgfqpoint{6.572727in}{0.474100in}}{\pgfqpoint{4.227273in}{3.318700in}}%
\pgfusepath{clip}%
\pgfsetbuttcap%
\pgfsetroundjoin%
\definecolor{currentfill}{rgb}{0.267004,0.004874,0.329415}%
\pgfsetfillcolor{currentfill}%
\pgfsetfillopacity{0.700000}%
\pgfsetlinewidth{0.000000pt}%
\definecolor{currentstroke}{rgb}{0.000000,0.000000,0.000000}%
\pgfsetstrokecolor{currentstroke}%
\pgfsetstrokeopacity{0.700000}%
\pgfsetdash{}{0pt}%
\pgfpathmoveto{\pgfqpoint{7.738347in}{1.465319in}}%
\pgfpathcurveto{\pgfqpoint{7.743391in}{1.465319in}}{\pgfqpoint{7.748228in}{1.467323in}}{\pgfqpoint{7.751795in}{1.470890in}}%
\pgfpathcurveto{\pgfqpoint{7.755361in}{1.474456in}}{\pgfqpoint{7.757365in}{1.479294in}}{\pgfqpoint{7.757365in}{1.484337in}}%
\pgfpathcurveto{\pgfqpoint{7.757365in}{1.489381in}}{\pgfqpoint{7.755361in}{1.494219in}}{\pgfqpoint{7.751795in}{1.497785in}}%
\pgfpathcurveto{\pgfqpoint{7.748228in}{1.501352in}}{\pgfqpoint{7.743391in}{1.503356in}}{\pgfqpoint{7.738347in}{1.503356in}}%
\pgfpathcurveto{\pgfqpoint{7.733303in}{1.503356in}}{\pgfqpoint{7.728466in}{1.501352in}}{\pgfqpoint{7.724899in}{1.497785in}}%
\pgfpathcurveto{\pgfqpoint{7.721333in}{1.494219in}}{\pgfqpoint{7.719329in}{1.489381in}}{\pgfqpoint{7.719329in}{1.484337in}}%
\pgfpathcurveto{\pgfqpoint{7.719329in}{1.479294in}}{\pgfqpoint{7.721333in}{1.474456in}}{\pgfqpoint{7.724899in}{1.470890in}}%
\pgfpathcurveto{\pgfqpoint{7.728466in}{1.467323in}}{\pgfqpoint{7.733303in}{1.465319in}}{\pgfqpoint{7.738347in}{1.465319in}}%
\pgfpathclose%
\pgfusepath{fill}%
\end{pgfscope}%
\begin{pgfscope}%
\pgfpathrectangle{\pgfqpoint{6.572727in}{0.474100in}}{\pgfqpoint{4.227273in}{3.318700in}}%
\pgfusepath{clip}%
\pgfsetbuttcap%
\pgfsetroundjoin%
\definecolor{currentfill}{rgb}{0.267004,0.004874,0.329415}%
\pgfsetfillcolor{currentfill}%
\pgfsetfillopacity{0.700000}%
\pgfsetlinewidth{0.000000pt}%
\definecolor{currentstroke}{rgb}{0.000000,0.000000,0.000000}%
\pgfsetstrokecolor{currentstroke}%
\pgfsetstrokeopacity{0.700000}%
\pgfsetdash{}{0pt}%
\pgfpathmoveto{\pgfqpoint{8.209600in}{1.614300in}}%
\pgfpathcurveto{\pgfqpoint{8.214644in}{1.614300in}}{\pgfqpoint{8.219482in}{1.616304in}}{\pgfqpoint{8.223048in}{1.619870in}}%
\pgfpathcurveto{\pgfqpoint{8.226614in}{1.623437in}}{\pgfqpoint{8.228618in}{1.628275in}}{\pgfqpoint{8.228618in}{1.633318in}}%
\pgfpathcurveto{\pgfqpoint{8.228618in}{1.638362in}}{\pgfqpoint{8.226614in}{1.643200in}}{\pgfqpoint{8.223048in}{1.646766in}}%
\pgfpathcurveto{\pgfqpoint{8.219482in}{1.650332in}}{\pgfqpoint{8.214644in}{1.652336in}}{\pgfqpoint{8.209600in}{1.652336in}}%
\pgfpathcurveto{\pgfqpoint{8.204556in}{1.652336in}}{\pgfqpoint{8.199719in}{1.650332in}}{\pgfqpoint{8.196152in}{1.646766in}}%
\pgfpathcurveto{\pgfqpoint{8.192586in}{1.643200in}}{\pgfqpoint{8.190582in}{1.638362in}}{\pgfqpoint{8.190582in}{1.633318in}}%
\pgfpathcurveto{\pgfqpoint{8.190582in}{1.628275in}}{\pgfqpoint{8.192586in}{1.623437in}}{\pgfqpoint{8.196152in}{1.619870in}}%
\pgfpathcurveto{\pgfqpoint{8.199719in}{1.616304in}}{\pgfqpoint{8.204556in}{1.614300in}}{\pgfqpoint{8.209600in}{1.614300in}}%
\pgfpathclose%
\pgfusepath{fill}%
\end{pgfscope}%
\begin{pgfscope}%
\pgfpathrectangle{\pgfqpoint{6.572727in}{0.474100in}}{\pgfqpoint{4.227273in}{3.318700in}}%
\pgfusepath{clip}%
\pgfsetbuttcap%
\pgfsetroundjoin%
\definecolor{currentfill}{rgb}{0.993248,0.906157,0.143936}%
\pgfsetfillcolor{currentfill}%
\pgfsetfillopacity{0.700000}%
\pgfsetlinewidth{0.000000pt}%
\definecolor{currentstroke}{rgb}{0.000000,0.000000,0.000000}%
\pgfsetstrokecolor{currentstroke}%
\pgfsetstrokeopacity{0.700000}%
\pgfsetdash{}{0pt}%
\pgfpathmoveto{\pgfqpoint{8.211597in}{3.272556in}}%
\pgfpathcurveto{\pgfqpoint{8.216641in}{3.272556in}}{\pgfqpoint{8.221478in}{3.274560in}}{\pgfqpoint{8.225045in}{3.278127in}}%
\pgfpathcurveto{\pgfqpoint{8.228611in}{3.281693in}}{\pgfqpoint{8.230615in}{3.286531in}}{\pgfqpoint{8.230615in}{3.291574in}}%
\pgfpathcurveto{\pgfqpoint{8.230615in}{3.296618in}}{\pgfqpoint{8.228611in}{3.301456in}}{\pgfqpoint{8.225045in}{3.305022in}}%
\pgfpathcurveto{\pgfqpoint{8.221478in}{3.308589in}}{\pgfqpoint{8.216641in}{3.310593in}}{\pgfqpoint{8.211597in}{3.310593in}}%
\pgfpathcurveto{\pgfqpoint{8.206553in}{3.310593in}}{\pgfqpoint{8.201715in}{3.308589in}}{\pgfqpoint{8.198149in}{3.305022in}}%
\pgfpathcurveto{\pgfqpoint{8.194583in}{3.301456in}}{\pgfqpoint{8.192579in}{3.296618in}}{\pgfqpoint{8.192579in}{3.291574in}}%
\pgfpathcurveto{\pgfqpoint{8.192579in}{3.286531in}}{\pgfqpoint{8.194583in}{3.281693in}}{\pgfqpoint{8.198149in}{3.278127in}}%
\pgfpathcurveto{\pgfqpoint{8.201715in}{3.274560in}}{\pgfqpoint{8.206553in}{3.272556in}}{\pgfqpoint{8.211597in}{3.272556in}}%
\pgfpathclose%
\pgfusepath{fill}%
\end{pgfscope}%
\begin{pgfscope}%
\pgfpathrectangle{\pgfqpoint{6.572727in}{0.474100in}}{\pgfqpoint{4.227273in}{3.318700in}}%
\pgfusepath{clip}%
\pgfsetbuttcap%
\pgfsetroundjoin%
\definecolor{currentfill}{rgb}{0.127568,0.566949,0.550556}%
\pgfsetfillcolor{currentfill}%
\pgfsetfillopacity{0.700000}%
\pgfsetlinewidth{0.000000pt}%
\definecolor{currentstroke}{rgb}{0.000000,0.000000,0.000000}%
\pgfsetstrokecolor{currentstroke}%
\pgfsetstrokeopacity{0.700000}%
\pgfsetdash{}{0pt}%
\pgfpathmoveto{\pgfqpoint{10.157788in}{1.575062in}}%
\pgfpathcurveto{\pgfqpoint{10.162831in}{1.575062in}}{\pgfqpoint{10.167669in}{1.577066in}}{\pgfqpoint{10.171236in}{1.580632in}}%
\pgfpathcurveto{\pgfqpoint{10.174802in}{1.584199in}}{\pgfqpoint{10.176806in}{1.589036in}}{\pgfqpoint{10.176806in}{1.594080in}}%
\pgfpathcurveto{\pgfqpoint{10.176806in}{1.599124in}}{\pgfqpoint{10.174802in}{1.603961in}}{\pgfqpoint{10.171236in}{1.607528in}}%
\pgfpathcurveto{\pgfqpoint{10.167669in}{1.611094in}}{\pgfqpoint{10.162831in}{1.613098in}}{\pgfqpoint{10.157788in}{1.613098in}}%
\pgfpathcurveto{\pgfqpoint{10.152744in}{1.613098in}}{\pgfqpoint{10.147906in}{1.611094in}}{\pgfqpoint{10.144340in}{1.607528in}}%
\pgfpathcurveto{\pgfqpoint{10.140773in}{1.603961in}}{\pgfqpoint{10.138770in}{1.599124in}}{\pgfqpoint{10.138770in}{1.594080in}}%
\pgfpathcurveto{\pgfqpoint{10.138770in}{1.589036in}}{\pgfqpoint{10.140773in}{1.584199in}}{\pgfqpoint{10.144340in}{1.580632in}}%
\pgfpathcurveto{\pgfqpoint{10.147906in}{1.577066in}}{\pgfqpoint{10.152744in}{1.575062in}}{\pgfqpoint{10.157788in}{1.575062in}}%
\pgfpathclose%
\pgfusepath{fill}%
\end{pgfscope}%
\begin{pgfscope}%
\pgfpathrectangle{\pgfqpoint{6.572727in}{0.474100in}}{\pgfqpoint{4.227273in}{3.318700in}}%
\pgfusepath{clip}%
\pgfsetbuttcap%
\pgfsetroundjoin%
\definecolor{currentfill}{rgb}{0.993248,0.906157,0.143936}%
\pgfsetfillcolor{currentfill}%
\pgfsetfillopacity{0.700000}%
\pgfsetlinewidth{0.000000pt}%
\definecolor{currentstroke}{rgb}{0.000000,0.000000,0.000000}%
\pgfsetstrokecolor{currentstroke}%
\pgfsetstrokeopacity{0.700000}%
\pgfsetdash{}{0pt}%
\pgfpathmoveto{\pgfqpoint{8.554084in}{2.730303in}}%
\pgfpathcurveto{\pgfqpoint{8.559127in}{2.730303in}}{\pgfqpoint{8.563965in}{2.732307in}}{\pgfqpoint{8.567532in}{2.735873in}}%
\pgfpathcurveto{\pgfqpoint{8.571098in}{2.739440in}}{\pgfqpoint{8.573102in}{2.744277in}}{\pgfqpoint{8.573102in}{2.749321in}}%
\pgfpathcurveto{\pgfqpoint{8.573102in}{2.754365in}}{\pgfqpoint{8.571098in}{2.759203in}}{\pgfqpoint{8.567532in}{2.762769in}}%
\pgfpathcurveto{\pgfqpoint{8.563965in}{2.766335in}}{\pgfqpoint{8.559127in}{2.768339in}}{\pgfqpoint{8.554084in}{2.768339in}}%
\pgfpathcurveto{\pgfqpoint{8.549040in}{2.768339in}}{\pgfqpoint{8.544202in}{2.766335in}}{\pgfqpoint{8.540636in}{2.762769in}}%
\pgfpathcurveto{\pgfqpoint{8.537070in}{2.759203in}}{\pgfqpoint{8.535066in}{2.754365in}}{\pgfqpoint{8.535066in}{2.749321in}}%
\pgfpathcurveto{\pgfqpoint{8.535066in}{2.744277in}}{\pgfqpoint{8.537070in}{2.739440in}}{\pgfqpoint{8.540636in}{2.735873in}}%
\pgfpathcurveto{\pgfqpoint{8.544202in}{2.732307in}}{\pgfqpoint{8.549040in}{2.730303in}}{\pgfqpoint{8.554084in}{2.730303in}}%
\pgfpathclose%
\pgfusepath{fill}%
\end{pgfscope}%
\begin{pgfscope}%
\pgfpathrectangle{\pgfqpoint{6.572727in}{0.474100in}}{\pgfqpoint{4.227273in}{3.318700in}}%
\pgfusepath{clip}%
\pgfsetbuttcap%
\pgfsetroundjoin%
\definecolor{currentfill}{rgb}{0.127568,0.566949,0.550556}%
\pgfsetfillcolor{currentfill}%
\pgfsetfillopacity{0.700000}%
\pgfsetlinewidth{0.000000pt}%
\definecolor{currentstroke}{rgb}{0.000000,0.000000,0.000000}%
\pgfsetstrokecolor{currentstroke}%
\pgfsetstrokeopacity{0.700000}%
\pgfsetdash{}{0pt}%
\pgfpathmoveto{\pgfqpoint{9.476601in}{1.818879in}}%
\pgfpathcurveto{\pgfqpoint{9.481645in}{1.818879in}}{\pgfqpoint{9.486483in}{1.820882in}}{\pgfqpoint{9.490049in}{1.824449in}}%
\pgfpathcurveto{\pgfqpoint{9.493615in}{1.828015in}}{\pgfqpoint{9.495619in}{1.832853in}}{\pgfqpoint{9.495619in}{1.837897in}}%
\pgfpathcurveto{\pgfqpoint{9.495619in}{1.842940in}}{\pgfqpoint{9.493615in}{1.847778in}}{\pgfqpoint{9.490049in}{1.851345in}}%
\pgfpathcurveto{\pgfqpoint{9.486483in}{1.854911in}}{\pgfqpoint{9.481645in}{1.856915in}}{\pgfqpoint{9.476601in}{1.856915in}}%
\pgfpathcurveto{\pgfqpoint{9.471558in}{1.856915in}}{\pgfqpoint{9.466720in}{1.854911in}}{\pgfqpoint{9.463153in}{1.851345in}}%
\pgfpathcurveto{\pgfqpoint{9.459587in}{1.847778in}}{\pgfqpoint{9.457583in}{1.842940in}}{\pgfqpoint{9.457583in}{1.837897in}}%
\pgfpathcurveto{\pgfqpoint{9.457583in}{1.832853in}}{\pgfqpoint{9.459587in}{1.828015in}}{\pgfqpoint{9.463153in}{1.824449in}}%
\pgfpathcurveto{\pgfqpoint{9.466720in}{1.820882in}}{\pgfqpoint{9.471558in}{1.818879in}}{\pgfqpoint{9.476601in}{1.818879in}}%
\pgfpathclose%
\pgfusepath{fill}%
\end{pgfscope}%
\begin{pgfscope}%
\pgfpathrectangle{\pgfqpoint{6.572727in}{0.474100in}}{\pgfqpoint{4.227273in}{3.318700in}}%
\pgfusepath{clip}%
\pgfsetbuttcap%
\pgfsetroundjoin%
\definecolor{currentfill}{rgb}{0.993248,0.906157,0.143936}%
\pgfsetfillcolor{currentfill}%
\pgfsetfillopacity{0.700000}%
\pgfsetlinewidth{0.000000pt}%
\definecolor{currentstroke}{rgb}{0.000000,0.000000,0.000000}%
\pgfsetstrokecolor{currentstroke}%
\pgfsetstrokeopacity{0.700000}%
\pgfsetdash{}{0pt}%
\pgfpathmoveto{\pgfqpoint{8.696904in}{3.217153in}}%
\pgfpathcurveto{\pgfqpoint{8.701948in}{3.217153in}}{\pgfqpoint{8.706786in}{3.219157in}}{\pgfqpoint{8.710352in}{3.222724in}}%
\pgfpathcurveto{\pgfqpoint{8.713919in}{3.226290in}}{\pgfqpoint{8.715923in}{3.231128in}}{\pgfqpoint{8.715923in}{3.236171in}}%
\pgfpathcurveto{\pgfqpoint{8.715923in}{3.241215in}}{\pgfqpoint{8.713919in}{3.246053in}}{\pgfqpoint{8.710352in}{3.249619in}}%
\pgfpathcurveto{\pgfqpoint{8.706786in}{3.253186in}}{\pgfqpoint{8.701948in}{3.255190in}}{\pgfqpoint{8.696904in}{3.255190in}}%
\pgfpathcurveto{\pgfqpoint{8.691861in}{3.255190in}}{\pgfqpoint{8.687023in}{3.253186in}}{\pgfqpoint{8.683457in}{3.249619in}}%
\pgfpathcurveto{\pgfqpoint{8.679890in}{3.246053in}}{\pgfqpoint{8.677886in}{3.241215in}}{\pgfqpoint{8.677886in}{3.236171in}}%
\pgfpathcurveto{\pgfqpoint{8.677886in}{3.231128in}}{\pgfqpoint{8.679890in}{3.226290in}}{\pgfqpoint{8.683457in}{3.222724in}}%
\pgfpathcurveto{\pgfqpoint{8.687023in}{3.219157in}}{\pgfqpoint{8.691861in}{3.217153in}}{\pgfqpoint{8.696904in}{3.217153in}}%
\pgfpathclose%
\pgfusepath{fill}%
\end{pgfscope}%
\begin{pgfscope}%
\pgfpathrectangle{\pgfqpoint{6.572727in}{0.474100in}}{\pgfqpoint{4.227273in}{3.318700in}}%
\pgfusepath{clip}%
\pgfsetbuttcap%
\pgfsetroundjoin%
\definecolor{currentfill}{rgb}{0.267004,0.004874,0.329415}%
\pgfsetfillcolor{currentfill}%
\pgfsetfillopacity{0.700000}%
\pgfsetlinewidth{0.000000pt}%
\definecolor{currentstroke}{rgb}{0.000000,0.000000,0.000000}%
\pgfsetstrokecolor{currentstroke}%
\pgfsetstrokeopacity{0.700000}%
\pgfsetdash{}{0pt}%
\pgfpathmoveto{\pgfqpoint{7.504100in}{1.742828in}}%
\pgfpathcurveto{\pgfqpoint{7.509143in}{1.742828in}}{\pgfqpoint{7.513981in}{1.744831in}}{\pgfqpoint{7.517547in}{1.748398in}}%
\pgfpathcurveto{\pgfqpoint{7.521114in}{1.751964in}}{\pgfqpoint{7.523118in}{1.756802in}}{\pgfqpoint{7.523118in}{1.761846in}}%
\pgfpathcurveto{\pgfqpoint{7.523118in}{1.766889in}}{\pgfqpoint{7.521114in}{1.771727in}}{\pgfqpoint{7.517547in}{1.775294in}}%
\pgfpathcurveto{\pgfqpoint{7.513981in}{1.778860in}}{\pgfqpoint{7.509143in}{1.780864in}}{\pgfqpoint{7.504100in}{1.780864in}}%
\pgfpathcurveto{\pgfqpoint{7.499056in}{1.780864in}}{\pgfqpoint{7.494218in}{1.778860in}}{\pgfqpoint{7.490652in}{1.775294in}}%
\pgfpathcurveto{\pgfqpoint{7.487085in}{1.771727in}}{\pgfqpoint{7.485081in}{1.766889in}}{\pgfqpoint{7.485081in}{1.761846in}}%
\pgfpathcurveto{\pgfqpoint{7.485081in}{1.756802in}}{\pgfqpoint{7.487085in}{1.751964in}}{\pgfqpoint{7.490652in}{1.748398in}}%
\pgfpathcurveto{\pgfqpoint{7.494218in}{1.744831in}}{\pgfqpoint{7.499056in}{1.742828in}}{\pgfqpoint{7.504100in}{1.742828in}}%
\pgfpathclose%
\pgfusepath{fill}%
\end{pgfscope}%
\begin{pgfscope}%
\pgfpathrectangle{\pgfqpoint{6.572727in}{0.474100in}}{\pgfqpoint{4.227273in}{3.318700in}}%
\pgfusepath{clip}%
\pgfsetbuttcap%
\pgfsetroundjoin%
\definecolor{currentfill}{rgb}{0.127568,0.566949,0.550556}%
\pgfsetfillcolor{currentfill}%
\pgfsetfillopacity{0.700000}%
\pgfsetlinewidth{0.000000pt}%
\definecolor{currentstroke}{rgb}{0.000000,0.000000,0.000000}%
\pgfsetstrokecolor{currentstroke}%
\pgfsetstrokeopacity{0.700000}%
\pgfsetdash{}{0pt}%
\pgfpathmoveto{\pgfqpoint{9.212377in}{1.544617in}}%
\pgfpathcurveto{\pgfqpoint{9.217420in}{1.544617in}}{\pgfqpoint{9.222258in}{1.546621in}}{\pgfqpoint{9.225824in}{1.550188in}}%
\pgfpathcurveto{\pgfqpoint{9.229391in}{1.553754in}}{\pgfqpoint{9.231395in}{1.558592in}}{\pgfqpoint{9.231395in}{1.563635in}}%
\pgfpathcurveto{\pgfqpoint{9.231395in}{1.568679in}}{\pgfqpoint{9.229391in}{1.573517in}}{\pgfqpoint{9.225824in}{1.577083in}}%
\pgfpathcurveto{\pgfqpoint{9.222258in}{1.580650in}}{\pgfqpoint{9.217420in}{1.582654in}}{\pgfqpoint{9.212377in}{1.582654in}}%
\pgfpathcurveto{\pgfqpoint{9.207333in}{1.582654in}}{\pgfqpoint{9.202495in}{1.580650in}}{\pgfqpoint{9.198929in}{1.577083in}}%
\pgfpathcurveto{\pgfqpoint{9.195362in}{1.573517in}}{\pgfqpoint{9.193358in}{1.568679in}}{\pgfqpoint{9.193358in}{1.563635in}}%
\pgfpathcurveto{\pgfqpoint{9.193358in}{1.558592in}}{\pgfqpoint{9.195362in}{1.553754in}}{\pgfqpoint{9.198929in}{1.550188in}}%
\pgfpathcurveto{\pgfqpoint{9.202495in}{1.546621in}}{\pgfqpoint{9.207333in}{1.544617in}}{\pgfqpoint{9.212377in}{1.544617in}}%
\pgfpathclose%
\pgfusepath{fill}%
\end{pgfscope}%
\begin{pgfscope}%
\pgfpathrectangle{\pgfqpoint{6.572727in}{0.474100in}}{\pgfqpoint{4.227273in}{3.318700in}}%
\pgfusepath{clip}%
\pgfsetbuttcap%
\pgfsetroundjoin%
\definecolor{currentfill}{rgb}{0.127568,0.566949,0.550556}%
\pgfsetfillcolor{currentfill}%
\pgfsetfillopacity{0.700000}%
\pgfsetlinewidth{0.000000pt}%
\definecolor{currentstroke}{rgb}{0.000000,0.000000,0.000000}%
\pgfsetstrokecolor{currentstroke}%
\pgfsetstrokeopacity{0.700000}%
\pgfsetdash{}{0pt}%
\pgfpathmoveto{\pgfqpoint{9.314282in}{1.097635in}}%
\pgfpathcurveto{\pgfqpoint{9.319326in}{1.097635in}}{\pgfqpoint{9.324164in}{1.099639in}}{\pgfqpoint{9.327730in}{1.103205in}}%
\pgfpathcurveto{\pgfqpoint{9.331297in}{1.106772in}}{\pgfqpoint{9.333300in}{1.111609in}}{\pgfqpoint{9.333300in}{1.116653in}}%
\pgfpathcurveto{\pgfqpoint{9.333300in}{1.121697in}}{\pgfqpoint{9.331297in}{1.126534in}}{\pgfqpoint{9.327730in}{1.130101in}}%
\pgfpathcurveto{\pgfqpoint{9.324164in}{1.133667in}}{\pgfqpoint{9.319326in}{1.135671in}}{\pgfqpoint{9.314282in}{1.135671in}}%
\pgfpathcurveto{\pgfqpoint{9.309239in}{1.135671in}}{\pgfqpoint{9.304401in}{1.133667in}}{\pgfqpoint{9.300834in}{1.130101in}}%
\pgfpathcurveto{\pgfqpoint{9.297268in}{1.126534in}}{\pgfqpoint{9.295264in}{1.121697in}}{\pgfqpoint{9.295264in}{1.116653in}}%
\pgfpathcurveto{\pgfqpoint{9.295264in}{1.111609in}}{\pgfqpoint{9.297268in}{1.106772in}}{\pgfqpoint{9.300834in}{1.103205in}}%
\pgfpathcurveto{\pgfqpoint{9.304401in}{1.099639in}}{\pgfqpoint{9.309239in}{1.097635in}}{\pgfqpoint{9.314282in}{1.097635in}}%
\pgfpathclose%
\pgfusepath{fill}%
\end{pgfscope}%
\begin{pgfscope}%
\pgfpathrectangle{\pgfqpoint{6.572727in}{0.474100in}}{\pgfqpoint{4.227273in}{3.318700in}}%
\pgfusepath{clip}%
\pgfsetbuttcap%
\pgfsetroundjoin%
\definecolor{currentfill}{rgb}{0.127568,0.566949,0.550556}%
\pgfsetfillcolor{currentfill}%
\pgfsetfillopacity{0.700000}%
\pgfsetlinewidth{0.000000pt}%
\definecolor{currentstroke}{rgb}{0.000000,0.000000,0.000000}%
\pgfsetstrokecolor{currentstroke}%
\pgfsetstrokeopacity{0.700000}%
\pgfsetdash{}{0pt}%
\pgfpathmoveto{\pgfqpoint{9.338851in}{1.852237in}}%
\pgfpathcurveto{\pgfqpoint{9.343895in}{1.852237in}}{\pgfqpoint{9.348733in}{1.854241in}}{\pgfqpoint{9.352299in}{1.857807in}}%
\pgfpathcurveto{\pgfqpoint{9.355866in}{1.861373in}}{\pgfqpoint{9.357870in}{1.866211in}}{\pgfqpoint{9.357870in}{1.871255in}}%
\pgfpathcurveto{\pgfqpoint{9.357870in}{1.876299in}}{\pgfqpoint{9.355866in}{1.881136in}}{\pgfqpoint{9.352299in}{1.884703in}}%
\pgfpathcurveto{\pgfqpoint{9.348733in}{1.888269in}}{\pgfqpoint{9.343895in}{1.890273in}}{\pgfqpoint{9.338851in}{1.890273in}}%
\pgfpathcurveto{\pgfqpoint{9.333808in}{1.890273in}}{\pgfqpoint{9.328970in}{1.888269in}}{\pgfqpoint{9.325404in}{1.884703in}}%
\pgfpathcurveto{\pgfqpoint{9.321837in}{1.881136in}}{\pgfqpoint{9.319833in}{1.876299in}}{\pgfqpoint{9.319833in}{1.871255in}}%
\pgfpathcurveto{\pgfqpoint{9.319833in}{1.866211in}}{\pgfqpoint{9.321837in}{1.861373in}}{\pgfqpoint{9.325404in}{1.857807in}}%
\pgfpathcurveto{\pgfqpoint{9.328970in}{1.854241in}}{\pgfqpoint{9.333808in}{1.852237in}}{\pgfqpoint{9.338851in}{1.852237in}}%
\pgfpathclose%
\pgfusepath{fill}%
\end{pgfscope}%
\begin{pgfscope}%
\pgfpathrectangle{\pgfqpoint{6.572727in}{0.474100in}}{\pgfqpoint{4.227273in}{3.318700in}}%
\pgfusepath{clip}%
\pgfsetbuttcap%
\pgfsetroundjoin%
\definecolor{currentfill}{rgb}{0.127568,0.566949,0.550556}%
\pgfsetfillcolor{currentfill}%
\pgfsetfillopacity{0.700000}%
\pgfsetlinewidth{0.000000pt}%
\definecolor{currentstroke}{rgb}{0.000000,0.000000,0.000000}%
\pgfsetstrokecolor{currentstroke}%
\pgfsetstrokeopacity{0.700000}%
\pgfsetdash{}{0pt}%
\pgfpathmoveto{\pgfqpoint{9.753532in}{1.396403in}}%
\pgfpathcurveto{\pgfqpoint{9.758576in}{1.396403in}}{\pgfqpoint{9.763414in}{1.398407in}}{\pgfqpoint{9.766980in}{1.401974in}}%
\pgfpathcurveto{\pgfqpoint{9.770547in}{1.405540in}}{\pgfqpoint{9.772550in}{1.410378in}}{\pgfqpoint{9.772550in}{1.415421in}}%
\pgfpathcurveto{\pgfqpoint{9.772550in}{1.420465in}}{\pgfqpoint{9.770547in}{1.425303in}}{\pgfqpoint{9.766980in}{1.428869in}}%
\pgfpathcurveto{\pgfqpoint{9.763414in}{1.432436in}}{\pgfqpoint{9.758576in}{1.434440in}}{\pgfqpoint{9.753532in}{1.434440in}}%
\pgfpathcurveto{\pgfqpoint{9.748489in}{1.434440in}}{\pgfqpoint{9.743651in}{1.432436in}}{\pgfqpoint{9.740084in}{1.428869in}}%
\pgfpathcurveto{\pgfqpoint{9.736518in}{1.425303in}}{\pgfqpoint{9.734514in}{1.420465in}}{\pgfqpoint{9.734514in}{1.415421in}}%
\pgfpathcurveto{\pgfqpoint{9.734514in}{1.410378in}}{\pgfqpoint{9.736518in}{1.405540in}}{\pgfqpoint{9.740084in}{1.401974in}}%
\pgfpathcurveto{\pgfqpoint{9.743651in}{1.398407in}}{\pgfqpoint{9.748489in}{1.396403in}}{\pgfqpoint{9.753532in}{1.396403in}}%
\pgfpathclose%
\pgfusepath{fill}%
\end{pgfscope}%
\begin{pgfscope}%
\pgfpathrectangle{\pgfqpoint{6.572727in}{0.474100in}}{\pgfqpoint{4.227273in}{3.318700in}}%
\pgfusepath{clip}%
\pgfsetbuttcap%
\pgfsetroundjoin%
\definecolor{currentfill}{rgb}{0.127568,0.566949,0.550556}%
\pgfsetfillcolor{currentfill}%
\pgfsetfillopacity{0.700000}%
\pgfsetlinewidth{0.000000pt}%
\definecolor{currentstroke}{rgb}{0.000000,0.000000,0.000000}%
\pgfsetstrokecolor{currentstroke}%
\pgfsetstrokeopacity{0.700000}%
\pgfsetdash{}{0pt}%
\pgfpathmoveto{\pgfqpoint{9.121166in}{2.209381in}}%
\pgfpathcurveto{\pgfqpoint{9.126209in}{2.209381in}}{\pgfqpoint{9.131047in}{2.211385in}}{\pgfqpoint{9.134613in}{2.214952in}}%
\pgfpathcurveto{\pgfqpoint{9.138180in}{2.218518in}}{\pgfqpoint{9.140184in}{2.223356in}}{\pgfqpoint{9.140184in}{2.228399in}}%
\pgfpathcurveto{\pgfqpoint{9.140184in}{2.233443in}}{\pgfqpoint{9.138180in}{2.238281in}}{\pgfqpoint{9.134613in}{2.241847in}}%
\pgfpathcurveto{\pgfqpoint{9.131047in}{2.245414in}}{\pgfqpoint{9.126209in}{2.247418in}}{\pgfqpoint{9.121166in}{2.247418in}}%
\pgfpathcurveto{\pgfqpoint{9.116122in}{2.247418in}}{\pgfqpoint{9.111284in}{2.245414in}}{\pgfqpoint{9.107718in}{2.241847in}}%
\pgfpathcurveto{\pgfqpoint{9.104151in}{2.238281in}}{\pgfqpoint{9.102147in}{2.233443in}}{\pgfqpoint{9.102147in}{2.228399in}}%
\pgfpathcurveto{\pgfqpoint{9.102147in}{2.223356in}}{\pgfqpoint{9.104151in}{2.218518in}}{\pgfqpoint{9.107718in}{2.214952in}}%
\pgfpathcurveto{\pgfqpoint{9.111284in}{2.211385in}}{\pgfqpoint{9.116122in}{2.209381in}}{\pgfqpoint{9.121166in}{2.209381in}}%
\pgfpathclose%
\pgfusepath{fill}%
\end{pgfscope}%
\begin{pgfscope}%
\pgfpathrectangle{\pgfqpoint{6.572727in}{0.474100in}}{\pgfqpoint{4.227273in}{3.318700in}}%
\pgfusepath{clip}%
\pgfsetbuttcap%
\pgfsetroundjoin%
\definecolor{currentfill}{rgb}{0.993248,0.906157,0.143936}%
\pgfsetfillcolor{currentfill}%
\pgfsetfillopacity{0.700000}%
\pgfsetlinewidth{0.000000pt}%
\definecolor{currentstroke}{rgb}{0.000000,0.000000,0.000000}%
\pgfsetstrokecolor{currentstroke}%
\pgfsetstrokeopacity{0.700000}%
\pgfsetdash{}{0pt}%
\pgfpathmoveto{\pgfqpoint{7.761253in}{2.659170in}}%
\pgfpathcurveto{\pgfqpoint{7.766297in}{2.659170in}}{\pgfqpoint{7.771135in}{2.661174in}}{\pgfqpoint{7.774701in}{2.664740in}}%
\pgfpathcurveto{\pgfqpoint{7.778267in}{2.668307in}}{\pgfqpoint{7.780271in}{2.673144in}}{\pgfqpoint{7.780271in}{2.678188in}}%
\pgfpathcurveto{\pgfqpoint{7.780271in}{2.683232in}}{\pgfqpoint{7.778267in}{2.688070in}}{\pgfqpoint{7.774701in}{2.691636in}}%
\pgfpathcurveto{\pgfqpoint{7.771135in}{2.695202in}}{\pgfqpoint{7.766297in}{2.697206in}}{\pgfqpoint{7.761253in}{2.697206in}}%
\pgfpathcurveto{\pgfqpoint{7.756209in}{2.697206in}}{\pgfqpoint{7.751372in}{2.695202in}}{\pgfqpoint{7.747805in}{2.691636in}}%
\pgfpathcurveto{\pgfqpoint{7.744239in}{2.688070in}}{\pgfqpoint{7.742235in}{2.683232in}}{\pgfqpoint{7.742235in}{2.678188in}}%
\pgfpathcurveto{\pgfqpoint{7.742235in}{2.673144in}}{\pgfqpoint{7.744239in}{2.668307in}}{\pgfqpoint{7.747805in}{2.664740in}}%
\pgfpathcurveto{\pgfqpoint{7.751372in}{2.661174in}}{\pgfqpoint{7.756209in}{2.659170in}}{\pgfqpoint{7.761253in}{2.659170in}}%
\pgfpathclose%
\pgfusepath{fill}%
\end{pgfscope}%
\begin{pgfscope}%
\pgfpathrectangle{\pgfqpoint{6.572727in}{0.474100in}}{\pgfqpoint{4.227273in}{3.318700in}}%
\pgfusepath{clip}%
\pgfsetbuttcap%
\pgfsetroundjoin%
\definecolor{currentfill}{rgb}{0.127568,0.566949,0.550556}%
\pgfsetfillcolor{currentfill}%
\pgfsetfillopacity{0.700000}%
\pgfsetlinewidth{0.000000pt}%
\definecolor{currentstroke}{rgb}{0.000000,0.000000,0.000000}%
\pgfsetstrokecolor{currentstroke}%
\pgfsetstrokeopacity{0.700000}%
\pgfsetdash{}{0pt}%
\pgfpathmoveto{\pgfqpoint{9.003812in}{1.826922in}}%
\pgfpathcurveto{\pgfqpoint{9.008856in}{1.826922in}}{\pgfqpoint{9.013694in}{1.828925in}}{\pgfqpoint{9.017260in}{1.832492in}}%
\pgfpathcurveto{\pgfqpoint{9.020827in}{1.836058in}}{\pgfqpoint{9.022831in}{1.840896in}}{\pgfqpoint{9.022831in}{1.845940in}}%
\pgfpathcurveto{\pgfqpoint{9.022831in}{1.850983in}}{\pgfqpoint{9.020827in}{1.855821in}}{\pgfqpoint{9.017260in}{1.859388in}}%
\pgfpathcurveto{\pgfqpoint{9.013694in}{1.862954in}}{\pgfqpoint{9.008856in}{1.864958in}}{\pgfqpoint{9.003812in}{1.864958in}}%
\pgfpathcurveto{\pgfqpoint{8.998769in}{1.864958in}}{\pgfqpoint{8.993931in}{1.862954in}}{\pgfqpoint{8.990365in}{1.859388in}}%
\pgfpathcurveto{\pgfqpoint{8.986798in}{1.855821in}}{\pgfqpoint{8.984794in}{1.850983in}}{\pgfqpoint{8.984794in}{1.845940in}}%
\pgfpathcurveto{\pgfqpoint{8.984794in}{1.840896in}}{\pgfqpoint{8.986798in}{1.836058in}}{\pgfqpoint{8.990365in}{1.832492in}}%
\pgfpathcurveto{\pgfqpoint{8.993931in}{1.828925in}}{\pgfqpoint{8.998769in}{1.826922in}}{\pgfqpoint{9.003812in}{1.826922in}}%
\pgfpathclose%
\pgfusepath{fill}%
\end{pgfscope}%
\begin{pgfscope}%
\pgfpathrectangle{\pgfqpoint{6.572727in}{0.474100in}}{\pgfqpoint{4.227273in}{3.318700in}}%
\pgfusepath{clip}%
\pgfsetbuttcap%
\pgfsetroundjoin%
\definecolor{currentfill}{rgb}{0.993248,0.906157,0.143936}%
\pgfsetfillcolor{currentfill}%
\pgfsetfillopacity{0.700000}%
\pgfsetlinewidth{0.000000pt}%
\definecolor{currentstroke}{rgb}{0.000000,0.000000,0.000000}%
\pgfsetstrokecolor{currentstroke}%
\pgfsetstrokeopacity{0.700000}%
\pgfsetdash{}{0pt}%
\pgfpathmoveto{\pgfqpoint{8.194852in}{3.195193in}}%
\pgfpathcurveto{\pgfqpoint{8.199896in}{3.195193in}}{\pgfqpoint{8.204734in}{3.197197in}}{\pgfqpoint{8.208300in}{3.200763in}}%
\pgfpathcurveto{\pgfqpoint{8.211866in}{3.204330in}}{\pgfqpoint{8.213870in}{3.209167in}}{\pgfqpoint{8.213870in}{3.214211in}}%
\pgfpathcurveto{\pgfqpoint{8.213870in}{3.219255in}}{\pgfqpoint{8.211866in}{3.224092in}}{\pgfqpoint{8.208300in}{3.227659in}}%
\pgfpathcurveto{\pgfqpoint{8.204734in}{3.231225in}}{\pgfqpoint{8.199896in}{3.233229in}}{\pgfqpoint{8.194852in}{3.233229in}}%
\pgfpathcurveto{\pgfqpoint{8.189808in}{3.233229in}}{\pgfqpoint{8.184971in}{3.231225in}}{\pgfqpoint{8.181404in}{3.227659in}}%
\pgfpathcurveto{\pgfqpoint{8.177838in}{3.224092in}}{\pgfqpoint{8.175834in}{3.219255in}}{\pgfqpoint{8.175834in}{3.214211in}}%
\pgfpathcurveto{\pgfqpoint{8.175834in}{3.209167in}}{\pgfqpoint{8.177838in}{3.204330in}}{\pgfqpoint{8.181404in}{3.200763in}}%
\pgfpathcurveto{\pgfqpoint{8.184971in}{3.197197in}}{\pgfqpoint{8.189808in}{3.195193in}}{\pgfqpoint{8.194852in}{3.195193in}}%
\pgfpathclose%
\pgfusepath{fill}%
\end{pgfscope}%
\begin{pgfscope}%
\pgfpathrectangle{\pgfqpoint{6.572727in}{0.474100in}}{\pgfqpoint{4.227273in}{3.318700in}}%
\pgfusepath{clip}%
\pgfsetbuttcap%
\pgfsetroundjoin%
\definecolor{currentfill}{rgb}{0.267004,0.004874,0.329415}%
\pgfsetfillcolor{currentfill}%
\pgfsetfillopacity{0.700000}%
\pgfsetlinewidth{0.000000pt}%
\definecolor{currentstroke}{rgb}{0.000000,0.000000,0.000000}%
\pgfsetstrokecolor{currentstroke}%
\pgfsetstrokeopacity{0.700000}%
\pgfsetdash{}{0pt}%
\pgfpathmoveto{\pgfqpoint{7.894513in}{1.834776in}}%
\pgfpathcurveto{\pgfqpoint{7.899556in}{1.834776in}}{\pgfqpoint{7.904394in}{1.836780in}}{\pgfqpoint{7.907961in}{1.840346in}}%
\pgfpathcurveto{\pgfqpoint{7.911527in}{1.843913in}}{\pgfqpoint{7.913531in}{1.848750in}}{\pgfqpoint{7.913531in}{1.853794in}}%
\pgfpathcurveto{\pgfqpoint{7.913531in}{1.858838in}}{\pgfqpoint{7.911527in}{1.863675in}}{\pgfqpoint{7.907961in}{1.867242in}}%
\pgfpathcurveto{\pgfqpoint{7.904394in}{1.870808in}}{\pgfqpoint{7.899556in}{1.872812in}}{\pgfqpoint{7.894513in}{1.872812in}}%
\pgfpathcurveto{\pgfqpoint{7.889469in}{1.872812in}}{\pgfqpoint{7.884631in}{1.870808in}}{\pgfqpoint{7.881065in}{1.867242in}}%
\pgfpathcurveto{\pgfqpoint{7.877498in}{1.863675in}}{\pgfqpoint{7.875494in}{1.858838in}}{\pgfqpoint{7.875494in}{1.853794in}}%
\pgfpathcurveto{\pgfqpoint{7.875494in}{1.848750in}}{\pgfqpoint{7.877498in}{1.843913in}}{\pgfqpoint{7.881065in}{1.840346in}}%
\pgfpathcurveto{\pgfqpoint{7.884631in}{1.836780in}}{\pgfqpoint{7.889469in}{1.834776in}}{\pgfqpoint{7.894513in}{1.834776in}}%
\pgfpathclose%
\pgfusepath{fill}%
\end{pgfscope}%
\begin{pgfscope}%
\pgfpathrectangle{\pgfqpoint{6.572727in}{0.474100in}}{\pgfqpoint{4.227273in}{3.318700in}}%
\pgfusepath{clip}%
\pgfsetbuttcap%
\pgfsetroundjoin%
\definecolor{currentfill}{rgb}{0.267004,0.004874,0.329415}%
\pgfsetfillcolor{currentfill}%
\pgfsetfillopacity{0.700000}%
\pgfsetlinewidth{0.000000pt}%
\definecolor{currentstroke}{rgb}{0.000000,0.000000,0.000000}%
\pgfsetstrokecolor{currentstroke}%
\pgfsetstrokeopacity{0.700000}%
\pgfsetdash{}{0pt}%
\pgfpathmoveto{\pgfqpoint{7.919408in}{2.049837in}}%
\pgfpathcurveto{\pgfqpoint{7.924452in}{2.049837in}}{\pgfqpoint{7.929289in}{2.051841in}}{\pgfqpoint{7.932856in}{2.055408in}}%
\pgfpathcurveto{\pgfqpoint{7.936422in}{2.058974in}}{\pgfqpoint{7.938426in}{2.063812in}}{\pgfqpoint{7.938426in}{2.068855in}}%
\pgfpathcurveto{\pgfqpoint{7.938426in}{2.073899in}}{\pgfqpoint{7.936422in}{2.078737in}}{\pgfqpoint{7.932856in}{2.082303in}}%
\pgfpathcurveto{\pgfqpoint{7.929289in}{2.085870in}}{\pgfqpoint{7.924452in}{2.087874in}}{\pgfqpoint{7.919408in}{2.087874in}}%
\pgfpathcurveto{\pgfqpoint{7.914364in}{2.087874in}}{\pgfqpoint{7.909527in}{2.085870in}}{\pgfqpoint{7.905960in}{2.082303in}}%
\pgfpathcurveto{\pgfqpoint{7.902394in}{2.078737in}}{\pgfqpoint{7.900390in}{2.073899in}}{\pgfqpoint{7.900390in}{2.068855in}}%
\pgfpathcurveto{\pgfqpoint{7.900390in}{2.063812in}}{\pgfqpoint{7.902394in}{2.058974in}}{\pgfqpoint{7.905960in}{2.055408in}}%
\pgfpathcurveto{\pgfqpoint{7.909527in}{2.051841in}}{\pgfqpoint{7.914364in}{2.049837in}}{\pgfqpoint{7.919408in}{2.049837in}}%
\pgfpathclose%
\pgfusepath{fill}%
\end{pgfscope}%
\begin{pgfscope}%
\pgfpathrectangle{\pgfqpoint{6.572727in}{0.474100in}}{\pgfqpoint{4.227273in}{3.318700in}}%
\pgfusepath{clip}%
\pgfsetbuttcap%
\pgfsetroundjoin%
\definecolor{currentfill}{rgb}{0.993248,0.906157,0.143936}%
\pgfsetfillcolor{currentfill}%
\pgfsetfillopacity{0.700000}%
\pgfsetlinewidth{0.000000pt}%
\definecolor{currentstroke}{rgb}{0.000000,0.000000,0.000000}%
\pgfsetstrokecolor{currentstroke}%
\pgfsetstrokeopacity{0.700000}%
\pgfsetdash{}{0pt}%
\pgfpathmoveto{\pgfqpoint{7.527580in}{2.592671in}}%
\pgfpathcurveto{\pgfqpoint{7.532624in}{2.592671in}}{\pgfqpoint{7.537462in}{2.594675in}}{\pgfqpoint{7.541028in}{2.598241in}}%
\pgfpathcurveto{\pgfqpoint{7.544594in}{2.601808in}}{\pgfqpoint{7.546598in}{2.606645in}}{\pgfqpoint{7.546598in}{2.611689in}}%
\pgfpathcurveto{\pgfqpoint{7.546598in}{2.616733in}}{\pgfqpoint{7.544594in}{2.621571in}}{\pgfqpoint{7.541028in}{2.625137in}}%
\pgfpathcurveto{\pgfqpoint{7.537462in}{2.628703in}}{\pgfqpoint{7.532624in}{2.630707in}}{\pgfqpoint{7.527580in}{2.630707in}}%
\pgfpathcurveto{\pgfqpoint{7.522537in}{2.630707in}}{\pgfqpoint{7.517699in}{2.628703in}}{\pgfqpoint{7.514132in}{2.625137in}}%
\pgfpathcurveto{\pgfqpoint{7.510566in}{2.621571in}}{\pgfqpoint{7.508562in}{2.616733in}}{\pgfqpoint{7.508562in}{2.611689in}}%
\pgfpathcurveto{\pgfqpoint{7.508562in}{2.606645in}}{\pgfqpoint{7.510566in}{2.601808in}}{\pgfqpoint{7.514132in}{2.598241in}}%
\pgfpathcurveto{\pgfqpoint{7.517699in}{2.594675in}}{\pgfqpoint{7.522537in}{2.592671in}}{\pgfqpoint{7.527580in}{2.592671in}}%
\pgfpathclose%
\pgfusepath{fill}%
\end{pgfscope}%
\begin{pgfscope}%
\pgfpathrectangle{\pgfqpoint{6.572727in}{0.474100in}}{\pgfqpoint{4.227273in}{3.318700in}}%
\pgfusepath{clip}%
\pgfsetbuttcap%
\pgfsetroundjoin%
\definecolor{currentfill}{rgb}{0.993248,0.906157,0.143936}%
\pgfsetfillcolor{currentfill}%
\pgfsetfillopacity{0.700000}%
\pgfsetlinewidth{0.000000pt}%
\definecolor{currentstroke}{rgb}{0.000000,0.000000,0.000000}%
\pgfsetstrokecolor{currentstroke}%
\pgfsetstrokeopacity{0.700000}%
\pgfsetdash{}{0pt}%
\pgfpathmoveto{\pgfqpoint{8.538993in}{2.840198in}}%
\pgfpathcurveto{\pgfqpoint{8.544037in}{2.840198in}}{\pgfqpoint{8.548875in}{2.842202in}}{\pgfqpoint{8.552441in}{2.845768in}}%
\pgfpathcurveto{\pgfqpoint{8.556007in}{2.849335in}}{\pgfqpoint{8.558011in}{2.854173in}}{\pgfqpoint{8.558011in}{2.859216in}}%
\pgfpathcurveto{\pgfqpoint{8.558011in}{2.864260in}}{\pgfqpoint{8.556007in}{2.869098in}}{\pgfqpoint{8.552441in}{2.872664in}}%
\pgfpathcurveto{\pgfqpoint{8.548875in}{2.876231in}}{\pgfqpoint{8.544037in}{2.878234in}}{\pgfqpoint{8.538993in}{2.878234in}}%
\pgfpathcurveto{\pgfqpoint{8.533949in}{2.878234in}}{\pgfqpoint{8.529112in}{2.876231in}}{\pgfqpoint{8.525545in}{2.872664in}}%
\pgfpathcurveto{\pgfqpoint{8.521979in}{2.869098in}}{\pgfqpoint{8.519975in}{2.864260in}}{\pgfqpoint{8.519975in}{2.859216in}}%
\pgfpathcurveto{\pgfqpoint{8.519975in}{2.854173in}}{\pgfqpoint{8.521979in}{2.849335in}}{\pgfqpoint{8.525545in}{2.845768in}}%
\pgfpathcurveto{\pgfqpoint{8.529112in}{2.842202in}}{\pgfqpoint{8.533949in}{2.840198in}}{\pgfqpoint{8.538993in}{2.840198in}}%
\pgfpathclose%
\pgfusepath{fill}%
\end{pgfscope}%
\begin{pgfscope}%
\pgfpathrectangle{\pgfqpoint{6.572727in}{0.474100in}}{\pgfqpoint{4.227273in}{3.318700in}}%
\pgfusepath{clip}%
\pgfsetbuttcap%
\pgfsetroundjoin%
\definecolor{currentfill}{rgb}{0.267004,0.004874,0.329415}%
\pgfsetfillcolor{currentfill}%
\pgfsetfillopacity{0.700000}%
\pgfsetlinewidth{0.000000pt}%
\definecolor{currentstroke}{rgb}{0.000000,0.000000,0.000000}%
\pgfsetstrokecolor{currentstroke}%
\pgfsetstrokeopacity{0.700000}%
\pgfsetdash{}{0pt}%
\pgfpathmoveto{\pgfqpoint{7.185606in}{1.533941in}}%
\pgfpathcurveto{\pgfqpoint{7.190649in}{1.533941in}}{\pgfqpoint{7.195487in}{1.535945in}}{\pgfqpoint{7.199053in}{1.539511in}}%
\pgfpathcurveto{\pgfqpoint{7.202620in}{1.543078in}}{\pgfqpoint{7.204624in}{1.547916in}}{\pgfqpoint{7.204624in}{1.552959in}}%
\pgfpathcurveto{\pgfqpoint{7.204624in}{1.558003in}}{\pgfqpoint{7.202620in}{1.562841in}}{\pgfqpoint{7.199053in}{1.566407in}}%
\pgfpathcurveto{\pgfqpoint{7.195487in}{1.569974in}}{\pgfqpoint{7.190649in}{1.571977in}}{\pgfqpoint{7.185606in}{1.571977in}}%
\pgfpathcurveto{\pgfqpoint{7.180562in}{1.571977in}}{\pgfqpoint{7.175724in}{1.569974in}}{\pgfqpoint{7.172158in}{1.566407in}}%
\pgfpathcurveto{\pgfqpoint{7.168591in}{1.562841in}}{\pgfqpoint{7.166587in}{1.558003in}}{\pgfqpoint{7.166587in}{1.552959in}}%
\pgfpathcurveto{\pgfqpoint{7.166587in}{1.547916in}}{\pgfqpoint{7.168591in}{1.543078in}}{\pgfqpoint{7.172158in}{1.539511in}}%
\pgfpathcurveto{\pgfqpoint{7.175724in}{1.535945in}}{\pgfqpoint{7.180562in}{1.533941in}}{\pgfqpoint{7.185606in}{1.533941in}}%
\pgfpathclose%
\pgfusepath{fill}%
\end{pgfscope}%
\begin{pgfscope}%
\pgfpathrectangle{\pgfqpoint{6.572727in}{0.474100in}}{\pgfqpoint{4.227273in}{3.318700in}}%
\pgfusepath{clip}%
\pgfsetbuttcap%
\pgfsetroundjoin%
\definecolor{currentfill}{rgb}{0.267004,0.004874,0.329415}%
\pgfsetfillcolor{currentfill}%
\pgfsetfillopacity{0.700000}%
\pgfsetlinewidth{0.000000pt}%
\definecolor{currentstroke}{rgb}{0.000000,0.000000,0.000000}%
\pgfsetstrokecolor{currentstroke}%
\pgfsetstrokeopacity{0.700000}%
\pgfsetdash{}{0pt}%
\pgfpathmoveto{\pgfqpoint{8.430377in}{1.511858in}}%
\pgfpathcurveto{\pgfqpoint{8.435420in}{1.511858in}}{\pgfqpoint{8.440258in}{1.513862in}}{\pgfqpoint{8.443825in}{1.517428in}}%
\pgfpathcurveto{\pgfqpoint{8.447391in}{1.520995in}}{\pgfqpoint{8.449395in}{1.525832in}}{\pgfqpoint{8.449395in}{1.530876in}}%
\pgfpathcurveto{\pgfqpoint{8.449395in}{1.535920in}}{\pgfqpoint{8.447391in}{1.540758in}}{\pgfqpoint{8.443825in}{1.544324in}}%
\pgfpathcurveto{\pgfqpoint{8.440258in}{1.547890in}}{\pgfqpoint{8.435420in}{1.549894in}}{\pgfqpoint{8.430377in}{1.549894in}}%
\pgfpathcurveto{\pgfqpoint{8.425333in}{1.549894in}}{\pgfqpoint{8.420495in}{1.547890in}}{\pgfqpoint{8.416929in}{1.544324in}}%
\pgfpathcurveto{\pgfqpoint{8.413363in}{1.540758in}}{\pgfqpoint{8.411359in}{1.535920in}}{\pgfqpoint{8.411359in}{1.530876in}}%
\pgfpathcurveto{\pgfqpoint{8.411359in}{1.525832in}}{\pgfqpoint{8.413363in}{1.520995in}}{\pgfqpoint{8.416929in}{1.517428in}}%
\pgfpathcurveto{\pgfqpoint{8.420495in}{1.513862in}}{\pgfqpoint{8.425333in}{1.511858in}}{\pgfqpoint{8.430377in}{1.511858in}}%
\pgfpathclose%
\pgfusepath{fill}%
\end{pgfscope}%
\begin{pgfscope}%
\pgfpathrectangle{\pgfqpoint{6.572727in}{0.474100in}}{\pgfqpoint{4.227273in}{3.318700in}}%
\pgfusepath{clip}%
\pgfsetbuttcap%
\pgfsetroundjoin%
\definecolor{currentfill}{rgb}{0.127568,0.566949,0.550556}%
\pgfsetfillcolor{currentfill}%
\pgfsetfillopacity{0.700000}%
\pgfsetlinewidth{0.000000pt}%
\definecolor{currentstroke}{rgb}{0.000000,0.000000,0.000000}%
\pgfsetstrokecolor{currentstroke}%
\pgfsetstrokeopacity{0.700000}%
\pgfsetdash{}{0pt}%
\pgfpathmoveto{\pgfqpoint{9.990781in}{1.756719in}}%
\pgfpathcurveto{\pgfqpoint{9.995825in}{1.756719in}}{\pgfqpoint{10.000662in}{1.758723in}}{\pgfqpoint{10.004229in}{1.762289in}}%
\pgfpathcurveto{\pgfqpoint{10.007795in}{1.765856in}}{\pgfqpoint{10.009799in}{1.770693in}}{\pgfqpoint{10.009799in}{1.775737in}}%
\pgfpathcurveto{\pgfqpoint{10.009799in}{1.780781in}}{\pgfqpoint{10.007795in}{1.785619in}}{\pgfqpoint{10.004229in}{1.789185in}}%
\pgfpathcurveto{\pgfqpoint{10.000662in}{1.792751in}}{\pgfqpoint{9.995825in}{1.794755in}}{\pgfqpoint{9.990781in}{1.794755in}}%
\pgfpathcurveto{\pgfqpoint{9.985737in}{1.794755in}}{\pgfqpoint{9.980900in}{1.792751in}}{\pgfqpoint{9.977333in}{1.789185in}}%
\pgfpathcurveto{\pgfqpoint{9.973767in}{1.785619in}}{\pgfqpoint{9.971763in}{1.780781in}}{\pgfqpoint{9.971763in}{1.775737in}}%
\pgfpathcurveto{\pgfqpoint{9.971763in}{1.770693in}}{\pgfqpoint{9.973767in}{1.765856in}}{\pgfqpoint{9.977333in}{1.762289in}}%
\pgfpathcurveto{\pgfqpoint{9.980900in}{1.758723in}}{\pgfqpoint{9.985737in}{1.756719in}}{\pgfqpoint{9.990781in}{1.756719in}}%
\pgfpathclose%
\pgfusepath{fill}%
\end{pgfscope}%
\begin{pgfscope}%
\pgfpathrectangle{\pgfqpoint{6.572727in}{0.474100in}}{\pgfqpoint{4.227273in}{3.318700in}}%
\pgfusepath{clip}%
\pgfsetbuttcap%
\pgfsetroundjoin%
\definecolor{currentfill}{rgb}{0.127568,0.566949,0.550556}%
\pgfsetfillcolor{currentfill}%
\pgfsetfillopacity{0.700000}%
\pgfsetlinewidth{0.000000pt}%
\definecolor{currentstroke}{rgb}{0.000000,0.000000,0.000000}%
\pgfsetstrokecolor{currentstroke}%
\pgfsetstrokeopacity{0.700000}%
\pgfsetdash{}{0pt}%
\pgfpathmoveto{\pgfqpoint{9.653179in}{1.280965in}}%
\pgfpathcurveto{\pgfqpoint{9.658223in}{1.280965in}}{\pgfqpoint{9.663060in}{1.282969in}}{\pgfqpoint{9.666627in}{1.286536in}}%
\pgfpathcurveto{\pgfqpoint{9.670193in}{1.290102in}}{\pgfqpoint{9.672197in}{1.294940in}}{\pgfqpoint{9.672197in}{1.299983in}}%
\pgfpathcurveto{\pgfqpoint{9.672197in}{1.305027in}}{\pgfqpoint{9.670193in}{1.309865in}}{\pgfqpoint{9.666627in}{1.313431in}}%
\pgfpathcurveto{\pgfqpoint{9.663060in}{1.316998in}}{\pgfqpoint{9.658223in}{1.319002in}}{\pgfqpoint{9.653179in}{1.319002in}}%
\pgfpathcurveto{\pgfqpoint{9.648135in}{1.319002in}}{\pgfqpoint{9.643298in}{1.316998in}}{\pgfqpoint{9.639731in}{1.313431in}}%
\pgfpathcurveto{\pgfqpoint{9.636165in}{1.309865in}}{\pgfqpoint{9.634161in}{1.305027in}}{\pgfqpoint{9.634161in}{1.299983in}}%
\pgfpathcurveto{\pgfqpoint{9.634161in}{1.294940in}}{\pgfqpoint{9.636165in}{1.290102in}}{\pgfqpoint{9.639731in}{1.286536in}}%
\pgfpathcurveto{\pgfqpoint{9.643298in}{1.282969in}}{\pgfqpoint{9.648135in}{1.280965in}}{\pgfqpoint{9.653179in}{1.280965in}}%
\pgfpathclose%
\pgfusepath{fill}%
\end{pgfscope}%
\begin{pgfscope}%
\pgfpathrectangle{\pgfqpoint{6.572727in}{0.474100in}}{\pgfqpoint{4.227273in}{3.318700in}}%
\pgfusepath{clip}%
\pgfsetbuttcap%
\pgfsetroundjoin%
\definecolor{currentfill}{rgb}{0.127568,0.566949,0.550556}%
\pgfsetfillcolor{currentfill}%
\pgfsetfillopacity{0.700000}%
\pgfsetlinewidth{0.000000pt}%
\definecolor{currentstroke}{rgb}{0.000000,0.000000,0.000000}%
\pgfsetstrokecolor{currentstroke}%
\pgfsetstrokeopacity{0.700000}%
\pgfsetdash{}{0pt}%
\pgfpathmoveto{\pgfqpoint{9.630854in}{1.570718in}}%
\pgfpathcurveto{\pgfqpoint{9.635898in}{1.570718in}}{\pgfqpoint{9.640735in}{1.572722in}}{\pgfqpoint{9.644302in}{1.576289in}}%
\pgfpathcurveto{\pgfqpoint{9.647868in}{1.579855in}}{\pgfqpoint{9.649872in}{1.584693in}}{\pgfqpoint{9.649872in}{1.589736in}}%
\pgfpathcurveto{\pgfqpoint{9.649872in}{1.594780in}}{\pgfqpoint{9.647868in}{1.599618in}}{\pgfqpoint{9.644302in}{1.603184in}}%
\pgfpathcurveto{\pgfqpoint{9.640735in}{1.606751in}}{\pgfqpoint{9.635898in}{1.608755in}}{\pgfqpoint{9.630854in}{1.608755in}}%
\pgfpathcurveto{\pgfqpoint{9.625810in}{1.608755in}}{\pgfqpoint{9.620972in}{1.606751in}}{\pgfqpoint{9.617406in}{1.603184in}}%
\pgfpathcurveto{\pgfqpoint{9.613840in}{1.599618in}}{\pgfqpoint{9.611836in}{1.594780in}}{\pgfqpoint{9.611836in}{1.589736in}}%
\pgfpathcurveto{\pgfqpoint{9.611836in}{1.584693in}}{\pgfqpoint{9.613840in}{1.579855in}}{\pgfqpoint{9.617406in}{1.576289in}}%
\pgfpathcurveto{\pgfqpoint{9.620972in}{1.572722in}}{\pgfqpoint{9.625810in}{1.570718in}}{\pgfqpoint{9.630854in}{1.570718in}}%
\pgfpathclose%
\pgfusepath{fill}%
\end{pgfscope}%
\begin{pgfscope}%
\pgfpathrectangle{\pgfqpoint{6.572727in}{0.474100in}}{\pgfqpoint{4.227273in}{3.318700in}}%
\pgfusepath{clip}%
\pgfsetbuttcap%
\pgfsetroundjoin%
\definecolor{currentfill}{rgb}{0.267004,0.004874,0.329415}%
\pgfsetfillcolor{currentfill}%
\pgfsetfillopacity{0.700000}%
\pgfsetlinewidth{0.000000pt}%
\definecolor{currentstroke}{rgb}{0.000000,0.000000,0.000000}%
\pgfsetstrokecolor{currentstroke}%
\pgfsetstrokeopacity{0.700000}%
\pgfsetdash{}{0pt}%
\pgfpathmoveto{\pgfqpoint{7.593807in}{1.159188in}}%
\pgfpathcurveto{\pgfqpoint{7.598851in}{1.159188in}}{\pgfqpoint{7.603689in}{1.161192in}}{\pgfqpoint{7.607255in}{1.164758in}}%
\pgfpathcurveto{\pgfqpoint{7.610822in}{1.168325in}}{\pgfqpoint{7.612825in}{1.173163in}}{\pgfqpoint{7.612825in}{1.178206in}}%
\pgfpathcurveto{\pgfqpoint{7.612825in}{1.183250in}}{\pgfqpoint{7.610822in}{1.188088in}}{\pgfqpoint{7.607255in}{1.191654in}}%
\pgfpathcurveto{\pgfqpoint{7.603689in}{1.195221in}}{\pgfqpoint{7.598851in}{1.197224in}}{\pgfqpoint{7.593807in}{1.197224in}}%
\pgfpathcurveto{\pgfqpoint{7.588764in}{1.197224in}}{\pgfqpoint{7.583926in}{1.195221in}}{\pgfqpoint{7.580359in}{1.191654in}}%
\pgfpathcurveto{\pgfqpoint{7.576793in}{1.188088in}}{\pgfqpoint{7.574789in}{1.183250in}}{\pgfqpoint{7.574789in}{1.178206in}}%
\pgfpathcurveto{\pgfqpoint{7.574789in}{1.173163in}}{\pgfqpoint{7.576793in}{1.168325in}}{\pgfqpoint{7.580359in}{1.164758in}}%
\pgfpathcurveto{\pgfqpoint{7.583926in}{1.161192in}}{\pgfqpoint{7.588764in}{1.159188in}}{\pgfqpoint{7.593807in}{1.159188in}}%
\pgfpathclose%
\pgfusepath{fill}%
\end{pgfscope}%
\begin{pgfscope}%
\pgfpathrectangle{\pgfqpoint{6.572727in}{0.474100in}}{\pgfqpoint{4.227273in}{3.318700in}}%
\pgfusepath{clip}%
\pgfsetbuttcap%
\pgfsetroundjoin%
\definecolor{currentfill}{rgb}{0.267004,0.004874,0.329415}%
\pgfsetfillcolor{currentfill}%
\pgfsetfillopacity{0.700000}%
\pgfsetlinewidth{0.000000pt}%
\definecolor{currentstroke}{rgb}{0.000000,0.000000,0.000000}%
\pgfsetstrokecolor{currentstroke}%
\pgfsetstrokeopacity{0.700000}%
\pgfsetdash{}{0pt}%
\pgfpathmoveto{\pgfqpoint{7.796717in}{1.533100in}}%
\pgfpathcurveto{\pgfqpoint{7.801761in}{1.533100in}}{\pgfqpoint{7.806599in}{1.535104in}}{\pgfqpoint{7.810165in}{1.538671in}}%
\pgfpathcurveto{\pgfqpoint{7.813732in}{1.542237in}}{\pgfqpoint{7.815736in}{1.547075in}}{\pgfqpoint{7.815736in}{1.552118in}}%
\pgfpathcurveto{\pgfqpoint{7.815736in}{1.557162in}}{\pgfqpoint{7.813732in}{1.562000in}}{\pgfqpoint{7.810165in}{1.565566in}}%
\pgfpathcurveto{\pgfqpoint{7.806599in}{1.569133in}}{\pgfqpoint{7.801761in}{1.571137in}}{\pgfqpoint{7.796717in}{1.571137in}}%
\pgfpathcurveto{\pgfqpoint{7.791674in}{1.571137in}}{\pgfqpoint{7.786836in}{1.569133in}}{\pgfqpoint{7.783270in}{1.565566in}}%
\pgfpathcurveto{\pgfqpoint{7.779703in}{1.562000in}}{\pgfqpoint{7.777699in}{1.557162in}}{\pgfqpoint{7.777699in}{1.552118in}}%
\pgfpathcurveto{\pgfqpoint{7.777699in}{1.547075in}}{\pgfqpoint{7.779703in}{1.542237in}}{\pgfqpoint{7.783270in}{1.538671in}}%
\pgfpathcurveto{\pgfqpoint{7.786836in}{1.535104in}}{\pgfqpoint{7.791674in}{1.533100in}}{\pgfqpoint{7.796717in}{1.533100in}}%
\pgfpathclose%
\pgfusepath{fill}%
\end{pgfscope}%
\begin{pgfscope}%
\pgfpathrectangle{\pgfqpoint{6.572727in}{0.474100in}}{\pgfqpoint{4.227273in}{3.318700in}}%
\pgfusepath{clip}%
\pgfsetbuttcap%
\pgfsetroundjoin%
\definecolor{currentfill}{rgb}{0.127568,0.566949,0.550556}%
\pgfsetfillcolor{currentfill}%
\pgfsetfillopacity{0.700000}%
\pgfsetlinewidth{0.000000pt}%
\definecolor{currentstroke}{rgb}{0.000000,0.000000,0.000000}%
\pgfsetstrokecolor{currentstroke}%
\pgfsetstrokeopacity{0.700000}%
\pgfsetdash{}{0pt}%
\pgfpathmoveto{\pgfqpoint{9.555610in}{1.856996in}}%
\pgfpathcurveto{\pgfqpoint{9.560654in}{1.856996in}}{\pgfqpoint{9.565491in}{1.859000in}}{\pgfqpoint{9.569058in}{1.862566in}}%
\pgfpathcurveto{\pgfqpoint{9.572624in}{1.866132in}}{\pgfqpoint{9.574628in}{1.870970in}}{\pgfqpoint{9.574628in}{1.876014in}}%
\pgfpathcurveto{\pgfqpoint{9.574628in}{1.881058in}}{\pgfqpoint{9.572624in}{1.885895in}}{\pgfqpoint{9.569058in}{1.889462in}}%
\pgfpathcurveto{\pgfqpoint{9.565491in}{1.893028in}}{\pgfqpoint{9.560654in}{1.895032in}}{\pgfqpoint{9.555610in}{1.895032in}}%
\pgfpathcurveto{\pgfqpoint{9.550566in}{1.895032in}}{\pgfqpoint{9.545728in}{1.893028in}}{\pgfqpoint{9.542162in}{1.889462in}}%
\pgfpathcurveto{\pgfqpoint{9.538596in}{1.885895in}}{\pgfqpoint{9.536592in}{1.881058in}}{\pgfqpoint{9.536592in}{1.876014in}}%
\pgfpathcurveto{\pgfqpoint{9.536592in}{1.870970in}}{\pgfqpoint{9.538596in}{1.866132in}}{\pgfqpoint{9.542162in}{1.862566in}}%
\pgfpathcurveto{\pgfqpoint{9.545728in}{1.859000in}}{\pgfqpoint{9.550566in}{1.856996in}}{\pgfqpoint{9.555610in}{1.856996in}}%
\pgfpathclose%
\pgfusepath{fill}%
\end{pgfscope}%
\begin{pgfscope}%
\pgfpathrectangle{\pgfqpoint{6.572727in}{0.474100in}}{\pgfqpoint{4.227273in}{3.318700in}}%
\pgfusepath{clip}%
\pgfsetbuttcap%
\pgfsetroundjoin%
\definecolor{currentfill}{rgb}{0.267004,0.004874,0.329415}%
\pgfsetfillcolor{currentfill}%
\pgfsetfillopacity{0.700000}%
\pgfsetlinewidth{0.000000pt}%
\definecolor{currentstroke}{rgb}{0.000000,0.000000,0.000000}%
\pgfsetstrokecolor{currentstroke}%
\pgfsetstrokeopacity{0.700000}%
\pgfsetdash{}{0pt}%
\pgfpathmoveto{\pgfqpoint{7.679271in}{0.886242in}}%
\pgfpathcurveto{\pgfqpoint{7.684315in}{0.886242in}}{\pgfqpoint{7.689153in}{0.888246in}}{\pgfqpoint{7.692719in}{0.891813in}}%
\pgfpathcurveto{\pgfqpoint{7.696286in}{0.895379in}}{\pgfqpoint{7.698290in}{0.900217in}}{\pgfqpoint{7.698290in}{0.905260in}}%
\pgfpathcurveto{\pgfqpoint{7.698290in}{0.910304in}}{\pgfqpoint{7.696286in}{0.915142in}}{\pgfqpoint{7.692719in}{0.918708in}}%
\pgfpathcurveto{\pgfqpoint{7.689153in}{0.922275in}}{\pgfqpoint{7.684315in}{0.924279in}}{\pgfqpoint{7.679271in}{0.924279in}}%
\pgfpathcurveto{\pgfqpoint{7.674228in}{0.924279in}}{\pgfqpoint{7.669390in}{0.922275in}}{\pgfqpoint{7.665824in}{0.918708in}}%
\pgfpathcurveto{\pgfqpoint{7.662257in}{0.915142in}}{\pgfqpoint{7.660253in}{0.910304in}}{\pgfqpoint{7.660253in}{0.905260in}}%
\pgfpathcurveto{\pgfqpoint{7.660253in}{0.900217in}}{\pgfqpoint{7.662257in}{0.895379in}}{\pgfqpoint{7.665824in}{0.891813in}}%
\pgfpathcurveto{\pgfqpoint{7.669390in}{0.888246in}}{\pgfqpoint{7.674228in}{0.886242in}}{\pgfqpoint{7.679271in}{0.886242in}}%
\pgfpathclose%
\pgfusepath{fill}%
\end{pgfscope}%
\begin{pgfscope}%
\pgfpathrectangle{\pgfqpoint{6.572727in}{0.474100in}}{\pgfqpoint{4.227273in}{3.318700in}}%
\pgfusepath{clip}%
\pgfsetbuttcap%
\pgfsetroundjoin%
\definecolor{currentfill}{rgb}{0.993248,0.906157,0.143936}%
\pgfsetfillcolor{currentfill}%
\pgfsetfillopacity{0.700000}%
\pgfsetlinewidth{0.000000pt}%
\definecolor{currentstroke}{rgb}{0.000000,0.000000,0.000000}%
\pgfsetstrokecolor{currentstroke}%
\pgfsetstrokeopacity{0.700000}%
\pgfsetdash{}{0pt}%
\pgfpathmoveto{\pgfqpoint{8.637455in}{2.661919in}}%
\pgfpathcurveto{\pgfqpoint{8.642498in}{2.661919in}}{\pgfqpoint{8.647336in}{2.663923in}}{\pgfqpoint{8.650903in}{2.667490in}}%
\pgfpathcurveto{\pgfqpoint{8.654469in}{2.671056in}}{\pgfqpoint{8.656473in}{2.675894in}}{\pgfqpoint{8.656473in}{2.680937in}}%
\pgfpathcurveto{\pgfqpoint{8.656473in}{2.685981in}}{\pgfqpoint{8.654469in}{2.690819in}}{\pgfqpoint{8.650903in}{2.694385in}}%
\pgfpathcurveto{\pgfqpoint{8.647336in}{2.697952in}}{\pgfqpoint{8.642498in}{2.699956in}}{\pgfqpoint{8.637455in}{2.699956in}}%
\pgfpathcurveto{\pgfqpoint{8.632411in}{2.699956in}}{\pgfqpoint{8.627573in}{2.697952in}}{\pgfqpoint{8.624007in}{2.694385in}}%
\pgfpathcurveto{\pgfqpoint{8.620440in}{2.690819in}}{\pgfqpoint{8.618437in}{2.685981in}}{\pgfqpoint{8.618437in}{2.680937in}}%
\pgfpathcurveto{\pgfqpoint{8.618437in}{2.675894in}}{\pgfqpoint{8.620440in}{2.671056in}}{\pgfqpoint{8.624007in}{2.667490in}}%
\pgfpathcurveto{\pgfqpoint{8.627573in}{2.663923in}}{\pgfqpoint{8.632411in}{2.661919in}}{\pgfqpoint{8.637455in}{2.661919in}}%
\pgfpathclose%
\pgfusepath{fill}%
\end{pgfscope}%
\begin{pgfscope}%
\pgfpathrectangle{\pgfqpoint{6.572727in}{0.474100in}}{\pgfqpoint{4.227273in}{3.318700in}}%
\pgfusepath{clip}%
\pgfsetbuttcap%
\pgfsetroundjoin%
\definecolor{currentfill}{rgb}{0.127568,0.566949,0.550556}%
\pgfsetfillcolor{currentfill}%
\pgfsetfillopacity{0.700000}%
\pgfsetlinewidth{0.000000pt}%
\definecolor{currentstroke}{rgb}{0.000000,0.000000,0.000000}%
\pgfsetstrokecolor{currentstroke}%
\pgfsetstrokeopacity{0.700000}%
\pgfsetdash{}{0pt}%
\pgfpathmoveto{\pgfqpoint{9.117483in}{1.724453in}}%
\pgfpathcurveto{\pgfqpoint{9.122527in}{1.724453in}}{\pgfqpoint{9.127364in}{1.726457in}}{\pgfqpoint{9.130931in}{1.730023in}}%
\pgfpathcurveto{\pgfqpoint{9.134497in}{1.733590in}}{\pgfqpoint{9.136501in}{1.738428in}}{\pgfqpoint{9.136501in}{1.743471in}}%
\pgfpathcurveto{\pgfqpoint{9.136501in}{1.748515in}}{\pgfqpoint{9.134497in}{1.753353in}}{\pgfqpoint{9.130931in}{1.756919in}}%
\pgfpathcurveto{\pgfqpoint{9.127364in}{1.760485in}}{\pgfqpoint{9.122527in}{1.762489in}}{\pgfqpoint{9.117483in}{1.762489in}}%
\pgfpathcurveto{\pgfqpoint{9.112439in}{1.762489in}}{\pgfqpoint{9.107602in}{1.760485in}}{\pgfqpoint{9.104035in}{1.756919in}}%
\pgfpathcurveto{\pgfqpoint{9.100469in}{1.753353in}}{\pgfqpoint{9.098465in}{1.748515in}}{\pgfqpoint{9.098465in}{1.743471in}}%
\pgfpathcurveto{\pgfqpoint{9.098465in}{1.738428in}}{\pgfqpoint{9.100469in}{1.733590in}}{\pgfqpoint{9.104035in}{1.730023in}}%
\pgfpathcurveto{\pgfqpoint{9.107602in}{1.726457in}}{\pgfqpoint{9.112439in}{1.724453in}}{\pgfqpoint{9.117483in}{1.724453in}}%
\pgfpathclose%
\pgfusepath{fill}%
\end{pgfscope}%
\begin{pgfscope}%
\pgfpathrectangle{\pgfqpoint{6.572727in}{0.474100in}}{\pgfqpoint{4.227273in}{3.318700in}}%
\pgfusepath{clip}%
\pgfsetbuttcap%
\pgfsetroundjoin%
\definecolor{currentfill}{rgb}{0.127568,0.566949,0.550556}%
\pgfsetfillcolor{currentfill}%
\pgfsetfillopacity{0.700000}%
\pgfsetlinewidth{0.000000pt}%
\definecolor{currentstroke}{rgb}{0.000000,0.000000,0.000000}%
\pgfsetstrokecolor{currentstroke}%
\pgfsetstrokeopacity{0.700000}%
\pgfsetdash{}{0pt}%
\pgfpathmoveto{\pgfqpoint{9.693644in}{1.789993in}}%
\pgfpathcurveto{\pgfqpoint{9.698688in}{1.789993in}}{\pgfqpoint{9.703526in}{1.791997in}}{\pgfqpoint{9.707092in}{1.795563in}}%
\pgfpathcurveto{\pgfqpoint{9.710658in}{1.799130in}}{\pgfqpoint{9.712662in}{1.803967in}}{\pgfqpoint{9.712662in}{1.809011in}}%
\pgfpathcurveto{\pgfqpoint{9.712662in}{1.814055in}}{\pgfqpoint{9.710658in}{1.818892in}}{\pgfqpoint{9.707092in}{1.822459in}}%
\pgfpathcurveto{\pgfqpoint{9.703526in}{1.826025in}}{\pgfqpoint{9.698688in}{1.828029in}}{\pgfqpoint{9.693644in}{1.828029in}}%
\pgfpathcurveto{\pgfqpoint{9.688600in}{1.828029in}}{\pgfqpoint{9.683763in}{1.826025in}}{\pgfqpoint{9.680196in}{1.822459in}}%
\pgfpathcurveto{\pgfqpoint{9.676630in}{1.818892in}}{\pgfqpoint{9.674626in}{1.814055in}}{\pgfqpoint{9.674626in}{1.809011in}}%
\pgfpathcurveto{\pgfqpoint{9.674626in}{1.803967in}}{\pgfqpoint{9.676630in}{1.799130in}}{\pgfqpoint{9.680196in}{1.795563in}}%
\pgfpathcurveto{\pgfqpoint{9.683763in}{1.791997in}}{\pgfqpoint{9.688600in}{1.789993in}}{\pgfqpoint{9.693644in}{1.789993in}}%
\pgfpathclose%
\pgfusepath{fill}%
\end{pgfscope}%
\begin{pgfscope}%
\pgfpathrectangle{\pgfqpoint{6.572727in}{0.474100in}}{\pgfqpoint{4.227273in}{3.318700in}}%
\pgfusepath{clip}%
\pgfsetbuttcap%
\pgfsetroundjoin%
\definecolor{currentfill}{rgb}{0.127568,0.566949,0.550556}%
\pgfsetfillcolor{currentfill}%
\pgfsetfillopacity{0.700000}%
\pgfsetlinewidth{0.000000pt}%
\definecolor{currentstroke}{rgb}{0.000000,0.000000,0.000000}%
\pgfsetstrokecolor{currentstroke}%
\pgfsetstrokeopacity{0.700000}%
\pgfsetdash{}{0pt}%
\pgfpathmoveto{\pgfqpoint{9.619127in}{1.971599in}}%
\pgfpathcurveto{\pgfqpoint{9.624170in}{1.971599in}}{\pgfqpoint{9.629008in}{1.973603in}}{\pgfqpoint{9.632574in}{1.977169in}}%
\pgfpathcurveto{\pgfqpoint{9.636141in}{1.980736in}}{\pgfqpoint{9.638145in}{1.985573in}}{\pgfqpoint{9.638145in}{1.990617in}}%
\pgfpathcurveto{\pgfqpoint{9.638145in}{1.995661in}}{\pgfqpoint{9.636141in}{2.000499in}}{\pgfqpoint{9.632574in}{2.004065in}}%
\pgfpathcurveto{\pgfqpoint{9.629008in}{2.007631in}}{\pgfqpoint{9.624170in}{2.009635in}}{\pgfqpoint{9.619127in}{2.009635in}}%
\pgfpathcurveto{\pgfqpoint{9.614083in}{2.009635in}}{\pgfqpoint{9.609245in}{2.007631in}}{\pgfqpoint{9.605679in}{2.004065in}}%
\pgfpathcurveto{\pgfqpoint{9.602112in}{2.000499in}}{\pgfqpoint{9.600108in}{1.995661in}}{\pgfqpoint{9.600108in}{1.990617in}}%
\pgfpathcurveto{\pgfqpoint{9.600108in}{1.985573in}}{\pgfqpoint{9.602112in}{1.980736in}}{\pgfqpoint{9.605679in}{1.977169in}}%
\pgfpathcurveto{\pgfqpoint{9.609245in}{1.973603in}}{\pgfqpoint{9.614083in}{1.971599in}}{\pgfqpoint{9.619127in}{1.971599in}}%
\pgfpathclose%
\pgfusepath{fill}%
\end{pgfscope}%
\begin{pgfscope}%
\pgfpathrectangle{\pgfqpoint{6.572727in}{0.474100in}}{\pgfqpoint{4.227273in}{3.318700in}}%
\pgfusepath{clip}%
\pgfsetbuttcap%
\pgfsetroundjoin%
\definecolor{currentfill}{rgb}{0.993248,0.906157,0.143936}%
\pgfsetfillcolor{currentfill}%
\pgfsetfillopacity{0.700000}%
\pgfsetlinewidth{0.000000pt}%
\definecolor{currentstroke}{rgb}{0.000000,0.000000,0.000000}%
\pgfsetstrokecolor{currentstroke}%
\pgfsetstrokeopacity{0.700000}%
\pgfsetdash{}{0pt}%
\pgfpathmoveto{\pgfqpoint{7.627012in}{2.515853in}}%
\pgfpathcurveto{\pgfqpoint{7.632056in}{2.515853in}}{\pgfqpoint{7.636894in}{2.517857in}}{\pgfqpoint{7.640460in}{2.521423in}}%
\pgfpathcurveto{\pgfqpoint{7.644027in}{2.524989in}}{\pgfqpoint{7.646031in}{2.529827in}}{\pgfqpoint{7.646031in}{2.534871in}}%
\pgfpathcurveto{\pgfqpoint{7.646031in}{2.539915in}}{\pgfqpoint{7.644027in}{2.544752in}}{\pgfqpoint{7.640460in}{2.548319in}}%
\pgfpathcurveto{\pgfqpoint{7.636894in}{2.551885in}}{\pgfqpoint{7.632056in}{2.553889in}}{\pgfqpoint{7.627012in}{2.553889in}}%
\pgfpathcurveto{\pgfqpoint{7.621969in}{2.553889in}}{\pgfqpoint{7.617131in}{2.551885in}}{\pgfqpoint{7.613565in}{2.548319in}}%
\pgfpathcurveto{\pgfqpoint{7.609998in}{2.544752in}}{\pgfqpoint{7.607994in}{2.539915in}}{\pgfqpoint{7.607994in}{2.534871in}}%
\pgfpathcurveto{\pgfqpoint{7.607994in}{2.529827in}}{\pgfqpoint{7.609998in}{2.524989in}}{\pgfqpoint{7.613565in}{2.521423in}}%
\pgfpathcurveto{\pgfqpoint{7.617131in}{2.517857in}}{\pgfqpoint{7.621969in}{2.515853in}}{\pgfqpoint{7.627012in}{2.515853in}}%
\pgfpathclose%
\pgfusepath{fill}%
\end{pgfscope}%
\begin{pgfscope}%
\pgfpathrectangle{\pgfqpoint{6.572727in}{0.474100in}}{\pgfqpoint{4.227273in}{3.318700in}}%
\pgfusepath{clip}%
\pgfsetbuttcap%
\pgfsetroundjoin%
\definecolor{currentfill}{rgb}{0.993248,0.906157,0.143936}%
\pgfsetfillcolor{currentfill}%
\pgfsetfillopacity{0.700000}%
\pgfsetlinewidth{0.000000pt}%
\definecolor{currentstroke}{rgb}{0.000000,0.000000,0.000000}%
\pgfsetstrokecolor{currentstroke}%
\pgfsetstrokeopacity{0.700000}%
\pgfsetdash{}{0pt}%
\pgfpathmoveto{\pgfqpoint{7.374058in}{3.103701in}}%
\pgfpathcurveto{\pgfqpoint{7.379101in}{3.103701in}}{\pgfqpoint{7.383939in}{3.105705in}}{\pgfqpoint{7.387505in}{3.109271in}}%
\pgfpathcurveto{\pgfqpoint{7.391072in}{3.112837in}}{\pgfqpoint{7.393076in}{3.117675in}}{\pgfqpoint{7.393076in}{3.122719in}}%
\pgfpathcurveto{\pgfqpoint{7.393076in}{3.127763in}}{\pgfqpoint{7.391072in}{3.132600in}}{\pgfqpoint{7.387505in}{3.136167in}}%
\pgfpathcurveto{\pgfqpoint{7.383939in}{3.139733in}}{\pgfqpoint{7.379101in}{3.141737in}}{\pgfqpoint{7.374058in}{3.141737in}}%
\pgfpathcurveto{\pgfqpoint{7.369014in}{3.141737in}}{\pgfqpoint{7.364176in}{3.139733in}}{\pgfqpoint{7.360610in}{3.136167in}}%
\pgfpathcurveto{\pgfqpoint{7.357043in}{3.132600in}}{\pgfqpoint{7.355039in}{3.127763in}}{\pgfqpoint{7.355039in}{3.122719in}}%
\pgfpathcurveto{\pgfqpoint{7.355039in}{3.117675in}}{\pgfqpoint{7.357043in}{3.112837in}}{\pgfqpoint{7.360610in}{3.109271in}}%
\pgfpathcurveto{\pgfqpoint{7.364176in}{3.105705in}}{\pgfqpoint{7.369014in}{3.103701in}}{\pgfqpoint{7.374058in}{3.103701in}}%
\pgfpathclose%
\pgfusepath{fill}%
\end{pgfscope}%
\begin{pgfscope}%
\pgfpathrectangle{\pgfqpoint{6.572727in}{0.474100in}}{\pgfqpoint{4.227273in}{3.318700in}}%
\pgfusepath{clip}%
\pgfsetbuttcap%
\pgfsetroundjoin%
\definecolor{currentfill}{rgb}{0.127568,0.566949,0.550556}%
\pgfsetfillcolor{currentfill}%
\pgfsetfillopacity{0.700000}%
\pgfsetlinewidth{0.000000pt}%
\definecolor{currentstroke}{rgb}{0.000000,0.000000,0.000000}%
\pgfsetstrokecolor{currentstroke}%
\pgfsetstrokeopacity{0.700000}%
\pgfsetdash{}{0pt}%
\pgfpathmoveto{\pgfqpoint{9.408185in}{1.668826in}}%
\pgfpathcurveto{\pgfqpoint{9.413229in}{1.668826in}}{\pgfqpoint{9.418067in}{1.670830in}}{\pgfqpoint{9.421633in}{1.674396in}}%
\pgfpathcurveto{\pgfqpoint{9.425199in}{1.677962in}}{\pgfqpoint{9.427203in}{1.682800in}}{\pgfqpoint{9.427203in}{1.687844in}}%
\pgfpathcurveto{\pgfqpoint{9.427203in}{1.692888in}}{\pgfqpoint{9.425199in}{1.697725in}}{\pgfqpoint{9.421633in}{1.701292in}}%
\pgfpathcurveto{\pgfqpoint{9.418067in}{1.704858in}}{\pgfqpoint{9.413229in}{1.706862in}}{\pgfqpoint{9.408185in}{1.706862in}}%
\pgfpathcurveto{\pgfqpoint{9.403141in}{1.706862in}}{\pgfqpoint{9.398304in}{1.704858in}}{\pgfqpoint{9.394737in}{1.701292in}}%
\pgfpathcurveto{\pgfqpoint{9.391171in}{1.697725in}}{\pgfqpoint{9.389167in}{1.692888in}}{\pgfqpoint{9.389167in}{1.687844in}}%
\pgfpathcurveto{\pgfqpoint{9.389167in}{1.682800in}}{\pgfqpoint{9.391171in}{1.677962in}}{\pgfqpoint{9.394737in}{1.674396in}}%
\pgfpathcurveto{\pgfqpoint{9.398304in}{1.670830in}}{\pgfqpoint{9.403141in}{1.668826in}}{\pgfqpoint{9.408185in}{1.668826in}}%
\pgfpathclose%
\pgfusepath{fill}%
\end{pgfscope}%
\begin{pgfscope}%
\pgfpathrectangle{\pgfqpoint{6.572727in}{0.474100in}}{\pgfqpoint{4.227273in}{3.318700in}}%
\pgfusepath{clip}%
\pgfsetbuttcap%
\pgfsetroundjoin%
\definecolor{currentfill}{rgb}{0.127568,0.566949,0.550556}%
\pgfsetfillcolor{currentfill}%
\pgfsetfillopacity{0.700000}%
\pgfsetlinewidth{0.000000pt}%
\definecolor{currentstroke}{rgb}{0.000000,0.000000,0.000000}%
\pgfsetstrokecolor{currentstroke}%
\pgfsetstrokeopacity{0.700000}%
\pgfsetdash{}{0pt}%
\pgfpathmoveto{\pgfqpoint{9.627874in}{0.937863in}}%
\pgfpathcurveto{\pgfqpoint{9.632918in}{0.937863in}}{\pgfqpoint{9.637756in}{0.939867in}}{\pgfqpoint{9.641322in}{0.943433in}}%
\pgfpathcurveto{\pgfqpoint{9.644888in}{0.947000in}}{\pgfqpoint{9.646892in}{0.951838in}}{\pgfqpoint{9.646892in}{0.956881in}}%
\pgfpathcurveto{\pgfqpoint{9.646892in}{0.961925in}}{\pgfqpoint{9.644888in}{0.966763in}}{\pgfqpoint{9.641322in}{0.970329in}}%
\pgfpathcurveto{\pgfqpoint{9.637756in}{0.973895in}}{\pgfqpoint{9.632918in}{0.975899in}}{\pgfqpoint{9.627874in}{0.975899in}}%
\pgfpathcurveto{\pgfqpoint{9.622830in}{0.975899in}}{\pgfqpoint{9.617993in}{0.973895in}}{\pgfqpoint{9.614426in}{0.970329in}}%
\pgfpathcurveto{\pgfqpoint{9.610860in}{0.966763in}}{\pgfqpoint{9.608856in}{0.961925in}}{\pgfqpoint{9.608856in}{0.956881in}}%
\pgfpathcurveto{\pgfqpoint{9.608856in}{0.951838in}}{\pgfqpoint{9.610860in}{0.947000in}}{\pgfqpoint{9.614426in}{0.943433in}}%
\pgfpathcurveto{\pgfqpoint{9.617993in}{0.939867in}}{\pgfqpoint{9.622830in}{0.937863in}}{\pgfqpoint{9.627874in}{0.937863in}}%
\pgfpathclose%
\pgfusepath{fill}%
\end{pgfscope}%
\begin{pgfscope}%
\pgfpathrectangle{\pgfqpoint{6.572727in}{0.474100in}}{\pgfqpoint{4.227273in}{3.318700in}}%
\pgfusepath{clip}%
\pgfsetbuttcap%
\pgfsetroundjoin%
\definecolor{currentfill}{rgb}{0.127568,0.566949,0.550556}%
\pgfsetfillcolor{currentfill}%
\pgfsetfillopacity{0.700000}%
\pgfsetlinewidth{0.000000pt}%
\definecolor{currentstroke}{rgb}{0.000000,0.000000,0.000000}%
\pgfsetstrokecolor{currentstroke}%
\pgfsetstrokeopacity{0.700000}%
\pgfsetdash{}{0pt}%
\pgfpathmoveto{\pgfqpoint{9.445242in}{1.557316in}}%
\pgfpathcurveto{\pgfqpoint{9.450286in}{1.557316in}}{\pgfqpoint{9.455124in}{1.559320in}}{\pgfqpoint{9.458690in}{1.562886in}}%
\pgfpathcurveto{\pgfqpoint{9.462257in}{1.566452in}}{\pgfqpoint{9.464260in}{1.571290in}}{\pgfqpoint{9.464260in}{1.576334in}}%
\pgfpathcurveto{\pgfqpoint{9.464260in}{1.581378in}}{\pgfqpoint{9.462257in}{1.586215in}}{\pgfqpoint{9.458690in}{1.589782in}}%
\pgfpathcurveto{\pgfqpoint{9.455124in}{1.593348in}}{\pgfqpoint{9.450286in}{1.595352in}}{\pgfqpoint{9.445242in}{1.595352in}}%
\pgfpathcurveto{\pgfqpoint{9.440199in}{1.595352in}}{\pgfqpoint{9.435361in}{1.593348in}}{\pgfqpoint{9.431794in}{1.589782in}}%
\pgfpathcurveto{\pgfqpoint{9.428228in}{1.586215in}}{\pgfqpoint{9.426224in}{1.581378in}}{\pgfqpoint{9.426224in}{1.576334in}}%
\pgfpathcurveto{\pgfqpoint{9.426224in}{1.571290in}}{\pgfqpoint{9.428228in}{1.566452in}}{\pgfqpoint{9.431794in}{1.562886in}}%
\pgfpathcurveto{\pgfqpoint{9.435361in}{1.559320in}}{\pgfqpoint{9.440199in}{1.557316in}}{\pgfqpoint{9.445242in}{1.557316in}}%
\pgfpathclose%
\pgfusepath{fill}%
\end{pgfscope}%
\begin{pgfscope}%
\pgfpathrectangle{\pgfqpoint{6.572727in}{0.474100in}}{\pgfqpoint{4.227273in}{3.318700in}}%
\pgfusepath{clip}%
\pgfsetbuttcap%
\pgfsetroundjoin%
\definecolor{currentfill}{rgb}{0.993248,0.906157,0.143936}%
\pgfsetfillcolor{currentfill}%
\pgfsetfillopacity{0.700000}%
\pgfsetlinewidth{0.000000pt}%
\definecolor{currentstroke}{rgb}{0.000000,0.000000,0.000000}%
\pgfsetstrokecolor{currentstroke}%
\pgfsetstrokeopacity{0.700000}%
\pgfsetdash{}{0pt}%
\pgfpathmoveto{\pgfqpoint{8.011715in}{2.692316in}}%
\pgfpathcurveto{\pgfqpoint{8.016759in}{2.692316in}}{\pgfqpoint{8.021597in}{2.694319in}}{\pgfqpoint{8.025163in}{2.697886in}}%
\pgfpathcurveto{\pgfqpoint{8.028730in}{2.701452in}}{\pgfqpoint{8.030733in}{2.706290in}}{\pgfqpoint{8.030733in}{2.711334in}}%
\pgfpathcurveto{\pgfqpoint{8.030733in}{2.716377in}}{\pgfqpoint{8.028730in}{2.721215in}}{\pgfqpoint{8.025163in}{2.724782in}}%
\pgfpathcurveto{\pgfqpoint{8.021597in}{2.728348in}}{\pgfqpoint{8.016759in}{2.730352in}}{\pgfqpoint{8.011715in}{2.730352in}}%
\pgfpathcurveto{\pgfqpoint{8.006672in}{2.730352in}}{\pgfqpoint{8.001834in}{2.728348in}}{\pgfqpoint{7.998267in}{2.724782in}}%
\pgfpathcurveto{\pgfqpoint{7.994701in}{2.721215in}}{\pgfqpoint{7.992697in}{2.716377in}}{\pgfqpoint{7.992697in}{2.711334in}}%
\pgfpathcurveto{\pgfqpoint{7.992697in}{2.706290in}}{\pgfqpoint{7.994701in}{2.701452in}}{\pgfqpoint{7.998267in}{2.697886in}}%
\pgfpathcurveto{\pgfqpoint{8.001834in}{2.694319in}}{\pgfqpoint{8.006672in}{2.692316in}}{\pgfqpoint{8.011715in}{2.692316in}}%
\pgfpathclose%
\pgfusepath{fill}%
\end{pgfscope}%
\begin{pgfscope}%
\pgfpathrectangle{\pgfqpoint{6.572727in}{0.474100in}}{\pgfqpoint{4.227273in}{3.318700in}}%
\pgfusepath{clip}%
\pgfsetbuttcap%
\pgfsetroundjoin%
\definecolor{currentfill}{rgb}{0.993248,0.906157,0.143936}%
\pgfsetfillcolor{currentfill}%
\pgfsetfillopacity{0.700000}%
\pgfsetlinewidth{0.000000pt}%
\definecolor{currentstroke}{rgb}{0.000000,0.000000,0.000000}%
\pgfsetstrokecolor{currentstroke}%
\pgfsetstrokeopacity{0.700000}%
\pgfsetdash{}{0pt}%
\pgfpathmoveto{\pgfqpoint{8.222164in}{2.777972in}}%
\pgfpathcurveto{\pgfqpoint{8.227207in}{2.777972in}}{\pgfqpoint{8.232045in}{2.779976in}}{\pgfqpoint{8.235612in}{2.783542in}}%
\pgfpathcurveto{\pgfqpoint{8.239178in}{2.787108in}}{\pgfqpoint{8.241182in}{2.791946in}}{\pgfqpoint{8.241182in}{2.796990in}}%
\pgfpathcurveto{\pgfqpoint{8.241182in}{2.802034in}}{\pgfqpoint{8.239178in}{2.806871in}}{\pgfqpoint{8.235612in}{2.810438in}}%
\pgfpathcurveto{\pgfqpoint{8.232045in}{2.814004in}}{\pgfqpoint{8.227207in}{2.816008in}}{\pgfqpoint{8.222164in}{2.816008in}}%
\pgfpathcurveto{\pgfqpoint{8.217120in}{2.816008in}}{\pgfqpoint{8.212282in}{2.814004in}}{\pgfqpoint{8.208716in}{2.810438in}}%
\pgfpathcurveto{\pgfqpoint{8.205149in}{2.806871in}}{\pgfqpoint{8.203146in}{2.802034in}}{\pgfqpoint{8.203146in}{2.796990in}}%
\pgfpathcurveto{\pgfqpoint{8.203146in}{2.791946in}}{\pgfqpoint{8.205149in}{2.787108in}}{\pgfqpoint{8.208716in}{2.783542in}}%
\pgfpathcurveto{\pgfqpoint{8.212282in}{2.779976in}}{\pgfqpoint{8.217120in}{2.777972in}}{\pgfqpoint{8.222164in}{2.777972in}}%
\pgfpathclose%
\pgfusepath{fill}%
\end{pgfscope}%
\begin{pgfscope}%
\pgfpathrectangle{\pgfqpoint{6.572727in}{0.474100in}}{\pgfqpoint{4.227273in}{3.318700in}}%
\pgfusepath{clip}%
\pgfsetbuttcap%
\pgfsetroundjoin%
\definecolor{currentfill}{rgb}{0.267004,0.004874,0.329415}%
\pgfsetfillcolor{currentfill}%
\pgfsetfillopacity{0.700000}%
\pgfsetlinewidth{0.000000pt}%
\definecolor{currentstroke}{rgb}{0.000000,0.000000,0.000000}%
\pgfsetstrokecolor{currentstroke}%
\pgfsetstrokeopacity{0.700000}%
\pgfsetdash{}{0pt}%
\pgfpathmoveto{\pgfqpoint{7.855825in}{1.851719in}}%
\pgfpathcurveto{\pgfqpoint{7.860868in}{1.851719in}}{\pgfqpoint{7.865706in}{1.853722in}}{\pgfqpoint{7.869272in}{1.857289in}}%
\pgfpathcurveto{\pgfqpoint{7.872839in}{1.860855in}}{\pgfqpoint{7.874843in}{1.865693in}}{\pgfqpoint{7.874843in}{1.870737in}}%
\pgfpathcurveto{\pgfqpoint{7.874843in}{1.875780in}}{\pgfqpoint{7.872839in}{1.880618in}}{\pgfqpoint{7.869272in}{1.884185in}}%
\pgfpathcurveto{\pgfqpoint{7.865706in}{1.887751in}}{\pgfqpoint{7.860868in}{1.889755in}}{\pgfqpoint{7.855825in}{1.889755in}}%
\pgfpathcurveto{\pgfqpoint{7.850781in}{1.889755in}}{\pgfqpoint{7.845943in}{1.887751in}}{\pgfqpoint{7.842377in}{1.884185in}}%
\pgfpathcurveto{\pgfqpoint{7.838810in}{1.880618in}}{\pgfqpoint{7.836806in}{1.875780in}}{\pgfqpoint{7.836806in}{1.870737in}}%
\pgfpathcurveto{\pgfqpoint{7.836806in}{1.865693in}}{\pgfqpoint{7.838810in}{1.860855in}}{\pgfqpoint{7.842377in}{1.857289in}}%
\pgfpathcurveto{\pgfqpoint{7.845943in}{1.853722in}}{\pgfqpoint{7.850781in}{1.851719in}}{\pgfqpoint{7.855825in}{1.851719in}}%
\pgfpathclose%
\pgfusepath{fill}%
\end{pgfscope}%
\begin{pgfscope}%
\pgfpathrectangle{\pgfqpoint{6.572727in}{0.474100in}}{\pgfqpoint{4.227273in}{3.318700in}}%
\pgfusepath{clip}%
\pgfsetbuttcap%
\pgfsetroundjoin%
\definecolor{currentfill}{rgb}{0.267004,0.004874,0.329415}%
\pgfsetfillcolor{currentfill}%
\pgfsetfillopacity{0.700000}%
\pgfsetlinewidth{0.000000pt}%
\definecolor{currentstroke}{rgb}{0.000000,0.000000,0.000000}%
\pgfsetstrokecolor{currentstroke}%
\pgfsetstrokeopacity{0.700000}%
\pgfsetdash{}{0pt}%
\pgfpathmoveto{\pgfqpoint{8.054335in}{1.534493in}}%
\pgfpathcurveto{\pgfqpoint{8.059378in}{1.534493in}}{\pgfqpoint{8.064216in}{1.536497in}}{\pgfqpoint{8.067783in}{1.540063in}}%
\pgfpathcurveto{\pgfqpoint{8.071349in}{1.543630in}}{\pgfqpoint{8.073353in}{1.548468in}}{\pgfqpoint{8.073353in}{1.553511in}}%
\pgfpathcurveto{\pgfqpoint{8.073353in}{1.558555in}}{\pgfqpoint{8.071349in}{1.563393in}}{\pgfqpoint{8.067783in}{1.566959in}}%
\pgfpathcurveto{\pgfqpoint{8.064216in}{1.570525in}}{\pgfqpoint{8.059378in}{1.572529in}}{\pgfqpoint{8.054335in}{1.572529in}}%
\pgfpathcurveto{\pgfqpoint{8.049291in}{1.572529in}}{\pgfqpoint{8.044453in}{1.570525in}}{\pgfqpoint{8.040887in}{1.566959in}}%
\pgfpathcurveto{\pgfqpoint{8.037321in}{1.563393in}}{\pgfqpoint{8.035317in}{1.558555in}}{\pgfqpoint{8.035317in}{1.553511in}}%
\pgfpathcurveto{\pgfqpoint{8.035317in}{1.548468in}}{\pgfqpoint{8.037321in}{1.543630in}}{\pgfqpoint{8.040887in}{1.540063in}}%
\pgfpathcurveto{\pgfqpoint{8.044453in}{1.536497in}}{\pgfqpoint{8.049291in}{1.534493in}}{\pgfqpoint{8.054335in}{1.534493in}}%
\pgfpathclose%
\pgfusepath{fill}%
\end{pgfscope}%
\begin{pgfscope}%
\pgfpathrectangle{\pgfqpoint{6.572727in}{0.474100in}}{\pgfqpoint{4.227273in}{3.318700in}}%
\pgfusepath{clip}%
\pgfsetbuttcap%
\pgfsetroundjoin%
\definecolor{currentfill}{rgb}{0.993248,0.906157,0.143936}%
\pgfsetfillcolor{currentfill}%
\pgfsetfillopacity{0.700000}%
\pgfsetlinewidth{0.000000pt}%
\definecolor{currentstroke}{rgb}{0.000000,0.000000,0.000000}%
\pgfsetstrokecolor{currentstroke}%
\pgfsetstrokeopacity{0.700000}%
\pgfsetdash{}{0pt}%
\pgfpathmoveto{\pgfqpoint{8.437098in}{2.756430in}}%
\pgfpathcurveto{\pgfqpoint{8.442142in}{2.756430in}}{\pgfqpoint{8.446979in}{2.758433in}}{\pgfqpoint{8.450546in}{2.762000in}}%
\pgfpathcurveto{\pgfqpoint{8.454112in}{2.765566in}}{\pgfqpoint{8.456116in}{2.770404in}}{\pgfqpoint{8.456116in}{2.775448in}}%
\pgfpathcurveto{\pgfqpoint{8.456116in}{2.780491in}}{\pgfqpoint{8.454112in}{2.785329in}}{\pgfqpoint{8.450546in}{2.788896in}}%
\pgfpathcurveto{\pgfqpoint{8.446979in}{2.792462in}}{\pgfqpoint{8.442142in}{2.794466in}}{\pgfqpoint{8.437098in}{2.794466in}}%
\pgfpathcurveto{\pgfqpoint{8.432054in}{2.794466in}}{\pgfqpoint{8.427217in}{2.792462in}}{\pgfqpoint{8.423650in}{2.788896in}}%
\pgfpathcurveto{\pgfqpoint{8.420084in}{2.785329in}}{\pgfqpoint{8.418080in}{2.780491in}}{\pgfqpoint{8.418080in}{2.775448in}}%
\pgfpathcurveto{\pgfqpoint{8.418080in}{2.770404in}}{\pgfqpoint{8.420084in}{2.765566in}}{\pgfqpoint{8.423650in}{2.762000in}}%
\pgfpathcurveto{\pgfqpoint{8.427217in}{2.758433in}}{\pgfqpoint{8.432054in}{2.756430in}}{\pgfqpoint{8.437098in}{2.756430in}}%
\pgfpathclose%
\pgfusepath{fill}%
\end{pgfscope}%
\begin{pgfscope}%
\pgfpathrectangle{\pgfqpoint{6.572727in}{0.474100in}}{\pgfqpoint{4.227273in}{3.318700in}}%
\pgfusepath{clip}%
\pgfsetbuttcap%
\pgfsetroundjoin%
\definecolor{currentfill}{rgb}{0.993248,0.906157,0.143936}%
\pgfsetfillcolor{currentfill}%
\pgfsetfillopacity{0.700000}%
\pgfsetlinewidth{0.000000pt}%
\definecolor{currentstroke}{rgb}{0.000000,0.000000,0.000000}%
\pgfsetstrokecolor{currentstroke}%
\pgfsetstrokeopacity{0.700000}%
\pgfsetdash{}{0pt}%
\pgfpathmoveto{\pgfqpoint{7.403810in}{2.898286in}}%
\pgfpathcurveto{\pgfqpoint{7.408854in}{2.898286in}}{\pgfqpoint{7.413692in}{2.900290in}}{\pgfqpoint{7.417258in}{2.903856in}}%
\pgfpathcurveto{\pgfqpoint{7.420825in}{2.907423in}}{\pgfqpoint{7.422829in}{2.912261in}}{\pgfqpoint{7.422829in}{2.917304in}}%
\pgfpathcurveto{\pgfqpoint{7.422829in}{2.922348in}}{\pgfqpoint{7.420825in}{2.927186in}}{\pgfqpoint{7.417258in}{2.930752in}}%
\pgfpathcurveto{\pgfqpoint{7.413692in}{2.934318in}}{\pgfqpoint{7.408854in}{2.936322in}}{\pgfqpoint{7.403810in}{2.936322in}}%
\pgfpathcurveto{\pgfqpoint{7.398767in}{2.936322in}}{\pgfqpoint{7.393929in}{2.934318in}}{\pgfqpoint{7.390363in}{2.930752in}}%
\pgfpathcurveto{\pgfqpoint{7.386796in}{2.927186in}}{\pgfqpoint{7.384792in}{2.922348in}}{\pgfqpoint{7.384792in}{2.917304in}}%
\pgfpathcurveto{\pgfqpoint{7.384792in}{2.912261in}}{\pgfqpoint{7.386796in}{2.907423in}}{\pgfqpoint{7.390363in}{2.903856in}}%
\pgfpathcurveto{\pgfqpoint{7.393929in}{2.900290in}}{\pgfqpoint{7.398767in}{2.898286in}}{\pgfqpoint{7.403810in}{2.898286in}}%
\pgfpathclose%
\pgfusepath{fill}%
\end{pgfscope}%
\begin{pgfscope}%
\pgfpathrectangle{\pgfqpoint{6.572727in}{0.474100in}}{\pgfqpoint{4.227273in}{3.318700in}}%
\pgfusepath{clip}%
\pgfsetbuttcap%
\pgfsetroundjoin%
\definecolor{currentfill}{rgb}{0.993248,0.906157,0.143936}%
\pgfsetfillcolor{currentfill}%
\pgfsetfillopacity{0.700000}%
\pgfsetlinewidth{0.000000pt}%
\definecolor{currentstroke}{rgb}{0.000000,0.000000,0.000000}%
\pgfsetstrokecolor{currentstroke}%
\pgfsetstrokeopacity{0.700000}%
\pgfsetdash{}{0pt}%
\pgfpathmoveto{\pgfqpoint{7.807939in}{2.644002in}}%
\pgfpathcurveto{\pgfqpoint{7.812983in}{2.644002in}}{\pgfqpoint{7.817821in}{2.646006in}}{\pgfqpoint{7.821387in}{2.649573in}}%
\pgfpathcurveto{\pgfqpoint{7.824954in}{2.653139in}}{\pgfqpoint{7.826958in}{2.657977in}}{\pgfqpoint{7.826958in}{2.663020in}}%
\pgfpathcurveto{\pgfqpoint{7.826958in}{2.668064in}}{\pgfqpoint{7.824954in}{2.672902in}}{\pgfqpoint{7.821387in}{2.676468in}}%
\pgfpathcurveto{\pgfqpoint{7.817821in}{2.680035in}}{\pgfqpoint{7.812983in}{2.682039in}}{\pgfqpoint{7.807939in}{2.682039in}}%
\pgfpathcurveto{\pgfqpoint{7.802896in}{2.682039in}}{\pgfqpoint{7.798058in}{2.680035in}}{\pgfqpoint{7.794492in}{2.676468in}}%
\pgfpathcurveto{\pgfqpoint{7.790925in}{2.672902in}}{\pgfqpoint{7.788921in}{2.668064in}}{\pgfqpoint{7.788921in}{2.663020in}}%
\pgfpathcurveto{\pgfqpoint{7.788921in}{2.657977in}}{\pgfqpoint{7.790925in}{2.653139in}}{\pgfqpoint{7.794492in}{2.649573in}}%
\pgfpathcurveto{\pgfqpoint{7.798058in}{2.646006in}}{\pgfqpoint{7.802896in}{2.644002in}}{\pgfqpoint{7.807939in}{2.644002in}}%
\pgfpathclose%
\pgfusepath{fill}%
\end{pgfscope}%
\begin{pgfscope}%
\pgfpathrectangle{\pgfqpoint{6.572727in}{0.474100in}}{\pgfqpoint{4.227273in}{3.318700in}}%
\pgfusepath{clip}%
\pgfsetbuttcap%
\pgfsetroundjoin%
\definecolor{currentfill}{rgb}{0.267004,0.004874,0.329415}%
\pgfsetfillcolor{currentfill}%
\pgfsetfillopacity{0.700000}%
\pgfsetlinewidth{0.000000pt}%
\definecolor{currentstroke}{rgb}{0.000000,0.000000,0.000000}%
\pgfsetstrokecolor{currentstroke}%
\pgfsetstrokeopacity{0.700000}%
\pgfsetdash{}{0pt}%
\pgfpathmoveto{\pgfqpoint{8.068671in}{1.512713in}}%
\pgfpathcurveto{\pgfqpoint{8.073715in}{1.512713in}}{\pgfqpoint{8.078553in}{1.514717in}}{\pgfqpoint{8.082119in}{1.518283in}}%
\pgfpathcurveto{\pgfqpoint{8.085686in}{1.521849in}}{\pgfqpoint{8.087689in}{1.526687in}}{\pgfqpoint{8.087689in}{1.531731in}}%
\pgfpathcurveto{\pgfqpoint{8.087689in}{1.536774in}}{\pgfqpoint{8.085686in}{1.541612in}}{\pgfqpoint{8.082119in}{1.545179in}}%
\pgfpathcurveto{\pgfqpoint{8.078553in}{1.548745in}}{\pgfqpoint{8.073715in}{1.550749in}}{\pgfqpoint{8.068671in}{1.550749in}}%
\pgfpathcurveto{\pgfqpoint{8.063628in}{1.550749in}}{\pgfqpoint{8.058790in}{1.548745in}}{\pgfqpoint{8.055223in}{1.545179in}}%
\pgfpathcurveto{\pgfqpoint{8.051657in}{1.541612in}}{\pgfqpoint{8.049653in}{1.536774in}}{\pgfqpoint{8.049653in}{1.531731in}}%
\pgfpathcurveto{\pgfqpoint{8.049653in}{1.526687in}}{\pgfqpoint{8.051657in}{1.521849in}}{\pgfqpoint{8.055223in}{1.518283in}}%
\pgfpathcurveto{\pgfqpoint{8.058790in}{1.514717in}}{\pgfqpoint{8.063628in}{1.512713in}}{\pgfqpoint{8.068671in}{1.512713in}}%
\pgfpathclose%
\pgfusepath{fill}%
\end{pgfscope}%
\begin{pgfscope}%
\pgfpathrectangle{\pgfqpoint{6.572727in}{0.474100in}}{\pgfqpoint{4.227273in}{3.318700in}}%
\pgfusepath{clip}%
\pgfsetbuttcap%
\pgfsetroundjoin%
\definecolor{currentfill}{rgb}{0.127568,0.566949,0.550556}%
\pgfsetfillcolor{currentfill}%
\pgfsetfillopacity{0.700000}%
\pgfsetlinewidth{0.000000pt}%
\definecolor{currentstroke}{rgb}{0.000000,0.000000,0.000000}%
\pgfsetstrokecolor{currentstroke}%
\pgfsetstrokeopacity{0.700000}%
\pgfsetdash{}{0pt}%
\pgfpathmoveto{\pgfqpoint{9.265946in}{2.034746in}}%
\pgfpathcurveto{\pgfqpoint{9.270990in}{2.034746in}}{\pgfqpoint{9.275827in}{2.036749in}}{\pgfqpoint{9.279394in}{2.040316in}}%
\pgfpathcurveto{\pgfqpoint{9.282960in}{2.043882in}}{\pgfqpoint{9.284964in}{2.048720in}}{\pgfqpoint{9.284964in}{2.053764in}}%
\pgfpathcurveto{\pgfqpoint{9.284964in}{2.058807in}}{\pgfqpoint{9.282960in}{2.063645in}}{\pgfqpoint{9.279394in}{2.067212in}}%
\pgfpathcurveto{\pgfqpoint{9.275827in}{2.070778in}}{\pgfqpoint{9.270990in}{2.072782in}}{\pgfqpoint{9.265946in}{2.072782in}}%
\pgfpathcurveto{\pgfqpoint{9.260902in}{2.072782in}}{\pgfqpoint{9.256064in}{2.070778in}}{\pgfqpoint{9.252498in}{2.067212in}}%
\pgfpathcurveto{\pgfqpoint{9.248932in}{2.063645in}}{\pgfqpoint{9.246928in}{2.058807in}}{\pgfqpoint{9.246928in}{2.053764in}}%
\pgfpathcurveto{\pgfqpoint{9.246928in}{2.048720in}}{\pgfqpoint{9.248932in}{2.043882in}}{\pgfqpoint{9.252498in}{2.040316in}}%
\pgfpathcurveto{\pgfqpoint{9.256064in}{2.036749in}}{\pgfqpoint{9.260902in}{2.034746in}}{\pgfqpoint{9.265946in}{2.034746in}}%
\pgfpathclose%
\pgfusepath{fill}%
\end{pgfscope}%
\begin{pgfscope}%
\pgfpathrectangle{\pgfqpoint{6.572727in}{0.474100in}}{\pgfqpoint{4.227273in}{3.318700in}}%
\pgfusepath{clip}%
\pgfsetbuttcap%
\pgfsetroundjoin%
\definecolor{currentfill}{rgb}{0.267004,0.004874,0.329415}%
\pgfsetfillcolor{currentfill}%
\pgfsetfillopacity{0.700000}%
\pgfsetlinewidth{0.000000pt}%
\definecolor{currentstroke}{rgb}{0.000000,0.000000,0.000000}%
\pgfsetstrokecolor{currentstroke}%
\pgfsetstrokeopacity{0.700000}%
\pgfsetdash{}{0pt}%
\pgfpathmoveto{\pgfqpoint{7.660278in}{1.132869in}}%
\pgfpathcurveto{\pgfqpoint{7.665322in}{1.132869in}}{\pgfqpoint{7.670159in}{1.134873in}}{\pgfqpoint{7.673726in}{1.138439in}}%
\pgfpathcurveto{\pgfqpoint{7.677292in}{1.142006in}}{\pgfqpoint{7.679296in}{1.146844in}}{\pgfqpoint{7.679296in}{1.151887in}}%
\pgfpathcurveto{\pgfqpoint{7.679296in}{1.156931in}}{\pgfqpoint{7.677292in}{1.161769in}}{\pgfqpoint{7.673726in}{1.165335in}}%
\pgfpathcurveto{\pgfqpoint{7.670159in}{1.168901in}}{\pgfqpoint{7.665322in}{1.170905in}}{\pgfqpoint{7.660278in}{1.170905in}}%
\pgfpathcurveto{\pgfqpoint{7.655234in}{1.170905in}}{\pgfqpoint{7.650396in}{1.168901in}}{\pgfqpoint{7.646830in}{1.165335in}}%
\pgfpathcurveto{\pgfqpoint{7.643264in}{1.161769in}}{\pgfqpoint{7.641260in}{1.156931in}}{\pgfqpoint{7.641260in}{1.151887in}}%
\pgfpathcurveto{\pgfqpoint{7.641260in}{1.146844in}}{\pgfqpoint{7.643264in}{1.142006in}}{\pgfqpoint{7.646830in}{1.138439in}}%
\pgfpathcurveto{\pgfqpoint{7.650396in}{1.134873in}}{\pgfqpoint{7.655234in}{1.132869in}}{\pgfqpoint{7.660278in}{1.132869in}}%
\pgfpathclose%
\pgfusepath{fill}%
\end{pgfscope}%
\begin{pgfscope}%
\pgfpathrectangle{\pgfqpoint{6.572727in}{0.474100in}}{\pgfqpoint{4.227273in}{3.318700in}}%
\pgfusepath{clip}%
\pgfsetbuttcap%
\pgfsetroundjoin%
\definecolor{currentfill}{rgb}{0.127568,0.566949,0.550556}%
\pgfsetfillcolor{currentfill}%
\pgfsetfillopacity{0.700000}%
\pgfsetlinewidth{0.000000pt}%
\definecolor{currentstroke}{rgb}{0.000000,0.000000,0.000000}%
\pgfsetstrokecolor{currentstroke}%
\pgfsetstrokeopacity{0.700000}%
\pgfsetdash{}{0pt}%
\pgfpathmoveto{\pgfqpoint{9.000927in}{1.801028in}}%
\pgfpathcurveto{\pgfqpoint{9.005970in}{1.801028in}}{\pgfqpoint{9.010808in}{1.803032in}}{\pgfqpoint{9.014375in}{1.806598in}}%
\pgfpathcurveto{\pgfqpoint{9.017941in}{1.810165in}}{\pgfqpoint{9.019945in}{1.815002in}}{\pgfqpoint{9.019945in}{1.820046in}}%
\pgfpathcurveto{\pgfqpoint{9.019945in}{1.825090in}}{\pgfqpoint{9.017941in}{1.829927in}}{\pgfqpoint{9.014375in}{1.833494in}}%
\pgfpathcurveto{\pgfqpoint{9.010808in}{1.837060in}}{\pgfqpoint{9.005970in}{1.839064in}}{\pgfqpoint{9.000927in}{1.839064in}}%
\pgfpathcurveto{\pgfqpoint{8.995883in}{1.839064in}}{\pgfqpoint{8.991045in}{1.837060in}}{\pgfqpoint{8.987479in}{1.833494in}}%
\pgfpathcurveto{\pgfqpoint{8.983912in}{1.829927in}}{\pgfqpoint{8.981908in}{1.825090in}}{\pgfqpoint{8.981908in}{1.820046in}}%
\pgfpathcurveto{\pgfqpoint{8.981908in}{1.815002in}}{\pgfqpoint{8.983912in}{1.810165in}}{\pgfqpoint{8.987479in}{1.806598in}}%
\pgfpathcurveto{\pgfqpoint{8.991045in}{1.803032in}}{\pgfqpoint{8.995883in}{1.801028in}}{\pgfqpoint{9.000927in}{1.801028in}}%
\pgfpathclose%
\pgfusepath{fill}%
\end{pgfscope}%
\begin{pgfscope}%
\pgfpathrectangle{\pgfqpoint{6.572727in}{0.474100in}}{\pgfqpoint{4.227273in}{3.318700in}}%
\pgfusepath{clip}%
\pgfsetbuttcap%
\pgfsetroundjoin%
\definecolor{currentfill}{rgb}{0.993248,0.906157,0.143936}%
\pgfsetfillcolor{currentfill}%
\pgfsetfillopacity{0.700000}%
\pgfsetlinewidth{0.000000pt}%
\definecolor{currentstroke}{rgb}{0.000000,0.000000,0.000000}%
\pgfsetstrokecolor{currentstroke}%
\pgfsetstrokeopacity{0.700000}%
\pgfsetdash{}{0pt}%
\pgfpathmoveto{\pgfqpoint{8.412896in}{2.490488in}}%
\pgfpathcurveto{\pgfqpoint{8.417940in}{2.490488in}}{\pgfqpoint{8.422778in}{2.492492in}}{\pgfqpoint{8.426344in}{2.496058in}}%
\pgfpathcurveto{\pgfqpoint{8.429910in}{2.499625in}}{\pgfqpoint{8.431914in}{2.504462in}}{\pgfqpoint{8.431914in}{2.509506in}}%
\pgfpathcurveto{\pgfqpoint{8.431914in}{2.514550in}}{\pgfqpoint{8.429910in}{2.519388in}}{\pgfqpoint{8.426344in}{2.522954in}}%
\pgfpathcurveto{\pgfqpoint{8.422778in}{2.526520in}}{\pgfqpoint{8.417940in}{2.528524in}}{\pgfqpoint{8.412896in}{2.528524in}}%
\pgfpathcurveto{\pgfqpoint{8.407852in}{2.528524in}}{\pgfqpoint{8.403015in}{2.526520in}}{\pgfqpoint{8.399448in}{2.522954in}}%
\pgfpathcurveto{\pgfqpoint{8.395882in}{2.519388in}}{\pgfqpoint{8.393878in}{2.514550in}}{\pgfqpoint{8.393878in}{2.509506in}}%
\pgfpathcurveto{\pgfqpoint{8.393878in}{2.504462in}}{\pgfqpoint{8.395882in}{2.499625in}}{\pgfqpoint{8.399448in}{2.496058in}}%
\pgfpathcurveto{\pgfqpoint{8.403015in}{2.492492in}}{\pgfqpoint{8.407852in}{2.490488in}}{\pgfqpoint{8.412896in}{2.490488in}}%
\pgfpathclose%
\pgfusepath{fill}%
\end{pgfscope}%
\begin{pgfscope}%
\pgfpathrectangle{\pgfqpoint{6.572727in}{0.474100in}}{\pgfqpoint{4.227273in}{3.318700in}}%
\pgfusepath{clip}%
\pgfsetbuttcap%
\pgfsetroundjoin%
\definecolor{currentfill}{rgb}{0.267004,0.004874,0.329415}%
\pgfsetfillcolor{currentfill}%
\pgfsetfillopacity{0.700000}%
\pgfsetlinewidth{0.000000pt}%
\definecolor{currentstroke}{rgb}{0.000000,0.000000,0.000000}%
\pgfsetstrokecolor{currentstroke}%
\pgfsetstrokeopacity{0.700000}%
\pgfsetdash{}{0pt}%
\pgfpathmoveto{\pgfqpoint{8.091446in}{1.666130in}}%
\pgfpathcurveto{\pgfqpoint{8.096490in}{1.666130in}}{\pgfqpoint{8.101328in}{1.668134in}}{\pgfqpoint{8.104894in}{1.671700in}}%
\pgfpathcurveto{\pgfqpoint{8.108461in}{1.675267in}}{\pgfqpoint{8.110464in}{1.680104in}}{\pgfqpoint{8.110464in}{1.685148in}}%
\pgfpathcurveto{\pgfqpoint{8.110464in}{1.690192in}}{\pgfqpoint{8.108461in}{1.695029in}}{\pgfqpoint{8.104894in}{1.698596in}}%
\pgfpathcurveto{\pgfqpoint{8.101328in}{1.702162in}}{\pgfqpoint{8.096490in}{1.704166in}}{\pgfqpoint{8.091446in}{1.704166in}}%
\pgfpathcurveto{\pgfqpoint{8.086403in}{1.704166in}}{\pgfqpoint{8.081565in}{1.702162in}}{\pgfqpoint{8.077998in}{1.698596in}}%
\pgfpathcurveto{\pgfqpoint{8.074432in}{1.695029in}}{\pgfqpoint{8.072428in}{1.690192in}}{\pgfqpoint{8.072428in}{1.685148in}}%
\pgfpathcurveto{\pgfqpoint{8.072428in}{1.680104in}}{\pgfqpoint{8.074432in}{1.675267in}}{\pgfqpoint{8.077998in}{1.671700in}}%
\pgfpathcurveto{\pgfqpoint{8.081565in}{1.668134in}}{\pgfqpoint{8.086403in}{1.666130in}}{\pgfqpoint{8.091446in}{1.666130in}}%
\pgfpathclose%
\pgfusepath{fill}%
\end{pgfscope}%
\begin{pgfscope}%
\pgfpathrectangle{\pgfqpoint{6.572727in}{0.474100in}}{\pgfqpoint{4.227273in}{3.318700in}}%
\pgfusepath{clip}%
\pgfsetbuttcap%
\pgfsetroundjoin%
\definecolor{currentfill}{rgb}{0.993248,0.906157,0.143936}%
\pgfsetfillcolor{currentfill}%
\pgfsetfillopacity{0.700000}%
\pgfsetlinewidth{0.000000pt}%
\definecolor{currentstroke}{rgb}{0.000000,0.000000,0.000000}%
\pgfsetstrokecolor{currentstroke}%
\pgfsetstrokeopacity{0.700000}%
\pgfsetdash{}{0pt}%
\pgfpathmoveto{\pgfqpoint{8.080122in}{2.596210in}}%
\pgfpathcurveto{\pgfqpoint{8.085166in}{2.596210in}}{\pgfqpoint{8.090004in}{2.598214in}}{\pgfqpoint{8.093570in}{2.601780in}}%
\pgfpathcurveto{\pgfqpoint{8.097137in}{2.605347in}}{\pgfqpoint{8.099140in}{2.610184in}}{\pgfqpoint{8.099140in}{2.615228in}}%
\pgfpathcurveto{\pgfqpoint{8.099140in}{2.620272in}}{\pgfqpoint{8.097137in}{2.625110in}}{\pgfqpoint{8.093570in}{2.628676in}}%
\pgfpathcurveto{\pgfqpoint{8.090004in}{2.632242in}}{\pgfqpoint{8.085166in}{2.634246in}}{\pgfqpoint{8.080122in}{2.634246in}}%
\pgfpathcurveto{\pgfqpoint{8.075079in}{2.634246in}}{\pgfqpoint{8.070241in}{2.632242in}}{\pgfqpoint{8.066674in}{2.628676in}}%
\pgfpathcurveto{\pgfqpoint{8.063108in}{2.625110in}}{\pgfqpoint{8.061104in}{2.620272in}}{\pgfqpoint{8.061104in}{2.615228in}}%
\pgfpathcurveto{\pgfqpoint{8.061104in}{2.610184in}}{\pgfqpoint{8.063108in}{2.605347in}}{\pgfqpoint{8.066674in}{2.601780in}}%
\pgfpathcurveto{\pgfqpoint{8.070241in}{2.598214in}}{\pgfqpoint{8.075079in}{2.596210in}}{\pgfqpoint{8.080122in}{2.596210in}}%
\pgfpathclose%
\pgfusepath{fill}%
\end{pgfscope}%
\begin{pgfscope}%
\pgfpathrectangle{\pgfqpoint{6.572727in}{0.474100in}}{\pgfqpoint{4.227273in}{3.318700in}}%
\pgfusepath{clip}%
\pgfsetbuttcap%
\pgfsetroundjoin%
\definecolor{currentfill}{rgb}{0.267004,0.004874,0.329415}%
\pgfsetfillcolor{currentfill}%
\pgfsetfillopacity{0.700000}%
\pgfsetlinewidth{0.000000pt}%
\definecolor{currentstroke}{rgb}{0.000000,0.000000,0.000000}%
\pgfsetstrokecolor{currentstroke}%
\pgfsetstrokeopacity{0.700000}%
\pgfsetdash{}{0pt}%
\pgfpathmoveto{\pgfqpoint{7.656938in}{2.234992in}}%
\pgfpathcurveto{\pgfqpoint{7.661981in}{2.234992in}}{\pgfqpoint{7.666819in}{2.236996in}}{\pgfqpoint{7.670386in}{2.240562in}}%
\pgfpathcurveto{\pgfqpoint{7.673952in}{2.244129in}}{\pgfqpoint{7.675956in}{2.248966in}}{\pgfqpoint{7.675956in}{2.254010in}}%
\pgfpathcurveto{\pgfqpoint{7.675956in}{2.259054in}}{\pgfqpoint{7.673952in}{2.263891in}}{\pgfqpoint{7.670386in}{2.267458in}}%
\pgfpathcurveto{\pgfqpoint{7.666819in}{2.271024in}}{\pgfqpoint{7.661981in}{2.273028in}}{\pgfqpoint{7.656938in}{2.273028in}}%
\pgfpathcurveto{\pgfqpoint{7.651894in}{2.273028in}}{\pgfqpoint{7.647056in}{2.271024in}}{\pgfqpoint{7.643490in}{2.267458in}}%
\pgfpathcurveto{\pgfqpoint{7.639923in}{2.263891in}}{\pgfqpoint{7.637920in}{2.259054in}}{\pgfqpoint{7.637920in}{2.254010in}}%
\pgfpathcurveto{\pgfqpoint{7.637920in}{2.248966in}}{\pgfqpoint{7.639923in}{2.244129in}}{\pgfqpoint{7.643490in}{2.240562in}}%
\pgfpathcurveto{\pgfqpoint{7.647056in}{2.236996in}}{\pgfqpoint{7.651894in}{2.234992in}}{\pgfqpoint{7.656938in}{2.234992in}}%
\pgfpathclose%
\pgfusepath{fill}%
\end{pgfscope}%
\begin{pgfscope}%
\pgfpathrectangle{\pgfqpoint{6.572727in}{0.474100in}}{\pgfqpoint{4.227273in}{3.318700in}}%
\pgfusepath{clip}%
\pgfsetbuttcap%
\pgfsetroundjoin%
\definecolor{currentfill}{rgb}{0.993248,0.906157,0.143936}%
\pgfsetfillcolor{currentfill}%
\pgfsetfillopacity{0.700000}%
\pgfsetlinewidth{0.000000pt}%
\definecolor{currentstroke}{rgb}{0.000000,0.000000,0.000000}%
\pgfsetstrokecolor{currentstroke}%
\pgfsetstrokeopacity{0.700000}%
\pgfsetdash{}{0pt}%
\pgfpathmoveto{\pgfqpoint{7.769723in}{2.903506in}}%
\pgfpathcurveto{\pgfqpoint{7.774767in}{2.903506in}}{\pgfqpoint{7.779605in}{2.905510in}}{\pgfqpoint{7.783171in}{2.909076in}}%
\pgfpathcurveto{\pgfqpoint{7.786738in}{2.912643in}}{\pgfqpoint{7.788742in}{2.917480in}}{\pgfqpoint{7.788742in}{2.922524in}}%
\pgfpathcurveto{\pgfqpoint{7.788742in}{2.927568in}}{\pgfqpoint{7.786738in}{2.932406in}}{\pgfqpoint{7.783171in}{2.935972in}}%
\pgfpathcurveto{\pgfqpoint{7.779605in}{2.939538in}}{\pgfqpoint{7.774767in}{2.941542in}}{\pgfqpoint{7.769723in}{2.941542in}}%
\pgfpathcurveto{\pgfqpoint{7.764680in}{2.941542in}}{\pgfqpoint{7.759842in}{2.939538in}}{\pgfqpoint{7.756276in}{2.935972in}}%
\pgfpathcurveto{\pgfqpoint{7.752709in}{2.932406in}}{\pgfqpoint{7.750705in}{2.927568in}}{\pgfqpoint{7.750705in}{2.922524in}}%
\pgfpathcurveto{\pgfqpoint{7.750705in}{2.917480in}}{\pgfqpoint{7.752709in}{2.912643in}}{\pgfqpoint{7.756276in}{2.909076in}}%
\pgfpathcurveto{\pgfqpoint{7.759842in}{2.905510in}}{\pgfqpoint{7.764680in}{2.903506in}}{\pgfqpoint{7.769723in}{2.903506in}}%
\pgfpathclose%
\pgfusepath{fill}%
\end{pgfscope}%
\begin{pgfscope}%
\pgfpathrectangle{\pgfqpoint{6.572727in}{0.474100in}}{\pgfqpoint{4.227273in}{3.318700in}}%
\pgfusepath{clip}%
\pgfsetbuttcap%
\pgfsetroundjoin%
\definecolor{currentfill}{rgb}{0.993248,0.906157,0.143936}%
\pgfsetfillcolor{currentfill}%
\pgfsetfillopacity{0.700000}%
\pgfsetlinewidth{0.000000pt}%
\definecolor{currentstroke}{rgb}{0.000000,0.000000,0.000000}%
\pgfsetstrokecolor{currentstroke}%
\pgfsetstrokeopacity{0.700000}%
\pgfsetdash{}{0pt}%
\pgfpathmoveto{\pgfqpoint{8.519379in}{2.415262in}}%
\pgfpathcurveto{\pgfqpoint{8.524423in}{2.415262in}}{\pgfqpoint{8.529260in}{2.417265in}}{\pgfqpoint{8.532827in}{2.420832in}}%
\pgfpathcurveto{\pgfqpoint{8.536393in}{2.424398in}}{\pgfqpoint{8.538397in}{2.429236in}}{\pgfqpoint{8.538397in}{2.434280in}}%
\pgfpathcurveto{\pgfqpoint{8.538397in}{2.439323in}}{\pgfqpoint{8.536393in}{2.444161in}}{\pgfqpoint{8.532827in}{2.447728in}}%
\pgfpathcurveto{\pgfqpoint{8.529260in}{2.451294in}}{\pgfqpoint{8.524423in}{2.453298in}}{\pgfqpoint{8.519379in}{2.453298in}}%
\pgfpathcurveto{\pgfqpoint{8.514335in}{2.453298in}}{\pgfqpoint{8.509497in}{2.451294in}}{\pgfqpoint{8.505931in}{2.447728in}}%
\pgfpathcurveto{\pgfqpoint{8.502365in}{2.444161in}}{\pgfqpoint{8.500361in}{2.439323in}}{\pgfqpoint{8.500361in}{2.434280in}}%
\pgfpathcurveto{\pgfqpoint{8.500361in}{2.429236in}}{\pgfqpoint{8.502365in}{2.424398in}}{\pgfqpoint{8.505931in}{2.420832in}}%
\pgfpathcurveto{\pgfqpoint{8.509497in}{2.417265in}}{\pgfqpoint{8.514335in}{2.415262in}}{\pgfqpoint{8.519379in}{2.415262in}}%
\pgfpathclose%
\pgfusepath{fill}%
\end{pgfscope}%
\begin{pgfscope}%
\pgfpathrectangle{\pgfqpoint{6.572727in}{0.474100in}}{\pgfqpoint{4.227273in}{3.318700in}}%
\pgfusepath{clip}%
\pgfsetbuttcap%
\pgfsetroundjoin%
\definecolor{currentfill}{rgb}{0.267004,0.004874,0.329415}%
\pgfsetfillcolor{currentfill}%
\pgfsetfillopacity{0.700000}%
\pgfsetlinewidth{0.000000pt}%
\definecolor{currentstroke}{rgb}{0.000000,0.000000,0.000000}%
\pgfsetstrokecolor{currentstroke}%
\pgfsetstrokeopacity{0.700000}%
\pgfsetdash{}{0pt}%
\pgfpathmoveto{\pgfqpoint{8.308376in}{1.689275in}}%
\pgfpathcurveto{\pgfqpoint{8.313419in}{1.689275in}}{\pgfqpoint{8.318257in}{1.691279in}}{\pgfqpoint{8.321824in}{1.694845in}}%
\pgfpathcurveto{\pgfqpoint{8.325390in}{1.698411in}}{\pgfqpoint{8.327394in}{1.703249in}}{\pgfqpoint{8.327394in}{1.708293in}}%
\pgfpathcurveto{\pgfqpoint{8.327394in}{1.713336in}}{\pgfqpoint{8.325390in}{1.718174in}}{\pgfqpoint{8.321824in}{1.721741in}}%
\pgfpathcurveto{\pgfqpoint{8.318257in}{1.725307in}}{\pgfqpoint{8.313419in}{1.727311in}}{\pgfqpoint{8.308376in}{1.727311in}}%
\pgfpathcurveto{\pgfqpoint{8.303332in}{1.727311in}}{\pgfqpoint{8.298494in}{1.725307in}}{\pgfqpoint{8.294928in}{1.721741in}}%
\pgfpathcurveto{\pgfqpoint{8.291362in}{1.718174in}}{\pgfqpoint{8.289358in}{1.713336in}}{\pgfqpoint{8.289358in}{1.708293in}}%
\pgfpathcurveto{\pgfqpoint{8.289358in}{1.703249in}}{\pgfqpoint{8.291362in}{1.698411in}}{\pgfqpoint{8.294928in}{1.694845in}}%
\pgfpathcurveto{\pgfqpoint{8.298494in}{1.691279in}}{\pgfqpoint{8.303332in}{1.689275in}}{\pgfqpoint{8.308376in}{1.689275in}}%
\pgfpathclose%
\pgfusepath{fill}%
\end{pgfscope}%
\begin{pgfscope}%
\pgfpathrectangle{\pgfqpoint{6.572727in}{0.474100in}}{\pgfqpoint{4.227273in}{3.318700in}}%
\pgfusepath{clip}%
\pgfsetbuttcap%
\pgfsetroundjoin%
\definecolor{currentfill}{rgb}{0.127568,0.566949,0.550556}%
\pgfsetfillcolor{currentfill}%
\pgfsetfillopacity{0.700000}%
\pgfsetlinewidth{0.000000pt}%
\definecolor{currentstroke}{rgb}{0.000000,0.000000,0.000000}%
\pgfsetstrokecolor{currentstroke}%
\pgfsetstrokeopacity{0.700000}%
\pgfsetdash{}{0pt}%
\pgfpathmoveto{\pgfqpoint{8.950710in}{1.619999in}}%
\pgfpathcurveto{\pgfqpoint{8.955754in}{1.619999in}}{\pgfqpoint{8.960592in}{1.622003in}}{\pgfqpoint{8.964158in}{1.625569in}}%
\pgfpathcurveto{\pgfqpoint{8.967724in}{1.629136in}}{\pgfqpoint{8.969728in}{1.633974in}}{\pgfqpoint{8.969728in}{1.639017in}}%
\pgfpathcurveto{\pgfqpoint{8.969728in}{1.644061in}}{\pgfqpoint{8.967724in}{1.648899in}}{\pgfqpoint{8.964158in}{1.652465in}}%
\pgfpathcurveto{\pgfqpoint{8.960592in}{1.656032in}}{\pgfqpoint{8.955754in}{1.658035in}}{\pgfqpoint{8.950710in}{1.658035in}}%
\pgfpathcurveto{\pgfqpoint{8.945666in}{1.658035in}}{\pgfqpoint{8.940829in}{1.656032in}}{\pgfqpoint{8.937262in}{1.652465in}}%
\pgfpathcurveto{\pgfqpoint{8.933696in}{1.648899in}}{\pgfqpoint{8.931692in}{1.644061in}}{\pgfqpoint{8.931692in}{1.639017in}}%
\pgfpathcurveto{\pgfqpoint{8.931692in}{1.633974in}}{\pgfqpoint{8.933696in}{1.629136in}}{\pgfqpoint{8.937262in}{1.625569in}}%
\pgfpathcurveto{\pgfqpoint{8.940829in}{1.622003in}}{\pgfqpoint{8.945666in}{1.619999in}}{\pgfqpoint{8.950710in}{1.619999in}}%
\pgfpathclose%
\pgfusepath{fill}%
\end{pgfscope}%
\begin{pgfscope}%
\pgfpathrectangle{\pgfqpoint{6.572727in}{0.474100in}}{\pgfqpoint{4.227273in}{3.318700in}}%
\pgfusepath{clip}%
\pgfsetbuttcap%
\pgfsetroundjoin%
\definecolor{currentfill}{rgb}{0.267004,0.004874,0.329415}%
\pgfsetfillcolor{currentfill}%
\pgfsetfillopacity{0.700000}%
\pgfsetlinewidth{0.000000pt}%
\definecolor{currentstroke}{rgb}{0.000000,0.000000,0.000000}%
\pgfsetstrokecolor{currentstroke}%
\pgfsetstrokeopacity{0.700000}%
\pgfsetdash{}{0pt}%
\pgfpathmoveto{\pgfqpoint{8.641015in}{1.955774in}}%
\pgfpathcurveto{\pgfqpoint{8.646058in}{1.955774in}}{\pgfqpoint{8.650896in}{1.957778in}}{\pgfqpoint{8.654463in}{1.961344in}}%
\pgfpathcurveto{\pgfqpoint{8.658029in}{1.964911in}}{\pgfqpoint{8.660033in}{1.969748in}}{\pgfqpoint{8.660033in}{1.974792in}}%
\pgfpathcurveto{\pgfqpoint{8.660033in}{1.979836in}}{\pgfqpoint{8.658029in}{1.984674in}}{\pgfqpoint{8.654463in}{1.988240in}}%
\pgfpathcurveto{\pgfqpoint{8.650896in}{1.991806in}}{\pgfqpoint{8.646058in}{1.993810in}}{\pgfqpoint{8.641015in}{1.993810in}}%
\pgfpathcurveto{\pgfqpoint{8.635971in}{1.993810in}}{\pgfqpoint{8.631133in}{1.991806in}}{\pgfqpoint{8.627567in}{1.988240in}}%
\pgfpathcurveto{\pgfqpoint{8.624000in}{1.984674in}}{\pgfqpoint{8.621997in}{1.979836in}}{\pgfqpoint{8.621997in}{1.974792in}}%
\pgfpathcurveto{\pgfqpoint{8.621997in}{1.969748in}}{\pgfqpoint{8.624000in}{1.964911in}}{\pgfqpoint{8.627567in}{1.961344in}}%
\pgfpathcurveto{\pgfqpoint{8.631133in}{1.957778in}}{\pgfqpoint{8.635971in}{1.955774in}}{\pgfqpoint{8.641015in}{1.955774in}}%
\pgfpathclose%
\pgfusepath{fill}%
\end{pgfscope}%
\begin{pgfscope}%
\pgfpathrectangle{\pgfqpoint{6.572727in}{0.474100in}}{\pgfqpoint{4.227273in}{3.318700in}}%
\pgfusepath{clip}%
\pgfsetbuttcap%
\pgfsetroundjoin%
\definecolor{currentfill}{rgb}{0.127568,0.566949,0.550556}%
\pgfsetfillcolor{currentfill}%
\pgfsetfillopacity{0.700000}%
\pgfsetlinewidth{0.000000pt}%
\definecolor{currentstroke}{rgb}{0.000000,0.000000,0.000000}%
\pgfsetstrokecolor{currentstroke}%
\pgfsetstrokeopacity{0.700000}%
\pgfsetdash{}{0pt}%
\pgfpathmoveto{\pgfqpoint{9.948102in}{2.050656in}}%
\pgfpathcurveto{\pgfqpoint{9.953145in}{2.050656in}}{\pgfqpoint{9.957983in}{2.052660in}}{\pgfqpoint{9.961550in}{2.056227in}}%
\pgfpathcurveto{\pgfqpoint{9.965116in}{2.059793in}}{\pgfqpoint{9.967120in}{2.064631in}}{\pgfqpoint{9.967120in}{2.069675in}}%
\pgfpathcurveto{\pgfqpoint{9.967120in}{2.074718in}}{\pgfqpoint{9.965116in}{2.079556in}}{\pgfqpoint{9.961550in}{2.083122in}}%
\pgfpathcurveto{\pgfqpoint{9.957983in}{2.086689in}}{\pgfqpoint{9.953145in}{2.088693in}}{\pgfqpoint{9.948102in}{2.088693in}}%
\pgfpathcurveto{\pgfqpoint{9.943058in}{2.088693in}}{\pgfqpoint{9.938220in}{2.086689in}}{\pgfqpoint{9.934654in}{2.083122in}}%
\pgfpathcurveto{\pgfqpoint{9.931088in}{2.079556in}}{\pgfqpoint{9.929084in}{2.074718in}}{\pgfqpoint{9.929084in}{2.069675in}}%
\pgfpathcurveto{\pgfqpoint{9.929084in}{2.064631in}}{\pgfqpoint{9.931088in}{2.059793in}}{\pgfqpoint{9.934654in}{2.056227in}}%
\pgfpathcurveto{\pgfqpoint{9.938220in}{2.052660in}}{\pgfqpoint{9.943058in}{2.050656in}}{\pgfqpoint{9.948102in}{2.050656in}}%
\pgfpathclose%
\pgfusepath{fill}%
\end{pgfscope}%
\begin{pgfscope}%
\pgfpathrectangle{\pgfqpoint{6.572727in}{0.474100in}}{\pgfqpoint{4.227273in}{3.318700in}}%
\pgfusepath{clip}%
\pgfsetbuttcap%
\pgfsetroundjoin%
\definecolor{currentfill}{rgb}{0.993248,0.906157,0.143936}%
\pgfsetfillcolor{currentfill}%
\pgfsetfillopacity{0.700000}%
\pgfsetlinewidth{0.000000pt}%
\definecolor{currentstroke}{rgb}{0.000000,0.000000,0.000000}%
\pgfsetstrokecolor{currentstroke}%
\pgfsetstrokeopacity{0.700000}%
\pgfsetdash{}{0pt}%
\pgfpathmoveto{\pgfqpoint{8.229313in}{2.717752in}}%
\pgfpathcurveto{\pgfqpoint{8.234357in}{2.717752in}}{\pgfqpoint{8.239194in}{2.719756in}}{\pgfqpoint{8.242761in}{2.723322in}}%
\pgfpathcurveto{\pgfqpoint{8.246327in}{2.726889in}}{\pgfqpoint{8.248331in}{2.731726in}}{\pgfqpoint{8.248331in}{2.736770in}}%
\pgfpathcurveto{\pgfqpoint{8.248331in}{2.741814in}}{\pgfqpoint{8.246327in}{2.746651in}}{\pgfqpoint{8.242761in}{2.750218in}}%
\pgfpathcurveto{\pgfqpoint{8.239194in}{2.753784in}}{\pgfqpoint{8.234357in}{2.755788in}}{\pgfqpoint{8.229313in}{2.755788in}}%
\pgfpathcurveto{\pgfqpoint{8.224269in}{2.755788in}}{\pgfqpoint{8.219431in}{2.753784in}}{\pgfqpoint{8.215865in}{2.750218in}}%
\pgfpathcurveto{\pgfqpoint{8.212299in}{2.746651in}}{\pgfqpoint{8.210295in}{2.741814in}}{\pgfqpoint{8.210295in}{2.736770in}}%
\pgfpathcurveto{\pgfqpoint{8.210295in}{2.731726in}}{\pgfqpoint{8.212299in}{2.726889in}}{\pgfqpoint{8.215865in}{2.723322in}}%
\pgfpathcurveto{\pgfqpoint{8.219431in}{2.719756in}}{\pgfqpoint{8.224269in}{2.717752in}}{\pgfqpoint{8.229313in}{2.717752in}}%
\pgfpathclose%
\pgfusepath{fill}%
\end{pgfscope}%
\begin{pgfscope}%
\pgfpathrectangle{\pgfqpoint{6.572727in}{0.474100in}}{\pgfqpoint{4.227273in}{3.318700in}}%
\pgfusepath{clip}%
\pgfsetbuttcap%
\pgfsetroundjoin%
\definecolor{currentfill}{rgb}{0.993248,0.906157,0.143936}%
\pgfsetfillcolor{currentfill}%
\pgfsetfillopacity{0.700000}%
\pgfsetlinewidth{0.000000pt}%
\definecolor{currentstroke}{rgb}{0.000000,0.000000,0.000000}%
\pgfsetstrokecolor{currentstroke}%
\pgfsetstrokeopacity{0.700000}%
\pgfsetdash{}{0pt}%
\pgfpathmoveto{\pgfqpoint{8.259005in}{2.258402in}}%
\pgfpathcurveto{\pgfqpoint{8.264049in}{2.258402in}}{\pgfqpoint{8.268887in}{2.260406in}}{\pgfqpoint{8.272453in}{2.263973in}}%
\pgfpathcurveto{\pgfqpoint{8.276019in}{2.267539in}}{\pgfqpoint{8.278023in}{2.272377in}}{\pgfqpoint{8.278023in}{2.277421in}}%
\pgfpathcurveto{\pgfqpoint{8.278023in}{2.282464in}}{\pgfqpoint{8.276019in}{2.287302in}}{\pgfqpoint{8.272453in}{2.290868in}}%
\pgfpathcurveto{\pgfqpoint{8.268887in}{2.294435in}}{\pgfqpoint{8.264049in}{2.296439in}}{\pgfqpoint{8.259005in}{2.296439in}}%
\pgfpathcurveto{\pgfqpoint{8.253961in}{2.296439in}}{\pgfqpoint{8.249124in}{2.294435in}}{\pgfqpoint{8.245557in}{2.290868in}}%
\pgfpathcurveto{\pgfqpoint{8.241991in}{2.287302in}}{\pgfqpoint{8.239987in}{2.282464in}}{\pgfqpoint{8.239987in}{2.277421in}}%
\pgfpathcurveto{\pgfqpoint{8.239987in}{2.272377in}}{\pgfqpoint{8.241991in}{2.267539in}}{\pgfqpoint{8.245557in}{2.263973in}}%
\pgfpathcurveto{\pgfqpoint{8.249124in}{2.260406in}}{\pgfqpoint{8.253961in}{2.258402in}}{\pgfqpoint{8.259005in}{2.258402in}}%
\pgfpathclose%
\pgfusepath{fill}%
\end{pgfscope}%
\begin{pgfscope}%
\pgfpathrectangle{\pgfqpoint{6.572727in}{0.474100in}}{\pgfqpoint{4.227273in}{3.318700in}}%
\pgfusepath{clip}%
\pgfsetbuttcap%
\pgfsetroundjoin%
\definecolor{currentfill}{rgb}{0.993248,0.906157,0.143936}%
\pgfsetfillcolor{currentfill}%
\pgfsetfillopacity{0.700000}%
\pgfsetlinewidth{0.000000pt}%
\definecolor{currentstroke}{rgb}{0.000000,0.000000,0.000000}%
\pgfsetstrokecolor{currentstroke}%
\pgfsetstrokeopacity{0.700000}%
\pgfsetdash{}{0pt}%
\pgfpathmoveto{\pgfqpoint{8.883793in}{3.147409in}}%
\pgfpathcurveto{\pgfqpoint{8.888837in}{3.147409in}}{\pgfqpoint{8.893675in}{3.149413in}}{\pgfqpoint{8.897241in}{3.152979in}}%
\pgfpathcurveto{\pgfqpoint{8.900807in}{3.156545in}}{\pgfqpoint{8.902811in}{3.161383in}}{\pgfqpoint{8.902811in}{3.166427in}}%
\pgfpathcurveto{\pgfqpoint{8.902811in}{3.171471in}}{\pgfqpoint{8.900807in}{3.176308in}}{\pgfqpoint{8.897241in}{3.179875in}}%
\pgfpathcurveto{\pgfqpoint{8.893675in}{3.183441in}}{\pgfqpoint{8.888837in}{3.185445in}}{\pgfqpoint{8.883793in}{3.185445in}}%
\pgfpathcurveto{\pgfqpoint{8.878749in}{3.185445in}}{\pgfqpoint{8.873912in}{3.183441in}}{\pgfqpoint{8.870345in}{3.179875in}}%
\pgfpathcurveto{\pgfqpoint{8.866779in}{3.176308in}}{\pgfqpoint{8.864775in}{3.171471in}}{\pgfqpoint{8.864775in}{3.166427in}}%
\pgfpathcurveto{\pgfqpoint{8.864775in}{3.161383in}}{\pgfqpoint{8.866779in}{3.156545in}}{\pgfqpoint{8.870345in}{3.152979in}}%
\pgfpathcurveto{\pgfqpoint{8.873912in}{3.149413in}}{\pgfqpoint{8.878749in}{3.147409in}}{\pgfqpoint{8.883793in}{3.147409in}}%
\pgfpathclose%
\pgfusepath{fill}%
\end{pgfscope}%
\begin{pgfscope}%
\pgfpathrectangle{\pgfqpoint{6.572727in}{0.474100in}}{\pgfqpoint{4.227273in}{3.318700in}}%
\pgfusepath{clip}%
\pgfsetbuttcap%
\pgfsetroundjoin%
\definecolor{currentfill}{rgb}{0.267004,0.004874,0.329415}%
\pgfsetfillcolor{currentfill}%
\pgfsetfillopacity{0.700000}%
\pgfsetlinewidth{0.000000pt}%
\definecolor{currentstroke}{rgb}{0.000000,0.000000,0.000000}%
\pgfsetstrokecolor{currentstroke}%
\pgfsetstrokeopacity{0.700000}%
\pgfsetdash{}{0pt}%
\pgfpathmoveto{\pgfqpoint{7.810273in}{1.561932in}}%
\pgfpathcurveto{\pgfqpoint{7.815317in}{1.561932in}}{\pgfqpoint{7.820155in}{1.563936in}}{\pgfqpoint{7.823721in}{1.567502in}}%
\pgfpathcurveto{\pgfqpoint{7.827288in}{1.571069in}}{\pgfqpoint{7.829292in}{1.575907in}}{\pgfqpoint{7.829292in}{1.580950in}}%
\pgfpathcurveto{\pgfqpoint{7.829292in}{1.585994in}}{\pgfqpoint{7.827288in}{1.590832in}}{\pgfqpoint{7.823721in}{1.594398in}}%
\pgfpathcurveto{\pgfqpoint{7.820155in}{1.597965in}}{\pgfqpoint{7.815317in}{1.599968in}}{\pgfqpoint{7.810273in}{1.599968in}}%
\pgfpathcurveto{\pgfqpoint{7.805230in}{1.599968in}}{\pgfqpoint{7.800392in}{1.597965in}}{\pgfqpoint{7.796826in}{1.594398in}}%
\pgfpathcurveto{\pgfqpoint{7.793259in}{1.590832in}}{\pgfqpoint{7.791255in}{1.585994in}}{\pgfqpoint{7.791255in}{1.580950in}}%
\pgfpathcurveto{\pgfqpoint{7.791255in}{1.575907in}}{\pgfqpoint{7.793259in}{1.571069in}}{\pgfqpoint{7.796826in}{1.567502in}}%
\pgfpathcurveto{\pgfqpoint{7.800392in}{1.563936in}}{\pgfqpoint{7.805230in}{1.561932in}}{\pgfqpoint{7.810273in}{1.561932in}}%
\pgfpathclose%
\pgfusepath{fill}%
\end{pgfscope}%
\begin{pgfscope}%
\pgfpathrectangle{\pgfqpoint{6.572727in}{0.474100in}}{\pgfqpoint{4.227273in}{3.318700in}}%
\pgfusepath{clip}%
\pgfsetbuttcap%
\pgfsetroundjoin%
\definecolor{currentfill}{rgb}{0.993248,0.906157,0.143936}%
\pgfsetfillcolor{currentfill}%
\pgfsetfillopacity{0.700000}%
\pgfsetlinewidth{0.000000pt}%
\definecolor{currentstroke}{rgb}{0.000000,0.000000,0.000000}%
\pgfsetstrokecolor{currentstroke}%
\pgfsetstrokeopacity{0.700000}%
\pgfsetdash{}{0pt}%
\pgfpathmoveto{\pgfqpoint{8.219733in}{3.388496in}}%
\pgfpathcurveto{\pgfqpoint{8.224777in}{3.388496in}}{\pgfqpoint{8.229614in}{3.390500in}}{\pgfqpoint{8.233181in}{3.394067in}}%
\pgfpathcurveto{\pgfqpoint{8.236747in}{3.397633in}}{\pgfqpoint{8.238751in}{3.402471in}}{\pgfqpoint{8.238751in}{3.407515in}}%
\pgfpathcurveto{\pgfqpoint{8.238751in}{3.412558in}}{\pgfqpoint{8.236747in}{3.417396in}}{\pgfqpoint{8.233181in}{3.420962in}}%
\pgfpathcurveto{\pgfqpoint{8.229614in}{3.424529in}}{\pgfqpoint{8.224777in}{3.426533in}}{\pgfqpoint{8.219733in}{3.426533in}}%
\pgfpathcurveto{\pgfqpoint{8.214689in}{3.426533in}}{\pgfqpoint{8.209851in}{3.424529in}}{\pgfqpoint{8.206285in}{3.420962in}}%
\pgfpathcurveto{\pgfqpoint{8.202719in}{3.417396in}}{\pgfqpoint{8.200715in}{3.412558in}}{\pgfqpoint{8.200715in}{3.407515in}}%
\pgfpathcurveto{\pgfqpoint{8.200715in}{3.402471in}}{\pgfqpoint{8.202719in}{3.397633in}}{\pgfqpoint{8.206285in}{3.394067in}}%
\pgfpathcurveto{\pgfqpoint{8.209851in}{3.390500in}}{\pgfqpoint{8.214689in}{3.388496in}}{\pgfqpoint{8.219733in}{3.388496in}}%
\pgfpathclose%
\pgfusepath{fill}%
\end{pgfscope}%
\begin{pgfscope}%
\pgfpathrectangle{\pgfqpoint{6.572727in}{0.474100in}}{\pgfqpoint{4.227273in}{3.318700in}}%
\pgfusepath{clip}%
\pgfsetbuttcap%
\pgfsetroundjoin%
\definecolor{currentfill}{rgb}{0.993248,0.906157,0.143936}%
\pgfsetfillcolor{currentfill}%
\pgfsetfillopacity{0.700000}%
\pgfsetlinewidth{0.000000pt}%
\definecolor{currentstroke}{rgb}{0.000000,0.000000,0.000000}%
\pgfsetstrokecolor{currentstroke}%
\pgfsetstrokeopacity{0.700000}%
\pgfsetdash{}{0pt}%
\pgfpathmoveto{\pgfqpoint{8.327065in}{2.924489in}}%
\pgfpathcurveto{\pgfqpoint{8.332109in}{2.924489in}}{\pgfqpoint{8.336947in}{2.926492in}}{\pgfqpoint{8.340513in}{2.930059in}}%
\pgfpathcurveto{\pgfqpoint{8.344080in}{2.933625in}}{\pgfqpoint{8.346084in}{2.938463in}}{\pgfqpoint{8.346084in}{2.943507in}}%
\pgfpathcurveto{\pgfqpoint{8.346084in}{2.948550in}}{\pgfqpoint{8.344080in}{2.953388in}}{\pgfqpoint{8.340513in}{2.956955in}}%
\pgfpathcurveto{\pgfqpoint{8.336947in}{2.960521in}}{\pgfqpoint{8.332109in}{2.962525in}}{\pgfqpoint{8.327065in}{2.962525in}}%
\pgfpathcurveto{\pgfqpoint{8.322022in}{2.962525in}}{\pgfqpoint{8.317184in}{2.960521in}}{\pgfqpoint{8.313618in}{2.956955in}}%
\pgfpathcurveto{\pgfqpoint{8.310051in}{2.953388in}}{\pgfqpoint{8.308047in}{2.948550in}}{\pgfqpoint{8.308047in}{2.943507in}}%
\pgfpathcurveto{\pgfqpoint{8.308047in}{2.938463in}}{\pgfqpoint{8.310051in}{2.933625in}}{\pgfqpoint{8.313618in}{2.930059in}}%
\pgfpathcurveto{\pgfqpoint{8.317184in}{2.926492in}}{\pgfqpoint{8.322022in}{2.924489in}}{\pgfqpoint{8.327065in}{2.924489in}}%
\pgfpathclose%
\pgfusepath{fill}%
\end{pgfscope}%
\begin{pgfscope}%
\pgfpathrectangle{\pgfqpoint{6.572727in}{0.474100in}}{\pgfqpoint{4.227273in}{3.318700in}}%
\pgfusepath{clip}%
\pgfsetbuttcap%
\pgfsetroundjoin%
\definecolor{currentfill}{rgb}{0.267004,0.004874,0.329415}%
\pgfsetfillcolor{currentfill}%
\pgfsetfillopacity{0.700000}%
\pgfsetlinewidth{0.000000pt}%
\definecolor{currentstroke}{rgb}{0.000000,0.000000,0.000000}%
\pgfsetstrokecolor{currentstroke}%
\pgfsetstrokeopacity{0.700000}%
\pgfsetdash{}{0pt}%
\pgfpathmoveto{\pgfqpoint{7.352729in}{1.174107in}}%
\pgfpathcurveto{\pgfqpoint{7.357773in}{1.174107in}}{\pgfqpoint{7.362610in}{1.176111in}}{\pgfqpoint{7.366177in}{1.179678in}}%
\pgfpathcurveto{\pgfqpoint{7.369743in}{1.183244in}}{\pgfqpoint{7.371747in}{1.188082in}}{\pgfqpoint{7.371747in}{1.193125in}}%
\pgfpathcurveto{\pgfqpoint{7.371747in}{1.198169in}}{\pgfqpoint{7.369743in}{1.203007in}}{\pgfqpoint{7.366177in}{1.206573in}}%
\pgfpathcurveto{\pgfqpoint{7.362610in}{1.210140in}}{\pgfqpoint{7.357773in}{1.212144in}}{\pgfqpoint{7.352729in}{1.212144in}}%
\pgfpathcurveto{\pgfqpoint{7.347685in}{1.212144in}}{\pgfqpoint{7.342847in}{1.210140in}}{\pgfqpoint{7.339281in}{1.206573in}}%
\pgfpathcurveto{\pgfqpoint{7.335715in}{1.203007in}}{\pgfqpoint{7.333711in}{1.198169in}}{\pgfqpoint{7.333711in}{1.193125in}}%
\pgfpathcurveto{\pgfqpoint{7.333711in}{1.188082in}}{\pgfqpoint{7.335715in}{1.183244in}}{\pgfqpoint{7.339281in}{1.179678in}}%
\pgfpathcurveto{\pgfqpoint{7.342847in}{1.176111in}}{\pgfqpoint{7.347685in}{1.174107in}}{\pgfqpoint{7.352729in}{1.174107in}}%
\pgfpathclose%
\pgfusepath{fill}%
\end{pgfscope}%
\begin{pgfscope}%
\pgfpathrectangle{\pgfqpoint{6.572727in}{0.474100in}}{\pgfqpoint{4.227273in}{3.318700in}}%
\pgfusepath{clip}%
\pgfsetbuttcap%
\pgfsetroundjoin%
\definecolor{currentfill}{rgb}{0.993248,0.906157,0.143936}%
\pgfsetfillcolor{currentfill}%
\pgfsetfillopacity{0.700000}%
\pgfsetlinewidth{0.000000pt}%
\definecolor{currentstroke}{rgb}{0.000000,0.000000,0.000000}%
\pgfsetstrokecolor{currentstroke}%
\pgfsetstrokeopacity{0.700000}%
\pgfsetdash{}{0pt}%
\pgfpathmoveto{\pgfqpoint{8.524468in}{2.854757in}}%
\pgfpathcurveto{\pgfqpoint{8.529512in}{2.854757in}}{\pgfqpoint{8.534350in}{2.856760in}}{\pgfqpoint{8.537916in}{2.860327in}}%
\pgfpathcurveto{\pgfqpoint{8.541482in}{2.863893in}}{\pgfqpoint{8.543486in}{2.868731in}}{\pgfqpoint{8.543486in}{2.873775in}}%
\pgfpathcurveto{\pgfqpoint{8.543486in}{2.878818in}}{\pgfqpoint{8.541482in}{2.883656in}}{\pgfqpoint{8.537916in}{2.887223in}}%
\pgfpathcurveto{\pgfqpoint{8.534350in}{2.890789in}}{\pgfqpoint{8.529512in}{2.892793in}}{\pgfqpoint{8.524468in}{2.892793in}}%
\pgfpathcurveto{\pgfqpoint{8.519424in}{2.892793in}}{\pgfqpoint{8.514587in}{2.890789in}}{\pgfqpoint{8.511020in}{2.887223in}}%
\pgfpathcurveto{\pgfqpoint{8.507454in}{2.883656in}}{\pgfqpoint{8.505450in}{2.878818in}}{\pgfqpoint{8.505450in}{2.873775in}}%
\pgfpathcurveto{\pgfqpoint{8.505450in}{2.868731in}}{\pgfqpoint{8.507454in}{2.863893in}}{\pgfqpoint{8.511020in}{2.860327in}}%
\pgfpathcurveto{\pgfqpoint{8.514587in}{2.856760in}}{\pgfqpoint{8.519424in}{2.854757in}}{\pgfqpoint{8.524468in}{2.854757in}}%
\pgfpathclose%
\pgfusepath{fill}%
\end{pgfscope}%
\begin{pgfscope}%
\pgfpathrectangle{\pgfqpoint{6.572727in}{0.474100in}}{\pgfqpoint{4.227273in}{3.318700in}}%
\pgfusepath{clip}%
\pgfsetbuttcap%
\pgfsetroundjoin%
\definecolor{currentfill}{rgb}{0.993248,0.906157,0.143936}%
\pgfsetfillcolor{currentfill}%
\pgfsetfillopacity{0.700000}%
\pgfsetlinewidth{0.000000pt}%
\definecolor{currentstroke}{rgb}{0.000000,0.000000,0.000000}%
\pgfsetstrokecolor{currentstroke}%
\pgfsetstrokeopacity{0.700000}%
\pgfsetdash{}{0pt}%
\pgfpathmoveto{\pgfqpoint{8.186872in}{3.570710in}}%
\pgfpathcurveto{\pgfqpoint{8.191915in}{3.570710in}}{\pgfqpoint{8.196753in}{3.572714in}}{\pgfqpoint{8.200320in}{3.576281in}}%
\pgfpathcurveto{\pgfqpoint{8.203886in}{3.579847in}}{\pgfqpoint{8.205890in}{3.584685in}}{\pgfqpoint{8.205890in}{3.589728in}}%
\pgfpathcurveto{\pgfqpoint{8.205890in}{3.594772in}}{\pgfqpoint{8.203886in}{3.599610in}}{\pgfqpoint{8.200320in}{3.603176in}}%
\pgfpathcurveto{\pgfqpoint{8.196753in}{3.606743in}}{\pgfqpoint{8.191915in}{3.608747in}}{\pgfqpoint{8.186872in}{3.608747in}}%
\pgfpathcurveto{\pgfqpoint{8.181828in}{3.608747in}}{\pgfqpoint{8.176990in}{3.606743in}}{\pgfqpoint{8.173424in}{3.603176in}}%
\pgfpathcurveto{\pgfqpoint{8.169857in}{3.599610in}}{\pgfqpoint{8.167854in}{3.594772in}}{\pgfqpoint{8.167854in}{3.589728in}}%
\pgfpathcurveto{\pgfqpoint{8.167854in}{3.584685in}}{\pgfqpoint{8.169857in}{3.579847in}}{\pgfqpoint{8.173424in}{3.576281in}}%
\pgfpathcurveto{\pgfqpoint{8.176990in}{3.572714in}}{\pgfqpoint{8.181828in}{3.570710in}}{\pgfqpoint{8.186872in}{3.570710in}}%
\pgfpathclose%
\pgfusepath{fill}%
\end{pgfscope}%
\begin{pgfscope}%
\pgfpathrectangle{\pgfqpoint{6.572727in}{0.474100in}}{\pgfqpoint{4.227273in}{3.318700in}}%
\pgfusepath{clip}%
\pgfsetbuttcap%
\pgfsetroundjoin%
\definecolor{currentfill}{rgb}{0.127568,0.566949,0.550556}%
\pgfsetfillcolor{currentfill}%
\pgfsetfillopacity{0.700000}%
\pgfsetlinewidth{0.000000pt}%
\definecolor{currentstroke}{rgb}{0.000000,0.000000,0.000000}%
\pgfsetstrokecolor{currentstroke}%
\pgfsetstrokeopacity{0.700000}%
\pgfsetdash{}{0pt}%
\pgfpathmoveto{\pgfqpoint{9.933332in}{1.329904in}}%
\pgfpathcurveto{\pgfqpoint{9.938376in}{1.329904in}}{\pgfqpoint{9.943214in}{1.331908in}}{\pgfqpoint{9.946780in}{1.335474in}}%
\pgfpathcurveto{\pgfqpoint{9.950347in}{1.339041in}}{\pgfqpoint{9.952351in}{1.343879in}}{\pgfqpoint{9.952351in}{1.348922in}}%
\pgfpathcurveto{\pgfqpoint{9.952351in}{1.353966in}}{\pgfqpoint{9.950347in}{1.358804in}}{\pgfqpoint{9.946780in}{1.362370in}}%
\pgfpathcurveto{\pgfqpoint{9.943214in}{1.365937in}}{\pgfqpoint{9.938376in}{1.367940in}}{\pgfqpoint{9.933332in}{1.367940in}}%
\pgfpathcurveto{\pgfqpoint{9.928289in}{1.367940in}}{\pgfqpoint{9.923451in}{1.365937in}}{\pgfqpoint{9.919885in}{1.362370in}}%
\pgfpathcurveto{\pgfqpoint{9.916318in}{1.358804in}}{\pgfqpoint{9.914314in}{1.353966in}}{\pgfqpoint{9.914314in}{1.348922in}}%
\pgfpathcurveto{\pgfqpoint{9.914314in}{1.343879in}}{\pgfqpoint{9.916318in}{1.339041in}}{\pgfqpoint{9.919885in}{1.335474in}}%
\pgfpathcurveto{\pgfqpoint{9.923451in}{1.331908in}}{\pgfqpoint{9.928289in}{1.329904in}}{\pgfqpoint{9.933332in}{1.329904in}}%
\pgfpathclose%
\pgfusepath{fill}%
\end{pgfscope}%
\begin{pgfscope}%
\pgfpathrectangle{\pgfqpoint{6.572727in}{0.474100in}}{\pgfqpoint{4.227273in}{3.318700in}}%
\pgfusepath{clip}%
\pgfsetbuttcap%
\pgfsetroundjoin%
\definecolor{currentfill}{rgb}{0.993248,0.906157,0.143936}%
\pgfsetfillcolor{currentfill}%
\pgfsetfillopacity{0.700000}%
\pgfsetlinewidth{0.000000pt}%
\definecolor{currentstroke}{rgb}{0.000000,0.000000,0.000000}%
\pgfsetstrokecolor{currentstroke}%
\pgfsetstrokeopacity{0.700000}%
\pgfsetdash{}{0pt}%
\pgfpathmoveto{\pgfqpoint{8.230374in}{2.422254in}}%
\pgfpathcurveto{\pgfqpoint{8.235418in}{2.422254in}}{\pgfqpoint{8.240256in}{2.424258in}}{\pgfqpoint{8.243822in}{2.427824in}}%
\pgfpathcurveto{\pgfqpoint{8.247389in}{2.431390in}}{\pgfqpoint{8.249392in}{2.436228in}}{\pgfqpoint{8.249392in}{2.441272in}}%
\pgfpathcurveto{\pgfqpoint{8.249392in}{2.446316in}}{\pgfqpoint{8.247389in}{2.451153in}}{\pgfqpoint{8.243822in}{2.454720in}}%
\pgfpathcurveto{\pgfqpoint{8.240256in}{2.458286in}}{\pgfqpoint{8.235418in}{2.460290in}}{\pgfqpoint{8.230374in}{2.460290in}}%
\pgfpathcurveto{\pgfqpoint{8.225331in}{2.460290in}}{\pgfqpoint{8.220493in}{2.458286in}}{\pgfqpoint{8.216926in}{2.454720in}}%
\pgfpathcurveto{\pgfqpoint{8.213360in}{2.451153in}}{\pgfqpoint{8.211356in}{2.446316in}}{\pgfqpoint{8.211356in}{2.441272in}}%
\pgfpathcurveto{\pgfqpoint{8.211356in}{2.436228in}}{\pgfqpoint{8.213360in}{2.431390in}}{\pgfqpoint{8.216926in}{2.427824in}}%
\pgfpathcurveto{\pgfqpoint{8.220493in}{2.424258in}}{\pgfqpoint{8.225331in}{2.422254in}}{\pgfqpoint{8.230374in}{2.422254in}}%
\pgfpathclose%
\pgfusepath{fill}%
\end{pgfscope}%
\begin{pgfscope}%
\pgfpathrectangle{\pgfqpoint{6.572727in}{0.474100in}}{\pgfqpoint{4.227273in}{3.318700in}}%
\pgfusepath{clip}%
\pgfsetbuttcap%
\pgfsetroundjoin%
\definecolor{currentfill}{rgb}{0.993248,0.906157,0.143936}%
\pgfsetfillcolor{currentfill}%
\pgfsetfillopacity{0.700000}%
\pgfsetlinewidth{0.000000pt}%
\definecolor{currentstroke}{rgb}{0.000000,0.000000,0.000000}%
\pgfsetstrokecolor{currentstroke}%
\pgfsetstrokeopacity{0.700000}%
\pgfsetdash{}{0pt}%
\pgfpathmoveto{\pgfqpoint{8.234453in}{2.957589in}}%
\pgfpathcurveto{\pgfqpoint{8.239497in}{2.957589in}}{\pgfqpoint{8.244335in}{2.959593in}}{\pgfqpoint{8.247901in}{2.963159in}}%
\pgfpathcurveto{\pgfqpoint{8.251468in}{2.966726in}}{\pgfqpoint{8.253472in}{2.971563in}}{\pgfqpoint{8.253472in}{2.976607in}}%
\pgfpathcurveto{\pgfqpoint{8.253472in}{2.981651in}}{\pgfqpoint{8.251468in}{2.986489in}}{\pgfqpoint{8.247901in}{2.990055in}}%
\pgfpathcurveto{\pgfqpoint{8.244335in}{2.993621in}}{\pgfqpoint{8.239497in}{2.995625in}}{\pgfqpoint{8.234453in}{2.995625in}}%
\pgfpathcurveto{\pgfqpoint{8.229410in}{2.995625in}}{\pgfqpoint{8.224572in}{2.993621in}}{\pgfqpoint{8.221006in}{2.990055in}}%
\pgfpathcurveto{\pgfqpoint{8.217439in}{2.986489in}}{\pgfqpoint{8.215435in}{2.981651in}}{\pgfqpoint{8.215435in}{2.976607in}}%
\pgfpathcurveto{\pgfqpoint{8.215435in}{2.971563in}}{\pgfqpoint{8.217439in}{2.966726in}}{\pgfqpoint{8.221006in}{2.963159in}}%
\pgfpathcurveto{\pgfqpoint{8.224572in}{2.959593in}}{\pgfqpoint{8.229410in}{2.957589in}}{\pgfqpoint{8.234453in}{2.957589in}}%
\pgfpathclose%
\pgfusepath{fill}%
\end{pgfscope}%
\begin{pgfscope}%
\pgfpathrectangle{\pgfqpoint{6.572727in}{0.474100in}}{\pgfqpoint{4.227273in}{3.318700in}}%
\pgfusepath{clip}%
\pgfsetbuttcap%
\pgfsetroundjoin%
\definecolor{currentfill}{rgb}{0.993248,0.906157,0.143936}%
\pgfsetfillcolor{currentfill}%
\pgfsetfillopacity{0.700000}%
\pgfsetlinewidth{0.000000pt}%
\definecolor{currentstroke}{rgb}{0.000000,0.000000,0.000000}%
\pgfsetstrokecolor{currentstroke}%
\pgfsetstrokeopacity{0.700000}%
\pgfsetdash{}{0pt}%
\pgfpathmoveto{\pgfqpoint{8.201388in}{2.585868in}}%
\pgfpathcurveto{\pgfqpoint{8.206432in}{2.585868in}}{\pgfqpoint{8.211269in}{2.587872in}}{\pgfqpoint{8.214836in}{2.591438in}}%
\pgfpathcurveto{\pgfqpoint{8.218402in}{2.595005in}}{\pgfqpoint{8.220406in}{2.599843in}}{\pgfqpoint{8.220406in}{2.604886in}}%
\pgfpathcurveto{\pgfqpoint{8.220406in}{2.609930in}}{\pgfqpoint{8.218402in}{2.614768in}}{\pgfqpoint{8.214836in}{2.618334in}}%
\pgfpathcurveto{\pgfqpoint{8.211269in}{2.621901in}}{\pgfqpoint{8.206432in}{2.623904in}}{\pgfqpoint{8.201388in}{2.623904in}}%
\pgfpathcurveto{\pgfqpoint{8.196344in}{2.623904in}}{\pgfqpoint{8.191506in}{2.621901in}}{\pgfqpoint{8.187940in}{2.618334in}}%
\pgfpathcurveto{\pgfqpoint{8.184374in}{2.614768in}}{\pgfqpoint{8.182370in}{2.609930in}}{\pgfqpoint{8.182370in}{2.604886in}}%
\pgfpathcurveto{\pgfqpoint{8.182370in}{2.599843in}}{\pgfqpoint{8.184374in}{2.595005in}}{\pgfqpoint{8.187940in}{2.591438in}}%
\pgfpathcurveto{\pgfqpoint{8.191506in}{2.587872in}}{\pgfqpoint{8.196344in}{2.585868in}}{\pgfqpoint{8.201388in}{2.585868in}}%
\pgfpathclose%
\pgfusepath{fill}%
\end{pgfscope}%
\begin{pgfscope}%
\pgfpathrectangle{\pgfqpoint{6.572727in}{0.474100in}}{\pgfqpoint{4.227273in}{3.318700in}}%
\pgfusepath{clip}%
\pgfsetbuttcap%
\pgfsetroundjoin%
\definecolor{currentfill}{rgb}{0.267004,0.004874,0.329415}%
\pgfsetfillcolor{currentfill}%
\pgfsetfillopacity{0.700000}%
\pgfsetlinewidth{0.000000pt}%
\definecolor{currentstroke}{rgb}{0.000000,0.000000,0.000000}%
\pgfsetstrokecolor{currentstroke}%
\pgfsetstrokeopacity{0.700000}%
\pgfsetdash{}{0pt}%
\pgfpathmoveto{\pgfqpoint{7.506977in}{1.539659in}}%
\pgfpathcurveto{\pgfqpoint{7.512021in}{1.539659in}}{\pgfqpoint{7.516859in}{1.541662in}}{\pgfqpoint{7.520425in}{1.545229in}}%
\pgfpathcurveto{\pgfqpoint{7.523992in}{1.548795in}}{\pgfqpoint{7.525995in}{1.553633in}}{\pgfqpoint{7.525995in}{1.558677in}}%
\pgfpathcurveto{\pgfqpoint{7.525995in}{1.563720in}}{\pgfqpoint{7.523992in}{1.568558in}}{\pgfqpoint{7.520425in}{1.572125in}}%
\pgfpathcurveto{\pgfqpoint{7.516859in}{1.575691in}}{\pgfqpoint{7.512021in}{1.577695in}}{\pgfqpoint{7.506977in}{1.577695in}}%
\pgfpathcurveto{\pgfqpoint{7.501934in}{1.577695in}}{\pgfqpoint{7.497096in}{1.575691in}}{\pgfqpoint{7.493529in}{1.572125in}}%
\pgfpathcurveto{\pgfqpoint{7.489963in}{1.568558in}}{\pgfqpoint{7.487959in}{1.563720in}}{\pgfqpoint{7.487959in}{1.558677in}}%
\pgfpathcurveto{\pgfqpoint{7.487959in}{1.553633in}}{\pgfqpoint{7.489963in}{1.548795in}}{\pgfqpoint{7.493529in}{1.545229in}}%
\pgfpathcurveto{\pgfqpoint{7.497096in}{1.541662in}}{\pgfqpoint{7.501934in}{1.539659in}}{\pgfqpoint{7.506977in}{1.539659in}}%
\pgfpathclose%
\pgfusepath{fill}%
\end{pgfscope}%
\begin{pgfscope}%
\pgfpathrectangle{\pgfqpoint{6.572727in}{0.474100in}}{\pgfqpoint{4.227273in}{3.318700in}}%
\pgfusepath{clip}%
\pgfsetbuttcap%
\pgfsetroundjoin%
\definecolor{currentfill}{rgb}{0.127568,0.566949,0.550556}%
\pgfsetfillcolor{currentfill}%
\pgfsetfillopacity{0.700000}%
\pgfsetlinewidth{0.000000pt}%
\definecolor{currentstroke}{rgb}{0.000000,0.000000,0.000000}%
\pgfsetstrokecolor{currentstroke}%
\pgfsetstrokeopacity{0.700000}%
\pgfsetdash{}{0pt}%
\pgfpathmoveto{\pgfqpoint{9.828348in}{1.199761in}}%
\pgfpathcurveto{\pgfqpoint{9.833391in}{1.199761in}}{\pgfqpoint{9.838229in}{1.201764in}}{\pgfqpoint{9.841795in}{1.205331in}}%
\pgfpathcurveto{\pgfqpoint{9.845362in}{1.208897in}}{\pgfqpoint{9.847366in}{1.213735in}}{\pgfqpoint{9.847366in}{1.218779in}}%
\pgfpathcurveto{\pgfqpoint{9.847366in}{1.223822in}}{\pgfqpoint{9.845362in}{1.228660in}}{\pgfqpoint{9.841795in}{1.232227in}}%
\pgfpathcurveto{\pgfqpoint{9.838229in}{1.235793in}}{\pgfqpoint{9.833391in}{1.237797in}}{\pgfqpoint{9.828348in}{1.237797in}}%
\pgfpathcurveto{\pgfqpoint{9.823304in}{1.237797in}}{\pgfqpoint{9.818466in}{1.235793in}}{\pgfqpoint{9.814900in}{1.232227in}}%
\pgfpathcurveto{\pgfqpoint{9.811333in}{1.228660in}}{\pgfqpoint{9.809329in}{1.223822in}}{\pgfqpoint{9.809329in}{1.218779in}}%
\pgfpathcurveto{\pgfqpoint{9.809329in}{1.213735in}}{\pgfqpoint{9.811333in}{1.208897in}}{\pgfqpoint{9.814900in}{1.205331in}}%
\pgfpathcurveto{\pgfqpoint{9.818466in}{1.201764in}}{\pgfqpoint{9.823304in}{1.199761in}}{\pgfqpoint{9.828348in}{1.199761in}}%
\pgfpathclose%
\pgfusepath{fill}%
\end{pgfscope}%
\begin{pgfscope}%
\pgfpathrectangle{\pgfqpoint{6.572727in}{0.474100in}}{\pgfqpoint{4.227273in}{3.318700in}}%
\pgfusepath{clip}%
\pgfsetbuttcap%
\pgfsetroundjoin%
\definecolor{currentfill}{rgb}{0.993248,0.906157,0.143936}%
\pgfsetfillcolor{currentfill}%
\pgfsetfillopacity{0.700000}%
\pgfsetlinewidth{0.000000pt}%
\definecolor{currentstroke}{rgb}{0.000000,0.000000,0.000000}%
\pgfsetstrokecolor{currentstroke}%
\pgfsetstrokeopacity{0.700000}%
\pgfsetdash{}{0pt}%
\pgfpathmoveto{\pgfqpoint{8.108587in}{2.496858in}}%
\pgfpathcurveto{\pgfqpoint{8.113631in}{2.496858in}}{\pgfqpoint{8.118468in}{2.498862in}}{\pgfqpoint{8.122035in}{2.502428in}}%
\pgfpathcurveto{\pgfqpoint{8.125601in}{2.505995in}}{\pgfqpoint{8.127605in}{2.510832in}}{\pgfqpoint{8.127605in}{2.515876in}}%
\pgfpathcurveto{\pgfqpoint{8.127605in}{2.520920in}}{\pgfqpoint{8.125601in}{2.525758in}}{\pgfqpoint{8.122035in}{2.529324in}}%
\pgfpathcurveto{\pgfqpoint{8.118468in}{2.532890in}}{\pgfqpoint{8.113631in}{2.534894in}}{\pgfqpoint{8.108587in}{2.534894in}}%
\pgfpathcurveto{\pgfqpoint{8.103543in}{2.534894in}}{\pgfqpoint{8.098706in}{2.532890in}}{\pgfqpoint{8.095139in}{2.529324in}}%
\pgfpathcurveto{\pgfqpoint{8.091573in}{2.525758in}}{\pgfqpoint{8.089569in}{2.520920in}}{\pgfqpoint{8.089569in}{2.515876in}}%
\pgfpathcurveto{\pgfqpoint{8.089569in}{2.510832in}}{\pgfqpoint{8.091573in}{2.505995in}}{\pgfqpoint{8.095139in}{2.502428in}}%
\pgfpathcurveto{\pgfqpoint{8.098706in}{2.498862in}}{\pgfqpoint{8.103543in}{2.496858in}}{\pgfqpoint{8.108587in}{2.496858in}}%
\pgfpathclose%
\pgfusepath{fill}%
\end{pgfscope}%
\begin{pgfscope}%
\pgfpathrectangle{\pgfqpoint{6.572727in}{0.474100in}}{\pgfqpoint{4.227273in}{3.318700in}}%
\pgfusepath{clip}%
\pgfsetbuttcap%
\pgfsetroundjoin%
\definecolor{currentfill}{rgb}{0.127568,0.566949,0.550556}%
\pgfsetfillcolor{currentfill}%
\pgfsetfillopacity{0.700000}%
\pgfsetlinewidth{0.000000pt}%
\definecolor{currentstroke}{rgb}{0.000000,0.000000,0.000000}%
\pgfsetstrokecolor{currentstroke}%
\pgfsetstrokeopacity{0.700000}%
\pgfsetdash{}{0pt}%
\pgfpathmoveto{\pgfqpoint{10.210872in}{1.493131in}}%
\pgfpathcurveto{\pgfqpoint{10.215916in}{1.493131in}}{\pgfqpoint{10.220754in}{1.495135in}}{\pgfqpoint{10.224320in}{1.498701in}}%
\pgfpathcurveto{\pgfqpoint{10.227887in}{1.502268in}}{\pgfqpoint{10.229890in}{1.507105in}}{\pgfqpoint{10.229890in}{1.512149in}}%
\pgfpathcurveto{\pgfqpoint{10.229890in}{1.517193in}}{\pgfqpoint{10.227887in}{1.522030in}}{\pgfqpoint{10.224320in}{1.525597in}}%
\pgfpathcurveto{\pgfqpoint{10.220754in}{1.529163in}}{\pgfqpoint{10.215916in}{1.531167in}}{\pgfqpoint{10.210872in}{1.531167in}}%
\pgfpathcurveto{\pgfqpoint{10.205829in}{1.531167in}}{\pgfqpoint{10.200991in}{1.529163in}}{\pgfqpoint{10.197424in}{1.525597in}}%
\pgfpathcurveto{\pgfqpoint{10.193858in}{1.522030in}}{\pgfqpoint{10.191854in}{1.517193in}}{\pgfqpoint{10.191854in}{1.512149in}}%
\pgfpathcurveto{\pgfqpoint{10.191854in}{1.507105in}}{\pgfqpoint{10.193858in}{1.502268in}}{\pgfqpoint{10.197424in}{1.498701in}}%
\pgfpathcurveto{\pgfqpoint{10.200991in}{1.495135in}}{\pgfqpoint{10.205829in}{1.493131in}}{\pgfqpoint{10.210872in}{1.493131in}}%
\pgfpathclose%
\pgfusepath{fill}%
\end{pgfscope}%
\begin{pgfscope}%
\pgfpathrectangle{\pgfqpoint{6.572727in}{0.474100in}}{\pgfqpoint{4.227273in}{3.318700in}}%
\pgfusepath{clip}%
\pgfsetbuttcap%
\pgfsetroundjoin%
\definecolor{currentfill}{rgb}{0.267004,0.004874,0.329415}%
\pgfsetfillcolor{currentfill}%
\pgfsetfillopacity{0.700000}%
\pgfsetlinewidth{0.000000pt}%
\definecolor{currentstroke}{rgb}{0.000000,0.000000,0.000000}%
\pgfsetstrokecolor{currentstroke}%
\pgfsetstrokeopacity{0.700000}%
\pgfsetdash{}{0pt}%
\pgfpathmoveto{\pgfqpoint{7.602098in}{1.563613in}}%
\pgfpathcurveto{\pgfqpoint{7.607142in}{1.563613in}}{\pgfqpoint{7.611980in}{1.565617in}}{\pgfqpoint{7.615546in}{1.569184in}}%
\pgfpathcurveto{\pgfqpoint{7.619113in}{1.572750in}}{\pgfqpoint{7.621117in}{1.577588in}}{\pgfqpoint{7.621117in}{1.582631in}}%
\pgfpathcurveto{\pgfqpoint{7.621117in}{1.587675in}}{\pgfqpoint{7.619113in}{1.592513in}}{\pgfqpoint{7.615546in}{1.596079in}}%
\pgfpathcurveto{\pgfqpoint{7.611980in}{1.599646in}}{\pgfqpoint{7.607142in}{1.601650in}}{\pgfqpoint{7.602098in}{1.601650in}}%
\pgfpathcurveto{\pgfqpoint{7.597055in}{1.601650in}}{\pgfqpoint{7.592217in}{1.599646in}}{\pgfqpoint{7.588651in}{1.596079in}}%
\pgfpathcurveto{\pgfqpoint{7.585084in}{1.592513in}}{\pgfqpoint{7.583080in}{1.587675in}}{\pgfqpoint{7.583080in}{1.582631in}}%
\pgfpathcurveto{\pgfqpoint{7.583080in}{1.577588in}}{\pgfqpoint{7.585084in}{1.572750in}}{\pgfqpoint{7.588651in}{1.569184in}}%
\pgfpathcurveto{\pgfqpoint{7.592217in}{1.565617in}}{\pgfqpoint{7.597055in}{1.563613in}}{\pgfqpoint{7.602098in}{1.563613in}}%
\pgfpathclose%
\pgfusepath{fill}%
\end{pgfscope}%
\begin{pgfscope}%
\pgfpathrectangle{\pgfqpoint{6.572727in}{0.474100in}}{\pgfqpoint{4.227273in}{3.318700in}}%
\pgfusepath{clip}%
\pgfsetbuttcap%
\pgfsetroundjoin%
\definecolor{currentfill}{rgb}{0.267004,0.004874,0.329415}%
\pgfsetfillcolor{currentfill}%
\pgfsetfillopacity{0.700000}%
\pgfsetlinewidth{0.000000pt}%
\definecolor{currentstroke}{rgb}{0.000000,0.000000,0.000000}%
\pgfsetstrokecolor{currentstroke}%
\pgfsetstrokeopacity{0.700000}%
\pgfsetdash{}{0pt}%
\pgfpathmoveto{\pgfqpoint{8.272134in}{1.498709in}}%
\pgfpathcurveto{\pgfqpoint{8.277178in}{1.498709in}}{\pgfqpoint{8.282016in}{1.500713in}}{\pgfqpoint{8.285582in}{1.504279in}}%
\pgfpathcurveto{\pgfqpoint{8.289148in}{1.507845in}}{\pgfqpoint{8.291152in}{1.512683in}}{\pgfqpoint{8.291152in}{1.517727in}}%
\pgfpathcurveto{\pgfqpoint{8.291152in}{1.522771in}}{\pgfqpoint{8.289148in}{1.527608in}}{\pgfqpoint{8.285582in}{1.531175in}}%
\pgfpathcurveto{\pgfqpoint{8.282016in}{1.534741in}}{\pgfqpoint{8.277178in}{1.536745in}}{\pgfqpoint{8.272134in}{1.536745in}}%
\pgfpathcurveto{\pgfqpoint{8.267091in}{1.536745in}}{\pgfqpoint{8.262253in}{1.534741in}}{\pgfqpoint{8.258686in}{1.531175in}}%
\pgfpathcurveto{\pgfqpoint{8.255120in}{1.527608in}}{\pgfqpoint{8.253116in}{1.522771in}}{\pgfqpoint{8.253116in}{1.517727in}}%
\pgfpathcurveto{\pgfqpoint{8.253116in}{1.512683in}}{\pgfqpoint{8.255120in}{1.507845in}}{\pgfqpoint{8.258686in}{1.504279in}}%
\pgfpathcurveto{\pgfqpoint{8.262253in}{1.500713in}}{\pgfqpoint{8.267091in}{1.498709in}}{\pgfqpoint{8.272134in}{1.498709in}}%
\pgfpathclose%
\pgfusepath{fill}%
\end{pgfscope}%
\begin{pgfscope}%
\pgfpathrectangle{\pgfqpoint{6.572727in}{0.474100in}}{\pgfqpoint{4.227273in}{3.318700in}}%
\pgfusepath{clip}%
\pgfsetbuttcap%
\pgfsetroundjoin%
\definecolor{currentfill}{rgb}{0.993248,0.906157,0.143936}%
\pgfsetfillcolor{currentfill}%
\pgfsetfillopacity{0.700000}%
\pgfsetlinewidth{0.000000pt}%
\definecolor{currentstroke}{rgb}{0.000000,0.000000,0.000000}%
\pgfsetstrokecolor{currentstroke}%
\pgfsetstrokeopacity{0.700000}%
\pgfsetdash{}{0pt}%
\pgfpathmoveto{\pgfqpoint{8.312181in}{3.374278in}}%
\pgfpathcurveto{\pgfqpoint{8.317225in}{3.374278in}}{\pgfqpoint{8.322063in}{3.376282in}}{\pgfqpoint{8.325629in}{3.379849in}}%
\pgfpathcurveto{\pgfqpoint{8.329195in}{3.383415in}}{\pgfqpoint{8.331199in}{3.388253in}}{\pgfqpoint{8.331199in}{3.393296in}}%
\pgfpathcurveto{\pgfqpoint{8.331199in}{3.398340in}}{\pgfqpoint{8.329195in}{3.403178in}}{\pgfqpoint{8.325629in}{3.406744in}}%
\pgfpathcurveto{\pgfqpoint{8.322063in}{3.410311in}}{\pgfqpoint{8.317225in}{3.412315in}}{\pgfqpoint{8.312181in}{3.412315in}}%
\pgfpathcurveto{\pgfqpoint{8.307138in}{3.412315in}}{\pgfqpoint{8.302300in}{3.410311in}}{\pgfqpoint{8.298733in}{3.406744in}}%
\pgfpathcurveto{\pgfqpoint{8.295167in}{3.403178in}}{\pgfqpoint{8.293163in}{3.398340in}}{\pgfqpoint{8.293163in}{3.393296in}}%
\pgfpathcurveto{\pgfqpoint{8.293163in}{3.388253in}}{\pgfqpoint{8.295167in}{3.383415in}}{\pgfqpoint{8.298733in}{3.379849in}}%
\pgfpathcurveto{\pgfqpoint{8.302300in}{3.376282in}}{\pgfqpoint{8.307138in}{3.374278in}}{\pgfqpoint{8.312181in}{3.374278in}}%
\pgfpathclose%
\pgfusepath{fill}%
\end{pgfscope}%
\begin{pgfscope}%
\pgfpathrectangle{\pgfqpoint{6.572727in}{0.474100in}}{\pgfqpoint{4.227273in}{3.318700in}}%
\pgfusepath{clip}%
\pgfsetbuttcap%
\pgfsetroundjoin%
\definecolor{currentfill}{rgb}{0.267004,0.004874,0.329415}%
\pgfsetfillcolor{currentfill}%
\pgfsetfillopacity{0.700000}%
\pgfsetlinewidth{0.000000pt}%
\definecolor{currentstroke}{rgb}{0.000000,0.000000,0.000000}%
\pgfsetstrokecolor{currentstroke}%
\pgfsetstrokeopacity{0.700000}%
\pgfsetdash{}{0pt}%
\pgfpathmoveto{\pgfqpoint{7.728138in}{1.908250in}}%
\pgfpathcurveto{\pgfqpoint{7.733182in}{1.908250in}}{\pgfqpoint{7.738019in}{1.910254in}}{\pgfqpoint{7.741586in}{1.913820in}}%
\pgfpathcurveto{\pgfqpoint{7.745152in}{1.917386in}}{\pgfqpoint{7.747156in}{1.922224in}}{\pgfqpoint{7.747156in}{1.927268in}}%
\pgfpathcurveto{\pgfqpoint{7.747156in}{1.932312in}}{\pgfqpoint{7.745152in}{1.937149in}}{\pgfqpoint{7.741586in}{1.940716in}}%
\pgfpathcurveto{\pgfqpoint{7.738019in}{1.944282in}}{\pgfqpoint{7.733182in}{1.946286in}}{\pgfqpoint{7.728138in}{1.946286in}}%
\pgfpathcurveto{\pgfqpoint{7.723094in}{1.946286in}}{\pgfqpoint{7.718257in}{1.944282in}}{\pgfqpoint{7.714690in}{1.940716in}}%
\pgfpathcurveto{\pgfqpoint{7.711124in}{1.937149in}}{\pgfqpoint{7.709120in}{1.932312in}}{\pgfqpoint{7.709120in}{1.927268in}}%
\pgfpathcurveto{\pgfqpoint{7.709120in}{1.922224in}}{\pgfqpoint{7.711124in}{1.917386in}}{\pgfqpoint{7.714690in}{1.913820in}}%
\pgfpathcurveto{\pgfqpoint{7.718257in}{1.910254in}}{\pgfqpoint{7.723094in}{1.908250in}}{\pgfqpoint{7.728138in}{1.908250in}}%
\pgfpathclose%
\pgfusepath{fill}%
\end{pgfscope}%
\begin{pgfscope}%
\pgfpathrectangle{\pgfqpoint{6.572727in}{0.474100in}}{\pgfqpoint{4.227273in}{3.318700in}}%
\pgfusepath{clip}%
\pgfsetbuttcap%
\pgfsetroundjoin%
\definecolor{currentfill}{rgb}{0.993248,0.906157,0.143936}%
\pgfsetfillcolor{currentfill}%
\pgfsetfillopacity{0.700000}%
\pgfsetlinewidth{0.000000pt}%
\definecolor{currentstroke}{rgb}{0.000000,0.000000,0.000000}%
\pgfsetstrokecolor{currentstroke}%
\pgfsetstrokeopacity{0.700000}%
\pgfsetdash{}{0pt}%
\pgfpathmoveto{\pgfqpoint{8.401447in}{3.040945in}}%
\pgfpathcurveto{\pgfqpoint{8.406491in}{3.040945in}}{\pgfqpoint{8.411328in}{3.042949in}}{\pgfqpoint{8.414895in}{3.046515in}}%
\pgfpathcurveto{\pgfqpoint{8.418461in}{3.050082in}}{\pgfqpoint{8.420465in}{3.054919in}}{\pgfqpoint{8.420465in}{3.059963in}}%
\pgfpathcurveto{\pgfqpoint{8.420465in}{3.065007in}}{\pgfqpoint{8.418461in}{3.069845in}}{\pgfqpoint{8.414895in}{3.073411in}}%
\pgfpathcurveto{\pgfqpoint{8.411328in}{3.076977in}}{\pgfqpoint{8.406491in}{3.078981in}}{\pgfqpoint{8.401447in}{3.078981in}}%
\pgfpathcurveto{\pgfqpoint{8.396403in}{3.078981in}}{\pgfqpoint{8.391565in}{3.076977in}}{\pgfqpoint{8.387999in}{3.073411in}}%
\pgfpathcurveto{\pgfqpoint{8.384433in}{3.069845in}}{\pgfqpoint{8.382429in}{3.065007in}}{\pgfqpoint{8.382429in}{3.059963in}}%
\pgfpathcurveto{\pgfqpoint{8.382429in}{3.054919in}}{\pgfqpoint{8.384433in}{3.050082in}}{\pgfqpoint{8.387999in}{3.046515in}}%
\pgfpathcurveto{\pgfqpoint{8.391565in}{3.042949in}}{\pgfqpoint{8.396403in}{3.040945in}}{\pgfqpoint{8.401447in}{3.040945in}}%
\pgfpathclose%
\pgfusepath{fill}%
\end{pgfscope}%
\begin{pgfscope}%
\pgfpathrectangle{\pgfqpoint{6.572727in}{0.474100in}}{\pgfqpoint{4.227273in}{3.318700in}}%
\pgfusepath{clip}%
\pgfsetbuttcap%
\pgfsetroundjoin%
\definecolor{currentfill}{rgb}{0.127568,0.566949,0.550556}%
\pgfsetfillcolor{currentfill}%
\pgfsetfillopacity{0.700000}%
\pgfsetlinewidth{0.000000pt}%
\definecolor{currentstroke}{rgb}{0.000000,0.000000,0.000000}%
\pgfsetstrokecolor{currentstroke}%
\pgfsetstrokeopacity{0.700000}%
\pgfsetdash{}{0pt}%
\pgfpathmoveto{\pgfqpoint{9.217785in}{1.548221in}}%
\pgfpathcurveto{\pgfqpoint{9.222829in}{1.548221in}}{\pgfqpoint{9.227667in}{1.550225in}}{\pgfqpoint{9.231233in}{1.553792in}}%
\pgfpathcurveto{\pgfqpoint{9.234800in}{1.557358in}}{\pgfqpoint{9.236804in}{1.562196in}}{\pgfqpoint{9.236804in}{1.567239in}}%
\pgfpathcurveto{\pgfqpoint{9.236804in}{1.572283in}}{\pgfqpoint{9.234800in}{1.577121in}}{\pgfqpoint{9.231233in}{1.580687in}}%
\pgfpathcurveto{\pgfqpoint{9.227667in}{1.584254in}}{\pgfqpoint{9.222829in}{1.586258in}}{\pgfqpoint{9.217785in}{1.586258in}}%
\pgfpathcurveto{\pgfqpoint{9.212742in}{1.586258in}}{\pgfqpoint{9.207904in}{1.584254in}}{\pgfqpoint{9.204338in}{1.580687in}}%
\pgfpathcurveto{\pgfqpoint{9.200771in}{1.577121in}}{\pgfqpoint{9.198767in}{1.572283in}}{\pgfqpoint{9.198767in}{1.567239in}}%
\pgfpathcurveto{\pgfqpoint{9.198767in}{1.562196in}}{\pgfqpoint{9.200771in}{1.557358in}}{\pgfqpoint{9.204338in}{1.553792in}}%
\pgfpathcurveto{\pgfqpoint{9.207904in}{1.550225in}}{\pgfqpoint{9.212742in}{1.548221in}}{\pgfqpoint{9.217785in}{1.548221in}}%
\pgfpathclose%
\pgfusepath{fill}%
\end{pgfscope}%
\begin{pgfscope}%
\pgfpathrectangle{\pgfqpoint{6.572727in}{0.474100in}}{\pgfqpoint{4.227273in}{3.318700in}}%
\pgfusepath{clip}%
\pgfsetbuttcap%
\pgfsetroundjoin%
\definecolor{currentfill}{rgb}{0.267004,0.004874,0.329415}%
\pgfsetfillcolor{currentfill}%
\pgfsetfillopacity{0.700000}%
\pgfsetlinewidth{0.000000pt}%
\definecolor{currentstroke}{rgb}{0.000000,0.000000,0.000000}%
\pgfsetstrokecolor{currentstroke}%
\pgfsetstrokeopacity{0.700000}%
\pgfsetdash{}{0pt}%
\pgfpathmoveto{\pgfqpoint{6.764876in}{1.942669in}}%
\pgfpathcurveto{\pgfqpoint{6.769920in}{1.942669in}}{\pgfqpoint{6.774757in}{1.944673in}}{\pgfqpoint{6.778324in}{1.948239in}}%
\pgfpathcurveto{\pgfqpoint{6.781890in}{1.951805in}}{\pgfqpoint{6.783894in}{1.956643in}}{\pgfqpoint{6.783894in}{1.961687in}}%
\pgfpathcurveto{\pgfqpoint{6.783894in}{1.966731in}}{\pgfqpoint{6.781890in}{1.971568in}}{\pgfqpoint{6.778324in}{1.975135in}}%
\pgfpathcurveto{\pgfqpoint{6.774757in}{1.978701in}}{\pgfqpoint{6.769920in}{1.980705in}}{\pgfqpoint{6.764876in}{1.980705in}}%
\pgfpathcurveto{\pgfqpoint{6.759832in}{1.980705in}}{\pgfqpoint{6.754995in}{1.978701in}}{\pgfqpoint{6.751428in}{1.975135in}}%
\pgfpathcurveto{\pgfqpoint{6.747862in}{1.971568in}}{\pgfqpoint{6.745858in}{1.966731in}}{\pgfqpoint{6.745858in}{1.961687in}}%
\pgfpathcurveto{\pgfqpoint{6.745858in}{1.956643in}}{\pgfqpoint{6.747862in}{1.951805in}}{\pgfqpoint{6.751428in}{1.948239in}}%
\pgfpathcurveto{\pgfqpoint{6.754995in}{1.944673in}}{\pgfqpoint{6.759832in}{1.942669in}}{\pgfqpoint{6.764876in}{1.942669in}}%
\pgfpathclose%
\pgfusepath{fill}%
\end{pgfscope}%
\begin{pgfscope}%
\pgfpathrectangle{\pgfqpoint{6.572727in}{0.474100in}}{\pgfqpoint{4.227273in}{3.318700in}}%
\pgfusepath{clip}%
\pgfsetbuttcap%
\pgfsetroundjoin%
\definecolor{currentfill}{rgb}{0.267004,0.004874,0.329415}%
\pgfsetfillcolor{currentfill}%
\pgfsetfillopacity{0.700000}%
\pgfsetlinewidth{0.000000pt}%
\definecolor{currentstroke}{rgb}{0.000000,0.000000,0.000000}%
\pgfsetstrokecolor{currentstroke}%
\pgfsetstrokeopacity{0.700000}%
\pgfsetdash{}{0pt}%
\pgfpathmoveto{\pgfqpoint{7.804196in}{0.954788in}}%
\pgfpathcurveto{\pgfqpoint{7.809240in}{0.954788in}}{\pgfqpoint{7.814078in}{0.956792in}}{\pgfqpoint{7.817644in}{0.960358in}}%
\pgfpathcurveto{\pgfqpoint{7.821211in}{0.963925in}}{\pgfqpoint{7.823214in}{0.968762in}}{\pgfqpoint{7.823214in}{0.973806in}}%
\pgfpathcurveto{\pgfqpoint{7.823214in}{0.978850in}}{\pgfqpoint{7.821211in}{0.983688in}}{\pgfqpoint{7.817644in}{0.987254in}}%
\pgfpathcurveto{\pgfqpoint{7.814078in}{0.990820in}}{\pgfqpoint{7.809240in}{0.992824in}}{\pgfqpoint{7.804196in}{0.992824in}}%
\pgfpathcurveto{\pgfqpoint{7.799153in}{0.992824in}}{\pgfqpoint{7.794315in}{0.990820in}}{\pgfqpoint{7.790748in}{0.987254in}}%
\pgfpathcurveto{\pgfqpoint{7.787182in}{0.983688in}}{\pgfqpoint{7.785178in}{0.978850in}}{\pgfqpoint{7.785178in}{0.973806in}}%
\pgfpathcurveto{\pgfqpoint{7.785178in}{0.968762in}}{\pgfqpoint{7.787182in}{0.963925in}}{\pgfqpoint{7.790748in}{0.960358in}}%
\pgfpathcurveto{\pgfqpoint{7.794315in}{0.956792in}}{\pgfqpoint{7.799153in}{0.954788in}}{\pgfqpoint{7.804196in}{0.954788in}}%
\pgfpathclose%
\pgfusepath{fill}%
\end{pgfscope}%
\begin{pgfscope}%
\pgfpathrectangle{\pgfqpoint{6.572727in}{0.474100in}}{\pgfqpoint{4.227273in}{3.318700in}}%
\pgfusepath{clip}%
\pgfsetbuttcap%
\pgfsetroundjoin%
\definecolor{currentfill}{rgb}{0.993248,0.906157,0.143936}%
\pgfsetfillcolor{currentfill}%
\pgfsetfillopacity{0.700000}%
\pgfsetlinewidth{0.000000pt}%
\definecolor{currentstroke}{rgb}{0.000000,0.000000,0.000000}%
\pgfsetstrokecolor{currentstroke}%
\pgfsetstrokeopacity{0.700000}%
\pgfsetdash{}{0pt}%
\pgfpathmoveto{\pgfqpoint{8.931081in}{3.117782in}}%
\pgfpathcurveto{\pgfqpoint{8.936125in}{3.117782in}}{\pgfqpoint{8.940963in}{3.119786in}}{\pgfqpoint{8.944529in}{3.123352in}}%
\pgfpathcurveto{\pgfqpoint{8.948096in}{3.126919in}}{\pgfqpoint{8.950100in}{3.131757in}}{\pgfqpoint{8.950100in}{3.136800in}}%
\pgfpathcurveto{\pgfqpoint{8.950100in}{3.141844in}}{\pgfqpoint{8.948096in}{3.146682in}}{\pgfqpoint{8.944529in}{3.150248in}}%
\pgfpathcurveto{\pgfqpoint{8.940963in}{3.153815in}}{\pgfqpoint{8.936125in}{3.155818in}}{\pgfqpoint{8.931081in}{3.155818in}}%
\pgfpathcurveto{\pgfqpoint{8.926038in}{3.155818in}}{\pgfqpoint{8.921200in}{3.153815in}}{\pgfqpoint{8.917634in}{3.150248in}}%
\pgfpathcurveto{\pgfqpoint{8.914067in}{3.146682in}}{\pgfqpoint{8.912063in}{3.141844in}}{\pgfqpoint{8.912063in}{3.136800in}}%
\pgfpathcurveto{\pgfqpoint{8.912063in}{3.131757in}}{\pgfqpoint{8.914067in}{3.126919in}}{\pgfqpoint{8.917634in}{3.123352in}}%
\pgfpathcurveto{\pgfqpoint{8.921200in}{3.119786in}}{\pgfqpoint{8.926038in}{3.117782in}}{\pgfqpoint{8.931081in}{3.117782in}}%
\pgfpathclose%
\pgfusepath{fill}%
\end{pgfscope}%
\begin{pgfscope}%
\pgfpathrectangle{\pgfqpoint{6.572727in}{0.474100in}}{\pgfqpoint{4.227273in}{3.318700in}}%
\pgfusepath{clip}%
\pgfsetbuttcap%
\pgfsetroundjoin%
\definecolor{currentfill}{rgb}{0.993248,0.906157,0.143936}%
\pgfsetfillcolor{currentfill}%
\pgfsetfillopacity{0.700000}%
\pgfsetlinewidth{0.000000pt}%
\definecolor{currentstroke}{rgb}{0.000000,0.000000,0.000000}%
\pgfsetstrokecolor{currentstroke}%
\pgfsetstrokeopacity{0.700000}%
\pgfsetdash{}{0pt}%
\pgfpathmoveto{\pgfqpoint{8.546253in}{3.269991in}}%
\pgfpathcurveto{\pgfqpoint{8.551297in}{3.269991in}}{\pgfqpoint{8.556135in}{3.271995in}}{\pgfqpoint{8.559701in}{3.275562in}}%
\pgfpathcurveto{\pgfqpoint{8.563267in}{3.279128in}}{\pgfqpoint{8.565271in}{3.283966in}}{\pgfqpoint{8.565271in}{3.289010in}}%
\pgfpathcurveto{\pgfqpoint{8.565271in}{3.294053in}}{\pgfqpoint{8.563267in}{3.298891in}}{\pgfqpoint{8.559701in}{3.302457in}}%
\pgfpathcurveto{\pgfqpoint{8.556135in}{3.306024in}}{\pgfqpoint{8.551297in}{3.308028in}}{\pgfqpoint{8.546253in}{3.308028in}}%
\pgfpathcurveto{\pgfqpoint{8.541210in}{3.308028in}}{\pgfqpoint{8.536372in}{3.306024in}}{\pgfqpoint{8.532805in}{3.302457in}}%
\pgfpathcurveto{\pgfqpoint{8.529239in}{3.298891in}}{\pgfqpoint{8.527235in}{3.294053in}}{\pgfqpoint{8.527235in}{3.289010in}}%
\pgfpathcurveto{\pgfqpoint{8.527235in}{3.283966in}}{\pgfqpoint{8.529239in}{3.279128in}}{\pgfqpoint{8.532805in}{3.275562in}}%
\pgfpathcurveto{\pgfqpoint{8.536372in}{3.271995in}}{\pgfqpoint{8.541210in}{3.269991in}}{\pgfqpoint{8.546253in}{3.269991in}}%
\pgfpathclose%
\pgfusepath{fill}%
\end{pgfscope}%
\begin{pgfscope}%
\pgfpathrectangle{\pgfqpoint{6.572727in}{0.474100in}}{\pgfqpoint{4.227273in}{3.318700in}}%
\pgfusepath{clip}%
\pgfsetbuttcap%
\pgfsetroundjoin%
\definecolor{currentfill}{rgb}{0.993248,0.906157,0.143936}%
\pgfsetfillcolor{currentfill}%
\pgfsetfillopacity{0.700000}%
\pgfsetlinewidth{0.000000pt}%
\definecolor{currentstroke}{rgb}{0.000000,0.000000,0.000000}%
\pgfsetstrokecolor{currentstroke}%
\pgfsetstrokeopacity{0.700000}%
\pgfsetdash{}{0pt}%
\pgfpathmoveto{\pgfqpoint{8.474563in}{2.276731in}}%
\pgfpathcurveto{\pgfqpoint{8.479607in}{2.276731in}}{\pgfqpoint{8.484445in}{2.278735in}}{\pgfqpoint{8.488011in}{2.282301in}}%
\pgfpathcurveto{\pgfqpoint{8.491578in}{2.285868in}}{\pgfqpoint{8.493581in}{2.290705in}}{\pgfqpoint{8.493581in}{2.295749in}}%
\pgfpathcurveto{\pgfqpoint{8.493581in}{2.300793in}}{\pgfqpoint{8.491578in}{2.305631in}}{\pgfqpoint{8.488011in}{2.309197in}}%
\pgfpathcurveto{\pgfqpoint{8.484445in}{2.312763in}}{\pgfqpoint{8.479607in}{2.314767in}}{\pgfqpoint{8.474563in}{2.314767in}}%
\pgfpathcurveto{\pgfqpoint{8.469520in}{2.314767in}}{\pgfqpoint{8.464682in}{2.312763in}}{\pgfqpoint{8.461115in}{2.309197in}}%
\pgfpathcurveto{\pgfqpoint{8.457549in}{2.305631in}}{\pgfqpoint{8.455545in}{2.300793in}}{\pgfqpoint{8.455545in}{2.295749in}}%
\pgfpathcurveto{\pgfqpoint{8.455545in}{2.290705in}}{\pgfqpoint{8.457549in}{2.285868in}}{\pgfqpoint{8.461115in}{2.282301in}}%
\pgfpathcurveto{\pgfqpoint{8.464682in}{2.278735in}}{\pgfqpoint{8.469520in}{2.276731in}}{\pgfqpoint{8.474563in}{2.276731in}}%
\pgfpathclose%
\pgfusepath{fill}%
\end{pgfscope}%
\begin{pgfscope}%
\pgfpathrectangle{\pgfqpoint{6.572727in}{0.474100in}}{\pgfqpoint{4.227273in}{3.318700in}}%
\pgfusepath{clip}%
\pgfsetbuttcap%
\pgfsetroundjoin%
\definecolor{currentfill}{rgb}{0.127568,0.566949,0.550556}%
\pgfsetfillcolor{currentfill}%
\pgfsetfillopacity{0.700000}%
\pgfsetlinewidth{0.000000pt}%
\definecolor{currentstroke}{rgb}{0.000000,0.000000,0.000000}%
\pgfsetstrokecolor{currentstroke}%
\pgfsetstrokeopacity{0.700000}%
\pgfsetdash{}{0pt}%
\pgfpathmoveto{\pgfqpoint{9.510394in}{1.409418in}}%
\pgfpathcurveto{\pgfqpoint{9.515437in}{1.409418in}}{\pgfqpoint{9.520275in}{1.411422in}}{\pgfqpoint{9.523841in}{1.414989in}}%
\pgfpathcurveto{\pgfqpoint{9.527408in}{1.418555in}}{\pgfqpoint{9.529412in}{1.423393in}}{\pgfqpoint{9.529412in}{1.428436in}}%
\pgfpathcurveto{\pgfqpoint{9.529412in}{1.433480in}}{\pgfqpoint{9.527408in}{1.438318in}}{\pgfqpoint{9.523841in}{1.441884in}}%
\pgfpathcurveto{\pgfqpoint{9.520275in}{1.445451in}}{\pgfqpoint{9.515437in}{1.447455in}}{\pgfqpoint{9.510394in}{1.447455in}}%
\pgfpathcurveto{\pgfqpoint{9.505350in}{1.447455in}}{\pgfqpoint{9.500512in}{1.445451in}}{\pgfqpoint{9.496946in}{1.441884in}}%
\pgfpathcurveto{\pgfqpoint{9.493379in}{1.438318in}}{\pgfqpoint{9.491375in}{1.433480in}}{\pgfqpoint{9.491375in}{1.428436in}}%
\pgfpathcurveto{\pgfqpoint{9.491375in}{1.423393in}}{\pgfqpoint{9.493379in}{1.418555in}}{\pgfqpoint{9.496946in}{1.414989in}}%
\pgfpathcurveto{\pgfqpoint{9.500512in}{1.411422in}}{\pgfqpoint{9.505350in}{1.409418in}}{\pgfqpoint{9.510394in}{1.409418in}}%
\pgfpathclose%
\pgfusepath{fill}%
\end{pgfscope}%
\begin{pgfscope}%
\pgfpathrectangle{\pgfqpoint{6.572727in}{0.474100in}}{\pgfqpoint{4.227273in}{3.318700in}}%
\pgfusepath{clip}%
\pgfsetbuttcap%
\pgfsetroundjoin%
\definecolor{currentfill}{rgb}{0.267004,0.004874,0.329415}%
\pgfsetfillcolor{currentfill}%
\pgfsetfillopacity{0.700000}%
\pgfsetlinewidth{0.000000pt}%
\definecolor{currentstroke}{rgb}{0.000000,0.000000,0.000000}%
\pgfsetstrokecolor{currentstroke}%
\pgfsetstrokeopacity{0.700000}%
\pgfsetdash{}{0pt}%
\pgfpathmoveto{\pgfqpoint{7.520916in}{1.256808in}}%
\pgfpathcurveto{\pgfqpoint{7.525959in}{1.256808in}}{\pgfqpoint{7.530797in}{1.258812in}}{\pgfqpoint{7.534363in}{1.262378in}}%
\pgfpathcurveto{\pgfqpoint{7.537930in}{1.265945in}}{\pgfqpoint{7.539934in}{1.270782in}}{\pgfqpoint{7.539934in}{1.275826in}}%
\pgfpathcurveto{\pgfqpoint{7.539934in}{1.280870in}}{\pgfqpoint{7.537930in}{1.285707in}}{\pgfqpoint{7.534363in}{1.289274in}}%
\pgfpathcurveto{\pgfqpoint{7.530797in}{1.292840in}}{\pgfqpoint{7.525959in}{1.294844in}}{\pgfqpoint{7.520916in}{1.294844in}}%
\pgfpathcurveto{\pgfqpoint{7.515872in}{1.294844in}}{\pgfqpoint{7.511034in}{1.292840in}}{\pgfqpoint{7.507468in}{1.289274in}}%
\pgfpathcurveto{\pgfqpoint{7.503901in}{1.285707in}}{\pgfqpoint{7.501897in}{1.280870in}}{\pgfqpoint{7.501897in}{1.275826in}}%
\pgfpathcurveto{\pgfqpoint{7.501897in}{1.270782in}}{\pgfqpoint{7.503901in}{1.265945in}}{\pgfqpoint{7.507468in}{1.262378in}}%
\pgfpathcurveto{\pgfqpoint{7.511034in}{1.258812in}}{\pgfqpoint{7.515872in}{1.256808in}}{\pgfqpoint{7.520916in}{1.256808in}}%
\pgfpathclose%
\pgfusepath{fill}%
\end{pgfscope}%
\begin{pgfscope}%
\pgfpathrectangle{\pgfqpoint{6.572727in}{0.474100in}}{\pgfqpoint{4.227273in}{3.318700in}}%
\pgfusepath{clip}%
\pgfsetbuttcap%
\pgfsetroundjoin%
\definecolor{currentfill}{rgb}{0.267004,0.004874,0.329415}%
\pgfsetfillcolor{currentfill}%
\pgfsetfillopacity{0.700000}%
\pgfsetlinewidth{0.000000pt}%
\definecolor{currentstroke}{rgb}{0.000000,0.000000,0.000000}%
\pgfsetstrokecolor{currentstroke}%
\pgfsetstrokeopacity{0.700000}%
\pgfsetdash{}{0pt}%
\pgfpathmoveto{\pgfqpoint{7.748101in}{1.562919in}}%
\pgfpathcurveto{\pgfqpoint{7.753145in}{1.562919in}}{\pgfqpoint{7.757983in}{1.564922in}}{\pgfqpoint{7.761549in}{1.568489in}}%
\pgfpathcurveto{\pgfqpoint{7.765116in}{1.572055in}}{\pgfqpoint{7.767120in}{1.576893in}}{\pgfqpoint{7.767120in}{1.581937in}}%
\pgfpathcurveto{\pgfqpoint{7.767120in}{1.586980in}}{\pgfqpoint{7.765116in}{1.591818in}}{\pgfqpoint{7.761549in}{1.595385in}}%
\pgfpathcurveto{\pgfqpoint{7.757983in}{1.598951in}}{\pgfqpoint{7.753145in}{1.600955in}}{\pgfqpoint{7.748101in}{1.600955in}}%
\pgfpathcurveto{\pgfqpoint{7.743058in}{1.600955in}}{\pgfqpoint{7.738220in}{1.598951in}}{\pgfqpoint{7.734654in}{1.595385in}}%
\pgfpathcurveto{\pgfqpoint{7.731087in}{1.591818in}}{\pgfqpoint{7.729083in}{1.586980in}}{\pgfqpoint{7.729083in}{1.581937in}}%
\pgfpathcurveto{\pgfqpoint{7.729083in}{1.576893in}}{\pgfqpoint{7.731087in}{1.572055in}}{\pgfqpoint{7.734654in}{1.568489in}}%
\pgfpathcurveto{\pgfqpoint{7.738220in}{1.564922in}}{\pgfqpoint{7.743058in}{1.562919in}}{\pgfqpoint{7.748101in}{1.562919in}}%
\pgfpathclose%
\pgfusepath{fill}%
\end{pgfscope}%
\begin{pgfscope}%
\pgfpathrectangle{\pgfqpoint{6.572727in}{0.474100in}}{\pgfqpoint{4.227273in}{3.318700in}}%
\pgfusepath{clip}%
\pgfsetbuttcap%
\pgfsetroundjoin%
\definecolor{currentfill}{rgb}{0.993248,0.906157,0.143936}%
\pgfsetfillcolor{currentfill}%
\pgfsetfillopacity{0.700000}%
\pgfsetlinewidth{0.000000pt}%
\definecolor{currentstroke}{rgb}{0.000000,0.000000,0.000000}%
\pgfsetstrokecolor{currentstroke}%
\pgfsetstrokeopacity{0.700000}%
\pgfsetdash{}{0pt}%
\pgfpathmoveto{\pgfqpoint{8.681375in}{2.920236in}}%
\pgfpathcurveto{\pgfqpoint{8.686419in}{2.920236in}}{\pgfqpoint{8.691256in}{2.922240in}}{\pgfqpoint{8.694823in}{2.925807in}}%
\pgfpathcurveto{\pgfqpoint{8.698389in}{2.929373in}}{\pgfqpoint{8.700393in}{2.934211in}}{\pgfqpoint{8.700393in}{2.939255in}}%
\pgfpathcurveto{\pgfqpoint{8.700393in}{2.944298in}}{\pgfqpoint{8.698389in}{2.949136in}}{\pgfqpoint{8.694823in}{2.952702in}}%
\pgfpathcurveto{\pgfqpoint{8.691256in}{2.956269in}}{\pgfqpoint{8.686419in}{2.958273in}}{\pgfqpoint{8.681375in}{2.958273in}}%
\pgfpathcurveto{\pgfqpoint{8.676331in}{2.958273in}}{\pgfqpoint{8.671493in}{2.956269in}}{\pgfqpoint{8.667927in}{2.952702in}}%
\pgfpathcurveto{\pgfqpoint{8.664361in}{2.949136in}}{\pgfqpoint{8.662357in}{2.944298in}}{\pgfqpoint{8.662357in}{2.939255in}}%
\pgfpathcurveto{\pgfqpoint{8.662357in}{2.934211in}}{\pgfqpoint{8.664361in}{2.929373in}}{\pgfqpoint{8.667927in}{2.925807in}}%
\pgfpathcurveto{\pgfqpoint{8.671493in}{2.922240in}}{\pgfqpoint{8.676331in}{2.920236in}}{\pgfqpoint{8.681375in}{2.920236in}}%
\pgfpathclose%
\pgfusepath{fill}%
\end{pgfscope}%
\begin{pgfscope}%
\pgfpathrectangle{\pgfqpoint{6.572727in}{0.474100in}}{\pgfqpoint{4.227273in}{3.318700in}}%
\pgfusepath{clip}%
\pgfsetbuttcap%
\pgfsetroundjoin%
\definecolor{currentfill}{rgb}{0.127568,0.566949,0.550556}%
\pgfsetfillcolor{currentfill}%
\pgfsetfillopacity{0.700000}%
\pgfsetlinewidth{0.000000pt}%
\definecolor{currentstroke}{rgb}{0.000000,0.000000,0.000000}%
\pgfsetstrokecolor{currentstroke}%
\pgfsetstrokeopacity{0.700000}%
\pgfsetdash{}{0pt}%
\pgfpathmoveto{\pgfqpoint{9.373092in}{1.367219in}}%
\pgfpathcurveto{\pgfqpoint{9.378136in}{1.367219in}}{\pgfqpoint{9.382973in}{1.369223in}}{\pgfqpoint{9.386540in}{1.372789in}}%
\pgfpathcurveto{\pgfqpoint{9.390106in}{1.376355in}}{\pgfqpoint{9.392110in}{1.381193in}}{\pgfqpoint{9.392110in}{1.386237in}}%
\pgfpathcurveto{\pgfqpoint{9.392110in}{1.391280in}}{\pgfqpoint{9.390106in}{1.396118in}}{\pgfqpoint{9.386540in}{1.399685in}}%
\pgfpathcurveto{\pgfqpoint{9.382973in}{1.403251in}}{\pgfqpoint{9.378136in}{1.405255in}}{\pgfqpoint{9.373092in}{1.405255in}}%
\pgfpathcurveto{\pgfqpoint{9.368048in}{1.405255in}}{\pgfqpoint{9.363210in}{1.403251in}}{\pgfqpoint{9.359644in}{1.399685in}}%
\pgfpathcurveto{\pgfqpoint{9.356078in}{1.396118in}}{\pgfqpoint{9.354074in}{1.391280in}}{\pgfqpoint{9.354074in}{1.386237in}}%
\pgfpathcurveto{\pgfqpoint{9.354074in}{1.381193in}}{\pgfqpoint{9.356078in}{1.376355in}}{\pgfqpoint{9.359644in}{1.372789in}}%
\pgfpathcurveto{\pgfqpoint{9.363210in}{1.369223in}}{\pgfqpoint{9.368048in}{1.367219in}}{\pgfqpoint{9.373092in}{1.367219in}}%
\pgfpathclose%
\pgfusepath{fill}%
\end{pgfscope}%
\begin{pgfscope}%
\pgfpathrectangle{\pgfqpoint{6.572727in}{0.474100in}}{\pgfqpoint{4.227273in}{3.318700in}}%
\pgfusepath{clip}%
\pgfsetbuttcap%
\pgfsetroundjoin%
\definecolor{currentfill}{rgb}{0.993248,0.906157,0.143936}%
\pgfsetfillcolor{currentfill}%
\pgfsetfillopacity{0.700000}%
\pgfsetlinewidth{0.000000pt}%
\definecolor{currentstroke}{rgb}{0.000000,0.000000,0.000000}%
\pgfsetstrokecolor{currentstroke}%
\pgfsetstrokeopacity{0.700000}%
\pgfsetdash{}{0pt}%
\pgfpathmoveto{\pgfqpoint{8.099575in}{2.425162in}}%
\pgfpathcurveto{\pgfqpoint{8.104618in}{2.425162in}}{\pgfqpoint{8.109456in}{2.427166in}}{\pgfqpoint{8.113023in}{2.430732in}}%
\pgfpathcurveto{\pgfqpoint{8.116589in}{2.434299in}}{\pgfqpoint{8.118593in}{2.439137in}}{\pgfqpoint{8.118593in}{2.444180in}}%
\pgfpathcurveto{\pgfqpoint{8.118593in}{2.449224in}}{\pgfqpoint{8.116589in}{2.454062in}}{\pgfqpoint{8.113023in}{2.457628in}}%
\pgfpathcurveto{\pgfqpoint{8.109456in}{2.461194in}}{\pgfqpoint{8.104618in}{2.463198in}}{\pgfqpoint{8.099575in}{2.463198in}}%
\pgfpathcurveto{\pgfqpoint{8.094531in}{2.463198in}}{\pgfqpoint{8.089693in}{2.461194in}}{\pgfqpoint{8.086127in}{2.457628in}}%
\pgfpathcurveto{\pgfqpoint{8.082561in}{2.454062in}}{\pgfqpoint{8.080557in}{2.449224in}}{\pgfqpoint{8.080557in}{2.444180in}}%
\pgfpathcurveto{\pgfqpoint{8.080557in}{2.439137in}}{\pgfqpoint{8.082561in}{2.434299in}}{\pgfqpoint{8.086127in}{2.430732in}}%
\pgfpathcurveto{\pgfqpoint{8.089693in}{2.427166in}}{\pgfqpoint{8.094531in}{2.425162in}}{\pgfqpoint{8.099575in}{2.425162in}}%
\pgfpathclose%
\pgfusepath{fill}%
\end{pgfscope}%
\begin{pgfscope}%
\pgfpathrectangle{\pgfqpoint{6.572727in}{0.474100in}}{\pgfqpoint{4.227273in}{3.318700in}}%
\pgfusepath{clip}%
\pgfsetbuttcap%
\pgfsetroundjoin%
\definecolor{currentfill}{rgb}{0.993248,0.906157,0.143936}%
\pgfsetfillcolor{currentfill}%
\pgfsetfillopacity{0.700000}%
\pgfsetlinewidth{0.000000pt}%
\definecolor{currentstroke}{rgb}{0.000000,0.000000,0.000000}%
\pgfsetstrokecolor{currentstroke}%
\pgfsetstrokeopacity{0.700000}%
\pgfsetdash{}{0pt}%
\pgfpathmoveto{\pgfqpoint{7.860710in}{3.220093in}}%
\pgfpathcurveto{\pgfqpoint{7.865754in}{3.220093in}}{\pgfqpoint{7.870592in}{3.222097in}}{\pgfqpoint{7.874158in}{3.225663in}}%
\pgfpathcurveto{\pgfqpoint{7.877725in}{3.229229in}}{\pgfqpoint{7.879728in}{3.234067in}}{\pgfqpoint{7.879728in}{3.239111in}}%
\pgfpathcurveto{\pgfqpoint{7.879728in}{3.244154in}}{\pgfqpoint{7.877725in}{3.248992in}}{\pgfqpoint{7.874158in}{3.252559in}}%
\pgfpathcurveto{\pgfqpoint{7.870592in}{3.256125in}}{\pgfqpoint{7.865754in}{3.258129in}}{\pgfqpoint{7.860710in}{3.258129in}}%
\pgfpathcurveto{\pgfqpoint{7.855667in}{3.258129in}}{\pgfqpoint{7.850829in}{3.256125in}}{\pgfqpoint{7.847262in}{3.252559in}}%
\pgfpathcurveto{\pgfqpoint{7.843696in}{3.248992in}}{\pgfqpoint{7.841692in}{3.244154in}}{\pgfqpoint{7.841692in}{3.239111in}}%
\pgfpathcurveto{\pgfqpoint{7.841692in}{3.234067in}}{\pgfqpoint{7.843696in}{3.229229in}}{\pgfqpoint{7.847262in}{3.225663in}}%
\pgfpathcurveto{\pgfqpoint{7.850829in}{3.222097in}}{\pgfqpoint{7.855667in}{3.220093in}}{\pgfqpoint{7.860710in}{3.220093in}}%
\pgfpathclose%
\pgfusepath{fill}%
\end{pgfscope}%
\begin{pgfscope}%
\pgfpathrectangle{\pgfqpoint{6.572727in}{0.474100in}}{\pgfqpoint{4.227273in}{3.318700in}}%
\pgfusepath{clip}%
\pgfsetbuttcap%
\pgfsetroundjoin%
\definecolor{currentfill}{rgb}{0.267004,0.004874,0.329415}%
\pgfsetfillcolor{currentfill}%
\pgfsetfillopacity{0.700000}%
\pgfsetlinewidth{0.000000pt}%
\definecolor{currentstroke}{rgb}{0.000000,0.000000,0.000000}%
\pgfsetstrokecolor{currentstroke}%
\pgfsetstrokeopacity{0.700000}%
\pgfsetdash{}{0pt}%
\pgfpathmoveto{\pgfqpoint{7.780012in}{1.831899in}}%
\pgfpathcurveto{\pgfqpoint{7.785056in}{1.831899in}}{\pgfqpoint{7.789894in}{1.833902in}}{\pgfqpoint{7.793460in}{1.837469in}}%
\pgfpathcurveto{\pgfqpoint{7.797027in}{1.841035in}}{\pgfqpoint{7.799030in}{1.845873in}}{\pgfqpoint{7.799030in}{1.850917in}}%
\pgfpathcurveto{\pgfqpoint{7.799030in}{1.855960in}}{\pgfqpoint{7.797027in}{1.860798in}}{\pgfqpoint{7.793460in}{1.864365in}}%
\pgfpathcurveto{\pgfqpoint{7.789894in}{1.867931in}}{\pgfqpoint{7.785056in}{1.869935in}}{\pgfqpoint{7.780012in}{1.869935in}}%
\pgfpathcurveto{\pgfqpoint{7.774969in}{1.869935in}}{\pgfqpoint{7.770131in}{1.867931in}}{\pgfqpoint{7.766564in}{1.864365in}}%
\pgfpathcurveto{\pgfqpoint{7.762998in}{1.860798in}}{\pgfqpoint{7.760994in}{1.855960in}}{\pgfqpoint{7.760994in}{1.850917in}}%
\pgfpathcurveto{\pgfqpoint{7.760994in}{1.845873in}}{\pgfqpoint{7.762998in}{1.841035in}}{\pgfqpoint{7.766564in}{1.837469in}}%
\pgfpathcurveto{\pgfqpoint{7.770131in}{1.833902in}}{\pgfqpoint{7.774969in}{1.831899in}}{\pgfqpoint{7.780012in}{1.831899in}}%
\pgfpathclose%
\pgfusepath{fill}%
\end{pgfscope}%
\begin{pgfscope}%
\pgfpathrectangle{\pgfqpoint{6.572727in}{0.474100in}}{\pgfqpoint{4.227273in}{3.318700in}}%
\pgfusepath{clip}%
\pgfsetbuttcap%
\pgfsetroundjoin%
\definecolor{currentfill}{rgb}{0.127568,0.566949,0.550556}%
\pgfsetfillcolor{currentfill}%
\pgfsetfillopacity{0.700000}%
\pgfsetlinewidth{0.000000pt}%
\definecolor{currentstroke}{rgb}{0.000000,0.000000,0.000000}%
\pgfsetstrokecolor{currentstroke}%
\pgfsetstrokeopacity{0.700000}%
\pgfsetdash{}{0pt}%
\pgfpathmoveto{\pgfqpoint{9.546283in}{2.084393in}}%
\pgfpathcurveto{\pgfqpoint{9.551327in}{2.084393in}}{\pgfqpoint{9.556165in}{2.086397in}}{\pgfqpoint{9.559731in}{2.089964in}}%
\pgfpathcurveto{\pgfqpoint{9.563298in}{2.093530in}}{\pgfqpoint{9.565302in}{2.098368in}}{\pgfqpoint{9.565302in}{2.103411in}}%
\pgfpathcurveto{\pgfqpoint{9.565302in}{2.108455in}}{\pgfqpoint{9.563298in}{2.113293in}}{\pgfqpoint{9.559731in}{2.116859in}}%
\pgfpathcurveto{\pgfqpoint{9.556165in}{2.120426in}}{\pgfqpoint{9.551327in}{2.122430in}}{\pgfqpoint{9.546283in}{2.122430in}}%
\pgfpathcurveto{\pgfqpoint{9.541240in}{2.122430in}}{\pgfqpoint{9.536402in}{2.120426in}}{\pgfqpoint{9.532836in}{2.116859in}}%
\pgfpathcurveto{\pgfqpoint{9.529269in}{2.113293in}}{\pgfqpoint{9.527265in}{2.108455in}}{\pgfqpoint{9.527265in}{2.103411in}}%
\pgfpathcurveto{\pgfqpoint{9.527265in}{2.098368in}}{\pgfqpoint{9.529269in}{2.093530in}}{\pgfqpoint{9.532836in}{2.089964in}}%
\pgfpathcurveto{\pgfqpoint{9.536402in}{2.086397in}}{\pgfqpoint{9.541240in}{2.084393in}}{\pgfqpoint{9.546283in}{2.084393in}}%
\pgfpathclose%
\pgfusepath{fill}%
\end{pgfscope}%
\begin{pgfscope}%
\pgfpathrectangle{\pgfqpoint{6.572727in}{0.474100in}}{\pgfqpoint{4.227273in}{3.318700in}}%
\pgfusepath{clip}%
\pgfsetbuttcap%
\pgfsetroundjoin%
\definecolor{currentfill}{rgb}{0.993248,0.906157,0.143936}%
\pgfsetfillcolor{currentfill}%
\pgfsetfillopacity{0.700000}%
\pgfsetlinewidth{0.000000pt}%
\definecolor{currentstroke}{rgb}{0.000000,0.000000,0.000000}%
\pgfsetstrokecolor{currentstroke}%
\pgfsetstrokeopacity{0.700000}%
\pgfsetdash{}{0pt}%
\pgfpathmoveto{\pgfqpoint{8.109146in}{2.733683in}}%
\pgfpathcurveto{\pgfqpoint{8.114190in}{2.733683in}}{\pgfqpoint{8.119028in}{2.735687in}}{\pgfqpoint{8.122594in}{2.739253in}}%
\pgfpathcurveto{\pgfqpoint{8.126161in}{2.742820in}}{\pgfqpoint{8.128164in}{2.747657in}}{\pgfqpoint{8.128164in}{2.752701in}}%
\pgfpathcurveto{\pgfqpoint{8.128164in}{2.757745in}}{\pgfqpoint{8.126161in}{2.762583in}}{\pgfqpoint{8.122594in}{2.766149in}}%
\pgfpathcurveto{\pgfqpoint{8.119028in}{2.769715in}}{\pgfqpoint{8.114190in}{2.771719in}}{\pgfqpoint{8.109146in}{2.771719in}}%
\pgfpathcurveto{\pgfqpoint{8.104103in}{2.771719in}}{\pgfqpoint{8.099265in}{2.769715in}}{\pgfqpoint{8.095698in}{2.766149in}}%
\pgfpathcurveto{\pgfqpoint{8.092132in}{2.762583in}}{\pgfqpoint{8.090128in}{2.757745in}}{\pgfqpoint{8.090128in}{2.752701in}}%
\pgfpathcurveto{\pgfqpoint{8.090128in}{2.747657in}}{\pgfqpoint{8.092132in}{2.742820in}}{\pgfqpoint{8.095698in}{2.739253in}}%
\pgfpathcurveto{\pgfqpoint{8.099265in}{2.735687in}}{\pgfqpoint{8.104103in}{2.733683in}}{\pgfqpoint{8.109146in}{2.733683in}}%
\pgfpathclose%
\pgfusepath{fill}%
\end{pgfscope}%
\begin{pgfscope}%
\pgfpathrectangle{\pgfqpoint{6.572727in}{0.474100in}}{\pgfqpoint{4.227273in}{3.318700in}}%
\pgfusepath{clip}%
\pgfsetbuttcap%
\pgfsetroundjoin%
\definecolor{currentfill}{rgb}{0.267004,0.004874,0.329415}%
\pgfsetfillcolor{currentfill}%
\pgfsetfillopacity{0.700000}%
\pgfsetlinewidth{0.000000pt}%
\definecolor{currentstroke}{rgb}{0.000000,0.000000,0.000000}%
\pgfsetstrokecolor{currentstroke}%
\pgfsetstrokeopacity{0.700000}%
\pgfsetdash{}{0pt}%
\pgfpathmoveto{\pgfqpoint{7.675055in}{1.437400in}}%
\pgfpathcurveto{\pgfqpoint{7.680099in}{1.437400in}}{\pgfqpoint{7.684937in}{1.439404in}}{\pgfqpoint{7.688503in}{1.442971in}}%
\pgfpathcurveto{\pgfqpoint{7.692069in}{1.446537in}}{\pgfqpoint{7.694073in}{1.451375in}}{\pgfqpoint{7.694073in}{1.456419in}}%
\pgfpathcurveto{\pgfqpoint{7.694073in}{1.461462in}}{\pgfqpoint{7.692069in}{1.466300in}}{\pgfqpoint{7.688503in}{1.469866in}}%
\pgfpathcurveto{\pgfqpoint{7.684937in}{1.473433in}}{\pgfqpoint{7.680099in}{1.475437in}}{\pgfqpoint{7.675055in}{1.475437in}}%
\pgfpathcurveto{\pgfqpoint{7.670012in}{1.475437in}}{\pgfqpoint{7.665174in}{1.473433in}}{\pgfqpoint{7.661607in}{1.469866in}}%
\pgfpathcurveto{\pgfqpoint{7.658041in}{1.466300in}}{\pgfqpoint{7.656037in}{1.461462in}}{\pgfqpoint{7.656037in}{1.456419in}}%
\pgfpathcurveto{\pgfqpoint{7.656037in}{1.451375in}}{\pgfqpoint{7.658041in}{1.446537in}}{\pgfqpoint{7.661607in}{1.442971in}}%
\pgfpathcurveto{\pgfqpoint{7.665174in}{1.439404in}}{\pgfqpoint{7.670012in}{1.437400in}}{\pgfqpoint{7.675055in}{1.437400in}}%
\pgfpathclose%
\pgfusepath{fill}%
\end{pgfscope}%
\begin{pgfscope}%
\pgfpathrectangle{\pgfqpoint{6.572727in}{0.474100in}}{\pgfqpoint{4.227273in}{3.318700in}}%
\pgfusepath{clip}%
\pgfsetbuttcap%
\pgfsetroundjoin%
\definecolor{currentfill}{rgb}{0.127568,0.566949,0.550556}%
\pgfsetfillcolor{currentfill}%
\pgfsetfillopacity{0.700000}%
\pgfsetlinewidth{0.000000pt}%
\definecolor{currentstroke}{rgb}{0.000000,0.000000,0.000000}%
\pgfsetstrokecolor{currentstroke}%
\pgfsetstrokeopacity{0.700000}%
\pgfsetdash{}{0pt}%
\pgfpathmoveto{\pgfqpoint{9.871804in}{1.994796in}}%
\pgfpathcurveto{\pgfqpoint{9.876847in}{1.994796in}}{\pgfqpoint{9.881685in}{1.996800in}}{\pgfqpoint{9.885252in}{2.000367in}}%
\pgfpathcurveto{\pgfqpoint{9.888818in}{2.003933in}}{\pgfqpoint{9.890822in}{2.008771in}}{\pgfqpoint{9.890822in}{2.013814in}}%
\pgfpathcurveto{\pgfqpoint{9.890822in}{2.018858in}}{\pgfqpoint{9.888818in}{2.023696in}}{\pgfqpoint{9.885252in}{2.027262in}}%
\pgfpathcurveto{\pgfqpoint{9.881685in}{2.030829in}}{\pgfqpoint{9.876847in}{2.032833in}}{\pgfqpoint{9.871804in}{2.032833in}}%
\pgfpathcurveto{\pgfqpoint{9.866760in}{2.032833in}}{\pgfqpoint{9.861922in}{2.030829in}}{\pgfqpoint{9.858356in}{2.027262in}}%
\pgfpathcurveto{\pgfqpoint{9.854790in}{2.023696in}}{\pgfqpoint{9.852786in}{2.018858in}}{\pgfqpoint{9.852786in}{2.013814in}}%
\pgfpathcurveto{\pgfqpoint{9.852786in}{2.008771in}}{\pgfqpoint{9.854790in}{2.003933in}}{\pgfqpoint{9.858356in}{2.000367in}}%
\pgfpathcurveto{\pgfqpoint{9.861922in}{1.996800in}}{\pgfqpoint{9.866760in}{1.994796in}}{\pgfqpoint{9.871804in}{1.994796in}}%
\pgfpathclose%
\pgfusepath{fill}%
\end{pgfscope}%
\begin{pgfscope}%
\pgfpathrectangle{\pgfqpoint{6.572727in}{0.474100in}}{\pgfqpoint{4.227273in}{3.318700in}}%
\pgfusepath{clip}%
\pgfsetbuttcap%
\pgfsetroundjoin%
\definecolor{currentfill}{rgb}{0.993248,0.906157,0.143936}%
\pgfsetfillcolor{currentfill}%
\pgfsetfillopacity{0.700000}%
\pgfsetlinewidth{0.000000pt}%
\definecolor{currentstroke}{rgb}{0.000000,0.000000,0.000000}%
\pgfsetstrokecolor{currentstroke}%
\pgfsetstrokeopacity{0.700000}%
\pgfsetdash{}{0pt}%
\pgfpathmoveto{\pgfqpoint{8.783733in}{3.099844in}}%
\pgfpathcurveto{\pgfqpoint{8.788777in}{3.099844in}}{\pgfqpoint{8.793615in}{3.101848in}}{\pgfqpoint{8.797181in}{3.105415in}}%
\pgfpathcurveto{\pgfqpoint{8.800747in}{3.108981in}}{\pgfqpoint{8.802751in}{3.113819in}}{\pgfqpoint{8.802751in}{3.118862in}}%
\pgfpathcurveto{\pgfqpoint{8.802751in}{3.123906in}}{\pgfqpoint{8.800747in}{3.128744in}}{\pgfqpoint{8.797181in}{3.132310in}}%
\pgfpathcurveto{\pgfqpoint{8.793615in}{3.135877in}}{\pgfqpoint{8.788777in}{3.137881in}}{\pgfqpoint{8.783733in}{3.137881in}}%
\pgfpathcurveto{\pgfqpoint{8.778690in}{3.137881in}}{\pgfqpoint{8.773852in}{3.135877in}}{\pgfqpoint{8.770285in}{3.132310in}}%
\pgfpathcurveto{\pgfqpoint{8.766719in}{3.128744in}}{\pgfqpoint{8.764715in}{3.123906in}}{\pgfqpoint{8.764715in}{3.118862in}}%
\pgfpathcurveto{\pgfqpoint{8.764715in}{3.113819in}}{\pgfqpoint{8.766719in}{3.108981in}}{\pgfqpoint{8.770285in}{3.105415in}}%
\pgfpathcurveto{\pgfqpoint{8.773852in}{3.101848in}}{\pgfqpoint{8.778690in}{3.099844in}}{\pgfqpoint{8.783733in}{3.099844in}}%
\pgfpathclose%
\pgfusepath{fill}%
\end{pgfscope}%
\begin{pgfscope}%
\pgfpathrectangle{\pgfqpoint{6.572727in}{0.474100in}}{\pgfqpoint{4.227273in}{3.318700in}}%
\pgfusepath{clip}%
\pgfsetbuttcap%
\pgfsetroundjoin%
\definecolor{currentfill}{rgb}{0.993248,0.906157,0.143936}%
\pgfsetfillcolor{currentfill}%
\pgfsetfillopacity{0.700000}%
\pgfsetlinewidth{0.000000pt}%
\definecolor{currentstroke}{rgb}{0.000000,0.000000,0.000000}%
\pgfsetstrokecolor{currentstroke}%
\pgfsetstrokeopacity{0.700000}%
\pgfsetdash{}{0pt}%
\pgfpathmoveto{\pgfqpoint{7.678613in}{2.835169in}}%
\pgfpathcurveto{\pgfqpoint{7.683657in}{2.835169in}}{\pgfqpoint{7.688494in}{2.837173in}}{\pgfqpoint{7.692061in}{2.840739in}}%
\pgfpathcurveto{\pgfqpoint{7.695627in}{2.844305in}}{\pgfqpoint{7.697631in}{2.849143in}}{\pgfqpoint{7.697631in}{2.854187in}}%
\pgfpathcurveto{\pgfqpoint{7.697631in}{2.859231in}}{\pgfqpoint{7.695627in}{2.864068in}}{\pgfqpoint{7.692061in}{2.867635in}}%
\pgfpathcurveto{\pgfqpoint{7.688494in}{2.871201in}}{\pgfqpoint{7.683657in}{2.873205in}}{\pgfqpoint{7.678613in}{2.873205in}}%
\pgfpathcurveto{\pgfqpoint{7.673569in}{2.873205in}}{\pgfqpoint{7.668731in}{2.871201in}}{\pgfqpoint{7.665165in}{2.867635in}}%
\pgfpathcurveto{\pgfqpoint{7.661599in}{2.864068in}}{\pgfqpoint{7.659595in}{2.859231in}}{\pgfqpoint{7.659595in}{2.854187in}}%
\pgfpathcurveto{\pgfqpoint{7.659595in}{2.849143in}}{\pgfqpoint{7.661599in}{2.844305in}}{\pgfqpoint{7.665165in}{2.840739in}}%
\pgfpathcurveto{\pgfqpoint{7.668731in}{2.837173in}}{\pgfqpoint{7.673569in}{2.835169in}}{\pgfqpoint{7.678613in}{2.835169in}}%
\pgfpathclose%
\pgfusepath{fill}%
\end{pgfscope}%
\begin{pgfscope}%
\pgfpathrectangle{\pgfqpoint{6.572727in}{0.474100in}}{\pgfqpoint{4.227273in}{3.318700in}}%
\pgfusepath{clip}%
\pgfsetbuttcap%
\pgfsetroundjoin%
\definecolor{currentfill}{rgb}{0.127568,0.566949,0.550556}%
\pgfsetfillcolor{currentfill}%
\pgfsetfillopacity{0.700000}%
\pgfsetlinewidth{0.000000pt}%
\definecolor{currentstroke}{rgb}{0.000000,0.000000,0.000000}%
\pgfsetstrokecolor{currentstroke}%
\pgfsetstrokeopacity{0.700000}%
\pgfsetdash{}{0pt}%
\pgfpathmoveto{\pgfqpoint{9.317019in}{1.614245in}}%
\pgfpathcurveto{\pgfqpoint{9.322062in}{1.614245in}}{\pgfqpoint{9.326900in}{1.616248in}}{\pgfqpoint{9.330466in}{1.619815in}}%
\pgfpathcurveto{\pgfqpoint{9.334033in}{1.623381in}}{\pgfqpoint{9.336037in}{1.628219in}}{\pgfqpoint{9.336037in}{1.633263in}}%
\pgfpathcurveto{\pgfqpoint{9.336037in}{1.638306in}}{\pgfqpoint{9.334033in}{1.643144in}}{\pgfqpoint{9.330466in}{1.646711in}}%
\pgfpathcurveto{\pgfqpoint{9.326900in}{1.650277in}}{\pgfqpoint{9.322062in}{1.652281in}}{\pgfqpoint{9.317019in}{1.652281in}}%
\pgfpathcurveto{\pgfqpoint{9.311975in}{1.652281in}}{\pgfqpoint{9.307137in}{1.650277in}}{\pgfqpoint{9.303571in}{1.646711in}}%
\pgfpathcurveto{\pgfqpoint{9.300004in}{1.643144in}}{\pgfqpoint{9.298000in}{1.638306in}}{\pgfqpoint{9.298000in}{1.633263in}}%
\pgfpathcurveto{\pgfqpoint{9.298000in}{1.628219in}}{\pgfqpoint{9.300004in}{1.623381in}}{\pgfqpoint{9.303571in}{1.619815in}}%
\pgfpathcurveto{\pgfqpoint{9.307137in}{1.616248in}}{\pgfqpoint{9.311975in}{1.614245in}}{\pgfqpoint{9.317019in}{1.614245in}}%
\pgfpathclose%
\pgfusepath{fill}%
\end{pgfscope}%
\begin{pgfscope}%
\pgfpathrectangle{\pgfqpoint{6.572727in}{0.474100in}}{\pgfqpoint{4.227273in}{3.318700in}}%
\pgfusepath{clip}%
\pgfsetbuttcap%
\pgfsetroundjoin%
\definecolor{currentfill}{rgb}{0.993248,0.906157,0.143936}%
\pgfsetfillcolor{currentfill}%
\pgfsetfillopacity{0.700000}%
\pgfsetlinewidth{0.000000pt}%
\definecolor{currentstroke}{rgb}{0.000000,0.000000,0.000000}%
\pgfsetstrokecolor{currentstroke}%
\pgfsetstrokeopacity{0.700000}%
\pgfsetdash{}{0pt}%
\pgfpathmoveto{\pgfqpoint{8.354403in}{3.258737in}}%
\pgfpathcurveto{\pgfqpoint{8.359447in}{3.258737in}}{\pgfqpoint{8.364285in}{3.260741in}}{\pgfqpoint{8.367851in}{3.264307in}}%
\pgfpathcurveto{\pgfqpoint{8.371417in}{3.267873in}}{\pgfqpoint{8.373421in}{3.272711in}}{\pgfqpoint{8.373421in}{3.277755in}}%
\pgfpathcurveto{\pgfqpoint{8.373421in}{3.282799in}}{\pgfqpoint{8.371417in}{3.287636in}}{\pgfqpoint{8.367851in}{3.291203in}}%
\pgfpathcurveto{\pgfqpoint{8.364285in}{3.294769in}}{\pgfqpoint{8.359447in}{3.296773in}}{\pgfqpoint{8.354403in}{3.296773in}}%
\pgfpathcurveto{\pgfqpoint{8.349359in}{3.296773in}}{\pgfqpoint{8.344522in}{3.294769in}}{\pgfqpoint{8.340955in}{3.291203in}}%
\pgfpathcurveto{\pgfqpoint{8.337389in}{3.287636in}}{\pgfqpoint{8.335385in}{3.282799in}}{\pgfqpoint{8.335385in}{3.277755in}}%
\pgfpathcurveto{\pgfqpoint{8.335385in}{3.272711in}}{\pgfqpoint{8.337389in}{3.267873in}}{\pgfqpoint{8.340955in}{3.264307in}}%
\pgfpathcurveto{\pgfqpoint{8.344522in}{3.260741in}}{\pgfqpoint{8.349359in}{3.258737in}}{\pgfqpoint{8.354403in}{3.258737in}}%
\pgfpathclose%
\pgfusepath{fill}%
\end{pgfscope}%
\begin{pgfscope}%
\pgfpathrectangle{\pgfqpoint{6.572727in}{0.474100in}}{\pgfqpoint{4.227273in}{3.318700in}}%
\pgfusepath{clip}%
\pgfsetbuttcap%
\pgfsetroundjoin%
\definecolor{currentfill}{rgb}{0.127568,0.566949,0.550556}%
\pgfsetfillcolor{currentfill}%
\pgfsetfillopacity{0.700000}%
\pgfsetlinewidth{0.000000pt}%
\definecolor{currentstroke}{rgb}{0.000000,0.000000,0.000000}%
\pgfsetstrokecolor{currentstroke}%
\pgfsetstrokeopacity{0.700000}%
\pgfsetdash{}{0pt}%
\pgfpathmoveto{\pgfqpoint{9.600668in}{1.649703in}}%
\pgfpathcurveto{\pgfqpoint{9.605712in}{1.649703in}}{\pgfqpoint{9.610550in}{1.651707in}}{\pgfqpoint{9.614116in}{1.655274in}}%
\pgfpathcurveto{\pgfqpoint{9.617682in}{1.658840in}}{\pgfqpoint{9.619686in}{1.663678in}}{\pgfqpoint{9.619686in}{1.668722in}}%
\pgfpathcurveto{\pgfqpoint{9.619686in}{1.673765in}}{\pgfqpoint{9.617682in}{1.678603in}}{\pgfqpoint{9.614116in}{1.682169in}}%
\pgfpathcurveto{\pgfqpoint{9.610550in}{1.685736in}}{\pgfqpoint{9.605712in}{1.687740in}}{\pgfqpoint{9.600668in}{1.687740in}}%
\pgfpathcurveto{\pgfqpoint{9.595624in}{1.687740in}}{\pgfqpoint{9.590787in}{1.685736in}}{\pgfqpoint{9.587220in}{1.682169in}}%
\pgfpathcurveto{\pgfqpoint{9.583654in}{1.678603in}}{\pgfqpoint{9.581650in}{1.673765in}}{\pgfqpoint{9.581650in}{1.668722in}}%
\pgfpathcurveto{\pgfqpoint{9.581650in}{1.663678in}}{\pgfqpoint{9.583654in}{1.658840in}}{\pgfqpoint{9.587220in}{1.655274in}}%
\pgfpathcurveto{\pgfqpoint{9.590787in}{1.651707in}}{\pgfqpoint{9.595624in}{1.649703in}}{\pgfqpoint{9.600668in}{1.649703in}}%
\pgfpathclose%
\pgfusepath{fill}%
\end{pgfscope}%
\begin{pgfscope}%
\pgfpathrectangle{\pgfqpoint{6.572727in}{0.474100in}}{\pgfqpoint{4.227273in}{3.318700in}}%
\pgfusepath{clip}%
\pgfsetbuttcap%
\pgfsetroundjoin%
\definecolor{currentfill}{rgb}{0.993248,0.906157,0.143936}%
\pgfsetfillcolor{currentfill}%
\pgfsetfillopacity{0.700000}%
\pgfsetlinewidth{0.000000pt}%
\definecolor{currentstroke}{rgb}{0.000000,0.000000,0.000000}%
\pgfsetstrokecolor{currentstroke}%
\pgfsetstrokeopacity{0.700000}%
\pgfsetdash{}{0pt}%
\pgfpathmoveto{\pgfqpoint{8.470465in}{3.123202in}}%
\pgfpathcurveto{\pgfqpoint{8.475509in}{3.123202in}}{\pgfqpoint{8.480347in}{3.125206in}}{\pgfqpoint{8.483913in}{3.128772in}}%
\pgfpathcurveto{\pgfqpoint{8.487480in}{3.132339in}}{\pgfqpoint{8.489483in}{3.137176in}}{\pgfqpoint{8.489483in}{3.142220in}}%
\pgfpathcurveto{\pgfqpoint{8.489483in}{3.147264in}}{\pgfqpoint{8.487480in}{3.152101in}}{\pgfqpoint{8.483913in}{3.155668in}}%
\pgfpathcurveto{\pgfqpoint{8.480347in}{3.159234in}}{\pgfqpoint{8.475509in}{3.161238in}}{\pgfqpoint{8.470465in}{3.161238in}}%
\pgfpathcurveto{\pgfqpoint{8.465422in}{3.161238in}}{\pgfqpoint{8.460584in}{3.159234in}}{\pgfqpoint{8.457017in}{3.155668in}}%
\pgfpathcurveto{\pgfqpoint{8.453451in}{3.152101in}}{\pgfqpoint{8.451447in}{3.147264in}}{\pgfqpoint{8.451447in}{3.142220in}}%
\pgfpathcurveto{\pgfqpoint{8.451447in}{3.137176in}}{\pgfqpoint{8.453451in}{3.132339in}}{\pgfqpoint{8.457017in}{3.128772in}}%
\pgfpathcurveto{\pgfqpoint{8.460584in}{3.125206in}}{\pgfqpoint{8.465422in}{3.123202in}}{\pgfqpoint{8.470465in}{3.123202in}}%
\pgfpathclose%
\pgfusepath{fill}%
\end{pgfscope}%
\begin{pgfscope}%
\pgfpathrectangle{\pgfqpoint{6.572727in}{0.474100in}}{\pgfqpoint{4.227273in}{3.318700in}}%
\pgfusepath{clip}%
\pgfsetbuttcap%
\pgfsetroundjoin%
\definecolor{currentfill}{rgb}{0.127568,0.566949,0.550556}%
\pgfsetfillcolor{currentfill}%
\pgfsetfillopacity{0.700000}%
\pgfsetlinewidth{0.000000pt}%
\definecolor{currentstroke}{rgb}{0.000000,0.000000,0.000000}%
\pgfsetstrokecolor{currentstroke}%
\pgfsetstrokeopacity{0.700000}%
\pgfsetdash{}{0pt}%
\pgfpathmoveto{\pgfqpoint{9.776351in}{1.140567in}}%
\pgfpathcurveto{\pgfqpoint{9.781395in}{1.140567in}}{\pgfqpoint{9.786233in}{1.142571in}}{\pgfqpoint{9.789799in}{1.146137in}}%
\pgfpathcurveto{\pgfqpoint{9.793365in}{1.149704in}}{\pgfqpoint{9.795369in}{1.154542in}}{\pgfqpoint{9.795369in}{1.159585in}}%
\pgfpathcurveto{\pgfqpoint{9.795369in}{1.164629in}}{\pgfqpoint{9.793365in}{1.169467in}}{\pgfqpoint{9.789799in}{1.173033in}}%
\pgfpathcurveto{\pgfqpoint{9.786233in}{1.176600in}}{\pgfqpoint{9.781395in}{1.178603in}}{\pgfqpoint{9.776351in}{1.178603in}}%
\pgfpathcurveto{\pgfqpoint{9.771307in}{1.178603in}}{\pgfqpoint{9.766470in}{1.176600in}}{\pgfqpoint{9.762903in}{1.173033in}}%
\pgfpathcurveto{\pgfqpoint{9.759337in}{1.169467in}}{\pgfqpoint{9.757333in}{1.164629in}}{\pgfqpoint{9.757333in}{1.159585in}}%
\pgfpathcurveto{\pgfqpoint{9.757333in}{1.154542in}}{\pgfqpoint{9.759337in}{1.149704in}}{\pgfqpoint{9.762903in}{1.146137in}}%
\pgfpathcurveto{\pgfqpoint{9.766470in}{1.142571in}}{\pgfqpoint{9.771307in}{1.140567in}}{\pgfqpoint{9.776351in}{1.140567in}}%
\pgfpathclose%
\pgfusepath{fill}%
\end{pgfscope}%
\begin{pgfscope}%
\pgfpathrectangle{\pgfqpoint{6.572727in}{0.474100in}}{\pgfqpoint{4.227273in}{3.318700in}}%
\pgfusepath{clip}%
\pgfsetbuttcap%
\pgfsetroundjoin%
\definecolor{currentfill}{rgb}{0.127568,0.566949,0.550556}%
\pgfsetfillcolor{currentfill}%
\pgfsetfillopacity{0.700000}%
\pgfsetlinewidth{0.000000pt}%
\definecolor{currentstroke}{rgb}{0.000000,0.000000,0.000000}%
\pgfsetstrokecolor{currentstroke}%
\pgfsetstrokeopacity{0.700000}%
\pgfsetdash{}{0pt}%
\pgfpathmoveto{\pgfqpoint{9.637120in}{1.766562in}}%
\pgfpathcurveto{\pgfqpoint{9.642163in}{1.766562in}}{\pgfqpoint{9.647001in}{1.768565in}}{\pgfqpoint{9.650568in}{1.772132in}}%
\pgfpathcurveto{\pgfqpoint{9.654134in}{1.775698in}}{\pgfqpoint{9.656138in}{1.780536in}}{\pgfqpoint{9.656138in}{1.785580in}}%
\pgfpathcurveto{\pgfqpoint{9.656138in}{1.790623in}}{\pgfqpoint{9.654134in}{1.795461in}}{\pgfqpoint{9.650568in}{1.799028in}}%
\pgfpathcurveto{\pgfqpoint{9.647001in}{1.802594in}}{\pgfqpoint{9.642163in}{1.804598in}}{\pgfqpoint{9.637120in}{1.804598in}}%
\pgfpathcurveto{\pgfqpoint{9.632076in}{1.804598in}}{\pgfqpoint{9.627238in}{1.802594in}}{\pgfqpoint{9.623672in}{1.799028in}}%
\pgfpathcurveto{\pgfqpoint{9.620105in}{1.795461in}}{\pgfqpoint{9.618102in}{1.790623in}}{\pgfqpoint{9.618102in}{1.785580in}}%
\pgfpathcurveto{\pgfqpoint{9.618102in}{1.780536in}}{\pgfqpoint{9.620105in}{1.775698in}}{\pgfqpoint{9.623672in}{1.772132in}}%
\pgfpathcurveto{\pgfqpoint{9.627238in}{1.768565in}}{\pgfqpoint{9.632076in}{1.766562in}}{\pgfqpoint{9.637120in}{1.766562in}}%
\pgfpathclose%
\pgfusepath{fill}%
\end{pgfscope}%
\begin{pgfscope}%
\pgfpathrectangle{\pgfqpoint{6.572727in}{0.474100in}}{\pgfqpoint{4.227273in}{3.318700in}}%
\pgfusepath{clip}%
\pgfsetbuttcap%
\pgfsetroundjoin%
\definecolor{currentfill}{rgb}{0.267004,0.004874,0.329415}%
\pgfsetfillcolor{currentfill}%
\pgfsetfillopacity{0.700000}%
\pgfsetlinewidth{0.000000pt}%
\definecolor{currentstroke}{rgb}{0.000000,0.000000,0.000000}%
\pgfsetstrokecolor{currentstroke}%
\pgfsetstrokeopacity{0.700000}%
\pgfsetdash{}{0pt}%
\pgfpathmoveto{\pgfqpoint{7.540105in}{1.666095in}}%
\pgfpathcurveto{\pgfqpoint{7.545149in}{1.666095in}}{\pgfqpoint{7.549986in}{1.668099in}}{\pgfqpoint{7.553553in}{1.671665in}}%
\pgfpathcurveto{\pgfqpoint{7.557119in}{1.675232in}}{\pgfqpoint{7.559123in}{1.680070in}}{\pgfqpoint{7.559123in}{1.685113in}}%
\pgfpathcurveto{\pgfqpoint{7.559123in}{1.690157in}}{\pgfqpoint{7.557119in}{1.694995in}}{\pgfqpoint{7.553553in}{1.698561in}}%
\pgfpathcurveto{\pgfqpoint{7.549986in}{1.702127in}}{\pgfqpoint{7.545149in}{1.704131in}}{\pgfqpoint{7.540105in}{1.704131in}}%
\pgfpathcurveto{\pgfqpoint{7.535061in}{1.704131in}}{\pgfqpoint{7.530224in}{1.702127in}}{\pgfqpoint{7.526657in}{1.698561in}}%
\pgfpathcurveto{\pgfqpoint{7.523091in}{1.694995in}}{\pgfqpoint{7.521087in}{1.690157in}}{\pgfqpoint{7.521087in}{1.685113in}}%
\pgfpathcurveto{\pgfqpoint{7.521087in}{1.680070in}}{\pgfqpoint{7.523091in}{1.675232in}}{\pgfqpoint{7.526657in}{1.671665in}}%
\pgfpathcurveto{\pgfqpoint{7.530224in}{1.668099in}}{\pgfqpoint{7.535061in}{1.666095in}}{\pgfqpoint{7.540105in}{1.666095in}}%
\pgfpathclose%
\pgfusepath{fill}%
\end{pgfscope}%
\begin{pgfscope}%
\pgfpathrectangle{\pgfqpoint{6.572727in}{0.474100in}}{\pgfqpoint{4.227273in}{3.318700in}}%
\pgfusepath{clip}%
\pgfsetbuttcap%
\pgfsetroundjoin%
\definecolor{currentfill}{rgb}{0.267004,0.004874,0.329415}%
\pgfsetfillcolor{currentfill}%
\pgfsetfillopacity{0.700000}%
\pgfsetlinewidth{0.000000pt}%
\definecolor{currentstroke}{rgb}{0.000000,0.000000,0.000000}%
\pgfsetstrokecolor{currentstroke}%
\pgfsetstrokeopacity{0.700000}%
\pgfsetdash{}{0pt}%
\pgfpathmoveto{\pgfqpoint{7.233386in}{0.960342in}}%
\pgfpathcurveto{\pgfqpoint{7.238430in}{0.960342in}}{\pgfqpoint{7.243268in}{0.962346in}}{\pgfqpoint{7.246834in}{0.965912in}}%
\pgfpathcurveto{\pgfqpoint{7.250401in}{0.969479in}}{\pgfqpoint{7.252405in}{0.974316in}}{\pgfqpoint{7.252405in}{0.979360in}}%
\pgfpathcurveto{\pgfqpoint{7.252405in}{0.984404in}}{\pgfqpoint{7.250401in}{0.989241in}}{\pgfqpoint{7.246834in}{0.992808in}}%
\pgfpathcurveto{\pgfqpoint{7.243268in}{0.996374in}}{\pgfqpoint{7.238430in}{0.998378in}}{\pgfqpoint{7.233386in}{0.998378in}}%
\pgfpathcurveto{\pgfqpoint{7.228343in}{0.998378in}}{\pgfqpoint{7.223505in}{0.996374in}}{\pgfqpoint{7.219939in}{0.992808in}}%
\pgfpathcurveto{\pgfqpoint{7.216372in}{0.989241in}}{\pgfqpoint{7.214368in}{0.984404in}}{\pgfqpoint{7.214368in}{0.979360in}}%
\pgfpathcurveto{\pgfqpoint{7.214368in}{0.974316in}}{\pgfqpoint{7.216372in}{0.969479in}}{\pgfqpoint{7.219939in}{0.965912in}}%
\pgfpathcurveto{\pgfqpoint{7.223505in}{0.962346in}}{\pgfqpoint{7.228343in}{0.960342in}}{\pgfqpoint{7.233386in}{0.960342in}}%
\pgfpathclose%
\pgfusepath{fill}%
\end{pgfscope}%
\begin{pgfscope}%
\pgfpathrectangle{\pgfqpoint{6.572727in}{0.474100in}}{\pgfqpoint{4.227273in}{3.318700in}}%
\pgfusepath{clip}%
\pgfsetbuttcap%
\pgfsetroundjoin%
\definecolor{currentfill}{rgb}{0.993248,0.906157,0.143936}%
\pgfsetfillcolor{currentfill}%
\pgfsetfillopacity{0.700000}%
\pgfsetlinewidth{0.000000pt}%
\definecolor{currentstroke}{rgb}{0.000000,0.000000,0.000000}%
\pgfsetstrokecolor{currentstroke}%
\pgfsetstrokeopacity{0.700000}%
\pgfsetdash{}{0pt}%
\pgfpathmoveto{\pgfqpoint{8.024822in}{2.439757in}}%
\pgfpathcurveto{\pgfqpoint{8.029866in}{2.439757in}}{\pgfqpoint{8.034704in}{2.441761in}}{\pgfqpoint{8.038270in}{2.445327in}}%
\pgfpathcurveto{\pgfqpoint{8.041837in}{2.448894in}}{\pgfqpoint{8.043840in}{2.453731in}}{\pgfqpoint{8.043840in}{2.458775in}}%
\pgfpathcurveto{\pgfqpoint{8.043840in}{2.463819in}}{\pgfqpoint{8.041837in}{2.468657in}}{\pgfqpoint{8.038270in}{2.472223in}}%
\pgfpathcurveto{\pgfqpoint{8.034704in}{2.475789in}}{\pgfqpoint{8.029866in}{2.477793in}}{\pgfqpoint{8.024822in}{2.477793in}}%
\pgfpathcurveto{\pgfqpoint{8.019779in}{2.477793in}}{\pgfqpoint{8.014941in}{2.475789in}}{\pgfqpoint{8.011374in}{2.472223in}}%
\pgfpathcurveto{\pgfqpoint{8.007808in}{2.468657in}}{\pgfqpoint{8.005804in}{2.463819in}}{\pgfqpoint{8.005804in}{2.458775in}}%
\pgfpathcurveto{\pgfqpoint{8.005804in}{2.453731in}}{\pgfqpoint{8.007808in}{2.448894in}}{\pgfqpoint{8.011374in}{2.445327in}}%
\pgfpathcurveto{\pgfqpoint{8.014941in}{2.441761in}}{\pgfqpoint{8.019779in}{2.439757in}}{\pgfqpoint{8.024822in}{2.439757in}}%
\pgfpathclose%
\pgfusepath{fill}%
\end{pgfscope}%
\begin{pgfscope}%
\pgfpathrectangle{\pgfqpoint{6.572727in}{0.474100in}}{\pgfqpoint{4.227273in}{3.318700in}}%
\pgfusepath{clip}%
\pgfsetbuttcap%
\pgfsetroundjoin%
\definecolor{currentfill}{rgb}{0.993248,0.906157,0.143936}%
\pgfsetfillcolor{currentfill}%
\pgfsetfillopacity{0.700000}%
\pgfsetlinewidth{0.000000pt}%
\definecolor{currentstroke}{rgb}{0.000000,0.000000,0.000000}%
\pgfsetstrokecolor{currentstroke}%
\pgfsetstrokeopacity{0.700000}%
\pgfsetdash{}{0pt}%
\pgfpathmoveto{\pgfqpoint{8.562464in}{2.565082in}}%
\pgfpathcurveto{\pgfqpoint{8.567508in}{2.565082in}}{\pgfqpoint{8.572345in}{2.567086in}}{\pgfqpoint{8.575912in}{2.570652in}}%
\pgfpathcurveto{\pgfqpoint{8.579478in}{2.574219in}}{\pgfqpoint{8.581482in}{2.579056in}}{\pgfqpoint{8.581482in}{2.584100in}}%
\pgfpathcurveto{\pgfqpoint{8.581482in}{2.589144in}}{\pgfqpoint{8.579478in}{2.593981in}}{\pgfqpoint{8.575912in}{2.597548in}}%
\pgfpathcurveto{\pgfqpoint{8.572345in}{2.601114in}}{\pgfqpoint{8.567508in}{2.603118in}}{\pgfqpoint{8.562464in}{2.603118in}}%
\pgfpathcurveto{\pgfqpoint{8.557420in}{2.603118in}}{\pgfqpoint{8.552582in}{2.601114in}}{\pgfqpoint{8.549016in}{2.597548in}}%
\pgfpathcurveto{\pgfqpoint{8.545450in}{2.593981in}}{\pgfqpoint{8.543446in}{2.589144in}}{\pgfqpoint{8.543446in}{2.584100in}}%
\pgfpathcurveto{\pgfqpoint{8.543446in}{2.579056in}}{\pgfqpoint{8.545450in}{2.574219in}}{\pgfqpoint{8.549016in}{2.570652in}}%
\pgfpathcurveto{\pgfqpoint{8.552582in}{2.567086in}}{\pgfqpoint{8.557420in}{2.565082in}}{\pgfqpoint{8.562464in}{2.565082in}}%
\pgfpathclose%
\pgfusepath{fill}%
\end{pgfscope}%
\begin{pgfscope}%
\pgfpathrectangle{\pgfqpoint{6.572727in}{0.474100in}}{\pgfqpoint{4.227273in}{3.318700in}}%
\pgfusepath{clip}%
\pgfsetbuttcap%
\pgfsetroundjoin%
\definecolor{currentfill}{rgb}{0.127568,0.566949,0.550556}%
\pgfsetfillcolor{currentfill}%
\pgfsetfillopacity{0.700000}%
\pgfsetlinewidth{0.000000pt}%
\definecolor{currentstroke}{rgb}{0.000000,0.000000,0.000000}%
\pgfsetstrokecolor{currentstroke}%
\pgfsetstrokeopacity{0.700000}%
\pgfsetdash{}{0pt}%
\pgfpathmoveto{\pgfqpoint{9.774988in}{1.386361in}}%
\pgfpathcurveto{\pgfqpoint{9.780032in}{1.386361in}}{\pgfqpoint{9.784870in}{1.388365in}}{\pgfqpoint{9.788436in}{1.391931in}}%
\pgfpathcurveto{\pgfqpoint{9.792002in}{1.395497in}}{\pgfqpoint{9.794006in}{1.400335in}}{\pgfqpoint{9.794006in}{1.405379in}}%
\pgfpathcurveto{\pgfqpoint{9.794006in}{1.410423in}}{\pgfqpoint{9.792002in}{1.415260in}}{\pgfqpoint{9.788436in}{1.418827in}}%
\pgfpathcurveto{\pgfqpoint{9.784870in}{1.422393in}}{\pgfqpoint{9.780032in}{1.424397in}}{\pgfqpoint{9.774988in}{1.424397in}}%
\pgfpathcurveto{\pgfqpoint{9.769944in}{1.424397in}}{\pgfqpoint{9.765107in}{1.422393in}}{\pgfqpoint{9.761540in}{1.418827in}}%
\pgfpathcurveto{\pgfqpoint{9.757974in}{1.415260in}}{\pgfqpoint{9.755970in}{1.410423in}}{\pgfqpoint{9.755970in}{1.405379in}}%
\pgfpathcurveto{\pgfqpoint{9.755970in}{1.400335in}}{\pgfqpoint{9.757974in}{1.395497in}}{\pgfqpoint{9.761540in}{1.391931in}}%
\pgfpathcurveto{\pgfqpoint{9.765107in}{1.388365in}}{\pgfqpoint{9.769944in}{1.386361in}}{\pgfqpoint{9.774988in}{1.386361in}}%
\pgfpathclose%
\pgfusepath{fill}%
\end{pgfscope}%
\begin{pgfscope}%
\pgfpathrectangle{\pgfqpoint{6.572727in}{0.474100in}}{\pgfqpoint{4.227273in}{3.318700in}}%
\pgfusepath{clip}%
\pgfsetbuttcap%
\pgfsetroundjoin%
\definecolor{currentfill}{rgb}{0.127568,0.566949,0.550556}%
\pgfsetfillcolor{currentfill}%
\pgfsetfillopacity{0.700000}%
\pgfsetlinewidth{0.000000pt}%
\definecolor{currentstroke}{rgb}{0.000000,0.000000,0.000000}%
\pgfsetstrokecolor{currentstroke}%
\pgfsetstrokeopacity{0.700000}%
\pgfsetdash{}{0pt}%
\pgfpathmoveto{\pgfqpoint{9.605057in}{1.352537in}}%
\pgfpathcurveto{\pgfqpoint{9.610100in}{1.352537in}}{\pgfqpoint{9.614938in}{1.354541in}}{\pgfqpoint{9.618505in}{1.358107in}}%
\pgfpathcurveto{\pgfqpoint{9.622071in}{1.361673in}}{\pgfqpoint{9.624075in}{1.366511in}}{\pgfqpoint{9.624075in}{1.371555in}}%
\pgfpathcurveto{\pgfqpoint{9.624075in}{1.376598in}}{\pgfqpoint{9.622071in}{1.381436in}}{\pgfqpoint{9.618505in}{1.385003in}}%
\pgfpathcurveto{\pgfqpoint{9.614938in}{1.388569in}}{\pgfqpoint{9.610100in}{1.390573in}}{\pgfqpoint{9.605057in}{1.390573in}}%
\pgfpathcurveto{\pgfqpoint{9.600013in}{1.390573in}}{\pgfqpoint{9.595175in}{1.388569in}}{\pgfqpoint{9.591609in}{1.385003in}}%
\pgfpathcurveto{\pgfqpoint{9.588042in}{1.381436in}}{\pgfqpoint{9.586039in}{1.376598in}}{\pgfqpoint{9.586039in}{1.371555in}}%
\pgfpathcurveto{\pgfqpoint{9.586039in}{1.366511in}}{\pgfqpoint{9.588042in}{1.361673in}}{\pgfqpoint{9.591609in}{1.358107in}}%
\pgfpathcurveto{\pgfqpoint{9.595175in}{1.354541in}}{\pgfqpoint{9.600013in}{1.352537in}}{\pgfqpoint{9.605057in}{1.352537in}}%
\pgfpathclose%
\pgfusepath{fill}%
\end{pgfscope}%
\begin{pgfscope}%
\pgfpathrectangle{\pgfqpoint{6.572727in}{0.474100in}}{\pgfqpoint{4.227273in}{3.318700in}}%
\pgfusepath{clip}%
\pgfsetbuttcap%
\pgfsetroundjoin%
\definecolor{currentfill}{rgb}{0.993248,0.906157,0.143936}%
\pgfsetfillcolor{currentfill}%
\pgfsetfillopacity{0.700000}%
\pgfsetlinewidth{0.000000pt}%
\definecolor{currentstroke}{rgb}{0.000000,0.000000,0.000000}%
\pgfsetstrokecolor{currentstroke}%
\pgfsetstrokeopacity{0.700000}%
\pgfsetdash{}{0pt}%
\pgfpathmoveto{\pgfqpoint{7.976908in}{2.879867in}}%
\pgfpathcurveto{\pgfqpoint{7.981951in}{2.879867in}}{\pgfqpoint{7.986789in}{2.881871in}}{\pgfqpoint{7.990355in}{2.885437in}}%
\pgfpathcurveto{\pgfqpoint{7.993922in}{2.889004in}}{\pgfqpoint{7.995926in}{2.893842in}}{\pgfqpoint{7.995926in}{2.898885in}}%
\pgfpathcurveto{\pgfqpoint{7.995926in}{2.903929in}}{\pgfqpoint{7.993922in}{2.908767in}}{\pgfqpoint{7.990355in}{2.912333in}}%
\pgfpathcurveto{\pgfqpoint{7.986789in}{2.915900in}}{\pgfqpoint{7.981951in}{2.917903in}}{\pgfqpoint{7.976908in}{2.917903in}}%
\pgfpathcurveto{\pgfqpoint{7.971864in}{2.917903in}}{\pgfqpoint{7.967026in}{2.915900in}}{\pgfqpoint{7.963460in}{2.912333in}}%
\pgfpathcurveto{\pgfqpoint{7.959893in}{2.908767in}}{\pgfqpoint{7.957889in}{2.903929in}}{\pgfqpoint{7.957889in}{2.898885in}}%
\pgfpathcurveto{\pgfqpoint{7.957889in}{2.893842in}}{\pgfqpoint{7.959893in}{2.889004in}}{\pgfqpoint{7.963460in}{2.885437in}}%
\pgfpathcurveto{\pgfqpoint{7.967026in}{2.881871in}}{\pgfqpoint{7.971864in}{2.879867in}}{\pgfqpoint{7.976908in}{2.879867in}}%
\pgfpathclose%
\pgfusepath{fill}%
\end{pgfscope}%
\begin{pgfscope}%
\pgfpathrectangle{\pgfqpoint{6.572727in}{0.474100in}}{\pgfqpoint{4.227273in}{3.318700in}}%
\pgfusepath{clip}%
\pgfsetbuttcap%
\pgfsetroundjoin%
\definecolor{currentfill}{rgb}{0.993248,0.906157,0.143936}%
\pgfsetfillcolor{currentfill}%
\pgfsetfillopacity{0.700000}%
\pgfsetlinewidth{0.000000pt}%
\definecolor{currentstroke}{rgb}{0.000000,0.000000,0.000000}%
\pgfsetstrokecolor{currentstroke}%
\pgfsetstrokeopacity{0.700000}%
\pgfsetdash{}{0pt}%
\pgfpathmoveto{\pgfqpoint{8.196999in}{2.845461in}}%
\pgfpathcurveto{\pgfqpoint{8.202043in}{2.845461in}}{\pgfqpoint{8.206881in}{2.847464in}}{\pgfqpoint{8.210447in}{2.851031in}}%
\pgfpathcurveto{\pgfqpoint{8.214013in}{2.854597in}}{\pgfqpoint{8.216017in}{2.859435in}}{\pgfqpoint{8.216017in}{2.864479in}}%
\pgfpathcurveto{\pgfqpoint{8.216017in}{2.869522in}}{\pgfqpoint{8.214013in}{2.874360in}}{\pgfqpoint{8.210447in}{2.877927in}}%
\pgfpathcurveto{\pgfqpoint{8.206881in}{2.881493in}}{\pgfqpoint{8.202043in}{2.883497in}}{\pgfqpoint{8.196999in}{2.883497in}}%
\pgfpathcurveto{\pgfqpoint{8.191956in}{2.883497in}}{\pgfqpoint{8.187118in}{2.881493in}}{\pgfqpoint{8.183551in}{2.877927in}}%
\pgfpathcurveto{\pgfqpoint{8.179985in}{2.874360in}}{\pgfqpoint{8.177981in}{2.869522in}}{\pgfqpoint{8.177981in}{2.864479in}}%
\pgfpathcurveto{\pgfqpoint{8.177981in}{2.859435in}}{\pgfqpoint{8.179985in}{2.854597in}}{\pgfqpoint{8.183551in}{2.851031in}}%
\pgfpathcurveto{\pgfqpoint{8.187118in}{2.847464in}}{\pgfqpoint{8.191956in}{2.845461in}}{\pgfqpoint{8.196999in}{2.845461in}}%
\pgfpathclose%
\pgfusepath{fill}%
\end{pgfscope}%
\begin{pgfscope}%
\pgfpathrectangle{\pgfqpoint{6.572727in}{0.474100in}}{\pgfqpoint{4.227273in}{3.318700in}}%
\pgfusepath{clip}%
\pgfsetbuttcap%
\pgfsetroundjoin%
\definecolor{currentfill}{rgb}{0.267004,0.004874,0.329415}%
\pgfsetfillcolor{currentfill}%
\pgfsetfillopacity{0.700000}%
\pgfsetlinewidth{0.000000pt}%
\definecolor{currentstroke}{rgb}{0.000000,0.000000,0.000000}%
\pgfsetstrokecolor{currentstroke}%
\pgfsetstrokeopacity{0.700000}%
\pgfsetdash{}{0pt}%
\pgfpathmoveto{\pgfqpoint{7.797632in}{1.270024in}}%
\pgfpathcurveto{\pgfqpoint{7.802675in}{1.270024in}}{\pgfqpoint{7.807513in}{1.272028in}}{\pgfqpoint{7.811079in}{1.275595in}}%
\pgfpathcurveto{\pgfqpoint{7.814646in}{1.279161in}}{\pgfqpoint{7.816650in}{1.283999in}}{\pgfqpoint{7.816650in}{1.289043in}}%
\pgfpathcurveto{\pgfqpoint{7.816650in}{1.294086in}}{\pgfqpoint{7.814646in}{1.298924in}}{\pgfqpoint{7.811079in}{1.302490in}}%
\pgfpathcurveto{\pgfqpoint{7.807513in}{1.306057in}}{\pgfqpoint{7.802675in}{1.308061in}}{\pgfqpoint{7.797632in}{1.308061in}}%
\pgfpathcurveto{\pgfqpoint{7.792588in}{1.308061in}}{\pgfqpoint{7.787750in}{1.306057in}}{\pgfqpoint{7.784184in}{1.302490in}}%
\pgfpathcurveto{\pgfqpoint{7.780617in}{1.298924in}}{\pgfqpoint{7.778613in}{1.294086in}}{\pgfqpoint{7.778613in}{1.289043in}}%
\pgfpathcurveto{\pgfqpoint{7.778613in}{1.283999in}}{\pgfqpoint{7.780617in}{1.279161in}}{\pgfqpoint{7.784184in}{1.275595in}}%
\pgfpathcurveto{\pgfqpoint{7.787750in}{1.272028in}}{\pgfqpoint{7.792588in}{1.270024in}}{\pgfqpoint{7.797632in}{1.270024in}}%
\pgfpathclose%
\pgfusepath{fill}%
\end{pgfscope}%
\begin{pgfscope}%
\pgfpathrectangle{\pgfqpoint{6.572727in}{0.474100in}}{\pgfqpoint{4.227273in}{3.318700in}}%
\pgfusepath{clip}%
\pgfsetbuttcap%
\pgfsetroundjoin%
\definecolor{currentfill}{rgb}{0.993248,0.906157,0.143936}%
\pgfsetfillcolor{currentfill}%
\pgfsetfillopacity{0.700000}%
\pgfsetlinewidth{0.000000pt}%
\definecolor{currentstroke}{rgb}{0.000000,0.000000,0.000000}%
\pgfsetstrokecolor{currentstroke}%
\pgfsetstrokeopacity{0.700000}%
\pgfsetdash{}{0pt}%
\pgfpathmoveto{\pgfqpoint{8.678931in}{2.602056in}}%
\pgfpathcurveto{\pgfqpoint{8.683974in}{2.602056in}}{\pgfqpoint{8.688812in}{2.604060in}}{\pgfqpoint{8.692379in}{2.607627in}}%
\pgfpathcurveto{\pgfqpoint{8.695945in}{2.611193in}}{\pgfqpoint{8.697949in}{2.616031in}}{\pgfqpoint{8.697949in}{2.621074in}}%
\pgfpathcurveto{\pgfqpoint{8.697949in}{2.626118in}}{\pgfqpoint{8.695945in}{2.630956in}}{\pgfqpoint{8.692379in}{2.634522in}}%
\pgfpathcurveto{\pgfqpoint{8.688812in}{2.638089in}}{\pgfqpoint{8.683974in}{2.640093in}}{\pgfqpoint{8.678931in}{2.640093in}}%
\pgfpathcurveto{\pgfqpoint{8.673887in}{2.640093in}}{\pgfqpoint{8.669049in}{2.638089in}}{\pgfqpoint{8.665483in}{2.634522in}}%
\pgfpathcurveto{\pgfqpoint{8.661916in}{2.630956in}}{\pgfqpoint{8.659913in}{2.626118in}}{\pgfqpoint{8.659913in}{2.621074in}}%
\pgfpathcurveto{\pgfqpoint{8.659913in}{2.616031in}}{\pgfqpoint{8.661916in}{2.611193in}}{\pgfqpoint{8.665483in}{2.607627in}}%
\pgfpathcurveto{\pgfqpoint{8.669049in}{2.604060in}}{\pgfqpoint{8.673887in}{2.602056in}}{\pgfqpoint{8.678931in}{2.602056in}}%
\pgfpathclose%
\pgfusepath{fill}%
\end{pgfscope}%
\begin{pgfscope}%
\pgfpathrectangle{\pgfqpoint{6.572727in}{0.474100in}}{\pgfqpoint{4.227273in}{3.318700in}}%
\pgfusepath{clip}%
\pgfsetbuttcap%
\pgfsetroundjoin%
\definecolor{currentfill}{rgb}{0.993248,0.906157,0.143936}%
\pgfsetfillcolor{currentfill}%
\pgfsetfillopacity{0.700000}%
\pgfsetlinewidth{0.000000pt}%
\definecolor{currentstroke}{rgb}{0.000000,0.000000,0.000000}%
\pgfsetstrokecolor{currentstroke}%
\pgfsetstrokeopacity{0.700000}%
\pgfsetdash{}{0pt}%
\pgfpathmoveto{\pgfqpoint{8.480039in}{2.271735in}}%
\pgfpathcurveto{\pgfqpoint{8.485082in}{2.271735in}}{\pgfqpoint{8.489920in}{2.273739in}}{\pgfqpoint{8.493487in}{2.277305in}}%
\pgfpathcurveto{\pgfqpoint{8.497053in}{2.280872in}}{\pgfqpoint{8.499057in}{2.285709in}}{\pgfqpoint{8.499057in}{2.290753in}}%
\pgfpathcurveto{\pgfqpoint{8.499057in}{2.295797in}}{\pgfqpoint{8.497053in}{2.300635in}}{\pgfqpoint{8.493487in}{2.304201in}}%
\pgfpathcurveto{\pgfqpoint{8.489920in}{2.307767in}}{\pgfqpoint{8.485082in}{2.309771in}}{\pgfqpoint{8.480039in}{2.309771in}}%
\pgfpathcurveto{\pgfqpoint{8.474995in}{2.309771in}}{\pgfqpoint{8.470157in}{2.307767in}}{\pgfqpoint{8.466591in}{2.304201in}}%
\pgfpathcurveto{\pgfqpoint{8.463024in}{2.300635in}}{\pgfqpoint{8.461021in}{2.295797in}}{\pgfqpoint{8.461021in}{2.290753in}}%
\pgfpathcurveto{\pgfqpoint{8.461021in}{2.285709in}}{\pgfqpoint{8.463024in}{2.280872in}}{\pgfqpoint{8.466591in}{2.277305in}}%
\pgfpathcurveto{\pgfqpoint{8.470157in}{2.273739in}}{\pgfqpoint{8.474995in}{2.271735in}}{\pgfqpoint{8.480039in}{2.271735in}}%
\pgfpathclose%
\pgfusepath{fill}%
\end{pgfscope}%
\begin{pgfscope}%
\pgfpathrectangle{\pgfqpoint{6.572727in}{0.474100in}}{\pgfqpoint{4.227273in}{3.318700in}}%
\pgfusepath{clip}%
\pgfsetbuttcap%
\pgfsetroundjoin%
\definecolor{currentfill}{rgb}{0.993248,0.906157,0.143936}%
\pgfsetfillcolor{currentfill}%
\pgfsetfillopacity{0.700000}%
\pgfsetlinewidth{0.000000pt}%
\definecolor{currentstroke}{rgb}{0.000000,0.000000,0.000000}%
\pgfsetstrokecolor{currentstroke}%
\pgfsetstrokeopacity{0.700000}%
\pgfsetdash{}{0pt}%
\pgfpathmoveto{\pgfqpoint{8.791698in}{3.449710in}}%
\pgfpathcurveto{\pgfqpoint{8.796741in}{3.449710in}}{\pgfqpoint{8.801579in}{3.451714in}}{\pgfqpoint{8.805145in}{3.455281in}}%
\pgfpathcurveto{\pgfqpoint{8.808712in}{3.458847in}}{\pgfqpoint{8.810716in}{3.463685in}}{\pgfqpoint{8.810716in}{3.468729in}}%
\pgfpathcurveto{\pgfqpoint{8.810716in}{3.473772in}}{\pgfqpoint{8.808712in}{3.478610in}}{\pgfqpoint{8.805145in}{3.482176in}}%
\pgfpathcurveto{\pgfqpoint{8.801579in}{3.485743in}}{\pgfqpoint{8.796741in}{3.487747in}}{\pgfqpoint{8.791698in}{3.487747in}}%
\pgfpathcurveto{\pgfqpoint{8.786654in}{3.487747in}}{\pgfqpoint{8.781816in}{3.485743in}}{\pgfqpoint{8.778250in}{3.482176in}}%
\pgfpathcurveto{\pgfqpoint{8.774683in}{3.478610in}}{\pgfqpoint{8.772679in}{3.473772in}}{\pgfqpoint{8.772679in}{3.468729in}}%
\pgfpathcurveto{\pgfqpoint{8.772679in}{3.463685in}}{\pgfqpoint{8.774683in}{3.458847in}}{\pgfqpoint{8.778250in}{3.455281in}}%
\pgfpathcurveto{\pgfqpoint{8.781816in}{3.451714in}}{\pgfqpoint{8.786654in}{3.449710in}}{\pgfqpoint{8.791698in}{3.449710in}}%
\pgfpathclose%
\pgfusepath{fill}%
\end{pgfscope}%
\begin{pgfscope}%
\pgfpathrectangle{\pgfqpoint{6.572727in}{0.474100in}}{\pgfqpoint{4.227273in}{3.318700in}}%
\pgfusepath{clip}%
\pgfsetbuttcap%
\pgfsetroundjoin%
\definecolor{currentfill}{rgb}{0.993248,0.906157,0.143936}%
\pgfsetfillcolor{currentfill}%
\pgfsetfillopacity{0.700000}%
\pgfsetlinewidth{0.000000pt}%
\definecolor{currentstroke}{rgb}{0.000000,0.000000,0.000000}%
\pgfsetstrokecolor{currentstroke}%
\pgfsetstrokeopacity{0.700000}%
\pgfsetdash{}{0pt}%
\pgfpathmoveto{\pgfqpoint{8.713837in}{2.149968in}}%
\pgfpathcurveto{\pgfqpoint{8.718880in}{2.149968in}}{\pgfqpoint{8.723718in}{2.151972in}}{\pgfqpoint{8.727284in}{2.155538in}}%
\pgfpathcurveto{\pgfqpoint{8.730851in}{2.159104in}}{\pgfqpoint{8.732855in}{2.163942in}}{\pgfqpoint{8.732855in}{2.168986in}}%
\pgfpathcurveto{\pgfqpoint{8.732855in}{2.174030in}}{\pgfqpoint{8.730851in}{2.178867in}}{\pgfqpoint{8.727284in}{2.182434in}}%
\pgfpathcurveto{\pgfqpoint{8.723718in}{2.186000in}}{\pgfqpoint{8.718880in}{2.188004in}}{\pgfqpoint{8.713837in}{2.188004in}}%
\pgfpathcurveto{\pgfqpoint{8.708793in}{2.188004in}}{\pgfqpoint{8.703955in}{2.186000in}}{\pgfqpoint{8.700389in}{2.182434in}}%
\pgfpathcurveto{\pgfqpoint{8.696822in}{2.178867in}}{\pgfqpoint{8.694818in}{2.174030in}}{\pgfqpoint{8.694818in}{2.168986in}}%
\pgfpathcurveto{\pgfqpoint{8.694818in}{2.163942in}}{\pgfqpoint{8.696822in}{2.159104in}}{\pgfqpoint{8.700389in}{2.155538in}}%
\pgfpathcurveto{\pgfqpoint{8.703955in}{2.151972in}}{\pgfqpoint{8.708793in}{2.149968in}}{\pgfqpoint{8.713837in}{2.149968in}}%
\pgfpathclose%
\pgfusepath{fill}%
\end{pgfscope}%
\begin{pgfscope}%
\pgfpathrectangle{\pgfqpoint{6.572727in}{0.474100in}}{\pgfqpoint{4.227273in}{3.318700in}}%
\pgfusepath{clip}%
\pgfsetbuttcap%
\pgfsetroundjoin%
\definecolor{currentfill}{rgb}{0.993248,0.906157,0.143936}%
\pgfsetfillcolor{currentfill}%
\pgfsetfillopacity{0.700000}%
\pgfsetlinewidth{0.000000pt}%
\definecolor{currentstroke}{rgb}{0.000000,0.000000,0.000000}%
\pgfsetstrokecolor{currentstroke}%
\pgfsetstrokeopacity{0.700000}%
\pgfsetdash{}{0pt}%
\pgfpathmoveto{\pgfqpoint{8.528419in}{3.090104in}}%
\pgfpathcurveto{\pgfqpoint{8.533463in}{3.090104in}}{\pgfqpoint{8.538301in}{3.092107in}}{\pgfqpoint{8.541867in}{3.095674in}}%
\pgfpathcurveto{\pgfqpoint{8.545434in}{3.099240in}}{\pgfqpoint{8.547438in}{3.104078in}}{\pgfqpoint{8.547438in}{3.109122in}}%
\pgfpathcurveto{\pgfqpoint{8.547438in}{3.114165in}}{\pgfqpoint{8.545434in}{3.119003in}}{\pgfqpoint{8.541867in}{3.122570in}}%
\pgfpathcurveto{\pgfqpoint{8.538301in}{3.126136in}}{\pgfqpoint{8.533463in}{3.128140in}}{\pgfqpoint{8.528419in}{3.128140in}}%
\pgfpathcurveto{\pgfqpoint{8.523376in}{3.128140in}}{\pgfqpoint{8.518538in}{3.126136in}}{\pgfqpoint{8.514972in}{3.122570in}}%
\pgfpathcurveto{\pgfqpoint{8.511405in}{3.119003in}}{\pgfqpoint{8.509401in}{3.114165in}}{\pgfqpoint{8.509401in}{3.109122in}}%
\pgfpathcurveto{\pgfqpoint{8.509401in}{3.104078in}}{\pgfqpoint{8.511405in}{3.099240in}}{\pgfqpoint{8.514972in}{3.095674in}}%
\pgfpathcurveto{\pgfqpoint{8.518538in}{3.092107in}}{\pgfqpoint{8.523376in}{3.090104in}}{\pgfqpoint{8.528419in}{3.090104in}}%
\pgfpathclose%
\pgfusepath{fill}%
\end{pgfscope}%
\begin{pgfscope}%
\pgfpathrectangle{\pgfqpoint{6.572727in}{0.474100in}}{\pgfqpoint{4.227273in}{3.318700in}}%
\pgfusepath{clip}%
\pgfsetbuttcap%
\pgfsetroundjoin%
\definecolor{currentfill}{rgb}{0.993248,0.906157,0.143936}%
\pgfsetfillcolor{currentfill}%
\pgfsetfillopacity{0.700000}%
\pgfsetlinewidth{0.000000pt}%
\definecolor{currentstroke}{rgb}{0.000000,0.000000,0.000000}%
\pgfsetstrokecolor{currentstroke}%
\pgfsetstrokeopacity{0.700000}%
\pgfsetdash{}{0pt}%
\pgfpathmoveto{\pgfqpoint{7.823149in}{2.667139in}}%
\pgfpathcurveto{\pgfqpoint{7.828193in}{2.667139in}}{\pgfqpoint{7.833031in}{2.669143in}}{\pgfqpoint{7.836597in}{2.672709in}}%
\pgfpathcurveto{\pgfqpoint{7.840164in}{2.676276in}}{\pgfqpoint{7.842167in}{2.681114in}}{\pgfqpoint{7.842167in}{2.686157in}}%
\pgfpathcurveto{\pgfqpoint{7.842167in}{2.691201in}}{\pgfqpoint{7.840164in}{2.696039in}}{\pgfqpoint{7.836597in}{2.699605in}}%
\pgfpathcurveto{\pgfqpoint{7.833031in}{2.703171in}}{\pgfqpoint{7.828193in}{2.705175in}}{\pgfqpoint{7.823149in}{2.705175in}}%
\pgfpathcurveto{\pgfqpoint{7.818106in}{2.705175in}}{\pgfqpoint{7.813268in}{2.703171in}}{\pgfqpoint{7.809701in}{2.699605in}}%
\pgfpathcurveto{\pgfqpoint{7.806135in}{2.696039in}}{\pgfqpoint{7.804131in}{2.691201in}}{\pgfqpoint{7.804131in}{2.686157in}}%
\pgfpathcurveto{\pgfqpoint{7.804131in}{2.681114in}}{\pgfqpoint{7.806135in}{2.676276in}}{\pgfqpoint{7.809701in}{2.672709in}}%
\pgfpathcurveto{\pgfqpoint{7.813268in}{2.669143in}}{\pgfqpoint{7.818106in}{2.667139in}}{\pgfqpoint{7.823149in}{2.667139in}}%
\pgfpathclose%
\pgfusepath{fill}%
\end{pgfscope}%
\begin{pgfscope}%
\pgfpathrectangle{\pgfqpoint{6.572727in}{0.474100in}}{\pgfqpoint{4.227273in}{3.318700in}}%
\pgfusepath{clip}%
\pgfsetbuttcap%
\pgfsetroundjoin%
\definecolor{currentfill}{rgb}{0.267004,0.004874,0.329415}%
\pgfsetfillcolor{currentfill}%
\pgfsetfillopacity{0.700000}%
\pgfsetlinewidth{0.000000pt}%
\definecolor{currentstroke}{rgb}{0.000000,0.000000,0.000000}%
\pgfsetstrokecolor{currentstroke}%
\pgfsetstrokeopacity{0.700000}%
\pgfsetdash{}{0pt}%
\pgfpathmoveto{\pgfqpoint{8.131231in}{1.692332in}}%
\pgfpathcurveto{\pgfqpoint{8.136275in}{1.692332in}}{\pgfqpoint{8.141113in}{1.694336in}}{\pgfqpoint{8.144679in}{1.697902in}}%
\pgfpathcurveto{\pgfqpoint{8.148245in}{1.701469in}}{\pgfqpoint{8.150249in}{1.706307in}}{\pgfqpoint{8.150249in}{1.711350in}}%
\pgfpathcurveto{\pgfqpoint{8.150249in}{1.716394in}}{\pgfqpoint{8.148245in}{1.721232in}}{\pgfqpoint{8.144679in}{1.724798in}}%
\pgfpathcurveto{\pgfqpoint{8.141113in}{1.728364in}}{\pgfqpoint{8.136275in}{1.730368in}}{\pgfqpoint{8.131231in}{1.730368in}}%
\pgfpathcurveto{\pgfqpoint{8.126187in}{1.730368in}}{\pgfqpoint{8.121350in}{1.728364in}}{\pgfqpoint{8.117783in}{1.724798in}}%
\pgfpathcurveto{\pgfqpoint{8.114217in}{1.721232in}}{\pgfqpoint{8.112213in}{1.716394in}}{\pgfqpoint{8.112213in}{1.711350in}}%
\pgfpathcurveto{\pgfqpoint{8.112213in}{1.706307in}}{\pgfqpoint{8.114217in}{1.701469in}}{\pgfqpoint{8.117783in}{1.697902in}}%
\pgfpathcurveto{\pgfqpoint{8.121350in}{1.694336in}}{\pgfqpoint{8.126187in}{1.692332in}}{\pgfqpoint{8.131231in}{1.692332in}}%
\pgfpathclose%
\pgfusepath{fill}%
\end{pgfscope}%
\begin{pgfscope}%
\pgfpathrectangle{\pgfqpoint{6.572727in}{0.474100in}}{\pgfqpoint{4.227273in}{3.318700in}}%
\pgfusepath{clip}%
\pgfsetbuttcap%
\pgfsetroundjoin%
\definecolor{currentfill}{rgb}{0.127568,0.566949,0.550556}%
\pgfsetfillcolor{currentfill}%
\pgfsetfillopacity{0.700000}%
\pgfsetlinewidth{0.000000pt}%
\definecolor{currentstroke}{rgb}{0.000000,0.000000,0.000000}%
\pgfsetstrokecolor{currentstroke}%
\pgfsetstrokeopacity{0.700000}%
\pgfsetdash{}{0pt}%
\pgfpathmoveto{\pgfqpoint{9.491682in}{1.739431in}}%
\pgfpathcurveto{\pgfqpoint{9.496726in}{1.739431in}}{\pgfqpoint{9.501564in}{1.741435in}}{\pgfqpoint{9.505130in}{1.745001in}}%
\pgfpathcurveto{\pgfqpoint{9.508697in}{1.748567in}}{\pgfqpoint{9.510701in}{1.753405in}}{\pgfqpoint{9.510701in}{1.758449in}}%
\pgfpathcurveto{\pgfqpoint{9.510701in}{1.763493in}}{\pgfqpoint{9.508697in}{1.768330in}}{\pgfqpoint{9.505130in}{1.771897in}}%
\pgfpathcurveto{\pgfqpoint{9.501564in}{1.775463in}}{\pgfqpoint{9.496726in}{1.777467in}}{\pgfqpoint{9.491682in}{1.777467in}}%
\pgfpathcurveto{\pgfqpoint{9.486639in}{1.777467in}}{\pgfqpoint{9.481801in}{1.775463in}}{\pgfqpoint{9.478235in}{1.771897in}}%
\pgfpathcurveto{\pgfqpoint{9.474668in}{1.768330in}}{\pgfqpoint{9.472664in}{1.763493in}}{\pgfqpoint{9.472664in}{1.758449in}}%
\pgfpathcurveto{\pgfqpoint{9.472664in}{1.753405in}}{\pgfqpoint{9.474668in}{1.748567in}}{\pgfqpoint{9.478235in}{1.745001in}}%
\pgfpathcurveto{\pgfqpoint{9.481801in}{1.741435in}}{\pgfqpoint{9.486639in}{1.739431in}}{\pgfqpoint{9.491682in}{1.739431in}}%
\pgfpathclose%
\pgfusepath{fill}%
\end{pgfscope}%
\begin{pgfscope}%
\pgfpathrectangle{\pgfqpoint{6.572727in}{0.474100in}}{\pgfqpoint{4.227273in}{3.318700in}}%
\pgfusepath{clip}%
\pgfsetbuttcap%
\pgfsetroundjoin%
\definecolor{currentfill}{rgb}{0.127568,0.566949,0.550556}%
\pgfsetfillcolor{currentfill}%
\pgfsetfillopacity{0.700000}%
\pgfsetlinewidth{0.000000pt}%
\definecolor{currentstroke}{rgb}{0.000000,0.000000,0.000000}%
\pgfsetstrokecolor{currentstroke}%
\pgfsetstrokeopacity{0.700000}%
\pgfsetdash{}{0pt}%
\pgfpathmoveto{\pgfqpoint{8.950341in}{1.323117in}}%
\pgfpathcurveto{\pgfqpoint{8.955385in}{1.323117in}}{\pgfqpoint{8.960223in}{1.325120in}}{\pgfqpoint{8.963789in}{1.328687in}}%
\pgfpathcurveto{\pgfqpoint{8.967356in}{1.332253in}}{\pgfqpoint{8.969360in}{1.337091in}}{\pgfqpoint{8.969360in}{1.342135in}}%
\pgfpathcurveto{\pgfqpoint{8.969360in}{1.347178in}}{\pgfqpoint{8.967356in}{1.352016in}}{\pgfqpoint{8.963789in}{1.355583in}}%
\pgfpathcurveto{\pgfqpoint{8.960223in}{1.359149in}}{\pgfqpoint{8.955385in}{1.361153in}}{\pgfqpoint{8.950341in}{1.361153in}}%
\pgfpathcurveto{\pgfqpoint{8.945298in}{1.361153in}}{\pgfqpoint{8.940460in}{1.359149in}}{\pgfqpoint{8.936894in}{1.355583in}}%
\pgfpathcurveto{\pgfqpoint{8.933327in}{1.352016in}}{\pgfqpoint{8.931323in}{1.347178in}}{\pgfqpoint{8.931323in}{1.342135in}}%
\pgfpathcurveto{\pgfqpoint{8.931323in}{1.337091in}}{\pgfqpoint{8.933327in}{1.332253in}}{\pgfqpoint{8.936894in}{1.328687in}}%
\pgfpathcurveto{\pgfqpoint{8.940460in}{1.325120in}}{\pgfqpoint{8.945298in}{1.323117in}}{\pgfqpoint{8.950341in}{1.323117in}}%
\pgfpathclose%
\pgfusepath{fill}%
\end{pgfscope}%
\begin{pgfscope}%
\pgfpathrectangle{\pgfqpoint{6.572727in}{0.474100in}}{\pgfqpoint{4.227273in}{3.318700in}}%
\pgfusepath{clip}%
\pgfsetbuttcap%
\pgfsetroundjoin%
\definecolor{currentfill}{rgb}{0.993248,0.906157,0.143936}%
\pgfsetfillcolor{currentfill}%
\pgfsetfillopacity{0.700000}%
\pgfsetlinewidth{0.000000pt}%
\definecolor{currentstroke}{rgb}{0.000000,0.000000,0.000000}%
\pgfsetstrokecolor{currentstroke}%
\pgfsetstrokeopacity{0.700000}%
\pgfsetdash{}{0pt}%
\pgfpathmoveto{\pgfqpoint{8.587236in}{2.927957in}}%
\pgfpathcurveto{\pgfqpoint{8.592280in}{2.927957in}}{\pgfqpoint{8.597117in}{2.929960in}}{\pgfqpoint{8.600684in}{2.933527in}}%
\pgfpathcurveto{\pgfqpoint{8.604250in}{2.937093in}}{\pgfqpoint{8.606254in}{2.941931in}}{\pgfqpoint{8.606254in}{2.946975in}}%
\pgfpathcurveto{\pgfqpoint{8.606254in}{2.952018in}}{\pgfqpoint{8.604250in}{2.956856in}}{\pgfqpoint{8.600684in}{2.960423in}}%
\pgfpathcurveto{\pgfqpoint{8.597117in}{2.963989in}}{\pgfqpoint{8.592280in}{2.965993in}}{\pgfqpoint{8.587236in}{2.965993in}}%
\pgfpathcurveto{\pgfqpoint{8.582192in}{2.965993in}}{\pgfqpoint{8.577354in}{2.963989in}}{\pgfqpoint{8.573788in}{2.960423in}}%
\pgfpathcurveto{\pgfqpoint{8.570222in}{2.956856in}}{\pgfqpoint{8.568218in}{2.952018in}}{\pgfqpoint{8.568218in}{2.946975in}}%
\pgfpathcurveto{\pgfqpoint{8.568218in}{2.941931in}}{\pgfqpoint{8.570222in}{2.937093in}}{\pgfqpoint{8.573788in}{2.933527in}}%
\pgfpathcurveto{\pgfqpoint{8.577354in}{2.929960in}}{\pgfqpoint{8.582192in}{2.927957in}}{\pgfqpoint{8.587236in}{2.927957in}}%
\pgfpathclose%
\pgfusepath{fill}%
\end{pgfscope}%
\begin{pgfscope}%
\pgfpathrectangle{\pgfqpoint{6.572727in}{0.474100in}}{\pgfqpoint{4.227273in}{3.318700in}}%
\pgfusepath{clip}%
\pgfsetbuttcap%
\pgfsetroundjoin%
\definecolor{currentfill}{rgb}{0.127568,0.566949,0.550556}%
\pgfsetfillcolor{currentfill}%
\pgfsetfillopacity{0.700000}%
\pgfsetlinewidth{0.000000pt}%
\definecolor{currentstroke}{rgb}{0.000000,0.000000,0.000000}%
\pgfsetstrokecolor{currentstroke}%
\pgfsetstrokeopacity{0.700000}%
\pgfsetdash{}{0pt}%
\pgfpathmoveto{\pgfqpoint{10.585804in}{1.305801in}}%
\pgfpathcurveto{\pgfqpoint{10.590848in}{1.305801in}}{\pgfqpoint{10.595686in}{1.307804in}}{\pgfqpoint{10.599252in}{1.311371in}}%
\pgfpathcurveto{\pgfqpoint{10.602818in}{1.314937in}}{\pgfqpoint{10.604822in}{1.319775in}}{\pgfqpoint{10.604822in}{1.324819in}}%
\pgfpathcurveto{\pgfqpoint{10.604822in}{1.329862in}}{\pgfqpoint{10.602818in}{1.334700in}}{\pgfqpoint{10.599252in}{1.338267in}}%
\pgfpathcurveto{\pgfqpoint{10.595686in}{1.341833in}}{\pgfqpoint{10.590848in}{1.343837in}}{\pgfqpoint{10.585804in}{1.343837in}}%
\pgfpathcurveto{\pgfqpoint{10.580760in}{1.343837in}}{\pgfqpoint{10.575923in}{1.341833in}}{\pgfqpoint{10.572356in}{1.338267in}}%
\pgfpathcurveto{\pgfqpoint{10.568790in}{1.334700in}}{\pgfqpoint{10.566786in}{1.329862in}}{\pgfqpoint{10.566786in}{1.324819in}}%
\pgfpathcurveto{\pgfqpoint{10.566786in}{1.319775in}}{\pgfqpoint{10.568790in}{1.314937in}}{\pgfqpoint{10.572356in}{1.311371in}}%
\pgfpathcurveto{\pgfqpoint{10.575923in}{1.307804in}}{\pgfqpoint{10.580760in}{1.305801in}}{\pgfqpoint{10.585804in}{1.305801in}}%
\pgfpathclose%
\pgfusepath{fill}%
\end{pgfscope}%
\begin{pgfscope}%
\pgfpathrectangle{\pgfqpoint{6.572727in}{0.474100in}}{\pgfqpoint{4.227273in}{3.318700in}}%
\pgfusepath{clip}%
\pgfsetbuttcap%
\pgfsetroundjoin%
\definecolor{currentfill}{rgb}{0.267004,0.004874,0.329415}%
\pgfsetfillcolor{currentfill}%
\pgfsetfillopacity{0.700000}%
\pgfsetlinewidth{0.000000pt}%
\definecolor{currentstroke}{rgb}{0.000000,0.000000,0.000000}%
\pgfsetstrokecolor{currentstroke}%
\pgfsetstrokeopacity{0.700000}%
\pgfsetdash{}{0pt}%
\pgfpathmoveto{\pgfqpoint{7.754195in}{1.367224in}}%
\pgfpathcurveto{\pgfqpoint{7.759238in}{1.367224in}}{\pgfqpoint{7.764076in}{1.369228in}}{\pgfqpoint{7.767643in}{1.372795in}}%
\pgfpathcurveto{\pgfqpoint{7.771209in}{1.376361in}}{\pgfqpoint{7.773213in}{1.381199in}}{\pgfqpoint{7.773213in}{1.386242in}}%
\pgfpathcurveto{\pgfqpoint{7.773213in}{1.391286in}}{\pgfqpoint{7.771209in}{1.396124in}}{\pgfqpoint{7.767643in}{1.399690in}}%
\pgfpathcurveto{\pgfqpoint{7.764076in}{1.403257in}}{\pgfqpoint{7.759238in}{1.405261in}}{\pgfqpoint{7.754195in}{1.405261in}}%
\pgfpathcurveto{\pgfqpoint{7.749151in}{1.405261in}}{\pgfqpoint{7.744313in}{1.403257in}}{\pgfqpoint{7.740747in}{1.399690in}}%
\pgfpathcurveto{\pgfqpoint{7.737180in}{1.396124in}}{\pgfqpoint{7.735177in}{1.391286in}}{\pgfqpoint{7.735177in}{1.386242in}}%
\pgfpathcurveto{\pgfqpoint{7.735177in}{1.381199in}}{\pgfqpoint{7.737180in}{1.376361in}}{\pgfqpoint{7.740747in}{1.372795in}}%
\pgfpathcurveto{\pgfqpoint{7.744313in}{1.369228in}}{\pgfqpoint{7.749151in}{1.367224in}}{\pgfqpoint{7.754195in}{1.367224in}}%
\pgfpathclose%
\pgfusepath{fill}%
\end{pgfscope}%
\begin{pgfscope}%
\pgfpathrectangle{\pgfqpoint{6.572727in}{0.474100in}}{\pgfqpoint{4.227273in}{3.318700in}}%
\pgfusepath{clip}%
\pgfsetbuttcap%
\pgfsetroundjoin%
\definecolor{currentfill}{rgb}{0.993248,0.906157,0.143936}%
\pgfsetfillcolor{currentfill}%
\pgfsetfillopacity{0.700000}%
\pgfsetlinewidth{0.000000pt}%
\definecolor{currentstroke}{rgb}{0.000000,0.000000,0.000000}%
\pgfsetstrokecolor{currentstroke}%
\pgfsetstrokeopacity{0.700000}%
\pgfsetdash{}{0pt}%
\pgfpathmoveto{\pgfqpoint{8.103378in}{2.845594in}}%
\pgfpathcurveto{\pgfqpoint{8.108421in}{2.845594in}}{\pgfqpoint{8.113259in}{2.847598in}}{\pgfqpoint{8.116826in}{2.851164in}}%
\pgfpathcurveto{\pgfqpoint{8.120392in}{2.854731in}}{\pgfqpoint{8.122396in}{2.859569in}}{\pgfqpoint{8.122396in}{2.864612in}}%
\pgfpathcurveto{\pgfqpoint{8.122396in}{2.869656in}}{\pgfqpoint{8.120392in}{2.874494in}}{\pgfqpoint{8.116826in}{2.878060in}}%
\pgfpathcurveto{\pgfqpoint{8.113259in}{2.881627in}}{\pgfqpoint{8.108421in}{2.883630in}}{\pgfqpoint{8.103378in}{2.883630in}}%
\pgfpathcurveto{\pgfqpoint{8.098334in}{2.883630in}}{\pgfqpoint{8.093496in}{2.881627in}}{\pgfqpoint{8.089930in}{2.878060in}}%
\pgfpathcurveto{\pgfqpoint{8.086364in}{2.874494in}}{\pgfqpoint{8.084360in}{2.869656in}}{\pgfqpoint{8.084360in}{2.864612in}}%
\pgfpathcurveto{\pgfqpoint{8.084360in}{2.859569in}}{\pgfqpoint{8.086364in}{2.854731in}}{\pgfqpoint{8.089930in}{2.851164in}}%
\pgfpathcurveto{\pgfqpoint{8.093496in}{2.847598in}}{\pgfqpoint{8.098334in}{2.845594in}}{\pgfqpoint{8.103378in}{2.845594in}}%
\pgfpathclose%
\pgfusepath{fill}%
\end{pgfscope}%
\begin{pgfscope}%
\pgfpathrectangle{\pgfqpoint{6.572727in}{0.474100in}}{\pgfqpoint{4.227273in}{3.318700in}}%
\pgfusepath{clip}%
\pgfsetbuttcap%
\pgfsetroundjoin%
\definecolor{currentfill}{rgb}{0.127568,0.566949,0.550556}%
\pgfsetfillcolor{currentfill}%
\pgfsetfillopacity{0.700000}%
\pgfsetlinewidth{0.000000pt}%
\definecolor{currentstroke}{rgb}{0.000000,0.000000,0.000000}%
\pgfsetstrokecolor{currentstroke}%
\pgfsetstrokeopacity{0.700000}%
\pgfsetdash{}{0pt}%
\pgfpathmoveto{\pgfqpoint{10.005022in}{1.611664in}}%
\pgfpathcurveto{\pgfqpoint{10.010066in}{1.611664in}}{\pgfqpoint{10.014904in}{1.613668in}}{\pgfqpoint{10.018470in}{1.617234in}}%
\pgfpathcurveto{\pgfqpoint{10.022037in}{1.620801in}}{\pgfqpoint{10.024041in}{1.625639in}}{\pgfqpoint{10.024041in}{1.630682in}}%
\pgfpathcurveto{\pgfqpoint{10.024041in}{1.635726in}}{\pgfqpoint{10.022037in}{1.640564in}}{\pgfqpoint{10.018470in}{1.644130in}}%
\pgfpathcurveto{\pgfqpoint{10.014904in}{1.647697in}}{\pgfqpoint{10.010066in}{1.649700in}}{\pgfqpoint{10.005022in}{1.649700in}}%
\pgfpathcurveto{\pgfqpoint{9.999979in}{1.649700in}}{\pgfqpoint{9.995141in}{1.647697in}}{\pgfqpoint{9.991575in}{1.644130in}}%
\pgfpathcurveto{\pgfqpoint{9.988008in}{1.640564in}}{\pgfqpoint{9.986004in}{1.635726in}}{\pgfqpoint{9.986004in}{1.630682in}}%
\pgfpathcurveto{\pgfqpoint{9.986004in}{1.625639in}}{\pgfqpoint{9.988008in}{1.620801in}}{\pgfqpoint{9.991575in}{1.617234in}}%
\pgfpathcurveto{\pgfqpoint{9.995141in}{1.613668in}}{\pgfqpoint{9.999979in}{1.611664in}}{\pgfqpoint{10.005022in}{1.611664in}}%
\pgfpathclose%
\pgfusepath{fill}%
\end{pgfscope}%
\begin{pgfscope}%
\pgfpathrectangle{\pgfqpoint{6.572727in}{0.474100in}}{\pgfqpoint{4.227273in}{3.318700in}}%
\pgfusepath{clip}%
\pgfsetbuttcap%
\pgfsetroundjoin%
\definecolor{currentfill}{rgb}{0.993248,0.906157,0.143936}%
\pgfsetfillcolor{currentfill}%
\pgfsetfillopacity{0.700000}%
\pgfsetlinewidth{0.000000pt}%
\definecolor{currentstroke}{rgb}{0.000000,0.000000,0.000000}%
\pgfsetstrokecolor{currentstroke}%
\pgfsetstrokeopacity{0.700000}%
\pgfsetdash{}{0pt}%
\pgfpathmoveto{\pgfqpoint{9.120544in}{3.130671in}}%
\pgfpathcurveto{\pgfqpoint{9.125587in}{3.130671in}}{\pgfqpoint{9.130425in}{3.132675in}}{\pgfqpoint{9.133992in}{3.136241in}}%
\pgfpathcurveto{\pgfqpoint{9.137558in}{3.139807in}}{\pgfqpoint{9.139562in}{3.144645in}}{\pgfqpoint{9.139562in}{3.149689in}}%
\pgfpathcurveto{\pgfqpoint{9.139562in}{3.154732in}}{\pgfqpoint{9.137558in}{3.159570in}}{\pgfqpoint{9.133992in}{3.163137in}}%
\pgfpathcurveto{\pgfqpoint{9.130425in}{3.166703in}}{\pgfqpoint{9.125587in}{3.168707in}}{\pgfqpoint{9.120544in}{3.168707in}}%
\pgfpathcurveto{\pgfqpoint{9.115500in}{3.168707in}}{\pgfqpoint{9.110662in}{3.166703in}}{\pgfqpoint{9.107096in}{3.163137in}}%
\pgfpathcurveto{\pgfqpoint{9.103529in}{3.159570in}}{\pgfqpoint{9.101526in}{3.154732in}}{\pgfqpoint{9.101526in}{3.149689in}}%
\pgfpathcurveto{\pgfqpoint{9.101526in}{3.144645in}}{\pgfqpoint{9.103529in}{3.139807in}}{\pgfqpoint{9.107096in}{3.136241in}}%
\pgfpathcurveto{\pgfqpoint{9.110662in}{3.132675in}}{\pgfqpoint{9.115500in}{3.130671in}}{\pgfqpoint{9.120544in}{3.130671in}}%
\pgfpathclose%
\pgfusepath{fill}%
\end{pgfscope}%
\begin{pgfscope}%
\pgfpathrectangle{\pgfqpoint{6.572727in}{0.474100in}}{\pgfqpoint{4.227273in}{3.318700in}}%
\pgfusepath{clip}%
\pgfsetbuttcap%
\pgfsetroundjoin%
\definecolor{currentfill}{rgb}{0.267004,0.004874,0.329415}%
\pgfsetfillcolor{currentfill}%
\pgfsetfillopacity{0.700000}%
\pgfsetlinewidth{0.000000pt}%
\definecolor{currentstroke}{rgb}{0.000000,0.000000,0.000000}%
\pgfsetstrokecolor{currentstroke}%
\pgfsetstrokeopacity{0.700000}%
\pgfsetdash{}{0pt}%
\pgfpathmoveto{\pgfqpoint{7.971187in}{1.676922in}}%
\pgfpathcurveto{\pgfqpoint{7.976230in}{1.676922in}}{\pgfqpoint{7.981068in}{1.678926in}}{\pgfqpoint{7.984634in}{1.682492in}}%
\pgfpathcurveto{\pgfqpoint{7.988201in}{1.686059in}}{\pgfqpoint{7.990205in}{1.690896in}}{\pgfqpoint{7.990205in}{1.695940in}}%
\pgfpathcurveto{\pgfqpoint{7.990205in}{1.700984in}}{\pgfqpoint{7.988201in}{1.705821in}}{\pgfqpoint{7.984634in}{1.709388in}}%
\pgfpathcurveto{\pgfqpoint{7.981068in}{1.712954in}}{\pgfqpoint{7.976230in}{1.714958in}}{\pgfqpoint{7.971187in}{1.714958in}}%
\pgfpathcurveto{\pgfqpoint{7.966143in}{1.714958in}}{\pgfqpoint{7.961305in}{1.712954in}}{\pgfqpoint{7.957739in}{1.709388in}}%
\pgfpathcurveto{\pgfqpoint{7.954172in}{1.705821in}}{\pgfqpoint{7.952168in}{1.700984in}}{\pgfqpoint{7.952168in}{1.695940in}}%
\pgfpathcurveto{\pgfqpoint{7.952168in}{1.690896in}}{\pgfqpoint{7.954172in}{1.686059in}}{\pgfqpoint{7.957739in}{1.682492in}}%
\pgfpathcurveto{\pgfqpoint{7.961305in}{1.678926in}}{\pgfqpoint{7.966143in}{1.676922in}}{\pgfqpoint{7.971187in}{1.676922in}}%
\pgfpathclose%
\pgfusepath{fill}%
\end{pgfscope}%
\begin{pgfscope}%
\pgfpathrectangle{\pgfqpoint{6.572727in}{0.474100in}}{\pgfqpoint{4.227273in}{3.318700in}}%
\pgfusepath{clip}%
\pgfsetbuttcap%
\pgfsetroundjoin%
\definecolor{currentfill}{rgb}{0.127568,0.566949,0.550556}%
\pgfsetfillcolor{currentfill}%
\pgfsetfillopacity{0.700000}%
\pgfsetlinewidth{0.000000pt}%
\definecolor{currentstroke}{rgb}{0.000000,0.000000,0.000000}%
\pgfsetstrokecolor{currentstroke}%
\pgfsetstrokeopacity{0.700000}%
\pgfsetdash{}{0pt}%
\pgfpathmoveto{\pgfqpoint{9.790444in}{1.436545in}}%
\pgfpathcurveto{\pgfqpoint{9.795488in}{1.436545in}}{\pgfqpoint{9.800326in}{1.438549in}}{\pgfqpoint{9.803892in}{1.442115in}}%
\pgfpathcurveto{\pgfqpoint{9.807459in}{1.445682in}}{\pgfqpoint{9.809463in}{1.450520in}}{\pgfqpoint{9.809463in}{1.455563in}}%
\pgfpathcurveto{\pgfqpoint{9.809463in}{1.460607in}}{\pgfqpoint{9.807459in}{1.465445in}}{\pgfqpoint{9.803892in}{1.469011in}}%
\pgfpathcurveto{\pgfqpoint{9.800326in}{1.472578in}}{\pgfqpoint{9.795488in}{1.474581in}}{\pgfqpoint{9.790444in}{1.474581in}}%
\pgfpathcurveto{\pgfqpoint{9.785401in}{1.474581in}}{\pgfqpoint{9.780563in}{1.472578in}}{\pgfqpoint{9.776997in}{1.469011in}}%
\pgfpathcurveto{\pgfqpoint{9.773430in}{1.465445in}}{\pgfqpoint{9.771426in}{1.460607in}}{\pgfqpoint{9.771426in}{1.455563in}}%
\pgfpathcurveto{\pgfqpoint{9.771426in}{1.450520in}}{\pgfqpoint{9.773430in}{1.445682in}}{\pgfqpoint{9.776997in}{1.442115in}}%
\pgfpathcurveto{\pgfqpoint{9.780563in}{1.438549in}}{\pgfqpoint{9.785401in}{1.436545in}}{\pgfqpoint{9.790444in}{1.436545in}}%
\pgfpathclose%
\pgfusepath{fill}%
\end{pgfscope}%
\begin{pgfscope}%
\pgfpathrectangle{\pgfqpoint{6.572727in}{0.474100in}}{\pgfqpoint{4.227273in}{3.318700in}}%
\pgfusepath{clip}%
\pgfsetbuttcap%
\pgfsetroundjoin%
\definecolor{currentfill}{rgb}{0.127568,0.566949,0.550556}%
\pgfsetfillcolor{currentfill}%
\pgfsetfillopacity{0.700000}%
\pgfsetlinewidth{0.000000pt}%
\definecolor{currentstroke}{rgb}{0.000000,0.000000,0.000000}%
\pgfsetstrokecolor{currentstroke}%
\pgfsetstrokeopacity{0.700000}%
\pgfsetdash{}{0pt}%
\pgfpathmoveto{\pgfqpoint{9.639419in}{2.179955in}}%
\pgfpathcurveto{\pgfqpoint{9.644463in}{2.179955in}}{\pgfqpoint{9.649300in}{2.181959in}}{\pgfqpoint{9.652867in}{2.185525in}}%
\pgfpathcurveto{\pgfqpoint{9.656433in}{2.189092in}}{\pgfqpoint{9.658437in}{2.193929in}}{\pgfqpoint{9.658437in}{2.198973in}}%
\pgfpathcurveto{\pgfqpoint{9.658437in}{2.204017in}}{\pgfqpoint{9.656433in}{2.208855in}}{\pgfqpoint{9.652867in}{2.212421in}}%
\pgfpathcurveto{\pgfqpoint{9.649300in}{2.215987in}}{\pgfqpoint{9.644463in}{2.217991in}}{\pgfqpoint{9.639419in}{2.217991in}}%
\pgfpathcurveto{\pgfqpoint{9.634375in}{2.217991in}}{\pgfqpoint{9.629538in}{2.215987in}}{\pgfqpoint{9.625971in}{2.212421in}}%
\pgfpathcurveto{\pgfqpoint{9.622405in}{2.208855in}}{\pgfqpoint{9.620401in}{2.204017in}}{\pgfqpoint{9.620401in}{2.198973in}}%
\pgfpathcurveto{\pgfqpoint{9.620401in}{2.193929in}}{\pgfqpoint{9.622405in}{2.189092in}}{\pgfqpoint{9.625971in}{2.185525in}}%
\pgfpathcurveto{\pgfqpoint{9.629538in}{2.181959in}}{\pgfqpoint{9.634375in}{2.179955in}}{\pgfqpoint{9.639419in}{2.179955in}}%
\pgfpathclose%
\pgfusepath{fill}%
\end{pgfscope}%
\begin{pgfscope}%
\pgfpathrectangle{\pgfqpoint{6.572727in}{0.474100in}}{\pgfqpoint{4.227273in}{3.318700in}}%
\pgfusepath{clip}%
\pgfsetbuttcap%
\pgfsetroundjoin%
\definecolor{currentfill}{rgb}{0.993248,0.906157,0.143936}%
\pgfsetfillcolor{currentfill}%
\pgfsetfillopacity{0.700000}%
\pgfsetlinewidth{0.000000pt}%
\definecolor{currentstroke}{rgb}{0.000000,0.000000,0.000000}%
\pgfsetstrokecolor{currentstroke}%
\pgfsetstrokeopacity{0.700000}%
\pgfsetdash{}{0pt}%
\pgfpathmoveto{\pgfqpoint{7.990502in}{3.329154in}}%
\pgfpathcurveto{\pgfqpoint{7.995546in}{3.329154in}}{\pgfqpoint{8.000383in}{3.331158in}}{\pgfqpoint{8.003950in}{3.334724in}}%
\pgfpathcurveto{\pgfqpoint{8.007516in}{3.338291in}}{\pgfqpoint{8.009520in}{3.343128in}}{\pgfqpoint{8.009520in}{3.348172in}}%
\pgfpathcurveto{\pgfqpoint{8.009520in}{3.353216in}}{\pgfqpoint{8.007516in}{3.358053in}}{\pgfqpoint{8.003950in}{3.361620in}}%
\pgfpathcurveto{\pgfqpoint{8.000383in}{3.365186in}}{\pgfqpoint{7.995546in}{3.367190in}}{\pgfqpoint{7.990502in}{3.367190in}}%
\pgfpathcurveto{\pgfqpoint{7.985458in}{3.367190in}}{\pgfqpoint{7.980621in}{3.365186in}}{\pgfqpoint{7.977054in}{3.361620in}}%
\pgfpathcurveto{\pgfqpoint{7.973488in}{3.358053in}}{\pgfqpoint{7.971484in}{3.353216in}}{\pgfqpoint{7.971484in}{3.348172in}}%
\pgfpathcurveto{\pgfqpoint{7.971484in}{3.343128in}}{\pgfqpoint{7.973488in}{3.338291in}}{\pgfqpoint{7.977054in}{3.334724in}}%
\pgfpathcurveto{\pgfqpoint{7.980621in}{3.331158in}}{\pgfqpoint{7.985458in}{3.329154in}}{\pgfqpoint{7.990502in}{3.329154in}}%
\pgfpathclose%
\pgfusepath{fill}%
\end{pgfscope}%
\begin{pgfscope}%
\pgfpathrectangle{\pgfqpoint{6.572727in}{0.474100in}}{\pgfqpoint{4.227273in}{3.318700in}}%
\pgfusepath{clip}%
\pgfsetbuttcap%
\pgfsetroundjoin%
\definecolor{currentfill}{rgb}{0.127568,0.566949,0.550556}%
\pgfsetfillcolor{currentfill}%
\pgfsetfillopacity{0.700000}%
\pgfsetlinewidth{0.000000pt}%
\definecolor{currentstroke}{rgb}{0.000000,0.000000,0.000000}%
\pgfsetstrokecolor{currentstroke}%
\pgfsetstrokeopacity{0.700000}%
\pgfsetdash{}{0pt}%
\pgfpathmoveto{\pgfqpoint{9.177129in}{1.651611in}}%
\pgfpathcurveto{\pgfqpoint{9.182173in}{1.651611in}}{\pgfqpoint{9.187011in}{1.653615in}}{\pgfqpoint{9.190577in}{1.657181in}}%
\pgfpathcurveto{\pgfqpoint{9.194144in}{1.660748in}}{\pgfqpoint{9.196148in}{1.665585in}}{\pgfqpoint{9.196148in}{1.670629in}}%
\pgfpathcurveto{\pgfqpoint{9.196148in}{1.675673in}}{\pgfqpoint{9.194144in}{1.680511in}}{\pgfqpoint{9.190577in}{1.684077in}}%
\pgfpathcurveto{\pgfqpoint{9.187011in}{1.687643in}}{\pgfqpoint{9.182173in}{1.689647in}}{\pgfqpoint{9.177129in}{1.689647in}}%
\pgfpathcurveto{\pgfqpoint{9.172086in}{1.689647in}}{\pgfqpoint{9.167248in}{1.687643in}}{\pgfqpoint{9.163682in}{1.684077in}}%
\pgfpathcurveto{\pgfqpoint{9.160115in}{1.680511in}}{\pgfqpoint{9.158111in}{1.675673in}}{\pgfqpoint{9.158111in}{1.670629in}}%
\pgfpathcurveto{\pgfqpoint{9.158111in}{1.665585in}}{\pgfqpoint{9.160115in}{1.660748in}}{\pgfqpoint{9.163682in}{1.657181in}}%
\pgfpathcurveto{\pgfqpoint{9.167248in}{1.653615in}}{\pgfqpoint{9.172086in}{1.651611in}}{\pgfqpoint{9.177129in}{1.651611in}}%
\pgfpathclose%
\pgfusepath{fill}%
\end{pgfscope}%
\begin{pgfscope}%
\pgfpathrectangle{\pgfqpoint{6.572727in}{0.474100in}}{\pgfqpoint{4.227273in}{3.318700in}}%
\pgfusepath{clip}%
\pgfsetbuttcap%
\pgfsetroundjoin%
\definecolor{currentfill}{rgb}{0.127568,0.566949,0.550556}%
\pgfsetfillcolor{currentfill}%
\pgfsetfillopacity{0.700000}%
\pgfsetlinewidth{0.000000pt}%
\definecolor{currentstroke}{rgb}{0.000000,0.000000,0.000000}%
\pgfsetstrokecolor{currentstroke}%
\pgfsetstrokeopacity{0.700000}%
\pgfsetdash{}{0pt}%
\pgfpathmoveto{\pgfqpoint{9.986951in}{1.493844in}}%
\pgfpathcurveto{\pgfqpoint{9.991995in}{1.493844in}}{\pgfqpoint{9.996833in}{1.495848in}}{\pgfqpoint{10.000399in}{1.499414in}}%
\pgfpathcurveto{\pgfqpoint{10.003966in}{1.502980in}}{\pgfqpoint{10.005969in}{1.507818in}}{\pgfqpoint{10.005969in}{1.512862in}}%
\pgfpathcurveto{\pgfqpoint{10.005969in}{1.517905in}}{\pgfqpoint{10.003966in}{1.522743in}}{\pgfqpoint{10.000399in}{1.526310in}}%
\pgfpathcurveto{\pgfqpoint{9.996833in}{1.529876in}}{\pgfqpoint{9.991995in}{1.531880in}}{\pgfqpoint{9.986951in}{1.531880in}}%
\pgfpathcurveto{\pgfqpoint{9.981908in}{1.531880in}}{\pgfqpoint{9.977070in}{1.529876in}}{\pgfqpoint{9.973503in}{1.526310in}}%
\pgfpathcurveto{\pgfqpoint{9.969937in}{1.522743in}}{\pgfqpoint{9.967933in}{1.517905in}}{\pgfqpoint{9.967933in}{1.512862in}}%
\pgfpathcurveto{\pgfqpoint{9.967933in}{1.507818in}}{\pgfqpoint{9.969937in}{1.502980in}}{\pgfqpoint{9.973503in}{1.499414in}}%
\pgfpathcurveto{\pgfqpoint{9.977070in}{1.495848in}}{\pgfqpoint{9.981908in}{1.493844in}}{\pgfqpoint{9.986951in}{1.493844in}}%
\pgfpathclose%
\pgfusepath{fill}%
\end{pgfscope}%
\begin{pgfscope}%
\pgfpathrectangle{\pgfqpoint{6.572727in}{0.474100in}}{\pgfqpoint{4.227273in}{3.318700in}}%
\pgfusepath{clip}%
\pgfsetbuttcap%
\pgfsetroundjoin%
\definecolor{currentfill}{rgb}{0.127568,0.566949,0.550556}%
\pgfsetfillcolor{currentfill}%
\pgfsetfillopacity{0.700000}%
\pgfsetlinewidth{0.000000pt}%
\definecolor{currentstroke}{rgb}{0.000000,0.000000,0.000000}%
\pgfsetstrokecolor{currentstroke}%
\pgfsetstrokeopacity{0.700000}%
\pgfsetdash{}{0pt}%
\pgfpathmoveto{\pgfqpoint{9.387374in}{1.176081in}}%
\pgfpathcurveto{\pgfqpoint{9.392418in}{1.176081in}}{\pgfqpoint{9.397256in}{1.178084in}}{\pgfqpoint{9.400822in}{1.181651in}}%
\pgfpathcurveto{\pgfqpoint{9.404389in}{1.185217in}}{\pgfqpoint{9.406393in}{1.190055in}}{\pgfqpoint{9.406393in}{1.195099in}}%
\pgfpathcurveto{\pgfqpoint{9.406393in}{1.200142in}}{\pgfqpoint{9.404389in}{1.204980in}}{\pgfqpoint{9.400822in}{1.208547in}}%
\pgfpathcurveto{\pgfqpoint{9.397256in}{1.212113in}}{\pgfqpoint{9.392418in}{1.214117in}}{\pgfqpoint{9.387374in}{1.214117in}}%
\pgfpathcurveto{\pgfqpoint{9.382331in}{1.214117in}}{\pgfqpoint{9.377493in}{1.212113in}}{\pgfqpoint{9.373927in}{1.208547in}}%
\pgfpathcurveto{\pgfqpoint{9.370360in}{1.204980in}}{\pgfqpoint{9.368356in}{1.200142in}}{\pgfqpoint{9.368356in}{1.195099in}}%
\pgfpathcurveto{\pgfqpoint{9.368356in}{1.190055in}}{\pgfqpoint{9.370360in}{1.185217in}}{\pgfqpoint{9.373927in}{1.181651in}}%
\pgfpathcurveto{\pgfqpoint{9.377493in}{1.178084in}}{\pgfqpoint{9.382331in}{1.176081in}}{\pgfqpoint{9.387374in}{1.176081in}}%
\pgfpathclose%
\pgfusepath{fill}%
\end{pgfscope}%
\begin{pgfscope}%
\pgfpathrectangle{\pgfqpoint{6.572727in}{0.474100in}}{\pgfqpoint{4.227273in}{3.318700in}}%
\pgfusepath{clip}%
\pgfsetbuttcap%
\pgfsetroundjoin%
\definecolor{currentfill}{rgb}{0.993248,0.906157,0.143936}%
\pgfsetfillcolor{currentfill}%
\pgfsetfillopacity{0.700000}%
\pgfsetlinewidth{0.000000pt}%
\definecolor{currentstroke}{rgb}{0.000000,0.000000,0.000000}%
\pgfsetstrokecolor{currentstroke}%
\pgfsetstrokeopacity{0.700000}%
\pgfsetdash{}{0pt}%
\pgfpathmoveto{\pgfqpoint{8.347673in}{2.781042in}}%
\pgfpathcurveto{\pgfqpoint{8.352717in}{2.781042in}}{\pgfqpoint{8.357555in}{2.783046in}}{\pgfqpoint{8.361121in}{2.786612in}}%
\pgfpathcurveto{\pgfqpoint{8.364688in}{2.790178in}}{\pgfqpoint{8.366692in}{2.795016in}}{\pgfqpoint{8.366692in}{2.800060in}}%
\pgfpathcurveto{\pgfqpoint{8.366692in}{2.805104in}}{\pgfqpoint{8.364688in}{2.809941in}}{\pgfqpoint{8.361121in}{2.813508in}}%
\pgfpathcurveto{\pgfqpoint{8.357555in}{2.817074in}}{\pgfqpoint{8.352717in}{2.819078in}}{\pgfqpoint{8.347673in}{2.819078in}}%
\pgfpathcurveto{\pgfqpoint{8.342630in}{2.819078in}}{\pgfqpoint{8.337792in}{2.817074in}}{\pgfqpoint{8.334226in}{2.813508in}}%
\pgfpathcurveto{\pgfqpoint{8.330659in}{2.809941in}}{\pgfqpoint{8.328655in}{2.805104in}}{\pgfqpoint{8.328655in}{2.800060in}}%
\pgfpathcurveto{\pgfqpoint{8.328655in}{2.795016in}}{\pgfqpoint{8.330659in}{2.790178in}}{\pgfqpoint{8.334226in}{2.786612in}}%
\pgfpathcurveto{\pgfqpoint{8.337792in}{2.783046in}}{\pgfqpoint{8.342630in}{2.781042in}}{\pgfqpoint{8.347673in}{2.781042in}}%
\pgfpathclose%
\pgfusepath{fill}%
\end{pgfscope}%
\begin{pgfscope}%
\pgfpathrectangle{\pgfqpoint{6.572727in}{0.474100in}}{\pgfqpoint{4.227273in}{3.318700in}}%
\pgfusepath{clip}%
\pgfsetbuttcap%
\pgfsetroundjoin%
\definecolor{currentfill}{rgb}{0.127568,0.566949,0.550556}%
\pgfsetfillcolor{currentfill}%
\pgfsetfillopacity{0.700000}%
\pgfsetlinewidth{0.000000pt}%
\definecolor{currentstroke}{rgb}{0.000000,0.000000,0.000000}%
\pgfsetstrokecolor{currentstroke}%
\pgfsetstrokeopacity{0.700000}%
\pgfsetdash{}{0pt}%
\pgfpathmoveto{\pgfqpoint{9.489269in}{1.817696in}}%
\pgfpathcurveto{\pgfqpoint{9.494313in}{1.817696in}}{\pgfqpoint{9.499151in}{1.819699in}}{\pgfqpoint{9.502717in}{1.823266in}}%
\pgfpathcurveto{\pgfqpoint{9.506284in}{1.826832in}}{\pgfqpoint{9.508288in}{1.831670in}}{\pgfqpoint{9.508288in}{1.836714in}}%
\pgfpathcurveto{\pgfqpoint{9.508288in}{1.841757in}}{\pgfqpoint{9.506284in}{1.846595in}}{\pgfqpoint{9.502717in}{1.850162in}}%
\pgfpathcurveto{\pgfqpoint{9.499151in}{1.853728in}}{\pgfqpoint{9.494313in}{1.855732in}}{\pgfqpoint{9.489269in}{1.855732in}}%
\pgfpathcurveto{\pgfqpoint{9.484226in}{1.855732in}}{\pgfqpoint{9.479388in}{1.853728in}}{\pgfqpoint{9.475822in}{1.850162in}}%
\pgfpathcurveto{\pgfqpoint{9.472255in}{1.846595in}}{\pgfqpoint{9.470251in}{1.841757in}}{\pgfqpoint{9.470251in}{1.836714in}}%
\pgfpathcurveto{\pgfqpoint{9.470251in}{1.831670in}}{\pgfqpoint{9.472255in}{1.826832in}}{\pgfqpoint{9.475822in}{1.823266in}}%
\pgfpathcurveto{\pgfqpoint{9.479388in}{1.819699in}}{\pgfqpoint{9.484226in}{1.817696in}}{\pgfqpoint{9.489269in}{1.817696in}}%
\pgfpathclose%
\pgfusepath{fill}%
\end{pgfscope}%
\begin{pgfscope}%
\pgfpathrectangle{\pgfqpoint{6.572727in}{0.474100in}}{\pgfqpoint{4.227273in}{3.318700in}}%
\pgfusepath{clip}%
\pgfsetbuttcap%
\pgfsetroundjoin%
\definecolor{currentfill}{rgb}{0.993248,0.906157,0.143936}%
\pgfsetfillcolor{currentfill}%
\pgfsetfillopacity{0.700000}%
\pgfsetlinewidth{0.000000pt}%
\definecolor{currentstroke}{rgb}{0.000000,0.000000,0.000000}%
\pgfsetstrokecolor{currentstroke}%
\pgfsetstrokeopacity{0.700000}%
\pgfsetdash{}{0pt}%
\pgfpathmoveto{\pgfqpoint{7.928335in}{3.454464in}}%
\pgfpathcurveto{\pgfqpoint{7.933379in}{3.454464in}}{\pgfqpoint{7.938216in}{3.456468in}}{\pgfqpoint{7.941783in}{3.460034in}}%
\pgfpathcurveto{\pgfqpoint{7.945349in}{3.463601in}}{\pgfqpoint{7.947353in}{3.468438in}}{\pgfqpoint{7.947353in}{3.473482in}}%
\pgfpathcurveto{\pgfqpoint{7.947353in}{3.478526in}}{\pgfqpoint{7.945349in}{3.483364in}}{\pgfqpoint{7.941783in}{3.486930in}}%
\pgfpathcurveto{\pgfqpoint{7.938216in}{3.490496in}}{\pgfqpoint{7.933379in}{3.492500in}}{\pgfqpoint{7.928335in}{3.492500in}}%
\pgfpathcurveto{\pgfqpoint{7.923291in}{3.492500in}}{\pgfqpoint{7.918454in}{3.490496in}}{\pgfqpoint{7.914887in}{3.486930in}}%
\pgfpathcurveto{\pgfqpoint{7.911321in}{3.483364in}}{\pgfqpoint{7.909317in}{3.478526in}}{\pgfqpoint{7.909317in}{3.473482in}}%
\pgfpathcurveto{\pgfqpoint{7.909317in}{3.468438in}}{\pgfqpoint{7.911321in}{3.463601in}}{\pgfqpoint{7.914887in}{3.460034in}}%
\pgfpathcurveto{\pgfqpoint{7.918454in}{3.456468in}}{\pgfqpoint{7.923291in}{3.454464in}}{\pgfqpoint{7.928335in}{3.454464in}}%
\pgfpathclose%
\pgfusepath{fill}%
\end{pgfscope}%
\begin{pgfscope}%
\pgfpathrectangle{\pgfqpoint{6.572727in}{0.474100in}}{\pgfqpoint{4.227273in}{3.318700in}}%
\pgfusepath{clip}%
\pgfsetbuttcap%
\pgfsetroundjoin%
\definecolor{currentfill}{rgb}{0.993248,0.906157,0.143936}%
\pgfsetfillcolor{currentfill}%
\pgfsetfillopacity{0.700000}%
\pgfsetlinewidth{0.000000pt}%
\definecolor{currentstroke}{rgb}{0.000000,0.000000,0.000000}%
\pgfsetstrokecolor{currentstroke}%
\pgfsetstrokeopacity{0.700000}%
\pgfsetdash{}{0pt}%
\pgfpathmoveto{\pgfqpoint{8.035604in}{2.904463in}}%
\pgfpathcurveto{\pgfqpoint{8.040648in}{2.904463in}}{\pgfqpoint{8.045486in}{2.906467in}}{\pgfqpoint{8.049052in}{2.910033in}}%
\pgfpathcurveto{\pgfqpoint{8.052619in}{2.913600in}}{\pgfqpoint{8.054623in}{2.918437in}}{\pgfqpoint{8.054623in}{2.923481in}}%
\pgfpathcurveto{\pgfqpoint{8.054623in}{2.928525in}}{\pgfqpoint{8.052619in}{2.933362in}}{\pgfqpoint{8.049052in}{2.936929in}}%
\pgfpathcurveto{\pgfqpoint{8.045486in}{2.940495in}}{\pgfqpoint{8.040648in}{2.942499in}}{\pgfqpoint{8.035604in}{2.942499in}}%
\pgfpathcurveto{\pgfqpoint{8.030561in}{2.942499in}}{\pgfqpoint{8.025723in}{2.940495in}}{\pgfqpoint{8.022157in}{2.936929in}}%
\pgfpathcurveto{\pgfqpoint{8.018590in}{2.933362in}}{\pgfqpoint{8.016586in}{2.928525in}}{\pgfqpoint{8.016586in}{2.923481in}}%
\pgfpathcurveto{\pgfqpoint{8.016586in}{2.918437in}}{\pgfqpoint{8.018590in}{2.913600in}}{\pgfqpoint{8.022157in}{2.910033in}}%
\pgfpathcurveto{\pgfqpoint{8.025723in}{2.906467in}}{\pgfqpoint{8.030561in}{2.904463in}}{\pgfqpoint{8.035604in}{2.904463in}}%
\pgfpathclose%
\pgfusepath{fill}%
\end{pgfscope}%
\begin{pgfscope}%
\pgfpathrectangle{\pgfqpoint{6.572727in}{0.474100in}}{\pgfqpoint{4.227273in}{3.318700in}}%
\pgfusepath{clip}%
\pgfsetbuttcap%
\pgfsetroundjoin%
\definecolor{currentfill}{rgb}{0.267004,0.004874,0.329415}%
\pgfsetfillcolor{currentfill}%
\pgfsetfillopacity{0.700000}%
\pgfsetlinewidth{0.000000pt}%
\definecolor{currentstroke}{rgb}{0.000000,0.000000,0.000000}%
\pgfsetstrokecolor{currentstroke}%
\pgfsetstrokeopacity{0.700000}%
\pgfsetdash{}{0pt}%
\pgfpathmoveto{\pgfqpoint{7.868367in}{2.022281in}}%
\pgfpathcurveto{\pgfqpoint{7.873410in}{2.022281in}}{\pgfqpoint{7.878248in}{2.024285in}}{\pgfqpoint{7.881814in}{2.027851in}}%
\pgfpathcurveto{\pgfqpoint{7.885381in}{2.031417in}}{\pgfqpoint{7.887385in}{2.036255in}}{\pgfqpoint{7.887385in}{2.041299in}}%
\pgfpathcurveto{\pgfqpoint{7.887385in}{2.046343in}}{\pgfqpoint{7.885381in}{2.051180in}}{\pgfqpoint{7.881814in}{2.054747in}}%
\pgfpathcurveto{\pgfqpoint{7.878248in}{2.058313in}}{\pgfqpoint{7.873410in}{2.060317in}}{\pgfqpoint{7.868367in}{2.060317in}}%
\pgfpathcurveto{\pgfqpoint{7.863323in}{2.060317in}}{\pgfqpoint{7.858485in}{2.058313in}}{\pgfqpoint{7.854919in}{2.054747in}}%
\pgfpathcurveto{\pgfqpoint{7.851352in}{2.051180in}}{\pgfqpoint{7.849348in}{2.046343in}}{\pgfqpoint{7.849348in}{2.041299in}}%
\pgfpathcurveto{\pgfqpoint{7.849348in}{2.036255in}}{\pgfqpoint{7.851352in}{2.031417in}}{\pgfqpoint{7.854919in}{2.027851in}}%
\pgfpathcurveto{\pgfqpoint{7.858485in}{2.024285in}}{\pgfqpoint{7.863323in}{2.022281in}}{\pgfqpoint{7.868367in}{2.022281in}}%
\pgfpathclose%
\pgfusepath{fill}%
\end{pgfscope}%
\begin{pgfscope}%
\pgfpathrectangle{\pgfqpoint{6.572727in}{0.474100in}}{\pgfqpoint{4.227273in}{3.318700in}}%
\pgfusepath{clip}%
\pgfsetbuttcap%
\pgfsetroundjoin%
\definecolor{currentfill}{rgb}{0.993248,0.906157,0.143936}%
\pgfsetfillcolor{currentfill}%
\pgfsetfillopacity{0.700000}%
\pgfsetlinewidth{0.000000pt}%
\definecolor{currentstroke}{rgb}{0.000000,0.000000,0.000000}%
\pgfsetstrokecolor{currentstroke}%
\pgfsetstrokeopacity{0.700000}%
\pgfsetdash{}{0pt}%
\pgfpathmoveto{\pgfqpoint{7.754510in}{2.362355in}}%
\pgfpathcurveto{\pgfqpoint{7.759553in}{2.362355in}}{\pgfqpoint{7.764391in}{2.364359in}}{\pgfqpoint{7.767958in}{2.367925in}}%
\pgfpathcurveto{\pgfqpoint{7.771524in}{2.371492in}}{\pgfqpoint{7.773528in}{2.376330in}}{\pgfqpoint{7.773528in}{2.381373in}}%
\pgfpathcurveto{\pgfqpoint{7.773528in}{2.386417in}}{\pgfqpoint{7.771524in}{2.391255in}}{\pgfqpoint{7.767958in}{2.394821in}}%
\pgfpathcurveto{\pgfqpoint{7.764391in}{2.398388in}}{\pgfqpoint{7.759553in}{2.400391in}}{\pgfqpoint{7.754510in}{2.400391in}}%
\pgfpathcurveto{\pgfqpoint{7.749466in}{2.400391in}}{\pgfqpoint{7.744628in}{2.398388in}}{\pgfqpoint{7.741062in}{2.394821in}}%
\pgfpathcurveto{\pgfqpoint{7.737495in}{2.391255in}}{\pgfqpoint{7.735492in}{2.386417in}}{\pgfqpoint{7.735492in}{2.381373in}}%
\pgfpathcurveto{\pgfqpoint{7.735492in}{2.376330in}}{\pgfqpoint{7.737495in}{2.371492in}}{\pgfqpoint{7.741062in}{2.367925in}}%
\pgfpathcurveto{\pgfqpoint{7.744628in}{2.364359in}}{\pgfqpoint{7.749466in}{2.362355in}}{\pgfqpoint{7.754510in}{2.362355in}}%
\pgfpathclose%
\pgfusepath{fill}%
\end{pgfscope}%
\begin{pgfscope}%
\pgfpathrectangle{\pgfqpoint{6.572727in}{0.474100in}}{\pgfqpoint{4.227273in}{3.318700in}}%
\pgfusepath{clip}%
\pgfsetbuttcap%
\pgfsetroundjoin%
\definecolor{currentfill}{rgb}{0.993248,0.906157,0.143936}%
\pgfsetfillcolor{currentfill}%
\pgfsetfillopacity{0.700000}%
\pgfsetlinewidth{0.000000pt}%
\definecolor{currentstroke}{rgb}{0.000000,0.000000,0.000000}%
\pgfsetstrokecolor{currentstroke}%
\pgfsetstrokeopacity{0.700000}%
\pgfsetdash{}{0pt}%
\pgfpathmoveto{\pgfqpoint{8.084904in}{3.434075in}}%
\pgfpathcurveto{\pgfqpoint{8.089947in}{3.434075in}}{\pgfqpoint{8.094785in}{3.436079in}}{\pgfqpoint{8.098352in}{3.439646in}}%
\pgfpathcurveto{\pgfqpoint{8.101918in}{3.443212in}}{\pgfqpoint{8.103922in}{3.448050in}}{\pgfqpoint{8.103922in}{3.453093in}}%
\pgfpathcurveto{\pgfqpoint{8.103922in}{3.458137in}}{\pgfqpoint{8.101918in}{3.462975in}}{\pgfqpoint{8.098352in}{3.466541in}}%
\pgfpathcurveto{\pgfqpoint{8.094785in}{3.470108in}}{\pgfqpoint{8.089947in}{3.472112in}}{\pgfqpoint{8.084904in}{3.472112in}}%
\pgfpathcurveto{\pgfqpoint{8.079860in}{3.472112in}}{\pgfqpoint{8.075022in}{3.470108in}}{\pgfqpoint{8.071456in}{3.466541in}}%
\pgfpathcurveto{\pgfqpoint{8.067889in}{3.462975in}}{\pgfqpoint{8.065886in}{3.458137in}}{\pgfqpoint{8.065886in}{3.453093in}}%
\pgfpathcurveto{\pgfqpoint{8.065886in}{3.448050in}}{\pgfqpoint{8.067889in}{3.443212in}}{\pgfqpoint{8.071456in}{3.439646in}}%
\pgfpathcurveto{\pgfqpoint{8.075022in}{3.436079in}}{\pgfqpoint{8.079860in}{3.434075in}}{\pgfqpoint{8.084904in}{3.434075in}}%
\pgfpathclose%
\pgfusepath{fill}%
\end{pgfscope}%
\begin{pgfscope}%
\pgfpathrectangle{\pgfqpoint{6.572727in}{0.474100in}}{\pgfqpoint{4.227273in}{3.318700in}}%
\pgfusepath{clip}%
\pgfsetbuttcap%
\pgfsetroundjoin%
\definecolor{currentfill}{rgb}{0.267004,0.004874,0.329415}%
\pgfsetfillcolor{currentfill}%
\pgfsetfillopacity{0.700000}%
\pgfsetlinewidth{0.000000pt}%
\definecolor{currentstroke}{rgb}{0.000000,0.000000,0.000000}%
\pgfsetstrokecolor{currentstroke}%
\pgfsetstrokeopacity{0.700000}%
\pgfsetdash{}{0pt}%
\pgfpathmoveto{\pgfqpoint{7.821623in}{1.684355in}}%
\pgfpathcurveto{\pgfqpoint{7.826667in}{1.684355in}}{\pgfqpoint{7.831504in}{1.686359in}}{\pgfqpoint{7.835071in}{1.689925in}}%
\pgfpathcurveto{\pgfqpoint{7.838637in}{1.693492in}}{\pgfqpoint{7.840641in}{1.698330in}}{\pgfqpoint{7.840641in}{1.703373in}}%
\pgfpathcurveto{\pgfqpoint{7.840641in}{1.708417in}}{\pgfqpoint{7.838637in}{1.713255in}}{\pgfqpoint{7.835071in}{1.716821in}}%
\pgfpathcurveto{\pgfqpoint{7.831504in}{1.720388in}}{\pgfqpoint{7.826667in}{1.722392in}}{\pgfqpoint{7.821623in}{1.722392in}}%
\pgfpathcurveto{\pgfqpoint{7.816579in}{1.722392in}}{\pgfqpoint{7.811742in}{1.720388in}}{\pgfqpoint{7.808175in}{1.716821in}}%
\pgfpathcurveto{\pgfqpoint{7.804609in}{1.713255in}}{\pgfqpoint{7.802605in}{1.708417in}}{\pgfqpoint{7.802605in}{1.703373in}}%
\pgfpathcurveto{\pgfqpoint{7.802605in}{1.698330in}}{\pgfqpoint{7.804609in}{1.693492in}}{\pgfqpoint{7.808175in}{1.689925in}}%
\pgfpathcurveto{\pgfqpoint{7.811742in}{1.686359in}}{\pgfqpoint{7.816579in}{1.684355in}}{\pgfqpoint{7.821623in}{1.684355in}}%
\pgfpathclose%
\pgfusepath{fill}%
\end{pgfscope}%
\begin{pgfscope}%
\pgfpathrectangle{\pgfqpoint{6.572727in}{0.474100in}}{\pgfqpoint{4.227273in}{3.318700in}}%
\pgfusepath{clip}%
\pgfsetbuttcap%
\pgfsetroundjoin%
\definecolor{currentfill}{rgb}{0.267004,0.004874,0.329415}%
\pgfsetfillcolor{currentfill}%
\pgfsetfillopacity{0.700000}%
\pgfsetlinewidth{0.000000pt}%
\definecolor{currentstroke}{rgb}{0.000000,0.000000,0.000000}%
\pgfsetstrokecolor{currentstroke}%
\pgfsetstrokeopacity{0.700000}%
\pgfsetdash{}{0pt}%
\pgfpathmoveto{\pgfqpoint{7.861638in}{1.659647in}}%
\pgfpathcurveto{\pgfqpoint{7.866682in}{1.659647in}}{\pgfqpoint{7.871519in}{1.661650in}}{\pgfqpoint{7.875086in}{1.665217in}}%
\pgfpathcurveto{\pgfqpoint{7.878652in}{1.668783in}}{\pgfqpoint{7.880656in}{1.673621in}}{\pgfqpoint{7.880656in}{1.678665in}}%
\pgfpathcurveto{\pgfqpoint{7.880656in}{1.683708in}}{\pgfqpoint{7.878652in}{1.688546in}}{\pgfqpoint{7.875086in}{1.692113in}}%
\pgfpathcurveto{\pgfqpoint{7.871519in}{1.695679in}}{\pgfqpoint{7.866682in}{1.697683in}}{\pgfqpoint{7.861638in}{1.697683in}}%
\pgfpathcurveto{\pgfqpoint{7.856594in}{1.697683in}}{\pgfqpoint{7.851756in}{1.695679in}}{\pgfqpoint{7.848190in}{1.692113in}}%
\pgfpathcurveto{\pgfqpoint{7.844624in}{1.688546in}}{\pgfqpoint{7.842620in}{1.683708in}}{\pgfqpoint{7.842620in}{1.678665in}}%
\pgfpathcurveto{\pgfqpoint{7.842620in}{1.673621in}}{\pgfqpoint{7.844624in}{1.668783in}}{\pgfqpoint{7.848190in}{1.665217in}}%
\pgfpathcurveto{\pgfqpoint{7.851756in}{1.661650in}}{\pgfqpoint{7.856594in}{1.659647in}}{\pgfqpoint{7.861638in}{1.659647in}}%
\pgfpathclose%
\pgfusepath{fill}%
\end{pgfscope}%
\begin{pgfscope}%
\pgfpathrectangle{\pgfqpoint{6.572727in}{0.474100in}}{\pgfqpoint{4.227273in}{3.318700in}}%
\pgfusepath{clip}%
\pgfsetbuttcap%
\pgfsetroundjoin%
\definecolor{currentfill}{rgb}{0.127568,0.566949,0.550556}%
\pgfsetfillcolor{currentfill}%
\pgfsetfillopacity{0.700000}%
\pgfsetlinewidth{0.000000pt}%
\definecolor{currentstroke}{rgb}{0.000000,0.000000,0.000000}%
\pgfsetstrokecolor{currentstroke}%
\pgfsetstrokeopacity{0.700000}%
\pgfsetdash{}{0pt}%
\pgfpathmoveto{\pgfqpoint{9.457914in}{0.971569in}}%
\pgfpathcurveto{\pgfqpoint{9.462957in}{0.971569in}}{\pgfqpoint{9.467795in}{0.973572in}}{\pgfqpoint{9.471361in}{0.977139in}}%
\pgfpathcurveto{\pgfqpoint{9.474928in}{0.980705in}}{\pgfqpoint{9.476932in}{0.985543in}}{\pgfqpoint{9.476932in}{0.990587in}}%
\pgfpathcurveto{\pgfqpoint{9.476932in}{0.995630in}}{\pgfqpoint{9.474928in}{1.000468in}}{\pgfqpoint{9.471361in}{1.004035in}}%
\pgfpathcurveto{\pgfqpoint{9.467795in}{1.007601in}}{\pgfqpoint{9.462957in}{1.009605in}}{\pgfqpoint{9.457914in}{1.009605in}}%
\pgfpathcurveto{\pgfqpoint{9.452870in}{1.009605in}}{\pgfqpoint{9.448032in}{1.007601in}}{\pgfqpoint{9.444466in}{1.004035in}}%
\pgfpathcurveto{\pgfqpoint{9.440899in}{1.000468in}}{\pgfqpoint{9.438895in}{0.995630in}}{\pgfqpoint{9.438895in}{0.990587in}}%
\pgfpathcurveto{\pgfqpoint{9.438895in}{0.985543in}}{\pgfqpoint{9.440899in}{0.980705in}}{\pgfqpoint{9.444466in}{0.977139in}}%
\pgfpathcurveto{\pgfqpoint{9.448032in}{0.973572in}}{\pgfqpoint{9.452870in}{0.971569in}}{\pgfqpoint{9.457914in}{0.971569in}}%
\pgfpathclose%
\pgfusepath{fill}%
\end{pgfscope}%
\begin{pgfscope}%
\pgfpathrectangle{\pgfqpoint{6.572727in}{0.474100in}}{\pgfqpoint{4.227273in}{3.318700in}}%
\pgfusepath{clip}%
\pgfsetbuttcap%
\pgfsetroundjoin%
\definecolor{currentfill}{rgb}{0.993248,0.906157,0.143936}%
\pgfsetfillcolor{currentfill}%
\pgfsetfillopacity{0.700000}%
\pgfsetlinewidth{0.000000pt}%
\definecolor{currentstroke}{rgb}{0.000000,0.000000,0.000000}%
\pgfsetstrokecolor{currentstroke}%
\pgfsetstrokeopacity{0.700000}%
\pgfsetdash{}{0pt}%
\pgfpathmoveto{\pgfqpoint{8.546718in}{2.815730in}}%
\pgfpathcurveto{\pgfqpoint{8.551761in}{2.815730in}}{\pgfqpoint{8.556599in}{2.817734in}}{\pgfqpoint{8.560166in}{2.821300in}}%
\pgfpathcurveto{\pgfqpoint{8.563732in}{2.824866in}}{\pgfqpoint{8.565736in}{2.829704in}}{\pgfqpoint{8.565736in}{2.834748in}}%
\pgfpathcurveto{\pgfqpoint{8.565736in}{2.839791in}}{\pgfqpoint{8.563732in}{2.844629in}}{\pgfqpoint{8.560166in}{2.848196in}}%
\pgfpathcurveto{\pgfqpoint{8.556599in}{2.851762in}}{\pgfqpoint{8.551761in}{2.853766in}}{\pgfqpoint{8.546718in}{2.853766in}}%
\pgfpathcurveto{\pgfqpoint{8.541674in}{2.853766in}}{\pgfqpoint{8.536836in}{2.851762in}}{\pgfqpoint{8.533270in}{2.848196in}}%
\pgfpathcurveto{\pgfqpoint{8.529703in}{2.844629in}}{\pgfqpoint{8.527700in}{2.839791in}}{\pgfqpoint{8.527700in}{2.834748in}}%
\pgfpathcurveto{\pgfqpoint{8.527700in}{2.829704in}}{\pgfqpoint{8.529703in}{2.824866in}}{\pgfqpoint{8.533270in}{2.821300in}}%
\pgfpathcurveto{\pgfqpoint{8.536836in}{2.817734in}}{\pgfqpoint{8.541674in}{2.815730in}}{\pgfqpoint{8.546718in}{2.815730in}}%
\pgfpathclose%
\pgfusepath{fill}%
\end{pgfscope}%
\begin{pgfscope}%
\pgfpathrectangle{\pgfqpoint{6.572727in}{0.474100in}}{\pgfqpoint{4.227273in}{3.318700in}}%
\pgfusepath{clip}%
\pgfsetbuttcap%
\pgfsetroundjoin%
\definecolor{currentfill}{rgb}{0.993248,0.906157,0.143936}%
\pgfsetfillcolor{currentfill}%
\pgfsetfillopacity{0.700000}%
\pgfsetlinewidth{0.000000pt}%
\definecolor{currentstroke}{rgb}{0.000000,0.000000,0.000000}%
\pgfsetstrokecolor{currentstroke}%
\pgfsetstrokeopacity{0.700000}%
\pgfsetdash{}{0pt}%
\pgfpathmoveto{\pgfqpoint{8.045324in}{2.953470in}}%
\pgfpathcurveto{\pgfqpoint{8.050368in}{2.953470in}}{\pgfqpoint{8.055206in}{2.955474in}}{\pgfqpoint{8.058772in}{2.959041in}}%
\pgfpathcurveto{\pgfqpoint{8.062339in}{2.962607in}}{\pgfqpoint{8.064342in}{2.967445in}}{\pgfqpoint{8.064342in}{2.972489in}}%
\pgfpathcurveto{\pgfqpoint{8.064342in}{2.977532in}}{\pgfqpoint{8.062339in}{2.982370in}}{\pgfqpoint{8.058772in}{2.985936in}}%
\pgfpathcurveto{\pgfqpoint{8.055206in}{2.989503in}}{\pgfqpoint{8.050368in}{2.991507in}}{\pgfqpoint{8.045324in}{2.991507in}}%
\pgfpathcurveto{\pgfqpoint{8.040281in}{2.991507in}}{\pgfqpoint{8.035443in}{2.989503in}}{\pgfqpoint{8.031876in}{2.985936in}}%
\pgfpathcurveto{\pgfqpoint{8.028310in}{2.982370in}}{\pgfqpoint{8.026306in}{2.977532in}}{\pgfqpoint{8.026306in}{2.972489in}}%
\pgfpathcurveto{\pgfqpoint{8.026306in}{2.967445in}}{\pgfqpoint{8.028310in}{2.962607in}}{\pgfqpoint{8.031876in}{2.959041in}}%
\pgfpathcurveto{\pgfqpoint{8.035443in}{2.955474in}}{\pgfqpoint{8.040281in}{2.953470in}}{\pgfqpoint{8.045324in}{2.953470in}}%
\pgfpathclose%
\pgfusepath{fill}%
\end{pgfscope}%
\begin{pgfscope}%
\pgfpathrectangle{\pgfqpoint{6.572727in}{0.474100in}}{\pgfqpoint{4.227273in}{3.318700in}}%
\pgfusepath{clip}%
\pgfsetbuttcap%
\pgfsetroundjoin%
\definecolor{currentfill}{rgb}{0.993248,0.906157,0.143936}%
\pgfsetfillcolor{currentfill}%
\pgfsetfillopacity{0.700000}%
\pgfsetlinewidth{0.000000pt}%
\definecolor{currentstroke}{rgb}{0.000000,0.000000,0.000000}%
\pgfsetstrokecolor{currentstroke}%
\pgfsetstrokeopacity{0.700000}%
\pgfsetdash{}{0pt}%
\pgfpathmoveto{\pgfqpoint{8.499750in}{2.760444in}}%
\pgfpathcurveto{\pgfqpoint{8.504794in}{2.760444in}}{\pgfqpoint{8.509632in}{2.762448in}}{\pgfqpoint{8.513198in}{2.766015in}}%
\pgfpathcurveto{\pgfqpoint{8.516765in}{2.769581in}}{\pgfqpoint{8.518769in}{2.774419in}}{\pgfqpoint{8.518769in}{2.779463in}}%
\pgfpathcurveto{\pgfqpoint{8.518769in}{2.784506in}}{\pgfqpoint{8.516765in}{2.789344in}}{\pgfqpoint{8.513198in}{2.792911in}}%
\pgfpathcurveto{\pgfqpoint{8.509632in}{2.796477in}}{\pgfqpoint{8.504794in}{2.798481in}}{\pgfqpoint{8.499750in}{2.798481in}}%
\pgfpathcurveto{\pgfqpoint{8.494707in}{2.798481in}}{\pgfqpoint{8.489869in}{2.796477in}}{\pgfqpoint{8.486303in}{2.792911in}}%
\pgfpathcurveto{\pgfqpoint{8.482736in}{2.789344in}}{\pgfqpoint{8.480732in}{2.784506in}}{\pgfqpoint{8.480732in}{2.779463in}}%
\pgfpathcurveto{\pgfqpoint{8.480732in}{2.774419in}}{\pgfqpoint{8.482736in}{2.769581in}}{\pgfqpoint{8.486303in}{2.766015in}}%
\pgfpathcurveto{\pgfqpoint{8.489869in}{2.762448in}}{\pgfqpoint{8.494707in}{2.760444in}}{\pgfqpoint{8.499750in}{2.760444in}}%
\pgfpathclose%
\pgfusepath{fill}%
\end{pgfscope}%
\begin{pgfscope}%
\pgfpathrectangle{\pgfqpoint{6.572727in}{0.474100in}}{\pgfqpoint{4.227273in}{3.318700in}}%
\pgfusepath{clip}%
\pgfsetbuttcap%
\pgfsetroundjoin%
\definecolor{currentfill}{rgb}{0.993248,0.906157,0.143936}%
\pgfsetfillcolor{currentfill}%
\pgfsetfillopacity{0.700000}%
\pgfsetlinewidth{0.000000pt}%
\definecolor{currentstroke}{rgb}{0.000000,0.000000,0.000000}%
\pgfsetstrokecolor{currentstroke}%
\pgfsetstrokeopacity{0.700000}%
\pgfsetdash{}{0pt}%
\pgfpathmoveto{\pgfqpoint{8.114727in}{2.352887in}}%
\pgfpathcurveto{\pgfqpoint{8.119771in}{2.352887in}}{\pgfqpoint{8.124609in}{2.354891in}}{\pgfqpoint{8.128175in}{2.358458in}}%
\pgfpathcurveto{\pgfqpoint{8.131742in}{2.362024in}}{\pgfqpoint{8.133746in}{2.366862in}}{\pgfqpoint{8.133746in}{2.371905in}}%
\pgfpathcurveto{\pgfqpoint{8.133746in}{2.376949in}}{\pgfqpoint{8.131742in}{2.381787in}}{\pgfqpoint{8.128175in}{2.385353in}}%
\pgfpathcurveto{\pgfqpoint{8.124609in}{2.388920in}}{\pgfqpoint{8.119771in}{2.390924in}}{\pgfqpoint{8.114727in}{2.390924in}}%
\pgfpathcurveto{\pgfqpoint{8.109684in}{2.390924in}}{\pgfqpoint{8.104846in}{2.388920in}}{\pgfqpoint{8.101280in}{2.385353in}}%
\pgfpathcurveto{\pgfqpoint{8.097713in}{2.381787in}}{\pgfqpoint{8.095709in}{2.376949in}}{\pgfqpoint{8.095709in}{2.371905in}}%
\pgfpathcurveto{\pgfqpoint{8.095709in}{2.366862in}}{\pgfqpoint{8.097713in}{2.362024in}}{\pgfqpoint{8.101280in}{2.358458in}}%
\pgfpathcurveto{\pgfqpoint{8.104846in}{2.354891in}}{\pgfqpoint{8.109684in}{2.352887in}}{\pgfqpoint{8.114727in}{2.352887in}}%
\pgfpathclose%
\pgfusepath{fill}%
\end{pgfscope}%
\begin{pgfscope}%
\pgfpathrectangle{\pgfqpoint{6.572727in}{0.474100in}}{\pgfqpoint{4.227273in}{3.318700in}}%
\pgfusepath{clip}%
\pgfsetbuttcap%
\pgfsetroundjoin%
\definecolor{currentfill}{rgb}{0.993248,0.906157,0.143936}%
\pgfsetfillcolor{currentfill}%
\pgfsetfillopacity{0.700000}%
\pgfsetlinewidth{0.000000pt}%
\definecolor{currentstroke}{rgb}{0.000000,0.000000,0.000000}%
\pgfsetstrokecolor{currentstroke}%
\pgfsetstrokeopacity{0.700000}%
\pgfsetdash{}{0pt}%
\pgfpathmoveto{\pgfqpoint{7.721510in}{2.648528in}}%
\pgfpathcurveto{\pgfqpoint{7.726554in}{2.648528in}}{\pgfqpoint{7.731391in}{2.650532in}}{\pgfqpoint{7.734958in}{2.654098in}}%
\pgfpathcurveto{\pgfqpoint{7.738524in}{2.657665in}}{\pgfqpoint{7.740528in}{2.662503in}}{\pgfqpoint{7.740528in}{2.667546in}}%
\pgfpathcurveto{\pgfqpoint{7.740528in}{2.672590in}}{\pgfqpoint{7.738524in}{2.677428in}}{\pgfqpoint{7.734958in}{2.680994in}}%
\pgfpathcurveto{\pgfqpoint{7.731391in}{2.684560in}}{\pgfqpoint{7.726554in}{2.686564in}}{\pgfqpoint{7.721510in}{2.686564in}}%
\pgfpathcurveto{\pgfqpoint{7.716466in}{2.686564in}}{\pgfqpoint{7.711629in}{2.684560in}}{\pgfqpoint{7.708062in}{2.680994in}}%
\pgfpathcurveto{\pgfqpoint{7.704496in}{2.677428in}}{\pgfqpoint{7.702492in}{2.672590in}}{\pgfqpoint{7.702492in}{2.667546in}}%
\pgfpathcurveto{\pgfqpoint{7.702492in}{2.662503in}}{\pgfqpoint{7.704496in}{2.657665in}}{\pgfqpoint{7.708062in}{2.654098in}}%
\pgfpathcurveto{\pgfqpoint{7.711629in}{2.650532in}}{\pgfqpoint{7.716466in}{2.648528in}}{\pgfqpoint{7.721510in}{2.648528in}}%
\pgfpathclose%
\pgfusepath{fill}%
\end{pgfscope}%
\begin{pgfscope}%
\pgfpathrectangle{\pgfqpoint{6.572727in}{0.474100in}}{\pgfqpoint{4.227273in}{3.318700in}}%
\pgfusepath{clip}%
\pgfsetbuttcap%
\pgfsetroundjoin%
\definecolor{currentfill}{rgb}{0.127568,0.566949,0.550556}%
\pgfsetfillcolor{currentfill}%
\pgfsetfillopacity{0.700000}%
\pgfsetlinewidth{0.000000pt}%
\definecolor{currentstroke}{rgb}{0.000000,0.000000,0.000000}%
\pgfsetstrokecolor{currentstroke}%
\pgfsetstrokeopacity{0.700000}%
\pgfsetdash{}{0pt}%
\pgfpathmoveto{\pgfqpoint{9.075194in}{2.058160in}}%
\pgfpathcurveto{\pgfqpoint{9.080238in}{2.058160in}}{\pgfqpoint{9.085076in}{2.060163in}}{\pgfqpoint{9.088642in}{2.063730in}}%
\pgfpathcurveto{\pgfqpoint{9.092208in}{2.067296in}}{\pgfqpoint{9.094212in}{2.072134in}}{\pgfqpoint{9.094212in}{2.077178in}}%
\pgfpathcurveto{\pgfqpoint{9.094212in}{2.082221in}}{\pgfqpoint{9.092208in}{2.087059in}}{\pgfqpoint{9.088642in}{2.090626in}}%
\pgfpathcurveto{\pgfqpoint{9.085076in}{2.094192in}}{\pgfqpoint{9.080238in}{2.096196in}}{\pgfqpoint{9.075194in}{2.096196in}}%
\pgfpathcurveto{\pgfqpoint{9.070151in}{2.096196in}}{\pgfqpoint{9.065313in}{2.094192in}}{\pgfqpoint{9.061746in}{2.090626in}}%
\pgfpathcurveto{\pgfqpoint{9.058180in}{2.087059in}}{\pgfqpoint{9.056176in}{2.082221in}}{\pgfqpoint{9.056176in}{2.077178in}}%
\pgfpathcurveto{\pgfqpoint{9.056176in}{2.072134in}}{\pgfqpoint{9.058180in}{2.067296in}}{\pgfqpoint{9.061746in}{2.063730in}}%
\pgfpathcurveto{\pgfqpoint{9.065313in}{2.060163in}}{\pgfqpoint{9.070151in}{2.058160in}}{\pgfqpoint{9.075194in}{2.058160in}}%
\pgfpathclose%
\pgfusepath{fill}%
\end{pgfscope}%
\begin{pgfscope}%
\pgfpathrectangle{\pgfqpoint{6.572727in}{0.474100in}}{\pgfqpoint{4.227273in}{3.318700in}}%
\pgfusepath{clip}%
\pgfsetbuttcap%
\pgfsetroundjoin%
\definecolor{currentfill}{rgb}{0.127568,0.566949,0.550556}%
\pgfsetfillcolor{currentfill}%
\pgfsetfillopacity{0.700000}%
\pgfsetlinewidth{0.000000pt}%
\definecolor{currentstroke}{rgb}{0.000000,0.000000,0.000000}%
\pgfsetstrokecolor{currentstroke}%
\pgfsetstrokeopacity{0.700000}%
\pgfsetdash{}{0pt}%
\pgfpathmoveto{\pgfqpoint{10.148880in}{2.046783in}}%
\pgfpathcurveto{\pgfqpoint{10.153924in}{2.046783in}}{\pgfqpoint{10.158762in}{2.048787in}}{\pgfqpoint{10.162328in}{2.052353in}}%
\pgfpathcurveto{\pgfqpoint{10.165894in}{2.055920in}}{\pgfqpoint{10.167898in}{2.060758in}}{\pgfqpoint{10.167898in}{2.065801in}}%
\pgfpathcurveto{\pgfqpoint{10.167898in}{2.070845in}}{\pgfqpoint{10.165894in}{2.075683in}}{\pgfqpoint{10.162328in}{2.079249in}}%
\pgfpathcurveto{\pgfqpoint{10.158762in}{2.082816in}}{\pgfqpoint{10.153924in}{2.084819in}}{\pgfqpoint{10.148880in}{2.084819in}}%
\pgfpathcurveto{\pgfqpoint{10.143836in}{2.084819in}}{\pgfqpoint{10.138999in}{2.082816in}}{\pgfqpoint{10.135432in}{2.079249in}}%
\pgfpathcurveto{\pgfqpoint{10.131866in}{2.075683in}}{\pgfqpoint{10.129862in}{2.070845in}}{\pgfqpoint{10.129862in}{2.065801in}}%
\pgfpathcurveto{\pgfqpoint{10.129862in}{2.060758in}}{\pgfqpoint{10.131866in}{2.055920in}}{\pgfqpoint{10.135432in}{2.052353in}}%
\pgfpathcurveto{\pgfqpoint{10.138999in}{2.048787in}}{\pgfqpoint{10.143836in}{2.046783in}}{\pgfqpoint{10.148880in}{2.046783in}}%
\pgfpathclose%
\pgfusepath{fill}%
\end{pgfscope}%
\begin{pgfscope}%
\pgfpathrectangle{\pgfqpoint{6.572727in}{0.474100in}}{\pgfqpoint{4.227273in}{3.318700in}}%
\pgfusepath{clip}%
\pgfsetbuttcap%
\pgfsetroundjoin%
\definecolor{currentfill}{rgb}{0.993248,0.906157,0.143936}%
\pgfsetfillcolor{currentfill}%
\pgfsetfillopacity{0.700000}%
\pgfsetlinewidth{0.000000pt}%
\definecolor{currentstroke}{rgb}{0.000000,0.000000,0.000000}%
\pgfsetstrokecolor{currentstroke}%
\pgfsetstrokeopacity{0.700000}%
\pgfsetdash{}{0pt}%
\pgfpathmoveto{\pgfqpoint{8.426245in}{2.775021in}}%
\pgfpathcurveto{\pgfqpoint{8.431289in}{2.775021in}}{\pgfqpoint{8.436126in}{2.777025in}}{\pgfqpoint{8.439693in}{2.780592in}}%
\pgfpathcurveto{\pgfqpoint{8.443259in}{2.784158in}}{\pgfqpoint{8.445263in}{2.788996in}}{\pgfqpoint{8.445263in}{2.794039in}}%
\pgfpathcurveto{\pgfqpoint{8.445263in}{2.799083in}}{\pgfqpoint{8.443259in}{2.803921in}}{\pgfqpoint{8.439693in}{2.807487in}}%
\pgfpathcurveto{\pgfqpoint{8.436126in}{2.811054in}}{\pgfqpoint{8.431289in}{2.813058in}}{\pgfqpoint{8.426245in}{2.813058in}}%
\pgfpathcurveto{\pgfqpoint{8.421201in}{2.813058in}}{\pgfqpoint{8.416363in}{2.811054in}}{\pgfqpoint{8.412797in}{2.807487in}}%
\pgfpathcurveto{\pgfqpoint{8.409231in}{2.803921in}}{\pgfqpoint{8.407227in}{2.799083in}}{\pgfqpoint{8.407227in}{2.794039in}}%
\pgfpathcurveto{\pgfqpoint{8.407227in}{2.788996in}}{\pgfqpoint{8.409231in}{2.784158in}}{\pgfqpoint{8.412797in}{2.780592in}}%
\pgfpathcurveto{\pgfqpoint{8.416363in}{2.777025in}}{\pgfqpoint{8.421201in}{2.775021in}}{\pgfqpoint{8.426245in}{2.775021in}}%
\pgfpathclose%
\pgfusepath{fill}%
\end{pgfscope}%
\begin{pgfscope}%
\pgfpathrectangle{\pgfqpoint{6.572727in}{0.474100in}}{\pgfqpoint{4.227273in}{3.318700in}}%
\pgfusepath{clip}%
\pgfsetbuttcap%
\pgfsetroundjoin%
\definecolor{currentfill}{rgb}{0.993248,0.906157,0.143936}%
\pgfsetfillcolor{currentfill}%
\pgfsetfillopacity{0.700000}%
\pgfsetlinewidth{0.000000pt}%
\definecolor{currentstroke}{rgb}{0.000000,0.000000,0.000000}%
\pgfsetstrokecolor{currentstroke}%
\pgfsetstrokeopacity{0.700000}%
\pgfsetdash{}{0pt}%
\pgfpathmoveto{\pgfqpoint{8.220088in}{2.371852in}}%
\pgfpathcurveto{\pgfqpoint{8.225131in}{2.371852in}}{\pgfqpoint{8.229969in}{2.373856in}}{\pgfqpoint{8.233536in}{2.377423in}}%
\pgfpathcurveto{\pgfqpoint{8.237102in}{2.380989in}}{\pgfqpoint{8.239106in}{2.385827in}}{\pgfqpoint{8.239106in}{2.390870in}}%
\pgfpathcurveto{\pgfqpoint{8.239106in}{2.395914in}}{\pgfqpoint{8.237102in}{2.400752in}}{\pgfqpoint{8.233536in}{2.404318in}}%
\pgfpathcurveto{\pgfqpoint{8.229969in}{2.407885in}}{\pgfqpoint{8.225131in}{2.409889in}}{\pgfqpoint{8.220088in}{2.409889in}}%
\pgfpathcurveto{\pgfqpoint{8.215044in}{2.409889in}}{\pgfqpoint{8.210206in}{2.407885in}}{\pgfqpoint{8.206640in}{2.404318in}}%
\pgfpathcurveto{\pgfqpoint{8.203074in}{2.400752in}}{\pgfqpoint{8.201070in}{2.395914in}}{\pgfqpoint{8.201070in}{2.390870in}}%
\pgfpathcurveto{\pgfqpoint{8.201070in}{2.385827in}}{\pgfqpoint{8.203074in}{2.380989in}}{\pgfqpoint{8.206640in}{2.377423in}}%
\pgfpathcurveto{\pgfqpoint{8.210206in}{2.373856in}}{\pgfqpoint{8.215044in}{2.371852in}}{\pgfqpoint{8.220088in}{2.371852in}}%
\pgfpathclose%
\pgfusepath{fill}%
\end{pgfscope}%
\begin{pgfscope}%
\pgfpathrectangle{\pgfqpoint{6.572727in}{0.474100in}}{\pgfqpoint{4.227273in}{3.318700in}}%
\pgfusepath{clip}%
\pgfsetbuttcap%
\pgfsetroundjoin%
\definecolor{currentfill}{rgb}{0.267004,0.004874,0.329415}%
\pgfsetfillcolor{currentfill}%
\pgfsetfillopacity{0.700000}%
\pgfsetlinewidth{0.000000pt}%
\definecolor{currentstroke}{rgb}{0.000000,0.000000,0.000000}%
\pgfsetstrokecolor{currentstroke}%
\pgfsetstrokeopacity{0.700000}%
\pgfsetdash{}{0pt}%
\pgfpathmoveto{\pgfqpoint{7.902854in}{1.510251in}}%
\pgfpathcurveto{\pgfqpoint{7.907897in}{1.510251in}}{\pgfqpoint{7.912735in}{1.512255in}}{\pgfqpoint{7.916302in}{1.515822in}}%
\pgfpathcurveto{\pgfqpoint{7.919868in}{1.519388in}}{\pgfqpoint{7.921872in}{1.524226in}}{\pgfqpoint{7.921872in}{1.529270in}}%
\pgfpathcurveto{\pgfqpoint{7.921872in}{1.534313in}}{\pgfqpoint{7.919868in}{1.539151in}}{\pgfqpoint{7.916302in}{1.542717in}}%
\pgfpathcurveto{\pgfqpoint{7.912735in}{1.546284in}}{\pgfqpoint{7.907897in}{1.548288in}}{\pgfqpoint{7.902854in}{1.548288in}}%
\pgfpathcurveto{\pgfqpoint{7.897810in}{1.548288in}}{\pgfqpoint{7.892972in}{1.546284in}}{\pgfqpoint{7.889406in}{1.542717in}}%
\pgfpathcurveto{\pgfqpoint{7.885839in}{1.539151in}}{\pgfqpoint{7.883836in}{1.534313in}}{\pgfqpoint{7.883836in}{1.529270in}}%
\pgfpathcurveto{\pgfqpoint{7.883836in}{1.524226in}}{\pgfqpoint{7.885839in}{1.519388in}}{\pgfqpoint{7.889406in}{1.515822in}}%
\pgfpathcurveto{\pgfqpoint{7.892972in}{1.512255in}}{\pgfqpoint{7.897810in}{1.510251in}}{\pgfqpoint{7.902854in}{1.510251in}}%
\pgfpathclose%
\pgfusepath{fill}%
\end{pgfscope}%
\begin{pgfscope}%
\pgfpathrectangle{\pgfqpoint{6.572727in}{0.474100in}}{\pgfqpoint{4.227273in}{3.318700in}}%
\pgfusepath{clip}%
\pgfsetbuttcap%
\pgfsetroundjoin%
\definecolor{currentfill}{rgb}{0.267004,0.004874,0.329415}%
\pgfsetfillcolor{currentfill}%
\pgfsetfillopacity{0.700000}%
\pgfsetlinewidth{0.000000pt}%
\definecolor{currentstroke}{rgb}{0.000000,0.000000,0.000000}%
\pgfsetstrokecolor{currentstroke}%
\pgfsetstrokeopacity{0.700000}%
\pgfsetdash{}{0pt}%
\pgfpathmoveto{\pgfqpoint{8.066402in}{1.466459in}}%
\pgfpathcurveto{\pgfqpoint{8.071446in}{1.466459in}}{\pgfqpoint{8.076283in}{1.468463in}}{\pgfqpoint{8.079850in}{1.472029in}}%
\pgfpathcurveto{\pgfqpoint{8.083416in}{1.475596in}}{\pgfqpoint{8.085420in}{1.480434in}}{\pgfqpoint{8.085420in}{1.485477in}}%
\pgfpathcurveto{\pgfqpoint{8.085420in}{1.490521in}}{\pgfqpoint{8.083416in}{1.495359in}}{\pgfqpoint{8.079850in}{1.498925in}}%
\pgfpathcurveto{\pgfqpoint{8.076283in}{1.502491in}}{\pgfqpoint{8.071446in}{1.504495in}}{\pgfqpoint{8.066402in}{1.504495in}}%
\pgfpathcurveto{\pgfqpoint{8.061358in}{1.504495in}}{\pgfqpoint{8.056521in}{1.502491in}}{\pgfqpoint{8.052954in}{1.498925in}}%
\pgfpathcurveto{\pgfqpoint{8.049388in}{1.495359in}}{\pgfqpoint{8.047384in}{1.490521in}}{\pgfqpoint{8.047384in}{1.485477in}}%
\pgfpathcurveto{\pgfqpoint{8.047384in}{1.480434in}}{\pgfqpoint{8.049388in}{1.475596in}}{\pgfqpoint{8.052954in}{1.472029in}}%
\pgfpathcurveto{\pgfqpoint{8.056521in}{1.468463in}}{\pgfqpoint{8.061358in}{1.466459in}}{\pgfqpoint{8.066402in}{1.466459in}}%
\pgfpathclose%
\pgfusepath{fill}%
\end{pgfscope}%
\begin{pgfscope}%
\pgfpathrectangle{\pgfqpoint{6.572727in}{0.474100in}}{\pgfqpoint{4.227273in}{3.318700in}}%
\pgfusepath{clip}%
\pgfsetbuttcap%
\pgfsetroundjoin%
\definecolor{currentfill}{rgb}{0.993248,0.906157,0.143936}%
\pgfsetfillcolor{currentfill}%
\pgfsetfillopacity{0.700000}%
\pgfsetlinewidth{0.000000pt}%
\definecolor{currentstroke}{rgb}{0.000000,0.000000,0.000000}%
\pgfsetstrokecolor{currentstroke}%
\pgfsetstrokeopacity{0.700000}%
\pgfsetdash{}{0pt}%
\pgfpathmoveto{\pgfqpoint{8.305048in}{2.846255in}}%
\pgfpathcurveto{\pgfqpoint{8.310092in}{2.846255in}}{\pgfqpoint{8.314930in}{2.848259in}}{\pgfqpoint{8.318496in}{2.851825in}}%
\pgfpathcurveto{\pgfqpoint{8.322063in}{2.855391in}}{\pgfqpoint{8.324067in}{2.860229in}}{\pgfqpoint{8.324067in}{2.865273in}}%
\pgfpathcurveto{\pgfqpoint{8.324067in}{2.870316in}}{\pgfqpoint{8.322063in}{2.875154in}}{\pgfqpoint{8.318496in}{2.878721in}}%
\pgfpathcurveto{\pgfqpoint{8.314930in}{2.882287in}}{\pgfqpoint{8.310092in}{2.884291in}}{\pgfqpoint{8.305048in}{2.884291in}}%
\pgfpathcurveto{\pgfqpoint{8.300005in}{2.884291in}}{\pgfqpoint{8.295167in}{2.882287in}}{\pgfqpoint{8.291601in}{2.878721in}}%
\pgfpathcurveto{\pgfqpoint{8.288034in}{2.875154in}}{\pgfqpoint{8.286030in}{2.870316in}}{\pgfqpoint{8.286030in}{2.865273in}}%
\pgfpathcurveto{\pgfqpoint{8.286030in}{2.860229in}}{\pgfqpoint{8.288034in}{2.855391in}}{\pgfqpoint{8.291601in}{2.851825in}}%
\pgfpathcurveto{\pgfqpoint{8.295167in}{2.848259in}}{\pgfqpoint{8.300005in}{2.846255in}}{\pgfqpoint{8.305048in}{2.846255in}}%
\pgfpathclose%
\pgfusepath{fill}%
\end{pgfscope}%
\begin{pgfscope}%
\pgfpathrectangle{\pgfqpoint{6.572727in}{0.474100in}}{\pgfqpoint{4.227273in}{3.318700in}}%
\pgfusepath{clip}%
\pgfsetbuttcap%
\pgfsetroundjoin%
\definecolor{currentfill}{rgb}{0.267004,0.004874,0.329415}%
\pgfsetfillcolor{currentfill}%
\pgfsetfillopacity{0.700000}%
\pgfsetlinewidth{0.000000pt}%
\definecolor{currentstroke}{rgb}{0.000000,0.000000,0.000000}%
\pgfsetstrokecolor{currentstroke}%
\pgfsetstrokeopacity{0.700000}%
\pgfsetdash{}{0pt}%
\pgfpathmoveto{\pgfqpoint{7.743296in}{1.659988in}}%
\pgfpathcurveto{\pgfqpoint{7.748339in}{1.659988in}}{\pgfqpoint{7.753177in}{1.661992in}}{\pgfqpoint{7.756743in}{1.665558in}}%
\pgfpathcurveto{\pgfqpoint{7.760310in}{1.669125in}}{\pgfqpoint{7.762314in}{1.673963in}}{\pgfqpoint{7.762314in}{1.679006in}}%
\pgfpathcurveto{\pgfqpoint{7.762314in}{1.684050in}}{\pgfqpoint{7.760310in}{1.688888in}}{\pgfqpoint{7.756743in}{1.692454in}}%
\pgfpathcurveto{\pgfqpoint{7.753177in}{1.696021in}}{\pgfqpoint{7.748339in}{1.698024in}}{\pgfqpoint{7.743296in}{1.698024in}}%
\pgfpathcurveto{\pgfqpoint{7.738252in}{1.698024in}}{\pgfqpoint{7.733414in}{1.696021in}}{\pgfqpoint{7.729848in}{1.692454in}}%
\pgfpathcurveto{\pgfqpoint{7.726281in}{1.688888in}}{\pgfqpoint{7.724277in}{1.684050in}}{\pgfqpoint{7.724277in}{1.679006in}}%
\pgfpathcurveto{\pgfqpoint{7.724277in}{1.673963in}}{\pgfqpoint{7.726281in}{1.669125in}}{\pgfqpoint{7.729848in}{1.665558in}}%
\pgfpathcurveto{\pgfqpoint{7.733414in}{1.661992in}}{\pgfqpoint{7.738252in}{1.659988in}}{\pgfqpoint{7.743296in}{1.659988in}}%
\pgfpathclose%
\pgfusepath{fill}%
\end{pgfscope}%
\begin{pgfscope}%
\pgfpathrectangle{\pgfqpoint{6.572727in}{0.474100in}}{\pgfqpoint{4.227273in}{3.318700in}}%
\pgfusepath{clip}%
\pgfsetbuttcap%
\pgfsetroundjoin%
\definecolor{currentfill}{rgb}{0.127568,0.566949,0.550556}%
\pgfsetfillcolor{currentfill}%
\pgfsetfillopacity{0.700000}%
\pgfsetlinewidth{0.000000pt}%
\definecolor{currentstroke}{rgb}{0.000000,0.000000,0.000000}%
\pgfsetstrokecolor{currentstroke}%
\pgfsetstrokeopacity{0.700000}%
\pgfsetdash{}{0pt}%
\pgfpathmoveto{\pgfqpoint{9.271095in}{1.896473in}}%
\pgfpathcurveto{\pgfqpoint{9.276139in}{1.896473in}}{\pgfqpoint{9.280977in}{1.898476in}}{\pgfqpoint{9.284543in}{1.902043in}}%
\pgfpathcurveto{\pgfqpoint{9.288109in}{1.905609in}}{\pgfqpoint{9.290113in}{1.910447in}}{\pgfqpoint{9.290113in}{1.915491in}}%
\pgfpathcurveto{\pgfqpoint{9.290113in}{1.920534in}}{\pgfqpoint{9.288109in}{1.925372in}}{\pgfqpoint{9.284543in}{1.928939in}}%
\pgfpathcurveto{\pgfqpoint{9.280977in}{1.932505in}}{\pgfqpoint{9.276139in}{1.934509in}}{\pgfqpoint{9.271095in}{1.934509in}}%
\pgfpathcurveto{\pgfqpoint{9.266051in}{1.934509in}}{\pgfqpoint{9.261214in}{1.932505in}}{\pgfqpoint{9.257647in}{1.928939in}}%
\pgfpathcurveto{\pgfqpoint{9.254081in}{1.925372in}}{\pgfqpoint{9.252077in}{1.920534in}}{\pgfqpoint{9.252077in}{1.915491in}}%
\pgfpathcurveto{\pgfqpoint{9.252077in}{1.910447in}}{\pgfqpoint{9.254081in}{1.905609in}}{\pgfqpoint{9.257647in}{1.902043in}}%
\pgfpathcurveto{\pgfqpoint{9.261214in}{1.898476in}}{\pgfqpoint{9.266051in}{1.896473in}}{\pgfqpoint{9.271095in}{1.896473in}}%
\pgfpathclose%
\pgfusepath{fill}%
\end{pgfscope}%
\begin{pgfscope}%
\pgfpathrectangle{\pgfqpoint{6.572727in}{0.474100in}}{\pgfqpoint{4.227273in}{3.318700in}}%
\pgfusepath{clip}%
\pgfsetbuttcap%
\pgfsetroundjoin%
\definecolor{currentfill}{rgb}{0.127568,0.566949,0.550556}%
\pgfsetfillcolor{currentfill}%
\pgfsetfillopacity{0.700000}%
\pgfsetlinewidth{0.000000pt}%
\definecolor{currentstroke}{rgb}{0.000000,0.000000,0.000000}%
\pgfsetstrokecolor{currentstroke}%
\pgfsetstrokeopacity{0.700000}%
\pgfsetdash{}{0pt}%
\pgfpathmoveto{\pgfqpoint{9.844824in}{1.009999in}}%
\pgfpathcurveto{\pgfqpoint{9.849868in}{1.009999in}}{\pgfqpoint{9.854706in}{1.012003in}}{\pgfqpoint{9.858272in}{1.015569in}}%
\pgfpathcurveto{\pgfqpoint{9.861838in}{1.019136in}}{\pgfqpoint{9.863842in}{1.023974in}}{\pgfqpoint{9.863842in}{1.029017in}}%
\pgfpathcurveto{\pgfqpoint{9.863842in}{1.034061in}}{\pgfqpoint{9.861838in}{1.038899in}}{\pgfqpoint{9.858272in}{1.042465in}}%
\pgfpathcurveto{\pgfqpoint{9.854706in}{1.046032in}}{\pgfqpoint{9.849868in}{1.048035in}}{\pgfqpoint{9.844824in}{1.048035in}}%
\pgfpathcurveto{\pgfqpoint{9.839780in}{1.048035in}}{\pgfqpoint{9.834943in}{1.046032in}}{\pgfqpoint{9.831376in}{1.042465in}}%
\pgfpathcurveto{\pgfqpoint{9.827810in}{1.038899in}}{\pgfqpoint{9.825806in}{1.034061in}}{\pgfqpoint{9.825806in}{1.029017in}}%
\pgfpathcurveto{\pgfqpoint{9.825806in}{1.023974in}}{\pgfqpoint{9.827810in}{1.019136in}}{\pgfqpoint{9.831376in}{1.015569in}}%
\pgfpathcurveto{\pgfqpoint{9.834943in}{1.012003in}}{\pgfqpoint{9.839780in}{1.009999in}}{\pgfqpoint{9.844824in}{1.009999in}}%
\pgfpathclose%
\pgfusepath{fill}%
\end{pgfscope}%
\begin{pgfscope}%
\pgfpathrectangle{\pgfqpoint{6.572727in}{0.474100in}}{\pgfqpoint{4.227273in}{3.318700in}}%
\pgfusepath{clip}%
\pgfsetbuttcap%
\pgfsetroundjoin%
\definecolor{currentfill}{rgb}{0.993248,0.906157,0.143936}%
\pgfsetfillcolor{currentfill}%
\pgfsetfillopacity{0.700000}%
\pgfsetlinewidth{0.000000pt}%
\definecolor{currentstroke}{rgb}{0.000000,0.000000,0.000000}%
\pgfsetstrokecolor{currentstroke}%
\pgfsetstrokeopacity{0.700000}%
\pgfsetdash{}{0pt}%
\pgfpathmoveto{\pgfqpoint{8.068550in}{3.021982in}}%
\pgfpathcurveto{\pgfqpoint{8.073594in}{3.021982in}}{\pgfqpoint{8.078432in}{3.023986in}}{\pgfqpoint{8.081998in}{3.027552in}}%
\pgfpathcurveto{\pgfqpoint{8.085565in}{3.031119in}}{\pgfqpoint{8.087569in}{3.035956in}}{\pgfqpoint{8.087569in}{3.041000in}}%
\pgfpathcurveto{\pgfqpoint{8.087569in}{3.046044in}}{\pgfqpoint{8.085565in}{3.050881in}}{\pgfqpoint{8.081998in}{3.054448in}}%
\pgfpathcurveto{\pgfqpoint{8.078432in}{3.058014in}}{\pgfqpoint{8.073594in}{3.060018in}}{\pgfqpoint{8.068550in}{3.060018in}}%
\pgfpathcurveto{\pgfqpoint{8.063507in}{3.060018in}}{\pgfqpoint{8.058669in}{3.058014in}}{\pgfqpoint{8.055103in}{3.054448in}}%
\pgfpathcurveto{\pgfqpoint{8.051536in}{3.050881in}}{\pgfqpoint{8.049532in}{3.046044in}}{\pgfqpoint{8.049532in}{3.041000in}}%
\pgfpathcurveto{\pgfqpoint{8.049532in}{3.035956in}}{\pgfqpoint{8.051536in}{3.031119in}}{\pgfqpoint{8.055103in}{3.027552in}}%
\pgfpathcurveto{\pgfqpoint{8.058669in}{3.023986in}}{\pgfqpoint{8.063507in}{3.021982in}}{\pgfqpoint{8.068550in}{3.021982in}}%
\pgfpathclose%
\pgfusepath{fill}%
\end{pgfscope}%
\begin{pgfscope}%
\pgfpathrectangle{\pgfqpoint{6.572727in}{0.474100in}}{\pgfqpoint{4.227273in}{3.318700in}}%
\pgfusepath{clip}%
\pgfsetbuttcap%
\pgfsetroundjoin%
\definecolor{currentfill}{rgb}{0.127568,0.566949,0.550556}%
\pgfsetfillcolor{currentfill}%
\pgfsetfillopacity{0.700000}%
\pgfsetlinewidth{0.000000pt}%
\definecolor{currentstroke}{rgb}{0.000000,0.000000,0.000000}%
\pgfsetstrokecolor{currentstroke}%
\pgfsetstrokeopacity{0.700000}%
\pgfsetdash{}{0pt}%
\pgfpathmoveto{\pgfqpoint{9.180485in}{1.696302in}}%
\pgfpathcurveto{\pgfqpoint{9.185529in}{1.696302in}}{\pgfqpoint{9.190367in}{1.698306in}}{\pgfqpoint{9.193933in}{1.701873in}}%
\pgfpathcurveto{\pgfqpoint{9.197500in}{1.705439in}}{\pgfqpoint{9.199504in}{1.710277in}}{\pgfqpoint{9.199504in}{1.715320in}}%
\pgfpathcurveto{\pgfqpoint{9.199504in}{1.720364in}}{\pgfqpoint{9.197500in}{1.725202in}}{\pgfqpoint{9.193933in}{1.728768in}}%
\pgfpathcurveto{\pgfqpoint{9.190367in}{1.732335in}}{\pgfqpoint{9.185529in}{1.734339in}}{\pgfqpoint{9.180485in}{1.734339in}}%
\pgfpathcurveto{\pgfqpoint{9.175442in}{1.734339in}}{\pgfqpoint{9.170604in}{1.732335in}}{\pgfqpoint{9.167038in}{1.728768in}}%
\pgfpathcurveto{\pgfqpoint{9.163471in}{1.725202in}}{\pgfqpoint{9.161467in}{1.720364in}}{\pgfqpoint{9.161467in}{1.715320in}}%
\pgfpathcurveto{\pgfqpoint{9.161467in}{1.710277in}}{\pgfqpoint{9.163471in}{1.705439in}}{\pgfqpoint{9.167038in}{1.701873in}}%
\pgfpathcurveto{\pgfqpoint{9.170604in}{1.698306in}}{\pgfqpoint{9.175442in}{1.696302in}}{\pgfqpoint{9.180485in}{1.696302in}}%
\pgfpathclose%
\pgfusepath{fill}%
\end{pgfscope}%
\begin{pgfscope}%
\pgfpathrectangle{\pgfqpoint{6.572727in}{0.474100in}}{\pgfqpoint{4.227273in}{3.318700in}}%
\pgfusepath{clip}%
\pgfsetbuttcap%
\pgfsetroundjoin%
\definecolor{currentfill}{rgb}{0.267004,0.004874,0.329415}%
\pgfsetfillcolor{currentfill}%
\pgfsetfillopacity{0.700000}%
\pgfsetlinewidth{0.000000pt}%
\definecolor{currentstroke}{rgb}{0.000000,0.000000,0.000000}%
\pgfsetstrokecolor{currentstroke}%
\pgfsetstrokeopacity{0.700000}%
\pgfsetdash{}{0pt}%
\pgfpathmoveto{\pgfqpoint{7.364351in}{1.572236in}}%
\pgfpathcurveto{\pgfqpoint{7.369395in}{1.572236in}}{\pgfqpoint{7.374233in}{1.574240in}}{\pgfqpoint{7.377799in}{1.577807in}}%
\pgfpathcurveto{\pgfqpoint{7.381366in}{1.581373in}}{\pgfqpoint{7.383370in}{1.586211in}}{\pgfqpoint{7.383370in}{1.591254in}}%
\pgfpathcurveto{\pgfqpoint{7.383370in}{1.596298in}}{\pgfqpoint{7.381366in}{1.601136in}}{\pgfqpoint{7.377799in}{1.604702in}}%
\pgfpathcurveto{\pgfqpoint{7.374233in}{1.608269in}}{\pgfqpoint{7.369395in}{1.610273in}}{\pgfqpoint{7.364351in}{1.610273in}}%
\pgfpathcurveto{\pgfqpoint{7.359308in}{1.610273in}}{\pgfqpoint{7.354470in}{1.608269in}}{\pgfqpoint{7.350904in}{1.604702in}}%
\pgfpathcurveto{\pgfqpoint{7.347337in}{1.601136in}}{\pgfqpoint{7.345333in}{1.596298in}}{\pgfqpoint{7.345333in}{1.591254in}}%
\pgfpathcurveto{\pgfqpoint{7.345333in}{1.586211in}}{\pgfqpoint{7.347337in}{1.581373in}}{\pgfqpoint{7.350904in}{1.577807in}}%
\pgfpathcurveto{\pgfqpoint{7.354470in}{1.574240in}}{\pgfqpoint{7.359308in}{1.572236in}}{\pgfqpoint{7.364351in}{1.572236in}}%
\pgfpathclose%
\pgfusepath{fill}%
\end{pgfscope}%
\begin{pgfscope}%
\pgfpathrectangle{\pgfqpoint{6.572727in}{0.474100in}}{\pgfqpoint{4.227273in}{3.318700in}}%
\pgfusepath{clip}%
\pgfsetbuttcap%
\pgfsetroundjoin%
\definecolor{currentfill}{rgb}{0.993248,0.906157,0.143936}%
\pgfsetfillcolor{currentfill}%
\pgfsetfillopacity{0.700000}%
\pgfsetlinewidth{0.000000pt}%
\definecolor{currentstroke}{rgb}{0.000000,0.000000,0.000000}%
\pgfsetstrokecolor{currentstroke}%
\pgfsetstrokeopacity{0.700000}%
\pgfsetdash{}{0pt}%
\pgfpathmoveto{\pgfqpoint{8.498929in}{3.046928in}}%
\pgfpathcurveto{\pgfqpoint{8.503973in}{3.046928in}}{\pgfqpoint{8.508811in}{3.048932in}}{\pgfqpoint{8.512377in}{3.052498in}}%
\pgfpathcurveto{\pgfqpoint{8.515944in}{3.056064in}}{\pgfqpoint{8.517947in}{3.060902in}}{\pgfqpoint{8.517947in}{3.065946in}}%
\pgfpathcurveto{\pgfqpoint{8.517947in}{3.070989in}}{\pgfqpoint{8.515944in}{3.075827in}}{\pgfqpoint{8.512377in}{3.079394in}}%
\pgfpathcurveto{\pgfqpoint{8.508811in}{3.082960in}}{\pgfqpoint{8.503973in}{3.084964in}}{\pgfqpoint{8.498929in}{3.084964in}}%
\pgfpathcurveto{\pgfqpoint{8.493886in}{3.084964in}}{\pgfqpoint{8.489048in}{3.082960in}}{\pgfqpoint{8.485481in}{3.079394in}}%
\pgfpathcurveto{\pgfqpoint{8.481915in}{3.075827in}}{\pgfqpoint{8.479911in}{3.070989in}}{\pgfqpoint{8.479911in}{3.065946in}}%
\pgfpathcurveto{\pgfqpoint{8.479911in}{3.060902in}}{\pgfqpoint{8.481915in}{3.056064in}}{\pgfqpoint{8.485481in}{3.052498in}}%
\pgfpathcurveto{\pgfqpoint{8.489048in}{3.048932in}}{\pgfqpoint{8.493886in}{3.046928in}}{\pgfqpoint{8.498929in}{3.046928in}}%
\pgfpathclose%
\pgfusepath{fill}%
\end{pgfscope}%
\begin{pgfscope}%
\pgfpathrectangle{\pgfqpoint{6.572727in}{0.474100in}}{\pgfqpoint{4.227273in}{3.318700in}}%
\pgfusepath{clip}%
\pgfsetbuttcap%
\pgfsetroundjoin%
\definecolor{currentfill}{rgb}{0.267004,0.004874,0.329415}%
\pgfsetfillcolor{currentfill}%
\pgfsetfillopacity{0.700000}%
\pgfsetlinewidth{0.000000pt}%
\definecolor{currentstroke}{rgb}{0.000000,0.000000,0.000000}%
\pgfsetstrokecolor{currentstroke}%
\pgfsetstrokeopacity{0.700000}%
\pgfsetdash{}{0pt}%
\pgfpathmoveto{\pgfqpoint{7.473740in}{1.447246in}}%
\pgfpathcurveto{\pgfqpoint{7.478783in}{1.447246in}}{\pgfqpoint{7.483621in}{1.449249in}}{\pgfqpoint{7.487188in}{1.452816in}}%
\pgfpathcurveto{\pgfqpoint{7.490754in}{1.456382in}}{\pgfqpoint{7.492758in}{1.461220in}}{\pgfqpoint{7.492758in}{1.466264in}}%
\pgfpathcurveto{\pgfqpoint{7.492758in}{1.471307in}}{\pgfqpoint{7.490754in}{1.476145in}}{\pgfqpoint{7.487188in}{1.479712in}}%
\pgfpathcurveto{\pgfqpoint{7.483621in}{1.483278in}}{\pgfqpoint{7.478783in}{1.485282in}}{\pgfqpoint{7.473740in}{1.485282in}}%
\pgfpathcurveto{\pgfqpoint{7.468696in}{1.485282in}}{\pgfqpoint{7.463858in}{1.483278in}}{\pgfqpoint{7.460292in}{1.479712in}}%
\pgfpathcurveto{\pgfqpoint{7.456726in}{1.476145in}}{\pgfqpoint{7.454722in}{1.471307in}}{\pgfqpoint{7.454722in}{1.466264in}}%
\pgfpathcurveto{\pgfqpoint{7.454722in}{1.461220in}}{\pgfqpoint{7.456726in}{1.456382in}}{\pgfqpoint{7.460292in}{1.452816in}}%
\pgfpathcurveto{\pgfqpoint{7.463858in}{1.449249in}}{\pgfqpoint{7.468696in}{1.447246in}}{\pgfqpoint{7.473740in}{1.447246in}}%
\pgfpathclose%
\pgfusepath{fill}%
\end{pgfscope}%
\begin{pgfscope}%
\pgfpathrectangle{\pgfqpoint{6.572727in}{0.474100in}}{\pgfqpoint{4.227273in}{3.318700in}}%
\pgfusepath{clip}%
\pgfsetbuttcap%
\pgfsetroundjoin%
\definecolor{currentfill}{rgb}{0.993248,0.906157,0.143936}%
\pgfsetfillcolor{currentfill}%
\pgfsetfillopacity{0.700000}%
\pgfsetlinewidth{0.000000pt}%
\definecolor{currentstroke}{rgb}{0.000000,0.000000,0.000000}%
\pgfsetstrokecolor{currentstroke}%
\pgfsetstrokeopacity{0.700000}%
\pgfsetdash{}{0pt}%
\pgfpathmoveto{\pgfqpoint{8.014006in}{2.652650in}}%
\pgfpathcurveto{\pgfqpoint{8.019050in}{2.652650in}}{\pgfqpoint{8.023888in}{2.654654in}}{\pgfqpoint{8.027454in}{2.658221in}}%
\pgfpathcurveto{\pgfqpoint{8.031021in}{2.661787in}}{\pgfqpoint{8.033024in}{2.666625in}}{\pgfqpoint{8.033024in}{2.671668in}}%
\pgfpathcurveto{\pgfqpoint{8.033024in}{2.676712in}}{\pgfqpoint{8.031021in}{2.681550in}}{\pgfqpoint{8.027454in}{2.685116in}}%
\pgfpathcurveto{\pgfqpoint{8.023888in}{2.688683in}}{\pgfqpoint{8.019050in}{2.690687in}}{\pgfqpoint{8.014006in}{2.690687in}}%
\pgfpathcurveto{\pgfqpoint{8.008963in}{2.690687in}}{\pgfqpoint{8.004125in}{2.688683in}}{\pgfqpoint{8.000558in}{2.685116in}}%
\pgfpathcurveto{\pgfqpoint{7.996992in}{2.681550in}}{\pgfqpoint{7.994988in}{2.676712in}}{\pgfqpoint{7.994988in}{2.671668in}}%
\pgfpathcurveto{\pgfqpoint{7.994988in}{2.666625in}}{\pgfqpoint{7.996992in}{2.661787in}}{\pgfqpoint{8.000558in}{2.658221in}}%
\pgfpathcurveto{\pgfqpoint{8.004125in}{2.654654in}}{\pgfqpoint{8.008963in}{2.652650in}}{\pgfqpoint{8.014006in}{2.652650in}}%
\pgfpathclose%
\pgfusepath{fill}%
\end{pgfscope}%
\begin{pgfscope}%
\pgfpathrectangle{\pgfqpoint{6.572727in}{0.474100in}}{\pgfqpoint{4.227273in}{3.318700in}}%
\pgfusepath{clip}%
\pgfsetbuttcap%
\pgfsetroundjoin%
\definecolor{currentfill}{rgb}{0.993248,0.906157,0.143936}%
\pgfsetfillcolor{currentfill}%
\pgfsetfillopacity{0.700000}%
\pgfsetlinewidth{0.000000pt}%
\definecolor{currentstroke}{rgb}{0.000000,0.000000,0.000000}%
\pgfsetstrokecolor{currentstroke}%
\pgfsetstrokeopacity{0.700000}%
\pgfsetdash{}{0pt}%
\pgfpathmoveto{\pgfqpoint{8.168779in}{3.109108in}}%
\pgfpathcurveto{\pgfqpoint{8.173822in}{3.109108in}}{\pgfqpoint{8.178660in}{3.111112in}}{\pgfqpoint{8.182227in}{3.114679in}}%
\pgfpathcurveto{\pgfqpoint{8.185793in}{3.118245in}}{\pgfqpoint{8.187797in}{3.123083in}}{\pgfqpoint{8.187797in}{3.128127in}}%
\pgfpathcurveto{\pgfqpoint{8.187797in}{3.133170in}}{\pgfqpoint{8.185793in}{3.138008in}}{\pgfqpoint{8.182227in}{3.141574in}}%
\pgfpathcurveto{\pgfqpoint{8.178660in}{3.145141in}}{\pgfqpoint{8.173822in}{3.147145in}}{\pgfqpoint{8.168779in}{3.147145in}}%
\pgfpathcurveto{\pgfqpoint{8.163735in}{3.147145in}}{\pgfqpoint{8.158897in}{3.145141in}}{\pgfqpoint{8.155331in}{3.141574in}}%
\pgfpathcurveto{\pgfqpoint{8.151764in}{3.138008in}}{\pgfqpoint{8.149761in}{3.133170in}}{\pgfqpoint{8.149761in}{3.128127in}}%
\pgfpathcurveto{\pgfqpoint{8.149761in}{3.123083in}}{\pgfqpoint{8.151764in}{3.118245in}}{\pgfqpoint{8.155331in}{3.114679in}}%
\pgfpathcurveto{\pgfqpoint{8.158897in}{3.111112in}}{\pgfqpoint{8.163735in}{3.109108in}}{\pgfqpoint{8.168779in}{3.109108in}}%
\pgfpathclose%
\pgfusepath{fill}%
\end{pgfscope}%
\begin{pgfscope}%
\pgfpathrectangle{\pgfqpoint{6.572727in}{0.474100in}}{\pgfqpoint{4.227273in}{3.318700in}}%
\pgfusepath{clip}%
\pgfsetbuttcap%
\pgfsetroundjoin%
\definecolor{currentfill}{rgb}{0.993248,0.906157,0.143936}%
\pgfsetfillcolor{currentfill}%
\pgfsetfillopacity{0.700000}%
\pgfsetlinewidth{0.000000pt}%
\definecolor{currentstroke}{rgb}{0.000000,0.000000,0.000000}%
\pgfsetstrokecolor{currentstroke}%
\pgfsetstrokeopacity{0.700000}%
\pgfsetdash{}{0pt}%
\pgfpathmoveto{\pgfqpoint{8.098024in}{2.938047in}}%
\pgfpathcurveto{\pgfqpoint{8.103068in}{2.938047in}}{\pgfqpoint{8.107905in}{2.940051in}}{\pgfqpoint{8.111472in}{2.943617in}}%
\pgfpathcurveto{\pgfqpoint{8.115038in}{2.947184in}}{\pgfqpoint{8.117042in}{2.952022in}}{\pgfqpoint{8.117042in}{2.957065in}}%
\pgfpathcurveto{\pgfqpoint{8.117042in}{2.962109in}}{\pgfqpoint{8.115038in}{2.966947in}}{\pgfqpoint{8.111472in}{2.970513in}}%
\pgfpathcurveto{\pgfqpoint{8.107905in}{2.974079in}}{\pgfqpoint{8.103068in}{2.976083in}}{\pgfqpoint{8.098024in}{2.976083in}}%
\pgfpathcurveto{\pgfqpoint{8.092980in}{2.976083in}}{\pgfqpoint{8.088143in}{2.974079in}}{\pgfqpoint{8.084576in}{2.970513in}}%
\pgfpathcurveto{\pgfqpoint{8.081010in}{2.966947in}}{\pgfqpoint{8.079006in}{2.962109in}}{\pgfqpoint{8.079006in}{2.957065in}}%
\pgfpathcurveto{\pgfqpoint{8.079006in}{2.952022in}}{\pgfqpoint{8.081010in}{2.947184in}}{\pgfqpoint{8.084576in}{2.943617in}}%
\pgfpathcurveto{\pgfqpoint{8.088143in}{2.940051in}}{\pgfqpoint{8.092980in}{2.938047in}}{\pgfqpoint{8.098024in}{2.938047in}}%
\pgfpathclose%
\pgfusepath{fill}%
\end{pgfscope}%
\begin{pgfscope}%
\pgfpathrectangle{\pgfqpoint{6.572727in}{0.474100in}}{\pgfqpoint{4.227273in}{3.318700in}}%
\pgfusepath{clip}%
\pgfsetbuttcap%
\pgfsetroundjoin%
\definecolor{currentfill}{rgb}{0.267004,0.004874,0.329415}%
\pgfsetfillcolor{currentfill}%
\pgfsetfillopacity{0.700000}%
\pgfsetlinewidth{0.000000pt}%
\definecolor{currentstroke}{rgb}{0.000000,0.000000,0.000000}%
\pgfsetstrokecolor{currentstroke}%
\pgfsetstrokeopacity{0.700000}%
\pgfsetdash{}{0pt}%
\pgfpathmoveto{\pgfqpoint{7.048983in}{1.682833in}}%
\pgfpathcurveto{\pgfqpoint{7.054027in}{1.682833in}}{\pgfqpoint{7.058864in}{1.684837in}}{\pgfqpoint{7.062431in}{1.688403in}}%
\pgfpathcurveto{\pgfqpoint{7.065997in}{1.691970in}}{\pgfqpoint{7.068001in}{1.696807in}}{\pgfqpoint{7.068001in}{1.701851in}}%
\pgfpathcurveto{\pgfqpoint{7.068001in}{1.706895in}}{\pgfqpoint{7.065997in}{1.711732in}}{\pgfqpoint{7.062431in}{1.715299in}}%
\pgfpathcurveto{\pgfqpoint{7.058864in}{1.718865in}}{\pgfqpoint{7.054027in}{1.720869in}}{\pgfqpoint{7.048983in}{1.720869in}}%
\pgfpathcurveto{\pgfqpoint{7.043939in}{1.720869in}}{\pgfqpoint{7.039102in}{1.718865in}}{\pgfqpoint{7.035535in}{1.715299in}}%
\pgfpathcurveto{\pgfqpoint{7.031969in}{1.711732in}}{\pgfqpoint{7.029965in}{1.706895in}}{\pgfqpoint{7.029965in}{1.701851in}}%
\pgfpathcurveto{\pgfqpoint{7.029965in}{1.696807in}}{\pgfqpoint{7.031969in}{1.691970in}}{\pgfqpoint{7.035535in}{1.688403in}}%
\pgfpathcurveto{\pgfqpoint{7.039102in}{1.684837in}}{\pgfqpoint{7.043939in}{1.682833in}}{\pgfqpoint{7.048983in}{1.682833in}}%
\pgfpathclose%
\pgfusepath{fill}%
\end{pgfscope}%
\begin{pgfscope}%
\pgfpathrectangle{\pgfqpoint{6.572727in}{0.474100in}}{\pgfqpoint{4.227273in}{3.318700in}}%
\pgfusepath{clip}%
\pgfsetbuttcap%
\pgfsetroundjoin%
\definecolor{currentfill}{rgb}{0.267004,0.004874,0.329415}%
\pgfsetfillcolor{currentfill}%
\pgfsetfillopacity{0.700000}%
\pgfsetlinewidth{0.000000pt}%
\definecolor{currentstroke}{rgb}{0.000000,0.000000,0.000000}%
\pgfsetstrokecolor{currentstroke}%
\pgfsetstrokeopacity{0.700000}%
\pgfsetdash{}{0pt}%
\pgfpathmoveto{\pgfqpoint{7.721729in}{1.385423in}}%
\pgfpathcurveto{\pgfqpoint{7.726773in}{1.385423in}}{\pgfqpoint{7.731611in}{1.387427in}}{\pgfqpoint{7.735177in}{1.390994in}}%
\pgfpathcurveto{\pgfqpoint{7.738744in}{1.394560in}}{\pgfqpoint{7.740747in}{1.399398in}}{\pgfqpoint{7.740747in}{1.404442in}}%
\pgfpathcurveto{\pgfqpoint{7.740747in}{1.409485in}}{\pgfqpoint{7.738744in}{1.414323in}}{\pgfqpoint{7.735177in}{1.417889in}}%
\pgfpathcurveto{\pgfqpoint{7.731611in}{1.421456in}}{\pgfqpoint{7.726773in}{1.423460in}}{\pgfqpoint{7.721729in}{1.423460in}}%
\pgfpathcurveto{\pgfqpoint{7.716686in}{1.423460in}}{\pgfqpoint{7.711848in}{1.421456in}}{\pgfqpoint{7.708281in}{1.417889in}}%
\pgfpathcurveto{\pgfqpoint{7.704715in}{1.414323in}}{\pgfqpoint{7.702711in}{1.409485in}}{\pgfqpoint{7.702711in}{1.404442in}}%
\pgfpathcurveto{\pgfqpoint{7.702711in}{1.399398in}}{\pgfqpoint{7.704715in}{1.394560in}}{\pgfqpoint{7.708281in}{1.390994in}}%
\pgfpathcurveto{\pgfqpoint{7.711848in}{1.387427in}}{\pgfqpoint{7.716686in}{1.385423in}}{\pgfqpoint{7.721729in}{1.385423in}}%
\pgfpathclose%
\pgfusepath{fill}%
\end{pgfscope}%
\begin{pgfscope}%
\pgfpathrectangle{\pgfqpoint{6.572727in}{0.474100in}}{\pgfqpoint{4.227273in}{3.318700in}}%
\pgfusepath{clip}%
\pgfsetbuttcap%
\pgfsetroundjoin%
\definecolor{currentfill}{rgb}{0.267004,0.004874,0.329415}%
\pgfsetfillcolor{currentfill}%
\pgfsetfillopacity{0.700000}%
\pgfsetlinewidth{0.000000pt}%
\definecolor{currentstroke}{rgb}{0.000000,0.000000,0.000000}%
\pgfsetstrokecolor{currentstroke}%
\pgfsetstrokeopacity{0.700000}%
\pgfsetdash{}{0pt}%
\pgfpathmoveto{\pgfqpoint{7.758603in}{1.121398in}}%
\pgfpathcurveto{\pgfqpoint{7.763647in}{1.121398in}}{\pgfqpoint{7.768484in}{1.123402in}}{\pgfqpoint{7.772051in}{1.126969in}}%
\pgfpathcurveto{\pgfqpoint{7.775617in}{1.130535in}}{\pgfqpoint{7.777621in}{1.135373in}}{\pgfqpoint{7.777621in}{1.140417in}}%
\pgfpathcurveto{\pgfqpoint{7.777621in}{1.145460in}}{\pgfqpoint{7.775617in}{1.150298in}}{\pgfqpoint{7.772051in}{1.153864in}}%
\pgfpathcurveto{\pgfqpoint{7.768484in}{1.157431in}}{\pgfqpoint{7.763647in}{1.159435in}}{\pgfqpoint{7.758603in}{1.159435in}}%
\pgfpathcurveto{\pgfqpoint{7.753559in}{1.159435in}}{\pgfqpoint{7.748721in}{1.157431in}}{\pgfqpoint{7.745155in}{1.153864in}}%
\pgfpathcurveto{\pgfqpoint{7.741589in}{1.150298in}}{\pgfqpoint{7.739585in}{1.145460in}}{\pgfqpoint{7.739585in}{1.140417in}}%
\pgfpathcurveto{\pgfqpoint{7.739585in}{1.135373in}}{\pgfqpoint{7.741589in}{1.130535in}}{\pgfqpoint{7.745155in}{1.126969in}}%
\pgfpathcurveto{\pgfqpoint{7.748721in}{1.123402in}}{\pgfqpoint{7.753559in}{1.121398in}}{\pgfqpoint{7.758603in}{1.121398in}}%
\pgfpathclose%
\pgfusepath{fill}%
\end{pgfscope}%
\begin{pgfscope}%
\pgfpathrectangle{\pgfqpoint{6.572727in}{0.474100in}}{\pgfqpoint{4.227273in}{3.318700in}}%
\pgfusepath{clip}%
\pgfsetbuttcap%
\pgfsetroundjoin%
\definecolor{currentfill}{rgb}{0.127568,0.566949,0.550556}%
\pgfsetfillcolor{currentfill}%
\pgfsetfillopacity{0.700000}%
\pgfsetlinewidth{0.000000pt}%
\definecolor{currentstroke}{rgb}{0.000000,0.000000,0.000000}%
\pgfsetstrokecolor{currentstroke}%
\pgfsetstrokeopacity{0.700000}%
\pgfsetdash{}{0pt}%
\pgfpathmoveto{\pgfqpoint{9.794202in}{1.524801in}}%
\pgfpathcurveto{\pgfqpoint{9.799246in}{1.524801in}}{\pgfqpoint{9.804084in}{1.526804in}}{\pgfqpoint{9.807650in}{1.530371in}}%
\pgfpathcurveto{\pgfqpoint{9.811217in}{1.533937in}}{\pgfqpoint{9.813221in}{1.538775in}}{\pgfqpoint{9.813221in}{1.543819in}}%
\pgfpathcurveto{\pgfqpoint{9.813221in}{1.548862in}}{\pgfqpoint{9.811217in}{1.553700in}}{\pgfqpoint{9.807650in}{1.557267in}}%
\pgfpathcurveto{\pgfqpoint{9.804084in}{1.560833in}}{\pgfqpoint{9.799246in}{1.562837in}}{\pgfqpoint{9.794202in}{1.562837in}}%
\pgfpathcurveto{\pgfqpoint{9.789159in}{1.562837in}}{\pgfqpoint{9.784321in}{1.560833in}}{\pgfqpoint{9.780755in}{1.557267in}}%
\pgfpathcurveto{\pgfqpoint{9.777188in}{1.553700in}}{\pgfqpoint{9.775184in}{1.548862in}}{\pgfqpoint{9.775184in}{1.543819in}}%
\pgfpathcurveto{\pgfqpoint{9.775184in}{1.538775in}}{\pgfqpoint{9.777188in}{1.533937in}}{\pgfqpoint{9.780755in}{1.530371in}}%
\pgfpathcurveto{\pgfqpoint{9.784321in}{1.526804in}}{\pgfqpoint{9.789159in}{1.524801in}}{\pgfqpoint{9.794202in}{1.524801in}}%
\pgfpathclose%
\pgfusepath{fill}%
\end{pgfscope}%
\begin{pgfscope}%
\pgfpathrectangle{\pgfqpoint{6.572727in}{0.474100in}}{\pgfqpoint{4.227273in}{3.318700in}}%
\pgfusepath{clip}%
\pgfsetbuttcap%
\pgfsetroundjoin%
\definecolor{currentfill}{rgb}{0.267004,0.004874,0.329415}%
\pgfsetfillcolor{currentfill}%
\pgfsetfillopacity{0.700000}%
\pgfsetlinewidth{0.000000pt}%
\definecolor{currentstroke}{rgb}{0.000000,0.000000,0.000000}%
\pgfsetstrokecolor{currentstroke}%
\pgfsetstrokeopacity{0.700000}%
\pgfsetdash{}{0pt}%
\pgfpathmoveto{\pgfqpoint{7.971867in}{1.517478in}}%
\pgfpathcurveto{\pgfqpoint{7.976911in}{1.517478in}}{\pgfqpoint{7.981749in}{1.519482in}}{\pgfqpoint{7.985315in}{1.523048in}}%
\pgfpathcurveto{\pgfqpoint{7.988881in}{1.526614in}}{\pgfqpoint{7.990885in}{1.531452in}}{\pgfqpoint{7.990885in}{1.536496in}}%
\pgfpathcurveto{\pgfqpoint{7.990885in}{1.541540in}}{\pgfqpoint{7.988881in}{1.546377in}}{\pgfqpoint{7.985315in}{1.549944in}}%
\pgfpathcurveto{\pgfqpoint{7.981749in}{1.553510in}}{\pgfqpoint{7.976911in}{1.555514in}}{\pgfqpoint{7.971867in}{1.555514in}}%
\pgfpathcurveto{\pgfqpoint{7.966824in}{1.555514in}}{\pgfqpoint{7.961986in}{1.553510in}}{\pgfqpoint{7.958419in}{1.549944in}}%
\pgfpathcurveto{\pgfqpoint{7.954853in}{1.546377in}}{\pgfqpoint{7.952849in}{1.541540in}}{\pgfqpoint{7.952849in}{1.536496in}}%
\pgfpathcurveto{\pgfqpoint{7.952849in}{1.531452in}}{\pgfqpoint{7.954853in}{1.526614in}}{\pgfqpoint{7.958419in}{1.523048in}}%
\pgfpathcurveto{\pgfqpoint{7.961986in}{1.519482in}}{\pgfqpoint{7.966824in}{1.517478in}}{\pgfqpoint{7.971867in}{1.517478in}}%
\pgfpathclose%
\pgfusepath{fill}%
\end{pgfscope}%
\begin{pgfscope}%
\pgfpathrectangle{\pgfqpoint{6.572727in}{0.474100in}}{\pgfqpoint{4.227273in}{3.318700in}}%
\pgfusepath{clip}%
\pgfsetbuttcap%
\pgfsetroundjoin%
\definecolor{currentfill}{rgb}{0.127568,0.566949,0.550556}%
\pgfsetfillcolor{currentfill}%
\pgfsetfillopacity{0.700000}%
\pgfsetlinewidth{0.000000pt}%
\definecolor{currentstroke}{rgb}{0.000000,0.000000,0.000000}%
\pgfsetstrokecolor{currentstroke}%
\pgfsetstrokeopacity{0.700000}%
\pgfsetdash{}{0pt}%
\pgfpathmoveto{\pgfqpoint{9.284878in}{2.442719in}}%
\pgfpathcurveto{\pgfqpoint{9.289922in}{2.442719in}}{\pgfqpoint{9.294759in}{2.444723in}}{\pgfqpoint{9.298326in}{2.448289in}}%
\pgfpathcurveto{\pgfqpoint{9.301892in}{2.451856in}}{\pgfqpoint{9.303896in}{2.456693in}}{\pgfqpoint{9.303896in}{2.461737in}}%
\pgfpathcurveto{\pgfqpoint{9.303896in}{2.466781in}}{\pgfqpoint{9.301892in}{2.471618in}}{\pgfqpoint{9.298326in}{2.475185in}}%
\pgfpathcurveto{\pgfqpoint{9.294759in}{2.478751in}}{\pgfqpoint{9.289922in}{2.480755in}}{\pgfqpoint{9.284878in}{2.480755in}}%
\pgfpathcurveto{\pgfqpoint{9.279834in}{2.480755in}}{\pgfqpoint{9.274997in}{2.478751in}}{\pgfqpoint{9.271430in}{2.475185in}}%
\pgfpathcurveto{\pgfqpoint{9.267864in}{2.471618in}}{\pgfqpoint{9.265860in}{2.466781in}}{\pgfqpoint{9.265860in}{2.461737in}}%
\pgfpathcurveto{\pgfqpoint{9.265860in}{2.456693in}}{\pgfqpoint{9.267864in}{2.451856in}}{\pgfqpoint{9.271430in}{2.448289in}}%
\pgfpathcurveto{\pgfqpoint{9.274997in}{2.444723in}}{\pgfqpoint{9.279834in}{2.442719in}}{\pgfqpoint{9.284878in}{2.442719in}}%
\pgfpathclose%
\pgfusepath{fill}%
\end{pgfscope}%
\begin{pgfscope}%
\pgfpathrectangle{\pgfqpoint{6.572727in}{0.474100in}}{\pgfqpoint{4.227273in}{3.318700in}}%
\pgfusepath{clip}%
\pgfsetbuttcap%
\pgfsetroundjoin%
\definecolor{currentfill}{rgb}{0.267004,0.004874,0.329415}%
\pgfsetfillcolor{currentfill}%
\pgfsetfillopacity{0.700000}%
\pgfsetlinewidth{0.000000pt}%
\definecolor{currentstroke}{rgb}{0.000000,0.000000,0.000000}%
\pgfsetstrokecolor{currentstroke}%
\pgfsetstrokeopacity{0.700000}%
\pgfsetdash{}{0pt}%
\pgfpathmoveto{\pgfqpoint{7.469351in}{1.666544in}}%
\pgfpathcurveto{\pgfqpoint{7.474395in}{1.666544in}}{\pgfqpoint{7.479233in}{1.668548in}}{\pgfqpoint{7.482799in}{1.672114in}}%
\pgfpathcurveto{\pgfqpoint{7.486366in}{1.675681in}}{\pgfqpoint{7.488370in}{1.680518in}}{\pgfqpoint{7.488370in}{1.685562in}}%
\pgfpathcurveto{\pgfqpoint{7.488370in}{1.690606in}}{\pgfqpoint{7.486366in}{1.695444in}}{\pgfqpoint{7.482799in}{1.699010in}}%
\pgfpathcurveto{\pgfqpoint{7.479233in}{1.702576in}}{\pgfqpoint{7.474395in}{1.704580in}}{\pgfqpoint{7.469351in}{1.704580in}}%
\pgfpathcurveto{\pgfqpoint{7.464308in}{1.704580in}}{\pgfqpoint{7.459470in}{1.702576in}}{\pgfqpoint{7.455904in}{1.699010in}}%
\pgfpathcurveto{\pgfqpoint{7.452337in}{1.695444in}}{\pgfqpoint{7.450333in}{1.690606in}}{\pgfqpoint{7.450333in}{1.685562in}}%
\pgfpathcurveto{\pgfqpoint{7.450333in}{1.680518in}}{\pgfqpoint{7.452337in}{1.675681in}}{\pgfqpoint{7.455904in}{1.672114in}}%
\pgfpathcurveto{\pgfqpoint{7.459470in}{1.668548in}}{\pgfqpoint{7.464308in}{1.666544in}}{\pgfqpoint{7.469351in}{1.666544in}}%
\pgfpathclose%
\pgfusepath{fill}%
\end{pgfscope}%
\begin{pgfscope}%
\pgfpathrectangle{\pgfqpoint{6.572727in}{0.474100in}}{\pgfqpoint{4.227273in}{3.318700in}}%
\pgfusepath{clip}%
\pgfsetbuttcap%
\pgfsetroundjoin%
\definecolor{currentfill}{rgb}{0.267004,0.004874,0.329415}%
\pgfsetfillcolor{currentfill}%
\pgfsetfillopacity{0.700000}%
\pgfsetlinewidth{0.000000pt}%
\definecolor{currentstroke}{rgb}{0.000000,0.000000,0.000000}%
\pgfsetstrokecolor{currentstroke}%
\pgfsetstrokeopacity{0.700000}%
\pgfsetdash{}{0pt}%
\pgfpathmoveto{\pgfqpoint{7.865279in}{1.582680in}}%
\pgfpathcurveto{\pgfqpoint{7.870323in}{1.582680in}}{\pgfqpoint{7.875161in}{1.584684in}}{\pgfqpoint{7.878727in}{1.588251in}}%
\pgfpathcurveto{\pgfqpoint{7.882293in}{1.591817in}}{\pgfqpoint{7.884297in}{1.596655in}}{\pgfqpoint{7.884297in}{1.601698in}}%
\pgfpathcurveto{\pgfqpoint{7.884297in}{1.606742in}}{\pgfqpoint{7.882293in}{1.611580in}}{\pgfqpoint{7.878727in}{1.615146in}}%
\pgfpathcurveto{\pgfqpoint{7.875161in}{1.618713in}}{\pgfqpoint{7.870323in}{1.620717in}}{\pgfqpoint{7.865279in}{1.620717in}}%
\pgfpathcurveto{\pgfqpoint{7.860235in}{1.620717in}}{\pgfqpoint{7.855398in}{1.618713in}}{\pgfqpoint{7.851831in}{1.615146in}}%
\pgfpathcurveto{\pgfqpoint{7.848265in}{1.611580in}}{\pgfqpoint{7.846261in}{1.606742in}}{\pgfqpoint{7.846261in}{1.601698in}}%
\pgfpathcurveto{\pgfqpoint{7.846261in}{1.596655in}}{\pgfqpoint{7.848265in}{1.591817in}}{\pgfqpoint{7.851831in}{1.588251in}}%
\pgfpathcurveto{\pgfqpoint{7.855398in}{1.584684in}}{\pgfqpoint{7.860235in}{1.582680in}}{\pgfqpoint{7.865279in}{1.582680in}}%
\pgfpathclose%
\pgfusepath{fill}%
\end{pgfscope}%
\begin{pgfscope}%
\pgfpathrectangle{\pgfqpoint{6.572727in}{0.474100in}}{\pgfqpoint{4.227273in}{3.318700in}}%
\pgfusepath{clip}%
\pgfsetbuttcap%
\pgfsetroundjoin%
\definecolor{currentfill}{rgb}{0.127568,0.566949,0.550556}%
\pgfsetfillcolor{currentfill}%
\pgfsetfillopacity{0.700000}%
\pgfsetlinewidth{0.000000pt}%
\definecolor{currentstroke}{rgb}{0.000000,0.000000,0.000000}%
\pgfsetstrokecolor{currentstroke}%
\pgfsetstrokeopacity{0.700000}%
\pgfsetdash{}{0pt}%
\pgfpathmoveto{\pgfqpoint{9.115555in}{1.376743in}}%
\pgfpathcurveto{\pgfqpoint{9.120599in}{1.376743in}}{\pgfqpoint{9.125437in}{1.378747in}}{\pgfqpoint{9.129003in}{1.382314in}}%
\pgfpathcurveto{\pgfqpoint{9.132569in}{1.385880in}}{\pgfqpoint{9.134573in}{1.390718in}}{\pgfqpoint{9.134573in}{1.395761in}}%
\pgfpathcurveto{\pgfqpoint{9.134573in}{1.400805in}}{\pgfqpoint{9.132569in}{1.405643in}}{\pgfqpoint{9.129003in}{1.409209in}}%
\pgfpathcurveto{\pgfqpoint{9.125437in}{1.412776in}}{\pgfqpoint{9.120599in}{1.414780in}}{\pgfqpoint{9.115555in}{1.414780in}}%
\pgfpathcurveto{\pgfqpoint{9.110511in}{1.414780in}}{\pgfqpoint{9.105674in}{1.412776in}}{\pgfqpoint{9.102107in}{1.409209in}}%
\pgfpathcurveto{\pgfqpoint{9.098541in}{1.405643in}}{\pgfqpoint{9.096537in}{1.400805in}}{\pgfqpoint{9.096537in}{1.395761in}}%
\pgfpathcurveto{\pgfqpoint{9.096537in}{1.390718in}}{\pgfqpoint{9.098541in}{1.385880in}}{\pgfqpoint{9.102107in}{1.382314in}}%
\pgfpathcurveto{\pgfqpoint{9.105674in}{1.378747in}}{\pgfqpoint{9.110511in}{1.376743in}}{\pgfqpoint{9.115555in}{1.376743in}}%
\pgfpathclose%
\pgfusepath{fill}%
\end{pgfscope}%
\begin{pgfscope}%
\pgfpathrectangle{\pgfqpoint{6.572727in}{0.474100in}}{\pgfqpoint{4.227273in}{3.318700in}}%
\pgfusepath{clip}%
\pgfsetbuttcap%
\pgfsetroundjoin%
\definecolor{currentfill}{rgb}{0.267004,0.004874,0.329415}%
\pgfsetfillcolor{currentfill}%
\pgfsetfillopacity{0.700000}%
\pgfsetlinewidth{0.000000pt}%
\definecolor{currentstroke}{rgb}{0.000000,0.000000,0.000000}%
\pgfsetstrokecolor{currentstroke}%
\pgfsetstrokeopacity{0.700000}%
\pgfsetdash{}{0pt}%
\pgfpathmoveto{\pgfqpoint{7.285794in}{1.542570in}}%
\pgfpathcurveto{\pgfqpoint{7.290837in}{1.542570in}}{\pgfqpoint{7.295675in}{1.544574in}}{\pgfqpoint{7.299241in}{1.548141in}}%
\pgfpathcurveto{\pgfqpoint{7.302808in}{1.551707in}}{\pgfqpoint{7.304812in}{1.556545in}}{\pgfqpoint{7.304812in}{1.561589in}}%
\pgfpathcurveto{\pgfqpoint{7.304812in}{1.566632in}}{\pgfqpoint{7.302808in}{1.571470in}}{\pgfqpoint{7.299241in}{1.575036in}}%
\pgfpathcurveto{\pgfqpoint{7.295675in}{1.578603in}}{\pgfqpoint{7.290837in}{1.580607in}}{\pgfqpoint{7.285794in}{1.580607in}}%
\pgfpathcurveto{\pgfqpoint{7.280750in}{1.580607in}}{\pgfqpoint{7.275912in}{1.578603in}}{\pgfqpoint{7.272346in}{1.575036in}}%
\pgfpathcurveto{\pgfqpoint{7.268779in}{1.571470in}}{\pgfqpoint{7.266775in}{1.566632in}}{\pgfqpoint{7.266775in}{1.561589in}}%
\pgfpathcurveto{\pgfqpoint{7.266775in}{1.556545in}}{\pgfqpoint{7.268779in}{1.551707in}}{\pgfqpoint{7.272346in}{1.548141in}}%
\pgfpathcurveto{\pgfqpoint{7.275912in}{1.544574in}}{\pgfqpoint{7.280750in}{1.542570in}}{\pgfqpoint{7.285794in}{1.542570in}}%
\pgfpathclose%
\pgfusepath{fill}%
\end{pgfscope}%
\begin{pgfscope}%
\pgfpathrectangle{\pgfqpoint{6.572727in}{0.474100in}}{\pgfqpoint{4.227273in}{3.318700in}}%
\pgfusepath{clip}%
\pgfsetbuttcap%
\pgfsetroundjoin%
\definecolor{currentfill}{rgb}{0.993248,0.906157,0.143936}%
\pgfsetfillcolor{currentfill}%
\pgfsetfillopacity{0.700000}%
\pgfsetlinewidth{0.000000pt}%
\definecolor{currentstroke}{rgb}{0.000000,0.000000,0.000000}%
\pgfsetstrokecolor{currentstroke}%
\pgfsetstrokeopacity{0.700000}%
\pgfsetdash{}{0pt}%
\pgfpathmoveto{\pgfqpoint{7.825966in}{3.077178in}}%
\pgfpathcurveto{\pgfqpoint{7.831010in}{3.077178in}}{\pgfqpoint{7.835848in}{3.079181in}}{\pgfqpoint{7.839414in}{3.082748in}}%
\pgfpathcurveto{\pgfqpoint{7.842981in}{3.086314in}}{\pgfqpoint{7.844985in}{3.091152in}}{\pgfqpoint{7.844985in}{3.096196in}}%
\pgfpathcurveto{\pgfqpoint{7.844985in}{3.101239in}}{\pgfqpoint{7.842981in}{3.106077in}}{\pgfqpoint{7.839414in}{3.109644in}}%
\pgfpathcurveto{\pgfqpoint{7.835848in}{3.113210in}}{\pgfqpoint{7.831010in}{3.115214in}}{\pgfqpoint{7.825966in}{3.115214in}}%
\pgfpathcurveto{\pgfqpoint{7.820923in}{3.115214in}}{\pgfqpoint{7.816085in}{3.113210in}}{\pgfqpoint{7.812519in}{3.109644in}}%
\pgfpathcurveto{\pgfqpoint{7.808952in}{3.106077in}}{\pgfqpoint{7.806948in}{3.101239in}}{\pgfqpoint{7.806948in}{3.096196in}}%
\pgfpathcurveto{\pgfqpoint{7.806948in}{3.091152in}}{\pgfqpoint{7.808952in}{3.086314in}}{\pgfqpoint{7.812519in}{3.082748in}}%
\pgfpathcurveto{\pgfqpoint{7.816085in}{3.079181in}}{\pgfqpoint{7.820923in}{3.077178in}}{\pgfqpoint{7.825966in}{3.077178in}}%
\pgfpathclose%
\pgfusepath{fill}%
\end{pgfscope}%
\begin{pgfscope}%
\pgfpathrectangle{\pgfqpoint{6.572727in}{0.474100in}}{\pgfqpoint{4.227273in}{3.318700in}}%
\pgfusepath{clip}%
\pgfsetbuttcap%
\pgfsetroundjoin%
\definecolor{currentfill}{rgb}{0.127568,0.566949,0.550556}%
\pgfsetfillcolor{currentfill}%
\pgfsetfillopacity{0.700000}%
\pgfsetlinewidth{0.000000pt}%
\definecolor{currentstroke}{rgb}{0.000000,0.000000,0.000000}%
\pgfsetstrokecolor{currentstroke}%
\pgfsetstrokeopacity{0.700000}%
\pgfsetdash{}{0pt}%
\pgfpathmoveto{\pgfqpoint{9.959873in}{1.537325in}}%
\pgfpathcurveto{\pgfqpoint{9.964917in}{1.537325in}}{\pgfqpoint{9.969755in}{1.539329in}}{\pgfqpoint{9.973321in}{1.542895in}}%
\pgfpathcurveto{\pgfqpoint{9.976888in}{1.546462in}}{\pgfqpoint{9.978891in}{1.551300in}}{\pgfqpoint{9.978891in}{1.556343in}}%
\pgfpathcurveto{\pgfqpoint{9.978891in}{1.561387in}}{\pgfqpoint{9.976888in}{1.566225in}}{\pgfqpoint{9.973321in}{1.569791in}}%
\pgfpathcurveto{\pgfqpoint{9.969755in}{1.573358in}}{\pgfqpoint{9.964917in}{1.575361in}}{\pgfqpoint{9.959873in}{1.575361in}}%
\pgfpathcurveto{\pgfqpoint{9.954830in}{1.575361in}}{\pgfqpoint{9.949992in}{1.573358in}}{\pgfqpoint{9.946425in}{1.569791in}}%
\pgfpathcurveto{\pgfqpoint{9.942859in}{1.566225in}}{\pgfqpoint{9.940855in}{1.561387in}}{\pgfqpoint{9.940855in}{1.556343in}}%
\pgfpathcurveto{\pgfqpoint{9.940855in}{1.551300in}}{\pgfqpoint{9.942859in}{1.546462in}}{\pgfqpoint{9.946425in}{1.542895in}}%
\pgfpathcurveto{\pgfqpoint{9.949992in}{1.539329in}}{\pgfqpoint{9.954830in}{1.537325in}}{\pgfqpoint{9.959873in}{1.537325in}}%
\pgfpathclose%
\pgfusepath{fill}%
\end{pgfscope}%
\begin{pgfscope}%
\pgfpathrectangle{\pgfqpoint{6.572727in}{0.474100in}}{\pgfqpoint{4.227273in}{3.318700in}}%
\pgfusepath{clip}%
\pgfsetbuttcap%
\pgfsetroundjoin%
\definecolor{currentfill}{rgb}{0.267004,0.004874,0.329415}%
\pgfsetfillcolor{currentfill}%
\pgfsetfillopacity{0.700000}%
\pgfsetlinewidth{0.000000pt}%
\definecolor{currentstroke}{rgb}{0.000000,0.000000,0.000000}%
\pgfsetstrokecolor{currentstroke}%
\pgfsetstrokeopacity{0.700000}%
\pgfsetdash{}{0pt}%
\pgfpathmoveto{\pgfqpoint{7.498734in}{1.647367in}}%
\pgfpathcurveto{\pgfqpoint{7.503777in}{1.647367in}}{\pgfqpoint{7.508615in}{1.649371in}}{\pgfqpoint{7.512182in}{1.652937in}}%
\pgfpathcurveto{\pgfqpoint{7.515748in}{1.656503in}}{\pgfqpoint{7.517752in}{1.661341in}}{\pgfqpoint{7.517752in}{1.666385in}}%
\pgfpathcurveto{\pgfqpoint{7.517752in}{1.671429in}}{\pgfqpoint{7.515748in}{1.676266in}}{\pgfqpoint{7.512182in}{1.679833in}}%
\pgfpathcurveto{\pgfqpoint{7.508615in}{1.683399in}}{\pgfqpoint{7.503777in}{1.685403in}}{\pgfqpoint{7.498734in}{1.685403in}}%
\pgfpathcurveto{\pgfqpoint{7.493690in}{1.685403in}}{\pgfqpoint{7.488852in}{1.683399in}}{\pgfqpoint{7.485286in}{1.679833in}}%
\pgfpathcurveto{\pgfqpoint{7.481719in}{1.676266in}}{\pgfqpoint{7.479716in}{1.671429in}}{\pgfqpoint{7.479716in}{1.666385in}}%
\pgfpathcurveto{\pgfqpoint{7.479716in}{1.661341in}}{\pgfqpoint{7.481719in}{1.656503in}}{\pgfqpoint{7.485286in}{1.652937in}}%
\pgfpathcurveto{\pgfqpoint{7.488852in}{1.649371in}}{\pgfqpoint{7.493690in}{1.647367in}}{\pgfqpoint{7.498734in}{1.647367in}}%
\pgfpathclose%
\pgfusepath{fill}%
\end{pgfscope}%
\begin{pgfscope}%
\pgfpathrectangle{\pgfqpoint{6.572727in}{0.474100in}}{\pgfqpoint{4.227273in}{3.318700in}}%
\pgfusepath{clip}%
\pgfsetbuttcap%
\pgfsetroundjoin%
\definecolor{currentfill}{rgb}{0.993248,0.906157,0.143936}%
\pgfsetfillcolor{currentfill}%
\pgfsetfillopacity{0.700000}%
\pgfsetlinewidth{0.000000pt}%
\definecolor{currentstroke}{rgb}{0.000000,0.000000,0.000000}%
\pgfsetstrokecolor{currentstroke}%
\pgfsetstrokeopacity{0.700000}%
\pgfsetdash{}{0pt}%
\pgfpathmoveto{\pgfqpoint{8.024086in}{3.334474in}}%
\pgfpathcurveto{\pgfqpoint{8.029130in}{3.334474in}}{\pgfqpoint{8.033967in}{3.336478in}}{\pgfqpoint{8.037534in}{3.340045in}}%
\pgfpathcurveto{\pgfqpoint{8.041100in}{3.343611in}}{\pgfqpoint{8.043104in}{3.348449in}}{\pgfqpoint{8.043104in}{3.353492in}}%
\pgfpathcurveto{\pgfqpoint{8.043104in}{3.358536in}}{\pgfqpoint{8.041100in}{3.363374in}}{\pgfqpoint{8.037534in}{3.366940in}}%
\pgfpathcurveto{\pgfqpoint{8.033967in}{3.370507in}}{\pgfqpoint{8.029130in}{3.372511in}}{\pgfqpoint{8.024086in}{3.372511in}}%
\pgfpathcurveto{\pgfqpoint{8.019042in}{3.372511in}}{\pgfqpoint{8.014205in}{3.370507in}}{\pgfqpoint{8.010638in}{3.366940in}}%
\pgfpathcurveto{\pgfqpoint{8.007072in}{3.363374in}}{\pgfqpoint{8.005068in}{3.358536in}}{\pgfqpoint{8.005068in}{3.353492in}}%
\pgfpathcurveto{\pgfqpoint{8.005068in}{3.348449in}}{\pgfqpoint{8.007072in}{3.343611in}}{\pgfqpoint{8.010638in}{3.340045in}}%
\pgfpathcurveto{\pgfqpoint{8.014205in}{3.336478in}}{\pgfqpoint{8.019042in}{3.334474in}}{\pgfqpoint{8.024086in}{3.334474in}}%
\pgfpathclose%
\pgfusepath{fill}%
\end{pgfscope}%
\begin{pgfscope}%
\pgfpathrectangle{\pgfqpoint{6.572727in}{0.474100in}}{\pgfqpoint{4.227273in}{3.318700in}}%
\pgfusepath{clip}%
\pgfsetbuttcap%
\pgfsetroundjoin%
\definecolor{currentfill}{rgb}{0.267004,0.004874,0.329415}%
\pgfsetfillcolor{currentfill}%
\pgfsetfillopacity{0.700000}%
\pgfsetlinewidth{0.000000pt}%
\definecolor{currentstroke}{rgb}{0.000000,0.000000,0.000000}%
\pgfsetstrokecolor{currentstroke}%
\pgfsetstrokeopacity{0.700000}%
\pgfsetdash{}{0pt}%
\pgfpathmoveto{\pgfqpoint{8.437606in}{1.424616in}}%
\pgfpathcurveto{\pgfqpoint{8.442650in}{1.424616in}}{\pgfqpoint{8.447487in}{1.426620in}}{\pgfqpoint{8.451054in}{1.430186in}}%
\pgfpathcurveto{\pgfqpoint{8.454620in}{1.433753in}}{\pgfqpoint{8.456624in}{1.438590in}}{\pgfqpoint{8.456624in}{1.443634in}}%
\pgfpathcurveto{\pgfqpoint{8.456624in}{1.448678in}}{\pgfqpoint{8.454620in}{1.453516in}}{\pgfqpoint{8.451054in}{1.457082in}}%
\pgfpathcurveto{\pgfqpoint{8.447487in}{1.460648in}}{\pgfqpoint{8.442650in}{1.462652in}}{\pgfqpoint{8.437606in}{1.462652in}}%
\pgfpathcurveto{\pgfqpoint{8.432562in}{1.462652in}}{\pgfqpoint{8.427724in}{1.460648in}}{\pgfqpoint{8.424158in}{1.457082in}}%
\pgfpathcurveto{\pgfqpoint{8.420592in}{1.453516in}}{\pgfqpoint{8.418588in}{1.448678in}}{\pgfqpoint{8.418588in}{1.443634in}}%
\pgfpathcurveto{\pgfqpoint{8.418588in}{1.438590in}}{\pgfqpoint{8.420592in}{1.433753in}}{\pgfqpoint{8.424158in}{1.430186in}}%
\pgfpathcurveto{\pgfqpoint{8.427724in}{1.426620in}}{\pgfqpoint{8.432562in}{1.424616in}}{\pgfqpoint{8.437606in}{1.424616in}}%
\pgfpathclose%
\pgfusepath{fill}%
\end{pgfscope}%
\begin{pgfscope}%
\pgfpathrectangle{\pgfqpoint{6.572727in}{0.474100in}}{\pgfqpoint{4.227273in}{3.318700in}}%
\pgfusepath{clip}%
\pgfsetbuttcap%
\pgfsetroundjoin%
\definecolor{currentfill}{rgb}{0.267004,0.004874,0.329415}%
\pgfsetfillcolor{currentfill}%
\pgfsetfillopacity{0.700000}%
\pgfsetlinewidth{0.000000pt}%
\definecolor{currentstroke}{rgb}{0.000000,0.000000,0.000000}%
\pgfsetstrokecolor{currentstroke}%
\pgfsetstrokeopacity{0.700000}%
\pgfsetdash{}{0pt}%
\pgfpathmoveto{\pgfqpoint{7.658236in}{1.289985in}}%
\pgfpathcurveto{\pgfqpoint{7.663280in}{1.289985in}}{\pgfqpoint{7.668117in}{1.291989in}}{\pgfqpoint{7.671684in}{1.295555in}}%
\pgfpathcurveto{\pgfqpoint{7.675250in}{1.299122in}}{\pgfqpoint{7.677254in}{1.303960in}}{\pgfqpoint{7.677254in}{1.309003in}}%
\pgfpathcurveto{\pgfqpoint{7.677254in}{1.314047in}}{\pgfqpoint{7.675250in}{1.318885in}}{\pgfqpoint{7.671684in}{1.322451in}}%
\pgfpathcurveto{\pgfqpoint{7.668117in}{1.326018in}}{\pgfqpoint{7.663280in}{1.328021in}}{\pgfqpoint{7.658236in}{1.328021in}}%
\pgfpathcurveto{\pgfqpoint{7.653192in}{1.328021in}}{\pgfqpoint{7.648354in}{1.326018in}}{\pgfqpoint{7.644788in}{1.322451in}}%
\pgfpathcurveto{\pgfqpoint{7.641222in}{1.318885in}}{\pgfqpoint{7.639218in}{1.314047in}}{\pgfqpoint{7.639218in}{1.309003in}}%
\pgfpathcurveto{\pgfqpoint{7.639218in}{1.303960in}}{\pgfqpoint{7.641222in}{1.299122in}}{\pgfqpoint{7.644788in}{1.295555in}}%
\pgfpathcurveto{\pgfqpoint{7.648354in}{1.291989in}}{\pgfqpoint{7.653192in}{1.289985in}}{\pgfqpoint{7.658236in}{1.289985in}}%
\pgfpathclose%
\pgfusepath{fill}%
\end{pgfscope}%
\begin{pgfscope}%
\pgfpathrectangle{\pgfqpoint{6.572727in}{0.474100in}}{\pgfqpoint{4.227273in}{3.318700in}}%
\pgfusepath{clip}%
\pgfsetbuttcap%
\pgfsetroundjoin%
\definecolor{currentfill}{rgb}{0.267004,0.004874,0.329415}%
\pgfsetfillcolor{currentfill}%
\pgfsetfillopacity{0.700000}%
\pgfsetlinewidth{0.000000pt}%
\definecolor{currentstroke}{rgb}{0.000000,0.000000,0.000000}%
\pgfsetstrokecolor{currentstroke}%
\pgfsetstrokeopacity{0.700000}%
\pgfsetdash{}{0pt}%
\pgfpathmoveto{\pgfqpoint{7.115113in}{1.658262in}}%
\pgfpathcurveto{\pgfqpoint{7.120156in}{1.658262in}}{\pgfqpoint{7.124994in}{1.660266in}}{\pgfqpoint{7.128560in}{1.663832in}}%
\pgfpathcurveto{\pgfqpoint{7.132127in}{1.667399in}}{\pgfqpoint{7.134131in}{1.672237in}}{\pgfqpoint{7.134131in}{1.677280in}}%
\pgfpathcurveto{\pgfqpoint{7.134131in}{1.682324in}}{\pgfqpoint{7.132127in}{1.687162in}}{\pgfqpoint{7.128560in}{1.690728in}}%
\pgfpathcurveto{\pgfqpoint{7.124994in}{1.694295in}}{\pgfqpoint{7.120156in}{1.696298in}}{\pgfqpoint{7.115113in}{1.696298in}}%
\pgfpathcurveto{\pgfqpoint{7.110069in}{1.696298in}}{\pgfqpoint{7.105231in}{1.694295in}}{\pgfqpoint{7.101665in}{1.690728in}}%
\pgfpathcurveto{\pgfqpoint{7.098098in}{1.687162in}}{\pgfqpoint{7.096094in}{1.682324in}}{\pgfqpoint{7.096094in}{1.677280in}}%
\pgfpathcurveto{\pgfqpoint{7.096094in}{1.672237in}}{\pgfqpoint{7.098098in}{1.667399in}}{\pgfqpoint{7.101665in}{1.663832in}}%
\pgfpathcurveto{\pgfqpoint{7.105231in}{1.660266in}}{\pgfqpoint{7.110069in}{1.658262in}}{\pgfqpoint{7.115113in}{1.658262in}}%
\pgfpathclose%
\pgfusepath{fill}%
\end{pgfscope}%
\begin{pgfscope}%
\pgfpathrectangle{\pgfqpoint{6.572727in}{0.474100in}}{\pgfqpoint{4.227273in}{3.318700in}}%
\pgfusepath{clip}%
\pgfsetbuttcap%
\pgfsetroundjoin%
\definecolor{currentfill}{rgb}{0.993248,0.906157,0.143936}%
\pgfsetfillcolor{currentfill}%
\pgfsetfillopacity{0.700000}%
\pgfsetlinewidth{0.000000pt}%
\definecolor{currentstroke}{rgb}{0.000000,0.000000,0.000000}%
\pgfsetstrokecolor{currentstroke}%
\pgfsetstrokeopacity{0.700000}%
\pgfsetdash{}{0pt}%
\pgfpathmoveto{\pgfqpoint{8.668257in}{3.175987in}}%
\pgfpathcurveto{\pgfqpoint{8.673300in}{3.175987in}}{\pgfqpoint{8.678138in}{3.177991in}}{\pgfqpoint{8.681705in}{3.181558in}}%
\pgfpathcurveto{\pgfqpoint{8.685271in}{3.185124in}}{\pgfqpoint{8.687275in}{3.189962in}}{\pgfqpoint{8.687275in}{3.195005in}}%
\pgfpathcurveto{\pgfqpoint{8.687275in}{3.200049in}}{\pgfqpoint{8.685271in}{3.204887in}}{\pgfqpoint{8.681705in}{3.208453in}}%
\pgfpathcurveto{\pgfqpoint{8.678138in}{3.212020in}}{\pgfqpoint{8.673300in}{3.214024in}}{\pgfqpoint{8.668257in}{3.214024in}}%
\pgfpathcurveto{\pgfqpoint{8.663213in}{3.214024in}}{\pgfqpoint{8.658375in}{3.212020in}}{\pgfqpoint{8.654809in}{3.208453in}}%
\pgfpathcurveto{\pgfqpoint{8.651243in}{3.204887in}}{\pgfqpoint{8.649239in}{3.200049in}}{\pgfqpoint{8.649239in}{3.195005in}}%
\pgfpathcurveto{\pgfqpoint{8.649239in}{3.189962in}}{\pgfqpoint{8.651243in}{3.185124in}}{\pgfqpoint{8.654809in}{3.181558in}}%
\pgfpathcurveto{\pgfqpoint{8.658375in}{3.177991in}}{\pgfqpoint{8.663213in}{3.175987in}}{\pgfqpoint{8.668257in}{3.175987in}}%
\pgfpathclose%
\pgfusepath{fill}%
\end{pgfscope}%
\begin{pgfscope}%
\pgfpathrectangle{\pgfqpoint{6.572727in}{0.474100in}}{\pgfqpoint{4.227273in}{3.318700in}}%
\pgfusepath{clip}%
\pgfsetbuttcap%
\pgfsetroundjoin%
\definecolor{currentfill}{rgb}{0.127568,0.566949,0.550556}%
\pgfsetfillcolor{currentfill}%
\pgfsetfillopacity{0.700000}%
\pgfsetlinewidth{0.000000pt}%
\definecolor{currentstroke}{rgb}{0.000000,0.000000,0.000000}%
\pgfsetstrokecolor{currentstroke}%
\pgfsetstrokeopacity{0.700000}%
\pgfsetdash{}{0pt}%
\pgfpathmoveto{\pgfqpoint{9.451032in}{1.267230in}}%
\pgfpathcurveto{\pgfqpoint{9.456076in}{1.267230in}}{\pgfqpoint{9.460914in}{1.269234in}}{\pgfqpoint{9.464480in}{1.272801in}}%
\pgfpathcurveto{\pgfqpoint{9.468047in}{1.276367in}}{\pgfqpoint{9.470050in}{1.281205in}}{\pgfqpoint{9.470050in}{1.286249in}}%
\pgfpathcurveto{\pgfqpoint{9.470050in}{1.291292in}}{\pgfqpoint{9.468047in}{1.296130in}}{\pgfqpoint{9.464480in}{1.299696in}}%
\pgfpathcurveto{\pgfqpoint{9.460914in}{1.303263in}}{\pgfqpoint{9.456076in}{1.305267in}}{\pgfqpoint{9.451032in}{1.305267in}}%
\pgfpathcurveto{\pgfqpoint{9.445989in}{1.305267in}}{\pgfqpoint{9.441151in}{1.303263in}}{\pgfqpoint{9.437584in}{1.299696in}}%
\pgfpathcurveto{\pgfqpoint{9.434018in}{1.296130in}}{\pgfqpoint{9.432014in}{1.291292in}}{\pgfqpoint{9.432014in}{1.286249in}}%
\pgfpathcurveto{\pgfqpoint{9.432014in}{1.281205in}}{\pgfqpoint{9.434018in}{1.276367in}}{\pgfqpoint{9.437584in}{1.272801in}}%
\pgfpathcurveto{\pgfqpoint{9.441151in}{1.269234in}}{\pgfqpoint{9.445989in}{1.267230in}}{\pgfqpoint{9.451032in}{1.267230in}}%
\pgfpathclose%
\pgfusepath{fill}%
\end{pgfscope}%
\begin{pgfscope}%
\pgfpathrectangle{\pgfqpoint{6.572727in}{0.474100in}}{\pgfqpoint{4.227273in}{3.318700in}}%
\pgfusepath{clip}%
\pgfsetbuttcap%
\pgfsetroundjoin%
\definecolor{currentfill}{rgb}{0.267004,0.004874,0.329415}%
\pgfsetfillcolor{currentfill}%
\pgfsetfillopacity{0.700000}%
\pgfsetlinewidth{0.000000pt}%
\definecolor{currentstroke}{rgb}{0.000000,0.000000,0.000000}%
\pgfsetstrokecolor{currentstroke}%
\pgfsetstrokeopacity{0.700000}%
\pgfsetdash{}{0pt}%
\pgfpathmoveto{\pgfqpoint{7.725862in}{1.814469in}}%
\pgfpathcurveto{\pgfqpoint{7.730905in}{1.814469in}}{\pgfqpoint{7.735743in}{1.816473in}}{\pgfqpoint{7.739310in}{1.820040in}}%
\pgfpathcurveto{\pgfqpoint{7.742876in}{1.823606in}}{\pgfqpoint{7.744880in}{1.828444in}}{\pgfqpoint{7.744880in}{1.833488in}}%
\pgfpathcurveto{\pgfqpoint{7.744880in}{1.838531in}}{\pgfqpoint{7.742876in}{1.843369in}}{\pgfqpoint{7.739310in}{1.846935in}}%
\pgfpathcurveto{\pgfqpoint{7.735743in}{1.850502in}}{\pgfqpoint{7.730905in}{1.852506in}}{\pgfqpoint{7.725862in}{1.852506in}}%
\pgfpathcurveto{\pgfqpoint{7.720818in}{1.852506in}}{\pgfqpoint{7.715980in}{1.850502in}}{\pgfqpoint{7.712414in}{1.846935in}}%
\pgfpathcurveto{\pgfqpoint{7.708847in}{1.843369in}}{\pgfqpoint{7.706844in}{1.838531in}}{\pgfqpoint{7.706844in}{1.833488in}}%
\pgfpathcurveto{\pgfqpoint{7.706844in}{1.828444in}}{\pgfqpoint{7.708847in}{1.823606in}}{\pgfqpoint{7.712414in}{1.820040in}}%
\pgfpathcurveto{\pgfqpoint{7.715980in}{1.816473in}}{\pgfqpoint{7.720818in}{1.814469in}}{\pgfqpoint{7.725862in}{1.814469in}}%
\pgfpathclose%
\pgfusepath{fill}%
\end{pgfscope}%
\begin{pgfscope}%
\pgfpathrectangle{\pgfqpoint{6.572727in}{0.474100in}}{\pgfqpoint{4.227273in}{3.318700in}}%
\pgfusepath{clip}%
\pgfsetbuttcap%
\pgfsetroundjoin%
\definecolor{currentfill}{rgb}{0.127568,0.566949,0.550556}%
\pgfsetfillcolor{currentfill}%
\pgfsetfillopacity{0.700000}%
\pgfsetlinewidth{0.000000pt}%
\definecolor{currentstroke}{rgb}{0.000000,0.000000,0.000000}%
\pgfsetstrokecolor{currentstroke}%
\pgfsetstrokeopacity{0.700000}%
\pgfsetdash{}{0pt}%
\pgfpathmoveto{\pgfqpoint{9.831010in}{1.605514in}}%
\pgfpathcurveto{\pgfqpoint{9.836054in}{1.605514in}}{\pgfqpoint{9.840892in}{1.607518in}}{\pgfqpoint{9.844458in}{1.611085in}}%
\pgfpathcurveto{\pgfqpoint{9.848025in}{1.614651in}}{\pgfqpoint{9.850028in}{1.619489in}}{\pgfqpoint{9.850028in}{1.624532in}}%
\pgfpathcurveto{\pgfqpoint{9.850028in}{1.629576in}}{\pgfqpoint{9.848025in}{1.634414in}}{\pgfqpoint{9.844458in}{1.637980in}}%
\pgfpathcurveto{\pgfqpoint{9.840892in}{1.641547in}}{\pgfqpoint{9.836054in}{1.643551in}}{\pgfqpoint{9.831010in}{1.643551in}}%
\pgfpathcurveto{\pgfqpoint{9.825967in}{1.643551in}}{\pgfqpoint{9.821129in}{1.641547in}}{\pgfqpoint{9.817562in}{1.637980in}}%
\pgfpathcurveto{\pgfqpoint{9.813996in}{1.634414in}}{\pgfqpoint{9.811992in}{1.629576in}}{\pgfqpoint{9.811992in}{1.624532in}}%
\pgfpathcurveto{\pgfqpoint{9.811992in}{1.619489in}}{\pgfqpoint{9.813996in}{1.614651in}}{\pgfqpoint{9.817562in}{1.611085in}}%
\pgfpathcurveto{\pgfqpoint{9.821129in}{1.607518in}}{\pgfqpoint{9.825967in}{1.605514in}}{\pgfqpoint{9.831010in}{1.605514in}}%
\pgfpathclose%
\pgfusepath{fill}%
\end{pgfscope}%
\begin{pgfscope}%
\pgfpathrectangle{\pgfqpoint{6.572727in}{0.474100in}}{\pgfqpoint{4.227273in}{3.318700in}}%
\pgfusepath{clip}%
\pgfsetbuttcap%
\pgfsetroundjoin%
\definecolor{currentfill}{rgb}{0.993248,0.906157,0.143936}%
\pgfsetfillcolor{currentfill}%
\pgfsetfillopacity{0.700000}%
\pgfsetlinewidth{0.000000pt}%
\definecolor{currentstroke}{rgb}{0.000000,0.000000,0.000000}%
\pgfsetstrokecolor{currentstroke}%
\pgfsetstrokeopacity{0.700000}%
\pgfsetdash{}{0pt}%
\pgfpathmoveto{\pgfqpoint{8.677976in}{2.753620in}}%
\pgfpathcurveto{\pgfqpoint{8.683019in}{2.753620in}}{\pgfqpoint{8.687857in}{2.755624in}}{\pgfqpoint{8.691424in}{2.759190in}}%
\pgfpathcurveto{\pgfqpoint{8.694990in}{2.762757in}}{\pgfqpoint{8.696994in}{2.767594in}}{\pgfqpoint{8.696994in}{2.772638in}}%
\pgfpathcurveto{\pgfqpoint{8.696994in}{2.777682in}}{\pgfqpoint{8.694990in}{2.782520in}}{\pgfqpoint{8.691424in}{2.786086in}}%
\pgfpathcurveto{\pgfqpoint{8.687857in}{2.789652in}}{\pgfqpoint{8.683019in}{2.791656in}}{\pgfqpoint{8.677976in}{2.791656in}}%
\pgfpathcurveto{\pgfqpoint{8.672932in}{2.791656in}}{\pgfqpoint{8.668094in}{2.789652in}}{\pgfqpoint{8.664528in}{2.786086in}}%
\pgfpathcurveto{\pgfqpoint{8.660961in}{2.782520in}}{\pgfqpoint{8.658958in}{2.777682in}}{\pgfqpoint{8.658958in}{2.772638in}}%
\pgfpathcurveto{\pgfqpoint{8.658958in}{2.767594in}}{\pgfqpoint{8.660961in}{2.762757in}}{\pgfqpoint{8.664528in}{2.759190in}}%
\pgfpathcurveto{\pgfqpoint{8.668094in}{2.755624in}}{\pgfqpoint{8.672932in}{2.753620in}}{\pgfqpoint{8.677976in}{2.753620in}}%
\pgfpathclose%
\pgfusepath{fill}%
\end{pgfscope}%
\begin{pgfscope}%
\pgfpathrectangle{\pgfqpoint{6.572727in}{0.474100in}}{\pgfqpoint{4.227273in}{3.318700in}}%
\pgfusepath{clip}%
\pgfsetbuttcap%
\pgfsetroundjoin%
\definecolor{currentfill}{rgb}{0.127568,0.566949,0.550556}%
\pgfsetfillcolor{currentfill}%
\pgfsetfillopacity{0.700000}%
\pgfsetlinewidth{0.000000pt}%
\definecolor{currentstroke}{rgb}{0.000000,0.000000,0.000000}%
\pgfsetstrokecolor{currentstroke}%
\pgfsetstrokeopacity{0.700000}%
\pgfsetdash{}{0pt}%
\pgfpathmoveto{\pgfqpoint{9.400929in}{1.170969in}}%
\pgfpathcurveto{\pgfqpoint{9.405973in}{1.170969in}}{\pgfqpoint{9.410810in}{1.172973in}}{\pgfqpoint{9.414377in}{1.176540in}}%
\pgfpathcurveto{\pgfqpoint{9.417943in}{1.180106in}}{\pgfqpoint{9.419947in}{1.184944in}}{\pgfqpoint{9.419947in}{1.189987in}}%
\pgfpathcurveto{\pgfqpoint{9.419947in}{1.195031in}}{\pgfqpoint{9.417943in}{1.199869in}}{\pgfqpoint{9.414377in}{1.203435in}}%
\pgfpathcurveto{\pgfqpoint{9.410810in}{1.207002in}}{\pgfqpoint{9.405973in}{1.209006in}}{\pgfqpoint{9.400929in}{1.209006in}}%
\pgfpathcurveto{\pgfqpoint{9.395885in}{1.209006in}}{\pgfqpoint{9.391048in}{1.207002in}}{\pgfqpoint{9.387481in}{1.203435in}}%
\pgfpathcurveto{\pgfqpoint{9.383915in}{1.199869in}}{\pgfqpoint{9.381911in}{1.195031in}}{\pgfqpoint{9.381911in}{1.189987in}}%
\pgfpathcurveto{\pgfqpoint{9.381911in}{1.184944in}}{\pgfqpoint{9.383915in}{1.180106in}}{\pgfqpoint{9.387481in}{1.176540in}}%
\pgfpathcurveto{\pgfqpoint{9.391048in}{1.172973in}}{\pgfqpoint{9.395885in}{1.170969in}}{\pgfqpoint{9.400929in}{1.170969in}}%
\pgfpathclose%
\pgfusepath{fill}%
\end{pgfscope}%
\begin{pgfscope}%
\pgfpathrectangle{\pgfqpoint{6.572727in}{0.474100in}}{\pgfqpoint{4.227273in}{3.318700in}}%
\pgfusepath{clip}%
\pgfsetbuttcap%
\pgfsetroundjoin%
\definecolor{currentfill}{rgb}{0.267004,0.004874,0.329415}%
\pgfsetfillcolor{currentfill}%
\pgfsetfillopacity{0.700000}%
\pgfsetlinewidth{0.000000pt}%
\definecolor{currentstroke}{rgb}{0.000000,0.000000,0.000000}%
\pgfsetstrokecolor{currentstroke}%
\pgfsetstrokeopacity{0.700000}%
\pgfsetdash{}{0pt}%
\pgfpathmoveto{\pgfqpoint{8.138448in}{1.632185in}}%
\pgfpathcurveto{\pgfqpoint{8.143492in}{1.632185in}}{\pgfqpoint{8.148329in}{1.634189in}}{\pgfqpoint{8.151896in}{1.637756in}}%
\pgfpathcurveto{\pgfqpoint{8.155462in}{1.641322in}}{\pgfqpoint{8.157466in}{1.646160in}}{\pgfqpoint{8.157466in}{1.651204in}}%
\pgfpathcurveto{\pgfqpoint{8.157466in}{1.656247in}}{\pgfqpoint{8.155462in}{1.661085in}}{\pgfqpoint{8.151896in}{1.664651in}}%
\pgfpathcurveto{\pgfqpoint{8.148329in}{1.668218in}}{\pgfqpoint{8.143492in}{1.670222in}}{\pgfqpoint{8.138448in}{1.670222in}}%
\pgfpathcurveto{\pgfqpoint{8.133404in}{1.670222in}}{\pgfqpoint{8.128567in}{1.668218in}}{\pgfqpoint{8.125000in}{1.664651in}}%
\pgfpathcurveto{\pgfqpoint{8.121434in}{1.661085in}}{\pgfqpoint{8.119430in}{1.656247in}}{\pgfqpoint{8.119430in}{1.651204in}}%
\pgfpathcurveto{\pgfqpoint{8.119430in}{1.646160in}}{\pgfqpoint{8.121434in}{1.641322in}}{\pgfqpoint{8.125000in}{1.637756in}}%
\pgfpathcurveto{\pgfqpoint{8.128567in}{1.634189in}}{\pgfqpoint{8.133404in}{1.632185in}}{\pgfqpoint{8.138448in}{1.632185in}}%
\pgfpathclose%
\pgfusepath{fill}%
\end{pgfscope}%
\begin{pgfscope}%
\pgfpathrectangle{\pgfqpoint{6.572727in}{0.474100in}}{\pgfqpoint{4.227273in}{3.318700in}}%
\pgfusepath{clip}%
\pgfsetbuttcap%
\pgfsetroundjoin%
\definecolor{currentfill}{rgb}{0.127568,0.566949,0.550556}%
\pgfsetfillcolor{currentfill}%
\pgfsetfillopacity{0.700000}%
\pgfsetlinewidth{0.000000pt}%
\definecolor{currentstroke}{rgb}{0.000000,0.000000,0.000000}%
\pgfsetstrokecolor{currentstroke}%
\pgfsetstrokeopacity{0.700000}%
\pgfsetdash{}{0pt}%
\pgfpathmoveto{\pgfqpoint{9.633024in}{1.413755in}}%
\pgfpathcurveto{\pgfqpoint{9.638067in}{1.413755in}}{\pgfqpoint{9.642905in}{1.415759in}}{\pgfqpoint{9.646472in}{1.419325in}}%
\pgfpathcurveto{\pgfqpoint{9.650038in}{1.422892in}}{\pgfqpoint{9.652042in}{1.427729in}}{\pgfqpoint{9.652042in}{1.432773in}}%
\pgfpathcurveto{\pgfqpoint{9.652042in}{1.437817in}}{\pgfqpoint{9.650038in}{1.442654in}}{\pgfqpoint{9.646472in}{1.446221in}}%
\pgfpathcurveto{\pgfqpoint{9.642905in}{1.449787in}}{\pgfqpoint{9.638067in}{1.451791in}}{\pgfqpoint{9.633024in}{1.451791in}}%
\pgfpathcurveto{\pgfqpoint{9.627980in}{1.451791in}}{\pgfqpoint{9.623142in}{1.449787in}}{\pgfqpoint{9.619576in}{1.446221in}}%
\pgfpathcurveto{\pgfqpoint{9.616009in}{1.442654in}}{\pgfqpoint{9.614006in}{1.437817in}}{\pgfqpoint{9.614006in}{1.432773in}}%
\pgfpathcurveto{\pgfqpoint{9.614006in}{1.427729in}}{\pgfqpoint{9.616009in}{1.422892in}}{\pgfqpoint{9.619576in}{1.419325in}}%
\pgfpathcurveto{\pgfqpoint{9.623142in}{1.415759in}}{\pgfqpoint{9.627980in}{1.413755in}}{\pgfqpoint{9.633024in}{1.413755in}}%
\pgfpathclose%
\pgfusepath{fill}%
\end{pgfscope}%
\begin{pgfscope}%
\pgfpathrectangle{\pgfqpoint{6.572727in}{0.474100in}}{\pgfqpoint{4.227273in}{3.318700in}}%
\pgfusepath{clip}%
\pgfsetbuttcap%
\pgfsetroundjoin%
\definecolor{currentfill}{rgb}{0.993248,0.906157,0.143936}%
\pgfsetfillcolor{currentfill}%
\pgfsetfillopacity{0.700000}%
\pgfsetlinewidth{0.000000pt}%
\definecolor{currentstroke}{rgb}{0.000000,0.000000,0.000000}%
\pgfsetstrokecolor{currentstroke}%
\pgfsetstrokeopacity{0.700000}%
\pgfsetdash{}{0pt}%
\pgfpathmoveto{\pgfqpoint{8.248668in}{3.412817in}}%
\pgfpathcurveto{\pgfqpoint{8.253712in}{3.412817in}}{\pgfqpoint{8.258549in}{3.414821in}}{\pgfqpoint{8.262116in}{3.418388in}}%
\pgfpathcurveto{\pgfqpoint{8.265682in}{3.421954in}}{\pgfqpoint{8.267686in}{3.426792in}}{\pgfqpoint{8.267686in}{3.431836in}}%
\pgfpathcurveto{\pgfqpoint{8.267686in}{3.436879in}}{\pgfqpoint{8.265682in}{3.441717in}}{\pgfqpoint{8.262116in}{3.445283in}}%
\pgfpathcurveto{\pgfqpoint{8.258549in}{3.448850in}}{\pgfqpoint{8.253712in}{3.450854in}}{\pgfqpoint{8.248668in}{3.450854in}}%
\pgfpathcurveto{\pgfqpoint{8.243624in}{3.450854in}}{\pgfqpoint{8.238787in}{3.448850in}}{\pgfqpoint{8.235220in}{3.445283in}}%
\pgfpathcurveto{\pgfqpoint{8.231654in}{3.441717in}}{\pgfqpoint{8.229650in}{3.436879in}}{\pgfqpoint{8.229650in}{3.431836in}}%
\pgfpathcurveto{\pgfqpoint{8.229650in}{3.426792in}}{\pgfqpoint{8.231654in}{3.421954in}}{\pgfqpoint{8.235220in}{3.418388in}}%
\pgfpathcurveto{\pgfqpoint{8.238787in}{3.414821in}}{\pgfqpoint{8.243624in}{3.412817in}}{\pgfqpoint{8.248668in}{3.412817in}}%
\pgfpathclose%
\pgfusepath{fill}%
\end{pgfscope}%
\begin{pgfscope}%
\pgfpathrectangle{\pgfqpoint{6.572727in}{0.474100in}}{\pgfqpoint{4.227273in}{3.318700in}}%
\pgfusepath{clip}%
\pgfsetbuttcap%
\pgfsetroundjoin%
\definecolor{currentfill}{rgb}{0.993248,0.906157,0.143936}%
\pgfsetfillcolor{currentfill}%
\pgfsetfillopacity{0.700000}%
\pgfsetlinewidth{0.000000pt}%
\definecolor{currentstroke}{rgb}{0.000000,0.000000,0.000000}%
\pgfsetstrokecolor{currentstroke}%
\pgfsetstrokeopacity{0.700000}%
\pgfsetdash{}{0pt}%
\pgfpathmoveto{\pgfqpoint{7.897783in}{2.893799in}}%
\pgfpathcurveto{\pgfqpoint{7.902826in}{2.893799in}}{\pgfqpoint{7.907664in}{2.895803in}}{\pgfqpoint{7.911230in}{2.899369in}}%
\pgfpathcurveto{\pgfqpoint{7.914797in}{2.902936in}}{\pgfqpoint{7.916801in}{2.907773in}}{\pgfqpoint{7.916801in}{2.912817in}}%
\pgfpathcurveto{\pgfqpoint{7.916801in}{2.917861in}}{\pgfqpoint{7.914797in}{2.922698in}}{\pgfqpoint{7.911230in}{2.926265in}}%
\pgfpathcurveto{\pgfqpoint{7.907664in}{2.929831in}}{\pgfqpoint{7.902826in}{2.931835in}}{\pgfqpoint{7.897783in}{2.931835in}}%
\pgfpathcurveto{\pgfqpoint{7.892739in}{2.931835in}}{\pgfqpoint{7.887901in}{2.929831in}}{\pgfqpoint{7.884335in}{2.926265in}}%
\pgfpathcurveto{\pgfqpoint{7.880768in}{2.922698in}}{\pgfqpoint{7.878764in}{2.917861in}}{\pgfqpoint{7.878764in}{2.912817in}}%
\pgfpathcurveto{\pgfqpoint{7.878764in}{2.907773in}}{\pgfqpoint{7.880768in}{2.902936in}}{\pgfqpoint{7.884335in}{2.899369in}}%
\pgfpathcurveto{\pgfqpoint{7.887901in}{2.895803in}}{\pgfqpoint{7.892739in}{2.893799in}}{\pgfqpoint{7.897783in}{2.893799in}}%
\pgfpathclose%
\pgfusepath{fill}%
\end{pgfscope}%
\begin{pgfscope}%
\pgfpathrectangle{\pgfqpoint{6.572727in}{0.474100in}}{\pgfqpoint{4.227273in}{3.318700in}}%
\pgfusepath{clip}%
\pgfsetbuttcap%
\pgfsetroundjoin%
\definecolor{currentfill}{rgb}{0.127568,0.566949,0.550556}%
\pgfsetfillcolor{currentfill}%
\pgfsetfillopacity{0.700000}%
\pgfsetlinewidth{0.000000pt}%
\definecolor{currentstroke}{rgb}{0.000000,0.000000,0.000000}%
\pgfsetstrokecolor{currentstroke}%
\pgfsetstrokeopacity{0.700000}%
\pgfsetdash{}{0pt}%
\pgfpathmoveto{\pgfqpoint{10.106096in}{1.447987in}}%
\pgfpathcurveto{\pgfqpoint{10.111139in}{1.447987in}}{\pgfqpoint{10.115977in}{1.449991in}}{\pgfqpoint{10.119543in}{1.453558in}}%
\pgfpathcurveto{\pgfqpoint{10.123110in}{1.457124in}}{\pgfqpoint{10.125114in}{1.461962in}}{\pgfqpoint{10.125114in}{1.467005in}}%
\pgfpathcurveto{\pgfqpoint{10.125114in}{1.472049in}}{\pgfqpoint{10.123110in}{1.476887in}}{\pgfqpoint{10.119543in}{1.480453in}}%
\pgfpathcurveto{\pgfqpoint{10.115977in}{1.484020in}}{\pgfqpoint{10.111139in}{1.486024in}}{\pgfqpoint{10.106096in}{1.486024in}}%
\pgfpathcurveto{\pgfqpoint{10.101052in}{1.486024in}}{\pgfqpoint{10.096214in}{1.484020in}}{\pgfqpoint{10.092648in}{1.480453in}}%
\pgfpathcurveto{\pgfqpoint{10.089081in}{1.476887in}}{\pgfqpoint{10.087077in}{1.472049in}}{\pgfqpoint{10.087077in}{1.467005in}}%
\pgfpathcurveto{\pgfqpoint{10.087077in}{1.461962in}}{\pgfqpoint{10.089081in}{1.457124in}}{\pgfqpoint{10.092648in}{1.453558in}}%
\pgfpathcurveto{\pgfqpoint{10.096214in}{1.449991in}}{\pgfqpoint{10.101052in}{1.447987in}}{\pgfqpoint{10.106096in}{1.447987in}}%
\pgfpathclose%
\pgfusepath{fill}%
\end{pgfscope}%
\begin{pgfscope}%
\pgfpathrectangle{\pgfqpoint{6.572727in}{0.474100in}}{\pgfqpoint{4.227273in}{3.318700in}}%
\pgfusepath{clip}%
\pgfsetbuttcap%
\pgfsetroundjoin%
\definecolor{currentfill}{rgb}{0.993248,0.906157,0.143936}%
\pgfsetfillcolor{currentfill}%
\pgfsetfillopacity{0.700000}%
\pgfsetlinewidth{0.000000pt}%
\definecolor{currentstroke}{rgb}{0.000000,0.000000,0.000000}%
\pgfsetstrokecolor{currentstroke}%
\pgfsetstrokeopacity{0.700000}%
\pgfsetdash{}{0pt}%
\pgfpathmoveto{\pgfqpoint{7.713252in}{3.082322in}}%
\pgfpathcurveto{\pgfqpoint{7.718295in}{3.082322in}}{\pgfqpoint{7.723133in}{3.084326in}}{\pgfqpoint{7.726699in}{3.087892in}}%
\pgfpathcurveto{\pgfqpoint{7.730266in}{3.091458in}}{\pgfqpoint{7.732270in}{3.096296in}}{\pgfqpoint{7.732270in}{3.101340in}}%
\pgfpathcurveto{\pgfqpoint{7.732270in}{3.106384in}}{\pgfqpoint{7.730266in}{3.111221in}}{\pgfqpoint{7.726699in}{3.114788in}}%
\pgfpathcurveto{\pgfqpoint{7.723133in}{3.118354in}}{\pgfqpoint{7.718295in}{3.120358in}}{\pgfqpoint{7.713252in}{3.120358in}}%
\pgfpathcurveto{\pgfqpoint{7.708208in}{3.120358in}}{\pgfqpoint{7.703370in}{3.118354in}}{\pgfqpoint{7.699804in}{3.114788in}}%
\pgfpathcurveto{\pgfqpoint{7.696237in}{3.111221in}}{\pgfqpoint{7.694233in}{3.106384in}}{\pgfqpoint{7.694233in}{3.101340in}}%
\pgfpathcurveto{\pgfqpoint{7.694233in}{3.096296in}}{\pgfqpoint{7.696237in}{3.091458in}}{\pgfqpoint{7.699804in}{3.087892in}}%
\pgfpathcurveto{\pgfqpoint{7.703370in}{3.084326in}}{\pgfqpoint{7.708208in}{3.082322in}}{\pgfqpoint{7.713252in}{3.082322in}}%
\pgfpathclose%
\pgfusepath{fill}%
\end{pgfscope}%
\begin{pgfscope}%
\pgfpathrectangle{\pgfqpoint{6.572727in}{0.474100in}}{\pgfqpoint{4.227273in}{3.318700in}}%
\pgfusepath{clip}%
\pgfsetbuttcap%
\pgfsetroundjoin%
\definecolor{currentfill}{rgb}{0.993248,0.906157,0.143936}%
\pgfsetfillcolor{currentfill}%
\pgfsetfillopacity{0.700000}%
\pgfsetlinewidth{0.000000pt}%
\definecolor{currentstroke}{rgb}{0.000000,0.000000,0.000000}%
\pgfsetstrokecolor{currentstroke}%
\pgfsetstrokeopacity{0.700000}%
\pgfsetdash{}{0pt}%
\pgfpathmoveto{\pgfqpoint{7.902055in}{2.390405in}}%
\pgfpathcurveto{\pgfqpoint{7.907099in}{2.390405in}}{\pgfqpoint{7.911937in}{2.392409in}}{\pgfqpoint{7.915503in}{2.395976in}}%
\pgfpathcurveto{\pgfqpoint{7.919070in}{2.399542in}}{\pgfqpoint{7.921073in}{2.404380in}}{\pgfqpoint{7.921073in}{2.409424in}}%
\pgfpathcurveto{\pgfqpoint{7.921073in}{2.414467in}}{\pgfqpoint{7.919070in}{2.419305in}}{\pgfqpoint{7.915503in}{2.422871in}}%
\pgfpathcurveto{\pgfqpoint{7.911937in}{2.426438in}}{\pgfqpoint{7.907099in}{2.428442in}}{\pgfqpoint{7.902055in}{2.428442in}}%
\pgfpathcurveto{\pgfqpoint{7.897012in}{2.428442in}}{\pgfqpoint{7.892174in}{2.426438in}}{\pgfqpoint{7.888607in}{2.422871in}}%
\pgfpathcurveto{\pgfqpoint{7.885041in}{2.419305in}}{\pgfqpoint{7.883037in}{2.414467in}}{\pgfqpoint{7.883037in}{2.409424in}}%
\pgfpathcurveto{\pgfqpoint{7.883037in}{2.404380in}}{\pgfqpoint{7.885041in}{2.399542in}}{\pgfqpoint{7.888607in}{2.395976in}}%
\pgfpathcurveto{\pgfqpoint{7.892174in}{2.392409in}}{\pgfqpoint{7.897012in}{2.390405in}}{\pgfqpoint{7.902055in}{2.390405in}}%
\pgfpathclose%
\pgfusepath{fill}%
\end{pgfscope}%
\begin{pgfscope}%
\pgfpathrectangle{\pgfqpoint{6.572727in}{0.474100in}}{\pgfqpoint{4.227273in}{3.318700in}}%
\pgfusepath{clip}%
\pgfsetbuttcap%
\pgfsetroundjoin%
\definecolor{currentfill}{rgb}{0.993248,0.906157,0.143936}%
\pgfsetfillcolor{currentfill}%
\pgfsetfillopacity{0.700000}%
\pgfsetlinewidth{0.000000pt}%
\definecolor{currentstroke}{rgb}{0.000000,0.000000,0.000000}%
\pgfsetstrokecolor{currentstroke}%
\pgfsetstrokeopacity{0.700000}%
\pgfsetdash{}{0pt}%
\pgfpathmoveto{\pgfqpoint{8.729553in}{2.518050in}}%
\pgfpathcurveto{\pgfqpoint{8.734596in}{2.518050in}}{\pgfqpoint{8.739434in}{2.520054in}}{\pgfqpoint{8.743001in}{2.523620in}}%
\pgfpathcurveto{\pgfqpoint{8.746567in}{2.527187in}}{\pgfqpoint{8.748571in}{2.532025in}}{\pgfqpoint{8.748571in}{2.537068in}}%
\pgfpathcurveto{\pgfqpoint{8.748571in}{2.542112in}}{\pgfqpoint{8.746567in}{2.546950in}}{\pgfqpoint{8.743001in}{2.550516in}}%
\pgfpathcurveto{\pgfqpoint{8.739434in}{2.554083in}}{\pgfqpoint{8.734596in}{2.556086in}}{\pgfqpoint{8.729553in}{2.556086in}}%
\pgfpathcurveto{\pgfqpoint{8.724509in}{2.556086in}}{\pgfqpoint{8.719671in}{2.554083in}}{\pgfqpoint{8.716105in}{2.550516in}}%
\pgfpathcurveto{\pgfqpoint{8.712538in}{2.546950in}}{\pgfqpoint{8.710535in}{2.542112in}}{\pgfqpoint{8.710535in}{2.537068in}}%
\pgfpathcurveto{\pgfqpoint{8.710535in}{2.532025in}}{\pgfqpoint{8.712538in}{2.527187in}}{\pgfqpoint{8.716105in}{2.523620in}}%
\pgfpathcurveto{\pgfqpoint{8.719671in}{2.520054in}}{\pgfqpoint{8.724509in}{2.518050in}}{\pgfqpoint{8.729553in}{2.518050in}}%
\pgfpathclose%
\pgfusepath{fill}%
\end{pgfscope}%
\begin{pgfscope}%
\pgfpathrectangle{\pgfqpoint{6.572727in}{0.474100in}}{\pgfqpoint{4.227273in}{3.318700in}}%
\pgfusepath{clip}%
\pgfsetbuttcap%
\pgfsetroundjoin%
\definecolor{currentfill}{rgb}{0.993248,0.906157,0.143936}%
\pgfsetfillcolor{currentfill}%
\pgfsetfillopacity{0.700000}%
\pgfsetlinewidth{0.000000pt}%
\definecolor{currentstroke}{rgb}{0.000000,0.000000,0.000000}%
\pgfsetstrokecolor{currentstroke}%
\pgfsetstrokeopacity{0.700000}%
\pgfsetdash{}{0pt}%
\pgfpathmoveto{\pgfqpoint{8.533761in}{2.657549in}}%
\pgfpathcurveto{\pgfqpoint{8.538805in}{2.657549in}}{\pgfqpoint{8.543643in}{2.659553in}}{\pgfqpoint{8.547209in}{2.663119in}}%
\pgfpathcurveto{\pgfqpoint{8.550776in}{2.666686in}}{\pgfqpoint{8.552780in}{2.671523in}}{\pgfqpoint{8.552780in}{2.676567in}}%
\pgfpathcurveto{\pgfqpoint{8.552780in}{2.681611in}}{\pgfqpoint{8.550776in}{2.686448in}}{\pgfqpoint{8.547209in}{2.690015in}}%
\pgfpathcurveto{\pgfqpoint{8.543643in}{2.693581in}}{\pgfqpoint{8.538805in}{2.695585in}}{\pgfqpoint{8.533761in}{2.695585in}}%
\pgfpathcurveto{\pgfqpoint{8.528718in}{2.695585in}}{\pgfqpoint{8.523880in}{2.693581in}}{\pgfqpoint{8.520314in}{2.690015in}}%
\pgfpathcurveto{\pgfqpoint{8.516747in}{2.686448in}}{\pgfqpoint{8.514743in}{2.681611in}}{\pgfqpoint{8.514743in}{2.676567in}}%
\pgfpathcurveto{\pgfqpoint{8.514743in}{2.671523in}}{\pgfqpoint{8.516747in}{2.666686in}}{\pgfqpoint{8.520314in}{2.663119in}}%
\pgfpathcurveto{\pgfqpoint{8.523880in}{2.659553in}}{\pgfqpoint{8.528718in}{2.657549in}}{\pgfqpoint{8.533761in}{2.657549in}}%
\pgfpathclose%
\pgfusepath{fill}%
\end{pgfscope}%
\begin{pgfscope}%
\pgfpathrectangle{\pgfqpoint{6.572727in}{0.474100in}}{\pgfqpoint{4.227273in}{3.318700in}}%
\pgfusepath{clip}%
\pgfsetbuttcap%
\pgfsetroundjoin%
\definecolor{currentfill}{rgb}{0.993248,0.906157,0.143936}%
\pgfsetfillcolor{currentfill}%
\pgfsetfillopacity{0.700000}%
\pgfsetlinewidth{0.000000pt}%
\definecolor{currentstroke}{rgb}{0.000000,0.000000,0.000000}%
\pgfsetstrokecolor{currentstroke}%
\pgfsetstrokeopacity{0.700000}%
\pgfsetdash{}{0pt}%
\pgfpathmoveto{\pgfqpoint{7.503841in}{2.633058in}}%
\pgfpathcurveto{\pgfqpoint{7.508884in}{2.633058in}}{\pgfqpoint{7.513722in}{2.635061in}}{\pgfqpoint{7.517289in}{2.638628in}}%
\pgfpathcurveto{\pgfqpoint{7.520855in}{2.642194in}}{\pgfqpoint{7.522859in}{2.647032in}}{\pgfqpoint{7.522859in}{2.652076in}}%
\pgfpathcurveto{\pgfqpoint{7.522859in}{2.657119in}}{\pgfqpoint{7.520855in}{2.661957in}}{\pgfqpoint{7.517289in}{2.665524in}}%
\pgfpathcurveto{\pgfqpoint{7.513722in}{2.669090in}}{\pgfqpoint{7.508884in}{2.671094in}}{\pgfqpoint{7.503841in}{2.671094in}}%
\pgfpathcurveto{\pgfqpoint{7.498797in}{2.671094in}}{\pgfqpoint{7.493959in}{2.669090in}}{\pgfqpoint{7.490393in}{2.665524in}}%
\pgfpathcurveto{\pgfqpoint{7.486826in}{2.661957in}}{\pgfqpoint{7.484823in}{2.657119in}}{\pgfqpoint{7.484823in}{2.652076in}}%
\pgfpathcurveto{\pgfqpoint{7.484823in}{2.647032in}}{\pgfqpoint{7.486826in}{2.642194in}}{\pgfqpoint{7.490393in}{2.638628in}}%
\pgfpathcurveto{\pgfqpoint{7.493959in}{2.635061in}}{\pgfqpoint{7.498797in}{2.633058in}}{\pgfqpoint{7.503841in}{2.633058in}}%
\pgfpathclose%
\pgfusepath{fill}%
\end{pgfscope}%
\begin{pgfscope}%
\pgfpathrectangle{\pgfqpoint{6.572727in}{0.474100in}}{\pgfqpoint{4.227273in}{3.318700in}}%
\pgfusepath{clip}%
\pgfsetbuttcap%
\pgfsetroundjoin%
\definecolor{currentfill}{rgb}{0.993248,0.906157,0.143936}%
\pgfsetfillcolor{currentfill}%
\pgfsetfillopacity{0.700000}%
\pgfsetlinewidth{0.000000pt}%
\definecolor{currentstroke}{rgb}{0.000000,0.000000,0.000000}%
\pgfsetstrokecolor{currentstroke}%
\pgfsetstrokeopacity{0.700000}%
\pgfsetdash{}{0pt}%
\pgfpathmoveto{\pgfqpoint{8.262997in}{2.979555in}}%
\pgfpathcurveto{\pgfqpoint{8.268041in}{2.979555in}}{\pgfqpoint{8.272879in}{2.981559in}}{\pgfqpoint{8.276445in}{2.985125in}}%
\pgfpathcurveto{\pgfqpoint{8.280012in}{2.988692in}}{\pgfqpoint{8.282016in}{2.993529in}}{\pgfqpoint{8.282016in}{2.998573in}}%
\pgfpathcurveto{\pgfqpoint{8.282016in}{3.003617in}}{\pgfqpoint{8.280012in}{3.008455in}}{\pgfqpoint{8.276445in}{3.012021in}}%
\pgfpathcurveto{\pgfqpoint{8.272879in}{3.015587in}}{\pgfqpoint{8.268041in}{3.017591in}}{\pgfqpoint{8.262997in}{3.017591in}}%
\pgfpathcurveto{\pgfqpoint{8.257954in}{3.017591in}}{\pgfqpoint{8.253116in}{3.015587in}}{\pgfqpoint{8.249550in}{3.012021in}}%
\pgfpathcurveto{\pgfqpoint{8.245983in}{3.008455in}}{\pgfqpoint{8.243979in}{3.003617in}}{\pgfqpoint{8.243979in}{2.998573in}}%
\pgfpathcurveto{\pgfqpoint{8.243979in}{2.993529in}}{\pgfqpoint{8.245983in}{2.988692in}}{\pgfqpoint{8.249550in}{2.985125in}}%
\pgfpathcurveto{\pgfqpoint{8.253116in}{2.981559in}}{\pgfqpoint{8.257954in}{2.979555in}}{\pgfqpoint{8.262997in}{2.979555in}}%
\pgfpathclose%
\pgfusepath{fill}%
\end{pgfscope}%
\begin{pgfscope}%
\pgfpathrectangle{\pgfqpoint{6.572727in}{0.474100in}}{\pgfqpoint{4.227273in}{3.318700in}}%
\pgfusepath{clip}%
\pgfsetbuttcap%
\pgfsetroundjoin%
\definecolor{currentfill}{rgb}{0.127568,0.566949,0.550556}%
\pgfsetfillcolor{currentfill}%
\pgfsetfillopacity{0.700000}%
\pgfsetlinewidth{0.000000pt}%
\definecolor{currentstroke}{rgb}{0.000000,0.000000,0.000000}%
\pgfsetstrokecolor{currentstroke}%
\pgfsetstrokeopacity{0.700000}%
\pgfsetdash{}{0pt}%
\pgfpathmoveto{\pgfqpoint{9.564049in}{1.277470in}}%
\pgfpathcurveto{\pgfqpoint{9.569093in}{1.277470in}}{\pgfqpoint{9.573930in}{1.279474in}}{\pgfqpoint{9.577497in}{1.283040in}}%
\pgfpathcurveto{\pgfqpoint{9.581063in}{1.286607in}}{\pgfqpoint{9.583067in}{1.291445in}}{\pgfqpoint{9.583067in}{1.296488in}}%
\pgfpathcurveto{\pgfqpoint{9.583067in}{1.301532in}}{\pgfqpoint{9.581063in}{1.306370in}}{\pgfqpoint{9.577497in}{1.309936in}}%
\pgfpathcurveto{\pgfqpoint{9.573930in}{1.313503in}}{\pgfqpoint{9.569093in}{1.315506in}}{\pgfqpoint{9.564049in}{1.315506in}}%
\pgfpathcurveto{\pgfqpoint{9.559005in}{1.315506in}}{\pgfqpoint{9.554167in}{1.313503in}}{\pgfqpoint{9.550601in}{1.309936in}}%
\pgfpathcurveto{\pgfqpoint{9.547035in}{1.306370in}}{\pgfqpoint{9.545031in}{1.301532in}}{\pgfqpoint{9.545031in}{1.296488in}}%
\pgfpathcurveto{\pgfqpoint{9.545031in}{1.291445in}}{\pgfqpoint{9.547035in}{1.286607in}}{\pgfqpoint{9.550601in}{1.283040in}}%
\pgfpathcurveto{\pgfqpoint{9.554167in}{1.279474in}}{\pgfqpoint{9.559005in}{1.277470in}}{\pgfqpoint{9.564049in}{1.277470in}}%
\pgfpathclose%
\pgfusepath{fill}%
\end{pgfscope}%
\begin{pgfscope}%
\pgfpathrectangle{\pgfqpoint{6.572727in}{0.474100in}}{\pgfqpoint{4.227273in}{3.318700in}}%
\pgfusepath{clip}%
\pgfsetbuttcap%
\pgfsetroundjoin%
\definecolor{currentfill}{rgb}{0.127568,0.566949,0.550556}%
\pgfsetfillcolor{currentfill}%
\pgfsetfillopacity{0.700000}%
\pgfsetlinewidth{0.000000pt}%
\definecolor{currentstroke}{rgb}{0.000000,0.000000,0.000000}%
\pgfsetstrokecolor{currentstroke}%
\pgfsetstrokeopacity{0.700000}%
\pgfsetdash{}{0pt}%
\pgfpathmoveto{\pgfqpoint{9.713501in}{2.068158in}}%
\pgfpathcurveto{\pgfqpoint{9.718545in}{2.068158in}}{\pgfqpoint{9.723383in}{2.070162in}}{\pgfqpoint{9.726949in}{2.073729in}}%
\pgfpathcurveto{\pgfqpoint{9.730515in}{2.077295in}}{\pgfqpoint{9.732519in}{2.082133in}}{\pgfqpoint{9.732519in}{2.087177in}}%
\pgfpathcurveto{\pgfqpoint{9.732519in}{2.092220in}}{\pgfqpoint{9.730515in}{2.097058in}}{\pgfqpoint{9.726949in}{2.100624in}}%
\pgfpathcurveto{\pgfqpoint{9.723383in}{2.104191in}}{\pgfqpoint{9.718545in}{2.106195in}}{\pgfqpoint{9.713501in}{2.106195in}}%
\pgfpathcurveto{\pgfqpoint{9.708457in}{2.106195in}}{\pgfqpoint{9.703620in}{2.104191in}}{\pgfqpoint{9.700053in}{2.100624in}}%
\pgfpathcurveto{\pgfqpoint{9.696487in}{2.097058in}}{\pgfqpoint{9.694483in}{2.092220in}}{\pgfqpoint{9.694483in}{2.087177in}}%
\pgfpathcurveto{\pgfqpoint{9.694483in}{2.082133in}}{\pgfqpoint{9.696487in}{2.077295in}}{\pgfqpoint{9.700053in}{2.073729in}}%
\pgfpathcurveto{\pgfqpoint{9.703620in}{2.070162in}}{\pgfqpoint{9.708457in}{2.068158in}}{\pgfqpoint{9.713501in}{2.068158in}}%
\pgfpathclose%
\pgfusepath{fill}%
\end{pgfscope}%
\begin{pgfscope}%
\pgfpathrectangle{\pgfqpoint{6.572727in}{0.474100in}}{\pgfqpoint{4.227273in}{3.318700in}}%
\pgfusepath{clip}%
\pgfsetbuttcap%
\pgfsetroundjoin%
\definecolor{currentfill}{rgb}{0.267004,0.004874,0.329415}%
\pgfsetfillcolor{currentfill}%
\pgfsetfillopacity{0.700000}%
\pgfsetlinewidth{0.000000pt}%
\definecolor{currentstroke}{rgb}{0.000000,0.000000,0.000000}%
\pgfsetstrokecolor{currentstroke}%
\pgfsetstrokeopacity{0.700000}%
\pgfsetdash{}{0pt}%
\pgfpathmoveto{\pgfqpoint{7.960718in}{1.606234in}}%
\pgfpathcurveto{\pgfqpoint{7.965762in}{1.606234in}}{\pgfqpoint{7.970599in}{1.608238in}}{\pgfqpoint{7.974166in}{1.611804in}}%
\pgfpathcurveto{\pgfqpoint{7.977732in}{1.615370in}}{\pgfqpoint{7.979736in}{1.620208in}}{\pgfqpoint{7.979736in}{1.625252in}}%
\pgfpathcurveto{\pgfqpoint{7.979736in}{1.630295in}}{\pgfqpoint{7.977732in}{1.635133in}}{\pgfqpoint{7.974166in}{1.638700in}}%
\pgfpathcurveto{\pgfqpoint{7.970599in}{1.642266in}}{\pgfqpoint{7.965762in}{1.644270in}}{\pgfqpoint{7.960718in}{1.644270in}}%
\pgfpathcurveto{\pgfqpoint{7.955674in}{1.644270in}}{\pgfqpoint{7.950836in}{1.642266in}}{\pgfqpoint{7.947270in}{1.638700in}}%
\pgfpathcurveto{\pgfqpoint{7.943704in}{1.635133in}}{\pgfqpoint{7.941700in}{1.630295in}}{\pgfqpoint{7.941700in}{1.625252in}}%
\pgfpathcurveto{\pgfqpoint{7.941700in}{1.620208in}}{\pgfqpoint{7.943704in}{1.615370in}}{\pgfqpoint{7.947270in}{1.611804in}}%
\pgfpathcurveto{\pgfqpoint{7.950836in}{1.608238in}}{\pgfqpoint{7.955674in}{1.606234in}}{\pgfqpoint{7.960718in}{1.606234in}}%
\pgfpathclose%
\pgfusepath{fill}%
\end{pgfscope}%
\begin{pgfscope}%
\pgfpathrectangle{\pgfqpoint{6.572727in}{0.474100in}}{\pgfqpoint{4.227273in}{3.318700in}}%
\pgfusepath{clip}%
\pgfsetbuttcap%
\pgfsetroundjoin%
\definecolor{currentfill}{rgb}{0.127568,0.566949,0.550556}%
\pgfsetfillcolor{currentfill}%
\pgfsetfillopacity{0.700000}%
\pgfsetlinewidth{0.000000pt}%
\definecolor{currentstroke}{rgb}{0.000000,0.000000,0.000000}%
\pgfsetstrokecolor{currentstroke}%
\pgfsetstrokeopacity{0.700000}%
\pgfsetdash{}{0pt}%
\pgfpathmoveto{\pgfqpoint{10.033390in}{1.032024in}}%
\pgfpathcurveto{\pgfqpoint{10.038434in}{1.032024in}}{\pgfqpoint{10.043272in}{1.034027in}}{\pgfqpoint{10.046838in}{1.037594in}}%
\pgfpathcurveto{\pgfqpoint{10.050404in}{1.041160in}}{\pgfqpoint{10.052408in}{1.045998in}}{\pgfqpoint{10.052408in}{1.051042in}}%
\pgfpathcurveto{\pgfqpoint{10.052408in}{1.056085in}}{\pgfqpoint{10.050404in}{1.060923in}}{\pgfqpoint{10.046838in}{1.064490in}}%
\pgfpathcurveto{\pgfqpoint{10.043272in}{1.068056in}}{\pgfqpoint{10.038434in}{1.070060in}}{\pgfqpoint{10.033390in}{1.070060in}}%
\pgfpathcurveto{\pgfqpoint{10.028346in}{1.070060in}}{\pgfqpoint{10.023509in}{1.068056in}}{\pgfqpoint{10.019942in}{1.064490in}}%
\pgfpathcurveto{\pgfqpoint{10.016376in}{1.060923in}}{\pgfqpoint{10.014372in}{1.056085in}}{\pgfqpoint{10.014372in}{1.051042in}}%
\pgfpathcurveto{\pgfqpoint{10.014372in}{1.045998in}}{\pgfqpoint{10.016376in}{1.041160in}}{\pgfqpoint{10.019942in}{1.037594in}}%
\pgfpathcurveto{\pgfqpoint{10.023509in}{1.034027in}}{\pgfqpoint{10.028346in}{1.032024in}}{\pgfqpoint{10.033390in}{1.032024in}}%
\pgfpathclose%
\pgfusepath{fill}%
\end{pgfscope}%
\begin{pgfscope}%
\pgfpathrectangle{\pgfqpoint{6.572727in}{0.474100in}}{\pgfqpoint{4.227273in}{3.318700in}}%
\pgfusepath{clip}%
\pgfsetbuttcap%
\pgfsetroundjoin%
\definecolor{currentfill}{rgb}{0.267004,0.004874,0.329415}%
\pgfsetfillcolor{currentfill}%
\pgfsetfillopacity{0.700000}%
\pgfsetlinewidth{0.000000pt}%
\definecolor{currentstroke}{rgb}{0.000000,0.000000,0.000000}%
\pgfsetstrokecolor{currentstroke}%
\pgfsetstrokeopacity{0.700000}%
\pgfsetdash{}{0pt}%
\pgfpathmoveto{\pgfqpoint{7.636301in}{1.799359in}}%
\pgfpathcurveto{\pgfqpoint{7.641344in}{1.799359in}}{\pgfqpoint{7.646182in}{1.801363in}}{\pgfqpoint{7.649749in}{1.804929in}}%
\pgfpathcurveto{\pgfqpoint{7.653315in}{1.808495in}}{\pgfqpoint{7.655319in}{1.813333in}}{\pgfqpoint{7.655319in}{1.818377in}}%
\pgfpathcurveto{\pgfqpoint{7.655319in}{1.823420in}}{\pgfqpoint{7.653315in}{1.828258in}}{\pgfqpoint{7.649749in}{1.831825in}}%
\pgfpathcurveto{\pgfqpoint{7.646182in}{1.835391in}}{\pgfqpoint{7.641344in}{1.837395in}}{\pgfqpoint{7.636301in}{1.837395in}}%
\pgfpathcurveto{\pgfqpoint{7.631257in}{1.837395in}}{\pgfqpoint{7.626419in}{1.835391in}}{\pgfqpoint{7.622853in}{1.831825in}}%
\pgfpathcurveto{\pgfqpoint{7.619286in}{1.828258in}}{\pgfqpoint{7.617283in}{1.823420in}}{\pgfqpoint{7.617283in}{1.818377in}}%
\pgfpathcurveto{\pgfqpoint{7.617283in}{1.813333in}}{\pgfqpoint{7.619286in}{1.808495in}}{\pgfqpoint{7.622853in}{1.804929in}}%
\pgfpathcurveto{\pgfqpoint{7.626419in}{1.801363in}}{\pgfqpoint{7.631257in}{1.799359in}}{\pgfqpoint{7.636301in}{1.799359in}}%
\pgfpathclose%
\pgfusepath{fill}%
\end{pgfscope}%
\begin{pgfscope}%
\pgfpathrectangle{\pgfqpoint{6.572727in}{0.474100in}}{\pgfqpoint{4.227273in}{3.318700in}}%
\pgfusepath{clip}%
\pgfsetbuttcap%
\pgfsetroundjoin%
\definecolor{currentfill}{rgb}{0.993248,0.906157,0.143936}%
\pgfsetfillcolor{currentfill}%
\pgfsetfillopacity{0.700000}%
\pgfsetlinewidth{0.000000pt}%
\definecolor{currentstroke}{rgb}{0.000000,0.000000,0.000000}%
\pgfsetstrokecolor{currentstroke}%
\pgfsetstrokeopacity{0.700000}%
\pgfsetdash{}{0pt}%
\pgfpathmoveto{\pgfqpoint{8.489685in}{2.659460in}}%
\pgfpathcurveto{\pgfqpoint{8.494729in}{2.659460in}}{\pgfqpoint{8.499567in}{2.661464in}}{\pgfqpoint{8.503133in}{2.665030in}}%
\pgfpathcurveto{\pgfqpoint{8.506700in}{2.668597in}}{\pgfqpoint{8.508703in}{2.673434in}}{\pgfqpoint{8.508703in}{2.678478in}}%
\pgfpathcurveto{\pgfqpoint{8.508703in}{2.683522in}}{\pgfqpoint{8.506700in}{2.688360in}}{\pgfqpoint{8.503133in}{2.691926in}}%
\pgfpathcurveto{\pgfqpoint{8.499567in}{2.695492in}}{\pgfqpoint{8.494729in}{2.697496in}}{\pgfqpoint{8.489685in}{2.697496in}}%
\pgfpathcurveto{\pgfqpoint{8.484642in}{2.697496in}}{\pgfqpoint{8.479804in}{2.695492in}}{\pgfqpoint{8.476237in}{2.691926in}}%
\pgfpathcurveto{\pgfqpoint{8.472671in}{2.688360in}}{\pgfqpoint{8.470667in}{2.683522in}}{\pgfqpoint{8.470667in}{2.678478in}}%
\pgfpathcurveto{\pgfqpoint{8.470667in}{2.673434in}}{\pgfqpoint{8.472671in}{2.668597in}}{\pgfqpoint{8.476237in}{2.665030in}}%
\pgfpathcurveto{\pgfqpoint{8.479804in}{2.661464in}}{\pgfqpoint{8.484642in}{2.659460in}}{\pgfqpoint{8.489685in}{2.659460in}}%
\pgfpathclose%
\pgfusepath{fill}%
\end{pgfscope}%
\begin{pgfscope}%
\pgfpathrectangle{\pgfqpoint{6.572727in}{0.474100in}}{\pgfqpoint{4.227273in}{3.318700in}}%
\pgfusepath{clip}%
\pgfsetbuttcap%
\pgfsetroundjoin%
\definecolor{currentfill}{rgb}{0.267004,0.004874,0.329415}%
\pgfsetfillcolor{currentfill}%
\pgfsetfillopacity{0.700000}%
\pgfsetlinewidth{0.000000pt}%
\definecolor{currentstroke}{rgb}{0.000000,0.000000,0.000000}%
\pgfsetstrokecolor{currentstroke}%
\pgfsetstrokeopacity{0.700000}%
\pgfsetdash{}{0pt}%
\pgfpathmoveto{\pgfqpoint{8.525579in}{1.453585in}}%
\pgfpathcurveto{\pgfqpoint{8.530622in}{1.453585in}}{\pgfqpoint{8.535460in}{1.455588in}}{\pgfqpoint{8.539027in}{1.459155in}}%
\pgfpathcurveto{\pgfqpoint{8.542593in}{1.462721in}}{\pgfqpoint{8.544597in}{1.467559in}}{\pgfqpoint{8.544597in}{1.472603in}}%
\pgfpathcurveto{\pgfqpoint{8.544597in}{1.477646in}}{\pgfqpoint{8.542593in}{1.482484in}}{\pgfqpoint{8.539027in}{1.486051in}}%
\pgfpathcurveto{\pgfqpoint{8.535460in}{1.489617in}}{\pgfqpoint{8.530622in}{1.491621in}}{\pgfqpoint{8.525579in}{1.491621in}}%
\pgfpathcurveto{\pgfqpoint{8.520535in}{1.491621in}}{\pgfqpoint{8.515697in}{1.489617in}}{\pgfqpoint{8.512131in}{1.486051in}}%
\pgfpathcurveto{\pgfqpoint{8.508564in}{1.482484in}}{\pgfqpoint{8.506561in}{1.477646in}}{\pgfqpoint{8.506561in}{1.472603in}}%
\pgfpathcurveto{\pgfqpoint{8.506561in}{1.467559in}}{\pgfqpoint{8.508564in}{1.462721in}}{\pgfqpoint{8.512131in}{1.459155in}}%
\pgfpathcurveto{\pgfqpoint{8.515697in}{1.455588in}}{\pgfqpoint{8.520535in}{1.453585in}}{\pgfqpoint{8.525579in}{1.453585in}}%
\pgfpathclose%
\pgfusepath{fill}%
\end{pgfscope}%
\begin{pgfscope}%
\pgfpathrectangle{\pgfqpoint{6.572727in}{0.474100in}}{\pgfqpoint{4.227273in}{3.318700in}}%
\pgfusepath{clip}%
\pgfsetbuttcap%
\pgfsetroundjoin%
\definecolor{currentfill}{rgb}{0.993248,0.906157,0.143936}%
\pgfsetfillcolor{currentfill}%
\pgfsetfillopacity{0.700000}%
\pgfsetlinewidth{0.000000pt}%
\definecolor{currentstroke}{rgb}{0.000000,0.000000,0.000000}%
\pgfsetstrokecolor{currentstroke}%
\pgfsetstrokeopacity{0.700000}%
\pgfsetdash{}{0pt}%
\pgfpathmoveto{\pgfqpoint{7.617807in}{3.468116in}}%
\pgfpathcurveto{\pgfqpoint{7.622850in}{3.468116in}}{\pgfqpoint{7.627688in}{3.470119in}}{\pgfqpoint{7.631254in}{3.473686in}}%
\pgfpathcurveto{\pgfqpoint{7.634821in}{3.477252in}}{\pgfqpoint{7.636825in}{3.482090in}}{\pgfqpoint{7.636825in}{3.487134in}}%
\pgfpathcurveto{\pgfqpoint{7.636825in}{3.492177in}}{\pgfqpoint{7.634821in}{3.497015in}}{\pgfqpoint{7.631254in}{3.500582in}}%
\pgfpathcurveto{\pgfqpoint{7.627688in}{3.504148in}}{\pgfqpoint{7.622850in}{3.506152in}}{\pgfqpoint{7.617807in}{3.506152in}}%
\pgfpathcurveto{\pgfqpoint{7.612763in}{3.506152in}}{\pgfqpoint{7.607925in}{3.504148in}}{\pgfqpoint{7.604359in}{3.500582in}}%
\pgfpathcurveto{\pgfqpoint{7.600792in}{3.497015in}}{\pgfqpoint{7.598788in}{3.492177in}}{\pgfqpoint{7.598788in}{3.487134in}}%
\pgfpathcurveto{\pgfqpoint{7.598788in}{3.482090in}}{\pgfqpoint{7.600792in}{3.477252in}}{\pgfqpoint{7.604359in}{3.473686in}}%
\pgfpathcurveto{\pgfqpoint{7.607925in}{3.470119in}}{\pgfqpoint{7.612763in}{3.468116in}}{\pgfqpoint{7.617807in}{3.468116in}}%
\pgfpathclose%
\pgfusepath{fill}%
\end{pgfscope}%
\begin{pgfscope}%
\pgfpathrectangle{\pgfqpoint{6.572727in}{0.474100in}}{\pgfqpoint{4.227273in}{3.318700in}}%
\pgfusepath{clip}%
\pgfsetbuttcap%
\pgfsetroundjoin%
\definecolor{currentfill}{rgb}{0.993248,0.906157,0.143936}%
\pgfsetfillcolor{currentfill}%
\pgfsetfillopacity{0.700000}%
\pgfsetlinewidth{0.000000pt}%
\definecolor{currentstroke}{rgb}{0.000000,0.000000,0.000000}%
\pgfsetstrokecolor{currentstroke}%
\pgfsetstrokeopacity{0.700000}%
\pgfsetdash{}{0pt}%
\pgfpathmoveto{\pgfqpoint{8.793839in}{2.745158in}}%
\pgfpathcurveto{\pgfqpoint{8.798882in}{2.745158in}}{\pgfqpoint{8.803720in}{2.747162in}}{\pgfqpoint{8.807287in}{2.750729in}}%
\pgfpathcurveto{\pgfqpoint{8.810853in}{2.754295in}}{\pgfqpoint{8.812857in}{2.759133in}}{\pgfqpoint{8.812857in}{2.764177in}}%
\pgfpathcurveto{\pgfqpoint{8.812857in}{2.769220in}}{\pgfqpoint{8.810853in}{2.774058in}}{\pgfqpoint{8.807287in}{2.777624in}}%
\pgfpathcurveto{\pgfqpoint{8.803720in}{2.781191in}}{\pgfqpoint{8.798882in}{2.783195in}}{\pgfqpoint{8.793839in}{2.783195in}}%
\pgfpathcurveto{\pgfqpoint{8.788795in}{2.783195in}}{\pgfqpoint{8.783957in}{2.781191in}}{\pgfqpoint{8.780391in}{2.777624in}}%
\pgfpathcurveto{\pgfqpoint{8.776824in}{2.774058in}}{\pgfqpoint{8.774821in}{2.769220in}}{\pgfqpoint{8.774821in}{2.764177in}}%
\pgfpathcurveto{\pgfqpoint{8.774821in}{2.759133in}}{\pgfqpoint{8.776824in}{2.754295in}}{\pgfqpoint{8.780391in}{2.750729in}}%
\pgfpathcurveto{\pgfqpoint{8.783957in}{2.747162in}}{\pgfqpoint{8.788795in}{2.745158in}}{\pgfqpoint{8.793839in}{2.745158in}}%
\pgfpathclose%
\pgfusepath{fill}%
\end{pgfscope}%
\begin{pgfscope}%
\pgfpathrectangle{\pgfqpoint{6.572727in}{0.474100in}}{\pgfqpoint{4.227273in}{3.318700in}}%
\pgfusepath{clip}%
\pgfsetbuttcap%
\pgfsetroundjoin%
\definecolor{currentfill}{rgb}{0.993248,0.906157,0.143936}%
\pgfsetfillcolor{currentfill}%
\pgfsetfillopacity{0.700000}%
\pgfsetlinewidth{0.000000pt}%
\definecolor{currentstroke}{rgb}{0.000000,0.000000,0.000000}%
\pgfsetstrokecolor{currentstroke}%
\pgfsetstrokeopacity{0.700000}%
\pgfsetdash{}{0pt}%
\pgfpathmoveto{\pgfqpoint{7.728160in}{2.616089in}}%
\pgfpathcurveto{\pgfqpoint{7.733204in}{2.616089in}}{\pgfqpoint{7.738042in}{2.618093in}}{\pgfqpoint{7.741608in}{2.621659in}}%
\pgfpathcurveto{\pgfqpoint{7.745175in}{2.625225in}}{\pgfqpoint{7.747179in}{2.630063in}}{\pgfqpoint{7.747179in}{2.635107in}}%
\pgfpathcurveto{\pgfqpoint{7.747179in}{2.640150in}}{\pgfqpoint{7.745175in}{2.644988in}}{\pgfqpoint{7.741608in}{2.648555in}}%
\pgfpathcurveto{\pgfqpoint{7.738042in}{2.652121in}}{\pgfqpoint{7.733204in}{2.654125in}}{\pgfqpoint{7.728160in}{2.654125in}}%
\pgfpathcurveto{\pgfqpoint{7.723117in}{2.654125in}}{\pgfqpoint{7.718279in}{2.652121in}}{\pgfqpoint{7.714713in}{2.648555in}}%
\pgfpathcurveto{\pgfqpoint{7.711146in}{2.644988in}}{\pgfqpoint{7.709142in}{2.640150in}}{\pgfqpoint{7.709142in}{2.635107in}}%
\pgfpathcurveto{\pgfqpoint{7.709142in}{2.630063in}}{\pgfqpoint{7.711146in}{2.625225in}}{\pgfqpoint{7.714713in}{2.621659in}}%
\pgfpathcurveto{\pgfqpoint{7.718279in}{2.618093in}}{\pgfqpoint{7.723117in}{2.616089in}}{\pgfqpoint{7.728160in}{2.616089in}}%
\pgfpathclose%
\pgfusepath{fill}%
\end{pgfscope}%
\begin{pgfscope}%
\pgfpathrectangle{\pgfqpoint{6.572727in}{0.474100in}}{\pgfqpoint{4.227273in}{3.318700in}}%
\pgfusepath{clip}%
\pgfsetbuttcap%
\pgfsetroundjoin%
\definecolor{currentfill}{rgb}{0.127568,0.566949,0.550556}%
\pgfsetfillcolor{currentfill}%
\pgfsetfillopacity{0.700000}%
\pgfsetlinewidth{0.000000pt}%
\definecolor{currentstroke}{rgb}{0.000000,0.000000,0.000000}%
\pgfsetstrokecolor{currentstroke}%
\pgfsetstrokeopacity{0.700000}%
\pgfsetdash{}{0pt}%
\pgfpathmoveto{\pgfqpoint{9.415018in}{1.716083in}}%
\pgfpathcurveto{\pgfqpoint{9.420062in}{1.716083in}}{\pgfqpoint{9.424900in}{1.718087in}}{\pgfqpoint{9.428466in}{1.721653in}}%
\pgfpathcurveto{\pgfqpoint{9.432033in}{1.725220in}}{\pgfqpoint{9.434037in}{1.730057in}}{\pgfqpoint{9.434037in}{1.735101in}}%
\pgfpathcurveto{\pgfqpoint{9.434037in}{1.740145in}}{\pgfqpoint{9.432033in}{1.744982in}}{\pgfqpoint{9.428466in}{1.748549in}}%
\pgfpathcurveto{\pgfqpoint{9.424900in}{1.752115in}}{\pgfqpoint{9.420062in}{1.754119in}}{\pgfqpoint{9.415018in}{1.754119in}}%
\pgfpathcurveto{\pgfqpoint{9.409975in}{1.754119in}}{\pgfqpoint{9.405137in}{1.752115in}}{\pgfqpoint{9.401571in}{1.748549in}}%
\pgfpathcurveto{\pgfqpoint{9.398004in}{1.744982in}}{\pgfqpoint{9.396000in}{1.740145in}}{\pgfqpoint{9.396000in}{1.735101in}}%
\pgfpathcurveto{\pgfqpoint{9.396000in}{1.730057in}}{\pgfqpoint{9.398004in}{1.725220in}}{\pgfqpoint{9.401571in}{1.721653in}}%
\pgfpathcurveto{\pgfqpoint{9.405137in}{1.718087in}}{\pgfqpoint{9.409975in}{1.716083in}}{\pgfqpoint{9.415018in}{1.716083in}}%
\pgfpathclose%
\pgfusepath{fill}%
\end{pgfscope}%
\begin{pgfscope}%
\pgfpathrectangle{\pgfqpoint{6.572727in}{0.474100in}}{\pgfqpoint{4.227273in}{3.318700in}}%
\pgfusepath{clip}%
\pgfsetbuttcap%
\pgfsetroundjoin%
\definecolor{currentfill}{rgb}{0.127568,0.566949,0.550556}%
\pgfsetfillcolor{currentfill}%
\pgfsetfillopacity{0.700000}%
\pgfsetlinewidth{0.000000pt}%
\definecolor{currentstroke}{rgb}{0.000000,0.000000,0.000000}%
\pgfsetstrokecolor{currentstroke}%
\pgfsetstrokeopacity{0.700000}%
\pgfsetdash{}{0pt}%
\pgfpathmoveto{\pgfqpoint{9.629667in}{1.910004in}}%
\pgfpathcurveto{\pgfqpoint{9.634711in}{1.910004in}}{\pgfqpoint{9.639548in}{1.912008in}}{\pgfqpoint{9.643115in}{1.915574in}}%
\pgfpathcurveto{\pgfqpoint{9.646681in}{1.919141in}}{\pgfqpoint{9.648685in}{1.923979in}}{\pgfqpoint{9.648685in}{1.929022in}}%
\pgfpathcurveto{\pgfqpoint{9.648685in}{1.934066in}}{\pgfqpoint{9.646681in}{1.938904in}}{\pgfqpoint{9.643115in}{1.942470in}}%
\pgfpathcurveto{\pgfqpoint{9.639548in}{1.946037in}}{\pgfqpoint{9.634711in}{1.948040in}}{\pgfqpoint{9.629667in}{1.948040in}}%
\pgfpathcurveto{\pgfqpoint{9.624623in}{1.948040in}}{\pgfqpoint{9.619786in}{1.946037in}}{\pgfqpoint{9.616219in}{1.942470in}}%
\pgfpathcurveto{\pgfqpoint{9.612653in}{1.938904in}}{\pgfqpoint{9.610649in}{1.934066in}}{\pgfqpoint{9.610649in}{1.929022in}}%
\pgfpathcurveto{\pgfqpoint{9.610649in}{1.923979in}}{\pgfqpoint{9.612653in}{1.919141in}}{\pgfqpoint{9.616219in}{1.915574in}}%
\pgfpathcurveto{\pgfqpoint{9.619786in}{1.912008in}}{\pgfqpoint{9.624623in}{1.910004in}}{\pgfqpoint{9.629667in}{1.910004in}}%
\pgfpathclose%
\pgfusepath{fill}%
\end{pgfscope}%
\begin{pgfscope}%
\pgfpathrectangle{\pgfqpoint{6.572727in}{0.474100in}}{\pgfqpoint{4.227273in}{3.318700in}}%
\pgfusepath{clip}%
\pgfsetbuttcap%
\pgfsetroundjoin%
\definecolor{currentfill}{rgb}{0.127568,0.566949,0.550556}%
\pgfsetfillcolor{currentfill}%
\pgfsetfillopacity{0.700000}%
\pgfsetlinewidth{0.000000pt}%
\definecolor{currentstroke}{rgb}{0.000000,0.000000,0.000000}%
\pgfsetstrokecolor{currentstroke}%
\pgfsetstrokeopacity{0.700000}%
\pgfsetdash{}{0pt}%
\pgfpathmoveto{\pgfqpoint{8.727090in}{1.930325in}}%
\pgfpathcurveto{\pgfqpoint{8.732134in}{1.930325in}}{\pgfqpoint{8.736972in}{1.932329in}}{\pgfqpoint{8.740538in}{1.935896in}}%
\pgfpathcurveto{\pgfqpoint{8.744105in}{1.939462in}}{\pgfqpoint{8.746108in}{1.944300in}}{\pgfqpoint{8.746108in}{1.949343in}}%
\pgfpathcurveto{\pgfqpoint{8.746108in}{1.954387in}}{\pgfqpoint{8.744105in}{1.959225in}}{\pgfqpoint{8.740538in}{1.962791in}}%
\pgfpathcurveto{\pgfqpoint{8.736972in}{1.966358in}}{\pgfqpoint{8.732134in}{1.968362in}}{\pgfqpoint{8.727090in}{1.968362in}}%
\pgfpathcurveto{\pgfqpoint{8.722047in}{1.968362in}}{\pgfqpoint{8.717209in}{1.966358in}}{\pgfqpoint{8.713642in}{1.962791in}}%
\pgfpathcurveto{\pgfqpoint{8.710076in}{1.959225in}}{\pgfqpoint{8.708072in}{1.954387in}}{\pgfqpoint{8.708072in}{1.949343in}}%
\pgfpathcurveto{\pgfqpoint{8.708072in}{1.944300in}}{\pgfqpoint{8.710076in}{1.939462in}}{\pgfqpoint{8.713642in}{1.935896in}}%
\pgfpathcurveto{\pgfqpoint{8.717209in}{1.932329in}}{\pgfqpoint{8.722047in}{1.930325in}}{\pgfqpoint{8.727090in}{1.930325in}}%
\pgfpathclose%
\pgfusepath{fill}%
\end{pgfscope}%
\begin{pgfscope}%
\pgfpathrectangle{\pgfqpoint{6.572727in}{0.474100in}}{\pgfqpoint{4.227273in}{3.318700in}}%
\pgfusepath{clip}%
\pgfsetbuttcap%
\pgfsetroundjoin%
\definecolor{currentfill}{rgb}{0.267004,0.004874,0.329415}%
\pgfsetfillcolor{currentfill}%
\pgfsetfillopacity{0.700000}%
\pgfsetlinewidth{0.000000pt}%
\definecolor{currentstroke}{rgb}{0.000000,0.000000,0.000000}%
\pgfsetstrokecolor{currentstroke}%
\pgfsetstrokeopacity{0.700000}%
\pgfsetdash{}{0pt}%
\pgfpathmoveto{\pgfqpoint{7.412554in}{1.354471in}}%
\pgfpathcurveto{\pgfqpoint{7.417598in}{1.354471in}}{\pgfqpoint{7.422436in}{1.356475in}}{\pgfqpoint{7.426002in}{1.360042in}}%
\pgfpathcurveto{\pgfqpoint{7.429568in}{1.363608in}}{\pgfqpoint{7.431572in}{1.368446in}}{\pgfqpoint{7.431572in}{1.373489in}}%
\pgfpathcurveto{\pgfqpoint{7.431572in}{1.378533in}}{\pgfqpoint{7.429568in}{1.383371in}}{\pgfqpoint{7.426002in}{1.386937in}}%
\pgfpathcurveto{\pgfqpoint{7.422436in}{1.390504in}}{\pgfqpoint{7.417598in}{1.392508in}}{\pgfqpoint{7.412554in}{1.392508in}}%
\pgfpathcurveto{\pgfqpoint{7.407510in}{1.392508in}}{\pgfqpoint{7.402673in}{1.390504in}}{\pgfqpoint{7.399106in}{1.386937in}}%
\pgfpathcurveto{\pgfqpoint{7.395540in}{1.383371in}}{\pgfqpoint{7.393536in}{1.378533in}}{\pgfqpoint{7.393536in}{1.373489in}}%
\pgfpathcurveto{\pgfqpoint{7.393536in}{1.368446in}}{\pgfqpoint{7.395540in}{1.363608in}}{\pgfqpoint{7.399106in}{1.360042in}}%
\pgfpathcurveto{\pgfqpoint{7.402673in}{1.356475in}}{\pgfqpoint{7.407510in}{1.354471in}}{\pgfqpoint{7.412554in}{1.354471in}}%
\pgfpathclose%
\pgfusepath{fill}%
\end{pgfscope}%
\begin{pgfscope}%
\pgfpathrectangle{\pgfqpoint{6.572727in}{0.474100in}}{\pgfqpoint{4.227273in}{3.318700in}}%
\pgfusepath{clip}%
\pgfsetbuttcap%
\pgfsetroundjoin%
\definecolor{currentfill}{rgb}{0.993248,0.906157,0.143936}%
\pgfsetfillcolor{currentfill}%
\pgfsetfillopacity{0.700000}%
\pgfsetlinewidth{0.000000pt}%
\definecolor{currentstroke}{rgb}{0.000000,0.000000,0.000000}%
\pgfsetstrokecolor{currentstroke}%
\pgfsetstrokeopacity{0.700000}%
\pgfsetdash{}{0pt}%
\pgfpathmoveto{\pgfqpoint{7.819516in}{2.709233in}}%
\pgfpathcurveto{\pgfqpoint{7.824560in}{2.709233in}}{\pgfqpoint{7.829398in}{2.711237in}}{\pgfqpoint{7.832964in}{2.714803in}}%
\pgfpathcurveto{\pgfqpoint{7.836531in}{2.718369in}}{\pgfqpoint{7.838534in}{2.723207in}}{\pgfqpoint{7.838534in}{2.728251in}}%
\pgfpathcurveto{\pgfqpoint{7.838534in}{2.733295in}}{\pgfqpoint{7.836531in}{2.738132in}}{\pgfqpoint{7.832964in}{2.741699in}}%
\pgfpathcurveto{\pgfqpoint{7.829398in}{2.745265in}}{\pgfqpoint{7.824560in}{2.747269in}}{\pgfqpoint{7.819516in}{2.747269in}}%
\pgfpathcurveto{\pgfqpoint{7.814473in}{2.747269in}}{\pgfqpoint{7.809635in}{2.745265in}}{\pgfqpoint{7.806068in}{2.741699in}}%
\pgfpathcurveto{\pgfqpoint{7.802502in}{2.738132in}}{\pgfqpoint{7.800498in}{2.733295in}}{\pgfqpoint{7.800498in}{2.728251in}}%
\pgfpathcurveto{\pgfqpoint{7.800498in}{2.723207in}}{\pgfqpoint{7.802502in}{2.718369in}}{\pgfqpoint{7.806068in}{2.714803in}}%
\pgfpathcurveto{\pgfqpoint{7.809635in}{2.711237in}}{\pgfqpoint{7.814473in}{2.709233in}}{\pgfqpoint{7.819516in}{2.709233in}}%
\pgfpathclose%
\pgfusepath{fill}%
\end{pgfscope}%
\begin{pgfscope}%
\pgfpathrectangle{\pgfqpoint{6.572727in}{0.474100in}}{\pgfqpoint{4.227273in}{3.318700in}}%
\pgfusepath{clip}%
\pgfsetbuttcap%
\pgfsetroundjoin%
\definecolor{currentfill}{rgb}{0.127568,0.566949,0.550556}%
\pgfsetfillcolor{currentfill}%
\pgfsetfillopacity{0.700000}%
\pgfsetlinewidth{0.000000pt}%
\definecolor{currentstroke}{rgb}{0.000000,0.000000,0.000000}%
\pgfsetstrokecolor{currentstroke}%
\pgfsetstrokeopacity{0.700000}%
\pgfsetdash{}{0pt}%
\pgfpathmoveto{\pgfqpoint{9.634047in}{1.544085in}}%
\pgfpathcurveto{\pgfqpoint{9.639091in}{1.544085in}}{\pgfqpoint{9.643928in}{1.546089in}}{\pgfqpoint{9.647495in}{1.549655in}}%
\pgfpathcurveto{\pgfqpoint{9.651061in}{1.553222in}}{\pgfqpoint{9.653065in}{1.558060in}}{\pgfqpoint{9.653065in}{1.563103in}}%
\pgfpathcurveto{\pgfqpoint{9.653065in}{1.568147in}}{\pgfqpoint{9.651061in}{1.572985in}}{\pgfqpoint{9.647495in}{1.576551in}}%
\pgfpathcurveto{\pgfqpoint{9.643928in}{1.580118in}}{\pgfqpoint{9.639091in}{1.582121in}}{\pgfqpoint{9.634047in}{1.582121in}}%
\pgfpathcurveto{\pgfqpoint{9.629003in}{1.582121in}}{\pgfqpoint{9.624165in}{1.580118in}}{\pgfqpoint{9.620599in}{1.576551in}}%
\pgfpathcurveto{\pgfqpoint{9.617033in}{1.572985in}}{\pgfqpoint{9.615029in}{1.568147in}}{\pgfqpoint{9.615029in}{1.563103in}}%
\pgfpathcurveto{\pgfqpoint{9.615029in}{1.558060in}}{\pgfqpoint{9.617033in}{1.553222in}}{\pgfqpoint{9.620599in}{1.549655in}}%
\pgfpathcurveto{\pgfqpoint{9.624165in}{1.546089in}}{\pgfqpoint{9.629003in}{1.544085in}}{\pgfqpoint{9.634047in}{1.544085in}}%
\pgfpathclose%
\pgfusepath{fill}%
\end{pgfscope}%
\begin{pgfscope}%
\pgfpathrectangle{\pgfqpoint{6.572727in}{0.474100in}}{\pgfqpoint{4.227273in}{3.318700in}}%
\pgfusepath{clip}%
\pgfsetbuttcap%
\pgfsetroundjoin%
\definecolor{currentfill}{rgb}{0.267004,0.004874,0.329415}%
\pgfsetfillcolor{currentfill}%
\pgfsetfillopacity{0.700000}%
\pgfsetlinewidth{0.000000pt}%
\definecolor{currentstroke}{rgb}{0.000000,0.000000,0.000000}%
\pgfsetstrokecolor{currentstroke}%
\pgfsetstrokeopacity{0.700000}%
\pgfsetdash{}{0pt}%
\pgfpathmoveto{\pgfqpoint{7.652595in}{1.913445in}}%
\pgfpathcurveto{\pgfqpoint{7.657639in}{1.913445in}}{\pgfqpoint{7.662477in}{1.915449in}}{\pgfqpoint{7.666043in}{1.919015in}}%
\pgfpathcurveto{\pgfqpoint{7.669610in}{1.922581in}}{\pgfqpoint{7.671613in}{1.927419in}}{\pgfqpoint{7.671613in}{1.932463in}}%
\pgfpathcurveto{\pgfqpoint{7.671613in}{1.937507in}}{\pgfqpoint{7.669610in}{1.942344in}}{\pgfqpoint{7.666043in}{1.945911in}}%
\pgfpathcurveto{\pgfqpoint{7.662477in}{1.949477in}}{\pgfqpoint{7.657639in}{1.951481in}}{\pgfqpoint{7.652595in}{1.951481in}}%
\pgfpathcurveto{\pgfqpoint{7.647552in}{1.951481in}}{\pgfqpoint{7.642714in}{1.949477in}}{\pgfqpoint{7.639147in}{1.945911in}}%
\pgfpathcurveto{\pgfqpoint{7.635581in}{1.942344in}}{\pgfqpoint{7.633577in}{1.937507in}}{\pgfqpoint{7.633577in}{1.932463in}}%
\pgfpathcurveto{\pgfqpoint{7.633577in}{1.927419in}}{\pgfqpoint{7.635581in}{1.922581in}}{\pgfqpoint{7.639147in}{1.919015in}}%
\pgfpathcurveto{\pgfqpoint{7.642714in}{1.915449in}}{\pgfqpoint{7.647552in}{1.913445in}}{\pgfqpoint{7.652595in}{1.913445in}}%
\pgfpathclose%
\pgfusepath{fill}%
\end{pgfscope}%
\begin{pgfscope}%
\pgfpathrectangle{\pgfqpoint{6.572727in}{0.474100in}}{\pgfqpoint{4.227273in}{3.318700in}}%
\pgfusepath{clip}%
\pgfsetbuttcap%
\pgfsetroundjoin%
\definecolor{currentfill}{rgb}{0.267004,0.004874,0.329415}%
\pgfsetfillcolor{currentfill}%
\pgfsetfillopacity{0.700000}%
\pgfsetlinewidth{0.000000pt}%
\definecolor{currentstroke}{rgb}{0.000000,0.000000,0.000000}%
\pgfsetstrokecolor{currentstroke}%
\pgfsetstrokeopacity{0.700000}%
\pgfsetdash{}{0pt}%
\pgfpathmoveto{\pgfqpoint{7.366921in}{1.149224in}}%
\pgfpathcurveto{\pgfqpoint{7.371964in}{1.149224in}}{\pgfqpoint{7.376802in}{1.151228in}}{\pgfqpoint{7.380369in}{1.154794in}}%
\pgfpathcurveto{\pgfqpoint{7.383935in}{1.158360in}}{\pgfqpoint{7.385939in}{1.163198in}}{\pgfqpoint{7.385939in}{1.168242in}}%
\pgfpathcurveto{\pgfqpoint{7.385939in}{1.173286in}}{\pgfqpoint{7.383935in}{1.178123in}}{\pgfqpoint{7.380369in}{1.181690in}}%
\pgfpathcurveto{\pgfqpoint{7.376802in}{1.185256in}}{\pgfqpoint{7.371964in}{1.187260in}}{\pgfqpoint{7.366921in}{1.187260in}}%
\pgfpathcurveto{\pgfqpoint{7.361877in}{1.187260in}}{\pgfqpoint{7.357039in}{1.185256in}}{\pgfqpoint{7.353473in}{1.181690in}}%
\pgfpathcurveto{\pgfqpoint{7.349907in}{1.178123in}}{\pgfqpoint{7.347903in}{1.173286in}}{\pgfqpoint{7.347903in}{1.168242in}}%
\pgfpathcurveto{\pgfqpoint{7.347903in}{1.163198in}}{\pgfqpoint{7.349907in}{1.158360in}}{\pgfqpoint{7.353473in}{1.154794in}}%
\pgfpathcurveto{\pgfqpoint{7.357039in}{1.151228in}}{\pgfqpoint{7.361877in}{1.149224in}}{\pgfqpoint{7.366921in}{1.149224in}}%
\pgfpathclose%
\pgfusepath{fill}%
\end{pgfscope}%
\begin{pgfscope}%
\pgfpathrectangle{\pgfqpoint{6.572727in}{0.474100in}}{\pgfqpoint{4.227273in}{3.318700in}}%
\pgfusepath{clip}%
\pgfsetbuttcap%
\pgfsetroundjoin%
\definecolor{currentfill}{rgb}{0.127568,0.566949,0.550556}%
\pgfsetfillcolor{currentfill}%
\pgfsetfillopacity{0.700000}%
\pgfsetlinewidth{0.000000pt}%
\definecolor{currentstroke}{rgb}{0.000000,0.000000,0.000000}%
\pgfsetstrokecolor{currentstroke}%
\pgfsetstrokeopacity{0.700000}%
\pgfsetdash{}{0pt}%
\pgfpathmoveto{\pgfqpoint{9.702031in}{1.746925in}}%
\pgfpathcurveto{\pgfqpoint{9.707075in}{1.746925in}}{\pgfqpoint{9.711913in}{1.748929in}}{\pgfqpoint{9.715479in}{1.752496in}}%
\pgfpathcurveto{\pgfqpoint{9.719046in}{1.756062in}}{\pgfqpoint{9.721049in}{1.760900in}}{\pgfqpoint{9.721049in}{1.765944in}}%
\pgfpathcurveto{\pgfqpoint{9.721049in}{1.770987in}}{\pgfqpoint{9.719046in}{1.775825in}}{\pgfqpoint{9.715479in}{1.779392in}}%
\pgfpathcurveto{\pgfqpoint{9.711913in}{1.782958in}}{\pgfqpoint{9.707075in}{1.784962in}}{\pgfqpoint{9.702031in}{1.784962in}}%
\pgfpathcurveto{\pgfqpoint{9.696988in}{1.784962in}}{\pgfqpoint{9.692150in}{1.782958in}}{\pgfqpoint{9.688583in}{1.779392in}}%
\pgfpathcurveto{\pgfqpoint{9.685017in}{1.775825in}}{\pgfqpoint{9.683013in}{1.770987in}}{\pgfqpoint{9.683013in}{1.765944in}}%
\pgfpathcurveto{\pgfqpoint{9.683013in}{1.760900in}}{\pgfqpoint{9.685017in}{1.756062in}}{\pgfqpoint{9.688583in}{1.752496in}}%
\pgfpathcurveto{\pgfqpoint{9.692150in}{1.748929in}}{\pgfqpoint{9.696988in}{1.746925in}}{\pgfqpoint{9.702031in}{1.746925in}}%
\pgfpathclose%
\pgfusepath{fill}%
\end{pgfscope}%
\begin{pgfscope}%
\pgfpathrectangle{\pgfqpoint{6.572727in}{0.474100in}}{\pgfqpoint{4.227273in}{3.318700in}}%
\pgfusepath{clip}%
\pgfsetbuttcap%
\pgfsetroundjoin%
\definecolor{currentfill}{rgb}{0.127568,0.566949,0.550556}%
\pgfsetfillcolor{currentfill}%
\pgfsetfillopacity{0.700000}%
\pgfsetlinewidth{0.000000pt}%
\definecolor{currentstroke}{rgb}{0.000000,0.000000,0.000000}%
\pgfsetstrokecolor{currentstroke}%
\pgfsetstrokeopacity{0.700000}%
\pgfsetdash{}{0pt}%
\pgfpathmoveto{\pgfqpoint{9.625621in}{1.853849in}}%
\pgfpathcurveto{\pgfqpoint{9.630665in}{1.853849in}}{\pgfqpoint{9.635503in}{1.855853in}}{\pgfqpoint{9.639069in}{1.859419in}}%
\pgfpathcurveto{\pgfqpoint{9.642635in}{1.862986in}}{\pgfqpoint{9.644639in}{1.867824in}}{\pgfqpoint{9.644639in}{1.872867in}}%
\pgfpathcurveto{\pgfqpoint{9.644639in}{1.877911in}}{\pgfqpoint{9.642635in}{1.882749in}}{\pgfqpoint{9.639069in}{1.886315in}}%
\pgfpathcurveto{\pgfqpoint{9.635503in}{1.889882in}}{\pgfqpoint{9.630665in}{1.891885in}}{\pgfqpoint{9.625621in}{1.891885in}}%
\pgfpathcurveto{\pgfqpoint{9.620577in}{1.891885in}}{\pgfqpoint{9.615740in}{1.889882in}}{\pgfqpoint{9.612173in}{1.886315in}}%
\pgfpathcurveto{\pgfqpoint{9.608607in}{1.882749in}}{\pgfqpoint{9.606603in}{1.877911in}}{\pgfqpoint{9.606603in}{1.872867in}}%
\pgfpathcurveto{\pgfqpoint{9.606603in}{1.867824in}}{\pgfqpoint{9.608607in}{1.862986in}}{\pgfqpoint{9.612173in}{1.859419in}}%
\pgfpathcurveto{\pgfqpoint{9.615740in}{1.855853in}}{\pgfqpoint{9.620577in}{1.853849in}}{\pgfqpoint{9.625621in}{1.853849in}}%
\pgfpathclose%
\pgfusepath{fill}%
\end{pgfscope}%
\begin{pgfscope}%
\pgfpathrectangle{\pgfqpoint{6.572727in}{0.474100in}}{\pgfqpoint{4.227273in}{3.318700in}}%
\pgfusepath{clip}%
\pgfsetbuttcap%
\pgfsetroundjoin%
\definecolor{currentfill}{rgb}{0.993248,0.906157,0.143936}%
\pgfsetfillcolor{currentfill}%
\pgfsetfillopacity{0.700000}%
\pgfsetlinewidth{0.000000pt}%
\definecolor{currentstroke}{rgb}{0.000000,0.000000,0.000000}%
\pgfsetstrokecolor{currentstroke}%
\pgfsetstrokeopacity{0.700000}%
\pgfsetdash{}{0pt}%
\pgfpathmoveto{\pgfqpoint{7.811504in}{3.250368in}}%
\pgfpathcurveto{\pgfqpoint{7.816548in}{3.250368in}}{\pgfqpoint{7.821386in}{3.252371in}}{\pgfqpoint{7.824952in}{3.255938in}}%
\pgfpathcurveto{\pgfqpoint{7.828519in}{3.259504in}}{\pgfqpoint{7.830523in}{3.264342in}}{\pgfqpoint{7.830523in}{3.269386in}}%
\pgfpathcurveto{\pgfqpoint{7.830523in}{3.274429in}}{\pgfqpoint{7.828519in}{3.279267in}}{\pgfqpoint{7.824952in}{3.282834in}}%
\pgfpathcurveto{\pgfqpoint{7.821386in}{3.286400in}}{\pgfqpoint{7.816548in}{3.288404in}}{\pgfqpoint{7.811504in}{3.288404in}}%
\pgfpathcurveto{\pgfqpoint{7.806461in}{3.288404in}}{\pgfqpoint{7.801623in}{3.286400in}}{\pgfqpoint{7.798057in}{3.282834in}}%
\pgfpathcurveto{\pgfqpoint{7.794490in}{3.279267in}}{\pgfqpoint{7.792486in}{3.274429in}}{\pgfqpoint{7.792486in}{3.269386in}}%
\pgfpathcurveto{\pgfqpoint{7.792486in}{3.264342in}}{\pgfqpoint{7.794490in}{3.259504in}}{\pgfqpoint{7.798057in}{3.255938in}}%
\pgfpathcurveto{\pgfqpoint{7.801623in}{3.252371in}}{\pgfqpoint{7.806461in}{3.250368in}}{\pgfqpoint{7.811504in}{3.250368in}}%
\pgfpathclose%
\pgfusepath{fill}%
\end{pgfscope}%
\begin{pgfscope}%
\pgfpathrectangle{\pgfqpoint{6.572727in}{0.474100in}}{\pgfqpoint{4.227273in}{3.318700in}}%
\pgfusepath{clip}%
\pgfsetbuttcap%
\pgfsetroundjoin%
\definecolor{currentfill}{rgb}{0.267004,0.004874,0.329415}%
\pgfsetfillcolor{currentfill}%
\pgfsetfillopacity{0.700000}%
\pgfsetlinewidth{0.000000pt}%
\definecolor{currentstroke}{rgb}{0.000000,0.000000,0.000000}%
\pgfsetstrokecolor{currentstroke}%
\pgfsetstrokeopacity{0.700000}%
\pgfsetdash{}{0pt}%
\pgfpathmoveto{\pgfqpoint{7.262638in}{2.201074in}}%
\pgfpathcurveto{\pgfqpoint{7.267681in}{2.201074in}}{\pgfqpoint{7.272519in}{2.203078in}}{\pgfqpoint{7.276086in}{2.206644in}}%
\pgfpathcurveto{\pgfqpoint{7.279652in}{2.210210in}}{\pgfqpoint{7.281656in}{2.215048in}}{\pgfqpoint{7.281656in}{2.220092in}}%
\pgfpathcurveto{\pgfqpoint{7.281656in}{2.225136in}}{\pgfqpoint{7.279652in}{2.229973in}}{\pgfqpoint{7.276086in}{2.233540in}}%
\pgfpathcurveto{\pgfqpoint{7.272519in}{2.237106in}}{\pgfqpoint{7.267681in}{2.239110in}}{\pgfqpoint{7.262638in}{2.239110in}}%
\pgfpathcurveto{\pgfqpoint{7.257594in}{2.239110in}}{\pgfqpoint{7.252756in}{2.237106in}}{\pgfqpoint{7.249190in}{2.233540in}}%
\pgfpathcurveto{\pgfqpoint{7.245623in}{2.229973in}}{\pgfqpoint{7.243620in}{2.225136in}}{\pgfqpoint{7.243620in}{2.220092in}}%
\pgfpathcurveto{\pgfqpoint{7.243620in}{2.215048in}}{\pgfqpoint{7.245623in}{2.210210in}}{\pgfqpoint{7.249190in}{2.206644in}}%
\pgfpathcurveto{\pgfqpoint{7.252756in}{2.203078in}}{\pgfqpoint{7.257594in}{2.201074in}}{\pgfqpoint{7.262638in}{2.201074in}}%
\pgfpathclose%
\pgfusepath{fill}%
\end{pgfscope}%
\begin{pgfscope}%
\pgfpathrectangle{\pgfqpoint{6.572727in}{0.474100in}}{\pgfqpoint{4.227273in}{3.318700in}}%
\pgfusepath{clip}%
\pgfsetbuttcap%
\pgfsetroundjoin%
\definecolor{currentfill}{rgb}{0.127568,0.566949,0.550556}%
\pgfsetfillcolor{currentfill}%
\pgfsetfillopacity{0.700000}%
\pgfsetlinewidth{0.000000pt}%
\definecolor{currentstroke}{rgb}{0.000000,0.000000,0.000000}%
\pgfsetstrokecolor{currentstroke}%
\pgfsetstrokeopacity{0.700000}%
\pgfsetdash{}{0pt}%
\pgfpathmoveto{\pgfqpoint{9.632931in}{2.304924in}}%
\pgfpathcurveto{\pgfqpoint{9.637975in}{2.304924in}}{\pgfqpoint{9.642812in}{2.306928in}}{\pgfqpoint{9.646379in}{2.310494in}}%
\pgfpathcurveto{\pgfqpoint{9.649945in}{2.314060in}}{\pgfqpoint{9.651949in}{2.318898in}}{\pgfqpoint{9.651949in}{2.323942in}}%
\pgfpathcurveto{\pgfqpoint{9.651949in}{2.328986in}}{\pgfqpoint{9.649945in}{2.333823in}}{\pgfqpoint{9.646379in}{2.337390in}}%
\pgfpathcurveto{\pgfqpoint{9.642812in}{2.340956in}}{\pgfqpoint{9.637975in}{2.342960in}}{\pgfqpoint{9.632931in}{2.342960in}}%
\pgfpathcurveto{\pgfqpoint{9.627887in}{2.342960in}}{\pgfqpoint{9.623050in}{2.340956in}}{\pgfqpoint{9.619483in}{2.337390in}}%
\pgfpathcurveto{\pgfqpoint{9.615917in}{2.333823in}}{\pgfqpoint{9.613913in}{2.328986in}}{\pgfqpoint{9.613913in}{2.323942in}}%
\pgfpathcurveto{\pgfqpoint{9.613913in}{2.318898in}}{\pgfqpoint{9.615917in}{2.314060in}}{\pgfqpoint{9.619483in}{2.310494in}}%
\pgfpathcurveto{\pgfqpoint{9.623050in}{2.306928in}}{\pgfqpoint{9.627887in}{2.304924in}}{\pgfqpoint{9.632931in}{2.304924in}}%
\pgfpathclose%
\pgfusepath{fill}%
\end{pgfscope}%
\begin{pgfscope}%
\pgfpathrectangle{\pgfqpoint{6.572727in}{0.474100in}}{\pgfqpoint{4.227273in}{3.318700in}}%
\pgfusepath{clip}%
\pgfsetbuttcap%
\pgfsetroundjoin%
\definecolor{currentfill}{rgb}{0.127568,0.566949,0.550556}%
\pgfsetfillcolor{currentfill}%
\pgfsetfillopacity{0.700000}%
\pgfsetlinewidth{0.000000pt}%
\definecolor{currentstroke}{rgb}{0.000000,0.000000,0.000000}%
\pgfsetstrokecolor{currentstroke}%
\pgfsetstrokeopacity{0.700000}%
\pgfsetdash{}{0pt}%
\pgfpathmoveto{\pgfqpoint{9.032504in}{1.096022in}}%
\pgfpathcurveto{\pgfqpoint{9.037547in}{1.096022in}}{\pgfqpoint{9.042385in}{1.098026in}}{\pgfqpoint{9.045951in}{1.101593in}}%
\pgfpathcurveto{\pgfqpoint{9.049518in}{1.105159in}}{\pgfqpoint{9.051522in}{1.109997in}}{\pgfqpoint{9.051522in}{1.115041in}}%
\pgfpathcurveto{\pgfqpoint{9.051522in}{1.120084in}}{\pgfqpoint{9.049518in}{1.124922in}}{\pgfqpoint{9.045951in}{1.128488in}}%
\pgfpathcurveto{\pgfqpoint{9.042385in}{1.132055in}}{\pgfqpoint{9.037547in}{1.134059in}}{\pgfqpoint{9.032504in}{1.134059in}}%
\pgfpathcurveto{\pgfqpoint{9.027460in}{1.134059in}}{\pgfqpoint{9.022622in}{1.132055in}}{\pgfqpoint{9.019056in}{1.128488in}}%
\pgfpathcurveto{\pgfqpoint{9.015489in}{1.124922in}}{\pgfqpoint{9.013485in}{1.120084in}}{\pgfqpoint{9.013485in}{1.115041in}}%
\pgfpathcurveto{\pgfqpoint{9.013485in}{1.109997in}}{\pgfqpoint{9.015489in}{1.105159in}}{\pgfqpoint{9.019056in}{1.101593in}}%
\pgfpathcurveto{\pgfqpoint{9.022622in}{1.098026in}}{\pgfqpoint{9.027460in}{1.096022in}}{\pgfqpoint{9.032504in}{1.096022in}}%
\pgfpathclose%
\pgfusepath{fill}%
\end{pgfscope}%
\begin{pgfscope}%
\pgfpathrectangle{\pgfqpoint{6.572727in}{0.474100in}}{\pgfqpoint{4.227273in}{3.318700in}}%
\pgfusepath{clip}%
\pgfsetbuttcap%
\pgfsetroundjoin%
\definecolor{currentfill}{rgb}{0.993248,0.906157,0.143936}%
\pgfsetfillcolor{currentfill}%
\pgfsetfillopacity{0.700000}%
\pgfsetlinewidth{0.000000pt}%
\definecolor{currentstroke}{rgb}{0.000000,0.000000,0.000000}%
\pgfsetstrokecolor{currentstroke}%
\pgfsetstrokeopacity{0.700000}%
\pgfsetdash{}{0pt}%
\pgfpathmoveto{\pgfqpoint{8.096654in}{3.223882in}}%
\pgfpathcurveto{\pgfqpoint{8.101698in}{3.223882in}}{\pgfqpoint{8.106536in}{3.225886in}}{\pgfqpoint{8.110102in}{3.229452in}}%
\pgfpathcurveto{\pgfqpoint{8.113668in}{3.233019in}}{\pgfqpoint{8.115672in}{3.237856in}}{\pgfqpoint{8.115672in}{3.242900in}}%
\pgfpathcurveto{\pgfqpoint{8.115672in}{3.247944in}}{\pgfqpoint{8.113668in}{3.252781in}}{\pgfqpoint{8.110102in}{3.256348in}}%
\pgfpathcurveto{\pgfqpoint{8.106536in}{3.259914in}}{\pgfqpoint{8.101698in}{3.261918in}}{\pgfqpoint{8.096654in}{3.261918in}}%
\pgfpathcurveto{\pgfqpoint{8.091610in}{3.261918in}}{\pgfqpoint{8.086773in}{3.259914in}}{\pgfqpoint{8.083206in}{3.256348in}}%
\pgfpathcurveto{\pgfqpoint{8.079640in}{3.252781in}}{\pgfqpoint{8.077636in}{3.247944in}}{\pgfqpoint{8.077636in}{3.242900in}}%
\pgfpathcurveto{\pgfqpoint{8.077636in}{3.237856in}}{\pgfqpoint{8.079640in}{3.233019in}}{\pgfqpoint{8.083206in}{3.229452in}}%
\pgfpathcurveto{\pgfqpoint{8.086773in}{3.225886in}}{\pgfqpoint{8.091610in}{3.223882in}}{\pgfqpoint{8.096654in}{3.223882in}}%
\pgfpathclose%
\pgfusepath{fill}%
\end{pgfscope}%
\begin{pgfscope}%
\pgfpathrectangle{\pgfqpoint{6.572727in}{0.474100in}}{\pgfqpoint{4.227273in}{3.318700in}}%
\pgfusepath{clip}%
\pgfsetbuttcap%
\pgfsetroundjoin%
\definecolor{currentfill}{rgb}{0.993248,0.906157,0.143936}%
\pgfsetfillcolor{currentfill}%
\pgfsetfillopacity{0.700000}%
\pgfsetlinewidth{0.000000pt}%
\definecolor{currentstroke}{rgb}{0.000000,0.000000,0.000000}%
\pgfsetstrokecolor{currentstroke}%
\pgfsetstrokeopacity{0.700000}%
\pgfsetdash{}{0pt}%
\pgfpathmoveto{\pgfqpoint{8.467173in}{3.114265in}}%
\pgfpathcurveto{\pgfqpoint{8.472217in}{3.114265in}}{\pgfqpoint{8.477054in}{3.116269in}}{\pgfqpoint{8.480621in}{3.119835in}}%
\pgfpathcurveto{\pgfqpoint{8.484187in}{3.123402in}}{\pgfqpoint{8.486191in}{3.128239in}}{\pgfqpoint{8.486191in}{3.133283in}}%
\pgfpathcurveto{\pgfqpoint{8.486191in}{3.138327in}}{\pgfqpoint{8.484187in}{3.143165in}}{\pgfqpoint{8.480621in}{3.146731in}}%
\pgfpathcurveto{\pgfqpoint{8.477054in}{3.150297in}}{\pgfqpoint{8.472217in}{3.152301in}}{\pgfqpoint{8.467173in}{3.152301in}}%
\pgfpathcurveto{\pgfqpoint{8.462129in}{3.152301in}}{\pgfqpoint{8.457292in}{3.150297in}}{\pgfqpoint{8.453725in}{3.146731in}}%
\pgfpathcurveto{\pgfqpoint{8.450159in}{3.143165in}}{\pgfqpoint{8.448155in}{3.138327in}}{\pgfqpoint{8.448155in}{3.133283in}}%
\pgfpathcurveto{\pgfqpoint{8.448155in}{3.128239in}}{\pgfqpoint{8.450159in}{3.123402in}}{\pgfqpoint{8.453725in}{3.119835in}}%
\pgfpathcurveto{\pgfqpoint{8.457292in}{3.116269in}}{\pgfqpoint{8.462129in}{3.114265in}}{\pgfqpoint{8.467173in}{3.114265in}}%
\pgfpathclose%
\pgfusepath{fill}%
\end{pgfscope}%
\begin{pgfscope}%
\pgfpathrectangle{\pgfqpoint{6.572727in}{0.474100in}}{\pgfqpoint{4.227273in}{3.318700in}}%
\pgfusepath{clip}%
\pgfsetbuttcap%
\pgfsetroundjoin%
\definecolor{currentfill}{rgb}{0.127568,0.566949,0.550556}%
\pgfsetfillcolor{currentfill}%
\pgfsetfillopacity{0.700000}%
\pgfsetlinewidth{0.000000pt}%
\definecolor{currentstroke}{rgb}{0.000000,0.000000,0.000000}%
\pgfsetstrokecolor{currentstroke}%
\pgfsetstrokeopacity{0.700000}%
\pgfsetdash{}{0pt}%
\pgfpathmoveto{\pgfqpoint{9.584894in}{1.203887in}}%
\pgfpathcurveto{\pgfqpoint{9.589938in}{1.203887in}}{\pgfqpoint{9.594776in}{1.205891in}}{\pgfqpoint{9.598342in}{1.209457in}}%
\pgfpathcurveto{\pgfqpoint{9.601909in}{1.213024in}}{\pgfqpoint{9.603913in}{1.217861in}}{\pgfqpoint{9.603913in}{1.222905in}}%
\pgfpathcurveto{\pgfqpoint{9.603913in}{1.227949in}}{\pgfqpoint{9.601909in}{1.232786in}}{\pgfqpoint{9.598342in}{1.236353in}}%
\pgfpathcurveto{\pgfqpoint{9.594776in}{1.239919in}}{\pgfqpoint{9.589938in}{1.241923in}}{\pgfqpoint{9.584894in}{1.241923in}}%
\pgfpathcurveto{\pgfqpoint{9.579851in}{1.241923in}}{\pgfqpoint{9.575013in}{1.239919in}}{\pgfqpoint{9.571447in}{1.236353in}}%
\pgfpathcurveto{\pgfqpoint{9.567880in}{1.232786in}}{\pgfqpoint{9.565876in}{1.227949in}}{\pgfqpoint{9.565876in}{1.222905in}}%
\pgfpathcurveto{\pgfqpoint{9.565876in}{1.217861in}}{\pgfqpoint{9.567880in}{1.213024in}}{\pgfqpoint{9.571447in}{1.209457in}}%
\pgfpathcurveto{\pgfqpoint{9.575013in}{1.205891in}}{\pgfqpoint{9.579851in}{1.203887in}}{\pgfqpoint{9.584894in}{1.203887in}}%
\pgfpathclose%
\pgfusepath{fill}%
\end{pgfscope}%
\begin{pgfscope}%
\pgfpathrectangle{\pgfqpoint{6.572727in}{0.474100in}}{\pgfqpoint{4.227273in}{3.318700in}}%
\pgfusepath{clip}%
\pgfsetbuttcap%
\pgfsetroundjoin%
\definecolor{currentfill}{rgb}{0.127568,0.566949,0.550556}%
\pgfsetfillcolor{currentfill}%
\pgfsetfillopacity{0.700000}%
\pgfsetlinewidth{0.000000pt}%
\definecolor{currentstroke}{rgb}{0.000000,0.000000,0.000000}%
\pgfsetstrokecolor{currentstroke}%
\pgfsetstrokeopacity{0.700000}%
\pgfsetdash{}{0pt}%
\pgfpathmoveto{\pgfqpoint{9.776371in}{1.543034in}}%
\pgfpathcurveto{\pgfqpoint{9.781414in}{1.543034in}}{\pgfqpoint{9.786252in}{1.545038in}}{\pgfqpoint{9.789819in}{1.548604in}}%
\pgfpathcurveto{\pgfqpoint{9.793385in}{1.552171in}}{\pgfqpoint{9.795389in}{1.557008in}}{\pgfqpoint{9.795389in}{1.562052in}}%
\pgfpathcurveto{\pgfqpoint{9.795389in}{1.567096in}}{\pgfqpoint{9.793385in}{1.571933in}}{\pgfqpoint{9.789819in}{1.575500in}}%
\pgfpathcurveto{\pgfqpoint{9.786252in}{1.579066in}}{\pgfqpoint{9.781414in}{1.581070in}}{\pgfqpoint{9.776371in}{1.581070in}}%
\pgfpathcurveto{\pgfqpoint{9.771327in}{1.581070in}}{\pgfqpoint{9.766489in}{1.579066in}}{\pgfqpoint{9.762923in}{1.575500in}}%
\pgfpathcurveto{\pgfqpoint{9.759356in}{1.571933in}}{\pgfqpoint{9.757352in}{1.567096in}}{\pgfqpoint{9.757352in}{1.562052in}}%
\pgfpathcurveto{\pgfqpoint{9.757352in}{1.557008in}}{\pgfqpoint{9.759356in}{1.552171in}}{\pgfqpoint{9.762923in}{1.548604in}}%
\pgfpathcurveto{\pgfqpoint{9.766489in}{1.545038in}}{\pgfqpoint{9.771327in}{1.543034in}}{\pgfqpoint{9.776371in}{1.543034in}}%
\pgfpathclose%
\pgfusepath{fill}%
\end{pgfscope}%
\begin{pgfscope}%
\pgfpathrectangle{\pgfqpoint{6.572727in}{0.474100in}}{\pgfqpoint{4.227273in}{3.318700in}}%
\pgfusepath{clip}%
\pgfsetbuttcap%
\pgfsetroundjoin%
\definecolor{currentfill}{rgb}{0.993248,0.906157,0.143936}%
\pgfsetfillcolor{currentfill}%
\pgfsetfillopacity{0.700000}%
\pgfsetlinewidth{0.000000pt}%
\definecolor{currentstroke}{rgb}{0.000000,0.000000,0.000000}%
\pgfsetstrokecolor{currentstroke}%
\pgfsetstrokeopacity{0.700000}%
\pgfsetdash{}{0pt}%
\pgfpathmoveto{\pgfqpoint{8.333781in}{2.515937in}}%
\pgfpathcurveto{\pgfqpoint{8.338824in}{2.515937in}}{\pgfqpoint{8.343662in}{2.517941in}}{\pgfqpoint{8.347229in}{2.521507in}}%
\pgfpathcurveto{\pgfqpoint{8.350795in}{2.525074in}}{\pgfqpoint{8.352799in}{2.529911in}}{\pgfqpoint{8.352799in}{2.534955in}}%
\pgfpathcurveto{\pgfqpoint{8.352799in}{2.539999in}}{\pgfqpoint{8.350795in}{2.544837in}}{\pgfqpoint{8.347229in}{2.548403in}}%
\pgfpathcurveto{\pgfqpoint{8.343662in}{2.551969in}}{\pgfqpoint{8.338824in}{2.553973in}}{\pgfqpoint{8.333781in}{2.553973in}}%
\pgfpathcurveto{\pgfqpoint{8.328737in}{2.553973in}}{\pgfqpoint{8.323899in}{2.551969in}}{\pgfqpoint{8.320333in}{2.548403in}}%
\pgfpathcurveto{\pgfqpoint{8.316766in}{2.544837in}}{\pgfqpoint{8.314763in}{2.539999in}}{\pgfqpoint{8.314763in}{2.534955in}}%
\pgfpathcurveto{\pgfqpoint{8.314763in}{2.529911in}}{\pgfqpoint{8.316766in}{2.525074in}}{\pgfqpoint{8.320333in}{2.521507in}}%
\pgfpathcurveto{\pgfqpoint{8.323899in}{2.517941in}}{\pgfqpoint{8.328737in}{2.515937in}}{\pgfqpoint{8.333781in}{2.515937in}}%
\pgfpathclose%
\pgfusepath{fill}%
\end{pgfscope}%
\begin{pgfscope}%
\pgfpathrectangle{\pgfqpoint{6.572727in}{0.474100in}}{\pgfqpoint{4.227273in}{3.318700in}}%
\pgfusepath{clip}%
\pgfsetbuttcap%
\pgfsetroundjoin%
\definecolor{currentfill}{rgb}{0.127568,0.566949,0.550556}%
\pgfsetfillcolor{currentfill}%
\pgfsetfillopacity{0.700000}%
\pgfsetlinewidth{0.000000pt}%
\definecolor{currentstroke}{rgb}{0.000000,0.000000,0.000000}%
\pgfsetstrokecolor{currentstroke}%
\pgfsetstrokeopacity{0.700000}%
\pgfsetdash{}{0pt}%
\pgfpathmoveto{\pgfqpoint{9.943738in}{1.958765in}}%
\pgfpathcurveto{\pgfqpoint{9.948782in}{1.958765in}}{\pgfqpoint{9.953619in}{1.960769in}}{\pgfqpoint{9.957186in}{1.964336in}}%
\pgfpathcurveto{\pgfqpoint{9.960752in}{1.967902in}}{\pgfqpoint{9.962756in}{1.972740in}}{\pgfqpoint{9.962756in}{1.977784in}}%
\pgfpathcurveto{\pgfqpoint{9.962756in}{1.982827in}}{\pgfqpoint{9.960752in}{1.987665in}}{\pgfqpoint{9.957186in}{1.991231in}}%
\pgfpathcurveto{\pgfqpoint{9.953619in}{1.994798in}}{\pgfqpoint{9.948782in}{1.996802in}}{\pgfqpoint{9.943738in}{1.996802in}}%
\pgfpathcurveto{\pgfqpoint{9.938694in}{1.996802in}}{\pgfqpoint{9.933856in}{1.994798in}}{\pgfqpoint{9.930290in}{1.991231in}}%
\pgfpathcurveto{\pgfqpoint{9.926724in}{1.987665in}}{\pgfqpoint{9.924720in}{1.982827in}}{\pgfqpoint{9.924720in}{1.977784in}}%
\pgfpathcurveto{\pgfqpoint{9.924720in}{1.972740in}}{\pgfqpoint{9.926724in}{1.967902in}}{\pgfqpoint{9.930290in}{1.964336in}}%
\pgfpathcurveto{\pgfqpoint{9.933856in}{1.960769in}}{\pgfqpoint{9.938694in}{1.958765in}}{\pgfqpoint{9.943738in}{1.958765in}}%
\pgfpathclose%
\pgfusepath{fill}%
\end{pgfscope}%
\begin{pgfscope}%
\pgfpathrectangle{\pgfqpoint{6.572727in}{0.474100in}}{\pgfqpoint{4.227273in}{3.318700in}}%
\pgfusepath{clip}%
\pgfsetbuttcap%
\pgfsetroundjoin%
\definecolor{currentfill}{rgb}{0.127568,0.566949,0.550556}%
\pgfsetfillcolor{currentfill}%
\pgfsetfillopacity{0.700000}%
\pgfsetlinewidth{0.000000pt}%
\definecolor{currentstroke}{rgb}{0.000000,0.000000,0.000000}%
\pgfsetstrokecolor{currentstroke}%
\pgfsetstrokeopacity{0.700000}%
\pgfsetdash{}{0pt}%
\pgfpathmoveto{\pgfqpoint{9.140389in}{2.030908in}}%
\pgfpathcurveto{\pgfqpoint{9.145433in}{2.030908in}}{\pgfqpoint{9.150271in}{2.032912in}}{\pgfqpoint{9.153837in}{2.036479in}}%
\pgfpathcurveto{\pgfqpoint{9.157403in}{2.040045in}}{\pgfqpoint{9.159407in}{2.044883in}}{\pgfqpoint{9.159407in}{2.049927in}}%
\pgfpathcurveto{\pgfqpoint{9.159407in}{2.054970in}}{\pgfqpoint{9.157403in}{2.059808in}}{\pgfqpoint{9.153837in}{2.063374in}}%
\pgfpathcurveto{\pgfqpoint{9.150271in}{2.066941in}}{\pgfqpoint{9.145433in}{2.068945in}}{\pgfqpoint{9.140389in}{2.068945in}}%
\pgfpathcurveto{\pgfqpoint{9.135345in}{2.068945in}}{\pgfqpoint{9.130508in}{2.066941in}}{\pgfqpoint{9.126941in}{2.063374in}}%
\pgfpathcurveto{\pgfqpoint{9.123375in}{2.059808in}}{\pgfqpoint{9.121371in}{2.054970in}}{\pgfqpoint{9.121371in}{2.049927in}}%
\pgfpathcurveto{\pgfqpoint{9.121371in}{2.044883in}}{\pgfqpoint{9.123375in}{2.040045in}}{\pgfqpoint{9.126941in}{2.036479in}}%
\pgfpathcurveto{\pgfqpoint{9.130508in}{2.032912in}}{\pgfqpoint{9.135345in}{2.030908in}}{\pgfqpoint{9.140389in}{2.030908in}}%
\pgfpathclose%
\pgfusepath{fill}%
\end{pgfscope}%
\begin{pgfscope}%
\pgfpathrectangle{\pgfqpoint{6.572727in}{0.474100in}}{\pgfqpoint{4.227273in}{3.318700in}}%
\pgfusepath{clip}%
\pgfsetbuttcap%
\pgfsetroundjoin%
\definecolor{currentfill}{rgb}{0.267004,0.004874,0.329415}%
\pgfsetfillcolor{currentfill}%
\pgfsetfillopacity{0.700000}%
\pgfsetlinewidth{0.000000pt}%
\definecolor{currentstroke}{rgb}{0.000000,0.000000,0.000000}%
\pgfsetstrokecolor{currentstroke}%
\pgfsetstrokeopacity{0.700000}%
\pgfsetdash{}{0pt}%
\pgfpathmoveto{\pgfqpoint{7.638007in}{1.450268in}}%
\pgfpathcurveto{\pgfqpoint{7.643051in}{1.450268in}}{\pgfqpoint{7.647889in}{1.452272in}}{\pgfqpoint{7.651455in}{1.455839in}}%
\pgfpathcurveto{\pgfqpoint{7.655021in}{1.459405in}}{\pgfqpoint{7.657025in}{1.464243in}}{\pgfqpoint{7.657025in}{1.469286in}}%
\pgfpathcurveto{\pgfqpoint{7.657025in}{1.474330in}}{\pgfqpoint{7.655021in}{1.479168in}}{\pgfqpoint{7.651455in}{1.482734in}}%
\pgfpathcurveto{\pgfqpoint{7.647889in}{1.486301in}}{\pgfqpoint{7.643051in}{1.488305in}}{\pgfqpoint{7.638007in}{1.488305in}}%
\pgfpathcurveto{\pgfqpoint{7.632964in}{1.488305in}}{\pgfqpoint{7.628126in}{1.486301in}}{\pgfqpoint{7.624559in}{1.482734in}}%
\pgfpathcurveto{\pgfqpoint{7.620993in}{1.479168in}}{\pgfqpoint{7.618989in}{1.474330in}}{\pgfqpoint{7.618989in}{1.469286in}}%
\pgfpathcurveto{\pgfqpoint{7.618989in}{1.464243in}}{\pgfqpoint{7.620993in}{1.459405in}}{\pgfqpoint{7.624559in}{1.455839in}}%
\pgfpathcurveto{\pgfqpoint{7.628126in}{1.452272in}}{\pgfqpoint{7.632964in}{1.450268in}}{\pgfqpoint{7.638007in}{1.450268in}}%
\pgfpathclose%
\pgfusepath{fill}%
\end{pgfscope}%
\begin{pgfscope}%
\pgfpathrectangle{\pgfqpoint{6.572727in}{0.474100in}}{\pgfqpoint{4.227273in}{3.318700in}}%
\pgfusepath{clip}%
\pgfsetbuttcap%
\pgfsetroundjoin%
\definecolor{currentfill}{rgb}{0.127568,0.566949,0.550556}%
\pgfsetfillcolor{currentfill}%
\pgfsetfillopacity{0.700000}%
\pgfsetlinewidth{0.000000pt}%
\definecolor{currentstroke}{rgb}{0.000000,0.000000,0.000000}%
\pgfsetstrokecolor{currentstroke}%
\pgfsetstrokeopacity{0.700000}%
\pgfsetdash{}{0pt}%
\pgfpathmoveto{\pgfqpoint{9.578576in}{1.060389in}}%
\pgfpathcurveto{\pgfqpoint{9.583620in}{1.060389in}}{\pgfqpoint{9.588458in}{1.062393in}}{\pgfqpoint{9.592024in}{1.065959in}}%
\pgfpathcurveto{\pgfqpoint{9.595590in}{1.069526in}}{\pgfqpoint{9.597594in}{1.074363in}}{\pgfqpoint{9.597594in}{1.079407in}}%
\pgfpathcurveto{\pgfqpoint{9.597594in}{1.084451in}}{\pgfqpoint{9.595590in}{1.089289in}}{\pgfqpoint{9.592024in}{1.092855in}}%
\pgfpathcurveto{\pgfqpoint{9.588458in}{1.096421in}}{\pgfqpoint{9.583620in}{1.098425in}}{\pgfqpoint{9.578576in}{1.098425in}}%
\pgfpathcurveto{\pgfqpoint{9.573532in}{1.098425in}}{\pgfqpoint{9.568695in}{1.096421in}}{\pgfqpoint{9.565128in}{1.092855in}}%
\pgfpathcurveto{\pgfqpoint{9.561562in}{1.089289in}}{\pgfqpoint{9.559558in}{1.084451in}}{\pgfqpoint{9.559558in}{1.079407in}}%
\pgfpathcurveto{\pgfqpoint{9.559558in}{1.074363in}}{\pgfqpoint{9.561562in}{1.069526in}}{\pgfqpoint{9.565128in}{1.065959in}}%
\pgfpathcurveto{\pgfqpoint{9.568695in}{1.062393in}}{\pgfqpoint{9.573532in}{1.060389in}}{\pgfqpoint{9.578576in}{1.060389in}}%
\pgfpathclose%
\pgfusepath{fill}%
\end{pgfscope}%
\begin{pgfscope}%
\pgfpathrectangle{\pgfqpoint{6.572727in}{0.474100in}}{\pgfqpoint{4.227273in}{3.318700in}}%
\pgfusepath{clip}%
\pgfsetbuttcap%
\pgfsetroundjoin%
\definecolor{currentfill}{rgb}{0.127568,0.566949,0.550556}%
\pgfsetfillcolor{currentfill}%
\pgfsetfillopacity{0.700000}%
\pgfsetlinewidth{0.000000pt}%
\definecolor{currentstroke}{rgb}{0.000000,0.000000,0.000000}%
\pgfsetstrokecolor{currentstroke}%
\pgfsetstrokeopacity{0.700000}%
\pgfsetdash{}{0pt}%
\pgfpathmoveto{\pgfqpoint{9.907620in}{1.701599in}}%
\pgfpathcurveto{\pgfqpoint{9.912663in}{1.701599in}}{\pgfqpoint{9.917501in}{1.703602in}}{\pgfqpoint{9.921068in}{1.707169in}}%
\pgfpathcurveto{\pgfqpoint{9.924634in}{1.710735in}}{\pgfqpoint{9.926638in}{1.715573in}}{\pgfqpoint{9.926638in}{1.720617in}}%
\pgfpathcurveto{\pgfqpoint{9.926638in}{1.725660in}}{\pgfqpoint{9.924634in}{1.730498in}}{\pgfqpoint{9.921068in}{1.734065in}}%
\pgfpathcurveto{\pgfqpoint{9.917501in}{1.737631in}}{\pgfqpoint{9.912663in}{1.739635in}}{\pgfqpoint{9.907620in}{1.739635in}}%
\pgfpathcurveto{\pgfqpoint{9.902576in}{1.739635in}}{\pgfqpoint{9.897738in}{1.737631in}}{\pgfqpoint{9.894172in}{1.734065in}}%
\pgfpathcurveto{\pgfqpoint{9.890605in}{1.730498in}}{\pgfqpoint{9.888602in}{1.725660in}}{\pgfqpoint{9.888602in}{1.720617in}}%
\pgfpathcurveto{\pgfqpoint{9.888602in}{1.715573in}}{\pgfqpoint{9.890605in}{1.710735in}}{\pgfqpoint{9.894172in}{1.707169in}}%
\pgfpathcurveto{\pgfqpoint{9.897738in}{1.703602in}}{\pgfqpoint{9.902576in}{1.701599in}}{\pgfqpoint{9.907620in}{1.701599in}}%
\pgfpathclose%
\pgfusepath{fill}%
\end{pgfscope}%
\begin{pgfscope}%
\pgfpathrectangle{\pgfqpoint{6.572727in}{0.474100in}}{\pgfqpoint{4.227273in}{3.318700in}}%
\pgfusepath{clip}%
\pgfsetbuttcap%
\pgfsetroundjoin%
\definecolor{currentfill}{rgb}{0.127568,0.566949,0.550556}%
\pgfsetfillcolor{currentfill}%
\pgfsetfillopacity{0.700000}%
\pgfsetlinewidth{0.000000pt}%
\definecolor{currentstroke}{rgb}{0.000000,0.000000,0.000000}%
\pgfsetstrokecolor{currentstroke}%
\pgfsetstrokeopacity{0.700000}%
\pgfsetdash{}{0pt}%
\pgfpathmoveto{\pgfqpoint{9.089290in}{1.684619in}}%
\pgfpathcurveto{\pgfqpoint{9.094334in}{1.684619in}}{\pgfqpoint{9.099172in}{1.686623in}}{\pgfqpoint{9.102738in}{1.690189in}}%
\pgfpathcurveto{\pgfqpoint{9.106305in}{1.693756in}}{\pgfqpoint{9.108308in}{1.698593in}}{\pgfqpoint{9.108308in}{1.703637in}}%
\pgfpathcurveto{\pgfqpoint{9.108308in}{1.708681in}}{\pgfqpoint{9.106305in}{1.713519in}}{\pgfqpoint{9.102738in}{1.717085in}}%
\pgfpathcurveto{\pgfqpoint{9.099172in}{1.720651in}}{\pgfqpoint{9.094334in}{1.722655in}}{\pgfqpoint{9.089290in}{1.722655in}}%
\pgfpathcurveto{\pgfqpoint{9.084247in}{1.722655in}}{\pgfqpoint{9.079409in}{1.720651in}}{\pgfqpoint{9.075842in}{1.717085in}}%
\pgfpathcurveto{\pgfqpoint{9.072276in}{1.713519in}}{\pgfqpoint{9.070272in}{1.708681in}}{\pgfqpoint{9.070272in}{1.703637in}}%
\pgfpathcurveto{\pgfqpoint{9.070272in}{1.698593in}}{\pgfqpoint{9.072276in}{1.693756in}}{\pgfqpoint{9.075842in}{1.690189in}}%
\pgfpathcurveto{\pgfqpoint{9.079409in}{1.686623in}}{\pgfqpoint{9.084247in}{1.684619in}}{\pgfqpoint{9.089290in}{1.684619in}}%
\pgfpathclose%
\pgfusepath{fill}%
\end{pgfscope}%
\begin{pgfscope}%
\pgfpathrectangle{\pgfqpoint{6.572727in}{0.474100in}}{\pgfqpoint{4.227273in}{3.318700in}}%
\pgfusepath{clip}%
\pgfsetbuttcap%
\pgfsetroundjoin%
\definecolor{currentfill}{rgb}{0.993248,0.906157,0.143936}%
\pgfsetfillcolor{currentfill}%
\pgfsetfillopacity{0.700000}%
\pgfsetlinewidth{0.000000pt}%
\definecolor{currentstroke}{rgb}{0.000000,0.000000,0.000000}%
\pgfsetstrokecolor{currentstroke}%
\pgfsetstrokeopacity{0.700000}%
\pgfsetdash{}{0pt}%
\pgfpathmoveto{\pgfqpoint{9.091430in}{2.458496in}}%
\pgfpathcurveto{\pgfqpoint{9.096473in}{2.458496in}}{\pgfqpoint{9.101311in}{2.460500in}}{\pgfqpoint{9.104878in}{2.464066in}}%
\pgfpathcurveto{\pgfqpoint{9.108444in}{2.467633in}}{\pgfqpoint{9.110448in}{2.472470in}}{\pgfqpoint{9.110448in}{2.477514in}}%
\pgfpathcurveto{\pgfqpoint{9.110448in}{2.482558in}}{\pgfqpoint{9.108444in}{2.487395in}}{\pgfqpoint{9.104878in}{2.490962in}}%
\pgfpathcurveto{\pgfqpoint{9.101311in}{2.494528in}}{\pgfqpoint{9.096473in}{2.496532in}}{\pgfqpoint{9.091430in}{2.496532in}}%
\pgfpathcurveto{\pgfqpoint{9.086386in}{2.496532in}}{\pgfqpoint{9.081548in}{2.494528in}}{\pgfqpoint{9.077982in}{2.490962in}}%
\pgfpathcurveto{\pgfqpoint{9.074415in}{2.487395in}}{\pgfqpoint{9.072412in}{2.482558in}}{\pgfqpoint{9.072412in}{2.477514in}}%
\pgfpathcurveto{\pgfqpoint{9.072412in}{2.472470in}}{\pgfqpoint{9.074415in}{2.467633in}}{\pgfqpoint{9.077982in}{2.464066in}}%
\pgfpathcurveto{\pgfqpoint{9.081548in}{2.460500in}}{\pgfqpoint{9.086386in}{2.458496in}}{\pgfqpoint{9.091430in}{2.458496in}}%
\pgfpathclose%
\pgfusepath{fill}%
\end{pgfscope}%
\begin{pgfscope}%
\pgfpathrectangle{\pgfqpoint{6.572727in}{0.474100in}}{\pgfqpoint{4.227273in}{3.318700in}}%
\pgfusepath{clip}%
\pgfsetbuttcap%
\pgfsetroundjoin%
\definecolor{currentfill}{rgb}{0.267004,0.004874,0.329415}%
\pgfsetfillcolor{currentfill}%
\pgfsetfillopacity{0.700000}%
\pgfsetlinewidth{0.000000pt}%
\definecolor{currentstroke}{rgb}{0.000000,0.000000,0.000000}%
\pgfsetstrokecolor{currentstroke}%
\pgfsetstrokeopacity{0.700000}%
\pgfsetdash{}{0pt}%
\pgfpathmoveto{\pgfqpoint{7.440829in}{1.701689in}}%
\pgfpathcurveto{\pgfqpoint{7.445873in}{1.701689in}}{\pgfqpoint{7.450711in}{1.703693in}}{\pgfqpoint{7.454277in}{1.707259in}}%
\pgfpathcurveto{\pgfqpoint{7.457844in}{1.710826in}}{\pgfqpoint{7.459847in}{1.715664in}}{\pgfqpoint{7.459847in}{1.720707in}}%
\pgfpathcurveto{\pgfqpoint{7.459847in}{1.725751in}}{\pgfqpoint{7.457844in}{1.730589in}}{\pgfqpoint{7.454277in}{1.734155in}}%
\pgfpathcurveto{\pgfqpoint{7.450711in}{1.737722in}}{\pgfqpoint{7.445873in}{1.739725in}}{\pgfqpoint{7.440829in}{1.739725in}}%
\pgfpathcurveto{\pgfqpoint{7.435786in}{1.739725in}}{\pgfqpoint{7.430948in}{1.737722in}}{\pgfqpoint{7.427381in}{1.734155in}}%
\pgfpathcurveto{\pgfqpoint{7.423815in}{1.730589in}}{\pgfqpoint{7.421811in}{1.725751in}}{\pgfqpoint{7.421811in}{1.720707in}}%
\pgfpathcurveto{\pgfqpoint{7.421811in}{1.715664in}}{\pgfqpoint{7.423815in}{1.710826in}}{\pgfqpoint{7.427381in}{1.707259in}}%
\pgfpathcurveto{\pgfqpoint{7.430948in}{1.703693in}}{\pgfqpoint{7.435786in}{1.701689in}}{\pgfqpoint{7.440829in}{1.701689in}}%
\pgfpathclose%
\pgfusepath{fill}%
\end{pgfscope}%
\begin{pgfscope}%
\pgfpathrectangle{\pgfqpoint{6.572727in}{0.474100in}}{\pgfqpoint{4.227273in}{3.318700in}}%
\pgfusepath{clip}%
\pgfsetbuttcap%
\pgfsetroundjoin%
\definecolor{currentfill}{rgb}{0.993248,0.906157,0.143936}%
\pgfsetfillcolor{currentfill}%
\pgfsetfillopacity{0.700000}%
\pgfsetlinewidth{0.000000pt}%
\definecolor{currentstroke}{rgb}{0.000000,0.000000,0.000000}%
\pgfsetstrokecolor{currentstroke}%
\pgfsetstrokeopacity{0.700000}%
\pgfsetdash{}{0pt}%
\pgfpathmoveto{\pgfqpoint{8.116962in}{2.866384in}}%
\pgfpathcurveto{\pgfqpoint{8.122006in}{2.866384in}}{\pgfqpoint{8.126844in}{2.868388in}}{\pgfqpoint{8.130410in}{2.871954in}}%
\pgfpathcurveto{\pgfqpoint{8.133977in}{2.875521in}}{\pgfqpoint{8.135980in}{2.880358in}}{\pgfqpoint{8.135980in}{2.885402in}}%
\pgfpathcurveto{\pgfqpoint{8.135980in}{2.890446in}}{\pgfqpoint{8.133977in}{2.895284in}}{\pgfqpoint{8.130410in}{2.898850in}}%
\pgfpathcurveto{\pgfqpoint{8.126844in}{2.902416in}}{\pgfqpoint{8.122006in}{2.904420in}}{\pgfqpoint{8.116962in}{2.904420in}}%
\pgfpathcurveto{\pgfqpoint{8.111919in}{2.904420in}}{\pgfqpoint{8.107081in}{2.902416in}}{\pgfqpoint{8.103514in}{2.898850in}}%
\pgfpathcurveto{\pgfqpoint{8.099948in}{2.895284in}}{\pgfqpoint{8.097944in}{2.890446in}}{\pgfqpoint{8.097944in}{2.885402in}}%
\pgfpathcurveto{\pgfqpoint{8.097944in}{2.880358in}}{\pgfqpoint{8.099948in}{2.875521in}}{\pgfqpoint{8.103514in}{2.871954in}}%
\pgfpathcurveto{\pgfqpoint{8.107081in}{2.868388in}}{\pgfqpoint{8.111919in}{2.866384in}}{\pgfqpoint{8.116962in}{2.866384in}}%
\pgfpathclose%
\pgfusepath{fill}%
\end{pgfscope}%
\begin{pgfscope}%
\pgfpathrectangle{\pgfqpoint{6.572727in}{0.474100in}}{\pgfqpoint{4.227273in}{3.318700in}}%
\pgfusepath{clip}%
\pgfsetbuttcap%
\pgfsetroundjoin%
\definecolor{currentfill}{rgb}{0.127568,0.566949,0.550556}%
\pgfsetfillcolor{currentfill}%
\pgfsetfillopacity{0.700000}%
\pgfsetlinewidth{0.000000pt}%
\definecolor{currentstroke}{rgb}{0.000000,0.000000,0.000000}%
\pgfsetstrokecolor{currentstroke}%
\pgfsetstrokeopacity{0.700000}%
\pgfsetdash{}{0pt}%
\pgfpathmoveto{\pgfqpoint{9.851922in}{1.855126in}}%
\pgfpathcurveto{\pgfqpoint{9.856966in}{1.855126in}}{\pgfqpoint{9.861804in}{1.857130in}}{\pgfqpoint{9.865370in}{1.860696in}}%
\pgfpathcurveto{\pgfqpoint{9.868937in}{1.864262in}}{\pgfqpoint{9.870941in}{1.869100in}}{\pgfqpoint{9.870941in}{1.874144in}}%
\pgfpathcurveto{\pgfqpoint{9.870941in}{1.879188in}}{\pgfqpoint{9.868937in}{1.884025in}}{\pgfqpoint{9.865370in}{1.887592in}}%
\pgfpathcurveto{\pgfqpoint{9.861804in}{1.891158in}}{\pgfqpoint{9.856966in}{1.893162in}}{\pgfqpoint{9.851922in}{1.893162in}}%
\pgfpathcurveto{\pgfqpoint{9.846879in}{1.893162in}}{\pgfqpoint{9.842041in}{1.891158in}}{\pgfqpoint{9.838475in}{1.887592in}}%
\pgfpathcurveto{\pgfqpoint{9.834908in}{1.884025in}}{\pgfqpoint{9.832904in}{1.879188in}}{\pgfqpoint{9.832904in}{1.874144in}}%
\pgfpathcurveto{\pgfqpoint{9.832904in}{1.869100in}}{\pgfqpoint{9.834908in}{1.864262in}}{\pgfqpoint{9.838475in}{1.860696in}}%
\pgfpathcurveto{\pgfqpoint{9.842041in}{1.857130in}}{\pgfqpoint{9.846879in}{1.855126in}}{\pgfqpoint{9.851922in}{1.855126in}}%
\pgfpathclose%
\pgfusepath{fill}%
\end{pgfscope}%
\begin{pgfscope}%
\pgfpathrectangle{\pgfqpoint{6.572727in}{0.474100in}}{\pgfqpoint{4.227273in}{3.318700in}}%
\pgfusepath{clip}%
\pgfsetbuttcap%
\pgfsetroundjoin%
\definecolor{currentfill}{rgb}{0.127568,0.566949,0.550556}%
\pgfsetfillcolor{currentfill}%
\pgfsetfillopacity{0.700000}%
\pgfsetlinewidth{0.000000pt}%
\definecolor{currentstroke}{rgb}{0.000000,0.000000,0.000000}%
\pgfsetstrokecolor{currentstroke}%
\pgfsetstrokeopacity{0.700000}%
\pgfsetdash{}{0pt}%
\pgfpathmoveto{\pgfqpoint{9.680322in}{1.483906in}}%
\pgfpathcurveto{\pgfqpoint{9.685366in}{1.483906in}}{\pgfqpoint{9.690204in}{1.485910in}}{\pgfqpoint{9.693770in}{1.489476in}}%
\pgfpathcurveto{\pgfqpoint{9.697337in}{1.493043in}}{\pgfqpoint{9.699340in}{1.497881in}}{\pgfqpoint{9.699340in}{1.502924in}}%
\pgfpathcurveto{\pgfqpoint{9.699340in}{1.507968in}}{\pgfqpoint{9.697337in}{1.512806in}}{\pgfqpoint{9.693770in}{1.516372in}}%
\pgfpathcurveto{\pgfqpoint{9.690204in}{1.519938in}}{\pgfqpoint{9.685366in}{1.521942in}}{\pgfqpoint{9.680322in}{1.521942in}}%
\pgfpathcurveto{\pgfqpoint{9.675279in}{1.521942in}}{\pgfqpoint{9.670441in}{1.519938in}}{\pgfqpoint{9.666874in}{1.516372in}}%
\pgfpathcurveto{\pgfqpoint{9.663308in}{1.512806in}}{\pgfqpoint{9.661304in}{1.507968in}}{\pgfqpoint{9.661304in}{1.502924in}}%
\pgfpathcurveto{\pgfqpoint{9.661304in}{1.497881in}}{\pgfqpoint{9.663308in}{1.493043in}}{\pgfqpoint{9.666874in}{1.489476in}}%
\pgfpathcurveto{\pgfqpoint{9.670441in}{1.485910in}}{\pgfqpoint{9.675279in}{1.483906in}}{\pgfqpoint{9.680322in}{1.483906in}}%
\pgfpathclose%
\pgfusepath{fill}%
\end{pgfscope}%
\begin{pgfscope}%
\pgfpathrectangle{\pgfqpoint{6.572727in}{0.474100in}}{\pgfqpoint{4.227273in}{3.318700in}}%
\pgfusepath{clip}%
\pgfsetbuttcap%
\pgfsetroundjoin%
\definecolor{currentfill}{rgb}{0.127568,0.566949,0.550556}%
\pgfsetfillcolor{currentfill}%
\pgfsetfillopacity{0.700000}%
\pgfsetlinewidth{0.000000pt}%
\definecolor{currentstroke}{rgb}{0.000000,0.000000,0.000000}%
\pgfsetstrokecolor{currentstroke}%
\pgfsetstrokeopacity{0.700000}%
\pgfsetdash{}{0pt}%
\pgfpathmoveto{\pgfqpoint{9.586439in}{1.204134in}}%
\pgfpathcurveto{\pgfqpoint{9.591483in}{1.204134in}}{\pgfqpoint{9.596321in}{1.206138in}}{\pgfqpoint{9.599887in}{1.209704in}}%
\pgfpathcurveto{\pgfqpoint{9.603454in}{1.213271in}}{\pgfqpoint{9.605458in}{1.218108in}}{\pgfqpoint{9.605458in}{1.223152in}}%
\pgfpathcurveto{\pgfqpoint{9.605458in}{1.228196in}}{\pgfqpoint{9.603454in}{1.233033in}}{\pgfqpoint{9.599887in}{1.236600in}}%
\pgfpathcurveto{\pgfqpoint{9.596321in}{1.240166in}}{\pgfqpoint{9.591483in}{1.242170in}}{\pgfqpoint{9.586439in}{1.242170in}}%
\pgfpathcurveto{\pgfqpoint{9.581396in}{1.242170in}}{\pgfqpoint{9.576558in}{1.240166in}}{\pgfqpoint{9.572992in}{1.236600in}}%
\pgfpathcurveto{\pgfqpoint{9.569425in}{1.233033in}}{\pgfqpoint{9.567421in}{1.228196in}}{\pgfqpoint{9.567421in}{1.223152in}}%
\pgfpathcurveto{\pgfqpoint{9.567421in}{1.218108in}}{\pgfqpoint{9.569425in}{1.213271in}}{\pgfqpoint{9.572992in}{1.209704in}}%
\pgfpathcurveto{\pgfqpoint{9.576558in}{1.206138in}}{\pgfqpoint{9.581396in}{1.204134in}}{\pgfqpoint{9.586439in}{1.204134in}}%
\pgfpathclose%
\pgfusepath{fill}%
\end{pgfscope}%
\begin{pgfscope}%
\pgfpathrectangle{\pgfqpoint{6.572727in}{0.474100in}}{\pgfqpoint{4.227273in}{3.318700in}}%
\pgfusepath{clip}%
\pgfsetbuttcap%
\pgfsetroundjoin%
\definecolor{currentfill}{rgb}{0.993248,0.906157,0.143936}%
\pgfsetfillcolor{currentfill}%
\pgfsetfillopacity{0.700000}%
\pgfsetlinewidth{0.000000pt}%
\definecolor{currentstroke}{rgb}{0.000000,0.000000,0.000000}%
\pgfsetstrokecolor{currentstroke}%
\pgfsetstrokeopacity{0.700000}%
\pgfsetdash{}{0pt}%
\pgfpathmoveto{\pgfqpoint{7.819947in}{3.075917in}}%
\pgfpathcurveto{\pgfqpoint{7.824991in}{3.075917in}}{\pgfqpoint{7.829829in}{3.077921in}}{\pgfqpoint{7.833395in}{3.081487in}}%
\pgfpathcurveto{\pgfqpoint{7.836962in}{3.085054in}}{\pgfqpoint{7.838965in}{3.089891in}}{\pgfqpoint{7.838965in}{3.094935in}}%
\pgfpathcurveto{\pgfqpoint{7.838965in}{3.099979in}}{\pgfqpoint{7.836962in}{3.104816in}}{\pgfqpoint{7.833395in}{3.108383in}}%
\pgfpathcurveto{\pgfqpoint{7.829829in}{3.111949in}}{\pgfqpoint{7.824991in}{3.113953in}}{\pgfqpoint{7.819947in}{3.113953in}}%
\pgfpathcurveto{\pgfqpoint{7.814904in}{3.113953in}}{\pgfqpoint{7.810066in}{3.111949in}}{\pgfqpoint{7.806499in}{3.108383in}}%
\pgfpathcurveto{\pgfqpoint{7.802933in}{3.104816in}}{\pgfqpoint{7.800929in}{3.099979in}}{\pgfqpoint{7.800929in}{3.094935in}}%
\pgfpathcurveto{\pgfqpoint{7.800929in}{3.089891in}}{\pgfqpoint{7.802933in}{3.085054in}}{\pgfqpoint{7.806499in}{3.081487in}}%
\pgfpathcurveto{\pgfqpoint{7.810066in}{3.077921in}}{\pgfqpoint{7.814904in}{3.075917in}}{\pgfqpoint{7.819947in}{3.075917in}}%
\pgfpathclose%
\pgfusepath{fill}%
\end{pgfscope}%
\begin{pgfscope}%
\pgfpathrectangle{\pgfqpoint{6.572727in}{0.474100in}}{\pgfqpoint{4.227273in}{3.318700in}}%
\pgfusepath{clip}%
\pgfsetbuttcap%
\pgfsetroundjoin%
\definecolor{currentfill}{rgb}{0.127568,0.566949,0.550556}%
\pgfsetfillcolor{currentfill}%
\pgfsetfillopacity{0.700000}%
\pgfsetlinewidth{0.000000pt}%
\definecolor{currentstroke}{rgb}{0.000000,0.000000,0.000000}%
\pgfsetstrokecolor{currentstroke}%
\pgfsetstrokeopacity{0.700000}%
\pgfsetdash{}{0pt}%
\pgfpathmoveto{\pgfqpoint{9.253764in}{1.221726in}}%
\pgfpathcurveto{\pgfqpoint{9.258807in}{1.221726in}}{\pgfqpoint{9.263645in}{1.223730in}}{\pgfqpoint{9.267211in}{1.227296in}}%
\pgfpathcurveto{\pgfqpoint{9.270778in}{1.230863in}}{\pgfqpoint{9.272782in}{1.235701in}}{\pgfqpoint{9.272782in}{1.240744in}}%
\pgfpathcurveto{\pgfqpoint{9.272782in}{1.245788in}}{\pgfqpoint{9.270778in}{1.250626in}}{\pgfqpoint{9.267211in}{1.254192in}}%
\pgfpathcurveto{\pgfqpoint{9.263645in}{1.257759in}}{\pgfqpoint{9.258807in}{1.259762in}}{\pgfqpoint{9.253764in}{1.259762in}}%
\pgfpathcurveto{\pgfqpoint{9.248720in}{1.259762in}}{\pgfqpoint{9.243882in}{1.257759in}}{\pgfqpoint{9.240316in}{1.254192in}}%
\pgfpathcurveto{\pgfqpoint{9.236749in}{1.250626in}}{\pgfqpoint{9.234745in}{1.245788in}}{\pgfqpoint{9.234745in}{1.240744in}}%
\pgfpathcurveto{\pgfqpoint{9.234745in}{1.235701in}}{\pgfqpoint{9.236749in}{1.230863in}}{\pgfqpoint{9.240316in}{1.227296in}}%
\pgfpathcurveto{\pgfqpoint{9.243882in}{1.223730in}}{\pgfqpoint{9.248720in}{1.221726in}}{\pgfqpoint{9.253764in}{1.221726in}}%
\pgfpathclose%
\pgfusepath{fill}%
\end{pgfscope}%
\begin{pgfscope}%
\pgfpathrectangle{\pgfqpoint{6.572727in}{0.474100in}}{\pgfqpoint{4.227273in}{3.318700in}}%
\pgfusepath{clip}%
\pgfsetbuttcap%
\pgfsetroundjoin%
\definecolor{currentfill}{rgb}{0.267004,0.004874,0.329415}%
\pgfsetfillcolor{currentfill}%
\pgfsetfillopacity{0.700000}%
\pgfsetlinewidth{0.000000pt}%
\definecolor{currentstroke}{rgb}{0.000000,0.000000,0.000000}%
\pgfsetstrokecolor{currentstroke}%
\pgfsetstrokeopacity{0.700000}%
\pgfsetdash{}{0pt}%
\pgfpathmoveto{\pgfqpoint{7.869275in}{1.624017in}}%
\pgfpathcurveto{\pgfqpoint{7.874319in}{1.624017in}}{\pgfqpoint{7.879156in}{1.626020in}}{\pgfqpoint{7.882723in}{1.629587in}}%
\pgfpathcurveto{\pgfqpoint{7.886289in}{1.633153in}}{\pgfqpoint{7.888293in}{1.637991in}}{\pgfqpoint{7.888293in}{1.643035in}}%
\pgfpathcurveto{\pgfqpoint{7.888293in}{1.648078in}}{\pgfqpoint{7.886289in}{1.652916in}}{\pgfqpoint{7.882723in}{1.656483in}}%
\pgfpathcurveto{\pgfqpoint{7.879156in}{1.660049in}}{\pgfqpoint{7.874319in}{1.662053in}}{\pgfqpoint{7.869275in}{1.662053in}}%
\pgfpathcurveto{\pgfqpoint{7.864231in}{1.662053in}}{\pgfqpoint{7.859393in}{1.660049in}}{\pgfqpoint{7.855827in}{1.656483in}}%
\pgfpathcurveto{\pgfqpoint{7.852261in}{1.652916in}}{\pgfqpoint{7.850257in}{1.648078in}}{\pgfqpoint{7.850257in}{1.643035in}}%
\pgfpathcurveto{\pgfqpoint{7.850257in}{1.637991in}}{\pgfqpoint{7.852261in}{1.633153in}}{\pgfqpoint{7.855827in}{1.629587in}}%
\pgfpathcurveto{\pgfqpoint{7.859393in}{1.626020in}}{\pgfqpoint{7.864231in}{1.624017in}}{\pgfqpoint{7.869275in}{1.624017in}}%
\pgfpathclose%
\pgfusepath{fill}%
\end{pgfscope}%
\begin{pgfscope}%
\pgfpathrectangle{\pgfqpoint{6.572727in}{0.474100in}}{\pgfqpoint{4.227273in}{3.318700in}}%
\pgfusepath{clip}%
\pgfsetbuttcap%
\pgfsetroundjoin%
\definecolor{currentfill}{rgb}{0.267004,0.004874,0.329415}%
\pgfsetfillcolor{currentfill}%
\pgfsetfillopacity{0.700000}%
\pgfsetlinewidth{0.000000pt}%
\definecolor{currentstroke}{rgb}{0.000000,0.000000,0.000000}%
\pgfsetstrokecolor{currentstroke}%
\pgfsetstrokeopacity{0.700000}%
\pgfsetdash{}{0pt}%
\pgfpathmoveto{\pgfqpoint{7.867221in}{1.526818in}}%
\pgfpathcurveto{\pgfqpoint{7.872264in}{1.526818in}}{\pgfqpoint{7.877102in}{1.528822in}}{\pgfqpoint{7.880669in}{1.532388in}}%
\pgfpathcurveto{\pgfqpoint{7.884235in}{1.535955in}}{\pgfqpoint{7.886239in}{1.540793in}}{\pgfqpoint{7.886239in}{1.545836in}}%
\pgfpathcurveto{\pgfqpoint{7.886239in}{1.550880in}}{\pgfqpoint{7.884235in}{1.555718in}}{\pgfqpoint{7.880669in}{1.559284in}}%
\pgfpathcurveto{\pgfqpoint{7.877102in}{1.562851in}}{\pgfqpoint{7.872264in}{1.564854in}}{\pgfqpoint{7.867221in}{1.564854in}}%
\pgfpathcurveto{\pgfqpoint{7.862177in}{1.564854in}}{\pgfqpoint{7.857339in}{1.562851in}}{\pgfqpoint{7.853773in}{1.559284in}}%
\pgfpathcurveto{\pgfqpoint{7.850206in}{1.555718in}}{\pgfqpoint{7.848202in}{1.550880in}}{\pgfqpoint{7.848202in}{1.545836in}}%
\pgfpathcurveto{\pgfqpoint{7.848202in}{1.540793in}}{\pgfqpoint{7.850206in}{1.535955in}}{\pgfqpoint{7.853773in}{1.532388in}}%
\pgfpathcurveto{\pgfqpoint{7.857339in}{1.528822in}}{\pgfqpoint{7.862177in}{1.526818in}}{\pgfqpoint{7.867221in}{1.526818in}}%
\pgfpathclose%
\pgfusepath{fill}%
\end{pgfscope}%
\begin{pgfscope}%
\pgfpathrectangle{\pgfqpoint{6.572727in}{0.474100in}}{\pgfqpoint{4.227273in}{3.318700in}}%
\pgfusepath{clip}%
\pgfsetbuttcap%
\pgfsetroundjoin%
\definecolor{currentfill}{rgb}{0.993248,0.906157,0.143936}%
\pgfsetfillcolor{currentfill}%
\pgfsetfillopacity{0.700000}%
\pgfsetlinewidth{0.000000pt}%
\definecolor{currentstroke}{rgb}{0.000000,0.000000,0.000000}%
\pgfsetstrokecolor{currentstroke}%
\pgfsetstrokeopacity{0.700000}%
\pgfsetdash{}{0pt}%
\pgfpathmoveto{\pgfqpoint{8.437324in}{2.843606in}}%
\pgfpathcurveto{\pgfqpoint{8.442367in}{2.843606in}}{\pgfqpoint{8.447205in}{2.845610in}}{\pgfqpoint{8.450772in}{2.849176in}}%
\pgfpathcurveto{\pgfqpoint{8.454338in}{2.852742in}}{\pgfqpoint{8.456342in}{2.857580in}}{\pgfqpoint{8.456342in}{2.862624in}}%
\pgfpathcurveto{\pgfqpoint{8.456342in}{2.867668in}}{\pgfqpoint{8.454338in}{2.872505in}}{\pgfqpoint{8.450772in}{2.876072in}}%
\pgfpathcurveto{\pgfqpoint{8.447205in}{2.879638in}}{\pgfqpoint{8.442367in}{2.881642in}}{\pgfqpoint{8.437324in}{2.881642in}}%
\pgfpathcurveto{\pgfqpoint{8.432280in}{2.881642in}}{\pgfqpoint{8.427442in}{2.879638in}}{\pgfqpoint{8.423876in}{2.876072in}}%
\pgfpathcurveto{\pgfqpoint{8.420310in}{2.872505in}}{\pgfqpoint{8.418306in}{2.867668in}}{\pgfqpoint{8.418306in}{2.862624in}}%
\pgfpathcurveto{\pgfqpoint{8.418306in}{2.857580in}}{\pgfqpoint{8.420310in}{2.852742in}}{\pgfqpoint{8.423876in}{2.849176in}}%
\pgfpathcurveto{\pgfqpoint{8.427442in}{2.845610in}}{\pgfqpoint{8.432280in}{2.843606in}}{\pgfqpoint{8.437324in}{2.843606in}}%
\pgfpathclose%
\pgfusepath{fill}%
\end{pgfscope}%
\begin{pgfscope}%
\pgfpathrectangle{\pgfqpoint{6.572727in}{0.474100in}}{\pgfqpoint{4.227273in}{3.318700in}}%
\pgfusepath{clip}%
\pgfsetbuttcap%
\pgfsetroundjoin%
\definecolor{currentfill}{rgb}{0.127568,0.566949,0.550556}%
\pgfsetfillcolor{currentfill}%
\pgfsetfillopacity{0.700000}%
\pgfsetlinewidth{0.000000pt}%
\definecolor{currentstroke}{rgb}{0.000000,0.000000,0.000000}%
\pgfsetstrokecolor{currentstroke}%
\pgfsetstrokeopacity{0.700000}%
\pgfsetdash{}{0pt}%
\pgfpathmoveto{\pgfqpoint{9.695518in}{1.589233in}}%
\pgfpathcurveto{\pgfqpoint{9.700562in}{1.589233in}}{\pgfqpoint{9.705399in}{1.591237in}}{\pgfqpoint{9.708966in}{1.594803in}}%
\pgfpathcurveto{\pgfqpoint{9.712532in}{1.598369in}}{\pgfqpoint{9.714536in}{1.603207in}}{\pgfqpoint{9.714536in}{1.608251in}}%
\pgfpathcurveto{\pgfqpoint{9.714536in}{1.613294in}}{\pgfqpoint{9.712532in}{1.618132in}}{\pgfqpoint{9.708966in}{1.621699in}}%
\pgfpathcurveto{\pgfqpoint{9.705399in}{1.625265in}}{\pgfqpoint{9.700562in}{1.627269in}}{\pgfqpoint{9.695518in}{1.627269in}}%
\pgfpathcurveto{\pgfqpoint{9.690474in}{1.627269in}}{\pgfqpoint{9.685636in}{1.625265in}}{\pgfqpoint{9.682070in}{1.621699in}}%
\pgfpathcurveto{\pgfqpoint{9.678504in}{1.618132in}}{\pgfqpoint{9.676500in}{1.613294in}}{\pgfqpoint{9.676500in}{1.608251in}}%
\pgfpathcurveto{\pgfqpoint{9.676500in}{1.603207in}}{\pgfqpoint{9.678504in}{1.598369in}}{\pgfqpoint{9.682070in}{1.594803in}}%
\pgfpathcurveto{\pgfqpoint{9.685636in}{1.591237in}}{\pgfqpoint{9.690474in}{1.589233in}}{\pgfqpoint{9.695518in}{1.589233in}}%
\pgfpathclose%
\pgfusepath{fill}%
\end{pgfscope}%
\begin{pgfscope}%
\pgfpathrectangle{\pgfqpoint{6.572727in}{0.474100in}}{\pgfqpoint{4.227273in}{3.318700in}}%
\pgfusepath{clip}%
\pgfsetbuttcap%
\pgfsetroundjoin%
\definecolor{currentfill}{rgb}{0.127568,0.566949,0.550556}%
\pgfsetfillcolor{currentfill}%
\pgfsetfillopacity{0.700000}%
\pgfsetlinewidth{0.000000pt}%
\definecolor{currentstroke}{rgb}{0.000000,0.000000,0.000000}%
\pgfsetstrokecolor{currentstroke}%
\pgfsetstrokeopacity{0.700000}%
\pgfsetdash{}{0pt}%
\pgfpathmoveto{\pgfqpoint{9.966105in}{1.089835in}}%
\pgfpathcurveto{\pgfqpoint{9.971149in}{1.089835in}}{\pgfqpoint{9.975986in}{1.091839in}}{\pgfqpoint{9.979553in}{1.095405in}}%
\pgfpathcurveto{\pgfqpoint{9.983119in}{1.098972in}}{\pgfqpoint{9.985123in}{1.103809in}}{\pgfqpoint{9.985123in}{1.108853in}}%
\pgfpathcurveto{\pgfqpoint{9.985123in}{1.113897in}}{\pgfqpoint{9.983119in}{1.118734in}}{\pgfqpoint{9.979553in}{1.122301in}}%
\pgfpathcurveto{\pgfqpoint{9.975986in}{1.125867in}}{\pgfqpoint{9.971149in}{1.127871in}}{\pgfqpoint{9.966105in}{1.127871in}}%
\pgfpathcurveto{\pgfqpoint{9.961061in}{1.127871in}}{\pgfqpoint{9.956224in}{1.125867in}}{\pgfqpoint{9.952657in}{1.122301in}}%
\pgfpathcurveto{\pgfqpoint{9.949091in}{1.118734in}}{\pgfqpoint{9.947087in}{1.113897in}}{\pgfqpoint{9.947087in}{1.108853in}}%
\pgfpathcurveto{\pgfqpoint{9.947087in}{1.103809in}}{\pgfqpoint{9.949091in}{1.098972in}}{\pgfqpoint{9.952657in}{1.095405in}}%
\pgfpathcurveto{\pgfqpoint{9.956224in}{1.091839in}}{\pgfqpoint{9.961061in}{1.089835in}}{\pgfqpoint{9.966105in}{1.089835in}}%
\pgfpathclose%
\pgfusepath{fill}%
\end{pgfscope}%
\begin{pgfscope}%
\pgfpathrectangle{\pgfqpoint{6.572727in}{0.474100in}}{\pgfqpoint{4.227273in}{3.318700in}}%
\pgfusepath{clip}%
\pgfsetbuttcap%
\pgfsetroundjoin%
\definecolor{currentfill}{rgb}{0.127568,0.566949,0.550556}%
\pgfsetfillcolor{currentfill}%
\pgfsetfillopacity{0.700000}%
\pgfsetlinewidth{0.000000pt}%
\definecolor{currentstroke}{rgb}{0.000000,0.000000,0.000000}%
\pgfsetstrokecolor{currentstroke}%
\pgfsetstrokeopacity{0.700000}%
\pgfsetdash{}{0pt}%
\pgfpathmoveto{\pgfqpoint{9.973458in}{1.738924in}}%
\pgfpathcurveto{\pgfqpoint{9.978502in}{1.738924in}}{\pgfqpoint{9.983339in}{1.740927in}}{\pgfqpoint{9.986906in}{1.744494in}}%
\pgfpathcurveto{\pgfqpoint{9.990472in}{1.748060in}}{\pgfqpoint{9.992476in}{1.752898in}}{\pgfqpoint{9.992476in}{1.757942in}}%
\pgfpathcurveto{\pgfqpoint{9.992476in}{1.762985in}}{\pgfqpoint{9.990472in}{1.767823in}}{\pgfqpoint{9.986906in}{1.771390in}}%
\pgfpathcurveto{\pgfqpoint{9.983339in}{1.774956in}}{\pgfqpoint{9.978502in}{1.776960in}}{\pgfqpoint{9.973458in}{1.776960in}}%
\pgfpathcurveto{\pgfqpoint{9.968414in}{1.776960in}}{\pgfqpoint{9.963577in}{1.774956in}}{\pgfqpoint{9.960010in}{1.771390in}}%
\pgfpathcurveto{\pgfqpoint{9.956444in}{1.767823in}}{\pgfqpoint{9.954440in}{1.762985in}}{\pgfqpoint{9.954440in}{1.757942in}}%
\pgfpathcurveto{\pgfqpoint{9.954440in}{1.752898in}}{\pgfqpoint{9.956444in}{1.748060in}}{\pgfqpoint{9.960010in}{1.744494in}}%
\pgfpathcurveto{\pgfqpoint{9.963577in}{1.740927in}}{\pgfqpoint{9.968414in}{1.738924in}}{\pgfqpoint{9.973458in}{1.738924in}}%
\pgfpathclose%
\pgfusepath{fill}%
\end{pgfscope}%
\begin{pgfscope}%
\pgfpathrectangle{\pgfqpoint{6.572727in}{0.474100in}}{\pgfqpoint{4.227273in}{3.318700in}}%
\pgfusepath{clip}%
\pgfsetbuttcap%
\pgfsetroundjoin%
\definecolor{currentfill}{rgb}{0.267004,0.004874,0.329415}%
\pgfsetfillcolor{currentfill}%
\pgfsetfillopacity{0.700000}%
\pgfsetlinewidth{0.000000pt}%
\definecolor{currentstroke}{rgb}{0.000000,0.000000,0.000000}%
\pgfsetstrokecolor{currentstroke}%
\pgfsetstrokeopacity{0.700000}%
\pgfsetdash{}{0pt}%
\pgfpathmoveto{\pgfqpoint{7.477651in}{1.358210in}}%
\pgfpathcurveto{\pgfqpoint{7.482694in}{1.358210in}}{\pgfqpoint{7.487532in}{1.360214in}}{\pgfqpoint{7.491098in}{1.363780in}}%
\pgfpathcurveto{\pgfqpoint{7.494665in}{1.367347in}}{\pgfqpoint{7.496669in}{1.372184in}}{\pgfqpoint{7.496669in}{1.377228in}}%
\pgfpathcurveto{\pgfqpoint{7.496669in}{1.382272in}}{\pgfqpoint{7.494665in}{1.387109in}}{\pgfqpoint{7.491098in}{1.390676in}}%
\pgfpathcurveto{\pgfqpoint{7.487532in}{1.394242in}}{\pgfqpoint{7.482694in}{1.396246in}}{\pgfqpoint{7.477651in}{1.396246in}}%
\pgfpathcurveto{\pgfqpoint{7.472607in}{1.396246in}}{\pgfqpoint{7.467769in}{1.394242in}}{\pgfqpoint{7.464203in}{1.390676in}}%
\pgfpathcurveto{\pgfqpoint{7.460636in}{1.387109in}}{\pgfqpoint{7.458632in}{1.382272in}}{\pgfqpoint{7.458632in}{1.377228in}}%
\pgfpathcurveto{\pgfqpoint{7.458632in}{1.372184in}}{\pgfqpoint{7.460636in}{1.367347in}}{\pgfqpoint{7.464203in}{1.363780in}}%
\pgfpathcurveto{\pgfqpoint{7.467769in}{1.360214in}}{\pgfqpoint{7.472607in}{1.358210in}}{\pgfqpoint{7.477651in}{1.358210in}}%
\pgfpathclose%
\pgfusepath{fill}%
\end{pgfscope}%
\begin{pgfscope}%
\pgfpathrectangle{\pgfqpoint{6.572727in}{0.474100in}}{\pgfqpoint{4.227273in}{3.318700in}}%
\pgfusepath{clip}%
\pgfsetbuttcap%
\pgfsetroundjoin%
\definecolor{currentfill}{rgb}{0.993248,0.906157,0.143936}%
\pgfsetfillcolor{currentfill}%
\pgfsetfillopacity{0.700000}%
\pgfsetlinewidth{0.000000pt}%
\definecolor{currentstroke}{rgb}{0.000000,0.000000,0.000000}%
\pgfsetstrokecolor{currentstroke}%
\pgfsetstrokeopacity{0.700000}%
\pgfsetdash{}{0pt}%
\pgfpathmoveto{\pgfqpoint{7.599573in}{2.965168in}}%
\pgfpathcurveto{\pgfqpoint{7.604616in}{2.965168in}}{\pgfqpoint{7.609454in}{2.967172in}}{\pgfqpoint{7.613020in}{2.970739in}}%
\pgfpathcurveto{\pgfqpoint{7.616587in}{2.974305in}}{\pgfqpoint{7.618591in}{2.979143in}}{\pgfqpoint{7.618591in}{2.984187in}}%
\pgfpathcurveto{\pgfqpoint{7.618591in}{2.989230in}}{\pgfqpoint{7.616587in}{2.994068in}}{\pgfqpoint{7.613020in}{2.997634in}}%
\pgfpathcurveto{\pgfqpoint{7.609454in}{3.001201in}}{\pgfqpoint{7.604616in}{3.003205in}}{\pgfqpoint{7.599573in}{3.003205in}}%
\pgfpathcurveto{\pgfqpoint{7.594529in}{3.003205in}}{\pgfqpoint{7.589691in}{3.001201in}}{\pgfqpoint{7.586125in}{2.997634in}}%
\pgfpathcurveto{\pgfqpoint{7.582558in}{2.994068in}}{\pgfqpoint{7.580554in}{2.989230in}}{\pgfqpoint{7.580554in}{2.984187in}}%
\pgfpathcurveto{\pgfqpoint{7.580554in}{2.979143in}}{\pgfqpoint{7.582558in}{2.974305in}}{\pgfqpoint{7.586125in}{2.970739in}}%
\pgfpathcurveto{\pgfqpoint{7.589691in}{2.967172in}}{\pgfqpoint{7.594529in}{2.965168in}}{\pgfqpoint{7.599573in}{2.965168in}}%
\pgfpathclose%
\pgfusepath{fill}%
\end{pgfscope}%
\begin{pgfscope}%
\pgfpathrectangle{\pgfqpoint{6.572727in}{0.474100in}}{\pgfqpoint{4.227273in}{3.318700in}}%
\pgfusepath{clip}%
\pgfsetbuttcap%
\pgfsetroundjoin%
\definecolor{currentfill}{rgb}{0.127568,0.566949,0.550556}%
\pgfsetfillcolor{currentfill}%
\pgfsetfillopacity{0.700000}%
\pgfsetlinewidth{0.000000pt}%
\definecolor{currentstroke}{rgb}{0.000000,0.000000,0.000000}%
\pgfsetstrokecolor{currentstroke}%
\pgfsetstrokeopacity{0.700000}%
\pgfsetdash{}{0pt}%
\pgfpathmoveto{\pgfqpoint{9.608392in}{1.238244in}}%
\pgfpathcurveto{\pgfqpoint{9.613436in}{1.238244in}}{\pgfqpoint{9.618274in}{1.240248in}}{\pgfqpoint{9.621840in}{1.243815in}}%
\pgfpathcurveto{\pgfqpoint{9.625406in}{1.247381in}}{\pgfqpoint{9.627410in}{1.252219in}}{\pgfqpoint{9.627410in}{1.257262in}}%
\pgfpathcurveto{\pgfqpoint{9.627410in}{1.262306in}}{\pgfqpoint{9.625406in}{1.267144in}}{\pgfqpoint{9.621840in}{1.270710in}}%
\pgfpathcurveto{\pgfqpoint{9.618274in}{1.274277in}}{\pgfqpoint{9.613436in}{1.276281in}}{\pgfqpoint{9.608392in}{1.276281in}}%
\pgfpathcurveto{\pgfqpoint{9.603348in}{1.276281in}}{\pgfqpoint{9.598511in}{1.274277in}}{\pgfqpoint{9.594944in}{1.270710in}}%
\pgfpathcurveto{\pgfqpoint{9.591378in}{1.267144in}}{\pgfqpoint{9.589374in}{1.262306in}}{\pgfqpoint{9.589374in}{1.257262in}}%
\pgfpathcurveto{\pgfqpoint{9.589374in}{1.252219in}}{\pgfqpoint{9.591378in}{1.247381in}}{\pgfqpoint{9.594944in}{1.243815in}}%
\pgfpathcurveto{\pgfqpoint{9.598511in}{1.240248in}}{\pgfqpoint{9.603348in}{1.238244in}}{\pgfqpoint{9.608392in}{1.238244in}}%
\pgfpathclose%
\pgfusepath{fill}%
\end{pgfscope}%
\begin{pgfscope}%
\pgfpathrectangle{\pgfqpoint{6.572727in}{0.474100in}}{\pgfqpoint{4.227273in}{3.318700in}}%
\pgfusepath{clip}%
\pgfsetbuttcap%
\pgfsetroundjoin%
\definecolor{currentfill}{rgb}{0.127568,0.566949,0.550556}%
\pgfsetfillcolor{currentfill}%
\pgfsetfillopacity{0.700000}%
\pgfsetlinewidth{0.000000pt}%
\definecolor{currentstroke}{rgb}{0.000000,0.000000,0.000000}%
\pgfsetstrokecolor{currentstroke}%
\pgfsetstrokeopacity{0.700000}%
\pgfsetdash{}{0pt}%
\pgfpathmoveto{\pgfqpoint{9.267081in}{1.716430in}}%
\pgfpathcurveto{\pgfqpoint{9.272125in}{1.716430in}}{\pgfqpoint{9.276962in}{1.718434in}}{\pgfqpoint{9.280529in}{1.722001in}}%
\pgfpathcurveto{\pgfqpoint{9.284095in}{1.725567in}}{\pgfqpoint{9.286099in}{1.730405in}}{\pgfqpoint{9.286099in}{1.735449in}}%
\pgfpathcurveto{\pgfqpoint{9.286099in}{1.740492in}}{\pgfqpoint{9.284095in}{1.745330in}}{\pgfqpoint{9.280529in}{1.748897in}}%
\pgfpathcurveto{\pgfqpoint{9.276962in}{1.752463in}}{\pgfqpoint{9.272125in}{1.754467in}}{\pgfqpoint{9.267081in}{1.754467in}}%
\pgfpathcurveto{\pgfqpoint{9.262037in}{1.754467in}}{\pgfqpoint{9.257199in}{1.752463in}}{\pgfqpoint{9.253633in}{1.748897in}}%
\pgfpathcurveto{\pgfqpoint{9.250067in}{1.745330in}}{\pgfqpoint{9.248063in}{1.740492in}}{\pgfqpoint{9.248063in}{1.735449in}}%
\pgfpathcurveto{\pgfqpoint{9.248063in}{1.730405in}}{\pgfqpoint{9.250067in}{1.725567in}}{\pgfqpoint{9.253633in}{1.722001in}}%
\pgfpathcurveto{\pgfqpoint{9.257199in}{1.718434in}}{\pgfqpoint{9.262037in}{1.716430in}}{\pgfqpoint{9.267081in}{1.716430in}}%
\pgfpathclose%
\pgfusepath{fill}%
\end{pgfscope}%
\begin{pgfscope}%
\pgfpathrectangle{\pgfqpoint{6.572727in}{0.474100in}}{\pgfqpoint{4.227273in}{3.318700in}}%
\pgfusepath{clip}%
\pgfsetbuttcap%
\pgfsetroundjoin%
\definecolor{currentfill}{rgb}{0.127568,0.566949,0.550556}%
\pgfsetfillcolor{currentfill}%
\pgfsetfillopacity{0.700000}%
\pgfsetlinewidth{0.000000pt}%
\definecolor{currentstroke}{rgb}{0.000000,0.000000,0.000000}%
\pgfsetstrokecolor{currentstroke}%
\pgfsetstrokeopacity{0.700000}%
\pgfsetdash{}{0pt}%
\pgfpathmoveto{\pgfqpoint{9.427330in}{1.235788in}}%
\pgfpathcurveto{\pgfqpoint{9.432374in}{1.235788in}}{\pgfqpoint{9.437212in}{1.237792in}}{\pgfqpoint{9.440778in}{1.241358in}}%
\pgfpathcurveto{\pgfqpoint{9.444344in}{1.244925in}}{\pgfqpoint{9.446348in}{1.249762in}}{\pgfqpoint{9.446348in}{1.254806in}}%
\pgfpathcurveto{\pgfqpoint{9.446348in}{1.259850in}}{\pgfqpoint{9.444344in}{1.264687in}}{\pgfqpoint{9.440778in}{1.268254in}}%
\pgfpathcurveto{\pgfqpoint{9.437212in}{1.271820in}}{\pgfqpoint{9.432374in}{1.273824in}}{\pgfqpoint{9.427330in}{1.273824in}}%
\pgfpathcurveto{\pgfqpoint{9.422287in}{1.273824in}}{\pgfqpoint{9.417449in}{1.271820in}}{\pgfqpoint{9.413882in}{1.268254in}}%
\pgfpathcurveto{\pgfqpoint{9.410316in}{1.264687in}}{\pgfqpoint{9.408312in}{1.259850in}}{\pgfqpoint{9.408312in}{1.254806in}}%
\pgfpathcurveto{\pgfqpoint{9.408312in}{1.249762in}}{\pgfqpoint{9.410316in}{1.244925in}}{\pgfqpoint{9.413882in}{1.241358in}}%
\pgfpathcurveto{\pgfqpoint{9.417449in}{1.237792in}}{\pgfqpoint{9.422287in}{1.235788in}}{\pgfqpoint{9.427330in}{1.235788in}}%
\pgfpathclose%
\pgfusepath{fill}%
\end{pgfscope}%
\begin{pgfscope}%
\pgfpathrectangle{\pgfqpoint{6.572727in}{0.474100in}}{\pgfqpoint{4.227273in}{3.318700in}}%
\pgfusepath{clip}%
\pgfsetbuttcap%
\pgfsetroundjoin%
\definecolor{currentfill}{rgb}{0.127568,0.566949,0.550556}%
\pgfsetfillcolor{currentfill}%
\pgfsetfillopacity{0.700000}%
\pgfsetlinewidth{0.000000pt}%
\definecolor{currentstroke}{rgb}{0.000000,0.000000,0.000000}%
\pgfsetstrokecolor{currentstroke}%
\pgfsetstrokeopacity{0.700000}%
\pgfsetdash{}{0pt}%
\pgfpathmoveto{\pgfqpoint{9.562831in}{1.527462in}}%
\pgfpathcurveto{\pgfqpoint{9.567874in}{1.527462in}}{\pgfqpoint{9.572712in}{1.529466in}}{\pgfqpoint{9.576279in}{1.533032in}}%
\pgfpathcurveto{\pgfqpoint{9.579845in}{1.536599in}}{\pgfqpoint{9.581849in}{1.541437in}}{\pgfqpoint{9.581849in}{1.546480in}}%
\pgfpathcurveto{\pgfqpoint{9.581849in}{1.551524in}}{\pgfqpoint{9.579845in}{1.556362in}}{\pgfqpoint{9.576279in}{1.559928in}}%
\pgfpathcurveto{\pgfqpoint{9.572712in}{1.563495in}}{\pgfqpoint{9.567874in}{1.565498in}}{\pgfqpoint{9.562831in}{1.565498in}}%
\pgfpathcurveto{\pgfqpoint{9.557787in}{1.565498in}}{\pgfqpoint{9.552949in}{1.563495in}}{\pgfqpoint{9.549383in}{1.559928in}}%
\pgfpathcurveto{\pgfqpoint{9.545816in}{1.556362in}}{\pgfqpoint{9.543813in}{1.551524in}}{\pgfqpoint{9.543813in}{1.546480in}}%
\pgfpathcurveto{\pgfqpoint{9.543813in}{1.541437in}}{\pgfqpoint{9.545816in}{1.536599in}}{\pgfqpoint{9.549383in}{1.533032in}}%
\pgfpathcurveto{\pgfqpoint{9.552949in}{1.529466in}}{\pgfqpoint{9.557787in}{1.527462in}}{\pgfqpoint{9.562831in}{1.527462in}}%
\pgfpathclose%
\pgfusepath{fill}%
\end{pgfscope}%
\begin{pgfscope}%
\pgfpathrectangle{\pgfqpoint{6.572727in}{0.474100in}}{\pgfqpoint{4.227273in}{3.318700in}}%
\pgfusepath{clip}%
\pgfsetbuttcap%
\pgfsetroundjoin%
\definecolor{currentfill}{rgb}{0.127568,0.566949,0.550556}%
\pgfsetfillcolor{currentfill}%
\pgfsetfillopacity{0.700000}%
\pgfsetlinewidth{0.000000pt}%
\definecolor{currentstroke}{rgb}{0.000000,0.000000,0.000000}%
\pgfsetstrokecolor{currentstroke}%
\pgfsetstrokeopacity{0.700000}%
\pgfsetdash{}{0pt}%
\pgfpathmoveto{\pgfqpoint{9.380313in}{1.627885in}}%
\pgfpathcurveto{\pgfqpoint{9.385357in}{1.627885in}}{\pgfqpoint{9.390194in}{1.629889in}}{\pgfqpoint{9.393761in}{1.633456in}}%
\pgfpathcurveto{\pgfqpoint{9.397327in}{1.637022in}}{\pgfqpoint{9.399331in}{1.641860in}}{\pgfqpoint{9.399331in}{1.646903in}}%
\pgfpathcurveto{\pgfqpoint{9.399331in}{1.651947in}}{\pgfqpoint{9.397327in}{1.656785in}}{\pgfqpoint{9.393761in}{1.660351in}}%
\pgfpathcurveto{\pgfqpoint{9.390194in}{1.663918in}}{\pgfqpoint{9.385357in}{1.665922in}}{\pgfqpoint{9.380313in}{1.665922in}}%
\pgfpathcurveto{\pgfqpoint{9.375269in}{1.665922in}}{\pgfqpoint{9.370431in}{1.663918in}}{\pgfqpoint{9.366865in}{1.660351in}}%
\pgfpathcurveto{\pgfqpoint{9.363299in}{1.656785in}}{\pgfqpoint{9.361295in}{1.651947in}}{\pgfqpoint{9.361295in}{1.646903in}}%
\pgfpathcurveto{\pgfqpoint{9.361295in}{1.641860in}}{\pgfqpoint{9.363299in}{1.637022in}}{\pgfqpoint{9.366865in}{1.633456in}}%
\pgfpathcurveto{\pgfqpoint{9.370431in}{1.629889in}}{\pgfqpoint{9.375269in}{1.627885in}}{\pgfqpoint{9.380313in}{1.627885in}}%
\pgfpathclose%
\pgfusepath{fill}%
\end{pgfscope}%
\begin{pgfscope}%
\pgfpathrectangle{\pgfqpoint{6.572727in}{0.474100in}}{\pgfqpoint{4.227273in}{3.318700in}}%
\pgfusepath{clip}%
\pgfsetbuttcap%
\pgfsetroundjoin%
\definecolor{currentfill}{rgb}{0.267004,0.004874,0.329415}%
\pgfsetfillcolor{currentfill}%
\pgfsetfillopacity{0.700000}%
\pgfsetlinewidth{0.000000pt}%
\definecolor{currentstroke}{rgb}{0.000000,0.000000,0.000000}%
\pgfsetstrokecolor{currentstroke}%
\pgfsetstrokeopacity{0.700000}%
\pgfsetdash{}{0pt}%
\pgfpathmoveto{\pgfqpoint{8.069414in}{0.999264in}}%
\pgfpathcurveto{\pgfqpoint{8.074458in}{0.999264in}}{\pgfqpoint{8.079296in}{1.001267in}}{\pgfqpoint{8.082862in}{1.004834in}}%
\pgfpathcurveto{\pgfqpoint{8.086429in}{1.008400in}}{\pgfqpoint{8.088432in}{1.013238in}}{\pgfqpoint{8.088432in}{1.018282in}}%
\pgfpathcurveto{\pgfqpoint{8.088432in}{1.023325in}}{\pgfqpoint{8.086429in}{1.028163in}}{\pgfqpoint{8.082862in}{1.031730in}}%
\pgfpathcurveto{\pgfqpoint{8.079296in}{1.035296in}}{\pgfqpoint{8.074458in}{1.037300in}}{\pgfqpoint{8.069414in}{1.037300in}}%
\pgfpathcurveto{\pgfqpoint{8.064371in}{1.037300in}}{\pgfqpoint{8.059533in}{1.035296in}}{\pgfqpoint{8.055966in}{1.031730in}}%
\pgfpathcurveto{\pgfqpoint{8.052400in}{1.028163in}}{\pgfqpoint{8.050396in}{1.023325in}}{\pgfqpoint{8.050396in}{1.018282in}}%
\pgfpathcurveto{\pgfqpoint{8.050396in}{1.013238in}}{\pgfqpoint{8.052400in}{1.008400in}}{\pgfqpoint{8.055966in}{1.004834in}}%
\pgfpathcurveto{\pgfqpoint{8.059533in}{1.001267in}}{\pgfqpoint{8.064371in}{0.999264in}}{\pgfqpoint{8.069414in}{0.999264in}}%
\pgfpathclose%
\pgfusepath{fill}%
\end{pgfscope}%
\begin{pgfscope}%
\pgfpathrectangle{\pgfqpoint{6.572727in}{0.474100in}}{\pgfqpoint{4.227273in}{3.318700in}}%
\pgfusepath{clip}%
\pgfsetbuttcap%
\pgfsetroundjoin%
\definecolor{currentfill}{rgb}{0.267004,0.004874,0.329415}%
\pgfsetfillcolor{currentfill}%
\pgfsetfillopacity{0.700000}%
\pgfsetlinewidth{0.000000pt}%
\definecolor{currentstroke}{rgb}{0.000000,0.000000,0.000000}%
\pgfsetstrokecolor{currentstroke}%
\pgfsetstrokeopacity{0.700000}%
\pgfsetdash{}{0pt}%
\pgfpathmoveto{\pgfqpoint{8.367846in}{1.773739in}}%
\pgfpathcurveto{\pgfqpoint{8.372890in}{1.773739in}}{\pgfqpoint{8.377728in}{1.775743in}}{\pgfqpoint{8.381294in}{1.779309in}}%
\pgfpathcurveto{\pgfqpoint{8.384860in}{1.782876in}}{\pgfqpoint{8.386864in}{1.787713in}}{\pgfqpoint{8.386864in}{1.792757in}}%
\pgfpathcurveto{\pgfqpoint{8.386864in}{1.797801in}}{\pgfqpoint{8.384860in}{1.802639in}}{\pgfqpoint{8.381294in}{1.806205in}}%
\pgfpathcurveto{\pgfqpoint{8.377728in}{1.809771in}}{\pgfqpoint{8.372890in}{1.811775in}}{\pgfqpoint{8.367846in}{1.811775in}}%
\pgfpathcurveto{\pgfqpoint{8.362802in}{1.811775in}}{\pgfqpoint{8.357965in}{1.809771in}}{\pgfqpoint{8.354398in}{1.806205in}}%
\pgfpathcurveto{\pgfqpoint{8.350832in}{1.802639in}}{\pgfqpoint{8.348828in}{1.797801in}}{\pgfqpoint{8.348828in}{1.792757in}}%
\pgfpathcurveto{\pgfqpoint{8.348828in}{1.787713in}}{\pgfqpoint{8.350832in}{1.782876in}}{\pgfqpoint{8.354398in}{1.779309in}}%
\pgfpathcurveto{\pgfqpoint{8.357965in}{1.775743in}}{\pgfqpoint{8.362802in}{1.773739in}}{\pgfqpoint{8.367846in}{1.773739in}}%
\pgfpathclose%
\pgfusepath{fill}%
\end{pgfscope}%
\begin{pgfscope}%
\pgfpathrectangle{\pgfqpoint{6.572727in}{0.474100in}}{\pgfqpoint{4.227273in}{3.318700in}}%
\pgfusepath{clip}%
\pgfsetbuttcap%
\pgfsetroundjoin%
\definecolor{currentfill}{rgb}{0.993248,0.906157,0.143936}%
\pgfsetfillcolor{currentfill}%
\pgfsetfillopacity{0.700000}%
\pgfsetlinewidth{0.000000pt}%
\definecolor{currentstroke}{rgb}{0.000000,0.000000,0.000000}%
\pgfsetstrokecolor{currentstroke}%
\pgfsetstrokeopacity{0.700000}%
\pgfsetdash{}{0pt}%
\pgfpathmoveto{\pgfqpoint{8.586375in}{3.009194in}}%
\pgfpathcurveto{\pgfqpoint{8.591418in}{3.009194in}}{\pgfqpoint{8.596256in}{3.011198in}}{\pgfqpoint{8.599823in}{3.014765in}}%
\pgfpathcurveto{\pgfqpoint{8.603389in}{3.018331in}}{\pgfqpoint{8.605393in}{3.023169in}}{\pgfqpoint{8.605393in}{3.028213in}}%
\pgfpathcurveto{\pgfqpoint{8.605393in}{3.033256in}}{\pgfqpoint{8.603389in}{3.038094in}}{\pgfqpoint{8.599823in}{3.041660in}}%
\pgfpathcurveto{\pgfqpoint{8.596256in}{3.045227in}}{\pgfqpoint{8.591418in}{3.047231in}}{\pgfqpoint{8.586375in}{3.047231in}}%
\pgfpathcurveto{\pgfqpoint{8.581331in}{3.047231in}}{\pgfqpoint{8.576493in}{3.045227in}}{\pgfqpoint{8.572927in}{3.041660in}}%
\pgfpathcurveto{\pgfqpoint{8.569361in}{3.038094in}}{\pgfqpoint{8.567357in}{3.033256in}}{\pgfqpoint{8.567357in}{3.028213in}}%
\pgfpathcurveto{\pgfqpoint{8.567357in}{3.023169in}}{\pgfqpoint{8.569361in}{3.018331in}}{\pgfqpoint{8.572927in}{3.014765in}}%
\pgfpathcurveto{\pgfqpoint{8.576493in}{3.011198in}}{\pgfqpoint{8.581331in}{3.009194in}}{\pgfqpoint{8.586375in}{3.009194in}}%
\pgfpathclose%
\pgfusepath{fill}%
\end{pgfscope}%
\begin{pgfscope}%
\pgfpathrectangle{\pgfqpoint{6.572727in}{0.474100in}}{\pgfqpoint{4.227273in}{3.318700in}}%
\pgfusepath{clip}%
\pgfsetbuttcap%
\pgfsetroundjoin%
\definecolor{currentfill}{rgb}{0.267004,0.004874,0.329415}%
\pgfsetfillcolor{currentfill}%
\pgfsetfillopacity{0.700000}%
\pgfsetlinewidth{0.000000pt}%
\definecolor{currentstroke}{rgb}{0.000000,0.000000,0.000000}%
\pgfsetstrokecolor{currentstroke}%
\pgfsetstrokeopacity{0.700000}%
\pgfsetdash{}{0pt}%
\pgfpathmoveto{\pgfqpoint{7.440425in}{1.897495in}}%
\pgfpathcurveto{\pgfqpoint{7.445468in}{1.897495in}}{\pgfqpoint{7.450306in}{1.899499in}}{\pgfqpoint{7.453872in}{1.903065in}}%
\pgfpathcurveto{\pgfqpoint{7.457439in}{1.906632in}}{\pgfqpoint{7.459443in}{1.911469in}}{\pgfqpoint{7.459443in}{1.916513in}}%
\pgfpathcurveto{\pgfqpoint{7.459443in}{1.921557in}}{\pgfqpoint{7.457439in}{1.926394in}}{\pgfqpoint{7.453872in}{1.929961in}}%
\pgfpathcurveto{\pgfqpoint{7.450306in}{1.933527in}}{\pgfqpoint{7.445468in}{1.935531in}}{\pgfqpoint{7.440425in}{1.935531in}}%
\pgfpathcurveto{\pgfqpoint{7.435381in}{1.935531in}}{\pgfqpoint{7.430543in}{1.933527in}}{\pgfqpoint{7.426977in}{1.929961in}}%
\pgfpathcurveto{\pgfqpoint{7.423410in}{1.926394in}}{\pgfqpoint{7.421406in}{1.921557in}}{\pgfqpoint{7.421406in}{1.916513in}}%
\pgfpathcurveto{\pgfqpoint{7.421406in}{1.911469in}}{\pgfqpoint{7.423410in}{1.906632in}}{\pgfqpoint{7.426977in}{1.903065in}}%
\pgfpathcurveto{\pgfqpoint{7.430543in}{1.899499in}}{\pgfqpoint{7.435381in}{1.897495in}}{\pgfqpoint{7.440425in}{1.897495in}}%
\pgfpathclose%
\pgfusepath{fill}%
\end{pgfscope}%
\begin{pgfscope}%
\pgfpathrectangle{\pgfqpoint{6.572727in}{0.474100in}}{\pgfqpoint{4.227273in}{3.318700in}}%
\pgfusepath{clip}%
\pgfsetbuttcap%
\pgfsetroundjoin%
\definecolor{currentfill}{rgb}{0.267004,0.004874,0.329415}%
\pgfsetfillcolor{currentfill}%
\pgfsetfillopacity{0.700000}%
\pgfsetlinewidth{0.000000pt}%
\definecolor{currentstroke}{rgb}{0.000000,0.000000,0.000000}%
\pgfsetstrokecolor{currentstroke}%
\pgfsetstrokeopacity{0.700000}%
\pgfsetdash{}{0pt}%
\pgfpathmoveto{\pgfqpoint{7.878370in}{1.419632in}}%
\pgfpathcurveto{\pgfqpoint{7.883413in}{1.419632in}}{\pgfqpoint{7.888251in}{1.421636in}}{\pgfqpoint{7.891818in}{1.425203in}}%
\pgfpathcurveto{\pgfqpoint{7.895384in}{1.428769in}}{\pgfqpoint{7.897388in}{1.433607in}}{\pgfqpoint{7.897388in}{1.438651in}}%
\pgfpathcurveto{\pgfqpoint{7.897388in}{1.443694in}}{\pgfqpoint{7.895384in}{1.448532in}}{\pgfqpoint{7.891818in}{1.452098in}}%
\pgfpathcurveto{\pgfqpoint{7.888251in}{1.455665in}}{\pgfqpoint{7.883413in}{1.457669in}}{\pgfqpoint{7.878370in}{1.457669in}}%
\pgfpathcurveto{\pgfqpoint{7.873326in}{1.457669in}}{\pgfqpoint{7.868488in}{1.455665in}}{\pgfqpoint{7.864922in}{1.452098in}}%
\pgfpathcurveto{\pgfqpoint{7.861356in}{1.448532in}}{\pgfqpoint{7.859352in}{1.443694in}}{\pgfqpoint{7.859352in}{1.438651in}}%
\pgfpathcurveto{\pgfqpoint{7.859352in}{1.433607in}}{\pgfqpoint{7.861356in}{1.428769in}}{\pgfqpoint{7.864922in}{1.425203in}}%
\pgfpathcurveto{\pgfqpoint{7.868488in}{1.421636in}}{\pgfqpoint{7.873326in}{1.419632in}}{\pgfqpoint{7.878370in}{1.419632in}}%
\pgfpathclose%
\pgfusepath{fill}%
\end{pgfscope}%
\begin{pgfscope}%
\pgfpathrectangle{\pgfqpoint{6.572727in}{0.474100in}}{\pgfqpoint{4.227273in}{3.318700in}}%
\pgfusepath{clip}%
\pgfsetbuttcap%
\pgfsetroundjoin%
\definecolor{currentfill}{rgb}{0.993248,0.906157,0.143936}%
\pgfsetfillcolor{currentfill}%
\pgfsetfillopacity{0.700000}%
\pgfsetlinewidth{0.000000pt}%
\definecolor{currentstroke}{rgb}{0.000000,0.000000,0.000000}%
\pgfsetstrokecolor{currentstroke}%
\pgfsetstrokeopacity{0.700000}%
\pgfsetdash{}{0pt}%
\pgfpathmoveto{\pgfqpoint{8.461573in}{2.691941in}}%
\pgfpathcurveto{\pgfqpoint{8.466616in}{2.691941in}}{\pgfqpoint{8.471454in}{2.693945in}}{\pgfqpoint{8.475021in}{2.697511in}}%
\pgfpathcurveto{\pgfqpoint{8.478587in}{2.701077in}}{\pgfqpoint{8.480591in}{2.705915in}}{\pgfqpoint{8.480591in}{2.710959in}}%
\pgfpathcurveto{\pgfqpoint{8.480591in}{2.716003in}}{\pgfqpoint{8.478587in}{2.720840in}}{\pgfqpoint{8.475021in}{2.724407in}}%
\pgfpathcurveto{\pgfqpoint{8.471454in}{2.727973in}}{\pgfqpoint{8.466616in}{2.729977in}}{\pgfqpoint{8.461573in}{2.729977in}}%
\pgfpathcurveto{\pgfqpoint{8.456529in}{2.729977in}}{\pgfqpoint{8.451691in}{2.727973in}}{\pgfqpoint{8.448125in}{2.724407in}}%
\pgfpathcurveto{\pgfqpoint{8.444558in}{2.720840in}}{\pgfqpoint{8.442555in}{2.716003in}}{\pgfqpoint{8.442555in}{2.710959in}}%
\pgfpathcurveto{\pgfqpoint{8.442555in}{2.705915in}}{\pgfqpoint{8.444558in}{2.701077in}}{\pgfqpoint{8.448125in}{2.697511in}}%
\pgfpathcurveto{\pgfqpoint{8.451691in}{2.693945in}}{\pgfqpoint{8.456529in}{2.691941in}}{\pgfqpoint{8.461573in}{2.691941in}}%
\pgfpathclose%
\pgfusepath{fill}%
\end{pgfscope}%
\begin{pgfscope}%
\pgfpathrectangle{\pgfqpoint{6.572727in}{0.474100in}}{\pgfqpoint{4.227273in}{3.318700in}}%
\pgfusepath{clip}%
\pgfsetbuttcap%
\pgfsetroundjoin%
\definecolor{currentfill}{rgb}{0.993248,0.906157,0.143936}%
\pgfsetfillcolor{currentfill}%
\pgfsetfillopacity{0.700000}%
\pgfsetlinewidth{0.000000pt}%
\definecolor{currentstroke}{rgb}{0.000000,0.000000,0.000000}%
\pgfsetstrokecolor{currentstroke}%
\pgfsetstrokeopacity{0.700000}%
\pgfsetdash{}{0pt}%
\pgfpathmoveto{\pgfqpoint{8.746234in}{2.582759in}}%
\pgfpathcurveto{\pgfqpoint{8.751278in}{2.582759in}}{\pgfqpoint{8.756115in}{2.584763in}}{\pgfqpoint{8.759682in}{2.588330in}}%
\pgfpathcurveto{\pgfqpoint{8.763248in}{2.591896in}}{\pgfqpoint{8.765252in}{2.596734in}}{\pgfqpoint{8.765252in}{2.601777in}}%
\pgfpathcurveto{\pgfqpoint{8.765252in}{2.606821in}}{\pgfqpoint{8.763248in}{2.611659in}}{\pgfqpoint{8.759682in}{2.615225in}}%
\pgfpathcurveto{\pgfqpoint{8.756115in}{2.618792in}}{\pgfqpoint{8.751278in}{2.620796in}}{\pgfqpoint{8.746234in}{2.620796in}}%
\pgfpathcurveto{\pgfqpoint{8.741190in}{2.620796in}}{\pgfqpoint{8.736353in}{2.618792in}}{\pgfqpoint{8.732786in}{2.615225in}}%
\pgfpathcurveto{\pgfqpoint{8.729220in}{2.611659in}}{\pgfqpoint{8.727216in}{2.606821in}}{\pgfqpoint{8.727216in}{2.601777in}}%
\pgfpathcurveto{\pgfqpoint{8.727216in}{2.596734in}}{\pgfqpoint{8.729220in}{2.591896in}}{\pgfqpoint{8.732786in}{2.588330in}}%
\pgfpathcurveto{\pgfqpoint{8.736353in}{2.584763in}}{\pgfqpoint{8.741190in}{2.582759in}}{\pgfqpoint{8.746234in}{2.582759in}}%
\pgfpathclose%
\pgfusepath{fill}%
\end{pgfscope}%
\begin{pgfscope}%
\pgfpathrectangle{\pgfqpoint{6.572727in}{0.474100in}}{\pgfqpoint{4.227273in}{3.318700in}}%
\pgfusepath{clip}%
\pgfsetbuttcap%
\pgfsetroundjoin%
\definecolor{currentfill}{rgb}{0.267004,0.004874,0.329415}%
\pgfsetfillcolor{currentfill}%
\pgfsetfillopacity{0.700000}%
\pgfsetlinewidth{0.000000pt}%
\definecolor{currentstroke}{rgb}{0.000000,0.000000,0.000000}%
\pgfsetstrokecolor{currentstroke}%
\pgfsetstrokeopacity{0.700000}%
\pgfsetdash{}{0pt}%
\pgfpathmoveto{\pgfqpoint{7.289971in}{1.980645in}}%
\pgfpathcurveto{\pgfqpoint{7.295014in}{1.980645in}}{\pgfqpoint{7.299852in}{1.982649in}}{\pgfqpoint{7.303419in}{1.986215in}}%
\pgfpathcurveto{\pgfqpoint{7.306985in}{1.989782in}}{\pgfqpoint{7.308989in}{1.994619in}}{\pgfqpoint{7.308989in}{1.999663in}}%
\pgfpathcurveto{\pgfqpoint{7.308989in}{2.004707in}}{\pgfqpoint{7.306985in}{2.009545in}}{\pgfqpoint{7.303419in}{2.013111in}}%
\pgfpathcurveto{\pgfqpoint{7.299852in}{2.016677in}}{\pgfqpoint{7.295014in}{2.018681in}}{\pgfqpoint{7.289971in}{2.018681in}}%
\pgfpathcurveto{\pgfqpoint{7.284927in}{2.018681in}}{\pgfqpoint{7.280089in}{2.016677in}}{\pgfqpoint{7.276523in}{2.013111in}}%
\pgfpathcurveto{\pgfqpoint{7.272957in}{2.009545in}}{\pgfqpoint{7.270953in}{2.004707in}}{\pgfqpoint{7.270953in}{1.999663in}}%
\pgfpathcurveto{\pgfqpoint{7.270953in}{1.994619in}}{\pgfqpoint{7.272957in}{1.989782in}}{\pgfqpoint{7.276523in}{1.986215in}}%
\pgfpathcurveto{\pgfqpoint{7.280089in}{1.982649in}}{\pgfqpoint{7.284927in}{1.980645in}}{\pgfqpoint{7.289971in}{1.980645in}}%
\pgfpathclose%
\pgfusepath{fill}%
\end{pgfscope}%
\begin{pgfscope}%
\pgfpathrectangle{\pgfqpoint{6.572727in}{0.474100in}}{\pgfqpoint{4.227273in}{3.318700in}}%
\pgfusepath{clip}%
\pgfsetbuttcap%
\pgfsetroundjoin%
\definecolor{currentfill}{rgb}{0.267004,0.004874,0.329415}%
\pgfsetfillcolor{currentfill}%
\pgfsetfillopacity{0.700000}%
\pgfsetlinewidth{0.000000pt}%
\definecolor{currentstroke}{rgb}{0.000000,0.000000,0.000000}%
\pgfsetstrokecolor{currentstroke}%
\pgfsetstrokeopacity{0.700000}%
\pgfsetdash{}{0pt}%
\pgfpathmoveto{\pgfqpoint{7.497976in}{1.766077in}}%
\pgfpathcurveto{\pgfqpoint{7.503019in}{1.766077in}}{\pgfqpoint{7.507857in}{1.768081in}}{\pgfqpoint{7.511424in}{1.771647in}}%
\pgfpathcurveto{\pgfqpoint{7.514990in}{1.775214in}}{\pgfqpoint{7.516994in}{1.780051in}}{\pgfqpoint{7.516994in}{1.785095in}}%
\pgfpathcurveto{\pgfqpoint{7.516994in}{1.790139in}}{\pgfqpoint{7.514990in}{1.794976in}}{\pgfqpoint{7.511424in}{1.798543in}}%
\pgfpathcurveto{\pgfqpoint{7.507857in}{1.802109in}}{\pgfqpoint{7.503019in}{1.804113in}}{\pgfqpoint{7.497976in}{1.804113in}}%
\pgfpathcurveto{\pgfqpoint{7.492932in}{1.804113in}}{\pgfqpoint{7.488094in}{1.802109in}}{\pgfqpoint{7.484528in}{1.798543in}}%
\pgfpathcurveto{\pgfqpoint{7.480961in}{1.794976in}}{\pgfqpoint{7.478958in}{1.790139in}}{\pgfqpoint{7.478958in}{1.785095in}}%
\pgfpathcurveto{\pgfqpoint{7.478958in}{1.780051in}}{\pgfqpoint{7.480961in}{1.775214in}}{\pgfqpoint{7.484528in}{1.771647in}}%
\pgfpathcurveto{\pgfqpoint{7.488094in}{1.768081in}}{\pgfqpoint{7.492932in}{1.766077in}}{\pgfqpoint{7.497976in}{1.766077in}}%
\pgfpathclose%
\pgfusepath{fill}%
\end{pgfscope}%
\begin{pgfscope}%
\pgfpathrectangle{\pgfqpoint{6.572727in}{0.474100in}}{\pgfqpoint{4.227273in}{3.318700in}}%
\pgfusepath{clip}%
\pgfsetbuttcap%
\pgfsetroundjoin%
\definecolor{currentfill}{rgb}{0.993248,0.906157,0.143936}%
\pgfsetfillcolor{currentfill}%
\pgfsetfillopacity{0.700000}%
\pgfsetlinewidth{0.000000pt}%
\definecolor{currentstroke}{rgb}{0.000000,0.000000,0.000000}%
\pgfsetstrokecolor{currentstroke}%
\pgfsetstrokeopacity{0.700000}%
\pgfsetdash{}{0pt}%
\pgfpathmoveto{\pgfqpoint{8.130568in}{2.846616in}}%
\pgfpathcurveto{\pgfqpoint{8.135612in}{2.846616in}}{\pgfqpoint{8.140450in}{2.848620in}}{\pgfqpoint{8.144016in}{2.852187in}}%
\pgfpathcurveto{\pgfqpoint{8.147583in}{2.855753in}}{\pgfqpoint{8.149586in}{2.860591in}}{\pgfqpoint{8.149586in}{2.865635in}}%
\pgfpathcurveto{\pgfqpoint{8.149586in}{2.870678in}}{\pgfqpoint{8.147583in}{2.875516in}}{\pgfqpoint{8.144016in}{2.879082in}}%
\pgfpathcurveto{\pgfqpoint{8.140450in}{2.882649in}}{\pgfqpoint{8.135612in}{2.884653in}}{\pgfqpoint{8.130568in}{2.884653in}}%
\pgfpathcurveto{\pgfqpoint{8.125525in}{2.884653in}}{\pgfqpoint{8.120687in}{2.882649in}}{\pgfqpoint{8.117120in}{2.879082in}}%
\pgfpathcurveto{\pgfqpoint{8.113554in}{2.875516in}}{\pgfqpoint{8.111550in}{2.870678in}}{\pgfqpoint{8.111550in}{2.865635in}}%
\pgfpathcurveto{\pgfqpoint{8.111550in}{2.860591in}}{\pgfqpoint{8.113554in}{2.855753in}}{\pgfqpoint{8.117120in}{2.852187in}}%
\pgfpathcurveto{\pgfqpoint{8.120687in}{2.848620in}}{\pgfqpoint{8.125525in}{2.846616in}}{\pgfqpoint{8.130568in}{2.846616in}}%
\pgfpathclose%
\pgfusepath{fill}%
\end{pgfscope}%
\begin{pgfscope}%
\pgfpathrectangle{\pgfqpoint{6.572727in}{0.474100in}}{\pgfqpoint{4.227273in}{3.318700in}}%
\pgfusepath{clip}%
\pgfsetbuttcap%
\pgfsetroundjoin%
\definecolor{currentfill}{rgb}{0.127568,0.566949,0.550556}%
\pgfsetfillcolor{currentfill}%
\pgfsetfillopacity{0.700000}%
\pgfsetlinewidth{0.000000pt}%
\definecolor{currentstroke}{rgb}{0.000000,0.000000,0.000000}%
\pgfsetstrokecolor{currentstroke}%
\pgfsetstrokeopacity{0.700000}%
\pgfsetdash{}{0pt}%
\pgfpathmoveto{\pgfqpoint{9.747110in}{1.775018in}}%
\pgfpathcurveto{\pgfqpoint{9.752153in}{1.775018in}}{\pgfqpoint{9.756991in}{1.777022in}}{\pgfqpoint{9.760558in}{1.780588in}}%
\pgfpathcurveto{\pgfqpoint{9.764124in}{1.784154in}}{\pgfqpoint{9.766128in}{1.788992in}}{\pgfqpoint{9.766128in}{1.794036in}}%
\pgfpathcurveto{\pgfqpoint{9.766128in}{1.799080in}}{\pgfqpoint{9.764124in}{1.803917in}}{\pgfqpoint{9.760558in}{1.807484in}}%
\pgfpathcurveto{\pgfqpoint{9.756991in}{1.811050in}}{\pgfqpoint{9.752153in}{1.813054in}}{\pgfqpoint{9.747110in}{1.813054in}}%
\pgfpathcurveto{\pgfqpoint{9.742066in}{1.813054in}}{\pgfqpoint{9.737228in}{1.811050in}}{\pgfqpoint{9.733662in}{1.807484in}}%
\pgfpathcurveto{\pgfqpoint{9.730095in}{1.803917in}}{\pgfqpoint{9.728092in}{1.799080in}}{\pgfqpoint{9.728092in}{1.794036in}}%
\pgfpathcurveto{\pgfqpoint{9.728092in}{1.788992in}}{\pgfqpoint{9.730095in}{1.784154in}}{\pgfqpoint{9.733662in}{1.780588in}}%
\pgfpathcurveto{\pgfqpoint{9.737228in}{1.777022in}}{\pgfqpoint{9.742066in}{1.775018in}}{\pgfqpoint{9.747110in}{1.775018in}}%
\pgfpathclose%
\pgfusepath{fill}%
\end{pgfscope}%
\begin{pgfscope}%
\pgfpathrectangle{\pgfqpoint{6.572727in}{0.474100in}}{\pgfqpoint{4.227273in}{3.318700in}}%
\pgfusepath{clip}%
\pgfsetbuttcap%
\pgfsetroundjoin%
\definecolor{currentfill}{rgb}{0.267004,0.004874,0.329415}%
\pgfsetfillcolor{currentfill}%
\pgfsetfillopacity{0.700000}%
\pgfsetlinewidth{0.000000pt}%
\definecolor{currentstroke}{rgb}{0.000000,0.000000,0.000000}%
\pgfsetstrokecolor{currentstroke}%
\pgfsetstrokeopacity{0.700000}%
\pgfsetdash{}{0pt}%
\pgfpathmoveto{\pgfqpoint{7.462407in}{1.478957in}}%
\pgfpathcurveto{\pgfqpoint{7.467451in}{1.478957in}}{\pgfqpoint{7.472288in}{1.480961in}}{\pgfqpoint{7.475855in}{1.484528in}}%
\pgfpathcurveto{\pgfqpoint{7.479421in}{1.488094in}}{\pgfqpoint{7.481425in}{1.492932in}}{\pgfqpoint{7.481425in}{1.497975in}}%
\pgfpathcurveto{\pgfqpoint{7.481425in}{1.503019in}}{\pgfqpoint{7.479421in}{1.507857in}}{\pgfqpoint{7.475855in}{1.511423in}}%
\pgfpathcurveto{\pgfqpoint{7.472288in}{1.514990in}}{\pgfqpoint{7.467451in}{1.516994in}}{\pgfqpoint{7.462407in}{1.516994in}}%
\pgfpathcurveto{\pgfqpoint{7.457363in}{1.516994in}}{\pgfqpoint{7.452526in}{1.514990in}}{\pgfqpoint{7.448959in}{1.511423in}}%
\pgfpathcurveto{\pgfqpoint{7.445393in}{1.507857in}}{\pgfqpoint{7.443389in}{1.503019in}}{\pgfqpoint{7.443389in}{1.497975in}}%
\pgfpathcurveto{\pgfqpoint{7.443389in}{1.492932in}}{\pgfqpoint{7.445393in}{1.488094in}}{\pgfqpoint{7.448959in}{1.484528in}}%
\pgfpathcurveto{\pgfqpoint{7.452526in}{1.480961in}}{\pgfqpoint{7.457363in}{1.478957in}}{\pgfqpoint{7.462407in}{1.478957in}}%
\pgfpathclose%
\pgfusepath{fill}%
\end{pgfscope}%
\begin{pgfscope}%
\pgfpathrectangle{\pgfqpoint{6.572727in}{0.474100in}}{\pgfqpoint{4.227273in}{3.318700in}}%
\pgfusepath{clip}%
\pgfsetbuttcap%
\pgfsetroundjoin%
\definecolor{currentfill}{rgb}{0.127568,0.566949,0.550556}%
\pgfsetfillcolor{currentfill}%
\pgfsetfillopacity{0.700000}%
\pgfsetlinewidth{0.000000pt}%
\definecolor{currentstroke}{rgb}{0.000000,0.000000,0.000000}%
\pgfsetstrokecolor{currentstroke}%
\pgfsetstrokeopacity{0.700000}%
\pgfsetdash{}{0pt}%
\pgfpathmoveto{\pgfqpoint{9.811303in}{0.704290in}}%
\pgfpathcurveto{\pgfqpoint{9.816347in}{0.704290in}}{\pgfqpoint{9.821185in}{0.706294in}}{\pgfqpoint{9.824751in}{0.709861in}}%
\pgfpathcurveto{\pgfqpoint{9.828318in}{0.713427in}}{\pgfqpoint{9.830322in}{0.718265in}}{\pgfqpoint{9.830322in}{0.723309in}}%
\pgfpathcurveto{\pgfqpoint{9.830322in}{0.728352in}}{\pgfqpoint{9.828318in}{0.733190in}}{\pgfqpoint{9.824751in}{0.736756in}}%
\pgfpathcurveto{\pgfqpoint{9.821185in}{0.740323in}}{\pgfqpoint{9.816347in}{0.742327in}}{\pgfqpoint{9.811303in}{0.742327in}}%
\pgfpathcurveto{\pgfqpoint{9.806260in}{0.742327in}}{\pgfqpoint{9.801422in}{0.740323in}}{\pgfqpoint{9.797856in}{0.736756in}}%
\pgfpathcurveto{\pgfqpoint{9.794289in}{0.733190in}}{\pgfqpoint{9.792285in}{0.728352in}}{\pgfqpoint{9.792285in}{0.723309in}}%
\pgfpathcurveto{\pgfqpoint{9.792285in}{0.718265in}}{\pgfqpoint{9.794289in}{0.713427in}}{\pgfqpoint{9.797856in}{0.709861in}}%
\pgfpathcurveto{\pgfqpoint{9.801422in}{0.706294in}}{\pgfqpoint{9.806260in}{0.704290in}}{\pgfqpoint{9.811303in}{0.704290in}}%
\pgfpathclose%
\pgfusepath{fill}%
\end{pgfscope}%
\begin{pgfscope}%
\pgfpathrectangle{\pgfqpoint{6.572727in}{0.474100in}}{\pgfqpoint{4.227273in}{3.318700in}}%
\pgfusepath{clip}%
\pgfsetbuttcap%
\pgfsetroundjoin%
\definecolor{currentfill}{rgb}{0.267004,0.004874,0.329415}%
\pgfsetfillcolor{currentfill}%
\pgfsetfillopacity{0.700000}%
\pgfsetlinewidth{0.000000pt}%
\definecolor{currentstroke}{rgb}{0.000000,0.000000,0.000000}%
\pgfsetstrokecolor{currentstroke}%
\pgfsetstrokeopacity{0.700000}%
\pgfsetdash{}{0pt}%
\pgfpathmoveto{\pgfqpoint{7.547652in}{1.714814in}}%
\pgfpathcurveto{\pgfqpoint{7.552696in}{1.714814in}}{\pgfqpoint{7.557534in}{1.716818in}}{\pgfqpoint{7.561100in}{1.720384in}}%
\pgfpathcurveto{\pgfqpoint{7.564667in}{1.723951in}}{\pgfqpoint{7.566671in}{1.728788in}}{\pgfqpoint{7.566671in}{1.733832in}}%
\pgfpathcurveto{\pgfqpoint{7.566671in}{1.738876in}}{\pgfqpoint{7.564667in}{1.743713in}}{\pgfqpoint{7.561100in}{1.747280in}}%
\pgfpathcurveto{\pgfqpoint{7.557534in}{1.750846in}}{\pgfqpoint{7.552696in}{1.752850in}}{\pgfqpoint{7.547652in}{1.752850in}}%
\pgfpathcurveto{\pgfqpoint{7.542609in}{1.752850in}}{\pgfqpoint{7.537771in}{1.750846in}}{\pgfqpoint{7.534205in}{1.747280in}}%
\pgfpathcurveto{\pgfqpoint{7.530638in}{1.743713in}}{\pgfqpoint{7.528634in}{1.738876in}}{\pgfqpoint{7.528634in}{1.733832in}}%
\pgfpathcurveto{\pgfqpoint{7.528634in}{1.728788in}}{\pgfqpoint{7.530638in}{1.723951in}}{\pgfqpoint{7.534205in}{1.720384in}}%
\pgfpathcurveto{\pgfqpoint{7.537771in}{1.716818in}}{\pgfqpoint{7.542609in}{1.714814in}}{\pgfqpoint{7.547652in}{1.714814in}}%
\pgfpathclose%
\pgfusepath{fill}%
\end{pgfscope}%
\begin{pgfscope}%
\pgfpathrectangle{\pgfqpoint{6.572727in}{0.474100in}}{\pgfqpoint{4.227273in}{3.318700in}}%
\pgfusepath{clip}%
\pgfsetbuttcap%
\pgfsetroundjoin%
\definecolor{currentfill}{rgb}{0.993248,0.906157,0.143936}%
\pgfsetfillcolor{currentfill}%
\pgfsetfillopacity{0.700000}%
\pgfsetlinewidth{0.000000pt}%
\definecolor{currentstroke}{rgb}{0.000000,0.000000,0.000000}%
\pgfsetstrokecolor{currentstroke}%
\pgfsetstrokeopacity{0.700000}%
\pgfsetdash{}{0pt}%
\pgfpathmoveto{\pgfqpoint{8.850450in}{2.869026in}}%
\pgfpathcurveto{\pgfqpoint{8.855493in}{2.869026in}}{\pgfqpoint{8.860331in}{2.871030in}}{\pgfqpoint{8.863898in}{2.874596in}}%
\pgfpathcurveto{\pgfqpoint{8.867464in}{2.878163in}}{\pgfqpoint{8.869468in}{2.883000in}}{\pgfqpoint{8.869468in}{2.888044in}}%
\pgfpathcurveto{\pgfqpoint{8.869468in}{2.893088in}}{\pgfqpoint{8.867464in}{2.897926in}}{\pgfqpoint{8.863898in}{2.901492in}}%
\pgfpathcurveto{\pgfqpoint{8.860331in}{2.905058in}}{\pgfqpoint{8.855493in}{2.907062in}}{\pgfqpoint{8.850450in}{2.907062in}}%
\pgfpathcurveto{\pgfqpoint{8.845406in}{2.907062in}}{\pgfqpoint{8.840568in}{2.905058in}}{\pgfqpoint{8.837002in}{2.901492in}}%
\pgfpathcurveto{\pgfqpoint{8.833435in}{2.897926in}}{\pgfqpoint{8.831432in}{2.893088in}}{\pgfqpoint{8.831432in}{2.888044in}}%
\pgfpathcurveto{\pgfqpoint{8.831432in}{2.883000in}}{\pgfqpoint{8.833435in}{2.878163in}}{\pgfqpoint{8.837002in}{2.874596in}}%
\pgfpathcurveto{\pgfqpoint{8.840568in}{2.871030in}}{\pgfqpoint{8.845406in}{2.869026in}}{\pgfqpoint{8.850450in}{2.869026in}}%
\pgfpathclose%
\pgfusepath{fill}%
\end{pgfscope}%
\begin{pgfscope}%
\pgfpathrectangle{\pgfqpoint{6.572727in}{0.474100in}}{\pgfqpoint{4.227273in}{3.318700in}}%
\pgfusepath{clip}%
\pgfsetbuttcap%
\pgfsetroundjoin%
\definecolor{currentfill}{rgb}{0.127568,0.566949,0.550556}%
\pgfsetfillcolor{currentfill}%
\pgfsetfillopacity{0.700000}%
\pgfsetlinewidth{0.000000pt}%
\definecolor{currentstroke}{rgb}{0.000000,0.000000,0.000000}%
\pgfsetstrokecolor{currentstroke}%
\pgfsetstrokeopacity{0.700000}%
\pgfsetdash{}{0pt}%
\pgfpathmoveto{\pgfqpoint{9.659783in}{1.737252in}}%
\pgfpathcurveto{\pgfqpoint{9.664826in}{1.737252in}}{\pgfqpoint{9.669664in}{1.739256in}}{\pgfqpoint{9.673230in}{1.742823in}}%
\pgfpathcurveto{\pgfqpoint{9.676797in}{1.746389in}}{\pgfqpoint{9.678801in}{1.751227in}}{\pgfqpoint{9.678801in}{1.756271in}}%
\pgfpathcurveto{\pgfqpoint{9.678801in}{1.761314in}}{\pgfqpoint{9.676797in}{1.766152in}}{\pgfqpoint{9.673230in}{1.769718in}}%
\pgfpathcurveto{\pgfqpoint{9.669664in}{1.773285in}}{\pgfqpoint{9.664826in}{1.775289in}}{\pgfqpoint{9.659783in}{1.775289in}}%
\pgfpathcurveto{\pgfqpoint{9.654739in}{1.775289in}}{\pgfqpoint{9.649901in}{1.773285in}}{\pgfqpoint{9.646335in}{1.769718in}}%
\pgfpathcurveto{\pgfqpoint{9.642768in}{1.766152in}}{\pgfqpoint{9.640764in}{1.761314in}}{\pgfqpoint{9.640764in}{1.756271in}}%
\pgfpathcurveto{\pgfqpoint{9.640764in}{1.751227in}}{\pgfqpoint{9.642768in}{1.746389in}}{\pgfqpoint{9.646335in}{1.742823in}}%
\pgfpathcurveto{\pgfqpoint{9.649901in}{1.739256in}}{\pgfqpoint{9.654739in}{1.737252in}}{\pgfqpoint{9.659783in}{1.737252in}}%
\pgfpathclose%
\pgfusepath{fill}%
\end{pgfscope}%
\begin{pgfscope}%
\pgfpathrectangle{\pgfqpoint{6.572727in}{0.474100in}}{\pgfqpoint{4.227273in}{3.318700in}}%
\pgfusepath{clip}%
\pgfsetbuttcap%
\pgfsetroundjoin%
\definecolor{currentfill}{rgb}{0.127568,0.566949,0.550556}%
\pgfsetfillcolor{currentfill}%
\pgfsetfillopacity{0.700000}%
\pgfsetlinewidth{0.000000pt}%
\definecolor{currentstroke}{rgb}{0.000000,0.000000,0.000000}%
\pgfsetstrokecolor{currentstroke}%
\pgfsetstrokeopacity{0.700000}%
\pgfsetdash{}{0pt}%
\pgfpathmoveto{\pgfqpoint{9.671296in}{0.892303in}}%
\pgfpathcurveto{\pgfqpoint{9.676339in}{0.892303in}}{\pgfqpoint{9.681177in}{0.894307in}}{\pgfqpoint{9.684743in}{0.897873in}}%
\pgfpathcurveto{\pgfqpoint{9.688310in}{0.901440in}}{\pgfqpoint{9.690314in}{0.906278in}}{\pgfqpoint{9.690314in}{0.911321in}}%
\pgfpathcurveto{\pgfqpoint{9.690314in}{0.916365in}}{\pgfqpoint{9.688310in}{0.921203in}}{\pgfqpoint{9.684743in}{0.924769in}}%
\pgfpathcurveto{\pgfqpoint{9.681177in}{0.928336in}}{\pgfqpoint{9.676339in}{0.930339in}}{\pgfqpoint{9.671296in}{0.930339in}}%
\pgfpathcurveto{\pgfqpoint{9.666252in}{0.930339in}}{\pgfqpoint{9.661414in}{0.928336in}}{\pgfqpoint{9.657848in}{0.924769in}}%
\pgfpathcurveto{\pgfqpoint{9.654281in}{0.921203in}}{\pgfqpoint{9.652277in}{0.916365in}}{\pgfqpoint{9.652277in}{0.911321in}}%
\pgfpathcurveto{\pgfqpoint{9.652277in}{0.906278in}}{\pgfqpoint{9.654281in}{0.901440in}}{\pgfqpoint{9.657848in}{0.897873in}}%
\pgfpathcurveto{\pgfqpoint{9.661414in}{0.894307in}}{\pgfqpoint{9.666252in}{0.892303in}}{\pgfqpoint{9.671296in}{0.892303in}}%
\pgfpathclose%
\pgfusepath{fill}%
\end{pgfscope}%
\begin{pgfscope}%
\pgfpathrectangle{\pgfqpoint{6.572727in}{0.474100in}}{\pgfqpoint{4.227273in}{3.318700in}}%
\pgfusepath{clip}%
\pgfsetbuttcap%
\pgfsetroundjoin%
\definecolor{currentfill}{rgb}{0.267004,0.004874,0.329415}%
\pgfsetfillcolor{currentfill}%
\pgfsetfillopacity{0.700000}%
\pgfsetlinewidth{0.000000pt}%
\definecolor{currentstroke}{rgb}{0.000000,0.000000,0.000000}%
\pgfsetstrokecolor{currentstroke}%
\pgfsetstrokeopacity{0.700000}%
\pgfsetdash{}{0pt}%
\pgfpathmoveto{\pgfqpoint{7.262959in}{2.226435in}}%
\pgfpathcurveto{\pgfqpoint{7.268003in}{2.226435in}}{\pgfqpoint{7.272841in}{2.228439in}}{\pgfqpoint{7.276407in}{2.232006in}}%
\pgfpathcurveto{\pgfqpoint{7.279974in}{2.235572in}}{\pgfqpoint{7.281977in}{2.240410in}}{\pgfqpoint{7.281977in}{2.245454in}}%
\pgfpathcurveto{\pgfqpoint{7.281977in}{2.250497in}}{\pgfqpoint{7.279974in}{2.255335in}}{\pgfqpoint{7.276407in}{2.258901in}}%
\pgfpathcurveto{\pgfqpoint{7.272841in}{2.262468in}}{\pgfqpoint{7.268003in}{2.264472in}}{\pgfqpoint{7.262959in}{2.264472in}}%
\pgfpathcurveto{\pgfqpoint{7.257916in}{2.264472in}}{\pgfqpoint{7.253078in}{2.262468in}}{\pgfqpoint{7.249511in}{2.258901in}}%
\pgfpathcurveto{\pgfqpoint{7.245945in}{2.255335in}}{\pgfqpoint{7.243941in}{2.250497in}}{\pgfqpoint{7.243941in}{2.245454in}}%
\pgfpathcurveto{\pgfqpoint{7.243941in}{2.240410in}}{\pgfqpoint{7.245945in}{2.235572in}}{\pgfqpoint{7.249511in}{2.232006in}}%
\pgfpathcurveto{\pgfqpoint{7.253078in}{2.228439in}}{\pgfqpoint{7.257916in}{2.226435in}}{\pgfqpoint{7.262959in}{2.226435in}}%
\pgfpathclose%
\pgfusepath{fill}%
\end{pgfscope}%
\begin{pgfscope}%
\pgfpathrectangle{\pgfqpoint{6.572727in}{0.474100in}}{\pgfqpoint{4.227273in}{3.318700in}}%
\pgfusepath{clip}%
\pgfsetbuttcap%
\pgfsetroundjoin%
\definecolor{currentfill}{rgb}{0.267004,0.004874,0.329415}%
\pgfsetfillcolor{currentfill}%
\pgfsetfillopacity{0.700000}%
\pgfsetlinewidth{0.000000pt}%
\definecolor{currentstroke}{rgb}{0.000000,0.000000,0.000000}%
\pgfsetstrokecolor{currentstroke}%
\pgfsetstrokeopacity{0.700000}%
\pgfsetdash{}{0pt}%
\pgfpathmoveto{\pgfqpoint{8.289269in}{1.385486in}}%
\pgfpathcurveto{\pgfqpoint{8.294313in}{1.385486in}}{\pgfqpoint{8.299151in}{1.387490in}}{\pgfqpoint{8.302717in}{1.391056in}}%
\pgfpathcurveto{\pgfqpoint{8.306284in}{1.394623in}}{\pgfqpoint{8.308288in}{1.399460in}}{\pgfqpoint{8.308288in}{1.404504in}}%
\pgfpathcurveto{\pgfqpoint{8.308288in}{1.409548in}}{\pgfqpoint{8.306284in}{1.414385in}}{\pgfqpoint{8.302717in}{1.417952in}}%
\pgfpathcurveto{\pgfqpoint{8.299151in}{1.421518in}}{\pgfqpoint{8.294313in}{1.423522in}}{\pgfqpoint{8.289269in}{1.423522in}}%
\pgfpathcurveto{\pgfqpoint{8.284226in}{1.423522in}}{\pgfqpoint{8.279388in}{1.421518in}}{\pgfqpoint{8.275822in}{1.417952in}}%
\pgfpathcurveto{\pgfqpoint{8.272255in}{1.414385in}}{\pgfqpoint{8.270251in}{1.409548in}}{\pgfqpoint{8.270251in}{1.404504in}}%
\pgfpathcurveto{\pgfqpoint{8.270251in}{1.399460in}}{\pgfqpoint{8.272255in}{1.394623in}}{\pgfqpoint{8.275822in}{1.391056in}}%
\pgfpathcurveto{\pgfqpoint{8.279388in}{1.387490in}}{\pgfqpoint{8.284226in}{1.385486in}}{\pgfqpoint{8.289269in}{1.385486in}}%
\pgfpathclose%
\pgfusepath{fill}%
\end{pgfscope}%
\begin{pgfscope}%
\pgfpathrectangle{\pgfqpoint{6.572727in}{0.474100in}}{\pgfqpoint{4.227273in}{3.318700in}}%
\pgfusepath{clip}%
\pgfsetbuttcap%
\pgfsetroundjoin%
\definecolor{currentfill}{rgb}{0.993248,0.906157,0.143936}%
\pgfsetfillcolor{currentfill}%
\pgfsetfillopacity{0.700000}%
\pgfsetlinewidth{0.000000pt}%
\definecolor{currentstroke}{rgb}{0.000000,0.000000,0.000000}%
\pgfsetstrokecolor{currentstroke}%
\pgfsetstrokeopacity{0.700000}%
\pgfsetdash{}{0pt}%
\pgfpathmoveto{\pgfqpoint{8.163009in}{2.507769in}}%
\pgfpathcurveto{\pgfqpoint{8.168052in}{2.507769in}}{\pgfqpoint{8.172890in}{2.509772in}}{\pgfqpoint{8.176457in}{2.513339in}}%
\pgfpathcurveto{\pgfqpoint{8.180023in}{2.516905in}}{\pgfqpoint{8.182027in}{2.521743in}}{\pgfqpoint{8.182027in}{2.526787in}}%
\pgfpathcurveto{\pgfqpoint{8.182027in}{2.531830in}}{\pgfqpoint{8.180023in}{2.536668in}}{\pgfqpoint{8.176457in}{2.540235in}}%
\pgfpathcurveto{\pgfqpoint{8.172890in}{2.543801in}}{\pgfqpoint{8.168052in}{2.545805in}}{\pgfqpoint{8.163009in}{2.545805in}}%
\pgfpathcurveto{\pgfqpoint{8.157965in}{2.545805in}}{\pgfqpoint{8.153127in}{2.543801in}}{\pgfqpoint{8.149561in}{2.540235in}}%
\pgfpathcurveto{\pgfqpoint{8.145994in}{2.536668in}}{\pgfqpoint{8.143991in}{2.531830in}}{\pgfqpoint{8.143991in}{2.526787in}}%
\pgfpathcurveto{\pgfqpoint{8.143991in}{2.521743in}}{\pgfqpoint{8.145994in}{2.516905in}}{\pgfqpoint{8.149561in}{2.513339in}}%
\pgfpathcurveto{\pgfqpoint{8.153127in}{2.509772in}}{\pgfqpoint{8.157965in}{2.507769in}}{\pgfqpoint{8.163009in}{2.507769in}}%
\pgfpathclose%
\pgfusepath{fill}%
\end{pgfscope}%
\begin{pgfscope}%
\pgfpathrectangle{\pgfqpoint{6.572727in}{0.474100in}}{\pgfqpoint{4.227273in}{3.318700in}}%
\pgfusepath{clip}%
\pgfsetbuttcap%
\pgfsetroundjoin%
\definecolor{currentfill}{rgb}{0.993248,0.906157,0.143936}%
\pgfsetfillcolor{currentfill}%
\pgfsetfillopacity{0.700000}%
\pgfsetlinewidth{0.000000pt}%
\definecolor{currentstroke}{rgb}{0.000000,0.000000,0.000000}%
\pgfsetstrokecolor{currentstroke}%
\pgfsetstrokeopacity{0.700000}%
\pgfsetdash{}{0pt}%
\pgfpathmoveto{\pgfqpoint{8.005111in}{2.639183in}}%
\pgfpathcurveto{\pgfqpoint{8.010155in}{2.639183in}}{\pgfqpoint{8.014993in}{2.641186in}}{\pgfqpoint{8.018559in}{2.644753in}}%
\pgfpathcurveto{\pgfqpoint{8.022126in}{2.648319in}}{\pgfqpoint{8.024129in}{2.653157in}}{\pgfqpoint{8.024129in}{2.658201in}}%
\pgfpathcurveto{\pgfqpoint{8.024129in}{2.663244in}}{\pgfqpoint{8.022126in}{2.668082in}}{\pgfqpoint{8.018559in}{2.671649in}}%
\pgfpathcurveto{\pgfqpoint{8.014993in}{2.675215in}}{\pgfqpoint{8.010155in}{2.677219in}}{\pgfqpoint{8.005111in}{2.677219in}}%
\pgfpathcurveto{\pgfqpoint{8.000068in}{2.677219in}}{\pgfqpoint{7.995230in}{2.675215in}}{\pgfqpoint{7.991663in}{2.671649in}}%
\pgfpathcurveto{\pgfqpoint{7.988097in}{2.668082in}}{\pgfqpoint{7.986093in}{2.663244in}}{\pgfqpoint{7.986093in}{2.658201in}}%
\pgfpathcurveto{\pgfqpoint{7.986093in}{2.653157in}}{\pgfqpoint{7.988097in}{2.648319in}}{\pgfqpoint{7.991663in}{2.644753in}}%
\pgfpathcurveto{\pgfqpoint{7.995230in}{2.641186in}}{\pgfqpoint{8.000068in}{2.639183in}}{\pgfqpoint{8.005111in}{2.639183in}}%
\pgfpathclose%
\pgfusepath{fill}%
\end{pgfscope}%
\begin{pgfscope}%
\pgfpathrectangle{\pgfqpoint{6.572727in}{0.474100in}}{\pgfqpoint{4.227273in}{3.318700in}}%
\pgfusepath{clip}%
\pgfsetbuttcap%
\pgfsetroundjoin%
\definecolor{currentfill}{rgb}{0.267004,0.004874,0.329415}%
\pgfsetfillcolor{currentfill}%
\pgfsetfillopacity{0.700000}%
\pgfsetlinewidth{0.000000pt}%
\definecolor{currentstroke}{rgb}{0.000000,0.000000,0.000000}%
\pgfsetstrokecolor{currentstroke}%
\pgfsetstrokeopacity{0.700000}%
\pgfsetdash{}{0pt}%
\pgfpathmoveto{\pgfqpoint{7.308244in}{1.984365in}}%
\pgfpathcurveto{\pgfqpoint{7.313287in}{1.984365in}}{\pgfqpoint{7.318125in}{1.986368in}}{\pgfqpoint{7.321692in}{1.989935in}}%
\pgfpathcurveto{\pgfqpoint{7.325258in}{1.993501in}}{\pgfqpoint{7.327262in}{1.998339in}}{\pgfqpoint{7.327262in}{2.003383in}}%
\pgfpathcurveto{\pgfqpoint{7.327262in}{2.008426in}}{\pgfqpoint{7.325258in}{2.013264in}}{\pgfqpoint{7.321692in}{2.016831in}}%
\pgfpathcurveto{\pgfqpoint{7.318125in}{2.020397in}}{\pgfqpoint{7.313287in}{2.022401in}}{\pgfqpoint{7.308244in}{2.022401in}}%
\pgfpathcurveto{\pgfqpoint{7.303200in}{2.022401in}}{\pgfqpoint{7.298362in}{2.020397in}}{\pgfqpoint{7.294796in}{2.016831in}}%
\pgfpathcurveto{\pgfqpoint{7.291229in}{2.013264in}}{\pgfqpoint{7.289226in}{2.008426in}}{\pgfqpoint{7.289226in}{2.003383in}}%
\pgfpathcurveto{\pgfqpoint{7.289226in}{1.998339in}}{\pgfqpoint{7.291229in}{1.993501in}}{\pgfqpoint{7.294796in}{1.989935in}}%
\pgfpathcurveto{\pgfqpoint{7.298362in}{1.986368in}}{\pgfqpoint{7.303200in}{1.984365in}}{\pgfqpoint{7.308244in}{1.984365in}}%
\pgfpathclose%
\pgfusepath{fill}%
\end{pgfscope}%
\begin{pgfscope}%
\pgfpathrectangle{\pgfqpoint{6.572727in}{0.474100in}}{\pgfqpoint{4.227273in}{3.318700in}}%
\pgfusepath{clip}%
\pgfsetbuttcap%
\pgfsetroundjoin%
\definecolor{currentfill}{rgb}{0.993248,0.906157,0.143936}%
\pgfsetfillcolor{currentfill}%
\pgfsetfillopacity{0.700000}%
\pgfsetlinewidth{0.000000pt}%
\definecolor{currentstroke}{rgb}{0.000000,0.000000,0.000000}%
\pgfsetstrokecolor{currentstroke}%
\pgfsetstrokeopacity{0.700000}%
\pgfsetdash{}{0pt}%
\pgfpathmoveto{\pgfqpoint{7.376047in}{2.820774in}}%
\pgfpathcurveto{\pgfqpoint{7.381090in}{2.820774in}}{\pgfqpoint{7.385928in}{2.822777in}}{\pgfqpoint{7.389495in}{2.826344in}}%
\pgfpathcurveto{\pgfqpoint{7.393061in}{2.829910in}}{\pgfqpoint{7.395065in}{2.834748in}}{\pgfqpoint{7.395065in}{2.839792in}}%
\pgfpathcurveto{\pgfqpoint{7.395065in}{2.844835in}}{\pgfqpoint{7.393061in}{2.849673in}}{\pgfqpoint{7.389495in}{2.853240in}}%
\pgfpathcurveto{\pgfqpoint{7.385928in}{2.856806in}}{\pgfqpoint{7.381090in}{2.858810in}}{\pgfqpoint{7.376047in}{2.858810in}}%
\pgfpathcurveto{\pgfqpoint{7.371003in}{2.858810in}}{\pgfqpoint{7.366165in}{2.856806in}}{\pgfqpoint{7.362599in}{2.853240in}}%
\pgfpathcurveto{\pgfqpoint{7.359032in}{2.849673in}}{\pgfqpoint{7.357029in}{2.844835in}}{\pgfqpoint{7.357029in}{2.839792in}}%
\pgfpathcurveto{\pgfqpoint{7.357029in}{2.834748in}}{\pgfqpoint{7.359032in}{2.829910in}}{\pgfqpoint{7.362599in}{2.826344in}}%
\pgfpathcurveto{\pgfqpoint{7.366165in}{2.822777in}}{\pgfqpoint{7.371003in}{2.820774in}}{\pgfqpoint{7.376047in}{2.820774in}}%
\pgfpathclose%
\pgfusepath{fill}%
\end{pgfscope}%
\begin{pgfscope}%
\pgfpathrectangle{\pgfqpoint{6.572727in}{0.474100in}}{\pgfqpoint{4.227273in}{3.318700in}}%
\pgfusepath{clip}%
\pgfsetbuttcap%
\pgfsetroundjoin%
\definecolor{currentfill}{rgb}{0.127568,0.566949,0.550556}%
\pgfsetfillcolor{currentfill}%
\pgfsetfillopacity{0.700000}%
\pgfsetlinewidth{0.000000pt}%
\definecolor{currentstroke}{rgb}{0.000000,0.000000,0.000000}%
\pgfsetstrokecolor{currentstroke}%
\pgfsetstrokeopacity{0.700000}%
\pgfsetdash{}{0pt}%
\pgfpathmoveto{\pgfqpoint{9.555297in}{1.832976in}}%
\pgfpathcurveto{\pgfqpoint{9.560341in}{1.832976in}}{\pgfqpoint{9.565179in}{1.834980in}}{\pgfqpoint{9.568745in}{1.838546in}}%
\pgfpathcurveto{\pgfqpoint{9.572312in}{1.842112in}}{\pgfqpoint{9.574315in}{1.846950in}}{\pgfqpoint{9.574315in}{1.851994in}}%
\pgfpathcurveto{\pgfqpoint{9.574315in}{1.857038in}}{\pgfqpoint{9.572312in}{1.861875in}}{\pgfqpoint{9.568745in}{1.865442in}}%
\pgfpathcurveto{\pgfqpoint{9.565179in}{1.869008in}}{\pgfqpoint{9.560341in}{1.871012in}}{\pgfqpoint{9.555297in}{1.871012in}}%
\pgfpathcurveto{\pgfqpoint{9.550254in}{1.871012in}}{\pgfqpoint{9.545416in}{1.869008in}}{\pgfqpoint{9.541849in}{1.865442in}}%
\pgfpathcurveto{\pgfqpoint{9.538283in}{1.861875in}}{\pgfqpoint{9.536279in}{1.857038in}}{\pgfqpoint{9.536279in}{1.851994in}}%
\pgfpathcurveto{\pgfqpoint{9.536279in}{1.846950in}}{\pgfqpoint{9.538283in}{1.842112in}}{\pgfqpoint{9.541849in}{1.838546in}}%
\pgfpathcurveto{\pgfqpoint{9.545416in}{1.834980in}}{\pgfqpoint{9.550254in}{1.832976in}}{\pgfqpoint{9.555297in}{1.832976in}}%
\pgfpathclose%
\pgfusepath{fill}%
\end{pgfscope}%
\begin{pgfscope}%
\pgfpathrectangle{\pgfqpoint{6.572727in}{0.474100in}}{\pgfqpoint{4.227273in}{3.318700in}}%
\pgfusepath{clip}%
\pgfsetbuttcap%
\pgfsetroundjoin%
\definecolor{currentfill}{rgb}{0.127568,0.566949,0.550556}%
\pgfsetfillcolor{currentfill}%
\pgfsetfillopacity{0.700000}%
\pgfsetlinewidth{0.000000pt}%
\definecolor{currentstroke}{rgb}{0.000000,0.000000,0.000000}%
\pgfsetstrokecolor{currentstroke}%
\pgfsetstrokeopacity{0.700000}%
\pgfsetdash{}{0pt}%
\pgfpathmoveto{\pgfqpoint{9.942156in}{1.618216in}}%
\pgfpathcurveto{\pgfqpoint{9.947199in}{1.618216in}}{\pgfqpoint{9.952037in}{1.620220in}}{\pgfqpoint{9.955604in}{1.623786in}}%
\pgfpathcurveto{\pgfqpoint{9.959170in}{1.627352in}}{\pgfqpoint{9.961174in}{1.632190in}}{\pgfqpoint{9.961174in}{1.637234in}}%
\pgfpathcurveto{\pgfqpoint{9.961174in}{1.642277in}}{\pgfqpoint{9.959170in}{1.647115in}}{\pgfqpoint{9.955604in}{1.650682in}}%
\pgfpathcurveto{\pgfqpoint{9.952037in}{1.654248in}}{\pgfqpoint{9.947199in}{1.656252in}}{\pgfqpoint{9.942156in}{1.656252in}}%
\pgfpathcurveto{\pgfqpoint{9.937112in}{1.656252in}}{\pgfqpoint{9.932274in}{1.654248in}}{\pgfqpoint{9.928708in}{1.650682in}}%
\pgfpathcurveto{\pgfqpoint{9.925141in}{1.647115in}}{\pgfqpoint{9.923138in}{1.642277in}}{\pgfqpoint{9.923138in}{1.637234in}}%
\pgfpathcurveto{\pgfqpoint{9.923138in}{1.632190in}}{\pgfqpoint{9.925141in}{1.627352in}}{\pgfqpoint{9.928708in}{1.623786in}}%
\pgfpathcurveto{\pgfqpoint{9.932274in}{1.620220in}}{\pgfqpoint{9.937112in}{1.618216in}}{\pgfqpoint{9.942156in}{1.618216in}}%
\pgfpathclose%
\pgfusepath{fill}%
\end{pgfscope}%
\begin{pgfscope}%
\pgfpathrectangle{\pgfqpoint{6.572727in}{0.474100in}}{\pgfqpoint{4.227273in}{3.318700in}}%
\pgfusepath{clip}%
\pgfsetbuttcap%
\pgfsetroundjoin%
\definecolor{currentfill}{rgb}{0.267004,0.004874,0.329415}%
\pgfsetfillcolor{currentfill}%
\pgfsetfillopacity{0.700000}%
\pgfsetlinewidth{0.000000pt}%
\definecolor{currentstroke}{rgb}{0.000000,0.000000,0.000000}%
\pgfsetstrokecolor{currentstroke}%
\pgfsetstrokeopacity{0.700000}%
\pgfsetdash{}{0pt}%
\pgfpathmoveto{\pgfqpoint{7.526435in}{1.784472in}}%
\pgfpathcurveto{\pgfqpoint{7.531478in}{1.784472in}}{\pgfqpoint{7.536316in}{1.786476in}}{\pgfqpoint{7.539883in}{1.790042in}}%
\pgfpathcurveto{\pgfqpoint{7.543449in}{1.793609in}}{\pgfqpoint{7.545453in}{1.798446in}}{\pgfqpoint{7.545453in}{1.803490in}}%
\pgfpathcurveto{\pgfqpoint{7.545453in}{1.808534in}}{\pgfqpoint{7.543449in}{1.813372in}}{\pgfqpoint{7.539883in}{1.816938in}}%
\pgfpathcurveto{\pgfqpoint{7.536316in}{1.820504in}}{\pgfqpoint{7.531478in}{1.822508in}}{\pgfqpoint{7.526435in}{1.822508in}}%
\pgfpathcurveto{\pgfqpoint{7.521391in}{1.822508in}}{\pgfqpoint{7.516553in}{1.820504in}}{\pgfqpoint{7.512987in}{1.816938in}}%
\pgfpathcurveto{\pgfqpoint{7.509420in}{1.813372in}}{\pgfqpoint{7.507417in}{1.808534in}}{\pgfqpoint{7.507417in}{1.803490in}}%
\pgfpathcurveto{\pgfqpoint{7.507417in}{1.798446in}}{\pgfqpoint{7.509420in}{1.793609in}}{\pgfqpoint{7.512987in}{1.790042in}}%
\pgfpathcurveto{\pgfqpoint{7.516553in}{1.786476in}}{\pgfqpoint{7.521391in}{1.784472in}}{\pgfqpoint{7.526435in}{1.784472in}}%
\pgfpathclose%
\pgfusepath{fill}%
\end{pgfscope}%
\begin{pgfscope}%
\pgfpathrectangle{\pgfqpoint{6.572727in}{0.474100in}}{\pgfqpoint{4.227273in}{3.318700in}}%
\pgfusepath{clip}%
\pgfsetbuttcap%
\pgfsetroundjoin%
\definecolor{currentfill}{rgb}{0.267004,0.004874,0.329415}%
\pgfsetfillcolor{currentfill}%
\pgfsetfillopacity{0.700000}%
\pgfsetlinewidth{0.000000pt}%
\definecolor{currentstroke}{rgb}{0.000000,0.000000,0.000000}%
\pgfsetstrokecolor{currentstroke}%
\pgfsetstrokeopacity{0.700000}%
\pgfsetdash{}{0pt}%
\pgfpathmoveto{\pgfqpoint{7.469826in}{1.255008in}}%
\pgfpathcurveto{\pgfqpoint{7.474869in}{1.255008in}}{\pgfqpoint{7.479707in}{1.257012in}}{\pgfqpoint{7.483273in}{1.260578in}}%
\pgfpathcurveto{\pgfqpoint{7.486840in}{1.264145in}}{\pgfqpoint{7.488844in}{1.268982in}}{\pgfqpoint{7.488844in}{1.274026in}}%
\pgfpathcurveto{\pgfqpoint{7.488844in}{1.279070in}}{\pgfqpoint{7.486840in}{1.283907in}}{\pgfqpoint{7.483273in}{1.287474in}}%
\pgfpathcurveto{\pgfqpoint{7.479707in}{1.291040in}}{\pgfqpoint{7.474869in}{1.293044in}}{\pgfqpoint{7.469826in}{1.293044in}}%
\pgfpathcurveto{\pgfqpoint{7.464782in}{1.293044in}}{\pgfqpoint{7.459944in}{1.291040in}}{\pgfqpoint{7.456378in}{1.287474in}}%
\pgfpathcurveto{\pgfqpoint{7.452811in}{1.283907in}}{\pgfqpoint{7.450807in}{1.279070in}}{\pgfqpoint{7.450807in}{1.274026in}}%
\pgfpathcurveto{\pgfqpoint{7.450807in}{1.268982in}}{\pgfqpoint{7.452811in}{1.264145in}}{\pgfqpoint{7.456378in}{1.260578in}}%
\pgfpathcurveto{\pgfqpoint{7.459944in}{1.257012in}}{\pgfqpoint{7.464782in}{1.255008in}}{\pgfqpoint{7.469826in}{1.255008in}}%
\pgfpathclose%
\pgfusepath{fill}%
\end{pgfscope}%
\begin{pgfscope}%
\pgfpathrectangle{\pgfqpoint{6.572727in}{0.474100in}}{\pgfqpoint{4.227273in}{3.318700in}}%
\pgfusepath{clip}%
\pgfsetbuttcap%
\pgfsetroundjoin%
\definecolor{currentfill}{rgb}{0.127568,0.566949,0.550556}%
\pgfsetfillcolor{currentfill}%
\pgfsetfillopacity{0.700000}%
\pgfsetlinewidth{0.000000pt}%
\definecolor{currentstroke}{rgb}{0.000000,0.000000,0.000000}%
\pgfsetstrokecolor{currentstroke}%
\pgfsetstrokeopacity{0.700000}%
\pgfsetdash{}{0pt}%
\pgfpathmoveto{\pgfqpoint{9.474662in}{1.874821in}}%
\pgfpathcurveto{\pgfqpoint{9.479706in}{1.874821in}}{\pgfqpoint{9.484544in}{1.876825in}}{\pgfqpoint{9.488110in}{1.880391in}}%
\pgfpathcurveto{\pgfqpoint{9.491677in}{1.883958in}}{\pgfqpoint{9.493681in}{1.888796in}}{\pgfqpoint{9.493681in}{1.893839in}}%
\pgfpathcurveto{\pgfqpoint{9.493681in}{1.898883in}}{\pgfqpoint{9.491677in}{1.903721in}}{\pgfqpoint{9.488110in}{1.907287in}}%
\pgfpathcurveto{\pgfqpoint{9.484544in}{1.910854in}}{\pgfqpoint{9.479706in}{1.912857in}}{\pgfqpoint{9.474662in}{1.912857in}}%
\pgfpathcurveto{\pgfqpoint{9.469619in}{1.912857in}}{\pgfqpoint{9.464781in}{1.910854in}}{\pgfqpoint{9.461215in}{1.907287in}}%
\pgfpathcurveto{\pgfqpoint{9.457648in}{1.903721in}}{\pgfqpoint{9.455644in}{1.898883in}}{\pgfqpoint{9.455644in}{1.893839in}}%
\pgfpathcurveto{\pgfqpoint{9.455644in}{1.888796in}}{\pgfqpoint{9.457648in}{1.883958in}}{\pgfqpoint{9.461215in}{1.880391in}}%
\pgfpathcurveto{\pgfqpoint{9.464781in}{1.876825in}}{\pgfqpoint{9.469619in}{1.874821in}}{\pgfqpoint{9.474662in}{1.874821in}}%
\pgfpathclose%
\pgfusepath{fill}%
\end{pgfscope}%
\begin{pgfscope}%
\pgfpathrectangle{\pgfqpoint{6.572727in}{0.474100in}}{\pgfqpoint{4.227273in}{3.318700in}}%
\pgfusepath{clip}%
\pgfsetbuttcap%
\pgfsetroundjoin%
\definecolor{currentfill}{rgb}{0.993248,0.906157,0.143936}%
\pgfsetfillcolor{currentfill}%
\pgfsetfillopacity{0.700000}%
\pgfsetlinewidth{0.000000pt}%
\definecolor{currentstroke}{rgb}{0.000000,0.000000,0.000000}%
\pgfsetstrokecolor{currentstroke}%
\pgfsetstrokeopacity{0.700000}%
\pgfsetdash{}{0pt}%
\pgfpathmoveto{\pgfqpoint{7.870710in}{2.787848in}}%
\pgfpathcurveto{\pgfqpoint{7.875753in}{2.787848in}}{\pgfqpoint{7.880591in}{2.789851in}}{\pgfqpoint{7.884158in}{2.793418in}}%
\pgfpathcurveto{\pgfqpoint{7.887724in}{2.796984in}}{\pgfqpoint{7.889728in}{2.801822in}}{\pgfqpoint{7.889728in}{2.806866in}}%
\pgfpathcurveto{\pgfqpoint{7.889728in}{2.811909in}}{\pgfqpoint{7.887724in}{2.816747in}}{\pgfqpoint{7.884158in}{2.820314in}}%
\pgfpathcurveto{\pgfqpoint{7.880591in}{2.823880in}}{\pgfqpoint{7.875753in}{2.825884in}}{\pgfqpoint{7.870710in}{2.825884in}}%
\pgfpathcurveto{\pgfqpoint{7.865666in}{2.825884in}}{\pgfqpoint{7.860828in}{2.823880in}}{\pgfqpoint{7.857262in}{2.820314in}}%
\pgfpathcurveto{\pgfqpoint{7.853695in}{2.816747in}}{\pgfqpoint{7.851692in}{2.811909in}}{\pgfqpoint{7.851692in}{2.806866in}}%
\pgfpathcurveto{\pgfqpoint{7.851692in}{2.801822in}}{\pgfqpoint{7.853695in}{2.796984in}}{\pgfqpoint{7.857262in}{2.793418in}}%
\pgfpathcurveto{\pgfqpoint{7.860828in}{2.789851in}}{\pgfqpoint{7.865666in}{2.787848in}}{\pgfqpoint{7.870710in}{2.787848in}}%
\pgfpathclose%
\pgfusepath{fill}%
\end{pgfscope}%
\begin{pgfscope}%
\pgfpathrectangle{\pgfqpoint{6.572727in}{0.474100in}}{\pgfqpoint{4.227273in}{3.318700in}}%
\pgfusepath{clip}%
\pgfsetbuttcap%
\pgfsetroundjoin%
\definecolor{currentfill}{rgb}{0.993248,0.906157,0.143936}%
\pgfsetfillcolor{currentfill}%
\pgfsetfillopacity{0.700000}%
\pgfsetlinewidth{0.000000pt}%
\definecolor{currentstroke}{rgb}{0.000000,0.000000,0.000000}%
\pgfsetstrokecolor{currentstroke}%
\pgfsetstrokeopacity{0.700000}%
\pgfsetdash{}{0pt}%
\pgfpathmoveto{\pgfqpoint{8.498297in}{3.250508in}}%
\pgfpathcurveto{\pgfqpoint{8.503341in}{3.250508in}}{\pgfqpoint{8.508178in}{3.252512in}}{\pgfqpoint{8.511745in}{3.256078in}}%
\pgfpathcurveto{\pgfqpoint{8.515311in}{3.259644in}}{\pgfqpoint{8.517315in}{3.264482in}}{\pgfqpoint{8.517315in}{3.269526in}}%
\pgfpathcurveto{\pgfqpoint{8.517315in}{3.274570in}}{\pgfqpoint{8.515311in}{3.279407in}}{\pgfqpoint{8.511745in}{3.282974in}}%
\pgfpathcurveto{\pgfqpoint{8.508178in}{3.286540in}}{\pgfqpoint{8.503341in}{3.288544in}}{\pgfqpoint{8.498297in}{3.288544in}}%
\pgfpathcurveto{\pgfqpoint{8.493253in}{3.288544in}}{\pgfqpoint{8.488416in}{3.286540in}}{\pgfqpoint{8.484849in}{3.282974in}}%
\pgfpathcurveto{\pgfqpoint{8.481283in}{3.279407in}}{\pgfqpoint{8.479279in}{3.274570in}}{\pgfqpoint{8.479279in}{3.269526in}}%
\pgfpathcurveto{\pgfqpoint{8.479279in}{3.264482in}}{\pgfqpoint{8.481283in}{3.259644in}}{\pgfqpoint{8.484849in}{3.256078in}}%
\pgfpathcurveto{\pgfqpoint{8.488416in}{3.252512in}}{\pgfqpoint{8.493253in}{3.250508in}}{\pgfqpoint{8.498297in}{3.250508in}}%
\pgfpathclose%
\pgfusepath{fill}%
\end{pgfscope}%
\begin{pgfscope}%
\pgfpathrectangle{\pgfqpoint{6.572727in}{0.474100in}}{\pgfqpoint{4.227273in}{3.318700in}}%
\pgfusepath{clip}%
\pgfsetbuttcap%
\pgfsetroundjoin%
\definecolor{currentfill}{rgb}{0.267004,0.004874,0.329415}%
\pgfsetfillcolor{currentfill}%
\pgfsetfillopacity{0.700000}%
\pgfsetlinewidth{0.000000pt}%
\definecolor{currentstroke}{rgb}{0.000000,0.000000,0.000000}%
\pgfsetstrokecolor{currentstroke}%
\pgfsetstrokeopacity{0.700000}%
\pgfsetdash{}{0pt}%
\pgfpathmoveto{\pgfqpoint{8.251208in}{1.723952in}}%
\pgfpathcurveto{\pgfqpoint{8.256252in}{1.723952in}}{\pgfqpoint{8.261089in}{1.725956in}}{\pgfqpoint{8.264656in}{1.729522in}}%
\pgfpathcurveto{\pgfqpoint{8.268222in}{1.733089in}}{\pgfqpoint{8.270226in}{1.737927in}}{\pgfqpoint{8.270226in}{1.742970in}}%
\pgfpathcurveto{\pgfqpoint{8.270226in}{1.748014in}}{\pgfqpoint{8.268222in}{1.752852in}}{\pgfqpoint{8.264656in}{1.756418in}}%
\pgfpathcurveto{\pgfqpoint{8.261089in}{1.759984in}}{\pgfqpoint{8.256252in}{1.761988in}}{\pgfqpoint{8.251208in}{1.761988in}}%
\pgfpathcurveto{\pgfqpoint{8.246164in}{1.761988in}}{\pgfqpoint{8.241326in}{1.759984in}}{\pgfqpoint{8.237760in}{1.756418in}}%
\pgfpathcurveto{\pgfqpoint{8.234194in}{1.752852in}}{\pgfqpoint{8.232190in}{1.748014in}}{\pgfqpoint{8.232190in}{1.742970in}}%
\pgfpathcurveto{\pgfqpoint{8.232190in}{1.737927in}}{\pgfqpoint{8.234194in}{1.733089in}}{\pgfqpoint{8.237760in}{1.729522in}}%
\pgfpathcurveto{\pgfqpoint{8.241326in}{1.725956in}}{\pgfqpoint{8.246164in}{1.723952in}}{\pgfqpoint{8.251208in}{1.723952in}}%
\pgfpathclose%
\pgfusepath{fill}%
\end{pgfscope}%
\begin{pgfscope}%
\pgfpathrectangle{\pgfqpoint{6.572727in}{0.474100in}}{\pgfqpoint{4.227273in}{3.318700in}}%
\pgfusepath{clip}%
\pgfsetbuttcap%
\pgfsetroundjoin%
\definecolor{currentfill}{rgb}{0.993248,0.906157,0.143936}%
\pgfsetfillcolor{currentfill}%
\pgfsetfillopacity{0.700000}%
\pgfsetlinewidth{0.000000pt}%
\definecolor{currentstroke}{rgb}{0.000000,0.000000,0.000000}%
\pgfsetstrokecolor{currentstroke}%
\pgfsetstrokeopacity{0.700000}%
\pgfsetdash{}{0pt}%
\pgfpathmoveto{\pgfqpoint{8.206880in}{2.951642in}}%
\pgfpathcurveto{\pgfqpoint{8.211923in}{2.951642in}}{\pgfqpoint{8.216761in}{2.953646in}}{\pgfqpoint{8.220328in}{2.957213in}}%
\pgfpathcurveto{\pgfqpoint{8.223894in}{2.960779in}}{\pgfqpoint{8.225898in}{2.965617in}}{\pgfqpoint{8.225898in}{2.970661in}}%
\pgfpathcurveto{\pgfqpoint{8.225898in}{2.975704in}}{\pgfqpoint{8.223894in}{2.980542in}}{\pgfqpoint{8.220328in}{2.984108in}}%
\pgfpathcurveto{\pgfqpoint{8.216761in}{2.987675in}}{\pgfqpoint{8.211923in}{2.989679in}}{\pgfqpoint{8.206880in}{2.989679in}}%
\pgfpathcurveto{\pgfqpoint{8.201836in}{2.989679in}}{\pgfqpoint{8.196998in}{2.987675in}}{\pgfqpoint{8.193432in}{2.984108in}}%
\pgfpathcurveto{\pgfqpoint{8.189865in}{2.980542in}}{\pgfqpoint{8.187862in}{2.975704in}}{\pgfqpoint{8.187862in}{2.970661in}}%
\pgfpathcurveto{\pgfqpoint{8.187862in}{2.965617in}}{\pgfqpoint{8.189865in}{2.960779in}}{\pgfqpoint{8.193432in}{2.957213in}}%
\pgfpathcurveto{\pgfqpoint{8.196998in}{2.953646in}}{\pgfqpoint{8.201836in}{2.951642in}}{\pgfqpoint{8.206880in}{2.951642in}}%
\pgfpathclose%
\pgfusepath{fill}%
\end{pgfscope}%
\begin{pgfscope}%
\pgfpathrectangle{\pgfqpoint{6.572727in}{0.474100in}}{\pgfqpoint{4.227273in}{3.318700in}}%
\pgfusepath{clip}%
\pgfsetbuttcap%
\pgfsetroundjoin%
\definecolor{currentfill}{rgb}{0.993248,0.906157,0.143936}%
\pgfsetfillcolor{currentfill}%
\pgfsetfillopacity{0.700000}%
\pgfsetlinewidth{0.000000pt}%
\definecolor{currentstroke}{rgb}{0.000000,0.000000,0.000000}%
\pgfsetstrokecolor{currentstroke}%
\pgfsetstrokeopacity{0.700000}%
\pgfsetdash{}{0pt}%
\pgfpathmoveto{\pgfqpoint{8.291446in}{2.948755in}}%
\pgfpathcurveto{\pgfqpoint{8.296490in}{2.948755in}}{\pgfqpoint{8.301327in}{2.950759in}}{\pgfqpoint{8.304894in}{2.954326in}}%
\pgfpathcurveto{\pgfqpoint{8.308460in}{2.957892in}}{\pgfqpoint{8.310464in}{2.962730in}}{\pgfqpoint{8.310464in}{2.967774in}}%
\pgfpathcurveto{\pgfqpoint{8.310464in}{2.972817in}}{\pgfqpoint{8.308460in}{2.977655in}}{\pgfqpoint{8.304894in}{2.981221in}}%
\pgfpathcurveto{\pgfqpoint{8.301327in}{2.984788in}}{\pgfqpoint{8.296490in}{2.986792in}}{\pgfqpoint{8.291446in}{2.986792in}}%
\pgfpathcurveto{\pgfqpoint{8.286402in}{2.986792in}}{\pgfqpoint{8.281565in}{2.984788in}}{\pgfqpoint{8.277998in}{2.981221in}}%
\pgfpathcurveto{\pgfqpoint{8.274432in}{2.977655in}}{\pgfqpoint{8.272428in}{2.972817in}}{\pgfqpoint{8.272428in}{2.967774in}}%
\pgfpathcurveto{\pgfqpoint{8.272428in}{2.962730in}}{\pgfqpoint{8.274432in}{2.957892in}}{\pgfqpoint{8.277998in}{2.954326in}}%
\pgfpathcurveto{\pgfqpoint{8.281565in}{2.950759in}}{\pgfqpoint{8.286402in}{2.948755in}}{\pgfqpoint{8.291446in}{2.948755in}}%
\pgfpathclose%
\pgfusepath{fill}%
\end{pgfscope}%
\begin{pgfscope}%
\pgfpathrectangle{\pgfqpoint{6.572727in}{0.474100in}}{\pgfqpoint{4.227273in}{3.318700in}}%
\pgfusepath{clip}%
\pgfsetbuttcap%
\pgfsetroundjoin%
\definecolor{currentfill}{rgb}{0.993248,0.906157,0.143936}%
\pgfsetfillcolor{currentfill}%
\pgfsetfillopacity{0.700000}%
\pgfsetlinewidth{0.000000pt}%
\definecolor{currentstroke}{rgb}{0.000000,0.000000,0.000000}%
\pgfsetstrokecolor{currentstroke}%
\pgfsetstrokeopacity{0.700000}%
\pgfsetdash{}{0pt}%
\pgfpathmoveto{\pgfqpoint{8.238555in}{2.941097in}}%
\pgfpathcurveto{\pgfqpoint{8.243599in}{2.941097in}}{\pgfqpoint{8.248437in}{2.943101in}}{\pgfqpoint{8.252003in}{2.946667in}}%
\pgfpathcurveto{\pgfqpoint{8.255569in}{2.950234in}}{\pgfqpoint{8.257573in}{2.955072in}}{\pgfqpoint{8.257573in}{2.960115in}}%
\pgfpathcurveto{\pgfqpoint{8.257573in}{2.965159in}}{\pgfqpoint{8.255569in}{2.969997in}}{\pgfqpoint{8.252003in}{2.973563in}}%
\pgfpathcurveto{\pgfqpoint{8.248437in}{2.977130in}}{\pgfqpoint{8.243599in}{2.979133in}}{\pgfqpoint{8.238555in}{2.979133in}}%
\pgfpathcurveto{\pgfqpoint{8.233511in}{2.979133in}}{\pgfqpoint{8.228674in}{2.977130in}}{\pgfqpoint{8.225107in}{2.973563in}}%
\pgfpathcurveto{\pgfqpoint{8.221541in}{2.969997in}}{\pgfqpoint{8.219537in}{2.965159in}}{\pgfqpoint{8.219537in}{2.960115in}}%
\pgfpathcurveto{\pgfqpoint{8.219537in}{2.955072in}}{\pgfqpoint{8.221541in}{2.950234in}}{\pgfqpoint{8.225107in}{2.946667in}}%
\pgfpathcurveto{\pgfqpoint{8.228674in}{2.943101in}}{\pgfqpoint{8.233511in}{2.941097in}}{\pgfqpoint{8.238555in}{2.941097in}}%
\pgfpathclose%
\pgfusepath{fill}%
\end{pgfscope}%
\begin{pgfscope}%
\pgfpathrectangle{\pgfqpoint{6.572727in}{0.474100in}}{\pgfqpoint{4.227273in}{3.318700in}}%
\pgfusepath{clip}%
\pgfsetbuttcap%
\pgfsetroundjoin%
\definecolor{currentfill}{rgb}{0.127568,0.566949,0.550556}%
\pgfsetfillcolor{currentfill}%
\pgfsetfillopacity{0.700000}%
\pgfsetlinewidth{0.000000pt}%
\definecolor{currentstroke}{rgb}{0.000000,0.000000,0.000000}%
\pgfsetstrokecolor{currentstroke}%
\pgfsetstrokeopacity{0.700000}%
\pgfsetdash{}{0pt}%
\pgfpathmoveto{\pgfqpoint{9.926121in}{1.456416in}}%
\pgfpathcurveto{\pgfqpoint{9.931165in}{1.456416in}}{\pgfqpoint{9.936003in}{1.458420in}}{\pgfqpoint{9.939569in}{1.461986in}}%
\pgfpathcurveto{\pgfqpoint{9.943136in}{1.465552in}}{\pgfqpoint{9.945140in}{1.470390in}}{\pgfqpoint{9.945140in}{1.475434in}}%
\pgfpathcurveto{\pgfqpoint{9.945140in}{1.480477in}}{\pgfqpoint{9.943136in}{1.485315in}}{\pgfqpoint{9.939569in}{1.488882in}}%
\pgfpathcurveto{\pgfqpoint{9.936003in}{1.492448in}}{\pgfqpoint{9.931165in}{1.494452in}}{\pgfqpoint{9.926121in}{1.494452in}}%
\pgfpathcurveto{\pgfqpoint{9.921078in}{1.494452in}}{\pgfqpoint{9.916240in}{1.492448in}}{\pgfqpoint{9.912674in}{1.488882in}}%
\pgfpathcurveto{\pgfqpoint{9.909107in}{1.485315in}}{\pgfqpoint{9.907103in}{1.480477in}}{\pgfqpoint{9.907103in}{1.475434in}}%
\pgfpathcurveto{\pgfqpoint{9.907103in}{1.470390in}}{\pgfqpoint{9.909107in}{1.465552in}}{\pgfqpoint{9.912674in}{1.461986in}}%
\pgfpathcurveto{\pgfqpoint{9.916240in}{1.458420in}}{\pgfqpoint{9.921078in}{1.456416in}}{\pgfqpoint{9.926121in}{1.456416in}}%
\pgfpathclose%
\pgfusepath{fill}%
\end{pgfscope}%
\begin{pgfscope}%
\pgfpathrectangle{\pgfqpoint{6.572727in}{0.474100in}}{\pgfqpoint{4.227273in}{3.318700in}}%
\pgfusepath{clip}%
\pgfsetbuttcap%
\pgfsetroundjoin%
\definecolor{currentfill}{rgb}{0.993248,0.906157,0.143936}%
\pgfsetfillcolor{currentfill}%
\pgfsetfillopacity{0.700000}%
\pgfsetlinewidth{0.000000pt}%
\definecolor{currentstroke}{rgb}{0.000000,0.000000,0.000000}%
\pgfsetstrokecolor{currentstroke}%
\pgfsetstrokeopacity{0.700000}%
\pgfsetdash{}{0pt}%
\pgfpathmoveto{\pgfqpoint{8.287123in}{2.686814in}}%
\pgfpathcurveto{\pgfqpoint{8.292167in}{2.686814in}}{\pgfqpoint{8.297004in}{2.688818in}}{\pgfqpoint{8.300571in}{2.692384in}}%
\pgfpathcurveto{\pgfqpoint{8.304137in}{2.695951in}}{\pgfqpoint{8.306141in}{2.700788in}}{\pgfqpoint{8.306141in}{2.705832in}}%
\pgfpathcurveto{\pgfqpoint{8.306141in}{2.710876in}}{\pgfqpoint{8.304137in}{2.715713in}}{\pgfqpoint{8.300571in}{2.719280in}}%
\pgfpathcurveto{\pgfqpoint{8.297004in}{2.722846in}}{\pgfqpoint{8.292167in}{2.724850in}}{\pgfqpoint{8.287123in}{2.724850in}}%
\pgfpathcurveto{\pgfqpoint{8.282079in}{2.724850in}}{\pgfqpoint{8.277242in}{2.722846in}}{\pgfqpoint{8.273675in}{2.719280in}}%
\pgfpathcurveto{\pgfqpoint{8.270109in}{2.715713in}}{\pgfqpoint{8.268105in}{2.710876in}}{\pgfqpoint{8.268105in}{2.705832in}}%
\pgfpathcurveto{\pgfqpoint{8.268105in}{2.700788in}}{\pgfqpoint{8.270109in}{2.695951in}}{\pgfqpoint{8.273675in}{2.692384in}}%
\pgfpathcurveto{\pgfqpoint{8.277242in}{2.688818in}}{\pgfqpoint{8.282079in}{2.686814in}}{\pgfqpoint{8.287123in}{2.686814in}}%
\pgfpathclose%
\pgfusepath{fill}%
\end{pgfscope}%
\begin{pgfscope}%
\pgfpathrectangle{\pgfqpoint{6.572727in}{0.474100in}}{\pgfqpoint{4.227273in}{3.318700in}}%
\pgfusepath{clip}%
\pgfsetbuttcap%
\pgfsetroundjoin%
\definecolor{currentfill}{rgb}{0.127568,0.566949,0.550556}%
\pgfsetfillcolor{currentfill}%
\pgfsetfillopacity{0.700000}%
\pgfsetlinewidth{0.000000pt}%
\definecolor{currentstroke}{rgb}{0.000000,0.000000,0.000000}%
\pgfsetstrokecolor{currentstroke}%
\pgfsetstrokeopacity{0.700000}%
\pgfsetdash{}{0pt}%
\pgfpathmoveto{\pgfqpoint{9.393143in}{1.425542in}}%
\pgfpathcurveto{\pgfqpoint{9.398187in}{1.425542in}}{\pgfqpoint{9.403025in}{1.427546in}}{\pgfqpoint{9.406591in}{1.431112in}}%
\pgfpathcurveto{\pgfqpoint{9.410157in}{1.434678in}}{\pgfqpoint{9.412161in}{1.439516in}}{\pgfqpoint{9.412161in}{1.444560in}}%
\pgfpathcurveto{\pgfqpoint{9.412161in}{1.449603in}}{\pgfqpoint{9.410157in}{1.454441in}}{\pgfqpoint{9.406591in}{1.458008in}}%
\pgfpathcurveto{\pgfqpoint{9.403025in}{1.461574in}}{\pgfqpoint{9.398187in}{1.463578in}}{\pgfqpoint{9.393143in}{1.463578in}}%
\pgfpathcurveto{\pgfqpoint{9.388099in}{1.463578in}}{\pgfqpoint{9.383262in}{1.461574in}}{\pgfqpoint{9.379695in}{1.458008in}}%
\pgfpathcurveto{\pgfqpoint{9.376129in}{1.454441in}}{\pgfqpoint{9.374125in}{1.449603in}}{\pgfqpoint{9.374125in}{1.444560in}}%
\pgfpathcurveto{\pgfqpoint{9.374125in}{1.439516in}}{\pgfqpoint{9.376129in}{1.434678in}}{\pgfqpoint{9.379695in}{1.431112in}}%
\pgfpathcurveto{\pgfqpoint{9.383262in}{1.427546in}}{\pgfqpoint{9.388099in}{1.425542in}}{\pgfqpoint{9.393143in}{1.425542in}}%
\pgfpathclose%
\pgfusepath{fill}%
\end{pgfscope}%
\begin{pgfscope}%
\pgfpathrectangle{\pgfqpoint{6.572727in}{0.474100in}}{\pgfqpoint{4.227273in}{3.318700in}}%
\pgfusepath{clip}%
\pgfsetbuttcap%
\pgfsetroundjoin%
\definecolor{currentfill}{rgb}{0.993248,0.906157,0.143936}%
\pgfsetfillcolor{currentfill}%
\pgfsetfillopacity{0.700000}%
\pgfsetlinewidth{0.000000pt}%
\definecolor{currentstroke}{rgb}{0.000000,0.000000,0.000000}%
\pgfsetstrokecolor{currentstroke}%
\pgfsetstrokeopacity{0.700000}%
\pgfsetdash{}{0pt}%
\pgfpathmoveto{\pgfqpoint{8.808060in}{3.306680in}}%
\pgfpathcurveto{\pgfqpoint{8.813104in}{3.306680in}}{\pgfqpoint{8.817941in}{3.308684in}}{\pgfqpoint{8.821508in}{3.312251in}}%
\pgfpathcurveto{\pgfqpoint{8.825074in}{3.315817in}}{\pgfqpoint{8.827078in}{3.320655in}}{\pgfqpoint{8.827078in}{3.325698in}}%
\pgfpathcurveto{\pgfqpoint{8.827078in}{3.330742in}}{\pgfqpoint{8.825074in}{3.335580in}}{\pgfqpoint{8.821508in}{3.339146in}}%
\pgfpathcurveto{\pgfqpoint{8.817941in}{3.342713in}}{\pgfqpoint{8.813104in}{3.344717in}}{\pgfqpoint{8.808060in}{3.344717in}}%
\pgfpathcurveto{\pgfqpoint{8.803016in}{3.344717in}}{\pgfqpoint{8.798178in}{3.342713in}}{\pgfqpoint{8.794612in}{3.339146in}}%
\pgfpathcurveto{\pgfqpoint{8.791046in}{3.335580in}}{\pgfqpoint{8.789042in}{3.330742in}}{\pgfqpoint{8.789042in}{3.325698in}}%
\pgfpathcurveto{\pgfqpoint{8.789042in}{3.320655in}}{\pgfqpoint{8.791046in}{3.315817in}}{\pgfqpoint{8.794612in}{3.312251in}}%
\pgfpathcurveto{\pgfqpoint{8.798178in}{3.308684in}}{\pgfqpoint{8.803016in}{3.306680in}}{\pgfqpoint{8.808060in}{3.306680in}}%
\pgfpathclose%
\pgfusepath{fill}%
\end{pgfscope}%
\begin{pgfscope}%
\pgfpathrectangle{\pgfqpoint{6.572727in}{0.474100in}}{\pgfqpoint{4.227273in}{3.318700in}}%
\pgfusepath{clip}%
\pgfsetbuttcap%
\pgfsetroundjoin%
\definecolor{currentfill}{rgb}{0.127568,0.566949,0.550556}%
\pgfsetfillcolor{currentfill}%
\pgfsetfillopacity{0.700000}%
\pgfsetlinewidth{0.000000pt}%
\definecolor{currentstroke}{rgb}{0.000000,0.000000,0.000000}%
\pgfsetstrokecolor{currentstroke}%
\pgfsetstrokeopacity{0.700000}%
\pgfsetdash{}{0pt}%
\pgfpathmoveto{\pgfqpoint{8.843353in}{0.939826in}}%
\pgfpathcurveto{\pgfqpoint{8.848397in}{0.939826in}}{\pgfqpoint{8.853235in}{0.941830in}}{\pgfqpoint{8.856801in}{0.945397in}}%
\pgfpathcurveto{\pgfqpoint{8.860368in}{0.948963in}}{\pgfqpoint{8.862371in}{0.953801in}}{\pgfqpoint{8.862371in}{0.958844in}}%
\pgfpathcurveto{\pgfqpoint{8.862371in}{0.963888in}}{\pgfqpoint{8.860368in}{0.968726in}}{\pgfqpoint{8.856801in}{0.972292in}}%
\pgfpathcurveto{\pgfqpoint{8.853235in}{0.975859in}}{\pgfqpoint{8.848397in}{0.977863in}}{\pgfqpoint{8.843353in}{0.977863in}}%
\pgfpathcurveto{\pgfqpoint{8.838310in}{0.977863in}}{\pgfqpoint{8.833472in}{0.975859in}}{\pgfqpoint{8.829905in}{0.972292in}}%
\pgfpathcurveto{\pgfqpoint{8.826339in}{0.968726in}}{\pgfqpoint{8.824335in}{0.963888in}}{\pgfqpoint{8.824335in}{0.958844in}}%
\pgfpathcurveto{\pgfqpoint{8.824335in}{0.953801in}}{\pgfqpoint{8.826339in}{0.948963in}}{\pgfqpoint{8.829905in}{0.945397in}}%
\pgfpathcurveto{\pgfqpoint{8.833472in}{0.941830in}}{\pgfqpoint{8.838310in}{0.939826in}}{\pgfqpoint{8.843353in}{0.939826in}}%
\pgfpathclose%
\pgfusepath{fill}%
\end{pgfscope}%
\begin{pgfscope}%
\pgfpathrectangle{\pgfqpoint{6.572727in}{0.474100in}}{\pgfqpoint{4.227273in}{3.318700in}}%
\pgfusepath{clip}%
\pgfsetbuttcap%
\pgfsetroundjoin%
\definecolor{currentfill}{rgb}{0.267004,0.004874,0.329415}%
\pgfsetfillcolor{currentfill}%
\pgfsetfillopacity{0.700000}%
\pgfsetlinewidth{0.000000pt}%
\definecolor{currentstroke}{rgb}{0.000000,0.000000,0.000000}%
\pgfsetstrokecolor{currentstroke}%
\pgfsetstrokeopacity{0.700000}%
\pgfsetdash{}{0pt}%
\pgfpathmoveto{\pgfqpoint{7.805746in}{1.656711in}}%
\pgfpathcurveto{\pgfqpoint{7.810790in}{1.656711in}}{\pgfqpoint{7.815627in}{1.658715in}}{\pgfqpoint{7.819194in}{1.662281in}}%
\pgfpathcurveto{\pgfqpoint{7.822760in}{1.665847in}}{\pgfqpoint{7.824764in}{1.670685in}}{\pgfqpoint{7.824764in}{1.675729in}}%
\pgfpathcurveto{\pgfqpoint{7.824764in}{1.680773in}}{\pgfqpoint{7.822760in}{1.685610in}}{\pgfqpoint{7.819194in}{1.689177in}}%
\pgfpathcurveto{\pgfqpoint{7.815627in}{1.692743in}}{\pgfqpoint{7.810790in}{1.694747in}}{\pgfqpoint{7.805746in}{1.694747in}}%
\pgfpathcurveto{\pgfqpoint{7.800702in}{1.694747in}}{\pgfqpoint{7.795864in}{1.692743in}}{\pgfqpoint{7.792298in}{1.689177in}}%
\pgfpathcurveto{\pgfqpoint{7.788732in}{1.685610in}}{\pgfqpoint{7.786728in}{1.680773in}}{\pgfqpoint{7.786728in}{1.675729in}}%
\pgfpathcurveto{\pgfqpoint{7.786728in}{1.670685in}}{\pgfqpoint{7.788732in}{1.665847in}}{\pgfqpoint{7.792298in}{1.662281in}}%
\pgfpathcurveto{\pgfqpoint{7.795864in}{1.658715in}}{\pgfqpoint{7.800702in}{1.656711in}}{\pgfqpoint{7.805746in}{1.656711in}}%
\pgfpathclose%
\pgfusepath{fill}%
\end{pgfscope}%
\begin{pgfscope}%
\pgfpathrectangle{\pgfqpoint{6.572727in}{0.474100in}}{\pgfqpoint{4.227273in}{3.318700in}}%
\pgfusepath{clip}%
\pgfsetbuttcap%
\pgfsetroundjoin%
\definecolor{currentfill}{rgb}{0.127568,0.566949,0.550556}%
\pgfsetfillcolor{currentfill}%
\pgfsetfillopacity{0.700000}%
\pgfsetlinewidth{0.000000pt}%
\definecolor{currentstroke}{rgb}{0.000000,0.000000,0.000000}%
\pgfsetstrokecolor{currentstroke}%
\pgfsetstrokeopacity{0.700000}%
\pgfsetdash{}{0pt}%
\pgfpathmoveto{\pgfqpoint{10.089422in}{1.318415in}}%
\pgfpathcurveto{\pgfqpoint{10.094466in}{1.318415in}}{\pgfqpoint{10.099304in}{1.320419in}}{\pgfqpoint{10.102870in}{1.323986in}}%
\pgfpathcurveto{\pgfqpoint{10.106437in}{1.327552in}}{\pgfqpoint{10.108441in}{1.332390in}}{\pgfqpoint{10.108441in}{1.337434in}}%
\pgfpathcurveto{\pgfqpoint{10.108441in}{1.342477in}}{\pgfqpoint{10.106437in}{1.347315in}}{\pgfqpoint{10.102870in}{1.350881in}}%
\pgfpathcurveto{\pgfqpoint{10.099304in}{1.354448in}}{\pgfqpoint{10.094466in}{1.356452in}}{\pgfqpoint{10.089422in}{1.356452in}}%
\pgfpathcurveto{\pgfqpoint{10.084379in}{1.356452in}}{\pgfqpoint{10.079541in}{1.354448in}}{\pgfqpoint{10.075975in}{1.350881in}}%
\pgfpathcurveto{\pgfqpoint{10.072408in}{1.347315in}}{\pgfqpoint{10.070404in}{1.342477in}}{\pgfqpoint{10.070404in}{1.337434in}}%
\pgfpathcurveto{\pgfqpoint{10.070404in}{1.332390in}}{\pgfqpoint{10.072408in}{1.327552in}}{\pgfqpoint{10.075975in}{1.323986in}}%
\pgfpathcurveto{\pgfqpoint{10.079541in}{1.320419in}}{\pgfqpoint{10.084379in}{1.318415in}}{\pgfqpoint{10.089422in}{1.318415in}}%
\pgfpathclose%
\pgfusepath{fill}%
\end{pgfscope}%
\begin{pgfscope}%
\pgfpathrectangle{\pgfqpoint{6.572727in}{0.474100in}}{\pgfqpoint{4.227273in}{3.318700in}}%
\pgfusepath{clip}%
\pgfsetbuttcap%
\pgfsetroundjoin%
\definecolor{currentfill}{rgb}{0.267004,0.004874,0.329415}%
\pgfsetfillcolor{currentfill}%
\pgfsetfillopacity{0.700000}%
\pgfsetlinewidth{0.000000pt}%
\definecolor{currentstroke}{rgb}{0.000000,0.000000,0.000000}%
\pgfsetstrokecolor{currentstroke}%
\pgfsetstrokeopacity{0.700000}%
\pgfsetdash{}{0pt}%
\pgfpathmoveto{\pgfqpoint{8.485526in}{1.551498in}}%
\pgfpathcurveto{\pgfqpoint{8.490569in}{1.551498in}}{\pgfqpoint{8.495407in}{1.553502in}}{\pgfqpoint{8.498974in}{1.557068in}}%
\pgfpathcurveto{\pgfqpoint{8.502540in}{1.560635in}}{\pgfqpoint{8.504544in}{1.565473in}}{\pgfqpoint{8.504544in}{1.570516in}}%
\pgfpathcurveto{\pgfqpoint{8.504544in}{1.575560in}}{\pgfqpoint{8.502540in}{1.580398in}}{\pgfqpoint{8.498974in}{1.583964in}}%
\pgfpathcurveto{\pgfqpoint{8.495407in}{1.587531in}}{\pgfqpoint{8.490569in}{1.589534in}}{\pgfqpoint{8.485526in}{1.589534in}}%
\pgfpathcurveto{\pgfqpoint{8.480482in}{1.589534in}}{\pgfqpoint{8.475644in}{1.587531in}}{\pgfqpoint{8.472078in}{1.583964in}}%
\pgfpathcurveto{\pgfqpoint{8.468512in}{1.580398in}}{\pgfqpoint{8.466508in}{1.575560in}}{\pgfqpoint{8.466508in}{1.570516in}}%
\pgfpathcurveto{\pgfqpoint{8.466508in}{1.565473in}}{\pgfqpoint{8.468512in}{1.560635in}}{\pgfqpoint{8.472078in}{1.557068in}}%
\pgfpathcurveto{\pgfqpoint{8.475644in}{1.553502in}}{\pgfqpoint{8.480482in}{1.551498in}}{\pgfqpoint{8.485526in}{1.551498in}}%
\pgfpathclose%
\pgfusepath{fill}%
\end{pgfscope}%
\begin{pgfscope}%
\pgfpathrectangle{\pgfqpoint{6.572727in}{0.474100in}}{\pgfqpoint{4.227273in}{3.318700in}}%
\pgfusepath{clip}%
\pgfsetbuttcap%
\pgfsetroundjoin%
\definecolor{currentfill}{rgb}{0.993248,0.906157,0.143936}%
\pgfsetfillcolor{currentfill}%
\pgfsetfillopacity{0.700000}%
\pgfsetlinewidth{0.000000pt}%
\definecolor{currentstroke}{rgb}{0.000000,0.000000,0.000000}%
\pgfsetstrokecolor{currentstroke}%
\pgfsetstrokeopacity{0.700000}%
\pgfsetdash{}{0pt}%
\pgfpathmoveto{\pgfqpoint{7.997013in}{2.824629in}}%
\pgfpathcurveto{\pgfqpoint{8.002056in}{2.824629in}}{\pgfqpoint{8.006894in}{2.826633in}}{\pgfqpoint{8.010460in}{2.830199in}}%
\pgfpathcurveto{\pgfqpoint{8.014027in}{2.833765in}}{\pgfqpoint{8.016031in}{2.838603in}}{\pgfqpoint{8.016031in}{2.843647in}}%
\pgfpathcurveto{\pgfqpoint{8.016031in}{2.848691in}}{\pgfqpoint{8.014027in}{2.853528in}}{\pgfqpoint{8.010460in}{2.857095in}}%
\pgfpathcurveto{\pgfqpoint{8.006894in}{2.860661in}}{\pgfqpoint{8.002056in}{2.862665in}}{\pgfqpoint{7.997013in}{2.862665in}}%
\pgfpathcurveto{\pgfqpoint{7.991969in}{2.862665in}}{\pgfqpoint{7.987131in}{2.860661in}}{\pgfqpoint{7.983565in}{2.857095in}}%
\pgfpathcurveto{\pgfqpoint{7.979998in}{2.853528in}}{\pgfqpoint{7.977994in}{2.848691in}}{\pgfqpoint{7.977994in}{2.843647in}}%
\pgfpathcurveto{\pgfqpoint{7.977994in}{2.838603in}}{\pgfqpoint{7.979998in}{2.833765in}}{\pgfqpoint{7.983565in}{2.830199in}}%
\pgfpathcurveto{\pgfqpoint{7.987131in}{2.826633in}}{\pgfqpoint{7.991969in}{2.824629in}}{\pgfqpoint{7.997013in}{2.824629in}}%
\pgfpathclose%
\pgfusepath{fill}%
\end{pgfscope}%
\begin{pgfscope}%
\pgfpathrectangle{\pgfqpoint{6.572727in}{0.474100in}}{\pgfqpoint{4.227273in}{3.318700in}}%
\pgfusepath{clip}%
\pgfsetbuttcap%
\pgfsetroundjoin%
\definecolor{currentfill}{rgb}{0.267004,0.004874,0.329415}%
\pgfsetfillcolor{currentfill}%
\pgfsetfillopacity{0.700000}%
\pgfsetlinewidth{0.000000pt}%
\definecolor{currentstroke}{rgb}{0.000000,0.000000,0.000000}%
\pgfsetstrokecolor{currentstroke}%
\pgfsetstrokeopacity{0.700000}%
\pgfsetdash{}{0pt}%
\pgfpathmoveto{\pgfqpoint{7.491708in}{1.637198in}}%
\pgfpathcurveto{\pgfqpoint{7.496752in}{1.637198in}}{\pgfqpoint{7.501590in}{1.639202in}}{\pgfqpoint{7.505156in}{1.642768in}}%
\pgfpathcurveto{\pgfqpoint{7.508723in}{1.646334in}}{\pgfqpoint{7.510727in}{1.651172in}}{\pgfqpoint{7.510727in}{1.656216in}}%
\pgfpathcurveto{\pgfqpoint{7.510727in}{1.661259in}}{\pgfqpoint{7.508723in}{1.666097in}}{\pgfqpoint{7.505156in}{1.669664in}}%
\pgfpathcurveto{\pgfqpoint{7.501590in}{1.673230in}}{\pgfqpoint{7.496752in}{1.675234in}}{\pgfqpoint{7.491708in}{1.675234in}}%
\pgfpathcurveto{\pgfqpoint{7.486665in}{1.675234in}}{\pgfqpoint{7.481827in}{1.673230in}}{\pgfqpoint{7.478261in}{1.669664in}}%
\pgfpathcurveto{\pgfqpoint{7.474694in}{1.666097in}}{\pgfqpoint{7.472690in}{1.661259in}}{\pgfqpoint{7.472690in}{1.656216in}}%
\pgfpathcurveto{\pgfqpoint{7.472690in}{1.651172in}}{\pgfqpoint{7.474694in}{1.646334in}}{\pgfqpoint{7.478261in}{1.642768in}}%
\pgfpathcurveto{\pgfqpoint{7.481827in}{1.639202in}}{\pgfqpoint{7.486665in}{1.637198in}}{\pgfqpoint{7.491708in}{1.637198in}}%
\pgfpathclose%
\pgfusepath{fill}%
\end{pgfscope}%
\begin{pgfscope}%
\pgfpathrectangle{\pgfqpoint{6.572727in}{0.474100in}}{\pgfqpoint{4.227273in}{3.318700in}}%
\pgfusepath{clip}%
\pgfsetbuttcap%
\pgfsetroundjoin%
\definecolor{currentfill}{rgb}{0.993248,0.906157,0.143936}%
\pgfsetfillcolor{currentfill}%
\pgfsetfillopacity{0.700000}%
\pgfsetlinewidth{0.000000pt}%
\definecolor{currentstroke}{rgb}{0.000000,0.000000,0.000000}%
\pgfsetstrokecolor{currentstroke}%
\pgfsetstrokeopacity{0.700000}%
\pgfsetdash{}{0pt}%
\pgfpathmoveto{\pgfqpoint{7.995655in}{3.249166in}}%
\pgfpathcurveto{\pgfqpoint{8.000699in}{3.249166in}}{\pgfqpoint{8.005537in}{3.251170in}}{\pgfqpoint{8.009103in}{3.254737in}}%
\pgfpathcurveto{\pgfqpoint{8.012670in}{3.258303in}}{\pgfqpoint{8.014674in}{3.263141in}}{\pgfqpoint{8.014674in}{3.268184in}}%
\pgfpathcurveto{\pgfqpoint{8.014674in}{3.273228in}}{\pgfqpoint{8.012670in}{3.278066in}}{\pgfqpoint{8.009103in}{3.281632in}}%
\pgfpathcurveto{\pgfqpoint{8.005537in}{3.285199in}}{\pgfqpoint{8.000699in}{3.287203in}}{\pgfqpoint{7.995655in}{3.287203in}}%
\pgfpathcurveto{\pgfqpoint{7.990612in}{3.287203in}}{\pgfqpoint{7.985774in}{3.285199in}}{\pgfqpoint{7.982208in}{3.281632in}}%
\pgfpathcurveto{\pgfqpoint{7.978641in}{3.278066in}}{\pgfqpoint{7.976637in}{3.273228in}}{\pgfqpoint{7.976637in}{3.268184in}}%
\pgfpathcurveto{\pgfqpoint{7.976637in}{3.263141in}}{\pgfqpoint{7.978641in}{3.258303in}}{\pgfqpoint{7.982208in}{3.254737in}}%
\pgfpathcurveto{\pgfqpoint{7.985774in}{3.251170in}}{\pgfqpoint{7.990612in}{3.249166in}}{\pgfqpoint{7.995655in}{3.249166in}}%
\pgfpathclose%
\pgfusepath{fill}%
\end{pgfscope}%
\begin{pgfscope}%
\pgfpathrectangle{\pgfqpoint{6.572727in}{0.474100in}}{\pgfqpoint{4.227273in}{3.318700in}}%
\pgfusepath{clip}%
\pgfsetbuttcap%
\pgfsetroundjoin%
\definecolor{currentfill}{rgb}{0.267004,0.004874,0.329415}%
\pgfsetfillcolor{currentfill}%
\pgfsetfillopacity{0.700000}%
\pgfsetlinewidth{0.000000pt}%
\definecolor{currentstroke}{rgb}{0.000000,0.000000,0.000000}%
\pgfsetstrokecolor{currentstroke}%
\pgfsetstrokeopacity{0.700000}%
\pgfsetdash{}{0pt}%
\pgfpathmoveto{\pgfqpoint{7.800085in}{1.030987in}}%
\pgfpathcurveto{\pgfqpoint{7.805129in}{1.030987in}}{\pgfqpoint{7.809967in}{1.032991in}}{\pgfqpoint{7.813533in}{1.036558in}}%
\pgfpathcurveto{\pgfqpoint{7.817099in}{1.040124in}}{\pgfqpoint{7.819103in}{1.044962in}}{\pgfqpoint{7.819103in}{1.050006in}}%
\pgfpathcurveto{\pgfqpoint{7.819103in}{1.055049in}}{\pgfqpoint{7.817099in}{1.059887in}}{\pgfqpoint{7.813533in}{1.063453in}}%
\pgfpathcurveto{\pgfqpoint{7.809967in}{1.067020in}}{\pgfqpoint{7.805129in}{1.069024in}}{\pgfqpoint{7.800085in}{1.069024in}}%
\pgfpathcurveto{\pgfqpoint{7.795042in}{1.069024in}}{\pgfqpoint{7.790204in}{1.067020in}}{\pgfqpoint{7.786637in}{1.063453in}}%
\pgfpathcurveto{\pgfqpoint{7.783071in}{1.059887in}}{\pgfqpoint{7.781067in}{1.055049in}}{\pgfqpoint{7.781067in}{1.050006in}}%
\pgfpathcurveto{\pgfqpoint{7.781067in}{1.044962in}}{\pgfqpoint{7.783071in}{1.040124in}}{\pgfqpoint{7.786637in}{1.036558in}}%
\pgfpathcurveto{\pgfqpoint{7.790204in}{1.032991in}}{\pgfqpoint{7.795042in}{1.030987in}}{\pgfqpoint{7.800085in}{1.030987in}}%
\pgfpathclose%
\pgfusepath{fill}%
\end{pgfscope}%
\begin{pgfscope}%
\pgfpathrectangle{\pgfqpoint{6.572727in}{0.474100in}}{\pgfqpoint{4.227273in}{3.318700in}}%
\pgfusepath{clip}%
\pgfsetbuttcap%
\pgfsetroundjoin%
\definecolor{currentfill}{rgb}{0.267004,0.004874,0.329415}%
\pgfsetfillcolor{currentfill}%
\pgfsetfillopacity{0.700000}%
\pgfsetlinewidth{0.000000pt}%
\definecolor{currentstroke}{rgb}{0.000000,0.000000,0.000000}%
\pgfsetstrokecolor{currentstroke}%
\pgfsetstrokeopacity{0.700000}%
\pgfsetdash{}{0pt}%
\pgfpathmoveto{\pgfqpoint{7.781376in}{1.560309in}}%
\pgfpathcurveto{\pgfqpoint{7.786420in}{1.560309in}}{\pgfqpoint{7.791258in}{1.562313in}}{\pgfqpoint{7.794824in}{1.565880in}}%
\pgfpathcurveto{\pgfqpoint{7.798391in}{1.569446in}}{\pgfqpoint{7.800395in}{1.574284in}}{\pgfqpoint{7.800395in}{1.579328in}}%
\pgfpathcurveto{\pgfqpoint{7.800395in}{1.584371in}}{\pgfqpoint{7.798391in}{1.589209in}}{\pgfqpoint{7.794824in}{1.592775in}}%
\pgfpathcurveto{\pgfqpoint{7.791258in}{1.596342in}}{\pgfqpoint{7.786420in}{1.598346in}}{\pgfqpoint{7.781376in}{1.598346in}}%
\pgfpathcurveto{\pgfqpoint{7.776333in}{1.598346in}}{\pgfqpoint{7.771495in}{1.596342in}}{\pgfqpoint{7.767929in}{1.592775in}}%
\pgfpathcurveto{\pgfqpoint{7.764362in}{1.589209in}}{\pgfqpoint{7.762358in}{1.584371in}}{\pgfqpoint{7.762358in}{1.579328in}}%
\pgfpathcurveto{\pgfqpoint{7.762358in}{1.574284in}}{\pgfqpoint{7.764362in}{1.569446in}}{\pgfqpoint{7.767929in}{1.565880in}}%
\pgfpathcurveto{\pgfqpoint{7.771495in}{1.562313in}}{\pgfqpoint{7.776333in}{1.560309in}}{\pgfqpoint{7.781376in}{1.560309in}}%
\pgfpathclose%
\pgfusepath{fill}%
\end{pgfscope}%
\begin{pgfscope}%
\pgfpathrectangle{\pgfqpoint{6.572727in}{0.474100in}}{\pgfqpoint{4.227273in}{3.318700in}}%
\pgfusepath{clip}%
\pgfsetbuttcap%
\pgfsetroundjoin%
\definecolor{currentfill}{rgb}{0.993248,0.906157,0.143936}%
\pgfsetfillcolor{currentfill}%
\pgfsetfillopacity{0.700000}%
\pgfsetlinewidth{0.000000pt}%
\definecolor{currentstroke}{rgb}{0.000000,0.000000,0.000000}%
\pgfsetstrokecolor{currentstroke}%
\pgfsetstrokeopacity{0.700000}%
\pgfsetdash{}{0pt}%
\pgfpathmoveto{\pgfqpoint{8.579273in}{2.920329in}}%
\pgfpathcurveto{\pgfqpoint{8.584317in}{2.920329in}}{\pgfqpoint{8.589154in}{2.922333in}}{\pgfqpoint{8.592721in}{2.925899in}}%
\pgfpathcurveto{\pgfqpoint{8.596287in}{2.929466in}}{\pgfqpoint{8.598291in}{2.934303in}}{\pgfqpoint{8.598291in}{2.939347in}}%
\pgfpathcurveto{\pgfqpoint{8.598291in}{2.944391in}}{\pgfqpoint{8.596287in}{2.949228in}}{\pgfqpoint{8.592721in}{2.952795in}}%
\pgfpathcurveto{\pgfqpoint{8.589154in}{2.956361in}}{\pgfqpoint{8.584317in}{2.958365in}}{\pgfqpoint{8.579273in}{2.958365in}}%
\pgfpathcurveto{\pgfqpoint{8.574229in}{2.958365in}}{\pgfqpoint{8.569392in}{2.956361in}}{\pgfqpoint{8.565825in}{2.952795in}}%
\pgfpathcurveto{\pgfqpoint{8.562259in}{2.949228in}}{\pgfqpoint{8.560255in}{2.944391in}}{\pgfqpoint{8.560255in}{2.939347in}}%
\pgfpathcurveto{\pgfqpoint{8.560255in}{2.934303in}}{\pgfqpoint{8.562259in}{2.929466in}}{\pgfqpoint{8.565825in}{2.925899in}}%
\pgfpathcurveto{\pgfqpoint{8.569392in}{2.922333in}}{\pgfqpoint{8.574229in}{2.920329in}}{\pgfqpoint{8.579273in}{2.920329in}}%
\pgfpathclose%
\pgfusepath{fill}%
\end{pgfscope}%
\begin{pgfscope}%
\pgfpathrectangle{\pgfqpoint{6.572727in}{0.474100in}}{\pgfqpoint{4.227273in}{3.318700in}}%
\pgfusepath{clip}%
\pgfsetbuttcap%
\pgfsetroundjoin%
\definecolor{currentfill}{rgb}{0.127568,0.566949,0.550556}%
\pgfsetfillcolor{currentfill}%
\pgfsetfillopacity{0.700000}%
\pgfsetlinewidth{0.000000pt}%
\definecolor{currentstroke}{rgb}{0.000000,0.000000,0.000000}%
\pgfsetstrokecolor{currentstroke}%
\pgfsetstrokeopacity{0.700000}%
\pgfsetdash{}{0pt}%
\pgfpathmoveto{\pgfqpoint{9.428551in}{1.520236in}}%
\pgfpathcurveto{\pgfqpoint{9.433595in}{1.520236in}}{\pgfqpoint{9.438433in}{1.522240in}}{\pgfqpoint{9.441999in}{1.525806in}}%
\pgfpathcurveto{\pgfqpoint{9.445566in}{1.529373in}}{\pgfqpoint{9.447570in}{1.534211in}}{\pgfqpoint{9.447570in}{1.539254in}}%
\pgfpathcurveto{\pgfqpoint{9.447570in}{1.544298in}}{\pgfqpoint{9.445566in}{1.549136in}}{\pgfqpoint{9.441999in}{1.552702in}}%
\pgfpathcurveto{\pgfqpoint{9.438433in}{1.556269in}}{\pgfqpoint{9.433595in}{1.558272in}}{\pgfqpoint{9.428551in}{1.558272in}}%
\pgfpathcurveto{\pgfqpoint{9.423508in}{1.558272in}}{\pgfqpoint{9.418670in}{1.556269in}}{\pgfqpoint{9.415104in}{1.552702in}}%
\pgfpathcurveto{\pgfqpoint{9.411537in}{1.549136in}}{\pgfqpoint{9.409533in}{1.544298in}}{\pgfqpoint{9.409533in}{1.539254in}}%
\pgfpathcurveto{\pgfqpoint{9.409533in}{1.534211in}}{\pgfqpoint{9.411537in}{1.529373in}}{\pgfqpoint{9.415104in}{1.525806in}}%
\pgfpathcurveto{\pgfqpoint{9.418670in}{1.522240in}}{\pgfqpoint{9.423508in}{1.520236in}}{\pgfqpoint{9.428551in}{1.520236in}}%
\pgfpathclose%
\pgfusepath{fill}%
\end{pgfscope}%
\begin{pgfscope}%
\pgfpathrectangle{\pgfqpoint{6.572727in}{0.474100in}}{\pgfqpoint{4.227273in}{3.318700in}}%
\pgfusepath{clip}%
\pgfsetbuttcap%
\pgfsetroundjoin%
\definecolor{currentfill}{rgb}{0.267004,0.004874,0.329415}%
\pgfsetfillcolor{currentfill}%
\pgfsetfillopacity{0.700000}%
\pgfsetlinewidth{0.000000pt}%
\definecolor{currentstroke}{rgb}{0.000000,0.000000,0.000000}%
\pgfsetstrokecolor{currentstroke}%
\pgfsetstrokeopacity{0.700000}%
\pgfsetdash{}{0pt}%
\pgfpathmoveto{\pgfqpoint{7.940758in}{1.740042in}}%
\pgfpathcurveto{\pgfqpoint{7.945801in}{1.740042in}}{\pgfqpoint{7.950639in}{1.742046in}}{\pgfqpoint{7.954206in}{1.745612in}}%
\pgfpathcurveto{\pgfqpoint{7.957772in}{1.749179in}}{\pgfqpoint{7.959776in}{1.754017in}}{\pgfqpoint{7.959776in}{1.759060in}}%
\pgfpathcurveto{\pgfqpoint{7.959776in}{1.764104in}}{\pgfqpoint{7.957772in}{1.768942in}}{\pgfqpoint{7.954206in}{1.772508in}}%
\pgfpathcurveto{\pgfqpoint{7.950639in}{1.776075in}}{\pgfqpoint{7.945801in}{1.778078in}}{\pgfqpoint{7.940758in}{1.778078in}}%
\pgfpathcurveto{\pgfqpoint{7.935714in}{1.778078in}}{\pgfqpoint{7.930876in}{1.776075in}}{\pgfqpoint{7.927310in}{1.772508in}}%
\pgfpathcurveto{\pgfqpoint{7.923743in}{1.768942in}}{\pgfqpoint{7.921740in}{1.764104in}}{\pgfqpoint{7.921740in}{1.759060in}}%
\pgfpathcurveto{\pgfqpoint{7.921740in}{1.754017in}}{\pgfqpoint{7.923743in}{1.749179in}}{\pgfqpoint{7.927310in}{1.745612in}}%
\pgfpathcurveto{\pgfqpoint{7.930876in}{1.742046in}}{\pgfqpoint{7.935714in}{1.740042in}}{\pgfqpoint{7.940758in}{1.740042in}}%
\pgfpathclose%
\pgfusepath{fill}%
\end{pgfscope}%
\begin{pgfscope}%
\pgfpathrectangle{\pgfqpoint{6.572727in}{0.474100in}}{\pgfqpoint{4.227273in}{3.318700in}}%
\pgfusepath{clip}%
\pgfsetbuttcap%
\pgfsetroundjoin%
\definecolor{currentfill}{rgb}{0.127568,0.566949,0.550556}%
\pgfsetfillcolor{currentfill}%
\pgfsetfillopacity{0.700000}%
\pgfsetlinewidth{0.000000pt}%
\definecolor{currentstroke}{rgb}{0.000000,0.000000,0.000000}%
\pgfsetstrokecolor{currentstroke}%
\pgfsetstrokeopacity{0.700000}%
\pgfsetdash{}{0pt}%
\pgfpathmoveto{\pgfqpoint{9.011288in}{1.514011in}}%
\pgfpathcurveto{\pgfqpoint{9.016332in}{1.514011in}}{\pgfqpoint{9.021170in}{1.516014in}}{\pgfqpoint{9.024736in}{1.519581in}}%
\pgfpathcurveto{\pgfqpoint{9.028303in}{1.523147in}}{\pgfqpoint{9.030307in}{1.527985in}}{\pgfqpoint{9.030307in}{1.533029in}}%
\pgfpathcurveto{\pgfqpoint{9.030307in}{1.538072in}}{\pgfqpoint{9.028303in}{1.542910in}}{\pgfqpoint{9.024736in}{1.546477in}}%
\pgfpathcurveto{\pgfqpoint{9.021170in}{1.550043in}}{\pgfqpoint{9.016332in}{1.552047in}}{\pgfqpoint{9.011288in}{1.552047in}}%
\pgfpathcurveto{\pgfqpoint{9.006245in}{1.552047in}}{\pgfqpoint{9.001407in}{1.550043in}}{\pgfqpoint{8.997841in}{1.546477in}}%
\pgfpathcurveto{\pgfqpoint{8.994274in}{1.542910in}}{\pgfqpoint{8.992270in}{1.538072in}}{\pgfqpoint{8.992270in}{1.533029in}}%
\pgfpathcurveto{\pgfqpoint{8.992270in}{1.527985in}}{\pgfqpoint{8.994274in}{1.523147in}}{\pgfqpoint{8.997841in}{1.519581in}}%
\pgfpathcurveto{\pgfqpoint{9.001407in}{1.516014in}}{\pgfqpoint{9.006245in}{1.514011in}}{\pgfqpoint{9.011288in}{1.514011in}}%
\pgfpathclose%
\pgfusepath{fill}%
\end{pgfscope}%
\begin{pgfscope}%
\pgfpathrectangle{\pgfqpoint{6.572727in}{0.474100in}}{\pgfqpoint{4.227273in}{3.318700in}}%
\pgfusepath{clip}%
\pgfsetbuttcap%
\pgfsetroundjoin%
\definecolor{currentfill}{rgb}{0.127568,0.566949,0.550556}%
\pgfsetfillcolor{currentfill}%
\pgfsetfillopacity{0.700000}%
\pgfsetlinewidth{0.000000pt}%
\definecolor{currentstroke}{rgb}{0.000000,0.000000,0.000000}%
\pgfsetstrokecolor{currentstroke}%
\pgfsetstrokeopacity{0.700000}%
\pgfsetdash{}{0pt}%
\pgfpathmoveto{\pgfqpoint{10.085005in}{0.856734in}}%
\pgfpathcurveto{\pgfqpoint{10.090049in}{0.856734in}}{\pgfqpoint{10.094887in}{0.858738in}}{\pgfqpoint{10.098453in}{0.862305in}}%
\pgfpathcurveto{\pgfqpoint{10.102019in}{0.865871in}}{\pgfqpoint{10.104023in}{0.870709in}}{\pgfqpoint{10.104023in}{0.875753in}}%
\pgfpathcurveto{\pgfqpoint{10.104023in}{0.880796in}}{\pgfqpoint{10.102019in}{0.885634in}}{\pgfqpoint{10.098453in}{0.889200in}}%
\pgfpathcurveto{\pgfqpoint{10.094887in}{0.892767in}}{\pgfqpoint{10.090049in}{0.894771in}}{\pgfqpoint{10.085005in}{0.894771in}}%
\pgfpathcurveto{\pgfqpoint{10.079962in}{0.894771in}}{\pgfqpoint{10.075124in}{0.892767in}}{\pgfqpoint{10.071557in}{0.889200in}}%
\pgfpathcurveto{\pgfqpoint{10.067991in}{0.885634in}}{\pgfqpoint{10.065987in}{0.880796in}}{\pgfqpoint{10.065987in}{0.875753in}}%
\pgfpathcurveto{\pgfqpoint{10.065987in}{0.870709in}}{\pgfqpoint{10.067991in}{0.865871in}}{\pgfqpoint{10.071557in}{0.862305in}}%
\pgfpathcurveto{\pgfqpoint{10.075124in}{0.858738in}}{\pgfqpoint{10.079962in}{0.856734in}}{\pgfqpoint{10.085005in}{0.856734in}}%
\pgfpathclose%
\pgfusepath{fill}%
\end{pgfscope}%
\begin{pgfscope}%
\pgfpathrectangle{\pgfqpoint{6.572727in}{0.474100in}}{\pgfqpoint{4.227273in}{3.318700in}}%
\pgfusepath{clip}%
\pgfsetbuttcap%
\pgfsetroundjoin%
\definecolor{currentfill}{rgb}{0.993248,0.906157,0.143936}%
\pgfsetfillcolor{currentfill}%
\pgfsetfillopacity{0.700000}%
\pgfsetlinewidth{0.000000pt}%
\definecolor{currentstroke}{rgb}{0.000000,0.000000,0.000000}%
\pgfsetstrokecolor{currentstroke}%
\pgfsetstrokeopacity{0.700000}%
\pgfsetdash{}{0pt}%
\pgfpathmoveto{\pgfqpoint{8.170251in}{2.777107in}}%
\pgfpathcurveto{\pgfqpoint{8.175295in}{2.777107in}}{\pgfqpoint{8.180133in}{2.779111in}}{\pgfqpoint{8.183699in}{2.782678in}}%
\pgfpathcurveto{\pgfqpoint{8.187265in}{2.786244in}}{\pgfqpoint{8.189269in}{2.791082in}}{\pgfqpoint{8.189269in}{2.796126in}}%
\pgfpathcurveto{\pgfqpoint{8.189269in}{2.801169in}}{\pgfqpoint{8.187265in}{2.806007in}}{\pgfqpoint{8.183699in}{2.809573in}}%
\pgfpathcurveto{\pgfqpoint{8.180133in}{2.813140in}}{\pgfqpoint{8.175295in}{2.815144in}}{\pgfqpoint{8.170251in}{2.815144in}}%
\pgfpathcurveto{\pgfqpoint{8.165207in}{2.815144in}}{\pgfqpoint{8.160370in}{2.813140in}}{\pgfqpoint{8.156803in}{2.809573in}}%
\pgfpathcurveto{\pgfqpoint{8.153237in}{2.806007in}}{\pgfqpoint{8.151233in}{2.801169in}}{\pgfqpoint{8.151233in}{2.796126in}}%
\pgfpathcurveto{\pgfqpoint{8.151233in}{2.791082in}}{\pgfqpoint{8.153237in}{2.786244in}}{\pgfqpoint{8.156803in}{2.782678in}}%
\pgfpathcurveto{\pgfqpoint{8.160370in}{2.779111in}}{\pgfqpoint{8.165207in}{2.777107in}}{\pgfqpoint{8.170251in}{2.777107in}}%
\pgfpathclose%
\pgfusepath{fill}%
\end{pgfscope}%
\begin{pgfscope}%
\pgfpathrectangle{\pgfqpoint{6.572727in}{0.474100in}}{\pgfqpoint{4.227273in}{3.318700in}}%
\pgfusepath{clip}%
\pgfsetbuttcap%
\pgfsetroundjoin%
\definecolor{currentfill}{rgb}{0.267004,0.004874,0.329415}%
\pgfsetfillcolor{currentfill}%
\pgfsetfillopacity{0.700000}%
\pgfsetlinewidth{0.000000pt}%
\definecolor{currentstroke}{rgb}{0.000000,0.000000,0.000000}%
\pgfsetstrokecolor{currentstroke}%
\pgfsetstrokeopacity{0.700000}%
\pgfsetdash{}{0pt}%
\pgfpathmoveto{\pgfqpoint{7.906933in}{1.507738in}}%
\pgfpathcurveto{\pgfqpoint{7.911977in}{1.507738in}}{\pgfqpoint{7.916814in}{1.509742in}}{\pgfqpoint{7.920381in}{1.513308in}}%
\pgfpathcurveto{\pgfqpoint{7.923947in}{1.516875in}}{\pgfqpoint{7.925951in}{1.521713in}}{\pgfqpoint{7.925951in}{1.526756in}}%
\pgfpathcurveto{\pgfqpoint{7.925951in}{1.531800in}}{\pgfqpoint{7.923947in}{1.536638in}}{\pgfqpoint{7.920381in}{1.540204in}}%
\pgfpathcurveto{\pgfqpoint{7.916814in}{1.543770in}}{\pgfqpoint{7.911977in}{1.545774in}}{\pgfqpoint{7.906933in}{1.545774in}}%
\pgfpathcurveto{\pgfqpoint{7.901889in}{1.545774in}}{\pgfqpoint{7.897052in}{1.543770in}}{\pgfqpoint{7.893485in}{1.540204in}}%
\pgfpathcurveto{\pgfqpoint{7.889919in}{1.536638in}}{\pgfqpoint{7.887915in}{1.531800in}}{\pgfqpoint{7.887915in}{1.526756in}}%
\pgfpathcurveto{\pgfqpoint{7.887915in}{1.521713in}}{\pgfqpoint{7.889919in}{1.516875in}}{\pgfqpoint{7.893485in}{1.513308in}}%
\pgfpathcurveto{\pgfqpoint{7.897052in}{1.509742in}}{\pgfqpoint{7.901889in}{1.507738in}}{\pgfqpoint{7.906933in}{1.507738in}}%
\pgfpathclose%
\pgfusepath{fill}%
\end{pgfscope}%
\begin{pgfscope}%
\pgfpathrectangle{\pgfqpoint{6.572727in}{0.474100in}}{\pgfqpoint{4.227273in}{3.318700in}}%
\pgfusepath{clip}%
\pgfsetbuttcap%
\pgfsetroundjoin%
\definecolor{currentfill}{rgb}{0.267004,0.004874,0.329415}%
\pgfsetfillcolor{currentfill}%
\pgfsetfillopacity{0.700000}%
\pgfsetlinewidth{0.000000pt}%
\definecolor{currentstroke}{rgb}{0.000000,0.000000,0.000000}%
\pgfsetstrokecolor{currentstroke}%
\pgfsetstrokeopacity{0.700000}%
\pgfsetdash{}{0pt}%
\pgfpathmoveto{\pgfqpoint{7.456083in}{0.898283in}}%
\pgfpathcurveto{\pgfqpoint{7.461127in}{0.898283in}}{\pgfqpoint{7.465965in}{0.900287in}}{\pgfqpoint{7.469531in}{0.903854in}}%
\pgfpathcurveto{\pgfqpoint{7.473097in}{0.907420in}}{\pgfqpoint{7.475101in}{0.912258in}}{\pgfqpoint{7.475101in}{0.917301in}}%
\pgfpathcurveto{\pgfqpoint{7.475101in}{0.922345in}}{\pgfqpoint{7.473097in}{0.927183in}}{\pgfqpoint{7.469531in}{0.930749in}}%
\pgfpathcurveto{\pgfqpoint{7.465965in}{0.934316in}}{\pgfqpoint{7.461127in}{0.936320in}}{\pgfqpoint{7.456083in}{0.936320in}}%
\pgfpathcurveto{\pgfqpoint{7.451039in}{0.936320in}}{\pgfqpoint{7.446202in}{0.934316in}}{\pgfqpoint{7.442635in}{0.930749in}}%
\pgfpathcurveto{\pgfqpoint{7.439069in}{0.927183in}}{\pgfqpoint{7.437065in}{0.922345in}}{\pgfqpoint{7.437065in}{0.917301in}}%
\pgfpathcurveto{\pgfqpoint{7.437065in}{0.912258in}}{\pgfqpoint{7.439069in}{0.907420in}}{\pgfqpoint{7.442635in}{0.903854in}}%
\pgfpathcurveto{\pgfqpoint{7.446202in}{0.900287in}}{\pgfqpoint{7.451039in}{0.898283in}}{\pgfqpoint{7.456083in}{0.898283in}}%
\pgfpathclose%
\pgfusepath{fill}%
\end{pgfscope}%
\begin{pgfscope}%
\pgfpathrectangle{\pgfqpoint{6.572727in}{0.474100in}}{\pgfqpoint{4.227273in}{3.318700in}}%
\pgfusepath{clip}%
\pgfsetbuttcap%
\pgfsetroundjoin%
\definecolor{currentfill}{rgb}{0.267004,0.004874,0.329415}%
\pgfsetfillcolor{currentfill}%
\pgfsetfillopacity{0.700000}%
\pgfsetlinewidth{0.000000pt}%
\definecolor{currentstroke}{rgb}{0.000000,0.000000,0.000000}%
\pgfsetstrokecolor{currentstroke}%
\pgfsetstrokeopacity{0.700000}%
\pgfsetdash{}{0pt}%
\pgfpathmoveto{\pgfqpoint{7.844510in}{1.237000in}}%
\pgfpathcurveto{\pgfqpoint{7.849554in}{1.237000in}}{\pgfqpoint{7.854392in}{1.239003in}}{\pgfqpoint{7.857958in}{1.242570in}}%
\pgfpathcurveto{\pgfqpoint{7.861525in}{1.246136in}}{\pgfqpoint{7.863528in}{1.250974in}}{\pgfqpoint{7.863528in}{1.256018in}}%
\pgfpathcurveto{\pgfqpoint{7.863528in}{1.261061in}}{\pgfqpoint{7.861525in}{1.265899in}}{\pgfqpoint{7.857958in}{1.269466in}}%
\pgfpathcurveto{\pgfqpoint{7.854392in}{1.273032in}}{\pgfqpoint{7.849554in}{1.275036in}}{\pgfqpoint{7.844510in}{1.275036in}}%
\pgfpathcurveto{\pgfqpoint{7.839467in}{1.275036in}}{\pgfqpoint{7.834629in}{1.273032in}}{\pgfqpoint{7.831062in}{1.269466in}}%
\pgfpathcurveto{\pgfqpoint{7.827496in}{1.265899in}}{\pgfqpoint{7.825492in}{1.261061in}}{\pgfqpoint{7.825492in}{1.256018in}}%
\pgfpathcurveto{\pgfqpoint{7.825492in}{1.250974in}}{\pgfqpoint{7.827496in}{1.246136in}}{\pgfqpoint{7.831062in}{1.242570in}}%
\pgfpathcurveto{\pgfqpoint{7.834629in}{1.239003in}}{\pgfqpoint{7.839467in}{1.237000in}}{\pgfqpoint{7.844510in}{1.237000in}}%
\pgfpathclose%
\pgfusepath{fill}%
\end{pgfscope}%
\begin{pgfscope}%
\pgfpathrectangle{\pgfqpoint{6.572727in}{0.474100in}}{\pgfqpoint{4.227273in}{3.318700in}}%
\pgfusepath{clip}%
\pgfsetbuttcap%
\pgfsetroundjoin%
\definecolor{currentfill}{rgb}{0.127568,0.566949,0.550556}%
\pgfsetfillcolor{currentfill}%
\pgfsetfillopacity{0.700000}%
\pgfsetlinewidth{0.000000pt}%
\definecolor{currentstroke}{rgb}{0.000000,0.000000,0.000000}%
\pgfsetstrokecolor{currentstroke}%
\pgfsetstrokeopacity{0.700000}%
\pgfsetdash{}{0pt}%
\pgfpathmoveto{\pgfqpoint{9.944993in}{0.943829in}}%
\pgfpathcurveto{\pgfqpoint{9.950036in}{0.943829in}}{\pgfqpoint{9.954874in}{0.945832in}}{\pgfqpoint{9.958441in}{0.949399in}}%
\pgfpathcurveto{\pgfqpoint{9.962007in}{0.952965in}}{\pgfqpoint{9.964011in}{0.957803in}}{\pgfqpoint{9.964011in}{0.962847in}}%
\pgfpathcurveto{\pgfqpoint{9.964011in}{0.967890in}}{\pgfqpoint{9.962007in}{0.972728in}}{\pgfqpoint{9.958441in}{0.976295in}}%
\pgfpathcurveto{\pgfqpoint{9.954874in}{0.979861in}}{\pgfqpoint{9.950036in}{0.981865in}}{\pgfqpoint{9.944993in}{0.981865in}}%
\pgfpathcurveto{\pgfqpoint{9.939949in}{0.981865in}}{\pgfqpoint{9.935111in}{0.979861in}}{\pgfqpoint{9.931545in}{0.976295in}}%
\pgfpathcurveto{\pgfqpoint{9.927978in}{0.972728in}}{\pgfqpoint{9.925975in}{0.967890in}}{\pgfqpoint{9.925975in}{0.962847in}}%
\pgfpathcurveto{\pgfqpoint{9.925975in}{0.957803in}}{\pgfqpoint{9.927978in}{0.952965in}}{\pgfqpoint{9.931545in}{0.949399in}}%
\pgfpathcurveto{\pgfqpoint{9.935111in}{0.945832in}}{\pgfqpoint{9.939949in}{0.943829in}}{\pgfqpoint{9.944993in}{0.943829in}}%
\pgfpathclose%
\pgfusepath{fill}%
\end{pgfscope}%
\begin{pgfscope}%
\pgfpathrectangle{\pgfqpoint{6.572727in}{0.474100in}}{\pgfqpoint{4.227273in}{3.318700in}}%
\pgfusepath{clip}%
\pgfsetbuttcap%
\pgfsetroundjoin%
\definecolor{currentfill}{rgb}{0.267004,0.004874,0.329415}%
\pgfsetfillcolor{currentfill}%
\pgfsetfillopacity{0.700000}%
\pgfsetlinewidth{0.000000pt}%
\definecolor{currentstroke}{rgb}{0.000000,0.000000,0.000000}%
\pgfsetstrokecolor{currentstroke}%
\pgfsetstrokeopacity{0.700000}%
\pgfsetdash{}{0pt}%
\pgfpathmoveto{\pgfqpoint{7.891715in}{1.694903in}}%
\pgfpathcurveto{\pgfqpoint{7.896759in}{1.694903in}}{\pgfqpoint{7.901597in}{1.696907in}}{\pgfqpoint{7.905163in}{1.700473in}}%
\pgfpathcurveto{\pgfqpoint{7.908729in}{1.704040in}}{\pgfqpoint{7.910733in}{1.708877in}}{\pgfqpoint{7.910733in}{1.713921in}}%
\pgfpathcurveto{\pgfqpoint{7.910733in}{1.718965in}}{\pgfqpoint{7.908729in}{1.723803in}}{\pgfqpoint{7.905163in}{1.727369in}}%
\pgfpathcurveto{\pgfqpoint{7.901597in}{1.730935in}}{\pgfqpoint{7.896759in}{1.732939in}}{\pgfqpoint{7.891715in}{1.732939in}}%
\pgfpathcurveto{\pgfqpoint{7.886672in}{1.732939in}}{\pgfqpoint{7.881834in}{1.730935in}}{\pgfqpoint{7.878267in}{1.727369in}}%
\pgfpathcurveto{\pgfqpoint{7.874701in}{1.723803in}}{\pgfqpoint{7.872697in}{1.718965in}}{\pgfqpoint{7.872697in}{1.713921in}}%
\pgfpathcurveto{\pgfqpoint{7.872697in}{1.708877in}}{\pgfqpoint{7.874701in}{1.704040in}}{\pgfqpoint{7.878267in}{1.700473in}}%
\pgfpathcurveto{\pgfqpoint{7.881834in}{1.696907in}}{\pgfqpoint{7.886672in}{1.694903in}}{\pgfqpoint{7.891715in}{1.694903in}}%
\pgfpathclose%
\pgfusepath{fill}%
\end{pgfscope}%
\begin{pgfscope}%
\pgfpathrectangle{\pgfqpoint{6.572727in}{0.474100in}}{\pgfqpoint{4.227273in}{3.318700in}}%
\pgfusepath{clip}%
\pgfsetbuttcap%
\pgfsetroundjoin%
\definecolor{currentfill}{rgb}{0.993248,0.906157,0.143936}%
\pgfsetfillcolor{currentfill}%
\pgfsetfillopacity{0.700000}%
\pgfsetlinewidth{0.000000pt}%
\definecolor{currentstroke}{rgb}{0.000000,0.000000,0.000000}%
\pgfsetstrokecolor{currentstroke}%
\pgfsetstrokeopacity{0.700000}%
\pgfsetdash{}{0pt}%
\pgfpathmoveto{\pgfqpoint{8.383135in}{2.445289in}}%
\pgfpathcurveto{\pgfqpoint{8.388178in}{2.445289in}}{\pgfqpoint{8.393016in}{2.447293in}}{\pgfqpoint{8.396583in}{2.450859in}}%
\pgfpathcurveto{\pgfqpoint{8.400149in}{2.454425in}}{\pgfqpoint{8.402153in}{2.459263in}}{\pgfqpoint{8.402153in}{2.464307in}}%
\pgfpathcurveto{\pgfqpoint{8.402153in}{2.469351in}}{\pgfqpoint{8.400149in}{2.474188in}}{\pgfqpoint{8.396583in}{2.477755in}}%
\pgfpathcurveto{\pgfqpoint{8.393016in}{2.481321in}}{\pgfqpoint{8.388178in}{2.483325in}}{\pgfqpoint{8.383135in}{2.483325in}}%
\pgfpathcurveto{\pgfqpoint{8.378091in}{2.483325in}}{\pgfqpoint{8.373253in}{2.481321in}}{\pgfqpoint{8.369687in}{2.477755in}}%
\pgfpathcurveto{\pgfqpoint{8.366120in}{2.474188in}}{\pgfqpoint{8.364117in}{2.469351in}}{\pgfqpoint{8.364117in}{2.464307in}}%
\pgfpathcurveto{\pgfqpoint{8.364117in}{2.459263in}}{\pgfqpoint{8.366120in}{2.454425in}}{\pgfqpoint{8.369687in}{2.450859in}}%
\pgfpathcurveto{\pgfqpoint{8.373253in}{2.447293in}}{\pgfqpoint{8.378091in}{2.445289in}}{\pgfqpoint{8.383135in}{2.445289in}}%
\pgfpathclose%
\pgfusepath{fill}%
\end{pgfscope}%
\begin{pgfscope}%
\pgfpathrectangle{\pgfqpoint{6.572727in}{0.474100in}}{\pgfqpoint{4.227273in}{3.318700in}}%
\pgfusepath{clip}%
\pgfsetbuttcap%
\pgfsetroundjoin%
\definecolor{currentfill}{rgb}{0.267004,0.004874,0.329415}%
\pgfsetfillcolor{currentfill}%
\pgfsetfillopacity{0.700000}%
\pgfsetlinewidth{0.000000pt}%
\definecolor{currentstroke}{rgb}{0.000000,0.000000,0.000000}%
\pgfsetstrokecolor{currentstroke}%
\pgfsetstrokeopacity{0.700000}%
\pgfsetdash{}{0pt}%
\pgfpathmoveto{\pgfqpoint{8.008951in}{1.930338in}}%
\pgfpathcurveto{\pgfqpoint{8.013995in}{1.930338in}}{\pgfqpoint{8.018833in}{1.932342in}}{\pgfqpoint{8.022399in}{1.935908in}}%
\pgfpathcurveto{\pgfqpoint{8.025966in}{1.939475in}}{\pgfqpoint{8.027969in}{1.944312in}}{\pgfqpoint{8.027969in}{1.949356in}}%
\pgfpathcurveto{\pgfqpoint{8.027969in}{1.954400in}}{\pgfqpoint{8.025966in}{1.959238in}}{\pgfqpoint{8.022399in}{1.962804in}}%
\pgfpathcurveto{\pgfqpoint{8.018833in}{1.966370in}}{\pgfqpoint{8.013995in}{1.968374in}}{\pgfqpoint{8.008951in}{1.968374in}}%
\pgfpathcurveto{\pgfqpoint{8.003908in}{1.968374in}}{\pgfqpoint{7.999070in}{1.966370in}}{\pgfqpoint{7.995503in}{1.962804in}}%
\pgfpathcurveto{\pgfqpoint{7.991937in}{1.959238in}}{\pgfqpoint{7.989933in}{1.954400in}}{\pgfqpoint{7.989933in}{1.949356in}}%
\pgfpathcurveto{\pgfqpoint{7.989933in}{1.944312in}}{\pgfqpoint{7.991937in}{1.939475in}}{\pgfqpoint{7.995503in}{1.935908in}}%
\pgfpathcurveto{\pgfqpoint{7.999070in}{1.932342in}}{\pgfqpoint{8.003908in}{1.930338in}}{\pgfqpoint{8.008951in}{1.930338in}}%
\pgfpathclose%
\pgfusepath{fill}%
\end{pgfscope}%
\begin{pgfscope}%
\pgfpathrectangle{\pgfqpoint{6.572727in}{0.474100in}}{\pgfqpoint{4.227273in}{3.318700in}}%
\pgfusepath{clip}%
\pgfsetbuttcap%
\pgfsetroundjoin%
\definecolor{currentfill}{rgb}{0.993248,0.906157,0.143936}%
\pgfsetfillcolor{currentfill}%
\pgfsetfillopacity{0.700000}%
\pgfsetlinewidth{0.000000pt}%
\definecolor{currentstroke}{rgb}{0.000000,0.000000,0.000000}%
\pgfsetstrokecolor{currentstroke}%
\pgfsetstrokeopacity{0.700000}%
\pgfsetdash{}{0pt}%
\pgfpathmoveto{\pgfqpoint{8.280721in}{3.009439in}}%
\pgfpathcurveto{\pgfqpoint{8.285764in}{3.009439in}}{\pgfqpoint{8.290602in}{3.011443in}}{\pgfqpoint{8.294168in}{3.015009in}}%
\pgfpathcurveto{\pgfqpoint{8.297735in}{3.018576in}}{\pgfqpoint{8.299739in}{3.023413in}}{\pgfqpoint{8.299739in}{3.028457in}}%
\pgfpathcurveto{\pgfqpoint{8.299739in}{3.033501in}}{\pgfqpoint{8.297735in}{3.038338in}}{\pgfqpoint{8.294168in}{3.041905in}}%
\pgfpathcurveto{\pgfqpoint{8.290602in}{3.045471in}}{\pgfqpoint{8.285764in}{3.047475in}}{\pgfqpoint{8.280721in}{3.047475in}}%
\pgfpathcurveto{\pgfqpoint{8.275677in}{3.047475in}}{\pgfqpoint{8.270839in}{3.045471in}}{\pgfqpoint{8.267273in}{3.041905in}}%
\pgfpathcurveto{\pgfqpoint{8.263706in}{3.038338in}}{\pgfqpoint{8.261702in}{3.033501in}}{\pgfqpoint{8.261702in}{3.028457in}}%
\pgfpathcurveto{\pgfqpoint{8.261702in}{3.023413in}}{\pgfqpoint{8.263706in}{3.018576in}}{\pgfqpoint{8.267273in}{3.015009in}}%
\pgfpathcurveto{\pgfqpoint{8.270839in}{3.011443in}}{\pgfqpoint{8.275677in}{3.009439in}}{\pgfqpoint{8.280721in}{3.009439in}}%
\pgfpathclose%
\pgfusepath{fill}%
\end{pgfscope}%
\begin{pgfscope}%
\pgfpathrectangle{\pgfqpoint{6.572727in}{0.474100in}}{\pgfqpoint{4.227273in}{3.318700in}}%
\pgfusepath{clip}%
\pgfsetbuttcap%
\pgfsetroundjoin%
\definecolor{currentfill}{rgb}{0.267004,0.004874,0.329415}%
\pgfsetfillcolor{currentfill}%
\pgfsetfillopacity{0.700000}%
\pgfsetlinewidth{0.000000pt}%
\definecolor{currentstroke}{rgb}{0.000000,0.000000,0.000000}%
\pgfsetstrokecolor{currentstroke}%
\pgfsetstrokeopacity{0.700000}%
\pgfsetdash{}{0pt}%
\pgfpathmoveto{\pgfqpoint{8.272375in}{1.625308in}}%
\pgfpathcurveto{\pgfqpoint{8.277419in}{1.625308in}}{\pgfqpoint{8.282256in}{1.627312in}}{\pgfqpoint{8.285823in}{1.630879in}}%
\pgfpathcurveto{\pgfqpoint{8.289389in}{1.634445in}}{\pgfqpoint{8.291393in}{1.639283in}}{\pgfqpoint{8.291393in}{1.644327in}}%
\pgfpathcurveto{\pgfqpoint{8.291393in}{1.649370in}}{\pgfqpoint{8.289389in}{1.654208in}}{\pgfqpoint{8.285823in}{1.657774in}}%
\pgfpathcurveto{\pgfqpoint{8.282256in}{1.661341in}}{\pgfqpoint{8.277419in}{1.663345in}}{\pgfqpoint{8.272375in}{1.663345in}}%
\pgfpathcurveto{\pgfqpoint{8.267331in}{1.663345in}}{\pgfqpoint{8.262494in}{1.661341in}}{\pgfqpoint{8.258927in}{1.657774in}}%
\pgfpathcurveto{\pgfqpoint{8.255361in}{1.654208in}}{\pgfqpoint{8.253357in}{1.649370in}}{\pgfqpoint{8.253357in}{1.644327in}}%
\pgfpathcurveto{\pgfqpoint{8.253357in}{1.639283in}}{\pgfqpoint{8.255361in}{1.634445in}}{\pgfqpoint{8.258927in}{1.630879in}}%
\pgfpathcurveto{\pgfqpoint{8.262494in}{1.627312in}}{\pgfqpoint{8.267331in}{1.625308in}}{\pgfqpoint{8.272375in}{1.625308in}}%
\pgfpathclose%
\pgfusepath{fill}%
\end{pgfscope}%
\begin{pgfscope}%
\pgfpathrectangle{\pgfqpoint{6.572727in}{0.474100in}}{\pgfqpoint{4.227273in}{3.318700in}}%
\pgfusepath{clip}%
\pgfsetbuttcap%
\pgfsetroundjoin%
\definecolor{currentfill}{rgb}{0.993248,0.906157,0.143936}%
\pgfsetfillcolor{currentfill}%
\pgfsetfillopacity{0.700000}%
\pgfsetlinewidth{0.000000pt}%
\definecolor{currentstroke}{rgb}{0.000000,0.000000,0.000000}%
\pgfsetstrokecolor{currentstroke}%
\pgfsetstrokeopacity{0.700000}%
\pgfsetdash{}{0pt}%
\pgfpathmoveto{\pgfqpoint{8.306057in}{3.026902in}}%
\pgfpathcurveto{\pgfqpoint{8.311101in}{3.026902in}}{\pgfqpoint{8.315939in}{3.028906in}}{\pgfqpoint{8.319505in}{3.032473in}}%
\pgfpathcurveto{\pgfqpoint{8.323071in}{3.036039in}}{\pgfqpoint{8.325075in}{3.040877in}}{\pgfqpoint{8.325075in}{3.045920in}}%
\pgfpathcurveto{\pgfqpoint{8.325075in}{3.050964in}}{\pgfqpoint{8.323071in}{3.055802in}}{\pgfqpoint{8.319505in}{3.059368in}}%
\pgfpathcurveto{\pgfqpoint{8.315939in}{3.062935in}}{\pgfqpoint{8.311101in}{3.064939in}}{\pgfqpoint{8.306057in}{3.064939in}}%
\pgfpathcurveto{\pgfqpoint{8.301013in}{3.064939in}}{\pgfqpoint{8.296176in}{3.062935in}}{\pgfqpoint{8.292609in}{3.059368in}}%
\pgfpathcurveto{\pgfqpoint{8.289043in}{3.055802in}}{\pgfqpoint{8.287039in}{3.050964in}}{\pgfqpoint{8.287039in}{3.045920in}}%
\pgfpathcurveto{\pgfqpoint{8.287039in}{3.040877in}}{\pgfqpoint{8.289043in}{3.036039in}}{\pgfqpoint{8.292609in}{3.032473in}}%
\pgfpathcurveto{\pgfqpoint{8.296176in}{3.028906in}}{\pgfqpoint{8.301013in}{3.026902in}}{\pgfqpoint{8.306057in}{3.026902in}}%
\pgfpathclose%
\pgfusepath{fill}%
\end{pgfscope}%
\begin{pgfscope}%
\pgfpathrectangle{\pgfqpoint{6.572727in}{0.474100in}}{\pgfqpoint{4.227273in}{3.318700in}}%
\pgfusepath{clip}%
\pgfsetbuttcap%
\pgfsetroundjoin%
\definecolor{currentfill}{rgb}{0.127568,0.566949,0.550556}%
\pgfsetfillcolor{currentfill}%
\pgfsetfillopacity{0.700000}%
\pgfsetlinewidth{0.000000pt}%
\definecolor{currentstroke}{rgb}{0.000000,0.000000,0.000000}%
\pgfsetstrokecolor{currentstroke}%
\pgfsetstrokeopacity{0.700000}%
\pgfsetdash{}{0pt}%
\pgfpathmoveto{\pgfqpoint{9.770135in}{1.553215in}}%
\pgfpathcurveto{\pgfqpoint{9.775179in}{1.553215in}}{\pgfqpoint{9.780017in}{1.555219in}}{\pgfqpoint{9.783583in}{1.558785in}}%
\pgfpathcurveto{\pgfqpoint{9.787150in}{1.562352in}}{\pgfqpoint{9.789154in}{1.567190in}}{\pgfqpoint{9.789154in}{1.572233in}}%
\pgfpathcurveto{\pgfqpoint{9.789154in}{1.577277in}}{\pgfqpoint{9.787150in}{1.582115in}}{\pgfqpoint{9.783583in}{1.585681in}}%
\pgfpathcurveto{\pgfqpoint{9.780017in}{1.589248in}}{\pgfqpoint{9.775179in}{1.591251in}}{\pgfqpoint{9.770135in}{1.591251in}}%
\pgfpathcurveto{\pgfqpoint{9.765092in}{1.591251in}}{\pgfqpoint{9.760254in}{1.589248in}}{\pgfqpoint{9.756688in}{1.585681in}}%
\pgfpathcurveto{\pgfqpoint{9.753121in}{1.582115in}}{\pgfqpoint{9.751117in}{1.577277in}}{\pgfqpoint{9.751117in}{1.572233in}}%
\pgfpathcurveto{\pgfqpoint{9.751117in}{1.567190in}}{\pgfqpoint{9.753121in}{1.562352in}}{\pgfqpoint{9.756688in}{1.558785in}}%
\pgfpathcurveto{\pgfqpoint{9.760254in}{1.555219in}}{\pgfqpoint{9.765092in}{1.553215in}}{\pgfqpoint{9.770135in}{1.553215in}}%
\pgfpathclose%
\pgfusepath{fill}%
\end{pgfscope}%
\begin{pgfscope}%
\pgfpathrectangle{\pgfqpoint{6.572727in}{0.474100in}}{\pgfqpoint{4.227273in}{3.318700in}}%
\pgfusepath{clip}%
\pgfsetbuttcap%
\pgfsetroundjoin%
\definecolor{currentfill}{rgb}{0.993248,0.906157,0.143936}%
\pgfsetfillcolor{currentfill}%
\pgfsetfillopacity{0.700000}%
\pgfsetlinewidth{0.000000pt}%
\definecolor{currentstroke}{rgb}{0.000000,0.000000,0.000000}%
\pgfsetstrokecolor{currentstroke}%
\pgfsetstrokeopacity{0.700000}%
\pgfsetdash{}{0pt}%
\pgfpathmoveto{\pgfqpoint{8.649131in}{3.249225in}}%
\pgfpathcurveto{\pgfqpoint{8.654175in}{3.249225in}}{\pgfqpoint{8.659012in}{3.251229in}}{\pgfqpoint{8.662579in}{3.254796in}}%
\pgfpathcurveto{\pgfqpoint{8.666145in}{3.258362in}}{\pgfqpoint{8.668149in}{3.263200in}}{\pgfqpoint{8.668149in}{3.268243in}}%
\pgfpathcurveto{\pgfqpoint{8.668149in}{3.273287in}}{\pgfqpoint{8.666145in}{3.278125in}}{\pgfqpoint{8.662579in}{3.281691in}}%
\pgfpathcurveto{\pgfqpoint{8.659012in}{3.285258in}}{\pgfqpoint{8.654175in}{3.287262in}}{\pgfqpoint{8.649131in}{3.287262in}}%
\pgfpathcurveto{\pgfqpoint{8.644087in}{3.287262in}}{\pgfqpoint{8.639250in}{3.285258in}}{\pgfqpoint{8.635683in}{3.281691in}}%
\pgfpathcurveto{\pgfqpoint{8.632117in}{3.278125in}}{\pgfqpoint{8.630113in}{3.273287in}}{\pgfqpoint{8.630113in}{3.268243in}}%
\pgfpathcurveto{\pgfqpoint{8.630113in}{3.263200in}}{\pgfqpoint{8.632117in}{3.258362in}}{\pgfqpoint{8.635683in}{3.254796in}}%
\pgfpathcurveto{\pgfqpoint{8.639250in}{3.251229in}}{\pgfqpoint{8.644087in}{3.249225in}}{\pgfqpoint{8.649131in}{3.249225in}}%
\pgfpathclose%
\pgfusepath{fill}%
\end{pgfscope}%
\begin{pgfscope}%
\pgfpathrectangle{\pgfqpoint{6.572727in}{0.474100in}}{\pgfqpoint{4.227273in}{3.318700in}}%
\pgfusepath{clip}%
\pgfsetbuttcap%
\pgfsetroundjoin%
\definecolor{currentfill}{rgb}{0.127568,0.566949,0.550556}%
\pgfsetfillcolor{currentfill}%
\pgfsetfillopacity{0.700000}%
\pgfsetlinewidth{0.000000pt}%
\definecolor{currentstroke}{rgb}{0.000000,0.000000,0.000000}%
\pgfsetstrokecolor{currentstroke}%
\pgfsetstrokeopacity{0.700000}%
\pgfsetdash{}{0pt}%
\pgfpathmoveto{\pgfqpoint{9.342677in}{1.071246in}}%
\pgfpathcurveto{\pgfqpoint{9.347721in}{1.071246in}}{\pgfqpoint{9.352558in}{1.073250in}}{\pgfqpoint{9.356125in}{1.076816in}}%
\pgfpathcurveto{\pgfqpoint{9.359691in}{1.080382in}}{\pgfqpoint{9.361695in}{1.085220in}}{\pgfqpoint{9.361695in}{1.090264in}}%
\pgfpathcurveto{\pgfqpoint{9.361695in}{1.095307in}}{\pgfqpoint{9.359691in}{1.100145in}}{\pgfqpoint{9.356125in}{1.103712in}}%
\pgfpathcurveto{\pgfqpoint{9.352558in}{1.107278in}}{\pgfqpoint{9.347721in}{1.109282in}}{\pgfqpoint{9.342677in}{1.109282in}}%
\pgfpathcurveto{\pgfqpoint{9.337633in}{1.109282in}}{\pgfqpoint{9.332796in}{1.107278in}}{\pgfqpoint{9.329229in}{1.103712in}}%
\pgfpathcurveto{\pgfqpoint{9.325663in}{1.100145in}}{\pgfqpoint{9.323659in}{1.095307in}}{\pgfqpoint{9.323659in}{1.090264in}}%
\pgfpathcurveto{\pgfqpoint{9.323659in}{1.085220in}}{\pgfqpoint{9.325663in}{1.080382in}}{\pgfqpoint{9.329229in}{1.076816in}}%
\pgfpathcurveto{\pgfqpoint{9.332796in}{1.073250in}}{\pgfqpoint{9.337633in}{1.071246in}}{\pgfqpoint{9.342677in}{1.071246in}}%
\pgfpathclose%
\pgfusepath{fill}%
\end{pgfscope}%
\begin{pgfscope}%
\pgfpathrectangle{\pgfqpoint{6.572727in}{0.474100in}}{\pgfqpoint{4.227273in}{3.318700in}}%
\pgfusepath{clip}%
\pgfsetbuttcap%
\pgfsetroundjoin%
\definecolor{currentfill}{rgb}{0.127568,0.566949,0.550556}%
\pgfsetfillcolor{currentfill}%
\pgfsetfillopacity{0.700000}%
\pgfsetlinewidth{0.000000pt}%
\definecolor{currentstroke}{rgb}{0.000000,0.000000,0.000000}%
\pgfsetstrokecolor{currentstroke}%
\pgfsetstrokeopacity{0.700000}%
\pgfsetdash{}{0pt}%
\pgfpathmoveto{\pgfqpoint{9.502821in}{1.553700in}}%
\pgfpathcurveto{\pgfqpoint{9.507864in}{1.553700in}}{\pgfqpoint{9.512702in}{1.555704in}}{\pgfqpoint{9.516268in}{1.559271in}}%
\pgfpathcurveto{\pgfqpoint{9.519835in}{1.562837in}}{\pgfqpoint{9.521839in}{1.567675in}}{\pgfqpoint{9.521839in}{1.572719in}}%
\pgfpathcurveto{\pgfqpoint{9.521839in}{1.577762in}}{\pgfqpoint{9.519835in}{1.582600in}}{\pgfqpoint{9.516268in}{1.586166in}}%
\pgfpathcurveto{\pgfqpoint{9.512702in}{1.589733in}}{\pgfqpoint{9.507864in}{1.591737in}}{\pgfqpoint{9.502821in}{1.591737in}}%
\pgfpathcurveto{\pgfqpoint{9.497777in}{1.591737in}}{\pgfqpoint{9.492939in}{1.589733in}}{\pgfqpoint{9.489373in}{1.586166in}}%
\pgfpathcurveto{\pgfqpoint{9.485806in}{1.582600in}}{\pgfqpoint{9.483802in}{1.577762in}}{\pgfqpoint{9.483802in}{1.572719in}}%
\pgfpathcurveto{\pgfqpoint{9.483802in}{1.567675in}}{\pgfqpoint{9.485806in}{1.562837in}}{\pgfqpoint{9.489373in}{1.559271in}}%
\pgfpathcurveto{\pgfqpoint{9.492939in}{1.555704in}}{\pgfqpoint{9.497777in}{1.553700in}}{\pgfqpoint{9.502821in}{1.553700in}}%
\pgfpathclose%
\pgfusepath{fill}%
\end{pgfscope}%
\begin{pgfscope}%
\pgfpathrectangle{\pgfqpoint{6.572727in}{0.474100in}}{\pgfqpoint{4.227273in}{3.318700in}}%
\pgfusepath{clip}%
\pgfsetbuttcap%
\pgfsetroundjoin%
\definecolor{currentfill}{rgb}{0.267004,0.004874,0.329415}%
\pgfsetfillcolor{currentfill}%
\pgfsetfillopacity{0.700000}%
\pgfsetlinewidth{0.000000pt}%
\definecolor{currentstroke}{rgb}{0.000000,0.000000,0.000000}%
\pgfsetstrokecolor{currentstroke}%
\pgfsetstrokeopacity{0.700000}%
\pgfsetdash{}{0pt}%
\pgfpathmoveto{\pgfqpoint{7.253956in}{1.404765in}}%
\pgfpathcurveto{\pgfqpoint{7.258999in}{1.404765in}}{\pgfqpoint{7.263837in}{1.406769in}}{\pgfqpoint{7.267404in}{1.410335in}}%
\pgfpathcurveto{\pgfqpoint{7.270970in}{1.413902in}}{\pgfqpoint{7.272974in}{1.418740in}}{\pgfqpoint{7.272974in}{1.423783in}}%
\pgfpathcurveto{\pgfqpoint{7.272974in}{1.428827in}}{\pgfqpoint{7.270970in}{1.433665in}}{\pgfqpoint{7.267404in}{1.437231in}}%
\pgfpathcurveto{\pgfqpoint{7.263837in}{1.440798in}}{\pgfqpoint{7.258999in}{1.442801in}}{\pgfqpoint{7.253956in}{1.442801in}}%
\pgfpathcurveto{\pgfqpoint{7.248912in}{1.442801in}}{\pgfqpoint{7.244074in}{1.440798in}}{\pgfqpoint{7.240508in}{1.437231in}}%
\pgfpathcurveto{\pgfqpoint{7.236942in}{1.433665in}}{\pgfqpoint{7.234938in}{1.428827in}}{\pgfqpoint{7.234938in}{1.423783in}}%
\pgfpathcurveto{\pgfqpoint{7.234938in}{1.418740in}}{\pgfqpoint{7.236942in}{1.413902in}}{\pgfqpoint{7.240508in}{1.410335in}}%
\pgfpathcurveto{\pgfqpoint{7.244074in}{1.406769in}}{\pgfqpoint{7.248912in}{1.404765in}}{\pgfqpoint{7.253956in}{1.404765in}}%
\pgfpathclose%
\pgfusepath{fill}%
\end{pgfscope}%
\begin{pgfscope}%
\pgfpathrectangle{\pgfqpoint{6.572727in}{0.474100in}}{\pgfqpoint{4.227273in}{3.318700in}}%
\pgfusepath{clip}%
\pgfsetbuttcap%
\pgfsetroundjoin%
\definecolor{currentfill}{rgb}{0.267004,0.004874,0.329415}%
\pgfsetfillcolor{currentfill}%
\pgfsetfillopacity{0.700000}%
\pgfsetlinewidth{0.000000pt}%
\definecolor{currentstroke}{rgb}{0.000000,0.000000,0.000000}%
\pgfsetstrokecolor{currentstroke}%
\pgfsetstrokeopacity{0.700000}%
\pgfsetdash{}{0pt}%
\pgfpathmoveto{\pgfqpoint{8.312186in}{1.696756in}}%
\pgfpathcurveto{\pgfqpoint{8.317230in}{1.696756in}}{\pgfqpoint{8.322068in}{1.698760in}}{\pgfqpoint{8.325634in}{1.702327in}}%
\pgfpathcurveto{\pgfqpoint{8.329200in}{1.705893in}}{\pgfqpoint{8.331204in}{1.710731in}}{\pgfqpoint{8.331204in}{1.715775in}}%
\pgfpathcurveto{\pgfqpoint{8.331204in}{1.720818in}}{\pgfqpoint{8.329200in}{1.725656in}}{\pgfqpoint{8.325634in}{1.729222in}}%
\pgfpathcurveto{\pgfqpoint{8.322068in}{1.732789in}}{\pgfqpoint{8.317230in}{1.734793in}}{\pgfqpoint{8.312186in}{1.734793in}}%
\pgfpathcurveto{\pgfqpoint{8.307142in}{1.734793in}}{\pgfqpoint{8.302305in}{1.732789in}}{\pgfqpoint{8.298738in}{1.729222in}}%
\pgfpathcurveto{\pgfqpoint{8.295172in}{1.725656in}}{\pgfqpoint{8.293168in}{1.720818in}}{\pgfqpoint{8.293168in}{1.715775in}}%
\pgfpathcurveto{\pgfqpoint{8.293168in}{1.710731in}}{\pgfqpoint{8.295172in}{1.705893in}}{\pgfqpoint{8.298738in}{1.702327in}}%
\pgfpathcurveto{\pgfqpoint{8.302305in}{1.698760in}}{\pgfqpoint{8.307142in}{1.696756in}}{\pgfqpoint{8.312186in}{1.696756in}}%
\pgfpathclose%
\pgfusepath{fill}%
\end{pgfscope}%
\begin{pgfscope}%
\pgfpathrectangle{\pgfqpoint{6.572727in}{0.474100in}}{\pgfqpoint{4.227273in}{3.318700in}}%
\pgfusepath{clip}%
\pgfsetbuttcap%
\pgfsetroundjoin%
\definecolor{currentfill}{rgb}{0.993248,0.906157,0.143936}%
\pgfsetfillcolor{currentfill}%
\pgfsetfillopacity{0.700000}%
\pgfsetlinewidth{0.000000pt}%
\definecolor{currentstroke}{rgb}{0.000000,0.000000,0.000000}%
\pgfsetstrokecolor{currentstroke}%
\pgfsetstrokeopacity{0.700000}%
\pgfsetdash{}{0pt}%
\pgfpathmoveto{\pgfqpoint{8.155875in}{3.323079in}}%
\pgfpathcurveto{\pgfqpoint{8.160918in}{3.323079in}}{\pgfqpoint{8.165756in}{3.325083in}}{\pgfqpoint{8.169322in}{3.328649in}}%
\pgfpathcurveto{\pgfqpoint{8.172889in}{3.332216in}}{\pgfqpoint{8.174893in}{3.337053in}}{\pgfqpoint{8.174893in}{3.342097in}}%
\pgfpathcurveto{\pgfqpoint{8.174893in}{3.347141in}}{\pgfqpoint{8.172889in}{3.351979in}}{\pgfqpoint{8.169322in}{3.355545in}}%
\pgfpathcurveto{\pgfqpoint{8.165756in}{3.359111in}}{\pgfqpoint{8.160918in}{3.361115in}}{\pgfqpoint{8.155875in}{3.361115in}}%
\pgfpathcurveto{\pgfqpoint{8.150831in}{3.361115in}}{\pgfqpoint{8.145993in}{3.359111in}}{\pgfqpoint{8.142427in}{3.355545in}}%
\pgfpathcurveto{\pgfqpoint{8.138860in}{3.351979in}}{\pgfqpoint{8.136856in}{3.347141in}}{\pgfqpoint{8.136856in}{3.342097in}}%
\pgfpathcurveto{\pgfqpoint{8.136856in}{3.337053in}}{\pgfqpoint{8.138860in}{3.332216in}}{\pgfqpoint{8.142427in}{3.328649in}}%
\pgfpathcurveto{\pgfqpoint{8.145993in}{3.325083in}}{\pgfqpoint{8.150831in}{3.323079in}}{\pgfqpoint{8.155875in}{3.323079in}}%
\pgfpathclose%
\pgfusepath{fill}%
\end{pgfscope}%
\begin{pgfscope}%
\pgfpathrectangle{\pgfqpoint{6.572727in}{0.474100in}}{\pgfqpoint{4.227273in}{3.318700in}}%
\pgfusepath{clip}%
\pgfsetbuttcap%
\pgfsetroundjoin%
\definecolor{currentfill}{rgb}{0.267004,0.004874,0.329415}%
\pgfsetfillcolor{currentfill}%
\pgfsetfillopacity{0.700000}%
\pgfsetlinewidth{0.000000pt}%
\definecolor{currentstroke}{rgb}{0.000000,0.000000,0.000000}%
\pgfsetstrokecolor{currentstroke}%
\pgfsetstrokeopacity{0.700000}%
\pgfsetdash{}{0pt}%
\pgfpathmoveto{\pgfqpoint{8.160949in}{1.371168in}}%
\pgfpathcurveto{\pgfqpoint{8.165993in}{1.371168in}}{\pgfqpoint{8.170831in}{1.373172in}}{\pgfqpoint{8.174397in}{1.376738in}}%
\pgfpathcurveto{\pgfqpoint{8.177964in}{1.380305in}}{\pgfqpoint{8.179968in}{1.385142in}}{\pgfqpoint{8.179968in}{1.390186in}}%
\pgfpathcurveto{\pgfqpoint{8.179968in}{1.395230in}}{\pgfqpoint{8.177964in}{1.400068in}}{\pgfqpoint{8.174397in}{1.403634in}}%
\pgfpathcurveto{\pgfqpoint{8.170831in}{1.407200in}}{\pgfqpoint{8.165993in}{1.409204in}}{\pgfqpoint{8.160949in}{1.409204in}}%
\pgfpathcurveto{\pgfqpoint{8.155906in}{1.409204in}}{\pgfqpoint{8.151068in}{1.407200in}}{\pgfqpoint{8.147502in}{1.403634in}}%
\pgfpathcurveto{\pgfqpoint{8.143935in}{1.400068in}}{\pgfqpoint{8.141931in}{1.395230in}}{\pgfqpoint{8.141931in}{1.390186in}}%
\pgfpathcurveto{\pgfqpoint{8.141931in}{1.385142in}}{\pgfqpoint{8.143935in}{1.380305in}}{\pgfqpoint{8.147502in}{1.376738in}}%
\pgfpathcurveto{\pgfqpoint{8.151068in}{1.373172in}}{\pgfqpoint{8.155906in}{1.371168in}}{\pgfqpoint{8.160949in}{1.371168in}}%
\pgfpathclose%
\pgfusepath{fill}%
\end{pgfscope}%
\begin{pgfscope}%
\pgfpathrectangle{\pgfqpoint{6.572727in}{0.474100in}}{\pgfqpoint{4.227273in}{3.318700in}}%
\pgfusepath{clip}%
\pgfsetbuttcap%
\pgfsetroundjoin%
\definecolor{currentfill}{rgb}{0.993248,0.906157,0.143936}%
\pgfsetfillcolor{currentfill}%
\pgfsetfillopacity{0.700000}%
\pgfsetlinewidth{0.000000pt}%
\definecolor{currentstroke}{rgb}{0.000000,0.000000,0.000000}%
\pgfsetstrokecolor{currentstroke}%
\pgfsetstrokeopacity{0.700000}%
\pgfsetdash{}{0pt}%
\pgfpathmoveto{\pgfqpoint{8.148663in}{2.752842in}}%
\pgfpathcurveto{\pgfqpoint{8.153706in}{2.752842in}}{\pgfqpoint{8.158544in}{2.754846in}}{\pgfqpoint{8.162111in}{2.758412in}}%
\pgfpathcurveto{\pgfqpoint{8.165677in}{2.761979in}}{\pgfqpoint{8.167681in}{2.766817in}}{\pgfqpoint{8.167681in}{2.771860in}}%
\pgfpathcurveto{\pgfqpoint{8.167681in}{2.776904in}}{\pgfqpoint{8.165677in}{2.781742in}}{\pgfqpoint{8.162111in}{2.785308in}}%
\pgfpathcurveto{\pgfqpoint{8.158544in}{2.788874in}}{\pgfqpoint{8.153706in}{2.790878in}}{\pgfqpoint{8.148663in}{2.790878in}}%
\pgfpathcurveto{\pgfqpoint{8.143619in}{2.790878in}}{\pgfqpoint{8.138781in}{2.788874in}}{\pgfqpoint{8.135215in}{2.785308in}}%
\pgfpathcurveto{\pgfqpoint{8.131648in}{2.781742in}}{\pgfqpoint{8.129645in}{2.776904in}}{\pgfqpoint{8.129645in}{2.771860in}}%
\pgfpathcurveto{\pgfqpoint{8.129645in}{2.766817in}}{\pgfqpoint{8.131648in}{2.761979in}}{\pgfqpoint{8.135215in}{2.758412in}}%
\pgfpathcurveto{\pgfqpoint{8.138781in}{2.754846in}}{\pgfqpoint{8.143619in}{2.752842in}}{\pgfqpoint{8.148663in}{2.752842in}}%
\pgfpathclose%
\pgfusepath{fill}%
\end{pgfscope}%
\begin{pgfscope}%
\pgfpathrectangle{\pgfqpoint{6.572727in}{0.474100in}}{\pgfqpoint{4.227273in}{3.318700in}}%
\pgfusepath{clip}%
\pgfsetbuttcap%
\pgfsetroundjoin%
\definecolor{currentfill}{rgb}{0.127568,0.566949,0.550556}%
\pgfsetfillcolor{currentfill}%
\pgfsetfillopacity{0.700000}%
\pgfsetlinewidth{0.000000pt}%
\definecolor{currentstroke}{rgb}{0.000000,0.000000,0.000000}%
\pgfsetstrokecolor{currentstroke}%
\pgfsetstrokeopacity{0.700000}%
\pgfsetdash{}{0pt}%
\pgfpathmoveto{\pgfqpoint{9.442863in}{1.462796in}}%
\pgfpathcurveto{\pgfqpoint{9.447906in}{1.462796in}}{\pgfqpoint{9.452744in}{1.464800in}}{\pgfqpoint{9.456311in}{1.468366in}}%
\pgfpathcurveto{\pgfqpoint{9.459877in}{1.471933in}}{\pgfqpoint{9.461881in}{1.476770in}}{\pgfqpoint{9.461881in}{1.481814in}}%
\pgfpathcurveto{\pgfqpoint{9.461881in}{1.486858in}}{\pgfqpoint{9.459877in}{1.491695in}}{\pgfqpoint{9.456311in}{1.495262in}}%
\pgfpathcurveto{\pgfqpoint{9.452744in}{1.498828in}}{\pgfqpoint{9.447906in}{1.500832in}}{\pgfqpoint{9.442863in}{1.500832in}}%
\pgfpathcurveto{\pgfqpoint{9.437819in}{1.500832in}}{\pgfqpoint{9.432981in}{1.498828in}}{\pgfqpoint{9.429415in}{1.495262in}}%
\pgfpathcurveto{\pgfqpoint{9.425848in}{1.491695in}}{\pgfqpoint{9.423845in}{1.486858in}}{\pgfqpoint{9.423845in}{1.481814in}}%
\pgfpathcurveto{\pgfqpoint{9.423845in}{1.476770in}}{\pgfqpoint{9.425848in}{1.471933in}}{\pgfqpoint{9.429415in}{1.468366in}}%
\pgfpathcurveto{\pgfqpoint{9.432981in}{1.464800in}}{\pgfqpoint{9.437819in}{1.462796in}}{\pgfqpoint{9.442863in}{1.462796in}}%
\pgfpathclose%
\pgfusepath{fill}%
\end{pgfscope}%
\begin{pgfscope}%
\pgfpathrectangle{\pgfqpoint{6.572727in}{0.474100in}}{\pgfqpoint{4.227273in}{3.318700in}}%
\pgfusepath{clip}%
\pgfsetbuttcap%
\pgfsetroundjoin%
\definecolor{currentfill}{rgb}{0.267004,0.004874,0.329415}%
\pgfsetfillcolor{currentfill}%
\pgfsetfillopacity{0.700000}%
\pgfsetlinewidth{0.000000pt}%
\definecolor{currentstroke}{rgb}{0.000000,0.000000,0.000000}%
\pgfsetstrokecolor{currentstroke}%
\pgfsetstrokeopacity{0.700000}%
\pgfsetdash{}{0pt}%
\pgfpathmoveto{\pgfqpoint{8.100618in}{1.792480in}}%
\pgfpathcurveto{\pgfqpoint{8.105662in}{1.792480in}}{\pgfqpoint{8.110500in}{1.794484in}}{\pgfqpoint{8.114066in}{1.798051in}}%
\pgfpathcurveto{\pgfqpoint{8.117633in}{1.801617in}}{\pgfqpoint{8.119637in}{1.806455in}}{\pgfqpoint{8.119637in}{1.811499in}}%
\pgfpathcurveto{\pgfqpoint{8.119637in}{1.816542in}}{\pgfqpoint{8.117633in}{1.821380in}}{\pgfqpoint{8.114066in}{1.824946in}}%
\pgfpathcurveto{\pgfqpoint{8.110500in}{1.828513in}}{\pgfqpoint{8.105662in}{1.830517in}}{\pgfqpoint{8.100618in}{1.830517in}}%
\pgfpathcurveto{\pgfqpoint{8.095575in}{1.830517in}}{\pgfqpoint{8.090737in}{1.828513in}}{\pgfqpoint{8.087171in}{1.824946in}}%
\pgfpathcurveto{\pgfqpoint{8.083604in}{1.821380in}}{\pgfqpoint{8.081600in}{1.816542in}}{\pgfqpoint{8.081600in}{1.811499in}}%
\pgfpathcurveto{\pgfqpoint{8.081600in}{1.806455in}}{\pgfqpoint{8.083604in}{1.801617in}}{\pgfqpoint{8.087171in}{1.798051in}}%
\pgfpathcurveto{\pgfqpoint{8.090737in}{1.794484in}}{\pgfqpoint{8.095575in}{1.792480in}}{\pgfqpoint{8.100618in}{1.792480in}}%
\pgfpathclose%
\pgfusepath{fill}%
\end{pgfscope}%
\begin{pgfscope}%
\pgfpathrectangle{\pgfqpoint{6.572727in}{0.474100in}}{\pgfqpoint{4.227273in}{3.318700in}}%
\pgfusepath{clip}%
\pgfsetbuttcap%
\pgfsetroundjoin%
\definecolor{currentfill}{rgb}{0.993248,0.906157,0.143936}%
\pgfsetfillcolor{currentfill}%
\pgfsetfillopacity{0.700000}%
\pgfsetlinewidth{0.000000pt}%
\definecolor{currentstroke}{rgb}{0.000000,0.000000,0.000000}%
\pgfsetstrokecolor{currentstroke}%
\pgfsetstrokeopacity{0.700000}%
\pgfsetdash{}{0pt}%
\pgfpathmoveto{\pgfqpoint{8.936811in}{2.362006in}}%
\pgfpathcurveto{\pgfqpoint{8.941855in}{2.362006in}}{\pgfqpoint{8.946692in}{2.364010in}}{\pgfqpoint{8.950259in}{2.367576in}}%
\pgfpathcurveto{\pgfqpoint{8.953825in}{2.371143in}}{\pgfqpoint{8.955829in}{2.375980in}}{\pgfqpoint{8.955829in}{2.381024in}}%
\pgfpathcurveto{\pgfqpoint{8.955829in}{2.386068in}}{\pgfqpoint{8.953825in}{2.390905in}}{\pgfqpoint{8.950259in}{2.394472in}}%
\pgfpathcurveto{\pgfqpoint{8.946692in}{2.398038in}}{\pgfqpoint{8.941855in}{2.400042in}}{\pgfqpoint{8.936811in}{2.400042in}}%
\pgfpathcurveto{\pgfqpoint{8.931767in}{2.400042in}}{\pgfqpoint{8.926929in}{2.398038in}}{\pgfqpoint{8.923363in}{2.394472in}}%
\pgfpathcurveto{\pgfqpoint{8.919797in}{2.390905in}}{\pgfqpoint{8.917793in}{2.386068in}}{\pgfqpoint{8.917793in}{2.381024in}}%
\pgfpathcurveto{\pgfqpoint{8.917793in}{2.375980in}}{\pgfqpoint{8.919797in}{2.371143in}}{\pgfqpoint{8.923363in}{2.367576in}}%
\pgfpathcurveto{\pgfqpoint{8.926929in}{2.364010in}}{\pgfqpoint{8.931767in}{2.362006in}}{\pgfqpoint{8.936811in}{2.362006in}}%
\pgfpathclose%
\pgfusepath{fill}%
\end{pgfscope}%
\begin{pgfscope}%
\pgfpathrectangle{\pgfqpoint{6.572727in}{0.474100in}}{\pgfqpoint{4.227273in}{3.318700in}}%
\pgfusepath{clip}%
\pgfsetbuttcap%
\pgfsetroundjoin%
\definecolor{currentfill}{rgb}{0.127568,0.566949,0.550556}%
\pgfsetfillcolor{currentfill}%
\pgfsetfillopacity{0.700000}%
\pgfsetlinewidth{0.000000pt}%
\definecolor{currentstroke}{rgb}{0.000000,0.000000,0.000000}%
\pgfsetstrokecolor{currentstroke}%
\pgfsetstrokeopacity{0.700000}%
\pgfsetdash{}{0pt}%
\pgfpathmoveto{\pgfqpoint{8.822130in}{1.345310in}}%
\pgfpathcurveto{\pgfqpoint{8.827174in}{1.345310in}}{\pgfqpoint{8.832012in}{1.347314in}}{\pgfqpoint{8.835578in}{1.350881in}}%
\pgfpathcurveto{\pgfqpoint{8.839144in}{1.354447in}}{\pgfqpoint{8.841148in}{1.359285in}}{\pgfqpoint{8.841148in}{1.364329in}}%
\pgfpathcurveto{\pgfqpoint{8.841148in}{1.369372in}}{\pgfqpoint{8.839144in}{1.374210in}}{\pgfqpoint{8.835578in}{1.377776in}}%
\pgfpathcurveto{\pgfqpoint{8.832012in}{1.381343in}}{\pgfqpoint{8.827174in}{1.383347in}}{\pgfqpoint{8.822130in}{1.383347in}}%
\pgfpathcurveto{\pgfqpoint{8.817086in}{1.383347in}}{\pgfqpoint{8.812249in}{1.381343in}}{\pgfqpoint{8.808682in}{1.377776in}}%
\pgfpathcurveto{\pgfqpoint{8.805116in}{1.374210in}}{\pgfqpoint{8.803112in}{1.369372in}}{\pgfqpoint{8.803112in}{1.364329in}}%
\pgfpathcurveto{\pgfqpoint{8.803112in}{1.359285in}}{\pgfqpoint{8.805116in}{1.354447in}}{\pgfqpoint{8.808682in}{1.350881in}}%
\pgfpathcurveto{\pgfqpoint{8.812249in}{1.347314in}}{\pgfqpoint{8.817086in}{1.345310in}}{\pgfqpoint{8.822130in}{1.345310in}}%
\pgfpathclose%
\pgfusepath{fill}%
\end{pgfscope}%
\begin{pgfscope}%
\pgfpathrectangle{\pgfqpoint{6.572727in}{0.474100in}}{\pgfqpoint{4.227273in}{3.318700in}}%
\pgfusepath{clip}%
\pgfsetbuttcap%
\pgfsetroundjoin%
\definecolor{currentfill}{rgb}{0.267004,0.004874,0.329415}%
\pgfsetfillcolor{currentfill}%
\pgfsetfillopacity{0.700000}%
\pgfsetlinewidth{0.000000pt}%
\definecolor{currentstroke}{rgb}{0.000000,0.000000,0.000000}%
\pgfsetstrokecolor{currentstroke}%
\pgfsetstrokeopacity{0.700000}%
\pgfsetdash{}{0pt}%
\pgfpathmoveto{\pgfqpoint{8.157294in}{1.335820in}}%
\pgfpathcurveto{\pgfqpoint{8.162337in}{1.335820in}}{\pgfqpoint{8.167175in}{1.337824in}}{\pgfqpoint{8.170742in}{1.341391in}}%
\pgfpathcurveto{\pgfqpoint{8.174308in}{1.344957in}}{\pgfqpoint{8.176312in}{1.349795in}}{\pgfqpoint{8.176312in}{1.354838in}}%
\pgfpathcurveto{\pgfqpoint{8.176312in}{1.359882in}}{\pgfqpoint{8.174308in}{1.364720in}}{\pgfqpoint{8.170742in}{1.368286in}}%
\pgfpathcurveto{\pgfqpoint{8.167175in}{1.371853in}}{\pgfqpoint{8.162337in}{1.373857in}}{\pgfqpoint{8.157294in}{1.373857in}}%
\pgfpathcurveto{\pgfqpoint{8.152250in}{1.373857in}}{\pgfqpoint{8.147412in}{1.371853in}}{\pgfqpoint{8.143846in}{1.368286in}}%
\pgfpathcurveto{\pgfqpoint{8.140280in}{1.364720in}}{\pgfqpoint{8.138276in}{1.359882in}}{\pgfqpoint{8.138276in}{1.354838in}}%
\pgfpathcurveto{\pgfqpoint{8.138276in}{1.349795in}}{\pgfqpoint{8.140280in}{1.344957in}}{\pgfqpoint{8.143846in}{1.341391in}}%
\pgfpathcurveto{\pgfqpoint{8.147412in}{1.337824in}}{\pgfqpoint{8.152250in}{1.335820in}}{\pgfqpoint{8.157294in}{1.335820in}}%
\pgfpathclose%
\pgfusepath{fill}%
\end{pgfscope}%
\begin{pgfscope}%
\pgfpathrectangle{\pgfqpoint{6.572727in}{0.474100in}}{\pgfqpoint{4.227273in}{3.318700in}}%
\pgfusepath{clip}%
\pgfsetbuttcap%
\pgfsetroundjoin%
\definecolor{currentfill}{rgb}{0.127568,0.566949,0.550556}%
\pgfsetfillcolor{currentfill}%
\pgfsetfillopacity{0.700000}%
\pgfsetlinewidth{0.000000pt}%
\definecolor{currentstroke}{rgb}{0.000000,0.000000,0.000000}%
\pgfsetstrokecolor{currentstroke}%
\pgfsetstrokeopacity{0.700000}%
\pgfsetdash{}{0pt}%
\pgfpathmoveto{\pgfqpoint{9.732150in}{1.571216in}}%
\pgfpathcurveto{\pgfqpoint{9.737193in}{1.571216in}}{\pgfqpoint{9.742031in}{1.573220in}}{\pgfqpoint{9.745597in}{1.576786in}}%
\pgfpathcurveto{\pgfqpoint{9.749164in}{1.580352in}}{\pgfqpoint{9.751168in}{1.585190in}}{\pgfqpoint{9.751168in}{1.590234in}}%
\pgfpathcurveto{\pgfqpoint{9.751168in}{1.595277in}}{\pgfqpoint{9.749164in}{1.600115in}}{\pgfqpoint{9.745597in}{1.603682in}}%
\pgfpathcurveto{\pgfqpoint{9.742031in}{1.607248in}}{\pgfqpoint{9.737193in}{1.609252in}}{\pgfqpoint{9.732150in}{1.609252in}}%
\pgfpathcurveto{\pgfqpoint{9.727106in}{1.609252in}}{\pgfqpoint{9.722268in}{1.607248in}}{\pgfqpoint{9.718702in}{1.603682in}}%
\pgfpathcurveto{\pgfqpoint{9.715135in}{1.600115in}}{\pgfqpoint{9.713131in}{1.595277in}}{\pgfqpoint{9.713131in}{1.590234in}}%
\pgfpathcurveto{\pgfqpoint{9.713131in}{1.585190in}}{\pgfqpoint{9.715135in}{1.580352in}}{\pgfqpoint{9.718702in}{1.576786in}}%
\pgfpathcurveto{\pgfqpoint{9.722268in}{1.573220in}}{\pgfqpoint{9.727106in}{1.571216in}}{\pgfqpoint{9.732150in}{1.571216in}}%
\pgfpathclose%
\pgfusepath{fill}%
\end{pgfscope}%
\begin{pgfscope}%
\pgfpathrectangle{\pgfqpoint{6.572727in}{0.474100in}}{\pgfqpoint{4.227273in}{3.318700in}}%
\pgfusepath{clip}%
\pgfsetbuttcap%
\pgfsetroundjoin%
\definecolor{currentfill}{rgb}{0.127568,0.566949,0.550556}%
\pgfsetfillcolor{currentfill}%
\pgfsetfillopacity{0.700000}%
\pgfsetlinewidth{0.000000pt}%
\definecolor{currentstroke}{rgb}{0.000000,0.000000,0.000000}%
\pgfsetstrokecolor{currentstroke}%
\pgfsetstrokeopacity{0.700000}%
\pgfsetdash{}{0pt}%
\pgfpathmoveto{\pgfqpoint{10.044806in}{1.251015in}}%
\pgfpathcurveto{\pgfqpoint{10.049850in}{1.251015in}}{\pgfqpoint{10.054688in}{1.253019in}}{\pgfqpoint{10.058254in}{1.256585in}}%
\pgfpathcurveto{\pgfqpoint{10.061821in}{1.260151in}}{\pgfqpoint{10.063824in}{1.264989in}}{\pgfqpoint{10.063824in}{1.270033in}}%
\pgfpathcurveto{\pgfqpoint{10.063824in}{1.275076in}}{\pgfqpoint{10.061821in}{1.279914in}}{\pgfqpoint{10.058254in}{1.283481in}}%
\pgfpathcurveto{\pgfqpoint{10.054688in}{1.287047in}}{\pgfqpoint{10.049850in}{1.289051in}}{\pgfqpoint{10.044806in}{1.289051in}}%
\pgfpathcurveto{\pgfqpoint{10.039763in}{1.289051in}}{\pgfqpoint{10.034925in}{1.287047in}}{\pgfqpoint{10.031358in}{1.283481in}}%
\pgfpathcurveto{\pgfqpoint{10.027792in}{1.279914in}}{\pgfqpoint{10.025788in}{1.275076in}}{\pgfqpoint{10.025788in}{1.270033in}}%
\pgfpathcurveto{\pgfqpoint{10.025788in}{1.264989in}}{\pgfqpoint{10.027792in}{1.260151in}}{\pgfqpoint{10.031358in}{1.256585in}}%
\pgfpathcurveto{\pgfqpoint{10.034925in}{1.253019in}}{\pgfqpoint{10.039763in}{1.251015in}}{\pgfqpoint{10.044806in}{1.251015in}}%
\pgfpathclose%
\pgfusepath{fill}%
\end{pgfscope}%
\begin{pgfscope}%
\pgfpathrectangle{\pgfqpoint{6.572727in}{0.474100in}}{\pgfqpoint{4.227273in}{3.318700in}}%
\pgfusepath{clip}%
\pgfsetbuttcap%
\pgfsetroundjoin%
\definecolor{currentfill}{rgb}{0.993248,0.906157,0.143936}%
\pgfsetfillcolor{currentfill}%
\pgfsetfillopacity{0.700000}%
\pgfsetlinewidth{0.000000pt}%
\definecolor{currentstroke}{rgb}{0.000000,0.000000,0.000000}%
\pgfsetstrokecolor{currentstroke}%
\pgfsetstrokeopacity{0.700000}%
\pgfsetdash{}{0pt}%
\pgfpathmoveto{\pgfqpoint{8.453272in}{2.739284in}}%
\pgfpathcurveto{\pgfqpoint{8.458315in}{2.739284in}}{\pgfqpoint{8.463153in}{2.741288in}}{\pgfqpoint{8.466720in}{2.744855in}}%
\pgfpathcurveto{\pgfqpoint{8.470286in}{2.748421in}}{\pgfqpoint{8.472290in}{2.753259in}}{\pgfqpoint{8.472290in}{2.758303in}}%
\pgfpathcurveto{\pgfqpoint{8.472290in}{2.763346in}}{\pgfqpoint{8.470286in}{2.768184in}}{\pgfqpoint{8.466720in}{2.771750in}}%
\pgfpathcurveto{\pgfqpoint{8.463153in}{2.775317in}}{\pgfqpoint{8.458315in}{2.777321in}}{\pgfqpoint{8.453272in}{2.777321in}}%
\pgfpathcurveto{\pgfqpoint{8.448228in}{2.777321in}}{\pgfqpoint{8.443390in}{2.775317in}}{\pgfqpoint{8.439824in}{2.771750in}}%
\pgfpathcurveto{\pgfqpoint{8.436257in}{2.768184in}}{\pgfqpoint{8.434254in}{2.763346in}}{\pgfqpoint{8.434254in}{2.758303in}}%
\pgfpathcurveto{\pgfqpoint{8.434254in}{2.753259in}}{\pgfqpoint{8.436257in}{2.748421in}}{\pgfqpoint{8.439824in}{2.744855in}}%
\pgfpathcurveto{\pgfqpoint{8.443390in}{2.741288in}}{\pgfqpoint{8.448228in}{2.739284in}}{\pgfqpoint{8.453272in}{2.739284in}}%
\pgfpathclose%
\pgfusepath{fill}%
\end{pgfscope}%
\begin{pgfscope}%
\pgfpathrectangle{\pgfqpoint{6.572727in}{0.474100in}}{\pgfqpoint{4.227273in}{3.318700in}}%
\pgfusepath{clip}%
\pgfsetbuttcap%
\pgfsetroundjoin%
\definecolor{currentfill}{rgb}{0.993248,0.906157,0.143936}%
\pgfsetfillcolor{currentfill}%
\pgfsetfillopacity{0.700000}%
\pgfsetlinewidth{0.000000pt}%
\definecolor{currentstroke}{rgb}{0.000000,0.000000,0.000000}%
\pgfsetstrokecolor{currentstroke}%
\pgfsetstrokeopacity{0.700000}%
\pgfsetdash{}{0pt}%
\pgfpathmoveto{\pgfqpoint{8.931081in}{2.486241in}}%
\pgfpathcurveto{\pgfqpoint{8.936125in}{2.486241in}}{\pgfqpoint{8.940962in}{2.488245in}}{\pgfqpoint{8.944529in}{2.491811in}}%
\pgfpathcurveto{\pgfqpoint{8.948095in}{2.495377in}}{\pgfqpoint{8.950099in}{2.500215in}}{\pgfqpoint{8.950099in}{2.505259in}}%
\pgfpathcurveto{\pgfqpoint{8.950099in}{2.510302in}}{\pgfqpoint{8.948095in}{2.515140in}}{\pgfqpoint{8.944529in}{2.518707in}}%
\pgfpathcurveto{\pgfqpoint{8.940962in}{2.522273in}}{\pgfqpoint{8.936125in}{2.524277in}}{\pgfqpoint{8.931081in}{2.524277in}}%
\pgfpathcurveto{\pgfqpoint{8.926037in}{2.524277in}}{\pgfqpoint{8.921200in}{2.522273in}}{\pgfqpoint{8.917633in}{2.518707in}}%
\pgfpathcurveto{\pgfqpoint{8.914067in}{2.515140in}}{\pgfqpoint{8.912063in}{2.510302in}}{\pgfqpoint{8.912063in}{2.505259in}}%
\pgfpathcurveto{\pgfqpoint{8.912063in}{2.500215in}}{\pgfqpoint{8.914067in}{2.495377in}}{\pgfqpoint{8.917633in}{2.491811in}}%
\pgfpathcurveto{\pgfqpoint{8.921200in}{2.488245in}}{\pgfqpoint{8.926037in}{2.486241in}}{\pgfqpoint{8.931081in}{2.486241in}}%
\pgfpathclose%
\pgfusepath{fill}%
\end{pgfscope}%
\begin{pgfscope}%
\pgfpathrectangle{\pgfqpoint{6.572727in}{0.474100in}}{\pgfqpoint{4.227273in}{3.318700in}}%
\pgfusepath{clip}%
\pgfsetbuttcap%
\pgfsetroundjoin%
\definecolor{currentfill}{rgb}{0.993248,0.906157,0.143936}%
\pgfsetfillcolor{currentfill}%
\pgfsetfillopacity{0.700000}%
\pgfsetlinewidth{0.000000pt}%
\definecolor{currentstroke}{rgb}{0.000000,0.000000,0.000000}%
\pgfsetstrokecolor{currentstroke}%
\pgfsetstrokeopacity{0.700000}%
\pgfsetdash{}{0pt}%
\pgfpathmoveto{\pgfqpoint{8.350032in}{3.018121in}}%
\pgfpathcurveto{\pgfqpoint{8.355076in}{3.018121in}}{\pgfqpoint{8.359914in}{3.020125in}}{\pgfqpoint{8.363480in}{3.023691in}}%
\pgfpathcurveto{\pgfqpoint{8.367047in}{3.027257in}}{\pgfqpoint{8.369050in}{3.032095in}}{\pgfqpoint{8.369050in}{3.037139in}}%
\pgfpathcurveto{\pgfqpoint{8.369050in}{3.042183in}}{\pgfqpoint{8.367047in}{3.047020in}}{\pgfqpoint{8.363480in}{3.050587in}}%
\pgfpathcurveto{\pgfqpoint{8.359914in}{3.054153in}}{\pgfqpoint{8.355076in}{3.056157in}}{\pgfqpoint{8.350032in}{3.056157in}}%
\pgfpathcurveto{\pgfqpoint{8.344989in}{3.056157in}}{\pgfqpoint{8.340151in}{3.054153in}}{\pgfqpoint{8.336584in}{3.050587in}}%
\pgfpathcurveto{\pgfqpoint{8.333018in}{3.047020in}}{\pgfqpoint{8.331014in}{3.042183in}}{\pgfqpoint{8.331014in}{3.037139in}}%
\pgfpathcurveto{\pgfqpoint{8.331014in}{3.032095in}}{\pgfqpoint{8.333018in}{3.027257in}}{\pgfqpoint{8.336584in}{3.023691in}}%
\pgfpathcurveto{\pgfqpoint{8.340151in}{3.020125in}}{\pgfqpoint{8.344989in}{3.018121in}}{\pgfqpoint{8.350032in}{3.018121in}}%
\pgfpathclose%
\pgfusepath{fill}%
\end{pgfscope}%
\begin{pgfscope}%
\pgfpathrectangle{\pgfqpoint{6.572727in}{0.474100in}}{\pgfqpoint{4.227273in}{3.318700in}}%
\pgfusepath{clip}%
\pgfsetbuttcap%
\pgfsetroundjoin%
\definecolor{currentfill}{rgb}{0.993248,0.906157,0.143936}%
\pgfsetfillcolor{currentfill}%
\pgfsetfillopacity{0.700000}%
\pgfsetlinewidth{0.000000pt}%
\definecolor{currentstroke}{rgb}{0.000000,0.000000,0.000000}%
\pgfsetstrokecolor{currentstroke}%
\pgfsetstrokeopacity{0.700000}%
\pgfsetdash{}{0pt}%
\pgfpathmoveto{\pgfqpoint{8.659534in}{2.980281in}}%
\pgfpathcurveto{\pgfqpoint{8.664578in}{2.980281in}}{\pgfqpoint{8.669415in}{2.982285in}}{\pgfqpoint{8.672982in}{2.985851in}}%
\pgfpathcurveto{\pgfqpoint{8.676548in}{2.989417in}}{\pgfqpoint{8.678552in}{2.994255in}}{\pgfqpoint{8.678552in}{2.999299in}}%
\pgfpathcurveto{\pgfqpoint{8.678552in}{3.004342in}}{\pgfqpoint{8.676548in}{3.009180in}}{\pgfqpoint{8.672982in}{3.012747in}}%
\pgfpathcurveto{\pgfqpoint{8.669415in}{3.016313in}}{\pgfqpoint{8.664578in}{3.018317in}}{\pgfqpoint{8.659534in}{3.018317in}}%
\pgfpathcurveto{\pgfqpoint{8.654490in}{3.018317in}}{\pgfqpoint{8.649653in}{3.016313in}}{\pgfqpoint{8.646086in}{3.012747in}}%
\pgfpathcurveto{\pgfqpoint{8.642520in}{3.009180in}}{\pgfqpoint{8.640516in}{3.004342in}}{\pgfqpoint{8.640516in}{2.999299in}}%
\pgfpathcurveto{\pgfqpoint{8.640516in}{2.994255in}}{\pgfqpoint{8.642520in}{2.989417in}}{\pgfqpoint{8.646086in}{2.985851in}}%
\pgfpathcurveto{\pgfqpoint{8.649653in}{2.982285in}}{\pgfqpoint{8.654490in}{2.980281in}}{\pgfqpoint{8.659534in}{2.980281in}}%
\pgfpathclose%
\pgfusepath{fill}%
\end{pgfscope}%
\begin{pgfscope}%
\pgfpathrectangle{\pgfqpoint{6.572727in}{0.474100in}}{\pgfqpoint{4.227273in}{3.318700in}}%
\pgfusepath{clip}%
\pgfsetbuttcap%
\pgfsetroundjoin%
\definecolor{currentfill}{rgb}{0.127568,0.566949,0.550556}%
\pgfsetfillcolor{currentfill}%
\pgfsetfillopacity{0.700000}%
\pgfsetlinewidth{0.000000pt}%
\definecolor{currentstroke}{rgb}{0.000000,0.000000,0.000000}%
\pgfsetstrokecolor{currentstroke}%
\pgfsetstrokeopacity{0.700000}%
\pgfsetdash{}{0pt}%
\pgfpathmoveto{\pgfqpoint{9.611851in}{1.516827in}}%
\pgfpathcurveto{\pgfqpoint{9.616894in}{1.516827in}}{\pgfqpoint{9.621732in}{1.518831in}}{\pgfqpoint{9.625299in}{1.522397in}}%
\pgfpathcurveto{\pgfqpoint{9.628865in}{1.525964in}}{\pgfqpoint{9.630869in}{1.530801in}}{\pgfqpoint{9.630869in}{1.535845in}}%
\pgfpathcurveto{\pgfqpoint{9.630869in}{1.540889in}}{\pgfqpoint{9.628865in}{1.545726in}}{\pgfqpoint{9.625299in}{1.549293in}}%
\pgfpathcurveto{\pgfqpoint{9.621732in}{1.552859in}}{\pgfqpoint{9.616894in}{1.554863in}}{\pgfqpoint{9.611851in}{1.554863in}}%
\pgfpathcurveto{\pgfqpoint{9.606807in}{1.554863in}}{\pgfqpoint{9.601969in}{1.552859in}}{\pgfqpoint{9.598403in}{1.549293in}}%
\pgfpathcurveto{\pgfqpoint{9.594836in}{1.545726in}}{\pgfqpoint{9.592833in}{1.540889in}}{\pgfqpoint{9.592833in}{1.535845in}}%
\pgfpathcurveto{\pgfqpoint{9.592833in}{1.530801in}}{\pgfqpoint{9.594836in}{1.525964in}}{\pgfqpoint{9.598403in}{1.522397in}}%
\pgfpathcurveto{\pgfqpoint{9.601969in}{1.518831in}}{\pgfqpoint{9.606807in}{1.516827in}}{\pgfqpoint{9.611851in}{1.516827in}}%
\pgfpathclose%
\pgfusepath{fill}%
\end{pgfscope}%
\begin{pgfscope}%
\pgfpathrectangle{\pgfqpoint{6.572727in}{0.474100in}}{\pgfqpoint{4.227273in}{3.318700in}}%
\pgfusepath{clip}%
\pgfsetbuttcap%
\pgfsetroundjoin%
\definecolor{currentfill}{rgb}{0.127568,0.566949,0.550556}%
\pgfsetfillcolor{currentfill}%
\pgfsetfillopacity{0.700000}%
\pgfsetlinewidth{0.000000pt}%
\definecolor{currentstroke}{rgb}{0.000000,0.000000,0.000000}%
\pgfsetstrokecolor{currentstroke}%
\pgfsetstrokeopacity{0.700000}%
\pgfsetdash{}{0pt}%
\pgfpathmoveto{\pgfqpoint{9.500164in}{1.549704in}}%
\pgfpathcurveto{\pgfqpoint{9.505207in}{1.549704in}}{\pgfqpoint{9.510045in}{1.551707in}}{\pgfqpoint{9.513612in}{1.555274in}}%
\pgfpathcurveto{\pgfqpoint{9.517178in}{1.558840in}}{\pgfqpoint{9.519182in}{1.563678in}}{\pgfqpoint{9.519182in}{1.568722in}}%
\pgfpathcurveto{\pgfqpoint{9.519182in}{1.573765in}}{\pgfqpoint{9.517178in}{1.578603in}}{\pgfqpoint{9.513612in}{1.582170in}}%
\pgfpathcurveto{\pgfqpoint{9.510045in}{1.585736in}}{\pgfqpoint{9.505207in}{1.587740in}}{\pgfqpoint{9.500164in}{1.587740in}}%
\pgfpathcurveto{\pgfqpoint{9.495120in}{1.587740in}}{\pgfqpoint{9.490282in}{1.585736in}}{\pgfqpoint{9.486716in}{1.582170in}}%
\pgfpathcurveto{\pgfqpoint{9.483150in}{1.578603in}}{\pgfqpoint{9.481146in}{1.573765in}}{\pgfqpoint{9.481146in}{1.568722in}}%
\pgfpathcurveto{\pgfqpoint{9.481146in}{1.563678in}}{\pgfqpoint{9.483150in}{1.558840in}}{\pgfqpoint{9.486716in}{1.555274in}}%
\pgfpathcurveto{\pgfqpoint{9.490282in}{1.551707in}}{\pgfqpoint{9.495120in}{1.549704in}}{\pgfqpoint{9.500164in}{1.549704in}}%
\pgfpathclose%
\pgfusepath{fill}%
\end{pgfscope}%
\begin{pgfscope}%
\pgfpathrectangle{\pgfqpoint{6.572727in}{0.474100in}}{\pgfqpoint{4.227273in}{3.318700in}}%
\pgfusepath{clip}%
\pgfsetbuttcap%
\pgfsetroundjoin%
\definecolor{currentfill}{rgb}{0.127568,0.566949,0.550556}%
\pgfsetfillcolor{currentfill}%
\pgfsetfillopacity{0.700000}%
\pgfsetlinewidth{0.000000pt}%
\definecolor{currentstroke}{rgb}{0.000000,0.000000,0.000000}%
\pgfsetstrokecolor{currentstroke}%
\pgfsetstrokeopacity{0.700000}%
\pgfsetdash{}{0pt}%
\pgfpathmoveto{\pgfqpoint{10.086043in}{1.522788in}}%
\pgfpathcurveto{\pgfqpoint{10.091087in}{1.522788in}}{\pgfqpoint{10.095925in}{1.524792in}}{\pgfqpoint{10.099491in}{1.528358in}}%
\pgfpathcurveto{\pgfqpoint{10.103057in}{1.531925in}}{\pgfqpoint{10.105061in}{1.536763in}}{\pgfqpoint{10.105061in}{1.541806in}}%
\pgfpathcurveto{\pgfqpoint{10.105061in}{1.546850in}}{\pgfqpoint{10.103057in}{1.551688in}}{\pgfqpoint{10.099491in}{1.555254in}}%
\pgfpathcurveto{\pgfqpoint{10.095925in}{1.558821in}}{\pgfqpoint{10.091087in}{1.560824in}}{\pgfqpoint{10.086043in}{1.560824in}}%
\pgfpathcurveto{\pgfqpoint{10.080999in}{1.560824in}}{\pgfqpoint{10.076162in}{1.558821in}}{\pgfqpoint{10.072595in}{1.555254in}}%
\pgfpathcurveto{\pgfqpoint{10.069029in}{1.551688in}}{\pgfqpoint{10.067025in}{1.546850in}}{\pgfqpoint{10.067025in}{1.541806in}}%
\pgfpathcurveto{\pgfqpoint{10.067025in}{1.536763in}}{\pgfqpoint{10.069029in}{1.531925in}}{\pgfqpoint{10.072595in}{1.528358in}}%
\pgfpathcurveto{\pgfqpoint{10.076162in}{1.524792in}}{\pgfqpoint{10.080999in}{1.522788in}}{\pgfqpoint{10.086043in}{1.522788in}}%
\pgfpathclose%
\pgfusepath{fill}%
\end{pgfscope}%
\begin{pgfscope}%
\pgfpathrectangle{\pgfqpoint{6.572727in}{0.474100in}}{\pgfqpoint{4.227273in}{3.318700in}}%
\pgfusepath{clip}%
\pgfsetbuttcap%
\pgfsetroundjoin%
\definecolor{currentfill}{rgb}{0.127568,0.566949,0.550556}%
\pgfsetfillcolor{currentfill}%
\pgfsetfillopacity{0.700000}%
\pgfsetlinewidth{0.000000pt}%
\definecolor{currentstroke}{rgb}{0.000000,0.000000,0.000000}%
\pgfsetstrokecolor{currentstroke}%
\pgfsetstrokeopacity{0.700000}%
\pgfsetdash{}{0pt}%
\pgfpathmoveto{\pgfqpoint{9.325643in}{1.186204in}}%
\pgfpathcurveto{\pgfqpoint{9.330686in}{1.186204in}}{\pgfqpoint{9.335524in}{1.188207in}}{\pgfqpoint{9.339090in}{1.191774in}}%
\pgfpathcurveto{\pgfqpoint{9.342657in}{1.195340in}}{\pgfqpoint{9.344661in}{1.200178in}}{\pgfqpoint{9.344661in}{1.205222in}}%
\pgfpathcurveto{\pgfqpoint{9.344661in}{1.210265in}}{\pgfqpoint{9.342657in}{1.215103in}}{\pgfqpoint{9.339090in}{1.218670in}}%
\pgfpathcurveto{\pgfqpoint{9.335524in}{1.222236in}}{\pgfqpoint{9.330686in}{1.224240in}}{\pgfqpoint{9.325643in}{1.224240in}}%
\pgfpathcurveto{\pgfqpoint{9.320599in}{1.224240in}}{\pgfqpoint{9.315761in}{1.222236in}}{\pgfqpoint{9.312195in}{1.218670in}}%
\pgfpathcurveto{\pgfqpoint{9.308628in}{1.215103in}}{\pgfqpoint{9.306624in}{1.210265in}}{\pgfqpoint{9.306624in}{1.205222in}}%
\pgfpathcurveto{\pgfqpoint{9.306624in}{1.200178in}}{\pgfqpoint{9.308628in}{1.195340in}}{\pgfqpoint{9.312195in}{1.191774in}}%
\pgfpathcurveto{\pgfqpoint{9.315761in}{1.188207in}}{\pgfqpoint{9.320599in}{1.186204in}}{\pgfqpoint{9.325643in}{1.186204in}}%
\pgfpathclose%
\pgfusepath{fill}%
\end{pgfscope}%
\begin{pgfscope}%
\pgfpathrectangle{\pgfqpoint{6.572727in}{0.474100in}}{\pgfqpoint{4.227273in}{3.318700in}}%
\pgfusepath{clip}%
\pgfsetbuttcap%
\pgfsetroundjoin%
\definecolor{currentfill}{rgb}{0.127568,0.566949,0.550556}%
\pgfsetfillcolor{currentfill}%
\pgfsetfillopacity{0.700000}%
\pgfsetlinewidth{0.000000pt}%
\definecolor{currentstroke}{rgb}{0.000000,0.000000,0.000000}%
\pgfsetstrokecolor{currentstroke}%
\pgfsetstrokeopacity{0.700000}%
\pgfsetdash{}{0pt}%
\pgfpathmoveto{\pgfqpoint{9.240433in}{1.450621in}}%
\pgfpathcurveto{\pgfqpoint{9.245477in}{1.450621in}}{\pgfqpoint{9.250314in}{1.452625in}}{\pgfqpoint{9.253881in}{1.456191in}}%
\pgfpathcurveto{\pgfqpoint{9.257447in}{1.459758in}}{\pgfqpoint{9.259451in}{1.464596in}}{\pgfqpoint{9.259451in}{1.469639in}}%
\pgfpathcurveto{\pgfqpoint{9.259451in}{1.474683in}}{\pgfqpoint{9.257447in}{1.479521in}}{\pgfqpoint{9.253881in}{1.483087in}}%
\pgfpathcurveto{\pgfqpoint{9.250314in}{1.486654in}}{\pgfqpoint{9.245477in}{1.488657in}}{\pgfqpoint{9.240433in}{1.488657in}}%
\pgfpathcurveto{\pgfqpoint{9.235389in}{1.488657in}}{\pgfqpoint{9.230552in}{1.486654in}}{\pgfqpoint{9.226985in}{1.483087in}}%
\pgfpathcurveto{\pgfqpoint{9.223419in}{1.479521in}}{\pgfqpoint{9.221415in}{1.474683in}}{\pgfqpoint{9.221415in}{1.469639in}}%
\pgfpathcurveto{\pgfqpoint{9.221415in}{1.464596in}}{\pgfqpoint{9.223419in}{1.459758in}}{\pgfqpoint{9.226985in}{1.456191in}}%
\pgfpathcurveto{\pgfqpoint{9.230552in}{1.452625in}}{\pgfqpoint{9.235389in}{1.450621in}}{\pgfqpoint{9.240433in}{1.450621in}}%
\pgfpathclose%
\pgfusepath{fill}%
\end{pgfscope}%
\begin{pgfscope}%
\pgfpathrectangle{\pgfqpoint{6.572727in}{0.474100in}}{\pgfqpoint{4.227273in}{3.318700in}}%
\pgfusepath{clip}%
\pgfsetbuttcap%
\pgfsetroundjoin%
\definecolor{currentfill}{rgb}{0.127568,0.566949,0.550556}%
\pgfsetfillcolor{currentfill}%
\pgfsetfillopacity{0.700000}%
\pgfsetlinewidth{0.000000pt}%
\definecolor{currentstroke}{rgb}{0.000000,0.000000,0.000000}%
\pgfsetstrokecolor{currentstroke}%
\pgfsetstrokeopacity{0.700000}%
\pgfsetdash{}{0pt}%
\pgfpathmoveto{\pgfqpoint{9.772256in}{1.959423in}}%
\pgfpathcurveto{\pgfqpoint{9.777300in}{1.959423in}}{\pgfqpoint{9.782138in}{1.961427in}}{\pgfqpoint{9.785704in}{1.964994in}}%
\pgfpathcurveto{\pgfqpoint{9.789270in}{1.968560in}}{\pgfqpoint{9.791274in}{1.973398in}}{\pgfqpoint{9.791274in}{1.978441in}}%
\pgfpathcurveto{\pgfqpoint{9.791274in}{1.983485in}}{\pgfqpoint{9.789270in}{1.988323in}}{\pgfqpoint{9.785704in}{1.991889in}}%
\pgfpathcurveto{\pgfqpoint{9.782138in}{1.995456in}}{\pgfqpoint{9.777300in}{1.997460in}}{\pgfqpoint{9.772256in}{1.997460in}}%
\pgfpathcurveto{\pgfqpoint{9.767213in}{1.997460in}}{\pgfqpoint{9.762375in}{1.995456in}}{\pgfqpoint{9.758808in}{1.991889in}}%
\pgfpathcurveto{\pgfqpoint{9.755242in}{1.988323in}}{\pgfqpoint{9.753238in}{1.983485in}}{\pgfqpoint{9.753238in}{1.978441in}}%
\pgfpathcurveto{\pgfqpoint{9.753238in}{1.973398in}}{\pgfqpoint{9.755242in}{1.968560in}}{\pgfqpoint{9.758808in}{1.964994in}}%
\pgfpathcurveto{\pgfqpoint{9.762375in}{1.961427in}}{\pgfqpoint{9.767213in}{1.959423in}}{\pgfqpoint{9.772256in}{1.959423in}}%
\pgfpathclose%
\pgfusepath{fill}%
\end{pgfscope}%
\begin{pgfscope}%
\pgfpathrectangle{\pgfqpoint{6.572727in}{0.474100in}}{\pgfqpoint{4.227273in}{3.318700in}}%
\pgfusepath{clip}%
\pgfsetbuttcap%
\pgfsetroundjoin%
\definecolor{currentfill}{rgb}{0.993248,0.906157,0.143936}%
\pgfsetfillcolor{currentfill}%
\pgfsetfillopacity{0.700000}%
\pgfsetlinewidth{0.000000pt}%
\definecolor{currentstroke}{rgb}{0.000000,0.000000,0.000000}%
\pgfsetstrokecolor{currentstroke}%
\pgfsetstrokeopacity{0.700000}%
\pgfsetdash{}{0pt}%
\pgfpathmoveto{\pgfqpoint{8.436285in}{2.620874in}}%
\pgfpathcurveto{\pgfqpoint{8.441329in}{2.620874in}}{\pgfqpoint{8.446167in}{2.622878in}}{\pgfqpoint{8.449733in}{2.626444in}}%
\pgfpathcurveto{\pgfqpoint{8.453300in}{2.630010in}}{\pgfqpoint{8.455303in}{2.634848in}}{\pgfqpoint{8.455303in}{2.639892in}}%
\pgfpathcurveto{\pgfqpoint{8.455303in}{2.644936in}}{\pgfqpoint{8.453300in}{2.649773in}}{\pgfqpoint{8.449733in}{2.653340in}}%
\pgfpathcurveto{\pgfqpoint{8.446167in}{2.656906in}}{\pgfqpoint{8.441329in}{2.658910in}}{\pgfqpoint{8.436285in}{2.658910in}}%
\pgfpathcurveto{\pgfqpoint{8.431242in}{2.658910in}}{\pgfqpoint{8.426404in}{2.656906in}}{\pgfqpoint{8.422837in}{2.653340in}}%
\pgfpathcurveto{\pgfqpoint{8.419271in}{2.649773in}}{\pgfqpoint{8.417267in}{2.644936in}}{\pgfqpoint{8.417267in}{2.639892in}}%
\pgfpathcurveto{\pgfqpoint{8.417267in}{2.634848in}}{\pgfqpoint{8.419271in}{2.630010in}}{\pgfqpoint{8.422837in}{2.626444in}}%
\pgfpathcurveto{\pgfqpoint{8.426404in}{2.622878in}}{\pgfqpoint{8.431242in}{2.620874in}}{\pgfqpoint{8.436285in}{2.620874in}}%
\pgfpathclose%
\pgfusepath{fill}%
\end{pgfscope}%
\begin{pgfscope}%
\pgfpathrectangle{\pgfqpoint{6.572727in}{0.474100in}}{\pgfqpoint{4.227273in}{3.318700in}}%
\pgfusepath{clip}%
\pgfsetbuttcap%
\pgfsetroundjoin%
\definecolor{currentfill}{rgb}{0.127568,0.566949,0.550556}%
\pgfsetfillcolor{currentfill}%
\pgfsetfillopacity{0.700000}%
\pgfsetlinewidth{0.000000pt}%
\definecolor{currentstroke}{rgb}{0.000000,0.000000,0.000000}%
\pgfsetstrokecolor{currentstroke}%
\pgfsetstrokeopacity{0.700000}%
\pgfsetdash{}{0pt}%
\pgfpathmoveto{\pgfqpoint{9.664250in}{1.776857in}}%
\pgfpathcurveto{\pgfqpoint{9.669293in}{1.776857in}}{\pgfqpoint{9.674131in}{1.778861in}}{\pgfqpoint{9.677698in}{1.782428in}}%
\pgfpathcurveto{\pgfqpoint{9.681264in}{1.785994in}}{\pgfqpoint{9.683268in}{1.790832in}}{\pgfqpoint{9.683268in}{1.795876in}}%
\pgfpathcurveto{\pgfqpoint{9.683268in}{1.800919in}}{\pgfqpoint{9.681264in}{1.805757in}}{\pgfqpoint{9.677698in}{1.809323in}}%
\pgfpathcurveto{\pgfqpoint{9.674131in}{1.812890in}}{\pgfqpoint{9.669293in}{1.814894in}}{\pgfqpoint{9.664250in}{1.814894in}}%
\pgfpathcurveto{\pgfqpoint{9.659206in}{1.814894in}}{\pgfqpoint{9.654368in}{1.812890in}}{\pgfqpoint{9.650802in}{1.809323in}}%
\pgfpathcurveto{\pgfqpoint{9.647235in}{1.805757in}}{\pgfqpoint{9.645232in}{1.800919in}}{\pgfqpoint{9.645232in}{1.795876in}}%
\pgfpathcurveto{\pgfqpoint{9.645232in}{1.790832in}}{\pgfqpoint{9.647235in}{1.785994in}}{\pgfqpoint{9.650802in}{1.782428in}}%
\pgfpathcurveto{\pgfqpoint{9.654368in}{1.778861in}}{\pgfqpoint{9.659206in}{1.776857in}}{\pgfqpoint{9.664250in}{1.776857in}}%
\pgfpathclose%
\pgfusepath{fill}%
\end{pgfscope}%
\begin{pgfscope}%
\pgfpathrectangle{\pgfqpoint{6.572727in}{0.474100in}}{\pgfqpoint{4.227273in}{3.318700in}}%
\pgfusepath{clip}%
\pgfsetbuttcap%
\pgfsetroundjoin%
\definecolor{currentfill}{rgb}{0.993248,0.906157,0.143936}%
\pgfsetfillcolor{currentfill}%
\pgfsetfillopacity{0.700000}%
\pgfsetlinewidth{0.000000pt}%
\definecolor{currentstroke}{rgb}{0.000000,0.000000,0.000000}%
\pgfsetstrokecolor{currentstroke}%
\pgfsetstrokeopacity{0.700000}%
\pgfsetdash{}{0pt}%
\pgfpathmoveto{\pgfqpoint{7.774562in}{2.617640in}}%
\pgfpathcurveto{\pgfqpoint{7.779606in}{2.617640in}}{\pgfqpoint{7.784444in}{2.619644in}}{\pgfqpoint{7.788010in}{2.623210in}}%
\pgfpathcurveto{\pgfqpoint{7.791577in}{2.626777in}}{\pgfqpoint{7.793580in}{2.631615in}}{\pgfqpoint{7.793580in}{2.636658in}}%
\pgfpathcurveto{\pgfqpoint{7.793580in}{2.641702in}}{\pgfqpoint{7.791577in}{2.646540in}}{\pgfqpoint{7.788010in}{2.650106in}}%
\pgfpathcurveto{\pgfqpoint{7.784444in}{2.653672in}}{\pgfqpoint{7.779606in}{2.655676in}}{\pgfqpoint{7.774562in}{2.655676in}}%
\pgfpathcurveto{\pgfqpoint{7.769519in}{2.655676in}}{\pgfqpoint{7.764681in}{2.653672in}}{\pgfqpoint{7.761114in}{2.650106in}}%
\pgfpathcurveto{\pgfqpoint{7.757548in}{2.646540in}}{\pgfqpoint{7.755544in}{2.641702in}}{\pgfqpoint{7.755544in}{2.636658in}}%
\pgfpathcurveto{\pgfqpoint{7.755544in}{2.631615in}}{\pgfqpoint{7.757548in}{2.626777in}}{\pgfqpoint{7.761114in}{2.623210in}}%
\pgfpathcurveto{\pgfqpoint{7.764681in}{2.619644in}}{\pgfqpoint{7.769519in}{2.617640in}}{\pgfqpoint{7.774562in}{2.617640in}}%
\pgfpathclose%
\pgfusepath{fill}%
\end{pgfscope}%
\begin{pgfscope}%
\pgfpathrectangle{\pgfqpoint{6.572727in}{0.474100in}}{\pgfqpoint{4.227273in}{3.318700in}}%
\pgfusepath{clip}%
\pgfsetbuttcap%
\pgfsetroundjoin%
\definecolor{currentfill}{rgb}{0.267004,0.004874,0.329415}%
\pgfsetfillcolor{currentfill}%
\pgfsetfillopacity{0.700000}%
\pgfsetlinewidth{0.000000pt}%
\definecolor{currentstroke}{rgb}{0.000000,0.000000,0.000000}%
\pgfsetstrokecolor{currentstroke}%
\pgfsetstrokeopacity{0.700000}%
\pgfsetdash{}{0pt}%
\pgfpathmoveto{\pgfqpoint{7.723091in}{1.511474in}}%
\pgfpathcurveto{\pgfqpoint{7.728135in}{1.511474in}}{\pgfqpoint{7.732972in}{1.513478in}}{\pgfqpoint{7.736539in}{1.517045in}}%
\pgfpathcurveto{\pgfqpoint{7.740105in}{1.520611in}}{\pgfqpoint{7.742109in}{1.525449in}}{\pgfqpoint{7.742109in}{1.530492in}}%
\pgfpathcurveto{\pgfqpoint{7.742109in}{1.535536in}}{\pgfqpoint{7.740105in}{1.540374in}}{\pgfqpoint{7.736539in}{1.543940in}}%
\pgfpathcurveto{\pgfqpoint{7.732972in}{1.547507in}}{\pgfqpoint{7.728135in}{1.549511in}}{\pgfqpoint{7.723091in}{1.549511in}}%
\pgfpathcurveto{\pgfqpoint{7.718047in}{1.549511in}}{\pgfqpoint{7.713210in}{1.547507in}}{\pgfqpoint{7.709643in}{1.543940in}}%
\pgfpathcurveto{\pgfqpoint{7.706077in}{1.540374in}}{\pgfqpoint{7.704073in}{1.535536in}}{\pgfqpoint{7.704073in}{1.530492in}}%
\pgfpathcurveto{\pgfqpoint{7.704073in}{1.525449in}}{\pgfqpoint{7.706077in}{1.520611in}}{\pgfqpoint{7.709643in}{1.517045in}}%
\pgfpathcurveto{\pgfqpoint{7.713210in}{1.513478in}}{\pgfqpoint{7.718047in}{1.511474in}}{\pgfqpoint{7.723091in}{1.511474in}}%
\pgfpathclose%
\pgfusepath{fill}%
\end{pgfscope}%
\begin{pgfscope}%
\pgfpathrectangle{\pgfqpoint{6.572727in}{0.474100in}}{\pgfqpoint{4.227273in}{3.318700in}}%
\pgfusepath{clip}%
\pgfsetbuttcap%
\pgfsetroundjoin%
\definecolor{currentfill}{rgb}{0.127568,0.566949,0.550556}%
\pgfsetfillcolor{currentfill}%
\pgfsetfillopacity{0.700000}%
\pgfsetlinewidth{0.000000pt}%
\definecolor{currentstroke}{rgb}{0.000000,0.000000,0.000000}%
\pgfsetstrokecolor{currentstroke}%
\pgfsetstrokeopacity{0.700000}%
\pgfsetdash{}{0pt}%
\pgfpathmoveto{\pgfqpoint{9.771242in}{2.205898in}}%
\pgfpathcurveto{\pgfqpoint{9.776285in}{2.205898in}}{\pgfqpoint{9.781123in}{2.207902in}}{\pgfqpoint{9.784689in}{2.211469in}}%
\pgfpathcurveto{\pgfqpoint{9.788256in}{2.215035in}}{\pgfqpoint{9.790260in}{2.219873in}}{\pgfqpoint{9.790260in}{2.224917in}}%
\pgfpathcurveto{\pgfqpoint{9.790260in}{2.229960in}}{\pgfqpoint{9.788256in}{2.234798in}}{\pgfqpoint{9.784689in}{2.238364in}}%
\pgfpathcurveto{\pgfqpoint{9.781123in}{2.241931in}}{\pgfqpoint{9.776285in}{2.243935in}}{\pgfqpoint{9.771242in}{2.243935in}}%
\pgfpathcurveto{\pgfqpoint{9.766198in}{2.243935in}}{\pgfqpoint{9.761360in}{2.241931in}}{\pgfqpoint{9.757794in}{2.238364in}}%
\pgfpathcurveto{\pgfqpoint{9.754227in}{2.234798in}}{\pgfqpoint{9.752223in}{2.229960in}}{\pgfqpoint{9.752223in}{2.224917in}}%
\pgfpathcurveto{\pgfqpoint{9.752223in}{2.219873in}}{\pgfqpoint{9.754227in}{2.215035in}}{\pgfqpoint{9.757794in}{2.211469in}}%
\pgfpathcurveto{\pgfqpoint{9.761360in}{2.207902in}}{\pgfqpoint{9.766198in}{2.205898in}}{\pgfqpoint{9.771242in}{2.205898in}}%
\pgfpathclose%
\pgfusepath{fill}%
\end{pgfscope}%
\begin{pgfscope}%
\pgfpathrectangle{\pgfqpoint{6.572727in}{0.474100in}}{\pgfqpoint{4.227273in}{3.318700in}}%
\pgfusepath{clip}%
\pgfsetbuttcap%
\pgfsetroundjoin%
\definecolor{currentfill}{rgb}{0.127568,0.566949,0.550556}%
\pgfsetfillcolor{currentfill}%
\pgfsetfillopacity{0.700000}%
\pgfsetlinewidth{0.000000pt}%
\definecolor{currentstroke}{rgb}{0.000000,0.000000,0.000000}%
\pgfsetstrokecolor{currentstroke}%
\pgfsetstrokeopacity{0.700000}%
\pgfsetdash{}{0pt}%
\pgfpathmoveto{\pgfqpoint{9.498366in}{2.238778in}}%
\pgfpathcurveto{\pgfqpoint{9.503410in}{2.238778in}}{\pgfqpoint{9.508248in}{2.240782in}}{\pgfqpoint{9.511814in}{2.244348in}}%
\pgfpathcurveto{\pgfqpoint{9.515381in}{2.247915in}}{\pgfqpoint{9.517385in}{2.252752in}}{\pgfqpoint{9.517385in}{2.257796in}}%
\pgfpathcurveto{\pgfqpoint{9.517385in}{2.262840in}}{\pgfqpoint{9.515381in}{2.267678in}}{\pgfqpoint{9.511814in}{2.271244in}}%
\pgfpathcurveto{\pgfqpoint{9.508248in}{2.274810in}}{\pgfqpoint{9.503410in}{2.276814in}}{\pgfqpoint{9.498366in}{2.276814in}}%
\pgfpathcurveto{\pgfqpoint{9.493323in}{2.276814in}}{\pgfqpoint{9.488485in}{2.274810in}}{\pgfqpoint{9.484919in}{2.271244in}}%
\pgfpathcurveto{\pgfqpoint{9.481352in}{2.267678in}}{\pgfqpoint{9.479348in}{2.262840in}}{\pgfqpoint{9.479348in}{2.257796in}}%
\pgfpathcurveto{\pgfqpoint{9.479348in}{2.252752in}}{\pgfqpoint{9.481352in}{2.247915in}}{\pgfqpoint{9.484919in}{2.244348in}}%
\pgfpathcurveto{\pgfqpoint{9.488485in}{2.240782in}}{\pgfqpoint{9.493323in}{2.238778in}}{\pgfqpoint{9.498366in}{2.238778in}}%
\pgfpathclose%
\pgfusepath{fill}%
\end{pgfscope}%
\begin{pgfscope}%
\pgfpathrectangle{\pgfqpoint{6.572727in}{0.474100in}}{\pgfqpoint{4.227273in}{3.318700in}}%
\pgfusepath{clip}%
\pgfsetbuttcap%
\pgfsetroundjoin%
\definecolor{currentfill}{rgb}{0.267004,0.004874,0.329415}%
\pgfsetfillcolor{currentfill}%
\pgfsetfillopacity{0.700000}%
\pgfsetlinewidth{0.000000pt}%
\definecolor{currentstroke}{rgb}{0.000000,0.000000,0.000000}%
\pgfsetstrokecolor{currentstroke}%
\pgfsetstrokeopacity{0.700000}%
\pgfsetdash{}{0pt}%
\pgfpathmoveto{\pgfqpoint{8.061790in}{1.745243in}}%
\pgfpathcurveto{\pgfqpoint{8.066834in}{1.745243in}}{\pgfqpoint{8.071671in}{1.747247in}}{\pgfqpoint{8.075238in}{1.750813in}}%
\pgfpathcurveto{\pgfqpoint{8.078804in}{1.754380in}}{\pgfqpoint{8.080808in}{1.759217in}}{\pgfqpoint{8.080808in}{1.764261in}}%
\pgfpathcurveto{\pgfqpoint{8.080808in}{1.769305in}}{\pgfqpoint{8.078804in}{1.774143in}}{\pgfqpoint{8.075238in}{1.777709in}}%
\pgfpathcurveto{\pgfqpoint{8.071671in}{1.781275in}}{\pgfqpoint{8.066834in}{1.783279in}}{\pgfqpoint{8.061790in}{1.783279in}}%
\pgfpathcurveto{\pgfqpoint{8.056746in}{1.783279in}}{\pgfqpoint{8.051908in}{1.781275in}}{\pgfqpoint{8.048342in}{1.777709in}}%
\pgfpathcurveto{\pgfqpoint{8.044776in}{1.774143in}}{\pgfqpoint{8.042772in}{1.769305in}}{\pgfqpoint{8.042772in}{1.764261in}}%
\pgfpathcurveto{\pgfqpoint{8.042772in}{1.759217in}}{\pgfqpoint{8.044776in}{1.754380in}}{\pgfqpoint{8.048342in}{1.750813in}}%
\pgfpathcurveto{\pgfqpoint{8.051908in}{1.747247in}}{\pgfqpoint{8.056746in}{1.745243in}}{\pgfqpoint{8.061790in}{1.745243in}}%
\pgfpathclose%
\pgfusepath{fill}%
\end{pgfscope}%
\begin{pgfscope}%
\pgfpathrectangle{\pgfqpoint{6.572727in}{0.474100in}}{\pgfqpoint{4.227273in}{3.318700in}}%
\pgfusepath{clip}%
\pgfsetbuttcap%
\pgfsetroundjoin%
\definecolor{currentfill}{rgb}{0.993248,0.906157,0.143936}%
\pgfsetfillcolor{currentfill}%
\pgfsetfillopacity{0.700000}%
\pgfsetlinewidth{0.000000pt}%
\definecolor{currentstroke}{rgb}{0.000000,0.000000,0.000000}%
\pgfsetstrokecolor{currentstroke}%
\pgfsetstrokeopacity{0.700000}%
\pgfsetdash{}{0pt}%
\pgfpathmoveto{\pgfqpoint{7.891853in}{2.547996in}}%
\pgfpathcurveto{\pgfqpoint{7.896897in}{2.547996in}}{\pgfqpoint{7.901735in}{2.550000in}}{\pgfqpoint{7.905301in}{2.553566in}}%
\pgfpathcurveto{\pgfqpoint{7.908867in}{2.557133in}}{\pgfqpoint{7.910871in}{2.561970in}}{\pgfqpoint{7.910871in}{2.567014in}}%
\pgfpathcurveto{\pgfqpoint{7.910871in}{2.572058in}}{\pgfqpoint{7.908867in}{2.576895in}}{\pgfqpoint{7.905301in}{2.580462in}}%
\pgfpathcurveto{\pgfqpoint{7.901735in}{2.584028in}}{\pgfqpoint{7.896897in}{2.586032in}}{\pgfqpoint{7.891853in}{2.586032in}}%
\pgfpathcurveto{\pgfqpoint{7.886809in}{2.586032in}}{\pgfqpoint{7.881972in}{2.584028in}}{\pgfqpoint{7.878405in}{2.580462in}}%
\pgfpathcurveto{\pgfqpoint{7.874839in}{2.576895in}}{\pgfqpoint{7.872835in}{2.572058in}}{\pgfqpoint{7.872835in}{2.567014in}}%
\pgfpathcurveto{\pgfqpoint{7.872835in}{2.561970in}}{\pgfqpoint{7.874839in}{2.557133in}}{\pgfqpoint{7.878405in}{2.553566in}}%
\pgfpathcurveto{\pgfqpoint{7.881972in}{2.550000in}}{\pgfqpoint{7.886809in}{2.547996in}}{\pgfqpoint{7.891853in}{2.547996in}}%
\pgfpathclose%
\pgfusepath{fill}%
\end{pgfscope}%
\begin{pgfscope}%
\pgfpathrectangle{\pgfqpoint{6.572727in}{0.474100in}}{\pgfqpoint{4.227273in}{3.318700in}}%
\pgfusepath{clip}%
\pgfsetbuttcap%
\pgfsetroundjoin%
\definecolor{currentfill}{rgb}{0.127568,0.566949,0.550556}%
\pgfsetfillcolor{currentfill}%
\pgfsetfillopacity{0.700000}%
\pgfsetlinewidth{0.000000pt}%
\definecolor{currentstroke}{rgb}{0.000000,0.000000,0.000000}%
\pgfsetstrokecolor{currentstroke}%
\pgfsetstrokeopacity{0.700000}%
\pgfsetdash{}{0pt}%
\pgfpathmoveto{\pgfqpoint{9.984279in}{1.527744in}}%
\pgfpathcurveto{\pgfqpoint{9.989322in}{1.527744in}}{\pgfqpoint{9.994160in}{1.529748in}}{\pgfqpoint{9.997727in}{1.533314in}}%
\pgfpathcurveto{\pgfqpoint{10.001293in}{1.536881in}}{\pgfqpoint{10.003297in}{1.541718in}}{\pgfqpoint{10.003297in}{1.546762in}}%
\pgfpathcurveto{\pgfqpoint{10.003297in}{1.551806in}}{\pgfqpoint{10.001293in}{1.556644in}}{\pgfqpoint{9.997727in}{1.560210in}}%
\pgfpathcurveto{\pgfqpoint{9.994160in}{1.563776in}}{\pgfqpoint{9.989322in}{1.565780in}}{\pgfqpoint{9.984279in}{1.565780in}}%
\pgfpathcurveto{\pgfqpoint{9.979235in}{1.565780in}}{\pgfqpoint{9.974397in}{1.563776in}}{\pgfqpoint{9.970831in}{1.560210in}}%
\pgfpathcurveto{\pgfqpoint{9.967264in}{1.556644in}}{\pgfqpoint{9.965261in}{1.551806in}}{\pgfqpoint{9.965261in}{1.546762in}}%
\pgfpathcurveto{\pgfqpoint{9.965261in}{1.541718in}}{\pgfqpoint{9.967264in}{1.536881in}}{\pgfqpoint{9.970831in}{1.533314in}}%
\pgfpathcurveto{\pgfqpoint{9.974397in}{1.529748in}}{\pgfqpoint{9.979235in}{1.527744in}}{\pgfqpoint{9.984279in}{1.527744in}}%
\pgfpathclose%
\pgfusepath{fill}%
\end{pgfscope}%
\begin{pgfscope}%
\pgfpathrectangle{\pgfqpoint{6.572727in}{0.474100in}}{\pgfqpoint{4.227273in}{3.318700in}}%
\pgfusepath{clip}%
\pgfsetbuttcap%
\pgfsetroundjoin%
\definecolor{currentfill}{rgb}{0.267004,0.004874,0.329415}%
\pgfsetfillcolor{currentfill}%
\pgfsetfillopacity{0.700000}%
\pgfsetlinewidth{0.000000pt}%
\definecolor{currentstroke}{rgb}{0.000000,0.000000,0.000000}%
\pgfsetstrokecolor{currentstroke}%
\pgfsetstrokeopacity{0.700000}%
\pgfsetdash{}{0pt}%
\pgfpathmoveto{\pgfqpoint{7.950170in}{2.125240in}}%
\pgfpathcurveto{\pgfqpoint{7.955214in}{2.125240in}}{\pgfqpoint{7.960052in}{2.127244in}}{\pgfqpoint{7.963618in}{2.130811in}}%
\pgfpathcurveto{\pgfqpoint{7.967185in}{2.134377in}}{\pgfqpoint{7.969189in}{2.139215in}}{\pgfqpoint{7.969189in}{2.144258in}}%
\pgfpathcurveto{\pgfqpoint{7.969189in}{2.149302in}}{\pgfqpoint{7.967185in}{2.154140in}}{\pgfqpoint{7.963618in}{2.157706in}}%
\pgfpathcurveto{\pgfqpoint{7.960052in}{2.161273in}}{\pgfqpoint{7.955214in}{2.163277in}}{\pgfqpoint{7.950170in}{2.163277in}}%
\pgfpathcurveto{\pgfqpoint{7.945127in}{2.163277in}}{\pgfqpoint{7.940289in}{2.161273in}}{\pgfqpoint{7.936723in}{2.157706in}}%
\pgfpathcurveto{\pgfqpoint{7.933156in}{2.154140in}}{\pgfqpoint{7.931152in}{2.149302in}}{\pgfqpoint{7.931152in}{2.144258in}}%
\pgfpathcurveto{\pgfqpoint{7.931152in}{2.139215in}}{\pgfqpoint{7.933156in}{2.134377in}}{\pgfqpoint{7.936723in}{2.130811in}}%
\pgfpathcurveto{\pgfqpoint{7.940289in}{2.127244in}}{\pgfqpoint{7.945127in}{2.125240in}}{\pgfqpoint{7.950170in}{2.125240in}}%
\pgfpathclose%
\pgfusepath{fill}%
\end{pgfscope}%
\begin{pgfscope}%
\pgfpathrectangle{\pgfqpoint{6.572727in}{0.474100in}}{\pgfqpoint{4.227273in}{3.318700in}}%
\pgfusepath{clip}%
\pgfsetbuttcap%
\pgfsetroundjoin%
\definecolor{currentfill}{rgb}{0.127568,0.566949,0.550556}%
\pgfsetfillcolor{currentfill}%
\pgfsetfillopacity{0.700000}%
\pgfsetlinewidth{0.000000pt}%
\definecolor{currentstroke}{rgb}{0.000000,0.000000,0.000000}%
\pgfsetstrokecolor{currentstroke}%
\pgfsetstrokeopacity{0.700000}%
\pgfsetdash{}{0pt}%
\pgfpathmoveto{\pgfqpoint{10.138325in}{1.669074in}}%
\pgfpathcurveto{\pgfqpoint{10.143369in}{1.669074in}}{\pgfqpoint{10.148207in}{1.671077in}}{\pgfqpoint{10.151773in}{1.674644in}}%
\pgfpathcurveto{\pgfqpoint{10.155340in}{1.678210in}}{\pgfqpoint{10.157343in}{1.683048in}}{\pgfqpoint{10.157343in}{1.688092in}}%
\pgfpathcurveto{\pgfqpoint{10.157343in}{1.693135in}}{\pgfqpoint{10.155340in}{1.697973in}}{\pgfqpoint{10.151773in}{1.701540in}}%
\pgfpathcurveto{\pgfqpoint{10.148207in}{1.705106in}}{\pgfqpoint{10.143369in}{1.707110in}}{\pgfqpoint{10.138325in}{1.707110in}}%
\pgfpathcurveto{\pgfqpoint{10.133282in}{1.707110in}}{\pgfqpoint{10.128444in}{1.705106in}}{\pgfqpoint{10.124877in}{1.701540in}}%
\pgfpathcurveto{\pgfqpoint{10.121311in}{1.697973in}}{\pgfqpoint{10.119307in}{1.693135in}}{\pgfqpoint{10.119307in}{1.688092in}}%
\pgfpathcurveto{\pgfqpoint{10.119307in}{1.683048in}}{\pgfqpoint{10.121311in}{1.678210in}}{\pgfqpoint{10.124877in}{1.674644in}}%
\pgfpathcurveto{\pgfqpoint{10.128444in}{1.671077in}}{\pgfqpoint{10.133282in}{1.669074in}}{\pgfqpoint{10.138325in}{1.669074in}}%
\pgfpathclose%
\pgfusepath{fill}%
\end{pgfscope}%
\begin{pgfscope}%
\pgfpathrectangle{\pgfqpoint{6.572727in}{0.474100in}}{\pgfqpoint{4.227273in}{3.318700in}}%
\pgfusepath{clip}%
\pgfsetbuttcap%
\pgfsetroundjoin%
\definecolor{currentfill}{rgb}{0.993248,0.906157,0.143936}%
\pgfsetfillcolor{currentfill}%
\pgfsetfillopacity{0.700000}%
\pgfsetlinewidth{0.000000pt}%
\definecolor{currentstroke}{rgb}{0.000000,0.000000,0.000000}%
\pgfsetstrokecolor{currentstroke}%
\pgfsetstrokeopacity{0.700000}%
\pgfsetdash{}{0pt}%
\pgfpathmoveto{\pgfqpoint{8.195980in}{2.714096in}}%
\pgfpathcurveto{\pgfqpoint{8.201024in}{2.714096in}}{\pgfqpoint{8.205861in}{2.716099in}}{\pgfqpoint{8.209428in}{2.719666in}}%
\pgfpathcurveto{\pgfqpoint{8.212994in}{2.723232in}}{\pgfqpoint{8.214998in}{2.728070in}}{\pgfqpoint{8.214998in}{2.733114in}}%
\pgfpathcurveto{\pgfqpoint{8.214998in}{2.738157in}}{\pgfqpoint{8.212994in}{2.742995in}}{\pgfqpoint{8.209428in}{2.746562in}}%
\pgfpathcurveto{\pgfqpoint{8.205861in}{2.750128in}}{\pgfqpoint{8.201024in}{2.752132in}}{\pgfqpoint{8.195980in}{2.752132in}}%
\pgfpathcurveto{\pgfqpoint{8.190936in}{2.752132in}}{\pgfqpoint{8.186099in}{2.750128in}}{\pgfqpoint{8.182532in}{2.746562in}}%
\pgfpathcurveto{\pgfqpoint{8.178966in}{2.742995in}}{\pgfqpoint{8.176962in}{2.738157in}}{\pgfqpoint{8.176962in}{2.733114in}}%
\pgfpathcurveto{\pgfqpoint{8.176962in}{2.728070in}}{\pgfqpoint{8.178966in}{2.723232in}}{\pgfqpoint{8.182532in}{2.719666in}}%
\pgfpathcurveto{\pgfqpoint{8.186099in}{2.716099in}}{\pgfqpoint{8.190936in}{2.714096in}}{\pgfqpoint{8.195980in}{2.714096in}}%
\pgfpathclose%
\pgfusepath{fill}%
\end{pgfscope}%
\begin{pgfscope}%
\pgfpathrectangle{\pgfqpoint{6.572727in}{0.474100in}}{\pgfqpoint{4.227273in}{3.318700in}}%
\pgfusepath{clip}%
\pgfsetbuttcap%
\pgfsetroundjoin%
\definecolor{currentfill}{rgb}{0.127568,0.566949,0.550556}%
\pgfsetfillcolor{currentfill}%
\pgfsetfillopacity{0.700000}%
\pgfsetlinewidth{0.000000pt}%
\definecolor{currentstroke}{rgb}{0.000000,0.000000,0.000000}%
\pgfsetstrokecolor{currentstroke}%
\pgfsetstrokeopacity{0.700000}%
\pgfsetdash{}{0pt}%
\pgfpathmoveto{\pgfqpoint{9.428478in}{1.614705in}}%
\pgfpathcurveto{\pgfqpoint{9.433522in}{1.614705in}}{\pgfqpoint{9.438360in}{1.616709in}}{\pgfqpoint{9.441926in}{1.620275in}}%
\pgfpathcurveto{\pgfqpoint{9.445492in}{1.623842in}}{\pgfqpoint{9.447496in}{1.628679in}}{\pgfqpoint{9.447496in}{1.633723in}}%
\pgfpathcurveto{\pgfqpoint{9.447496in}{1.638767in}}{\pgfqpoint{9.445492in}{1.643605in}}{\pgfqpoint{9.441926in}{1.647171in}}%
\pgfpathcurveto{\pgfqpoint{9.438360in}{1.650737in}}{\pgfqpoint{9.433522in}{1.652741in}}{\pgfqpoint{9.428478in}{1.652741in}}%
\pgfpathcurveto{\pgfqpoint{9.423434in}{1.652741in}}{\pgfqpoint{9.418597in}{1.650737in}}{\pgfqpoint{9.415030in}{1.647171in}}%
\pgfpathcurveto{\pgfqpoint{9.411464in}{1.643605in}}{\pgfqpoint{9.409460in}{1.638767in}}{\pgfqpoint{9.409460in}{1.633723in}}%
\pgfpathcurveto{\pgfqpoint{9.409460in}{1.628679in}}{\pgfqpoint{9.411464in}{1.623842in}}{\pgfqpoint{9.415030in}{1.620275in}}%
\pgfpathcurveto{\pgfqpoint{9.418597in}{1.616709in}}{\pgfqpoint{9.423434in}{1.614705in}}{\pgfqpoint{9.428478in}{1.614705in}}%
\pgfpathclose%
\pgfusepath{fill}%
\end{pgfscope}%
\begin{pgfscope}%
\pgfpathrectangle{\pgfqpoint{6.572727in}{0.474100in}}{\pgfqpoint{4.227273in}{3.318700in}}%
\pgfusepath{clip}%
\pgfsetbuttcap%
\pgfsetroundjoin%
\definecolor{currentfill}{rgb}{0.127568,0.566949,0.550556}%
\pgfsetfillcolor{currentfill}%
\pgfsetfillopacity{0.700000}%
\pgfsetlinewidth{0.000000pt}%
\definecolor{currentstroke}{rgb}{0.000000,0.000000,0.000000}%
\pgfsetstrokecolor{currentstroke}%
\pgfsetstrokeopacity{0.700000}%
\pgfsetdash{}{0pt}%
\pgfpathmoveto{\pgfqpoint{10.116195in}{1.308328in}}%
\pgfpathcurveto{\pgfqpoint{10.121239in}{1.308328in}}{\pgfqpoint{10.126077in}{1.310332in}}{\pgfqpoint{10.129643in}{1.313898in}}%
\pgfpathcurveto{\pgfqpoint{10.133209in}{1.317465in}}{\pgfqpoint{10.135213in}{1.322303in}}{\pgfqpoint{10.135213in}{1.327346in}}%
\pgfpathcurveto{\pgfqpoint{10.135213in}{1.332390in}}{\pgfqpoint{10.133209in}{1.337228in}}{\pgfqpoint{10.129643in}{1.340794in}}%
\pgfpathcurveto{\pgfqpoint{10.126077in}{1.344360in}}{\pgfqpoint{10.121239in}{1.346364in}}{\pgfqpoint{10.116195in}{1.346364in}}%
\pgfpathcurveto{\pgfqpoint{10.111152in}{1.346364in}}{\pgfqpoint{10.106314in}{1.344360in}}{\pgfqpoint{10.102747in}{1.340794in}}%
\pgfpathcurveto{\pgfqpoint{10.099181in}{1.337228in}}{\pgfqpoint{10.097177in}{1.332390in}}{\pgfqpoint{10.097177in}{1.327346in}}%
\pgfpathcurveto{\pgfqpoint{10.097177in}{1.322303in}}{\pgfqpoint{10.099181in}{1.317465in}}{\pgfqpoint{10.102747in}{1.313898in}}%
\pgfpathcurveto{\pgfqpoint{10.106314in}{1.310332in}}{\pgfqpoint{10.111152in}{1.308328in}}{\pgfqpoint{10.116195in}{1.308328in}}%
\pgfpathclose%
\pgfusepath{fill}%
\end{pgfscope}%
\begin{pgfscope}%
\pgfpathrectangle{\pgfqpoint{6.572727in}{0.474100in}}{\pgfqpoint{4.227273in}{3.318700in}}%
\pgfusepath{clip}%
\pgfsetbuttcap%
\pgfsetroundjoin%
\definecolor{currentfill}{rgb}{0.993248,0.906157,0.143936}%
\pgfsetfillcolor{currentfill}%
\pgfsetfillopacity{0.700000}%
\pgfsetlinewidth{0.000000pt}%
\definecolor{currentstroke}{rgb}{0.000000,0.000000,0.000000}%
\pgfsetstrokecolor{currentstroke}%
\pgfsetstrokeopacity{0.700000}%
\pgfsetdash{}{0pt}%
\pgfpathmoveto{\pgfqpoint{8.103464in}{2.665122in}}%
\pgfpathcurveto{\pgfqpoint{8.108508in}{2.665122in}}{\pgfqpoint{8.113345in}{2.667126in}}{\pgfqpoint{8.116912in}{2.670692in}}%
\pgfpathcurveto{\pgfqpoint{8.120478in}{2.674258in}}{\pgfqpoint{8.122482in}{2.679096in}}{\pgfqpoint{8.122482in}{2.684140in}}%
\pgfpathcurveto{\pgfqpoint{8.122482in}{2.689183in}}{\pgfqpoint{8.120478in}{2.694021in}}{\pgfqpoint{8.116912in}{2.697588in}}%
\pgfpathcurveto{\pgfqpoint{8.113345in}{2.701154in}}{\pgfqpoint{8.108508in}{2.703158in}}{\pgfqpoint{8.103464in}{2.703158in}}%
\pgfpathcurveto{\pgfqpoint{8.098420in}{2.703158in}}{\pgfqpoint{8.093582in}{2.701154in}}{\pgfqpoint{8.090016in}{2.697588in}}%
\pgfpathcurveto{\pgfqpoint{8.086450in}{2.694021in}}{\pgfqpoint{8.084446in}{2.689183in}}{\pgfqpoint{8.084446in}{2.684140in}}%
\pgfpathcurveto{\pgfqpoint{8.084446in}{2.679096in}}{\pgfqpoint{8.086450in}{2.674258in}}{\pgfqpoint{8.090016in}{2.670692in}}%
\pgfpathcurveto{\pgfqpoint{8.093582in}{2.667126in}}{\pgfqpoint{8.098420in}{2.665122in}}{\pgfqpoint{8.103464in}{2.665122in}}%
\pgfpathclose%
\pgfusepath{fill}%
\end{pgfscope}%
\begin{pgfscope}%
\pgfpathrectangle{\pgfqpoint{6.572727in}{0.474100in}}{\pgfqpoint{4.227273in}{3.318700in}}%
\pgfusepath{clip}%
\pgfsetbuttcap%
\pgfsetroundjoin%
\definecolor{currentfill}{rgb}{0.127568,0.566949,0.550556}%
\pgfsetfillcolor{currentfill}%
\pgfsetfillopacity{0.700000}%
\pgfsetlinewidth{0.000000pt}%
\definecolor{currentstroke}{rgb}{0.000000,0.000000,0.000000}%
\pgfsetstrokecolor{currentstroke}%
\pgfsetstrokeopacity{0.700000}%
\pgfsetdash{}{0pt}%
\pgfpathmoveto{\pgfqpoint{9.523161in}{1.114760in}}%
\pgfpathcurveto{\pgfqpoint{9.528205in}{1.114760in}}{\pgfqpoint{9.533043in}{1.116764in}}{\pgfqpoint{9.536609in}{1.120331in}}%
\pgfpathcurveto{\pgfqpoint{9.540176in}{1.123897in}}{\pgfqpoint{9.542180in}{1.128735in}}{\pgfqpoint{9.542180in}{1.133779in}}%
\pgfpathcurveto{\pgfqpoint{9.542180in}{1.138822in}}{\pgfqpoint{9.540176in}{1.143660in}}{\pgfqpoint{9.536609in}{1.147226in}}%
\pgfpathcurveto{\pgfqpoint{9.533043in}{1.150793in}}{\pgfqpoint{9.528205in}{1.152797in}}{\pgfqpoint{9.523161in}{1.152797in}}%
\pgfpathcurveto{\pgfqpoint{9.518118in}{1.152797in}}{\pgfqpoint{9.513280in}{1.150793in}}{\pgfqpoint{9.509714in}{1.147226in}}%
\pgfpathcurveto{\pgfqpoint{9.506147in}{1.143660in}}{\pgfqpoint{9.504143in}{1.138822in}}{\pgfqpoint{9.504143in}{1.133779in}}%
\pgfpathcurveto{\pgfqpoint{9.504143in}{1.128735in}}{\pgfqpoint{9.506147in}{1.123897in}}{\pgfqpoint{9.509714in}{1.120331in}}%
\pgfpathcurveto{\pgfqpoint{9.513280in}{1.116764in}}{\pgfqpoint{9.518118in}{1.114760in}}{\pgfqpoint{9.523161in}{1.114760in}}%
\pgfpathclose%
\pgfusepath{fill}%
\end{pgfscope}%
\begin{pgfscope}%
\pgfpathrectangle{\pgfqpoint{6.572727in}{0.474100in}}{\pgfqpoint{4.227273in}{3.318700in}}%
\pgfusepath{clip}%
\pgfsetbuttcap%
\pgfsetroundjoin%
\definecolor{currentfill}{rgb}{0.127568,0.566949,0.550556}%
\pgfsetfillcolor{currentfill}%
\pgfsetfillopacity{0.700000}%
\pgfsetlinewidth{0.000000pt}%
\definecolor{currentstroke}{rgb}{0.000000,0.000000,0.000000}%
\pgfsetstrokecolor{currentstroke}%
\pgfsetstrokeopacity{0.700000}%
\pgfsetdash{}{0pt}%
\pgfpathmoveto{\pgfqpoint{9.822215in}{1.796532in}}%
\pgfpathcurveto{\pgfqpoint{9.827259in}{1.796532in}}{\pgfqpoint{9.832097in}{1.798536in}}{\pgfqpoint{9.835663in}{1.802102in}}%
\pgfpathcurveto{\pgfqpoint{9.839229in}{1.805669in}}{\pgfqpoint{9.841233in}{1.810506in}}{\pgfqpoint{9.841233in}{1.815550in}}%
\pgfpathcurveto{\pgfqpoint{9.841233in}{1.820594in}}{\pgfqpoint{9.839229in}{1.825432in}}{\pgfqpoint{9.835663in}{1.828998in}}%
\pgfpathcurveto{\pgfqpoint{9.832097in}{1.832564in}}{\pgfqpoint{9.827259in}{1.834568in}}{\pgfqpoint{9.822215in}{1.834568in}}%
\pgfpathcurveto{\pgfqpoint{9.817171in}{1.834568in}}{\pgfqpoint{9.812334in}{1.832564in}}{\pgfqpoint{9.808767in}{1.828998in}}%
\pgfpathcurveto{\pgfqpoint{9.805201in}{1.825432in}}{\pgfqpoint{9.803197in}{1.820594in}}{\pgfqpoint{9.803197in}{1.815550in}}%
\pgfpathcurveto{\pgfqpoint{9.803197in}{1.810506in}}{\pgfqpoint{9.805201in}{1.805669in}}{\pgfqpoint{9.808767in}{1.802102in}}%
\pgfpathcurveto{\pgfqpoint{9.812334in}{1.798536in}}{\pgfqpoint{9.817171in}{1.796532in}}{\pgfqpoint{9.822215in}{1.796532in}}%
\pgfpathclose%
\pgfusepath{fill}%
\end{pgfscope}%
\begin{pgfscope}%
\pgfpathrectangle{\pgfqpoint{6.572727in}{0.474100in}}{\pgfqpoint{4.227273in}{3.318700in}}%
\pgfusepath{clip}%
\pgfsetbuttcap%
\pgfsetroundjoin%
\definecolor{currentfill}{rgb}{0.127568,0.566949,0.550556}%
\pgfsetfillcolor{currentfill}%
\pgfsetfillopacity{0.700000}%
\pgfsetlinewidth{0.000000pt}%
\definecolor{currentstroke}{rgb}{0.000000,0.000000,0.000000}%
\pgfsetstrokecolor{currentstroke}%
\pgfsetstrokeopacity{0.700000}%
\pgfsetdash{}{0pt}%
\pgfpathmoveto{\pgfqpoint{10.196545in}{1.241149in}}%
\pgfpathcurveto{\pgfqpoint{10.201589in}{1.241149in}}{\pgfqpoint{10.206427in}{1.243153in}}{\pgfqpoint{10.209993in}{1.246719in}}%
\pgfpathcurveto{\pgfqpoint{10.213560in}{1.250286in}}{\pgfqpoint{10.215563in}{1.255124in}}{\pgfqpoint{10.215563in}{1.260167in}}%
\pgfpathcurveto{\pgfqpoint{10.215563in}{1.265211in}}{\pgfqpoint{10.213560in}{1.270049in}}{\pgfqpoint{10.209993in}{1.273615in}}%
\pgfpathcurveto{\pgfqpoint{10.206427in}{1.277182in}}{\pgfqpoint{10.201589in}{1.279185in}}{\pgfqpoint{10.196545in}{1.279185in}}%
\pgfpathcurveto{\pgfqpoint{10.191502in}{1.279185in}}{\pgfqpoint{10.186664in}{1.277182in}}{\pgfqpoint{10.183097in}{1.273615in}}%
\pgfpathcurveto{\pgfqpoint{10.179531in}{1.270049in}}{\pgfqpoint{10.177527in}{1.265211in}}{\pgfqpoint{10.177527in}{1.260167in}}%
\pgfpathcurveto{\pgfqpoint{10.177527in}{1.255124in}}{\pgfqpoint{10.179531in}{1.250286in}}{\pgfqpoint{10.183097in}{1.246719in}}%
\pgfpathcurveto{\pgfqpoint{10.186664in}{1.243153in}}{\pgfqpoint{10.191502in}{1.241149in}}{\pgfqpoint{10.196545in}{1.241149in}}%
\pgfpathclose%
\pgfusepath{fill}%
\end{pgfscope}%
\begin{pgfscope}%
\pgfpathrectangle{\pgfqpoint{6.572727in}{0.474100in}}{\pgfqpoint{4.227273in}{3.318700in}}%
\pgfusepath{clip}%
\pgfsetbuttcap%
\pgfsetroundjoin%
\definecolor{currentfill}{rgb}{0.127568,0.566949,0.550556}%
\pgfsetfillcolor{currentfill}%
\pgfsetfillopacity{0.700000}%
\pgfsetlinewidth{0.000000pt}%
\definecolor{currentstroke}{rgb}{0.000000,0.000000,0.000000}%
\pgfsetstrokecolor{currentstroke}%
\pgfsetstrokeopacity{0.700000}%
\pgfsetdash{}{0pt}%
\pgfpathmoveto{\pgfqpoint{9.040576in}{1.839627in}}%
\pgfpathcurveto{\pgfqpoint{9.045619in}{1.839627in}}{\pgfqpoint{9.050457in}{1.841631in}}{\pgfqpoint{9.054023in}{1.845198in}}%
\pgfpathcurveto{\pgfqpoint{9.057590in}{1.848764in}}{\pgfqpoint{9.059594in}{1.853602in}}{\pgfqpoint{9.059594in}{1.858646in}}%
\pgfpathcurveto{\pgfqpoint{9.059594in}{1.863689in}}{\pgfqpoint{9.057590in}{1.868527in}}{\pgfqpoint{9.054023in}{1.872093in}}%
\pgfpathcurveto{\pgfqpoint{9.050457in}{1.875660in}}{\pgfqpoint{9.045619in}{1.877664in}}{\pgfqpoint{9.040576in}{1.877664in}}%
\pgfpathcurveto{\pgfqpoint{9.035532in}{1.877664in}}{\pgfqpoint{9.030694in}{1.875660in}}{\pgfqpoint{9.027128in}{1.872093in}}%
\pgfpathcurveto{\pgfqpoint{9.023561in}{1.868527in}}{\pgfqpoint{9.021557in}{1.863689in}}{\pgfqpoint{9.021557in}{1.858646in}}%
\pgfpathcurveto{\pgfqpoint{9.021557in}{1.853602in}}{\pgfqpoint{9.023561in}{1.848764in}}{\pgfqpoint{9.027128in}{1.845198in}}%
\pgfpathcurveto{\pgfqpoint{9.030694in}{1.841631in}}{\pgfqpoint{9.035532in}{1.839627in}}{\pgfqpoint{9.040576in}{1.839627in}}%
\pgfpathclose%
\pgfusepath{fill}%
\end{pgfscope}%
\begin{pgfscope}%
\pgfpathrectangle{\pgfqpoint{6.572727in}{0.474100in}}{\pgfqpoint{4.227273in}{3.318700in}}%
\pgfusepath{clip}%
\pgfsetbuttcap%
\pgfsetroundjoin%
\definecolor{currentfill}{rgb}{0.127568,0.566949,0.550556}%
\pgfsetfillcolor{currentfill}%
\pgfsetfillopacity{0.700000}%
\pgfsetlinewidth{0.000000pt}%
\definecolor{currentstroke}{rgb}{0.000000,0.000000,0.000000}%
\pgfsetstrokecolor{currentstroke}%
\pgfsetstrokeopacity{0.700000}%
\pgfsetdash{}{0pt}%
\pgfpathmoveto{\pgfqpoint{9.724121in}{1.318908in}}%
\pgfpathcurveto{\pgfqpoint{9.729165in}{1.318908in}}{\pgfqpoint{9.734003in}{1.320912in}}{\pgfqpoint{9.737569in}{1.324478in}}%
\pgfpathcurveto{\pgfqpoint{9.741135in}{1.328045in}}{\pgfqpoint{9.743139in}{1.332883in}}{\pgfqpoint{9.743139in}{1.337926in}}%
\pgfpathcurveto{\pgfqpoint{9.743139in}{1.342970in}}{\pgfqpoint{9.741135in}{1.347808in}}{\pgfqpoint{9.737569in}{1.351374in}}%
\pgfpathcurveto{\pgfqpoint{9.734003in}{1.354941in}}{\pgfqpoint{9.729165in}{1.356944in}}{\pgfqpoint{9.724121in}{1.356944in}}%
\pgfpathcurveto{\pgfqpoint{9.719077in}{1.356944in}}{\pgfqpoint{9.714240in}{1.354941in}}{\pgfqpoint{9.710673in}{1.351374in}}%
\pgfpathcurveto{\pgfqpoint{9.707107in}{1.347808in}}{\pgfqpoint{9.705103in}{1.342970in}}{\pgfqpoint{9.705103in}{1.337926in}}%
\pgfpathcurveto{\pgfqpoint{9.705103in}{1.332883in}}{\pgfqpoint{9.707107in}{1.328045in}}{\pgfqpoint{9.710673in}{1.324478in}}%
\pgfpathcurveto{\pgfqpoint{9.714240in}{1.320912in}}{\pgfqpoint{9.719077in}{1.318908in}}{\pgfqpoint{9.724121in}{1.318908in}}%
\pgfpathclose%
\pgfusepath{fill}%
\end{pgfscope}%
\begin{pgfscope}%
\pgfpathrectangle{\pgfqpoint{6.572727in}{0.474100in}}{\pgfqpoint{4.227273in}{3.318700in}}%
\pgfusepath{clip}%
\pgfsetbuttcap%
\pgfsetroundjoin%
\definecolor{currentfill}{rgb}{0.127568,0.566949,0.550556}%
\pgfsetfillcolor{currentfill}%
\pgfsetfillopacity{0.700000}%
\pgfsetlinewidth{0.000000pt}%
\definecolor{currentstroke}{rgb}{0.000000,0.000000,0.000000}%
\pgfsetstrokecolor{currentstroke}%
\pgfsetstrokeopacity{0.700000}%
\pgfsetdash{}{0pt}%
\pgfpathmoveto{\pgfqpoint{10.414029in}{1.713492in}}%
\pgfpathcurveto{\pgfqpoint{10.419072in}{1.713492in}}{\pgfqpoint{10.423910in}{1.715496in}}{\pgfqpoint{10.427477in}{1.719062in}}%
\pgfpathcurveto{\pgfqpoint{10.431043in}{1.722629in}}{\pgfqpoint{10.433047in}{1.727466in}}{\pgfqpoint{10.433047in}{1.732510in}}%
\pgfpathcurveto{\pgfqpoint{10.433047in}{1.737554in}}{\pgfqpoint{10.431043in}{1.742391in}}{\pgfqpoint{10.427477in}{1.745958in}}%
\pgfpathcurveto{\pgfqpoint{10.423910in}{1.749524in}}{\pgfqpoint{10.419072in}{1.751528in}}{\pgfqpoint{10.414029in}{1.751528in}}%
\pgfpathcurveto{\pgfqpoint{10.408985in}{1.751528in}}{\pgfqpoint{10.404147in}{1.749524in}}{\pgfqpoint{10.400581in}{1.745958in}}%
\pgfpathcurveto{\pgfqpoint{10.397015in}{1.742391in}}{\pgfqpoint{10.395011in}{1.737554in}}{\pgfqpoint{10.395011in}{1.732510in}}%
\pgfpathcurveto{\pgfqpoint{10.395011in}{1.727466in}}{\pgfqpoint{10.397015in}{1.722629in}}{\pgfqpoint{10.400581in}{1.719062in}}%
\pgfpathcurveto{\pgfqpoint{10.404147in}{1.715496in}}{\pgfqpoint{10.408985in}{1.713492in}}{\pgfqpoint{10.414029in}{1.713492in}}%
\pgfpathclose%
\pgfusepath{fill}%
\end{pgfscope}%
\begin{pgfscope}%
\pgfpathrectangle{\pgfqpoint{6.572727in}{0.474100in}}{\pgfqpoint{4.227273in}{3.318700in}}%
\pgfusepath{clip}%
\pgfsetbuttcap%
\pgfsetroundjoin%
\definecolor{currentfill}{rgb}{0.267004,0.004874,0.329415}%
\pgfsetfillcolor{currentfill}%
\pgfsetfillopacity{0.700000}%
\pgfsetlinewidth{0.000000pt}%
\definecolor{currentstroke}{rgb}{0.000000,0.000000,0.000000}%
\pgfsetstrokecolor{currentstroke}%
\pgfsetstrokeopacity{0.700000}%
\pgfsetdash{}{0pt}%
\pgfpathmoveto{\pgfqpoint{8.703118in}{0.803521in}}%
\pgfpathcurveto{\pgfqpoint{8.708162in}{0.803521in}}{\pgfqpoint{8.713000in}{0.805525in}}{\pgfqpoint{8.716566in}{0.809092in}}%
\pgfpathcurveto{\pgfqpoint{8.720133in}{0.812658in}}{\pgfqpoint{8.722136in}{0.817496in}}{\pgfqpoint{8.722136in}{0.822540in}}%
\pgfpathcurveto{\pgfqpoint{8.722136in}{0.827583in}}{\pgfqpoint{8.720133in}{0.832421in}}{\pgfqpoint{8.716566in}{0.835987in}}%
\pgfpathcurveto{\pgfqpoint{8.713000in}{0.839554in}}{\pgfqpoint{8.708162in}{0.841558in}}{\pgfqpoint{8.703118in}{0.841558in}}%
\pgfpathcurveto{\pgfqpoint{8.698075in}{0.841558in}}{\pgfqpoint{8.693237in}{0.839554in}}{\pgfqpoint{8.689670in}{0.835987in}}%
\pgfpathcurveto{\pgfqpoint{8.686104in}{0.832421in}}{\pgfqpoint{8.684100in}{0.827583in}}{\pgfqpoint{8.684100in}{0.822540in}}%
\pgfpathcurveto{\pgfqpoint{8.684100in}{0.817496in}}{\pgfqpoint{8.686104in}{0.812658in}}{\pgfqpoint{8.689670in}{0.809092in}}%
\pgfpathcurveto{\pgfqpoint{8.693237in}{0.805525in}}{\pgfqpoint{8.698075in}{0.803521in}}{\pgfqpoint{8.703118in}{0.803521in}}%
\pgfpathclose%
\pgfusepath{fill}%
\end{pgfscope}%
\begin{pgfscope}%
\pgfpathrectangle{\pgfqpoint{6.572727in}{0.474100in}}{\pgfqpoint{4.227273in}{3.318700in}}%
\pgfusepath{clip}%
\pgfsetbuttcap%
\pgfsetroundjoin%
\definecolor{currentfill}{rgb}{0.127568,0.566949,0.550556}%
\pgfsetfillcolor{currentfill}%
\pgfsetfillopacity{0.700000}%
\pgfsetlinewidth{0.000000pt}%
\definecolor{currentstroke}{rgb}{0.000000,0.000000,0.000000}%
\pgfsetstrokecolor{currentstroke}%
\pgfsetstrokeopacity{0.700000}%
\pgfsetdash{}{0pt}%
\pgfpathmoveto{\pgfqpoint{9.323233in}{2.160461in}}%
\pgfpathcurveto{\pgfqpoint{9.328276in}{2.160461in}}{\pgfqpoint{9.333114in}{2.162465in}}{\pgfqpoint{9.336680in}{2.166032in}}%
\pgfpathcurveto{\pgfqpoint{9.340247in}{2.169598in}}{\pgfqpoint{9.342251in}{2.174436in}}{\pgfqpoint{9.342251in}{2.179479in}}%
\pgfpathcurveto{\pgfqpoint{9.342251in}{2.184523in}}{\pgfqpoint{9.340247in}{2.189361in}}{\pgfqpoint{9.336680in}{2.192927in}}%
\pgfpathcurveto{\pgfqpoint{9.333114in}{2.196494in}}{\pgfqpoint{9.328276in}{2.198498in}}{\pgfqpoint{9.323233in}{2.198498in}}%
\pgfpathcurveto{\pgfqpoint{9.318189in}{2.198498in}}{\pgfqpoint{9.313351in}{2.196494in}}{\pgfqpoint{9.309785in}{2.192927in}}%
\pgfpathcurveto{\pgfqpoint{9.306218in}{2.189361in}}{\pgfqpoint{9.304214in}{2.184523in}}{\pgfqpoint{9.304214in}{2.179479in}}%
\pgfpathcurveto{\pgfqpoint{9.304214in}{2.174436in}}{\pgfqpoint{9.306218in}{2.169598in}}{\pgfqpoint{9.309785in}{2.166032in}}%
\pgfpathcurveto{\pgfqpoint{9.313351in}{2.162465in}}{\pgfqpoint{9.318189in}{2.160461in}}{\pgfqpoint{9.323233in}{2.160461in}}%
\pgfpathclose%
\pgfusepath{fill}%
\end{pgfscope}%
\begin{pgfscope}%
\pgfpathrectangle{\pgfqpoint{6.572727in}{0.474100in}}{\pgfqpoint{4.227273in}{3.318700in}}%
\pgfusepath{clip}%
\pgfsetbuttcap%
\pgfsetroundjoin%
\definecolor{currentfill}{rgb}{0.267004,0.004874,0.329415}%
\pgfsetfillcolor{currentfill}%
\pgfsetfillopacity{0.700000}%
\pgfsetlinewidth{0.000000pt}%
\definecolor{currentstroke}{rgb}{0.000000,0.000000,0.000000}%
\pgfsetstrokecolor{currentstroke}%
\pgfsetstrokeopacity{0.700000}%
\pgfsetdash{}{0pt}%
\pgfpathmoveto{\pgfqpoint{7.856344in}{1.168467in}}%
\pgfpathcurveto{\pgfqpoint{7.861387in}{1.168467in}}{\pgfqpoint{7.866225in}{1.170470in}}{\pgfqpoint{7.869792in}{1.174037in}}%
\pgfpathcurveto{\pgfqpoint{7.873358in}{1.177603in}}{\pgfqpoint{7.875362in}{1.182441in}}{\pgfqpoint{7.875362in}{1.187485in}}%
\pgfpathcurveto{\pgfqpoint{7.875362in}{1.192528in}}{\pgfqpoint{7.873358in}{1.197366in}}{\pgfqpoint{7.869792in}{1.200933in}}%
\pgfpathcurveto{\pgfqpoint{7.866225in}{1.204499in}}{\pgfqpoint{7.861387in}{1.206503in}}{\pgfqpoint{7.856344in}{1.206503in}}%
\pgfpathcurveto{\pgfqpoint{7.851300in}{1.206503in}}{\pgfqpoint{7.846462in}{1.204499in}}{\pgfqpoint{7.842896in}{1.200933in}}%
\pgfpathcurveto{\pgfqpoint{7.839330in}{1.197366in}}{\pgfqpoint{7.837326in}{1.192528in}}{\pgfqpoint{7.837326in}{1.187485in}}%
\pgfpathcurveto{\pgfqpoint{7.837326in}{1.182441in}}{\pgfqpoint{7.839330in}{1.177603in}}{\pgfqpoint{7.842896in}{1.174037in}}%
\pgfpathcurveto{\pgfqpoint{7.846462in}{1.170470in}}{\pgfqpoint{7.851300in}{1.168467in}}{\pgfqpoint{7.856344in}{1.168467in}}%
\pgfpathclose%
\pgfusepath{fill}%
\end{pgfscope}%
\begin{pgfscope}%
\pgfpathrectangle{\pgfqpoint{6.572727in}{0.474100in}}{\pgfqpoint{4.227273in}{3.318700in}}%
\pgfusepath{clip}%
\pgfsetbuttcap%
\pgfsetroundjoin%
\definecolor{currentfill}{rgb}{0.267004,0.004874,0.329415}%
\pgfsetfillcolor{currentfill}%
\pgfsetfillopacity{0.700000}%
\pgfsetlinewidth{0.000000pt}%
\definecolor{currentstroke}{rgb}{0.000000,0.000000,0.000000}%
\pgfsetstrokecolor{currentstroke}%
\pgfsetstrokeopacity{0.700000}%
\pgfsetdash{}{0pt}%
\pgfpathmoveto{\pgfqpoint{7.767347in}{1.636610in}}%
\pgfpathcurveto{\pgfqpoint{7.772391in}{1.636610in}}{\pgfqpoint{7.777228in}{1.638613in}}{\pgfqpoint{7.780795in}{1.642180in}}%
\pgfpathcurveto{\pgfqpoint{7.784361in}{1.645746in}}{\pgfqpoint{7.786365in}{1.650584in}}{\pgfqpoint{7.786365in}{1.655628in}}%
\pgfpathcurveto{\pgfqpoint{7.786365in}{1.660671in}}{\pgfqpoint{7.784361in}{1.665509in}}{\pgfqpoint{7.780795in}{1.669076in}}%
\pgfpathcurveto{\pgfqpoint{7.777228in}{1.672642in}}{\pgfqpoint{7.772391in}{1.674646in}}{\pgfqpoint{7.767347in}{1.674646in}}%
\pgfpathcurveto{\pgfqpoint{7.762303in}{1.674646in}}{\pgfqpoint{7.757466in}{1.672642in}}{\pgfqpoint{7.753899in}{1.669076in}}%
\pgfpathcurveto{\pgfqpoint{7.750333in}{1.665509in}}{\pgfqpoint{7.748329in}{1.660671in}}{\pgfqpoint{7.748329in}{1.655628in}}%
\pgfpathcurveto{\pgfqpoint{7.748329in}{1.650584in}}{\pgfqpoint{7.750333in}{1.645746in}}{\pgfqpoint{7.753899in}{1.642180in}}%
\pgfpathcurveto{\pgfqpoint{7.757466in}{1.638613in}}{\pgfqpoint{7.762303in}{1.636610in}}{\pgfqpoint{7.767347in}{1.636610in}}%
\pgfpathclose%
\pgfusepath{fill}%
\end{pgfscope}%
\begin{pgfscope}%
\pgfpathrectangle{\pgfqpoint{6.572727in}{0.474100in}}{\pgfqpoint{4.227273in}{3.318700in}}%
\pgfusepath{clip}%
\pgfsetbuttcap%
\pgfsetroundjoin%
\definecolor{currentfill}{rgb}{0.127568,0.566949,0.550556}%
\pgfsetfillcolor{currentfill}%
\pgfsetfillopacity{0.700000}%
\pgfsetlinewidth{0.000000pt}%
\definecolor{currentstroke}{rgb}{0.000000,0.000000,0.000000}%
\pgfsetstrokecolor{currentstroke}%
\pgfsetstrokeopacity{0.700000}%
\pgfsetdash{}{0pt}%
\pgfpathmoveto{\pgfqpoint{9.706365in}{2.097766in}}%
\pgfpathcurveto{\pgfqpoint{9.711408in}{2.097766in}}{\pgfqpoint{9.716246in}{2.099770in}}{\pgfqpoint{9.719813in}{2.103336in}}%
\pgfpathcurveto{\pgfqpoint{9.723379in}{2.106902in}}{\pgfqpoint{9.725383in}{2.111740in}}{\pgfqpoint{9.725383in}{2.116784in}}%
\pgfpathcurveto{\pgfqpoint{9.725383in}{2.121828in}}{\pgfqpoint{9.723379in}{2.126665in}}{\pgfqpoint{9.719813in}{2.130232in}}%
\pgfpathcurveto{\pgfqpoint{9.716246in}{2.133798in}}{\pgfqpoint{9.711408in}{2.135802in}}{\pgfqpoint{9.706365in}{2.135802in}}%
\pgfpathcurveto{\pgfqpoint{9.701321in}{2.135802in}}{\pgfqpoint{9.696483in}{2.133798in}}{\pgfqpoint{9.692917in}{2.130232in}}%
\pgfpathcurveto{\pgfqpoint{9.689350in}{2.126665in}}{\pgfqpoint{9.687347in}{2.121828in}}{\pgfqpoint{9.687347in}{2.116784in}}%
\pgfpathcurveto{\pgfqpoint{9.687347in}{2.111740in}}{\pgfqpoint{9.689350in}{2.106902in}}{\pgfqpoint{9.692917in}{2.103336in}}%
\pgfpathcurveto{\pgfqpoint{9.696483in}{2.099770in}}{\pgfqpoint{9.701321in}{2.097766in}}{\pgfqpoint{9.706365in}{2.097766in}}%
\pgfpathclose%
\pgfusepath{fill}%
\end{pgfscope}%
\begin{pgfscope}%
\pgfpathrectangle{\pgfqpoint{6.572727in}{0.474100in}}{\pgfqpoint{4.227273in}{3.318700in}}%
\pgfusepath{clip}%
\pgfsetbuttcap%
\pgfsetroundjoin%
\definecolor{currentfill}{rgb}{0.267004,0.004874,0.329415}%
\pgfsetfillcolor{currentfill}%
\pgfsetfillopacity{0.700000}%
\pgfsetlinewidth{0.000000pt}%
\definecolor{currentstroke}{rgb}{0.000000,0.000000,0.000000}%
\pgfsetstrokecolor{currentstroke}%
\pgfsetstrokeopacity{0.700000}%
\pgfsetdash{}{0pt}%
\pgfpathmoveto{\pgfqpoint{7.866135in}{1.688261in}}%
\pgfpathcurveto{\pgfqpoint{7.871179in}{1.688261in}}{\pgfqpoint{7.876017in}{1.690265in}}{\pgfqpoint{7.879583in}{1.693831in}}%
\pgfpathcurveto{\pgfqpoint{7.883150in}{1.697397in}}{\pgfqpoint{7.885153in}{1.702235in}}{\pgfqpoint{7.885153in}{1.707279in}}%
\pgfpathcurveto{\pgfqpoint{7.885153in}{1.712323in}}{\pgfqpoint{7.883150in}{1.717160in}}{\pgfqpoint{7.879583in}{1.720727in}}%
\pgfpathcurveto{\pgfqpoint{7.876017in}{1.724293in}}{\pgfqpoint{7.871179in}{1.726297in}}{\pgfqpoint{7.866135in}{1.726297in}}%
\pgfpathcurveto{\pgfqpoint{7.861092in}{1.726297in}}{\pgfqpoint{7.856254in}{1.724293in}}{\pgfqpoint{7.852687in}{1.720727in}}%
\pgfpathcurveto{\pgfqpoint{7.849121in}{1.717160in}}{\pgfqpoint{7.847117in}{1.712323in}}{\pgfqpoint{7.847117in}{1.707279in}}%
\pgfpathcurveto{\pgfqpoint{7.847117in}{1.702235in}}{\pgfqpoint{7.849121in}{1.697397in}}{\pgfqpoint{7.852687in}{1.693831in}}%
\pgfpathcurveto{\pgfqpoint{7.856254in}{1.690265in}}{\pgfqpoint{7.861092in}{1.688261in}}{\pgfqpoint{7.866135in}{1.688261in}}%
\pgfpathclose%
\pgfusepath{fill}%
\end{pgfscope}%
\begin{pgfscope}%
\pgfpathrectangle{\pgfqpoint{6.572727in}{0.474100in}}{\pgfqpoint{4.227273in}{3.318700in}}%
\pgfusepath{clip}%
\pgfsetbuttcap%
\pgfsetroundjoin%
\definecolor{currentfill}{rgb}{0.993248,0.906157,0.143936}%
\pgfsetfillcolor{currentfill}%
\pgfsetfillopacity{0.700000}%
\pgfsetlinewidth{0.000000pt}%
\definecolor{currentstroke}{rgb}{0.000000,0.000000,0.000000}%
\pgfsetstrokecolor{currentstroke}%
\pgfsetstrokeopacity{0.700000}%
\pgfsetdash{}{0pt}%
\pgfpathmoveto{\pgfqpoint{8.416112in}{2.689800in}}%
\pgfpathcurveto{\pgfqpoint{8.421155in}{2.689800in}}{\pgfqpoint{8.425993in}{2.691804in}}{\pgfqpoint{8.429560in}{2.695370in}}%
\pgfpathcurveto{\pgfqpoint{8.433126in}{2.698937in}}{\pgfqpoint{8.435130in}{2.703775in}}{\pgfqpoint{8.435130in}{2.708818in}}%
\pgfpathcurveto{\pgfqpoint{8.435130in}{2.713862in}}{\pgfqpoint{8.433126in}{2.718700in}}{\pgfqpoint{8.429560in}{2.722266in}}%
\pgfpathcurveto{\pgfqpoint{8.425993in}{2.725833in}}{\pgfqpoint{8.421155in}{2.727836in}}{\pgfqpoint{8.416112in}{2.727836in}}%
\pgfpathcurveto{\pgfqpoint{8.411068in}{2.727836in}}{\pgfqpoint{8.406230in}{2.725833in}}{\pgfqpoint{8.402664in}{2.722266in}}%
\pgfpathcurveto{\pgfqpoint{8.399097in}{2.718700in}}{\pgfqpoint{8.397094in}{2.713862in}}{\pgfqpoint{8.397094in}{2.708818in}}%
\pgfpathcurveto{\pgfqpoint{8.397094in}{2.703775in}}{\pgfqpoint{8.399097in}{2.698937in}}{\pgfqpoint{8.402664in}{2.695370in}}%
\pgfpathcurveto{\pgfqpoint{8.406230in}{2.691804in}}{\pgfqpoint{8.411068in}{2.689800in}}{\pgfqpoint{8.416112in}{2.689800in}}%
\pgfpathclose%
\pgfusepath{fill}%
\end{pgfscope}%
\begin{pgfscope}%
\pgfpathrectangle{\pgfqpoint{6.572727in}{0.474100in}}{\pgfqpoint{4.227273in}{3.318700in}}%
\pgfusepath{clip}%
\pgfsetbuttcap%
\pgfsetroundjoin%
\definecolor{currentfill}{rgb}{0.127568,0.566949,0.550556}%
\pgfsetfillcolor{currentfill}%
\pgfsetfillopacity{0.700000}%
\pgfsetlinewidth{0.000000pt}%
\definecolor{currentstroke}{rgb}{0.000000,0.000000,0.000000}%
\pgfsetstrokecolor{currentstroke}%
\pgfsetstrokeopacity{0.700000}%
\pgfsetdash{}{0pt}%
\pgfpathmoveto{\pgfqpoint{9.629258in}{1.664976in}}%
\pgfpathcurveto{\pgfqpoint{9.634302in}{1.664976in}}{\pgfqpoint{9.639140in}{1.666980in}}{\pgfqpoint{9.642706in}{1.670546in}}%
\pgfpathcurveto{\pgfqpoint{9.646273in}{1.674113in}}{\pgfqpoint{9.648277in}{1.678951in}}{\pgfqpoint{9.648277in}{1.683994in}}%
\pgfpathcurveto{\pgfqpoint{9.648277in}{1.689038in}}{\pgfqpoint{9.646273in}{1.693876in}}{\pgfqpoint{9.642706in}{1.697442in}}%
\pgfpathcurveto{\pgfqpoint{9.639140in}{1.701009in}}{\pgfqpoint{9.634302in}{1.703012in}}{\pgfqpoint{9.629258in}{1.703012in}}%
\pgfpathcurveto{\pgfqpoint{9.624215in}{1.703012in}}{\pgfqpoint{9.619377in}{1.701009in}}{\pgfqpoint{9.615811in}{1.697442in}}%
\pgfpathcurveto{\pgfqpoint{9.612244in}{1.693876in}}{\pgfqpoint{9.610240in}{1.689038in}}{\pgfqpoint{9.610240in}{1.683994in}}%
\pgfpathcurveto{\pgfqpoint{9.610240in}{1.678951in}}{\pgfqpoint{9.612244in}{1.674113in}}{\pgfqpoint{9.615811in}{1.670546in}}%
\pgfpathcurveto{\pgfqpoint{9.619377in}{1.666980in}}{\pgfqpoint{9.624215in}{1.664976in}}{\pgfqpoint{9.629258in}{1.664976in}}%
\pgfpathclose%
\pgfusepath{fill}%
\end{pgfscope}%
\begin{pgfscope}%
\pgfpathrectangle{\pgfqpoint{6.572727in}{0.474100in}}{\pgfqpoint{4.227273in}{3.318700in}}%
\pgfusepath{clip}%
\pgfsetbuttcap%
\pgfsetroundjoin%
\definecolor{currentfill}{rgb}{0.127568,0.566949,0.550556}%
\pgfsetfillcolor{currentfill}%
\pgfsetfillopacity{0.700000}%
\pgfsetlinewidth{0.000000pt}%
\definecolor{currentstroke}{rgb}{0.000000,0.000000,0.000000}%
\pgfsetstrokecolor{currentstroke}%
\pgfsetstrokeopacity{0.700000}%
\pgfsetdash{}{0pt}%
\pgfpathmoveto{\pgfqpoint{8.962418in}{1.287431in}}%
\pgfpathcurveto{\pgfqpoint{8.967462in}{1.287431in}}{\pgfqpoint{8.972299in}{1.289435in}}{\pgfqpoint{8.975866in}{1.293001in}}%
\pgfpathcurveto{\pgfqpoint{8.979432in}{1.296568in}}{\pgfqpoint{8.981436in}{1.301405in}}{\pgfqpoint{8.981436in}{1.306449in}}%
\pgfpathcurveto{\pgfqpoint{8.981436in}{1.311493in}}{\pgfqpoint{8.979432in}{1.316331in}}{\pgfqpoint{8.975866in}{1.319897in}}%
\pgfpathcurveto{\pgfqpoint{8.972299in}{1.323463in}}{\pgfqpoint{8.967462in}{1.325467in}}{\pgfqpoint{8.962418in}{1.325467in}}%
\pgfpathcurveto{\pgfqpoint{8.957374in}{1.325467in}}{\pgfqpoint{8.952536in}{1.323463in}}{\pgfqpoint{8.948970in}{1.319897in}}%
\pgfpathcurveto{\pgfqpoint{8.945404in}{1.316331in}}{\pgfqpoint{8.943400in}{1.311493in}}{\pgfqpoint{8.943400in}{1.306449in}}%
\pgfpathcurveto{\pgfqpoint{8.943400in}{1.301405in}}{\pgfqpoint{8.945404in}{1.296568in}}{\pgfqpoint{8.948970in}{1.293001in}}%
\pgfpathcurveto{\pgfqpoint{8.952536in}{1.289435in}}{\pgfqpoint{8.957374in}{1.287431in}}{\pgfqpoint{8.962418in}{1.287431in}}%
\pgfpathclose%
\pgfusepath{fill}%
\end{pgfscope}%
\begin{pgfscope}%
\pgfpathrectangle{\pgfqpoint{6.572727in}{0.474100in}}{\pgfqpoint{4.227273in}{3.318700in}}%
\pgfusepath{clip}%
\pgfsetbuttcap%
\pgfsetroundjoin%
\definecolor{currentfill}{rgb}{0.127568,0.566949,0.550556}%
\pgfsetfillcolor{currentfill}%
\pgfsetfillopacity{0.700000}%
\pgfsetlinewidth{0.000000pt}%
\definecolor{currentstroke}{rgb}{0.000000,0.000000,0.000000}%
\pgfsetstrokecolor{currentstroke}%
\pgfsetstrokeopacity{0.700000}%
\pgfsetdash{}{0pt}%
\pgfpathmoveto{\pgfqpoint{9.790870in}{1.879159in}}%
\pgfpathcurveto{\pgfqpoint{9.795913in}{1.879159in}}{\pgfqpoint{9.800751in}{1.881163in}}{\pgfqpoint{9.804317in}{1.884729in}}%
\pgfpathcurveto{\pgfqpoint{9.807884in}{1.888296in}}{\pgfqpoint{9.809888in}{1.893134in}}{\pgfqpoint{9.809888in}{1.898177in}}%
\pgfpathcurveto{\pgfqpoint{9.809888in}{1.903221in}}{\pgfqpoint{9.807884in}{1.908059in}}{\pgfqpoint{9.804317in}{1.911625in}}%
\pgfpathcurveto{\pgfqpoint{9.800751in}{1.915192in}}{\pgfqpoint{9.795913in}{1.917195in}}{\pgfqpoint{9.790870in}{1.917195in}}%
\pgfpathcurveto{\pgfqpoint{9.785826in}{1.917195in}}{\pgfqpoint{9.780988in}{1.915192in}}{\pgfqpoint{9.777422in}{1.911625in}}%
\pgfpathcurveto{\pgfqpoint{9.773855in}{1.908059in}}{\pgfqpoint{9.771851in}{1.903221in}}{\pgfqpoint{9.771851in}{1.898177in}}%
\pgfpathcurveto{\pgfqpoint{9.771851in}{1.893134in}}{\pgfqpoint{9.773855in}{1.888296in}}{\pgfqpoint{9.777422in}{1.884729in}}%
\pgfpathcurveto{\pgfqpoint{9.780988in}{1.881163in}}{\pgfqpoint{9.785826in}{1.879159in}}{\pgfqpoint{9.790870in}{1.879159in}}%
\pgfpathclose%
\pgfusepath{fill}%
\end{pgfscope}%
\begin{pgfscope}%
\pgfpathrectangle{\pgfqpoint{6.572727in}{0.474100in}}{\pgfqpoint{4.227273in}{3.318700in}}%
\pgfusepath{clip}%
\pgfsetbuttcap%
\pgfsetroundjoin%
\definecolor{currentfill}{rgb}{0.267004,0.004874,0.329415}%
\pgfsetfillcolor{currentfill}%
\pgfsetfillopacity{0.700000}%
\pgfsetlinewidth{0.000000pt}%
\definecolor{currentstroke}{rgb}{0.000000,0.000000,0.000000}%
\pgfsetstrokecolor{currentstroke}%
\pgfsetstrokeopacity{0.700000}%
\pgfsetdash{}{0pt}%
\pgfpathmoveto{\pgfqpoint{7.402153in}{1.269046in}}%
\pgfpathcurveto{\pgfqpoint{7.407197in}{1.269046in}}{\pgfqpoint{7.412035in}{1.271050in}}{\pgfqpoint{7.415601in}{1.274617in}}%
\pgfpathcurveto{\pgfqpoint{7.419168in}{1.278183in}}{\pgfqpoint{7.421171in}{1.283021in}}{\pgfqpoint{7.421171in}{1.288064in}}%
\pgfpathcurveto{\pgfqpoint{7.421171in}{1.293108in}}{\pgfqpoint{7.419168in}{1.297946in}}{\pgfqpoint{7.415601in}{1.301512in}}%
\pgfpathcurveto{\pgfqpoint{7.412035in}{1.305079in}}{\pgfqpoint{7.407197in}{1.307083in}}{\pgfqpoint{7.402153in}{1.307083in}}%
\pgfpathcurveto{\pgfqpoint{7.397110in}{1.307083in}}{\pgfqpoint{7.392272in}{1.305079in}}{\pgfqpoint{7.388705in}{1.301512in}}%
\pgfpathcurveto{\pgfqpoint{7.385139in}{1.297946in}}{\pgfqpoint{7.383135in}{1.293108in}}{\pgfqpoint{7.383135in}{1.288064in}}%
\pgfpathcurveto{\pgfqpoint{7.383135in}{1.283021in}}{\pgfqpoint{7.385139in}{1.278183in}}{\pgfqpoint{7.388705in}{1.274617in}}%
\pgfpathcurveto{\pgfqpoint{7.392272in}{1.271050in}}{\pgfqpoint{7.397110in}{1.269046in}}{\pgfqpoint{7.402153in}{1.269046in}}%
\pgfpathclose%
\pgfusepath{fill}%
\end{pgfscope}%
\begin{pgfscope}%
\pgfpathrectangle{\pgfqpoint{6.572727in}{0.474100in}}{\pgfqpoint{4.227273in}{3.318700in}}%
\pgfusepath{clip}%
\pgfsetbuttcap%
\pgfsetroundjoin%
\definecolor{currentfill}{rgb}{0.127568,0.566949,0.550556}%
\pgfsetfillcolor{currentfill}%
\pgfsetfillopacity{0.700000}%
\pgfsetlinewidth{0.000000pt}%
\definecolor{currentstroke}{rgb}{0.000000,0.000000,0.000000}%
\pgfsetstrokecolor{currentstroke}%
\pgfsetstrokeopacity{0.700000}%
\pgfsetdash{}{0pt}%
\pgfpathmoveto{\pgfqpoint{9.794170in}{2.131494in}}%
\pgfpathcurveto{\pgfqpoint{9.799213in}{2.131494in}}{\pgfqpoint{9.804051in}{2.133498in}}{\pgfqpoint{9.807617in}{2.137065in}}%
\pgfpathcurveto{\pgfqpoint{9.811184in}{2.140631in}}{\pgfqpoint{9.813188in}{2.145469in}}{\pgfqpoint{9.813188in}{2.150512in}}%
\pgfpathcurveto{\pgfqpoint{9.813188in}{2.155556in}}{\pgfqpoint{9.811184in}{2.160394in}}{\pgfqpoint{9.807617in}{2.163960in}}%
\pgfpathcurveto{\pgfqpoint{9.804051in}{2.167527in}}{\pgfqpoint{9.799213in}{2.169531in}}{\pgfqpoint{9.794170in}{2.169531in}}%
\pgfpathcurveto{\pgfqpoint{9.789126in}{2.169531in}}{\pgfqpoint{9.784288in}{2.167527in}}{\pgfqpoint{9.780722in}{2.163960in}}%
\pgfpathcurveto{\pgfqpoint{9.777155in}{2.160394in}}{\pgfqpoint{9.775151in}{2.155556in}}{\pgfqpoint{9.775151in}{2.150512in}}%
\pgfpathcurveto{\pgfqpoint{9.775151in}{2.145469in}}{\pgfqpoint{9.777155in}{2.140631in}}{\pgfqpoint{9.780722in}{2.137065in}}%
\pgfpathcurveto{\pgfqpoint{9.784288in}{2.133498in}}{\pgfqpoint{9.789126in}{2.131494in}}{\pgfqpoint{9.794170in}{2.131494in}}%
\pgfpathclose%
\pgfusepath{fill}%
\end{pgfscope}%
\begin{pgfscope}%
\pgfpathrectangle{\pgfqpoint{6.572727in}{0.474100in}}{\pgfqpoint{4.227273in}{3.318700in}}%
\pgfusepath{clip}%
\pgfsetbuttcap%
\pgfsetroundjoin%
\definecolor{currentfill}{rgb}{0.267004,0.004874,0.329415}%
\pgfsetfillcolor{currentfill}%
\pgfsetfillopacity{0.700000}%
\pgfsetlinewidth{0.000000pt}%
\definecolor{currentstroke}{rgb}{0.000000,0.000000,0.000000}%
\pgfsetstrokecolor{currentstroke}%
\pgfsetstrokeopacity{0.700000}%
\pgfsetdash{}{0pt}%
\pgfpathmoveto{\pgfqpoint{7.970063in}{1.267306in}}%
\pgfpathcurveto{\pgfqpoint{7.975106in}{1.267306in}}{\pgfqpoint{7.979944in}{1.269310in}}{\pgfqpoint{7.983511in}{1.272877in}}%
\pgfpathcurveto{\pgfqpoint{7.987077in}{1.276443in}}{\pgfqpoint{7.989081in}{1.281281in}}{\pgfqpoint{7.989081in}{1.286324in}}%
\pgfpathcurveto{\pgfqpoint{7.989081in}{1.291368in}}{\pgfqpoint{7.987077in}{1.296206in}}{\pgfqpoint{7.983511in}{1.299772in}}%
\pgfpathcurveto{\pgfqpoint{7.979944in}{1.303339in}}{\pgfqpoint{7.975106in}{1.305343in}}{\pgfqpoint{7.970063in}{1.305343in}}%
\pgfpathcurveto{\pgfqpoint{7.965019in}{1.305343in}}{\pgfqpoint{7.960181in}{1.303339in}}{\pgfqpoint{7.956615in}{1.299772in}}%
\pgfpathcurveto{\pgfqpoint{7.953048in}{1.296206in}}{\pgfqpoint{7.951045in}{1.291368in}}{\pgfqpoint{7.951045in}{1.286324in}}%
\pgfpathcurveto{\pgfqpoint{7.951045in}{1.281281in}}{\pgfqpoint{7.953048in}{1.276443in}}{\pgfqpoint{7.956615in}{1.272877in}}%
\pgfpathcurveto{\pgfqpoint{7.960181in}{1.269310in}}{\pgfqpoint{7.965019in}{1.267306in}}{\pgfqpoint{7.970063in}{1.267306in}}%
\pgfpathclose%
\pgfusepath{fill}%
\end{pgfscope}%
\begin{pgfscope}%
\pgfpathrectangle{\pgfqpoint{6.572727in}{0.474100in}}{\pgfqpoint{4.227273in}{3.318700in}}%
\pgfusepath{clip}%
\pgfsetbuttcap%
\pgfsetroundjoin%
\definecolor{currentfill}{rgb}{0.267004,0.004874,0.329415}%
\pgfsetfillcolor{currentfill}%
\pgfsetfillopacity{0.700000}%
\pgfsetlinewidth{0.000000pt}%
\definecolor{currentstroke}{rgb}{0.000000,0.000000,0.000000}%
\pgfsetstrokecolor{currentstroke}%
\pgfsetstrokeopacity{0.700000}%
\pgfsetdash{}{0pt}%
\pgfpathmoveto{\pgfqpoint{7.979843in}{1.638995in}}%
\pgfpathcurveto{\pgfqpoint{7.984887in}{1.638995in}}{\pgfqpoint{7.989724in}{1.640999in}}{\pgfqpoint{7.993291in}{1.644565in}}%
\pgfpathcurveto{\pgfqpoint{7.996857in}{1.648132in}}{\pgfqpoint{7.998861in}{1.652970in}}{\pgfqpoint{7.998861in}{1.658013in}}%
\pgfpathcurveto{\pgfqpoint{7.998861in}{1.663057in}}{\pgfqpoint{7.996857in}{1.667895in}}{\pgfqpoint{7.993291in}{1.671461in}}%
\pgfpathcurveto{\pgfqpoint{7.989724in}{1.675028in}}{\pgfqpoint{7.984887in}{1.677031in}}{\pgfqpoint{7.979843in}{1.677031in}}%
\pgfpathcurveto{\pgfqpoint{7.974799in}{1.677031in}}{\pgfqpoint{7.969962in}{1.675028in}}{\pgfqpoint{7.966395in}{1.671461in}}%
\pgfpathcurveto{\pgfqpoint{7.962829in}{1.667895in}}{\pgfqpoint{7.960825in}{1.663057in}}{\pgfqpoint{7.960825in}{1.658013in}}%
\pgfpathcurveto{\pgfqpoint{7.960825in}{1.652970in}}{\pgfqpoint{7.962829in}{1.648132in}}{\pgfqpoint{7.966395in}{1.644565in}}%
\pgfpathcurveto{\pgfqpoint{7.969962in}{1.640999in}}{\pgfqpoint{7.974799in}{1.638995in}}{\pgfqpoint{7.979843in}{1.638995in}}%
\pgfpathclose%
\pgfusepath{fill}%
\end{pgfscope}%
\begin{pgfscope}%
\pgfpathrectangle{\pgfqpoint{6.572727in}{0.474100in}}{\pgfqpoint{4.227273in}{3.318700in}}%
\pgfusepath{clip}%
\pgfsetbuttcap%
\pgfsetroundjoin%
\definecolor{currentfill}{rgb}{0.993248,0.906157,0.143936}%
\pgfsetfillcolor{currentfill}%
\pgfsetfillopacity{0.700000}%
\pgfsetlinewidth{0.000000pt}%
\definecolor{currentstroke}{rgb}{0.000000,0.000000,0.000000}%
\pgfsetstrokecolor{currentstroke}%
\pgfsetstrokeopacity{0.700000}%
\pgfsetdash{}{0pt}%
\pgfpathmoveto{\pgfqpoint{7.864756in}{3.042516in}}%
\pgfpathcurveto{\pgfqpoint{7.869800in}{3.042516in}}{\pgfqpoint{7.874638in}{3.044520in}}{\pgfqpoint{7.878204in}{3.048087in}}%
\pgfpathcurveto{\pgfqpoint{7.881771in}{3.051653in}}{\pgfqpoint{7.883775in}{3.056491in}}{\pgfqpoint{7.883775in}{3.061535in}}%
\pgfpathcurveto{\pgfqpoint{7.883775in}{3.066578in}}{\pgfqpoint{7.881771in}{3.071416in}}{\pgfqpoint{7.878204in}{3.074982in}}%
\pgfpathcurveto{\pgfqpoint{7.874638in}{3.078549in}}{\pgfqpoint{7.869800in}{3.080553in}}{\pgfqpoint{7.864756in}{3.080553in}}%
\pgfpathcurveto{\pgfqpoint{7.859713in}{3.080553in}}{\pgfqpoint{7.854875in}{3.078549in}}{\pgfqpoint{7.851309in}{3.074982in}}%
\pgfpathcurveto{\pgfqpoint{7.847742in}{3.071416in}}{\pgfqpoint{7.845738in}{3.066578in}}{\pgfqpoint{7.845738in}{3.061535in}}%
\pgfpathcurveto{\pgfqpoint{7.845738in}{3.056491in}}{\pgfqpoint{7.847742in}{3.051653in}}{\pgfqpoint{7.851309in}{3.048087in}}%
\pgfpathcurveto{\pgfqpoint{7.854875in}{3.044520in}}{\pgfqpoint{7.859713in}{3.042516in}}{\pgfqpoint{7.864756in}{3.042516in}}%
\pgfpathclose%
\pgfusepath{fill}%
\end{pgfscope}%
\begin{pgfscope}%
\pgfpathrectangle{\pgfqpoint{6.572727in}{0.474100in}}{\pgfqpoint{4.227273in}{3.318700in}}%
\pgfusepath{clip}%
\pgfsetbuttcap%
\pgfsetroundjoin%
\definecolor{currentfill}{rgb}{0.127568,0.566949,0.550556}%
\pgfsetfillcolor{currentfill}%
\pgfsetfillopacity{0.700000}%
\pgfsetlinewidth{0.000000pt}%
\definecolor{currentstroke}{rgb}{0.000000,0.000000,0.000000}%
\pgfsetstrokecolor{currentstroke}%
\pgfsetstrokeopacity{0.700000}%
\pgfsetdash{}{0pt}%
\pgfpathmoveto{\pgfqpoint{9.403020in}{1.046733in}}%
\pgfpathcurveto{\pgfqpoint{9.408063in}{1.046733in}}{\pgfqpoint{9.412901in}{1.048737in}}{\pgfqpoint{9.416467in}{1.052303in}}%
\pgfpathcurveto{\pgfqpoint{9.420034in}{1.055870in}}{\pgfqpoint{9.422038in}{1.060708in}}{\pgfqpoint{9.422038in}{1.065751in}}%
\pgfpathcurveto{\pgfqpoint{9.422038in}{1.070795in}}{\pgfqpoint{9.420034in}{1.075633in}}{\pgfqpoint{9.416467in}{1.079199in}}%
\pgfpathcurveto{\pgfqpoint{9.412901in}{1.082766in}}{\pgfqpoint{9.408063in}{1.084769in}}{\pgfqpoint{9.403020in}{1.084769in}}%
\pgfpathcurveto{\pgfqpoint{9.397976in}{1.084769in}}{\pgfqpoint{9.393138in}{1.082766in}}{\pgfqpoint{9.389572in}{1.079199in}}%
\pgfpathcurveto{\pgfqpoint{9.386005in}{1.075633in}}{\pgfqpoint{9.384001in}{1.070795in}}{\pgfqpoint{9.384001in}{1.065751in}}%
\pgfpathcurveto{\pgfqpoint{9.384001in}{1.060708in}}{\pgfqpoint{9.386005in}{1.055870in}}{\pgfqpoint{9.389572in}{1.052303in}}%
\pgfpathcurveto{\pgfqpoint{9.393138in}{1.048737in}}{\pgfqpoint{9.397976in}{1.046733in}}{\pgfqpoint{9.403020in}{1.046733in}}%
\pgfpathclose%
\pgfusepath{fill}%
\end{pgfscope}%
\begin{pgfscope}%
\pgfpathrectangle{\pgfqpoint{6.572727in}{0.474100in}}{\pgfqpoint{4.227273in}{3.318700in}}%
\pgfusepath{clip}%
\pgfsetbuttcap%
\pgfsetroundjoin%
\definecolor{currentfill}{rgb}{0.267004,0.004874,0.329415}%
\pgfsetfillcolor{currentfill}%
\pgfsetfillopacity{0.700000}%
\pgfsetlinewidth{0.000000pt}%
\definecolor{currentstroke}{rgb}{0.000000,0.000000,0.000000}%
\pgfsetstrokecolor{currentstroke}%
\pgfsetstrokeopacity{0.700000}%
\pgfsetdash{}{0pt}%
\pgfpathmoveto{\pgfqpoint{7.918575in}{2.149727in}}%
\pgfpathcurveto{\pgfqpoint{7.923619in}{2.149727in}}{\pgfqpoint{7.928456in}{2.151731in}}{\pgfqpoint{7.932023in}{2.155297in}}%
\pgfpathcurveto{\pgfqpoint{7.935589in}{2.158864in}}{\pgfqpoint{7.937593in}{2.163702in}}{\pgfqpoint{7.937593in}{2.168745in}}%
\pgfpathcurveto{\pgfqpoint{7.937593in}{2.173789in}}{\pgfqpoint{7.935589in}{2.178627in}}{\pgfqpoint{7.932023in}{2.182193in}}%
\pgfpathcurveto{\pgfqpoint{7.928456in}{2.185759in}}{\pgfqpoint{7.923619in}{2.187763in}}{\pgfqpoint{7.918575in}{2.187763in}}%
\pgfpathcurveto{\pgfqpoint{7.913531in}{2.187763in}}{\pgfqpoint{7.908694in}{2.185759in}}{\pgfqpoint{7.905127in}{2.182193in}}%
\pgfpathcurveto{\pgfqpoint{7.901561in}{2.178627in}}{\pgfqpoint{7.899557in}{2.173789in}}{\pgfqpoint{7.899557in}{2.168745in}}%
\pgfpathcurveto{\pgfqpoint{7.899557in}{2.163702in}}{\pgfqpoint{7.901561in}{2.158864in}}{\pgfqpoint{7.905127in}{2.155297in}}%
\pgfpathcurveto{\pgfqpoint{7.908694in}{2.151731in}}{\pgfqpoint{7.913531in}{2.149727in}}{\pgfqpoint{7.918575in}{2.149727in}}%
\pgfpathclose%
\pgfusepath{fill}%
\end{pgfscope}%
\begin{pgfscope}%
\pgfpathrectangle{\pgfqpoint{6.572727in}{0.474100in}}{\pgfqpoint{4.227273in}{3.318700in}}%
\pgfusepath{clip}%
\pgfsetbuttcap%
\pgfsetroundjoin%
\definecolor{currentfill}{rgb}{0.267004,0.004874,0.329415}%
\pgfsetfillcolor{currentfill}%
\pgfsetfillopacity{0.700000}%
\pgfsetlinewidth{0.000000pt}%
\definecolor{currentstroke}{rgb}{0.000000,0.000000,0.000000}%
\pgfsetstrokecolor{currentstroke}%
\pgfsetstrokeopacity{0.700000}%
\pgfsetdash{}{0pt}%
\pgfpathmoveto{\pgfqpoint{8.004817in}{1.225383in}}%
\pgfpathcurveto{\pgfqpoint{8.009861in}{1.225383in}}{\pgfqpoint{8.014699in}{1.227387in}}{\pgfqpoint{8.018265in}{1.230953in}}%
\pgfpathcurveto{\pgfqpoint{8.021832in}{1.234520in}}{\pgfqpoint{8.023835in}{1.239357in}}{\pgfqpoint{8.023835in}{1.244401in}}%
\pgfpathcurveto{\pgfqpoint{8.023835in}{1.249445in}}{\pgfqpoint{8.021832in}{1.254283in}}{\pgfqpoint{8.018265in}{1.257849in}}%
\pgfpathcurveto{\pgfqpoint{8.014699in}{1.261415in}}{\pgfqpoint{8.009861in}{1.263419in}}{\pgfqpoint{8.004817in}{1.263419in}}%
\pgfpathcurveto{\pgfqpoint{7.999774in}{1.263419in}}{\pgfqpoint{7.994936in}{1.261415in}}{\pgfqpoint{7.991369in}{1.257849in}}%
\pgfpathcurveto{\pgfqpoint{7.987803in}{1.254283in}}{\pgfqpoint{7.985799in}{1.249445in}}{\pgfqpoint{7.985799in}{1.244401in}}%
\pgfpathcurveto{\pgfqpoint{7.985799in}{1.239357in}}{\pgfqpoint{7.987803in}{1.234520in}}{\pgfqpoint{7.991369in}{1.230953in}}%
\pgfpathcurveto{\pgfqpoint{7.994936in}{1.227387in}}{\pgfqpoint{7.999774in}{1.225383in}}{\pgfqpoint{8.004817in}{1.225383in}}%
\pgfpathclose%
\pgfusepath{fill}%
\end{pgfscope}%
\begin{pgfscope}%
\pgfpathrectangle{\pgfqpoint{6.572727in}{0.474100in}}{\pgfqpoint{4.227273in}{3.318700in}}%
\pgfusepath{clip}%
\pgfsetbuttcap%
\pgfsetroundjoin%
\definecolor{currentfill}{rgb}{0.267004,0.004874,0.329415}%
\pgfsetfillcolor{currentfill}%
\pgfsetfillopacity{0.700000}%
\pgfsetlinewidth{0.000000pt}%
\definecolor{currentstroke}{rgb}{0.000000,0.000000,0.000000}%
\pgfsetstrokecolor{currentstroke}%
\pgfsetstrokeopacity{0.700000}%
\pgfsetdash{}{0pt}%
\pgfpathmoveto{\pgfqpoint{8.196048in}{1.382449in}}%
\pgfpathcurveto{\pgfqpoint{8.201092in}{1.382449in}}{\pgfqpoint{8.205930in}{1.384453in}}{\pgfqpoint{8.209496in}{1.388019in}}%
\pgfpathcurveto{\pgfqpoint{8.213063in}{1.391585in}}{\pgfqpoint{8.215066in}{1.396423in}}{\pgfqpoint{8.215066in}{1.401467in}}%
\pgfpathcurveto{\pgfqpoint{8.215066in}{1.406511in}}{\pgfqpoint{8.213063in}{1.411348in}}{\pgfqpoint{8.209496in}{1.414915in}}%
\pgfpathcurveto{\pgfqpoint{8.205930in}{1.418481in}}{\pgfqpoint{8.201092in}{1.420485in}}{\pgfqpoint{8.196048in}{1.420485in}}%
\pgfpathcurveto{\pgfqpoint{8.191005in}{1.420485in}}{\pgfqpoint{8.186167in}{1.418481in}}{\pgfqpoint{8.182600in}{1.414915in}}%
\pgfpathcurveto{\pgfqpoint{8.179034in}{1.411348in}}{\pgfqpoint{8.177030in}{1.406511in}}{\pgfqpoint{8.177030in}{1.401467in}}%
\pgfpathcurveto{\pgfqpoint{8.177030in}{1.396423in}}{\pgfqpoint{8.179034in}{1.391585in}}{\pgfqpoint{8.182600in}{1.388019in}}%
\pgfpathcurveto{\pgfqpoint{8.186167in}{1.384453in}}{\pgfqpoint{8.191005in}{1.382449in}}{\pgfqpoint{8.196048in}{1.382449in}}%
\pgfpathclose%
\pgfusepath{fill}%
\end{pgfscope}%
\begin{pgfscope}%
\pgfpathrectangle{\pgfqpoint{6.572727in}{0.474100in}}{\pgfqpoint{4.227273in}{3.318700in}}%
\pgfusepath{clip}%
\pgfsetbuttcap%
\pgfsetroundjoin%
\definecolor{currentfill}{rgb}{0.993248,0.906157,0.143936}%
\pgfsetfillcolor{currentfill}%
\pgfsetfillopacity{0.700000}%
\pgfsetlinewidth{0.000000pt}%
\definecolor{currentstroke}{rgb}{0.000000,0.000000,0.000000}%
\pgfsetstrokecolor{currentstroke}%
\pgfsetstrokeopacity{0.700000}%
\pgfsetdash{}{0pt}%
\pgfpathmoveto{\pgfqpoint{8.345686in}{2.515201in}}%
\pgfpathcurveto{\pgfqpoint{8.350730in}{2.515201in}}{\pgfqpoint{8.355568in}{2.517205in}}{\pgfqpoint{8.359134in}{2.520771in}}%
\pgfpathcurveto{\pgfqpoint{8.362700in}{2.524338in}}{\pgfqpoint{8.364704in}{2.529175in}}{\pgfqpoint{8.364704in}{2.534219in}}%
\pgfpathcurveto{\pgfqpoint{8.364704in}{2.539263in}}{\pgfqpoint{8.362700in}{2.544101in}}{\pgfqpoint{8.359134in}{2.547667in}}%
\pgfpathcurveto{\pgfqpoint{8.355568in}{2.551233in}}{\pgfqpoint{8.350730in}{2.553237in}}{\pgfqpoint{8.345686in}{2.553237in}}%
\pgfpathcurveto{\pgfqpoint{8.340642in}{2.553237in}}{\pgfqpoint{8.335805in}{2.551233in}}{\pgfqpoint{8.332238in}{2.547667in}}%
\pgfpathcurveto{\pgfqpoint{8.328672in}{2.544101in}}{\pgfqpoint{8.326668in}{2.539263in}}{\pgfqpoint{8.326668in}{2.534219in}}%
\pgfpathcurveto{\pgfqpoint{8.326668in}{2.529175in}}{\pgfqpoint{8.328672in}{2.524338in}}{\pgfqpoint{8.332238in}{2.520771in}}%
\pgfpathcurveto{\pgfqpoint{8.335805in}{2.517205in}}{\pgfqpoint{8.340642in}{2.515201in}}{\pgfqpoint{8.345686in}{2.515201in}}%
\pgfpathclose%
\pgfusepath{fill}%
\end{pgfscope}%
\begin{pgfscope}%
\pgfpathrectangle{\pgfqpoint{6.572727in}{0.474100in}}{\pgfqpoint{4.227273in}{3.318700in}}%
\pgfusepath{clip}%
\pgfsetbuttcap%
\pgfsetroundjoin%
\definecolor{currentfill}{rgb}{0.127568,0.566949,0.550556}%
\pgfsetfillcolor{currentfill}%
\pgfsetfillopacity{0.700000}%
\pgfsetlinewidth{0.000000pt}%
\definecolor{currentstroke}{rgb}{0.000000,0.000000,0.000000}%
\pgfsetstrokecolor{currentstroke}%
\pgfsetstrokeopacity{0.700000}%
\pgfsetdash{}{0pt}%
\pgfpathmoveto{\pgfqpoint{9.600133in}{1.406614in}}%
\pgfpathcurveto{\pgfqpoint{9.605177in}{1.406614in}}{\pgfqpoint{9.610015in}{1.408618in}}{\pgfqpoint{9.613581in}{1.412185in}}%
\pgfpathcurveto{\pgfqpoint{9.617147in}{1.415751in}}{\pgfqpoint{9.619151in}{1.420589in}}{\pgfqpoint{9.619151in}{1.425632in}}%
\pgfpathcurveto{\pgfqpoint{9.619151in}{1.430676in}}{\pgfqpoint{9.617147in}{1.435514in}}{\pgfqpoint{9.613581in}{1.439080in}}%
\pgfpathcurveto{\pgfqpoint{9.610015in}{1.442647in}}{\pgfqpoint{9.605177in}{1.444651in}}{\pgfqpoint{9.600133in}{1.444651in}}%
\pgfpathcurveto{\pgfqpoint{9.595089in}{1.444651in}}{\pgfqpoint{9.590252in}{1.442647in}}{\pgfqpoint{9.586685in}{1.439080in}}%
\pgfpathcurveto{\pgfqpoint{9.583119in}{1.435514in}}{\pgfqpoint{9.581115in}{1.430676in}}{\pgfqpoint{9.581115in}{1.425632in}}%
\pgfpathcurveto{\pgfqpoint{9.581115in}{1.420589in}}{\pgfqpoint{9.583119in}{1.415751in}}{\pgfqpoint{9.586685in}{1.412185in}}%
\pgfpathcurveto{\pgfqpoint{9.590252in}{1.408618in}}{\pgfqpoint{9.595089in}{1.406614in}}{\pgfqpoint{9.600133in}{1.406614in}}%
\pgfpathclose%
\pgfusepath{fill}%
\end{pgfscope}%
\begin{pgfscope}%
\pgfpathrectangle{\pgfqpoint{6.572727in}{0.474100in}}{\pgfqpoint{4.227273in}{3.318700in}}%
\pgfusepath{clip}%
\pgfsetbuttcap%
\pgfsetroundjoin%
\definecolor{currentfill}{rgb}{0.267004,0.004874,0.329415}%
\pgfsetfillcolor{currentfill}%
\pgfsetfillopacity{0.700000}%
\pgfsetlinewidth{0.000000pt}%
\definecolor{currentstroke}{rgb}{0.000000,0.000000,0.000000}%
\pgfsetstrokecolor{currentstroke}%
\pgfsetstrokeopacity{0.700000}%
\pgfsetdash{}{0pt}%
\pgfpathmoveto{\pgfqpoint{8.391248in}{1.036682in}}%
\pgfpathcurveto{\pgfqpoint{8.396292in}{1.036682in}}{\pgfqpoint{8.401129in}{1.038685in}}{\pgfqpoint{8.404696in}{1.042252in}}%
\pgfpathcurveto{\pgfqpoint{8.408262in}{1.045818in}}{\pgfqpoint{8.410266in}{1.050656in}}{\pgfqpoint{8.410266in}{1.055700in}}%
\pgfpathcurveto{\pgfqpoint{8.410266in}{1.060743in}}{\pgfqpoint{8.408262in}{1.065581in}}{\pgfqpoint{8.404696in}{1.069148in}}%
\pgfpathcurveto{\pgfqpoint{8.401129in}{1.072714in}}{\pgfqpoint{8.396292in}{1.074718in}}{\pgfqpoint{8.391248in}{1.074718in}}%
\pgfpathcurveto{\pgfqpoint{8.386204in}{1.074718in}}{\pgfqpoint{8.381366in}{1.072714in}}{\pgfqpoint{8.377800in}{1.069148in}}%
\pgfpathcurveto{\pgfqpoint{8.374234in}{1.065581in}}{\pgfqpoint{8.372230in}{1.060743in}}{\pgfqpoint{8.372230in}{1.055700in}}%
\pgfpathcurveto{\pgfqpoint{8.372230in}{1.050656in}}{\pgfqpoint{8.374234in}{1.045818in}}{\pgfqpoint{8.377800in}{1.042252in}}%
\pgfpathcurveto{\pgfqpoint{8.381366in}{1.038685in}}{\pgfqpoint{8.386204in}{1.036682in}}{\pgfqpoint{8.391248in}{1.036682in}}%
\pgfpathclose%
\pgfusepath{fill}%
\end{pgfscope}%
\begin{pgfscope}%
\pgfpathrectangle{\pgfqpoint{6.572727in}{0.474100in}}{\pgfqpoint{4.227273in}{3.318700in}}%
\pgfusepath{clip}%
\pgfsetbuttcap%
\pgfsetroundjoin%
\definecolor{currentfill}{rgb}{0.127568,0.566949,0.550556}%
\pgfsetfillcolor{currentfill}%
\pgfsetfillopacity{0.700000}%
\pgfsetlinewidth{0.000000pt}%
\definecolor{currentstroke}{rgb}{0.000000,0.000000,0.000000}%
\pgfsetstrokecolor{currentstroke}%
\pgfsetstrokeopacity{0.700000}%
\pgfsetdash{}{0pt}%
\pgfpathmoveto{\pgfqpoint{9.446347in}{1.380990in}}%
\pgfpathcurveto{\pgfqpoint{9.451391in}{1.380990in}}{\pgfqpoint{9.456229in}{1.382994in}}{\pgfqpoint{9.459795in}{1.386560in}}%
\pgfpathcurveto{\pgfqpoint{9.463362in}{1.390127in}}{\pgfqpoint{9.465366in}{1.394964in}}{\pgfqpoint{9.465366in}{1.400008in}}%
\pgfpathcurveto{\pgfqpoint{9.465366in}{1.405052in}}{\pgfqpoint{9.463362in}{1.409889in}}{\pgfqpoint{9.459795in}{1.413456in}}%
\pgfpathcurveto{\pgfqpoint{9.456229in}{1.417022in}}{\pgfqpoint{9.451391in}{1.419026in}}{\pgfqpoint{9.446347in}{1.419026in}}%
\pgfpathcurveto{\pgfqpoint{9.441304in}{1.419026in}}{\pgfqpoint{9.436466in}{1.417022in}}{\pgfqpoint{9.432900in}{1.413456in}}%
\pgfpathcurveto{\pgfqpoint{9.429333in}{1.409889in}}{\pgfqpoint{9.427329in}{1.405052in}}{\pgfqpoint{9.427329in}{1.400008in}}%
\pgfpathcurveto{\pgfqpoint{9.427329in}{1.394964in}}{\pgfqpoint{9.429333in}{1.390127in}}{\pgfqpoint{9.432900in}{1.386560in}}%
\pgfpathcurveto{\pgfqpoint{9.436466in}{1.382994in}}{\pgfqpoint{9.441304in}{1.380990in}}{\pgfqpoint{9.446347in}{1.380990in}}%
\pgfpathclose%
\pgfusepath{fill}%
\end{pgfscope}%
\begin{pgfscope}%
\pgfpathrectangle{\pgfqpoint{6.572727in}{0.474100in}}{\pgfqpoint{4.227273in}{3.318700in}}%
\pgfusepath{clip}%
\pgfsetbuttcap%
\pgfsetroundjoin%
\definecolor{currentfill}{rgb}{0.993248,0.906157,0.143936}%
\pgfsetfillcolor{currentfill}%
\pgfsetfillopacity{0.700000}%
\pgfsetlinewidth{0.000000pt}%
\definecolor{currentstroke}{rgb}{0.000000,0.000000,0.000000}%
\pgfsetstrokecolor{currentstroke}%
\pgfsetstrokeopacity{0.700000}%
\pgfsetdash{}{0pt}%
\pgfpathmoveto{\pgfqpoint{8.137568in}{2.558601in}}%
\pgfpathcurveto{\pgfqpoint{8.142612in}{2.558601in}}{\pgfqpoint{8.147450in}{2.560605in}}{\pgfqpoint{8.151016in}{2.564171in}}%
\pgfpathcurveto{\pgfqpoint{8.154582in}{2.567738in}}{\pgfqpoint{8.156586in}{2.572575in}}{\pgfqpoint{8.156586in}{2.577619in}}%
\pgfpathcurveto{\pgfqpoint{8.156586in}{2.582663in}}{\pgfqpoint{8.154582in}{2.587500in}}{\pgfqpoint{8.151016in}{2.591067in}}%
\pgfpathcurveto{\pgfqpoint{8.147450in}{2.594633in}}{\pgfqpoint{8.142612in}{2.596637in}}{\pgfqpoint{8.137568in}{2.596637in}}%
\pgfpathcurveto{\pgfqpoint{8.132524in}{2.596637in}}{\pgfqpoint{8.127687in}{2.594633in}}{\pgfqpoint{8.124120in}{2.591067in}}%
\pgfpathcurveto{\pgfqpoint{8.120554in}{2.587500in}}{\pgfqpoint{8.118550in}{2.582663in}}{\pgfqpoint{8.118550in}{2.577619in}}%
\pgfpathcurveto{\pgfqpoint{8.118550in}{2.572575in}}{\pgfqpoint{8.120554in}{2.567738in}}{\pgfqpoint{8.124120in}{2.564171in}}%
\pgfpathcurveto{\pgfqpoint{8.127687in}{2.560605in}}{\pgfqpoint{8.132524in}{2.558601in}}{\pgfqpoint{8.137568in}{2.558601in}}%
\pgfpathclose%
\pgfusepath{fill}%
\end{pgfscope}%
\begin{pgfscope}%
\pgfpathrectangle{\pgfqpoint{6.572727in}{0.474100in}}{\pgfqpoint{4.227273in}{3.318700in}}%
\pgfusepath{clip}%
\pgfsetbuttcap%
\pgfsetroundjoin%
\definecolor{currentfill}{rgb}{0.993248,0.906157,0.143936}%
\pgfsetfillcolor{currentfill}%
\pgfsetfillopacity{0.700000}%
\pgfsetlinewidth{0.000000pt}%
\definecolor{currentstroke}{rgb}{0.000000,0.000000,0.000000}%
\pgfsetstrokecolor{currentstroke}%
\pgfsetstrokeopacity{0.700000}%
\pgfsetdash{}{0pt}%
\pgfpathmoveto{\pgfqpoint{8.005069in}{2.682855in}}%
\pgfpathcurveto{\pgfqpoint{8.010113in}{2.682855in}}{\pgfqpoint{8.014950in}{2.684859in}}{\pgfqpoint{8.018517in}{2.688425in}}%
\pgfpathcurveto{\pgfqpoint{8.022083in}{2.691992in}}{\pgfqpoint{8.024087in}{2.696829in}}{\pgfqpoint{8.024087in}{2.701873in}}%
\pgfpathcurveto{\pgfqpoint{8.024087in}{2.706917in}}{\pgfqpoint{8.022083in}{2.711755in}}{\pgfqpoint{8.018517in}{2.715321in}}%
\pgfpathcurveto{\pgfqpoint{8.014950in}{2.718887in}}{\pgfqpoint{8.010113in}{2.720891in}}{\pgfqpoint{8.005069in}{2.720891in}}%
\pgfpathcurveto{\pgfqpoint{8.000025in}{2.720891in}}{\pgfqpoint{7.995188in}{2.718887in}}{\pgfqpoint{7.991621in}{2.715321in}}%
\pgfpathcurveto{\pgfqpoint{7.988055in}{2.711755in}}{\pgfqpoint{7.986051in}{2.706917in}}{\pgfqpoint{7.986051in}{2.701873in}}%
\pgfpathcurveto{\pgfqpoint{7.986051in}{2.696829in}}{\pgfqpoint{7.988055in}{2.691992in}}{\pgfqpoint{7.991621in}{2.688425in}}%
\pgfpathcurveto{\pgfqpoint{7.995188in}{2.684859in}}{\pgfqpoint{8.000025in}{2.682855in}}{\pgfqpoint{8.005069in}{2.682855in}}%
\pgfpathclose%
\pgfusepath{fill}%
\end{pgfscope}%
\begin{pgfscope}%
\pgfpathrectangle{\pgfqpoint{6.572727in}{0.474100in}}{\pgfqpoint{4.227273in}{3.318700in}}%
\pgfusepath{clip}%
\pgfsetbuttcap%
\pgfsetroundjoin%
\definecolor{currentfill}{rgb}{0.267004,0.004874,0.329415}%
\pgfsetfillcolor{currentfill}%
\pgfsetfillopacity{0.700000}%
\pgfsetlinewidth{0.000000pt}%
\definecolor{currentstroke}{rgb}{0.000000,0.000000,0.000000}%
\pgfsetstrokecolor{currentstroke}%
\pgfsetstrokeopacity{0.700000}%
\pgfsetdash{}{0pt}%
\pgfpathmoveto{\pgfqpoint{8.138261in}{1.416787in}}%
\pgfpathcurveto{\pgfqpoint{8.143304in}{1.416787in}}{\pgfqpoint{8.148142in}{1.418791in}}{\pgfqpoint{8.151709in}{1.422358in}}%
\pgfpathcurveto{\pgfqpoint{8.155275in}{1.425924in}}{\pgfqpoint{8.157279in}{1.430762in}}{\pgfqpoint{8.157279in}{1.435805in}}%
\pgfpathcurveto{\pgfqpoint{8.157279in}{1.440849in}}{\pgfqpoint{8.155275in}{1.445687in}}{\pgfqpoint{8.151709in}{1.449253in}}%
\pgfpathcurveto{\pgfqpoint{8.148142in}{1.452820in}}{\pgfqpoint{8.143304in}{1.454824in}}{\pgfqpoint{8.138261in}{1.454824in}}%
\pgfpathcurveto{\pgfqpoint{8.133217in}{1.454824in}}{\pgfqpoint{8.128379in}{1.452820in}}{\pgfqpoint{8.124813in}{1.449253in}}%
\pgfpathcurveto{\pgfqpoint{8.121247in}{1.445687in}}{\pgfqpoint{8.119243in}{1.440849in}}{\pgfqpoint{8.119243in}{1.435805in}}%
\pgfpathcurveto{\pgfqpoint{8.119243in}{1.430762in}}{\pgfqpoint{8.121247in}{1.425924in}}{\pgfqpoint{8.124813in}{1.422358in}}%
\pgfpathcurveto{\pgfqpoint{8.128379in}{1.418791in}}{\pgfqpoint{8.133217in}{1.416787in}}{\pgfqpoint{8.138261in}{1.416787in}}%
\pgfpathclose%
\pgfusepath{fill}%
\end{pgfscope}%
\begin{pgfscope}%
\pgfpathrectangle{\pgfqpoint{6.572727in}{0.474100in}}{\pgfqpoint{4.227273in}{3.318700in}}%
\pgfusepath{clip}%
\pgfsetbuttcap%
\pgfsetroundjoin%
\definecolor{currentfill}{rgb}{0.993248,0.906157,0.143936}%
\pgfsetfillcolor{currentfill}%
\pgfsetfillopacity{0.700000}%
\pgfsetlinewidth{0.000000pt}%
\definecolor{currentstroke}{rgb}{0.000000,0.000000,0.000000}%
\pgfsetstrokecolor{currentstroke}%
\pgfsetstrokeopacity{0.700000}%
\pgfsetdash{}{0pt}%
\pgfpathmoveto{\pgfqpoint{8.041117in}{2.656134in}}%
\pgfpathcurveto{\pgfqpoint{8.046161in}{2.656134in}}{\pgfqpoint{8.050999in}{2.658138in}}{\pgfqpoint{8.054565in}{2.661704in}}%
\pgfpathcurveto{\pgfqpoint{8.058132in}{2.665271in}}{\pgfqpoint{8.060136in}{2.670109in}}{\pgfqpoint{8.060136in}{2.675152in}}%
\pgfpathcurveto{\pgfqpoint{8.060136in}{2.680196in}}{\pgfqpoint{8.058132in}{2.685034in}}{\pgfqpoint{8.054565in}{2.688600in}}%
\pgfpathcurveto{\pgfqpoint{8.050999in}{2.692167in}}{\pgfqpoint{8.046161in}{2.694170in}}{\pgfqpoint{8.041117in}{2.694170in}}%
\pgfpathcurveto{\pgfqpoint{8.036074in}{2.694170in}}{\pgfqpoint{8.031236in}{2.692167in}}{\pgfqpoint{8.027670in}{2.688600in}}%
\pgfpathcurveto{\pgfqpoint{8.024103in}{2.685034in}}{\pgfqpoint{8.022099in}{2.680196in}}{\pgfqpoint{8.022099in}{2.675152in}}%
\pgfpathcurveto{\pgfqpoint{8.022099in}{2.670109in}}{\pgfqpoint{8.024103in}{2.665271in}}{\pgfqpoint{8.027670in}{2.661704in}}%
\pgfpathcurveto{\pgfqpoint{8.031236in}{2.658138in}}{\pgfqpoint{8.036074in}{2.656134in}}{\pgfqpoint{8.041117in}{2.656134in}}%
\pgfpathclose%
\pgfusepath{fill}%
\end{pgfscope}%
\begin{pgfscope}%
\pgfpathrectangle{\pgfqpoint{6.572727in}{0.474100in}}{\pgfqpoint{4.227273in}{3.318700in}}%
\pgfusepath{clip}%
\pgfsetbuttcap%
\pgfsetroundjoin%
\definecolor{currentfill}{rgb}{0.267004,0.004874,0.329415}%
\pgfsetfillcolor{currentfill}%
\pgfsetfillopacity{0.700000}%
\pgfsetlinewidth{0.000000pt}%
\definecolor{currentstroke}{rgb}{0.000000,0.000000,0.000000}%
\pgfsetstrokecolor{currentstroke}%
\pgfsetstrokeopacity{0.700000}%
\pgfsetdash{}{0pt}%
\pgfpathmoveto{\pgfqpoint{8.234294in}{1.491787in}}%
\pgfpathcurveto{\pgfqpoint{8.239338in}{1.491787in}}{\pgfqpoint{8.244176in}{1.493791in}}{\pgfqpoint{8.247742in}{1.497357in}}%
\pgfpathcurveto{\pgfqpoint{8.251309in}{1.500924in}}{\pgfqpoint{8.253313in}{1.505762in}}{\pgfqpoint{8.253313in}{1.510805in}}%
\pgfpathcurveto{\pgfqpoint{8.253313in}{1.515849in}}{\pgfqpoint{8.251309in}{1.520687in}}{\pgfqpoint{8.247742in}{1.524253in}}%
\pgfpathcurveto{\pgfqpoint{8.244176in}{1.527820in}}{\pgfqpoint{8.239338in}{1.529823in}}{\pgfqpoint{8.234294in}{1.529823in}}%
\pgfpathcurveto{\pgfqpoint{8.229251in}{1.529823in}}{\pgfqpoint{8.224413in}{1.527820in}}{\pgfqpoint{8.220847in}{1.524253in}}%
\pgfpathcurveto{\pgfqpoint{8.217280in}{1.520687in}}{\pgfqpoint{8.215276in}{1.515849in}}{\pgfqpoint{8.215276in}{1.510805in}}%
\pgfpathcurveto{\pgfqpoint{8.215276in}{1.505762in}}{\pgfqpoint{8.217280in}{1.500924in}}{\pgfqpoint{8.220847in}{1.497357in}}%
\pgfpathcurveto{\pgfqpoint{8.224413in}{1.493791in}}{\pgfqpoint{8.229251in}{1.491787in}}{\pgfqpoint{8.234294in}{1.491787in}}%
\pgfpathclose%
\pgfusepath{fill}%
\end{pgfscope}%
\begin{pgfscope}%
\pgfpathrectangle{\pgfqpoint{6.572727in}{0.474100in}}{\pgfqpoint{4.227273in}{3.318700in}}%
\pgfusepath{clip}%
\pgfsetbuttcap%
\pgfsetroundjoin%
\definecolor{currentfill}{rgb}{0.267004,0.004874,0.329415}%
\pgfsetfillcolor{currentfill}%
\pgfsetfillopacity{0.700000}%
\pgfsetlinewidth{0.000000pt}%
\definecolor{currentstroke}{rgb}{0.000000,0.000000,0.000000}%
\pgfsetstrokecolor{currentstroke}%
\pgfsetstrokeopacity{0.700000}%
\pgfsetdash{}{0pt}%
\pgfpathmoveto{\pgfqpoint{7.417881in}{2.117797in}}%
\pgfpathcurveto{\pgfqpoint{7.422924in}{2.117797in}}{\pgfqpoint{7.427762in}{2.119801in}}{\pgfqpoint{7.431329in}{2.123368in}}%
\pgfpathcurveto{\pgfqpoint{7.434895in}{2.126934in}}{\pgfqpoint{7.436899in}{2.131772in}}{\pgfqpoint{7.436899in}{2.136815in}}%
\pgfpathcurveto{\pgfqpoint{7.436899in}{2.141859in}}{\pgfqpoint{7.434895in}{2.146697in}}{\pgfqpoint{7.431329in}{2.150263in}}%
\pgfpathcurveto{\pgfqpoint{7.427762in}{2.153830in}}{\pgfqpoint{7.422924in}{2.155834in}}{\pgfqpoint{7.417881in}{2.155834in}}%
\pgfpathcurveto{\pgfqpoint{7.412837in}{2.155834in}}{\pgfqpoint{7.407999in}{2.153830in}}{\pgfqpoint{7.404433in}{2.150263in}}%
\pgfpathcurveto{\pgfqpoint{7.400866in}{2.146697in}}{\pgfqpoint{7.398863in}{2.141859in}}{\pgfqpoint{7.398863in}{2.136815in}}%
\pgfpathcurveto{\pgfqpoint{7.398863in}{2.131772in}}{\pgfqpoint{7.400866in}{2.126934in}}{\pgfqpoint{7.404433in}{2.123368in}}%
\pgfpathcurveto{\pgfqpoint{7.407999in}{2.119801in}}{\pgfqpoint{7.412837in}{2.117797in}}{\pgfqpoint{7.417881in}{2.117797in}}%
\pgfpathclose%
\pgfusepath{fill}%
\end{pgfscope}%
\begin{pgfscope}%
\pgfpathrectangle{\pgfqpoint{6.572727in}{0.474100in}}{\pgfqpoint{4.227273in}{3.318700in}}%
\pgfusepath{clip}%
\pgfsetbuttcap%
\pgfsetroundjoin%
\definecolor{currentfill}{rgb}{0.267004,0.004874,0.329415}%
\pgfsetfillcolor{currentfill}%
\pgfsetfillopacity{0.700000}%
\pgfsetlinewidth{0.000000pt}%
\definecolor{currentstroke}{rgb}{0.000000,0.000000,0.000000}%
\pgfsetstrokecolor{currentstroke}%
\pgfsetstrokeopacity{0.700000}%
\pgfsetdash{}{0pt}%
\pgfpathmoveto{\pgfqpoint{7.862585in}{1.140430in}}%
\pgfpathcurveto{\pgfqpoint{7.867629in}{1.140430in}}{\pgfqpoint{7.872467in}{1.142434in}}{\pgfqpoint{7.876033in}{1.146001in}}%
\pgfpathcurveto{\pgfqpoint{7.879600in}{1.149567in}}{\pgfqpoint{7.881604in}{1.154405in}}{\pgfqpoint{7.881604in}{1.159448in}}%
\pgfpathcurveto{\pgfqpoint{7.881604in}{1.164492in}}{\pgfqpoint{7.879600in}{1.169330in}}{\pgfqpoint{7.876033in}{1.172896in}}%
\pgfpathcurveto{\pgfqpoint{7.872467in}{1.176463in}}{\pgfqpoint{7.867629in}{1.178467in}}{\pgfqpoint{7.862585in}{1.178467in}}%
\pgfpathcurveto{\pgfqpoint{7.857542in}{1.178467in}}{\pgfqpoint{7.852704in}{1.176463in}}{\pgfqpoint{7.849138in}{1.172896in}}%
\pgfpathcurveto{\pgfqpoint{7.845571in}{1.169330in}}{\pgfqpoint{7.843567in}{1.164492in}}{\pgfqpoint{7.843567in}{1.159448in}}%
\pgfpathcurveto{\pgfqpoint{7.843567in}{1.154405in}}{\pgfqpoint{7.845571in}{1.149567in}}{\pgfqpoint{7.849138in}{1.146001in}}%
\pgfpathcurveto{\pgfqpoint{7.852704in}{1.142434in}}{\pgfqpoint{7.857542in}{1.140430in}}{\pgfqpoint{7.862585in}{1.140430in}}%
\pgfpathclose%
\pgfusepath{fill}%
\end{pgfscope}%
\begin{pgfscope}%
\pgfpathrectangle{\pgfqpoint{6.572727in}{0.474100in}}{\pgfqpoint{4.227273in}{3.318700in}}%
\pgfusepath{clip}%
\pgfsetbuttcap%
\pgfsetroundjoin%
\definecolor{currentfill}{rgb}{0.267004,0.004874,0.329415}%
\pgfsetfillcolor{currentfill}%
\pgfsetfillopacity{0.700000}%
\pgfsetlinewidth{0.000000pt}%
\definecolor{currentstroke}{rgb}{0.000000,0.000000,0.000000}%
\pgfsetstrokecolor{currentstroke}%
\pgfsetstrokeopacity{0.700000}%
\pgfsetdash{}{0pt}%
\pgfpathmoveto{\pgfqpoint{8.169487in}{1.757332in}}%
\pgfpathcurveto{\pgfqpoint{8.174530in}{1.757332in}}{\pgfqpoint{8.179368in}{1.759336in}}{\pgfqpoint{8.182935in}{1.762902in}}%
\pgfpathcurveto{\pgfqpoint{8.186501in}{1.766468in}}{\pgfqpoint{8.188505in}{1.771306in}}{\pgfqpoint{8.188505in}{1.776350in}}%
\pgfpathcurveto{\pgfqpoint{8.188505in}{1.781393in}}{\pgfqpoint{8.186501in}{1.786231in}}{\pgfqpoint{8.182935in}{1.789798in}}%
\pgfpathcurveto{\pgfqpoint{8.179368in}{1.793364in}}{\pgfqpoint{8.174530in}{1.795368in}}{\pgfqpoint{8.169487in}{1.795368in}}%
\pgfpathcurveto{\pgfqpoint{8.164443in}{1.795368in}}{\pgfqpoint{8.159605in}{1.793364in}}{\pgfqpoint{8.156039in}{1.789798in}}%
\pgfpathcurveto{\pgfqpoint{8.152473in}{1.786231in}}{\pgfqpoint{8.150469in}{1.781393in}}{\pgfqpoint{8.150469in}{1.776350in}}%
\pgfpathcurveto{\pgfqpoint{8.150469in}{1.771306in}}{\pgfqpoint{8.152473in}{1.766468in}}{\pgfqpoint{8.156039in}{1.762902in}}%
\pgfpathcurveto{\pgfqpoint{8.159605in}{1.759336in}}{\pgfqpoint{8.164443in}{1.757332in}}{\pgfqpoint{8.169487in}{1.757332in}}%
\pgfpathclose%
\pgfusepath{fill}%
\end{pgfscope}%
\begin{pgfscope}%
\pgfpathrectangle{\pgfqpoint{6.572727in}{0.474100in}}{\pgfqpoint{4.227273in}{3.318700in}}%
\pgfusepath{clip}%
\pgfsetbuttcap%
\pgfsetroundjoin%
\definecolor{currentfill}{rgb}{0.267004,0.004874,0.329415}%
\pgfsetfillcolor{currentfill}%
\pgfsetfillopacity{0.700000}%
\pgfsetlinewidth{0.000000pt}%
\definecolor{currentstroke}{rgb}{0.000000,0.000000,0.000000}%
\pgfsetstrokecolor{currentstroke}%
\pgfsetstrokeopacity{0.700000}%
\pgfsetdash{}{0pt}%
\pgfpathmoveto{\pgfqpoint{7.683748in}{1.933337in}}%
\pgfpathcurveto{\pgfqpoint{7.688792in}{1.933337in}}{\pgfqpoint{7.693629in}{1.935341in}}{\pgfqpoint{7.697196in}{1.938907in}}%
\pgfpathcurveto{\pgfqpoint{7.700762in}{1.942474in}}{\pgfqpoint{7.702766in}{1.947311in}}{\pgfqpoint{7.702766in}{1.952355in}}%
\pgfpathcurveto{\pgfqpoint{7.702766in}{1.957399in}}{\pgfqpoint{7.700762in}{1.962236in}}{\pgfqpoint{7.697196in}{1.965803in}}%
\pgfpathcurveto{\pgfqpoint{7.693629in}{1.969369in}}{\pgfqpoint{7.688792in}{1.971373in}}{\pgfqpoint{7.683748in}{1.971373in}}%
\pgfpathcurveto{\pgfqpoint{7.678704in}{1.971373in}}{\pgfqpoint{7.673867in}{1.969369in}}{\pgfqpoint{7.670300in}{1.965803in}}%
\pgfpathcurveto{\pgfqpoint{7.666734in}{1.962236in}}{\pgfqpoint{7.664730in}{1.957399in}}{\pgfqpoint{7.664730in}{1.952355in}}%
\pgfpathcurveto{\pgfqpoint{7.664730in}{1.947311in}}{\pgfqpoint{7.666734in}{1.942474in}}{\pgfqpoint{7.670300in}{1.938907in}}%
\pgfpathcurveto{\pgfqpoint{7.673867in}{1.935341in}}{\pgfqpoint{7.678704in}{1.933337in}}{\pgfqpoint{7.683748in}{1.933337in}}%
\pgfpathclose%
\pgfusepath{fill}%
\end{pgfscope}%
\begin{pgfscope}%
\pgfpathrectangle{\pgfqpoint{6.572727in}{0.474100in}}{\pgfqpoint{4.227273in}{3.318700in}}%
\pgfusepath{clip}%
\pgfsetbuttcap%
\pgfsetroundjoin%
\definecolor{currentfill}{rgb}{0.127568,0.566949,0.550556}%
\pgfsetfillcolor{currentfill}%
\pgfsetfillopacity{0.700000}%
\pgfsetlinewidth{0.000000pt}%
\definecolor{currentstroke}{rgb}{0.000000,0.000000,0.000000}%
\pgfsetstrokecolor{currentstroke}%
\pgfsetstrokeopacity{0.700000}%
\pgfsetdash{}{0pt}%
\pgfpathmoveto{\pgfqpoint{9.038013in}{1.557534in}}%
\pgfpathcurveto{\pgfqpoint{9.043056in}{1.557534in}}{\pgfqpoint{9.047894in}{1.559538in}}{\pgfqpoint{9.051461in}{1.563104in}}%
\pgfpathcurveto{\pgfqpoint{9.055027in}{1.566671in}}{\pgfqpoint{9.057031in}{1.571508in}}{\pgfqpoint{9.057031in}{1.576552in}}%
\pgfpathcurveto{\pgfqpoint{9.057031in}{1.581596in}}{\pgfqpoint{9.055027in}{1.586434in}}{\pgfqpoint{9.051461in}{1.590000in}}%
\pgfpathcurveto{\pgfqpoint{9.047894in}{1.593566in}}{\pgfqpoint{9.043056in}{1.595570in}}{\pgfqpoint{9.038013in}{1.595570in}}%
\pgfpathcurveto{\pgfqpoint{9.032969in}{1.595570in}}{\pgfqpoint{9.028131in}{1.593566in}}{\pgfqpoint{9.024565in}{1.590000in}}%
\pgfpathcurveto{\pgfqpoint{9.020998in}{1.586434in}}{\pgfqpoint{9.018995in}{1.581596in}}{\pgfqpoint{9.018995in}{1.576552in}}%
\pgfpathcurveto{\pgfqpoint{9.018995in}{1.571508in}}{\pgfqpoint{9.020998in}{1.566671in}}{\pgfqpoint{9.024565in}{1.563104in}}%
\pgfpathcurveto{\pgfqpoint{9.028131in}{1.559538in}}{\pgfqpoint{9.032969in}{1.557534in}}{\pgfqpoint{9.038013in}{1.557534in}}%
\pgfpathclose%
\pgfusepath{fill}%
\end{pgfscope}%
\begin{pgfscope}%
\pgfpathrectangle{\pgfqpoint{6.572727in}{0.474100in}}{\pgfqpoint{4.227273in}{3.318700in}}%
\pgfusepath{clip}%
\pgfsetbuttcap%
\pgfsetroundjoin%
\definecolor{currentfill}{rgb}{0.993248,0.906157,0.143936}%
\pgfsetfillcolor{currentfill}%
\pgfsetfillopacity{0.700000}%
\pgfsetlinewidth{0.000000pt}%
\definecolor{currentstroke}{rgb}{0.000000,0.000000,0.000000}%
\pgfsetstrokecolor{currentstroke}%
\pgfsetstrokeopacity{0.700000}%
\pgfsetdash{}{0pt}%
\pgfpathmoveto{\pgfqpoint{8.481208in}{2.287272in}}%
\pgfpathcurveto{\pgfqpoint{8.486252in}{2.287272in}}{\pgfqpoint{8.491090in}{2.289276in}}{\pgfqpoint{8.494656in}{2.292842in}}%
\pgfpathcurveto{\pgfqpoint{8.498223in}{2.296409in}}{\pgfqpoint{8.500226in}{2.301246in}}{\pgfqpoint{8.500226in}{2.306290in}}%
\pgfpathcurveto{\pgfqpoint{8.500226in}{2.311334in}}{\pgfqpoint{8.498223in}{2.316171in}}{\pgfqpoint{8.494656in}{2.319738in}}%
\pgfpathcurveto{\pgfqpoint{8.491090in}{2.323304in}}{\pgfqpoint{8.486252in}{2.325308in}}{\pgfqpoint{8.481208in}{2.325308in}}%
\pgfpathcurveto{\pgfqpoint{8.476165in}{2.325308in}}{\pgfqpoint{8.471327in}{2.323304in}}{\pgfqpoint{8.467760in}{2.319738in}}%
\pgfpathcurveto{\pgfqpoint{8.464194in}{2.316171in}}{\pgfqpoint{8.462190in}{2.311334in}}{\pgfqpoint{8.462190in}{2.306290in}}%
\pgfpathcurveto{\pgfqpoint{8.462190in}{2.301246in}}{\pgfqpoint{8.464194in}{2.296409in}}{\pgfqpoint{8.467760in}{2.292842in}}%
\pgfpathcurveto{\pgfqpoint{8.471327in}{2.289276in}}{\pgfqpoint{8.476165in}{2.287272in}}{\pgfqpoint{8.481208in}{2.287272in}}%
\pgfpathclose%
\pgfusepath{fill}%
\end{pgfscope}%
\begin{pgfscope}%
\pgfpathrectangle{\pgfqpoint{6.572727in}{0.474100in}}{\pgfqpoint{4.227273in}{3.318700in}}%
\pgfusepath{clip}%
\pgfsetbuttcap%
\pgfsetroundjoin%
\definecolor{currentfill}{rgb}{0.127568,0.566949,0.550556}%
\pgfsetfillcolor{currentfill}%
\pgfsetfillopacity{0.700000}%
\pgfsetlinewidth{0.000000pt}%
\definecolor{currentstroke}{rgb}{0.000000,0.000000,0.000000}%
\pgfsetstrokecolor{currentstroke}%
\pgfsetstrokeopacity{0.700000}%
\pgfsetdash{}{0pt}%
\pgfpathmoveto{\pgfqpoint{9.314467in}{1.888475in}}%
\pgfpathcurveto{\pgfqpoint{9.319511in}{1.888475in}}{\pgfqpoint{9.324349in}{1.890478in}}{\pgfqpoint{9.327915in}{1.894045in}}%
\pgfpathcurveto{\pgfqpoint{9.331481in}{1.897611in}}{\pgfqpoint{9.333485in}{1.902449in}}{\pgfqpoint{9.333485in}{1.907493in}}%
\pgfpathcurveto{\pgfqpoint{9.333485in}{1.912536in}}{\pgfqpoint{9.331481in}{1.917374in}}{\pgfqpoint{9.327915in}{1.920941in}}%
\pgfpathcurveto{\pgfqpoint{9.324349in}{1.924507in}}{\pgfqpoint{9.319511in}{1.926511in}}{\pgfqpoint{9.314467in}{1.926511in}}%
\pgfpathcurveto{\pgfqpoint{9.309423in}{1.926511in}}{\pgfqpoint{9.304586in}{1.924507in}}{\pgfqpoint{9.301019in}{1.920941in}}%
\pgfpathcurveto{\pgfqpoint{9.297453in}{1.917374in}}{\pgfqpoint{9.295449in}{1.912536in}}{\pgfqpoint{9.295449in}{1.907493in}}%
\pgfpathcurveto{\pgfqpoint{9.295449in}{1.902449in}}{\pgfqpoint{9.297453in}{1.897611in}}{\pgfqpoint{9.301019in}{1.894045in}}%
\pgfpathcurveto{\pgfqpoint{9.304586in}{1.890478in}}{\pgfqpoint{9.309423in}{1.888475in}}{\pgfqpoint{9.314467in}{1.888475in}}%
\pgfpathclose%
\pgfusepath{fill}%
\end{pgfscope}%
\begin{pgfscope}%
\pgfpathrectangle{\pgfqpoint{6.572727in}{0.474100in}}{\pgfqpoint{4.227273in}{3.318700in}}%
\pgfusepath{clip}%
\pgfsetbuttcap%
\pgfsetroundjoin%
\definecolor{currentfill}{rgb}{0.267004,0.004874,0.329415}%
\pgfsetfillcolor{currentfill}%
\pgfsetfillopacity{0.700000}%
\pgfsetlinewidth{0.000000pt}%
\definecolor{currentstroke}{rgb}{0.000000,0.000000,0.000000}%
\pgfsetstrokecolor{currentstroke}%
\pgfsetstrokeopacity{0.700000}%
\pgfsetdash{}{0pt}%
\pgfpathmoveto{\pgfqpoint{7.843992in}{0.899686in}}%
\pgfpathcurveto{\pgfqpoint{7.849036in}{0.899686in}}{\pgfqpoint{7.853873in}{0.901690in}}{\pgfqpoint{7.857440in}{0.905256in}}%
\pgfpathcurveto{\pgfqpoint{7.861006in}{0.908823in}}{\pgfqpoint{7.863010in}{0.913661in}}{\pgfqpoint{7.863010in}{0.918704in}}%
\pgfpathcurveto{\pgfqpoint{7.863010in}{0.923748in}}{\pgfqpoint{7.861006in}{0.928586in}}{\pgfqpoint{7.857440in}{0.932152in}}%
\pgfpathcurveto{\pgfqpoint{7.853873in}{0.935718in}}{\pgfqpoint{7.849036in}{0.937722in}}{\pgfqpoint{7.843992in}{0.937722in}}%
\pgfpathcurveto{\pgfqpoint{7.838948in}{0.937722in}}{\pgfqpoint{7.834110in}{0.935718in}}{\pgfqpoint{7.830544in}{0.932152in}}%
\pgfpathcurveto{\pgfqpoint{7.826978in}{0.928586in}}{\pgfqpoint{7.824974in}{0.923748in}}{\pgfqpoint{7.824974in}{0.918704in}}%
\pgfpathcurveto{\pgfqpoint{7.824974in}{0.913661in}}{\pgfqpoint{7.826978in}{0.908823in}}{\pgfqpoint{7.830544in}{0.905256in}}%
\pgfpathcurveto{\pgfqpoint{7.834110in}{0.901690in}}{\pgfqpoint{7.838948in}{0.899686in}}{\pgfqpoint{7.843992in}{0.899686in}}%
\pgfpathclose%
\pgfusepath{fill}%
\end{pgfscope}%
\begin{pgfscope}%
\pgfpathrectangle{\pgfqpoint{6.572727in}{0.474100in}}{\pgfqpoint{4.227273in}{3.318700in}}%
\pgfusepath{clip}%
\pgfsetbuttcap%
\pgfsetroundjoin%
\definecolor{currentfill}{rgb}{0.993248,0.906157,0.143936}%
\pgfsetfillcolor{currentfill}%
\pgfsetfillopacity{0.700000}%
\pgfsetlinewidth{0.000000pt}%
\definecolor{currentstroke}{rgb}{0.000000,0.000000,0.000000}%
\pgfsetstrokecolor{currentstroke}%
\pgfsetstrokeopacity{0.700000}%
\pgfsetdash{}{0pt}%
\pgfpathmoveto{\pgfqpoint{8.501472in}{2.815219in}}%
\pgfpathcurveto{\pgfqpoint{8.506515in}{2.815219in}}{\pgfqpoint{8.511353in}{2.817223in}}{\pgfqpoint{8.514919in}{2.820790in}}%
\pgfpathcurveto{\pgfqpoint{8.518486in}{2.824356in}}{\pgfqpoint{8.520490in}{2.829194in}}{\pgfqpoint{8.520490in}{2.834237in}}%
\pgfpathcurveto{\pgfqpoint{8.520490in}{2.839281in}}{\pgfqpoint{8.518486in}{2.844119in}}{\pgfqpoint{8.514919in}{2.847685in}}%
\pgfpathcurveto{\pgfqpoint{8.511353in}{2.851252in}}{\pgfqpoint{8.506515in}{2.853256in}}{\pgfqpoint{8.501472in}{2.853256in}}%
\pgfpathcurveto{\pgfqpoint{8.496428in}{2.853256in}}{\pgfqpoint{8.491590in}{2.851252in}}{\pgfqpoint{8.488024in}{2.847685in}}%
\pgfpathcurveto{\pgfqpoint{8.484457in}{2.844119in}}{\pgfqpoint{8.482453in}{2.839281in}}{\pgfqpoint{8.482453in}{2.834237in}}%
\pgfpathcurveto{\pgfqpoint{8.482453in}{2.829194in}}{\pgfqpoint{8.484457in}{2.824356in}}{\pgfqpoint{8.488024in}{2.820790in}}%
\pgfpathcurveto{\pgfqpoint{8.491590in}{2.817223in}}{\pgfqpoint{8.496428in}{2.815219in}}{\pgfqpoint{8.501472in}{2.815219in}}%
\pgfpathclose%
\pgfusepath{fill}%
\end{pgfscope}%
\begin{pgfscope}%
\pgfpathrectangle{\pgfqpoint{6.572727in}{0.474100in}}{\pgfqpoint{4.227273in}{3.318700in}}%
\pgfusepath{clip}%
\pgfsetbuttcap%
\pgfsetroundjoin%
\definecolor{currentfill}{rgb}{0.993248,0.906157,0.143936}%
\pgfsetfillcolor{currentfill}%
\pgfsetfillopacity{0.700000}%
\pgfsetlinewidth{0.000000pt}%
\definecolor{currentstroke}{rgb}{0.000000,0.000000,0.000000}%
\pgfsetstrokecolor{currentstroke}%
\pgfsetstrokeopacity{0.700000}%
\pgfsetdash{}{0pt}%
\pgfpathmoveto{\pgfqpoint{8.110796in}{2.866503in}}%
\pgfpathcurveto{\pgfqpoint{8.115839in}{2.866503in}}{\pgfqpoint{8.120677in}{2.868507in}}{\pgfqpoint{8.124244in}{2.872073in}}%
\pgfpathcurveto{\pgfqpoint{8.127810in}{2.875640in}}{\pgfqpoint{8.129814in}{2.880477in}}{\pgfqpoint{8.129814in}{2.885521in}}%
\pgfpathcurveto{\pgfqpoint{8.129814in}{2.890565in}}{\pgfqpoint{8.127810in}{2.895403in}}{\pgfqpoint{8.124244in}{2.898969in}}%
\pgfpathcurveto{\pgfqpoint{8.120677in}{2.902535in}}{\pgfqpoint{8.115839in}{2.904539in}}{\pgfqpoint{8.110796in}{2.904539in}}%
\pgfpathcurveto{\pgfqpoint{8.105752in}{2.904539in}}{\pgfqpoint{8.100914in}{2.902535in}}{\pgfqpoint{8.097348in}{2.898969in}}%
\pgfpathcurveto{\pgfqpoint{8.093781in}{2.895403in}}{\pgfqpoint{8.091778in}{2.890565in}}{\pgfqpoint{8.091778in}{2.885521in}}%
\pgfpathcurveto{\pgfqpoint{8.091778in}{2.880477in}}{\pgfqpoint{8.093781in}{2.875640in}}{\pgfqpoint{8.097348in}{2.872073in}}%
\pgfpathcurveto{\pgfqpoint{8.100914in}{2.868507in}}{\pgfqpoint{8.105752in}{2.866503in}}{\pgfqpoint{8.110796in}{2.866503in}}%
\pgfpathclose%
\pgfusepath{fill}%
\end{pgfscope}%
\begin{pgfscope}%
\pgfpathrectangle{\pgfqpoint{6.572727in}{0.474100in}}{\pgfqpoint{4.227273in}{3.318700in}}%
\pgfusepath{clip}%
\pgfsetbuttcap%
\pgfsetroundjoin%
\definecolor{currentfill}{rgb}{0.267004,0.004874,0.329415}%
\pgfsetfillcolor{currentfill}%
\pgfsetfillopacity{0.700000}%
\pgfsetlinewidth{0.000000pt}%
\definecolor{currentstroke}{rgb}{0.000000,0.000000,0.000000}%
\pgfsetstrokecolor{currentstroke}%
\pgfsetstrokeopacity{0.700000}%
\pgfsetdash{}{0pt}%
\pgfpathmoveto{\pgfqpoint{8.297181in}{1.904391in}}%
\pgfpathcurveto{\pgfqpoint{8.302224in}{1.904391in}}{\pgfqpoint{8.307062in}{1.906395in}}{\pgfqpoint{8.310629in}{1.909961in}}%
\pgfpathcurveto{\pgfqpoint{8.314195in}{1.913528in}}{\pgfqpoint{8.316199in}{1.918365in}}{\pgfqpoint{8.316199in}{1.923409in}}%
\pgfpathcurveto{\pgfqpoint{8.316199in}{1.928453in}}{\pgfqpoint{8.314195in}{1.933291in}}{\pgfqpoint{8.310629in}{1.936857in}}%
\pgfpathcurveto{\pgfqpoint{8.307062in}{1.940423in}}{\pgfqpoint{8.302224in}{1.942427in}}{\pgfqpoint{8.297181in}{1.942427in}}%
\pgfpathcurveto{\pgfqpoint{8.292137in}{1.942427in}}{\pgfqpoint{8.287299in}{1.940423in}}{\pgfqpoint{8.283733in}{1.936857in}}%
\pgfpathcurveto{\pgfqpoint{8.280167in}{1.933291in}}{\pgfqpoint{8.278163in}{1.928453in}}{\pgfqpoint{8.278163in}{1.923409in}}%
\pgfpathcurveto{\pgfqpoint{8.278163in}{1.918365in}}{\pgfqpoint{8.280167in}{1.913528in}}{\pgfqpoint{8.283733in}{1.909961in}}%
\pgfpathcurveto{\pgfqpoint{8.287299in}{1.906395in}}{\pgfqpoint{8.292137in}{1.904391in}}{\pgfqpoint{8.297181in}{1.904391in}}%
\pgfpathclose%
\pgfusepath{fill}%
\end{pgfscope}%
\begin{pgfscope}%
\pgfpathrectangle{\pgfqpoint{6.572727in}{0.474100in}}{\pgfqpoint{4.227273in}{3.318700in}}%
\pgfusepath{clip}%
\pgfsetbuttcap%
\pgfsetroundjoin%
\definecolor{currentfill}{rgb}{0.127568,0.566949,0.550556}%
\pgfsetfillcolor{currentfill}%
\pgfsetfillopacity{0.700000}%
\pgfsetlinewidth{0.000000pt}%
\definecolor{currentstroke}{rgb}{0.000000,0.000000,0.000000}%
\pgfsetstrokecolor{currentstroke}%
\pgfsetstrokeopacity{0.700000}%
\pgfsetdash{}{0pt}%
\pgfpathmoveto{\pgfqpoint{8.738845in}{1.420968in}}%
\pgfpathcurveto{\pgfqpoint{8.743889in}{1.420968in}}{\pgfqpoint{8.748727in}{1.422972in}}{\pgfqpoint{8.752293in}{1.426538in}}%
\pgfpathcurveto{\pgfqpoint{8.755860in}{1.430105in}}{\pgfqpoint{8.757863in}{1.434942in}}{\pgfqpoint{8.757863in}{1.439986in}}%
\pgfpathcurveto{\pgfqpoint{8.757863in}{1.445030in}}{\pgfqpoint{8.755860in}{1.449867in}}{\pgfqpoint{8.752293in}{1.453434in}}%
\pgfpathcurveto{\pgfqpoint{8.748727in}{1.457000in}}{\pgfqpoint{8.743889in}{1.459004in}}{\pgfqpoint{8.738845in}{1.459004in}}%
\pgfpathcurveto{\pgfqpoint{8.733802in}{1.459004in}}{\pgfqpoint{8.728964in}{1.457000in}}{\pgfqpoint{8.725397in}{1.453434in}}%
\pgfpathcurveto{\pgfqpoint{8.721831in}{1.449867in}}{\pgfqpoint{8.719827in}{1.445030in}}{\pgfqpoint{8.719827in}{1.439986in}}%
\pgfpathcurveto{\pgfqpoint{8.719827in}{1.434942in}}{\pgfqpoint{8.721831in}{1.430105in}}{\pgfqpoint{8.725397in}{1.426538in}}%
\pgfpathcurveto{\pgfqpoint{8.728964in}{1.422972in}}{\pgfqpoint{8.733802in}{1.420968in}}{\pgfqpoint{8.738845in}{1.420968in}}%
\pgfpathclose%
\pgfusepath{fill}%
\end{pgfscope}%
\begin{pgfscope}%
\pgfpathrectangle{\pgfqpoint{6.572727in}{0.474100in}}{\pgfqpoint{4.227273in}{3.318700in}}%
\pgfusepath{clip}%
\pgfsetbuttcap%
\pgfsetroundjoin%
\definecolor{currentfill}{rgb}{0.127568,0.566949,0.550556}%
\pgfsetfillcolor{currentfill}%
\pgfsetfillopacity{0.700000}%
\pgfsetlinewidth{0.000000pt}%
\definecolor{currentstroke}{rgb}{0.000000,0.000000,0.000000}%
\pgfsetstrokecolor{currentstroke}%
\pgfsetstrokeopacity{0.700000}%
\pgfsetdash{}{0pt}%
\pgfpathmoveto{\pgfqpoint{9.144984in}{1.286895in}}%
\pgfpathcurveto{\pgfqpoint{9.150027in}{1.286895in}}{\pgfqpoint{9.154865in}{1.288899in}}{\pgfqpoint{9.158431in}{1.292466in}}%
\pgfpathcurveto{\pgfqpoint{9.161998in}{1.296032in}}{\pgfqpoint{9.164002in}{1.300870in}}{\pgfqpoint{9.164002in}{1.305914in}}%
\pgfpathcurveto{\pgfqpoint{9.164002in}{1.310957in}}{\pgfqpoint{9.161998in}{1.315795in}}{\pgfqpoint{9.158431in}{1.319361in}}%
\pgfpathcurveto{\pgfqpoint{9.154865in}{1.322928in}}{\pgfqpoint{9.150027in}{1.324932in}}{\pgfqpoint{9.144984in}{1.324932in}}%
\pgfpathcurveto{\pgfqpoint{9.139940in}{1.324932in}}{\pgfqpoint{9.135102in}{1.322928in}}{\pgfqpoint{9.131536in}{1.319361in}}%
\pgfpathcurveto{\pgfqpoint{9.127969in}{1.315795in}}{\pgfqpoint{9.125965in}{1.310957in}}{\pgfqpoint{9.125965in}{1.305914in}}%
\pgfpathcurveto{\pgfqpoint{9.125965in}{1.300870in}}{\pgfqpoint{9.127969in}{1.296032in}}{\pgfqpoint{9.131536in}{1.292466in}}%
\pgfpathcurveto{\pgfqpoint{9.135102in}{1.288899in}}{\pgfqpoint{9.139940in}{1.286895in}}{\pgfqpoint{9.144984in}{1.286895in}}%
\pgfpathclose%
\pgfusepath{fill}%
\end{pgfscope}%
\begin{pgfscope}%
\pgfpathrectangle{\pgfqpoint{6.572727in}{0.474100in}}{\pgfqpoint{4.227273in}{3.318700in}}%
\pgfusepath{clip}%
\pgfsetbuttcap%
\pgfsetroundjoin%
\definecolor{currentfill}{rgb}{0.127568,0.566949,0.550556}%
\pgfsetfillcolor{currentfill}%
\pgfsetfillopacity{0.700000}%
\pgfsetlinewidth{0.000000pt}%
\definecolor{currentstroke}{rgb}{0.000000,0.000000,0.000000}%
\pgfsetstrokecolor{currentstroke}%
\pgfsetstrokeopacity{0.700000}%
\pgfsetdash{}{0pt}%
\pgfpathmoveto{\pgfqpoint{9.338292in}{1.466118in}}%
\pgfpathcurveto{\pgfqpoint{9.343335in}{1.466118in}}{\pgfqpoint{9.348173in}{1.468122in}}{\pgfqpoint{9.351740in}{1.471688in}}%
\pgfpathcurveto{\pgfqpoint{9.355306in}{1.475255in}}{\pgfqpoint{9.357310in}{1.480092in}}{\pgfqpoint{9.357310in}{1.485136in}}%
\pgfpathcurveto{\pgfqpoint{9.357310in}{1.490180in}}{\pgfqpoint{9.355306in}{1.495018in}}{\pgfqpoint{9.351740in}{1.498584in}}%
\pgfpathcurveto{\pgfqpoint{9.348173in}{1.502150in}}{\pgfqpoint{9.343335in}{1.504154in}}{\pgfqpoint{9.338292in}{1.504154in}}%
\pgfpathcurveto{\pgfqpoint{9.333248in}{1.504154in}}{\pgfqpoint{9.328410in}{1.502150in}}{\pgfqpoint{9.324844in}{1.498584in}}%
\pgfpathcurveto{\pgfqpoint{9.321277in}{1.495018in}}{\pgfqpoint{9.319274in}{1.490180in}}{\pgfqpoint{9.319274in}{1.485136in}}%
\pgfpathcurveto{\pgfqpoint{9.319274in}{1.480092in}}{\pgfqpoint{9.321277in}{1.475255in}}{\pgfqpoint{9.324844in}{1.471688in}}%
\pgfpathcurveto{\pgfqpoint{9.328410in}{1.468122in}}{\pgfqpoint{9.333248in}{1.466118in}}{\pgfqpoint{9.338292in}{1.466118in}}%
\pgfpathclose%
\pgfusepath{fill}%
\end{pgfscope}%
\begin{pgfscope}%
\pgfpathrectangle{\pgfqpoint{6.572727in}{0.474100in}}{\pgfqpoint{4.227273in}{3.318700in}}%
\pgfusepath{clip}%
\pgfsetbuttcap%
\pgfsetroundjoin%
\definecolor{currentfill}{rgb}{0.993248,0.906157,0.143936}%
\pgfsetfillcolor{currentfill}%
\pgfsetfillopacity{0.700000}%
\pgfsetlinewidth{0.000000pt}%
\definecolor{currentstroke}{rgb}{0.000000,0.000000,0.000000}%
\pgfsetstrokecolor{currentstroke}%
\pgfsetstrokeopacity{0.700000}%
\pgfsetdash{}{0pt}%
\pgfpathmoveto{\pgfqpoint{8.082784in}{2.719942in}}%
\pgfpathcurveto{\pgfqpoint{8.087827in}{2.719942in}}{\pgfqpoint{8.092665in}{2.721946in}}{\pgfqpoint{8.096232in}{2.725512in}}%
\pgfpathcurveto{\pgfqpoint{8.099798in}{2.729079in}}{\pgfqpoint{8.101802in}{2.733917in}}{\pgfqpoint{8.101802in}{2.738960in}}%
\pgfpathcurveto{\pgfqpoint{8.101802in}{2.744004in}}{\pgfqpoint{8.099798in}{2.748842in}}{\pgfqpoint{8.096232in}{2.752408in}}%
\pgfpathcurveto{\pgfqpoint{8.092665in}{2.755975in}}{\pgfqpoint{8.087827in}{2.757978in}}{\pgfqpoint{8.082784in}{2.757978in}}%
\pgfpathcurveto{\pgfqpoint{8.077740in}{2.757978in}}{\pgfqpoint{8.072902in}{2.755975in}}{\pgfqpoint{8.069336in}{2.752408in}}%
\pgfpathcurveto{\pgfqpoint{8.065769in}{2.748842in}}{\pgfqpoint{8.063766in}{2.744004in}}{\pgfqpoint{8.063766in}{2.738960in}}%
\pgfpathcurveto{\pgfqpoint{8.063766in}{2.733917in}}{\pgfqpoint{8.065769in}{2.729079in}}{\pgfqpoint{8.069336in}{2.725512in}}%
\pgfpathcurveto{\pgfqpoint{8.072902in}{2.721946in}}{\pgfqpoint{8.077740in}{2.719942in}}{\pgfqpoint{8.082784in}{2.719942in}}%
\pgfpathclose%
\pgfusepath{fill}%
\end{pgfscope}%
\begin{pgfscope}%
\pgfpathrectangle{\pgfqpoint{6.572727in}{0.474100in}}{\pgfqpoint{4.227273in}{3.318700in}}%
\pgfusepath{clip}%
\pgfsetbuttcap%
\pgfsetroundjoin%
\definecolor{currentfill}{rgb}{0.127568,0.566949,0.550556}%
\pgfsetfillcolor{currentfill}%
\pgfsetfillopacity{0.700000}%
\pgfsetlinewidth{0.000000pt}%
\definecolor{currentstroke}{rgb}{0.000000,0.000000,0.000000}%
\pgfsetstrokecolor{currentstroke}%
\pgfsetstrokeopacity{0.700000}%
\pgfsetdash{}{0pt}%
\pgfpathmoveto{\pgfqpoint{10.523575in}{1.485629in}}%
\pgfpathcurveto{\pgfqpoint{10.528619in}{1.485629in}}{\pgfqpoint{10.533457in}{1.487633in}}{\pgfqpoint{10.537023in}{1.491199in}}%
\pgfpathcurveto{\pgfqpoint{10.540590in}{1.494766in}}{\pgfqpoint{10.542593in}{1.499603in}}{\pgfqpoint{10.542593in}{1.504647in}}%
\pgfpathcurveto{\pgfqpoint{10.542593in}{1.509691in}}{\pgfqpoint{10.540590in}{1.514529in}}{\pgfqpoint{10.537023in}{1.518095in}}%
\pgfpathcurveto{\pgfqpoint{10.533457in}{1.521661in}}{\pgfqpoint{10.528619in}{1.523665in}}{\pgfqpoint{10.523575in}{1.523665in}}%
\pgfpathcurveto{\pgfqpoint{10.518532in}{1.523665in}}{\pgfqpoint{10.513694in}{1.521661in}}{\pgfqpoint{10.510127in}{1.518095in}}%
\pgfpathcurveto{\pgfqpoint{10.506561in}{1.514529in}}{\pgfqpoint{10.504557in}{1.509691in}}{\pgfqpoint{10.504557in}{1.504647in}}%
\pgfpathcurveto{\pgfqpoint{10.504557in}{1.499603in}}{\pgfqpoint{10.506561in}{1.494766in}}{\pgfqpoint{10.510127in}{1.491199in}}%
\pgfpathcurveto{\pgfqpoint{10.513694in}{1.487633in}}{\pgfqpoint{10.518532in}{1.485629in}}{\pgfqpoint{10.523575in}{1.485629in}}%
\pgfpathclose%
\pgfusepath{fill}%
\end{pgfscope}%
\begin{pgfscope}%
\pgfpathrectangle{\pgfqpoint{6.572727in}{0.474100in}}{\pgfqpoint{4.227273in}{3.318700in}}%
\pgfusepath{clip}%
\pgfsetbuttcap%
\pgfsetroundjoin%
\definecolor{currentfill}{rgb}{0.993248,0.906157,0.143936}%
\pgfsetfillcolor{currentfill}%
\pgfsetfillopacity{0.700000}%
\pgfsetlinewidth{0.000000pt}%
\definecolor{currentstroke}{rgb}{0.000000,0.000000,0.000000}%
\pgfsetstrokecolor{currentstroke}%
\pgfsetstrokeopacity{0.700000}%
\pgfsetdash{}{0pt}%
\pgfpathmoveto{\pgfqpoint{7.985853in}{3.130262in}}%
\pgfpathcurveto{\pgfqpoint{7.990897in}{3.130262in}}{\pgfqpoint{7.995735in}{3.132266in}}{\pgfqpoint{7.999301in}{3.135832in}}%
\pgfpathcurveto{\pgfqpoint{8.002868in}{3.139399in}}{\pgfqpoint{8.004872in}{3.144236in}}{\pgfqpoint{8.004872in}{3.149280in}}%
\pgfpathcurveto{\pgfqpoint{8.004872in}{3.154324in}}{\pgfqpoint{8.002868in}{3.159162in}}{\pgfqpoint{7.999301in}{3.162728in}}%
\pgfpathcurveto{\pgfqpoint{7.995735in}{3.166294in}}{\pgfqpoint{7.990897in}{3.168298in}}{\pgfqpoint{7.985853in}{3.168298in}}%
\pgfpathcurveto{\pgfqpoint{7.980810in}{3.168298in}}{\pgfqpoint{7.975972in}{3.166294in}}{\pgfqpoint{7.972406in}{3.162728in}}%
\pgfpathcurveto{\pgfqpoint{7.968839in}{3.159162in}}{\pgfqpoint{7.966835in}{3.154324in}}{\pgfqpoint{7.966835in}{3.149280in}}%
\pgfpathcurveto{\pgfqpoint{7.966835in}{3.144236in}}{\pgfqpoint{7.968839in}{3.139399in}}{\pgfqpoint{7.972406in}{3.135832in}}%
\pgfpathcurveto{\pgfqpoint{7.975972in}{3.132266in}}{\pgfqpoint{7.980810in}{3.130262in}}{\pgfqpoint{7.985853in}{3.130262in}}%
\pgfpathclose%
\pgfusepath{fill}%
\end{pgfscope}%
\begin{pgfscope}%
\pgfpathrectangle{\pgfqpoint{6.572727in}{0.474100in}}{\pgfqpoint{4.227273in}{3.318700in}}%
\pgfusepath{clip}%
\pgfsetbuttcap%
\pgfsetroundjoin%
\definecolor{currentfill}{rgb}{0.127568,0.566949,0.550556}%
\pgfsetfillcolor{currentfill}%
\pgfsetfillopacity{0.700000}%
\pgfsetlinewidth{0.000000pt}%
\definecolor{currentstroke}{rgb}{0.000000,0.000000,0.000000}%
\pgfsetstrokecolor{currentstroke}%
\pgfsetstrokeopacity{0.700000}%
\pgfsetdash{}{0pt}%
\pgfpathmoveto{\pgfqpoint{9.121314in}{1.549120in}}%
\pgfpathcurveto{\pgfqpoint{9.126358in}{1.549120in}}{\pgfqpoint{9.131196in}{1.551124in}}{\pgfqpoint{9.134762in}{1.554690in}}%
\pgfpathcurveto{\pgfqpoint{9.138329in}{1.558256in}}{\pgfqpoint{9.140332in}{1.563094in}}{\pgfqpoint{9.140332in}{1.568138in}}%
\pgfpathcurveto{\pgfqpoint{9.140332in}{1.573182in}}{\pgfqpoint{9.138329in}{1.578019in}}{\pgfqpoint{9.134762in}{1.581586in}}%
\pgfpathcurveto{\pgfqpoint{9.131196in}{1.585152in}}{\pgfqpoint{9.126358in}{1.587156in}}{\pgfqpoint{9.121314in}{1.587156in}}%
\pgfpathcurveto{\pgfqpoint{9.116271in}{1.587156in}}{\pgfqpoint{9.111433in}{1.585152in}}{\pgfqpoint{9.107866in}{1.581586in}}%
\pgfpathcurveto{\pgfqpoint{9.104300in}{1.578019in}}{\pgfqpoint{9.102296in}{1.573182in}}{\pgfqpoint{9.102296in}{1.568138in}}%
\pgfpathcurveto{\pgfqpoint{9.102296in}{1.563094in}}{\pgfqpoint{9.104300in}{1.558256in}}{\pgfqpoint{9.107866in}{1.554690in}}%
\pgfpathcurveto{\pgfqpoint{9.111433in}{1.551124in}}{\pgfqpoint{9.116271in}{1.549120in}}{\pgfqpoint{9.121314in}{1.549120in}}%
\pgfpathclose%
\pgfusepath{fill}%
\end{pgfscope}%
\begin{pgfscope}%
\pgfpathrectangle{\pgfqpoint{6.572727in}{0.474100in}}{\pgfqpoint{4.227273in}{3.318700in}}%
\pgfusepath{clip}%
\pgfsetbuttcap%
\pgfsetroundjoin%
\definecolor{currentfill}{rgb}{0.127568,0.566949,0.550556}%
\pgfsetfillcolor{currentfill}%
\pgfsetfillopacity{0.700000}%
\pgfsetlinewidth{0.000000pt}%
\definecolor{currentstroke}{rgb}{0.000000,0.000000,0.000000}%
\pgfsetstrokecolor{currentstroke}%
\pgfsetstrokeopacity{0.700000}%
\pgfsetdash{}{0pt}%
\pgfpathmoveto{\pgfqpoint{9.732590in}{1.861824in}}%
\pgfpathcurveto{\pgfqpoint{9.737634in}{1.861824in}}{\pgfqpoint{9.742472in}{1.863828in}}{\pgfqpoint{9.746038in}{1.867394in}}%
\pgfpathcurveto{\pgfqpoint{9.749605in}{1.870960in}}{\pgfqpoint{9.751608in}{1.875798in}}{\pgfqpoint{9.751608in}{1.880842in}}%
\pgfpathcurveto{\pgfqpoint{9.751608in}{1.885885in}}{\pgfqpoint{9.749605in}{1.890723in}}{\pgfqpoint{9.746038in}{1.894290in}}%
\pgfpathcurveto{\pgfqpoint{9.742472in}{1.897856in}}{\pgfqpoint{9.737634in}{1.899860in}}{\pgfqpoint{9.732590in}{1.899860in}}%
\pgfpathcurveto{\pgfqpoint{9.727547in}{1.899860in}}{\pgfqpoint{9.722709in}{1.897856in}}{\pgfqpoint{9.719142in}{1.894290in}}%
\pgfpathcurveto{\pgfqpoint{9.715576in}{1.890723in}}{\pgfqpoint{9.713572in}{1.885885in}}{\pgfqpoint{9.713572in}{1.880842in}}%
\pgfpathcurveto{\pgfqpoint{9.713572in}{1.875798in}}{\pgfqpoint{9.715576in}{1.870960in}}{\pgfqpoint{9.719142in}{1.867394in}}%
\pgfpathcurveto{\pgfqpoint{9.722709in}{1.863828in}}{\pgfqpoint{9.727547in}{1.861824in}}{\pgfqpoint{9.732590in}{1.861824in}}%
\pgfpathclose%
\pgfusepath{fill}%
\end{pgfscope}%
\begin{pgfscope}%
\pgfpathrectangle{\pgfqpoint{6.572727in}{0.474100in}}{\pgfqpoint{4.227273in}{3.318700in}}%
\pgfusepath{clip}%
\pgfsetbuttcap%
\pgfsetroundjoin%
\definecolor{currentfill}{rgb}{0.267004,0.004874,0.329415}%
\pgfsetfillcolor{currentfill}%
\pgfsetfillopacity{0.700000}%
\pgfsetlinewidth{0.000000pt}%
\definecolor{currentstroke}{rgb}{0.000000,0.000000,0.000000}%
\pgfsetstrokecolor{currentstroke}%
\pgfsetstrokeopacity{0.700000}%
\pgfsetdash{}{0pt}%
\pgfpathmoveto{\pgfqpoint{7.764354in}{1.425265in}}%
\pgfpathcurveto{\pgfqpoint{7.769397in}{1.425265in}}{\pgfqpoint{7.774235in}{1.427269in}}{\pgfqpoint{7.777802in}{1.430836in}}%
\pgfpathcurveto{\pgfqpoint{7.781368in}{1.434402in}}{\pgfqpoint{7.783372in}{1.439240in}}{\pgfqpoint{7.783372in}{1.444283in}}%
\pgfpathcurveto{\pgfqpoint{7.783372in}{1.449327in}}{\pgfqpoint{7.781368in}{1.454165in}}{\pgfqpoint{7.777802in}{1.457731in}}%
\pgfpathcurveto{\pgfqpoint{7.774235in}{1.461298in}}{\pgfqpoint{7.769397in}{1.463302in}}{\pgfqpoint{7.764354in}{1.463302in}}%
\pgfpathcurveto{\pgfqpoint{7.759310in}{1.463302in}}{\pgfqpoint{7.754472in}{1.461298in}}{\pgfqpoint{7.750906in}{1.457731in}}%
\pgfpathcurveto{\pgfqpoint{7.747340in}{1.454165in}}{\pgfqpoint{7.745336in}{1.449327in}}{\pgfqpoint{7.745336in}{1.444283in}}%
\pgfpathcurveto{\pgfqpoint{7.745336in}{1.439240in}}{\pgfqpoint{7.747340in}{1.434402in}}{\pgfqpoint{7.750906in}{1.430836in}}%
\pgfpathcurveto{\pgfqpoint{7.754472in}{1.427269in}}{\pgfqpoint{7.759310in}{1.425265in}}{\pgfqpoint{7.764354in}{1.425265in}}%
\pgfpathclose%
\pgfusepath{fill}%
\end{pgfscope}%
\begin{pgfscope}%
\pgfpathrectangle{\pgfqpoint{6.572727in}{0.474100in}}{\pgfqpoint{4.227273in}{3.318700in}}%
\pgfusepath{clip}%
\pgfsetbuttcap%
\pgfsetroundjoin%
\definecolor{currentfill}{rgb}{0.127568,0.566949,0.550556}%
\pgfsetfillcolor{currentfill}%
\pgfsetfillopacity{0.700000}%
\pgfsetlinewidth{0.000000pt}%
\definecolor{currentstroke}{rgb}{0.000000,0.000000,0.000000}%
\pgfsetstrokecolor{currentstroke}%
\pgfsetstrokeopacity{0.700000}%
\pgfsetdash{}{0pt}%
\pgfpathmoveto{\pgfqpoint{10.185394in}{1.587454in}}%
\pgfpathcurveto{\pgfqpoint{10.190437in}{1.587454in}}{\pgfqpoint{10.195275in}{1.589458in}}{\pgfqpoint{10.198841in}{1.593025in}}%
\pgfpathcurveto{\pgfqpoint{10.202408in}{1.596591in}}{\pgfqpoint{10.204412in}{1.601429in}}{\pgfqpoint{10.204412in}{1.606472in}}%
\pgfpathcurveto{\pgfqpoint{10.204412in}{1.611516in}}{\pgfqpoint{10.202408in}{1.616354in}}{\pgfqpoint{10.198841in}{1.619920in}}%
\pgfpathcurveto{\pgfqpoint{10.195275in}{1.623487in}}{\pgfqpoint{10.190437in}{1.625491in}}{\pgfqpoint{10.185394in}{1.625491in}}%
\pgfpathcurveto{\pgfqpoint{10.180350in}{1.625491in}}{\pgfqpoint{10.175512in}{1.623487in}}{\pgfqpoint{10.171946in}{1.619920in}}%
\pgfpathcurveto{\pgfqpoint{10.168379in}{1.616354in}}{\pgfqpoint{10.166375in}{1.611516in}}{\pgfqpoint{10.166375in}{1.606472in}}%
\pgfpathcurveto{\pgfqpoint{10.166375in}{1.601429in}}{\pgfqpoint{10.168379in}{1.596591in}}{\pgfqpoint{10.171946in}{1.593025in}}%
\pgfpathcurveto{\pgfqpoint{10.175512in}{1.589458in}}{\pgfqpoint{10.180350in}{1.587454in}}{\pgfqpoint{10.185394in}{1.587454in}}%
\pgfpathclose%
\pgfusepath{fill}%
\end{pgfscope}%
\begin{pgfscope}%
\pgfpathrectangle{\pgfqpoint{6.572727in}{0.474100in}}{\pgfqpoint{4.227273in}{3.318700in}}%
\pgfusepath{clip}%
\pgfsetbuttcap%
\pgfsetroundjoin%
\definecolor{currentfill}{rgb}{0.127568,0.566949,0.550556}%
\pgfsetfillcolor{currentfill}%
\pgfsetfillopacity{0.700000}%
\pgfsetlinewidth{0.000000pt}%
\definecolor{currentstroke}{rgb}{0.000000,0.000000,0.000000}%
\pgfsetstrokecolor{currentstroke}%
\pgfsetstrokeopacity{0.700000}%
\pgfsetdash{}{0pt}%
\pgfpathmoveto{\pgfqpoint{9.875598in}{1.573172in}}%
\pgfpathcurveto{\pgfqpoint{9.880641in}{1.573172in}}{\pgfqpoint{9.885479in}{1.575176in}}{\pgfqpoint{9.889046in}{1.578743in}}%
\pgfpathcurveto{\pgfqpoint{9.892612in}{1.582309in}}{\pgfqpoint{9.894616in}{1.587147in}}{\pgfqpoint{9.894616in}{1.592190in}}%
\pgfpathcurveto{\pgfqpoint{9.894616in}{1.597234in}}{\pgfqpoint{9.892612in}{1.602072in}}{\pgfqpoint{9.889046in}{1.605638in}}%
\pgfpathcurveto{\pgfqpoint{9.885479in}{1.609205in}}{\pgfqpoint{9.880641in}{1.611209in}}{\pgfqpoint{9.875598in}{1.611209in}}%
\pgfpathcurveto{\pgfqpoint{9.870554in}{1.611209in}}{\pgfqpoint{9.865716in}{1.609205in}}{\pgfqpoint{9.862150in}{1.605638in}}%
\pgfpathcurveto{\pgfqpoint{9.858583in}{1.602072in}}{\pgfqpoint{9.856580in}{1.597234in}}{\pgfqpoint{9.856580in}{1.592190in}}%
\pgfpathcurveto{\pgfqpoint{9.856580in}{1.587147in}}{\pgfqpoint{9.858583in}{1.582309in}}{\pgfqpoint{9.862150in}{1.578743in}}%
\pgfpathcurveto{\pgfqpoint{9.865716in}{1.575176in}}{\pgfqpoint{9.870554in}{1.573172in}}{\pgfqpoint{9.875598in}{1.573172in}}%
\pgfpathclose%
\pgfusepath{fill}%
\end{pgfscope}%
\begin{pgfscope}%
\pgfpathrectangle{\pgfqpoint{6.572727in}{0.474100in}}{\pgfqpoint{4.227273in}{3.318700in}}%
\pgfusepath{clip}%
\pgfsetbuttcap%
\pgfsetroundjoin%
\definecolor{currentfill}{rgb}{0.127568,0.566949,0.550556}%
\pgfsetfillcolor{currentfill}%
\pgfsetfillopacity{0.700000}%
\pgfsetlinewidth{0.000000pt}%
\definecolor{currentstroke}{rgb}{0.000000,0.000000,0.000000}%
\pgfsetstrokecolor{currentstroke}%
\pgfsetstrokeopacity{0.700000}%
\pgfsetdash{}{0pt}%
\pgfpathmoveto{\pgfqpoint{9.118270in}{1.198404in}}%
\pgfpathcurveto{\pgfqpoint{9.123314in}{1.198404in}}{\pgfqpoint{9.128151in}{1.200407in}}{\pgfqpoint{9.131718in}{1.203974in}}%
\pgfpathcurveto{\pgfqpoint{9.135284in}{1.207540in}}{\pgfqpoint{9.137288in}{1.212378in}}{\pgfqpoint{9.137288in}{1.217422in}}%
\pgfpathcurveto{\pgfqpoint{9.137288in}{1.222465in}}{\pgfqpoint{9.135284in}{1.227303in}}{\pgfqpoint{9.131718in}{1.230870in}}%
\pgfpathcurveto{\pgfqpoint{9.128151in}{1.234436in}}{\pgfqpoint{9.123314in}{1.236440in}}{\pgfqpoint{9.118270in}{1.236440in}}%
\pgfpathcurveto{\pgfqpoint{9.113226in}{1.236440in}}{\pgfqpoint{9.108389in}{1.234436in}}{\pgfqpoint{9.104822in}{1.230870in}}%
\pgfpathcurveto{\pgfqpoint{9.101256in}{1.227303in}}{\pgfqpoint{9.099252in}{1.222465in}}{\pgfqpoint{9.099252in}{1.217422in}}%
\pgfpathcurveto{\pgfqpoint{9.099252in}{1.212378in}}{\pgfqpoint{9.101256in}{1.207540in}}{\pgfqpoint{9.104822in}{1.203974in}}%
\pgfpathcurveto{\pgfqpoint{9.108389in}{1.200407in}}{\pgfqpoint{9.113226in}{1.198404in}}{\pgfqpoint{9.118270in}{1.198404in}}%
\pgfpathclose%
\pgfusepath{fill}%
\end{pgfscope}%
\begin{pgfscope}%
\pgfpathrectangle{\pgfqpoint{6.572727in}{0.474100in}}{\pgfqpoint{4.227273in}{3.318700in}}%
\pgfusepath{clip}%
\pgfsetbuttcap%
\pgfsetroundjoin%
\definecolor{currentfill}{rgb}{0.267004,0.004874,0.329415}%
\pgfsetfillcolor{currentfill}%
\pgfsetfillopacity{0.700000}%
\pgfsetlinewidth{0.000000pt}%
\definecolor{currentstroke}{rgb}{0.000000,0.000000,0.000000}%
\pgfsetstrokecolor{currentstroke}%
\pgfsetstrokeopacity{0.700000}%
\pgfsetdash{}{0pt}%
\pgfpathmoveto{\pgfqpoint{8.034419in}{1.573831in}}%
\pgfpathcurveto{\pgfqpoint{8.039463in}{1.573831in}}{\pgfqpoint{8.044301in}{1.575834in}}{\pgfqpoint{8.047867in}{1.579401in}}%
\pgfpathcurveto{\pgfqpoint{8.051433in}{1.582967in}}{\pgfqpoint{8.053437in}{1.587805in}}{\pgfqpoint{8.053437in}{1.592849in}}%
\pgfpathcurveto{\pgfqpoint{8.053437in}{1.597892in}}{\pgfqpoint{8.051433in}{1.602730in}}{\pgfqpoint{8.047867in}{1.606297in}}%
\pgfpathcurveto{\pgfqpoint{8.044301in}{1.609863in}}{\pgfqpoint{8.039463in}{1.611867in}}{\pgfqpoint{8.034419in}{1.611867in}}%
\pgfpathcurveto{\pgfqpoint{8.029375in}{1.611867in}}{\pgfqpoint{8.024538in}{1.609863in}}{\pgfqpoint{8.020971in}{1.606297in}}%
\pgfpathcurveto{\pgfqpoint{8.017405in}{1.602730in}}{\pgfqpoint{8.015401in}{1.597892in}}{\pgfqpoint{8.015401in}{1.592849in}}%
\pgfpathcurveto{\pgfqpoint{8.015401in}{1.587805in}}{\pgfqpoint{8.017405in}{1.582967in}}{\pgfqpoint{8.020971in}{1.579401in}}%
\pgfpathcurveto{\pgfqpoint{8.024538in}{1.575834in}}{\pgfqpoint{8.029375in}{1.573831in}}{\pgfqpoint{8.034419in}{1.573831in}}%
\pgfpathclose%
\pgfusepath{fill}%
\end{pgfscope}%
\begin{pgfscope}%
\pgfpathrectangle{\pgfqpoint{6.572727in}{0.474100in}}{\pgfqpoint{4.227273in}{3.318700in}}%
\pgfusepath{clip}%
\pgfsetbuttcap%
\pgfsetroundjoin%
\definecolor{currentfill}{rgb}{0.993248,0.906157,0.143936}%
\pgfsetfillcolor{currentfill}%
\pgfsetfillopacity{0.700000}%
\pgfsetlinewidth{0.000000pt}%
\definecolor{currentstroke}{rgb}{0.000000,0.000000,0.000000}%
\pgfsetstrokecolor{currentstroke}%
\pgfsetstrokeopacity{0.700000}%
\pgfsetdash{}{0pt}%
\pgfpathmoveto{\pgfqpoint{7.957217in}{2.776938in}}%
\pgfpathcurveto{\pgfqpoint{7.962261in}{2.776938in}}{\pgfqpoint{7.967098in}{2.778941in}}{\pgfqpoint{7.970665in}{2.782508in}}%
\pgfpathcurveto{\pgfqpoint{7.974231in}{2.786074in}}{\pgfqpoint{7.976235in}{2.790912in}}{\pgfqpoint{7.976235in}{2.795956in}}%
\pgfpathcurveto{\pgfqpoint{7.976235in}{2.800999in}}{\pgfqpoint{7.974231in}{2.805837in}}{\pgfqpoint{7.970665in}{2.809404in}}%
\pgfpathcurveto{\pgfqpoint{7.967098in}{2.812970in}}{\pgfqpoint{7.962261in}{2.814974in}}{\pgfqpoint{7.957217in}{2.814974in}}%
\pgfpathcurveto{\pgfqpoint{7.952173in}{2.814974in}}{\pgfqpoint{7.947335in}{2.812970in}}{\pgfqpoint{7.943769in}{2.809404in}}%
\pgfpathcurveto{\pgfqpoint{7.940203in}{2.805837in}}{\pgfqpoint{7.938199in}{2.800999in}}{\pgfqpoint{7.938199in}{2.795956in}}%
\pgfpathcurveto{\pgfqpoint{7.938199in}{2.790912in}}{\pgfqpoint{7.940203in}{2.786074in}}{\pgfqpoint{7.943769in}{2.782508in}}%
\pgfpathcurveto{\pgfqpoint{7.947335in}{2.778941in}}{\pgfqpoint{7.952173in}{2.776938in}}{\pgfqpoint{7.957217in}{2.776938in}}%
\pgfpathclose%
\pgfusepath{fill}%
\end{pgfscope}%
\begin{pgfscope}%
\pgfpathrectangle{\pgfqpoint{6.572727in}{0.474100in}}{\pgfqpoint{4.227273in}{3.318700in}}%
\pgfusepath{clip}%
\pgfsetbuttcap%
\pgfsetroundjoin%
\definecolor{currentfill}{rgb}{0.267004,0.004874,0.329415}%
\pgfsetfillcolor{currentfill}%
\pgfsetfillopacity{0.700000}%
\pgfsetlinewidth{0.000000pt}%
\definecolor{currentstroke}{rgb}{0.000000,0.000000,0.000000}%
\pgfsetstrokecolor{currentstroke}%
\pgfsetstrokeopacity{0.700000}%
\pgfsetdash{}{0pt}%
\pgfpathmoveto{\pgfqpoint{7.447726in}{1.621090in}}%
\pgfpathcurveto{\pgfqpoint{7.452769in}{1.621090in}}{\pgfqpoint{7.457607in}{1.623094in}}{\pgfqpoint{7.461173in}{1.626661in}}%
\pgfpathcurveto{\pgfqpoint{7.464740in}{1.630227in}}{\pgfqpoint{7.466744in}{1.635065in}}{\pgfqpoint{7.466744in}{1.640108in}}%
\pgfpathcurveto{\pgfqpoint{7.466744in}{1.645152in}}{\pgfqpoint{7.464740in}{1.649990in}}{\pgfqpoint{7.461173in}{1.653556in}}%
\pgfpathcurveto{\pgfqpoint{7.457607in}{1.657123in}}{\pgfqpoint{7.452769in}{1.659127in}}{\pgfqpoint{7.447726in}{1.659127in}}%
\pgfpathcurveto{\pgfqpoint{7.442682in}{1.659127in}}{\pgfqpoint{7.437844in}{1.657123in}}{\pgfqpoint{7.434278in}{1.653556in}}%
\pgfpathcurveto{\pgfqpoint{7.430711in}{1.649990in}}{\pgfqpoint{7.428707in}{1.645152in}}{\pgfqpoint{7.428707in}{1.640108in}}%
\pgfpathcurveto{\pgfqpoint{7.428707in}{1.635065in}}{\pgfqpoint{7.430711in}{1.630227in}}{\pgfqpoint{7.434278in}{1.626661in}}%
\pgfpathcurveto{\pgfqpoint{7.437844in}{1.623094in}}{\pgfqpoint{7.442682in}{1.621090in}}{\pgfqpoint{7.447726in}{1.621090in}}%
\pgfpathclose%
\pgfusepath{fill}%
\end{pgfscope}%
\begin{pgfscope}%
\pgfpathrectangle{\pgfqpoint{6.572727in}{0.474100in}}{\pgfqpoint{4.227273in}{3.318700in}}%
\pgfusepath{clip}%
\pgfsetbuttcap%
\pgfsetroundjoin%
\definecolor{currentfill}{rgb}{0.993248,0.906157,0.143936}%
\pgfsetfillcolor{currentfill}%
\pgfsetfillopacity{0.700000}%
\pgfsetlinewidth{0.000000pt}%
\definecolor{currentstroke}{rgb}{0.000000,0.000000,0.000000}%
\pgfsetstrokecolor{currentstroke}%
\pgfsetstrokeopacity{0.700000}%
\pgfsetdash{}{0pt}%
\pgfpathmoveto{\pgfqpoint{8.375764in}{3.050893in}}%
\pgfpathcurveto{\pgfqpoint{8.380807in}{3.050893in}}{\pgfqpoint{8.385645in}{3.052897in}}{\pgfqpoint{8.389211in}{3.056463in}}%
\pgfpathcurveto{\pgfqpoint{8.392778in}{3.060029in}}{\pgfqpoint{8.394782in}{3.064867in}}{\pgfqpoint{8.394782in}{3.069911in}}%
\pgfpathcurveto{\pgfqpoint{8.394782in}{3.074954in}}{\pgfqpoint{8.392778in}{3.079792in}}{\pgfqpoint{8.389211in}{3.083359in}}%
\pgfpathcurveto{\pgfqpoint{8.385645in}{3.086925in}}{\pgfqpoint{8.380807in}{3.088929in}}{\pgfqpoint{8.375764in}{3.088929in}}%
\pgfpathcurveto{\pgfqpoint{8.370720in}{3.088929in}}{\pgfqpoint{8.365882in}{3.086925in}}{\pgfqpoint{8.362316in}{3.083359in}}%
\pgfpathcurveto{\pgfqpoint{8.358749in}{3.079792in}}{\pgfqpoint{8.356745in}{3.074954in}}{\pgfqpoint{8.356745in}{3.069911in}}%
\pgfpathcurveto{\pgfqpoint{8.356745in}{3.064867in}}{\pgfqpoint{8.358749in}{3.060029in}}{\pgfqpoint{8.362316in}{3.056463in}}%
\pgfpathcurveto{\pgfqpoint{8.365882in}{3.052897in}}{\pgfqpoint{8.370720in}{3.050893in}}{\pgfqpoint{8.375764in}{3.050893in}}%
\pgfpathclose%
\pgfusepath{fill}%
\end{pgfscope}%
\begin{pgfscope}%
\pgfpathrectangle{\pgfqpoint{6.572727in}{0.474100in}}{\pgfqpoint{4.227273in}{3.318700in}}%
\pgfusepath{clip}%
\pgfsetbuttcap%
\pgfsetroundjoin%
\definecolor{currentfill}{rgb}{0.993248,0.906157,0.143936}%
\pgfsetfillcolor{currentfill}%
\pgfsetfillopacity{0.700000}%
\pgfsetlinewidth{0.000000pt}%
\definecolor{currentstroke}{rgb}{0.000000,0.000000,0.000000}%
\pgfsetstrokecolor{currentstroke}%
\pgfsetstrokeopacity{0.700000}%
\pgfsetdash{}{0pt}%
\pgfpathmoveto{\pgfqpoint{7.971135in}{2.536822in}}%
\pgfpathcurveto{\pgfqpoint{7.976178in}{2.536822in}}{\pgfqpoint{7.981016in}{2.538826in}}{\pgfqpoint{7.984582in}{2.542392in}}%
\pgfpathcurveto{\pgfqpoint{7.988149in}{2.545959in}}{\pgfqpoint{7.990153in}{2.550797in}}{\pgfqpoint{7.990153in}{2.555840in}}%
\pgfpathcurveto{\pgfqpoint{7.990153in}{2.560884in}}{\pgfqpoint{7.988149in}{2.565722in}}{\pgfqpoint{7.984582in}{2.569288in}}%
\pgfpathcurveto{\pgfqpoint{7.981016in}{2.572855in}}{\pgfqpoint{7.976178in}{2.574858in}}{\pgfqpoint{7.971135in}{2.574858in}}%
\pgfpathcurveto{\pgfqpoint{7.966091in}{2.574858in}}{\pgfqpoint{7.961253in}{2.572855in}}{\pgfqpoint{7.957687in}{2.569288in}}%
\pgfpathcurveto{\pgfqpoint{7.954120in}{2.565722in}}{\pgfqpoint{7.952116in}{2.560884in}}{\pgfqpoint{7.952116in}{2.555840in}}%
\pgfpathcurveto{\pgfqpoint{7.952116in}{2.550797in}}{\pgfqpoint{7.954120in}{2.545959in}}{\pgfqpoint{7.957687in}{2.542392in}}%
\pgfpathcurveto{\pgfqpoint{7.961253in}{2.538826in}}{\pgfqpoint{7.966091in}{2.536822in}}{\pgfqpoint{7.971135in}{2.536822in}}%
\pgfpathclose%
\pgfusepath{fill}%
\end{pgfscope}%
\begin{pgfscope}%
\pgfpathrectangle{\pgfqpoint{6.572727in}{0.474100in}}{\pgfqpoint{4.227273in}{3.318700in}}%
\pgfusepath{clip}%
\pgfsetbuttcap%
\pgfsetroundjoin%
\definecolor{currentfill}{rgb}{0.993248,0.906157,0.143936}%
\pgfsetfillcolor{currentfill}%
\pgfsetfillopacity{0.700000}%
\pgfsetlinewidth{0.000000pt}%
\definecolor{currentstroke}{rgb}{0.000000,0.000000,0.000000}%
\pgfsetstrokecolor{currentstroke}%
\pgfsetstrokeopacity{0.700000}%
\pgfsetdash{}{0pt}%
\pgfpathmoveto{\pgfqpoint{7.306662in}{3.033135in}}%
\pgfpathcurveto{\pgfqpoint{7.311706in}{3.033135in}}{\pgfqpoint{7.316544in}{3.035139in}}{\pgfqpoint{7.320110in}{3.038705in}}%
\pgfpathcurveto{\pgfqpoint{7.323677in}{3.042272in}}{\pgfqpoint{7.325681in}{3.047110in}}{\pgfqpoint{7.325681in}{3.052153in}}%
\pgfpathcurveto{\pgfqpoint{7.325681in}{3.057197in}}{\pgfqpoint{7.323677in}{3.062035in}}{\pgfqpoint{7.320110in}{3.065601in}}%
\pgfpathcurveto{\pgfqpoint{7.316544in}{3.069168in}}{\pgfqpoint{7.311706in}{3.071171in}}{\pgfqpoint{7.306662in}{3.071171in}}%
\pgfpathcurveto{\pgfqpoint{7.301619in}{3.071171in}}{\pgfqpoint{7.296781in}{3.069168in}}{\pgfqpoint{7.293215in}{3.065601in}}%
\pgfpathcurveto{\pgfqpoint{7.289648in}{3.062035in}}{\pgfqpoint{7.287644in}{3.057197in}}{\pgfqpoint{7.287644in}{3.052153in}}%
\pgfpathcurveto{\pgfqpoint{7.287644in}{3.047110in}}{\pgfqpoint{7.289648in}{3.042272in}}{\pgfqpoint{7.293215in}{3.038705in}}%
\pgfpathcurveto{\pgfqpoint{7.296781in}{3.035139in}}{\pgfqpoint{7.301619in}{3.033135in}}{\pgfqpoint{7.306662in}{3.033135in}}%
\pgfpathclose%
\pgfusepath{fill}%
\end{pgfscope}%
\begin{pgfscope}%
\pgfpathrectangle{\pgfqpoint{6.572727in}{0.474100in}}{\pgfqpoint{4.227273in}{3.318700in}}%
\pgfusepath{clip}%
\pgfsetbuttcap%
\pgfsetroundjoin%
\definecolor{currentfill}{rgb}{0.267004,0.004874,0.329415}%
\pgfsetfillcolor{currentfill}%
\pgfsetfillopacity{0.700000}%
\pgfsetlinewidth{0.000000pt}%
\definecolor{currentstroke}{rgb}{0.000000,0.000000,0.000000}%
\pgfsetstrokecolor{currentstroke}%
\pgfsetstrokeopacity{0.700000}%
\pgfsetdash{}{0pt}%
\pgfpathmoveto{\pgfqpoint{7.989772in}{1.855730in}}%
\pgfpathcurveto{\pgfqpoint{7.994815in}{1.855730in}}{\pgfqpoint{7.999653in}{1.857734in}}{\pgfqpoint{8.003220in}{1.861300in}}%
\pgfpathcurveto{\pgfqpoint{8.006786in}{1.864867in}}{\pgfqpoint{8.008790in}{1.869705in}}{\pgfqpoint{8.008790in}{1.874748in}}%
\pgfpathcurveto{\pgfqpoint{8.008790in}{1.879792in}}{\pgfqpoint{8.006786in}{1.884630in}}{\pgfqpoint{8.003220in}{1.888196in}}%
\pgfpathcurveto{\pgfqpoint{7.999653in}{1.891762in}}{\pgfqpoint{7.994815in}{1.893766in}}{\pgfqpoint{7.989772in}{1.893766in}}%
\pgfpathcurveto{\pgfqpoint{7.984728in}{1.893766in}}{\pgfqpoint{7.979890in}{1.891762in}}{\pgfqpoint{7.976324in}{1.888196in}}%
\pgfpathcurveto{\pgfqpoint{7.972757in}{1.884630in}}{\pgfqpoint{7.970754in}{1.879792in}}{\pgfqpoint{7.970754in}{1.874748in}}%
\pgfpathcurveto{\pgfqpoint{7.970754in}{1.869705in}}{\pgfqpoint{7.972757in}{1.864867in}}{\pgfqpoint{7.976324in}{1.861300in}}%
\pgfpathcurveto{\pgfqpoint{7.979890in}{1.857734in}}{\pgfqpoint{7.984728in}{1.855730in}}{\pgfqpoint{7.989772in}{1.855730in}}%
\pgfpathclose%
\pgfusepath{fill}%
\end{pgfscope}%
\begin{pgfscope}%
\pgfpathrectangle{\pgfqpoint{6.572727in}{0.474100in}}{\pgfqpoint{4.227273in}{3.318700in}}%
\pgfusepath{clip}%
\pgfsetbuttcap%
\pgfsetroundjoin%
\definecolor{currentfill}{rgb}{0.267004,0.004874,0.329415}%
\pgfsetfillcolor{currentfill}%
\pgfsetfillopacity{0.700000}%
\pgfsetlinewidth{0.000000pt}%
\definecolor{currentstroke}{rgb}{0.000000,0.000000,0.000000}%
\pgfsetstrokecolor{currentstroke}%
\pgfsetstrokeopacity{0.700000}%
\pgfsetdash{}{0pt}%
\pgfpathmoveto{\pgfqpoint{8.164127in}{0.869917in}}%
\pgfpathcurveto{\pgfqpoint{8.169171in}{0.869917in}}{\pgfqpoint{8.174009in}{0.871921in}}{\pgfqpoint{8.177575in}{0.875487in}}%
\pgfpathcurveto{\pgfqpoint{8.181142in}{0.879054in}}{\pgfqpoint{8.183145in}{0.883891in}}{\pgfqpoint{8.183145in}{0.888935in}}%
\pgfpathcurveto{\pgfqpoint{8.183145in}{0.893979in}}{\pgfqpoint{8.181142in}{0.898816in}}{\pgfqpoint{8.177575in}{0.902383in}}%
\pgfpathcurveto{\pgfqpoint{8.174009in}{0.905949in}}{\pgfqpoint{8.169171in}{0.907953in}}{\pgfqpoint{8.164127in}{0.907953in}}%
\pgfpathcurveto{\pgfqpoint{8.159084in}{0.907953in}}{\pgfqpoint{8.154246in}{0.905949in}}{\pgfqpoint{8.150679in}{0.902383in}}%
\pgfpathcurveto{\pgfqpoint{8.147113in}{0.898816in}}{\pgfqpoint{8.145109in}{0.893979in}}{\pgfqpoint{8.145109in}{0.888935in}}%
\pgfpathcurveto{\pgfqpoint{8.145109in}{0.883891in}}{\pgfqpoint{8.147113in}{0.879054in}}{\pgfqpoint{8.150679in}{0.875487in}}%
\pgfpathcurveto{\pgfqpoint{8.154246in}{0.871921in}}{\pgfqpoint{8.159084in}{0.869917in}}{\pgfqpoint{8.164127in}{0.869917in}}%
\pgfpathclose%
\pgfusepath{fill}%
\end{pgfscope}%
\begin{pgfscope}%
\pgfpathrectangle{\pgfqpoint{6.572727in}{0.474100in}}{\pgfqpoint{4.227273in}{3.318700in}}%
\pgfusepath{clip}%
\pgfsetbuttcap%
\pgfsetroundjoin%
\definecolor{currentfill}{rgb}{0.993248,0.906157,0.143936}%
\pgfsetfillcolor{currentfill}%
\pgfsetfillopacity{0.700000}%
\pgfsetlinewidth{0.000000pt}%
\definecolor{currentstroke}{rgb}{0.000000,0.000000,0.000000}%
\pgfsetstrokecolor{currentstroke}%
\pgfsetstrokeopacity{0.700000}%
\pgfsetdash{}{0pt}%
\pgfpathmoveto{\pgfqpoint{8.302321in}{2.891813in}}%
\pgfpathcurveto{\pgfqpoint{8.307365in}{2.891813in}}{\pgfqpoint{8.312202in}{2.893817in}}{\pgfqpoint{8.315769in}{2.897383in}}%
\pgfpathcurveto{\pgfqpoint{8.319335in}{2.900950in}}{\pgfqpoint{8.321339in}{2.905787in}}{\pgfqpoint{8.321339in}{2.910831in}}%
\pgfpathcurveto{\pgfqpoint{8.321339in}{2.915875in}}{\pgfqpoint{8.319335in}{2.920713in}}{\pgfqpoint{8.315769in}{2.924279in}}%
\pgfpathcurveto{\pgfqpoint{8.312202in}{2.927845in}}{\pgfqpoint{8.307365in}{2.929849in}}{\pgfqpoint{8.302321in}{2.929849in}}%
\pgfpathcurveto{\pgfqpoint{8.297277in}{2.929849in}}{\pgfqpoint{8.292440in}{2.927845in}}{\pgfqpoint{8.288873in}{2.924279in}}%
\pgfpathcurveto{\pgfqpoint{8.285307in}{2.920713in}}{\pgfqpoint{8.283303in}{2.915875in}}{\pgfqpoint{8.283303in}{2.910831in}}%
\pgfpathcurveto{\pgfqpoint{8.283303in}{2.905787in}}{\pgfqpoint{8.285307in}{2.900950in}}{\pgfqpoint{8.288873in}{2.897383in}}%
\pgfpathcurveto{\pgfqpoint{8.292440in}{2.893817in}}{\pgfqpoint{8.297277in}{2.891813in}}{\pgfqpoint{8.302321in}{2.891813in}}%
\pgfpathclose%
\pgfusepath{fill}%
\end{pgfscope}%
\begin{pgfscope}%
\pgfpathrectangle{\pgfqpoint{6.572727in}{0.474100in}}{\pgfqpoint{4.227273in}{3.318700in}}%
\pgfusepath{clip}%
\pgfsetbuttcap%
\pgfsetroundjoin%
\definecolor{currentfill}{rgb}{0.267004,0.004874,0.329415}%
\pgfsetfillcolor{currentfill}%
\pgfsetfillopacity{0.700000}%
\pgfsetlinewidth{0.000000pt}%
\definecolor{currentstroke}{rgb}{0.000000,0.000000,0.000000}%
\pgfsetstrokecolor{currentstroke}%
\pgfsetstrokeopacity{0.700000}%
\pgfsetdash{}{0pt}%
\pgfpathmoveto{\pgfqpoint{7.920534in}{1.620271in}}%
\pgfpathcurveto{\pgfqpoint{7.925577in}{1.620271in}}{\pgfqpoint{7.930415in}{1.622275in}}{\pgfqpoint{7.933982in}{1.625842in}}%
\pgfpathcurveto{\pgfqpoint{7.937548in}{1.629408in}}{\pgfqpoint{7.939552in}{1.634246in}}{\pgfqpoint{7.939552in}{1.639290in}}%
\pgfpathcurveto{\pgfqpoint{7.939552in}{1.644333in}}{\pgfqpoint{7.937548in}{1.649171in}}{\pgfqpoint{7.933982in}{1.652737in}}%
\pgfpathcurveto{\pgfqpoint{7.930415in}{1.656304in}}{\pgfqpoint{7.925577in}{1.658308in}}{\pgfqpoint{7.920534in}{1.658308in}}%
\pgfpathcurveto{\pgfqpoint{7.915490in}{1.658308in}}{\pgfqpoint{7.910652in}{1.656304in}}{\pgfqpoint{7.907086in}{1.652737in}}%
\pgfpathcurveto{\pgfqpoint{7.903520in}{1.649171in}}{\pgfqpoint{7.901516in}{1.644333in}}{\pgfqpoint{7.901516in}{1.639290in}}%
\pgfpathcurveto{\pgfqpoint{7.901516in}{1.634246in}}{\pgfqpoint{7.903520in}{1.629408in}}{\pgfqpoint{7.907086in}{1.625842in}}%
\pgfpathcurveto{\pgfqpoint{7.910652in}{1.622275in}}{\pgfqpoint{7.915490in}{1.620271in}}{\pgfqpoint{7.920534in}{1.620271in}}%
\pgfpathclose%
\pgfusepath{fill}%
\end{pgfscope}%
\begin{pgfscope}%
\pgfpathrectangle{\pgfqpoint{6.572727in}{0.474100in}}{\pgfqpoint{4.227273in}{3.318700in}}%
\pgfusepath{clip}%
\pgfsetbuttcap%
\pgfsetroundjoin%
\definecolor{currentfill}{rgb}{0.267004,0.004874,0.329415}%
\pgfsetfillcolor{currentfill}%
\pgfsetfillopacity{0.700000}%
\pgfsetlinewidth{0.000000pt}%
\definecolor{currentstroke}{rgb}{0.000000,0.000000,0.000000}%
\pgfsetstrokecolor{currentstroke}%
\pgfsetstrokeopacity{0.700000}%
\pgfsetdash{}{0pt}%
\pgfpathmoveto{\pgfqpoint{7.973924in}{1.447028in}}%
\pgfpathcurveto{\pgfqpoint{7.978968in}{1.447028in}}{\pgfqpoint{7.983806in}{1.449032in}}{\pgfqpoint{7.987372in}{1.452598in}}%
\pgfpathcurveto{\pgfqpoint{7.990938in}{1.456165in}}{\pgfqpoint{7.992942in}{1.461002in}}{\pgfqpoint{7.992942in}{1.466046in}}%
\pgfpathcurveto{\pgfqpoint{7.992942in}{1.471090in}}{\pgfqpoint{7.990938in}{1.475928in}}{\pgfqpoint{7.987372in}{1.479494in}}%
\pgfpathcurveto{\pgfqpoint{7.983806in}{1.483060in}}{\pgfqpoint{7.978968in}{1.485064in}}{\pgfqpoint{7.973924in}{1.485064in}}%
\pgfpathcurveto{\pgfqpoint{7.968880in}{1.485064in}}{\pgfqpoint{7.964043in}{1.483060in}}{\pgfqpoint{7.960476in}{1.479494in}}%
\pgfpathcurveto{\pgfqpoint{7.956910in}{1.475928in}}{\pgfqpoint{7.954906in}{1.471090in}}{\pgfqpoint{7.954906in}{1.466046in}}%
\pgfpathcurveto{\pgfqpoint{7.954906in}{1.461002in}}{\pgfqpoint{7.956910in}{1.456165in}}{\pgfqpoint{7.960476in}{1.452598in}}%
\pgfpathcurveto{\pgfqpoint{7.964043in}{1.449032in}}{\pgfqpoint{7.968880in}{1.447028in}}{\pgfqpoint{7.973924in}{1.447028in}}%
\pgfpathclose%
\pgfusepath{fill}%
\end{pgfscope}%
\begin{pgfscope}%
\pgfpathrectangle{\pgfqpoint{6.572727in}{0.474100in}}{\pgfqpoint{4.227273in}{3.318700in}}%
\pgfusepath{clip}%
\pgfsetbuttcap%
\pgfsetroundjoin%
\definecolor{currentfill}{rgb}{0.127568,0.566949,0.550556}%
\pgfsetfillcolor{currentfill}%
\pgfsetfillopacity{0.700000}%
\pgfsetlinewidth{0.000000pt}%
\definecolor{currentstroke}{rgb}{0.000000,0.000000,0.000000}%
\pgfsetstrokecolor{currentstroke}%
\pgfsetstrokeopacity{0.700000}%
\pgfsetdash{}{0pt}%
\pgfpathmoveto{\pgfqpoint{9.695945in}{2.223342in}}%
\pgfpathcurveto{\pgfqpoint{9.700988in}{2.223342in}}{\pgfqpoint{9.705826in}{2.225346in}}{\pgfqpoint{9.709392in}{2.228913in}}%
\pgfpathcurveto{\pgfqpoint{9.712959in}{2.232479in}}{\pgfqpoint{9.714963in}{2.237317in}}{\pgfqpoint{9.714963in}{2.242360in}}%
\pgfpathcurveto{\pgfqpoint{9.714963in}{2.247404in}}{\pgfqpoint{9.712959in}{2.252242in}}{\pgfqpoint{9.709392in}{2.255808in}}%
\pgfpathcurveto{\pgfqpoint{9.705826in}{2.259375in}}{\pgfqpoint{9.700988in}{2.261379in}}{\pgfqpoint{9.695945in}{2.261379in}}%
\pgfpathcurveto{\pgfqpoint{9.690901in}{2.261379in}}{\pgfqpoint{9.686063in}{2.259375in}}{\pgfqpoint{9.682497in}{2.255808in}}%
\pgfpathcurveto{\pgfqpoint{9.678930in}{2.252242in}}{\pgfqpoint{9.676926in}{2.247404in}}{\pgfqpoint{9.676926in}{2.242360in}}%
\pgfpathcurveto{\pgfqpoint{9.676926in}{2.237317in}}{\pgfqpoint{9.678930in}{2.232479in}}{\pgfqpoint{9.682497in}{2.228913in}}%
\pgfpathcurveto{\pgfqpoint{9.686063in}{2.225346in}}{\pgfqpoint{9.690901in}{2.223342in}}{\pgfqpoint{9.695945in}{2.223342in}}%
\pgfpathclose%
\pgfusepath{fill}%
\end{pgfscope}%
\begin{pgfscope}%
\pgfpathrectangle{\pgfqpoint{6.572727in}{0.474100in}}{\pgfqpoint{4.227273in}{3.318700in}}%
\pgfusepath{clip}%
\pgfsetbuttcap%
\pgfsetroundjoin%
\definecolor{currentfill}{rgb}{0.267004,0.004874,0.329415}%
\pgfsetfillcolor{currentfill}%
\pgfsetfillopacity{0.700000}%
\pgfsetlinewidth{0.000000pt}%
\definecolor{currentstroke}{rgb}{0.000000,0.000000,0.000000}%
\pgfsetstrokecolor{currentstroke}%
\pgfsetstrokeopacity{0.700000}%
\pgfsetdash{}{0pt}%
\pgfpathmoveto{\pgfqpoint{8.414693in}{1.290831in}}%
\pgfpathcurveto{\pgfqpoint{8.419737in}{1.290831in}}{\pgfqpoint{8.424574in}{1.292835in}}{\pgfqpoint{8.428141in}{1.296401in}}%
\pgfpathcurveto{\pgfqpoint{8.431707in}{1.299967in}}{\pgfqpoint{8.433711in}{1.304805in}}{\pgfqpoint{8.433711in}{1.309849in}}%
\pgfpathcurveto{\pgfqpoint{8.433711in}{1.314892in}}{\pgfqpoint{8.431707in}{1.319730in}}{\pgfqpoint{8.428141in}{1.323297in}}%
\pgfpathcurveto{\pgfqpoint{8.424574in}{1.326863in}}{\pgfqpoint{8.419737in}{1.328867in}}{\pgfqpoint{8.414693in}{1.328867in}}%
\pgfpathcurveto{\pgfqpoint{8.409649in}{1.328867in}}{\pgfqpoint{8.404811in}{1.326863in}}{\pgfqpoint{8.401245in}{1.323297in}}%
\pgfpathcurveto{\pgfqpoint{8.397679in}{1.319730in}}{\pgfqpoint{8.395675in}{1.314892in}}{\pgfqpoint{8.395675in}{1.309849in}}%
\pgfpathcurveto{\pgfqpoint{8.395675in}{1.304805in}}{\pgfqpoint{8.397679in}{1.299967in}}{\pgfqpoint{8.401245in}{1.296401in}}%
\pgfpathcurveto{\pgfqpoint{8.404811in}{1.292835in}}{\pgfqpoint{8.409649in}{1.290831in}}{\pgfqpoint{8.414693in}{1.290831in}}%
\pgfpathclose%
\pgfusepath{fill}%
\end{pgfscope}%
\begin{pgfscope}%
\pgfpathrectangle{\pgfqpoint{6.572727in}{0.474100in}}{\pgfqpoint{4.227273in}{3.318700in}}%
\pgfusepath{clip}%
\pgfsetbuttcap%
\pgfsetroundjoin%
\definecolor{currentfill}{rgb}{0.993248,0.906157,0.143936}%
\pgfsetfillcolor{currentfill}%
\pgfsetfillopacity{0.700000}%
\pgfsetlinewidth{0.000000pt}%
\definecolor{currentstroke}{rgb}{0.000000,0.000000,0.000000}%
\pgfsetstrokecolor{currentstroke}%
\pgfsetstrokeopacity{0.700000}%
\pgfsetdash{}{0pt}%
\pgfpathmoveto{\pgfqpoint{8.411561in}{2.200381in}}%
\pgfpathcurveto{\pgfqpoint{8.416604in}{2.200381in}}{\pgfqpoint{8.421442in}{2.202385in}}{\pgfqpoint{8.425009in}{2.205952in}}%
\pgfpathcurveto{\pgfqpoint{8.428575in}{2.209518in}}{\pgfqpoint{8.430579in}{2.214356in}}{\pgfqpoint{8.430579in}{2.219399in}}%
\pgfpathcurveto{\pgfqpoint{8.430579in}{2.224443in}}{\pgfqpoint{8.428575in}{2.229281in}}{\pgfqpoint{8.425009in}{2.232847in}}%
\pgfpathcurveto{\pgfqpoint{8.421442in}{2.236414in}}{\pgfqpoint{8.416604in}{2.238418in}}{\pgfqpoint{8.411561in}{2.238418in}}%
\pgfpathcurveto{\pgfqpoint{8.406517in}{2.238418in}}{\pgfqpoint{8.401679in}{2.236414in}}{\pgfqpoint{8.398113in}{2.232847in}}%
\pgfpathcurveto{\pgfqpoint{8.394547in}{2.229281in}}{\pgfqpoint{8.392543in}{2.224443in}}{\pgfqpoint{8.392543in}{2.219399in}}%
\pgfpathcurveto{\pgfqpoint{8.392543in}{2.214356in}}{\pgfqpoint{8.394547in}{2.209518in}}{\pgfqpoint{8.398113in}{2.205952in}}%
\pgfpathcurveto{\pgfqpoint{8.401679in}{2.202385in}}{\pgfqpoint{8.406517in}{2.200381in}}{\pgfqpoint{8.411561in}{2.200381in}}%
\pgfpathclose%
\pgfusepath{fill}%
\end{pgfscope}%
\begin{pgfscope}%
\pgfpathrectangle{\pgfqpoint{6.572727in}{0.474100in}}{\pgfqpoint{4.227273in}{3.318700in}}%
\pgfusepath{clip}%
\pgfsetbuttcap%
\pgfsetroundjoin%
\definecolor{currentfill}{rgb}{0.267004,0.004874,0.329415}%
\pgfsetfillcolor{currentfill}%
\pgfsetfillopacity{0.700000}%
\pgfsetlinewidth{0.000000pt}%
\definecolor{currentstroke}{rgb}{0.000000,0.000000,0.000000}%
\pgfsetstrokecolor{currentstroke}%
\pgfsetstrokeopacity{0.700000}%
\pgfsetdash{}{0pt}%
\pgfpathmoveto{\pgfqpoint{7.618611in}{1.585276in}}%
\pgfpathcurveto{\pgfqpoint{7.623655in}{1.585276in}}{\pgfqpoint{7.628493in}{1.587280in}}{\pgfqpoint{7.632059in}{1.590846in}}%
\pgfpathcurveto{\pgfqpoint{7.635625in}{1.594413in}}{\pgfqpoint{7.637629in}{1.599251in}}{\pgfqpoint{7.637629in}{1.604294in}}%
\pgfpathcurveto{\pgfqpoint{7.637629in}{1.609338in}}{\pgfqpoint{7.635625in}{1.614176in}}{\pgfqpoint{7.632059in}{1.617742in}}%
\pgfpathcurveto{\pgfqpoint{7.628493in}{1.621309in}}{\pgfqpoint{7.623655in}{1.623312in}}{\pgfqpoint{7.618611in}{1.623312in}}%
\pgfpathcurveto{\pgfqpoint{7.613568in}{1.623312in}}{\pgfqpoint{7.608730in}{1.621309in}}{\pgfqpoint{7.605163in}{1.617742in}}%
\pgfpathcurveto{\pgfqpoint{7.601597in}{1.614176in}}{\pgfqpoint{7.599593in}{1.609338in}}{\pgfqpoint{7.599593in}{1.604294in}}%
\pgfpathcurveto{\pgfqpoint{7.599593in}{1.599251in}}{\pgfqpoint{7.601597in}{1.594413in}}{\pgfqpoint{7.605163in}{1.590846in}}%
\pgfpathcurveto{\pgfqpoint{7.608730in}{1.587280in}}{\pgfqpoint{7.613568in}{1.585276in}}{\pgfqpoint{7.618611in}{1.585276in}}%
\pgfpathclose%
\pgfusepath{fill}%
\end{pgfscope}%
\begin{pgfscope}%
\pgfpathrectangle{\pgfqpoint{6.572727in}{0.474100in}}{\pgfqpoint{4.227273in}{3.318700in}}%
\pgfusepath{clip}%
\pgfsetbuttcap%
\pgfsetroundjoin%
\definecolor{currentfill}{rgb}{0.993248,0.906157,0.143936}%
\pgfsetfillcolor{currentfill}%
\pgfsetfillopacity{0.700000}%
\pgfsetlinewidth{0.000000pt}%
\definecolor{currentstroke}{rgb}{0.000000,0.000000,0.000000}%
\pgfsetstrokecolor{currentstroke}%
\pgfsetstrokeopacity{0.700000}%
\pgfsetdash{}{0pt}%
\pgfpathmoveto{\pgfqpoint{8.589034in}{3.287556in}}%
\pgfpathcurveto{\pgfqpoint{8.594078in}{3.287556in}}{\pgfqpoint{8.598916in}{3.289560in}}{\pgfqpoint{8.602482in}{3.293126in}}%
\pgfpathcurveto{\pgfqpoint{8.606049in}{3.296693in}}{\pgfqpoint{8.608052in}{3.301530in}}{\pgfqpoint{8.608052in}{3.306574in}}%
\pgfpathcurveto{\pgfqpoint{8.608052in}{3.311618in}}{\pgfqpoint{8.606049in}{3.316456in}}{\pgfqpoint{8.602482in}{3.320022in}}%
\pgfpathcurveto{\pgfqpoint{8.598916in}{3.323588in}}{\pgfqpoint{8.594078in}{3.325592in}}{\pgfqpoint{8.589034in}{3.325592in}}%
\pgfpathcurveto{\pgfqpoint{8.583991in}{3.325592in}}{\pgfqpoint{8.579153in}{3.323588in}}{\pgfqpoint{8.575586in}{3.320022in}}%
\pgfpathcurveto{\pgfqpoint{8.572020in}{3.316456in}}{\pgfqpoint{8.570016in}{3.311618in}}{\pgfqpoint{8.570016in}{3.306574in}}%
\pgfpathcurveto{\pgfqpoint{8.570016in}{3.301530in}}{\pgfqpoint{8.572020in}{3.296693in}}{\pgfqpoint{8.575586in}{3.293126in}}%
\pgfpathcurveto{\pgfqpoint{8.579153in}{3.289560in}}{\pgfqpoint{8.583991in}{3.287556in}}{\pgfqpoint{8.589034in}{3.287556in}}%
\pgfpathclose%
\pgfusepath{fill}%
\end{pgfscope}%
\begin{pgfscope}%
\pgfpathrectangle{\pgfqpoint{6.572727in}{0.474100in}}{\pgfqpoint{4.227273in}{3.318700in}}%
\pgfusepath{clip}%
\pgfsetbuttcap%
\pgfsetroundjoin%
\definecolor{currentfill}{rgb}{0.993248,0.906157,0.143936}%
\pgfsetfillcolor{currentfill}%
\pgfsetfillopacity{0.700000}%
\pgfsetlinewidth{0.000000pt}%
\definecolor{currentstroke}{rgb}{0.000000,0.000000,0.000000}%
\pgfsetstrokecolor{currentstroke}%
\pgfsetstrokeopacity{0.700000}%
\pgfsetdash{}{0pt}%
\pgfpathmoveto{\pgfqpoint{7.887958in}{2.919524in}}%
\pgfpathcurveto{\pgfqpoint{7.893001in}{2.919524in}}{\pgfqpoint{7.897839in}{2.921528in}}{\pgfqpoint{7.901406in}{2.925094in}}%
\pgfpathcurveto{\pgfqpoint{7.904972in}{2.928660in}}{\pgfqpoint{7.906976in}{2.933498in}}{\pgfqpoint{7.906976in}{2.938542in}}%
\pgfpathcurveto{\pgfqpoint{7.906976in}{2.943585in}}{\pgfqpoint{7.904972in}{2.948423in}}{\pgfqpoint{7.901406in}{2.951990in}}%
\pgfpathcurveto{\pgfqpoint{7.897839in}{2.955556in}}{\pgfqpoint{7.893001in}{2.957560in}}{\pgfqpoint{7.887958in}{2.957560in}}%
\pgfpathcurveto{\pgfqpoint{7.882914in}{2.957560in}}{\pgfqpoint{7.878076in}{2.955556in}}{\pgfqpoint{7.874510in}{2.951990in}}%
\pgfpathcurveto{\pgfqpoint{7.870944in}{2.948423in}}{\pgfqpoint{7.868940in}{2.943585in}}{\pgfqpoint{7.868940in}{2.938542in}}%
\pgfpathcurveto{\pgfqpoint{7.868940in}{2.933498in}}{\pgfqpoint{7.870944in}{2.928660in}}{\pgfqpoint{7.874510in}{2.925094in}}%
\pgfpathcurveto{\pgfqpoint{7.878076in}{2.921528in}}{\pgfqpoint{7.882914in}{2.919524in}}{\pgfqpoint{7.887958in}{2.919524in}}%
\pgfpathclose%
\pgfusepath{fill}%
\end{pgfscope}%
\begin{pgfscope}%
\pgfpathrectangle{\pgfqpoint{6.572727in}{0.474100in}}{\pgfqpoint{4.227273in}{3.318700in}}%
\pgfusepath{clip}%
\pgfsetbuttcap%
\pgfsetroundjoin%
\definecolor{currentfill}{rgb}{0.993248,0.906157,0.143936}%
\pgfsetfillcolor{currentfill}%
\pgfsetfillopacity{0.700000}%
\pgfsetlinewidth{0.000000pt}%
\definecolor{currentstroke}{rgb}{0.000000,0.000000,0.000000}%
\pgfsetstrokecolor{currentstroke}%
\pgfsetstrokeopacity{0.700000}%
\pgfsetdash{}{0pt}%
\pgfpathmoveto{\pgfqpoint{8.083421in}{2.874764in}}%
\pgfpathcurveto{\pgfqpoint{8.088464in}{2.874764in}}{\pgfqpoint{8.093302in}{2.876768in}}{\pgfqpoint{8.096868in}{2.880334in}}%
\pgfpathcurveto{\pgfqpoint{8.100435in}{2.883900in}}{\pgfqpoint{8.102439in}{2.888738in}}{\pgfqpoint{8.102439in}{2.893782in}}%
\pgfpathcurveto{\pgfqpoint{8.102439in}{2.898826in}}{\pgfqpoint{8.100435in}{2.903663in}}{\pgfqpoint{8.096868in}{2.907230in}}%
\pgfpathcurveto{\pgfqpoint{8.093302in}{2.910796in}}{\pgfqpoint{8.088464in}{2.912800in}}{\pgfqpoint{8.083421in}{2.912800in}}%
\pgfpathcurveto{\pgfqpoint{8.078377in}{2.912800in}}{\pgfqpoint{8.073539in}{2.910796in}}{\pgfqpoint{8.069973in}{2.907230in}}%
\pgfpathcurveto{\pgfqpoint{8.066406in}{2.903663in}}{\pgfqpoint{8.064402in}{2.898826in}}{\pgfqpoint{8.064402in}{2.893782in}}%
\pgfpathcurveto{\pgfqpoint{8.064402in}{2.888738in}}{\pgfqpoint{8.066406in}{2.883900in}}{\pgfqpoint{8.069973in}{2.880334in}}%
\pgfpathcurveto{\pgfqpoint{8.073539in}{2.876768in}}{\pgfqpoint{8.078377in}{2.874764in}}{\pgfqpoint{8.083421in}{2.874764in}}%
\pgfpathclose%
\pgfusepath{fill}%
\end{pgfscope}%
\begin{pgfscope}%
\pgfpathrectangle{\pgfqpoint{6.572727in}{0.474100in}}{\pgfqpoint{4.227273in}{3.318700in}}%
\pgfusepath{clip}%
\pgfsetbuttcap%
\pgfsetroundjoin%
\definecolor{currentfill}{rgb}{0.127568,0.566949,0.550556}%
\pgfsetfillcolor{currentfill}%
\pgfsetfillopacity{0.700000}%
\pgfsetlinewidth{0.000000pt}%
\definecolor{currentstroke}{rgb}{0.000000,0.000000,0.000000}%
\pgfsetstrokecolor{currentstroke}%
\pgfsetstrokeopacity{0.700000}%
\pgfsetdash{}{0pt}%
\pgfpathmoveto{\pgfqpoint{9.490124in}{1.760659in}}%
\pgfpathcurveto{\pgfqpoint{9.495167in}{1.760659in}}{\pgfqpoint{9.500005in}{1.762663in}}{\pgfqpoint{9.503572in}{1.766229in}}%
\pgfpathcurveto{\pgfqpoint{9.507138in}{1.769796in}}{\pgfqpoint{9.509142in}{1.774634in}}{\pgfqpoint{9.509142in}{1.779677in}}%
\pgfpathcurveto{\pgfqpoint{9.509142in}{1.784721in}}{\pgfqpoint{9.507138in}{1.789559in}}{\pgfqpoint{9.503572in}{1.793125in}}%
\pgfpathcurveto{\pgfqpoint{9.500005in}{1.796692in}}{\pgfqpoint{9.495167in}{1.798695in}}{\pgfqpoint{9.490124in}{1.798695in}}%
\pgfpathcurveto{\pgfqpoint{9.485080in}{1.798695in}}{\pgfqpoint{9.480242in}{1.796692in}}{\pgfqpoint{9.476676in}{1.793125in}}%
\pgfpathcurveto{\pgfqpoint{9.473109in}{1.789559in}}{\pgfqpoint{9.471106in}{1.784721in}}{\pgfqpoint{9.471106in}{1.779677in}}%
\pgfpathcurveto{\pgfqpoint{9.471106in}{1.774634in}}{\pgfqpoint{9.473109in}{1.769796in}}{\pgfqpoint{9.476676in}{1.766229in}}%
\pgfpathcurveto{\pgfqpoint{9.480242in}{1.762663in}}{\pgfqpoint{9.485080in}{1.760659in}}{\pgfqpoint{9.490124in}{1.760659in}}%
\pgfpathclose%
\pgfusepath{fill}%
\end{pgfscope}%
\begin{pgfscope}%
\pgfpathrectangle{\pgfqpoint{6.572727in}{0.474100in}}{\pgfqpoint{4.227273in}{3.318700in}}%
\pgfusepath{clip}%
\pgfsetbuttcap%
\pgfsetroundjoin%
\definecolor{currentfill}{rgb}{0.267004,0.004874,0.329415}%
\pgfsetfillcolor{currentfill}%
\pgfsetfillopacity{0.700000}%
\pgfsetlinewidth{0.000000pt}%
\definecolor{currentstroke}{rgb}{0.000000,0.000000,0.000000}%
\pgfsetstrokecolor{currentstroke}%
\pgfsetstrokeopacity{0.700000}%
\pgfsetdash{}{0pt}%
\pgfpathmoveto{\pgfqpoint{7.472172in}{1.564153in}}%
\pgfpathcurveto{\pgfqpoint{7.477216in}{1.564153in}}{\pgfqpoint{7.482054in}{1.566157in}}{\pgfqpoint{7.485620in}{1.569723in}}%
\pgfpathcurveto{\pgfqpoint{7.489187in}{1.573289in}}{\pgfqpoint{7.491191in}{1.578127in}}{\pgfqpoint{7.491191in}{1.583171in}}%
\pgfpathcurveto{\pgfqpoint{7.491191in}{1.588214in}}{\pgfqpoint{7.489187in}{1.593052in}}{\pgfqpoint{7.485620in}{1.596619in}}%
\pgfpathcurveto{\pgfqpoint{7.482054in}{1.600185in}}{\pgfqpoint{7.477216in}{1.602189in}}{\pgfqpoint{7.472172in}{1.602189in}}%
\pgfpathcurveto{\pgfqpoint{7.467129in}{1.602189in}}{\pgfqpoint{7.462291in}{1.600185in}}{\pgfqpoint{7.458725in}{1.596619in}}%
\pgfpathcurveto{\pgfqpoint{7.455158in}{1.593052in}}{\pgfqpoint{7.453154in}{1.588214in}}{\pgfqpoint{7.453154in}{1.583171in}}%
\pgfpathcurveto{\pgfqpoint{7.453154in}{1.578127in}}{\pgfqpoint{7.455158in}{1.573289in}}{\pgfqpoint{7.458725in}{1.569723in}}%
\pgfpathcurveto{\pgfqpoint{7.462291in}{1.566157in}}{\pgfqpoint{7.467129in}{1.564153in}}{\pgfqpoint{7.472172in}{1.564153in}}%
\pgfpathclose%
\pgfusepath{fill}%
\end{pgfscope}%
\begin{pgfscope}%
\pgfpathrectangle{\pgfqpoint{6.572727in}{0.474100in}}{\pgfqpoint{4.227273in}{3.318700in}}%
\pgfusepath{clip}%
\pgfsetbuttcap%
\pgfsetroundjoin%
\definecolor{currentfill}{rgb}{0.127568,0.566949,0.550556}%
\pgfsetfillcolor{currentfill}%
\pgfsetfillopacity{0.700000}%
\pgfsetlinewidth{0.000000pt}%
\definecolor{currentstroke}{rgb}{0.000000,0.000000,0.000000}%
\pgfsetstrokecolor{currentstroke}%
\pgfsetstrokeopacity{0.700000}%
\pgfsetdash{}{0pt}%
\pgfpathmoveto{\pgfqpoint{8.958383in}{1.694388in}}%
\pgfpathcurveto{\pgfqpoint{8.963427in}{1.694388in}}{\pgfqpoint{8.968265in}{1.696392in}}{\pgfqpoint{8.971831in}{1.699958in}}%
\pgfpathcurveto{\pgfqpoint{8.975398in}{1.703524in}}{\pgfqpoint{8.977402in}{1.708362in}}{\pgfqpoint{8.977402in}{1.713406in}}%
\pgfpathcurveto{\pgfqpoint{8.977402in}{1.718450in}}{\pgfqpoint{8.975398in}{1.723287in}}{\pgfqpoint{8.971831in}{1.726854in}}%
\pgfpathcurveto{\pgfqpoint{8.968265in}{1.730420in}}{\pgfqpoint{8.963427in}{1.732424in}}{\pgfqpoint{8.958383in}{1.732424in}}%
\pgfpathcurveto{\pgfqpoint{8.953340in}{1.732424in}}{\pgfqpoint{8.948502in}{1.730420in}}{\pgfqpoint{8.944936in}{1.726854in}}%
\pgfpathcurveto{\pgfqpoint{8.941369in}{1.723287in}}{\pgfqpoint{8.939365in}{1.718450in}}{\pgfqpoint{8.939365in}{1.713406in}}%
\pgfpathcurveto{\pgfqpoint{8.939365in}{1.708362in}}{\pgfqpoint{8.941369in}{1.703524in}}{\pgfqpoint{8.944936in}{1.699958in}}%
\pgfpathcurveto{\pgfqpoint{8.948502in}{1.696392in}}{\pgfqpoint{8.953340in}{1.694388in}}{\pgfqpoint{8.958383in}{1.694388in}}%
\pgfpathclose%
\pgfusepath{fill}%
\end{pgfscope}%
\begin{pgfscope}%
\pgfpathrectangle{\pgfqpoint{6.572727in}{0.474100in}}{\pgfqpoint{4.227273in}{3.318700in}}%
\pgfusepath{clip}%
\pgfsetbuttcap%
\pgfsetroundjoin%
\definecolor{currentfill}{rgb}{0.127568,0.566949,0.550556}%
\pgfsetfillcolor{currentfill}%
\pgfsetfillopacity{0.700000}%
\pgfsetlinewidth{0.000000pt}%
\definecolor{currentstroke}{rgb}{0.000000,0.000000,0.000000}%
\pgfsetstrokecolor{currentstroke}%
\pgfsetstrokeopacity{0.700000}%
\pgfsetdash{}{0pt}%
\pgfpathmoveto{\pgfqpoint{9.053885in}{1.734843in}}%
\pgfpathcurveto{\pgfqpoint{9.058929in}{1.734843in}}{\pgfqpoint{9.063766in}{1.736846in}}{\pgfqpoint{9.067333in}{1.740413in}}%
\pgfpathcurveto{\pgfqpoint{9.070899in}{1.743979in}}{\pgfqpoint{9.072903in}{1.748817in}}{\pgfqpoint{9.072903in}{1.753861in}}%
\pgfpathcurveto{\pgfqpoint{9.072903in}{1.758904in}}{\pgfqpoint{9.070899in}{1.763742in}}{\pgfqpoint{9.067333in}{1.767309in}}%
\pgfpathcurveto{\pgfqpoint{9.063766in}{1.770875in}}{\pgfqpoint{9.058929in}{1.772879in}}{\pgfqpoint{9.053885in}{1.772879in}}%
\pgfpathcurveto{\pgfqpoint{9.048841in}{1.772879in}}{\pgfqpoint{9.044004in}{1.770875in}}{\pgfqpoint{9.040437in}{1.767309in}}%
\pgfpathcurveto{\pgfqpoint{9.036871in}{1.763742in}}{\pgfqpoint{9.034867in}{1.758904in}}{\pgfqpoint{9.034867in}{1.753861in}}%
\pgfpathcurveto{\pgfqpoint{9.034867in}{1.748817in}}{\pgfqpoint{9.036871in}{1.743979in}}{\pgfqpoint{9.040437in}{1.740413in}}%
\pgfpathcurveto{\pgfqpoint{9.044004in}{1.736846in}}{\pgfqpoint{9.048841in}{1.734843in}}{\pgfqpoint{9.053885in}{1.734843in}}%
\pgfpathclose%
\pgfusepath{fill}%
\end{pgfscope}%
\begin{pgfscope}%
\pgfpathrectangle{\pgfqpoint{6.572727in}{0.474100in}}{\pgfqpoint{4.227273in}{3.318700in}}%
\pgfusepath{clip}%
\pgfsetbuttcap%
\pgfsetroundjoin%
\definecolor{currentfill}{rgb}{0.127568,0.566949,0.550556}%
\pgfsetfillcolor{currentfill}%
\pgfsetfillopacity{0.700000}%
\pgfsetlinewidth{0.000000pt}%
\definecolor{currentstroke}{rgb}{0.000000,0.000000,0.000000}%
\pgfsetstrokecolor{currentstroke}%
\pgfsetstrokeopacity{0.700000}%
\pgfsetdash{}{0pt}%
\pgfpathmoveto{\pgfqpoint{9.016784in}{2.025438in}}%
\pgfpathcurveto{\pgfqpoint{9.021828in}{2.025438in}}{\pgfqpoint{9.026666in}{2.027442in}}{\pgfqpoint{9.030232in}{2.031009in}}%
\pgfpathcurveto{\pgfqpoint{9.033799in}{2.034575in}}{\pgfqpoint{9.035802in}{2.039413in}}{\pgfqpoint{9.035802in}{2.044457in}}%
\pgfpathcurveto{\pgfqpoint{9.035802in}{2.049500in}}{\pgfqpoint{9.033799in}{2.054338in}}{\pgfqpoint{9.030232in}{2.057904in}}%
\pgfpathcurveto{\pgfqpoint{9.026666in}{2.061471in}}{\pgfqpoint{9.021828in}{2.063475in}}{\pgfqpoint{9.016784in}{2.063475in}}%
\pgfpathcurveto{\pgfqpoint{9.011741in}{2.063475in}}{\pgfqpoint{9.006903in}{2.061471in}}{\pgfqpoint{9.003336in}{2.057904in}}%
\pgfpathcurveto{\pgfqpoint{8.999770in}{2.054338in}}{\pgfqpoint{8.997766in}{2.049500in}}{\pgfqpoint{8.997766in}{2.044457in}}%
\pgfpathcurveto{\pgfqpoint{8.997766in}{2.039413in}}{\pgfqpoint{8.999770in}{2.034575in}}{\pgfqpoint{9.003336in}{2.031009in}}%
\pgfpathcurveto{\pgfqpoint{9.006903in}{2.027442in}}{\pgfqpoint{9.011741in}{2.025438in}}{\pgfqpoint{9.016784in}{2.025438in}}%
\pgfpathclose%
\pgfusepath{fill}%
\end{pgfscope}%
\begin{pgfscope}%
\pgfpathrectangle{\pgfqpoint{6.572727in}{0.474100in}}{\pgfqpoint{4.227273in}{3.318700in}}%
\pgfusepath{clip}%
\pgfsetbuttcap%
\pgfsetroundjoin%
\definecolor{currentfill}{rgb}{0.267004,0.004874,0.329415}%
\pgfsetfillcolor{currentfill}%
\pgfsetfillopacity{0.700000}%
\pgfsetlinewidth{0.000000pt}%
\definecolor{currentstroke}{rgb}{0.000000,0.000000,0.000000}%
\pgfsetstrokecolor{currentstroke}%
\pgfsetstrokeopacity{0.700000}%
\pgfsetdash{}{0pt}%
\pgfpathmoveto{\pgfqpoint{7.885119in}{1.287685in}}%
\pgfpathcurveto{\pgfqpoint{7.890163in}{1.287685in}}{\pgfqpoint{7.895000in}{1.289688in}}{\pgfqpoint{7.898567in}{1.293255in}}%
\pgfpathcurveto{\pgfqpoint{7.902133in}{1.296821in}}{\pgfqpoint{7.904137in}{1.301659in}}{\pgfqpoint{7.904137in}{1.306703in}}%
\pgfpathcurveto{\pgfqpoint{7.904137in}{1.311746in}}{\pgfqpoint{7.902133in}{1.316584in}}{\pgfqpoint{7.898567in}{1.320151in}}%
\pgfpathcurveto{\pgfqpoint{7.895000in}{1.323717in}}{\pgfqpoint{7.890163in}{1.325721in}}{\pgfqpoint{7.885119in}{1.325721in}}%
\pgfpathcurveto{\pgfqpoint{7.880075in}{1.325721in}}{\pgfqpoint{7.875238in}{1.323717in}}{\pgfqpoint{7.871671in}{1.320151in}}%
\pgfpathcurveto{\pgfqpoint{7.868105in}{1.316584in}}{\pgfqpoint{7.866101in}{1.311746in}}{\pgfqpoint{7.866101in}{1.306703in}}%
\pgfpathcurveto{\pgfqpoint{7.866101in}{1.301659in}}{\pgfqpoint{7.868105in}{1.296821in}}{\pgfqpoint{7.871671in}{1.293255in}}%
\pgfpathcurveto{\pgfqpoint{7.875238in}{1.289688in}}{\pgfqpoint{7.880075in}{1.287685in}}{\pgfqpoint{7.885119in}{1.287685in}}%
\pgfpathclose%
\pgfusepath{fill}%
\end{pgfscope}%
\begin{pgfscope}%
\pgfpathrectangle{\pgfqpoint{6.572727in}{0.474100in}}{\pgfqpoint{4.227273in}{3.318700in}}%
\pgfusepath{clip}%
\pgfsetbuttcap%
\pgfsetroundjoin%
\definecolor{currentfill}{rgb}{0.993248,0.906157,0.143936}%
\pgfsetfillcolor{currentfill}%
\pgfsetfillopacity{0.700000}%
\pgfsetlinewidth{0.000000pt}%
\definecolor{currentstroke}{rgb}{0.000000,0.000000,0.000000}%
\pgfsetstrokecolor{currentstroke}%
\pgfsetstrokeopacity{0.700000}%
\pgfsetdash{}{0pt}%
\pgfpathmoveto{\pgfqpoint{7.830735in}{2.568934in}}%
\pgfpathcurveto{\pgfqpoint{7.835779in}{2.568934in}}{\pgfqpoint{7.840616in}{2.570938in}}{\pgfqpoint{7.844183in}{2.574504in}}%
\pgfpathcurveto{\pgfqpoint{7.847749in}{2.578071in}}{\pgfqpoint{7.849753in}{2.582908in}}{\pgfqpoint{7.849753in}{2.587952in}}%
\pgfpathcurveto{\pgfqpoint{7.849753in}{2.592996in}}{\pgfqpoint{7.847749in}{2.597834in}}{\pgfqpoint{7.844183in}{2.601400in}}%
\pgfpathcurveto{\pgfqpoint{7.840616in}{2.604966in}}{\pgfqpoint{7.835779in}{2.606970in}}{\pgfqpoint{7.830735in}{2.606970in}}%
\pgfpathcurveto{\pgfqpoint{7.825691in}{2.606970in}}{\pgfqpoint{7.820853in}{2.604966in}}{\pgfqpoint{7.817287in}{2.601400in}}%
\pgfpathcurveto{\pgfqpoint{7.813721in}{2.597834in}}{\pgfqpoint{7.811717in}{2.592996in}}{\pgfqpoint{7.811717in}{2.587952in}}%
\pgfpathcurveto{\pgfqpoint{7.811717in}{2.582908in}}{\pgfqpoint{7.813721in}{2.578071in}}{\pgfqpoint{7.817287in}{2.574504in}}%
\pgfpathcurveto{\pgfqpoint{7.820853in}{2.570938in}}{\pgfqpoint{7.825691in}{2.568934in}}{\pgfqpoint{7.830735in}{2.568934in}}%
\pgfpathclose%
\pgfusepath{fill}%
\end{pgfscope}%
\begin{pgfscope}%
\pgfpathrectangle{\pgfqpoint{6.572727in}{0.474100in}}{\pgfqpoint{4.227273in}{3.318700in}}%
\pgfusepath{clip}%
\pgfsetbuttcap%
\pgfsetroundjoin%
\definecolor{currentfill}{rgb}{0.267004,0.004874,0.329415}%
\pgfsetfillcolor{currentfill}%
\pgfsetfillopacity{0.700000}%
\pgfsetlinewidth{0.000000pt}%
\definecolor{currentstroke}{rgb}{0.000000,0.000000,0.000000}%
\pgfsetstrokecolor{currentstroke}%
\pgfsetstrokeopacity{0.700000}%
\pgfsetdash{}{0pt}%
\pgfpathmoveto{\pgfqpoint{7.642837in}{1.536358in}}%
\pgfpathcurveto{\pgfqpoint{7.647881in}{1.536358in}}{\pgfqpoint{7.652719in}{1.538361in}}{\pgfqpoint{7.656285in}{1.541928in}}%
\pgfpathcurveto{\pgfqpoint{7.659851in}{1.545494in}}{\pgfqpoint{7.661855in}{1.550332in}}{\pgfqpoint{7.661855in}{1.555376in}}%
\pgfpathcurveto{\pgfqpoint{7.661855in}{1.560419in}}{\pgfqpoint{7.659851in}{1.565257in}}{\pgfqpoint{7.656285in}{1.568824in}}%
\pgfpathcurveto{\pgfqpoint{7.652719in}{1.572390in}}{\pgfqpoint{7.647881in}{1.574394in}}{\pgfqpoint{7.642837in}{1.574394in}}%
\pgfpathcurveto{\pgfqpoint{7.637793in}{1.574394in}}{\pgfqpoint{7.632956in}{1.572390in}}{\pgfqpoint{7.629389in}{1.568824in}}%
\pgfpathcurveto{\pgfqpoint{7.625823in}{1.565257in}}{\pgfqpoint{7.623819in}{1.560419in}}{\pgfqpoint{7.623819in}{1.555376in}}%
\pgfpathcurveto{\pgfqpoint{7.623819in}{1.550332in}}{\pgfqpoint{7.625823in}{1.545494in}}{\pgfqpoint{7.629389in}{1.541928in}}%
\pgfpathcurveto{\pgfqpoint{7.632956in}{1.538361in}}{\pgfqpoint{7.637793in}{1.536358in}}{\pgfqpoint{7.642837in}{1.536358in}}%
\pgfpathclose%
\pgfusepath{fill}%
\end{pgfscope}%
\begin{pgfscope}%
\pgfpathrectangle{\pgfqpoint{6.572727in}{0.474100in}}{\pgfqpoint{4.227273in}{3.318700in}}%
\pgfusepath{clip}%
\pgfsetbuttcap%
\pgfsetroundjoin%
\definecolor{currentfill}{rgb}{0.993248,0.906157,0.143936}%
\pgfsetfillcolor{currentfill}%
\pgfsetfillopacity{0.700000}%
\pgfsetlinewidth{0.000000pt}%
\definecolor{currentstroke}{rgb}{0.000000,0.000000,0.000000}%
\pgfsetstrokecolor{currentstroke}%
\pgfsetstrokeopacity{0.700000}%
\pgfsetdash{}{0pt}%
\pgfpathmoveto{\pgfqpoint{8.592896in}{2.761975in}}%
\pgfpathcurveto{\pgfqpoint{8.597940in}{2.761975in}}{\pgfqpoint{8.602778in}{2.763979in}}{\pgfqpoint{8.606344in}{2.767545in}}%
\pgfpathcurveto{\pgfqpoint{8.609910in}{2.771111in}}{\pgfqpoint{8.611914in}{2.775949in}}{\pgfqpoint{8.611914in}{2.780993in}}%
\pgfpathcurveto{\pgfqpoint{8.611914in}{2.786037in}}{\pgfqpoint{8.609910in}{2.790874in}}{\pgfqpoint{8.606344in}{2.794441in}}%
\pgfpathcurveto{\pgfqpoint{8.602778in}{2.798007in}}{\pgfqpoint{8.597940in}{2.800011in}}{\pgfqpoint{8.592896in}{2.800011in}}%
\pgfpathcurveto{\pgfqpoint{8.587853in}{2.800011in}}{\pgfqpoint{8.583015in}{2.798007in}}{\pgfqpoint{8.579448in}{2.794441in}}%
\pgfpathcurveto{\pgfqpoint{8.575882in}{2.790874in}}{\pgfqpoint{8.573878in}{2.786037in}}{\pgfqpoint{8.573878in}{2.780993in}}%
\pgfpathcurveto{\pgfqpoint{8.573878in}{2.775949in}}{\pgfqpoint{8.575882in}{2.771111in}}{\pgfqpoint{8.579448in}{2.767545in}}%
\pgfpathcurveto{\pgfqpoint{8.583015in}{2.763979in}}{\pgfqpoint{8.587853in}{2.761975in}}{\pgfqpoint{8.592896in}{2.761975in}}%
\pgfpathclose%
\pgfusepath{fill}%
\end{pgfscope}%
\begin{pgfscope}%
\pgfpathrectangle{\pgfqpoint{6.572727in}{0.474100in}}{\pgfqpoint{4.227273in}{3.318700in}}%
\pgfusepath{clip}%
\pgfsetbuttcap%
\pgfsetroundjoin%
\definecolor{currentfill}{rgb}{0.993248,0.906157,0.143936}%
\pgfsetfillcolor{currentfill}%
\pgfsetfillopacity{0.700000}%
\pgfsetlinewidth{0.000000pt}%
\definecolor{currentstroke}{rgb}{0.000000,0.000000,0.000000}%
\pgfsetstrokecolor{currentstroke}%
\pgfsetstrokeopacity{0.700000}%
\pgfsetdash{}{0pt}%
\pgfpathmoveto{\pgfqpoint{8.318046in}{2.796986in}}%
\pgfpathcurveto{\pgfqpoint{8.323090in}{2.796986in}}{\pgfqpoint{8.327928in}{2.798990in}}{\pgfqpoint{8.331494in}{2.802556in}}%
\pgfpathcurveto{\pgfqpoint{8.335061in}{2.806122in}}{\pgfqpoint{8.337065in}{2.810960in}}{\pgfqpoint{8.337065in}{2.816004in}}%
\pgfpathcurveto{\pgfqpoint{8.337065in}{2.821048in}}{\pgfqpoint{8.335061in}{2.825885in}}{\pgfqpoint{8.331494in}{2.829452in}}%
\pgfpathcurveto{\pgfqpoint{8.327928in}{2.833018in}}{\pgfqpoint{8.323090in}{2.835022in}}{\pgfqpoint{8.318046in}{2.835022in}}%
\pgfpathcurveto{\pgfqpoint{8.313003in}{2.835022in}}{\pgfqpoint{8.308165in}{2.833018in}}{\pgfqpoint{8.304599in}{2.829452in}}%
\pgfpathcurveto{\pgfqpoint{8.301032in}{2.825885in}}{\pgfqpoint{8.299028in}{2.821048in}}{\pgfqpoint{8.299028in}{2.816004in}}%
\pgfpathcurveto{\pgfqpoint{8.299028in}{2.810960in}}{\pgfqpoint{8.301032in}{2.806122in}}{\pgfqpoint{8.304599in}{2.802556in}}%
\pgfpathcurveto{\pgfqpoint{8.308165in}{2.798990in}}{\pgfqpoint{8.313003in}{2.796986in}}{\pgfqpoint{8.318046in}{2.796986in}}%
\pgfpathclose%
\pgfusepath{fill}%
\end{pgfscope}%
\begin{pgfscope}%
\pgfpathrectangle{\pgfqpoint{6.572727in}{0.474100in}}{\pgfqpoint{4.227273in}{3.318700in}}%
\pgfusepath{clip}%
\pgfsetbuttcap%
\pgfsetroundjoin%
\definecolor{currentfill}{rgb}{0.993248,0.906157,0.143936}%
\pgfsetfillcolor{currentfill}%
\pgfsetfillopacity{0.700000}%
\pgfsetlinewidth{0.000000pt}%
\definecolor{currentstroke}{rgb}{0.000000,0.000000,0.000000}%
\pgfsetstrokecolor{currentstroke}%
\pgfsetstrokeopacity{0.700000}%
\pgfsetdash{}{0pt}%
\pgfpathmoveto{\pgfqpoint{8.800912in}{2.999439in}}%
\pgfpathcurveto{\pgfqpoint{8.805956in}{2.999439in}}{\pgfqpoint{8.810794in}{3.001443in}}{\pgfqpoint{8.814360in}{3.005009in}}%
\pgfpathcurveto{\pgfqpoint{8.817926in}{3.008575in}}{\pgfqpoint{8.819930in}{3.013413in}}{\pgfqpoint{8.819930in}{3.018457in}}%
\pgfpathcurveto{\pgfqpoint{8.819930in}{3.023500in}}{\pgfqpoint{8.817926in}{3.028338in}}{\pgfqpoint{8.814360in}{3.031905in}}%
\pgfpathcurveto{\pgfqpoint{8.810794in}{3.035471in}}{\pgfqpoint{8.805956in}{3.037475in}}{\pgfqpoint{8.800912in}{3.037475in}}%
\pgfpathcurveto{\pgfqpoint{8.795869in}{3.037475in}}{\pgfqpoint{8.791031in}{3.035471in}}{\pgfqpoint{8.787464in}{3.031905in}}%
\pgfpathcurveto{\pgfqpoint{8.783898in}{3.028338in}}{\pgfqpoint{8.781894in}{3.023500in}}{\pgfqpoint{8.781894in}{3.018457in}}%
\pgfpathcurveto{\pgfqpoint{8.781894in}{3.013413in}}{\pgfqpoint{8.783898in}{3.008575in}}{\pgfqpoint{8.787464in}{3.005009in}}%
\pgfpathcurveto{\pgfqpoint{8.791031in}{3.001443in}}{\pgfqpoint{8.795869in}{2.999439in}}{\pgfqpoint{8.800912in}{2.999439in}}%
\pgfpathclose%
\pgfusepath{fill}%
\end{pgfscope}%
\begin{pgfscope}%
\pgfpathrectangle{\pgfqpoint{6.572727in}{0.474100in}}{\pgfqpoint{4.227273in}{3.318700in}}%
\pgfusepath{clip}%
\pgfsetbuttcap%
\pgfsetroundjoin%
\definecolor{currentfill}{rgb}{0.993248,0.906157,0.143936}%
\pgfsetfillcolor{currentfill}%
\pgfsetfillopacity{0.700000}%
\pgfsetlinewidth{0.000000pt}%
\definecolor{currentstroke}{rgb}{0.000000,0.000000,0.000000}%
\pgfsetstrokecolor{currentstroke}%
\pgfsetstrokeopacity{0.700000}%
\pgfsetdash{}{0pt}%
\pgfpathmoveto{\pgfqpoint{8.195678in}{3.175359in}}%
\pgfpathcurveto{\pgfqpoint{8.200722in}{3.175359in}}{\pgfqpoint{8.205560in}{3.177363in}}{\pgfqpoint{8.209126in}{3.180929in}}%
\pgfpathcurveto{\pgfqpoint{8.212693in}{3.184495in}}{\pgfqpoint{8.214696in}{3.189333in}}{\pgfqpoint{8.214696in}{3.194377in}}%
\pgfpathcurveto{\pgfqpoint{8.214696in}{3.199420in}}{\pgfqpoint{8.212693in}{3.204258in}}{\pgfqpoint{8.209126in}{3.207825in}}%
\pgfpathcurveto{\pgfqpoint{8.205560in}{3.211391in}}{\pgfqpoint{8.200722in}{3.213395in}}{\pgfqpoint{8.195678in}{3.213395in}}%
\pgfpathcurveto{\pgfqpoint{8.190635in}{3.213395in}}{\pgfqpoint{8.185797in}{3.211391in}}{\pgfqpoint{8.182230in}{3.207825in}}%
\pgfpathcurveto{\pgfqpoint{8.178664in}{3.204258in}}{\pgfqpoint{8.176660in}{3.199420in}}{\pgfqpoint{8.176660in}{3.194377in}}%
\pgfpathcurveto{\pgfqpoint{8.176660in}{3.189333in}}{\pgfqpoint{8.178664in}{3.184495in}}{\pgfqpoint{8.182230in}{3.180929in}}%
\pgfpathcurveto{\pgfqpoint{8.185797in}{3.177363in}}{\pgfqpoint{8.190635in}{3.175359in}}{\pgfqpoint{8.195678in}{3.175359in}}%
\pgfpathclose%
\pgfusepath{fill}%
\end{pgfscope}%
\begin{pgfscope}%
\pgfpathrectangle{\pgfqpoint{6.572727in}{0.474100in}}{\pgfqpoint{4.227273in}{3.318700in}}%
\pgfusepath{clip}%
\pgfsetbuttcap%
\pgfsetroundjoin%
\definecolor{currentfill}{rgb}{0.993248,0.906157,0.143936}%
\pgfsetfillcolor{currentfill}%
\pgfsetfillopacity{0.700000}%
\pgfsetlinewidth{0.000000pt}%
\definecolor{currentstroke}{rgb}{0.000000,0.000000,0.000000}%
\pgfsetstrokecolor{currentstroke}%
\pgfsetstrokeopacity{0.700000}%
\pgfsetdash{}{0pt}%
\pgfpathmoveto{\pgfqpoint{8.146326in}{3.083046in}}%
\pgfpathcurveto{\pgfqpoint{8.151370in}{3.083046in}}{\pgfqpoint{8.156208in}{3.085049in}}{\pgfqpoint{8.159774in}{3.088616in}}%
\pgfpathcurveto{\pgfqpoint{8.163341in}{3.092182in}}{\pgfqpoint{8.165345in}{3.097020in}}{\pgfqpoint{8.165345in}{3.102064in}}%
\pgfpathcurveto{\pgfqpoint{8.165345in}{3.107107in}}{\pgfqpoint{8.163341in}{3.111945in}}{\pgfqpoint{8.159774in}{3.115512in}}%
\pgfpathcurveto{\pgfqpoint{8.156208in}{3.119078in}}{\pgfqpoint{8.151370in}{3.121082in}}{\pgfqpoint{8.146326in}{3.121082in}}%
\pgfpathcurveto{\pgfqpoint{8.141283in}{3.121082in}}{\pgfqpoint{8.136445in}{3.119078in}}{\pgfqpoint{8.132879in}{3.115512in}}%
\pgfpathcurveto{\pgfqpoint{8.129312in}{3.111945in}}{\pgfqpoint{8.127308in}{3.107107in}}{\pgfqpoint{8.127308in}{3.102064in}}%
\pgfpathcurveto{\pgfqpoint{8.127308in}{3.097020in}}{\pgfqpoint{8.129312in}{3.092182in}}{\pgfqpoint{8.132879in}{3.088616in}}%
\pgfpathcurveto{\pgfqpoint{8.136445in}{3.085049in}}{\pgfqpoint{8.141283in}{3.083046in}}{\pgfqpoint{8.146326in}{3.083046in}}%
\pgfpathclose%
\pgfusepath{fill}%
\end{pgfscope}%
\begin{pgfscope}%
\pgfpathrectangle{\pgfqpoint{6.572727in}{0.474100in}}{\pgfqpoint{4.227273in}{3.318700in}}%
\pgfusepath{clip}%
\pgfsetbuttcap%
\pgfsetroundjoin%
\definecolor{currentfill}{rgb}{0.993248,0.906157,0.143936}%
\pgfsetfillcolor{currentfill}%
\pgfsetfillopacity{0.700000}%
\pgfsetlinewidth{0.000000pt}%
\definecolor{currentstroke}{rgb}{0.000000,0.000000,0.000000}%
\pgfsetstrokecolor{currentstroke}%
\pgfsetstrokeopacity{0.700000}%
\pgfsetdash{}{0pt}%
\pgfpathmoveto{\pgfqpoint{7.797254in}{2.711690in}}%
\pgfpathcurveto{\pgfqpoint{7.802297in}{2.711690in}}{\pgfqpoint{7.807135in}{2.713694in}}{\pgfqpoint{7.810702in}{2.717261in}}%
\pgfpathcurveto{\pgfqpoint{7.814268in}{2.720827in}}{\pgfqpoint{7.816272in}{2.725665in}}{\pgfqpoint{7.816272in}{2.730708in}}%
\pgfpathcurveto{\pgfqpoint{7.816272in}{2.735752in}}{\pgfqpoint{7.814268in}{2.740590in}}{\pgfqpoint{7.810702in}{2.744156in}}%
\pgfpathcurveto{\pgfqpoint{7.807135in}{2.747723in}}{\pgfqpoint{7.802297in}{2.749727in}}{\pgfqpoint{7.797254in}{2.749727in}}%
\pgfpathcurveto{\pgfqpoint{7.792210in}{2.749727in}}{\pgfqpoint{7.787372in}{2.747723in}}{\pgfqpoint{7.783806in}{2.744156in}}%
\pgfpathcurveto{\pgfqpoint{7.780239in}{2.740590in}}{\pgfqpoint{7.778236in}{2.735752in}}{\pgfqpoint{7.778236in}{2.730708in}}%
\pgfpathcurveto{\pgfqpoint{7.778236in}{2.725665in}}{\pgfqpoint{7.780239in}{2.720827in}}{\pgfqpoint{7.783806in}{2.717261in}}%
\pgfpathcurveto{\pgfqpoint{7.787372in}{2.713694in}}{\pgfqpoint{7.792210in}{2.711690in}}{\pgfqpoint{7.797254in}{2.711690in}}%
\pgfpathclose%
\pgfusepath{fill}%
\end{pgfscope}%
\begin{pgfscope}%
\pgfpathrectangle{\pgfqpoint{6.572727in}{0.474100in}}{\pgfqpoint{4.227273in}{3.318700in}}%
\pgfusepath{clip}%
\pgfsetbuttcap%
\pgfsetroundjoin%
\definecolor{currentfill}{rgb}{0.267004,0.004874,0.329415}%
\pgfsetfillcolor{currentfill}%
\pgfsetfillopacity{0.700000}%
\pgfsetlinewidth{0.000000pt}%
\definecolor{currentstroke}{rgb}{0.000000,0.000000,0.000000}%
\pgfsetstrokecolor{currentstroke}%
\pgfsetstrokeopacity{0.700000}%
\pgfsetdash{}{0pt}%
\pgfpathmoveto{\pgfqpoint{7.873351in}{2.025215in}}%
\pgfpathcurveto{\pgfqpoint{7.878395in}{2.025215in}}{\pgfqpoint{7.883232in}{2.027219in}}{\pgfqpoint{7.886799in}{2.030785in}}%
\pgfpathcurveto{\pgfqpoint{7.890365in}{2.034352in}}{\pgfqpoint{7.892369in}{2.039190in}}{\pgfqpoint{7.892369in}{2.044233in}}%
\pgfpathcurveto{\pgfqpoint{7.892369in}{2.049277in}}{\pgfqpoint{7.890365in}{2.054115in}}{\pgfqpoint{7.886799in}{2.057681in}}%
\pgfpathcurveto{\pgfqpoint{7.883232in}{2.061248in}}{\pgfqpoint{7.878395in}{2.063251in}}{\pgfqpoint{7.873351in}{2.063251in}}%
\pgfpathcurveto{\pgfqpoint{7.868307in}{2.063251in}}{\pgfqpoint{7.863469in}{2.061248in}}{\pgfqpoint{7.859903in}{2.057681in}}%
\pgfpathcurveto{\pgfqpoint{7.856337in}{2.054115in}}{\pgfqpoint{7.854333in}{2.049277in}}{\pgfqpoint{7.854333in}{2.044233in}}%
\pgfpathcurveto{\pgfqpoint{7.854333in}{2.039190in}}{\pgfqpoint{7.856337in}{2.034352in}}{\pgfqpoint{7.859903in}{2.030785in}}%
\pgfpathcurveto{\pgfqpoint{7.863469in}{2.027219in}}{\pgfqpoint{7.868307in}{2.025215in}}{\pgfqpoint{7.873351in}{2.025215in}}%
\pgfpathclose%
\pgfusepath{fill}%
\end{pgfscope}%
\begin{pgfscope}%
\pgfpathrectangle{\pgfqpoint{6.572727in}{0.474100in}}{\pgfqpoint{4.227273in}{3.318700in}}%
\pgfusepath{clip}%
\pgfsetbuttcap%
\pgfsetroundjoin%
\definecolor{currentfill}{rgb}{0.993248,0.906157,0.143936}%
\pgfsetfillcolor{currentfill}%
\pgfsetfillopacity{0.700000}%
\pgfsetlinewidth{0.000000pt}%
\definecolor{currentstroke}{rgb}{0.000000,0.000000,0.000000}%
\pgfsetstrokecolor{currentstroke}%
\pgfsetstrokeopacity{0.700000}%
\pgfsetdash{}{0pt}%
\pgfpathmoveto{\pgfqpoint{8.143509in}{2.575959in}}%
\pgfpathcurveto{\pgfqpoint{8.148553in}{2.575959in}}{\pgfqpoint{8.153390in}{2.577963in}}{\pgfqpoint{8.156957in}{2.581529in}}%
\pgfpathcurveto{\pgfqpoint{8.160523in}{2.585096in}}{\pgfqpoint{8.162527in}{2.589934in}}{\pgfqpoint{8.162527in}{2.594977in}}%
\pgfpathcurveto{\pgfqpoint{8.162527in}{2.600021in}}{\pgfqpoint{8.160523in}{2.604859in}}{\pgfqpoint{8.156957in}{2.608425in}}%
\pgfpathcurveto{\pgfqpoint{8.153390in}{2.611992in}}{\pgfqpoint{8.148553in}{2.613995in}}{\pgfqpoint{8.143509in}{2.613995in}}%
\pgfpathcurveto{\pgfqpoint{8.138465in}{2.613995in}}{\pgfqpoint{8.133627in}{2.611992in}}{\pgfqpoint{8.130061in}{2.608425in}}%
\pgfpathcurveto{\pgfqpoint{8.126495in}{2.604859in}}{\pgfqpoint{8.124491in}{2.600021in}}{\pgfqpoint{8.124491in}{2.594977in}}%
\pgfpathcurveto{\pgfqpoint{8.124491in}{2.589934in}}{\pgfqpoint{8.126495in}{2.585096in}}{\pgfqpoint{8.130061in}{2.581529in}}%
\pgfpathcurveto{\pgfqpoint{8.133627in}{2.577963in}}{\pgfqpoint{8.138465in}{2.575959in}}{\pgfqpoint{8.143509in}{2.575959in}}%
\pgfpathclose%
\pgfusepath{fill}%
\end{pgfscope}%
\begin{pgfscope}%
\pgfpathrectangle{\pgfqpoint{6.572727in}{0.474100in}}{\pgfqpoint{4.227273in}{3.318700in}}%
\pgfusepath{clip}%
\pgfsetbuttcap%
\pgfsetroundjoin%
\definecolor{currentfill}{rgb}{0.127568,0.566949,0.550556}%
\pgfsetfillcolor{currentfill}%
\pgfsetfillopacity{0.700000}%
\pgfsetlinewidth{0.000000pt}%
\definecolor{currentstroke}{rgb}{0.000000,0.000000,0.000000}%
\pgfsetstrokecolor{currentstroke}%
\pgfsetstrokeopacity{0.700000}%
\pgfsetdash{}{0pt}%
\pgfpathmoveto{\pgfqpoint{10.118185in}{1.860835in}}%
\pgfpathcurveto{\pgfqpoint{10.123228in}{1.860835in}}{\pgfqpoint{10.128066in}{1.862839in}}{\pgfqpoint{10.131632in}{1.866406in}}%
\pgfpathcurveto{\pgfqpoint{10.135199in}{1.869972in}}{\pgfqpoint{10.137203in}{1.874810in}}{\pgfqpoint{10.137203in}{1.879854in}}%
\pgfpathcurveto{\pgfqpoint{10.137203in}{1.884897in}}{\pgfqpoint{10.135199in}{1.889735in}}{\pgfqpoint{10.131632in}{1.893301in}}%
\pgfpathcurveto{\pgfqpoint{10.128066in}{1.896868in}}{\pgfqpoint{10.123228in}{1.898872in}}{\pgfqpoint{10.118185in}{1.898872in}}%
\pgfpathcurveto{\pgfqpoint{10.113141in}{1.898872in}}{\pgfqpoint{10.108303in}{1.896868in}}{\pgfqpoint{10.104737in}{1.893301in}}%
\pgfpathcurveto{\pgfqpoint{10.101170in}{1.889735in}}{\pgfqpoint{10.099166in}{1.884897in}}{\pgfqpoint{10.099166in}{1.879854in}}%
\pgfpathcurveto{\pgfqpoint{10.099166in}{1.874810in}}{\pgfqpoint{10.101170in}{1.869972in}}{\pgfqpoint{10.104737in}{1.866406in}}%
\pgfpathcurveto{\pgfqpoint{10.108303in}{1.862839in}}{\pgfqpoint{10.113141in}{1.860835in}}{\pgfqpoint{10.118185in}{1.860835in}}%
\pgfpathclose%
\pgfusepath{fill}%
\end{pgfscope}%
\begin{pgfscope}%
\pgfpathrectangle{\pgfqpoint{6.572727in}{0.474100in}}{\pgfqpoint{4.227273in}{3.318700in}}%
\pgfusepath{clip}%
\pgfsetbuttcap%
\pgfsetroundjoin%
\definecolor{currentfill}{rgb}{0.993248,0.906157,0.143936}%
\pgfsetfillcolor{currentfill}%
\pgfsetfillopacity{0.700000}%
\pgfsetlinewidth{0.000000pt}%
\definecolor{currentstroke}{rgb}{0.000000,0.000000,0.000000}%
\pgfsetstrokecolor{currentstroke}%
\pgfsetstrokeopacity{0.700000}%
\pgfsetdash{}{0pt}%
\pgfpathmoveto{\pgfqpoint{8.077415in}{2.819816in}}%
\pgfpathcurveto{\pgfqpoint{8.082459in}{2.819816in}}{\pgfqpoint{8.087297in}{2.821820in}}{\pgfqpoint{8.090863in}{2.825386in}}%
\pgfpathcurveto{\pgfqpoint{8.094429in}{2.828953in}}{\pgfqpoint{8.096433in}{2.833791in}}{\pgfqpoint{8.096433in}{2.838834in}}%
\pgfpathcurveto{\pgfqpoint{8.096433in}{2.843878in}}{\pgfqpoint{8.094429in}{2.848716in}}{\pgfqpoint{8.090863in}{2.852282in}}%
\pgfpathcurveto{\pgfqpoint{8.087297in}{2.855849in}}{\pgfqpoint{8.082459in}{2.857852in}}{\pgfqpoint{8.077415in}{2.857852in}}%
\pgfpathcurveto{\pgfqpoint{8.072371in}{2.857852in}}{\pgfqpoint{8.067534in}{2.855849in}}{\pgfqpoint{8.063967in}{2.852282in}}%
\pgfpathcurveto{\pgfqpoint{8.060401in}{2.848716in}}{\pgfqpoint{8.058397in}{2.843878in}}{\pgfqpoint{8.058397in}{2.838834in}}%
\pgfpathcurveto{\pgfqpoint{8.058397in}{2.833791in}}{\pgfqpoint{8.060401in}{2.828953in}}{\pgfqpoint{8.063967in}{2.825386in}}%
\pgfpathcurveto{\pgfqpoint{8.067534in}{2.821820in}}{\pgfqpoint{8.072371in}{2.819816in}}{\pgfqpoint{8.077415in}{2.819816in}}%
\pgfpathclose%
\pgfusepath{fill}%
\end{pgfscope}%
\begin{pgfscope}%
\pgfpathrectangle{\pgfqpoint{6.572727in}{0.474100in}}{\pgfqpoint{4.227273in}{3.318700in}}%
\pgfusepath{clip}%
\pgfsetbuttcap%
\pgfsetroundjoin%
\definecolor{currentfill}{rgb}{0.993248,0.906157,0.143936}%
\pgfsetfillcolor{currentfill}%
\pgfsetfillopacity{0.700000}%
\pgfsetlinewidth{0.000000pt}%
\definecolor{currentstroke}{rgb}{0.000000,0.000000,0.000000}%
\pgfsetstrokecolor{currentstroke}%
\pgfsetstrokeopacity{0.700000}%
\pgfsetdash{}{0pt}%
\pgfpathmoveto{\pgfqpoint{8.196651in}{3.193915in}}%
\pgfpathcurveto{\pgfqpoint{8.201695in}{3.193915in}}{\pgfqpoint{8.206533in}{3.195919in}}{\pgfqpoint{8.210099in}{3.199485in}}%
\pgfpathcurveto{\pgfqpoint{8.213665in}{3.203051in}}{\pgfqpoint{8.215669in}{3.207889in}}{\pgfqpoint{8.215669in}{3.212933in}}%
\pgfpathcurveto{\pgfqpoint{8.215669in}{3.217977in}}{\pgfqpoint{8.213665in}{3.222814in}}{\pgfqpoint{8.210099in}{3.226381in}}%
\pgfpathcurveto{\pgfqpoint{8.206533in}{3.229947in}}{\pgfqpoint{8.201695in}{3.231951in}}{\pgfqpoint{8.196651in}{3.231951in}}%
\pgfpathcurveto{\pgfqpoint{8.191607in}{3.231951in}}{\pgfqpoint{8.186770in}{3.229947in}}{\pgfqpoint{8.183203in}{3.226381in}}%
\pgfpathcurveto{\pgfqpoint{8.179637in}{3.222814in}}{\pgfqpoint{8.177633in}{3.217977in}}{\pgfqpoint{8.177633in}{3.212933in}}%
\pgfpathcurveto{\pgfqpoint{8.177633in}{3.207889in}}{\pgfqpoint{8.179637in}{3.203051in}}{\pgfqpoint{8.183203in}{3.199485in}}%
\pgfpathcurveto{\pgfqpoint{8.186770in}{3.195919in}}{\pgfqpoint{8.191607in}{3.193915in}}{\pgfqpoint{8.196651in}{3.193915in}}%
\pgfpathclose%
\pgfusepath{fill}%
\end{pgfscope}%
\begin{pgfscope}%
\pgfpathrectangle{\pgfqpoint{6.572727in}{0.474100in}}{\pgfqpoint{4.227273in}{3.318700in}}%
\pgfusepath{clip}%
\pgfsetbuttcap%
\pgfsetroundjoin%
\definecolor{currentfill}{rgb}{0.993248,0.906157,0.143936}%
\pgfsetfillcolor{currentfill}%
\pgfsetfillopacity{0.700000}%
\pgfsetlinewidth{0.000000pt}%
\definecolor{currentstroke}{rgb}{0.000000,0.000000,0.000000}%
\pgfsetstrokecolor{currentstroke}%
\pgfsetstrokeopacity{0.700000}%
\pgfsetdash{}{0pt}%
\pgfpathmoveto{\pgfqpoint{8.566652in}{2.775355in}}%
\pgfpathcurveto{\pgfqpoint{8.571695in}{2.775355in}}{\pgfqpoint{8.576533in}{2.777359in}}{\pgfqpoint{8.580100in}{2.780926in}}%
\pgfpathcurveto{\pgfqpoint{8.583666in}{2.784492in}}{\pgfqpoint{8.585670in}{2.789330in}}{\pgfqpoint{8.585670in}{2.794374in}}%
\pgfpathcurveto{\pgfqpoint{8.585670in}{2.799417in}}{\pgfqpoint{8.583666in}{2.804255in}}{\pgfqpoint{8.580100in}{2.807821in}}%
\pgfpathcurveto{\pgfqpoint{8.576533in}{2.811388in}}{\pgfqpoint{8.571695in}{2.813392in}}{\pgfqpoint{8.566652in}{2.813392in}}%
\pgfpathcurveto{\pgfqpoint{8.561608in}{2.813392in}}{\pgfqpoint{8.556770in}{2.811388in}}{\pgfqpoint{8.553204in}{2.807821in}}%
\pgfpathcurveto{\pgfqpoint{8.549638in}{2.804255in}}{\pgfqpoint{8.547634in}{2.799417in}}{\pgfqpoint{8.547634in}{2.794374in}}%
\pgfpathcurveto{\pgfqpoint{8.547634in}{2.789330in}}{\pgfqpoint{8.549638in}{2.784492in}}{\pgfqpoint{8.553204in}{2.780926in}}%
\pgfpathcurveto{\pgfqpoint{8.556770in}{2.777359in}}{\pgfqpoint{8.561608in}{2.775355in}}{\pgfqpoint{8.566652in}{2.775355in}}%
\pgfpathclose%
\pgfusepath{fill}%
\end{pgfscope}%
\begin{pgfscope}%
\pgfpathrectangle{\pgfqpoint{6.572727in}{0.474100in}}{\pgfqpoint{4.227273in}{3.318700in}}%
\pgfusepath{clip}%
\pgfsetbuttcap%
\pgfsetroundjoin%
\definecolor{currentfill}{rgb}{0.127568,0.566949,0.550556}%
\pgfsetfillcolor{currentfill}%
\pgfsetfillopacity{0.700000}%
\pgfsetlinewidth{0.000000pt}%
\definecolor{currentstroke}{rgb}{0.000000,0.000000,0.000000}%
\pgfsetstrokecolor{currentstroke}%
\pgfsetstrokeopacity{0.700000}%
\pgfsetdash{}{0pt}%
\pgfpathmoveto{\pgfqpoint{9.713458in}{1.760334in}}%
\pgfpathcurveto{\pgfqpoint{9.718502in}{1.760334in}}{\pgfqpoint{9.723339in}{1.762338in}}{\pgfqpoint{9.726906in}{1.765904in}}%
\pgfpathcurveto{\pgfqpoint{9.730472in}{1.769471in}}{\pgfqpoint{9.732476in}{1.774308in}}{\pgfqpoint{9.732476in}{1.779352in}}%
\pgfpathcurveto{\pgfqpoint{9.732476in}{1.784396in}}{\pgfqpoint{9.730472in}{1.789233in}}{\pgfqpoint{9.726906in}{1.792800in}}%
\pgfpathcurveto{\pgfqpoint{9.723339in}{1.796366in}}{\pgfqpoint{9.718502in}{1.798370in}}{\pgfqpoint{9.713458in}{1.798370in}}%
\pgfpathcurveto{\pgfqpoint{9.708414in}{1.798370in}}{\pgfqpoint{9.703577in}{1.796366in}}{\pgfqpoint{9.700010in}{1.792800in}}%
\pgfpathcurveto{\pgfqpoint{9.696444in}{1.789233in}}{\pgfqpoint{9.694440in}{1.784396in}}{\pgfqpoint{9.694440in}{1.779352in}}%
\pgfpathcurveto{\pgfqpoint{9.694440in}{1.774308in}}{\pgfqpoint{9.696444in}{1.769471in}}{\pgfqpoint{9.700010in}{1.765904in}}%
\pgfpathcurveto{\pgfqpoint{9.703577in}{1.762338in}}{\pgfqpoint{9.708414in}{1.760334in}}{\pgfqpoint{9.713458in}{1.760334in}}%
\pgfpathclose%
\pgfusepath{fill}%
\end{pgfscope}%
\begin{pgfscope}%
\pgfpathrectangle{\pgfqpoint{6.572727in}{0.474100in}}{\pgfqpoint{4.227273in}{3.318700in}}%
\pgfusepath{clip}%
\pgfsetbuttcap%
\pgfsetroundjoin%
\definecolor{currentfill}{rgb}{0.127568,0.566949,0.550556}%
\pgfsetfillcolor{currentfill}%
\pgfsetfillopacity{0.700000}%
\pgfsetlinewidth{0.000000pt}%
\definecolor{currentstroke}{rgb}{0.000000,0.000000,0.000000}%
\pgfsetstrokecolor{currentstroke}%
\pgfsetstrokeopacity{0.700000}%
\pgfsetdash{}{0pt}%
\pgfpathmoveto{\pgfqpoint{9.549198in}{0.981676in}}%
\pgfpathcurveto{\pgfqpoint{9.554242in}{0.981676in}}{\pgfqpoint{9.559080in}{0.983680in}}{\pgfqpoint{9.562646in}{0.987246in}}%
\pgfpathcurveto{\pgfqpoint{9.566212in}{0.990813in}}{\pgfqpoint{9.568216in}{0.995651in}}{\pgfqpoint{9.568216in}{1.000694in}}%
\pgfpathcurveto{\pgfqpoint{9.568216in}{1.005738in}}{\pgfqpoint{9.566212in}{1.010576in}}{\pgfqpoint{9.562646in}{1.014142in}}%
\pgfpathcurveto{\pgfqpoint{9.559080in}{1.017708in}}{\pgfqpoint{9.554242in}{1.019712in}}{\pgfqpoint{9.549198in}{1.019712in}}%
\pgfpathcurveto{\pgfqpoint{9.544154in}{1.019712in}}{\pgfqpoint{9.539317in}{1.017708in}}{\pgfqpoint{9.535750in}{1.014142in}}%
\pgfpathcurveto{\pgfqpoint{9.532184in}{1.010576in}}{\pgfqpoint{9.530180in}{1.005738in}}{\pgfqpoint{9.530180in}{1.000694in}}%
\pgfpathcurveto{\pgfqpoint{9.530180in}{0.995651in}}{\pgfqpoint{9.532184in}{0.990813in}}{\pgfqpoint{9.535750in}{0.987246in}}%
\pgfpathcurveto{\pgfqpoint{9.539317in}{0.983680in}}{\pgfqpoint{9.544154in}{0.981676in}}{\pgfqpoint{9.549198in}{0.981676in}}%
\pgfpathclose%
\pgfusepath{fill}%
\end{pgfscope}%
\begin{pgfscope}%
\pgfpathrectangle{\pgfqpoint{6.572727in}{0.474100in}}{\pgfqpoint{4.227273in}{3.318700in}}%
\pgfusepath{clip}%
\pgfsetbuttcap%
\pgfsetroundjoin%
\definecolor{currentfill}{rgb}{0.127568,0.566949,0.550556}%
\pgfsetfillcolor{currentfill}%
\pgfsetfillopacity{0.700000}%
\pgfsetlinewidth{0.000000pt}%
\definecolor{currentstroke}{rgb}{0.000000,0.000000,0.000000}%
\pgfsetstrokecolor{currentstroke}%
\pgfsetstrokeopacity{0.700000}%
\pgfsetdash{}{0pt}%
\pgfpathmoveto{\pgfqpoint{9.938517in}{1.284958in}}%
\pgfpathcurveto{\pgfqpoint{9.943561in}{1.284958in}}{\pgfqpoint{9.948398in}{1.286962in}}{\pgfqpoint{9.951965in}{1.290529in}}%
\pgfpathcurveto{\pgfqpoint{9.955531in}{1.294095in}}{\pgfqpoint{9.957535in}{1.298933in}}{\pgfqpoint{9.957535in}{1.303977in}}%
\pgfpathcurveto{\pgfqpoint{9.957535in}{1.309020in}}{\pgfqpoint{9.955531in}{1.313858in}}{\pgfqpoint{9.951965in}{1.317424in}}%
\pgfpathcurveto{\pgfqpoint{9.948398in}{1.320991in}}{\pgfqpoint{9.943561in}{1.322995in}}{\pgfqpoint{9.938517in}{1.322995in}}%
\pgfpathcurveto{\pgfqpoint{9.933473in}{1.322995in}}{\pgfqpoint{9.928636in}{1.320991in}}{\pgfqpoint{9.925069in}{1.317424in}}%
\pgfpathcurveto{\pgfqpoint{9.921503in}{1.313858in}}{\pgfqpoint{9.919499in}{1.309020in}}{\pgfqpoint{9.919499in}{1.303977in}}%
\pgfpathcurveto{\pgfqpoint{9.919499in}{1.298933in}}{\pgfqpoint{9.921503in}{1.294095in}}{\pgfqpoint{9.925069in}{1.290529in}}%
\pgfpathcurveto{\pgfqpoint{9.928636in}{1.286962in}}{\pgfqpoint{9.933473in}{1.284958in}}{\pgfqpoint{9.938517in}{1.284958in}}%
\pgfpathclose%
\pgfusepath{fill}%
\end{pgfscope}%
\begin{pgfscope}%
\pgfpathrectangle{\pgfqpoint{6.572727in}{0.474100in}}{\pgfqpoint{4.227273in}{3.318700in}}%
\pgfusepath{clip}%
\pgfsetbuttcap%
\pgfsetroundjoin%
\definecolor{currentfill}{rgb}{0.127568,0.566949,0.550556}%
\pgfsetfillcolor{currentfill}%
\pgfsetfillopacity{0.700000}%
\pgfsetlinewidth{0.000000pt}%
\definecolor{currentstroke}{rgb}{0.000000,0.000000,0.000000}%
\pgfsetstrokecolor{currentstroke}%
\pgfsetstrokeopacity{0.700000}%
\pgfsetdash{}{0pt}%
\pgfpathmoveto{\pgfqpoint{9.637918in}{1.755463in}}%
\pgfpathcurveto{\pgfqpoint{9.642962in}{1.755463in}}{\pgfqpoint{9.647800in}{1.757466in}}{\pgfqpoint{9.651366in}{1.761033in}}%
\pgfpathcurveto{\pgfqpoint{9.654933in}{1.764599in}}{\pgfqpoint{9.656937in}{1.769437in}}{\pgfqpoint{9.656937in}{1.774481in}}%
\pgfpathcurveto{\pgfqpoint{9.656937in}{1.779524in}}{\pgfqpoint{9.654933in}{1.784362in}}{\pgfqpoint{9.651366in}{1.787929in}}%
\pgfpathcurveto{\pgfqpoint{9.647800in}{1.791495in}}{\pgfqpoint{9.642962in}{1.793499in}}{\pgfqpoint{9.637918in}{1.793499in}}%
\pgfpathcurveto{\pgfqpoint{9.632875in}{1.793499in}}{\pgfqpoint{9.628037in}{1.791495in}}{\pgfqpoint{9.624471in}{1.787929in}}%
\pgfpathcurveto{\pgfqpoint{9.620904in}{1.784362in}}{\pgfqpoint{9.618900in}{1.779524in}}{\pgfqpoint{9.618900in}{1.774481in}}%
\pgfpathcurveto{\pgfqpoint{9.618900in}{1.769437in}}{\pgfqpoint{9.620904in}{1.764599in}}{\pgfqpoint{9.624471in}{1.761033in}}%
\pgfpathcurveto{\pgfqpoint{9.628037in}{1.757466in}}{\pgfqpoint{9.632875in}{1.755463in}}{\pgfqpoint{9.637918in}{1.755463in}}%
\pgfpathclose%
\pgfusepath{fill}%
\end{pgfscope}%
\begin{pgfscope}%
\pgfpathrectangle{\pgfqpoint{6.572727in}{0.474100in}}{\pgfqpoint{4.227273in}{3.318700in}}%
\pgfusepath{clip}%
\pgfsetbuttcap%
\pgfsetroundjoin%
\definecolor{currentfill}{rgb}{0.127568,0.566949,0.550556}%
\pgfsetfillcolor{currentfill}%
\pgfsetfillopacity{0.700000}%
\pgfsetlinewidth{0.000000pt}%
\definecolor{currentstroke}{rgb}{0.000000,0.000000,0.000000}%
\pgfsetstrokecolor{currentstroke}%
\pgfsetstrokeopacity{0.700000}%
\pgfsetdash{}{0pt}%
\pgfpathmoveto{\pgfqpoint{9.935366in}{1.275067in}}%
\pgfpathcurveto{\pgfqpoint{9.940410in}{1.275067in}}{\pgfqpoint{9.945248in}{1.277071in}}{\pgfqpoint{9.948814in}{1.280637in}}%
\pgfpathcurveto{\pgfqpoint{9.952381in}{1.284204in}}{\pgfqpoint{9.954385in}{1.289041in}}{\pgfqpoint{9.954385in}{1.294085in}}%
\pgfpathcurveto{\pgfqpoint{9.954385in}{1.299129in}}{\pgfqpoint{9.952381in}{1.303966in}}{\pgfqpoint{9.948814in}{1.307533in}}%
\pgfpathcurveto{\pgfqpoint{9.945248in}{1.311099in}}{\pgfqpoint{9.940410in}{1.313103in}}{\pgfqpoint{9.935366in}{1.313103in}}%
\pgfpathcurveto{\pgfqpoint{9.930323in}{1.313103in}}{\pgfqpoint{9.925485in}{1.311099in}}{\pgfqpoint{9.921919in}{1.307533in}}%
\pgfpathcurveto{\pgfqpoint{9.918352in}{1.303966in}}{\pgfqpoint{9.916348in}{1.299129in}}{\pgfqpoint{9.916348in}{1.294085in}}%
\pgfpathcurveto{\pgfqpoint{9.916348in}{1.289041in}}{\pgfqpoint{9.918352in}{1.284204in}}{\pgfqpoint{9.921919in}{1.280637in}}%
\pgfpathcurveto{\pgfqpoint{9.925485in}{1.277071in}}{\pgfqpoint{9.930323in}{1.275067in}}{\pgfqpoint{9.935366in}{1.275067in}}%
\pgfpathclose%
\pgfusepath{fill}%
\end{pgfscope}%
\begin{pgfscope}%
\pgfpathrectangle{\pgfqpoint{6.572727in}{0.474100in}}{\pgfqpoint{4.227273in}{3.318700in}}%
\pgfusepath{clip}%
\pgfsetbuttcap%
\pgfsetroundjoin%
\definecolor{currentfill}{rgb}{0.127568,0.566949,0.550556}%
\pgfsetfillcolor{currentfill}%
\pgfsetfillopacity{0.700000}%
\pgfsetlinewidth{0.000000pt}%
\definecolor{currentstroke}{rgb}{0.000000,0.000000,0.000000}%
\pgfsetstrokecolor{currentstroke}%
\pgfsetstrokeopacity{0.700000}%
\pgfsetdash{}{0pt}%
\pgfpathmoveto{\pgfqpoint{9.632648in}{1.617971in}}%
\pgfpathcurveto{\pgfqpoint{9.637691in}{1.617971in}}{\pgfqpoint{9.642529in}{1.619974in}}{\pgfqpoint{9.646095in}{1.623541in}}%
\pgfpathcurveto{\pgfqpoint{9.649662in}{1.627107in}}{\pgfqpoint{9.651666in}{1.631945in}}{\pgfqpoint{9.651666in}{1.636989in}}%
\pgfpathcurveto{\pgfqpoint{9.651666in}{1.642032in}}{\pgfqpoint{9.649662in}{1.646870in}}{\pgfqpoint{9.646095in}{1.650437in}}%
\pgfpathcurveto{\pgfqpoint{9.642529in}{1.654003in}}{\pgfqpoint{9.637691in}{1.656007in}}{\pgfqpoint{9.632648in}{1.656007in}}%
\pgfpathcurveto{\pgfqpoint{9.627604in}{1.656007in}}{\pgfqpoint{9.622766in}{1.654003in}}{\pgfqpoint{9.619200in}{1.650437in}}%
\pgfpathcurveto{\pgfqpoint{9.615633in}{1.646870in}}{\pgfqpoint{9.613629in}{1.642032in}}{\pgfqpoint{9.613629in}{1.636989in}}%
\pgfpathcurveto{\pgfqpoint{9.613629in}{1.631945in}}{\pgfqpoint{9.615633in}{1.627107in}}{\pgfqpoint{9.619200in}{1.623541in}}%
\pgfpathcurveto{\pgfqpoint{9.622766in}{1.619974in}}{\pgfqpoint{9.627604in}{1.617971in}}{\pgfqpoint{9.632648in}{1.617971in}}%
\pgfpathclose%
\pgfusepath{fill}%
\end{pgfscope}%
\begin{pgfscope}%
\pgfpathrectangle{\pgfqpoint{6.572727in}{0.474100in}}{\pgfqpoint{4.227273in}{3.318700in}}%
\pgfusepath{clip}%
\pgfsetbuttcap%
\pgfsetroundjoin%
\definecolor{currentfill}{rgb}{0.267004,0.004874,0.329415}%
\pgfsetfillcolor{currentfill}%
\pgfsetfillopacity{0.700000}%
\pgfsetlinewidth{0.000000pt}%
\definecolor{currentstroke}{rgb}{0.000000,0.000000,0.000000}%
\pgfsetstrokecolor{currentstroke}%
\pgfsetstrokeopacity{0.700000}%
\pgfsetdash{}{0pt}%
\pgfpathmoveto{\pgfqpoint{8.044212in}{1.054902in}}%
\pgfpathcurveto{\pgfqpoint{8.049255in}{1.054902in}}{\pgfqpoint{8.054093in}{1.056906in}}{\pgfqpoint{8.057660in}{1.060472in}}%
\pgfpathcurveto{\pgfqpoint{8.061226in}{1.064038in}}{\pgfqpoint{8.063230in}{1.068876in}}{\pgfqpoint{8.063230in}{1.073920in}}%
\pgfpathcurveto{\pgfqpoint{8.063230in}{1.078964in}}{\pgfqpoint{8.061226in}{1.083801in}}{\pgfqpoint{8.057660in}{1.087368in}}%
\pgfpathcurveto{\pgfqpoint{8.054093in}{1.090934in}}{\pgfqpoint{8.049255in}{1.092938in}}{\pgfqpoint{8.044212in}{1.092938in}}%
\pgfpathcurveto{\pgfqpoint{8.039168in}{1.092938in}}{\pgfqpoint{8.034330in}{1.090934in}}{\pgfqpoint{8.030764in}{1.087368in}}%
\pgfpathcurveto{\pgfqpoint{8.027197in}{1.083801in}}{\pgfqpoint{8.025194in}{1.078964in}}{\pgfqpoint{8.025194in}{1.073920in}}%
\pgfpathcurveto{\pgfqpoint{8.025194in}{1.068876in}}{\pgfqpoint{8.027197in}{1.064038in}}{\pgfqpoint{8.030764in}{1.060472in}}%
\pgfpathcurveto{\pgfqpoint{8.034330in}{1.056906in}}{\pgfqpoint{8.039168in}{1.054902in}}{\pgfqpoint{8.044212in}{1.054902in}}%
\pgfpathclose%
\pgfusepath{fill}%
\end{pgfscope}%
\begin{pgfscope}%
\pgfpathrectangle{\pgfqpoint{6.572727in}{0.474100in}}{\pgfqpoint{4.227273in}{3.318700in}}%
\pgfusepath{clip}%
\pgfsetbuttcap%
\pgfsetroundjoin%
\definecolor{currentfill}{rgb}{0.267004,0.004874,0.329415}%
\pgfsetfillcolor{currentfill}%
\pgfsetfillopacity{0.700000}%
\pgfsetlinewidth{0.000000pt}%
\definecolor{currentstroke}{rgb}{0.000000,0.000000,0.000000}%
\pgfsetstrokecolor{currentstroke}%
\pgfsetstrokeopacity{0.700000}%
\pgfsetdash{}{0pt}%
\pgfpathmoveto{\pgfqpoint{7.670830in}{1.274676in}}%
\pgfpathcurveto{\pgfqpoint{7.675874in}{1.274676in}}{\pgfqpoint{7.680712in}{1.276680in}}{\pgfqpoint{7.684278in}{1.280246in}}%
\pgfpathcurveto{\pgfqpoint{7.687845in}{1.283813in}}{\pgfqpoint{7.689848in}{1.288650in}}{\pgfqpoint{7.689848in}{1.293694in}}%
\pgfpathcurveto{\pgfqpoint{7.689848in}{1.298738in}}{\pgfqpoint{7.687845in}{1.303575in}}{\pgfqpoint{7.684278in}{1.307142in}}%
\pgfpathcurveto{\pgfqpoint{7.680712in}{1.310708in}}{\pgfqpoint{7.675874in}{1.312712in}}{\pgfqpoint{7.670830in}{1.312712in}}%
\pgfpathcurveto{\pgfqpoint{7.665787in}{1.312712in}}{\pgfqpoint{7.660949in}{1.310708in}}{\pgfqpoint{7.657382in}{1.307142in}}%
\pgfpathcurveto{\pgfqpoint{7.653816in}{1.303575in}}{\pgfqpoint{7.651812in}{1.298738in}}{\pgfqpoint{7.651812in}{1.293694in}}%
\pgfpathcurveto{\pgfqpoint{7.651812in}{1.288650in}}{\pgfqpoint{7.653816in}{1.283813in}}{\pgfqpoint{7.657382in}{1.280246in}}%
\pgfpathcurveto{\pgfqpoint{7.660949in}{1.276680in}}{\pgfqpoint{7.665787in}{1.274676in}}{\pgfqpoint{7.670830in}{1.274676in}}%
\pgfpathclose%
\pgfusepath{fill}%
\end{pgfscope}%
\begin{pgfscope}%
\pgfpathrectangle{\pgfqpoint{6.572727in}{0.474100in}}{\pgfqpoint{4.227273in}{3.318700in}}%
\pgfusepath{clip}%
\pgfsetbuttcap%
\pgfsetroundjoin%
\definecolor{currentfill}{rgb}{0.127568,0.566949,0.550556}%
\pgfsetfillcolor{currentfill}%
\pgfsetfillopacity{0.700000}%
\pgfsetlinewidth{0.000000pt}%
\definecolor{currentstroke}{rgb}{0.000000,0.000000,0.000000}%
\pgfsetstrokecolor{currentstroke}%
\pgfsetstrokeopacity{0.700000}%
\pgfsetdash{}{0pt}%
\pgfpathmoveto{\pgfqpoint{9.727923in}{1.884600in}}%
\pgfpathcurveto{\pgfqpoint{9.732967in}{1.884600in}}{\pgfqpoint{9.737804in}{1.886604in}}{\pgfqpoint{9.741371in}{1.890170in}}%
\pgfpathcurveto{\pgfqpoint{9.744937in}{1.893737in}}{\pgfqpoint{9.746941in}{1.898575in}}{\pgfqpoint{9.746941in}{1.903618in}}%
\pgfpathcurveto{\pgfqpoint{9.746941in}{1.908662in}}{\pgfqpoint{9.744937in}{1.913500in}}{\pgfqpoint{9.741371in}{1.917066in}}%
\pgfpathcurveto{\pgfqpoint{9.737804in}{1.920632in}}{\pgfqpoint{9.732967in}{1.922636in}}{\pgfqpoint{9.727923in}{1.922636in}}%
\pgfpathcurveto{\pgfqpoint{9.722879in}{1.922636in}}{\pgfqpoint{9.718042in}{1.920632in}}{\pgfqpoint{9.714475in}{1.917066in}}%
\pgfpathcurveto{\pgfqpoint{9.710909in}{1.913500in}}{\pgfqpoint{9.708905in}{1.908662in}}{\pgfqpoint{9.708905in}{1.903618in}}%
\pgfpathcurveto{\pgfqpoint{9.708905in}{1.898575in}}{\pgfqpoint{9.710909in}{1.893737in}}{\pgfqpoint{9.714475in}{1.890170in}}%
\pgfpathcurveto{\pgfqpoint{9.718042in}{1.886604in}}{\pgfqpoint{9.722879in}{1.884600in}}{\pgfqpoint{9.727923in}{1.884600in}}%
\pgfpathclose%
\pgfusepath{fill}%
\end{pgfscope}%
\begin{pgfscope}%
\pgfpathrectangle{\pgfqpoint{6.572727in}{0.474100in}}{\pgfqpoint{4.227273in}{3.318700in}}%
\pgfusepath{clip}%
\pgfsetbuttcap%
\pgfsetroundjoin%
\definecolor{currentfill}{rgb}{0.267004,0.004874,0.329415}%
\pgfsetfillcolor{currentfill}%
\pgfsetfillopacity{0.700000}%
\pgfsetlinewidth{0.000000pt}%
\definecolor{currentstroke}{rgb}{0.000000,0.000000,0.000000}%
\pgfsetstrokecolor{currentstroke}%
\pgfsetstrokeopacity{0.700000}%
\pgfsetdash{}{0pt}%
\pgfpathmoveto{\pgfqpoint{7.486500in}{1.381530in}}%
\pgfpathcurveto{\pgfqpoint{7.491544in}{1.381530in}}{\pgfqpoint{7.496381in}{1.383533in}}{\pgfqpoint{7.499948in}{1.387100in}}%
\pgfpathcurveto{\pgfqpoint{7.503514in}{1.390666in}}{\pgfqpoint{7.505518in}{1.395504in}}{\pgfqpoint{7.505518in}{1.400548in}}%
\pgfpathcurveto{\pgfqpoint{7.505518in}{1.405591in}}{\pgfqpoint{7.503514in}{1.410429in}}{\pgfqpoint{7.499948in}{1.413996in}}%
\pgfpathcurveto{\pgfqpoint{7.496381in}{1.417562in}}{\pgfqpoint{7.491544in}{1.419566in}}{\pgfqpoint{7.486500in}{1.419566in}}%
\pgfpathcurveto{\pgfqpoint{7.481456in}{1.419566in}}{\pgfqpoint{7.476619in}{1.417562in}}{\pgfqpoint{7.473052in}{1.413996in}}%
\pgfpathcurveto{\pgfqpoint{7.469486in}{1.410429in}}{\pgfqpoint{7.467482in}{1.405591in}}{\pgfqpoint{7.467482in}{1.400548in}}%
\pgfpathcurveto{\pgfqpoint{7.467482in}{1.395504in}}{\pgfqpoint{7.469486in}{1.390666in}}{\pgfqpoint{7.473052in}{1.387100in}}%
\pgfpathcurveto{\pgfqpoint{7.476619in}{1.383533in}}{\pgfqpoint{7.481456in}{1.381530in}}{\pgfqpoint{7.486500in}{1.381530in}}%
\pgfpathclose%
\pgfusepath{fill}%
\end{pgfscope}%
\begin{pgfscope}%
\pgfpathrectangle{\pgfqpoint{6.572727in}{0.474100in}}{\pgfqpoint{4.227273in}{3.318700in}}%
\pgfusepath{clip}%
\pgfsetbuttcap%
\pgfsetroundjoin%
\definecolor{currentfill}{rgb}{0.267004,0.004874,0.329415}%
\pgfsetfillcolor{currentfill}%
\pgfsetfillopacity{0.700000}%
\pgfsetlinewidth{0.000000pt}%
\definecolor{currentstroke}{rgb}{0.000000,0.000000,0.000000}%
\pgfsetstrokecolor{currentstroke}%
\pgfsetstrokeopacity{0.700000}%
\pgfsetdash{}{0pt}%
\pgfpathmoveto{\pgfqpoint{7.183260in}{1.310549in}}%
\pgfpathcurveto{\pgfqpoint{7.188303in}{1.310549in}}{\pgfqpoint{7.193141in}{1.312553in}}{\pgfqpoint{7.196708in}{1.316119in}}%
\pgfpathcurveto{\pgfqpoint{7.200274in}{1.319686in}}{\pgfqpoint{7.202278in}{1.324523in}}{\pgfqpoint{7.202278in}{1.329567in}}%
\pgfpathcurveto{\pgfqpoint{7.202278in}{1.334611in}}{\pgfqpoint{7.200274in}{1.339449in}}{\pgfqpoint{7.196708in}{1.343015in}}%
\pgfpathcurveto{\pgfqpoint{7.193141in}{1.346581in}}{\pgfqpoint{7.188303in}{1.348585in}}{\pgfqpoint{7.183260in}{1.348585in}}%
\pgfpathcurveto{\pgfqpoint{7.178216in}{1.348585in}}{\pgfqpoint{7.173378in}{1.346581in}}{\pgfqpoint{7.169812in}{1.343015in}}%
\pgfpathcurveto{\pgfqpoint{7.166245in}{1.339449in}}{\pgfqpoint{7.164241in}{1.334611in}}{\pgfqpoint{7.164241in}{1.329567in}}%
\pgfpathcurveto{\pgfqpoint{7.164241in}{1.324523in}}{\pgfqpoint{7.166245in}{1.319686in}}{\pgfqpoint{7.169812in}{1.316119in}}%
\pgfpathcurveto{\pgfqpoint{7.173378in}{1.312553in}}{\pgfqpoint{7.178216in}{1.310549in}}{\pgfqpoint{7.183260in}{1.310549in}}%
\pgfpathclose%
\pgfusepath{fill}%
\end{pgfscope}%
\begin{pgfscope}%
\pgfpathrectangle{\pgfqpoint{6.572727in}{0.474100in}}{\pgfqpoint{4.227273in}{3.318700in}}%
\pgfusepath{clip}%
\pgfsetbuttcap%
\pgfsetroundjoin%
\definecolor{currentfill}{rgb}{0.993248,0.906157,0.143936}%
\pgfsetfillcolor{currentfill}%
\pgfsetfillopacity{0.700000}%
\pgfsetlinewidth{0.000000pt}%
\definecolor{currentstroke}{rgb}{0.000000,0.000000,0.000000}%
\pgfsetstrokecolor{currentstroke}%
\pgfsetstrokeopacity{0.700000}%
\pgfsetdash{}{0pt}%
\pgfpathmoveto{\pgfqpoint{8.591138in}{2.724221in}}%
\pgfpathcurveto{\pgfqpoint{8.596182in}{2.724221in}}{\pgfqpoint{8.601020in}{2.726225in}}{\pgfqpoint{8.604586in}{2.729792in}}%
\pgfpathcurveto{\pgfqpoint{8.608153in}{2.733358in}}{\pgfqpoint{8.610156in}{2.738196in}}{\pgfqpoint{8.610156in}{2.743239in}}%
\pgfpathcurveto{\pgfqpoint{8.610156in}{2.748283in}}{\pgfqpoint{8.608153in}{2.753121in}}{\pgfqpoint{8.604586in}{2.756687in}}%
\pgfpathcurveto{\pgfqpoint{8.601020in}{2.760254in}}{\pgfqpoint{8.596182in}{2.762258in}}{\pgfqpoint{8.591138in}{2.762258in}}%
\pgfpathcurveto{\pgfqpoint{8.586095in}{2.762258in}}{\pgfqpoint{8.581257in}{2.760254in}}{\pgfqpoint{8.577690in}{2.756687in}}%
\pgfpathcurveto{\pgfqpoint{8.574124in}{2.753121in}}{\pgfqpoint{8.572120in}{2.748283in}}{\pgfqpoint{8.572120in}{2.743239in}}%
\pgfpathcurveto{\pgfqpoint{8.572120in}{2.738196in}}{\pgfqpoint{8.574124in}{2.733358in}}{\pgfqpoint{8.577690in}{2.729792in}}%
\pgfpathcurveto{\pgfqpoint{8.581257in}{2.726225in}}{\pgfqpoint{8.586095in}{2.724221in}}{\pgfqpoint{8.591138in}{2.724221in}}%
\pgfpathclose%
\pgfusepath{fill}%
\end{pgfscope}%
\begin{pgfscope}%
\pgfpathrectangle{\pgfqpoint{6.572727in}{0.474100in}}{\pgfqpoint{4.227273in}{3.318700in}}%
\pgfusepath{clip}%
\pgfsetbuttcap%
\pgfsetroundjoin%
\definecolor{currentfill}{rgb}{0.267004,0.004874,0.329415}%
\pgfsetfillcolor{currentfill}%
\pgfsetfillopacity{0.700000}%
\pgfsetlinewidth{0.000000pt}%
\definecolor{currentstroke}{rgb}{0.000000,0.000000,0.000000}%
\pgfsetstrokecolor{currentstroke}%
\pgfsetstrokeopacity{0.700000}%
\pgfsetdash{}{0pt}%
\pgfpathmoveto{\pgfqpoint{7.203953in}{2.308809in}}%
\pgfpathcurveto{\pgfqpoint{7.208997in}{2.308809in}}{\pgfqpoint{7.213834in}{2.310813in}}{\pgfqpoint{7.217401in}{2.314379in}}%
\pgfpathcurveto{\pgfqpoint{7.220967in}{2.317946in}}{\pgfqpoint{7.222971in}{2.322783in}}{\pgfqpoint{7.222971in}{2.327827in}}%
\pgfpathcurveto{\pgfqpoint{7.222971in}{2.332871in}}{\pgfqpoint{7.220967in}{2.337708in}}{\pgfqpoint{7.217401in}{2.341275in}}%
\pgfpathcurveto{\pgfqpoint{7.213834in}{2.344841in}}{\pgfqpoint{7.208997in}{2.346845in}}{\pgfqpoint{7.203953in}{2.346845in}}%
\pgfpathcurveto{\pgfqpoint{7.198909in}{2.346845in}}{\pgfqpoint{7.194071in}{2.344841in}}{\pgfqpoint{7.190505in}{2.341275in}}%
\pgfpathcurveto{\pgfqpoint{7.186939in}{2.337708in}}{\pgfqpoint{7.184935in}{2.332871in}}{\pgfqpoint{7.184935in}{2.327827in}}%
\pgfpathcurveto{\pgfqpoint{7.184935in}{2.322783in}}{\pgfqpoint{7.186939in}{2.317946in}}{\pgfqpoint{7.190505in}{2.314379in}}%
\pgfpathcurveto{\pgfqpoint{7.194071in}{2.310813in}}{\pgfqpoint{7.198909in}{2.308809in}}{\pgfqpoint{7.203953in}{2.308809in}}%
\pgfpathclose%
\pgfusepath{fill}%
\end{pgfscope}%
\begin{pgfscope}%
\pgfpathrectangle{\pgfqpoint{6.572727in}{0.474100in}}{\pgfqpoint{4.227273in}{3.318700in}}%
\pgfusepath{clip}%
\pgfsetbuttcap%
\pgfsetroundjoin%
\definecolor{currentfill}{rgb}{0.993248,0.906157,0.143936}%
\pgfsetfillcolor{currentfill}%
\pgfsetfillopacity{0.700000}%
\pgfsetlinewidth{0.000000pt}%
\definecolor{currentstroke}{rgb}{0.000000,0.000000,0.000000}%
\pgfsetstrokecolor{currentstroke}%
\pgfsetstrokeopacity{0.700000}%
\pgfsetdash{}{0pt}%
\pgfpathmoveto{\pgfqpoint{8.146426in}{2.435695in}}%
\pgfpathcurveto{\pgfqpoint{8.151469in}{2.435695in}}{\pgfqpoint{8.156307in}{2.437699in}}{\pgfqpoint{8.159873in}{2.441265in}}%
\pgfpathcurveto{\pgfqpoint{8.163440in}{2.444831in}}{\pgfqpoint{8.165444in}{2.449669in}}{\pgfqpoint{8.165444in}{2.454713in}}%
\pgfpathcurveto{\pgfqpoint{8.165444in}{2.459756in}}{\pgfqpoint{8.163440in}{2.464594in}}{\pgfqpoint{8.159873in}{2.468161in}}%
\pgfpathcurveto{\pgfqpoint{8.156307in}{2.471727in}}{\pgfqpoint{8.151469in}{2.473731in}}{\pgfqpoint{8.146426in}{2.473731in}}%
\pgfpathcurveto{\pgfqpoint{8.141382in}{2.473731in}}{\pgfqpoint{8.136544in}{2.471727in}}{\pgfqpoint{8.132978in}{2.468161in}}%
\pgfpathcurveto{\pgfqpoint{8.129411in}{2.464594in}}{\pgfqpoint{8.127407in}{2.459756in}}{\pgfqpoint{8.127407in}{2.454713in}}%
\pgfpathcurveto{\pgfqpoint{8.127407in}{2.449669in}}{\pgfqpoint{8.129411in}{2.444831in}}{\pgfqpoint{8.132978in}{2.441265in}}%
\pgfpathcurveto{\pgfqpoint{8.136544in}{2.437699in}}{\pgfqpoint{8.141382in}{2.435695in}}{\pgfqpoint{8.146426in}{2.435695in}}%
\pgfpathclose%
\pgfusepath{fill}%
\end{pgfscope}%
\begin{pgfscope}%
\pgfpathrectangle{\pgfqpoint{6.572727in}{0.474100in}}{\pgfqpoint{4.227273in}{3.318700in}}%
\pgfusepath{clip}%
\pgfsetbuttcap%
\pgfsetroundjoin%
\definecolor{currentfill}{rgb}{0.127568,0.566949,0.550556}%
\pgfsetfillcolor{currentfill}%
\pgfsetfillopacity{0.700000}%
\pgfsetlinewidth{0.000000pt}%
\definecolor{currentstroke}{rgb}{0.000000,0.000000,0.000000}%
\pgfsetstrokecolor{currentstroke}%
\pgfsetstrokeopacity{0.700000}%
\pgfsetdash{}{0pt}%
\pgfpathmoveto{\pgfqpoint{9.958396in}{2.297978in}}%
\pgfpathcurveto{\pgfqpoint{9.963440in}{2.297978in}}{\pgfqpoint{9.968278in}{2.299982in}}{\pgfqpoint{9.971844in}{2.303549in}}%
\pgfpathcurveto{\pgfqpoint{9.975411in}{2.307115in}}{\pgfqpoint{9.977414in}{2.311953in}}{\pgfqpoint{9.977414in}{2.316996in}}%
\pgfpathcurveto{\pgfqpoint{9.977414in}{2.322040in}}{\pgfqpoint{9.975411in}{2.326878in}}{\pgfqpoint{9.971844in}{2.330444in}}%
\pgfpathcurveto{\pgfqpoint{9.968278in}{2.334011in}}{\pgfqpoint{9.963440in}{2.336015in}}{\pgfqpoint{9.958396in}{2.336015in}}%
\pgfpathcurveto{\pgfqpoint{9.953353in}{2.336015in}}{\pgfqpoint{9.948515in}{2.334011in}}{\pgfqpoint{9.944948in}{2.330444in}}%
\pgfpathcurveto{\pgfqpoint{9.941382in}{2.326878in}}{\pgfqpoint{9.939378in}{2.322040in}}{\pgfqpoint{9.939378in}{2.316996in}}%
\pgfpathcurveto{\pgfqpoint{9.939378in}{2.311953in}}{\pgfqpoint{9.941382in}{2.307115in}}{\pgfqpoint{9.944948in}{2.303549in}}%
\pgfpathcurveto{\pgfqpoint{9.948515in}{2.299982in}}{\pgfqpoint{9.953353in}{2.297978in}}{\pgfqpoint{9.958396in}{2.297978in}}%
\pgfpathclose%
\pgfusepath{fill}%
\end{pgfscope}%
\begin{pgfscope}%
\pgfpathrectangle{\pgfqpoint{6.572727in}{0.474100in}}{\pgfqpoint{4.227273in}{3.318700in}}%
\pgfusepath{clip}%
\pgfsetbuttcap%
\pgfsetroundjoin%
\definecolor{currentfill}{rgb}{0.993248,0.906157,0.143936}%
\pgfsetfillcolor{currentfill}%
\pgfsetfillopacity{0.700000}%
\pgfsetlinewidth{0.000000pt}%
\definecolor{currentstroke}{rgb}{0.000000,0.000000,0.000000}%
\pgfsetstrokecolor{currentstroke}%
\pgfsetstrokeopacity{0.700000}%
\pgfsetdash{}{0pt}%
\pgfpathmoveto{\pgfqpoint{8.131257in}{2.251934in}}%
\pgfpathcurveto{\pgfqpoint{8.136300in}{2.251934in}}{\pgfqpoint{8.141138in}{2.253938in}}{\pgfqpoint{8.144704in}{2.257505in}}%
\pgfpathcurveto{\pgfqpoint{8.148271in}{2.261071in}}{\pgfqpoint{8.150275in}{2.265909in}}{\pgfqpoint{8.150275in}{2.270952in}}%
\pgfpathcurveto{\pgfqpoint{8.150275in}{2.275996in}}{\pgfqpoint{8.148271in}{2.280834in}}{\pgfqpoint{8.144704in}{2.284400in}}%
\pgfpathcurveto{\pgfqpoint{8.141138in}{2.287967in}}{\pgfqpoint{8.136300in}{2.289971in}}{\pgfqpoint{8.131257in}{2.289971in}}%
\pgfpathcurveto{\pgfqpoint{8.126213in}{2.289971in}}{\pgfqpoint{8.121375in}{2.287967in}}{\pgfqpoint{8.117809in}{2.284400in}}%
\pgfpathcurveto{\pgfqpoint{8.114242in}{2.280834in}}{\pgfqpoint{8.112238in}{2.275996in}}{\pgfqpoint{8.112238in}{2.270952in}}%
\pgfpathcurveto{\pgfqpoint{8.112238in}{2.265909in}}{\pgfqpoint{8.114242in}{2.261071in}}{\pgfqpoint{8.117809in}{2.257505in}}%
\pgfpathcurveto{\pgfqpoint{8.121375in}{2.253938in}}{\pgfqpoint{8.126213in}{2.251934in}}{\pgfqpoint{8.131257in}{2.251934in}}%
\pgfpathclose%
\pgfusepath{fill}%
\end{pgfscope}%
\begin{pgfscope}%
\pgfpathrectangle{\pgfqpoint{6.572727in}{0.474100in}}{\pgfqpoint{4.227273in}{3.318700in}}%
\pgfusepath{clip}%
\pgfsetbuttcap%
\pgfsetroundjoin%
\definecolor{currentfill}{rgb}{0.267004,0.004874,0.329415}%
\pgfsetfillcolor{currentfill}%
\pgfsetfillopacity{0.700000}%
\pgfsetlinewidth{0.000000pt}%
\definecolor{currentstroke}{rgb}{0.000000,0.000000,0.000000}%
\pgfsetstrokecolor{currentstroke}%
\pgfsetstrokeopacity{0.700000}%
\pgfsetdash{}{0pt}%
\pgfpathmoveto{\pgfqpoint{8.039184in}{1.823604in}}%
\pgfpathcurveto{\pgfqpoint{8.044228in}{1.823604in}}{\pgfqpoint{8.049065in}{1.825608in}}{\pgfqpoint{8.052632in}{1.829175in}}%
\pgfpathcurveto{\pgfqpoint{8.056198in}{1.832741in}}{\pgfqpoint{8.058202in}{1.837579in}}{\pgfqpoint{8.058202in}{1.842623in}}%
\pgfpathcurveto{\pgfqpoint{8.058202in}{1.847666in}}{\pgfqpoint{8.056198in}{1.852504in}}{\pgfqpoint{8.052632in}{1.856070in}}%
\pgfpathcurveto{\pgfqpoint{8.049065in}{1.859637in}}{\pgfqpoint{8.044228in}{1.861641in}}{\pgfqpoint{8.039184in}{1.861641in}}%
\pgfpathcurveto{\pgfqpoint{8.034140in}{1.861641in}}{\pgfqpoint{8.029303in}{1.859637in}}{\pgfqpoint{8.025736in}{1.856070in}}%
\pgfpathcurveto{\pgfqpoint{8.022170in}{1.852504in}}{\pgfqpoint{8.020166in}{1.847666in}}{\pgfqpoint{8.020166in}{1.842623in}}%
\pgfpathcurveto{\pgfqpoint{8.020166in}{1.837579in}}{\pgfqpoint{8.022170in}{1.832741in}}{\pgfqpoint{8.025736in}{1.829175in}}%
\pgfpathcurveto{\pgfqpoint{8.029303in}{1.825608in}}{\pgfqpoint{8.034140in}{1.823604in}}{\pgfqpoint{8.039184in}{1.823604in}}%
\pgfpathclose%
\pgfusepath{fill}%
\end{pgfscope}%
\begin{pgfscope}%
\pgfpathrectangle{\pgfqpoint{6.572727in}{0.474100in}}{\pgfqpoint{4.227273in}{3.318700in}}%
\pgfusepath{clip}%
\pgfsetbuttcap%
\pgfsetroundjoin%
\definecolor{currentfill}{rgb}{0.267004,0.004874,0.329415}%
\pgfsetfillcolor{currentfill}%
\pgfsetfillopacity{0.700000}%
\pgfsetlinewidth{0.000000pt}%
\definecolor{currentstroke}{rgb}{0.000000,0.000000,0.000000}%
\pgfsetstrokecolor{currentstroke}%
\pgfsetstrokeopacity{0.700000}%
\pgfsetdash{}{0pt}%
\pgfpathmoveto{\pgfqpoint{8.147547in}{1.813627in}}%
\pgfpathcurveto{\pgfqpoint{8.152590in}{1.813627in}}{\pgfqpoint{8.157428in}{1.815631in}}{\pgfqpoint{8.160995in}{1.819198in}}%
\pgfpathcurveto{\pgfqpoint{8.164561in}{1.822764in}}{\pgfqpoint{8.166565in}{1.827602in}}{\pgfqpoint{8.166565in}{1.832646in}}%
\pgfpathcurveto{\pgfqpoint{8.166565in}{1.837689in}}{\pgfqpoint{8.164561in}{1.842527in}}{\pgfqpoint{8.160995in}{1.846094in}}%
\pgfpathcurveto{\pgfqpoint{8.157428in}{1.849660in}}{\pgfqpoint{8.152590in}{1.851664in}}{\pgfqpoint{8.147547in}{1.851664in}}%
\pgfpathcurveto{\pgfqpoint{8.142503in}{1.851664in}}{\pgfqpoint{8.137665in}{1.849660in}}{\pgfqpoint{8.134099in}{1.846094in}}%
\pgfpathcurveto{\pgfqpoint{8.130532in}{1.842527in}}{\pgfqpoint{8.128529in}{1.837689in}}{\pgfqpoint{8.128529in}{1.832646in}}%
\pgfpathcurveto{\pgfqpoint{8.128529in}{1.827602in}}{\pgfqpoint{8.130532in}{1.822764in}}{\pgfqpoint{8.134099in}{1.819198in}}%
\pgfpathcurveto{\pgfqpoint{8.137665in}{1.815631in}}{\pgfqpoint{8.142503in}{1.813627in}}{\pgfqpoint{8.147547in}{1.813627in}}%
\pgfpathclose%
\pgfusepath{fill}%
\end{pgfscope}%
\begin{pgfscope}%
\pgfpathrectangle{\pgfqpoint{6.572727in}{0.474100in}}{\pgfqpoint{4.227273in}{3.318700in}}%
\pgfusepath{clip}%
\pgfsetbuttcap%
\pgfsetroundjoin%
\definecolor{currentfill}{rgb}{0.267004,0.004874,0.329415}%
\pgfsetfillcolor{currentfill}%
\pgfsetfillopacity{0.700000}%
\pgfsetlinewidth{0.000000pt}%
\definecolor{currentstroke}{rgb}{0.000000,0.000000,0.000000}%
\pgfsetstrokecolor{currentstroke}%
\pgfsetstrokeopacity{0.700000}%
\pgfsetdash{}{0pt}%
\pgfpathmoveto{\pgfqpoint{7.346350in}{1.388649in}}%
\pgfpathcurveto{\pgfqpoint{7.351394in}{1.388649in}}{\pgfqpoint{7.356232in}{1.390653in}}{\pgfqpoint{7.359798in}{1.394220in}}%
\pgfpathcurveto{\pgfqpoint{7.363365in}{1.397786in}}{\pgfqpoint{7.365368in}{1.402624in}}{\pgfqpoint{7.365368in}{1.407667in}}%
\pgfpathcurveto{\pgfqpoint{7.365368in}{1.412711in}}{\pgfqpoint{7.363365in}{1.417549in}}{\pgfqpoint{7.359798in}{1.421115in}}%
\pgfpathcurveto{\pgfqpoint{7.356232in}{1.424682in}}{\pgfqpoint{7.351394in}{1.426686in}}{\pgfqpoint{7.346350in}{1.426686in}}%
\pgfpathcurveto{\pgfqpoint{7.341307in}{1.426686in}}{\pgfqpoint{7.336469in}{1.424682in}}{\pgfqpoint{7.332902in}{1.421115in}}%
\pgfpathcurveto{\pgfqpoint{7.329336in}{1.417549in}}{\pgfqpoint{7.327332in}{1.412711in}}{\pgfqpoint{7.327332in}{1.407667in}}%
\pgfpathcurveto{\pgfqpoint{7.327332in}{1.402624in}}{\pgfqpoint{7.329336in}{1.397786in}}{\pgfqpoint{7.332902in}{1.394220in}}%
\pgfpathcurveto{\pgfqpoint{7.336469in}{1.390653in}}{\pgfqpoint{7.341307in}{1.388649in}}{\pgfqpoint{7.346350in}{1.388649in}}%
\pgfpathclose%
\pgfusepath{fill}%
\end{pgfscope}%
\begin{pgfscope}%
\pgfpathrectangle{\pgfqpoint{6.572727in}{0.474100in}}{\pgfqpoint{4.227273in}{3.318700in}}%
\pgfusepath{clip}%
\pgfsetbuttcap%
\pgfsetroundjoin%
\definecolor{currentfill}{rgb}{0.127568,0.566949,0.550556}%
\pgfsetfillcolor{currentfill}%
\pgfsetfillopacity{0.700000}%
\pgfsetlinewidth{0.000000pt}%
\definecolor{currentstroke}{rgb}{0.000000,0.000000,0.000000}%
\pgfsetstrokecolor{currentstroke}%
\pgfsetstrokeopacity{0.700000}%
\pgfsetdash{}{0pt}%
\pgfpathmoveto{\pgfqpoint{9.787159in}{1.349847in}}%
\pgfpathcurveto{\pgfqpoint{9.792203in}{1.349847in}}{\pgfqpoint{9.797040in}{1.351851in}}{\pgfqpoint{9.800607in}{1.355418in}}%
\pgfpathcurveto{\pgfqpoint{9.804173in}{1.358984in}}{\pgfqpoint{9.806177in}{1.363822in}}{\pgfqpoint{9.806177in}{1.368865in}}%
\pgfpathcurveto{\pgfqpoint{9.806177in}{1.373909in}}{\pgfqpoint{9.804173in}{1.378747in}}{\pgfqpoint{9.800607in}{1.382313in}}%
\pgfpathcurveto{\pgfqpoint{9.797040in}{1.385880in}}{\pgfqpoint{9.792203in}{1.387884in}}{\pgfqpoint{9.787159in}{1.387884in}}%
\pgfpathcurveto{\pgfqpoint{9.782115in}{1.387884in}}{\pgfqpoint{9.777278in}{1.385880in}}{\pgfqpoint{9.773711in}{1.382313in}}%
\pgfpathcurveto{\pgfqpoint{9.770145in}{1.378747in}}{\pgfqpoint{9.768141in}{1.373909in}}{\pgfqpoint{9.768141in}{1.368865in}}%
\pgfpathcurveto{\pgfqpoint{9.768141in}{1.363822in}}{\pgfqpoint{9.770145in}{1.358984in}}{\pgfqpoint{9.773711in}{1.355418in}}%
\pgfpathcurveto{\pgfqpoint{9.777278in}{1.351851in}}{\pgfqpoint{9.782115in}{1.349847in}}{\pgfqpoint{9.787159in}{1.349847in}}%
\pgfpathclose%
\pgfusepath{fill}%
\end{pgfscope}%
\begin{pgfscope}%
\pgfpathrectangle{\pgfqpoint{6.572727in}{0.474100in}}{\pgfqpoint{4.227273in}{3.318700in}}%
\pgfusepath{clip}%
\pgfsetbuttcap%
\pgfsetroundjoin%
\definecolor{currentfill}{rgb}{0.993248,0.906157,0.143936}%
\pgfsetfillcolor{currentfill}%
\pgfsetfillopacity{0.700000}%
\pgfsetlinewidth{0.000000pt}%
\definecolor{currentstroke}{rgb}{0.000000,0.000000,0.000000}%
\pgfsetstrokecolor{currentstroke}%
\pgfsetstrokeopacity{0.700000}%
\pgfsetdash{}{0pt}%
\pgfpathmoveto{\pgfqpoint{8.359853in}{2.527548in}}%
\pgfpathcurveto{\pgfqpoint{8.364896in}{2.527548in}}{\pgfqpoint{8.369734in}{2.529552in}}{\pgfqpoint{8.373300in}{2.533118in}}%
\pgfpathcurveto{\pgfqpoint{8.376867in}{2.536684in}}{\pgfqpoint{8.378871in}{2.541522in}}{\pgfqpoint{8.378871in}{2.546566in}}%
\pgfpathcurveto{\pgfqpoint{8.378871in}{2.551610in}}{\pgfqpoint{8.376867in}{2.556447in}}{\pgfqpoint{8.373300in}{2.560014in}}%
\pgfpathcurveto{\pgfqpoint{8.369734in}{2.563580in}}{\pgfqpoint{8.364896in}{2.565584in}}{\pgfqpoint{8.359853in}{2.565584in}}%
\pgfpathcurveto{\pgfqpoint{8.354809in}{2.565584in}}{\pgfqpoint{8.349971in}{2.563580in}}{\pgfqpoint{8.346405in}{2.560014in}}%
\pgfpathcurveto{\pgfqpoint{8.342838in}{2.556447in}}{\pgfqpoint{8.340834in}{2.551610in}}{\pgfqpoint{8.340834in}{2.546566in}}%
\pgfpathcurveto{\pgfqpoint{8.340834in}{2.541522in}}{\pgfqpoint{8.342838in}{2.536684in}}{\pgfqpoint{8.346405in}{2.533118in}}%
\pgfpathcurveto{\pgfqpoint{8.349971in}{2.529552in}}{\pgfqpoint{8.354809in}{2.527548in}}{\pgfqpoint{8.359853in}{2.527548in}}%
\pgfpathclose%
\pgfusepath{fill}%
\end{pgfscope}%
\begin{pgfscope}%
\pgfpathrectangle{\pgfqpoint{6.572727in}{0.474100in}}{\pgfqpoint{4.227273in}{3.318700in}}%
\pgfusepath{clip}%
\pgfsetbuttcap%
\pgfsetroundjoin%
\definecolor{currentfill}{rgb}{0.127568,0.566949,0.550556}%
\pgfsetfillcolor{currentfill}%
\pgfsetfillopacity{0.700000}%
\pgfsetlinewidth{0.000000pt}%
\definecolor{currentstroke}{rgb}{0.000000,0.000000,0.000000}%
\pgfsetstrokecolor{currentstroke}%
\pgfsetstrokeopacity{0.700000}%
\pgfsetdash{}{0pt}%
\pgfpathmoveto{\pgfqpoint{9.451051in}{1.268015in}}%
\pgfpathcurveto{\pgfqpoint{9.456095in}{1.268015in}}{\pgfqpoint{9.460933in}{1.270019in}}{\pgfqpoint{9.464499in}{1.273585in}}%
\pgfpathcurveto{\pgfqpoint{9.468066in}{1.277152in}}{\pgfqpoint{9.470070in}{1.281990in}}{\pgfqpoint{9.470070in}{1.287033in}}%
\pgfpathcurveto{\pgfqpoint{9.470070in}{1.292077in}}{\pgfqpoint{9.468066in}{1.296915in}}{\pgfqpoint{9.464499in}{1.300481in}}%
\pgfpathcurveto{\pgfqpoint{9.460933in}{1.304047in}}{\pgfqpoint{9.456095in}{1.306051in}}{\pgfqpoint{9.451051in}{1.306051in}}%
\pgfpathcurveto{\pgfqpoint{9.446008in}{1.306051in}}{\pgfqpoint{9.441170in}{1.304047in}}{\pgfqpoint{9.437604in}{1.300481in}}%
\pgfpathcurveto{\pgfqpoint{9.434037in}{1.296915in}}{\pgfqpoint{9.432033in}{1.292077in}}{\pgfqpoint{9.432033in}{1.287033in}}%
\pgfpathcurveto{\pgfqpoint{9.432033in}{1.281990in}}{\pgfqpoint{9.434037in}{1.277152in}}{\pgfqpoint{9.437604in}{1.273585in}}%
\pgfpathcurveto{\pgfqpoint{9.441170in}{1.270019in}}{\pgfqpoint{9.446008in}{1.268015in}}{\pgfqpoint{9.451051in}{1.268015in}}%
\pgfpathclose%
\pgfusepath{fill}%
\end{pgfscope}%
\begin{pgfscope}%
\pgfpathrectangle{\pgfqpoint{6.572727in}{0.474100in}}{\pgfqpoint{4.227273in}{3.318700in}}%
\pgfusepath{clip}%
\pgfsetbuttcap%
\pgfsetroundjoin%
\definecolor{currentfill}{rgb}{0.993248,0.906157,0.143936}%
\pgfsetfillcolor{currentfill}%
\pgfsetfillopacity{0.700000}%
\pgfsetlinewidth{0.000000pt}%
\definecolor{currentstroke}{rgb}{0.000000,0.000000,0.000000}%
\pgfsetstrokecolor{currentstroke}%
\pgfsetstrokeopacity{0.700000}%
\pgfsetdash{}{0pt}%
\pgfpathmoveto{\pgfqpoint{8.622912in}{2.789715in}}%
\pgfpathcurveto{\pgfqpoint{8.627956in}{2.789715in}}{\pgfqpoint{8.632794in}{2.791718in}}{\pgfqpoint{8.636360in}{2.795285in}}%
\pgfpathcurveto{\pgfqpoint{8.639927in}{2.798851in}}{\pgfqpoint{8.641931in}{2.803689in}}{\pgfqpoint{8.641931in}{2.808733in}}%
\pgfpathcurveto{\pgfqpoint{8.641931in}{2.813776in}}{\pgfqpoint{8.639927in}{2.818614in}}{\pgfqpoint{8.636360in}{2.822181in}}%
\pgfpathcurveto{\pgfqpoint{8.632794in}{2.825747in}}{\pgfqpoint{8.627956in}{2.827751in}}{\pgfqpoint{8.622912in}{2.827751in}}%
\pgfpathcurveto{\pgfqpoint{8.617869in}{2.827751in}}{\pgfqpoint{8.613031in}{2.825747in}}{\pgfqpoint{8.609465in}{2.822181in}}%
\pgfpathcurveto{\pgfqpoint{8.605898in}{2.818614in}}{\pgfqpoint{8.603894in}{2.813776in}}{\pgfqpoint{8.603894in}{2.808733in}}%
\pgfpathcurveto{\pgfqpoint{8.603894in}{2.803689in}}{\pgfqpoint{8.605898in}{2.798851in}}{\pgfqpoint{8.609465in}{2.795285in}}%
\pgfpathcurveto{\pgfqpoint{8.613031in}{2.791718in}}{\pgfqpoint{8.617869in}{2.789715in}}{\pgfqpoint{8.622912in}{2.789715in}}%
\pgfpathclose%
\pgfusepath{fill}%
\end{pgfscope}%
\begin{pgfscope}%
\pgfpathrectangle{\pgfqpoint{6.572727in}{0.474100in}}{\pgfqpoint{4.227273in}{3.318700in}}%
\pgfusepath{clip}%
\pgfsetbuttcap%
\pgfsetroundjoin%
\definecolor{currentfill}{rgb}{0.127568,0.566949,0.550556}%
\pgfsetfillcolor{currentfill}%
\pgfsetfillopacity{0.700000}%
\pgfsetlinewidth{0.000000pt}%
\definecolor{currentstroke}{rgb}{0.000000,0.000000,0.000000}%
\pgfsetstrokecolor{currentstroke}%
\pgfsetstrokeopacity{0.700000}%
\pgfsetdash{}{0pt}%
\pgfpathmoveto{\pgfqpoint{9.626565in}{2.293616in}}%
\pgfpathcurveto{\pgfqpoint{9.631608in}{2.293616in}}{\pgfqpoint{9.636446in}{2.295620in}}{\pgfqpoint{9.640013in}{2.299187in}}%
\pgfpathcurveto{\pgfqpoint{9.643579in}{2.302753in}}{\pgfqpoint{9.645583in}{2.307591in}}{\pgfqpoint{9.645583in}{2.312634in}}%
\pgfpathcurveto{\pgfqpoint{9.645583in}{2.317678in}}{\pgfqpoint{9.643579in}{2.322516in}}{\pgfqpoint{9.640013in}{2.326082in}}%
\pgfpathcurveto{\pgfqpoint{9.636446in}{2.329649in}}{\pgfqpoint{9.631608in}{2.331653in}}{\pgfqpoint{9.626565in}{2.331653in}}%
\pgfpathcurveto{\pgfqpoint{9.621521in}{2.331653in}}{\pgfqpoint{9.616683in}{2.329649in}}{\pgfqpoint{9.613117in}{2.326082in}}%
\pgfpathcurveto{\pgfqpoint{9.609551in}{2.322516in}}{\pgfqpoint{9.607547in}{2.317678in}}{\pgfqpoint{9.607547in}{2.312634in}}%
\pgfpathcurveto{\pgfqpoint{9.607547in}{2.307591in}}{\pgfqpoint{9.609551in}{2.302753in}}{\pgfqpoint{9.613117in}{2.299187in}}%
\pgfpathcurveto{\pgfqpoint{9.616683in}{2.295620in}}{\pgfqpoint{9.621521in}{2.293616in}}{\pgfqpoint{9.626565in}{2.293616in}}%
\pgfpathclose%
\pgfusepath{fill}%
\end{pgfscope}%
\begin{pgfscope}%
\pgfpathrectangle{\pgfqpoint{6.572727in}{0.474100in}}{\pgfqpoint{4.227273in}{3.318700in}}%
\pgfusepath{clip}%
\pgfsetbuttcap%
\pgfsetroundjoin%
\definecolor{currentfill}{rgb}{0.127568,0.566949,0.550556}%
\pgfsetfillcolor{currentfill}%
\pgfsetfillopacity{0.700000}%
\pgfsetlinewidth{0.000000pt}%
\definecolor{currentstroke}{rgb}{0.000000,0.000000,0.000000}%
\pgfsetstrokecolor{currentstroke}%
\pgfsetstrokeopacity{0.700000}%
\pgfsetdash{}{0pt}%
\pgfpathmoveto{\pgfqpoint{10.067406in}{1.084939in}}%
\pgfpathcurveto{\pgfqpoint{10.072449in}{1.084939in}}{\pgfqpoint{10.077287in}{1.086943in}}{\pgfqpoint{10.080853in}{1.090509in}}%
\pgfpathcurveto{\pgfqpoint{10.084420in}{1.094076in}}{\pgfqpoint{10.086424in}{1.098914in}}{\pgfqpoint{10.086424in}{1.103957in}}%
\pgfpathcurveto{\pgfqpoint{10.086424in}{1.109001in}}{\pgfqpoint{10.084420in}{1.113839in}}{\pgfqpoint{10.080853in}{1.117405in}}%
\pgfpathcurveto{\pgfqpoint{10.077287in}{1.120972in}}{\pgfqpoint{10.072449in}{1.122975in}}{\pgfqpoint{10.067406in}{1.122975in}}%
\pgfpathcurveto{\pgfqpoint{10.062362in}{1.122975in}}{\pgfqpoint{10.057524in}{1.120972in}}{\pgfqpoint{10.053958in}{1.117405in}}%
\pgfpathcurveto{\pgfqpoint{10.050391in}{1.113839in}}{\pgfqpoint{10.048387in}{1.109001in}}{\pgfqpoint{10.048387in}{1.103957in}}%
\pgfpathcurveto{\pgfqpoint{10.048387in}{1.098914in}}{\pgfqpoint{10.050391in}{1.094076in}}{\pgfqpoint{10.053958in}{1.090509in}}%
\pgfpathcurveto{\pgfqpoint{10.057524in}{1.086943in}}{\pgfqpoint{10.062362in}{1.084939in}}{\pgfqpoint{10.067406in}{1.084939in}}%
\pgfpathclose%
\pgfusepath{fill}%
\end{pgfscope}%
\begin{pgfscope}%
\pgfpathrectangle{\pgfqpoint{6.572727in}{0.474100in}}{\pgfqpoint{4.227273in}{3.318700in}}%
\pgfusepath{clip}%
\pgfsetbuttcap%
\pgfsetroundjoin%
\definecolor{currentfill}{rgb}{0.267004,0.004874,0.329415}%
\pgfsetfillcolor{currentfill}%
\pgfsetfillopacity{0.700000}%
\pgfsetlinewidth{0.000000pt}%
\definecolor{currentstroke}{rgb}{0.000000,0.000000,0.000000}%
\pgfsetstrokecolor{currentstroke}%
\pgfsetstrokeopacity{0.700000}%
\pgfsetdash{}{0pt}%
\pgfpathmoveto{\pgfqpoint{7.406433in}{1.076929in}}%
\pgfpathcurveto{\pgfqpoint{7.411477in}{1.076929in}}{\pgfqpoint{7.416315in}{1.078933in}}{\pgfqpoint{7.419881in}{1.082500in}}%
\pgfpathcurveto{\pgfqpoint{7.423448in}{1.086066in}}{\pgfqpoint{7.425451in}{1.090904in}}{\pgfqpoint{7.425451in}{1.095947in}}%
\pgfpathcurveto{\pgfqpoint{7.425451in}{1.100991in}}{\pgfqpoint{7.423448in}{1.105829in}}{\pgfqpoint{7.419881in}{1.109395in}}%
\pgfpathcurveto{\pgfqpoint{7.416315in}{1.112962in}}{\pgfqpoint{7.411477in}{1.114966in}}{\pgfqpoint{7.406433in}{1.114966in}}%
\pgfpathcurveto{\pgfqpoint{7.401390in}{1.114966in}}{\pgfqpoint{7.396552in}{1.112962in}}{\pgfqpoint{7.392985in}{1.109395in}}%
\pgfpathcurveto{\pgfqpoint{7.389419in}{1.105829in}}{\pgfqpoint{7.387415in}{1.100991in}}{\pgfqpoint{7.387415in}{1.095947in}}%
\pgfpathcurveto{\pgfqpoint{7.387415in}{1.090904in}}{\pgfqpoint{7.389419in}{1.086066in}}{\pgfqpoint{7.392985in}{1.082500in}}%
\pgfpathcurveto{\pgfqpoint{7.396552in}{1.078933in}}{\pgfqpoint{7.401390in}{1.076929in}}{\pgfqpoint{7.406433in}{1.076929in}}%
\pgfpathclose%
\pgfusepath{fill}%
\end{pgfscope}%
\begin{pgfscope}%
\pgfpathrectangle{\pgfqpoint{6.572727in}{0.474100in}}{\pgfqpoint{4.227273in}{3.318700in}}%
\pgfusepath{clip}%
\pgfsetbuttcap%
\pgfsetroundjoin%
\definecolor{currentfill}{rgb}{0.267004,0.004874,0.329415}%
\pgfsetfillcolor{currentfill}%
\pgfsetfillopacity{0.700000}%
\pgfsetlinewidth{0.000000pt}%
\definecolor{currentstroke}{rgb}{0.000000,0.000000,0.000000}%
\pgfsetstrokecolor{currentstroke}%
\pgfsetstrokeopacity{0.700000}%
\pgfsetdash{}{0pt}%
\pgfpathmoveto{\pgfqpoint{7.584939in}{0.924488in}}%
\pgfpathcurveto{\pgfqpoint{7.589983in}{0.924488in}}{\pgfqpoint{7.594821in}{0.926492in}}{\pgfqpoint{7.598387in}{0.930059in}}%
\pgfpathcurveto{\pgfqpoint{7.601954in}{0.933625in}}{\pgfqpoint{7.603957in}{0.938463in}}{\pgfqpoint{7.603957in}{0.943507in}}%
\pgfpathcurveto{\pgfqpoint{7.603957in}{0.948550in}}{\pgfqpoint{7.601954in}{0.953388in}}{\pgfqpoint{7.598387in}{0.956954in}}%
\pgfpathcurveto{\pgfqpoint{7.594821in}{0.960521in}}{\pgfqpoint{7.589983in}{0.962525in}}{\pgfqpoint{7.584939in}{0.962525in}}%
\pgfpathcurveto{\pgfqpoint{7.579896in}{0.962525in}}{\pgfqpoint{7.575058in}{0.960521in}}{\pgfqpoint{7.571491in}{0.956954in}}%
\pgfpathcurveto{\pgfqpoint{7.567925in}{0.953388in}}{\pgfqpoint{7.565921in}{0.948550in}}{\pgfqpoint{7.565921in}{0.943507in}}%
\pgfpathcurveto{\pgfqpoint{7.565921in}{0.938463in}}{\pgfqpoint{7.567925in}{0.933625in}}{\pgfqpoint{7.571491in}{0.930059in}}%
\pgfpathcurveto{\pgfqpoint{7.575058in}{0.926492in}}{\pgfqpoint{7.579896in}{0.924488in}}{\pgfqpoint{7.584939in}{0.924488in}}%
\pgfpathclose%
\pgfusepath{fill}%
\end{pgfscope}%
\begin{pgfscope}%
\pgfpathrectangle{\pgfqpoint{6.572727in}{0.474100in}}{\pgfqpoint{4.227273in}{3.318700in}}%
\pgfusepath{clip}%
\pgfsetbuttcap%
\pgfsetroundjoin%
\definecolor{currentfill}{rgb}{0.267004,0.004874,0.329415}%
\pgfsetfillcolor{currentfill}%
\pgfsetfillopacity{0.700000}%
\pgfsetlinewidth{0.000000pt}%
\definecolor{currentstroke}{rgb}{0.000000,0.000000,0.000000}%
\pgfsetstrokecolor{currentstroke}%
\pgfsetstrokeopacity{0.700000}%
\pgfsetdash{}{0pt}%
\pgfpathmoveto{\pgfqpoint{7.637912in}{1.399358in}}%
\pgfpathcurveto{\pgfqpoint{7.642955in}{1.399358in}}{\pgfqpoint{7.647793in}{1.401362in}}{\pgfqpoint{7.651360in}{1.404928in}}%
\pgfpathcurveto{\pgfqpoint{7.654926in}{1.408494in}}{\pgfqpoint{7.656930in}{1.413332in}}{\pgfqpoint{7.656930in}{1.418376in}}%
\pgfpathcurveto{\pgfqpoint{7.656930in}{1.423420in}}{\pgfqpoint{7.654926in}{1.428257in}}{\pgfqpoint{7.651360in}{1.431824in}}%
\pgfpathcurveto{\pgfqpoint{7.647793in}{1.435390in}}{\pgfqpoint{7.642955in}{1.437394in}}{\pgfqpoint{7.637912in}{1.437394in}}%
\pgfpathcurveto{\pgfqpoint{7.632868in}{1.437394in}}{\pgfqpoint{7.628030in}{1.435390in}}{\pgfqpoint{7.624464in}{1.431824in}}%
\pgfpathcurveto{\pgfqpoint{7.620898in}{1.428257in}}{\pgfqpoint{7.618894in}{1.423420in}}{\pgfqpoint{7.618894in}{1.418376in}}%
\pgfpathcurveto{\pgfqpoint{7.618894in}{1.413332in}}{\pgfqpoint{7.620898in}{1.408494in}}{\pgfqpoint{7.624464in}{1.404928in}}%
\pgfpathcurveto{\pgfqpoint{7.628030in}{1.401362in}}{\pgfqpoint{7.632868in}{1.399358in}}{\pgfqpoint{7.637912in}{1.399358in}}%
\pgfpathclose%
\pgfusepath{fill}%
\end{pgfscope}%
\begin{pgfscope}%
\pgfpathrectangle{\pgfqpoint{6.572727in}{0.474100in}}{\pgfqpoint{4.227273in}{3.318700in}}%
\pgfusepath{clip}%
\pgfsetbuttcap%
\pgfsetroundjoin%
\definecolor{currentfill}{rgb}{0.267004,0.004874,0.329415}%
\pgfsetfillcolor{currentfill}%
\pgfsetfillopacity{0.700000}%
\pgfsetlinewidth{0.000000pt}%
\definecolor{currentstroke}{rgb}{0.000000,0.000000,0.000000}%
\pgfsetstrokecolor{currentstroke}%
\pgfsetstrokeopacity{0.700000}%
\pgfsetdash{}{0pt}%
\pgfpathmoveto{\pgfqpoint{7.696788in}{1.261967in}}%
\pgfpathcurveto{\pgfqpoint{7.701831in}{1.261967in}}{\pgfqpoint{7.706669in}{1.263971in}}{\pgfqpoint{7.710236in}{1.267537in}}%
\pgfpathcurveto{\pgfqpoint{7.713802in}{1.271103in}}{\pgfqpoint{7.715806in}{1.275941in}}{\pgfqpoint{7.715806in}{1.280985in}}%
\pgfpathcurveto{\pgfqpoint{7.715806in}{1.286028in}}{\pgfqpoint{7.713802in}{1.290866in}}{\pgfqpoint{7.710236in}{1.294433in}}%
\pgfpathcurveto{\pgfqpoint{7.706669in}{1.297999in}}{\pgfqpoint{7.701831in}{1.300003in}}{\pgfqpoint{7.696788in}{1.300003in}}%
\pgfpathcurveto{\pgfqpoint{7.691744in}{1.300003in}}{\pgfqpoint{7.686906in}{1.297999in}}{\pgfqpoint{7.683340in}{1.294433in}}%
\pgfpathcurveto{\pgfqpoint{7.679773in}{1.290866in}}{\pgfqpoint{7.677770in}{1.286028in}}{\pgfqpoint{7.677770in}{1.280985in}}%
\pgfpathcurveto{\pgfqpoint{7.677770in}{1.275941in}}{\pgfqpoint{7.679773in}{1.271103in}}{\pgfqpoint{7.683340in}{1.267537in}}%
\pgfpathcurveto{\pgfqpoint{7.686906in}{1.263971in}}{\pgfqpoint{7.691744in}{1.261967in}}{\pgfqpoint{7.696788in}{1.261967in}}%
\pgfpathclose%
\pgfusepath{fill}%
\end{pgfscope}%
\begin{pgfscope}%
\pgfpathrectangle{\pgfqpoint{6.572727in}{0.474100in}}{\pgfqpoint{4.227273in}{3.318700in}}%
\pgfusepath{clip}%
\pgfsetbuttcap%
\pgfsetroundjoin%
\definecolor{currentfill}{rgb}{0.993248,0.906157,0.143936}%
\pgfsetfillcolor{currentfill}%
\pgfsetfillopacity{0.700000}%
\pgfsetlinewidth{0.000000pt}%
\definecolor{currentstroke}{rgb}{0.000000,0.000000,0.000000}%
\pgfsetstrokecolor{currentstroke}%
\pgfsetstrokeopacity{0.700000}%
\pgfsetdash{}{0pt}%
\pgfpathmoveto{\pgfqpoint{7.882664in}{2.320399in}}%
\pgfpathcurveto{\pgfqpoint{7.887708in}{2.320399in}}{\pgfqpoint{7.892545in}{2.322403in}}{\pgfqpoint{7.896112in}{2.325969in}}%
\pgfpathcurveto{\pgfqpoint{7.899678in}{2.329536in}}{\pgfqpoint{7.901682in}{2.334373in}}{\pgfqpoint{7.901682in}{2.339417in}}%
\pgfpathcurveto{\pgfqpoint{7.901682in}{2.344461in}}{\pgfqpoint{7.899678in}{2.349299in}}{\pgfqpoint{7.896112in}{2.352865in}}%
\pgfpathcurveto{\pgfqpoint{7.892545in}{2.356431in}}{\pgfqpoint{7.887708in}{2.358435in}}{\pgfqpoint{7.882664in}{2.358435in}}%
\pgfpathcurveto{\pgfqpoint{7.877620in}{2.358435in}}{\pgfqpoint{7.872783in}{2.356431in}}{\pgfqpoint{7.869216in}{2.352865in}}%
\pgfpathcurveto{\pgfqpoint{7.865650in}{2.349299in}}{\pgfqpoint{7.863646in}{2.344461in}}{\pgfqpoint{7.863646in}{2.339417in}}%
\pgfpathcurveto{\pgfqpoint{7.863646in}{2.334373in}}{\pgfqpoint{7.865650in}{2.329536in}}{\pgfqpoint{7.869216in}{2.325969in}}%
\pgfpathcurveto{\pgfqpoint{7.872783in}{2.322403in}}{\pgfqpoint{7.877620in}{2.320399in}}{\pgfqpoint{7.882664in}{2.320399in}}%
\pgfpathclose%
\pgfusepath{fill}%
\end{pgfscope}%
\begin{pgfscope}%
\pgfpathrectangle{\pgfqpoint{6.572727in}{0.474100in}}{\pgfqpoint{4.227273in}{3.318700in}}%
\pgfusepath{clip}%
\pgfsetbuttcap%
\pgfsetroundjoin%
\definecolor{currentfill}{rgb}{0.127568,0.566949,0.550556}%
\pgfsetfillcolor{currentfill}%
\pgfsetfillopacity{0.700000}%
\pgfsetlinewidth{0.000000pt}%
\definecolor{currentstroke}{rgb}{0.000000,0.000000,0.000000}%
\pgfsetstrokecolor{currentstroke}%
\pgfsetstrokeopacity{0.700000}%
\pgfsetdash{}{0pt}%
\pgfpathmoveto{\pgfqpoint{9.492020in}{2.408279in}}%
\pgfpathcurveto{\pgfqpoint{9.497064in}{2.408279in}}{\pgfqpoint{9.501902in}{2.410283in}}{\pgfqpoint{9.505468in}{2.413849in}}%
\pgfpathcurveto{\pgfqpoint{9.509035in}{2.417415in}}{\pgfqpoint{9.511039in}{2.422253in}}{\pgfqpoint{9.511039in}{2.427297in}}%
\pgfpathcurveto{\pgfqpoint{9.511039in}{2.432341in}}{\pgfqpoint{9.509035in}{2.437178in}}{\pgfqpoint{9.505468in}{2.440745in}}%
\pgfpathcurveto{\pgfqpoint{9.501902in}{2.444311in}}{\pgfqpoint{9.497064in}{2.446315in}}{\pgfqpoint{9.492020in}{2.446315in}}%
\pgfpathcurveto{\pgfqpoint{9.486977in}{2.446315in}}{\pgfqpoint{9.482139in}{2.444311in}}{\pgfqpoint{9.478573in}{2.440745in}}%
\pgfpathcurveto{\pgfqpoint{9.475006in}{2.437178in}}{\pgfqpoint{9.473002in}{2.432341in}}{\pgfqpoint{9.473002in}{2.427297in}}%
\pgfpathcurveto{\pgfqpoint{9.473002in}{2.422253in}}{\pgfqpoint{9.475006in}{2.417415in}}{\pgfqpoint{9.478573in}{2.413849in}}%
\pgfpathcurveto{\pgfqpoint{9.482139in}{2.410283in}}{\pgfqpoint{9.486977in}{2.408279in}}{\pgfqpoint{9.492020in}{2.408279in}}%
\pgfpathclose%
\pgfusepath{fill}%
\end{pgfscope}%
\begin{pgfscope}%
\pgfpathrectangle{\pgfqpoint{6.572727in}{0.474100in}}{\pgfqpoint{4.227273in}{3.318700in}}%
\pgfusepath{clip}%
\pgfsetbuttcap%
\pgfsetroundjoin%
\definecolor{currentfill}{rgb}{0.993248,0.906157,0.143936}%
\pgfsetfillcolor{currentfill}%
\pgfsetfillopacity{0.700000}%
\pgfsetlinewidth{0.000000pt}%
\definecolor{currentstroke}{rgb}{0.000000,0.000000,0.000000}%
\pgfsetstrokecolor{currentstroke}%
\pgfsetstrokeopacity{0.700000}%
\pgfsetdash{}{0pt}%
\pgfpathmoveto{\pgfqpoint{8.117132in}{2.619036in}}%
\pgfpathcurveto{\pgfqpoint{8.122176in}{2.619036in}}{\pgfqpoint{8.127013in}{2.621040in}}{\pgfqpoint{8.130580in}{2.624606in}}%
\pgfpathcurveto{\pgfqpoint{8.134146in}{2.628173in}}{\pgfqpoint{8.136150in}{2.633010in}}{\pgfqpoint{8.136150in}{2.638054in}}%
\pgfpathcurveto{\pgfqpoint{8.136150in}{2.643098in}}{\pgfqpoint{8.134146in}{2.647936in}}{\pgfqpoint{8.130580in}{2.651502in}}%
\pgfpathcurveto{\pgfqpoint{8.127013in}{2.655068in}}{\pgfqpoint{8.122176in}{2.657072in}}{\pgfqpoint{8.117132in}{2.657072in}}%
\pgfpathcurveto{\pgfqpoint{8.112088in}{2.657072in}}{\pgfqpoint{8.107250in}{2.655068in}}{\pgfqpoint{8.103684in}{2.651502in}}%
\pgfpathcurveto{\pgfqpoint{8.100118in}{2.647936in}}{\pgfqpoint{8.098114in}{2.643098in}}{\pgfqpoint{8.098114in}{2.638054in}}%
\pgfpathcurveto{\pgfqpoint{8.098114in}{2.633010in}}{\pgfqpoint{8.100118in}{2.628173in}}{\pgfqpoint{8.103684in}{2.624606in}}%
\pgfpathcurveto{\pgfqpoint{8.107250in}{2.621040in}}{\pgfqpoint{8.112088in}{2.619036in}}{\pgfqpoint{8.117132in}{2.619036in}}%
\pgfpathclose%
\pgfusepath{fill}%
\end{pgfscope}%
\begin{pgfscope}%
\pgfpathrectangle{\pgfqpoint{6.572727in}{0.474100in}}{\pgfqpoint{4.227273in}{3.318700in}}%
\pgfusepath{clip}%
\pgfsetbuttcap%
\pgfsetroundjoin%
\definecolor{currentfill}{rgb}{0.267004,0.004874,0.329415}%
\pgfsetfillcolor{currentfill}%
\pgfsetfillopacity{0.700000}%
\pgfsetlinewidth{0.000000pt}%
\definecolor{currentstroke}{rgb}{0.000000,0.000000,0.000000}%
\pgfsetstrokecolor{currentstroke}%
\pgfsetstrokeopacity{0.700000}%
\pgfsetdash{}{0pt}%
\pgfpathmoveto{\pgfqpoint{7.589798in}{1.772762in}}%
\pgfpathcurveto{\pgfqpoint{7.594842in}{1.772762in}}{\pgfqpoint{7.599680in}{1.774766in}}{\pgfqpoint{7.603246in}{1.778333in}}%
\pgfpathcurveto{\pgfqpoint{7.606812in}{1.781899in}}{\pgfqpoint{7.608816in}{1.786737in}}{\pgfqpoint{7.608816in}{1.791780in}}%
\pgfpathcurveto{\pgfqpoint{7.608816in}{1.796824in}}{\pgfqpoint{7.606812in}{1.801662in}}{\pgfqpoint{7.603246in}{1.805228in}}%
\pgfpathcurveto{\pgfqpoint{7.599680in}{1.808795in}}{\pgfqpoint{7.594842in}{1.810799in}}{\pgfqpoint{7.589798in}{1.810799in}}%
\pgfpathcurveto{\pgfqpoint{7.584754in}{1.810799in}}{\pgfqpoint{7.579917in}{1.808795in}}{\pgfqpoint{7.576350in}{1.805228in}}%
\pgfpathcurveto{\pgfqpoint{7.572784in}{1.801662in}}{\pgfqpoint{7.570780in}{1.796824in}}{\pgfqpoint{7.570780in}{1.791780in}}%
\pgfpathcurveto{\pgfqpoint{7.570780in}{1.786737in}}{\pgfqpoint{7.572784in}{1.781899in}}{\pgfqpoint{7.576350in}{1.778333in}}%
\pgfpathcurveto{\pgfqpoint{7.579917in}{1.774766in}}{\pgfqpoint{7.584754in}{1.772762in}}{\pgfqpoint{7.589798in}{1.772762in}}%
\pgfpathclose%
\pgfusepath{fill}%
\end{pgfscope}%
\begin{pgfscope}%
\pgfpathrectangle{\pgfqpoint{6.572727in}{0.474100in}}{\pgfqpoint{4.227273in}{3.318700in}}%
\pgfusepath{clip}%
\pgfsetbuttcap%
\pgfsetroundjoin%
\definecolor{currentfill}{rgb}{0.127568,0.566949,0.550556}%
\pgfsetfillcolor{currentfill}%
\pgfsetfillopacity{0.700000}%
\pgfsetlinewidth{0.000000pt}%
\definecolor{currentstroke}{rgb}{0.000000,0.000000,0.000000}%
\pgfsetstrokecolor{currentstroke}%
\pgfsetstrokeopacity{0.700000}%
\pgfsetdash{}{0pt}%
\pgfpathmoveto{\pgfqpoint{9.617024in}{1.391560in}}%
\pgfpathcurveto{\pgfqpoint{9.622067in}{1.391560in}}{\pgfqpoint{9.626905in}{1.393564in}}{\pgfqpoint{9.630471in}{1.397130in}}%
\pgfpathcurveto{\pgfqpoint{9.634038in}{1.400697in}}{\pgfqpoint{9.636042in}{1.405534in}}{\pgfqpoint{9.636042in}{1.410578in}}%
\pgfpathcurveto{\pgfqpoint{9.636042in}{1.415622in}}{\pgfqpoint{9.634038in}{1.420459in}}{\pgfqpoint{9.630471in}{1.424026in}}%
\pgfpathcurveto{\pgfqpoint{9.626905in}{1.427592in}}{\pgfqpoint{9.622067in}{1.429596in}}{\pgfqpoint{9.617024in}{1.429596in}}%
\pgfpathcurveto{\pgfqpoint{9.611980in}{1.429596in}}{\pgfqpoint{9.607142in}{1.427592in}}{\pgfqpoint{9.603576in}{1.424026in}}%
\pgfpathcurveto{\pgfqpoint{9.600009in}{1.420459in}}{\pgfqpoint{9.598005in}{1.415622in}}{\pgfqpoint{9.598005in}{1.410578in}}%
\pgfpathcurveto{\pgfqpoint{9.598005in}{1.405534in}}{\pgfqpoint{9.600009in}{1.400697in}}{\pgfqpoint{9.603576in}{1.397130in}}%
\pgfpathcurveto{\pgfqpoint{9.607142in}{1.393564in}}{\pgfqpoint{9.611980in}{1.391560in}}{\pgfqpoint{9.617024in}{1.391560in}}%
\pgfpathclose%
\pgfusepath{fill}%
\end{pgfscope}%
\begin{pgfscope}%
\pgfpathrectangle{\pgfqpoint{6.572727in}{0.474100in}}{\pgfqpoint{4.227273in}{3.318700in}}%
\pgfusepath{clip}%
\pgfsetbuttcap%
\pgfsetroundjoin%
\definecolor{currentfill}{rgb}{0.267004,0.004874,0.329415}%
\pgfsetfillcolor{currentfill}%
\pgfsetfillopacity{0.700000}%
\pgfsetlinewidth{0.000000pt}%
\definecolor{currentstroke}{rgb}{0.000000,0.000000,0.000000}%
\pgfsetstrokecolor{currentstroke}%
\pgfsetstrokeopacity{0.700000}%
\pgfsetdash{}{0pt}%
\pgfpathmoveto{\pgfqpoint{7.555599in}{1.889150in}}%
\pgfpathcurveto{\pgfqpoint{7.560642in}{1.889150in}}{\pgfqpoint{7.565480in}{1.891154in}}{\pgfqpoint{7.569047in}{1.894721in}}%
\pgfpathcurveto{\pgfqpoint{7.572613in}{1.898287in}}{\pgfqpoint{7.574617in}{1.903125in}}{\pgfqpoint{7.574617in}{1.908169in}}%
\pgfpathcurveto{\pgfqpoint{7.574617in}{1.913212in}}{\pgfqpoint{7.572613in}{1.918050in}}{\pgfqpoint{7.569047in}{1.921616in}}%
\pgfpathcurveto{\pgfqpoint{7.565480in}{1.925183in}}{\pgfqpoint{7.560642in}{1.927187in}}{\pgfqpoint{7.555599in}{1.927187in}}%
\pgfpathcurveto{\pgfqpoint{7.550555in}{1.927187in}}{\pgfqpoint{7.545717in}{1.925183in}}{\pgfqpoint{7.542151in}{1.921616in}}%
\pgfpathcurveto{\pgfqpoint{7.538584in}{1.918050in}}{\pgfqpoint{7.536581in}{1.913212in}}{\pgfqpoint{7.536581in}{1.908169in}}%
\pgfpathcurveto{\pgfqpoint{7.536581in}{1.903125in}}{\pgfqpoint{7.538584in}{1.898287in}}{\pgfqpoint{7.542151in}{1.894721in}}%
\pgfpathcurveto{\pgfqpoint{7.545717in}{1.891154in}}{\pgfqpoint{7.550555in}{1.889150in}}{\pgfqpoint{7.555599in}{1.889150in}}%
\pgfpathclose%
\pgfusepath{fill}%
\end{pgfscope}%
\begin{pgfscope}%
\pgfpathrectangle{\pgfqpoint{6.572727in}{0.474100in}}{\pgfqpoint{4.227273in}{3.318700in}}%
\pgfusepath{clip}%
\pgfsetbuttcap%
\pgfsetroundjoin%
\definecolor{currentfill}{rgb}{0.267004,0.004874,0.329415}%
\pgfsetfillcolor{currentfill}%
\pgfsetfillopacity{0.700000}%
\pgfsetlinewidth{0.000000pt}%
\definecolor{currentstroke}{rgb}{0.000000,0.000000,0.000000}%
\pgfsetstrokecolor{currentstroke}%
\pgfsetstrokeopacity{0.700000}%
\pgfsetdash{}{0pt}%
\pgfpathmoveto{\pgfqpoint{7.765132in}{1.467946in}}%
\pgfpathcurveto{\pgfqpoint{7.770175in}{1.467946in}}{\pgfqpoint{7.775013in}{1.469950in}}{\pgfqpoint{7.778580in}{1.473517in}}%
\pgfpathcurveto{\pgfqpoint{7.782146in}{1.477083in}}{\pgfqpoint{7.784150in}{1.481921in}}{\pgfqpoint{7.784150in}{1.486965in}}%
\pgfpathcurveto{\pgfqpoint{7.784150in}{1.492008in}}{\pgfqpoint{7.782146in}{1.496846in}}{\pgfqpoint{7.778580in}{1.500412in}}%
\pgfpathcurveto{\pgfqpoint{7.775013in}{1.503979in}}{\pgfqpoint{7.770175in}{1.505983in}}{\pgfqpoint{7.765132in}{1.505983in}}%
\pgfpathcurveto{\pgfqpoint{7.760088in}{1.505983in}}{\pgfqpoint{7.755250in}{1.503979in}}{\pgfqpoint{7.751684in}{1.500412in}}%
\pgfpathcurveto{\pgfqpoint{7.748117in}{1.496846in}}{\pgfqpoint{7.746114in}{1.492008in}}{\pgfqpoint{7.746114in}{1.486965in}}%
\pgfpathcurveto{\pgfqpoint{7.746114in}{1.481921in}}{\pgfqpoint{7.748117in}{1.477083in}}{\pgfqpoint{7.751684in}{1.473517in}}%
\pgfpathcurveto{\pgfqpoint{7.755250in}{1.469950in}}{\pgfqpoint{7.760088in}{1.467946in}}{\pgfqpoint{7.765132in}{1.467946in}}%
\pgfpathclose%
\pgfusepath{fill}%
\end{pgfscope}%
\begin{pgfscope}%
\pgfpathrectangle{\pgfqpoint{6.572727in}{0.474100in}}{\pgfqpoint{4.227273in}{3.318700in}}%
\pgfusepath{clip}%
\pgfsetbuttcap%
\pgfsetroundjoin%
\definecolor{currentfill}{rgb}{0.993248,0.906157,0.143936}%
\pgfsetfillcolor{currentfill}%
\pgfsetfillopacity{0.700000}%
\pgfsetlinewidth{0.000000pt}%
\definecolor{currentstroke}{rgb}{0.000000,0.000000,0.000000}%
\pgfsetstrokecolor{currentstroke}%
\pgfsetstrokeopacity{0.700000}%
\pgfsetdash{}{0pt}%
\pgfpathmoveto{\pgfqpoint{8.390330in}{2.447414in}}%
\pgfpathcurveto{\pgfqpoint{8.395374in}{2.447414in}}{\pgfqpoint{8.400212in}{2.449418in}}{\pgfqpoint{8.403778in}{2.452985in}}%
\pgfpathcurveto{\pgfqpoint{8.407344in}{2.456551in}}{\pgfqpoint{8.409348in}{2.461389in}}{\pgfqpoint{8.409348in}{2.466433in}}%
\pgfpathcurveto{\pgfqpoint{8.409348in}{2.471476in}}{\pgfqpoint{8.407344in}{2.476314in}}{\pgfqpoint{8.403778in}{2.479880in}}%
\pgfpathcurveto{\pgfqpoint{8.400212in}{2.483447in}}{\pgfqpoint{8.395374in}{2.485451in}}{\pgfqpoint{8.390330in}{2.485451in}}%
\pgfpathcurveto{\pgfqpoint{8.385286in}{2.485451in}}{\pgfqpoint{8.380449in}{2.483447in}}{\pgfqpoint{8.376882in}{2.479880in}}%
\pgfpathcurveto{\pgfqpoint{8.373316in}{2.476314in}}{\pgfqpoint{8.371312in}{2.471476in}}{\pgfqpoint{8.371312in}{2.466433in}}%
\pgfpathcurveto{\pgfqpoint{8.371312in}{2.461389in}}{\pgfqpoint{8.373316in}{2.456551in}}{\pgfqpoint{8.376882in}{2.452985in}}%
\pgfpathcurveto{\pgfqpoint{8.380449in}{2.449418in}}{\pgfqpoint{8.385286in}{2.447414in}}{\pgfqpoint{8.390330in}{2.447414in}}%
\pgfpathclose%
\pgfusepath{fill}%
\end{pgfscope}%
\begin{pgfscope}%
\pgfpathrectangle{\pgfqpoint{6.572727in}{0.474100in}}{\pgfqpoint{4.227273in}{3.318700in}}%
\pgfusepath{clip}%
\pgfsetbuttcap%
\pgfsetroundjoin%
\definecolor{currentfill}{rgb}{0.267004,0.004874,0.329415}%
\pgfsetfillcolor{currentfill}%
\pgfsetfillopacity{0.700000}%
\pgfsetlinewidth{0.000000pt}%
\definecolor{currentstroke}{rgb}{0.000000,0.000000,0.000000}%
\pgfsetstrokecolor{currentstroke}%
\pgfsetstrokeopacity{0.700000}%
\pgfsetdash{}{0pt}%
\pgfpathmoveto{\pgfqpoint{7.505418in}{1.258751in}}%
\pgfpathcurveto{\pgfqpoint{7.510462in}{1.258751in}}{\pgfqpoint{7.515299in}{1.260755in}}{\pgfqpoint{7.518866in}{1.264321in}}%
\pgfpathcurveto{\pgfqpoint{7.522432in}{1.267888in}}{\pgfqpoint{7.524436in}{1.272725in}}{\pgfqpoint{7.524436in}{1.277769in}}%
\pgfpathcurveto{\pgfqpoint{7.524436in}{1.282813in}}{\pgfqpoint{7.522432in}{1.287650in}}{\pgfqpoint{7.518866in}{1.291217in}}%
\pgfpathcurveto{\pgfqpoint{7.515299in}{1.294783in}}{\pgfqpoint{7.510462in}{1.296787in}}{\pgfqpoint{7.505418in}{1.296787in}}%
\pgfpathcurveto{\pgfqpoint{7.500374in}{1.296787in}}{\pgfqpoint{7.495536in}{1.294783in}}{\pgfqpoint{7.491970in}{1.291217in}}%
\pgfpathcurveto{\pgfqpoint{7.488404in}{1.287650in}}{\pgfqpoint{7.486400in}{1.282813in}}{\pgfqpoint{7.486400in}{1.277769in}}%
\pgfpathcurveto{\pgfqpoint{7.486400in}{1.272725in}}{\pgfqpoint{7.488404in}{1.267888in}}{\pgfqpoint{7.491970in}{1.264321in}}%
\pgfpathcurveto{\pgfqpoint{7.495536in}{1.260755in}}{\pgfqpoint{7.500374in}{1.258751in}}{\pgfqpoint{7.505418in}{1.258751in}}%
\pgfpathclose%
\pgfusepath{fill}%
\end{pgfscope}%
\begin{pgfscope}%
\pgfpathrectangle{\pgfqpoint{6.572727in}{0.474100in}}{\pgfqpoint{4.227273in}{3.318700in}}%
\pgfusepath{clip}%
\pgfsetbuttcap%
\pgfsetroundjoin%
\definecolor{currentfill}{rgb}{0.127568,0.566949,0.550556}%
\pgfsetfillcolor{currentfill}%
\pgfsetfillopacity{0.700000}%
\pgfsetlinewidth{0.000000pt}%
\definecolor{currentstroke}{rgb}{0.000000,0.000000,0.000000}%
\pgfsetstrokecolor{currentstroke}%
\pgfsetstrokeopacity{0.700000}%
\pgfsetdash{}{0pt}%
\pgfpathmoveto{\pgfqpoint{9.308547in}{1.978042in}}%
\pgfpathcurveto{\pgfqpoint{9.313590in}{1.978042in}}{\pgfqpoint{9.318428in}{1.980045in}}{\pgfqpoint{9.321995in}{1.983612in}}%
\pgfpathcurveto{\pgfqpoint{9.325561in}{1.987178in}}{\pgfqpoint{9.327565in}{1.992016in}}{\pgfqpoint{9.327565in}{1.997060in}}%
\pgfpathcurveto{\pgfqpoint{9.327565in}{2.002103in}}{\pgfqpoint{9.325561in}{2.006941in}}{\pgfqpoint{9.321995in}{2.010508in}}%
\pgfpathcurveto{\pgfqpoint{9.318428in}{2.014074in}}{\pgfqpoint{9.313590in}{2.016078in}}{\pgfqpoint{9.308547in}{2.016078in}}%
\pgfpathcurveto{\pgfqpoint{9.303503in}{2.016078in}}{\pgfqpoint{9.298665in}{2.014074in}}{\pgfqpoint{9.295099in}{2.010508in}}%
\pgfpathcurveto{\pgfqpoint{9.291532in}{2.006941in}}{\pgfqpoint{9.289529in}{2.002103in}}{\pgfqpoint{9.289529in}{1.997060in}}%
\pgfpathcurveto{\pgfqpoint{9.289529in}{1.992016in}}{\pgfqpoint{9.291532in}{1.987178in}}{\pgfqpoint{9.295099in}{1.983612in}}%
\pgfpathcurveto{\pgfqpoint{9.298665in}{1.980045in}}{\pgfqpoint{9.303503in}{1.978042in}}{\pgfqpoint{9.308547in}{1.978042in}}%
\pgfpathclose%
\pgfusepath{fill}%
\end{pgfscope}%
\begin{pgfscope}%
\pgfpathrectangle{\pgfqpoint{6.572727in}{0.474100in}}{\pgfqpoint{4.227273in}{3.318700in}}%
\pgfusepath{clip}%
\pgfsetbuttcap%
\pgfsetroundjoin%
\definecolor{currentfill}{rgb}{0.267004,0.004874,0.329415}%
\pgfsetfillcolor{currentfill}%
\pgfsetfillopacity{0.700000}%
\pgfsetlinewidth{0.000000pt}%
\definecolor{currentstroke}{rgb}{0.000000,0.000000,0.000000}%
\pgfsetstrokecolor{currentstroke}%
\pgfsetstrokeopacity{0.700000}%
\pgfsetdash{}{0pt}%
\pgfpathmoveto{\pgfqpoint{7.873616in}{1.609758in}}%
\pgfpathcurveto{\pgfqpoint{7.878660in}{1.609758in}}{\pgfqpoint{7.883498in}{1.611762in}}{\pgfqpoint{7.887064in}{1.615328in}}%
\pgfpathcurveto{\pgfqpoint{7.890630in}{1.618894in}}{\pgfqpoint{7.892634in}{1.623732in}}{\pgfqpoint{7.892634in}{1.628776in}}%
\pgfpathcurveto{\pgfqpoint{7.892634in}{1.633820in}}{\pgfqpoint{7.890630in}{1.638657in}}{\pgfqpoint{7.887064in}{1.642224in}}%
\pgfpathcurveto{\pgfqpoint{7.883498in}{1.645790in}}{\pgfqpoint{7.878660in}{1.647794in}}{\pgfqpoint{7.873616in}{1.647794in}}%
\pgfpathcurveto{\pgfqpoint{7.868572in}{1.647794in}}{\pgfqpoint{7.863735in}{1.645790in}}{\pgfqpoint{7.860168in}{1.642224in}}%
\pgfpathcurveto{\pgfqpoint{7.856602in}{1.638657in}}{\pgfqpoint{7.854598in}{1.633820in}}{\pgfqpoint{7.854598in}{1.628776in}}%
\pgfpathcurveto{\pgfqpoint{7.854598in}{1.623732in}}{\pgfqpoint{7.856602in}{1.618894in}}{\pgfqpoint{7.860168in}{1.615328in}}%
\pgfpathcurveto{\pgfqpoint{7.863735in}{1.611762in}}{\pgfqpoint{7.868572in}{1.609758in}}{\pgfqpoint{7.873616in}{1.609758in}}%
\pgfpathclose%
\pgfusepath{fill}%
\end{pgfscope}%
\begin{pgfscope}%
\pgfpathrectangle{\pgfqpoint{6.572727in}{0.474100in}}{\pgfqpoint{4.227273in}{3.318700in}}%
\pgfusepath{clip}%
\pgfsetbuttcap%
\pgfsetroundjoin%
\definecolor{currentfill}{rgb}{0.127568,0.566949,0.550556}%
\pgfsetfillcolor{currentfill}%
\pgfsetfillopacity{0.700000}%
\pgfsetlinewidth{0.000000pt}%
\definecolor{currentstroke}{rgb}{0.000000,0.000000,0.000000}%
\pgfsetstrokecolor{currentstroke}%
\pgfsetstrokeopacity{0.700000}%
\pgfsetdash{}{0pt}%
\pgfpathmoveto{\pgfqpoint{9.908448in}{1.458857in}}%
\pgfpathcurveto{\pgfqpoint{9.913492in}{1.458857in}}{\pgfqpoint{9.918329in}{1.460861in}}{\pgfqpoint{9.921896in}{1.464428in}}%
\pgfpathcurveto{\pgfqpoint{9.925462in}{1.467994in}}{\pgfqpoint{9.927466in}{1.472832in}}{\pgfqpoint{9.927466in}{1.477876in}}%
\pgfpathcurveto{\pgfqpoint{9.927466in}{1.482919in}}{\pgfqpoint{9.925462in}{1.487757in}}{\pgfqpoint{9.921896in}{1.491323in}}%
\pgfpathcurveto{\pgfqpoint{9.918329in}{1.494890in}}{\pgfqpoint{9.913492in}{1.496894in}}{\pgfqpoint{9.908448in}{1.496894in}}%
\pgfpathcurveto{\pgfqpoint{9.903404in}{1.496894in}}{\pgfqpoint{9.898566in}{1.494890in}}{\pgfqpoint{9.895000in}{1.491323in}}%
\pgfpathcurveto{\pgfqpoint{9.891434in}{1.487757in}}{\pgfqpoint{9.889430in}{1.482919in}}{\pgfqpoint{9.889430in}{1.477876in}}%
\pgfpathcurveto{\pgfqpoint{9.889430in}{1.472832in}}{\pgfqpoint{9.891434in}{1.467994in}}{\pgfqpoint{9.895000in}{1.464428in}}%
\pgfpathcurveto{\pgfqpoint{9.898566in}{1.460861in}}{\pgfqpoint{9.903404in}{1.458857in}}{\pgfqpoint{9.908448in}{1.458857in}}%
\pgfpathclose%
\pgfusepath{fill}%
\end{pgfscope}%
\begin{pgfscope}%
\pgfpathrectangle{\pgfqpoint{6.572727in}{0.474100in}}{\pgfqpoint{4.227273in}{3.318700in}}%
\pgfusepath{clip}%
\pgfsetbuttcap%
\pgfsetroundjoin%
\definecolor{currentfill}{rgb}{0.267004,0.004874,0.329415}%
\pgfsetfillcolor{currentfill}%
\pgfsetfillopacity{0.700000}%
\pgfsetlinewidth{0.000000pt}%
\definecolor{currentstroke}{rgb}{0.000000,0.000000,0.000000}%
\pgfsetstrokecolor{currentstroke}%
\pgfsetstrokeopacity{0.700000}%
\pgfsetdash{}{0pt}%
\pgfpathmoveto{\pgfqpoint{7.982655in}{1.692162in}}%
\pgfpathcurveto{\pgfqpoint{7.987699in}{1.692162in}}{\pgfqpoint{7.992537in}{1.694165in}}{\pgfqpoint{7.996103in}{1.697732in}}%
\pgfpathcurveto{\pgfqpoint{7.999669in}{1.701298in}}{\pgfqpoint{8.001673in}{1.706136in}}{\pgfqpoint{8.001673in}{1.711180in}}%
\pgfpathcurveto{\pgfqpoint{8.001673in}{1.716223in}}{\pgfqpoint{7.999669in}{1.721061in}}{\pgfqpoint{7.996103in}{1.724628in}}%
\pgfpathcurveto{\pgfqpoint{7.992537in}{1.728194in}}{\pgfqpoint{7.987699in}{1.730198in}}{\pgfqpoint{7.982655in}{1.730198in}}%
\pgfpathcurveto{\pgfqpoint{7.977612in}{1.730198in}}{\pgfqpoint{7.972774in}{1.728194in}}{\pgfqpoint{7.969207in}{1.724628in}}%
\pgfpathcurveto{\pgfqpoint{7.965641in}{1.721061in}}{\pgfqpoint{7.963637in}{1.716223in}}{\pgfqpoint{7.963637in}{1.711180in}}%
\pgfpathcurveto{\pgfqpoint{7.963637in}{1.706136in}}{\pgfqpoint{7.965641in}{1.701298in}}{\pgfqpoint{7.969207in}{1.697732in}}%
\pgfpathcurveto{\pgfqpoint{7.972774in}{1.694165in}}{\pgfqpoint{7.977612in}{1.692162in}}{\pgfqpoint{7.982655in}{1.692162in}}%
\pgfpathclose%
\pgfusepath{fill}%
\end{pgfscope}%
\begin{pgfscope}%
\pgfpathrectangle{\pgfqpoint{6.572727in}{0.474100in}}{\pgfqpoint{4.227273in}{3.318700in}}%
\pgfusepath{clip}%
\pgfsetbuttcap%
\pgfsetroundjoin%
\definecolor{currentfill}{rgb}{0.993248,0.906157,0.143936}%
\pgfsetfillcolor{currentfill}%
\pgfsetfillopacity{0.700000}%
\pgfsetlinewidth{0.000000pt}%
\definecolor{currentstroke}{rgb}{0.000000,0.000000,0.000000}%
\pgfsetstrokecolor{currentstroke}%
\pgfsetstrokeopacity{0.700000}%
\pgfsetdash{}{0pt}%
\pgfpathmoveto{\pgfqpoint{7.948766in}{2.438947in}}%
\pgfpathcurveto{\pgfqpoint{7.953810in}{2.438947in}}{\pgfqpoint{7.958647in}{2.440951in}}{\pgfqpoint{7.962214in}{2.444517in}}%
\pgfpathcurveto{\pgfqpoint{7.965780in}{2.448084in}}{\pgfqpoint{7.967784in}{2.452922in}}{\pgfqpoint{7.967784in}{2.457965in}}%
\pgfpathcurveto{\pgfqpoint{7.967784in}{2.463009in}}{\pgfqpoint{7.965780in}{2.467847in}}{\pgfqpoint{7.962214in}{2.471413in}}%
\pgfpathcurveto{\pgfqpoint{7.958647in}{2.474980in}}{\pgfqpoint{7.953810in}{2.476983in}}{\pgfqpoint{7.948766in}{2.476983in}}%
\pgfpathcurveto{\pgfqpoint{7.943722in}{2.476983in}}{\pgfqpoint{7.938885in}{2.474980in}}{\pgfqpoint{7.935318in}{2.471413in}}%
\pgfpathcurveto{\pgfqpoint{7.931752in}{2.467847in}}{\pgfqpoint{7.929748in}{2.463009in}}{\pgfqpoint{7.929748in}{2.457965in}}%
\pgfpathcurveto{\pgfqpoint{7.929748in}{2.452922in}}{\pgfqpoint{7.931752in}{2.448084in}}{\pgfqpoint{7.935318in}{2.444517in}}%
\pgfpathcurveto{\pgfqpoint{7.938885in}{2.440951in}}{\pgfqpoint{7.943722in}{2.438947in}}{\pgfqpoint{7.948766in}{2.438947in}}%
\pgfpathclose%
\pgfusepath{fill}%
\end{pgfscope}%
\begin{pgfscope}%
\pgfpathrectangle{\pgfqpoint{6.572727in}{0.474100in}}{\pgfqpoint{4.227273in}{3.318700in}}%
\pgfusepath{clip}%
\pgfsetbuttcap%
\pgfsetroundjoin%
\definecolor{currentfill}{rgb}{0.993248,0.906157,0.143936}%
\pgfsetfillcolor{currentfill}%
\pgfsetfillopacity{0.700000}%
\pgfsetlinewidth{0.000000pt}%
\definecolor{currentstroke}{rgb}{0.000000,0.000000,0.000000}%
\pgfsetstrokecolor{currentstroke}%
\pgfsetstrokeopacity{0.700000}%
\pgfsetdash{}{0pt}%
\pgfpathmoveto{\pgfqpoint{7.267644in}{3.017274in}}%
\pgfpathcurveto{\pgfqpoint{7.272687in}{3.017274in}}{\pgfqpoint{7.277525in}{3.019278in}}{\pgfqpoint{7.281091in}{3.022845in}}%
\pgfpathcurveto{\pgfqpoint{7.284658in}{3.026411in}}{\pgfqpoint{7.286662in}{3.031249in}}{\pgfqpoint{7.286662in}{3.036292in}}%
\pgfpathcurveto{\pgfqpoint{7.286662in}{3.041336in}}{\pgfqpoint{7.284658in}{3.046174in}}{\pgfqpoint{7.281091in}{3.049740in}}%
\pgfpathcurveto{\pgfqpoint{7.277525in}{3.053307in}}{\pgfqpoint{7.272687in}{3.055311in}}{\pgfqpoint{7.267644in}{3.055311in}}%
\pgfpathcurveto{\pgfqpoint{7.262600in}{3.055311in}}{\pgfqpoint{7.257762in}{3.053307in}}{\pgfqpoint{7.254196in}{3.049740in}}%
\pgfpathcurveto{\pgfqpoint{7.250629in}{3.046174in}}{\pgfqpoint{7.248625in}{3.041336in}}{\pgfqpoint{7.248625in}{3.036292in}}%
\pgfpathcurveto{\pgfqpoint{7.248625in}{3.031249in}}{\pgfqpoint{7.250629in}{3.026411in}}{\pgfqpoint{7.254196in}{3.022845in}}%
\pgfpathcurveto{\pgfqpoint{7.257762in}{3.019278in}}{\pgfqpoint{7.262600in}{3.017274in}}{\pgfqpoint{7.267644in}{3.017274in}}%
\pgfpathclose%
\pgfusepath{fill}%
\end{pgfscope}%
\begin{pgfscope}%
\pgfpathrectangle{\pgfqpoint{6.572727in}{0.474100in}}{\pgfqpoint{4.227273in}{3.318700in}}%
\pgfusepath{clip}%
\pgfsetbuttcap%
\pgfsetroundjoin%
\definecolor{currentfill}{rgb}{0.127568,0.566949,0.550556}%
\pgfsetfillcolor{currentfill}%
\pgfsetfillopacity{0.700000}%
\pgfsetlinewidth{0.000000pt}%
\definecolor{currentstroke}{rgb}{0.000000,0.000000,0.000000}%
\pgfsetstrokecolor{currentstroke}%
\pgfsetstrokeopacity{0.700000}%
\pgfsetdash{}{0pt}%
\pgfpathmoveto{\pgfqpoint{9.764049in}{1.895044in}}%
\pgfpathcurveto{\pgfqpoint{9.769093in}{1.895044in}}{\pgfqpoint{9.773931in}{1.897048in}}{\pgfqpoint{9.777497in}{1.900614in}}%
\pgfpathcurveto{\pgfqpoint{9.781064in}{1.904181in}}{\pgfqpoint{9.783067in}{1.909018in}}{\pgfqpoint{9.783067in}{1.914062in}}%
\pgfpathcurveto{\pgfqpoint{9.783067in}{1.919106in}}{\pgfqpoint{9.781064in}{1.923943in}}{\pgfqpoint{9.777497in}{1.927510in}}%
\pgfpathcurveto{\pgfqpoint{9.773931in}{1.931076in}}{\pgfqpoint{9.769093in}{1.933080in}}{\pgfqpoint{9.764049in}{1.933080in}}%
\pgfpathcurveto{\pgfqpoint{9.759006in}{1.933080in}}{\pgfqpoint{9.754168in}{1.931076in}}{\pgfqpoint{9.750601in}{1.927510in}}%
\pgfpathcurveto{\pgfqpoint{9.747035in}{1.923943in}}{\pgfqpoint{9.745031in}{1.919106in}}{\pgfqpoint{9.745031in}{1.914062in}}%
\pgfpathcurveto{\pgfqpoint{9.745031in}{1.909018in}}{\pgfqpoint{9.747035in}{1.904181in}}{\pgfqpoint{9.750601in}{1.900614in}}%
\pgfpathcurveto{\pgfqpoint{9.754168in}{1.897048in}}{\pgfqpoint{9.759006in}{1.895044in}}{\pgfqpoint{9.764049in}{1.895044in}}%
\pgfpathclose%
\pgfusepath{fill}%
\end{pgfscope}%
\begin{pgfscope}%
\pgfpathrectangle{\pgfqpoint{6.572727in}{0.474100in}}{\pgfqpoint{4.227273in}{3.318700in}}%
\pgfusepath{clip}%
\pgfsetbuttcap%
\pgfsetroundjoin%
\definecolor{currentfill}{rgb}{0.993248,0.906157,0.143936}%
\pgfsetfillcolor{currentfill}%
\pgfsetfillopacity{0.700000}%
\pgfsetlinewidth{0.000000pt}%
\definecolor{currentstroke}{rgb}{0.000000,0.000000,0.000000}%
\pgfsetstrokecolor{currentstroke}%
\pgfsetstrokeopacity{0.700000}%
\pgfsetdash{}{0pt}%
\pgfpathmoveto{\pgfqpoint{7.877260in}{2.295941in}}%
\pgfpathcurveto{\pgfqpoint{7.882304in}{2.295941in}}{\pgfqpoint{7.887142in}{2.297945in}}{\pgfqpoint{7.890708in}{2.301511in}}%
\pgfpathcurveto{\pgfqpoint{7.894275in}{2.305078in}}{\pgfqpoint{7.896279in}{2.309916in}}{\pgfqpoint{7.896279in}{2.314959in}}%
\pgfpathcurveto{\pgfqpoint{7.896279in}{2.320003in}}{\pgfqpoint{7.894275in}{2.324841in}}{\pgfqpoint{7.890708in}{2.328407in}}%
\pgfpathcurveto{\pgfqpoint{7.887142in}{2.331973in}}{\pgfqpoint{7.882304in}{2.333977in}}{\pgfqpoint{7.877260in}{2.333977in}}%
\pgfpathcurveto{\pgfqpoint{7.872217in}{2.333977in}}{\pgfqpoint{7.867379in}{2.331973in}}{\pgfqpoint{7.863813in}{2.328407in}}%
\pgfpathcurveto{\pgfqpoint{7.860246in}{2.324841in}}{\pgfqpoint{7.858242in}{2.320003in}}{\pgfqpoint{7.858242in}{2.314959in}}%
\pgfpathcurveto{\pgfqpoint{7.858242in}{2.309916in}}{\pgfqpoint{7.860246in}{2.305078in}}{\pgfqpoint{7.863813in}{2.301511in}}%
\pgfpathcurveto{\pgfqpoint{7.867379in}{2.297945in}}{\pgfqpoint{7.872217in}{2.295941in}}{\pgfqpoint{7.877260in}{2.295941in}}%
\pgfpathclose%
\pgfusepath{fill}%
\end{pgfscope}%
\begin{pgfscope}%
\pgfpathrectangle{\pgfqpoint{6.572727in}{0.474100in}}{\pgfqpoint{4.227273in}{3.318700in}}%
\pgfusepath{clip}%
\pgfsetbuttcap%
\pgfsetroundjoin%
\definecolor{currentfill}{rgb}{0.993248,0.906157,0.143936}%
\pgfsetfillcolor{currentfill}%
\pgfsetfillopacity{0.700000}%
\pgfsetlinewidth{0.000000pt}%
\definecolor{currentstroke}{rgb}{0.000000,0.000000,0.000000}%
\pgfsetstrokecolor{currentstroke}%
\pgfsetstrokeopacity{0.700000}%
\pgfsetdash{}{0pt}%
\pgfpathmoveto{\pgfqpoint{8.057528in}{2.667054in}}%
\pgfpathcurveto{\pgfqpoint{8.062571in}{2.667054in}}{\pgfqpoint{8.067409in}{2.669058in}}{\pgfqpoint{8.070976in}{2.672624in}}%
\pgfpathcurveto{\pgfqpoint{8.074542in}{2.676190in}}{\pgfqpoint{8.076546in}{2.681028in}}{\pgfqpoint{8.076546in}{2.686072in}}%
\pgfpathcurveto{\pgfqpoint{8.076546in}{2.691116in}}{\pgfqpoint{8.074542in}{2.695953in}}{\pgfqpoint{8.070976in}{2.699520in}}%
\pgfpathcurveto{\pgfqpoint{8.067409in}{2.703086in}}{\pgfqpoint{8.062571in}{2.705090in}}{\pgfqpoint{8.057528in}{2.705090in}}%
\pgfpathcurveto{\pgfqpoint{8.052484in}{2.705090in}}{\pgfqpoint{8.047646in}{2.703086in}}{\pgfqpoint{8.044080in}{2.699520in}}%
\pgfpathcurveto{\pgfqpoint{8.040513in}{2.695953in}}{\pgfqpoint{8.038510in}{2.691116in}}{\pgfqpoint{8.038510in}{2.686072in}}%
\pgfpathcurveto{\pgfqpoint{8.038510in}{2.681028in}}{\pgfqpoint{8.040513in}{2.676190in}}{\pgfqpoint{8.044080in}{2.672624in}}%
\pgfpathcurveto{\pgfqpoint{8.047646in}{2.669058in}}{\pgfqpoint{8.052484in}{2.667054in}}{\pgfqpoint{8.057528in}{2.667054in}}%
\pgfpathclose%
\pgfusepath{fill}%
\end{pgfscope}%
\begin{pgfscope}%
\pgfpathrectangle{\pgfqpoint{6.572727in}{0.474100in}}{\pgfqpoint{4.227273in}{3.318700in}}%
\pgfusepath{clip}%
\pgfsetbuttcap%
\pgfsetroundjoin%
\definecolor{currentfill}{rgb}{0.127568,0.566949,0.550556}%
\pgfsetfillcolor{currentfill}%
\pgfsetfillopacity{0.700000}%
\pgfsetlinewidth{0.000000pt}%
\definecolor{currentstroke}{rgb}{0.000000,0.000000,0.000000}%
\pgfsetstrokecolor{currentstroke}%
\pgfsetstrokeopacity{0.700000}%
\pgfsetdash{}{0pt}%
\pgfpathmoveto{\pgfqpoint{9.430775in}{1.906199in}}%
\pgfpathcurveto{\pgfqpoint{9.435819in}{1.906199in}}{\pgfqpoint{9.440656in}{1.908203in}}{\pgfqpoint{9.444223in}{1.911769in}}%
\pgfpathcurveto{\pgfqpoint{9.447789in}{1.915336in}}{\pgfqpoint{9.449793in}{1.920173in}}{\pgfqpoint{9.449793in}{1.925217in}}%
\pgfpathcurveto{\pgfqpoint{9.449793in}{1.930261in}}{\pgfqpoint{9.447789in}{1.935099in}}{\pgfqpoint{9.444223in}{1.938665in}}%
\pgfpathcurveto{\pgfqpoint{9.440656in}{1.942231in}}{\pgfqpoint{9.435819in}{1.944235in}}{\pgfqpoint{9.430775in}{1.944235in}}%
\pgfpathcurveto{\pgfqpoint{9.425731in}{1.944235in}}{\pgfqpoint{9.420894in}{1.942231in}}{\pgfqpoint{9.417327in}{1.938665in}}%
\pgfpathcurveto{\pgfqpoint{9.413761in}{1.935099in}}{\pgfqpoint{9.411757in}{1.930261in}}{\pgfqpoint{9.411757in}{1.925217in}}%
\pgfpathcurveto{\pgfqpoint{9.411757in}{1.920173in}}{\pgfqpoint{9.413761in}{1.915336in}}{\pgfqpoint{9.417327in}{1.911769in}}%
\pgfpathcurveto{\pgfqpoint{9.420894in}{1.908203in}}{\pgfqpoint{9.425731in}{1.906199in}}{\pgfqpoint{9.430775in}{1.906199in}}%
\pgfpathclose%
\pgfusepath{fill}%
\end{pgfscope}%
\begin{pgfscope}%
\pgfpathrectangle{\pgfqpoint{6.572727in}{0.474100in}}{\pgfqpoint{4.227273in}{3.318700in}}%
\pgfusepath{clip}%
\pgfsetbuttcap%
\pgfsetroundjoin%
\definecolor{currentfill}{rgb}{0.267004,0.004874,0.329415}%
\pgfsetfillcolor{currentfill}%
\pgfsetfillopacity{0.700000}%
\pgfsetlinewidth{0.000000pt}%
\definecolor{currentstroke}{rgb}{0.000000,0.000000,0.000000}%
\pgfsetstrokecolor{currentstroke}%
\pgfsetstrokeopacity{0.700000}%
\pgfsetdash{}{0pt}%
\pgfpathmoveto{\pgfqpoint{7.415492in}{2.120343in}}%
\pgfpathcurveto{\pgfqpoint{7.420535in}{2.120343in}}{\pgfqpoint{7.425373in}{2.122347in}}{\pgfqpoint{7.428940in}{2.125913in}}%
\pgfpathcurveto{\pgfqpoint{7.432506in}{2.129480in}}{\pgfqpoint{7.434510in}{2.134317in}}{\pgfqpoint{7.434510in}{2.139361in}}%
\pgfpathcurveto{\pgfqpoint{7.434510in}{2.144405in}}{\pgfqpoint{7.432506in}{2.149242in}}{\pgfqpoint{7.428940in}{2.152809in}}%
\pgfpathcurveto{\pgfqpoint{7.425373in}{2.156375in}}{\pgfqpoint{7.420535in}{2.158379in}}{\pgfqpoint{7.415492in}{2.158379in}}%
\pgfpathcurveto{\pgfqpoint{7.410448in}{2.158379in}}{\pgfqpoint{7.405610in}{2.156375in}}{\pgfqpoint{7.402044in}{2.152809in}}%
\pgfpathcurveto{\pgfqpoint{7.398477in}{2.149242in}}{\pgfqpoint{7.396474in}{2.144405in}}{\pgfqpoint{7.396474in}{2.139361in}}%
\pgfpathcurveto{\pgfqpoint{7.396474in}{2.134317in}}{\pgfqpoint{7.398477in}{2.129480in}}{\pgfqpoint{7.402044in}{2.125913in}}%
\pgfpathcurveto{\pgfqpoint{7.405610in}{2.122347in}}{\pgfqpoint{7.410448in}{2.120343in}}{\pgfqpoint{7.415492in}{2.120343in}}%
\pgfpathclose%
\pgfusepath{fill}%
\end{pgfscope}%
\begin{pgfscope}%
\pgfpathrectangle{\pgfqpoint{6.572727in}{0.474100in}}{\pgfqpoint{4.227273in}{3.318700in}}%
\pgfusepath{clip}%
\pgfsetbuttcap%
\pgfsetroundjoin%
\definecolor{currentfill}{rgb}{0.993248,0.906157,0.143936}%
\pgfsetfillcolor{currentfill}%
\pgfsetfillopacity{0.700000}%
\pgfsetlinewidth{0.000000pt}%
\definecolor{currentstroke}{rgb}{0.000000,0.000000,0.000000}%
\pgfsetstrokecolor{currentstroke}%
\pgfsetstrokeopacity{0.700000}%
\pgfsetdash{}{0pt}%
\pgfpathmoveto{\pgfqpoint{8.174750in}{2.681549in}}%
\pgfpathcurveto{\pgfqpoint{8.179794in}{2.681549in}}{\pgfqpoint{8.184632in}{2.683552in}}{\pgfqpoint{8.188198in}{2.687119in}}%
\pgfpathcurveto{\pgfqpoint{8.191764in}{2.690685in}}{\pgfqpoint{8.193768in}{2.695523in}}{\pgfqpoint{8.193768in}{2.700567in}}%
\pgfpathcurveto{\pgfqpoint{8.193768in}{2.705610in}}{\pgfqpoint{8.191764in}{2.710448in}}{\pgfqpoint{8.188198in}{2.714015in}}%
\pgfpathcurveto{\pgfqpoint{8.184632in}{2.717581in}}{\pgfqpoint{8.179794in}{2.719585in}}{\pgfqpoint{8.174750in}{2.719585in}}%
\pgfpathcurveto{\pgfqpoint{8.169706in}{2.719585in}}{\pgfqpoint{8.164869in}{2.717581in}}{\pgfqpoint{8.161302in}{2.714015in}}%
\pgfpathcurveto{\pgfqpoint{8.157736in}{2.710448in}}{\pgfqpoint{8.155732in}{2.705610in}}{\pgfqpoint{8.155732in}{2.700567in}}%
\pgfpathcurveto{\pgfqpoint{8.155732in}{2.695523in}}{\pgfqpoint{8.157736in}{2.690685in}}{\pgfqpoint{8.161302in}{2.687119in}}%
\pgfpathcurveto{\pgfqpoint{8.164869in}{2.683552in}}{\pgfqpoint{8.169706in}{2.681549in}}{\pgfqpoint{8.174750in}{2.681549in}}%
\pgfpathclose%
\pgfusepath{fill}%
\end{pgfscope}%
\begin{pgfscope}%
\pgfpathrectangle{\pgfqpoint{6.572727in}{0.474100in}}{\pgfqpoint{4.227273in}{3.318700in}}%
\pgfusepath{clip}%
\pgfsetbuttcap%
\pgfsetroundjoin%
\definecolor{currentfill}{rgb}{0.267004,0.004874,0.329415}%
\pgfsetfillcolor{currentfill}%
\pgfsetfillopacity{0.700000}%
\pgfsetlinewidth{0.000000pt}%
\definecolor{currentstroke}{rgb}{0.000000,0.000000,0.000000}%
\pgfsetstrokecolor{currentstroke}%
\pgfsetstrokeopacity{0.700000}%
\pgfsetdash{}{0pt}%
\pgfpathmoveto{\pgfqpoint{7.873512in}{1.432447in}}%
\pgfpathcurveto{\pgfqpoint{7.878555in}{1.432447in}}{\pgfqpoint{7.883393in}{1.434451in}}{\pgfqpoint{7.886959in}{1.438017in}}%
\pgfpathcurveto{\pgfqpoint{7.890526in}{1.441583in}}{\pgfqpoint{7.892530in}{1.446421in}}{\pgfqpoint{7.892530in}{1.451465in}}%
\pgfpathcurveto{\pgfqpoint{7.892530in}{1.456509in}}{\pgfqpoint{7.890526in}{1.461346in}}{\pgfqpoint{7.886959in}{1.464913in}}%
\pgfpathcurveto{\pgfqpoint{7.883393in}{1.468479in}}{\pgfqpoint{7.878555in}{1.470483in}}{\pgfqpoint{7.873512in}{1.470483in}}%
\pgfpathcurveto{\pgfqpoint{7.868468in}{1.470483in}}{\pgfqpoint{7.863630in}{1.468479in}}{\pgfqpoint{7.860064in}{1.464913in}}%
\pgfpathcurveto{\pgfqpoint{7.856497in}{1.461346in}}{\pgfqpoint{7.854493in}{1.456509in}}{\pgfqpoint{7.854493in}{1.451465in}}%
\pgfpathcurveto{\pgfqpoint{7.854493in}{1.446421in}}{\pgfqpoint{7.856497in}{1.441583in}}{\pgfqpoint{7.860064in}{1.438017in}}%
\pgfpathcurveto{\pgfqpoint{7.863630in}{1.434451in}}{\pgfqpoint{7.868468in}{1.432447in}}{\pgfqpoint{7.873512in}{1.432447in}}%
\pgfpathclose%
\pgfusepath{fill}%
\end{pgfscope}%
\begin{pgfscope}%
\pgfpathrectangle{\pgfqpoint{6.572727in}{0.474100in}}{\pgfqpoint{4.227273in}{3.318700in}}%
\pgfusepath{clip}%
\pgfsetbuttcap%
\pgfsetroundjoin%
\definecolor{currentfill}{rgb}{0.993248,0.906157,0.143936}%
\pgfsetfillcolor{currentfill}%
\pgfsetfillopacity{0.700000}%
\pgfsetlinewidth{0.000000pt}%
\definecolor{currentstroke}{rgb}{0.000000,0.000000,0.000000}%
\pgfsetstrokecolor{currentstroke}%
\pgfsetstrokeopacity{0.700000}%
\pgfsetdash{}{0pt}%
\pgfpathmoveto{\pgfqpoint{8.491858in}{2.950591in}}%
\pgfpathcurveto{\pgfqpoint{8.496901in}{2.950591in}}{\pgfqpoint{8.501739in}{2.952594in}}{\pgfqpoint{8.505306in}{2.956161in}}%
\pgfpathcurveto{\pgfqpoint{8.508872in}{2.959727in}}{\pgfqpoint{8.510876in}{2.964565in}}{\pgfqpoint{8.510876in}{2.969609in}}%
\pgfpathcurveto{\pgfqpoint{8.510876in}{2.974652in}}{\pgfqpoint{8.508872in}{2.979490in}}{\pgfqpoint{8.505306in}{2.983057in}}%
\pgfpathcurveto{\pgfqpoint{8.501739in}{2.986623in}}{\pgfqpoint{8.496901in}{2.988627in}}{\pgfqpoint{8.491858in}{2.988627in}}%
\pgfpathcurveto{\pgfqpoint{8.486814in}{2.988627in}}{\pgfqpoint{8.481976in}{2.986623in}}{\pgfqpoint{8.478410in}{2.983057in}}%
\pgfpathcurveto{\pgfqpoint{8.474843in}{2.979490in}}{\pgfqpoint{8.472840in}{2.974652in}}{\pgfqpoint{8.472840in}{2.969609in}}%
\pgfpathcurveto{\pgfqpoint{8.472840in}{2.964565in}}{\pgfqpoint{8.474843in}{2.959727in}}{\pgfqpoint{8.478410in}{2.956161in}}%
\pgfpathcurveto{\pgfqpoint{8.481976in}{2.952594in}}{\pgfqpoint{8.486814in}{2.950591in}}{\pgfqpoint{8.491858in}{2.950591in}}%
\pgfpathclose%
\pgfusepath{fill}%
\end{pgfscope}%
\begin{pgfscope}%
\pgfpathrectangle{\pgfqpoint{6.572727in}{0.474100in}}{\pgfqpoint{4.227273in}{3.318700in}}%
\pgfusepath{clip}%
\pgfsetbuttcap%
\pgfsetroundjoin%
\definecolor{currentfill}{rgb}{0.267004,0.004874,0.329415}%
\pgfsetfillcolor{currentfill}%
\pgfsetfillopacity{0.700000}%
\pgfsetlinewidth{0.000000pt}%
\definecolor{currentstroke}{rgb}{0.000000,0.000000,0.000000}%
\pgfsetstrokecolor{currentstroke}%
\pgfsetstrokeopacity{0.700000}%
\pgfsetdash{}{0pt}%
\pgfpathmoveto{\pgfqpoint{7.709426in}{1.329065in}}%
\pgfpathcurveto{\pgfqpoint{7.714470in}{1.329065in}}{\pgfqpoint{7.719308in}{1.331069in}}{\pgfqpoint{7.722874in}{1.334636in}}%
\pgfpathcurveto{\pgfqpoint{7.726440in}{1.338202in}}{\pgfqpoint{7.728444in}{1.343040in}}{\pgfqpoint{7.728444in}{1.348083in}}%
\pgfpathcurveto{\pgfqpoint{7.728444in}{1.353127in}}{\pgfqpoint{7.726440in}{1.357965in}}{\pgfqpoint{7.722874in}{1.361531in}}%
\pgfpathcurveto{\pgfqpoint{7.719308in}{1.365098in}}{\pgfqpoint{7.714470in}{1.367102in}}{\pgfqpoint{7.709426in}{1.367102in}}%
\pgfpathcurveto{\pgfqpoint{7.704382in}{1.367102in}}{\pgfqpoint{7.699545in}{1.365098in}}{\pgfqpoint{7.695978in}{1.361531in}}%
\pgfpathcurveto{\pgfqpoint{7.692412in}{1.357965in}}{\pgfqpoint{7.690408in}{1.353127in}}{\pgfqpoint{7.690408in}{1.348083in}}%
\pgfpathcurveto{\pgfqpoint{7.690408in}{1.343040in}}{\pgfqpoint{7.692412in}{1.338202in}}{\pgfqpoint{7.695978in}{1.334636in}}%
\pgfpathcurveto{\pgfqpoint{7.699545in}{1.331069in}}{\pgfqpoint{7.704382in}{1.329065in}}{\pgfqpoint{7.709426in}{1.329065in}}%
\pgfpathclose%
\pgfusepath{fill}%
\end{pgfscope}%
\begin{pgfscope}%
\pgfpathrectangle{\pgfqpoint{6.572727in}{0.474100in}}{\pgfqpoint{4.227273in}{3.318700in}}%
\pgfusepath{clip}%
\pgfsetbuttcap%
\pgfsetroundjoin%
\definecolor{currentfill}{rgb}{0.267004,0.004874,0.329415}%
\pgfsetfillcolor{currentfill}%
\pgfsetfillopacity{0.700000}%
\pgfsetlinewidth{0.000000pt}%
\definecolor{currentstroke}{rgb}{0.000000,0.000000,0.000000}%
\pgfsetstrokecolor{currentstroke}%
\pgfsetstrokeopacity{0.700000}%
\pgfsetdash{}{0pt}%
\pgfpathmoveto{\pgfqpoint{8.343571in}{1.163724in}}%
\pgfpathcurveto{\pgfqpoint{8.348614in}{1.163724in}}{\pgfqpoint{8.353452in}{1.165728in}}{\pgfqpoint{8.357018in}{1.169294in}}%
\pgfpathcurveto{\pgfqpoint{8.360585in}{1.172861in}}{\pgfqpoint{8.362589in}{1.177698in}}{\pgfqpoint{8.362589in}{1.182742in}}%
\pgfpathcurveto{\pgfqpoint{8.362589in}{1.187786in}}{\pgfqpoint{8.360585in}{1.192624in}}{\pgfqpoint{8.357018in}{1.196190in}}%
\pgfpathcurveto{\pgfqpoint{8.353452in}{1.199756in}}{\pgfqpoint{8.348614in}{1.201760in}}{\pgfqpoint{8.343571in}{1.201760in}}%
\pgfpathcurveto{\pgfqpoint{8.338527in}{1.201760in}}{\pgfqpoint{8.333689in}{1.199756in}}{\pgfqpoint{8.330123in}{1.196190in}}%
\pgfpathcurveto{\pgfqpoint{8.326556in}{1.192624in}}{\pgfqpoint{8.324552in}{1.187786in}}{\pgfqpoint{8.324552in}{1.182742in}}%
\pgfpathcurveto{\pgfqpoint{8.324552in}{1.177698in}}{\pgfqpoint{8.326556in}{1.172861in}}{\pgfqpoint{8.330123in}{1.169294in}}%
\pgfpathcurveto{\pgfqpoint{8.333689in}{1.165728in}}{\pgfqpoint{8.338527in}{1.163724in}}{\pgfqpoint{8.343571in}{1.163724in}}%
\pgfpathclose%
\pgfusepath{fill}%
\end{pgfscope}%
\begin{pgfscope}%
\pgfpathrectangle{\pgfqpoint{6.572727in}{0.474100in}}{\pgfqpoint{4.227273in}{3.318700in}}%
\pgfusepath{clip}%
\pgfsetbuttcap%
\pgfsetroundjoin%
\definecolor{currentfill}{rgb}{0.267004,0.004874,0.329415}%
\pgfsetfillcolor{currentfill}%
\pgfsetfillopacity{0.700000}%
\pgfsetlinewidth{0.000000pt}%
\definecolor{currentstroke}{rgb}{0.000000,0.000000,0.000000}%
\pgfsetstrokecolor{currentstroke}%
\pgfsetstrokeopacity{0.700000}%
\pgfsetdash{}{0pt}%
\pgfpathmoveto{\pgfqpoint{7.792378in}{1.357340in}}%
\pgfpathcurveto{\pgfqpoint{7.797421in}{1.357340in}}{\pgfqpoint{7.802259in}{1.359344in}}{\pgfqpoint{7.805825in}{1.362910in}}%
\pgfpathcurveto{\pgfqpoint{7.809392in}{1.366477in}}{\pgfqpoint{7.811396in}{1.371314in}}{\pgfqpoint{7.811396in}{1.376358in}}%
\pgfpathcurveto{\pgfqpoint{7.811396in}{1.381402in}}{\pgfqpoint{7.809392in}{1.386239in}}{\pgfqpoint{7.805825in}{1.389806in}}%
\pgfpathcurveto{\pgfqpoint{7.802259in}{1.393372in}}{\pgfqpoint{7.797421in}{1.395376in}}{\pgfqpoint{7.792378in}{1.395376in}}%
\pgfpathcurveto{\pgfqpoint{7.787334in}{1.395376in}}{\pgfqpoint{7.782496in}{1.393372in}}{\pgfqpoint{7.778930in}{1.389806in}}%
\pgfpathcurveto{\pgfqpoint{7.775363in}{1.386239in}}{\pgfqpoint{7.773359in}{1.381402in}}{\pgfqpoint{7.773359in}{1.376358in}}%
\pgfpathcurveto{\pgfqpoint{7.773359in}{1.371314in}}{\pgfqpoint{7.775363in}{1.366477in}}{\pgfqpoint{7.778930in}{1.362910in}}%
\pgfpathcurveto{\pgfqpoint{7.782496in}{1.359344in}}{\pgfqpoint{7.787334in}{1.357340in}}{\pgfqpoint{7.792378in}{1.357340in}}%
\pgfpathclose%
\pgfusepath{fill}%
\end{pgfscope}%
\begin{pgfscope}%
\pgfpathrectangle{\pgfqpoint{6.572727in}{0.474100in}}{\pgfqpoint{4.227273in}{3.318700in}}%
\pgfusepath{clip}%
\pgfsetbuttcap%
\pgfsetroundjoin%
\definecolor{currentfill}{rgb}{0.267004,0.004874,0.329415}%
\pgfsetfillcolor{currentfill}%
\pgfsetfillopacity{0.700000}%
\pgfsetlinewidth{0.000000pt}%
\definecolor{currentstroke}{rgb}{0.000000,0.000000,0.000000}%
\pgfsetstrokecolor{currentstroke}%
\pgfsetstrokeopacity{0.700000}%
\pgfsetdash{}{0pt}%
\pgfpathmoveto{\pgfqpoint{7.940147in}{1.639148in}}%
\pgfpathcurveto{\pgfqpoint{7.945191in}{1.639148in}}{\pgfqpoint{7.950028in}{1.641151in}}{\pgfqpoint{7.953595in}{1.644718in}}%
\pgfpathcurveto{\pgfqpoint{7.957161in}{1.648284in}}{\pgfqpoint{7.959165in}{1.653122in}}{\pgfqpoint{7.959165in}{1.658166in}}%
\pgfpathcurveto{\pgfqpoint{7.959165in}{1.663209in}}{\pgfqpoint{7.957161in}{1.668047in}}{\pgfqpoint{7.953595in}{1.671614in}}%
\pgfpathcurveto{\pgfqpoint{7.950028in}{1.675180in}}{\pgfqpoint{7.945191in}{1.677184in}}{\pgfqpoint{7.940147in}{1.677184in}}%
\pgfpathcurveto{\pgfqpoint{7.935103in}{1.677184in}}{\pgfqpoint{7.930265in}{1.675180in}}{\pgfqpoint{7.926699in}{1.671614in}}%
\pgfpathcurveto{\pgfqpoint{7.923133in}{1.668047in}}{\pgfqpoint{7.921129in}{1.663209in}}{\pgfqpoint{7.921129in}{1.658166in}}%
\pgfpathcurveto{\pgfqpoint{7.921129in}{1.653122in}}{\pgfqpoint{7.923133in}{1.648284in}}{\pgfqpoint{7.926699in}{1.644718in}}%
\pgfpathcurveto{\pgfqpoint{7.930265in}{1.641151in}}{\pgfqpoint{7.935103in}{1.639148in}}{\pgfqpoint{7.940147in}{1.639148in}}%
\pgfpathclose%
\pgfusepath{fill}%
\end{pgfscope}%
\begin{pgfscope}%
\pgfpathrectangle{\pgfqpoint{6.572727in}{0.474100in}}{\pgfqpoint{4.227273in}{3.318700in}}%
\pgfusepath{clip}%
\pgfsetbuttcap%
\pgfsetroundjoin%
\definecolor{currentfill}{rgb}{0.267004,0.004874,0.329415}%
\pgfsetfillcolor{currentfill}%
\pgfsetfillopacity{0.700000}%
\pgfsetlinewidth{0.000000pt}%
\definecolor{currentstroke}{rgb}{0.000000,0.000000,0.000000}%
\pgfsetstrokecolor{currentstroke}%
\pgfsetstrokeopacity{0.700000}%
\pgfsetdash{}{0pt}%
\pgfpathmoveto{\pgfqpoint{8.380058in}{1.280001in}}%
\pgfpathcurveto{\pgfqpoint{8.385102in}{1.280001in}}{\pgfqpoint{8.389940in}{1.282005in}}{\pgfqpoint{8.393506in}{1.285572in}}%
\pgfpathcurveto{\pgfqpoint{8.397073in}{1.289138in}}{\pgfqpoint{8.399077in}{1.293976in}}{\pgfqpoint{8.399077in}{1.299019in}}%
\pgfpathcurveto{\pgfqpoint{8.399077in}{1.304063in}}{\pgfqpoint{8.397073in}{1.308901in}}{\pgfqpoint{8.393506in}{1.312467in}}%
\pgfpathcurveto{\pgfqpoint{8.389940in}{1.316034in}}{\pgfqpoint{8.385102in}{1.318038in}}{\pgfqpoint{8.380058in}{1.318038in}}%
\pgfpathcurveto{\pgfqpoint{8.375015in}{1.318038in}}{\pgfqpoint{8.370177in}{1.316034in}}{\pgfqpoint{8.366611in}{1.312467in}}%
\pgfpathcurveto{\pgfqpoint{8.363044in}{1.308901in}}{\pgfqpoint{8.361040in}{1.304063in}}{\pgfqpoint{8.361040in}{1.299019in}}%
\pgfpathcurveto{\pgfqpoint{8.361040in}{1.293976in}}{\pgfqpoint{8.363044in}{1.289138in}}{\pgfqpoint{8.366611in}{1.285572in}}%
\pgfpathcurveto{\pgfqpoint{8.370177in}{1.282005in}}{\pgfqpoint{8.375015in}{1.280001in}}{\pgfqpoint{8.380058in}{1.280001in}}%
\pgfpathclose%
\pgfusepath{fill}%
\end{pgfscope}%
\begin{pgfscope}%
\pgfpathrectangle{\pgfqpoint{6.572727in}{0.474100in}}{\pgfqpoint{4.227273in}{3.318700in}}%
\pgfusepath{clip}%
\pgfsetbuttcap%
\pgfsetroundjoin%
\definecolor{currentfill}{rgb}{0.127568,0.566949,0.550556}%
\pgfsetfillcolor{currentfill}%
\pgfsetfillopacity{0.700000}%
\pgfsetlinewidth{0.000000pt}%
\definecolor{currentstroke}{rgb}{0.000000,0.000000,0.000000}%
\pgfsetstrokecolor{currentstroke}%
\pgfsetstrokeopacity{0.700000}%
\pgfsetdash{}{0pt}%
\pgfpathmoveto{\pgfqpoint{9.698185in}{1.945955in}}%
\pgfpathcurveto{\pgfqpoint{9.703229in}{1.945955in}}{\pgfqpoint{9.708067in}{1.947959in}}{\pgfqpoint{9.711633in}{1.951525in}}%
\pgfpathcurveto{\pgfqpoint{9.715199in}{1.955092in}}{\pgfqpoint{9.717203in}{1.959929in}}{\pgfqpoint{9.717203in}{1.964973in}}%
\pgfpathcurveto{\pgfqpoint{9.717203in}{1.970017in}}{\pgfqpoint{9.715199in}{1.974855in}}{\pgfqpoint{9.711633in}{1.978421in}}%
\pgfpathcurveto{\pgfqpoint{9.708067in}{1.981987in}}{\pgfqpoint{9.703229in}{1.983991in}}{\pgfqpoint{9.698185in}{1.983991in}}%
\pgfpathcurveto{\pgfqpoint{9.693141in}{1.983991in}}{\pgfqpoint{9.688304in}{1.981987in}}{\pgfqpoint{9.684737in}{1.978421in}}%
\pgfpathcurveto{\pgfqpoint{9.681171in}{1.974855in}}{\pgfqpoint{9.679167in}{1.970017in}}{\pgfqpoint{9.679167in}{1.964973in}}%
\pgfpathcurveto{\pgfqpoint{9.679167in}{1.959929in}}{\pgfqpoint{9.681171in}{1.955092in}}{\pgfqpoint{9.684737in}{1.951525in}}%
\pgfpathcurveto{\pgfqpoint{9.688304in}{1.947959in}}{\pgfqpoint{9.693141in}{1.945955in}}{\pgfqpoint{9.698185in}{1.945955in}}%
\pgfpathclose%
\pgfusepath{fill}%
\end{pgfscope}%
\begin{pgfscope}%
\pgfpathrectangle{\pgfqpoint{6.572727in}{0.474100in}}{\pgfqpoint{4.227273in}{3.318700in}}%
\pgfusepath{clip}%
\pgfsetbuttcap%
\pgfsetroundjoin%
\definecolor{currentfill}{rgb}{0.127568,0.566949,0.550556}%
\pgfsetfillcolor{currentfill}%
\pgfsetfillopacity{0.700000}%
\pgfsetlinewidth{0.000000pt}%
\definecolor{currentstroke}{rgb}{0.000000,0.000000,0.000000}%
\pgfsetstrokecolor{currentstroke}%
\pgfsetstrokeopacity{0.700000}%
\pgfsetdash{}{0pt}%
\pgfpathmoveto{\pgfqpoint{9.397700in}{1.280531in}}%
\pgfpathcurveto{\pgfqpoint{9.402744in}{1.280531in}}{\pgfqpoint{9.407582in}{1.282535in}}{\pgfqpoint{9.411148in}{1.286102in}}%
\pgfpathcurveto{\pgfqpoint{9.414715in}{1.289668in}}{\pgfqpoint{9.416719in}{1.294506in}}{\pgfqpoint{9.416719in}{1.299550in}}%
\pgfpathcurveto{\pgfqpoint{9.416719in}{1.304593in}}{\pgfqpoint{9.414715in}{1.309431in}}{\pgfqpoint{9.411148in}{1.312997in}}%
\pgfpathcurveto{\pgfqpoint{9.407582in}{1.316564in}}{\pgfqpoint{9.402744in}{1.318568in}}{\pgfqpoint{9.397700in}{1.318568in}}%
\pgfpathcurveto{\pgfqpoint{9.392657in}{1.318568in}}{\pgfqpoint{9.387819in}{1.316564in}}{\pgfqpoint{9.384253in}{1.312997in}}%
\pgfpathcurveto{\pgfqpoint{9.380686in}{1.309431in}}{\pgfqpoint{9.378682in}{1.304593in}}{\pgfqpoint{9.378682in}{1.299550in}}%
\pgfpathcurveto{\pgfqpoint{9.378682in}{1.294506in}}{\pgfqpoint{9.380686in}{1.289668in}}{\pgfqpoint{9.384253in}{1.286102in}}%
\pgfpathcurveto{\pgfqpoint{9.387819in}{1.282535in}}{\pgfqpoint{9.392657in}{1.280531in}}{\pgfqpoint{9.397700in}{1.280531in}}%
\pgfpathclose%
\pgfusepath{fill}%
\end{pgfscope}%
\begin{pgfscope}%
\pgfpathrectangle{\pgfqpoint{6.572727in}{0.474100in}}{\pgfqpoint{4.227273in}{3.318700in}}%
\pgfusepath{clip}%
\pgfsetbuttcap%
\pgfsetroundjoin%
\definecolor{currentfill}{rgb}{0.267004,0.004874,0.329415}%
\pgfsetfillcolor{currentfill}%
\pgfsetfillopacity{0.700000}%
\pgfsetlinewidth{0.000000pt}%
\definecolor{currentstroke}{rgb}{0.000000,0.000000,0.000000}%
\pgfsetstrokecolor{currentstroke}%
\pgfsetstrokeopacity{0.700000}%
\pgfsetdash{}{0pt}%
\pgfpathmoveto{\pgfqpoint{8.575376in}{1.526192in}}%
\pgfpathcurveto{\pgfqpoint{8.580420in}{1.526192in}}{\pgfqpoint{8.585258in}{1.528196in}}{\pgfqpoint{8.588824in}{1.531762in}}%
\pgfpathcurveto{\pgfqpoint{8.592390in}{1.535328in}}{\pgfqpoint{8.594394in}{1.540166in}}{\pgfqpoint{8.594394in}{1.545210in}}%
\pgfpathcurveto{\pgfqpoint{8.594394in}{1.550254in}}{\pgfqpoint{8.592390in}{1.555091in}}{\pgfqpoint{8.588824in}{1.558658in}}%
\pgfpathcurveto{\pgfqpoint{8.585258in}{1.562224in}}{\pgfqpoint{8.580420in}{1.564228in}}{\pgfqpoint{8.575376in}{1.564228in}}%
\pgfpathcurveto{\pgfqpoint{8.570333in}{1.564228in}}{\pgfqpoint{8.565495in}{1.562224in}}{\pgfqpoint{8.561928in}{1.558658in}}%
\pgfpathcurveto{\pgfqpoint{8.558362in}{1.555091in}}{\pgfqpoint{8.556358in}{1.550254in}}{\pgfqpoint{8.556358in}{1.545210in}}%
\pgfpathcurveto{\pgfqpoint{8.556358in}{1.540166in}}{\pgfqpoint{8.558362in}{1.535328in}}{\pgfqpoint{8.561928in}{1.531762in}}%
\pgfpathcurveto{\pgfqpoint{8.565495in}{1.528196in}}{\pgfqpoint{8.570333in}{1.526192in}}{\pgfqpoint{8.575376in}{1.526192in}}%
\pgfpathclose%
\pgfusepath{fill}%
\end{pgfscope}%
\begin{pgfscope}%
\pgfpathrectangle{\pgfqpoint{6.572727in}{0.474100in}}{\pgfqpoint{4.227273in}{3.318700in}}%
\pgfusepath{clip}%
\pgfsetbuttcap%
\pgfsetroundjoin%
\definecolor{currentfill}{rgb}{0.993248,0.906157,0.143936}%
\pgfsetfillcolor{currentfill}%
\pgfsetfillopacity{0.700000}%
\pgfsetlinewidth{0.000000pt}%
\definecolor{currentstroke}{rgb}{0.000000,0.000000,0.000000}%
\pgfsetstrokecolor{currentstroke}%
\pgfsetstrokeopacity{0.700000}%
\pgfsetdash{}{0pt}%
\pgfpathmoveto{\pgfqpoint{7.961889in}{2.994783in}}%
\pgfpathcurveto{\pgfqpoint{7.966932in}{2.994783in}}{\pgfqpoint{7.971770in}{2.996787in}}{\pgfqpoint{7.975336in}{3.000354in}}%
\pgfpathcurveto{\pgfqpoint{7.978903in}{3.003920in}}{\pgfqpoint{7.980907in}{3.008758in}}{\pgfqpoint{7.980907in}{3.013801in}}%
\pgfpathcurveto{\pgfqpoint{7.980907in}{3.018845in}}{\pgfqpoint{7.978903in}{3.023683in}}{\pgfqpoint{7.975336in}{3.027249in}}%
\pgfpathcurveto{\pgfqpoint{7.971770in}{3.030816in}}{\pgfqpoint{7.966932in}{3.032820in}}{\pgfqpoint{7.961889in}{3.032820in}}%
\pgfpathcurveto{\pgfqpoint{7.956845in}{3.032820in}}{\pgfqpoint{7.952007in}{3.030816in}}{\pgfqpoint{7.948441in}{3.027249in}}%
\pgfpathcurveto{\pgfqpoint{7.944874in}{3.023683in}}{\pgfqpoint{7.942870in}{3.018845in}}{\pgfqpoint{7.942870in}{3.013801in}}%
\pgfpathcurveto{\pgfqpoint{7.942870in}{3.008758in}}{\pgfqpoint{7.944874in}{3.003920in}}{\pgfqpoint{7.948441in}{3.000354in}}%
\pgfpathcurveto{\pgfqpoint{7.952007in}{2.996787in}}{\pgfqpoint{7.956845in}{2.994783in}}{\pgfqpoint{7.961889in}{2.994783in}}%
\pgfpathclose%
\pgfusepath{fill}%
\end{pgfscope}%
\begin{pgfscope}%
\pgfpathrectangle{\pgfqpoint{6.572727in}{0.474100in}}{\pgfqpoint{4.227273in}{3.318700in}}%
\pgfusepath{clip}%
\pgfsetbuttcap%
\pgfsetroundjoin%
\definecolor{currentfill}{rgb}{0.127568,0.566949,0.550556}%
\pgfsetfillcolor{currentfill}%
\pgfsetfillopacity{0.700000}%
\pgfsetlinewidth{0.000000pt}%
\definecolor{currentstroke}{rgb}{0.000000,0.000000,0.000000}%
\pgfsetstrokecolor{currentstroke}%
\pgfsetstrokeopacity{0.700000}%
\pgfsetdash{}{0pt}%
\pgfpathmoveto{\pgfqpoint{9.692532in}{2.127383in}}%
\pgfpathcurveto{\pgfqpoint{9.697576in}{2.127383in}}{\pgfqpoint{9.702413in}{2.129387in}}{\pgfqpoint{9.705980in}{2.132953in}}%
\pgfpathcurveto{\pgfqpoint{9.709546in}{2.136519in}}{\pgfqpoint{9.711550in}{2.141357in}}{\pgfqpoint{9.711550in}{2.146401in}}%
\pgfpathcurveto{\pgfqpoint{9.711550in}{2.151444in}}{\pgfqpoint{9.709546in}{2.156282in}}{\pgfqpoint{9.705980in}{2.159849in}}%
\pgfpathcurveto{\pgfqpoint{9.702413in}{2.163415in}}{\pgfqpoint{9.697576in}{2.165419in}}{\pgfqpoint{9.692532in}{2.165419in}}%
\pgfpathcurveto{\pgfqpoint{9.687488in}{2.165419in}}{\pgfqpoint{9.682651in}{2.163415in}}{\pgfqpoint{9.679084in}{2.159849in}}%
\pgfpathcurveto{\pgfqpoint{9.675518in}{2.156282in}}{\pgfqpoint{9.673514in}{2.151444in}}{\pgfqpoint{9.673514in}{2.146401in}}%
\pgfpathcurveto{\pgfqpoint{9.673514in}{2.141357in}}{\pgfqpoint{9.675518in}{2.136519in}}{\pgfqpoint{9.679084in}{2.132953in}}%
\pgfpathcurveto{\pgfqpoint{9.682651in}{2.129387in}}{\pgfqpoint{9.687488in}{2.127383in}}{\pgfqpoint{9.692532in}{2.127383in}}%
\pgfpathclose%
\pgfusepath{fill}%
\end{pgfscope}%
\begin{pgfscope}%
\pgfpathrectangle{\pgfqpoint{6.572727in}{0.474100in}}{\pgfqpoint{4.227273in}{3.318700in}}%
\pgfusepath{clip}%
\pgfsetbuttcap%
\pgfsetroundjoin%
\definecolor{currentfill}{rgb}{0.267004,0.004874,0.329415}%
\pgfsetfillcolor{currentfill}%
\pgfsetfillopacity{0.700000}%
\pgfsetlinewidth{0.000000pt}%
\definecolor{currentstroke}{rgb}{0.000000,0.000000,0.000000}%
\pgfsetstrokecolor{currentstroke}%
\pgfsetstrokeopacity{0.700000}%
\pgfsetdash{}{0pt}%
\pgfpathmoveto{\pgfqpoint{8.102790in}{1.286592in}}%
\pgfpathcurveto{\pgfqpoint{8.107833in}{1.286592in}}{\pgfqpoint{8.112671in}{1.288596in}}{\pgfqpoint{8.116238in}{1.292163in}}%
\pgfpathcurveto{\pgfqpoint{8.119804in}{1.295729in}}{\pgfqpoint{8.121808in}{1.300567in}}{\pgfqpoint{8.121808in}{1.305611in}}%
\pgfpathcurveto{\pgfqpoint{8.121808in}{1.310654in}}{\pgfqpoint{8.119804in}{1.315492in}}{\pgfqpoint{8.116238in}{1.319058in}}%
\pgfpathcurveto{\pgfqpoint{8.112671in}{1.322625in}}{\pgfqpoint{8.107833in}{1.324629in}}{\pgfqpoint{8.102790in}{1.324629in}}%
\pgfpathcurveto{\pgfqpoint{8.097746in}{1.324629in}}{\pgfqpoint{8.092908in}{1.322625in}}{\pgfqpoint{8.089342in}{1.319058in}}%
\pgfpathcurveto{\pgfqpoint{8.085775in}{1.315492in}}{\pgfqpoint{8.083772in}{1.310654in}}{\pgfqpoint{8.083772in}{1.305611in}}%
\pgfpathcurveto{\pgfqpoint{8.083772in}{1.300567in}}{\pgfqpoint{8.085775in}{1.295729in}}{\pgfqpoint{8.089342in}{1.292163in}}%
\pgfpathcurveto{\pgfqpoint{8.092908in}{1.288596in}}{\pgfqpoint{8.097746in}{1.286592in}}{\pgfqpoint{8.102790in}{1.286592in}}%
\pgfpathclose%
\pgfusepath{fill}%
\end{pgfscope}%
\begin{pgfscope}%
\pgfpathrectangle{\pgfqpoint{6.572727in}{0.474100in}}{\pgfqpoint{4.227273in}{3.318700in}}%
\pgfusepath{clip}%
\pgfsetbuttcap%
\pgfsetroundjoin%
\definecolor{currentfill}{rgb}{0.267004,0.004874,0.329415}%
\pgfsetfillcolor{currentfill}%
\pgfsetfillopacity{0.700000}%
\pgfsetlinewidth{0.000000pt}%
\definecolor{currentstroke}{rgb}{0.000000,0.000000,0.000000}%
\pgfsetstrokecolor{currentstroke}%
\pgfsetstrokeopacity{0.700000}%
\pgfsetdash{}{0pt}%
\pgfpathmoveto{\pgfqpoint{7.747129in}{1.225143in}}%
\pgfpathcurveto{\pgfqpoint{7.752173in}{1.225143in}}{\pgfqpoint{7.757010in}{1.227147in}}{\pgfqpoint{7.760577in}{1.230713in}}%
\pgfpathcurveto{\pgfqpoint{7.764143in}{1.234280in}}{\pgfqpoint{7.766147in}{1.239118in}}{\pgfqpoint{7.766147in}{1.244161in}}%
\pgfpathcurveto{\pgfqpoint{7.766147in}{1.249205in}}{\pgfqpoint{7.764143in}{1.254043in}}{\pgfqpoint{7.760577in}{1.257609in}}%
\pgfpathcurveto{\pgfqpoint{7.757010in}{1.261176in}}{\pgfqpoint{7.752173in}{1.263179in}}{\pgfqpoint{7.747129in}{1.263179in}}%
\pgfpathcurveto{\pgfqpoint{7.742085in}{1.263179in}}{\pgfqpoint{7.737248in}{1.261176in}}{\pgfqpoint{7.733681in}{1.257609in}}%
\pgfpathcurveto{\pgfqpoint{7.730115in}{1.254043in}}{\pgfqpoint{7.728111in}{1.249205in}}{\pgfqpoint{7.728111in}{1.244161in}}%
\pgfpathcurveto{\pgfqpoint{7.728111in}{1.239118in}}{\pgfqpoint{7.730115in}{1.234280in}}{\pgfqpoint{7.733681in}{1.230713in}}%
\pgfpathcurveto{\pgfqpoint{7.737248in}{1.227147in}}{\pgfqpoint{7.742085in}{1.225143in}}{\pgfqpoint{7.747129in}{1.225143in}}%
\pgfpathclose%
\pgfusepath{fill}%
\end{pgfscope}%
\begin{pgfscope}%
\pgfpathrectangle{\pgfqpoint{6.572727in}{0.474100in}}{\pgfqpoint{4.227273in}{3.318700in}}%
\pgfusepath{clip}%
\pgfsetbuttcap%
\pgfsetroundjoin%
\definecolor{currentfill}{rgb}{0.993248,0.906157,0.143936}%
\pgfsetfillcolor{currentfill}%
\pgfsetfillopacity{0.700000}%
\pgfsetlinewidth{0.000000pt}%
\definecolor{currentstroke}{rgb}{0.000000,0.000000,0.000000}%
\pgfsetstrokecolor{currentstroke}%
\pgfsetstrokeopacity{0.700000}%
\pgfsetdash{}{0pt}%
\pgfpathmoveto{\pgfqpoint{8.573791in}{3.275027in}}%
\pgfpathcurveto{\pgfqpoint{8.578835in}{3.275027in}}{\pgfqpoint{8.583672in}{3.277031in}}{\pgfqpoint{8.587239in}{3.280597in}}%
\pgfpathcurveto{\pgfqpoint{8.590805in}{3.284163in}}{\pgfqpoint{8.592809in}{3.289001in}}{\pgfqpoint{8.592809in}{3.294045in}}%
\pgfpathcurveto{\pgfqpoint{8.592809in}{3.299088in}}{\pgfqpoint{8.590805in}{3.303926in}}{\pgfqpoint{8.587239in}{3.307493in}}%
\pgfpathcurveto{\pgfqpoint{8.583672in}{3.311059in}}{\pgfqpoint{8.578835in}{3.313063in}}{\pgfqpoint{8.573791in}{3.313063in}}%
\pgfpathcurveto{\pgfqpoint{8.568747in}{3.313063in}}{\pgfqpoint{8.563909in}{3.311059in}}{\pgfqpoint{8.560343in}{3.307493in}}%
\pgfpathcurveto{\pgfqpoint{8.556777in}{3.303926in}}{\pgfqpoint{8.554773in}{3.299088in}}{\pgfqpoint{8.554773in}{3.294045in}}%
\pgfpathcurveto{\pgfqpoint{8.554773in}{3.289001in}}{\pgfqpoint{8.556777in}{3.284163in}}{\pgfqpoint{8.560343in}{3.280597in}}%
\pgfpathcurveto{\pgfqpoint{8.563909in}{3.277031in}}{\pgfqpoint{8.568747in}{3.275027in}}{\pgfqpoint{8.573791in}{3.275027in}}%
\pgfpathclose%
\pgfusepath{fill}%
\end{pgfscope}%
\begin{pgfscope}%
\pgfpathrectangle{\pgfqpoint{6.572727in}{0.474100in}}{\pgfqpoint{4.227273in}{3.318700in}}%
\pgfusepath{clip}%
\pgfsetbuttcap%
\pgfsetroundjoin%
\definecolor{currentfill}{rgb}{0.993248,0.906157,0.143936}%
\pgfsetfillcolor{currentfill}%
\pgfsetfillopacity{0.700000}%
\pgfsetlinewidth{0.000000pt}%
\definecolor{currentstroke}{rgb}{0.000000,0.000000,0.000000}%
\pgfsetstrokecolor{currentstroke}%
\pgfsetstrokeopacity{0.700000}%
\pgfsetdash{}{0pt}%
\pgfpathmoveto{\pgfqpoint{7.415957in}{2.471074in}}%
\pgfpathcurveto{\pgfqpoint{7.421001in}{2.471074in}}{\pgfqpoint{7.425839in}{2.473078in}}{\pgfqpoint{7.429405in}{2.476644in}}%
\pgfpathcurveto{\pgfqpoint{7.432972in}{2.480211in}}{\pgfqpoint{7.434976in}{2.485048in}}{\pgfqpoint{7.434976in}{2.490092in}}%
\pgfpathcurveto{\pgfqpoint{7.434976in}{2.495136in}}{\pgfqpoint{7.432972in}{2.499973in}}{\pgfqpoint{7.429405in}{2.503540in}}%
\pgfpathcurveto{\pgfqpoint{7.425839in}{2.507106in}}{\pgfqpoint{7.421001in}{2.509110in}}{\pgfqpoint{7.415957in}{2.509110in}}%
\pgfpathcurveto{\pgfqpoint{7.410914in}{2.509110in}}{\pgfqpoint{7.406076in}{2.507106in}}{\pgfqpoint{7.402510in}{2.503540in}}%
\pgfpathcurveto{\pgfqpoint{7.398943in}{2.499973in}}{\pgfqpoint{7.396939in}{2.495136in}}{\pgfqpoint{7.396939in}{2.490092in}}%
\pgfpathcurveto{\pgfqpoint{7.396939in}{2.485048in}}{\pgfqpoint{7.398943in}{2.480211in}}{\pgfqpoint{7.402510in}{2.476644in}}%
\pgfpathcurveto{\pgfqpoint{7.406076in}{2.473078in}}{\pgfqpoint{7.410914in}{2.471074in}}{\pgfqpoint{7.415957in}{2.471074in}}%
\pgfpathclose%
\pgfusepath{fill}%
\end{pgfscope}%
\begin{pgfscope}%
\pgfpathrectangle{\pgfqpoint{6.572727in}{0.474100in}}{\pgfqpoint{4.227273in}{3.318700in}}%
\pgfusepath{clip}%
\pgfsetbuttcap%
\pgfsetroundjoin%
\definecolor{currentfill}{rgb}{0.993248,0.906157,0.143936}%
\pgfsetfillcolor{currentfill}%
\pgfsetfillopacity{0.700000}%
\pgfsetlinewidth{0.000000pt}%
\definecolor{currentstroke}{rgb}{0.000000,0.000000,0.000000}%
\pgfsetstrokecolor{currentstroke}%
\pgfsetstrokeopacity{0.700000}%
\pgfsetdash{}{0pt}%
\pgfpathmoveto{\pgfqpoint{8.074452in}{2.975412in}}%
\pgfpathcurveto{\pgfqpoint{8.079496in}{2.975412in}}{\pgfqpoint{8.084333in}{2.977416in}}{\pgfqpoint{8.087900in}{2.980982in}}%
\pgfpathcurveto{\pgfqpoint{8.091466in}{2.984549in}}{\pgfqpoint{8.093470in}{2.989386in}}{\pgfqpoint{8.093470in}{2.994430in}}%
\pgfpathcurveto{\pgfqpoint{8.093470in}{2.999474in}}{\pgfqpoint{8.091466in}{3.004311in}}{\pgfqpoint{8.087900in}{3.007878in}}%
\pgfpathcurveto{\pgfqpoint{8.084333in}{3.011444in}}{\pgfqpoint{8.079496in}{3.013448in}}{\pgfqpoint{8.074452in}{3.013448in}}%
\pgfpathcurveto{\pgfqpoint{8.069408in}{3.013448in}}{\pgfqpoint{8.064571in}{3.011444in}}{\pgfqpoint{8.061004in}{3.007878in}}%
\pgfpathcurveto{\pgfqpoint{8.057438in}{3.004311in}}{\pgfqpoint{8.055434in}{2.999474in}}{\pgfqpoint{8.055434in}{2.994430in}}%
\pgfpathcurveto{\pgfqpoint{8.055434in}{2.989386in}}{\pgfqpoint{8.057438in}{2.984549in}}{\pgfqpoint{8.061004in}{2.980982in}}%
\pgfpathcurveto{\pgfqpoint{8.064571in}{2.977416in}}{\pgfqpoint{8.069408in}{2.975412in}}{\pgfqpoint{8.074452in}{2.975412in}}%
\pgfpathclose%
\pgfusepath{fill}%
\end{pgfscope}%
\begin{pgfscope}%
\pgfpathrectangle{\pgfqpoint{6.572727in}{0.474100in}}{\pgfqpoint{4.227273in}{3.318700in}}%
\pgfusepath{clip}%
\pgfsetbuttcap%
\pgfsetroundjoin%
\definecolor{currentfill}{rgb}{0.267004,0.004874,0.329415}%
\pgfsetfillcolor{currentfill}%
\pgfsetfillopacity{0.700000}%
\pgfsetlinewidth{0.000000pt}%
\definecolor{currentstroke}{rgb}{0.000000,0.000000,0.000000}%
\pgfsetstrokecolor{currentstroke}%
\pgfsetstrokeopacity{0.700000}%
\pgfsetdash{}{0pt}%
\pgfpathmoveto{\pgfqpoint{8.042473in}{1.200590in}}%
\pgfpathcurveto{\pgfqpoint{8.047517in}{1.200590in}}{\pgfqpoint{8.052355in}{1.202594in}}{\pgfqpoint{8.055921in}{1.206160in}}%
\pgfpathcurveto{\pgfqpoint{8.059488in}{1.209727in}}{\pgfqpoint{8.061492in}{1.214564in}}{\pgfqpoint{8.061492in}{1.219608in}}%
\pgfpathcurveto{\pgfqpoint{8.061492in}{1.224652in}}{\pgfqpoint{8.059488in}{1.229489in}}{\pgfqpoint{8.055921in}{1.233056in}}%
\pgfpathcurveto{\pgfqpoint{8.052355in}{1.236622in}}{\pgfqpoint{8.047517in}{1.238626in}}{\pgfqpoint{8.042473in}{1.238626in}}%
\pgfpathcurveto{\pgfqpoint{8.037430in}{1.238626in}}{\pgfqpoint{8.032592in}{1.236622in}}{\pgfqpoint{8.029026in}{1.233056in}}%
\pgfpathcurveto{\pgfqpoint{8.025459in}{1.229489in}}{\pgfqpoint{8.023455in}{1.224652in}}{\pgfqpoint{8.023455in}{1.219608in}}%
\pgfpathcurveto{\pgfqpoint{8.023455in}{1.214564in}}{\pgfqpoint{8.025459in}{1.209727in}}{\pgfqpoint{8.029026in}{1.206160in}}%
\pgfpathcurveto{\pgfqpoint{8.032592in}{1.202594in}}{\pgfqpoint{8.037430in}{1.200590in}}{\pgfqpoint{8.042473in}{1.200590in}}%
\pgfpathclose%
\pgfusepath{fill}%
\end{pgfscope}%
\begin{pgfscope}%
\pgfpathrectangle{\pgfqpoint{6.572727in}{0.474100in}}{\pgfqpoint{4.227273in}{3.318700in}}%
\pgfusepath{clip}%
\pgfsetbuttcap%
\pgfsetroundjoin%
\definecolor{currentfill}{rgb}{0.993248,0.906157,0.143936}%
\pgfsetfillcolor{currentfill}%
\pgfsetfillopacity{0.700000}%
\pgfsetlinewidth{0.000000pt}%
\definecolor{currentstroke}{rgb}{0.000000,0.000000,0.000000}%
\pgfsetstrokecolor{currentstroke}%
\pgfsetstrokeopacity{0.700000}%
\pgfsetdash{}{0pt}%
\pgfpathmoveto{\pgfqpoint{8.057104in}{2.418049in}}%
\pgfpathcurveto{\pgfqpoint{8.062148in}{2.418049in}}{\pgfqpoint{8.066985in}{2.420053in}}{\pgfqpoint{8.070552in}{2.423619in}}%
\pgfpathcurveto{\pgfqpoint{8.074118in}{2.427186in}}{\pgfqpoint{8.076122in}{2.432024in}}{\pgfqpoint{8.076122in}{2.437067in}}%
\pgfpathcurveto{\pgfqpoint{8.076122in}{2.442111in}}{\pgfqpoint{8.074118in}{2.446949in}}{\pgfqpoint{8.070552in}{2.450515in}}%
\pgfpathcurveto{\pgfqpoint{8.066985in}{2.454082in}}{\pgfqpoint{8.062148in}{2.456085in}}{\pgfqpoint{8.057104in}{2.456085in}}%
\pgfpathcurveto{\pgfqpoint{8.052060in}{2.456085in}}{\pgfqpoint{8.047223in}{2.454082in}}{\pgfqpoint{8.043656in}{2.450515in}}%
\pgfpathcurveto{\pgfqpoint{8.040090in}{2.446949in}}{\pgfqpoint{8.038086in}{2.442111in}}{\pgfqpoint{8.038086in}{2.437067in}}%
\pgfpathcurveto{\pgfqpoint{8.038086in}{2.432024in}}{\pgfqpoint{8.040090in}{2.427186in}}{\pgfqpoint{8.043656in}{2.423619in}}%
\pgfpathcurveto{\pgfqpoint{8.047223in}{2.420053in}}{\pgfqpoint{8.052060in}{2.418049in}}{\pgfqpoint{8.057104in}{2.418049in}}%
\pgfpathclose%
\pgfusepath{fill}%
\end{pgfscope}%
\begin{pgfscope}%
\pgfpathrectangle{\pgfqpoint{6.572727in}{0.474100in}}{\pgfqpoint{4.227273in}{3.318700in}}%
\pgfusepath{clip}%
\pgfsetbuttcap%
\pgfsetroundjoin%
\definecolor{currentfill}{rgb}{0.127568,0.566949,0.550556}%
\pgfsetfillcolor{currentfill}%
\pgfsetfillopacity{0.700000}%
\pgfsetlinewidth{0.000000pt}%
\definecolor{currentstroke}{rgb}{0.000000,0.000000,0.000000}%
\pgfsetstrokecolor{currentstroke}%
\pgfsetstrokeopacity{0.700000}%
\pgfsetdash{}{0pt}%
\pgfpathmoveto{\pgfqpoint{9.446345in}{1.870362in}}%
\pgfpathcurveto{\pgfqpoint{9.451389in}{1.870362in}}{\pgfqpoint{9.456226in}{1.872366in}}{\pgfqpoint{9.459793in}{1.875932in}}%
\pgfpathcurveto{\pgfqpoint{9.463359in}{1.879499in}}{\pgfqpoint{9.465363in}{1.884336in}}{\pgfqpoint{9.465363in}{1.889380in}}%
\pgfpathcurveto{\pgfqpoint{9.465363in}{1.894424in}}{\pgfqpoint{9.463359in}{1.899261in}}{\pgfqpoint{9.459793in}{1.902828in}}%
\pgfpathcurveto{\pgfqpoint{9.456226in}{1.906394in}}{\pgfqpoint{9.451389in}{1.908398in}}{\pgfqpoint{9.446345in}{1.908398in}}%
\pgfpathcurveto{\pgfqpoint{9.441301in}{1.908398in}}{\pgfqpoint{9.436463in}{1.906394in}}{\pgfqpoint{9.432897in}{1.902828in}}%
\pgfpathcurveto{\pgfqpoint{9.429331in}{1.899261in}}{\pgfqpoint{9.427327in}{1.894424in}}{\pgfqpoint{9.427327in}{1.889380in}}%
\pgfpathcurveto{\pgfqpoint{9.427327in}{1.884336in}}{\pgfqpoint{9.429331in}{1.879499in}}{\pgfqpoint{9.432897in}{1.875932in}}%
\pgfpathcurveto{\pgfqpoint{9.436463in}{1.872366in}}{\pgfqpoint{9.441301in}{1.870362in}}{\pgfqpoint{9.446345in}{1.870362in}}%
\pgfpathclose%
\pgfusepath{fill}%
\end{pgfscope}%
\begin{pgfscope}%
\pgfpathrectangle{\pgfqpoint{6.572727in}{0.474100in}}{\pgfqpoint{4.227273in}{3.318700in}}%
\pgfusepath{clip}%
\pgfsetbuttcap%
\pgfsetroundjoin%
\definecolor{currentfill}{rgb}{0.127568,0.566949,0.550556}%
\pgfsetfillcolor{currentfill}%
\pgfsetfillopacity{0.700000}%
\pgfsetlinewidth{0.000000pt}%
\definecolor{currentstroke}{rgb}{0.000000,0.000000,0.000000}%
\pgfsetstrokecolor{currentstroke}%
\pgfsetstrokeopacity{0.700000}%
\pgfsetdash{}{0pt}%
\pgfpathmoveto{\pgfqpoint{9.648030in}{1.465150in}}%
\pgfpathcurveto{\pgfqpoint{9.653074in}{1.465150in}}{\pgfqpoint{9.657912in}{1.467154in}}{\pgfqpoint{9.661478in}{1.470720in}}%
\pgfpathcurveto{\pgfqpoint{9.665044in}{1.474287in}}{\pgfqpoint{9.667048in}{1.479125in}}{\pgfqpoint{9.667048in}{1.484168in}}%
\pgfpathcurveto{\pgfqpoint{9.667048in}{1.489212in}}{\pgfqpoint{9.665044in}{1.494050in}}{\pgfqpoint{9.661478in}{1.497616in}}%
\pgfpathcurveto{\pgfqpoint{9.657912in}{1.501183in}}{\pgfqpoint{9.653074in}{1.503186in}}{\pgfqpoint{9.648030in}{1.503186in}}%
\pgfpathcurveto{\pgfqpoint{9.642986in}{1.503186in}}{\pgfqpoint{9.638149in}{1.501183in}}{\pgfqpoint{9.634582in}{1.497616in}}%
\pgfpathcurveto{\pgfqpoint{9.631016in}{1.494050in}}{\pgfqpoint{9.629012in}{1.489212in}}{\pgfqpoint{9.629012in}{1.484168in}}%
\pgfpathcurveto{\pgfqpoint{9.629012in}{1.479125in}}{\pgfqpoint{9.631016in}{1.474287in}}{\pgfqpoint{9.634582in}{1.470720in}}%
\pgfpathcurveto{\pgfqpoint{9.638149in}{1.467154in}}{\pgfqpoint{9.642986in}{1.465150in}}{\pgfqpoint{9.648030in}{1.465150in}}%
\pgfpathclose%
\pgfusepath{fill}%
\end{pgfscope}%
\begin{pgfscope}%
\pgfpathrectangle{\pgfqpoint{6.572727in}{0.474100in}}{\pgfqpoint{4.227273in}{3.318700in}}%
\pgfusepath{clip}%
\pgfsetbuttcap%
\pgfsetroundjoin%
\definecolor{currentfill}{rgb}{0.267004,0.004874,0.329415}%
\pgfsetfillcolor{currentfill}%
\pgfsetfillopacity{0.700000}%
\pgfsetlinewidth{0.000000pt}%
\definecolor{currentstroke}{rgb}{0.000000,0.000000,0.000000}%
\pgfsetstrokecolor{currentstroke}%
\pgfsetstrokeopacity{0.700000}%
\pgfsetdash{}{0pt}%
\pgfpathmoveto{\pgfqpoint{7.824022in}{1.601359in}}%
\pgfpathcurveto{\pgfqpoint{7.829066in}{1.601359in}}{\pgfqpoint{7.833904in}{1.603363in}}{\pgfqpoint{7.837470in}{1.606929in}}%
\pgfpathcurveto{\pgfqpoint{7.841037in}{1.610496in}}{\pgfqpoint{7.843040in}{1.615333in}}{\pgfqpoint{7.843040in}{1.620377in}}%
\pgfpathcurveto{\pgfqpoint{7.843040in}{1.625421in}}{\pgfqpoint{7.841037in}{1.630259in}}{\pgfqpoint{7.837470in}{1.633825in}}%
\pgfpathcurveto{\pgfqpoint{7.833904in}{1.637391in}}{\pgfqpoint{7.829066in}{1.639395in}}{\pgfqpoint{7.824022in}{1.639395in}}%
\pgfpathcurveto{\pgfqpoint{7.818979in}{1.639395in}}{\pgfqpoint{7.814141in}{1.637391in}}{\pgfqpoint{7.810574in}{1.633825in}}%
\pgfpathcurveto{\pgfqpoint{7.807008in}{1.630259in}}{\pgfqpoint{7.805004in}{1.625421in}}{\pgfqpoint{7.805004in}{1.620377in}}%
\pgfpathcurveto{\pgfqpoint{7.805004in}{1.615333in}}{\pgfqpoint{7.807008in}{1.610496in}}{\pgfqpoint{7.810574in}{1.606929in}}%
\pgfpathcurveto{\pgfqpoint{7.814141in}{1.603363in}}{\pgfqpoint{7.818979in}{1.601359in}}{\pgfqpoint{7.824022in}{1.601359in}}%
\pgfpathclose%
\pgfusepath{fill}%
\end{pgfscope}%
\begin{pgfscope}%
\pgfpathrectangle{\pgfqpoint{6.572727in}{0.474100in}}{\pgfqpoint{4.227273in}{3.318700in}}%
\pgfusepath{clip}%
\pgfsetbuttcap%
\pgfsetroundjoin%
\definecolor{currentfill}{rgb}{0.993248,0.906157,0.143936}%
\pgfsetfillcolor{currentfill}%
\pgfsetfillopacity{0.700000}%
\pgfsetlinewidth{0.000000pt}%
\definecolor{currentstroke}{rgb}{0.000000,0.000000,0.000000}%
\pgfsetstrokecolor{currentstroke}%
\pgfsetstrokeopacity{0.700000}%
\pgfsetdash{}{0pt}%
\pgfpathmoveto{\pgfqpoint{8.428063in}{3.165375in}}%
\pgfpathcurveto{\pgfqpoint{8.433107in}{3.165375in}}{\pgfqpoint{8.437945in}{3.167379in}}{\pgfqpoint{8.441511in}{3.170946in}}%
\pgfpathcurveto{\pgfqpoint{8.445078in}{3.174512in}}{\pgfqpoint{8.447081in}{3.179350in}}{\pgfqpoint{8.447081in}{3.184393in}}%
\pgfpathcurveto{\pgfqpoint{8.447081in}{3.189437in}}{\pgfqpoint{8.445078in}{3.194275in}}{\pgfqpoint{8.441511in}{3.197841in}}%
\pgfpathcurveto{\pgfqpoint{8.437945in}{3.201408in}}{\pgfqpoint{8.433107in}{3.203412in}}{\pgfqpoint{8.428063in}{3.203412in}}%
\pgfpathcurveto{\pgfqpoint{8.423020in}{3.203412in}}{\pgfqpoint{8.418182in}{3.201408in}}{\pgfqpoint{8.414615in}{3.197841in}}%
\pgfpathcurveto{\pgfqpoint{8.411049in}{3.194275in}}{\pgfqpoint{8.409045in}{3.189437in}}{\pgfqpoint{8.409045in}{3.184393in}}%
\pgfpathcurveto{\pgfqpoint{8.409045in}{3.179350in}}{\pgfqpoint{8.411049in}{3.174512in}}{\pgfqpoint{8.414615in}{3.170946in}}%
\pgfpathcurveto{\pgfqpoint{8.418182in}{3.167379in}}{\pgfqpoint{8.423020in}{3.165375in}}{\pgfqpoint{8.428063in}{3.165375in}}%
\pgfpathclose%
\pgfusepath{fill}%
\end{pgfscope}%
\begin{pgfscope}%
\pgfpathrectangle{\pgfqpoint{6.572727in}{0.474100in}}{\pgfqpoint{4.227273in}{3.318700in}}%
\pgfusepath{clip}%
\pgfsetbuttcap%
\pgfsetroundjoin%
\definecolor{currentfill}{rgb}{0.127568,0.566949,0.550556}%
\pgfsetfillcolor{currentfill}%
\pgfsetfillopacity{0.700000}%
\pgfsetlinewidth{0.000000pt}%
\definecolor{currentstroke}{rgb}{0.000000,0.000000,0.000000}%
\pgfsetstrokecolor{currentstroke}%
\pgfsetstrokeopacity{0.700000}%
\pgfsetdash{}{0pt}%
\pgfpathmoveto{\pgfqpoint{9.258780in}{1.220225in}}%
\pgfpathcurveto{\pgfqpoint{9.263824in}{1.220225in}}{\pgfqpoint{9.268662in}{1.222229in}}{\pgfqpoint{9.272228in}{1.225795in}}%
\pgfpathcurveto{\pgfqpoint{9.275794in}{1.229362in}}{\pgfqpoint{9.277798in}{1.234199in}}{\pgfqpoint{9.277798in}{1.239243in}}%
\pgfpathcurveto{\pgfqpoint{9.277798in}{1.244287in}}{\pgfqpoint{9.275794in}{1.249125in}}{\pgfqpoint{9.272228in}{1.252691in}}%
\pgfpathcurveto{\pgfqpoint{9.268662in}{1.256257in}}{\pgfqpoint{9.263824in}{1.258261in}}{\pgfqpoint{9.258780in}{1.258261in}}%
\pgfpathcurveto{\pgfqpoint{9.253736in}{1.258261in}}{\pgfqpoint{9.248899in}{1.256257in}}{\pgfqpoint{9.245332in}{1.252691in}}%
\pgfpathcurveto{\pgfqpoint{9.241766in}{1.249125in}}{\pgfqpoint{9.239762in}{1.244287in}}{\pgfqpoint{9.239762in}{1.239243in}}%
\pgfpathcurveto{\pgfqpoint{9.239762in}{1.234199in}}{\pgfqpoint{9.241766in}{1.229362in}}{\pgfqpoint{9.245332in}{1.225795in}}%
\pgfpathcurveto{\pgfqpoint{9.248899in}{1.222229in}}{\pgfqpoint{9.253736in}{1.220225in}}{\pgfqpoint{9.258780in}{1.220225in}}%
\pgfpathclose%
\pgfusepath{fill}%
\end{pgfscope}%
\begin{pgfscope}%
\pgfpathrectangle{\pgfqpoint{6.572727in}{0.474100in}}{\pgfqpoint{4.227273in}{3.318700in}}%
\pgfusepath{clip}%
\pgfsetbuttcap%
\pgfsetroundjoin%
\definecolor{currentfill}{rgb}{0.993248,0.906157,0.143936}%
\pgfsetfillcolor{currentfill}%
\pgfsetfillopacity{0.700000}%
\pgfsetlinewidth{0.000000pt}%
\definecolor{currentstroke}{rgb}{0.000000,0.000000,0.000000}%
\pgfsetstrokecolor{currentstroke}%
\pgfsetstrokeopacity{0.700000}%
\pgfsetdash{}{0pt}%
\pgfpathmoveto{\pgfqpoint{7.915753in}{2.943696in}}%
\pgfpathcurveto{\pgfqpoint{7.920796in}{2.943696in}}{\pgfqpoint{7.925634in}{2.945700in}}{\pgfqpoint{7.929200in}{2.949267in}}%
\pgfpathcurveto{\pgfqpoint{7.932767in}{2.952833in}}{\pgfqpoint{7.934771in}{2.957671in}}{\pgfqpoint{7.934771in}{2.962714in}}%
\pgfpathcurveto{\pgfqpoint{7.934771in}{2.967758in}}{\pgfqpoint{7.932767in}{2.972596in}}{\pgfqpoint{7.929200in}{2.976162in}}%
\pgfpathcurveto{\pgfqpoint{7.925634in}{2.979729in}}{\pgfqpoint{7.920796in}{2.981733in}}{\pgfqpoint{7.915753in}{2.981733in}}%
\pgfpathcurveto{\pgfqpoint{7.910709in}{2.981733in}}{\pgfqpoint{7.905871in}{2.979729in}}{\pgfqpoint{7.902305in}{2.976162in}}%
\pgfpathcurveto{\pgfqpoint{7.898738in}{2.972596in}}{\pgfqpoint{7.896734in}{2.967758in}}{\pgfqpoint{7.896734in}{2.962714in}}%
\pgfpathcurveto{\pgfqpoint{7.896734in}{2.957671in}}{\pgfqpoint{7.898738in}{2.952833in}}{\pgfqpoint{7.902305in}{2.949267in}}%
\pgfpathcurveto{\pgfqpoint{7.905871in}{2.945700in}}{\pgfqpoint{7.910709in}{2.943696in}}{\pgfqpoint{7.915753in}{2.943696in}}%
\pgfpathclose%
\pgfusepath{fill}%
\end{pgfscope}%
\begin{pgfscope}%
\pgfpathrectangle{\pgfqpoint{6.572727in}{0.474100in}}{\pgfqpoint{4.227273in}{3.318700in}}%
\pgfusepath{clip}%
\pgfsetbuttcap%
\pgfsetroundjoin%
\definecolor{currentfill}{rgb}{0.127568,0.566949,0.550556}%
\pgfsetfillcolor{currentfill}%
\pgfsetfillopacity{0.700000}%
\pgfsetlinewidth{0.000000pt}%
\definecolor{currentstroke}{rgb}{0.000000,0.000000,0.000000}%
\pgfsetstrokecolor{currentstroke}%
\pgfsetstrokeopacity{0.700000}%
\pgfsetdash{}{0pt}%
\pgfpathmoveto{\pgfqpoint{9.496340in}{1.471711in}}%
\pgfpathcurveto{\pgfqpoint{9.501384in}{1.471711in}}{\pgfqpoint{9.506222in}{1.473715in}}{\pgfqpoint{9.509788in}{1.477281in}}%
\pgfpathcurveto{\pgfqpoint{9.513354in}{1.480848in}}{\pgfqpoint{9.515358in}{1.485686in}}{\pgfqpoint{9.515358in}{1.490729in}}%
\pgfpathcurveto{\pgfqpoint{9.515358in}{1.495773in}}{\pgfqpoint{9.513354in}{1.500611in}}{\pgfqpoint{9.509788in}{1.504177in}}%
\pgfpathcurveto{\pgfqpoint{9.506222in}{1.507744in}}{\pgfqpoint{9.501384in}{1.509747in}}{\pgfqpoint{9.496340in}{1.509747in}}%
\pgfpathcurveto{\pgfqpoint{9.491296in}{1.509747in}}{\pgfqpoint{9.486459in}{1.507744in}}{\pgfqpoint{9.482892in}{1.504177in}}%
\pgfpathcurveto{\pgfqpoint{9.479326in}{1.500611in}}{\pgfqpoint{9.477322in}{1.495773in}}{\pgfqpoint{9.477322in}{1.490729in}}%
\pgfpathcurveto{\pgfqpoint{9.477322in}{1.485686in}}{\pgfqpoint{9.479326in}{1.480848in}}{\pgfqpoint{9.482892in}{1.477281in}}%
\pgfpathcurveto{\pgfqpoint{9.486459in}{1.473715in}}{\pgfqpoint{9.491296in}{1.471711in}}{\pgfqpoint{9.496340in}{1.471711in}}%
\pgfpathclose%
\pgfusepath{fill}%
\end{pgfscope}%
\begin{pgfscope}%
\pgfpathrectangle{\pgfqpoint{6.572727in}{0.474100in}}{\pgfqpoint{4.227273in}{3.318700in}}%
\pgfusepath{clip}%
\pgfsetbuttcap%
\pgfsetroundjoin%
\definecolor{currentfill}{rgb}{0.127568,0.566949,0.550556}%
\pgfsetfillcolor{currentfill}%
\pgfsetfillopacity{0.700000}%
\pgfsetlinewidth{0.000000pt}%
\definecolor{currentstroke}{rgb}{0.000000,0.000000,0.000000}%
\pgfsetstrokecolor{currentstroke}%
\pgfsetstrokeopacity{0.700000}%
\pgfsetdash{}{0pt}%
\pgfpathmoveto{\pgfqpoint{10.133801in}{1.006054in}}%
\pgfpathcurveto{\pgfqpoint{10.138845in}{1.006054in}}{\pgfqpoint{10.143683in}{1.008058in}}{\pgfqpoint{10.147249in}{1.011624in}}%
\pgfpathcurveto{\pgfqpoint{10.150816in}{1.015191in}}{\pgfqpoint{10.152820in}{1.020028in}}{\pgfqpoint{10.152820in}{1.025072in}}%
\pgfpathcurveto{\pgfqpoint{10.152820in}{1.030116in}}{\pgfqpoint{10.150816in}{1.034954in}}{\pgfqpoint{10.147249in}{1.038520in}}%
\pgfpathcurveto{\pgfqpoint{10.143683in}{1.042086in}}{\pgfqpoint{10.138845in}{1.044090in}}{\pgfqpoint{10.133801in}{1.044090in}}%
\pgfpathcurveto{\pgfqpoint{10.128758in}{1.044090in}}{\pgfqpoint{10.123920in}{1.042086in}}{\pgfqpoint{10.120354in}{1.038520in}}%
\pgfpathcurveto{\pgfqpoint{10.116787in}{1.034954in}}{\pgfqpoint{10.114783in}{1.030116in}}{\pgfqpoint{10.114783in}{1.025072in}}%
\pgfpathcurveto{\pgfqpoint{10.114783in}{1.020028in}}{\pgfqpoint{10.116787in}{1.015191in}}{\pgfqpoint{10.120354in}{1.011624in}}%
\pgfpathcurveto{\pgfqpoint{10.123920in}{1.008058in}}{\pgfqpoint{10.128758in}{1.006054in}}{\pgfqpoint{10.133801in}{1.006054in}}%
\pgfpathclose%
\pgfusepath{fill}%
\end{pgfscope}%
\begin{pgfscope}%
\pgfpathrectangle{\pgfqpoint{6.572727in}{0.474100in}}{\pgfqpoint{4.227273in}{3.318700in}}%
\pgfusepath{clip}%
\pgfsetbuttcap%
\pgfsetroundjoin%
\definecolor{currentfill}{rgb}{0.127568,0.566949,0.550556}%
\pgfsetfillcolor{currentfill}%
\pgfsetfillopacity{0.700000}%
\pgfsetlinewidth{0.000000pt}%
\definecolor{currentstroke}{rgb}{0.000000,0.000000,0.000000}%
\pgfsetstrokecolor{currentstroke}%
\pgfsetstrokeopacity{0.700000}%
\pgfsetdash{}{0pt}%
\pgfpathmoveto{\pgfqpoint{9.791706in}{1.650366in}}%
\pgfpathcurveto{\pgfqpoint{9.796749in}{1.650366in}}{\pgfqpoint{9.801587in}{1.652370in}}{\pgfqpoint{9.805153in}{1.655937in}}%
\pgfpathcurveto{\pgfqpoint{9.808720in}{1.659503in}}{\pgfqpoint{9.810724in}{1.664341in}}{\pgfqpoint{9.810724in}{1.669384in}}%
\pgfpathcurveto{\pgfqpoint{9.810724in}{1.674428in}}{\pgfqpoint{9.808720in}{1.679266in}}{\pgfqpoint{9.805153in}{1.682832in}}%
\pgfpathcurveto{\pgfqpoint{9.801587in}{1.686399in}}{\pgfqpoint{9.796749in}{1.688403in}}{\pgfqpoint{9.791706in}{1.688403in}}%
\pgfpathcurveto{\pgfqpoint{9.786662in}{1.688403in}}{\pgfqpoint{9.781824in}{1.686399in}}{\pgfqpoint{9.778258in}{1.682832in}}%
\pgfpathcurveto{\pgfqpoint{9.774691in}{1.679266in}}{\pgfqpoint{9.772687in}{1.674428in}}{\pgfqpoint{9.772687in}{1.669384in}}%
\pgfpathcurveto{\pgfqpoint{9.772687in}{1.664341in}}{\pgfqpoint{9.774691in}{1.659503in}}{\pgfqpoint{9.778258in}{1.655937in}}%
\pgfpathcurveto{\pgfqpoint{9.781824in}{1.652370in}}{\pgfqpoint{9.786662in}{1.650366in}}{\pgfqpoint{9.791706in}{1.650366in}}%
\pgfpathclose%
\pgfusepath{fill}%
\end{pgfscope}%
\begin{pgfscope}%
\pgfpathrectangle{\pgfqpoint{6.572727in}{0.474100in}}{\pgfqpoint{4.227273in}{3.318700in}}%
\pgfusepath{clip}%
\pgfsetbuttcap%
\pgfsetroundjoin%
\definecolor{currentfill}{rgb}{0.993248,0.906157,0.143936}%
\pgfsetfillcolor{currentfill}%
\pgfsetfillopacity{0.700000}%
\pgfsetlinewidth{0.000000pt}%
\definecolor{currentstroke}{rgb}{0.000000,0.000000,0.000000}%
\pgfsetstrokecolor{currentstroke}%
\pgfsetstrokeopacity{0.700000}%
\pgfsetdash{}{0pt}%
\pgfpathmoveto{\pgfqpoint{8.327714in}{2.964758in}}%
\pgfpathcurveto{\pgfqpoint{8.332758in}{2.964758in}}{\pgfqpoint{8.337595in}{2.966762in}}{\pgfqpoint{8.341162in}{2.970329in}}%
\pgfpathcurveto{\pgfqpoint{8.344728in}{2.973895in}}{\pgfqpoint{8.346732in}{2.978733in}}{\pgfqpoint{8.346732in}{2.983777in}}%
\pgfpathcurveto{\pgfqpoint{8.346732in}{2.988820in}}{\pgfqpoint{8.344728in}{2.993658in}}{\pgfqpoint{8.341162in}{2.997224in}}%
\pgfpathcurveto{\pgfqpoint{8.337595in}{3.000791in}}{\pgfqpoint{8.332758in}{3.002795in}}{\pgfqpoint{8.327714in}{3.002795in}}%
\pgfpathcurveto{\pgfqpoint{8.322670in}{3.002795in}}{\pgfqpoint{8.317833in}{3.000791in}}{\pgfqpoint{8.314266in}{2.997224in}}%
\pgfpathcurveto{\pgfqpoint{8.310700in}{2.993658in}}{\pgfqpoint{8.308696in}{2.988820in}}{\pgfqpoint{8.308696in}{2.983777in}}%
\pgfpathcurveto{\pgfqpoint{8.308696in}{2.978733in}}{\pgfqpoint{8.310700in}{2.973895in}}{\pgfqpoint{8.314266in}{2.970329in}}%
\pgfpathcurveto{\pgfqpoint{8.317833in}{2.966762in}}{\pgfqpoint{8.322670in}{2.964758in}}{\pgfqpoint{8.327714in}{2.964758in}}%
\pgfpathclose%
\pgfusepath{fill}%
\end{pgfscope}%
\begin{pgfscope}%
\pgfpathrectangle{\pgfqpoint{6.572727in}{0.474100in}}{\pgfqpoint{4.227273in}{3.318700in}}%
\pgfusepath{clip}%
\pgfsetbuttcap%
\pgfsetroundjoin%
\definecolor{currentfill}{rgb}{0.127568,0.566949,0.550556}%
\pgfsetfillcolor{currentfill}%
\pgfsetfillopacity{0.700000}%
\pgfsetlinewidth{0.000000pt}%
\definecolor{currentstroke}{rgb}{0.000000,0.000000,0.000000}%
\pgfsetstrokecolor{currentstroke}%
\pgfsetstrokeopacity{0.700000}%
\pgfsetdash{}{0pt}%
\pgfpathmoveto{\pgfqpoint{9.386556in}{1.606792in}}%
\pgfpathcurveto{\pgfqpoint{9.391600in}{1.606792in}}{\pgfqpoint{9.396438in}{1.608796in}}{\pgfqpoint{9.400004in}{1.612362in}}%
\pgfpathcurveto{\pgfqpoint{9.403570in}{1.615929in}}{\pgfqpoint{9.405574in}{1.620767in}}{\pgfqpoint{9.405574in}{1.625810in}}%
\pgfpathcurveto{\pgfqpoint{9.405574in}{1.630854in}}{\pgfqpoint{9.403570in}{1.635692in}}{\pgfqpoint{9.400004in}{1.639258in}}%
\pgfpathcurveto{\pgfqpoint{9.396438in}{1.642825in}}{\pgfqpoint{9.391600in}{1.644828in}}{\pgfqpoint{9.386556in}{1.644828in}}%
\pgfpathcurveto{\pgfqpoint{9.381513in}{1.644828in}}{\pgfqpoint{9.376675in}{1.642825in}}{\pgfqpoint{9.373108in}{1.639258in}}%
\pgfpathcurveto{\pgfqpoint{9.369542in}{1.635692in}}{\pgfqpoint{9.367538in}{1.630854in}}{\pgfqpoint{9.367538in}{1.625810in}}%
\pgfpathcurveto{\pgfqpoint{9.367538in}{1.620767in}}{\pgfqpoint{9.369542in}{1.615929in}}{\pgfqpoint{9.373108in}{1.612362in}}%
\pgfpathcurveto{\pgfqpoint{9.376675in}{1.608796in}}{\pgfqpoint{9.381513in}{1.606792in}}{\pgfqpoint{9.386556in}{1.606792in}}%
\pgfpathclose%
\pgfusepath{fill}%
\end{pgfscope}%
\begin{pgfscope}%
\pgfpathrectangle{\pgfqpoint{6.572727in}{0.474100in}}{\pgfqpoint{4.227273in}{3.318700in}}%
\pgfusepath{clip}%
\pgfsetbuttcap%
\pgfsetroundjoin%
\definecolor{currentfill}{rgb}{0.267004,0.004874,0.329415}%
\pgfsetfillcolor{currentfill}%
\pgfsetfillopacity{0.700000}%
\pgfsetlinewidth{0.000000pt}%
\definecolor{currentstroke}{rgb}{0.000000,0.000000,0.000000}%
\pgfsetstrokecolor{currentstroke}%
\pgfsetstrokeopacity{0.700000}%
\pgfsetdash{}{0pt}%
\pgfpathmoveto{\pgfqpoint{7.638030in}{1.288159in}}%
\pgfpathcurveto{\pgfqpoint{7.643074in}{1.288159in}}{\pgfqpoint{7.647912in}{1.290163in}}{\pgfqpoint{7.651478in}{1.293729in}}%
\pgfpathcurveto{\pgfqpoint{7.655045in}{1.297296in}}{\pgfqpoint{7.657049in}{1.302133in}}{\pgfqpoint{7.657049in}{1.307177in}}%
\pgfpathcurveto{\pgfqpoint{7.657049in}{1.312221in}}{\pgfqpoint{7.655045in}{1.317058in}}{\pgfqpoint{7.651478in}{1.320625in}}%
\pgfpathcurveto{\pgfqpoint{7.647912in}{1.324191in}}{\pgfqpoint{7.643074in}{1.326195in}}{\pgfqpoint{7.638030in}{1.326195in}}%
\pgfpathcurveto{\pgfqpoint{7.632987in}{1.326195in}}{\pgfqpoint{7.628149in}{1.324191in}}{\pgfqpoint{7.624583in}{1.320625in}}%
\pgfpathcurveto{\pgfqpoint{7.621016in}{1.317058in}}{\pgfqpoint{7.619012in}{1.312221in}}{\pgfqpoint{7.619012in}{1.307177in}}%
\pgfpathcurveto{\pgfqpoint{7.619012in}{1.302133in}}{\pgfqpoint{7.621016in}{1.297296in}}{\pgfqpoint{7.624583in}{1.293729in}}%
\pgfpathcurveto{\pgfqpoint{7.628149in}{1.290163in}}{\pgfqpoint{7.632987in}{1.288159in}}{\pgfqpoint{7.638030in}{1.288159in}}%
\pgfpathclose%
\pgfusepath{fill}%
\end{pgfscope}%
\begin{pgfscope}%
\pgfpathrectangle{\pgfqpoint{6.572727in}{0.474100in}}{\pgfqpoint{4.227273in}{3.318700in}}%
\pgfusepath{clip}%
\pgfsetbuttcap%
\pgfsetroundjoin%
\definecolor{currentfill}{rgb}{0.993248,0.906157,0.143936}%
\pgfsetfillcolor{currentfill}%
\pgfsetfillopacity{0.700000}%
\pgfsetlinewidth{0.000000pt}%
\definecolor{currentstroke}{rgb}{0.000000,0.000000,0.000000}%
\pgfsetstrokecolor{currentstroke}%
\pgfsetstrokeopacity{0.700000}%
\pgfsetdash{}{0pt}%
\pgfpathmoveto{\pgfqpoint{7.604546in}{2.376934in}}%
\pgfpathcurveto{\pgfqpoint{7.609590in}{2.376934in}}{\pgfqpoint{7.614427in}{2.378938in}}{\pgfqpoint{7.617994in}{2.382504in}}%
\pgfpathcurveto{\pgfqpoint{7.621560in}{2.386070in}}{\pgfqpoint{7.623564in}{2.390908in}}{\pgfqpoint{7.623564in}{2.395952in}}%
\pgfpathcurveto{\pgfqpoint{7.623564in}{2.400995in}}{\pgfqpoint{7.621560in}{2.405833in}}{\pgfqpoint{7.617994in}{2.409400in}}%
\pgfpathcurveto{\pgfqpoint{7.614427in}{2.412966in}}{\pgfqpoint{7.609590in}{2.414970in}}{\pgfqpoint{7.604546in}{2.414970in}}%
\pgfpathcurveto{\pgfqpoint{7.599502in}{2.414970in}}{\pgfqpoint{7.594665in}{2.412966in}}{\pgfqpoint{7.591098in}{2.409400in}}%
\pgfpathcurveto{\pgfqpoint{7.587532in}{2.405833in}}{\pgfqpoint{7.585528in}{2.400995in}}{\pgfqpoint{7.585528in}{2.395952in}}%
\pgfpathcurveto{\pgfqpoint{7.585528in}{2.390908in}}{\pgfqpoint{7.587532in}{2.386070in}}{\pgfqpoint{7.591098in}{2.382504in}}%
\pgfpathcurveto{\pgfqpoint{7.594665in}{2.378938in}}{\pgfqpoint{7.599502in}{2.376934in}}{\pgfqpoint{7.604546in}{2.376934in}}%
\pgfpathclose%
\pgfusepath{fill}%
\end{pgfscope}%
\begin{pgfscope}%
\pgfpathrectangle{\pgfqpoint{6.572727in}{0.474100in}}{\pgfqpoint{4.227273in}{3.318700in}}%
\pgfusepath{clip}%
\pgfsetbuttcap%
\pgfsetroundjoin%
\definecolor{currentfill}{rgb}{0.993248,0.906157,0.143936}%
\pgfsetfillcolor{currentfill}%
\pgfsetfillopacity{0.700000}%
\pgfsetlinewidth{0.000000pt}%
\definecolor{currentstroke}{rgb}{0.000000,0.000000,0.000000}%
\pgfsetstrokecolor{currentstroke}%
\pgfsetstrokeopacity{0.700000}%
\pgfsetdash{}{0pt}%
\pgfpathmoveto{\pgfqpoint{8.385452in}{2.802472in}}%
\pgfpathcurveto{\pgfqpoint{8.390496in}{2.802472in}}{\pgfqpoint{8.395334in}{2.804476in}}{\pgfqpoint{8.398900in}{2.808043in}}%
\pgfpathcurveto{\pgfqpoint{8.402467in}{2.811609in}}{\pgfqpoint{8.404470in}{2.816447in}}{\pgfqpoint{8.404470in}{2.821490in}}%
\pgfpathcurveto{\pgfqpoint{8.404470in}{2.826534in}}{\pgfqpoint{8.402467in}{2.831372in}}{\pgfqpoint{8.398900in}{2.834938in}}%
\pgfpathcurveto{\pgfqpoint{8.395334in}{2.838505in}}{\pgfqpoint{8.390496in}{2.840509in}}{\pgfqpoint{8.385452in}{2.840509in}}%
\pgfpathcurveto{\pgfqpoint{8.380409in}{2.840509in}}{\pgfqpoint{8.375571in}{2.838505in}}{\pgfqpoint{8.372004in}{2.834938in}}%
\pgfpathcurveto{\pgfqpoint{8.368438in}{2.831372in}}{\pgfqpoint{8.366434in}{2.826534in}}{\pgfqpoint{8.366434in}{2.821490in}}%
\pgfpathcurveto{\pgfqpoint{8.366434in}{2.816447in}}{\pgfqpoint{8.368438in}{2.811609in}}{\pgfqpoint{8.372004in}{2.808043in}}%
\pgfpathcurveto{\pgfqpoint{8.375571in}{2.804476in}}{\pgfqpoint{8.380409in}{2.802472in}}{\pgfqpoint{8.385452in}{2.802472in}}%
\pgfpathclose%
\pgfusepath{fill}%
\end{pgfscope}%
\begin{pgfscope}%
\pgfpathrectangle{\pgfqpoint{6.572727in}{0.474100in}}{\pgfqpoint{4.227273in}{3.318700in}}%
\pgfusepath{clip}%
\pgfsetbuttcap%
\pgfsetroundjoin%
\definecolor{currentfill}{rgb}{0.993248,0.906157,0.143936}%
\pgfsetfillcolor{currentfill}%
\pgfsetfillopacity{0.700000}%
\pgfsetlinewidth{0.000000pt}%
\definecolor{currentstroke}{rgb}{0.000000,0.000000,0.000000}%
\pgfsetstrokecolor{currentstroke}%
\pgfsetstrokeopacity{0.700000}%
\pgfsetdash{}{0pt}%
\pgfpathmoveto{\pgfqpoint{8.414996in}{2.907545in}}%
\pgfpathcurveto{\pgfqpoint{8.420040in}{2.907545in}}{\pgfqpoint{8.424878in}{2.909549in}}{\pgfqpoint{8.428444in}{2.913115in}}%
\pgfpathcurveto{\pgfqpoint{8.432010in}{2.916682in}}{\pgfqpoint{8.434014in}{2.921520in}}{\pgfqpoint{8.434014in}{2.926563in}}%
\pgfpathcurveto{\pgfqpoint{8.434014in}{2.931607in}}{\pgfqpoint{8.432010in}{2.936445in}}{\pgfqpoint{8.428444in}{2.940011in}}%
\pgfpathcurveto{\pgfqpoint{8.424878in}{2.943577in}}{\pgfqpoint{8.420040in}{2.945581in}}{\pgfqpoint{8.414996in}{2.945581in}}%
\pgfpathcurveto{\pgfqpoint{8.409952in}{2.945581in}}{\pgfqpoint{8.405115in}{2.943577in}}{\pgfqpoint{8.401548in}{2.940011in}}%
\pgfpathcurveto{\pgfqpoint{8.397982in}{2.936445in}}{\pgfqpoint{8.395978in}{2.931607in}}{\pgfqpoint{8.395978in}{2.926563in}}%
\pgfpathcurveto{\pgfqpoint{8.395978in}{2.921520in}}{\pgfqpoint{8.397982in}{2.916682in}}{\pgfqpoint{8.401548in}{2.913115in}}%
\pgfpathcurveto{\pgfqpoint{8.405115in}{2.909549in}}{\pgfqpoint{8.409952in}{2.907545in}}{\pgfqpoint{8.414996in}{2.907545in}}%
\pgfpathclose%
\pgfusepath{fill}%
\end{pgfscope}%
\begin{pgfscope}%
\pgfpathrectangle{\pgfqpoint{6.572727in}{0.474100in}}{\pgfqpoint{4.227273in}{3.318700in}}%
\pgfusepath{clip}%
\pgfsetbuttcap%
\pgfsetroundjoin%
\definecolor{currentfill}{rgb}{0.127568,0.566949,0.550556}%
\pgfsetfillcolor{currentfill}%
\pgfsetfillopacity{0.700000}%
\pgfsetlinewidth{0.000000pt}%
\definecolor{currentstroke}{rgb}{0.000000,0.000000,0.000000}%
\pgfsetstrokecolor{currentstroke}%
\pgfsetstrokeopacity{0.700000}%
\pgfsetdash{}{0pt}%
\pgfpathmoveto{\pgfqpoint{9.486798in}{1.234783in}}%
\pgfpathcurveto{\pgfqpoint{9.491841in}{1.234783in}}{\pgfqpoint{9.496679in}{1.236787in}}{\pgfqpoint{9.500246in}{1.240353in}}%
\pgfpathcurveto{\pgfqpoint{9.503812in}{1.243920in}}{\pgfqpoint{9.505816in}{1.248757in}}{\pgfqpoint{9.505816in}{1.253801in}}%
\pgfpathcurveto{\pgfqpoint{9.505816in}{1.258845in}}{\pgfqpoint{9.503812in}{1.263682in}}{\pgfqpoint{9.500246in}{1.267249in}}%
\pgfpathcurveto{\pgfqpoint{9.496679in}{1.270815in}}{\pgfqpoint{9.491841in}{1.272819in}}{\pgfqpoint{9.486798in}{1.272819in}}%
\pgfpathcurveto{\pgfqpoint{9.481754in}{1.272819in}}{\pgfqpoint{9.476916in}{1.270815in}}{\pgfqpoint{9.473350in}{1.267249in}}%
\pgfpathcurveto{\pgfqpoint{9.469783in}{1.263682in}}{\pgfqpoint{9.467780in}{1.258845in}}{\pgfqpoint{9.467780in}{1.253801in}}%
\pgfpathcurveto{\pgfqpoint{9.467780in}{1.248757in}}{\pgfqpoint{9.469783in}{1.243920in}}{\pgfqpoint{9.473350in}{1.240353in}}%
\pgfpathcurveto{\pgfqpoint{9.476916in}{1.236787in}}{\pgfqpoint{9.481754in}{1.234783in}}{\pgfqpoint{9.486798in}{1.234783in}}%
\pgfpathclose%
\pgfusepath{fill}%
\end{pgfscope}%
\begin{pgfscope}%
\pgfpathrectangle{\pgfqpoint{6.572727in}{0.474100in}}{\pgfqpoint{4.227273in}{3.318700in}}%
\pgfusepath{clip}%
\pgfsetbuttcap%
\pgfsetroundjoin%
\definecolor{currentfill}{rgb}{0.267004,0.004874,0.329415}%
\pgfsetfillcolor{currentfill}%
\pgfsetfillopacity{0.700000}%
\pgfsetlinewidth{0.000000pt}%
\definecolor{currentstroke}{rgb}{0.000000,0.000000,0.000000}%
\pgfsetstrokecolor{currentstroke}%
\pgfsetstrokeopacity{0.700000}%
\pgfsetdash{}{0pt}%
\pgfpathmoveto{\pgfqpoint{7.913414in}{1.778866in}}%
\pgfpathcurveto{\pgfqpoint{7.918458in}{1.778866in}}{\pgfqpoint{7.923295in}{1.780870in}}{\pgfqpoint{7.926862in}{1.784437in}}%
\pgfpathcurveto{\pgfqpoint{7.930428in}{1.788003in}}{\pgfqpoint{7.932432in}{1.792841in}}{\pgfqpoint{7.932432in}{1.797884in}}%
\pgfpathcurveto{\pgfqpoint{7.932432in}{1.802928in}}{\pgfqpoint{7.930428in}{1.807766in}}{\pgfqpoint{7.926862in}{1.811332in}}%
\pgfpathcurveto{\pgfqpoint{7.923295in}{1.814899in}}{\pgfqpoint{7.918458in}{1.816903in}}{\pgfqpoint{7.913414in}{1.816903in}}%
\pgfpathcurveto{\pgfqpoint{7.908370in}{1.816903in}}{\pgfqpoint{7.903532in}{1.814899in}}{\pgfqpoint{7.899966in}{1.811332in}}%
\pgfpathcurveto{\pgfqpoint{7.896400in}{1.807766in}}{\pgfqpoint{7.894396in}{1.802928in}}{\pgfqpoint{7.894396in}{1.797884in}}%
\pgfpathcurveto{\pgfqpoint{7.894396in}{1.792841in}}{\pgfqpoint{7.896400in}{1.788003in}}{\pgfqpoint{7.899966in}{1.784437in}}%
\pgfpathcurveto{\pgfqpoint{7.903532in}{1.780870in}}{\pgfqpoint{7.908370in}{1.778866in}}{\pgfqpoint{7.913414in}{1.778866in}}%
\pgfpathclose%
\pgfusepath{fill}%
\end{pgfscope}%
\begin{pgfscope}%
\pgfpathrectangle{\pgfqpoint{6.572727in}{0.474100in}}{\pgfqpoint{4.227273in}{3.318700in}}%
\pgfusepath{clip}%
\pgfsetbuttcap%
\pgfsetroundjoin%
\definecolor{currentfill}{rgb}{0.267004,0.004874,0.329415}%
\pgfsetfillcolor{currentfill}%
\pgfsetfillopacity{0.700000}%
\pgfsetlinewidth{0.000000pt}%
\definecolor{currentstroke}{rgb}{0.000000,0.000000,0.000000}%
\pgfsetstrokecolor{currentstroke}%
\pgfsetstrokeopacity{0.700000}%
\pgfsetdash{}{0pt}%
\pgfpathmoveto{\pgfqpoint{7.821927in}{2.085040in}}%
\pgfpathcurveto{\pgfqpoint{7.826971in}{2.085040in}}{\pgfqpoint{7.831808in}{2.087044in}}{\pgfqpoint{7.835375in}{2.090610in}}%
\pgfpathcurveto{\pgfqpoint{7.838941in}{2.094176in}}{\pgfqpoint{7.840945in}{2.099014in}}{\pgfqpoint{7.840945in}{2.104058in}}%
\pgfpathcurveto{\pgfqpoint{7.840945in}{2.109101in}}{\pgfqpoint{7.838941in}{2.113939in}}{\pgfqpoint{7.835375in}{2.117506in}}%
\pgfpathcurveto{\pgfqpoint{7.831808in}{2.121072in}}{\pgfqpoint{7.826971in}{2.123076in}}{\pgfqpoint{7.821927in}{2.123076in}}%
\pgfpathcurveto{\pgfqpoint{7.816883in}{2.123076in}}{\pgfqpoint{7.812045in}{2.121072in}}{\pgfqpoint{7.808479in}{2.117506in}}%
\pgfpathcurveto{\pgfqpoint{7.804913in}{2.113939in}}{\pgfqpoint{7.802909in}{2.109101in}}{\pgfqpoint{7.802909in}{2.104058in}}%
\pgfpathcurveto{\pgfqpoint{7.802909in}{2.099014in}}{\pgfqpoint{7.804913in}{2.094176in}}{\pgfqpoint{7.808479in}{2.090610in}}%
\pgfpathcurveto{\pgfqpoint{7.812045in}{2.087044in}}{\pgfqpoint{7.816883in}{2.085040in}}{\pgfqpoint{7.821927in}{2.085040in}}%
\pgfpathclose%
\pgfusepath{fill}%
\end{pgfscope}%
\begin{pgfscope}%
\pgfpathrectangle{\pgfqpoint{6.572727in}{0.474100in}}{\pgfqpoint{4.227273in}{3.318700in}}%
\pgfusepath{clip}%
\pgfsetbuttcap%
\pgfsetroundjoin%
\definecolor{currentfill}{rgb}{0.993248,0.906157,0.143936}%
\pgfsetfillcolor{currentfill}%
\pgfsetfillopacity{0.700000}%
\pgfsetlinewidth{0.000000pt}%
\definecolor{currentstroke}{rgb}{0.000000,0.000000,0.000000}%
\pgfsetstrokecolor{currentstroke}%
\pgfsetstrokeopacity{0.700000}%
\pgfsetdash{}{0pt}%
\pgfpathmoveto{\pgfqpoint{8.164135in}{2.450441in}}%
\pgfpathcurveto{\pgfqpoint{8.169179in}{2.450441in}}{\pgfqpoint{8.174016in}{2.452445in}}{\pgfqpoint{8.177583in}{2.456011in}}%
\pgfpathcurveto{\pgfqpoint{8.181149in}{2.459578in}}{\pgfqpoint{8.183153in}{2.464415in}}{\pgfqpoint{8.183153in}{2.469459in}}%
\pgfpathcurveto{\pgfqpoint{8.183153in}{2.474503in}}{\pgfqpoint{8.181149in}{2.479340in}}{\pgfqpoint{8.177583in}{2.482907in}}%
\pgfpathcurveto{\pgfqpoint{8.174016in}{2.486473in}}{\pgfqpoint{8.169179in}{2.488477in}}{\pgfqpoint{8.164135in}{2.488477in}}%
\pgfpathcurveto{\pgfqpoint{8.159091in}{2.488477in}}{\pgfqpoint{8.154253in}{2.486473in}}{\pgfqpoint{8.150687in}{2.482907in}}%
\pgfpathcurveto{\pgfqpoint{8.147121in}{2.479340in}}{\pgfqpoint{8.145117in}{2.474503in}}{\pgfqpoint{8.145117in}{2.469459in}}%
\pgfpathcurveto{\pgfqpoint{8.145117in}{2.464415in}}{\pgfqpoint{8.147121in}{2.459578in}}{\pgfqpoint{8.150687in}{2.456011in}}%
\pgfpathcurveto{\pgfqpoint{8.154253in}{2.452445in}}{\pgfqpoint{8.159091in}{2.450441in}}{\pgfqpoint{8.164135in}{2.450441in}}%
\pgfpathclose%
\pgfusepath{fill}%
\end{pgfscope}%
\begin{pgfscope}%
\pgfpathrectangle{\pgfqpoint{6.572727in}{0.474100in}}{\pgfqpoint{4.227273in}{3.318700in}}%
\pgfusepath{clip}%
\pgfsetbuttcap%
\pgfsetroundjoin%
\definecolor{currentfill}{rgb}{0.993248,0.906157,0.143936}%
\pgfsetfillcolor{currentfill}%
\pgfsetfillopacity{0.700000}%
\pgfsetlinewidth{0.000000pt}%
\definecolor{currentstroke}{rgb}{0.000000,0.000000,0.000000}%
\pgfsetstrokecolor{currentstroke}%
\pgfsetstrokeopacity{0.700000}%
\pgfsetdash{}{0pt}%
\pgfpathmoveto{\pgfqpoint{7.910866in}{2.671667in}}%
\pgfpathcurveto{\pgfqpoint{7.915910in}{2.671667in}}{\pgfqpoint{7.920748in}{2.673671in}}{\pgfqpoint{7.924314in}{2.677237in}}%
\pgfpathcurveto{\pgfqpoint{7.927881in}{2.680804in}}{\pgfqpoint{7.929885in}{2.685641in}}{\pgfqpoint{7.929885in}{2.690685in}}%
\pgfpathcurveto{\pgfqpoint{7.929885in}{2.695729in}}{\pgfqpoint{7.927881in}{2.700566in}}{\pgfqpoint{7.924314in}{2.704133in}}%
\pgfpathcurveto{\pgfqpoint{7.920748in}{2.707699in}}{\pgfqpoint{7.915910in}{2.709703in}}{\pgfqpoint{7.910866in}{2.709703in}}%
\pgfpathcurveto{\pgfqpoint{7.905823in}{2.709703in}}{\pgfqpoint{7.900985in}{2.707699in}}{\pgfqpoint{7.897419in}{2.704133in}}%
\pgfpathcurveto{\pgfqpoint{7.893852in}{2.700566in}}{\pgfqpoint{7.891848in}{2.695729in}}{\pgfqpoint{7.891848in}{2.690685in}}%
\pgfpathcurveto{\pgfqpoint{7.891848in}{2.685641in}}{\pgfqpoint{7.893852in}{2.680804in}}{\pgfqpoint{7.897419in}{2.677237in}}%
\pgfpathcurveto{\pgfqpoint{7.900985in}{2.673671in}}{\pgfqpoint{7.905823in}{2.671667in}}{\pgfqpoint{7.910866in}{2.671667in}}%
\pgfpathclose%
\pgfusepath{fill}%
\end{pgfscope}%
\begin{pgfscope}%
\pgfpathrectangle{\pgfqpoint{6.572727in}{0.474100in}}{\pgfqpoint{4.227273in}{3.318700in}}%
\pgfusepath{clip}%
\pgfsetbuttcap%
\pgfsetroundjoin%
\definecolor{currentfill}{rgb}{0.267004,0.004874,0.329415}%
\pgfsetfillcolor{currentfill}%
\pgfsetfillopacity{0.700000}%
\pgfsetlinewidth{0.000000pt}%
\definecolor{currentstroke}{rgb}{0.000000,0.000000,0.000000}%
\pgfsetstrokecolor{currentstroke}%
\pgfsetstrokeopacity{0.700000}%
\pgfsetdash{}{0pt}%
\pgfpathmoveto{\pgfqpoint{7.957590in}{2.138552in}}%
\pgfpathcurveto{\pgfqpoint{7.962633in}{2.138552in}}{\pgfqpoint{7.967471in}{2.140556in}}{\pgfqpoint{7.971037in}{2.144122in}}%
\pgfpathcurveto{\pgfqpoint{7.974604in}{2.147689in}}{\pgfqpoint{7.976608in}{2.152527in}}{\pgfqpoint{7.976608in}{2.157570in}}%
\pgfpathcurveto{\pgfqpoint{7.976608in}{2.162614in}}{\pgfqpoint{7.974604in}{2.167452in}}{\pgfqpoint{7.971037in}{2.171018in}}%
\pgfpathcurveto{\pgfqpoint{7.967471in}{2.174585in}}{\pgfqpoint{7.962633in}{2.176588in}}{\pgfqpoint{7.957590in}{2.176588in}}%
\pgfpathcurveto{\pgfqpoint{7.952546in}{2.176588in}}{\pgfqpoint{7.947708in}{2.174585in}}{\pgfqpoint{7.944142in}{2.171018in}}%
\pgfpathcurveto{\pgfqpoint{7.940575in}{2.167452in}}{\pgfqpoint{7.938571in}{2.162614in}}{\pgfqpoint{7.938571in}{2.157570in}}%
\pgfpathcurveto{\pgfqpoint{7.938571in}{2.152527in}}{\pgfqpoint{7.940575in}{2.147689in}}{\pgfqpoint{7.944142in}{2.144122in}}%
\pgfpathcurveto{\pgfqpoint{7.947708in}{2.140556in}}{\pgfqpoint{7.952546in}{2.138552in}}{\pgfqpoint{7.957590in}{2.138552in}}%
\pgfpathclose%
\pgfusepath{fill}%
\end{pgfscope}%
\begin{pgfscope}%
\pgfpathrectangle{\pgfqpoint{6.572727in}{0.474100in}}{\pgfqpoint{4.227273in}{3.318700in}}%
\pgfusepath{clip}%
\pgfsetbuttcap%
\pgfsetroundjoin%
\definecolor{currentfill}{rgb}{0.127568,0.566949,0.550556}%
\pgfsetfillcolor{currentfill}%
\pgfsetfillopacity{0.700000}%
\pgfsetlinewidth{0.000000pt}%
\definecolor{currentstroke}{rgb}{0.000000,0.000000,0.000000}%
\pgfsetstrokecolor{currentstroke}%
\pgfsetstrokeopacity{0.700000}%
\pgfsetdash{}{0pt}%
\pgfpathmoveto{\pgfqpoint{10.137840in}{1.609582in}}%
\pgfpathcurveto{\pgfqpoint{10.142883in}{1.609582in}}{\pgfqpoint{10.147721in}{1.611586in}}{\pgfqpoint{10.151288in}{1.615152in}}%
\pgfpathcurveto{\pgfqpoint{10.154854in}{1.618718in}}{\pgfqpoint{10.156858in}{1.623556in}}{\pgfqpoint{10.156858in}{1.628600in}}%
\pgfpathcurveto{\pgfqpoint{10.156858in}{1.633644in}}{\pgfqpoint{10.154854in}{1.638481in}}{\pgfqpoint{10.151288in}{1.642048in}}%
\pgfpathcurveto{\pgfqpoint{10.147721in}{1.645614in}}{\pgfqpoint{10.142883in}{1.647618in}}{\pgfqpoint{10.137840in}{1.647618in}}%
\pgfpathcurveto{\pgfqpoint{10.132796in}{1.647618in}}{\pgfqpoint{10.127958in}{1.645614in}}{\pgfqpoint{10.124392in}{1.642048in}}%
\pgfpathcurveto{\pgfqpoint{10.120825in}{1.638481in}}{\pgfqpoint{10.118822in}{1.633644in}}{\pgfqpoint{10.118822in}{1.628600in}}%
\pgfpathcurveto{\pgfqpoint{10.118822in}{1.623556in}}{\pgfqpoint{10.120825in}{1.618718in}}{\pgfqpoint{10.124392in}{1.615152in}}%
\pgfpathcurveto{\pgfqpoint{10.127958in}{1.611586in}}{\pgfqpoint{10.132796in}{1.609582in}}{\pgfqpoint{10.137840in}{1.609582in}}%
\pgfpathclose%
\pgfusepath{fill}%
\end{pgfscope}%
\begin{pgfscope}%
\pgfpathrectangle{\pgfqpoint{6.572727in}{0.474100in}}{\pgfqpoint{4.227273in}{3.318700in}}%
\pgfusepath{clip}%
\pgfsetbuttcap%
\pgfsetroundjoin%
\definecolor{currentfill}{rgb}{0.267004,0.004874,0.329415}%
\pgfsetfillcolor{currentfill}%
\pgfsetfillopacity{0.700000}%
\pgfsetlinewidth{0.000000pt}%
\definecolor{currentstroke}{rgb}{0.000000,0.000000,0.000000}%
\pgfsetstrokecolor{currentstroke}%
\pgfsetstrokeopacity{0.700000}%
\pgfsetdash{}{0pt}%
\pgfpathmoveto{\pgfqpoint{7.864466in}{1.589054in}}%
\pgfpathcurveto{\pgfqpoint{7.869509in}{1.589054in}}{\pgfqpoint{7.874347in}{1.591058in}}{\pgfqpoint{7.877913in}{1.594625in}}%
\pgfpathcurveto{\pgfqpoint{7.881480in}{1.598191in}}{\pgfqpoint{7.883484in}{1.603029in}}{\pgfqpoint{7.883484in}{1.608073in}}%
\pgfpathcurveto{\pgfqpoint{7.883484in}{1.613116in}}{\pgfqpoint{7.881480in}{1.617954in}}{\pgfqpoint{7.877913in}{1.621520in}}%
\pgfpathcurveto{\pgfqpoint{7.874347in}{1.625087in}}{\pgfqpoint{7.869509in}{1.627091in}}{\pgfqpoint{7.864466in}{1.627091in}}%
\pgfpathcurveto{\pgfqpoint{7.859422in}{1.627091in}}{\pgfqpoint{7.854584in}{1.625087in}}{\pgfqpoint{7.851018in}{1.621520in}}%
\pgfpathcurveto{\pgfqpoint{7.847451in}{1.617954in}}{\pgfqpoint{7.845447in}{1.613116in}}{\pgfqpoint{7.845447in}{1.608073in}}%
\pgfpathcurveto{\pgfqpoint{7.845447in}{1.603029in}}{\pgfqpoint{7.847451in}{1.598191in}}{\pgfqpoint{7.851018in}{1.594625in}}%
\pgfpathcurveto{\pgfqpoint{7.854584in}{1.591058in}}{\pgfqpoint{7.859422in}{1.589054in}}{\pgfqpoint{7.864466in}{1.589054in}}%
\pgfpathclose%
\pgfusepath{fill}%
\end{pgfscope}%
\begin{pgfscope}%
\pgfpathrectangle{\pgfqpoint{6.572727in}{0.474100in}}{\pgfqpoint{4.227273in}{3.318700in}}%
\pgfusepath{clip}%
\pgfsetbuttcap%
\pgfsetroundjoin%
\definecolor{currentfill}{rgb}{0.127568,0.566949,0.550556}%
\pgfsetfillcolor{currentfill}%
\pgfsetfillopacity{0.700000}%
\pgfsetlinewidth{0.000000pt}%
\definecolor{currentstroke}{rgb}{0.000000,0.000000,0.000000}%
\pgfsetstrokecolor{currentstroke}%
\pgfsetstrokeopacity{0.700000}%
\pgfsetdash{}{0pt}%
\pgfpathmoveto{\pgfqpoint{9.629775in}{1.528945in}}%
\pgfpathcurveto{\pgfqpoint{9.634819in}{1.528945in}}{\pgfqpoint{9.639657in}{1.530949in}}{\pgfqpoint{9.643223in}{1.534515in}}%
\pgfpathcurveto{\pgfqpoint{9.646790in}{1.538081in}}{\pgfqpoint{9.648793in}{1.542919in}}{\pgfqpoint{9.648793in}{1.547963in}}%
\pgfpathcurveto{\pgfqpoint{9.648793in}{1.553006in}}{\pgfqpoint{9.646790in}{1.557844in}}{\pgfqpoint{9.643223in}{1.561411in}}%
\pgfpathcurveto{\pgfqpoint{9.639657in}{1.564977in}}{\pgfqpoint{9.634819in}{1.566981in}}{\pgfqpoint{9.629775in}{1.566981in}}%
\pgfpathcurveto{\pgfqpoint{9.624732in}{1.566981in}}{\pgfqpoint{9.619894in}{1.564977in}}{\pgfqpoint{9.616327in}{1.561411in}}%
\pgfpathcurveto{\pgfqpoint{9.612761in}{1.557844in}}{\pgfqpoint{9.610757in}{1.553006in}}{\pgfqpoint{9.610757in}{1.547963in}}%
\pgfpathcurveto{\pgfqpoint{9.610757in}{1.542919in}}{\pgfqpoint{9.612761in}{1.538081in}}{\pgfqpoint{9.616327in}{1.534515in}}%
\pgfpathcurveto{\pgfqpoint{9.619894in}{1.530949in}}{\pgfqpoint{9.624732in}{1.528945in}}{\pgfqpoint{9.629775in}{1.528945in}}%
\pgfpathclose%
\pgfusepath{fill}%
\end{pgfscope}%
\begin{pgfscope}%
\pgfpathrectangle{\pgfqpoint{6.572727in}{0.474100in}}{\pgfqpoint{4.227273in}{3.318700in}}%
\pgfusepath{clip}%
\pgfsetbuttcap%
\pgfsetroundjoin%
\definecolor{currentfill}{rgb}{0.127568,0.566949,0.550556}%
\pgfsetfillcolor{currentfill}%
\pgfsetfillopacity{0.700000}%
\pgfsetlinewidth{0.000000pt}%
\definecolor{currentstroke}{rgb}{0.000000,0.000000,0.000000}%
\pgfsetstrokecolor{currentstroke}%
\pgfsetstrokeopacity{0.700000}%
\pgfsetdash{}{0pt}%
\pgfpathmoveto{\pgfqpoint{9.785555in}{1.629689in}}%
\pgfpathcurveto{\pgfqpoint{9.790598in}{1.629689in}}{\pgfqpoint{9.795436in}{1.631693in}}{\pgfqpoint{9.799002in}{1.635260in}}%
\pgfpathcurveto{\pgfqpoint{9.802569in}{1.638826in}}{\pgfqpoint{9.804573in}{1.643664in}}{\pgfqpoint{9.804573in}{1.648708in}}%
\pgfpathcurveto{\pgfqpoint{9.804573in}{1.653751in}}{\pgfqpoint{9.802569in}{1.658589in}}{\pgfqpoint{9.799002in}{1.662155in}}%
\pgfpathcurveto{\pgfqpoint{9.795436in}{1.665722in}}{\pgfqpoint{9.790598in}{1.667726in}}{\pgfqpoint{9.785555in}{1.667726in}}%
\pgfpathcurveto{\pgfqpoint{9.780511in}{1.667726in}}{\pgfqpoint{9.775673in}{1.665722in}}{\pgfqpoint{9.772107in}{1.662155in}}%
\pgfpathcurveto{\pgfqpoint{9.768540in}{1.658589in}}{\pgfqpoint{9.766536in}{1.653751in}}{\pgfqpoint{9.766536in}{1.648708in}}%
\pgfpathcurveto{\pgfqpoint{9.766536in}{1.643664in}}{\pgfqpoint{9.768540in}{1.638826in}}{\pgfqpoint{9.772107in}{1.635260in}}%
\pgfpathcurveto{\pgfqpoint{9.775673in}{1.631693in}}{\pgfqpoint{9.780511in}{1.629689in}}{\pgfqpoint{9.785555in}{1.629689in}}%
\pgfpathclose%
\pgfusepath{fill}%
\end{pgfscope}%
\begin{pgfscope}%
\pgfpathrectangle{\pgfqpoint{6.572727in}{0.474100in}}{\pgfqpoint{4.227273in}{3.318700in}}%
\pgfusepath{clip}%
\pgfsetbuttcap%
\pgfsetroundjoin%
\definecolor{currentfill}{rgb}{0.127568,0.566949,0.550556}%
\pgfsetfillcolor{currentfill}%
\pgfsetfillopacity{0.700000}%
\pgfsetlinewidth{0.000000pt}%
\definecolor{currentstroke}{rgb}{0.000000,0.000000,0.000000}%
\pgfsetstrokecolor{currentstroke}%
\pgfsetstrokeopacity{0.700000}%
\pgfsetdash{}{0pt}%
\pgfpathmoveto{\pgfqpoint{9.315914in}{1.238980in}}%
\pgfpathcurveto{\pgfqpoint{9.320957in}{1.238980in}}{\pgfqpoint{9.325795in}{1.240983in}}{\pgfqpoint{9.329362in}{1.244550in}}%
\pgfpathcurveto{\pgfqpoint{9.332928in}{1.248116in}}{\pgfqpoint{9.334932in}{1.252954in}}{\pgfqpoint{9.334932in}{1.257998in}}%
\pgfpathcurveto{\pgfqpoint{9.334932in}{1.263041in}}{\pgfqpoint{9.332928in}{1.267879in}}{\pgfqpoint{9.329362in}{1.271446in}}%
\pgfpathcurveto{\pgfqpoint{9.325795in}{1.275012in}}{\pgfqpoint{9.320957in}{1.277016in}}{\pgfqpoint{9.315914in}{1.277016in}}%
\pgfpathcurveto{\pgfqpoint{9.310870in}{1.277016in}}{\pgfqpoint{9.306032in}{1.275012in}}{\pgfqpoint{9.302466in}{1.271446in}}%
\pgfpathcurveto{\pgfqpoint{9.298899in}{1.267879in}}{\pgfqpoint{9.296896in}{1.263041in}}{\pgfqpoint{9.296896in}{1.257998in}}%
\pgfpathcurveto{\pgfqpoint{9.296896in}{1.252954in}}{\pgfqpoint{9.298899in}{1.248116in}}{\pgfqpoint{9.302466in}{1.244550in}}%
\pgfpathcurveto{\pgfqpoint{9.306032in}{1.240983in}}{\pgfqpoint{9.310870in}{1.238980in}}{\pgfqpoint{9.315914in}{1.238980in}}%
\pgfpathclose%
\pgfusepath{fill}%
\end{pgfscope}%
\begin{pgfscope}%
\pgfpathrectangle{\pgfqpoint{6.572727in}{0.474100in}}{\pgfqpoint{4.227273in}{3.318700in}}%
\pgfusepath{clip}%
\pgfsetbuttcap%
\pgfsetroundjoin%
\definecolor{currentfill}{rgb}{0.993248,0.906157,0.143936}%
\pgfsetfillcolor{currentfill}%
\pgfsetfillopacity{0.700000}%
\pgfsetlinewidth{0.000000pt}%
\definecolor{currentstroke}{rgb}{0.000000,0.000000,0.000000}%
\pgfsetstrokecolor{currentstroke}%
\pgfsetstrokeopacity{0.700000}%
\pgfsetdash{}{0pt}%
\pgfpathmoveto{\pgfqpoint{8.526535in}{2.933432in}}%
\pgfpathcurveto{\pgfqpoint{8.531579in}{2.933432in}}{\pgfqpoint{8.536417in}{2.935436in}}{\pgfqpoint{8.539983in}{2.939002in}}%
\pgfpathcurveto{\pgfqpoint{8.543550in}{2.942569in}}{\pgfqpoint{8.545553in}{2.947406in}}{\pgfqpoint{8.545553in}{2.952450in}}%
\pgfpathcurveto{\pgfqpoint{8.545553in}{2.957494in}}{\pgfqpoint{8.543550in}{2.962331in}}{\pgfqpoint{8.539983in}{2.965898in}}%
\pgfpathcurveto{\pgfqpoint{8.536417in}{2.969464in}}{\pgfqpoint{8.531579in}{2.971468in}}{\pgfqpoint{8.526535in}{2.971468in}}%
\pgfpathcurveto{\pgfqpoint{8.521492in}{2.971468in}}{\pgfqpoint{8.516654in}{2.969464in}}{\pgfqpoint{8.513087in}{2.965898in}}%
\pgfpathcurveto{\pgfqpoint{8.509521in}{2.962331in}}{\pgfqpoint{8.507517in}{2.957494in}}{\pgfqpoint{8.507517in}{2.952450in}}%
\pgfpathcurveto{\pgfqpoint{8.507517in}{2.947406in}}{\pgfqpoint{8.509521in}{2.942569in}}{\pgfqpoint{8.513087in}{2.939002in}}%
\pgfpathcurveto{\pgfqpoint{8.516654in}{2.935436in}}{\pgfqpoint{8.521492in}{2.933432in}}{\pgfqpoint{8.526535in}{2.933432in}}%
\pgfpathclose%
\pgfusepath{fill}%
\end{pgfscope}%
\begin{pgfscope}%
\pgfpathrectangle{\pgfqpoint{6.572727in}{0.474100in}}{\pgfqpoint{4.227273in}{3.318700in}}%
\pgfusepath{clip}%
\pgfsetbuttcap%
\pgfsetroundjoin%
\definecolor{currentfill}{rgb}{0.267004,0.004874,0.329415}%
\pgfsetfillcolor{currentfill}%
\pgfsetfillopacity{0.700000}%
\pgfsetlinewidth{0.000000pt}%
\definecolor{currentstroke}{rgb}{0.000000,0.000000,0.000000}%
\pgfsetstrokecolor{currentstroke}%
\pgfsetstrokeopacity{0.700000}%
\pgfsetdash{}{0pt}%
\pgfpathmoveto{\pgfqpoint{7.632062in}{1.188426in}}%
\pgfpathcurveto{\pgfqpoint{7.637105in}{1.188426in}}{\pgfqpoint{7.641943in}{1.190430in}}{\pgfqpoint{7.645510in}{1.193996in}}%
\pgfpathcurveto{\pgfqpoint{7.649076in}{1.197563in}}{\pgfqpoint{7.651080in}{1.202400in}}{\pgfqpoint{7.651080in}{1.207444in}}%
\pgfpathcurveto{\pgfqpoint{7.651080in}{1.212488in}}{\pgfqpoint{7.649076in}{1.217325in}}{\pgfqpoint{7.645510in}{1.220892in}}%
\pgfpathcurveto{\pgfqpoint{7.641943in}{1.224458in}}{\pgfqpoint{7.637105in}{1.226462in}}{\pgfqpoint{7.632062in}{1.226462in}}%
\pgfpathcurveto{\pgfqpoint{7.627018in}{1.226462in}}{\pgfqpoint{7.622180in}{1.224458in}}{\pgfqpoint{7.618614in}{1.220892in}}%
\pgfpathcurveto{\pgfqpoint{7.615047in}{1.217325in}}{\pgfqpoint{7.613044in}{1.212488in}}{\pgfqpoint{7.613044in}{1.207444in}}%
\pgfpathcurveto{\pgfqpoint{7.613044in}{1.202400in}}{\pgfqpoint{7.615047in}{1.197563in}}{\pgfqpoint{7.618614in}{1.193996in}}%
\pgfpathcurveto{\pgfqpoint{7.622180in}{1.190430in}}{\pgfqpoint{7.627018in}{1.188426in}}{\pgfqpoint{7.632062in}{1.188426in}}%
\pgfpathclose%
\pgfusepath{fill}%
\end{pgfscope}%
\begin{pgfscope}%
\pgfpathrectangle{\pgfqpoint{6.572727in}{0.474100in}}{\pgfqpoint{4.227273in}{3.318700in}}%
\pgfusepath{clip}%
\pgfsetbuttcap%
\pgfsetroundjoin%
\definecolor{currentfill}{rgb}{0.127568,0.566949,0.550556}%
\pgfsetfillcolor{currentfill}%
\pgfsetfillopacity{0.700000}%
\pgfsetlinewidth{0.000000pt}%
\definecolor{currentstroke}{rgb}{0.000000,0.000000,0.000000}%
\pgfsetstrokecolor{currentstroke}%
\pgfsetstrokeopacity{0.700000}%
\pgfsetdash{}{0pt}%
\pgfpathmoveto{\pgfqpoint{9.806089in}{1.309000in}}%
\pgfpathcurveto{\pgfqpoint{9.811133in}{1.309000in}}{\pgfqpoint{9.815971in}{1.311004in}}{\pgfqpoint{9.819537in}{1.314570in}}%
\pgfpathcurveto{\pgfqpoint{9.823103in}{1.318137in}}{\pgfqpoint{9.825107in}{1.322975in}}{\pgfqpoint{9.825107in}{1.328018in}}%
\pgfpathcurveto{\pgfqpoint{9.825107in}{1.333062in}}{\pgfqpoint{9.823103in}{1.337900in}}{\pgfqpoint{9.819537in}{1.341466in}}%
\pgfpathcurveto{\pgfqpoint{9.815971in}{1.345033in}}{\pgfqpoint{9.811133in}{1.347036in}}{\pgfqpoint{9.806089in}{1.347036in}}%
\pgfpathcurveto{\pgfqpoint{9.801045in}{1.347036in}}{\pgfqpoint{9.796208in}{1.345033in}}{\pgfqpoint{9.792641in}{1.341466in}}%
\pgfpathcurveto{\pgfqpoint{9.789075in}{1.337900in}}{\pgfqpoint{9.787071in}{1.333062in}}{\pgfqpoint{9.787071in}{1.328018in}}%
\pgfpathcurveto{\pgfqpoint{9.787071in}{1.322975in}}{\pgfqpoint{9.789075in}{1.318137in}}{\pgfqpoint{9.792641in}{1.314570in}}%
\pgfpathcurveto{\pgfqpoint{9.796208in}{1.311004in}}{\pgfqpoint{9.801045in}{1.309000in}}{\pgfqpoint{9.806089in}{1.309000in}}%
\pgfpathclose%
\pgfusepath{fill}%
\end{pgfscope}%
\begin{pgfscope}%
\pgfpathrectangle{\pgfqpoint{6.572727in}{0.474100in}}{\pgfqpoint{4.227273in}{3.318700in}}%
\pgfusepath{clip}%
\pgfsetbuttcap%
\pgfsetroundjoin%
\definecolor{currentfill}{rgb}{0.267004,0.004874,0.329415}%
\pgfsetfillcolor{currentfill}%
\pgfsetfillopacity{0.700000}%
\pgfsetlinewidth{0.000000pt}%
\definecolor{currentstroke}{rgb}{0.000000,0.000000,0.000000}%
\pgfsetstrokecolor{currentstroke}%
\pgfsetstrokeopacity{0.700000}%
\pgfsetdash{}{0pt}%
\pgfpathmoveto{\pgfqpoint{7.583548in}{1.188066in}}%
\pgfpathcurveto{\pgfqpoint{7.588592in}{1.188066in}}{\pgfqpoint{7.593429in}{1.190070in}}{\pgfqpoint{7.596996in}{1.193636in}}%
\pgfpathcurveto{\pgfqpoint{7.600562in}{1.197202in}}{\pgfqpoint{7.602566in}{1.202040in}}{\pgfqpoint{7.602566in}{1.207084in}}%
\pgfpathcurveto{\pgfqpoint{7.602566in}{1.212128in}}{\pgfqpoint{7.600562in}{1.216965in}}{\pgfqpoint{7.596996in}{1.220532in}}%
\pgfpathcurveto{\pgfqpoint{7.593429in}{1.224098in}}{\pgfqpoint{7.588592in}{1.226102in}}{\pgfqpoint{7.583548in}{1.226102in}}%
\pgfpathcurveto{\pgfqpoint{7.578504in}{1.226102in}}{\pgfqpoint{7.573666in}{1.224098in}}{\pgfqpoint{7.570100in}{1.220532in}}%
\pgfpathcurveto{\pgfqpoint{7.566534in}{1.216965in}}{\pgfqpoint{7.564530in}{1.212128in}}{\pgfqpoint{7.564530in}{1.207084in}}%
\pgfpathcurveto{\pgfqpoint{7.564530in}{1.202040in}}{\pgfqpoint{7.566534in}{1.197202in}}{\pgfqpoint{7.570100in}{1.193636in}}%
\pgfpathcurveto{\pgfqpoint{7.573666in}{1.190070in}}{\pgfqpoint{7.578504in}{1.188066in}}{\pgfqpoint{7.583548in}{1.188066in}}%
\pgfpathclose%
\pgfusepath{fill}%
\end{pgfscope}%
\begin{pgfscope}%
\pgfpathrectangle{\pgfqpoint{6.572727in}{0.474100in}}{\pgfqpoint{4.227273in}{3.318700in}}%
\pgfusepath{clip}%
\pgfsetbuttcap%
\pgfsetroundjoin%
\definecolor{currentfill}{rgb}{0.267004,0.004874,0.329415}%
\pgfsetfillcolor{currentfill}%
\pgfsetfillopacity{0.700000}%
\pgfsetlinewidth{0.000000pt}%
\definecolor{currentstroke}{rgb}{0.000000,0.000000,0.000000}%
\pgfsetstrokecolor{currentstroke}%
\pgfsetstrokeopacity{0.700000}%
\pgfsetdash{}{0pt}%
\pgfpathmoveto{\pgfqpoint{7.070485in}{1.250366in}}%
\pgfpathcurveto{\pgfqpoint{7.075528in}{1.250366in}}{\pgfqpoint{7.080366in}{1.252370in}}{\pgfqpoint{7.083933in}{1.255937in}}%
\pgfpathcurveto{\pgfqpoint{7.087499in}{1.259503in}}{\pgfqpoint{7.089503in}{1.264341in}}{\pgfqpoint{7.089503in}{1.269385in}}%
\pgfpathcurveto{\pgfqpoint{7.089503in}{1.274428in}}{\pgfqpoint{7.087499in}{1.279266in}}{\pgfqpoint{7.083933in}{1.282832in}}%
\pgfpathcurveto{\pgfqpoint{7.080366in}{1.286399in}}{\pgfqpoint{7.075528in}{1.288403in}}{\pgfqpoint{7.070485in}{1.288403in}}%
\pgfpathcurveto{\pgfqpoint{7.065441in}{1.288403in}}{\pgfqpoint{7.060603in}{1.286399in}}{\pgfqpoint{7.057037in}{1.282832in}}%
\pgfpathcurveto{\pgfqpoint{7.053470in}{1.279266in}}{\pgfqpoint{7.051467in}{1.274428in}}{\pgfqpoint{7.051467in}{1.269385in}}%
\pgfpathcurveto{\pgfqpoint{7.051467in}{1.264341in}}{\pgfqpoint{7.053470in}{1.259503in}}{\pgfqpoint{7.057037in}{1.255937in}}%
\pgfpathcurveto{\pgfqpoint{7.060603in}{1.252370in}}{\pgfqpoint{7.065441in}{1.250366in}}{\pgfqpoint{7.070485in}{1.250366in}}%
\pgfpathclose%
\pgfusepath{fill}%
\end{pgfscope}%
\begin{pgfscope}%
\pgfpathrectangle{\pgfqpoint{6.572727in}{0.474100in}}{\pgfqpoint{4.227273in}{3.318700in}}%
\pgfusepath{clip}%
\pgfsetbuttcap%
\pgfsetroundjoin%
\definecolor{currentfill}{rgb}{0.267004,0.004874,0.329415}%
\pgfsetfillcolor{currentfill}%
\pgfsetfillopacity{0.700000}%
\pgfsetlinewidth{0.000000pt}%
\definecolor{currentstroke}{rgb}{0.000000,0.000000,0.000000}%
\pgfsetstrokecolor{currentstroke}%
\pgfsetstrokeopacity{0.700000}%
\pgfsetdash{}{0pt}%
\pgfpathmoveto{\pgfqpoint{7.793212in}{1.589804in}}%
\pgfpathcurveto{\pgfqpoint{7.798255in}{1.589804in}}{\pgfqpoint{7.803093in}{1.591808in}}{\pgfqpoint{7.806660in}{1.595374in}}%
\pgfpathcurveto{\pgfqpoint{7.810226in}{1.598941in}}{\pgfqpoint{7.812230in}{1.603778in}}{\pgfqpoint{7.812230in}{1.608822in}}%
\pgfpathcurveto{\pgfqpoint{7.812230in}{1.613866in}}{\pgfqpoint{7.810226in}{1.618704in}}{\pgfqpoint{7.806660in}{1.622270in}}%
\pgfpathcurveto{\pgfqpoint{7.803093in}{1.625836in}}{\pgfqpoint{7.798255in}{1.627840in}}{\pgfqpoint{7.793212in}{1.627840in}}%
\pgfpathcurveto{\pgfqpoint{7.788168in}{1.627840in}}{\pgfqpoint{7.783330in}{1.625836in}}{\pgfqpoint{7.779764in}{1.622270in}}%
\pgfpathcurveto{\pgfqpoint{7.776197in}{1.618704in}}{\pgfqpoint{7.774194in}{1.613866in}}{\pgfqpoint{7.774194in}{1.608822in}}%
\pgfpathcurveto{\pgfqpoint{7.774194in}{1.603778in}}{\pgfqpoint{7.776197in}{1.598941in}}{\pgfqpoint{7.779764in}{1.595374in}}%
\pgfpathcurveto{\pgfqpoint{7.783330in}{1.591808in}}{\pgfqpoint{7.788168in}{1.589804in}}{\pgfqpoint{7.793212in}{1.589804in}}%
\pgfpathclose%
\pgfusepath{fill}%
\end{pgfscope}%
\begin{pgfscope}%
\pgfpathrectangle{\pgfqpoint{6.572727in}{0.474100in}}{\pgfqpoint{4.227273in}{3.318700in}}%
\pgfusepath{clip}%
\pgfsetbuttcap%
\pgfsetroundjoin%
\definecolor{currentfill}{rgb}{0.993248,0.906157,0.143936}%
\pgfsetfillcolor{currentfill}%
\pgfsetfillopacity{0.700000}%
\pgfsetlinewidth{0.000000pt}%
\definecolor{currentstroke}{rgb}{0.000000,0.000000,0.000000}%
\pgfsetstrokecolor{currentstroke}%
\pgfsetstrokeopacity{0.700000}%
\pgfsetdash{}{0pt}%
\pgfpathmoveto{\pgfqpoint{8.054637in}{2.485720in}}%
\pgfpathcurveto{\pgfqpoint{8.059681in}{2.485720in}}{\pgfqpoint{8.064518in}{2.487724in}}{\pgfqpoint{8.068085in}{2.491291in}}%
\pgfpathcurveto{\pgfqpoint{8.071651in}{2.494857in}}{\pgfqpoint{8.073655in}{2.499695in}}{\pgfqpoint{8.073655in}{2.504738in}}%
\pgfpathcurveto{\pgfqpoint{8.073655in}{2.509782in}}{\pgfqpoint{8.071651in}{2.514620in}}{\pgfqpoint{8.068085in}{2.518186in}}%
\pgfpathcurveto{\pgfqpoint{8.064518in}{2.521753in}}{\pgfqpoint{8.059681in}{2.523757in}}{\pgfqpoint{8.054637in}{2.523757in}}%
\pgfpathcurveto{\pgfqpoint{8.049593in}{2.523757in}}{\pgfqpoint{8.044755in}{2.521753in}}{\pgfqpoint{8.041189in}{2.518186in}}%
\pgfpathcurveto{\pgfqpoint{8.037623in}{2.514620in}}{\pgfqpoint{8.035619in}{2.509782in}}{\pgfqpoint{8.035619in}{2.504738in}}%
\pgfpathcurveto{\pgfqpoint{8.035619in}{2.499695in}}{\pgfqpoint{8.037623in}{2.494857in}}{\pgfqpoint{8.041189in}{2.491291in}}%
\pgfpathcurveto{\pgfqpoint{8.044755in}{2.487724in}}{\pgfqpoint{8.049593in}{2.485720in}}{\pgfqpoint{8.054637in}{2.485720in}}%
\pgfpathclose%
\pgfusepath{fill}%
\end{pgfscope}%
\begin{pgfscope}%
\pgfpathrectangle{\pgfqpoint{6.572727in}{0.474100in}}{\pgfqpoint{4.227273in}{3.318700in}}%
\pgfusepath{clip}%
\pgfsetbuttcap%
\pgfsetroundjoin%
\definecolor{currentfill}{rgb}{0.267004,0.004874,0.329415}%
\pgfsetfillcolor{currentfill}%
\pgfsetfillopacity{0.700000}%
\pgfsetlinewidth{0.000000pt}%
\definecolor{currentstroke}{rgb}{0.000000,0.000000,0.000000}%
\pgfsetstrokecolor{currentstroke}%
\pgfsetstrokeopacity{0.700000}%
\pgfsetdash{}{0pt}%
\pgfpathmoveto{\pgfqpoint{7.599768in}{1.282891in}}%
\pgfpathcurveto{\pgfqpoint{7.604812in}{1.282891in}}{\pgfqpoint{7.609650in}{1.284895in}}{\pgfqpoint{7.613216in}{1.288461in}}%
\pgfpathcurveto{\pgfqpoint{7.616782in}{1.292028in}}{\pgfqpoint{7.618786in}{1.296866in}}{\pgfqpoint{7.618786in}{1.301909in}}%
\pgfpathcurveto{\pgfqpoint{7.618786in}{1.306953in}}{\pgfqpoint{7.616782in}{1.311791in}}{\pgfqpoint{7.613216in}{1.315357in}}%
\pgfpathcurveto{\pgfqpoint{7.609650in}{1.318924in}}{\pgfqpoint{7.604812in}{1.320927in}}{\pgfqpoint{7.599768in}{1.320927in}}%
\pgfpathcurveto{\pgfqpoint{7.594724in}{1.320927in}}{\pgfqpoint{7.589887in}{1.318924in}}{\pgfqpoint{7.586320in}{1.315357in}}%
\pgfpathcurveto{\pgfqpoint{7.582754in}{1.311791in}}{\pgfqpoint{7.580750in}{1.306953in}}{\pgfqpoint{7.580750in}{1.301909in}}%
\pgfpathcurveto{\pgfqpoint{7.580750in}{1.296866in}}{\pgfqpoint{7.582754in}{1.292028in}}{\pgfqpoint{7.586320in}{1.288461in}}%
\pgfpathcurveto{\pgfqpoint{7.589887in}{1.284895in}}{\pgfqpoint{7.594724in}{1.282891in}}{\pgfqpoint{7.599768in}{1.282891in}}%
\pgfpathclose%
\pgfusepath{fill}%
\end{pgfscope}%
\begin{pgfscope}%
\pgfpathrectangle{\pgfqpoint{6.572727in}{0.474100in}}{\pgfqpoint{4.227273in}{3.318700in}}%
\pgfusepath{clip}%
\pgfsetbuttcap%
\pgfsetroundjoin%
\definecolor{currentfill}{rgb}{0.993248,0.906157,0.143936}%
\pgfsetfillcolor{currentfill}%
\pgfsetfillopacity{0.700000}%
\pgfsetlinewidth{0.000000pt}%
\definecolor{currentstroke}{rgb}{0.000000,0.000000,0.000000}%
\pgfsetstrokecolor{currentstroke}%
\pgfsetstrokeopacity{0.700000}%
\pgfsetdash{}{0pt}%
\pgfpathmoveto{\pgfqpoint{8.375172in}{3.020818in}}%
\pgfpathcurveto{\pgfqpoint{8.380216in}{3.020818in}}{\pgfqpoint{8.385054in}{3.022822in}}{\pgfqpoint{8.388620in}{3.026388in}}%
\pgfpathcurveto{\pgfqpoint{8.392187in}{3.029954in}}{\pgfqpoint{8.394191in}{3.034792in}}{\pgfqpoint{8.394191in}{3.039836in}}%
\pgfpathcurveto{\pgfqpoint{8.394191in}{3.044880in}}{\pgfqpoint{8.392187in}{3.049717in}}{\pgfqpoint{8.388620in}{3.053284in}}%
\pgfpathcurveto{\pgfqpoint{8.385054in}{3.056850in}}{\pgfqpoint{8.380216in}{3.058854in}}{\pgfqpoint{8.375172in}{3.058854in}}%
\pgfpathcurveto{\pgfqpoint{8.370129in}{3.058854in}}{\pgfqpoint{8.365291in}{3.056850in}}{\pgfqpoint{8.361725in}{3.053284in}}%
\pgfpathcurveto{\pgfqpoint{8.358158in}{3.049717in}}{\pgfqpoint{8.356154in}{3.044880in}}{\pgfqpoint{8.356154in}{3.039836in}}%
\pgfpathcurveto{\pgfqpoint{8.356154in}{3.034792in}}{\pgfqpoint{8.358158in}{3.029954in}}{\pgfqpoint{8.361725in}{3.026388in}}%
\pgfpathcurveto{\pgfqpoint{8.365291in}{3.022822in}}{\pgfqpoint{8.370129in}{3.020818in}}{\pgfqpoint{8.375172in}{3.020818in}}%
\pgfpathclose%
\pgfusepath{fill}%
\end{pgfscope}%
\begin{pgfscope}%
\pgfpathrectangle{\pgfqpoint{6.572727in}{0.474100in}}{\pgfqpoint{4.227273in}{3.318700in}}%
\pgfusepath{clip}%
\pgfsetbuttcap%
\pgfsetroundjoin%
\definecolor{currentfill}{rgb}{0.993248,0.906157,0.143936}%
\pgfsetfillcolor{currentfill}%
\pgfsetfillopacity{0.700000}%
\pgfsetlinewidth{0.000000pt}%
\definecolor{currentstroke}{rgb}{0.000000,0.000000,0.000000}%
\pgfsetstrokecolor{currentstroke}%
\pgfsetstrokeopacity{0.700000}%
\pgfsetdash{}{0pt}%
\pgfpathmoveto{\pgfqpoint{7.989577in}{2.763777in}}%
\pgfpathcurveto{\pgfqpoint{7.994620in}{2.763777in}}{\pgfqpoint{7.999458in}{2.765781in}}{\pgfqpoint{8.003025in}{2.769348in}}%
\pgfpathcurveto{\pgfqpoint{8.006591in}{2.772914in}}{\pgfqpoint{8.008595in}{2.777752in}}{\pgfqpoint{8.008595in}{2.782795in}}%
\pgfpathcurveto{\pgfqpoint{8.008595in}{2.787839in}}{\pgfqpoint{8.006591in}{2.792677in}}{\pgfqpoint{8.003025in}{2.796243in}}%
\pgfpathcurveto{\pgfqpoint{7.999458in}{2.799810in}}{\pgfqpoint{7.994620in}{2.801814in}}{\pgfqpoint{7.989577in}{2.801814in}}%
\pgfpathcurveto{\pgfqpoint{7.984533in}{2.801814in}}{\pgfqpoint{7.979695in}{2.799810in}}{\pgfqpoint{7.976129in}{2.796243in}}%
\pgfpathcurveto{\pgfqpoint{7.972562in}{2.792677in}}{\pgfqpoint{7.970559in}{2.787839in}}{\pgfqpoint{7.970559in}{2.782795in}}%
\pgfpathcurveto{\pgfqpoint{7.970559in}{2.777752in}}{\pgfqpoint{7.972562in}{2.772914in}}{\pgfqpoint{7.976129in}{2.769348in}}%
\pgfpathcurveto{\pgfqpoint{7.979695in}{2.765781in}}{\pgfqpoint{7.984533in}{2.763777in}}{\pgfqpoint{7.989577in}{2.763777in}}%
\pgfpathclose%
\pgfusepath{fill}%
\end{pgfscope}%
\begin{pgfscope}%
\pgfpathrectangle{\pgfqpoint{6.572727in}{0.474100in}}{\pgfqpoint{4.227273in}{3.318700in}}%
\pgfusepath{clip}%
\pgfsetbuttcap%
\pgfsetroundjoin%
\definecolor{currentfill}{rgb}{0.267004,0.004874,0.329415}%
\pgfsetfillcolor{currentfill}%
\pgfsetfillopacity{0.700000}%
\pgfsetlinewidth{0.000000pt}%
\definecolor{currentstroke}{rgb}{0.000000,0.000000,0.000000}%
\pgfsetstrokecolor{currentstroke}%
\pgfsetstrokeopacity{0.700000}%
\pgfsetdash{}{0pt}%
\pgfpathmoveto{\pgfqpoint{8.127369in}{1.497664in}}%
\pgfpathcurveto{\pgfqpoint{8.132413in}{1.497664in}}{\pgfqpoint{8.137251in}{1.499668in}}{\pgfqpoint{8.140817in}{1.503234in}}%
\pgfpathcurveto{\pgfqpoint{8.144384in}{1.506801in}}{\pgfqpoint{8.146387in}{1.511638in}}{\pgfqpoint{8.146387in}{1.516682in}}%
\pgfpathcurveto{\pgfqpoint{8.146387in}{1.521726in}}{\pgfqpoint{8.144384in}{1.526564in}}{\pgfqpoint{8.140817in}{1.530130in}}%
\pgfpathcurveto{\pgfqpoint{8.137251in}{1.533696in}}{\pgfqpoint{8.132413in}{1.535700in}}{\pgfqpoint{8.127369in}{1.535700in}}%
\pgfpathcurveto{\pgfqpoint{8.122326in}{1.535700in}}{\pgfqpoint{8.117488in}{1.533696in}}{\pgfqpoint{8.113921in}{1.530130in}}%
\pgfpathcurveto{\pgfqpoint{8.110355in}{1.526564in}}{\pgfqpoint{8.108351in}{1.521726in}}{\pgfqpoint{8.108351in}{1.516682in}}%
\pgfpathcurveto{\pgfqpoint{8.108351in}{1.511638in}}{\pgfqpoint{8.110355in}{1.506801in}}{\pgfqpoint{8.113921in}{1.503234in}}%
\pgfpathcurveto{\pgfqpoint{8.117488in}{1.499668in}}{\pgfqpoint{8.122326in}{1.497664in}}{\pgfqpoint{8.127369in}{1.497664in}}%
\pgfpathclose%
\pgfusepath{fill}%
\end{pgfscope}%
\begin{pgfscope}%
\pgfpathrectangle{\pgfqpoint{6.572727in}{0.474100in}}{\pgfqpoint{4.227273in}{3.318700in}}%
\pgfusepath{clip}%
\pgfsetbuttcap%
\pgfsetroundjoin%
\definecolor{currentfill}{rgb}{0.127568,0.566949,0.550556}%
\pgfsetfillcolor{currentfill}%
\pgfsetfillopacity{0.700000}%
\pgfsetlinewidth{0.000000pt}%
\definecolor{currentstroke}{rgb}{0.000000,0.000000,0.000000}%
\pgfsetstrokecolor{currentstroke}%
\pgfsetstrokeopacity{0.700000}%
\pgfsetdash{}{0pt}%
\pgfpathmoveto{\pgfqpoint{9.781117in}{1.506190in}}%
\pgfpathcurveto{\pgfqpoint{9.786161in}{1.506190in}}{\pgfqpoint{9.790999in}{1.508194in}}{\pgfqpoint{9.794565in}{1.511760in}}%
\pgfpathcurveto{\pgfqpoint{9.798132in}{1.515326in}}{\pgfqpoint{9.800136in}{1.520164in}}{\pgfqpoint{9.800136in}{1.525208in}}%
\pgfpathcurveto{\pgfqpoint{9.800136in}{1.530252in}}{\pgfqpoint{9.798132in}{1.535089in}}{\pgfqpoint{9.794565in}{1.538656in}}%
\pgfpathcurveto{\pgfqpoint{9.790999in}{1.542222in}}{\pgfqpoint{9.786161in}{1.544226in}}{\pgfqpoint{9.781117in}{1.544226in}}%
\pgfpathcurveto{\pgfqpoint{9.776074in}{1.544226in}}{\pgfqpoint{9.771236in}{1.542222in}}{\pgfqpoint{9.767670in}{1.538656in}}%
\pgfpathcurveto{\pgfqpoint{9.764103in}{1.535089in}}{\pgfqpoint{9.762099in}{1.530252in}}{\pgfqpoint{9.762099in}{1.525208in}}%
\pgfpathcurveto{\pgfqpoint{9.762099in}{1.520164in}}{\pgfqpoint{9.764103in}{1.515326in}}{\pgfqpoint{9.767670in}{1.511760in}}%
\pgfpathcurveto{\pgfqpoint{9.771236in}{1.508194in}}{\pgfqpoint{9.776074in}{1.506190in}}{\pgfqpoint{9.781117in}{1.506190in}}%
\pgfpathclose%
\pgfusepath{fill}%
\end{pgfscope}%
\begin{pgfscope}%
\pgfpathrectangle{\pgfqpoint{6.572727in}{0.474100in}}{\pgfqpoint{4.227273in}{3.318700in}}%
\pgfusepath{clip}%
\pgfsetbuttcap%
\pgfsetroundjoin%
\definecolor{currentfill}{rgb}{0.127568,0.566949,0.550556}%
\pgfsetfillcolor{currentfill}%
\pgfsetfillopacity{0.700000}%
\pgfsetlinewidth{0.000000pt}%
\definecolor{currentstroke}{rgb}{0.000000,0.000000,0.000000}%
\pgfsetstrokecolor{currentstroke}%
\pgfsetstrokeopacity{0.700000}%
\pgfsetdash{}{0pt}%
\pgfpathmoveto{\pgfqpoint{9.497266in}{1.564599in}}%
\pgfpathcurveto{\pgfqpoint{9.502309in}{1.564599in}}{\pgfqpoint{9.507147in}{1.566603in}}{\pgfqpoint{9.510713in}{1.570169in}}%
\pgfpathcurveto{\pgfqpoint{9.514280in}{1.573736in}}{\pgfqpoint{9.516284in}{1.578574in}}{\pgfqpoint{9.516284in}{1.583617in}}%
\pgfpathcurveto{\pgfqpoint{9.516284in}{1.588661in}}{\pgfqpoint{9.514280in}{1.593499in}}{\pgfqpoint{9.510713in}{1.597065in}}%
\pgfpathcurveto{\pgfqpoint{9.507147in}{1.600631in}}{\pgfqpoint{9.502309in}{1.602635in}}{\pgfqpoint{9.497266in}{1.602635in}}%
\pgfpathcurveto{\pgfqpoint{9.492222in}{1.602635in}}{\pgfqpoint{9.487384in}{1.600631in}}{\pgfqpoint{9.483818in}{1.597065in}}%
\pgfpathcurveto{\pgfqpoint{9.480251in}{1.593499in}}{\pgfqpoint{9.478247in}{1.588661in}}{\pgfqpoint{9.478247in}{1.583617in}}%
\pgfpathcurveto{\pgfqpoint{9.478247in}{1.578574in}}{\pgfqpoint{9.480251in}{1.573736in}}{\pgfqpoint{9.483818in}{1.570169in}}%
\pgfpathcurveto{\pgfqpoint{9.487384in}{1.566603in}}{\pgfqpoint{9.492222in}{1.564599in}}{\pgfqpoint{9.497266in}{1.564599in}}%
\pgfpathclose%
\pgfusepath{fill}%
\end{pgfscope}%
\begin{pgfscope}%
\pgfpathrectangle{\pgfqpoint{6.572727in}{0.474100in}}{\pgfqpoint{4.227273in}{3.318700in}}%
\pgfusepath{clip}%
\pgfsetbuttcap%
\pgfsetroundjoin%
\definecolor{currentfill}{rgb}{0.267004,0.004874,0.329415}%
\pgfsetfillcolor{currentfill}%
\pgfsetfillopacity{0.700000}%
\pgfsetlinewidth{0.000000pt}%
\definecolor{currentstroke}{rgb}{0.000000,0.000000,0.000000}%
\pgfsetstrokecolor{currentstroke}%
\pgfsetstrokeopacity{0.700000}%
\pgfsetdash{}{0pt}%
\pgfpathmoveto{\pgfqpoint{7.090086in}{1.496827in}}%
\pgfpathcurveto{\pgfqpoint{7.095130in}{1.496827in}}{\pgfqpoint{7.099967in}{1.498831in}}{\pgfqpoint{7.103534in}{1.502397in}}%
\pgfpathcurveto{\pgfqpoint{7.107100in}{1.505963in}}{\pgfqpoint{7.109104in}{1.510801in}}{\pgfqpoint{7.109104in}{1.515845in}}%
\pgfpathcurveto{\pgfqpoint{7.109104in}{1.520889in}}{\pgfqpoint{7.107100in}{1.525726in}}{\pgfqpoint{7.103534in}{1.529293in}}%
\pgfpathcurveto{\pgfqpoint{7.099967in}{1.532859in}}{\pgfqpoint{7.095130in}{1.534863in}}{\pgfqpoint{7.090086in}{1.534863in}}%
\pgfpathcurveto{\pgfqpoint{7.085042in}{1.534863in}}{\pgfqpoint{7.080205in}{1.532859in}}{\pgfqpoint{7.076638in}{1.529293in}}%
\pgfpathcurveto{\pgfqpoint{7.073072in}{1.525726in}}{\pgfqpoint{7.071068in}{1.520889in}}{\pgfqpoint{7.071068in}{1.515845in}}%
\pgfpathcurveto{\pgfqpoint{7.071068in}{1.510801in}}{\pgfqpoint{7.073072in}{1.505963in}}{\pgfqpoint{7.076638in}{1.502397in}}%
\pgfpathcurveto{\pgfqpoint{7.080205in}{1.498831in}}{\pgfqpoint{7.085042in}{1.496827in}}{\pgfqpoint{7.090086in}{1.496827in}}%
\pgfpathclose%
\pgfusepath{fill}%
\end{pgfscope}%
\begin{pgfscope}%
\pgfpathrectangle{\pgfqpoint{6.572727in}{0.474100in}}{\pgfqpoint{4.227273in}{3.318700in}}%
\pgfusepath{clip}%
\pgfsetbuttcap%
\pgfsetroundjoin%
\definecolor{currentfill}{rgb}{0.993248,0.906157,0.143936}%
\pgfsetfillcolor{currentfill}%
\pgfsetfillopacity{0.700000}%
\pgfsetlinewidth{0.000000pt}%
\definecolor{currentstroke}{rgb}{0.000000,0.000000,0.000000}%
\pgfsetstrokecolor{currentstroke}%
\pgfsetstrokeopacity{0.700000}%
\pgfsetdash{}{0pt}%
\pgfpathmoveto{\pgfqpoint{8.236202in}{3.175845in}}%
\pgfpathcurveto{\pgfqpoint{8.241246in}{3.175845in}}{\pgfqpoint{8.246084in}{3.177849in}}{\pgfqpoint{8.249650in}{3.181415in}}%
\pgfpathcurveto{\pgfqpoint{8.253217in}{3.184982in}}{\pgfqpoint{8.255220in}{3.189820in}}{\pgfqpoint{8.255220in}{3.194863in}}%
\pgfpathcurveto{\pgfqpoint{8.255220in}{3.199907in}}{\pgfqpoint{8.253217in}{3.204745in}}{\pgfqpoint{8.249650in}{3.208311in}}%
\pgfpathcurveto{\pgfqpoint{8.246084in}{3.211878in}}{\pgfqpoint{8.241246in}{3.213881in}}{\pgfqpoint{8.236202in}{3.213881in}}%
\pgfpathcurveto{\pgfqpoint{8.231159in}{3.213881in}}{\pgfqpoint{8.226321in}{3.211878in}}{\pgfqpoint{8.222754in}{3.208311in}}%
\pgfpathcurveto{\pgfqpoint{8.219188in}{3.204745in}}{\pgfqpoint{8.217184in}{3.199907in}}{\pgfqpoint{8.217184in}{3.194863in}}%
\pgfpathcurveto{\pgfqpoint{8.217184in}{3.189820in}}{\pgfqpoint{8.219188in}{3.184982in}}{\pgfqpoint{8.222754in}{3.181415in}}%
\pgfpathcurveto{\pgfqpoint{8.226321in}{3.177849in}}{\pgfqpoint{8.231159in}{3.175845in}}{\pgfqpoint{8.236202in}{3.175845in}}%
\pgfpathclose%
\pgfusepath{fill}%
\end{pgfscope}%
\begin{pgfscope}%
\pgfpathrectangle{\pgfqpoint{6.572727in}{0.474100in}}{\pgfqpoint{4.227273in}{3.318700in}}%
\pgfusepath{clip}%
\pgfsetbuttcap%
\pgfsetroundjoin%
\definecolor{currentfill}{rgb}{0.993248,0.906157,0.143936}%
\pgfsetfillcolor{currentfill}%
\pgfsetfillopacity{0.700000}%
\pgfsetlinewidth{0.000000pt}%
\definecolor{currentstroke}{rgb}{0.000000,0.000000,0.000000}%
\pgfsetstrokecolor{currentstroke}%
\pgfsetstrokeopacity{0.700000}%
\pgfsetdash{}{0pt}%
\pgfpathmoveto{\pgfqpoint{7.645761in}{2.969141in}}%
\pgfpathcurveto{\pgfqpoint{7.650804in}{2.969141in}}{\pgfqpoint{7.655642in}{2.971145in}}{\pgfqpoint{7.659208in}{2.974712in}}%
\pgfpathcurveto{\pgfqpoint{7.662775in}{2.978278in}}{\pgfqpoint{7.664779in}{2.983116in}}{\pgfqpoint{7.664779in}{2.988159in}}%
\pgfpathcurveto{\pgfqpoint{7.664779in}{2.993203in}}{\pgfqpoint{7.662775in}{2.998041in}}{\pgfqpoint{7.659208in}{3.001607in}}%
\pgfpathcurveto{\pgfqpoint{7.655642in}{3.005174in}}{\pgfqpoint{7.650804in}{3.007178in}}{\pgfqpoint{7.645761in}{3.007178in}}%
\pgfpathcurveto{\pgfqpoint{7.640717in}{3.007178in}}{\pgfqpoint{7.635879in}{3.005174in}}{\pgfqpoint{7.632313in}{3.001607in}}%
\pgfpathcurveto{\pgfqpoint{7.628746in}{2.998041in}}{\pgfqpoint{7.626742in}{2.993203in}}{\pgfqpoint{7.626742in}{2.988159in}}%
\pgfpathcurveto{\pgfqpoint{7.626742in}{2.983116in}}{\pgfqpoint{7.628746in}{2.978278in}}{\pgfqpoint{7.632313in}{2.974712in}}%
\pgfpathcurveto{\pgfqpoint{7.635879in}{2.971145in}}{\pgfqpoint{7.640717in}{2.969141in}}{\pgfqpoint{7.645761in}{2.969141in}}%
\pgfpathclose%
\pgfusepath{fill}%
\end{pgfscope}%
\begin{pgfscope}%
\pgfpathrectangle{\pgfqpoint{6.572727in}{0.474100in}}{\pgfqpoint{4.227273in}{3.318700in}}%
\pgfusepath{clip}%
\pgfsetbuttcap%
\pgfsetroundjoin%
\definecolor{currentfill}{rgb}{0.993248,0.906157,0.143936}%
\pgfsetfillcolor{currentfill}%
\pgfsetfillopacity{0.700000}%
\pgfsetlinewidth{0.000000pt}%
\definecolor{currentstroke}{rgb}{0.000000,0.000000,0.000000}%
\pgfsetstrokecolor{currentstroke}%
\pgfsetstrokeopacity{0.700000}%
\pgfsetdash{}{0pt}%
\pgfpathmoveto{\pgfqpoint{8.310008in}{2.679224in}}%
\pgfpathcurveto{\pgfqpoint{8.315052in}{2.679224in}}{\pgfqpoint{8.319890in}{2.681228in}}{\pgfqpoint{8.323456in}{2.684794in}}%
\pgfpathcurveto{\pgfqpoint{8.327023in}{2.688361in}}{\pgfqpoint{8.329026in}{2.693199in}}{\pgfqpoint{8.329026in}{2.698242in}}%
\pgfpathcurveto{\pgfqpoint{8.329026in}{2.703286in}}{\pgfqpoint{8.327023in}{2.708124in}}{\pgfqpoint{8.323456in}{2.711690in}}%
\pgfpathcurveto{\pgfqpoint{8.319890in}{2.715257in}}{\pgfqpoint{8.315052in}{2.717260in}}{\pgfqpoint{8.310008in}{2.717260in}}%
\pgfpathcurveto{\pgfqpoint{8.304965in}{2.717260in}}{\pgfqpoint{8.300127in}{2.715257in}}{\pgfqpoint{8.296560in}{2.711690in}}%
\pgfpathcurveto{\pgfqpoint{8.292994in}{2.708124in}}{\pgfqpoint{8.290990in}{2.703286in}}{\pgfqpoint{8.290990in}{2.698242in}}%
\pgfpathcurveto{\pgfqpoint{8.290990in}{2.693199in}}{\pgfqpoint{8.292994in}{2.688361in}}{\pgfqpoint{8.296560in}{2.684794in}}%
\pgfpathcurveto{\pgfqpoint{8.300127in}{2.681228in}}{\pgfqpoint{8.304965in}{2.679224in}}{\pgfqpoint{8.310008in}{2.679224in}}%
\pgfpathclose%
\pgfusepath{fill}%
\end{pgfscope}%
\begin{pgfscope}%
\pgfpathrectangle{\pgfqpoint{6.572727in}{0.474100in}}{\pgfqpoint{4.227273in}{3.318700in}}%
\pgfusepath{clip}%
\pgfsetbuttcap%
\pgfsetroundjoin%
\definecolor{currentfill}{rgb}{0.127568,0.566949,0.550556}%
\pgfsetfillcolor{currentfill}%
\pgfsetfillopacity{0.700000}%
\pgfsetlinewidth{0.000000pt}%
\definecolor{currentstroke}{rgb}{0.000000,0.000000,0.000000}%
\pgfsetstrokecolor{currentstroke}%
\pgfsetstrokeopacity{0.700000}%
\pgfsetdash{}{0pt}%
\pgfpathmoveto{\pgfqpoint{9.612338in}{1.447560in}}%
\pgfpathcurveto{\pgfqpoint{9.617382in}{1.447560in}}{\pgfqpoint{9.622220in}{1.449564in}}{\pgfqpoint{9.625786in}{1.453131in}}%
\pgfpathcurveto{\pgfqpoint{9.629353in}{1.456697in}}{\pgfqpoint{9.631357in}{1.461535in}}{\pgfqpoint{9.631357in}{1.466579in}}%
\pgfpathcurveto{\pgfqpoint{9.631357in}{1.471622in}}{\pgfqpoint{9.629353in}{1.476460in}}{\pgfqpoint{9.625786in}{1.480026in}}%
\pgfpathcurveto{\pgfqpoint{9.622220in}{1.483593in}}{\pgfqpoint{9.617382in}{1.485597in}}{\pgfqpoint{9.612338in}{1.485597in}}%
\pgfpathcurveto{\pgfqpoint{9.607295in}{1.485597in}}{\pgfqpoint{9.602457in}{1.483593in}}{\pgfqpoint{9.598891in}{1.480026in}}%
\pgfpathcurveto{\pgfqpoint{9.595324in}{1.476460in}}{\pgfqpoint{9.593320in}{1.471622in}}{\pgfqpoint{9.593320in}{1.466579in}}%
\pgfpathcurveto{\pgfqpoint{9.593320in}{1.461535in}}{\pgfqpoint{9.595324in}{1.456697in}}{\pgfqpoint{9.598891in}{1.453131in}}%
\pgfpathcurveto{\pgfqpoint{9.602457in}{1.449564in}}{\pgfqpoint{9.607295in}{1.447560in}}{\pgfqpoint{9.612338in}{1.447560in}}%
\pgfpathclose%
\pgfusepath{fill}%
\end{pgfscope}%
\begin{pgfscope}%
\pgfpathrectangle{\pgfqpoint{6.572727in}{0.474100in}}{\pgfqpoint{4.227273in}{3.318700in}}%
\pgfusepath{clip}%
\pgfsetbuttcap%
\pgfsetroundjoin%
\definecolor{currentfill}{rgb}{0.127568,0.566949,0.550556}%
\pgfsetfillcolor{currentfill}%
\pgfsetfillopacity{0.700000}%
\pgfsetlinewidth{0.000000pt}%
\definecolor{currentstroke}{rgb}{0.000000,0.000000,0.000000}%
\pgfsetstrokecolor{currentstroke}%
\pgfsetstrokeopacity{0.700000}%
\pgfsetdash{}{0pt}%
\pgfpathmoveto{\pgfqpoint{9.350576in}{1.352473in}}%
\pgfpathcurveto{\pgfqpoint{9.355619in}{1.352473in}}{\pgfqpoint{9.360457in}{1.354477in}}{\pgfqpoint{9.364024in}{1.358043in}}%
\pgfpathcurveto{\pgfqpoint{9.367590in}{1.361610in}}{\pgfqpoint{9.369594in}{1.366447in}}{\pgfqpoint{9.369594in}{1.371491in}}%
\pgfpathcurveto{\pgfqpoint{9.369594in}{1.376535in}}{\pgfqpoint{9.367590in}{1.381373in}}{\pgfqpoint{9.364024in}{1.384939in}}%
\pgfpathcurveto{\pgfqpoint{9.360457in}{1.388505in}}{\pgfqpoint{9.355619in}{1.390509in}}{\pgfqpoint{9.350576in}{1.390509in}}%
\pgfpathcurveto{\pgfqpoint{9.345532in}{1.390509in}}{\pgfqpoint{9.340694in}{1.388505in}}{\pgfqpoint{9.337128in}{1.384939in}}%
\pgfpathcurveto{\pgfqpoint{9.333561in}{1.381373in}}{\pgfqpoint{9.331558in}{1.376535in}}{\pgfqpoint{9.331558in}{1.371491in}}%
\pgfpathcurveto{\pgfqpoint{9.331558in}{1.366447in}}{\pgfqpoint{9.333561in}{1.361610in}}{\pgfqpoint{9.337128in}{1.358043in}}%
\pgfpathcurveto{\pgfqpoint{9.340694in}{1.354477in}}{\pgfqpoint{9.345532in}{1.352473in}}{\pgfqpoint{9.350576in}{1.352473in}}%
\pgfpathclose%
\pgfusepath{fill}%
\end{pgfscope}%
\begin{pgfscope}%
\pgfpathrectangle{\pgfqpoint{6.572727in}{0.474100in}}{\pgfqpoint{4.227273in}{3.318700in}}%
\pgfusepath{clip}%
\pgfsetbuttcap%
\pgfsetroundjoin%
\definecolor{currentfill}{rgb}{0.993248,0.906157,0.143936}%
\pgfsetfillcolor{currentfill}%
\pgfsetfillopacity{0.700000}%
\pgfsetlinewidth{0.000000pt}%
\definecolor{currentstroke}{rgb}{0.000000,0.000000,0.000000}%
\pgfsetstrokecolor{currentstroke}%
\pgfsetstrokeopacity{0.700000}%
\pgfsetdash{}{0pt}%
\pgfpathmoveto{\pgfqpoint{8.407916in}{2.925847in}}%
\pgfpathcurveto{\pgfqpoint{8.412960in}{2.925847in}}{\pgfqpoint{8.417797in}{2.927851in}}{\pgfqpoint{8.421364in}{2.931417in}}%
\pgfpathcurveto{\pgfqpoint{8.424930in}{2.934984in}}{\pgfqpoint{8.426934in}{2.939822in}}{\pgfqpoint{8.426934in}{2.944865in}}%
\pgfpathcurveto{\pgfqpoint{8.426934in}{2.949909in}}{\pgfqpoint{8.424930in}{2.954747in}}{\pgfqpoint{8.421364in}{2.958313in}}%
\pgfpathcurveto{\pgfqpoint{8.417797in}{2.961880in}}{\pgfqpoint{8.412960in}{2.963883in}}{\pgfqpoint{8.407916in}{2.963883in}}%
\pgfpathcurveto{\pgfqpoint{8.402872in}{2.963883in}}{\pgfqpoint{8.398035in}{2.961880in}}{\pgfqpoint{8.394468in}{2.958313in}}%
\pgfpathcurveto{\pgfqpoint{8.390902in}{2.954747in}}{\pgfqpoint{8.388898in}{2.949909in}}{\pgfqpoint{8.388898in}{2.944865in}}%
\pgfpathcurveto{\pgfqpoint{8.388898in}{2.939822in}}{\pgfqpoint{8.390902in}{2.934984in}}{\pgfqpoint{8.394468in}{2.931417in}}%
\pgfpathcurveto{\pgfqpoint{8.398035in}{2.927851in}}{\pgfqpoint{8.402872in}{2.925847in}}{\pgfqpoint{8.407916in}{2.925847in}}%
\pgfpathclose%
\pgfusepath{fill}%
\end{pgfscope}%
\begin{pgfscope}%
\pgfpathrectangle{\pgfqpoint{6.572727in}{0.474100in}}{\pgfqpoint{4.227273in}{3.318700in}}%
\pgfusepath{clip}%
\pgfsetbuttcap%
\pgfsetroundjoin%
\definecolor{currentfill}{rgb}{0.127568,0.566949,0.550556}%
\pgfsetfillcolor{currentfill}%
\pgfsetfillopacity{0.700000}%
\pgfsetlinewidth{0.000000pt}%
\definecolor{currentstroke}{rgb}{0.000000,0.000000,0.000000}%
\pgfsetstrokecolor{currentstroke}%
\pgfsetstrokeopacity{0.700000}%
\pgfsetdash{}{0pt}%
\pgfpathmoveto{\pgfqpoint{9.464407in}{1.748516in}}%
\pgfpathcurveto{\pgfqpoint{9.469451in}{1.748516in}}{\pgfqpoint{9.474288in}{1.750520in}}{\pgfqpoint{9.477855in}{1.754087in}}%
\pgfpathcurveto{\pgfqpoint{9.481421in}{1.757653in}}{\pgfqpoint{9.483425in}{1.762491in}}{\pgfqpoint{9.483425in}{1.767534in}}%
\pgfpathcurveto{\pgfqpoint{9.483425in}{1.772578in}}{\pgfqpoint{9.481421in}{1.777416in}}{\pgfqpoint{9.477855in}{1.780982in}}%
\pgfpathcurveto{\pgfqpoint{9.474288in}{1.784549in}}{\pgfqpoint{9.469451in}{1.786553in}}{\pgfqpoint{9.464407in}{1.786553in}}%
\pgfpathcurveto{\pgfqpoint{9.459363in}{1.786553in}}{\pgfqpoint{9.454526in}{1.784549in}}{\pgfqpoint{9.450959in}{1.780982in}}%
\pgfpathcurveto{\pgfqpoint{9.447393in}{1.777416in}}{\pgfqpoint{9.445389in}{1.772578in}}{\pgfqpoint{9.445389in}{1.767534in}}%
\pgfpathcurveto{\pgfqpoint{9.445389in}{1.762491in}}{\pgfqpoint{9.447393in}{1.757653in}}{\pgfqpoint{9.450959in}{1.754087in}}%
\pgfpathcurveto{\pgfqpoint{9.454526in}{1.750520in}}{\pgfqpoint{9.459363in}{1.748516in}}{\pgfqpoint{9.464407in}{1.748516in}}%
\pgfpathclose%
\pgfusepath{fill}%
\end{pgfscope}%
\begin{pgfscope}%
\pgfpathrectangle{\pgfqpoint{6.572727in}{0.474100in}}{\pgfqpoint{4.227273in}{3.318700in}}%
\pgfusepath{clip}%
\pgfsetbuttcap%
\pgfsetroundjoin%
\definecolor{currentfill}{rgb}{0.993248,0.906157,0.143936}%
\pgfsetfillcolor{currentfill}%
\pgfsetfillopacity{0.700000}%
\pgfsetlinewidth{0.000000pt}%
\definecolor{currentstroke}{rgb}{0.000000,0.000000,0.000000}%
\pgfsetstrokecolor{currentstroke}%
\pgfsetstrokeopacity{0.700000}%
\pgfsetdash{}{0pt}%
\pgfpathmoveto{\pgfqpoint{8.176808in}{2.543123in}}%
\pgfpathcurveto{\pgfqpoint{8.181852in}{2.543123in}}{\pgfqpoint{8.186689in}{2.545127in}}{\pgfqpoint{8.190256in}{2.548693in}}%
\pgfpathcurveto{\pgfqpoint{8.193822in}{2.552260in}}{\pgfqpoint{8.195826in}{2.557098in}}{\pgfqpoint{8.195826in}{2.562141in}}%
\pgfpathcurveto{\pgfqpoint{8.195826in}{2.567185in}}{\pgfqpoint{8.193822in}{2.572023in}}{\pgfqpoint{8.190256in}{2.575589in}}%
\pgfpathcurveto{\pgfqpoint{8.186689in}{2.579156in}}{\pgfqpoint{8.181852in}{2.581159in}}{\pgfqpoint{8.176808in}{2.581159in}}%
\pgfpathcurveto{\pgfqpoint{8.171764in}{2.581159in}}{\pgfqpoint{8.166926in}{2.579156in}}{\pgfqpoint{8.163360in}{2.575589in}}%
\pgfpathcurveto{\pgfqpoint{8.159794in}{2.572023in}}{\pgfqpoint{8.157790in}{2.567185in}}{\pgfqpoint{8.157790in}{2.562141in}}%
\pgfpathcurveto{\pgfqpoint{8.157790in}{2.557098in}}{\pgfqpoint{8.159794in}{2.552260in}}{\pgfqpoint{8.163360in}{2.548693in}}%
\pgfpathcurveto{\pgfqpoint{8.166926in}{2.545127in}}{\pgfqpoint{8.171764in}{2.543123in}}{\pgfqpoint{8.176808in}{2.543123in}}%
\pgfpathclose%
\pgfusepath{fill}%
\end{pgfscope}%
\begin{pgfscope}%
\pgfpathrectangle{\pgfqpoint{6.572727in}{0.474100in}}{\pgfqpoint{4.227273in}{3.318700in}}%
\pgfusepath{clip}%
\pgfsetbuttcap%
\pgfsetroundjoin%
\definecolor{currentfill}{rgb}{0.127568,0.566949,0.550556}%
\pgfsetfillcolor{currentfill}%
\pgfsetfillopacity{0.700000}%
\pgfsetlinewidth{0.000000pt}%
\definecolor{currentstroke}{rgb}{0.000000,0.000000,0.000000}%
\pgfsetstrokecolor{currentstroke}%
\pgfsetstrokeopacity{0.700000}%
\pgfsetdash{}{0pt}%
\pgfpathmoveto{\pgfqpoint{9.450364in}{1.850104in}}%
\pgfpathcurveto{\pgfqpoint{9.455408in}{1.850104in}}{\pgfqpoint{9.460246in}{1.852108in}}{\pgfqpoint{9.463812in}{1.855674in}}%
\pgfpathcurveto{\pgfqpoint{9.467379in}{1.859240in}}{\pgfqpoint{9.469382in}{1.864078in}}{\pgfqpoint{9.469382in}{1.869122in}}%
\pgfpathcurveto{\pgfqpoint{9.469382in}{1.874166in}}{\pgfqpoint{9.467379in}{1.879003in}}{\pgfqpoint{9.463812in}{1.882570in}}%
\pgfpathcurveto{\pgfqpoint{9.460246in}{1.886136in}}{\pgfqpoint{9.455408in}{1.888140in}}{\pgfqpoint{9.450364in}{1.888140in}}%
\pgfpathcurveto{\pgfqpoint{9.445321in}{1.888140in}}{\pgfqpoint{9.440483in}{1.886136in}}{\pgfqpoint{9.436916in}{1.882570in}}%
\pgfpathcurveto{\pgfqpoint{9.433350in}{1.879003in}}{\pgfqpoint{9.431346in}{1.874166in}}{\pgfqpoint{9.431346in}{1.869122in}}%
\pgfpathcurveto{\pgfqpoint{9.431346in}{1.864078in}}{\pgfqpoint{9.433350in}{1.859240in}}{\pgfqpoint{9.436916in}{1.855674in}}%
\pgfpathcurveto{\pgfqpoint{9.440483in}{1.852108in}}{\pgfqpoint{9.445321in}{1.850104in}}{\pgfqpoint{9.450364in}{1.850104in}}%
\pgfpathclose%
\pgfusepath{fill}%
\end{pgfscope}%
\begin{pgfscope}%
\pgfpathrectangle{\pgfqpoint{6.572727in}{0.474100in}}{\pgfqpoint{4.227273in}{3.318700in}}%
\pgfusepath{clip}%
\pgfsetbuttcap%
\pgfsetroundjoin%
\definecolor{currentfill}{rgb}{0.993248,0.906157,0.143936}%
\pgfsetfillcolor{currentfill}%
\pgfsetfillopacity{0.700000}%
\pgfsetlinewidth{0.000000pt}%
\definecolor{currentstroke}{rgb}{0.000000,0.000000,0.000000}%
\pgfsetstrokecolor{currentstroke}%
\pgfsetstrokeopacity{0.700000}%
\pgfsetdash{}{0pt}%
\pgfpathmoveto{\pgfqpoint{9.039734in}{3.262138in}}%
\pgfpathcurveto{\pgfqpoint{9.044777in}{3.262138in}}{\pgfqpoint{9.049615in}{3.264142in}}{\pgfqpoint{9.053182in}{3.267708in}}%
\pgfpathcurveto{\pgfqpoint{9.056748in}{3.271274in}}{\pgfqpoint{9.058752in}{3.276112in}}{\pgfqpoint{9.058752in}{3.281156in}}%
\pgfpathcurveto{\pgfqpoint{9.058752in}{3.286200in}}{\pgfqpoint{9.056748in}{3.291037in}}{\pgfqpoint{9.053182in}{3.294604in}}%
\pgfpathcurveto{\pgfqpoint{9.049615in}{3.298170in}}{\pgfqpoint{9.044777in}{3.300174in}}{\pgfqpoint{9.039734in}{3.300174in}}%
\pgfpathcurveto{\pgfqpoint{9.034690in}{3.300174in}}{\pgfqpoint{9.029852in}{3.298170in}}{\pgfqpoint{9.026286in}{3.294604in}}%
\pgfpathcurveto{\pgfqpoint{9.022720in}{3.291037in}}{\pgfqpoint{9.020716in}{3.286200in}}{\pgfqpoint{9.020716in}{3.281156in}}%
\pgfpathcurveto{\pgfqpoint{9.020716in}{3.276112in}}{\pgfqpoint{9.022720in}{3.271274in}}{\pgfqpoint{9.026286in}{3.267708in}}%
\pgfpathcurveto{\pgfqpoint{9.029852in}{3.264142in}}{\pgfqpoint{9.034690in}{3.262138in}}{\pgfqpoint{9.039734in}{3.262138in}}%
\pgfpathclose%
\pgfusepath{fill}%
\end{pgfscope}%
\begin{pgfscope}%
\pgfpathrectangle{\pgfqpoint{6.572727in}{0.474100in}}{\pgfqpoint{4.227273in}{3.318700in}}%
\pgfusepath{clip}%
\pgfsetbuttcap%
\pgfsetroundjoin%
\definecolor{currentfill}{rgb}{0.267004,0.004874,0.329415}%
\pgfsetfillcolor{currentfill}%
\pgfsetfillopacity{0.700000}%
\pgfsetlinewidth{0.000000pt}%
\definecolor{currentstroke}{rgb}{0.000000,0.000000,0.000000}%
\pgfsetstrokecolor{currentstroke}%
\pgfsetstrokeopacity{0.700000}%
\pgfsetdash{}{0pt}%
\pgfpathmoveto{\pgfqpoint{8.006949in}{1.357047in}}%
\pgfpathcurveto{\pgfqpoint{8.011993in}{1.357047in}}{\pgfqpoint{8.016831in}{1.359051in}}{\pgfqpoint{8.020397in}{1.362618in}}%
\pgfpathcurveto{\pgfqpoint{8.023963in}{1.366184in}}{\pgfqpoint{8.025967in}{1.371022in}}{\pgfqpoint{8.025967in}{1.376066in}}%
\pgfpathcurveto{\pgfqpoint{8.025967in}{1.381109in}}{\pgfqpoint{8.023963in}{1.385947in}}{\pgfqpoint{8.020397in}{1.389513in}}%
\pgfpathcurveto{\pgfqpoint{8.016831in}{1.393080in}}{\pgfqpoint{8.011993in}{1.395084in}}{\pgfqpoint{8.006949in}{1.395084in}}%
\pgfpathcurveto{\pgfqpoint{8.001905in}{1.395084in}}{\pgfqpoint{7.997068in}{1.393080in}}{\pgfqpoint{7.993501in}{1.389513in}}%
\pgfpathcurveto{\pgfqpoint{7.989935in}{1.385947in}}{\pgfqpoint{7.987931in}{1.381109in}}{\pgfqpoint{7.987931in}{1.376066in}}%
\pgfpathcurveto{\pgfqpoint{7.987931in}{1.371022in}}{\pgfqpoint{7.989935in}{1.366184in}}{\pgfqpoint{7.993501in}{1.362618in}}%
\pgfpathcurveto{\pgfqpoint{7.997068in}{1.359051in}}{\pgfqpoint{8.001905in}{1.357047in}}{\pgfqpoint{8.006949in}{1.357047in}}%
\pgfpathclose%
\pgfusepath{fill}%
\end{pgfscope}%
\begin{pgfscope}%
\pgfpathrectangle{\pgfqpoint{6.572727in}{0.474100in}}{\pgfqpoint{4.227273in}{3.318700in}}%
\pgfusepath{clip}%
\pgfsetbuttcap%
\pgfsetroundjoin%
\definecolor{currentfill}{rgb}{0.267004,0.004874,0.329415}%
\pgfsetfillcolor{currentfill}%
\pgfsetfillopacity{0.700000}%
\pgfsetlinewidth{0.000000pt}%
\definecolor{currentstroke}{rgb}{0.000000,0.000000,0.000000}%
\pgfsetstrokecolor{currentstroke}%
\pgfsetstrokeopacity{0.700000}%
\pgfsetdash{}{0pt}%
\pgfpathmoveto{\pgfqpoint{7.678302in}{0.972630in}}%
\pgfpathcurveto{\pgfqpoint{7.683346in}{0.972630in}}{\pgfqpoint{7.688184in}{0.974634in}}{\pgfqpoint{7.691750in}{0.978200in}}%
\pgfpathcurveto{\pgfqpoint{7.695317in}{0.981767in}}{\pgfqpoint{7.697321in}{0.986604in}}{\pgfqpoint{7.697321in}{0.991648in}}%
\pgfpathcurveto{\pgfqpoint{7.697321in}{0.996692in}}{\pgfqpoint{7.695317in}{1.001529in}}{\pgfqpoint{7.691750in}{1.005096in}}%
\pgfpathcurveto{\pgfqpoint{7.688184in}{1.008662in}}{\pgfqpoint{7.683346in}{1.010666in}}{\pgfqpoint{7.678302in}{1.010666in}}%
\pgfpathcurveto{\pgfqpoint{7.673259in}{1.010666in}}{\pgfqpoint{7.668421in}{1.008662in}}{\pgfqpoint{7.664855in}{1.005096in}}%
\pgfpathcurveto{\pgfqpoint{7.661288in}{1.001529in}}{\pgfqpoint{7.659284in}{0.996692in}}{\pgfqpoint{7.659284in}{0.991648in}}%
\pgfpathcurveto{\pgfqpoint{7.659284in}{0.986604in}}{\pgfqpoint{7.661288in}{0.981767in}}{\pgfqpoint{7.664855in}{0.978200in}}%
\pgfpathcurveto{\pgfqpoint{7.668421in}{0.974634in}}{\pgfqpoint{7.673259in}{0.972630in}}{\pgfqpoint{7.678302in}{0.972630in}}%
\pgfpathclose%
\pgfusepath{fill}%
\end{pgfscope}%
\begin{pgfscope}%
\pgfpathrectangle{\pgfqpoint{6.572727in}{0.474100in}}{\pgfqpoint{4.227273in}{3.318700in}}%
\pgfusepath{clip}%
\pgfsetbuttcap%
\pgfsetroundjoin%
\definecolor{currentfill}{rgb}{0.127568,0.566949,0.550556}%
\pgfsetfillcolor{currentfill}%
\pgfsetfillopacity{0.700000}%
\pgfsetlinewidth{0.000000pt}%
\definecolor{currentstroke}{rgb}{0.000000,0.000000,0.000000}%
\pgfsetstrokecolor{currentstroke}%
\pgfsetstrokeopacity{0.700000}%
\pgfsetdash{}{0pt}%
\pgfpathmoveto{\pgfqpoint{9.677000in}{1.228677in}}%
\pgfpathcurveto{\pgfqpoint{9.682043in}{1.228677in}}{\pgfqpoint{9.686881in}{1.230680in}}{\pgfqpoint{9.690448in}{1.234247in}}%
\pgfpathcurveto{\pgfqpoint{9.694014in}{1.237813in}}{\pgfqpoint{9.696018in}{1.242651in}}{\pgfqpoint{9.696018in}{1.247695in}}%
\pgfpathcurveto{\pgfqpoint{9.696018in}{1.252738in}}{\pgfqpoint{9.694014in}{1.257576in}}{\pgfqpoint{9.690448in}{1.261143in}}%
\pgfpathcurveto{\pgfqpoint{9.686881in}{1.264709in}}{\pgfqpoint{9.682043in}{1.266713in}}{\pgfqpoint{9.677000in}{1.266713in}}%
\pgfpathcurveto{\pgfqpoint{9.671956in}{1.266713in}}{\pgfqpoint{9.667118in}{1.264709in}}{\pgfqpoint{9.663552in}{1.261143in}}%
\pgfpathcurveto{\pgfqpoint{9.659985in}{1.257576in}}{\pgfqpoint{9.657982in}{1.252738in}}{\pgfqpoint{9.657982in}{1.247695in}}%
\pgfpathcurveto{\pgfqpoint{9.657982in}{1.242651in}}{\pgfqpoint{9.659985in}{1.237813in}}{\pgfqpoint{9.663552in}{1.234247in}}%
\pgfpathcurveto{\pgfqpoint{9.667118in}{1.230680in}}{\pgfqpoint{9.671956in}{1.228677in}}{\pgfqpoint{9.677000in}{1.228677in}}%
\pgfpathclose%
\pgfusepath{fill}%
\end{pgfscope}%
\begin{pgfscope}%
\pgfpathrectangle{\pgfqpoint{6.572727in}{0.474100in}}{\pgfqpoint{4.227273in}{3.318700in}}%
\pgfusepath{clip}%
\pgfsetbuttcap%
\pgfsetroundjoin%
\definecolor{currentfill}{rgb}{0.267004,0.004874,0.329415}%
\pgfsetfillcolor{currentfill}%
\pgfsetfillopacity{0.700000}%
\pgfsetlinewidth{0.000000pt}%
\definecolor{currentstroke}{rgb}{0.000000,0.000000,0.000000}%
\pgfsetstrokecolor{currentstroke}%
\pgfsetstrokeopacity{0.700000}%
\pgfsetdash{}{0pt}%
\pgfpathmoveto{\pgfqpoint{7.666068in}{1.317929in}}%
\pgfpathcurveto{\pgfqpoint{7.671112in}{1.317929in}}{\pgfqpoint{7.675950in}{1.319932in}}{\pgfqpoint{7.679516in}{1.323499in}}%
\pgfpathcurveto{\pgfqpoint{7.683083in}{1.327065in}}{\pgfqpoint{7.685086in}{1.331903in}}{\pgfqpoint{7.685086in}{1.336947in}}%
\pgfpathcurveto{\pgfqpoint{7.685086in}{1.341990in}}{\pgfqpoint{7.683083in}{1.346828in}}{\pgfqpoint{7.679516in}{1.350395in}}%
\pgfpathcurveto{\pgfqpoint{7.675950in}{1.353961in}}{\pgfqpoint{7.671112in}{1.355965in}}{\pgfqpoint{7.666068in}{1.355965in}}%
\pgfpathcurveto{\pgfqpoint{7.661025in}{1.355965in}}{\pgfqpoint{7.656187in}{1.353961in}}{\pgfqpoint{7.652620in}{1.350395in}}%
\pgfpathcurveto{\pgfqpoint{7.649054in}{1.346828in}}{\pgfqpoint{7.647050in}{1.341990in}}{\pgfqpoint{7.647050in}{1.336947in}}%
\pgfpathcurveto{\pgfqpoint{7.647050in}{1.331903in}}{\pgfqpoint{7.649054in}{1.327065in}}{\pgfqpoint{7.652620in}{1.323499in}}%
\pgfpathcurveto{\pgfqpoint{7.656187in}{1.319932in}}{\pgfqpoint{7.661025in}{1.317929in}}{\pgfqpoint{7.666068in}{1.317929in}}%
\pgfpathclose%
\pgfusepath{fill}%
\end{pgfscope}%
\begin{pgfscope}%
\pgfpathrectangle{\pgfqpoint{6.572727in}{0.474100in}}{\pgfqpoint{4.227273in}{3.318700in}}%
\pgfusepath{clip}%
\pgfsetbuttcap%
\pgfsetroundjoin%
\definecolor{currentfill}{rgb}{0.127568,0.566949,0.550556}%
\pgfsetfillcolor{currentfill}%
\pgfsetfillopacity{0.700000}%
\pgfsetlinewidth{0.000000pt}%
\definecolor{currentstroke}{rgb}{0.000000,0.000000,0.000000}%
\pgfsetstrokecolor{currentstroke}%
\pgfsetstrokeopacity{0.700000}%
\pgfsetdash{}{0pt}%
\pgfpathmoveto{\pgfqpoint{9.636833in}{1.728385in}}%
\pgfpathcurveto{\pgfqpoint{9.641876in}{1.728385in}}{\pgfqpoint{9.646714in}{1.730389in}}{\pgfqpoint{9.650280in}{1.733955in}}%
\pgfpathcurveto{\pgfqpoint{9.653847in}{1.737522in}}{\pgfqpoint{9.655851in}{1.742359in}}{\pgfqpoint{9.655851in}{1.747403in}}%
\pgfpathcurveto{\pgfqpoint{9.655851in}{1.752447in}}{\pgfqpoint{9.653847in}{1.757285in}}{\pgfqpoint{9.650280in}{1.760851in}}%
\pgfpathcurveto{\pgfqpoint{9.646714in}{1.764417in}}{\pgfqpoint{9.641876in}{1.766421in}}{\pgfqpoint{9.636833in}{1.766421in}}%
\pgfpathcurveto{\pgfqpoint{9.631789in}{1.766421in}}{\pgfqpoint{9.626951in}{1.764417in}}{\pgfqpoint{9.623385in}{1.760851in}}%
\pgfpathcurveto{\pgfqpoint{9.619818in}{1.757285in}}{\pgfqpoint{9.617814in}{1.752447in}}{\pgfqpoint{9.617814in}{1.747403in}}%
\pgfpathcurveto{\pgfqpoint{9.617814in}{1.742359in}}{\pgfqpoint{9.619818in}{1.737522in}}{\pgfqpoint{9.623385in}{1.733955in}}%
\pgfpathcurveto{\pgfqpoint{9.626951in}{1.730389in}}{\pgfqpoint{9.631789in}{1.728385in}}{\pgfqpoint{9.636833in}{1.728385in}}%
\pgfpathclose%
\pgfusepath{fill}%
\end{pgfscope}%
\begin{pgfscope}%
\pgfpathrectangle{\pgfqpoint{6.572727in}{0.474100in}}{\pgfqpoint{4.227273in}{3.318700in}}%
\pgfusepath{clip}%
\pgfsetbuttcap%
\pgfsetroundjoin%
\definecolor{currentfill}{rgb}{0.127568,0.566949,0.550556}%
\pgfsetfillcolor{currentfill}%
\pgfsetfillopacity{0.700000}%
\pgfsetlinewidth{0.000000pt}%
\definecolor{currentstroke}{rgb}{0.000000,0.000000,0.000000}%
\pgfsetstrokecolor{currentstroke}%
\pgfsetstrokeopacity{0.700000}%
\pgfsetdash{}{0pt}%
\pgfpathmoveto{\pgfqpoint{9.946611in}{1.902850in}}%
\pgfpathcurveto{\pgfqpoint{9.951655in}{1.902850in}}{\pgfqpoint{9.956493in}{1.904853in}}{\pgfqpoint{9.960059in}{1.908420in}}%
\pgfpathcurveto{\pgfqpoint{9.963625in}{1.911986in}}{\pgfqpoint{9.965629in}{1.916824in}}{\pgfqpoint{9.965629in}{1.921868in}}%
\pgfpathcurveto{\pgfqpoint{9.965629in}{1.926911in}}{\pgfqpoint{9.963625in}{1.931749in}}{\pgfqpoint{9.960059in}{1.935316in}}%
\pgfpathcurveto{\pgfqpoint{9.956493in}{1.938882in}}{\pgfqpoint{9.951655in}{1.940886in}}{\pgfqpoint{9.946611in}{1.940886in}}%
\pgfpathcurveto{\pgfqpoint{9.941568in}{1.940886in}}{\pgfqpoint{9.936730in}{1.938882in}}{\pgfqpoint{9.933163in}{1.935316in}}%
\pgfpathcurveto{\pgfqpoint{9.929597in}{1.931749in}}{\pgfqpoint{9.927593in}{1.926911in}}{\pgfqpoint{9.927593in}{1.921868in}}%
\pgfpathcurveto{\pgfqpoint{9.927593in}{1.916824in}}{\pgfqpoint{9.929597in}{1.911986in}}{\pgfqpoint{9.933163in}{1.908420in}}%
\pgfpathcurveto{\pgfqpoint{9.936730in}{1.904853in}}{\pgfqpoint{9.941568in}{1.902850in}}{\pgfqpoint{9.946611in}{1.902850in}}%
\pgfpathclose%
\pgfusepath{fill}%
\end{pgfscope}%
\begin{pgfscope}%
\pgfpathrectangle{\pgfqpoint{6.572727in}{0.474100in}}{\pgfqpoint{4.227273in}{3.318700in}}%
\pgfusepath{clip}%
\pgfsetbuttcap%
\pgfsetroundjoin%
\definecolor{currentfill}{rgb}{0.267004,0.004874,0.329415}%
\pgfsetfillcolor{currentfill}%
\pgfsetfillopacity{0.700000}%
\pgfsetlinewidth{0.000000pt}%
\definecolor{currentstroke}{rgb}{0.000000,0.000000,0.000000}%
\pgfsetstrokecolor{currentstroke}%
\pgfsetstrokeopacity{0.700000}%
\pgfsetdash{}{0pt}%
\pgfpathmoveto{\pgfqpoint{8.294514in}{1.454814in}}%
\pgfpathcurveto{\pgfqpoint{8.299558in}{1.454814in}}{\pgfqpoint{8.304396in}{1.456818in}}{\pgfqpoint{8.307962in}{1.460385in}}%
\pgfpathcurveto{\pgfqpoint{8.311528in}{1.463951in}}{\pgfqpoint{8.313532in}{1.468789in}}{\pgfqpoint{8.313532in}{1.473833in}}%
\pgfpathcurveto{\pgfqpoint{8.313532in}{1.478876in}}{\pgfqpoint{8.311528in}{1.483714in}}{\pgfqpoint{8.307962in}{1.487280in}}%
\pgfpathcurveto{\pgfqpoint{8.304396in}{1.490847in}}{\pgfqpoint{8.299558in}{1.492851in}}{\pgfqpoint{8.294514in}{1.492851in}}%
\pgfpathcurveto{\pgfqpoint{8.289470in}{1.492851in}}{\pgfqpoint{8.284633in}{1.490847in}}{\pgfqpoint{8.281066in}{1.487280in}}%
\pgfpathcurveto{\pgfqpoint{8.277500in}{1.483714in}}{\pgfqpoint{8.275496in}{1.478876in}}{\pgfqpoint{8.275496in}{1.473833in}}%
\pgfpathcurveto{\pgfqpoint{8.275496in}{1.468789in}}{\pgfqpoint{8.277500in}{1.463951in}}{\pgfqpoint{8.281066in}{1.460385in}}%
\pgfpathcurveto{\pgfqpoint{8.284633in}{1.456818in}}{\pgfqpoint{8.289470in}{1.454814in}}{\pgfqpoint{8.294514in}{1.454814in}}%
\pgfpathclose%
\pgfusepath{fill}%
\end{pgfscope}%
\begin{pgfscope}%
\pgfpathrectangle{\pgfqpoint{6.572727in}{0.474100in}}{\pgfqpoint{4.227273in}{3.318700in}}%
\pgfusepath{clip}%
\pgfsetbuttcap%
\pgfsetroundjoin%
\definecolor{currentfill}{rgb}{0.267004,0.004874,0.329415}%
\pgfsetfillcolor{currentfill}%
\pgfsetfillopacity{0.700000}%
\pgfsetlinewidth{0.000000pt}%
\definecolor{currentstroke}{rgb}{0.000000,0.000000,0.000000}%
\pgfsetstrokecolor{currentstroke}%
\pgfsetstrokeopacity{0.700000}%
\pgfsetdash{}{0pt}%
\pgfpathmoveto{\pgfqpoint{7.598975in}{1.203723in}}%
\pgfpathcurveto{\pgfqpoint{7.604018in}{1.203723in}}{\pgfqpoint{7.608856in}{1.205727in}}{\pgfqpoint{7.612422in}{1.209293in}}%
\pgfpathcurveto{\pgfqpoint{7.615989in}{1.212859in}}{\pgfqpoint{7.617993in}{1.217697in}}{\pgfqpoint{7.617993in}{1.222741in}}%
\pgfpathcurveto{\pgfqpoint{7.617993in}{1.227784in}}{\pgfqpoint{7.615989in}{1.232622in}}{\pgfqpoint{7.612422in}{1.236189in}}%
\pgfpathcurveto{\pgfqpoint{7.608856in}{1.239755in}}{\pgfqpoint{7.604018in}{1.241759in}}{\pgfqpoint{7.598975in}{1.241759in}}%
\pgfpathcurveto{\pgfqpoint{7.593931in}{1.241759in}}{\pgfqpoint{7.589093in}{1.239755in}}{\pgfqpoint{7.585527in}{1.236189in}}%
\pgfpathcurveto{\pgfqpoint{7.581960in}{1.232622in}}{\pgfqpoint{7.579956in}{1.227784in}}{\pgfqpoint{7.579956in}{1.222741in}}%
\pgfpathcurveto{\pgfqpoint{7.579956in}{1.217697in}}{\pgfqpoint{7.581960in}{1.212859in}}{\pgfqpoint{7.585527in}{1.209293in}}%
\pgfpathcurveto{\pgfqpoint{7.589093in}{1.205727in}}{\pgfqpoint{7.593931in}{1.203723in}}{\pgfqpoint{7.598975in}{1.203723in}}%
\pgfpathclose%
\pgfusepath{fill}%
\end{pgfscope}%
\begin{pgfscope}%
\pgfpathrectangle{\pgfqpoint{6.572727in}{0.474100in}}{\pgfqpoint{4.227273in}{3.318700in}}%
\pgfusepath{clip}%
\pgfsetbuttcap%
\pgfsetroundjoin%
\definecolor{currentfill}{rgb}{0.993248,0.906157,0.143936}%
\pgfsetfillcolor{currentfill}%
\pgfsetfillopacity{0.700000}%
\pgfsetlinewidth{0.000000pt}%
\definecolor{currentstroke}{rgb}{0.000000,0.000000,0.000000}%
\pgfsetstrokecolor{currentstroke}%
\pgfsetstrokeopacity{0.700000}%
\pgfsetdash{}{0pt}%
\pgfpathmoveto{\pgfqpoint{7.973042in}{3.331380in}}%
\pgfpathcurveto{\pgfqpoint{7.978086in}{3.331380in}}{\pgfqpoint{7.982924in}{3.333384in}}{\pgfqpoint{7.986490in}{3.336950in}}%
\pgfpathcurveto{\pgfqpoint{7.990057in}{3.340516in}}{\pgfqpoint{7.992061in}{3.345354in}}{\pgfqpoint{7.992061in}{3.350398in}}%
\pgfpathcurveto{\pgfqpoint{7.992061in}{3.355441in}}{\pgfqpoint{7.990057in}{3.360279in}}{\pgfqpoint{7.986490in}{3.363846in}}%
\pgfpathcurveto{\pgfqpoint{7.982924in}{3.367412in}}{\pgfqpoint{7.978086in}{3.369416in}}{\pgfqpoint{7.973042in}{3.369416in}}%
\pgfpathcurveto{\pgfqpoint{7.967999in}{3.369416in}}{\pgfqpoint{7.963161in}{3.367412in}}{\pgfqpoint{7.959595in}{3.363846in}}%
\pgfpathcurveto{\pgfqpoint{7.956028in}{3.360279in}}{\pgfqpoint{7.954024in}{3.355441in}}{\pgfqpoint{7.954024in}{3.350398in}}%
\pgfpathcurveto{\pgfqpoint{7.954024in}{3.345354in}}{\pgfqpoint{7.956028in}{3.340516in}}{\pgfqpoint{7.959595in}{3.336950in}}%
\pgfpathcurveto{\pgfqpoint{7.963161in}{3.333384in}}{\pgfqpoint{7.967999in}{3.331380in}}{\pgfqpoint{7.973042in}{3.331380in}}%
\pgfpathclose%
\pgfusepath{fill}%
\end{pgfscope}%
\begin{pgfscope}%
\pgfpathrectangle{\pgfqpoint{6.572727in}{0.474100in}}{\pgfqpoint{4.227273in}{3.318700in}}%
\pgfusepath{clip}%
\pgfsetbuttcap%
\pgfsetroundjoin%
\definecolor{currentfill}{rgb}{0.267004,0.004874,0.329415}%
\pgfsetfillcolor{currentfill}%
\pgfsetfillopacity{0.700000}%
\pgfsetlinewidth{0.000000pt}%
\definecolor{currentstroke}{rgb}{0.000000,0.000000,0.000000}%
\pgfsetstrokecolor{currentstroke}%
\pgfsetstrokeopacity{0.700000}%
\pgfsetdash{}{0pt}%
\pgfpathmoveto{\pgfqpoint{7.046167in}{1.754201in}}%
\pgfpathcurveto{\pgfqpoint{7.051210in}{1.754201in}}{\pgfqpoint{7.056048in}{1.756205in}}{\pgfqpoint{7.059614in}{1.759771in}}%
\pgfpathcurveto{\pgfqpoint{7.063181in}{1.763338in}}{\pgfqpoint{7.065185in}{1.768175in}}{\pgfqpoint{7.065185in}{1.773219in}}%
\pgfpathcurveto{\pgfqpoint{7.065185in}{1.778263in}}{\pgfqpoint{7.063181in}{1.783101in}}{\pgfqpoint{7.059614in}{1.786667in}}%
\pgfpathcurveto{\pgfqpoint{7.056048in}{1.790233in}}{\pgfqpoint{7.051210in}{1.792237in}}{\pgfqpoint{7.046167in}{1.792237in}}%
\pgfpathcurveto{\pgfqpoint{7.041123in}{1.792237in}}{\pgfqpoint{7.036285in}{1.790233in}}{\pgfqpoint{7.032719in}{1.786667in}}%
\pgfpathcurveto{\pgfqpoint{7.029152in}{1.783101in}}{\pgfqpoint{7.027148in}{1.778263in}}{\pgfqpoint{7.027148in}{1.773219in}}%
\pgfpathcurveto{\pgfqpoint{7.027148in}{1.768175in}}{\pgfqpoint{7.029152in}{1.763338in}}{\pgfqpoint{7.032719in}{1.759771in}}%
\pgfpathcurveto{\pgfqpoint{7.036285in}{1.756205in}}{\pgfqpoint{7.041123in}{1.754201in}}{\pgfqpoint{7.046167in}{1.754201in}}%
\pgfpathclose%
\pgfusepath{fill}%
\end{pgfscope}%
\begin{pgfscope}%
\pgfpathrectangle{\pgfqpoint{6.572727in}{0.474100in}}{\pgfqpoint{4.227273in}{3.318700in}}%
\pgfusepath{clip}%
\pgfsetbuttcap%
\pgfsetroundjoin%
\definecolor{currentfill}{rgb}{0.993248,0.906157,0.143936}%
\pgfsetfillcolor{currentfill}%
\pgfsetfillopacity{0.700000}%
\pgfsetlinewidth{0.000000pt}%
\definecolor{currentstroke}{rgb}{0.000000,0.000000,0.000000}%
\pgfsetstrokecolor{currentstroke}%
\pgfsetstrokeopacity{0.700000}%
\pgfsetdash{}{0pt}%
\pgfpathmoveto{\pgfqpoint{8.564128in}{2.906799in}}%
\pgfpathcurveto{\pgfqpoint{8.569171in}{2.906799in}}{\pgfqpoint{8.574009in}{2.908803in}}{\pgfqpoint{8.577575in}{2.912370in}}%
\pgfpathcurveto{\pgfqpoint{8.581142in}{2.915936in}}{\pgfqpoint{8.583146in}{2.920774in}}{\pgfqpoint{8.583146in}{2.925817in}}%
\pgfpathcurveto{\pgfqpoint{8.583146in}{2.930861in}}{\pgfqpoint{8.581142in}{2.935699in}}{\pgfqpoint{8.577575in}{2.939265in}}%
\pgfpathcurveto{\pgfqpoint{8.574009in}{2.942832in}}{\pgfqpoint{8.569171in}{2.944836in}}{\pgfqpoint{8.564128in}{2.944836in}}%
\pgfpathcurveto{\pgfqpoint{8.559084in}{2.944836in}}{\pgfqpoint{8.554246in}{2.942832in}}{\pgfqpoint{8.550680in}{2.939265in}}%
\pgfpathcurveto{\pgfqpoint{8.547113in}{2.935699in}}{\pgfqpoint{8.545109in}{2.930861in}}{\pgfqpoint{8.545109in}{2.925817in}}%
\pgfpathcurveto{\pgfqpoint{8.545109in}{2.920774in}}{\pgfqpoint{8.547113in}{2.915936in}}{\pgfqpoint{8.550680in}{2.912370in}}%
\pgfpathcurveto{\pgfqpoint{8.554246in}{2.908803in}}{\pgfqpoint{8.559084in}{2.906799in}}{\pgfqpoint{8.564128in}{2.906799in}}%
\pgfpathclose%
\pgfusepath{fill}%
\end{pgfscope}%
\begin{pgfscope}%
\pgfpathrectangle{\pgfqpoint{6.572727in}{0.474100in}}{\pgfqpoint{4.227273in}{3.318700in}}%
\pgfusepath{clip}%
\pgfsetbuttcap%
\pgfsetroundjoin%
\definecolor{currentfill}{rgb}{0.127568,0.566949,0.550556}%
\pgfsetfillcolor{currentfill}%
\pgfsetfillopacity{0.700000}%
\pgfsetlinewidth{0.000000pt}%
\definecolor{currentstroke}{rgb}{0.000000,0.000000,0.000000}%
\pgfsetstrokecolor{currentstroke}%
\pgfsetstrokeopacity{0.700000}%
\pgfsetdash{}{0pt}%
\pgfpathmoveto{\pgfqpoint{10.011983in}{1.213553in}}%
\pgfpathcurveto{\pgfqpoint{10.017026in}{1.213553in}}{\pgfqpoint{10.021864in}{1.215557in}}{\pgfqpoint{10.025431in}{1.219123in}}%
\pgfpathcurveto{\pgfqpoint{10.028997in}{1.222690in}}{\pgfqpoint{10.031001in}{1.227528in}}{\pgfqpoint{10.031001in}{1.232571in}}%
\pgfpathcurveto{\pgfqpoint{10.031001in}{1.237615in}}{\pgfqpoint{10.028997in}{1.242453in}}{\pgfqpoint{10.025431in}{1.246019in}}%
\pgfpathcurveto{\pgfqpoint{10.021864in}{1.249586in}}{\pgfqpoint{10.017026in}{1.251589in}}{\pgfqpoint{10.011983in}{1.251589in}}%
\pgfpathcurveto{\pgfqpoint{10.006939in}{1.251589in}}{\pgfqpoint{10.002101in}{1.249586in}}{\pgfqpoint{9.998535in}{1.246019in}}%
\pgfpathcurveto{\pgfqpoint{9.994968in}{1.242453in}}{\pgfqpoint{9.992965in}{1.237615in}}{\pgfqpoint{9.992965in}{1.232571in}}%
\pgfpathcurveto{\pgfqpoint{9.992965in}{1.227528in}}{\pgfqpoint{9.994968in}{1.222690in}}{\pgfqpoint{9.998535in}{1.219123in}}%
\pgfpathcurveto{\pgfqpoint{10.002101in}{1.215557in}}{\pgfqpoint{10.006939in}{1.213553in}}{\pgfqpoint{10.011983in}{1.213553in}}%
\pgfpathclose%
\pgfusepath{fill}%
\end{pgfscope}%
\begin{pgfscope}%
\pgfpathrectangle{\pgfqpoint{6.572727in}{0.474100in}}{\pgfqpoint{4.227273in}{3.318700in}}%
\pgfusepath{clip}%
\pgfsetbuttcap%
\pgfsetroundjoin%
\definecolor{currentfill}{rgb}{0.127568,0.566949,0.550556}%
\pgfsetfillcolor{currentfill}%
\pgfsetfillopacity{0.700000}%
\pgfsetlinewidth{0.000000pt}%
\definecolor{currentstroke}{rgb}{0.000000,0.000000,0.000000}%
\pgfsetstrokecolor{currentstroke}%
\pgfsetstrokeopacity{0.700000}%
\pgfsetdash{}{0pt}%
\pgfpathmoveto{\pgfqpoint{9.948581in}{1.465199in}}%
\pgfpathcurveto{\pgfqpoint{9.953625in}{1.465199in}}{\pgfqpoint{9.958462in}{1.467202in}}{\pgfqpoint{9.962029in}{1.470769in}}%
\pgfpathcurveto{\pgfqpoint{9.965595in}{1.474335in}}{\pgfqpoint{9.967599in}{1.479173in}}{\pgfqpoint{9.967599in}{1.484217in}}%
\pgfpathcurveto{\pgfqpoint{9.967599in}{1.489260in}}{\pgfqpoint{9.965595in}{1.494098in}}{\pgfqpoint{9.962029in}{1.497665in}}%
\pgfpathcurveto{\pgfqpoint{9.958462in}{1.501231in}}{\pgfqpoint{9.953625in}{1.503235in}}{\pgfqpoint{9.948581in}{1.503235in}}%
\pgfpathcurveto{\pgfqpoint{9.943537in}{1.503235in}}{\pgfqpoint{9.938700in}{1.501231in}}{\pgfqpoint{9.935133in}{1.497665in}}%
\pgfpathcurveto{\pgfqpoint{9.931567in}{1.494098in}}{\pgfqpoint{9.929563in}{1.489260in}}{\pgfqpoint{9.929563in}{1.484217in}}%
\pgfpathcurveto{\pgfqpoint{9.929563in}{1.479173in}}{\pgfqpoint{9.931567in}{1.474335in}}{\pgfqpoint{9.935133in}{1.470769in}}%
\pgfpathcurveto{\pgfqpoint{9.938700in}{1.467202in}}{\pgfqpoint{9.943537in}{1.465199in}}{\pgfqpoint{9.948581in}{1.465199in}}%
\pgfpathclose%
\pgfusepath{fill}%
\end{pgfscope}%
\begin{pgfscope}%
\pgfpathrectangle{\pgfqpoint{6.572727in}{0.474100in}}{\pgfqpoint{4.227273in}{3.318700in}}%
\pgfusepath{clip}%
\pgfsetbuttcap%
\pgfsetroundjoin%
\definecolor{currentfill}{rgb}{0.993248,0.906157,0.143936}%
\pgfsetfillcolor{currentfill}%
\pgfsetfillopacity{0.700000}%
\pgfsetlinewidth{0.000000pt}%
\definecolor{currentstroke}{rgb}{0.000000,0.000000,0.000000}%
\pgfsetstrokecolor{currentstroke}%
\pgfsetstrokeopacity{0.700000}%
\pgfsetdash{}{0pt}%
\pgfpathmoveto{\pgfqpoint{8.444460in}{2.961181in}}%
\pgfpathcurveto{\pgfqpoint{8.449504in}{2.961181in}}{\pgfqpoint{8.454342in}{2.963185in}}{\pgfqpoint{8.457908in}{2.966751in}}%
\pgfpathcurveto{\pgfqpoint{8.461475in}{2.970317in}}{\pgfqpoint{8.463479in}{2.975155in}}{\pgfqpoint{8.463479in}{2.980199in}}%
\pgfpathcurveto{\pgfqpoint{8.463479in}{2.985243in}}{\pgfqpoint{8.461475in}{2.990080in}}{\pgfqpoint{8.457908in}{2.993647in}}%
\pgfpathcurveto{\pgfqpoint{8.454342in}{2.997213in}}{\pgfqpoint{8.449504in}{2.999217in}}{\pgfqpoint{8.444460in}{2.999217in}}%
\pgfpathcurveto{\pgfqpoint{8.439417in}{2.999217in}}{\pgfqpoint{8.434579in}{2.997213in}}{\pgfqpoint{8.431013in}{2.993647in}}%
\pgfpathcurveto{\pgfqpoint{8.427446in}{2.990080in}}{\pgfqpoint{8.425442in}{2.985243in}}{\pgfqpoint{8.425442in}{2.980199in}}%
\pgfpathcurveto{\pgfqpoint{8.425442in}{2.975155in}}{\pgfqpoint{8.427446in}{2.970317in}}{\pgfqpoint{8.431013in}{2.966751in}}%
\pgfpathcurveto{\pgfqpoint{8.434579in}{2.963185in}}{\pgfqpoint{8.439417in}{2.961181in}}{\pgfqpoint{8.444460in}{2.961181in}}%
\pgfpathclose%
\pgfusepath{fill}%
\end{pgfscope}%
\begin{pgfscope}%
\pgfpathrectangle{\pgfqpoint{6.572727in}{0.474100in}}{\pgfqpoint{4.227273in}{3.318700in}}%
\pgfusepath{clip}%
\pgfsetbuttcap%
\pgfsetroundjoin%
\definecolor{currentfill}{rgb}{0.267004,0.004874,0.329415}%
\pgfsetfillcolor{currentfill}%
\pgfsetfillopacity{0.700000}%
\pgfsetlinewidth{0.000000pt}%
\definecolor{currentstroke}{rgb}{0.000000,0.000000,0.000000}%
\pgfsetstrokecolor{currentstroke}%
\pgfsetstrokeopacity{0.700000}%
\pgfsetdash{}{0pt}%
\pgfpathmoveto{\pgfqpoint{7.196845in}{1.258952in}}%
\pgfpathcurveto{\pgfqpoint{7.201889in}{1.258952in}}{\pgfqpoint{7.206727in}{1.260956in}}{\pgfqpoint{7.210293in}{1.264522in}}%
\pgfpathcurveto{\pgfqpoint{7.213860in}{1.268089in}}{\pgfqpoint{7.215864in}{1.272926in}}{\pgfqpoint{7.215864in}{1.277970in}}%
\pgfpathcurveto{\pgfqpoint{7.215864in}{1.283014in}}{\pgfqpoint{7.213860in}{1.287851in}}{\pgfqpoint{7.210293in}{1.291418in}}%
\pgfpathcurveto{\pgfqpoint{7.206727in}{1.294984in}}{\pgfqpoint{7.201889in}{1.296988in}}{\pgfqpoint{7.196845in}{1.296988in}}%
\pgfpathcurveto{\pgfqpoint{7.191802in}{1.296988in}}{\pgfqpoint{7.186964in}{1.294984in}}{\pgfqpoint{7.183398in}{1.291418in}}%
\pgfpathcurveto{\pgfqpoint{7.179831in}{1.287851in}}{\pgfqpoint{7.177827in}{1.283014in}}{\pgfqpoint{7.177827in}{1.277970in}}%
\pgfpathcurveto{\pgfqpoint{7.177827in}{1.272926in}}{\pgfqpoint{7.179831in}{1.268089in}}{\pgfqpoint{7.183398in}{1.264522in}}%
\pgfpathcurveto{\pgfqpoint{7.186964in}{1.260956in}}{\pgfqpoint{7.191802in}{1.258952in}}{\pgfqpoint{7.196845in}{1.258952in}}%
\pgfpathclose%
\pgfusepath{fill}%
\end{pgfscope}%
\begin{pgfscope}%
\pgfpathrectangle{\pgfqpoint{6.572727in}{0.474100in}}{\pgfqpoint{4.227273in}{3.318700in}}%
\pgfusepath{clip}%
\pgfsetbuttcap%
\pgfsetroundjoin%
\definecolor{currentfill}{rgb}{0.993248,0.906157,0.143936}%
\pgfsetfillcolor{currentfill}%
\pgfsetfillopacity{0.700000}%
\pgfsetlinewidth{0.000000pt}%
\definecolor{currentstroke}{rgb}{0.000000,0.000000,0.000000}%
\pgfsetstrokecolor{currentstroke}%
\pgfsetstrokeopacity{0.700000}%
\pgfsetdash{}{0pt}%
\pgfpathmoveto{\pgfqpoint{8.261624in}{2.751186in}}%
\pgfpathcurveto{\pgfqpoint{8.266668in}{2.751186in}}{\pgfqpoint{8.271506in}{2.753190in}}{\pgfqpoint{8.275072in}{2.756756in}}%
\pgfpathcurveto{\pgfqpoint{8.278639in}{2.760322in}}{\pgfqpoint{8.280643in}{2.765160in}}{\pgfqpoint{8.280643in}{2.770204in}}%
\pgfpathcurveto{\pgfqpoint{8.280643in}{2.775248in}}{\pgfqpoint{8.278639in}{2.780085in}}{\pgfqpoint{8.275072in}{2.783652in}}%
\pgfpathcurveto{\pgfqpoint{8.271506in}{2.787218in}}{\pgfqpoint{8.266668in}{2.789222in}}{\pgfqpoint{8.261624in}{2.789222in}}%
\pgfpathcurveto{\pgfqpoint{8.256581in}{2.789222in}}{\pgfqpoint{8.251743in}{2.787218in}}{\pgfqpoint{8.248177in}{2.783652in}}%
\pgfpathcurveto{\pgfqpoint{8.244610in}{2.780085in}}{\pgfqpoint{8.242606in}{2.775248in}}{\pgfqpoint{8.242606in}{2.770204in}}%
\pgfpathcurveto{\pgfqpoint{8.242606in}{2.765160in}}{\pgfqpoint{8.244610in}{2.760322in}}{\pgfqpoint{8.248177in}{2.756756in}}%
\pgfpathcurveto{\pgfqpoint{8.251743in}{2.753190in}}{\pgfqpoint{8.256581in}{2.751186in}}{\pgfqpoint{8.261624in}{2.751186in}}%
\pgfpathclose%
\pgfusepath{fill}%
\end{pgfscope}%
\begin{pgfscope}%
\pgfpathrectangle{\pgfqpoint{6.572727in}{0.474100in}}{\pgfqpoint{4.227273in}{3.318700in}}%
\pgfusepath{clip}%
\pgfsetbuttcap%
\pgfsetroundjoin%
\definecolor{currentfill}{rgb}{0.127568,0.566949,0.550556}%
\pgfsetfillcolor{currentfill}%
\pgfsetfillopacity{0.700000}%
\pgfsetlinewidth{0.000000pt}%
\definecolor{currentstroke}{rgb}{0.000000,0.000000,0.000000}%
\pgfsetstrokecolor{currentstroke}%
\pgfsetstrokeopacity{0.700000}%
\pgfsetdash{}{0pt}%
\pgfpathmoveto{\pgfqpoint{9.804221in}{1.719766in}}%
\pgfpathcurveto{\pgfqpoint{9.809265in}{1.719766in}}{\pgfqpoint{9.814103in}{1.721770in}}{\pgfqpoint{9.817669in}{1.725336in}}%
\pgfpathcurveto{\pgfqpoint{9.821236in}{1.728903in}}{\pgfqpoint{9.823239in}{1.733741in}}{\pgfqpoint{9.823239in}{1.738784in}}%
\pgfpathcurveto{\pgfqpoint{9.823239in}{1.743828in}}{\pgfqpoint{9.821236in}{1.748666in}}{\pgfqpoint{9.817669in}{1.752232in}}%
\pgfpathcurveto{\pgfqpoint{9.814103in}{1.755799in}}{\pgfqpoint{9.809265in}{1.757802in}}{\pgfqpoint{9.804221in}{1.757802in}}%
\pgfpathcurveto{\pgfqpoint{9.799178in}{1.757802in}}{\pgfqpoint{9.794340in}{1.755799in}}{\pgfqpoint{9.790773in}{1.752232in}}%
\pgfpathcurveto{\pgfqpoint{9.787207in}{1.748666in}}{\pgfqpoint{9.785203in}{1.743828in}}{\pgfqpoint{9.785203in}{1.738784in}}%
\pgfpathcurveto{\pgfqpoint{9.785203in}{1.733741in}}{\pgfqpoint{9.787207in}{1.728903in}}{\pgfqpoint{9.790773in}{1.725336in}}%
\pgfpathcurveto{\pgfqpoint{9.794340in}{1.721770in}}{\pgfqpoint{9.799178in}{1.719766in}}{\pgfqpoint{9.804221in}{1.719766in}}%
\pgfpathclose%
\pgfusepath{fill}%
\end{pgfscope}%
\begin{pgfscope}%
\pgfpathrectangle{\pgfqpoint{6.572727in}{0.474100in}}{\pgfqpoint{4.227273in}{3.318700in}}%
\pgfusepath{clip}%
\pgfsetbuttcap%
\pgfsetroundjoin%
\definecolor{currentfill}{rgb}{0.127568,0.566949,0.550556}%
\pgfsetfillcolor{currentfill}%
\pgfsetfillopacity{0.700000}%
\pgfsetlinewidth{0.000000pt}%
\definecolor{currentstroke}{rgb}{0.000000,0.000000,0.000000}%
\pgfsetstrokecolor{currentstroke}%
\pgfsetstrokeopacity{0.700000}%
\pgfsetdash{}{0pt}%
\pgfpathmoveto{\pgfqpoint{9.471324in}{1.421160in}}%
\pgfpathcurveto{\pgfqpoint{9.476368in}{1.421160in}}{\pgfqpoint{9.481206in}{1.423164in}}{\pgfqpoint{9.484772in}{1.426730in}}%
\pgfpathcurveto{\pgfqpoint{9.488339in}{1.430297in}}{\pgfqpoint{9.490343in}{1.435135in}}{\pgfqpoint{9.490343in}{1.440178in}}%
\pgfpathcurveto{\pgfqpoint{9.490343in}{1.445222in}}{\pgfqpoint{9.488339in}{1.450060in}}{\pgfqpoint{9.484772in}{1.453626in}}%
\pgfpathcurveto{\pgfqpoint{9.481206in}{1.457192in}}{\pgfqpoint{9.476368in}{1.459196in}}{\pgfqpoint{9.471324in}{1.459196in}}%
\pgfpathcurveto{\pgfqpoint{9.466281in}{1.459196in}}{\pgfqpoint{9.461443in}{1.457192in}}{\pgfqpoint{9.457877in}{1.453626in}}%
\pgfpathcurveto{\pgfqpoint{9.454310in}{1.450060in}}{\pgfqpoint{9.452306in}{1.445222in}}{\pgfqpoint{9.452306in}{1.440178in}}%
\pgfpathcurveto{\pgfqpoint{9.452306in}{1.435135in}}{\pgfqpoint{9.454310in}{1.430297in}}{\pgfqpoint{9.457877in}{1.426730in}}%
\pgfpathcurveto{\pgfqpoint{9.461443in}{1.423164in}}{\pgfqpoint{9.466281in}{1.421160in}}{\pgfqpoint{9.471324in}{1.421160in}}%
\pgfpathclose%
\pgfusepath{fill}%
\end{pgfscope}%
\begin{pgfscope}%
\pgfpathrectangle{\pgfqpoint{6.572727in}{0.474100in}}{\pgfqpoint{4.227273in}{3.318700in}}%
\pgfusepath{clip}%
\pgfsetbuttcap%
\pgfsetroundjoin%
\definecolor{currentfill}{rgb}{0.127568,0.566949,0.550556}%
\pgfsetfillcolor{currentfill}%
\pgfsetfillopacity{0.700000}%
\pgfsetlinewidth{0.000000pt}%
\definecolor{currentstroke}{rgb}{0.000000,0.000000,0.000000}%
\pgfsetstrokecolor{currentstroke}%
\pgfsetstrokeopacity{0.700000}%
\pgfsetdash{}{0pt}%
\pgfpathmoveto{\pgfqpoint{9.078876in}{2.326257in}}%
\pgfpathcurveto{\pgfqpoint{9.083920in}{2.326257in}}{\pgfqpoint{9.088758in}{2.328261in}}{\pgfqpoint{9.092324in}{2.331827in}}%
\pgfpathcurveto{\pgfqpoint{9.095891in}{2.335394in}}{\pgfqpoint{9.097894in}{2.340232in}}{\pgfqpoint{9.097894in}{2.345275in}}%
\pgfpathcurveto{\pgfqpoint{9.097894in}{2.350319in}}{\pgfqpoint{9.095891in}{2.355157in}}{\pgfqpoint{9.092324in}{2.358723in}}%
\pgfpathcurveto{\pgfqpoint{9.088758in}{2.362289in}}{\pgfqpoint{9.083920in}{2.364293in}}{\pgfqpoint{9.078876in}{2.364293in}}%
\pgfpathcurveto{\pgfqpoint{9.073833in}{2.364293in}}{\pgfqpoint{9.068995in}{2.362289in}}{\pgfqpoint{9.065428in}{2.358723in}}%
\pgfpathcurveto{\pgfqpoint{9.061862in}{2.355157in}}{\pgfqpoint{9.059858in}{2.350319in}}{\pgfqpoint{9.059858in}{2.345275in}}%
\pgfpathcurveto{\pgfqpoint{9.059858in}{2.340232in}}{\pgfqpoint{9.061862in}{2.335394in}}{\pgfqpoint{9.065428in}{2.331827in}}%
\pgfpathcurveto{\pgfqpoint{9.068995in}{2.328261in}}{\pgfqpoint{9.073833in}{2.326257in}}{\pgfqpoint{9.078876in}{2.326257in}}%
\pgfpathclose%
\pgfusepath{fill}%
\end{pgfscope}%
\begin{pgfscope}%
\pgfpathrectangle{\pgfqpoint{6.572727in}{0.474100in}}{\pgfqpoint{4.227273in}{3.318700in}}%
\pgfusepath{clip}%
\pgfsetbuttcap%
\pgfsetroundjoin%
\definecolor{currentfill}{rgb}{0.267004,0.004874,0.329415}%
\pgfsetfillcolor{currentfill}%
\pgfsetfillopacity{0.700000}%
\pgfsetlinewidth{0.000000pt}%
\definecolor{currentstroke}{rgb}{0.000000,0.000000,0.000000}%
\pgfsetstrokecolor{currentstroke}%
\pgfsetstrokeopacity{0.700000}%
\pgfsetdash{}{0pt}%
\pgfpathmoveto{\pgfqpoint{8.182845in}{1.344732in}}%
\pgfpathcurveto{\pgfqpoint{8.187889in}{1.344732in}}{\pgfqpoint{8.192727in}{1.346736in}}{\pgfqpoint{8.196293in}{1.350302in}}%
\pgfpathcurveto{\pgfqpoint{8.199860in}{1.353869in}}{\pgfqpoint{8.201863in}{1.358706in}}{\pgfqpoint{8.201863in}{1.363750in}}%
\pgfpathcurveto{\pgfqpoint{8.201863in}{1.368794in}}{\pgfqpoint{8.199860in}{1.373631in}}{\pgfqpoint{8.196293in}{1.377198in}}%
\pgfpathcurveto{\pgfqpoint{8.192727in}{1.380764in}}{\pgfqpoint{8.187889in}{1.382768in}}{\pgfqpoint{8.182845in}{1.382768in}}%
\pgfpathcurveto{\pgfqpoint{8.177802in}{1.382768in}}{\pgfqpoint{8.172964in}{1.380764in}}{\pgfqpoint{8.169397in}{1.377198in}}%
\pgfpathcurveto{\pgfqpoint{8.165831in}{1.373631in}}{\pgfqpoint{8.163827in}{1.368794in}}{\pgfqpoint{8.163827in}{1.363750in}}%
\pgfpathcurveto{\pgfqpoint{8.163827in}{1.358706in}}{\pgfqpoint{8.165831in}{1.353869in}}{\pgfqpoint{8.169397in}{1.350302in}}%
\pgfpathcurveto{\pgfqpoint{8.172964in}{1.346736in}}{\pgfqpoint{8.177802in}{1.344732in}}{\pgfqpoint{8.182845in}{1.344732in}}%
\pgfpathclose%
\pgfusepath{fill}%
\end{pgfscope}%
\begin{pgfscope}%
\pgfpathrectangle{\pgfqpoint{6.572727in}{0.474100in}}{\pgfqpoint{4.227273in}{3.318700in}}%
\pgfusepath{clip}%
\pgfsetbuttcap%
\pgfsetroundjoin%
\definecolor{currentfill}{rgb}{0.267004,0.004874,0.329415}%
\pgfsetfillcolor{currentfill}%
\pgfsetfillopacity{0.700000}%
\pgfsetlinewidth{0.000000pt}%
\definecolor{currentstroke}{rgb}{0.000000,0.000000,0.000000}%
\pgfsetstrokecolor{currentstroke}%
\pgfsetstrokeopacity{0.700000}%
\pgfsetdash{}{0pt}%
\pgfpathmoveto{\pgfqpoint{7.817637in}{1.871247in}}%
\pgfpathcurveto{\pgfqpoint{7.822681in}{1.871247in}}{\pgfqpoint{7.827519in}{1.873251in}}{\pgfqpoint{7.831085in}{1.876817in}}%
\pgfpathcurveto{\pgfqpoint{7.834652in}{1.880384in}}{\pgfqpoint{7.836655in}{1.885221in}}{\pgfqpoint{7.836655in}{1.890265in}}%
\pgfpathcurveto{\pgfqpoint{7.836655in}{1.895309in}}{\pgfqpoint{7.834652in}{1.900147in}}{\pgfqpoint{7.831085in}{1.903713in}}%
\pgfpathcurveto{\pgfqpoint{7.827519in}{1.907279in}}{\pgfqpoint{7.822681in}{1.909283in}}{\pgfqpoint{7.817637in}{1.909283in}}%
\pgfpathcurveto{\pgfqpoint{7.812594in}{1.909283in}}{\pgfqpoint{7.807756in}{1.907279in}}{\pgfqpoint{7.804189in}{1.903713in}}%
\pgfpathcurveto{\pgfqpoint{7.800623in}{1.900147in}}{\pgfqpoint{7.798619in}{1.895309in}}{\pgfqpoint{7.798619in}{1.890265in}}%
\pgfpathcurveto{\pgfqpoint{7.798619in}{1.885221in}}{\pgfqpoint{7.800623in}{1.880384in}}{\pgfqpoint{7.804189in}{1.876817in}}%
\pgfpathcurveto{\pgfqpoint{7.807756in}{1.873251in}}{\pgfqpoint{7.812594in}{1.871247in}}{\pgfqpoint{7.817637in}{1.871247in}}%
\pgfpathclose%
\pgfusepath{fill}%
\end{pgfscope}%
\begin{pgfscope}%
\pgfpathrectangle{\pgfqpoint{6.572727in}{0.474100in}}{\pgfqpoint{4.227273in}{3.318700in}}%
\pgfusepath{clip}%
\pgfsetbuttcap%
\pgfsetroundjoin%
\definecolor{currentfill}{rgb}{0.267004,0.004874,0.329415}%
\pgfsetfillcolor{currentfill}%
\pgfsetfillopacity{0.700000}%
\pgfsetlinewidth{0.000000pt}%
\definecolor{currentstroke}{rgb}{0.000000,0.000000,0.000000}%
\pgfsetstrokecolor{currentstroke}%
\pgfsetstrokeopacity{0.700000}%
\pgfsetdash{}{0pt}%
\pgfpathmoveto{\pgfqpoint{7.984190in}{0.946791in}}%
\pgfpathcurveto{\pgfqpoint{7.989234in}{0.946791in}}{\pgfqpoint{7.994072in}{0.948795in}}{\pgfqpoint{7.997638in}{0.952361in}}%
\pgfpathcurveto{\pgfqpoint{8.001204in}{0.955928in}}{\pgfqpoint{8.003208in}{0.960765in}}{\pgfqpoint{8.003208in}{0.965809in}}%
\pgfpathcurveto{\pgfqpoint{8.003208in}{0.970853in}}{\pgfqpoint{8.001204in}{0.975690in}}{\pgfqpoint{7.997638in}{0.979257in}}%
\pgfpathcurveto{\pgfqpoint{7.994072in}{0.982823in}}{\pgfqpoint{7.989234in}{0.984827in}}{\pgfqpoint{7.984190in}{0.984827in}}%
\pgfpathcurveto{\pgfqpoint{7.979146in}{0.984827in}}{\pgfqpoint{7.974309in}{0.982823in}}{\pgfqpoint{7.970742in}{0.979257in}}%
\pgfpathcurveto{\pgfqpoint{7.967176in}{0.975690in}}{\pgfqpoint{7.965172in}{0.970853in}}{\pgfqpoint{7.965172in}{0.965809in}}%
\pgfpathcurveto{\pgfqpoint{7.965172in}{0.960765in}}{\pgfqpoint{7.967176in}{0.955928in}}{\pgfqpoint{7.970742in}{0.952361in}}%
\pgfpathcurveto{\pgfqpoint{7.974309in}{0.948795in}}{\pgfqpoint{7.979146in}{0.946791in}}{\pgfqpoint{7.984190in}{0.946791in}}%
\pgfpathclose%
\pgfusepath{fill}%
\end{pgfscope}%
\begin{pgfscope}%
\pgfpathrectangle{\pgfqpoint{6.572727in}{0.474100in}}{\pgfqpoint{4.227273in}{3.318700in}}%
\pgfusepath{clip}%
\pgfsetbuttcap%
\pgfsetroundjoin%
\definecolor{currentfill}{rgb}{0.993248,0.906157,0.143936}%
\pgfsetfillcolor{currentfill}%
\pgfsetfillopacity{0.700000}%
\pgfsetlinewidth{0.000000pt}%
\definecolor{currentstroke}{rgb}{0.000000,0.000000,0.000000}%
\pgfsetstrokecolor{currentstroke}%
\pgfsetstrokeopacity{0.700000}%
\pgfsetdash{}{0pt}%
\pgfpathmoveto{\pgfqpoint{8.648901in}{3.033687in}}%
\pgfpathcurveto{\pgfqpoint{8.653945in}{3.033687in}}{\pgfqpoint{8.658783in}{3.035691in}}{\pgfqpoint{8.662349in}{3.039257in}}%
\pgfpathcurveto{\pgfqpoint{8.665916in}{3.042824in}}{\pgfqpoint{8.667920in}{3.047661in}}{\pgfqpoint{8.667920in}{3.052705in}}%
\pgfpathcurveto{\pgfqpoint{8.667920in}{3.057749in}}{\pgfqpoint{8.665916in}{3.062586in}}{\pgfqpoint{8.662349in}{3.066153in}}%
\pgfpathcurveto{\pgfqpoint{8.658783in}{3.069719in}}{\pgfqpoint{8.653945in}{3.071723in}}{\pgfqpoint{8.648901in}{3.071723in}}%
\pgfpathcurveto{\pgfqpoint{8.643858in}{3.071723in}}{\pgfqpoint{8.639020in}{3.069719in}}{\pgfqpoint{8.635454in}{3.066153in}}%
\pgfpathcurveto{\pgfqpoint{8.631887in}{3.062586in}}{\pgfqpoint{8.629883in}{3.057749in}}{\pgfqpoint{8.629883in}{3.052705in}}%
\pgfpathcurveto{\pgfqpoint{8.629883in}{3.047661in}}{\pgfqpoint{8.631887in}{3.042824in}}{\pgfqpoint{8.635454in}{3.039257in}}%
\pgfpathcurveto{\pgfqpoint{8.639020in}{3.035691in}}{\pgfqpoint{8.643858in}{3.033687in}}{\pgfqpoint{8.648901in}{3.033687in}}%
\pgfpathclose%
\pgfusepath{fill}%
\end{pgfscope}%
\begin{pgfscope}%
\pgfpathrectangle{\pgfqpoint{6.572727in}{0.474100in}}{\pgfqpoint{4.227273in}{3.318700in}}%
\pgfusepath{clip}%
\pgfsetbuttcap%
\pgfsetroundjoin%
\definecolor{currentfill}{rgb}{0.993248,0.906157,0.143936}%
\pgfsetfillcolor{currentfill}%
\pgfsetfillopacity{0.700000}%
\pgfsetlinewidth{0.000000pt}%
\definecolor{currentstroke}{rgb}{0.000000,0.000000,0.000000}%
\pgfsetstrokecolor{currentstroke}%
\pgfsetstrokeopacity{0.700000}%
\pgfsetdash{}{0pt}%
\pgfpathmoveto{\pgfqpoint{7.866339in}{2.696607in}}%
\pgfpathcurveto{\pgfqpoint{7.871383in}{2.696607in}}{\pgfqpoint{7.876221in}{2.698611in}}{\pgfqpoint{7.879787in}{2.702177in}}%
\pgfpathcurveto{\pgfqpoint{7.883354in}{2.705744in}}{\pgfqpoint{7.885357in}{2.710582in}}{\pgfqpoint{7.885357in}{2.715625in}}%
\pgfpathcurveto{\pgfqpoint{7.885357in}{2.720669in}}{\pgfqpoint{7.883354in}{2.725507in}}{\pgfqpoint{7.879787in}{2.729073in}}%
\pgfpathcurveto{\pgfqpoint{7.876221in}{2.732639in}}{\pgfqpoint{7.871383in}{2.734643in}}{\pgfqpoint{7.866339in}{2.734643in}}%
\pgfpathcurveto{\pgfqpoint{7.861296in}{2.734643in}}{\pgfqpoint{7.856458in}{2.732639in}}{\pgfqpoint{7.852891in}{2.729073in}}%
\pgfpathcurveto{\pgfqpoint{7.849325in}{2.725507in}}{\pgfqpoint{7.847321in}{2.720669in}}{\pgfqpoint{7.847321in}{2.715625in}}%
\pgfpathcurveto{\pgfqpoint{7.847321in}{2.710582in}}{\pgfqpoint{7.849325in}{2.705744in}}{\pgfqpoint{7.852891in}{2.702177in}}%
\pgfpathcurveto{\pgfqpoint{7.856458in}{2.698611in}}{\pgfqpoint{7.861296in}{2.696607in}}{\pgfqpoint{7.866339in}{2.696607in}}%
\pgfpathclose%
\pgfusepath{fill}%
\end{pgfscope}%
\begin{pgfscope}%
\pgfpathrectangle{\pgfqpoint{6.572727in}{0.474100in}}{\pgfqpoint{4.227273in}{3.318700in}}%
\pgfusepath{clip}%
\pgfsetbuttcap%
\pgfsetroundjoin%
\definecolor{currentfill}{rgb}{0.993248,0.906157,0.143936}%
\pgfsetfillcolor{currentfill}%
\pgfsetfillopacity{0.700000}%
\pgfsetlinewidth{0.000000pt}%
\definecolor{currentstroke}{rgb}{0.000000,0.000000,0.000000}%
\pgfsetstrokecolor{currentstroke}%
\pgfsetstrokeopacity{0.700000}%
\pgfsetdash{}{0pt}%
\pgfpathmoveto{\pgfqpoint{8.411086in}{2.846783in}}%
\pgfpathcurveto{\pgfqpoint{8.416129in}{2.846783in}}{\pgfqpoint{8.420967in}{2.848787in}}{\pgfqpoint{8.424534in}{2.852353in}}%
\pgfpathcurveto{\pgfqpoint{8.428100in}{2.855919in}}{\pgfqpoint{8.430104in}{2.860757in}}{\pgfqpoint{8.430104in}{2.865801in}}%
\pgfpathcurveto{\pgfqpoint{8.430104in}{2.870845in}}{\pgfqpoint{8.428100in}{2.875682in}}{\pgfqpoint{8.424534in}{2.879249in}}%
\pgfpathcurveto{\pgfqpoint{8.420967in}{2.882815in}}{\pgfqpoint{8.416129in}{2.884819in}}{\pgfqpoint{8.411086in}{2.884819in}}%
\pgfpathcurveto{\pgfqpoint{8.406042in}{2.884819in}}{\pgfqpoint{8.401204in}{2.882815in}}{\pgfqpoint{8.397638in}{2.879249in}}%
\pgfpathcurveto{\pgfqpoint{8.394071in}{2.875682in}}{\pgfqpoint{8.392068in}{2.870845in}}{\pgfqpoint{8.392068in}{2.865801in}}%
\pgfpathcurveto{\pgfqpoint{8.392068in}{2.860757in}}{\pgfqpoint{8.394071in}{2.855919in}}{\pgfqpoint{8.397638in}{2.852353in}}%
\pgfpathcurveto{\pgfqpoint{8.401204in}{2.848787in}}{\pgfqpoint{8.406042in}{2.846783in}}{\pgfqpoint{8.411086in}{2.846783in}}%
\pgfpathclose%
\pgfusepath{fill}%
\end{pgfscope}%
\begin{pgfscope}%
\pgfpathrectangle{\pgfqpoint{6.572727in}{0.474100in}}{\pgfqpoint{4.227273in}{3.318700in}}%
\pgfusepath{clip}%
\pgfsetbuttcap%
\pgfsetroundjoin%
\definecolor{currentfill}{rgb}{0.267004,0.004874,0.329415}%
\pgfsetfillcolor{currentfill}%
\pgfsetfillopacity{0.700000}%
\pgfsetlinewidth{0.000000pt}%
\definecolor{currentstroke}{rgb}{0.000000,0.000000,0.000000}%
\pgfsetstrokecolor{currentstroke}%
\pgfsetstrokeopacity{0.700000}%
\pgfsetdash{}{0pt}%
\pgfpathmoveto{\pgfqpoint{7.301251in}{1.737872in}}%
\pgfpathcurveto{\pgfqpoint{7.306294in}{1.737872in}}{\pgfqpoint{7.311132in}{1.739876in}}{\pgfqpoint{7.314698in}{1.743442in}}%
\pgfpathcurveto{\pgfqpoint{7.318265in}{1.747008in}}{\pgfqpoint{7.320269in}{1.751846in}}{\pgfqpoint{7.320269in}{1.756890in}}%
\pgfpathcurveto{\pgfqpoint{7.320269in}{1.761934in}}{\pgfqpoint{7.318265in}{1.766771in}}{\pgfqpoint{7.314698in}{1.770338in}}%
\pgfpathcurveto{\pgfqpoint{7.311132in}{1.773904in}}{\pgfqpoint{7.306294in}{1.775908in}}{\pgfqpoint{7.301251in}{1.775908in}}%
\pgfpathcurveto{\pgfqpoint{7.296207in}{1.775908in}}{\pgfqpoint{7.291369in}{1.773904in}}{\pgfqpoint{7.287803in}{1.770338in}}%
\pgfpathcurveto{\pgfqpoint{7.284236in}{1.766771in}}{\pgfqpoint{7.282232in}{1.761934in}}{\pgfqpoint{7.282232in}{1.756890in}}%
\pgfpathcurveto{\pgfqpoint{7.282232in}{1.751846in}}{\pgfqpoint{7.284236in}{1.747008in}}{\pgfqpoint{7.287803in}{1.743442in}}%
\pgfpathcurveto{\pgfqpoint{7.291369in}{1.739876in}}{\pgfqpoint{7.296207in}{1.737872in}}{\pgfqpoint{7.301251in}{1.737872in}}%
\pgfpathclose%
\pgfusepath{fill}%
\end{pgfscope}%
\begin{pgfscope}%
\pgfpathrectangle{\pgfqpoint{6.572727in}{0.474100in}}{\pgfqpoint{4.227273in}{3.318700in}}%
\pgfusepath{clip}%
\pgfsetbuttcap%
\pgfsetroundjoin%
\definecolor{currentfill}{rgb}{0.127568,0.566949,0.550556}%
\pgfsetfillcolor{currentfill}%
\pgfsetfillopacity{0.700000}%
\pgfsetlinewidth{0.000000pt}%
\definecolor{currentstroke}{rgb}{0.000000,0.000000,0.000000}%
\pgfsetstrokecolor{currentstroke}%
\pgfsetstrokeopacity{0.700000}%
\pgfsetdash{}{0pt}%
\pgfpathmoveto{\pgfqpoint{9.701916in}{1.218985in}}%
\pgfpathcurveto{\pgfqpoint{9.706960in}{1.218985in}}{\pgfqpoint{9.711798in}{1.220989in}}{\pgfqpoint{9.715364in}{1.224556in}}%
\pgfpathcurveto{\pgfqpoint{9.718931in}{1.228122in}}{\pgfqpoint{9.720935in}{1.232960in}}{\pgfqpoint{9.720935in}{1.238003in}}%
\pgfpathcurveto{\pgfqpoint{9.720935in}{1.243047in}}{\pgfqpoint{9.718931in}{1.247885in}}{\pgfqpoint{9.715364in}{1.251451in}}%
\pgfpathcurveto{\pgfqpoint{9.711798in}{1.255018in}}{\pgfqpoint{9.706960in}{1.257022in}}{\pgfqpoint{9.701916in}{1.257022in}}%
\pgfpathcurveto{\pgfqpoint{9.696873in}{1.257022in}}{\pgfqpoint{9.692035in}{1.255018in}}{\pgfqpoint{9.688469in}{1.251451in}}%
\pgfpathcurveto{\pgfqpoint{9.684902in}{1.247885in}}{\pgfqpoint{9.682898in}{1.243047in}}{\pgfqpoint{9.682898in}{1.238003in}}%
\pgfpathcurveto{\pgfqpoint{9.682898in}{1.232960in}}{\pgfqpoint{9.684902in}{1.228122in}}{\pgfqpoint{9.688469in}{1.224556in}}%
\pgfpathcurveto{\pgfqpoint{9.692035in}{1.220989in}}{\pgfqpoint{9.696873in}{1.218985in}}{\pgfqpoint{9.701916in}{1.218985in}}%
\pgfpathclose%
\pgfusepath{fill}%
\end{pgfscope}%
\begin{pgfscope}%
\pgfpathrectangle{\pgfqpoint{6.572727in}{0.474100in}}{\pgfqpoint{4.227273in}{3.318700in}}%
\pgfusepath{clip}%
\pgfsetbuttcap%
\pgfsetroundjoin%
\definecolor{currentfill}{rgb}{0.127568,0.566949,0.550556}%
\pgfsetfillcolor{currentfill}%
\pgfsetfillopacity{0.700000}%
\pgfsetlinewidth{0.000000pt}%
\definecolor{currentstroke}{rgb}{0.000000,0.000000,0.000000}%
\pgfsetstrokecolor{currentstroke}%
\pgfsetstrokeopacity{0.700000}%
\pgfsetdash{}{0pt}%
\pgfpathmoveto{\pgfqpoint{9.674932in}{2.171266in}}%
\pgfpathcurveto{\pgfqpoint{9.679975in}{2.171266in}}{\pgfqpoint{9.684813in}{2.173270in}}{\pgfqpoint{9.688379in}{2.176836in}}%
\pgfpathcurveto{\pgfqpoint{9.691946in}{2.180403in}}{\pgfqpoint{9.693950in}{2.185241in}}{\pgfqpoint{9.693950in}{2.190284in}}%
\pgfpathcurveto{\pgfqpoint{9.693950in}{2.195328in}}{\pgfqpoint{9.691946in}{2.200166in}}{\pgfqpoint{9.688379in}{2.203732in}}%
\pgfpathcurveto{\pgfqpoint{9.684813in}{2.207299in}}{\pgfqpoint{9.679975in}{2.209302in}}{\pgfqpoint{9.674932in}{2.209302in}}%
\pgfpathcurveto{\pgfqpoint{9.669888in}{2.209302in}}{\pgfqpoint{9.665050in}{2.207299in}}{\pgfqpoint{9.661484in}{2.203732in}}%
\pgfpathcurveto{\pgfqpoint{9.657917in}{2.200166in}}{\pgfqpoint{9.655913in}{2.195328in}}{\pgfqpoint{9.655913in}{2.190284in}}%
\pgfpathcurveto{\pgfqpoint{9.655913in}{2.185241in}}{\pgfqpoint{9.657917in}{2.180403in}}{\pgfqpoint{9.661484in}{2.176836in}}%
\pgfpathcurveto{\pgfqpoint{9.665050in}{2.173270in}}{\pgfqpoint{9.669888in}{2.171266in}}{\pgfqpoint{9.674932in}{2.171266in}}%
\pgfpathclose%
\pgfusepath{fill}%
\end{pgfscope}%
\begin{pgfscope}%
\pgfpathrectangle{\pgfqpoint{6.572727in}{0.474100in}}{\pgfqpoint{4.227273in}{3.318700in}}%
\pgfusepath{clip}%
\pgfsetbuttcap%
\pgfsetroundjoin%
\definecolor{currentfill}{rgb}{0.267004,0.004874,0.329415}%
\pgfsetfillcolor{currentfill}%
\pgfsetfillopacity{0.700000}%
\pgfsetlinewidth{0.000000pt}%
\definecolor{currentstroke}{rgb}{0.000000,0.000000,0.000000}%
\pgfsetstrokecolor{currentstroke}%
\pgfsetstrokeopacity{0.700000}%
\pgfsetdash{}{0pt}%
\pgfpathmoveto{\pgfqpoint{8.138498in}{2.002133in}}%
\pgfpathcurveto{\pgfqpoint{8.143542in}{2.002133in}}{\pgfqpoint{8.148380in}{2.004137in}}{\pgfqpoint{8.151946in}{2.007703in}}%
\pgfpathcurveto{\pgfqpoint{8.155512in}{2.011270in}}{\pgfqpoint{8.157516in}{2.016108in}}{\pgfqpoint{8.157516in}{2.021151in}}%
\pgfpathcurveto{\pgfqpoint{8.157516in}{2.026195in}}{\pgfqpoint{8.155512in}{2.031033in}}{\pgfqpoint{8.151946in}{2.034599in}}%
\pgfpathcurveto{\pgfqpoint{8.148380in}{2.038165in}}{\pgfqpoint{8.143542in}{2.040169in}}{\pgfqpoint{8.138498in}{2.040169in}}%
\pgfpathcurveto{\pgfqpoint{8.133454in}{2.040169in}}{\pgfqpoint{8.128617in}{2.038165in}}{\pgfqpoint{8.125050in}{2.034599in}}%
\pgfpathcurveto{\pgfqpoint{8.121484in}{2.031033in}}{\pgfqpoint{8.119480in}{2.026195in}}{\pgfqpoint{8.119480in}{2.021151in}}%
\pgfpathcurveto{\pgfqpoint{8.119480in}{2.016108in}}{\pgfqpoint{8.121484in}{2.011270in}}{\pgfqpoint{8.125050in}{2.007703in}}%
\pgfpathcurveto{\pgfqpoint{8.128617in}{2.004137in}}{\pgfqpoint{8.133454in}{2.002133in}}{\pgfqpoint{8.138498in}{2.002133in}}%
\pgfpathclose%
\pgfusepath{fill}%
\end{pgfscope}%
\begin{pgfscope}%
\pgfpathrectangle{\pgfqpoint{6.572727in}{0.474100in}}{\pgfqpoint{4.227273in}{3.318700in}}%
\pgfusepath{clip}%
\pgfsetbuttcap%
\pgfsetroundjoin%
\definecolor{currentfill}{rgb}{0.127568,0.566949,0.550556}%
\pgfsetfillcolor{currentfill}%
\pgfsetfillopacity{0.700000}%
\pgfsetlinewidth{0.000000pt}%
\definecolor{currentstroke}{rgb}{0.000000,0.000000,0.000000}%
\pgfsetstrokecolor{currentstroke}%
\pgfsetstrokeopacity{0.700000}%
\pgfsetdash{}{0pt}%
\pgfpathmoveto{\pgfqpoint{9.293637in}{1.740326in}}%
\pgfpathcurveto{\pgfqpoint{9.298680in}{1.740326in}}{\pgfqpoint{9.303518in}{1.742330in}}{\pgfqpoint{9.307084in}{1.745897in}}%
\pgfpathcurveto{\pgfqpoint{9.310651in}{1.749463in}}{\pgfqpoint{9.312655in}{1.754301in}}{\pgfqpoint{9.312655in}{1.759345in}}%
\pgfpathcurveto{\pgfqpoint{9.312655in}{1.764388in}}{\pgfqpoint{9.310651in}{1.769226in}}{\pgfqpoint{9.307084in}{1.772792in}}%
\pgfpathcurveto{\pgfqpoint{9.303518in}{1.776359in}}{\pgfqpoint{9.298680in}{1.778363in}}{\pgfqpoint{9.293637in}{1.778363in}}%
\pgfpathcurveto{\pgfqpoint{9.288593in}{1.778363in}}{\pgfqpoint{9.283755in}{1.776359in}}{\pgfqpoint{9.280189in}{1.772792in}}%
\pgfpathcurveto{\pgfqpoint{9.276622in}{1.769226in}}{\pgfqpoint{9.274618in}{1.764388in}}{\pgfqpoint{9.274618in}{1.759345in}}%
\pgfpathcurveto{\pgfqpoint{9.274618in}{1.754301in}}{\pgfqpoint{9.276622in}{1.749463in}}{\pgfqpoint{9.280189in}{1.745897in}}%
\pgfpathcurveto{\pgfqpoint{9.283755in}{1.742330in}}{\pgfqpoint{9.288593in}{1.740326in}}{\pgfqpoint{9.293637in}{1.740326in}}%
\pgfpathclose%
\pgfusepath{fill}%
\end{pgfscope}%
\begin{pgfscope}%
\pgfpathrectangle{\pgfqpoint{6.572727in}{0.474100in}}{\pgfqpoint{4.227273in}{3.318700in}}%
\pgfusepath{clip}%
\pgfsetbuttcap%
\pgfsetroundjoin%
\definecolor{currentfill}{rgb}{0.127568,0.566949,0.550556}%
\pgfsetfillcolor{currentfill}%
\pgfsetfillopacity{0.700000}%
\pgfsetlinewidth{0.000000pt}%
\definecolor{currentstroke}{rgb}{0.000000,0.000000,0.000000}%
\pgfsetstrokecolor{currentstroke}%
\pgfsetstrokeopacity{0.700000}%
\pgfsetdash{}{0pt}%
\pgfpathmoveto{\pgfqpoint{9.159888in}{1.566051in}}%
\pgfpathcurveto{\pgfqpoint{9.164932in}{1.566051in}}{\pgfqpoint{9.169769in}{1.568055in}}{\pgfqpoint{9.173336in}{1.571621in}}%
\pgfpathcurveto{\pgfqpoint{9.176902in}{1.575188in}}{\pgfqpoint{9.178906in}{1.580025in}}{\pgfqpoint{9.178906in}{1.585069in}}%
\pgfpathcurveto{\pgfqpoint{9.178906in}{1.590113in}}{\pgfqpoint{9.176902in}{1.594951in}}{\pgfqpoint{9.173336in}{1.598517in}}%
\pgfpathcurveto{\pgfqpoint{9.169769in}{1.602083in}}{\pgfqpoint{9.164932in}{1.604087in}}{\pgfqpoint{9.159888in}{1.604087in}}%
\pgfpathcurveto{\pgfqpoint{9.154844in}{1.604087in}}{\pgfqpoint{9.150006in}{1.602083in}}{\pgfqpoint{9.146440in}{1.598517in}}%
\pgfpathcurveto{\pgfqpoint{9.142874in}{1.594951in}}{\pgfqpoint{9.140870in}{1.590113in}}{\pgfqpoint{9.140870in}{1.585069in}}%
\pgfpathcurveto{\pgfqpoint{9.140870in}{1.580025in}}{\pgfqpoint{9.142874in}{1.575188in}}{\pgfqpoint{9.146440in}{1.571621in}}%
\pgfpathcurveto{\pgfqpoint{9.150006in}{1.568055in}}{\pgfqpoint{9.154844in}{1.566051in}}{\pgfqpoint{9.159888in}{1.566051in}}%
\pgfpathclose%
\pgfusepath{fill}%
\end{pgfscope}%
\begin{pgfscope}%
\pgfpathrectangle{\pgfqpoint{6.572727in}{0.474100in}}{\pgfqpoint{4.227273in}{3.318700in}}%
\pgfusepath{clip}%
\pgfsetbuttcap%
\pgfsetroundjoin%
\definecolor{currentfill}{rgb}{0.127568,0.566949,0.550556}%
\pgfsetfillcolor{currentfill}%
\pgfsetfillopacity{0.700000}%
\pgfsetlinewidth{0.000000pt}%
\definecolor{currentstroke}{rgb}{0.000000,0.000000,0.000000}%
\pgfsetstrokecolor{currentstroke}%
\pgfsetstrokeopacity{0.700000}%
\pgfsetdash{}{0pt}%
\pgfpathmoveto{\pgfqpoint{9.820642in}{1.508687in}}%
\pgfpathcurveto{\pgfqpoint{9.825686in}{1.508687in}}{\pgfqpoint{9.830523in}{1.510691in}}{\pgfqpoint{9.834090in}{1.514257in}}%
\pgfpathcurveto{\pgfqpoint{9.837656in}{1.517823in}}{\pgfqpoint{9.839660in}{1.522661in}}{\pgfqpoint{9.839660in}{1.527705in}}%
\pgfpathcurveto{\pgfqpoint{9.839660in}{1.532749in}}{\pgfqpoint{9.837656in}{1.537586in}}{\pgfqpoint{9.834090in}{1.541153in}}%
\pgfpathcurveto{\pgfqpoint{9.830523in}{1.544719in}}{\pgfqpoint{9.825686in}{1.546723in}}{\pgfqpoint{9.820642in}{1.546723in}}%
\pgfpathcurveto{\pgfqpoint{9.815598in}{1.546723in}}{\pgfqpoint{9.810760in}{1.544719in}}{\pgfqpoint{9.807194in}{1.541153in}}%
\pgfpathcurveto{\pgfqpoint{9.803628in}{1.537586in}}{\pgfqpoint{9.801624in}{1.532749in}}{\pgfqpoint{9.801624in}{1.527705in}}%
\pgfpathcurveto{\pgfqpoint{9.801624in}{1.522661in}}{\pgfqpoint{9.803628in}{1.517823in}}{\pgfqpoint{9.807194in}{1.514257in}}%
\pgfpathcurveto{\pgfqpoint{9.810760in}{1.510691in}}{\pgfqpoint{9.815598in}{1.508687in}}{\pgfqpoint{9.820642in}{1.508687in}}%
\pgfpathclose%
\pgfusepath{fill}%
\end{pgfscope}%
\begin{pgfscope}%
\pgfpathrectangle{\pgfqpoint{6.572727in}{0.474100in}}{\pgfqpoint{4.227273in}{3.318700in}}%
\pgfusepath{clip}%
\pgfsetbuttcap%
\pgfsetroundjoin%
\definecolor{currentfill}{rgb}{0.267004,0.004874,0.329415}%
\pgfsetfillcolor{currentfill}%
\pgfsetfillopacity{0.700000}%
\pgfsetlinewidth{0.000000pt}%
\definecolor{currentstroke}{rgb}{0.000000,0.000000,0.000000}%
\pgfsetstrokecolor{currentstroke}%
\pgfsetstrokeopacity{0.700000}%
\pgfsetdash{}{0pt}%
\pgfpathmoveto{\pgfqpoint{7.495660in}{1.923523in}}%
\pgfpathcurveto{\pgfqpoint{7.500704in}{1.923523in}}{\pgfqpoint{7.505541in}{1.925526in}}{\pgfqpoint{7.509108in}{1.929093in}}%
\pgfpathcurveto{\pgfqpoint{7.512674in}{1.932659in}}{\pgfqpoint{7.514678in}{1.937497in}}{\pgfqpoint{7.514678in}{1.942541in}}%
\pgfpathcurveto{\pgfqpoint{7.514678in}{1.947584in}}{\pgfqpoint{7.512674in}{1.952422in}}{\pgfqpoint{7.509108in}{1.955989in}}%
\pgfpathcurveto{\pgfqpoint{7.505541in}{1.959555in}}{\pgfqpoint{7.500704in}{1.961559in}}{\pgfqpoint{7.495660in}{1.961559in}}%
\pgfpathcurveto{\pgfqpoint{7.490616in}{1.961559in}}{\pgfqpoint{7.485778in}{1.959555in}}{\pgfqpoint{7.482212in}{1.955989in}}%
\pgfpathcurveto{\pgfqpoint{7.478646in}{1.952422in}}{\pgfqpoint{7.476642in}{1.947584in}}{\pgfqpoint{7.476642in}{1.942541in}}%
\pgfpathcurveto{\pgfqpoint{7.476642in}{1.937497in}}{\pgfqpoint{7.478646in}{1.932659in}}{\pgfqpoint{7.482212in}{1.929093in}}%
\pgfpathcurveto{\pgfqpoint{7.485778in}{1.925526in}}{\pgfqpoint{7.490616in}{1.923523in}}{\pgfqpoint{7.495660in}{1.923523in}}%
\pgfpathclose%
\pgfusepath{fill}%
\end{pgfscope}%
\begin{pgfscope}%
\pgfpathrectangle{\pgfqpoint{6.572727in}{0.474100in}}{\pgfqpoint{4.227273in}{3.318700in}}%
\pgfusepath{clip}%
\pgfsetbuttcap%
\pgfsetroundjoin%
\definecolor{currentfill}{rgb}{0.993248,0.906157,0.143936}%
\pgfsetfillcolor{currentfill}%
\pgfsetfillopacity{0.700000}%
\pgfsetlinewidth{0.000000pt}%
\definecolor{currentstroke}{rgb}{0.000000,0.000000,0.000000}%
\pgfsetstrokecolor{currentstroke}%
\pgfsetstrokeopacity{0.700000}%
\pgfsetdash{}{0pt}%
\pgfpathmoveto{\pgfqpoint{8.061798in}{3.460992in}}%
\pgfpathcurveto{\pgfqpoint{8.066841in}{3.460992in}}{\pgfqpoint{8.071679in}{3.462996in}}{\pgfqpoint{8.075245in}{3.466562in}}%
\pgfpathcurveto{\pgfqpoint{8.078812in}{3.470129in}}{\pgfqpoint{8.080816in}{3.474966in}}{\pgfqpoint{8.080816in}{3.480010in}}%
\pgfpathcurveto{\pgfqpoint{8.080816in}{3.485054in}}{\pgfqpoint{8.078812in}{3.489892in}}{\pgfqpoint{8.075245in}{3.493458in}}%
\pgfpathcurveto{\pgfqpoint{8.071679in}{3.497024in}}{\pgfqpoint{8.066841in}{3.499028in}}{\pgfqpoint{8.061798in}{3.499028in}}%
\pgfpathcurveto{\pgfqpoint{8.056754in}{3.499028in}}{\pgfqpoint{8.051916in}{3.497024in}}{\pgfqpoint{8.048350in}{3.493458in}}%
\pgfpathcurveto{\pgfqpoint{8.044783in}{3.489892in}}{\pgfqpoint{8.042779in}{3.485054in}}{\pgfqpoint{8.042779in}{3.480010in}}%
\pgfpathcurveto{\pgfqpoint{8.042779in}{3.474966in}}{\pgfqpoint{8.044783in}{3.470129in}}{\pgfqpoint{8.048350in}{3.466562in}}%
\pgfpathcurveto{\pgfqpoint{8.051916in}{3.462996in}}{\pgfqpoint{8.056754in}{3.460992in}}{\pgfqpoint{8.061798in}{3.460992in}}%
\pgfpathclose%
\pgfusepath{fill}%
\end{pgfscope}%
\begin{pgfscope}%
\pgfpathrectangle{\pgfqpoint{6.572727in}{0.474100in}}{\pgfqpoint{4.227273in}{3.318700in}}%
\pgfusepath{clip}%
\pgfsetbuttcap%
\pgfsetroundjoin%
\definecolor{currentfill}{rgb}{0.127568,0.566949,0.550556}%
\pgfsetfillcolor{currentfill}%
\pgfsetfillopacity{0.700000}%
\pgfsetlinewidth{0.000000pt}%
\definecolor{currentstroke}{rgb}{0.000000,0.000000,0.000000}%
\pgfsetstrokecolor{currentstroke}%
\pgfsetstrokeopacity{0.700000}%
\pgfsetdash{}{0pt}%
\pgfpathmoveto{\pgfqpoint{9.826447in}{1.331738in}}%
\pgfpathcurveto{\pgfqpoint{9.831491in}{1.331738in}}{\pgfqpoint{9.836329in}{1.333742in}}{\pgfqpoint{9.839895in}{1.337308in}}%
\pgfpathcurveto{\pgfqpoint{9.843461in}{1.340874in}}{\pgfqpoint{9.845465in}{1.345712in}}{\pgfqpoint{9.845465in}{1.350756in}}%
\pgfpathcurveto{\pgfqpoint{9.845465in}{1.355800in}}{\pgfqpoint{9.843461in}{1.360637in}}{\pgfqpoint{9.839895in}{1.364204in}}%
\pgfpathcurveto{\pgfqpoint{9.836329in}{1.367770in}}{\pgfqpoint{9.831491in}{1.369774in}}{\pgfqpoint{9.826447in}{1.369774in}}%
\pgfpathcurveto{\pgfqpoint{9.821404in}{1.369774in}}{\pgfqpoint{9.816566in}{1.367770in}}{\pgfqpoint{9.812999in}{1.364204in}}%
\pgfpathcurveto{\pgfqpoint{9.809433in}{1.360637in}}{\pgfqpoint{9.807429in}{1.355800in}}{\pgfqpoint{9.807429in}{1.350756in}}%
\pgfpathcurveto{\pgfqpoint{9.807429in}{1.345712in}}{\pgfqpoint{9.809433in}{1.340874in}}{\pgfqpoint{9.812999in}{1.337308in}}%
\pgfpathcurveto{\pgfqpoint{9.816566in}{1.333742in}}{\pgfqpoint{9.821404in}{1.331738in}}{\pgfqpoint{9.826447in}{1.331738in}}%
\pgfpathclose%
\pgfusepath{fill}%
\end{pgfscope}%
\begin{pgfscope}%
\pgfpathrectangle{\pgfqpoint{6.572727in}{0.474100in}}{\pgfqpoint{4.227273in}{3.318700in}}%
\pgfusepath{clip}%
\pgfsetbuttcap%
\pgfsetroundjoin%
\definecolor{currentfill}{rgb}{0.127568,0.566949,0.550556}%
\pgfsetfillcolor{currentfill}%
\pgfsetfillopacity{0.700000}%
\pgfsetlinewidth{0.000000pt}%
\definecolor{currentstroke}{rgb}{0.000000,0.000000,0.000000}%
\pgfsetstrokecolor{currentstroke}%
\pgfsetstrokeopacity{0.700000}%
\pgfsetdash{}{0pt}%
\pgfpathmoveto{\pgfqpoint{10.093114in}{1.259359in}}%
\pgfpathcurveto{\pgfqpoint{10.098158in}{1.259359in}}{\pgfqpoint{10.102996in}{1.261363in}}{\pgfqpoint{10.106562in}{1.264929in}}%
\pgfpathcurveto{\pgfqpoint{10.110129in}{1.268496in}}{\pgfqpoint{10.112133in}{1.273334in}}{\pgfqpoint{10.112133in}{1.278377in}}%
\pgfpathcurveto{\pgfqpoint{10.112133in}{1.283421in}}{\pgfqpoint{10.110129in}{1.288259in}}{\pgfqpoint{10.106562in}{1.291825in}}%
\pgfpathcurveto{\pgfqpoint{10.102996in}{1.295391in}}{\pgfqpoint{10.098158in}{1.297395in}}{\pgfqpoint{10.093114in}{1.297395in}}%
\pgfpathcurveto{\pgfqpoint{10.088071in}{1.297395in}}{\pgfqpoint{10.083233in}{1.295391in}}{\pgfqpoint{10.079667in}{1.291825in}}%
\pgfpathcurveto{\pgfqpoint{10.076100in}{1.288259in}}{\pgfqpoint{10.074096in}{1.283421in}}{\pgfqpoint{10.074096in}{1.278377in}}%
\pgfpathcurveto{\pgfqpoint{10.074096in}{1.273334in}}{\pgfqpoint{10.076100in}{1.268496in}}{\pgfqpoint{10.079667in}{1.264929in}}%
\pgfpathcurveto{\pgfqpoint{10.083233in}{1.261363in}}{\pgfqpoint{10.088071in}{1.259359in}}{\pgfqpoint{10.093114in}{1.259359in}}%
\pgfpathclose%
\pgfusepath{fill}%
\end{pgfscope}%
\begin{pgfscope}%
\pgfpathrectangle{\pgfqpoint{6.572727in}{0.474100in}}{\pgfqpoint{4.227273in}{3.318700in}}%
\pgfusepath{clip}%
\pgfsetbuttcap%
\pgfsetroundjoin%
\definecolor{currentfill}{rgb}{0.127568,0.566949,0.550556}%
\pgfsetfillcolor{currentfill}%
\pgfsetfillopacity{0.700000}%
\pgfsetlinewidth{0.000000pt}%
\definecolor{currentstroke}{rgb}{0.000000,0.000000,0.000000}%
\pgfsetstrokecolor{currentstroke}%
\pgfsetstrokeopacity{0.700000}%
\pgfsetdash{}{0pt}%
\pgfpathmoveto{\pgfqpoint{9.257791in}{2.241229in}}%
\pgfpathcurveto{\pgfqpoint{9.262835in}{2.241229in}}{\pgfqpoint{9.267673in}{2.243233in}}{\pgfqpoint{9.271239in}{2.246799in}}%
\pgfpathcurveto{\pgfqpoint{9.274806in}{2.250366in}}{\pgfqpoint{9.276810in}{2.255204in}}{\pgfqpoint{9.276810in}{2.260247in}}%
\pgfpathcurveto{\pgfqpoint{9.276810in}{2.265291in}}{\pgfqpoint{9.274806in}{2.270129in}}{\pgfqpoint{9.271239in}{2.273695in}}%
\pgfpathcurveto{\pgfqpoint{9.267673in}{2.277262in}}{\pgfqpoint{9.262835in}{2.279265in}}{\pgfqpoint{9.257791in}{2.279265in}}%
\pgfpathcurveto{\pgfqpoint{9.252748in}{2.279265in}}{\pgfqpoint{9.247910in}{2.277262in}}{\pgfqpoint{9.244344in}{2.273695in}}%
\pgfpathcurveto{\pgfqpoint{9.240777in}{2.270129in}}{\pgfqpoint{9.238773in}{2.265291in}}{\pgfqpoint{9.238773in}{2.260247in}}%
\pgfpathcurveto{\pgfqpoint{9.238773in}{2.255204in}}{\pgfqpoint{9.240777in}{2.250366in}}{\pgfqpoint{9.244344in}{2.246799in}}%
\pgfpathcurveto{\pgfqpoint{9.247910in}{2.243233in}}{\pgfqpoint{9.252748in}{2.241229in}}{\pgfqpoint{9.257791in}{2.241229in}}%
\pgfpathclose%
\pgfusepath{fill}%
\end{pgfscope}%
\begin{pgfscope}%
\pgfpathrectangle{\pgfqpoint{6.572727in}{0.474100in}}{\pgfqpoint{4.227273in}{3.318700in}}%
\pgfusepath{clip}%
\pgfsetbuttcap%
\pgfsetroundjoin%
\definecolor{currentfill}{rgb}{0.267004,0.004874,0.329415}%
\pgfsetfillcolor{currentfill}%
\pgfsetfillopacity{0.700000}%
\pgfsetlinewidth{0.000000pt}%
\definecolor{currentstroke}{rgb}{0.000000,0.000000,0.000000}%
\pgfsetstrokecolor{currentstroke}%
\pgfsetstrokeopacity{0.700000}%
\pgfsetdash{}{0pt}%
\pgfpathmoveto{\pgfqpoint{8.069953in}{1.296751in}}%
\pgfpathcurveto{\pgfqpoint{8.074997in}{1.296751in}}{\pgfqpoint{8.079834in}{1.298755in}}{\pgfqpoint{8.083401in}{1.302322in}}%
\pgfpathcurveto{\pgfqpoint{8.086967in}{1.305888in}}{\pgfqpoint{8.088971in}{1.310726in}}{\pgfqpoint{8.088971in}{1.315769in}}%
\pgfpathcurveto{\pgfqpoint{8.088971in}{1.320813in}}{\pgfqpoint{8.086967in}{1.325651in}}{\pgfqpoint{8.083401in}{1.329217in}}%
\pgfpathcurveto{\pgfqpoint{8.079834in}{1.332784in}}{\pgfqpoint{8.074997in}{1.334788in}}{\pgfqpoint{8.069953in}{1.334788in}}%
\pgfpathcurveto{\pgfqpoint{8.064909in}{1.334788in}}{\pgfqpoint{8.060071in}{1.332784in}}{\pgfqpoint{8.056505in}{1.329217in}}%
\pgfpathcurveto{\pgfqpoint{8.052939in}{1.325651in}}{\pgfqpoint{8.050935in}{1.320813in}}{\pgfqpoint{8.050935in}{1.315769in}}%
\pgfpathcurveto{\pgfqpoint{8.050935in}{1.310726in}}{\pgfqpoint{8.052939in}{1.305888in}}{\pgfqpoint{8.056505in}{1.302322in}}%
\pgfpathcurveto{\pgfqpoint{8.060071in}{1.298755in}}{\pgfqpoint{8.064909in}{1.296751in}}{\pgfqpoint{8.069953in}{1.296751in}}%
\pgfpathclose%
\pgfusepath{fill}%
\end{pgfscope}%
\begin{pgfscope}%
\pgfpathrectangle{\pgfqpoint{6.572727in}{0.474100in}}{\pgfqpoint{4.227273in}{3.318700in}}%
\pgfusepath{clip}%
\pgfsetbuttcap%
\pgfsetroundjoin%
\definecolor{currentfill}{rgb}{0.127568,0.566949,0.550556}%
\pgfsetfillcolor{currentfill}%
\pgfsetfillopacity{0.700000}%
\pgfsetlinewidth{0.000000pt}%
\definecolor{currentstroke}{rgb}{0.000000,0.000000,0.000000}%
\pgfsetstrokecolor{currentstroke}%
\pgfsetstrokeopacity{0.700000}%
\pgfsetdash{}{0pt}%
\pgfpathmoveto{\pgfqpoint{10.028396in}{1.690765in}}%
\pgfpathcurveto{\pgfqpoint{10.033440in}{1.690765in}}{\pgfqpoint{10.038278in}{1.692769in}}{\pgfqpoint{10.041844in}{1.696335in}}%
\pgfpathcurveto{\pgfqpoint{10.045411in}{1.699902in}}{\pgfqpoint{10.047414in}{1.704739in}}{\pgfqpoint{10.047414in}{1.709783in}}%
\pgfpathcurveto{\pgfqpoint{10.047414in}{1.714827in}}{\pgfqpoint{10.045411in}{1.719665in}}{\pgfqpoint{10.041844in}{1.723231in}}%
\pgfpathcurveto{\pgfqpoint{10.038278in}{1.726797in}}{\pgfqpoint{10.033440in}{1.728801in}}{\pgfqpoint{10.028396in}{1.728801in}}%
\pgfpathcurveto{\pgfqpoint{10.023353in}{1.728801in}}{\pgfqpoint{10.018515in}{1.726797in}}{\pgfqpoint{10.014948in}{1.723231in}}%
\pgfpathcurveto{\pgfqpoint{10.011382in}{1.719665in}}{\pgfqpoint{10.009378in}{1.714827in}}{\pgfqpoint{10.009378in}{1.709783in}}%
\pgfpathcurveto{\pgfqpoint{10.009378in}{1.704739in}}{\pgfqpoint{10.011382in}{1.699902in}}{\pgfqpoint{10.014948in}{1.696335in}}%
\pgfpathcurveto{\pgfqpoint{10.018515in}{1.692769in}}{\pgfqpoint{10.023353in}{1.690765in}}{\pgfqpoint{10.028396in}{1.690765in}}%
\pgfpathclose%
\pgfusepath{fill}%
\end{pgfscope}%
\begin{pgfscope}%
\pgfpathrectangle{\pgfqpoint{6.572727in}{0.474100in}}{\pgfqpoint{4.227273in}{3.318700in}}%
\pgfusepath{clip}%
\pgfsetbuttcap%
\pgfsetroundjoin%
\definecolor{currentfill}{rgb}{0.993248,0.906157,0.143936}%
\pgfsetfillcolor{currentfill}%
\pgfsetfillopacity{0.700000}%
\pgfsetlinewidth{0.000000pt}%
\definecolor{currentstroke}{rgb}{0.000000,0.000000,0.000000}%
\pgfsetstrokecolor{currentstroke}%
\pgfsetstrokeopacity{0.700000}%
\pgfsetdash{}{0pt}%
\pgfpathmoveto{\pgfqpoint{8.604466in}{3.565402in}}%
\pgfpathcurveto{\pgfqpoint{8.609510in}{3.565402in}}{\pgfqpoint{8.614348in}{3.567406in}}{\pgfqpoint{8.617914in}{3.570972in}}%
\pgfpathcurveto{\pgfqpoint{8.621481in}{3.574539in}}{\pgfqpoint{8.623485in}{3.579376in}}{\pgfqpoint{8.623485in}{3.584420in}}%
\pgfpathcurveto{\pgfqpoint{8.623485in}{3.589464in}}{\pgfqpoint{8.621481in}{3.594301in}}{\pgfqpoint{8.617914in}{3.597868in}}%
\pgfpathcurveto{\pgfqpoint{8.614348in}{3.601434in}}{\pgfqpoint{8.609510in}{3.603438in}}{\pgfqpoint{8.604466in}{3.603438in}}%
\pgfpathcurveto{\pgfqpoint{8.599423in}{3.603438in}}{\pgfqpoint{8.594585in}{3.601434in}}{\pgfqpoint{8.591019in}{3.597868in}}%
\pgfpathcurveto{\pgfqpoint{8.587452in}{3.594301in}}{\pgfqpoint{8.585448in}{3.589464in}}{\pgfqpoint{8.585448in}{3.584420in}}%
\pgfpathcurveto{\pgfqpoint{8.585448in}{3.579376in}}{\pgfqpoint{8.587452in}{3.574539in}}{\pgfqpoint{8.591019in}{3.570972in}}%
\pgfpathcurveto{\pgfqpoint{8.594585in}{3.567406in}}{\pgfqpoint{8.599423in}{3.565402in}}{\pgfqpoint{8.604466in}{3.565402in}}%
\pgfpathclose%
\pgfusepath{fill}%
\end{pgfscope}%
\begin{pgfscope}%
\pgfpathrectangle{\pgfqpoint{6.572727in}{0.474100in}}{\pgfqpoint{4.227273in}{3.318700in}}%
\pgfusepath{clip}%
\pgfsetbuttcap%
\pgfsetroundjoin%
\definecolor{currentfill}{rgb}{0.993248,0.906157,0.143936}%
\pgfsetfillcolor{currentfill}%
\pgfsetfillopacity{0.700000}%
\pgfsetlinewidth{0.000000pt}%
\definecolor{currentstroke}{rgb}{0.000000,0.000000,0.000000}%
\pgfsetstrokecolor{currentstroke}%
\pgfsetstrokeopacity{0.700000}%
\pgfsetdash{}{0pt}%
\pgfpathmoveto{\pgfqpoint{7.838982in}{2.831448in}}%
\pgfpathcurveto{\pgfqpoint{7.844026in}{2.831448in}}{\pgfqpoint{7.848863in}{2.833451in}}{\pgfqpoint{7.852430in}{2.837018in}}%
\pgfpathcurveto{\pgfqpoint{7.855996in}{2.840584in}}{\pgfqpoint{7.858000in}{2.845422in}}{\pgfqpoint{7.858000in}{2.850466in}}%
\pgfpathcurveto{\pgfqpoint{7.858000in}{2.855509in}}{\pgfqpoint{7.855996in}{2.860347in}}{\pgfqpoint{7.852430in}{2.863914in}}%
\pgfpathcurveto{\pgfqpoint{7.848863in}{2.867480in}}{\pgfqpoint{7.844026in}{2.869484in}}{\pgfqpoint{7.838982in}{2.869484in}}%
\pgfpathcurveto{\pgfqpoint{7.833938in}{2.869484in}}{\pgfqpoint{7.829100in}{2.867480in}}{\pgfqpoint{7.825534in}{2.863914in}}%
\pgfpathcurveto{\pgfqpoint{7.821968in}{2.860347in}}{\pgfqpoint{7.819964in}{2.855509in}}{\pgfqpoint{7.819964in}{2.850466in}}%
\pgfpathcurveto{\pgfqpoint{7.819964in}{2.845422in}}{\pgfqpoint{7.821968in}{2.840584in}}{\pgfqpoint{7.825534in}{2.837018in}}%
\pgfpathcurveto{\pgfqpoint{7.829100in}{2.833451in}}{\pgfqpoint{7.833938in}{2.831448in}}{\pgfqpoint{7.838982in}{2.831448in}}%
\pgfpathclose%
\pgfusepath{fill}%
\end{pgfscope}%
\begin{pgfscope}%
\pgfpathrectangle{\pgfqpoint{6.572727in}{0.474100in}}{\pgfqpoint{4.227273in}{3.318700in}}%
\pgfusepath{clip}%
\pgfsetbuttcap%
\pgfsetroundjoin%
\definecolor{currentfill}{rgb}{0.267004,0.004874,0.329415}%
\pgfsetfillcolor{currentfill}%
\pgfsetfillopacity{0.700000}%
\pgfsetlinewidth{0.000000pt}%
\definecolor{currentstroke}{rgb}{0.000000,0.000000,0.000000}%
\pgfsetstrokecolor{currentstroke}%
\pgfsetstrokeopacity{0.700000}%
\pgfsetdash{}{0pt}%
\pgfpathmoveto{\pgfqpoint{7.485068in}{1.086535in}}%
\pgfpathcurveto{\pgfqpoint{7.490112in}{1.086535in}}{\pgfqpoint{7.494949in}{1.088538in}}{\pgfqpoint{7.498516in}{1.092105in}}%
\pgfpathcurveto{\pgfqpoint{7.502082in}{1.095671in}}{\pgfqpoint{7.504086in}{1.100509in}}{\pgfqpoint{7.504086in}{1.105553in}}%
\pgfpathcurveto{\pgfqpoint{7.504086in}{1.110596in}}{\pgfqpoint{7.502082in}{1.115434in}}{\pgfqpoint{7.498516in}{1.119001in}}%
\pgfpathcurveto{\pgfqpoint{7.494949in}{1.122567in}}{\pgfqpoint{7.490112in}{1.124571in}}{\pgfqpoint{7.485068in}{1.124571in}}%
\pgfpathcurveto{\pgfqpoint{7.480024in}{1.124571in}}{\pgfqpoint{7.475187in}{1.122567in}}{\pgfqpoint{7.471620in}{1.119001in}}%
\pgfpathcurveto{\pgfqpoint{7.468054in}{1.115434in}}{\pgfqpoint{7.466050in}{1.110596in}}{\pgfqpoint{7.466050in}{1.105553in}}%
\pgfpathcurveto{\pgfqpoint{7.466050in}{1.100509in}}{\pgfqpoint{7.468054in}{1.095671in}}{\pgfqpoint{7.471620in}{1.092105in}}%
\pgfpathcurveto{\pgfqpoint{7.475187in}{1.088538in}}{\pgfqpoint{7.480024in}{1.086535in}}{\pgfqpoint{7.485068in}{1.086535in}}%
\pgfpathclose%
\pgfusepath{fill}%
\end{pgfscope}%
\begin{pgfscope}%
\pgfpathrectangle{\pgfqpoint{6.572727in}{0.474100in}}{\pgfqpoint{4.227273in}{3.318700in}}%
\pgfusepath{clip}%
\pgfsetbuttcap%
\pgfsetroundjoin%
\definecolor{currentfill}{rgb}{0.267004,0.004874,0.329415}%
\pgfsetfillcolor{currentfill}%
\pgfsetfillopacity{0.700000}%
\pgfsetlinewidth{0.000000pt}%
\definecolor{currentstroke}{rgb}{0.000000,0.000000,0.000000}%
\pgfsetstrokecolor{currentstroke}%
\pgfsetstrokeopacity{0.700000}%
\pgfsetdash{}{0pt}%
\pgfpathmoveto{\pgfqpoint{7.363760in}{1.436366in}}%
\pgfpathcurveto{\pgfqpoint{7.368803in}{1.436366in}}{\pgfqpoint{7.373641in}{1.438370in}}{\pgfqpoint{7.377207in}{1.441936in}}%
\pgfpathcurveto{\pgfqpoint{7.380774in}{1.445502in}}{\pgfqpoint{7.382778in}{1.450340in}}{\pgfqpoint{7.382778in}{1.455384in}}%
\pgfpathcurveto{\pgfqpoint{7.382778in}{1.460428in}}{\pgfqpoint{7.380774in}{1.465265in}}{\pgfqpoint{7.377207in}{1.468832in}}%
\pgfpathcurveto{\pgfqpoint{7.373641in}{1.472398in}}{\pgfqpoint{7.368803in}{1.474402in}}{\pgfqpoint{7.363760in}{1.474402in}}%
\pgfpathcurveto{\pgfqpoint{7.358716in}{1.474402in}}{\pgfqpoint{7.353878in}{1.472398in}}{\pgfqpoint{7.350312in}{1.468832in}}%
\pgfpathcurveto{\pgfqpoint{7.346745in}{1.465265in}}{\pgfqpoint{7.344741in}{1.460428in}}{\pgfqpoint{7.344741in}{1.455384in}}%
\pgfpathcurveto{\pgfqpoint{7.344741in}{1.450340in}}{\pgfqpoint{7.346745in}{1.445502in}}{\pgfqpoint{7.350312in}{1.441936in}}%
\pgfpathcurveto{\pgfqpoint{7.353878in}{1.438370in}}{\pgfqpoint{7.358716in}{1.436366in}}{\pgfqpoint{7.363760in}{1.436366in}}%
\pgfpathclose%
\pgfusepath{fill}%
\end{pgfscope}%
\begin{pgfscope}%
\pgfpathrectangle{\pgfqpoint{6.572727in}{0.474100in}}{\pgfqpoint{4.227273in}{3.318700in}}%
\pgfusepath{clip}%
\pgfsetbuttcap%
\pgfsetroundjoin%
\definecolor{currentfill}{rgb}{0.267004,0.004874,0.329415}%
\pgfsetfillcolor{currentfill}%
\pgfsetfillopacity{0.700000}%
\pgfsetlinewidth{0.000000pt}%
\definecolor{currentstroke}{rgb}{0.000000,0.000000,0.000000}%
\pgfsetstrokecolor{currentstroke}%
\pgfsetstrokeopacity{0.700000}%
\pgfsetdash{}{0pt}%
\pgfpathmoveto{\pgfqpoint{7.886747in}{2.037674in}}%
\pgfpathcurveto{\pgfqpoint{7.891791in}{2.037674in}}{\pgfqpoint{7.896629in}{2.039678in}}{\pgfqpoint{7.900195in}{2.043244in}}%
\pgfpathcurveto{\pgfqpoint{7.903761in}{2.046811in}}{\pgfqpoint{7.905765in}{2.051648in}}{\pgfqpoint{7.905765in}{2.056692in}}%
\pgfpathcurveto{\pgfqpoint{7.905765in}{2.061736in}}{\pgfqpoint{7.903761in}{2.066573in}}{\pgfqpoint{7.900195in}{2.070140in}}%
\pgfpathcurveto{\pgfqpoint{7.896629in}{2.073706in}}{\pgfqpoint{7.891791in}{2.075710in}}{\pgfqpoint{7.886747in}{2.075710in}}%
\pgfpathcurveto{\pgfqpoint{7.881703in}{2.075710in}}{\pgfqpoint{7.876866in}{2.073706in}}{\pgfqpoint{7.873299in}{2.070140in}}%
\pgfpathcurveto{\pgfqpoint{7.869733in}{2.066573in}}{\pgfqpoint{7.867729in}{2.061736in}}{\pgfqpoint{7.867729in}{2.056692in}}%
\pgfpathcurveto{\pgfqpoint{7.867729in}{2.051648in}}{\pgfqpoint{7.869733in}{2.046811in}}{\pgfqpoint{7.873299in}{2.043244in}}%
\pgfpathcurveto{\pgfqpoint{7.876866in}{2.039678in}}{\pgfqpoint{7.881703in}{2.037674in}}{\pgfqpoint{7.886747in}{2.037674in}}%
\pgfpathclose%
\pgfusepath{fill}%
\end{pgfscope}%
\begin{pgfscope}%
\pgfpathrectangle{\pgfqpoint{6.572727in}{0.474100in}}{\pgfqpoint{4.227273in}{3.318700in}}%
\pgfusepath{clip}%
\pgfsetbuttcap%
\pgfsetroundjoin%
\definecolor{currentfill}{rgb}{0.267004,0.004874,0.329415}%
\pgfsetfillcolor{currentfill}%
\pgfsetfillopacity{0.700000}%
\pgfsetlinewidth{0.000000pt}%
\definecolor{currentstroke}{rgb}{0.000000,0.000000,0.000000}%
\pgfsetstrokecolor{currentstroke}%
\pgfsetstrokeopacity{0.700000}%
\pgfsetdash{}{0pt}%
\pgfpathmoveto{\pgfqpoint{7.008253in}{2.027720in}}%
\pgfpathcurveto{\pgfqpoint{7.013296in}{2.027720in}}{\pgfqpoint{7.018134in}{2.029724in}}{\pgfqpoint{7.021700in}{2.033290in}}%
\pgfpathcurveto{\pgfqpoint{7.025267in}{2.036857in}}{\pgfqpoint{7.027271in}{2.041694in}}{\pgfqpoint{7.027271in}{2.046738in}}%
\pgfpathcurveto{\pgfqpoint{7.027271in}{2.051782in}}{\pgfqpoint{7.025267in}{2.056619in}}{\pgfqpoint{7.021700in}{2.060186in}}%
\pgfpathcurveto{\pgfqpoint{7.018134in}{2.063752in}}{\pgfqpoint{7.013296in}{2.065756in}}{\pgfqpoint{7.008253in}{2.065756in}}%
\pgfpathcurveto{\pgfqpoint{7.003209in}{2.065756in}}{\pgfqpoint{6.998371in}{2.063752in}}{\pgfqpoint{6.994805in}{2.060186in}}%
\pgfpathcurveto{\pgfqpoint{6.991238in}{2.056619in}}{\pgfqpoint{6.989234in}{2.051782in}}{\pgfqpoint{6.989234in}{2.046738in}}%
\pgfpathcurveto{\pgfqpoint{6.989234in}{2.041694in}}{\pgfqpoint{6.991238in}{2.036857in}}{\pgfqpoint{6.994805in}{2.033290in}}%
\pgfpathcurveto{\pgfqpoint{6.998371in}{2.029724in}}{\pgfqpoint{7.003209in}{2.027720in}}{\pgfqpoint{7.008253in}{2.027720in}}%
\pgfpathclose%
\pgfusepath{fill}%
\end{pgfscope}%
\begin{pgfscope}%
\pgfpathrectangle{\pgfqpoint{6.572727in}{0.474100in}}{\pgfqpoint{4.227273in}{3.318700in}}%
\pgfusepath{clip}%
\pgfsetbuttcap%
\pgfsetroundjoin%
\definecolor{currentfill}{rgb}{0.267004,0.004874,0.329415}%
\pgfsetfillcolor{currentfill}%
\pgfsetfillopacity{0.700000}%
\pgfsetlinewidth{0.000000pt}%
\definecolor{currentstroke}{rgb}{0.000000,0.000000,0.000000}%
\pgfsetstrokecolor{currentstroke}%
\pgfsetstrokeopacity{0.700000}%
\pgfsetdash{}{0pt}%
\pgfpathmoveto{\pgfqpoint{7.762669in}{1.978695in}}%
\pgfpathcurveto{\pgfqpoint{7.767713in}{1.978695in}}{\pgfqpoint{7.772551in}{1.980699in}}{\pgfqpoint{7.776117in}{1.984265in}}%
\pgfpathcurveto{\pgfqpoint{7.779684in}{1.987831in}}{\pgfqpoint{7.781688in}{1.992669in}}{\pgfqpoint{7.781688in}{1.997713in}}%
\pgfpathcurveto{\pgfqpoint{7.781688in}{2.002756in}}{\pgfqpoint{7.779684in}{2.007594in}}{\pgfqpoint{7.776117in}{2.011161in}}%
\pgfpathcurveto{\pgfqpoint{7.772551in}{2.014727in}}{\pgfqpoint{7.767713in}{2.016731in}}{\pgfqpoint{7.762669in}{2.016731in}}%
\pgfpathcurveto{\pgfqpoint{7.757626in}{2.016731in}}{\pgfqpoint{7.752788in}{2.014727in}}{\pgfqpoint{7.749222in}{2.011161in}}%
\pgfpathcurveto{\pgfqpoint{7.745655in}{2.007594in}}{\pgfqpoint{7.743651in}{2.002756in}}{\pgfqpoint{7.743651in}{1.997713in}}%
\pgfpathcurveto{\pgfqpoint{7.743651in}{1.992669in}}{\pgfqpoint{7.745655in}{1.987831in}}{\pgfqpoint{7.749222in}{1.984265in}}%
\pgfpathcurveto{\pgfqpoint{7.752788in}{1.980699in}}{\pgfqpoint{7.757626in}{1.978695in}}{\pgfqpoint{7.762669in}{1.978695in}}%
\pgfpathclose%
\pgfusepath{fill}%
\end{pgfscope}%
\begin{pgfscope}%
\pgfpathrectangle{\pgfqpoint{6.572727in}{0.474100in}}{\pgfqpoint{4.227273in}{3.318700in}}%
\pgfusepath{clip}%
\pgfsetbuttcap%
\pgfsetroundjoin%
\definecolor{currentfill}{rgb}{0.267004,0.004874,0.329415}%
\pgfsetfillcolor{currentfill}%
\pgfsetfillopacity{0.700000}%
\pgfsetlinewidth{0.000000pt}%
\definecolor{currentstroke}{rgb}{0.000000,0.000000,0.000000}%
\pgfsetstrokecolor{currentstroke}%
\pgfsetstrokeopacity{0.700000}%
\pgfsetdash{}{0pt}%
\pgfpathmoveto{\pgfqpoint{8.051190in}{1.564707in}}%
\pgfpathcurveto{\pgfqpoint{8.056233in}{1.564707in}}{\pgfqpoint{8.061071in}{1.566711in}}{\pgfqpoint{8.064638in}{1.570277in}}%
\pgfpathcurveto{\pgfqpoint{8.068204in}{1.573843in}}{\pgfqpoint{8.070208in}{1.578681in}}{\pgfqpoint{8.070208in}{1.583725in}}%
\pgfpathcurveto{\pgfqpoint{8.070208in}{1.588769in}}{\pgfqpoint{8.068204in}{1.593606in}}{\pgfqpoint{8.064638in}{1.597173in}}%
\pgfpathcurveto{\pgfqpoint{8.061071in}{1.600739in}}{\pgfqpoint{8.056233in}{1.602743in}}{\pgfqpoint{8.051190in}{1.602743in}}%
\pgfpathcurveto{\pgfqpoint{8.046146in}{1.602743in}}{\pgfqpoint{8.041308in}{1.600739in}}{\pgfqpoint{8.037742in}{1.597173in}}%
\pgfpathcurveto{\pgfqpoint{8.034176in}{1.593606in}}{\pgfqpoint{8.032172in}{1.588769in}}{\pgfqpoint{8.032172in}{1.583725in}}%
\pgfpathcurveto{\pgfqpoint{8.032172in}{1.578681in}}{\pgfqpoint{8.034176in}{1.573843in}}{\pgfqpoint{8.037742in}{1.570277in}}%
\pgfpathcurveto{\pgfqpoint{8.041308in}{1.566711in}}{\pgfqpoint{8.046146in}{1.564707in}}{\pgfqpoint{8.051190in}{1.564707in}}%
\pgfpathclose%
\pgfusepath{fill}%
\end{pgfscope}%
\begin{pgfscope}%
\pgfpathrectangle{\pgfqpoint{6.572727in}{0.474100in}}{\pgfqpoint{4.227273in}{3.318700in}}%
\pgfusepath{clip}%
\pgfsetbuttcap%
\pgfsetroundjoin%
\definecolor{currentfill}{rgb}{0.993248,0.906157,0.143936}%
\pgfsetfillcolor{currentfill}%
\pgfsetfillopacity{0.700000}%
\pgfsetlinewidth{0.000000pt}%
\definecolor{currentstroke}{rgb}{0.000000,0.000000,0.000000}%
\pgfsetstrokecolor{currentstroke}%
\pgfsetstrokeopacity{0.700000}%
\pgfsetdash{}{0pt}%
\pgfpathmoveto{\pgfqpoint{7.897754in}{2.676451in}}%
\pgfpathcurveto{\pgfqpoint{7.902798in}{2.676451in}}{\pgfqpoint{7.907635in}{2.678455in}}{\pgfqpoint{7.911202in}{2.682022in}}%
\pgfpathcurveto{\pgfqpoint{7.914768in}{2.685588in}}{\pgfqpoint{7.916772in}{2.690426in}}{\pgfqpoint{7.916772in}{2.695470in}}%
\pgfpathcurveto{\pgfqpoint{7.916772in}{2.700513in}}{\pgfqpoint{7.914768in}{2.705351in}}{\pgfqpoint{7.911202in}{2.708917in}}%
\pgfpathcurveto{\pgfqpoint{7.907635in}{2.712484in}}{\pgfqpoint{7.902798in}{2.714488in}}{\pgfqpoint{7.897754in}{2.714488in}}%
\pgfpathcurveto{\pgfqpoint{7.892710in}{2.714488in}}{\pgfqpoint{7.887872in}{2.712484in}}{\pgfqpoint{7.884306in}{2.708917in}}%
\pgfpathcurveto{\pgfqpoint{7.880740in}{2.705351in}}{\pgfqpoint{7.878736in}{2.700513in}}{\pgfqpoint{7.878736in}{2.695470in}}%
\pgfpathcurveto{\pgfqpoint{7.878736in}{2.690426in}}{\pgfqpoint{7.880740in}{2.685588in}}{\pgfqpoint{7.884306in}{2.682022in}}%
\pgfpathcurveto{\pgfqpoint{7.887872in}{2.678455in}}{\pgfqpoint{7.892710in}{2.676451in}}{\pgfqpoint{7.897754in}{2.676451in}}%
\pgfpathclose%
\pgfusepath{fill}%
\end{pgfscope}%
\begin{pgfscope}%
\pgfpathrectangle{\pgfqpoint{6.572727in}{0.474100in}}{\pgfqpoint{4.227273in}{3.318700in}}%
\pgfusepath{clip}%
\pgfsetbuttcap%
\pgfsetroundjoin%
\definecolor{currentfill}{rgb}{0.267004,0.004874,0.329415}%
\pgfsetfillcolor{currentfill}%
\pgfsetfillopacity{0.700000}%
\pgfsetlinewidth{0.000000pt}%
\definecolor{currentstroke}{rgb}{0.000000,0.000000,0.000000}%
\pgfsetstrokecolor{currentstroke}%
\pgfsetstrokeopacity{0.700000}%
\pgfsetdash{}{0pt}%
\pgfpathmoveto{\pgfqpoint{8.591351in}{1.895249in}}%
\pgfpathcurveto{\pgfqpoint{8.596394in}{1.895249in}}{\pgfqpoint{8.601232in}{1.897253in}}{\pgfqpoint{8.604799in}{1.900819in}}%
\pgfpathcurveto{\pgfqpoint{8.608365in}{1.904385in}}{\pgfqpoint{8.610369in}{1.909223in}}{\pgfqpoint{8.610369in}{1.914267in}}%
\pgfpathcurveto{\pgfqpoint{8.610369in}{1.919311in}}{\pgfqpoint{8.608365in}{1.924148in}}{\pgfqpoint{8.604799in}{1.927715in}}%
\pgfpathcurveto{\pgfqpoint{8.601232in}{1.931281in}}{\pgfqpoint{8.596394in}{1.933285in}}{\pgfqpoint{8.591351in}{1.933285in}}%
\pgfpathcurveto{\pgfqpoint{8.586307in}{1.933285in}}{\pgfqpoint{8.581469in}{1.931281in}}{\pgfqpoint{8.577903in}{1.927715in}}%
\pgfpathcurveto{\pgfqpoint{8.574336in}{1.924148in}}{\pgfqpoint{8.572333in}{1.919311in}}{\pgfqpoint{8.572333in}{1.914267in}}%
\pgfpathcurveto{\pgfqpoint{8.572333in}{1.909223in}}{\pgfqpoint{8.574336in}{1.904385in}}{\pgfqpoint{8.577903in}{1.900819in}}%
\pgfpathcurveto{\pgfqpoint{8.581469in}{1.897253in}}{\pgfqpoint{8.586307in}{1.895249in}}{\pgfqpoint{8.591351in}{1.895249in}}%
\pgfpathclose%
\pgfusepath{fill}%
\end{pgfscope}%
\begin{pgfscope}%
\pgfpathrectangle{\pgfqpoint{6.572727in}{0.474100in}}{\pgfqpoint{4.227273in}{3.318700in}}%
\pgfusepath{clip}%
\pgfsetbuttcap%
\pgfsetroundjoin%
\definecolor{currentfill}{rgb}{0.267004,0.004874,0.329415}%
\pgfsetfillcolor{currentfill}%
\pgfsetfillopacity{0.700000}%
\pgfsetlinewidth{0.000000pt}%
\definecolor{currentstroke}{rgb}{0.000000,0.000000,0.000000}%
\pgfsetstrokecolor{currentstroke}%
\pgfsetstrokeopacity{0.700000}%
\pgfsetdash{}{0pt}%
\pgfpathmoveto{\pgfqpoint{8.008299in}{2.068373in}}%
\pgfpathcurveto{\pgfqpoint{8.013343in}{2.068373in}}{\pgfqpoint{8.018181in}{2.070377in}}{\pgfqpoint{8.021747in}{2.073943in}}%
\pgfpathcurveto{\pgfqpoint{8.025313in}{2.077510in}}{\pgfqpoint{8.027317in}{2.082347in}}{\pgfqpoint{8.027317in}{2.087391in}}%
\pgfpathcurveto{\pgfqpoint{8.027317in}{2.092435in}}{\pgfqpoint{8.025313in}{2.097272in}}{\pgfqpoint{8.021747in}{2.100839in}}%
\pgfpathcurveto{\pgfqpoint{8.018181in}{2.104405in}}{\pgfqpoint{8.013343in}{2.106409in}}{\pgfqpoint{8.008299in}{2.106409in}}%
\pgfpathcurveto{\pgfqpoint{8.003255in}{2.106409in}}{\pgfqpoint{7.998418in}{2.104405in}}{\pgfqpoint{7.994851in}{2.100839in}}%
\pgfpathcurveto{\pgfqpoint{7.991285in}{2.097272in}}{\pgfqpoint{7.989281in}{2.092435in}}{\pgfqpoint{7.989281in}{2.087391in}}%
\pgfpathcurveto{\pgfqpoint{7.989281in}{2.082347in}}{\pgfqpoint{7.991285in}{2.077510in}}{\pgfqpoint{7.994851in}{2.073943in}}%
\pgfpathcurveto{\pgfqpoint{7.998418in}{2.070377in}}{\pgfqpoint{8.003255in}{2.068373in}}{\pgfqpoint{8.008299in}{2.068373in}}%
\pgfpathclose%
\pgfusepath{fill}%
\end{pgfscope}%
\begin{pgfscope}%
\pgfpathrectangle{\pgfqpoint{6.572727in}{0.474100in}}{\pgfqpoint{4.227273in}{3.318700in}}%
\pgfusepath{clip}%
\pgfsetbuttcap%
\pgfsetroundjoin%
\definecolor{currentfill}{rgb}{0.993248,0.906157,0.143936}%
\pgfsetfillcolor{currentfill}%
\pgfsetfillopacity{0.700000}%
\pgfsetlinewidth{0.000000pt}%
\definecolor{currentstroke}{rgb}{0.000000,0.000000,0.000000}%
\pgfsetstrokecolor{currentstroke}%
\pgfsetstrokeopacity{0.700000}%
\pgfsetdash{}{0pt}%
\pgfpathmoveto{\pgfqpoint{8.025682in}{2.796327in}}%
\pgfpathcurveto{\pgfqpoint{8.030726in}{2.796327in}}{\pgfqpoint{8.035564in}{2.798331in}}{\pgfqpoint{8.039130in}{2.801898in}}%
\pgfpathcurveto{\pgfqpoint{8.042696in}{2.805464in}}{\pgfqpoint{8.044700in}{2.810302in}}{\pgfqpoint{8.044700in}{2.815345in}}%
\pgfpathcurveto{\pgfqpoint{8.044700in}{2.820389in}}{\pgfqpoint{8.042696in}{2.825227in}}{\pgfqpoint{8.039130in}{2.828793in}}%
\pgfpathcurveto{\pgfqpoint{8.035564in}{2.832360in}}{\pgfqpoint{8.030726in}{2.834364in}}{\pgfqpoint{8.025682in}{2.834364in}}%
\pgfpathcurveto{\pgfqpoint{8.020638in}{2.834364in}}{\pgfqpoint{8.015801in}{2.832360in}}{\pgfqpoint{8.012234in}{2.828793in}}%
\pgfpathcurveto{\pgfqpoint{8.008668in}{2.825227in}}{\pgfqpoint{8.006664in}{2.820389in}}{\pgfqpoint{8.006664in}{2.815345in}}%
\pgfpathcurveto{\pgfqpoint{8.006664in}{2.810302in}}{\pgfqpoint{8.008668in}{2.805464in}}{\pgfqpoint{8.012234in}{2.801898in}}%
\pgfpathcurveto{\pgfqpoint{8.015801in}{2.798331in}}{\pgfqpoint{8.020638in}{2.796327in}}{\pgfqpoint{8.025682in}{2.796327in}}%
\pgfpathclose%
\pgfusepath{fill}%
\end{pgfscope}%
\begin{pgfscope}%
\pgfpathrectangle{\pgfqpoint{6.572727in}{0.474100in}}{\pgfqpoint{4.227273in}{3.318700in}}%
\pgfusepath{clip}%
\pgfsetbuttcap%
\pgfsetroundjoin%
\definecolor{currentfill}{rgb}{0.993248,0.906157,0.143936}%
\pgfsetfillcolor{currentfill}%
\pgfsetfillopacity{0.700000}%
\pgfsetlinewidth{0.000000pt}%
\definecolor{currentstroke}{rgb}{0.000000,0.000000,0.000000}%
\pgfsetstrokecolor{currentstroke}%
\pgfsetstrokeopacity{0.700000}%
\pgfsetdash{}{0pt}%
\pgfpathmoveto{\pgfqpoint{7.696817in}{2.961298in}}%
\pgfpathcurveto{\pgfqpoint{7.701861in}{2.961298in}}{\pgfqpoint{7.706699in}{2.963302in}}{\pgfqpoint{7.710265in}{2.966868in}}%
\pgfpathcurveto{\pgfqpoint{7.713831in}{2.970434in}}{\pgfqpoint{7.715835in}{2.975272in}}{\pgfqpoint{7.715835in}{2.980316in}}%
\pgfpathcurveto{\pgfqpoint{7.715835in}{2.985360in}}{\pgfqpoint{7.713831in}{2.990197in}}{\pgfqpoint{7.710265in}{2.993764in}}%
\pgfpathcurveto{\pgfqpoint{7.706699in}{2.997330in}}{\pgfqpoint{7.701861in}{2.999334in}}{\pgfqpoint{7.696817in}{2.999334in}}%
\pgfpathcurveto{\pgfqpoint{7.691774in}{2.999334in}}{\pgfqpoint{7.686936in}{2.997330in}}{\pgfqpoint{7.683369in}{2.993764in}}%
\pgfpathcurveto{\pgfqpoint{7.679803in}{2.990197in}}{\pgfqpoint{7.677799in}{2.985360in}}{\pgfqpoint{7.677799in}{2.980316in}}%
\pgfpathcurveto{\pgfqpoint{7.677799in}{2.975272in}}{\pgfqpoint{7.679803in}{2.970434in}}{\pgfqpoint{7.683369in}{2.966868in}}%
\pgfpathcurveto{\pgfqpoint{7.686936in}{2.963302in}}{\pgfqpoint{7.691774in}{2.961298in}}{\pgfqpoint{7.696817in}{2.961298in}}%
\pgfpathclose%
\pgfusepath{fill}%
\end{pgfscope}%
\begin{pgfscope}%
\pgfpathrectangle{\pgfqpoint{6.572727in}{0.474100in}}{\pgfqpoint{4.227273in}{3.318700in}}%
\pgfusepath{clip}%
\pgfsetbuttcap%
\pgfsetroundjoin%
\definecolor{currentfill}{rgb}{0.127568,0.566949,0.550556}%
\pgfsetfillcolor{currentfill}%
\pgfsetfillopacity{0.700000}%
\pgfsetlinewidth{0.000000pt}%
\definecolor{currentstroke}{rgb}{0.000000,0.000000,0.000000}%
\pgfsetstrokecolor{currentstroke}%
\pgfsetstrokeopacity{0.700000}%
\pgfsetdash{}{0pt}%
\pgfpathmoveto{\pgfqpoint{9.263842in}{1.358747in}}%
\pgfpathcurveto{\pgfqpoint{9.268886in}{1.358747in}}{\pgfqpoint{9.273724in}{1.360751in}}{\pgfqpoint{9.277290in}{1.364318in}}%
\pgfpathcurveto{\pgfqpoint{9.280857in}{1.367884in}}{\pgfqpoint{9.282860in}{1.372722in}}{\pgfqpoint{9.282860in}{1.377766in}}%
\pgfpathcurveto{\pgfqpoint{9.282860in}{1.382809in}}{\pgfqpoint{9.280857in}{1.387647in}}{\pgfqpoint{9.277290in}{1.391213in}}%
\pgfpathcurveto{\pgfqpoint{9.273724in}{1.394780in}}{\pgfqpoint{9.268886in}{1.396784in}}{\pgfqpoint{9.263842in}{1.396784in}}%
\pgfpathcurveto{\pgfqpoint{9.258799in}{1.396784in}}{\pgfqpoint{9.253961in}{1.394780in}}{\pgfqpoint{9.250394in}{1.391213in}}%
\pgfpathcurveto{\pgfqpoint{9.246828in}{1.387647in}}{\pgfqpoint{9.244824in}{1.382809in}}{\pgfqpoint{9.244824in}{1.377766in}}%
\pgfpathcurveto{\pgfqpoint{9.244824in}{1.372722in}}{\pgfqpoint{9.246828in}{1.367884in}}{\pgfqpoint{9.250394in}{1.364318in}}%
\pgfpathcurveto{\pgfqpoint{9.253961in}{1.360751in}}{\pgfqpoint{9.258799in}{1.358747in}}{\pgfqpoint{9.263842in}{1.358747in}}%
\pgfpathclose%
\pgfusepath{fill}%
\end{pgfscope}%
\begin{pgfscope}%
\pgfpathrectangle{\pgfqpoint{6.572727in}{0.474100in}}{\pgfqpoint{4.227273in}{3.318700in}}%
\pgfusepath{clip}%
\pgfsetbuttcap%
\pgfsetroundjoin%
\definecolor{currentfill}{rgb}{0.127568,0.566949,0.550556}%
\pgfsetfillcolor{currentfill}%
\pgfsetfillopacity{0.700000}%
\pgfsetlinewidth{0.000000pt}%
\definecolor{currentstroke}{rgb}{0.000000,0.000000,0.000000}%
\pgfsetstrokecolor{currentstroke}%
\pgfsetstrokeopacity{0.700000}%
\pgfsetdash{}{0pt}%
\pgfpathmoveto{\pgfqpoint{9.568155in}{1.588647in}}%
\pgfpathcurveto{\pgfqpoint{9.573199in}{1.588647in}}{\pgfqpoint{9.578036in}{1.590651in}}{\pgfqpoint{9.581603in}{1.594217in}}%
\pgfpathcurveto{\pgfqpoint{9.585169in}{1.597784in}}{\pgfqpoint{9.587173in}{1.602621in}}{\pgfqpoint{9.587173in}{1.607665in}}%
\pgfpathcurveto{\pgfqpoint{9.587173in}{1.612709in}}{\pgfqpoint{9.585169in}{1.617547in}}{\pgfqpoint{9.581603in}{1.621113in}}%
\pgfpathcurveto{\pgfqpoint{9.578036in}{1.624679in}}{\pgfqpoint{9.573199in}{1.626683in}}{\pgfqpoint{9.568155in}{1.626683in}}%
\pgfpathcurveto{\pgfqpoint{9.563111in}{1.626683in}}{\pgfqpoint{9.558273in}{1.624679in}}{\pgfqpoint{9.554707in}{1.621113in}}%
\pgfpathcurveto{\pgfqpoint{9.551141in}{1.617547in}}{\pgfqpoint{9.549137in}{1.612709in}}{\pgfqpoint{9.549137in}{1.607665in}}%
\pgfpathcurveto{\pgfqpoint{9.549137in}{1.602621in}}{\pgfqpoint{9.551141in}{1.597784in}}{\pgfqpoint{9.554707in}{1.594217in}}%
\pgfpathcurveto{\pgfqpoint{9.558273in}{1.590651in}}{\pgfqpoint{9.563111in}{1.588647in}}{\pgfqpoint{9.568155in}{1.588647in}}%
\pgfpathclose%
\pgfusepath{fill}%
\end{pgfscope}%
\begin{pgfscope}%
\pgfpathrectangle{\pgfqpoint{6.572727in}{0.474100in}}{\pgfqpoint{4.227273in}{3.318700in}}%
\pgfusepath{clip}%
\pgfsetbuttcap%
\pgfsetroundjoin%
\definecolor{currentfill}{rgb}{0.127568,0.566949,0.550556}%
\pgfsetfillcolor{currentfill}%
\pgfsetfillopacity{0.700000}%
\pgfsetlinewidth{0.000000pt}%
\definecolor{currentstroke}{rgb}{0.000000,0.000000,0.000000}%
\pgfsetstrokecolor{currentstroke}%
\pgfsetstrokeopacity{0.700000}%
\pgfsetdash{}{0pt}%
\pgfpathmoveto{\pgfqpoint{9.920171in}{1.154693in}}%
\pgfpathcurveto{\pgfqpoint{9.925215in}{1.154693in}}{\pgfqpoint{9.930053in}{1.156696in}}{\pgfqpoint{9.933619in}{1.160263in}}%
\pgfpathcurveto{\pgfqpoint{9.937186in}{1.163829in}}{\pgfqpoint{9.939190in}{1.168667in}}{\pgfqpoint{9.939190in}{1.173711in}}%
\pgfpathcurveto{\pgfqpoint{9.939190in}{1.178754in}}{\pgfqpoint{9.937186in}{1.183592in}}{\pgfqpoint{9.933619in}{1.187159in}}%
\pgfpathcurveto{\pgfqpoint{9.930053in}{1.190725in}}{\pgfqpoint{9.925215in}{1.192729in}}{\pgfqpoint{9.920171in}{1.192729in}}%
\pgfpathcurveto{\pgfqpoint{9.915128in}{1.192729in}}{\pgfqpoint{9.910290in}{1.190725in}}{\pgfqpoint{9.906724in}{1.187159in}}%
\pgfpathcurveto{\pgfqpoint{9.903157in}{1.183592in}}{\pgfqpoint{9.901153in}{1.178754in}}{\pgfqpoint{9.901153in}{1.173711in}}%
\pgfpathcurveto{\pgfqpoint{9.901153in}{1.168667in}}{\pgfqpoint{9.903157in}{1.163829in}}{\pgfqpoint{9.906724in}{1.160263in}}%
\pgfpathcurveto{\pgfqpoint{9.910290in}{1.156696in}}{\pgfqpoint{9.915128in}{1.154693in}}{\pgfqpoint{9.920171in}{1.154693in}}%
\pgfpathclose%
\pgfusepath{fill}%
\end{pgfscope}%
\begin{pgfscope}%
\pgfpathrectangle{\pgfqpoint{6.572727in}{0.474100in}}{\pgfqpoint{4.227273in}{3.318700in}}%
\pgfusepath{clip}%
\pgfsetbuttcap%
\pgfsetroundjoin%
\definecolor{currentfill}{rgb}{0.993248,0.906157,0.143936}%
\pgfsetfillcolor{currentfill}%
\pgfsetfillopacity{0.700000}%
\pgfsetlinewidth{0.000000pt}%
\definecolor{currentstroke}{rgb}{0.000000,0.000000,0.000000}%
\pgfsetstrokecolor{currentstroke}%
\pgfsetstrokeopacity{0.700000}%
\pgfsetdash{}{0pt}%
\pgfpathmoveto{\pgfqpoint{7.993456in}{3.370458in}}%
\pgfpathcurveto{\pgfqpoint{7.998500in}{3.370458in}}{\pgfqpoint{8.003338in}{3.372462in}}{\pgfqpoint{8.006904in}{3.376028in}}%
\pgfpathcurveto{\pgfqpoint{8.010470in}{3.379595in}}{\pgfqpoint{8.012474in}{3.384433in}}{\pgfqpoint{8.012474in}{3.389476in}}%
\pgfpathcurveto{\pgfqpoint{8.012474in}{3.394520in}}{\pgfqpoint{8.010470in}{3.399358in}}{\pgfqpoint{8.006904in}{3.402924in}}%
\pgfpathcurveto{\pgfqpoint{8.003338in}{3.406491in}}{\pgfqpoint{7.998500in}{3.408494in}}{\pgfqpoint{7.993456in}{3.408494in}}%
\pgfpathcurveto{\pgfqpoint{7.988412in}{3.408494in}}{\pgfqpoint{7.983575in}{3.406491in}}{\pgfqpoint{7.980008in}{3.402924in}}%
\pgfpathcurveto{\pgfqpoint{7.976442in}{3.399358in}}{\pgfqpoint{7.974438in}{3.394520in}}{\pgfqpoint{7.974438in}{3.389476in}}%
\pgfpathcurveto{\pgfqpoint{7.974438in}{3.384433in}}{\pgfqpoint{7.976442in}{3.379595in}}{\pgfqpoint{7.980008in}{3.376028in}}%
\pgfpathcurveto{\pgfqpoint{7.983575in}{3.372462in}}{\pgfqpoint{7.988412in}{3.370458in}}{\pgfqpoint{7.993456in}{3.370458in}}%
\pgfpathclose%
\pgfusepath{fill}%
\end{pgfscope}%
\begin{pgfscope}%
\pgfpathrectangle{\pgfqpoint{6.572727in}{0.474100in}}{\pgfqpoint{4.227273in}{3.318700in}}%
\pgfusepath{clip}%
\pgfsetbuttcap%
\pgfsetroundjoin%
\definecolor{currentfill}{rgb}{0.267004,0.004874,0.329415}%
\pgfsetfillcolor{currentfill}%
\pgfsetfillopacity{0.700000}%
\pgfsetlinewidth{0.000000pt}%
\definecolor{currentstroke}{rgb}{0.000000,0.000000,0.000000}%
\pgfsetstrokecolor{currentstroke}%
\pgfsetstrokeopacity{0.700000}%
\pgfsetdash{}{0pt}%
\pgfpathmoveto{\pgfqpoint{7.379559in}{1.546080in}}%
\pgfpathcurveto{\pgfqpoint{7.384602in}{1.546080in}}{\pgfqpoint{7.389440in}{1.548083in}}{\pgfqpoint{7.393007in}{1.551650in}}%
\pgfpathcurveto{\pgfqpoint{7.396573in}{1.555216in}}{\pgfqpoint{7.398577in}{1.560054in}}{\pgfqpoint{7.398577in}{1.565098in}}%
\pgfpathcurveto{\pgfqpoint{7.398577in}{1.570141in}}{\pgfqpoint{7.396573in}{1.574979in}}{\pgfqpoint{7.393007in}{1.578546in}}%
\pgfpathcurveto{\pgfqpoint{7.389440in}{1.582112in}}{\pgfqpoint{7.384602in}{1.584116in}}{\pgfqpoint{7.379559in}{1.584116in}}%
\pgfpathcurveto{\pgfqpoint{7.374515in}{1.584116in}}{\pgfqpoint{7.369677in}{1.582112in}}{\pgfqpoint{7.366111in}{1.578546in}}%
\pgfpathcurveto{\pgfqpoint{7.362544in}{1.574979in}}{\pgfqpoint{7.360541in}{1.570141in}}{\pgfqpoint{7.360541in}{1.565098in}}%
\pgfpathcurveto{\pgfqpoint{7.360541in}{1.560054in}}{\pgfqpoint{7.362544in}{1.555216in}}{\pgfqpoint{7.366111in}{1.551650in}}%
\pgfpathcurveto{\pgfqpoint{7.369677in}{1.548083in}}{\pgfqpoint{7.374515in}{1.546080in}}{\pgfqpoint{7.379559in}{1.546080in}}%
\pgfpathclose%
\pgfusepath{fill}%
\end{pgfscope}%
\begin{pgfscope}%
\pgfpathrectangle{\pgfqpoint{6.572727in}{0.474100in}}{\pgfqpoint{4.227273in}{3.318700in}}%
\pgfusepath{clip}%
\pgfsetbuttcap%
\pgfsetroundjoin%
\definecolor{currentfill}{rgb}{0.993248,0.906157,0.143936}%
\pgfsetfillcolor{currentfill}%
\pgfsetfillopacity{0.700000}%
\pgfsetlinewidth{0.000000pt}%
\definecolor{currentstroke}{rgb}{0.000000,0.000000,0.000000}%
\pgfsetstrokecolor{currentstroke}%
\pgfsetstrokeopacity{0.700000}%
\pgfsetdash{}{0pt}%
\pgfpathmoveto{\pgfqpoint{8.224182in}{2.813490in}}%
\pgfpathcurveto{\pgfqpoint{8.229226in}{2.813490in}}{\pgfqpoint{8.234064in}{2.815494in}}{\pgfqpoint{8.237630in}{2.819061in}}%
\pgfpathcurveto{\pgfqpoint{8.241197in}{2.822627in}}{\pgfqpoint{8.243201in}{2.827465in}}{\pgfqpoint{8.243201in}{2.832509in}}%
\pgfpathcurveto{\pgfqpoint{8.243201in}{2.837552in}}{\pgfqpoint{8.241197in}{2.842390in}}{\pgfqpoint{8.237630in}{2.845956in}}%
\pgfpathcurveto{\pgfqpoint{8.234064in}{2.849523in}}{\pgfqpoint{8.229226in}{2.851527in}}{\pgfqpoint{8.224182in}{2.851527in}}%
\pgfpathcurveto{\pgfqpoint{8.219139in}{2.851527in}}{\pgfqpoint{8.214301in}{2.849523in}}{\pgfqpoint{8.210735in}{2.845956in}}%
\pgfpathcurveto{\pgfqpoint{8.207168in}{2.842390in}}{\pgfqpoint{8.205164in}{2.837552in}}{\pgfqpoint{8.205164in}{2.832509in}}%
\pgfpathcurveto{\pgfqpoint{8.205164in}{2.827465in}}{\pgfqpoint{8.207168in}{2.822627in}}{\pgfqpoint{8.210735in}{2.819061in}}%
\pgfpathcurveto{\pgfqpoint{8.214301in}{2.815494in}}{\pgfqpoint{8.219139in}{2.813490in}}{\pgfqpoint{8.224182in}{2.813490in}}%
\pgfpathclose%
\pgfusepath{fill}%
\end{pgfscope}%
\begin{pgfscope}%
\pgfpathrectangle{\pgfqpoint{6.572727in}{0.474100in}}{\pgfqpoint{4.227273in}{3.318700in}}%
\pgfusepath{clip}%
\pgfsetbuttcap%
\pgfsetroundjoin%
\definecolor{currentfill}{rgb}{0.127568,0.566949,0.550556}%
\pgfsetfillcolor{currentfill}%
\pgfsetfillopacity{0.700000}%
\pgfsetlinewidth{0.000000pt}%
\definecolor{currentstroke}{rgb}{0.000000,0.000000,0.000000}%
\pgfsetstrokecolor{currentstroke}%
\pgfsetstrokeopacity{0.700000}%
\pgfsetdash{}{0pt}%
\pgfpathmoveto{\pgfqpoint{9.570389in}{1.144786in}}%
\pgfpathcurveto{\pgfqpoint{9.575432in}{1.144786in}}{\pgfqpoint{9.580270in}{1.146790in}}{\pgfqpoint{9.583836in}{1.150356in}}%
\pgfpathcurveto{\pgfqpoint{9.587403in}{1.153922in}}{\pgfqpoint{9.589407in}{1.158760in}}{\pgfqpoint{9.589407in}{1.163804in}}%
\pgfpathcurveto{\pgfqpoint{9.589407in}{1.168847in}}{\pgfqpoint{9.587403in}{1.173685in}}{\pgfqpoint{9.583836in}{1.177252in}}%
\pgfpathcurveto{\pgfqpoint{9.580270in}{1.180818in}}{\pgfqpoint{9.575432in}{1.182822in}}{\pgfqpoint{9.570389in}{1.182822in}}%
\pgfpathcurveto{\pgfqpoint{9.565345in}{1.182822in}}{\pgfqpoint{9.560507in}{1.180818in}}{\pgfqpoint{9.556941in}{1.177252in}}%
\pgfpathcurveto{\pgfqpoint{9.553374in}{1.173685in}}{\pgfqpoint{9.551370in}{1.168847in}}{\pgfqpoint{9.551370in}{1.163804in}}%
\pgfpathcurveto{\pgfqpoint{9.551370in}{1.158760in}}{\pgfqpoint{9.553374in}{1.153922in}}{\pgfqpoint{9.556941in}{1.150356in}}%
\pgfpathcurveto{\pgfqpoint{9.560507in}{1.146790in}}{\pgfqpoint{9.565345in}{1.144786in}}{\pgfqpoint{9.570389in}{1.144786in}}%
\pgfpathclose%
\pgfusepath{fill}%
\end{pgfscope}%
\begin{pgfscope}%
\pgfpathrectangle{\pgfqpoint{6.572727in}{0.474100in}}{\pgfqpoint{4.227273in}{3.318700in}}%
\pgfusepath{clip}%
\pgfsetbuttcap%
\pgfsetroundjoin%
\definecolor{currentfill}{rgb}{0.127568,0.566949,0.550556}%
\pgfsetfillcolor{currentfill}%
\pgfsetfillopacity{0.700000}%
\pgfsetlinewidth{0.000000pt}%
\definecolor{currentstroke}{rgb}{0.000000,0.000000,0.000000}%
\pgfsetstrokecolor{currentstroke}%
\pgfsetstrokeopacity{0.700000}%
\pgfsetdash{}{0pt}%
\pgfpathmoveto{\pgfqpoint{9.253134in}{1.280960in}}%
\pgfpathcurveto{\pgfqpoint{9.258177in}{1.280960in}}{\pgfqpoint{9.263015in}{1.282964in}}{\pgfqpoint{9.266581in}{1.286530in}}%
\pgfpathcurveto{\pgfqpoint{9.270148in}{1.290097in}}{\pgfqpoint{9.272152in}{1.294935in}}{\pgfqpoint{9.272152in}{1.299978in}}%
\pgfpathcurveto{\pgfqpoint{9.272152in}{1.305022in}}{\pgfqpoint{9.270148in}{1.309860in}}{\pgfqpoint{9.266581in}{1.313426in}}%
\pgfpathcurveto{\pgfqpoint{9.263015in}{1.316993in}}{\pgfqpoint{9.258177in}{1.318996in}}{\pgfqpoint{9.253134in}{1.318996in}}%
\pgfpathcurveto{\pgfqpoint{9.248090in}{1.318996in}}{\pgfqpoint{9.243252in}{1.316993in}}{\pgfqpoint{9.239686in}{1.313426in}}%
\pgfpathcurveto{\pgfqpoint{9.236119in}{1.309860in}}{\pgfqpoint{9.234115in}{1.305022in}}{\pgfqpoint{9.234115in}{1.299978in}}%
\pgfpathcurveto{\pgfqpoint{9.234115in}{1.294935in}}{\pgfqpoint{9.236119in}{1.290097in}}{\pgfqpoint{9.239686in}{1.286530in}}%
\pgfpathcurveto{\pgfqpoint{9.243252in}{1.282964in}}{\pgfqpoint{9.248090in}{1.280960in}}{\pgfqpoint{9.253134in}{1.280960in}}%
\pgfpathclose%
\pgfusepath{fill}%
\end{pgfscope}%
\begin{pgfscope}%
\pgfpathrectangle{\pgfqpoint{6.572727in}{0.474100in}}{\pgfqpoint{4.227273in}{3.318700in}}%
\pgfusepath{clip}%
\pgfsetbuttcap%
\pgfsetroundjoin%
\definecolor{currentfill}{rgb}{0.993248,0.906157,0.143936}%
\pgfsetfillcolor{currentfill}%
\pgfsetfillopacity{0.700000}%
\pgfsetlinewidth{0.000000pt}%
\definecolor{currentstroke}{rgb}{0.000000,0.000000,0.000000}%
\pgfsetstrokecolor{currentstroke}%
\pgfsetstrokeopacity{0.700000}%
\pgfsetdash{}{0pt}%
\pgfpathmoveto{\pgfqpoint{8.658814in}{2.451817in}}%
\pgfpathcurveto{\pgfqpoint{8.663858in}{2.451817in}}{\pgfqpoint{8.668696in}{2.453821in}}{\pgfqpoint{8.672262in}{2.457388in}}%
\pgfpathcurveto{\pgfqpoint{8.675829in}{2.460954in}}{\pgfqpoint{8.677833in}{2.465792in}}{\pgfqpoint{8.677833in}{2.470836in}}%
\pgfpathcurveto{\pgfqpoint{8.677833in}{2.475879in}}{\pgfqpoint{8.675829in}{2.480717in}}{\pgfqpoint{8.672262in}{2.484283in}}%
\pgfpathcurveto{\pgfqpoint{8.668696in}{2.487850in}}{\pgfqpoint{8.663858in}{2.489854in}}{\pgfqpoint{8.658814in}{2.489854in}}%
\pgfpathcurveto{\pgfqpoint{8.653771in}{2.489854in}}{\pgfqpoint{8.648933in}{2.487850in}}{\pgfqpoint{8.645367in}{2.484283in}}%
\pgfpathcurveto{\pgfqpoint{8.641800in}{2.480717in}}{\pgfqpoint{8.639796in}{2.475879in}}{\pgfqpoint{8.639796in}{2.470836in}}%
\pgfpathcurveto{\pgfqpoint{8.639796in}{2.465792in}}{\pgfqpoint{8.641800in}{2.460954in}}{\pgfqpoint{8.645367in}{2.457388in}}%
\pgfpathcurveto{\pgfqpoint{8.648933in}{2.453821in}}{\pgfqpoint{8.653771in}{2.451817in}}{\pgfqpoint{8.658814in}{2.451817in}}%
\pgfpathclose%
\pgfusepath{fill}%
\end{pgfscope}%
\begin{pgfscope}%
\pgfpathrectangle{\pgfqpoint{6.572727in}{0.474100in}}{\pgfqpoint{4.227273in}{3.318700in}}%
\pgfusepath{clip}%
\pgfsetbuttcap%
\pgfsetroundjoin%
\definecolor{currentfill}{rgb}{0.267004,0.004874,0.329415}%
\pgfsetfillcolor{currentfill}%
\pgfsetfillopacity{0.700000}%
\pgfsetlinewidth{0.000000pt}%
\definecolor{currentstroke}{rgb}{0.000000,0.000000,0.000000}%
\pgfsetstrokecolor{currentstroke}%
\pgfsetstrokeopacity{0.700000}%
\pgfsetdash{}{0pt}%
\pgfpathmoveto{\pgfqpoint{7.867878in}{0.993533in}}%
\pgfpathcurveto{\pgfqpoint{7.872922in}{0.993533in}}{\pgfqpoint{7.877760in}{0.995537in}}{\pgfqpoint{7.881326in}{0.999103in}}%
\pgfpathcurveto{\pgfqpoint{7.884892in}{1.002669in}}{\pgfqpoint{7.886896in}{1.007507in}}{\pgfqpoint{7.886896in}{1.012551in}}%
\pgfpathcurveto{\pgfqpoint{7.886896in}{1.017595in}}{\pgfqpoint{7.884892in}{1.022432in}}{\pgfqpoint{7.881326in}{1.025999in}}%
\pgfpathcurveto{\pgfqpoint{7.877760in}{1.029565in}}{\pgfqpoint{7.872922in}{1.031569in}}{\pgfqpoint{7.867878in}{1.031569in}}%
\pgfpathcurveto{\pgfqpoint{7.862834in}{1.031569in}}{\pgfqpoint{7.857997in}{1.029565in}}{\pgfqpoint{7.854430in}{1.025999in}}%
\pgfpathcurveto{\pgfqpoint{7.850864in}{1.022432in}}{\pgfqpoint{7.848860in}{1.017595in}}{\pgfqpoint{7.848860in}{1.012551in}}%
\pgfpathcurveto{\pgfqpoint{7.848860in}{1.007507in}}{\pgfqpoint{7.850864in}{1.002669in}}{\pgfqpoint{7.854430in}{0.999103in}}%
\pgfpathcurveto{\pgfqpoint{7.857997in}{0.995537in}}{\pgfqpoint{7.862834in}{0.993533in}}{\pgfqpoint{7.867878in}{0.993533in}}%
\pgfpathclose%
\pgfusepath{fill}%
\end{pgfscope}%
\begin{pgfscope}%
\pgfpathrectangle{\pgfqpoint{6.572727in}{0.474100in}}{\pgfqpoint{4.227273in}{3.318700in}}%
\pgfusepath{clip}%
\pgfsetbuttcap%
\pgfsetroundjoin%
\definecolor{currentfill}{rgb}{0.127568,0.566949,0.550556}%
\pgfsetfillcolor{currentfill}%
\pgfsetfillopacity{0.700000}%
\pgfsetlinewidth{0.000000pt}%
\definecolor{currentstroke}{rgb}{0.000000,0.000000,0.000000}%
\pgfsetstrokecolor{currentstroke}%
\pgfsetstrokeopacity{0.700000}%
\pgfsetdash{}{0pt}%
\pgfpathmoveto{\pgfqpoint{9.600641in}{1.318018in}}%
\pgfpathcurveto{\pgfqpoint{9.605684in}{1.318018in}}{\pgfqpoint{9.610522in}{1.320021in}}{\pgfqpoint{9.614088in}{1.323588in}}%
\pgfpathcurveto{\pgfqpoint{9.617655in}{1.327154in}}{\pgfqpoint{9.619659in}{1.331992in}}{\pgfqpoint{9.619659in}{1.337036in}}%
\pgfpathcurveto{\pgfqpoint{9.619659in}{1.342079in}}{\pgfqpoint{9.617655in}{1.346917in}}{\pgfqpoint{9.614088in}{1.350484in}}%
\pgfpathcurveto{\pgfqpoint{9.610522in}{1.354050in}}{\pgfqpoint{9.605684in}{1.356054in}}{\pgfqpoint{9.600641in}{1.356054in}}%
\pgfpathcurveto{\pgfqpoint{9.595597in}{1.356054in}}{\pgfqpoint{9.590759in}{1.354050in}}{\pgfqpoint{9.587193in}{1.350484in}}%
\pgfpathcurveto{\pgfqpoint{9.583626in}{1.346917in}}{\pgfqpoint{9.581622in}{1.342079in}}{\pgfqpoint{9.581622in}{1.337036in}}%
\pgfpathcurveto{\pgfqpoint{9.581622in}{1.331992in}}{\pgfqpoint{9.583626in}{1.327154in}}{\pgfqpoint{9.587193in}{1.323588in}}%
\pgfpathcurveto{\pgfqpoint{9.590759in}{1.320021in}}{\pgfqpoint{9.595597in}{1.318018in}}{\pgfqpoint{9.600641in}{1.318018in}}%
\pgfpathclose%
\pgfusepath{fill}%
\end{pgfscope}%
\begin{pgfscope}%
\pgfpathrectangle{\pgfqpoint{6.572727in}{0.474100in}}{\pgfqpoint{4.227273in}{3.318700in}}%
\pgfusepath{clip}%
\pgfsetbuttcap%
\pgfsetroundjoin%
\definecolor{currentfill}{rgb}{0.993248,0.906157,0.143936}%
\pgfsetfillcolor{currentfill}%
\pgfsetfillopacity{0.700000}%
\pgfsetlinewidth{0.000000pt}%
\definecolor{currentstroke}{rgb}{0.000000,0.000000,0.000000}%
\pgfsetstrokecolor{currentstroke}%
\pgfsetstrokeopacity{0.700000}%
\pgfsetdash{}{0pt}%
\pgfpathmoveto{\pgfqpoint{7.764700in}{3.369982in}}%
\pgfpathcurveto{\pgfqpoint{7.769744in}{3.369982in}}{\pgfqpoint{7.774582in}{3.371986in}}{\pgfqpoint{7.778148in}{3.375552in}}%
\pgfpathcurveto{\pgfqpoint{7.781714in}{3.379119in}}{\pgfqpoint{7.783718in}{3.383956in}}{\pgfqpoint{7.783718in}{3.389000in}}%
\pgfpathcurveto{\pgfqpoint{7.783718in}{3.394044in}}{\pgfqpoint{7.781714in}{3.398882in}}{\pgfqpoint{7.778148in}{3.402448in}}%
\pgfpathcurveto{\pgfqpoint{7.774582in}{3.406014in}}{\pgfqpoint{7.769744in}{3.408018in}}{\pgfqpoint{7.764700in}{3.408018in}}%
\pgfpathcurveto{\pgfqpoint{7.759657in}{3.408018in}}{\pgfqpoint{7.754819in}{3.406014in}}{\pgfqpoint{7.751252in}{3.402448in}}%
\pgfpathcurveto{\pgfqpoint{7.747686in}{3.398882in}}{\pgfqpoint{7.745682in}{3.394044in}}{\pgfqpoint{7.745682in}{3.389000in}}%
\pgfpathcurveto{\pgfqpoint{7.745682in}{3.383956in}}{\pgfqpoint{7.747686in}{3.379119in}}{\pgfqpoint{7.751252in}{3.375552in}}%
\pgfpathcurveto{\pgfqpoint{7.754819in}{3.371986in}}{\pgfqpoint{7.759657in}{3.369982in}}{\pgfqpoint{7.764700in}{3.369982in}}%
\pgfpathclose%
\pgfusepath{fill}%
\end{pgfscope}%
\begin{pgfscope}%
\pgfpathrectangle{\pgfqpoint{6.572727in}{0.474100in}}{\pgfqpoint{4.227273in}{3.318700in}}%
\pgfusepath{clip}%
\pgfsetbuttcap%
\pgfsetroundjoin%
\definecolor{currentfill}{rgb}{0.127568,0.566949,0.550556}%
\pgfsetfillcolor{currentfill}%
\pgfsetfillopacity{0.700000}%
\pgfsetlinewidth{0.000000pt}%
\definecolor{currentstroke}{rgb}{0.000000,0.000000,0.000000}%
\pgfsetstrokecolor{currentstroke}%
\pgfsetstrokeopacity{0.700000}%
\pgfsetdash{}{0pt}%
\pgfpathmoveto{\pgfqpoint{10.262451in}{1.458038in}}%
\pgfpathcurveto{\pgfqpoint{10.267495in}{1.458038in}}{\pgfqpoint{10.272332in}{1.460042in}}{\pgfqpoint{10.275899in}{1.463608in}}%
\pgfpathcurveto{\pgfqpoint{10.279465in}{1.467175in}}{\pgfqpoint{10.281469in}{1.472013in}}{\pgfqpoint{10.281469in}{1.477056in}}%
\pgfpathcurveto{\pgfqpoint{10.281469in}{1.482100in}}{\pgfqpoint{10.279465in}{1.486938in}}{\pgfqpoint{10.275899in}{1.490504in}}%
\pgfpathcurveto{\pgfqpoint{10.272332in}{1.494070in}}{\pgfqpoint{10.267495in}{1.496074in}}{\pgfqpoint{10.262451in}{1.496074in}}%
\pgfpathcurveto{\pgfqpoint{10.257407in}{1.496074in}}{\pgfqpoint{10.252570in}{1.494070in}}{\pgfqpoint{10.249003in}{1.490504in}}%
\pgfpathcurveto{\pgfqpoint{10.245437in}{1.486938in}}{\pgfqpoint{10.243433in}{1.482100in}}{\pgfqpoint{10.243433in}{1.477056in}}%
\pgfpathcurveto{\pgfqpoint{10.243433in}{1.472013in}}{\pgfqpoint{10.245437in}{1.467175in}}{\pgfqpoint{10.249003in}{1.463608in}}%
\pgfpathcurveto{\pgfqpoint{10.252570in}{1.460042in}}{\pgfqpoint{10.257407in}{1.458038in}}{\pgfqpoint{10.262451in}{1.458038in}}%
\pgfpathclose%
\pgfusepath{fill}%
\end{pgfscope}%
\begin{pgfscope}%
\pgfpathrectangle{\pgfqpoint{6.572727in}{0.474100in}}{\pgfqpoint{4.227273in}{3.318700in}}%
\pgfusepath{clip}%
\pgfsetbuttcap%
\pgfsetroundjoin%
\definecolor{currentfill}{rgb}{0.127568,0.566949,0.550556}%
\pgfsetfillcolor{currentfill}%
\pgfsetfillopacity{0.700000}%
\pgfsetlinewidth{0.000000pt}%
\definecolor{currentstroke}{rgb}{0.000000,0.000000,0.000000}%
\pgfsetstrokecolor{currentstroke}%
\pgfsetstrokeopacity{0.700000}%
\pgfsetdash{}{0pt}%
\pgfpathmoveto{\pgfqpoint{9.255775in}{1.308261in}}%
\pgfpathcurveto{\pgfqpoint{9.260819in}{1.308261in}}{\pgfqpoint{9.265656in}{1.310265in}}{\pgfqpoint{9.269223in}{1.313831in}}%
\pgfpathcurveto{\pgfqpoint{9.272789in}{1.317397in}}{\pgfqpoint{9.274793in}{1.322235in}}{\pgfqpoint{9.274793in}{1.327279in}}%
\pgfpathcurveto{\pgfqpoint{9.274793in}{1.332323in}}{\pgfqpoint{9.272789in}{1.337160in}}{\pgfqpoint{9.269223in}{1.340727in}}%
\pgfpathcurveto{\pgfqpoint{9.265656in}{1.344293in}}{\pgfqpoint{9.260819in}{1.346297in}}{\pgfqpoint{9.255775in}{1.346297in}}%
\pgfpathcurveto{\pgfqpoint{9.250731in}{1.346297in}}{\pgfqpoint{9.245894in}{1.344293in}}{\pgfqpoint{9.242327in}{1.340727in}}%
\pgfpathcurveto{\pgfqpoint{9.238761in}{1.337160in}}{\pgfqpoint{9.236757in}{1.332323in}}{\pgfqpoint{9.236757in}{1.327279in}}%
\pgfpathcurveto{\pgfqpoint{9.236757in}{1.322235in}}{\pgfqpoint{9.238761in}{1.317397in}}{\pgfqpoint{9.242327in}{1.313831in}}%
\pgfpathcurveto{\pgfqpoint{9.245894in}{1.310265in}}{\pgfqpoint{9.250731in}{1.308261in}}{\pgfqpoint{9.255775in}{1.308261in}}%
\pgfpathclose%
\pgfusepath{fill}%
\end{pgfscope}%
\begin{pgfscope}%
\pgfpathrectangle{\pgfqpoint{6.572727in}{0.474100in}}{\pgfqpoint{4.227273in}{3.318700in}}%
\pgfusepath{clip}%
\pgfsetbuttcap%
\pgfsetroundjoin%
\definecolor{currentfill}{rgb}{0.267004,0.004874,0.329415}%
\pgfsetfillcolor{currentfill}%
\pgfsetfillopacity{0.700000}%
\pgfsetlinewidth{0.000000pt}%
\definecolor{currentstroke}{rgb}{0.000000,0.000000,0.000000}%
\pgfsetstrokecolor{currentstroke}%
\pgfsetstrokeopacity{0.700000}%
\pgfsetdash{}{0pt}%
\pgfpathmoveto{\pgfqpoint{7.691415in}{1.439742in}}%
\pgfpathcurveto{\pgfqpoint{7.696459in}{1.439742in}}{\pgfqpoint{7.701297in}{1.441746in}}{\pgfqpoint{7.704863in}{1.445312in}}%
\pgfpathcurveto{\pgfqpoint{7.708430in}{1.448879in}}{\pgfqpoint{7.710433in}{1.453716in}}{\pgfqpoint{7.710433in}{1.458760in}}%
\pgfpathcurveto{\pgfqpoint{7.710433in}{1.463804in}}{\pgfqpoint{7.708430in}{1.468641in}}{\pgfqpoint{7.704863in}{1.472208in}}%
\pgfpathcurveto{\pgfqpoint{7.701297in}{1.475774in}}{\pgfqpoint{7.696459in}{1.477778in}}{\pgfqpoint{7.691415in}{1.477778in}}%
\pgfpathcurveto{\pgfqpoint{7.686372in}{1.477778in}}{\pgfqpoint{7.681534in}{1.475774in}}{\pgfqpoint{7.677967in}{1.472208in}}%
\pgfpathcurveto{\pgfqpoint{7.674401in}{1.468641in}}{\pgfqpoint{7.672397in}{1.463804in}}{\pgfqpoint{7.672397in}{1.458760in}}%
\pgfpathcurveto{\pgfqpoint{7.672397in}{1.453716in}}{\pgfqpoint{7.674401in}{1.448879in}}{\pgfqpoint{7.677967in}{1.445312in}}%
\pgfpathcurveto{\pgfqpoint{7.681534in}{1.441746in}}{\pgfqpoint{7.686372in}{1.439742in}}{\pgfqpoint{7.691415in}{1.439742in}}%
\pgfpathclose%
\pgfusepath{fill}%
\end{pgfscope}%
\begin{pgfscope}%
\pgfpathrectangle{\pgfqpoint{6.572727in}{0.474100in}}{\pgfqpoint{4.227273in}{3.318700in}}%
\pgfusepath{clip}%
\pgfsetbuttcap%
\pgfsetroundjoin%
\definecolor{currentfill}{rgb}{0.993248,0.906157,0.143936}%
\pgfsetfillcolor{currentfill}%
\pgfsetfillopacity{0.700000}%
\pgfsetlinewidth{0.000000pt}%
\definecolor{currentstroke}{rgb}{0.000000,0.000000,0.000000}%
\pgfsetstrokecolor{currentstroke}%
\pgfsetstrokeopacity{0.700000}%
\pgfsetdash{}{0pt}%
\pgfpathmoveto{\pgfqpoint{7.949581in}{2.582726in}}%
\pgfpathcurveto{\pgfqpoint{7.954624in}{2.582726in}}{\pgfqpoint{7.959462in}{2.584729in}}{\pgfqpoint{7.963029in}{2.588296in}}%
\pgfpathcurveto{\pgfqpoint{7.966595in}{2.591862in}}{\pgfqpoint{7.968599in}{2.596700in}}{\pgfqpoint{7.968599in}{2.601744in}}%
\pgfpathcurveto{\pgfqpoint{7.968599in}{2.606787in}}{\pgfqpoint{7.966595in}{2.611625in}}{\pgfqpoint{7.963029in}{2.615192in}}%
\pgfpathcurveto{\pgfqpoint{7.959462in}{2.618758in}}{\pgfqpoint{7.954624in}{2.620762in}}{\pgfqpoint{7.949581in}{2.620762in}}%
\pgfpathcurveto{\pgfqpoint{7.944537in}{2.620762in}}{\pgfqpoint{7.939699in}{2.618758in}}{\pgfqpoint{7.936133in}{2.615192in}}%
\pgfpathcurveto{\pgfqpoint{7.932566in}{2.611625in}}{\pgfqpoint{7.930563in}{2.606787in}}{\pgfqpoint{7.930563in}{2.601744in}}%
\pgfpathcurveto{\pgfqpoint{7.930563in}{2.596700in}}{\pgfqpoint{7.932566in}{2.591862in}}{\pgfqpoint{7.936133in}{2.588296in}}%
\pgfpathcurveto{\pgfqpoint{7.939699in}{2.584729in}}{\pgfqpoint{7.944537in}{2.582726in}}{\pgfqpoint{7.949581in}{2.582726in}}%
\pgfpathclose%
\pgfusepath{fill}%
\end{pgfscope}%
\begin{pgfscope}%
\pgfpathrectangle{\pgfqpoint{6.572727in}{0.474100in}}{\pgfqpoint{4.227273in}{3.318700in}}%
\pgfusepath{clip}%
\pgfsetbuttcap%
\pgfsetroundjoin%
\definecolor{currentfill}{rgb}{0.267004,0.004874,0.329415}%
\pgfsetfillcolor{currentfill}%
\pgfsetfillopacity{0.700000}%
\pgfsetlinewidth{0.000000pt}%
\definecolor{currentstroke}{rgb}{0.000000,0.000000,0.000000}%
\pgfsetstrokecolor{currentstroke}%
\pgfsetstrokeopacity{0.700000}%
\pgfsetdash{}{0pt}%
\pgfpathmoveto{\pgfqpoint{7.651143in}{1.871028in}}%
\pgfpathcurveto{\pgfqpoint{7.656187in}{1.871028in}}{\pgfqpoint{7.661025in}{1.873031in}}{\pgfqpoint{7.664591in}{1.876598in}}%
\pgfpathcurveto{\pgfqpoint{7.668157in}{1.880164in}}{\pgfqpoint{7.670161in}{1.885002in}}{\pgfqpoint{7.670161in}{1.890046in}}%
\pgfpathcurveto{\pgfqpoint{7.670161in}{1.895089in}}{\pgfqpoint{7.668157in}{1.899927in}}{\pgfqpoint{7.664591in}{1.903494in}}%
\pgfpathcurveto{\pgfqpoint{7.661025in}{1.907060in}}{\pgfqpoint{7.656187in}{1.909064in}}{\pgfqpoint{7.651143in}{1.909064in}}%
\pgfpathcurveto{\pgfqpoint{7.646099in}{1.909064in}}{\pgfqpoint{7.641262in}{1.907060in}}{\pgfqpoint{7.637695in}{1.903494in}}%
\pgfpathcurveto{\pgfqpoint{7.634129in}{1.899927in}}{\pgfqpoint{7.632125in}{1.895089in}}{\pgfqpoint{7.632125in}{1.890046in}}%
\pgfpathcurveto{\pgfqpoint{7.632125in}{1.885002in}}{\pgfqpoint{7.634129in}{1.880164in}}{\pgfqpoint{7.637695in}{1.876598in}}%
\pgfpathcurveto{\pgfqpoint{7.641262in}{1.873031in}}{\pgfqpoint{7.646099in}{1.871028in}}{\pgfqpoint{7.651143in}{1.871028in}}%
\pgfpathclose%
\pgfusepath{fill}%
\end{pgfscope}%
\begin{pgfscope}%
\pgfpathrectangle{\pgfqpoint{6.572727in}{0.474100in}}{\pgfqpoint{4.227273in}{3.318700in}}%
\pgfusepath{clip}%
\pgfsetbuttcap%
\pgfsetroundjoin%
\definecolor{currentfill}{rgb}{0.993248,0.906157,0.143936}%
\pgfsetfillcolor{currentfill}%
\pgfsetfillopacity{0.700000}%
\pgfsetlinewidth{0.000000pt}%
\definecolor{currentstroke}{rgb}{0.000000,0.000000,0.000000}%
\pgfsetstrokecolor{currentstroke}%
\pgfsetstrokeopacity{0.700000}%
\pgfsetdash{}{0pt}%
\pgfpathmoveto{\pgfqpoint{7.946003in}{2.235074in}}%
\pgfpathcurveto{\pgfqpoint{7.951046in}{2.235074in}}{\pgfqpoint{7.955884in}{2.237078in}}{\pgfqpoint{7.959451in}{2.240644in}}%
\pgfpathcurveto{\pgfqpoint{7.963017in}{2.244211in}}{\pgfqpoint{7.965021in}{2.249049in}}{\pgfqpoint{7.965021in}{2.254092in}}%
\pgfpathcurveto{\pgfqpoint{7.965021in}{2.259136in}}{\pgfqpoint{7.963017in}{2.263974in}}{\pgfqpoint{7.959451in}{2.267540in}}%
\pgfpathcurveto{\pgfqpoint{7.955884in}{2.271107in}}{\pgfqpoint{7.951046in}{2.273110in}}{\pgfqpoint{7.946003in}{2.273110in}}%
\pgfpathcurveto{\pgfqpoint{7.940959in}{2.273110in}}{\pgfqpoint{7.936121in}{2.271107in}}{\pgfqpoint{7.932555in}{2.267540in}}%
\pgfpathcurveto{\pgfqpoint{7.928988in}{2.263974in}}{\pgfqpoint{7.926985in}{2.259136in}}{\pgfqpoint{7.926985in}{2.254092in}}%
\pgfpathcurveto{\pgfqpoint{7.926985in}{2.249049in}}{\pgfqpoint{7.928988in}{2.244211in}}{\pgfqpoint{7.932555in}{2.240644in}}%
\pgfpathcurveto{\pgfqpoint{7.936121in}{2.237078in}}{\pgfqpoint{7.940959in}{2.235074in}}{\pgfqpoint{7.946003in}{2.235074in}}%
\pgfpathclose%
\pgfusepath{fill}%
\end{pgfscope}%
\begin{pgfscope}%
\pgfpathrectangle{\pgfqpoint{6.572727in}{0.474100in}}{\pgfqpoint{4.227273in}{3.318700in}}%
\pgfusepath{clip}%
\pgfsetbuttcap%
\pgfsetroundjoin%
\definecolor{currentfill}{rgb}{0.267004,0.004874,0.329415}%
\pgfsetfillcolor{currentfill}%
\pgfsetfillopacity{0.700000}%
\pgfsetlinewidth{0.000000pt}%
\definecolor{currentstroke}{rgb}{0.000000,0.000000,0.000000}%
\pgfsetstrokecolor{currentstroke}%
\pgfsetstrokeopacity{0.700000}%
\pgfsetdash{}{0pt}%
\pgfpathmoveto{\pgfqpoint{7.762822in}{1.076009in}}%
\pgfpathcurveto{\pgfqpoint{7.767866in}{1.076009in}}{\pgfqpoint{7.772703in}{1.078013in}}{\pgfqpoint{7.776270in}{1.081579in}}%
\pgfpathcurveto{\pgfqpoint{7.779836in}{1.085146in}}{\pgfqpoint{7.781840in}{1.089983in}}{\pgfqpoint{7.781840in}{1.095027in}}%
\pgfpathcurveto{\pgfqpoint{7.781840in}{1.100071in}}{\pgfqpoint{7.779836in}{1.104909in}}{\pgfqpoint{7.776270in}{1.108475in}}%
\pgfpathcurveto{\pgfqpoint{7.772703in}{1.112041in}}{\pgfqpoint{7.767866in}{1.114045in}}{\pgfqpoint{7.762822in}{1.114045in}}%
\pgfpathcurveto{\pgfqpoint{7.757778in}{1.114045in}}{\pgfqpoint{7.752941in}{1.112041in}}{\pgfqpoint{7.749374in}{1.108475in}}%
\pgfpathcurveto{\pgfqpoint{7.745808in}{1.104909in}}{\pgfqpoint{7.743804in}{1.100071in}}{\pgfqpoint{7.743804in}{1.095027in}}%
\pgfpathcurveto{\pgfqpoint{7.743804in}{1.089983in}}{\pgfqpoint{7.745808in}{1.085146in}}{\pgfqpoint{7.749374in}{1.081579in}}%
\pgfpathcurveto{\pgfqpoint{7.752941in}{1.078013in}}{\pgfqpoint{7.757778in}{1.076009in}}{\pgfqpoint{7.762822in}{1.076009in}}%
\pgfpathclose%
\pgfusepath{fill}%
\end{pgfscope}%
\begin{pgfscope}%
\pgfpathrectangle{\pgfqpoint{6.572727in}{0.474100in}}{\pgfqpoint{4.227273in}{3.318700in}}%
\pgfusepath{clip}%
\pgfsetbuttcap%
\pgfsetroundjoin%
\definecolor{currentfill}{rgb}{0.267004,0.004874,0.329415}%
\pgfsetfillcolor{currentfill}%
\pgfsetfillopacity{0.700000}%
\pgfsetlinewidth{0.000000pt}%
\definecolor{currentstroke}{rgb}{0.000000,0.000000,0.000000}%
\pgfsetstrokecolor{currentstroke}%
\pgfsetstrokeopacity{0.700000}%
\pgfsetdash{}{0pt}%
\pgfpathmoveto{\pgfqpoint{8.124940in}{1.511999in}}%
\pgfpathcurveto{\pgfqpoint{8.129983in}{1.511999in}}{\pgfqpoint{8.134821in}{1.514003in}}{\pgfqpoint{8.138387in}{1.517569in}}%
\pgfpathcurveto{\pgfqpoint{8.141954in}{1.521136in}}{\pgfqpoint{8.143958in}{1.525973in}}{\pgfqpoint{8.143958in}{1.531017in}}%
\pgfpathcurveto{\pgfqpoint{8.143958in}{1.536061in}}{\pgfqpoint{8.141954in}{1.540899in}}{\pgfqpoint{8.138387in}{1.544465in}}%
\pgfpathcurveto{\pgfqpoint{8.134821in}{1.548031in}}{\pgfqpoint{8.129983in}{1.550035in}}{\pgfqpoint{8.124940in}{1.550035in}}%
\pgfpathcurveto{\pgfqpoint{8.119896in}{1.550035in}}{\pgfqpoint{8.115058in}{1.548031in}}{\pgfqpoint{8.111492in}{1.544465in}}%
\pgfpathcurveto{\pgfqpoint{8.107925in}{1.540899in}}{\pgfqpoint{8.105921in}{1.536061in}}{\pgfqpoint{8.105921in}{1.531017in}}%
\pgfpathcurveto{\pgfqpoint{8.105921in}{1.525973in}}{\pgfqpoint{8.107925in}{1.521136in}}{\pgfqpoint{8.111492in}{1.517569in}}%
\pgfpathcurveto{\pgfqpoint{8.115058in}{1.514003in}}{\pgfqpoint{8.119896in}{1.511999in}}{\pgfqpoint{8.124940in}{1.511999in}}%
\pgfpathclose%
\pgfusepath{fill}%
\end{pgfscope}%
\begin{pgfscope}%
\pgfpathrectangle{\pgfqpoint{6.572727in}{0.474100in}}{\pgfqpoint{4.227273in}{3.318700in}}%
\pgfusepath{clip}%
\pgfsetbuttcap%
\pgfsetroundjoin%
\definecolor{currentfill}{rgb}{0.127568,0.566949,0.550556}%
\pgfsetfillcolor{currentfill}%
\pgfsetfillopacity{0.700000}%
\pgfsetlinewidth{0.000000pt}%
\definecolor{currentstroke}{rgb}{0.000000,0.000000,0.000000}%
\pgfsetstrokecolor{currentstroke}%
\pgfsetstrokeopacity{0.700000}%
\pgfsetdash{}{0pt}%
\pgfpathmoveto{\pgfqpoint{10.010201in}{1.875300in}}%
\pgfpathcurveto{\pgfqpoint{10.015244in}{1.875300in}}{\pgfqpoint{10.020082in}{1.877304in}}{\pgfqpoint{10.023649in}{1.880870in}}%
\pgfpathcurveto{\pgfqpoint{10.027215in}{1.884437in}}{\pgfqpoint{10.029219in}{1.889275in}}{\pgfqpoint{10.029219in}{1.894318in}}%
\pgfpathcurveto{\pgfqpoint{10.029219in}{1.899362in}}{\pgfqpoint{10.027215in}{1.904200in}}{\pgfqpoint{10.023649in}{1.907766in}}%
\pgfpathcurveto{\pgfqpoint{10.020082in}{1.911332in}}{\pgfqpoint{10.015244in}{1.913336in}}{\pgfqpoint{10.010201in}{1.913336in}}%
\pgfpathcurveto{\pgfqpoint{10.005157in}{1.913336in}}{\pgfqpoint{10.000319in}{1.911332in}}{\pgfqpoint{9.996753in}{1.907766in}}%
\pgfpathcurveto{\pgfqpoint{9.993186in}{1.904200in}}{\pgfqpoint{9.991183in}{1.899362in}}{\pgfqpoint{9.991183in}{1.894318in}}%
\pgfpathcurveto{\pgfqpoint{9.991183in}{1.889275in}}{\pgfqpoint{9.993186in}{1.884437in}}{\pgfqpoint{9.996753in}{1.880870in}}%
\pgfpathcurveto{\pgfqpoint{10.000319in}{1.877304in}}{\pgfqpoint{10.005157in}{1.875300in}}{\pgfqpoint{10.010201in}{1.875300in}}%
\pgfpathclose%
\pgfusepath{fill}%
\end{pgfscope}%
\begin{pgfscope}%
\pgfpathrectangle{\pgfqpoint{6.572727in}{0.474100in}}{\pgfqpoint{4.227273in}{3.318700in}}%
\pgfusepath{clip}%
\pgfsetbuttcap%
\pgfsetroundjoin%
\definecolor{currentfill}{rgb}{0.993248,0.906157,0.143936}%
\pgfsetfillcolor{currentfill}%
\pgfsetfillopacity{0.700000}%
\pgfsetlinewidth{0.000000pt}%
\definecolor{currentstroke}{rgb}{0.000000,0.000000,0.000000}%
\pgfsetstrokecolor{currentstroke}%
\pgfsetstrokeopacity{0.700000}%
\pgfsetdash{}{0pt}%
\pgfpathmoveto{\pgfqpoint{8.010571in}{2.680736in}}%
\pgfpathcurveto{\pgfqpoint{8.015614in}{2.680736in}}{\pgfqpoint{8.020452in}{2.682740in}}{\pgfqpoint{8.024018in}{2.686307in}}%
\pgfpathcurveto{\pgfqpoint{8.027585in}{2.689873in}}{\pgfqpoint{8.029589in}{2.694711in}}{\pgfqpoint{8.029589in}{2.699754in}}%
\pgfpathcurveto{\pgfqpoint{8.029589in}{2.704798in}}{\pgfqpoint{8.027585in}{2.709636in}}{\pgfqpoint{8.024018in}{2.713202in}}%
\pgfpathcurveto{\pgfqpoint{8.020452in}{2.716769in}}{\pgfqpoint{8.015614in}{2.718773in}}{\pgfqpoint{8.010571in}{2.718773in}}%
\pgfpathcurveto{\pgfqpoint{8.005527in}{2.718773in}}{\pgfqpoint{8.000689in}{2.716769in}}{\pgfqpoint{7.997123in}{2.713202in}}%
\pgfpathcurveto{\pgfqpoint{7.993556in}{2.709636in}}{\pgfqpoint{7.991552in}{2.704798in}}{\pgfqpoint{7.991552in}{2.699754in}}%
\pgfpathcurveto{\pgfqpoint{7.991552in}{2.694711in}}{\pgfqpoint{7.993556in}{2.689873in}}{\pgfqpoint{7.997123in}{2.686307in}}%
\pgfpathcurveto{\pgfqpoint{8.000689in}{2.682740in}}{\pgfqpoint{8.005527in}{2.680736in}}{\pgfqpoint{8.010571in}{2.680736in}}%
\pgfpathclose%
\pgfusepath{fill}%
\end{pgfscope}%
\begin{pgfscope}%
\pgfpathrectangle{\pgfqpoint{6.572727in}{0.474100in}}{\pgfqpoint{4.227273in}{3.318700in}}%
\pgfusepath{clip}%
\pgfsetbuttcap%
\pgfsetroundjoin%
\definecolor{currentfill}{rgb}{0.127568,0.566949,0.550556}%
\pgfsetfillcolor{currentfill}%
\pgfsetfillopacity{0.700000}%
\pgfsetlinewidth{0.000000pt}%
\definecolor{currentstroke}{rgb}{0.000000,0.000000,0.000000}%
\pgfsetstrokecolor{currentstroke}%
\pgfsetstrokeopacity{0.700000}%
\pgfsetdash{}{0pt}%
\pgfpathmoveto{\pgfqpoint{10.233045in}{1.081419in}}%
\pgfpathcurveto{\pgfqpoint{10.238089in}{1.081419in}}{\pgfqpoint{10.242927in}{1.083422in}}{\pgfqpoint{10.246493in}{1.086989in}}%
\pgfpathcurveto{\pgfqpoint{10.250060in}{1.090555in}}{\pgfqpoint{10.252064in}{1.095393in}}{\pgfqpoint{10.252064in}{1.100437in}}%
\pgfpathcurveto{\pgfqpoint{10.252064in}{1.105480in}}{\pgfqpoint{10.250060in}{1.110318in}}{\pgfqpoint{10.246493in}{1.113885in}}%
\pgfpathcurveto{\pgfqpoint{10.242927in}{1.117451in}}{\pgfqpoint{10.238089in}{1.119455in}}{\pgfqpoint{10.233045in}{1.119455in}}%
\pgfpathcurveto{\pgfqpoint{10.228002in}{1.119455in}}{\pgfqpoint{10.223164in}{1.117451in}}{\pgfqpoint{10.219598in}{1.113885in}}%
\pgfpathcurveto{\pgfqpoint{10.216031in}{1.110318in}}{\pgfqpoint{10.214027in}{1.105480in}}{\pgfqpoint{10.214027in}{1.100437in}}%
\pgfpathcurveto{\pgfqpoint{10.214027in}{1.095393in}}{\pgfqpoint{10.216031in}{1.090555in}}{\pgfqpoint{10.219598in}{1.086989in}}%
\pgfpathcurveto{\pgfqpoint{10.223164in}{1.083422in}}{\pgfqpoint{10.228002in}{1.081419in}}{\pgfqpoint{10.233045in}{1.081419in}}%
\pgfpathclose%
\pgfusepath{fill}%
\end{pgfscope}%
\begin{pgfscope}%
\pgfpathrectangle{\pgfqpoint{6.572727in}{0.474100in}}{\pgfqpoint{4.227273in}{3.318700in}}%
\pgfusepath{clip}%
\pgfsetbuttcap%
\pgfsetroundjoin%
\definecolor{currentfill}{rgb}{0.267004,0.004874,0.329415}%
\pgfsetfillcolor{currentfill}%
\pgfsetfillopacity{0.700000}%
\pgfsetlinewidth{0.000000pt}%
\definecolor{currentstroke}{rgb}{0.000000,0.000000,0.000000}%
\pgfsetstrokecolor{currentstroke}%
\pgfsetstrokeopacity{0.700000}%
\pgfsetdash{}{0pt}%
\pgfpathmoveto{\pgfqpoint{7.491528in}{1.482655in}}%
\pgfpathcurveto{\pgfqpoint{7.496572in}{1.482655in}}{\pgfqpoint{7.501409in}{1.484659in}}{\pgfqpoint{7.504976in}{1.488225in}}%
\pgfpathcurveto{\pgfqpoint{7.508542in}{1.491791in}}{\pgfqpoint{7.510546in}{1.496629in}}{\pgfqpoint{7.510546in}{1.501673in}}%
\pgfpathcurveto{\pgfqpoint{7.510546in}{1.506716in}}{\pgfqpoint{7.508542in}{1.511554in}}{\pgfqpoint{7.504976in}{1.515121in}}%
\pgfpathcurveto{\pgfqpoint{7.501409in}{1.518687in}}{\pgfqpoint{7.496572in}{1.520691in}}{\pgfqpoint{7.491528in}{1.520691in}}%
\pgfpathcurveto{\pgfqpoint{7.486484in}{1.520691in}}{\pgfqpoint{7.481646in}{1.518687in}}{\pgfqpoint{7.478080in}{1.515121in}}%
\pgfpathcurveto{\pgfqpoint{7.474514in}{1.511554in}}{\pgfqpoint{7.472510in}{1.506716in}}{\pgfqpoint{7.472510in}{1.501673in}}%
\pgfpathcurveto{\pgfqpoint{7.472510in}{1.496629in}}{\pgfqpoint{7.474514in}{1.491791in}}{\pgfqpoint{7.478080in}{1.488225in}}%
\pgfpathcurveto{\pgfqpoint{7.481646in}{1.484659in}}{\pgfqpoint{7.486484in}{1.482655in}}{\pgfqpoint{7.491528in}{1.482655in}}%
\pgfpathclose%
\pgfusepath{fill}%
\end{pgfscope}%
\begin{pgfscope}%
\pgfpathrectangle{\pgfqpoint{6.572727in}{0.474100in}}{\pgfqpoint{4.227273in}{3.318700in}}%
\pgfusepath{clip}%
\pgfsetbuttcap%
\pgfsetroundjoin%
\definecolor{currentfill}{rgb}{0.267004,0.004874,0.329415}%
\pgfsetfillcolor{currentfill}%
\pgfsetfillopacity{0.700000}%
\pgfsetlinewidth{0.000000pt}%
\definecolor{currentstroke}{rgb}{0.000000,0.000000,0.000000}%
\pgfsetstrokecolor{currentstroke}%
\pgfsetstrokeopacity{0.700000}%
\pgfsetdash{}{0pt}%
\pgfpathmoveto{\pgfqpoint{8.031887in}{1.319203in}}%
\pgfpathcurveto{\pgfqpoint{8.036930in}{1.319203in}}{\pgfqpoint{8.041768in}{1.321207in}}{\pgfqpoint{8.045334in}{1.324773in}}%
\pgfpathcurveto{\pgfqpoint{8.048901in}{1.328340in}}{\pgfqpoint{8.050905in}{1.333177in}}{\pgfqpoint{8.050905in}{1.338221in}}%
\pgfpathcurveto{\pgfqpoint{8.050905in}{1.343265in}}{\pgfqpoint{8.048901in}{1.348103in}}{\pgfqpoint{8.045334in}{1.351669in}}%
\pgfpathcurveto{\pgfqpoint{8.041768in}{1.355235in}}{\pgfqpoint{8.036930in}{1.357239in}}{\pgfqpoint{8.031887in}{1.357239in}}%
\pgfpathcurveto{\pgfqpoint{8.026843in}{1.357239in}}{\pgfqpoint{8.022005in}{1.355235in}}{\pgfqpoint{8.018439in}{1.351669in}}%
\pgfpathcurveto{\pgfqpoint{8.014872in}{1.348103in}}{\pgfqpoint{8.012868in}{1.343265in}}{\pgfqpoint{8.012868in}{1.338221in}}%
\pgfpathcurveto{\pgfqpoint{8.012868in}{1.333177in}}{\pgfqpoint{8.014872in}{1.328340in}}{\pgfqpoint{8.018439in}{1.324773in}}%
\pgfpathcurveto{\pgfqpoint{8.022005in}{1.321207in}}{\pgfqpoint{8.026843in}{1.319203in}}{\pgfqpoint{8.031887in}{1.319203in}}%
\pgfpathclose%
\pgfusepath{fill}%
\end{pgfscope}%
\begin{pgfscope}%
\pgfpathrectangle{\pgfqpoint{6.572727in}{0.474100in}}{\pgfqpoint{4.227273in}{3.318700in}}%
\pgfusepath{clip}%
\pgfsetbuttcap%
\pgfsetroundjoin%
\definecolor{currentfill}{rgb}{0.993248,0.906157,0.143936}%
\pgfsetfillcolor{currentfill}%
\pgfsetfillopacity{0.700000}%
\pgfsetlinewidth{0.000000pt}%
\definecolor{currentstroke}{rgb}{0.000000,0.000000,0.000000}%
\pgfsetstrokecolor{currentstroke}%
\pgfsetstrokeopacity{0.700000}%
\pgfsetdash{}{0pt}%
\pgfpathmoveto{\pgfqpoint{8.073258in}{2.338818in}}%
\pgfpathcurveto{\pgfqpoint{8.078302in}{2.338818in}}{\pgfqpoint{8.083139in}{2.340822in}}{\pgfqpoint{8.086706in}{2.344389in}}%
\pgfpathcurveto{\pgfqpoint{8.090272in}{2.347955in}}{\pgfqpoint{8.092276in}{2.352793in}}{\pgfqpoint{8.092276in}{2.357837in}}%
\pgfpathcurveto{\pgfqpoint{8.092276in}{2.362880in}}{\pgfqpoint{8.090272in}{2.367718in}}{\pgfqpoint{8.086706in}{2.371284in}}%
\pgfpathcurveto{\pgfqpoint{8.083139in}{2.374851in}}{\pgfqpoint{8.078302in}{2.376855in}}{\pgfqpoint{8.073258in}{2.376855in}}%
\pgfpathcurveto{\pgfqpoint{8.068214in}{2.376855in}}{\pgfqpoint{8.063376in}{2.374851in}}{\pgfqpoint{8.059810in}{2.371284in}}%
\pgfpathcurveto{\pgfqpoint{8.056244in}{2.367718in}}{\pgfqpoint{8.054240in}{2.362880in}}{\pgfqpoint{8.054240in}{2.357837in}}%
\pgfpathcurveto{\pgfqpoint{8.054240in}{2.352793in}}{\pgfqpoint{8.056244in}{2.347955in}}{\pgfqpoint{8.059810in}{2.344389in}}%
\pgfpathcurveto{\pgfqpoint{8.063376in}{2.340822in}}{\pgfqpoint{8.068214in}{2.338818in}}{\pgfqpoint{8.073258in}{2.338818in}}%
\pgfpathclose%
\pgfusepath{fill}%
\end{pgfscope}%
\begin{pgfscope}%
\pgfpathrectangle{\pgfqpoint{6.572727in}{0.474100in}}{\pgfqpoint{4.227273in}{3.318700in}}%
\pgfusepath{clip}%
\pgfsetbuttcap%
\pgfsetroundjoin%
\definecolor{currentfill}{rgb}{0.267004,0.004874,0.329415}%
\pgfsetfillcolor{currentfill}%
\pgfsetfillopacity{0.700000}%
\pgfsetlinewidth{0.000000pt}%
\definecolor{currentstroke}{rgb}{0.000000,0.000000,0.000000}%
\pgfsetstrokecolor{currentstroke}%
\pgfsetstrokeopacity{0.700000}%
\pgfsetdash{}{0pt}%
\pgfpathmoveto{\pgfqpoint{7.636168in}{1.266710in}}%
\pgfpathcurveto{\pgfqpoint{7.641212in}{1.266710in}}{\pgfqpoint{7.646049in}{1.268714in}}{\pgfqpoint{7.649616in}{1.272280in}}%
\pgfpathcurveto{\pgfqpoint{7.653182in}{1.275847in}}{\pgfqpoint{7.655186in}{1.280685in}}{\pgfqpoint{7.655186in}{1.285728in}}%
\pgfpathcurveto{\pgfqpoint{7.655186in}{1.290772in}}{\pgfqpoint{7.653182in}{1.295610in}}{\pgfqpoint{7.649616in}{1.299176in}}%
\pgfpathcurveto{\pgfqpoint{7.646049in}{1.302743in}}{\pgfqpoint{7.641212in}{1.304746in}}{\pgfqpoint{7.636168in}{1.304746in}}%
\pgfpathcurveto{\pgfqpoint{7.631124in}{1.304746in}}{\pgfqpoint{7.626286in}{1.302743in}}{\pgfqpoint{7.622720in}{1.299176in}}%
\pgfpathcurveto{\pgfqpoint{7.619154in}{1.295610in}}{\pgfqpoint{7.617150in}{1.290772in}}{\pgfqpoint{7.617150in}{1.285728in}}%
\pgfpathcurveto{\pgfqpoint{7.617150in}{1.280685in}}{\pgfqpoint{7.619154in}{1.275847in}}{\pgfqpoint{7.622720in}{1.272280in}}%
\pgfpathcurveto{\pgfqpoint{7.626286in}{1.268714in}}{\pgfqpoint{7.631124in}{1.266710in}}{\pgfqpoint{7.636168in}{1.266710in}}%
\pgfpathclose%
\pgfusepath{fill}%
\end{pgfscope}%
\begin{pgfscope}%
\pgfpathrectangle{\pgfqpoint{6.572727in}{0.474100in}}{\pgfqpoint{4.227273in}{3.318700in}}%
\pgfusepath{clip}%
\pgfsetbuttcap%
\pgfsetroundjoin%
\definecolor{currentfill}{rgb}{0.127568,0.566949,0.550556}%
\pgfsetfillcolor{currentfill}%
\pgfsetfillopacity{0.700000}%
\pgfsetlinewidth{0.000000pt}%
\definecolor{currentstroke}{rgb}{0.000000,0.000000,0.000000}%
\pgfsetstrokecolor{currentstroke}%
\pgfsetstrokeopacity{0.700000}%
\pgfsetdash{}{0pt}%
\pgfpathmoveto{\pgfqpoint{9.859125in}{1.753705in}}%
\pgfpathcurveto{\pgfqpoint{9.864169in}{1.753705in}}{\pgfqpoint{9.869007in}{1.755709in}}{\pgfqpoint{9.872573in}{1.759275in}}%
\pgfpathcurveto{\pgfqpoint{9.876139in}{1.762842in}}{\pgfqpoint{9.878143in}{1.767680in}}{\pgfqpoint{9.878143in}{1.772723in}}%
\pgfpathcurveto{\pgfqpoint{9.878143in}{1.777767in}}{\pgfqpoint{9.876139in}{1.782605in}}{\pgfqpoint{9.872573in}{1.786171in}}%
\pgfpathcurveto{\pgfqpoint{9.869007in}{1.789738in}}{\pgfqpoint{9.864169in}{1.791741in}}{\pgfqpoint{9.859125in}{1.791741in}}%
\pgfpathcurveto{\pgfqpoint{9.854082in}{1.791741in}}{\pgfqpoint{9.849244in}{1.789738in}}{\pgfqpoint{9.845677in}{1.786171in}}%
\pgfpathcurveto{\pgfqpoint{9.842111in}{1.782605in}}{\pgfqpoint{9.840107in}{1.777767in}}{\pgfqpoint{9.840107in}{1.772723in}}%
\pgfpathcurveto{\pgfqpoint{9.840107in}{1.767680in}}{\pgfqpoint{9.842111in}{1.762842in}}{\pgfqpoint{9.845677in}{1.759275in}}%
\pgfpathcurveto{\pgfqpoint{9.849244in}{1.755709in}}{\pgfqpoint{9.854082in}{1.753705in}}{\pgfqpoint{9.859125in}{1.753705in}}%
\pgfpathclose%
\pgfusepath{fill}%
\end{pgfscope}%
\begin{pgfscope}%
\pgfpathrectangle{\pgfqpoint{6.572727in}{0.474100in}}{\pgfqpoint{4.227273in}{3.318700in}}%
\pgfusepath{clip}%
\pgfsetbuttcap%
\pgfsetroundjoin%
\definecolor{currentfill}{rgb}{0.267004,0.004874,0.329415}%
\pgfsetfillcolor{currentfill}%
\pgfsetfillopacity{0.700000}%
\pgfsetlinewidth{0.000000pt}%
\definecolor{currentstroke}{rgb}{0.000000,0.000000,0.000000}%
\pgfsetstrokecolor{currentstroke}%
\pgfsetstrokeopacity{0.700000}%
\pgfsetdash{}{0pt}%
\pgfpathmoveto{\pgfqpoint{7.248329in}{1.327273in}}%
\pgfpathcurveto{\pgfqpoint{7.253373in}{1.327273in}}{\pgfqpoint{7.258211in}{1.329277in}}{\pgfqpoint{7.261777in}{1.332843in}}%
\pgfpathcurveto{\pgfqpoint{7.265344in}{1.336410in}}{\pgfqpoint{7.267347in}{1.341248in}}{\pgfqpoint{7.267347in}{1.346291in}}%
\pgfpathcurveto{\pgfqpoint{7.267347in}{1.351335in}}{\pgfqpoint{7.265344in}{1.356173in}}{\pgfqpoint{7.261777in}{1.359739in}}%
\pgfpathcurveto{\pgfqpoint{7.258211in}{1.363306in}}{\pgfqpoint{7.253373in}{1.365309in}}{\pgfqpoint{7.248329in}{1.365309in}}%
\pgfpathcurveto{\pgfqpoint{7.243286in}{1.365309in}}{\pgfqpoint{7.238448in}{1.363306in}}{\pgfqpoint{7.234881in}{1.359739in}}%
\pgfpathcurveto{\pgfqpoint{7.231315in}{1.356173in}}{\pgfqpoint{7.229311in}{1.351335in}}{\pgfqpoint{7.229311in}{1.346291in}}%
\pgfpathcurveto{\pgfqpoint{7.229311in}{1.341248in}}{\pgfqpoint{7.231315in}{1.336410in}}{\pgfqpoint{7.234881in}{1.332843in}}%
\pgfpathcurveto{\pgfqpoint{7.238448in}{1.329277in}}{\pgfqpoint{7.243286in}{1.327273in}}{\pgfqpoint{7.248329in}{1.327273in}}%
\pgfpathclose%
\pgfusepath{fill}%
\end{pgfscope}%
\begin{pgfscope}%
\pgfpathrectangle{\pgfqpoint{6.572727in}{0.474100in}}{\pgfqpoint{4.227273in}{3.318700in}}%
\pgfusepath{clip}%
\pgfsetbuttcap%
\pgfsetroundjoin%
\definecolor{currentfill}{rgb}{0.127568,0.566949,0.550556}%
\pgfsetfillcolor{currentfill}%
\pgfsetfillopacity{0.700000}%
\pgfsetlinewidth{0.000000pt}%
\definecolor{currentstroke}{rgb}{0.000000,0.000000,0.000000}%
\pgfsetstrokecolor{currentstroke}%
\pgfsetstrokeopacity{0.700000}%
\pgfsetdash{}{0pt}%
\pgfpathmoveto{\pgfqpoint{9.765802in}{1.949143in}}%
\pgfpathcurveto{\pgfqpoint{9.770846in}{1.949143in}}{\pgfqpoint{9.775684in}{1.951147in}}{\pgfqpoint{9.779250in}{1.954713in}}%
\pgfpathcurveto{\pgfqpoint{9.782817in}{1.958280in}}{\pgfqpoint{9.784820in}{1.963117in}}{\pgfqpoint{9.784820in}{1.968161in}}%
\pgfpathcurveto{\pgfqpoint{9.784820in}{1.973205in}}{\pgfqpoint{9.782817in}{1.978043in}}{\pgfqpoint{9.779250in}{1.981609in}}%
\pgfpathcurveto{\pgfqpoint{9.775684in}{1.985175in}}{\pgfqpoint{9.770846in}{1.987179in}}{\pgfqpoint{9.765802in}{1.987179in}}%
\pgfpathcurveto{\pgfqpoint{9.760759in}{1.987179in}}{\pgfqpoint{9.755921in}{1.985175in}}{\pgfqpoint{9.752354in}{1.981609in}}%
\pgfpathcurveto{\pgfqpoint{9.748788in}{1.978043in}}{\pgfqpoint{9.746784in}{1.973205in}}{\pgfqpoint{9.746784in}{1.968161in}}%
\pgfpathcurveto{\pgfqpoint{9.746784in}{1.963117in}}{\pgfqpoint{9.748788in}{1.958280in}}{\pgfqpoint{9.752354in}{1.954713in}}%
\pgfpathcurveto{\pgfqpoint{9.755921in}{1.951147in}}{\pgfqpoint{9.760759in}{1.949143in}}{\pgfqpoint{9.765802in}{1.949143in}}%
\pgfpathclose%
\pgfusepath{fill}%
\end{pgfscope}%
\begin{pgfscope}%
\pgfpathrectangle{\pgfqpoint{6.572727in}{0.474100in}}{\pgfqpoint{4.227273in}{3.318700in}}%
\pgfusepath{clip}%
\pgfsetbuttcap%
\pgfsetroundjoin%
\definecolor{currentfill}{rgb}{0.993248,0.906157,0.143936}%
\pgfsetfillcolor{currentfill}%
\pgfsetfillopacity{0.700000}%
\pgfsetlinewidth{0.000000pt}%
\definecolor{currentstroke}{rgb}{0.000000,0.000000,0.000000}%
\pgfsetstrokecolor{currentstroke}%
\pgfsetstrokeopacity{0.700000}%
\pgfsetdash{}{0pt}%
\pgfpathmoveto{\pgfqpoint{8.297741in}{3.301622in}}%
\pgfpathcurveto{\pgfqpoint{8.302785in}{3.301622in}}{\pgfqpoint{8.307623in}{3.303626in}}{\pgfqpoint{8.311189in}{3.307192in}}%
\pgfpathcurveto{\pgfqpoint{8.314756in}{3.310758in}}{\pgfqpoint{8.316759in}{3.315596in}}{\pgfqpoint{8.316759in}{3.320640in}}%
\pgfpathcurveto{\pgfqpoint{8.316759in}{3.325684in}}{\pgfqpoint{8.314756in}{3.330521in}}{\pgfqpoint{8.311189in}{3.334088in}}%
\pgfpathcurveto{\pgfqpoint{8.307623in}{3.337654in}}{\pgfqpoint{8.302785in}{3.339658in}}{\pgfqpoint{8.297741in}{3.339658in}}%
\pgfpathcurveto{\pgfqpoint{8.292698in}{3.339658in}}{\pgfqpoint{8.287860in}{3.337654in}}{\pgfqpoint{8.284293in}{3.334088in}}%
\pgfpathcurveto{\pgfqpoint{8.280727in}{3.330521in}}{\pgfqpoint{8.278723in}{3.325684in}}{\pgfqpoint{8.278723in}{3.320640in}}%
\pgfpathcurveto{\pgfqpoint{8.278723in}{3.315596in}}{\pgfqpoint{8.280727in}{3.310758in}}{\pgfqpoint{8.284293in}{3.307192in}}%
\pgfpathcurveto{\pgfqpoint{8.287860in}{3.303626in}}{\pgfqpoint{8.292698in}{3.301622in}}{\pgfqpoint{8.297741in}{3.301622in}}%
\pgfpathclose%
\pgfusepath{fill}%
\end{pgfscope}%
\begin{pgfscope}%
\pgfpathrectangle{\pgfqpoint{6.572727in}{0.474100in}}{\pgfqpoint{4.227273in}{3.318700in}}%
\pgfusepath{clip}%
\pgfsetbuttcap%
\pgfsetroundjoin%
\definecolor{currentfill}{rgb}{0.267004,0.004874,0.329415}%
\pgfsetfillcolor{currentfill}%
\pgfsetfillopacity{0.700000}%
\pgfsetlinewidth{0.000000pt}%
\definecolor{currentstroke}{rgb}{0.000000,0.000000,0.000000}%
\pgfsetstrokecolor{currentstroke}%
\pgfsetstrokeopacity{0.700000}%
\pgfsetdash{}{0pt}%
\pgfpathmoveto{\pgfqpoint{8.332766in}{1.731264in}}%
\pgfpathcurveto{\pgfqpoint{8.337809in}{1.731264in}}{\pgfqpoint{8.342647in}{1.733268in}}{\pgfqpoint{8.346214in}{1.736834in}}%
\pgfpathcurveto{\pgfqpoint{8.349780in}{1.740401in}}{\pgfqpoint{8.351784in}{1.745239in}}{\pgfqpoint{8.351784in}{1.750282in}}%
\pgfpathcurveto{\pgfqpoint{8.351784in}{1.755326in}}{\pgfqpoint{8.349780in}{1.760164in}}{\pgfqpoint{8.346214in}{1.763730in}}%
\pgfpathcurveto{\pgfqpoint{8.342647in}{1.767297in}}{\pgfqpoint{8.337809in}{1.769300in}}{\pgfqpoint{8.332766in}{1.769300in}}%
\pgfpathcurveto{\pgfqpoint{8.327722in}{1.769300in}}{\pgfqpoint{8.322884in}{1.767297in}}{\pgfqpoint{8.319318in}{1.763730in}}%
\pgfpathcurveto{\pgfqpoint{8.315751in}{1.760164in}}{\pgfqpoint{8.313748in}{1.755326in}}{\pgfqpoint{8.313748in}{1.750282in}}%
\pgfpathcurveto{\pgfqpoint{8.313748in}{1.745239in}}{\pgfqpoint{8.315751in}{1.740401in}}{\pgfqpoint{8.319318in}{1.736834in}}%
\pgfpathcurveto{\pgfqpoint{8.322884in}{1.733268in}}{\pgfqpoint{8.327722in}{1.731264in}}{\pgfqpoint{8.332766in}{1.731264in}}%
\pgfpathclose%
\pgfusepath{fill}%
\end{pgfscope}%
\begin{pgfscope}%
\pgfpathrectangle{\pgfqpoint{6.572727in}{0.474100in}}{\pgfqpoint{4.227273in}{3.318700in}}%
\pgfusepath{clip}%
\pgfsetbuttcap%
\pgfsetroundjoin%
\definecolor{currentfill}{rgb}{0.127568,0.566949,0.550556}%
\pgfsetfillcolor{currentfill}%
\pgfsetfillopacity{0.700000}%
\pgfsetlinewidth{0.000000pt}%
\definecolor{currentstroke}{rgb}{0.000000,0.000000,0.000000}%
\pgfsetstrokecolor{currentstroke}%
\pgfsetstrokeopacity{0.700000}%
\pgfsetdash{}{0pt}%
\pgfpathmoveto{\pgfqpoint{9.636829in}{1.362323in}}%
\pgfpathcurveto{\pgfqpoint{9.641873in}{1.362323in}}{\pgfqpoint{9.646711in}{1.364327in}}{\pgfqpoint{9.650277in}{1.367893in}}%
\pgfpathcurveto{\pgfqpoint{9.653843in}{1.371459in}}{\pgfqpoint{9.655847in}{1.376297in}}{\pgfqpoint{9.655847in}{1.381341in}}%
\pgfpathcurveto{\pgfqpoint{9.655847in}{1.386385in}}{\pgfqpoint{9.653843in}{1.391222in}}{\pgfqpoint{9.650277in}{1.394789in}}%
\pgfpathcurveto{\pgfqpoint{9.646711in}{1.398355in}}{\pgfqpoint{9.641873in}{1.400359in}}{\pgfqpoint{9.636829in}{1.400359in}}%
\pgfpathcurveto{\pgfqpoint{9.631786in}{1.400359in}}{\pgfqpoint{9.626948in}{1.398355in}}{\pgfqpoint{9.623381in}{1.394789in}}%
\pgfpathcurveto{\pgfqpoint{9.619815in}{1.391222in}}{\pgfqpoint{9.617811in}{1.386385in}}{\pgfqpoint{9.617811in}{1.381341in}}%
\pgfpathcurveto{\pgfqpoint{9.617811in}{1.376297in}}{\pgfqpoint{9.619815in}{1.371459in}}{\pgfqpoint{9.623381in}{1.367893in}}%
\pgfpathcurveto{\pgfqpoint{9.626948in}{1.364327in}}{\pgfqpoint{9.631786in}{1.362323in}}{\pgfqpoint{9.636829in}{1.362323in}}%
\pgfpathclose%
\pgfusepath{fill}%
\end{pgfscope}%
\begin{pgfscope}%
\pgfpathrectangle{\pgfqpoint{6.572727in}{0.474100in}}{\pgfqpoint{4.227273in}{3.318700in}}%
\pgfusepath{clip}%
\pgfsetbuttcap%
\pgfsetroundjoin%
\definecolor{currentfill}{rgb}{0.993248,0.906157,0.143936}%
\pgfsetfillcolor{currentfill}%
\pgfsetfillopacity{0.700000}%
\pgfsetlinewidth{0.000000pt}%
\definecolor{currentstroke}{rgb}{0.000000,0.000000,0.000000}%
\pgfsetstrokecolor{currentstroke}%
\pgfsetstrokeopacity{0.700000}%
\pgfsetdash{}{0pt}%
\pgfpathmoveto{\pgfqpoint{7.857265in}{3.089012in}}%
\pgfpathcurveto{\pgfqpoint{7.862309in}{3.089012in}}{\pgfqpoint{7.867147in}{3.091015in}}{\pgfqpoint{7.870713in}{3.094582in}}%
\pgfpathcurveto{\pgfqpoint{7.874280in}{3.098148in}}{\pgfqpoint{7.876283in}{3.102986in}}{\pgfqpoint{7.876283in}{3.108030in}}%
\pgfpathcurveto{\pgfqpoint{7.876283in}{3.113073in}}{\pgfqpoint{7.874280in}{3.117911in}}{\pgfqpoint{7.870713in}{3.121478in}}%
\pgfpathcurveto{\pgfqpoint{7.867147in}{3.125044in}}{\pgfqpoint{7.862309in}{3.127048in}}{\pgfqpoint{7.857265in}{3.127048in}}%
\pgfpathcurveto{\pgfqpoint{7.852222in}{3.127048in}}{\pgfqpoint{7.847384in}{3.125044in}}{\pgfqpoint{7.843817in}{3.121478in}}%
\pgfpathcurveto{\pgfqpoint{7.840251in}{3.117911in}}{\pgfqpoint{7.838247in}{3.113073in}}{\pgfqpoint{7.838247in}{3.108030in}}%
\pgfpathcurveto{\pgfqpoint{7.838247in}{3.102986in}}{\pgfqpoint{7.840251in}{3.098148in}}{\pgfqpoint{7.843817in}{3.094582in}}%
\pgfpathcurveto{\pgfqpoint{7.847384in}{3.091015in}}{\pgfqpoint{7.852222in}{3.089012in}}{\pgfqpoint{7.857265in}{3.089012in}}%
\pgfpathclose%
\pgfusepath{fill}%
\end{pgfscope}%
\begin{pgfscope}%
\pgfpathrectangle{\pgfqpoint{6.572727in}{0.474100in}}{\pgfqpoint{4.227273in}{3.318700in}}%
\pgfusepath{clip}%
\pgfsetbuttcap%
\pgfsetroundjoin%
\definecolor{currentfill}{rgb}{0.993248,0.906157,0.143936}%
\pgfsetfillcolor{currentfill}%
\pgfsetfillopacity{0.700000}%
\pgfsetlinewidth{0.000000pt}%
\definecolor{currentstroke}{rgb}{0.000000,0.000000,0.000000}%
\pgfsetstrokecolor{currentstroke}%
\pgfsetstrokeopacity{0.700000}%
\pgfsetdash{}{0pt}%
\pgfpathmoveto{\pgfqpoint{7.903005in}{2.754068in}}%
\pgfpathcurveto{\pgfqpoint{7.908049in}{2.754068in}}{\pgfqpoint{7.912886in}{2.756072in}}{\pgfqpoint{7.916453in}{2.759638in}}%
\pgfpathcurveto{\pgfqpoint{7.920019in}{2.763205in}}{\pgfqpoint{7.922023in}{2.768042in}}{\pgfqpoint{7.922023in}{2.773086in}}%
\pgfpathcurveto{\pgfqpoint{7.922023in}{2.778130in}}{\pgfqpoint{7.920019in}{2.782967in}}{\pgfqpoint{7.916453in}{2.786534in}}%
\pgfpathcurveto{\pgfqpoint{7.912886in}{2.790100in}}{\pgfqpoint{7.908049in}{2.792104in}}{\pgfqpoint{7.903005in}{2.792104in}}%
\pgfpathcurveto{\pgfqpoint{7.897961in}{2.792104in}}{\pgfqpoint{7.893124in}{2.790100in}}{\pgfqpoint{7.889557in}{2.786534in}}%
\pgfpathcurveto{\pgfqpoint{7.885991in}{2.782967in}}{\pgfqpoint{7.883987in}{2.778130in}}{\pgfqpoint{7.883987in}{2.773086in}}%
\pgfpathcurveto{\pgfqpoint{7.883987in}{2.768042in}}{\pgfqpoint{7.885991in}{2.763205in}}{\pgfqpoint{7.889557in}{2.759638in}}%
\pgfpathcurveto{\pgfqpoint{7.893124in}{2.756072in}}{\pgfqpoint{7.897961in}{2.754068in}}{\pgfqpoint{7.903005in}{2.754068in}}%
\pgfpathclose%
\pgfusepath{fill}%
\end{pgfscope}%
\begin{pgfscope}%
\pgfpathrectangle{\pgfqpoint{6.572727in}{0.474100in}}{\pgfqpoint{4.227273in}{3.318700in}}%
\pgfusepath{clip}%
\pgfsetbuttcap%
\pgfsetroundjoin%
\definecolor{currentfill}{rgb}{0.993248,0.906157,0.143936}%
\pgfsetfillcolor{currentfill}%
\pgfsetfillopacity{0.700000}%
\pgfsetlinewidth{0.000000pt}%
\definecolor{currentstroke}{rgb}{0.000000,0.000000,0.000000}%
\pgfsetstrokecolor{currentstroke}%
\pgfsetstrokeopacity{0.700000}%
\pgfsetdash{}{0pt}%
\pgfpathmoveto{\pgfqpoint{8.694913in}{2.835703in}}%
\pgfpathcurveto{\pgfqpoint{8.699956in}{2.835703in}}{\pgfqpoint{8.704794in}{2.837707in}}{\pgfqpoint{8.708361in}{2.841274in}}%
\pgfpathcurveto{\pgfqpoint{8.711927in}{2.844840in}}{\pgfqpoint{8.713931in}{2.849678in}}{\pgfqpoint{8.713931in}{2.854721in}}%
\pgfpathcurveto{\pgfqpoint{8.713931in}{2.859765in}}{\pgfqpoint{8.711927in}{2.864603in}}{\pgfqpoint{8.708361in}{2.868169in}}%
\pgfpathcurveto{\pgfqpoint{8.704794in}{2.871736in}}{\pgfqpoint{8.699956in}{2.873740in}}{\pgfqpoint{8.694913in}{2.873740in}}%
\pgfpathcurveto{\pgfqpoint{8.689869in}{2.873740in}}{\pgfqpoint{8.685031in}{2.871736in}}{\pgfqpoint{8.681465in}{2.868169in}}%
\pgfpathcurveto{\pgfqpoint{8.677898in}{2.864603in}}{\pgfqpoint{8.675895in}{2.859765in}}{\pgfqpoint{8.675895in}{2.854721in}}%
\pgfpathcurveto{\pgfqpoint{8.675895in}{2.849678in}}{\pgfqpoint{8.677898in}{2.844840in}}{\pgfqpoint{8.681465in}{2.841274in}}%
\pgfpathcurveto{\pgfqpoint{8.685031in}{2.837707in}}{\pgfqpoint{8.689869in}{2.835703in}}{\pgfqpoint{8.694913in}{2.835703in}}%
\pgfpathclose%
\pgfusepath{fill}%
\end{pgfscope}%
\begin{pgfscope}%
\pgfpathrectangle{\pgfqpoint{6.572727in}{0.474100in}}{\pgfqpoint{4.227273in}{3.318700in}}%
\pgfusepath{clip}%
\pgfsetbuttcap%
\pgfsetroundjoin%
\definecolor{currentfill}{rgb}{0.267004,0.004874,0.329415}%
\pgfsetfillcolor{currentfill}%
\pgfsetfillopacity{0.700000}%
\pgfsetlinewidth{0.000000pt}%
\definecolor{currentstroke}{rgb}{0.000000,0.000000,0.000000}%
\pgfsetstrokecolor{currentstroke}%
\pgfsetstrokeopacity{0.700000}%
\pgfsetdash{}{0pt}%
\pgfpathmoveto{\pgfqpoint{7.486427in}{1.070516in}}%
\pgfpathcurveto{\pgfqpoint{7.491471in}{1.070516in}}{\pgfqpoint{7.496309in}{1.072519in}}{\pgfqpoint{7.499875in}{1.076086in}}%
\pgfpathcurveto{\pgfqpoint{7.503441in}{1.079652in}}{\pgfqpoint{7.505445in}{1.084490in}}{\pgfqpoint{7.505445in}{1.089534in}}%
\pgfpathcurveto{\pgfqpoint{7.505445in}{1.094577in}}{\pgfqpoint{7.503441in}{1.099415in}}{\pgfqpoint{7.499875in}{1.102982in}}%
\pgfpathcurveto{\pgfqpoint{7.496309in}{1.106548in}}{\pgfqpoint{7.491471in}{1.108552in}}{\pgfqpoint{7.486427in}{1.108552in}}%
\pgfpathcurveto{\pgfqpoint{7.481383in}{1.108552in}}{\pgfqpoint{7.476546in}{1.106548in}}{\pgfqpoint{7.472979in}{1.102982in}}%
\pgfpathcurveto{\pgfqpoint{7.469413in}{1.099415in}}{\pgfqpoint{7.467409in}{1.094577in}}{\pgfqpoint{7.467409in}{1.089534in}}%
\pgfpathcurveto{\pgfqpoint{7.467409in}{1.084490in}}{\pgfqpoint{7.469413in}{1.079652in}}{\pgfqpoint{7.472979in}{1.076086in}}%
\pgfpathcurveto{\pgfqpoint{7.476546in}{1.072519in}}{\pgfqpoint{7.481383in}{1.070516in}}{\pgfqpoint{7.486427in}{1.070516in}}%
\pgfpathclose%
\pgfusepath{fill}%
\end{pgfscope}%
\begin{pgfscope}%
\pgfpathrectangle{\pgfqpoint{6.572727in}{0.474100in}}{\pgfqpoint{4.227273in}{3.318700in}}%
\pgfusepath{clip}%
\pgfsetbuttcap%
\pgfsetroundjoin%
\definecolor{currentfill}{rgb}{0.267004,0.004874,0.329415}%
\pgfsetfillcolor{currentfill}%
\pgfsetfillopacity{0.700000}%
\pgfsetlinewidth{0.000000pt}%
\definecolor{currentstroke}{rgb}{0.000000,0.000000,0.000000}%
\pgfsetstrokecolor{currentstroke}%
\pgfsetstrokeopacity{0.700000}%
\pgfsetdash{}{0pt}%
\pgfpathmoveto{\pgfqpoint{7.674943in}{1.334929in}}%
\pgfpathcurveto{\pgfqpoint{7.679987in}{1.334929in}}{\pgfqpoint{7.684825in}{1.336933in}}{\pgfqpoint{7.688391in}{1.340499in}}%
\pgfpathcurveto{\pgfqpoint{7.691957in}{1.344066in}}{\pgfqpoint{7.693961in}{1.348904in}}{\pgfqpoint{7.693961in}{1.353947in}}%
\pgfpathcurveto{\pgfqpoint{7.693961in}{1.358991in}}{\pgfqpoint{7.691957in}{1.363829in}}{\pgfqpoint{7.688391in}{1.367395in}}%
\pgfpathcurveto{\pgfqpoint{7.684825in}{1.370962in}}{\pgfqpoint{7.679987in}{1.372965in}}{\pgfqpoint{7.674943in}{1.372965in}}%
\pgfpathcurveto{\pgfqpoint{7.669899in}{1.372965in}}{\pgfqpoint{7.665062in}{1.370962in}}{\pgfqpoint{7.661495in}{1.367395in}}%
\pgfpathcurveto{\pgfqpoint{7.657929in}{1.363829in}}{\pgfqpoint{7.655925in}{1.358991in}}{\pgfqpoint{7.655925in}{1.353947in}}%
\pgfpathcurveto{\pgfqpoint{7.655925in}{1.348904in}}{\pgfqpoint{7.657929in}{1.344066in}}{\pgfqpoint{7.661495in}{1.340499in}}%
\pgfpathcurveto{\pgfqpoint{7.665062in}{1.336933in}}{\pgfqpoint{7.669899in}{1.334929in}}{\pgfqpoint{7.674943in}{1.334929in}}%
\pgfpathclose%
\pgfusepath{fill}%
\end{pgfscope}%
\begin{pgfscope}%
\pgfpathrectangle{\pgfqpoint{6.572727in}{0.474100in}}{\pgfqpoint{4.227273in}{3.318700in}}%
\pgfusepath{clip}%
\pgfsetbuttcap%
\pgfsetroundjoin%
\definecolor{currentfill}{rgb}{0.993248,0.906157,0.143936}%
\pgfsetfillcolor{currentfill}%
\pgfsetfillopacity{0.700000}%
\pgfsetlinewidth{0.000000pt}%
\definecolor{currentstroke}{rgb}{0.000000,0.000000,0.000000}%
\pgfsetstrokecolor{currentstroke}%
\pgfsetstrokeopacity{0.700000}%
\pgfsetdash{}{0pt}%
\pgfpathmoveto{\pgfqpoint{8.347863in}{2.480927in}}%
\pgfpathcurveto{\pgfqpoint{8.352907in}{2.480927in}}{\pgfqpoint{8.357745in}{2.482931in}}{\pgfqpoint{8.361311in}{2.486497in}}%
\pgfpathcurveto{\pgfqpoint{8.364878in}{2.490064in}}{\pgfqpoint{8.366882in}{2.494902in}}{\pgfqpoint{8.366882in}{2.499945in}}%
\pgfpathcurveto{\pgfqpoint{8.366882in}{2.504989in}}{\pgfqpoint{8.364878in}{2.509827in}}{\pgfqpoint{8.361311in}{2.513393in}}%
\pgfpathcurveto{\pgfqpoint{8.357745in}{2.516959in}}{\pgfqpoint{8.352907in}{2.518963in}}{\pgfqpoint{8.347863in}{2.518963in}}%
\pgfpathcurveto{\pgfqpoint{8.342820in}{2.518963in}}{\pgfqpoint{8.337982in}{2.516959in}}{\pgfqpoint{8.334416in}{2.513393in}}%
\pgfpathcurveto{\pgfqpoint{8.330849in}{2.509827in}}{\pgfqpoint{8.328845in}{2.504989in}}{\pgfqpoint{8.328845in}{2.499945in}}%
\pgfpathcurveto{\pgfqpoint{8.328845in}{2.494902in}}{\pgfqpoint{8.330849in}{2.490064in}}{\pgfqpoint{8.334416in}{2.486497in}}%
\pgfpathcurveto{\pgfqpoint{8.337982in}{2.482931in}}{\pgfqpoint{8.342820in}{2.480927in}}{\pgfqpoint{8.347863in}{2.480927in}}%
\pgfpathclose%
\pgfusepath{fill}%
\end{pgfscope}%
\begin{pgfscope}%
\pgfpathrectangle{\pgfqpoint{6.572727in}{0.474100in}}{\pgfqpoint{4.227273in}{3.318700in}}%
\pgfusepath{clip}%
\pgfsetbuttcap%
\pgfsetroundjoin%
\definecolor{currentfill}{rgb}{0.127568,0.566949,0.550556}%
\pgfsetfillcolor{currentfill}%
\pgfsetfillopacity{0.700000}%
\pgfsetlinewidth{0.000000pt}%
\definecolor{currentstroke}{rgb}{0.000000,0.000000,0.000000}%
\pgfsetstrokecolor{currentstroke}%
\pgfsetstrokeopacity{0.700000}%
\pgfsetdash{}{0pt}%
\pgfpathmoveto{\pgfqpoint{9.203207in}{1.116841in}}%
\pgfpathcurveto{\pgfqpoint{9.208251in}{1.116841in}}{\pgfqpoint{9.213089in}{1.118845in}}{\pgfqpoint{9.216655in}{1.122411in}}%
\pgfpathcurveto{\pgfqpoint{9.220222in}{1.125978in}}{\pgfqpoint{9.222226in}{1.130815in}}{\pgfqpoint{9.222226in}{1.135859in}}%
\pgfpathcurveto{\pgfqpoint{9.222226in}{1.140903in}}{\pgfqpoint{9.220222in}{1.145741in}}{\pgfqpoint{9.216655in}{1.149307in}}%
\pgfpathcurveto{\pgfqpoint{9.213089in}{1.152873in}}{\pgfqpoint{9.208251in}{1.154877in}}{\pgfqpoint{9.203207in}{1.154877in}}%
\pgfpathcurveto{\pgfqpoint{9.198164in}{1.154877in}}{\pgfqpoint{9.193326in}{1.152873in}}{\pgfqpoint{9.189760in}{1.149307in}}%
\pgfpathcurveto{\pgfqpoint{9.186193in}{1.145741in}}{\pgfqpoint{9.184189in}{1.140903in}}{\pgfqpoint{9.184189in}{1.135859in}}%
\pgfpathcurveto{\pgfqpoint{9.184189in}{1.130815in}}{\pgfqpoint{9.186193in}{1.125978in}}{\pgfqpoint{9.189760in}{1.122411in}}%
\pgfpathcurveto{\pgfqpoint{9.193326in}{1.118845in}}{\pgfqpoint{9.198164in}{1.116841in}}{\pgfqpoint{9.203207in}{1.116841in}}%
\pgfpathclose%
\pgfusepath{fill}%
\end{pgfscope}%
\begin{pgfscope}%
\pgfpathrectangle{\pgfqpoint{6.572727in}{0.474100in}}{\pgfqpoint{4.227273in}{3.318700in}}%
\pgfusepath{clip}%
\pgfsetbuttcap%
\pgfsetroundjoin%
\definecolor{currentfill}{rgb}{0.127568,0.566949,0.550556}%
\pgfsetfillcolor{currentfill}%
\pgfsetfillopacity{0.700000}%
\pgfsetlinewidth{0.000000pt}%
\definecolor{currentstroke}{rgb}{0.000000,0.000000,0.000000}%
\pgfsetstrokecolor{currentstroke}%
\pgfsetstrokeopacity{0.700000}%
\pgfsetdash{}{0pt}%
\pgfpathmoveto{\pgfqpoint{9.020750in}{0.960771in}}%
\pgfpathcurveto{\pgfqpoint{9.025794in}{0.960771in}}{\pgfqpoint{9.030632in}{0.962774in}}{\pgfqpoint{9.034198in}{0.966341in}}%
\pgfpathcurveto{\pgfqpoint{9.037765in}{0.969907in}}{\pgfqpoint{9.039769in}{0.974745in}}{\pgfqpoint{9.039769in}{0.979789in}}%
\pgfpathcurveto{\pgfqpoint{9.039769in}{0.984832in}}{\pgfqpoint{9.037765in}{0.989670in}}{\pgfqpoint{9.034198in}{0.993237in}}%
\pgfpathcurveto{\pgfqpoint{9.030632in}{0.996803in}}{\pgfqpoint{9.025794in}{0.998807in}}{\pgfqpoint{9.020750in}{0.998807in}}%
\pgfpathcurveto{\pgfqpoint{9.015707in}{0.998807in}}{\pgfqpoint{9.010869in}{0.996803in}}{\pgfqpoint{9.007303in}{0.993237in}}%
\pgfpathcurveto{\pgfqpoint{9.003736in}{0.989670in}}{\pgfqpoint{9.001732in}{0.984832in}}{\pgfqpoint{9.001732in}{0.979789in}}%
\pgfpathcurveto{\pgfqpoint{9.001732in}{0.974745in}}{\pgfqpoint{9.003736in}{0.969907in}}{\pgfqpoint{9.007303in}{0.966341in}}%
\pgfpathcurveto{\pgfqpoint{9.010869in}{0.962774in}}{\pgfqpoint{9.015707in}{0.960771in}}{\pgfqpoint{9.020750in}{0.960771in}}%
\pgfpathclose%
\pgfusepath{fill}%
\end{pgfscope}%
\begin{pgfscope}%
\pgfpathrectangle{\pgfqpoint{6.572727in}{0.474100in}}{\pgfqpoint{4.227273in}{3.318700in}}%
\pgfusepath{clip}%
\pgfsetbuttcap%
\pgfsetroundjoin%
\definecolor{currentfill}{rgb}{0.993248,0.906157,0.143936}%
\pgfsetfillcolor{currentfill}%
\pgfsetfillopacity{0.700000}%
\pgfsetlinewidth{0.000000pt}%
\definecolor{currentstroke}{rgb}{0.000000,0.000000,0.000000}%
\pgfsetstrokecolor{currentstroke}%
\pgfsetstrokeopacity{0.700000}%
\pgfsetdash{}{0pt}%
\pgfpathmoveto{\pgfqpoint{8.228015in}{2.887898in}}%
\pgfpathcurveto{\pgfqpoint{8.233058in}{2.887898in}}{\pgfqpoint{8.237896in}{2.889902in}}{\pgfqpoint{8.241463in}{2.893469in}}%
\pgfpathcurveto{\pgfqpoint{8.245029in}{2.897035in}}{\pgfqpoint{8.247033in}{2.901873in}}{\pgfqpoint{8.247033in}{2.906917in}}%
\pgfpathcurveto{\pgfqpoint{8.247033in}{2.911960in}}{\pgfqpoint{8.245029in}{2.916798in}}{\pgfqpoint{8.241463in}{2.920364in}}%
\pgfpathcurveto{\pgfqpoint{8.237896in}{2.923931in}}{\pgfqpoint{8.233058in}{2.925935in}}{\pgfqpoint{8.228015in}{2.925935in}}%
\pgfpathcurveto{\pgfqpoint{8.222971in}{2.925935in}}{\pgfqpoint{8.218133in}{2.923931in}}{\pgfqpoint{8.214567in}{2.920364in}}%
\pgfpathcurveto{\pgfqpoint{8.211000in}{2.916798in}}{\pgfqpoint{8.208997in}{2.911960in}}{\pgfqpoint{8.208997in}{2.906917in}}%
\pgfpathcurveto{\pgfqpoint{8.208997in}{2.901873in}}{\pgfqpoint{8.211000in}{2.897035in}}{\pgfqpoint{8.214567in}{2.893469in}}%
\pgfpathcurveto{\pgfqpoint{8.218133in}{2.889902in}}{\pgfqpoint{8.222971in}{2.887898in}}{\pgfqpoint{8.228015in}{2.887898in}}%
\pgfpathclose%
\pgfusepath{fill}%
\end{pgfscope}%
\begin{pgfscope}%
\pgfpathrectangle{\pgfqpoint{6.572727in}{0.474100in}}{\pgfqpoint{4.227273in}{3.318700in}}%
\pgfusepath{clip}%
\pgfsetbuttcap%
\pgfsetroundjoin%
\definecolor{currentfill}{rgb}{0.127568,0.566949,0.550556}%
\pgfsetfillcolor{currentfill}%
\pgfsetfillopacity{0.700000}%
\pgfsetlinewidth{0.000000pt}%
\definecolor{currentstroke}{rgb}{0.000000,0.000000,0.000000}%
\pgfsetstrokecolor{currentstroke}%
\pgfsetstrokeopacity{0.700000}%
\pgfsetdash{}{0pt}%
\pgfpathmoveto{\pgfqpoint{9.623363in}{1.299352in}}%
\pgfpathcurveto{\pgfqpoint{9.628406in}{1.299352in}}{\pgfqpoint{9.633244in}{1.301355in}}{\pgfqpoint{9.636810in}{1.304922in}}%
\pgfpathcurveto{\pgfqpoint{9.640377in}{1.308488in}}{\pgfqpoint{9.642381in}{1.313326in}}{\pgfqpoint{9.642381in}{1.318370in}}%
\pgfpathcurveto{\pgfqpoint{9.642381in}{1.323413in}}{\pgfqpoint{9.640377in}{1.328251in}}{\pgfqpoint{9.636810in}{1.331818in}}%
\pgfpathcurveto{\pgfqpoint{9.633244in}{1.335384in}}{\pgfqpoint{9.628406in}{1.337388in}}{\pgfqpoint{9.623363in}{1.337388in}}%
\pgfpathcurveto{\pgfqpoint{9.618319in}{1.337388in}}{\pgfqpoint{9.613481in}{1.335384in}}{\pgfqpoint{9.609915in}{1.331818in}}%
\pgfpathcurveto{\pgfqpoint{9.606348in}{1.328251in}}{\pgfqpoint{9.604344in}{1.323413in}}{\pgfqpoint{9.604344in}{1.318370in}}%
\pgfpathcurveto{\pgfqpoint{9.604344in}{1.313326in}}{\pgfqpoint{9.606348in}{1.308488in}}{\pgfqpoint{9.609915in}{1.304922in}}%
\pgfpathcurveto{\pgfqpoint{9.613481in}{1.301355in}}{\pgfqpoint{9.618319in}{1.299352in}}{\pgfqpoint{9.623363in}{1.299352in}}%
\pgfpathclose%
\pgfusepath{fill}%
\end{pgfscope}%
\begin{pgfscope}%
\pgfpathrectangle{\pgfqpoint{6.572727in}{0.474100in}}{\pgfqpoint{4.227273in}{3.318700in}}%
\pgfusepath{clip}%
\pgfsetbuttcap%
\pgfsetroundjoin%
\definecolor{currentfill}{rgb}{0.127568,0.566949,0.550556}%
\pgfsetfillcolor{currentfill}%
\pgfsetfillopacity{0.700000}%
\pgfsetlinewidth{0.000000pt}%
\definecolor{currentstroke}{rgb}{0.000000,0.000000,0.000000}%
\pgfsetstrokecolor{currentstroke}%
\pgfsetstrokeopacity{0.700000}%
\pgfsetdash{}{0pt}%
\pgfpathmoveto{\pgfqpoint{9.234232in}{1.370013in}}%
\pgfpathcurveto{\pgfqpoint{9.239276in}{1.370013in}}{\pgfqpoint{9.244114in}{1.372017in}}{\pgfqpoint{9.247680in}{1.375583in}}%
\pgfpathcurveto{\pgfqpoint{9.251246in}{1.379150in}}{\pgfqpoint{9.253250in}{1.383987in}}{\pgfqpoint{9.253250in}{1.389031in}}%
\pgfpathcurveto{\pgfqpoint{9.253250in}{1.394075in}}{\pgfqpoint{9.251246in}{1.398913in}}{\pgfqpoint{9.247680in}{1.402479in}}%
\pgfpathcurveto{\pgfqpoint{9.244114in}{1.406045in}}{\pgfqpoint{9.239276in}{1.408049in}}{\pgfqpoint{9.234232in}{1.408049in}}%
\pgfpathcurveto{\pgfqpoint{9.229188in}{1.408049in}}{\pgfqpoint{9.224351in}{1.406045in}}{\pgfqpoint{9.220784in}{1.402479in}}%
\pgfpathcurveto{\pgfqpoint{9.217218in}{1.398913in}}{\pgfqpoint{9.215214in}{1.394075in}}{\pgfqpoint{9.215214in}{1.389031in}}%
\pgfpathcurveto{\pgfqpoint{9.215214in}{1.383987in}}{\pgfqpoint{9.217218in}{1.379150in}}{\pgfqpoint{9.220784in}{1.375583in}}%
\pgfpathcurveto{\pgfqpoint{9.224351in}{1.372017in}}{\pgfqpoint{9.229188in}{1.370013in}}{\pgfqpoint{9.234232in}{1.370013in}}%
\pgfpathclose%
\pgfusepath{fill}%
\end{pgfscope}%
\begin{pgfscope}%
\pgfpathrectangle{\pgfqpoint{6.572727in}{0.474100in}}{\pgfqpoint{4.227273in}{3.318700in}}%
\pgfusepath{clip}%
\pgfsetbuttcap%
\pgfsetroundjoin%
\definecolor{currentfill}{rgb}{0.267004,0.004874,0.329415}%
\pgfsetfillcolor{currentfill}%
\pgfsetfillopacity{0.700000}%
\pgfsetlinewidth{0.000000pt}%
\definecolor{currentstroke}{rgb}{0.000000,0.000000,0.000000}%
\pgfsetstrokecolor{currentstroke}%
\pgfsetstrokeopacity{0.700000}%
\pgfsetdash{}{0pt}%
\pgfpathmoveto{\pgfqpoint{7.543704in}{1.881352in}}%
\pgfpathcurveto{\pgfqpoint{7.548748in}{1.881352in}}{\pgfqpoint{7.553585in}{1.883356in}}{\pgfqpoint{7.557152in}{1.886923in}}%
\pgfpathcurveto{\pgfqpoint{7.560718in}{1.890489in}}{\pgfqpoint{7.562722in}{1.895327in}}{\pgfqpoint{7.562722in}{1.900371in}}%
\pgfpathcurveto{\pgfqpoint{7.562722in}{1.905414in}}{\pgfqpoint{7.560718in}{1.910252in}}{\pgfqpoint{7.557152in}{1.913818in}}%
\pgfpathcurveto{\pgfqpoint{7.553585in}{1.917385in}}{\pgfqpoint{7.548748in}{1.919389in}}{\pgfqpoint{7.543704in}{1.919389in}}%
\pgfpathcurveto{\pgfqpoint{7.538660in}{1.919389in}}{\pgfqpoint{7.533823in}{1.917385in}}{\pgfqpoint{7.530256in}{1.913818in}}%
\pgfpathcurveto{\pgfqpoint{7.526690in}{1.910252in}}{\pgfqpoint{7.524686in}{1.905414in}}{\pgfqpoint{7.524686in}{1.900371in}}%
\pgfpathcurveto{\pgfqpoint{7.524686in}{1.895327in}}{\pgfqpoint{7.526690in}{1.890489in}}{\pgfqpoint{7.530256in}{1.886923in}}%
\pgfpathcurveto{\pgfqpoint{7.533823in}{1.883356in}}{\pgfqpoint{7.538660in}{1.881352in}}{\pgfqpoint{7.543704in}{1.881352in}}%
\pgfpathclose%
\pgfusepath{fill}%
\end{pgfscope}%
\begin{pgfscope}%
\pgfpathrectangle{\pgfqpoint{6.572727in}{0.474100in}}{\pgfqpoint{4.227273in}{3.318700in}}%
\pgfusepath{clip}%
\pgfsetbuttcap%
\pgfsetroundjoin%
\definecolor{currentfill}{rgb}{0.267004,0.004874,0.329415}%
\pgfsetfillcolor{currentfill}%
\pgfsetfillopacity{0.700000}%
\pgfsetlinewidth{0.000000pt}%
\definecolor{currentstroke}{rgb}{0.000000,0.000000,0.000000}%
\pgfsetstrokecolor{currentstroke}%
\pgfsetstrokeopacity{0.700000}%
\pgfsetdash{}{0pt}%
\pgfpathmoveto{\pgfqpoint{7.281759in}{1.846736in}}%
\pgfpathcurveto{\pgfqpoint{7.286803in}{1.846736in}}{\pgfqpoint{7.291640in}{1.848740in}}{\pgfqpoint{7.295207in}{1.852306in}}%
\pgfpathcurveto{\pgfqpoint{7.298773in}{1.855873in}}{\pgfqpoint{7.300777in}{1.860711in}}{\pgfqpoint{7.300777in}{1.865754in}}%
\pgfpathcurveto{\pgfqpoint{7.300777in}{1.870798in}}{\pgfqpoint{7.298773in}{1.875636in}}{\pgfqpoint{7.295207in}{1.879202in}}%
\pgfpathcurveto{\pgfqpoint{7.291640in}{1.882769in}}{\pgfqpoint{7.286803in}{1.884772in}}{\pgfqpoint{7.281759in}{1.884772in}}%
\pgfpathcurveto{\pgfqpoint{7.276715in}{1.884772in}}{\pgfqpoint{7.271878in}{1.882769in}}{\pgfqpoint{7.268311in}{1.879202in}}%
\pgfpathcurveto{\pgfqpoint{7.264745in}{1.875636in}}{\pgfqpoint{7.262741in}{1.870798in}}{\pgfqpoint{7.262741in}{1.865754in}}%
\pgfpathcurveto{\pgfqpoint{7.262741in}{1.860711in}}{\pgfqpoint{7.264745in}{1.855873in}}{\pgfqpoint{7.268311in}{1.852306in}}%
\pgfpathcurveto{\pgfqpoint{7.271878in}{1.848740in}}{\pgfqpoint{7.276715in}{1.846736in}}{\pgfqpoint{7.281759in}{1.846736in}}%
\pgfpathclose%
\pgfusepath{fill}%
\end{pgfscope}%
\begin{pgfscope}%
\pgfpathrectangle{\pgfqpoint{6.572727in}{0.474100in}}{\pgfqpoint{4.227273in}{3.318700in}}%
\pgfusepath{clip}%
\pgfsetbuttcap%
\pgfsetroundjoin%
\definecolor{currentfill}{rgb}{0.267004,0.004874,0.329415}%
\pgfsetfillcolor{currentfill}%
\pgfsetfillopacity{0.700000}%
\pgfsetlinewidth{0.000000pt}%
\definecolor{currentstroke}{rgb}{0.000000,0.000000,0.000000}%
\pgfsetstrokecolor{currentstroke}%
\pgfsetstrokeopacity{0.700000}%
\pgfsetdash{}{0pt}%
\pgfpathmoveto{\pgfqpoint{7.755262in}{2.153809in}}%
\pgfpathcurveto{\pgfqpoint{7.760305in}{2.153809in}}{\pgfqpoint{7.765143in}{2.155813in}}{\pgfqpoint{7.768710in}{2.159379in}}%
\pgfpathcurveto{\pgfqpoint{7.772276in}{2.162945in}}{\pgfqpoint{7.774280in}{2.167783in}}{\pgfqpoint{7.774280in}{2.172827in}}%
\pgfpathcurveto{\pgfqpoint{7.774280in}{2.177870in}}{\pgfqpoint{7.772276in}{2.182708in}}{\pgfqpoint{7.768710in}{2.186275in}}%
\pgfpathcurveto{\pgfqpoint{7.765143in}{2.189841in}}{\pgfqpoint{7.760305in}{2.191845in}}{\pgfqpoint{7.755262in}{2.191845in}}%
\pgfpathcurveto{\pgfqpoint{7.750218in}{2.191845in}}{\pgfqpoint{7.745380in}{2.189841in}}{\pgfqpoint{7.741814in}{2.186275in}}%
\pgfpathcurveto{\pgfqpoint{7.738248in}{2.182708in}}{\pgfqpoint{7.736244in}{2.177870in}}{\pgfqpoint{7.736244in}{2.172827in}}%
\pgfpathcurveto{\pgfqpoint{7.736244in}{2.167783in}}{\pgfqpoint{7.738248in}{2.162945in}}{\pgfqpoint{7.741814in}{2.159379in}}%
\pgfpathcurveto{\pgfqpoint{7.745380in}{2.155813in}}{\pgfqpoint{7.750218in}{2.153809in}}{\pgfqpoint{7.755262in}{2.153809in}}%
\pgfpathclose%
\pgfusepath{fill}%
\end{pgfscope}%
\begin{pgfscope}%
\pgfpathrectangle{\pgfqpoint{6.572727in}{0.474100in}}{\pgfqpoint{4.227273in}{3.318700in}}%
\pgfusepath{clip}%
\pgfsetbuttcap%
\pgfsetroundjoin%
\definecolor{currentfill}{rgb}{0.993248,0.906157,0.143936}%
\pgfsetfillcolor{currentfill}%
\pgfsetfillopacity{0.700000}%
\pgfsetlinewidth{0.000000pt}%
\definecolor{currentstroke}{rgb}{0.000000,0.000000,0.000000}%
\pgfsetstrokecolor{currentstroke}%
\pgfsetstrokeopacity{0.700000}%
\pgfsetdash{}{0pt}%
\pgfpathmoveto{\pgfqpoint{8.323257in}{2.408679in}}%
\pgfpathcurveto{\pgfqpoint{8.328301in}{2.408679in}}{\pgfqpoint{8.333139in}{2.410683in}}{\pgfqpoint{8.336705in}{2.414250in}}%
\pgfpathcurveto{\pgfqpoint{8.340271in}{2.417816in}}{\pgfqpoint{8.342275in}{2.422654in}}{\pgfqpoint{8.342275in}{2.427698in}}%
\pgfpathcurveto{\pgfqpoint{8.342275in}{2.432741in}}{\pgfqpoint{8.340271in}{2.437579in}}{\pgfqpoint{8.336705in}{2.441145in}}%
\pgfpathcurveto{\pgfqpoint{8.333139in}{2.444712in}}{\pgfqpoint{8.328301in}{2.446716in}}{\pgfqpoint{8.323257in}{2.446716in}}%
\pgfpathcurveto{\pgfqpoint{8.318213in}{2.446716in}}{\pgfqpoint{8.313376in}{2.444712in}}{\pgfqpoint{8.309809in}{2.441145in}}%
\pgfpathcurveto{\pgfqpoint{8.306243in}{2.437579in}}{\pgfqpoint{8.304239in}{2.432741in}}{\pgfqpoint{8.304239in}{2.427698in}}%
\pgfpathcurveto{\pgfqpoint{8.304239in}{2.422654in}}{\pgfqpoint{8.306243in}{2.417816in}}{\pgfqpoint{8.309809in}{2.414250in}}%
\pgfpathcurveto{\pgfqpoint{8.313376in}{2.410683in}}{\pgfqpoint{8.318213in}{2.408679in}}{\pgfqpoint{8.323257in}{2.408679in}}%
\pgfpathclose%
\pgfusepath{fill}%
\end{pgfscope}%
\begin{pgfscope}%
\pgfpathrectangle{\pgfqpoint{6.572727in}{0.474100in}}{\pgfqpoint{4.227273in}{3.318700in}}%
\pgfusepath{clip}%
\pgfsetbuttcap%
\pgfsetroundjoin%
\definecolor{currentfill}{rgb}{0.267004,0.004874,0.329415}%
\pgfsetfillcolor{currentfill}%
\pgfsetfillopacity{0.700000}%
\pgfsetlinewidth{0.000000pt}%
\definecolor{currentstroke}{rgb}{0.000000,0.000000,0.000000}%
\pgfsetstrokecolor{currentstroke}%
\pgfsetstrokeopacity{0.700000}%
\pgfsetdash{}{0pt}%
\pgfpathmoveto{\pgfqpoint{8.300424in}{1.418839in}}%
\pgfpathcurveto{\pgfqpoint{8.305468in}{1.418839in}}{\pgfqpoint{8.310306in}{1.420843in}}{\pgfqpoint{8.313872in}{1.424409in}}%
\pgfpathcurveto{\pgfqpoint{8.317439in}{1.427975in}}{\pgfqpoint{8.319442in}{1.432813in}}{\pgfqpoint{8.319442in}{1.437857in}}%
\pgfpathcurveto{\pgfqpoint{8.319442in}{1.442900in}}{\pgfqpoint{8.317439in}{1.447738in}}{\pgfqpoint{8.313872in}{1.451305in}}%
\pgfpathcurveto{\pgfqpoint{8.310306in}{1.454871in}}{\pgfqpoint{8.305468in}{1.456875in}}{\pgfqpoint{8.300424in}{1.456875in}}%
\pgfpathcurveto{\pgfqpoint{8.295381in}{1.456875in}}{\pgfqpoint{8.290543in}{1.454871in}}{\pgfqpoint{8.286976in}{1.451305in}}%
\pgfpathcurveto{\pgfqpoint{8.283410in}{1.447738in}}{\pgfqpoint{8.281406in}{1.442900in}}{\pgfqpoint{8.281406in}{1.437857in}}%
\pgfpathcurveto{\pgfqpoint{8.281406in}{1.432813in}}{\pgfqpoint{8.283410in}{1.427975in}}{\pgfqpoint{8.286976in}{1.424409in}}%
\pgfpathcurveto{\pgfqpoint{8.290543in}{1.420843in}}{\pgfqpoint{8.295381in}{1.418839in}}{\pgfqpoint{8.300424in}{1.418839in}}%
\pgfpathclose%
\pgfusepath{fill}%
\end{pgfscope}%
\begin{pgfscope}%
\pgfpathrectangle{\pgfqpoint{6.572727in}{0.474100in}}{\pgfqpoint{4.227273in}{3.318700in}}%
\pgfusepath{clip}%
\pgfsetbuttcap%
\pgfsetroundjoin%
\definecolor{currentfill}{rgb}{0.127568,0.566949,0.550556}%
\pgfsetfillcolor{currentfill}%
\pgfsetfillopacity{0.700000}%
\pgfsetlinewidth{0.000000pt}%
\definecolor{currentstroke}{rgb}{0.000000,0.000000,0.000000}%
\pgfsetstrokecolor{currentstroke}%
\pgfsetstrokeopacity{0.700000}%
\pgfsetdash{}{0pt}%
\pgfpathmoveto{\pgfqpoint{9.664183in}{1.336627in}}%
\pgfpathcurveto{\pgfqpoint{9.669227in}{1.336627in}}{\pgfqpoint{9.674065in}{1.338631in}}{\pgfqpoint{9.677631in}{1.342197in}}%
\pgfpathcurveto{\pgfqpoint{9.681198in}{1.345764in}}{\pgfqpoint{9.683202in}{1.350602in}}{\pgfqpoint{9.683202in}{1.355645in}}%
\pgfpathcurveto{\pgfqpoint{9.683202in}{1.360689in}}{\pgfqpoint{9.681198in}{1.365527in}}{\pgfqpoint{9.677631in}{1.369093in}}%
\pgfpathcurveto{\pgfqpoint{9.674065in}{1.372660in}}{\pgfqpoint{9.669227in}{1.374663in}}{\pgfqpoint{9.664183in}{1.374663in}}%
\pgfpathcurveto{\pgfqpoint{9.659140in}{1.374663in}}{\pgfqpoint{9.654302in}{1.372660in}}{\pgfqpoint{9.650736in}{1.369093in}}%
\pgfpathcurveto{\pgfqpoint{9.647169in}{1.365527in}}{\pgfqpoint{9.645165in}{1.360689in}}{\pgfqpoint{9.645165in}{1.355645in}}%
\pgfpathcurveto{\pgfqpoint{9.645165in}{1.350602in}}{\pgfqpoint{9.647169in}{1.345764in}}{\pgfqpoint{9.650736in}{1.342197in}}%
\pgfpathcurveto{\pgfqpoint{9.654302in}{1.338631in}}{\pgfqpoint{9.659140in}{1.336627in}}{\pgfqpoint{9.664183in}{1.336627in}}%
\pgfpathclose%
\pgfusepath{fill}%
\end{pgfscope}%
\begin{pgfscope}%
\pgfpathrectangle{\pgfqpoint{6.572727in}{0.474100in}}{\pgfqpoint{4.227273in}{3.318700in}}%
\pgfusepath{clip}%
\pgfsetbuttcap%
\pgfsetroundjoin%
\definecolor{currentfill}{rgb}{0.267004,0.004874,0.329415}%
\pgfsetfillcolor{currentfill}%
\pgfsetfillopacity{0.700000}%
\pgfsetlinewidth{0.000000pt}%
\definecolor{currentstroke}{rgb}{0.000000,0.000000,0.000000}%
\pgfsetstrokecolor{currentstroke}%
\pgfsetstrokeopacity{0.700000}%
\pgfsetdash{}{0pt}%
\pgfpathmoveto{\pgfqpoint{7.833822in}{1.085122in}}%
\pgfpathcurveto{\pgfqpoint{7.838865in}{1.085122in}}{\pgfqpoint{7.843703in}{1.087126in}}{\pgfqpoint{7.847269in}{1.090692in}}%
\pgfpathcurveto{\pgfqpoint{7.850836in}{1.094259in}}{\pgfqpoint{7.852840in}{1.099096in}}{\pgfqpoint{7.852840in}{1.104140in}}%
\pgfpathcurveto{\pgfqpoint{7.852840in}{1.109184in}}{\pgfqpoint{7.850836in}{1.114021in}}{\pgfqpoint{7.847269in}{1.117588in}}%
\pgfpathcurveto{\pgfqpoint{7.843703in}{1.121154in}}{\pgfqpoint{7.838865in}{1.123158in}}{\pgfqpoint{7.833822in}{1.123158in}}%
\pgfpathcurveto{\pgfqpoint{7.828778in}{1.123158in}}{\pgfqpoint{7.823940in}{1.121154in}}{\pgfqpoint{7.820374in}{1.117588in}}%
\pgfpathcurveto{\pgfqpoint{7.816807in}{1.114021in}}{\pgfqpoint{7.814803in}{1.109184in}}{\pgfqpoint{7.814803in}{1.104140in}}%
\pgfpathcurveto{\pgfqpoint{7.814803in}{1.099096in}}{\pgfqpoint{7.816807in}{1.094259in}}{\pgfqpoint{7.820374in}{1.090692in}}%
\pgfpathcurveto{\pgfqpoint{7.823940in}{1.087126in}}{\pgfqpoint{7.828778in}{1.085122in}}{\pgfqpoint{7.833822in}{1.085122in}}%
\pgfpathclose%
\pgfusepath{fill}%
\end{pgfscope}%
\begin{pgfscope}%
\pgfpathrectangle{\pgfqpoint{6.572727in}{0.474100in}}{\pgfqpoint{4.227273in}{3.318700in}}%
\pgfusepath{clip}%
\pgfsetbuttcap%
\pgfsetroundjoin%
\definecolor{currentfill}{rgb}{0.993248,0.906157,0.143936}%
\pgfsetfillcolor{currentfill}%
\pgfsetfillopacity{0.700000}%
\pgfsetlinewidth{0.000000pt}%
\definecolor{currentstroke}{rgb}{0.000000,0.000000,0.000000}%
\pgfsetstrokecolor{currentstroke}%
\pgfsetstrokeopacity{0.700000}%
\pgfsetdash{}{0pt}%
\pgfpathmoveto{\pgfqpoint{8.321344in}{3.141455in}}%
\pgfpathcurveto{\pgfqpoint{8.326388in}{3.141455in}}{\pgfqpoint{8.331226in}{3.143459in}}{\pgfqpoint{8.334792in}{3.147025in}}%
\pgfpathcurveto{\pgfqpoint{8.338359in}{3.150592in}}{\pgfqpoint{8.340363in}{3.155429in}}{\pgfqpoint{8.340363in}{3.160473in}}%
\pgfpathcurveto{\pgfqpoint{8.340363in}{3.165517in}}{\pgfqpoint{8.338359in}{3.170354in}}{\pgfqpoint{8.334792in}{3.173921in}}%
\pgfpathcurveto{\pgfqpoint{8.331226in}{3.177487in}}{\pgfqpoint{8.326388in}{3.179491in}}{\pgfqpoint{8.321344in}{3.179491in}}%
\pgfpathcurveto{\pgfqpoint{8.316301in}{3.179491in}}{\pgfqpoint{8.311463in}{3.177487in}}{\pgfqpoint{8.307897in}{3.173921in}}%
\pgfpathcurveto{\pgfqpoint{8.304330in}{3.170354in}}{\pgfqpoint{8.302326in}{3.165517in}}{\pgfqpoint{8.302326in}{3.160473in}}%
\pgfpathcurveto{\pgfqpoint{8.302326in}{3.155429in}}{\pgfqpoint{8.304330in}{3.150592in}}{\pgfqpoint{8.307897in}{3.147025in}}%
\pgfpathcurveto{\pgfqpoint{8.311463in}{3.143459in}}{\pgfqpoint{8.316301in}{3.141455in}}{\pgfqpoint{8.321344in}{3.141455in}}%
\pgfpathclose%
\pgfusepath{fill}%
\end{pgfscope}%
\begin{pgfscope}%
\pgfpathrectangle{\pgfqpoint{6.572727in}{0.474100in}}{\pgfqpoint{4.227273in}{3.318700in}}%
\pgfusepath{clip}%
\pgfsetbuttcap%
\pgfsetroundjoin%
\definecolor{currentfill}{rgb}{0.993248,0.906157,0.143936}%
\pgfsetfillcolor{currentfill}%
\pgfsetfillopacity{0.700000}%
\pgfsetlinewidth{0.000000pt}%
\definecolor{currentstroke}{rgb}{0.000000,0.000000,0.000000}%
\pgfsetstrokecolor{currentstroke}%
\pgfsetstrokeopacity{0.700000}%
\pgfsetdash{}{0pt}%
\pgfpathmoveto{\pgfqpoint{8.114934in}{2.271897in}}%
\pgfpathcurveto{\pgfqpoint{8.119978in}{2.271897in}}{\pgfqpoint{8.124815in}{2.273900in}}{\pgfqpoint{8.128382in}{2.277467in}}%
\pgfpathcurveto{\pgfqpoint{8.131948in}{2.281033in}}{\pgfqpoint{8.133952in}{2.285871in}}{\pgfqpoint{8.133952in}{2.290915in}}%
\pgfpathcurveto{\pgfqpoint{8.133952in}{2.295958in}}{\pgfqpoint{8.131948in}{2.300796in}}{\pgfqpoint{8.128382in}{2.304363in}}%
\pgfpathcurveto{\pgfqpoint{8.124815in}{2.307929in}}{\pgfqpoint{8.119978in}{2.309933in}}{\pgfqpoint{8.114934in}{2.309933in}}%
\pgfpathcurveto{\pgfqpoint{8.109890in}{2.309933in}}{\pgfqpoint{8.105053in}{2.307929in}}{\pgfqpoint{8.101486in}{2.304363in}}%
\pgfpathcurveto{\pgfqpoint{8.097920in}{2.300796in}}{\pgfqpoint{8.095916in}{2.295958in}}{\pgfqpoint{8.095916in}{2.290915in}}%
\pgfpathcurveto{\pgfqpoint{8.095916in}{2.285871in}}{\pgfqpoint{8.097920in}{2.281033in}}{\pgfqpoint{8.101486in}{2.277467in}}%
\pgfpathcurveto{\pgfqpoint{8.105053in}{2.273900in}}{\pgfqpoint{8.109890in}{2.271897in}}{\pgfqpoint{8.114934in}{2.271897in}}%
\pgfpathclose%
\pgfusepath{fill}%
\end{pgfscope}%
\begin{pgfscope}%
\pgfpathrectangle{\pgfqpoint{6.572727in}{0.474100in}}{\pgfqpoint{4.227273in}{3.318700in}}%
\pgfusepath{clip}%
\pgfsetbuttcap%
\pgfsetroundjoin%
\definecolor{currentfill}{rgb}{0.127568,0.566949,0.550556}%
\pgfsetfillcolor{currentfill}%
\pgfsetfillopacity{0.700000}%
\pgfsetlinewidth{0.000000pt}%
\definecolor{currentstroke}{rgb}{0.000000,0.000000,0.000000}%
\pgfsetstrokecolor{currentstroke}%
\pgfsetstrokeopacity{0.700000}%
\pgfsetdash{}{0pt}%
\pgfpathmoveto{\pgfqpoint{9.606244in}{1.582986in}}%
\pgfpathcurveto{\pgfqpoint{9.611288in}{1.582986in}}{\pgfqpoint{9.616126in}{1.584990in}}{\pgfqpoint{9.619692in}{1.588556in}}%
\pgfpathcurveto{\pgfqpoint{9.623258in}{1.592122in}}{\pgfqpoint{9.625262in}{1.596960in}}{\pgfqpoint{9.625262in}{1.602004in}}%
\pgfpathcurveto{\pgfqpoint{9.625262in}{1.607047in}}{\pgfqpoint{9.623258in}{1.611885in}}{\pgfqpoint{9.619692in}{1.615452in}}%
\pgfpathcurveto{\pgfqpoint{9.616126in}{1.619018in}}{\pgfqpoint{9.611288in}{1.621022in}}{\pgfqpoint{9.606244in}{1.621022in}}%
\pgfpathcurveto{\pgfqpoint{9.601201in}{1.621022in}}{\pgfqpoint{9.596363in}{1.619018in}}{\pgfqpoint{9.592796in}{1.615452in}}%
\pgfpathcurveto{\pgfqpoint{9.589230in}{1.611885in}}{\pgfqpoint{9.587226in}{1.607047in}}{\pgfqpoint{9.587226in}{1.602004in}}%
\pgfpathcurveto{\pgfqpoint{9.587226in}{1.596960in}}{\pgfqpoint{9.589230in}{1.592122in}}{\pgfqpoint{9.592796in}{1.588556in}}%
\pgfpathcurveto{\pgfqpoint{9.596363in}{1.584990in}}{\pgfqpoint{9.601201in}{1.582986in}}{\pgfqpoint{9.606244in}{1.582986in}}%
\pgfpathclose%
\pgfusepath{fill}%
\end{pgfscope}%
\begin{pgfscope}%
\pgfpathrectangle{\pgfqpoint{6.572727in}{0.474100in}}{\pgfqpoint{4.227273in}{3.318700in}}%
\pgfusepath{clip}%
\pgfsetbuttcap%
\pgfsetroundjoin%
\definecolor{currentfill}{rgb}{0.993248,0.906157,0.143936}%
\pgfsetfillcolor{currentfill}%
\pgfsetfillopacity{0.700000}%
\pgfsetlinewidth{0.000000pt}%
\definecolor{currentstroke}{rgb}{0.000000,0.000000,0.000000}%
\pgfsetstrokecolor{currentstroke}%
\pgfsetstrokeopacity{0.700000}%
\pgfsetdash{}{0pt}%
\pgfpathmoveto{\pgfqpoint{7.802058in}{2.341009in}}%
\pgfpathcurveto{\pgfqpoint{7.807101in}{2.341009in}}{\pgfqpoint{7.811939in}{2.343012in}}{\pgfqpoint{7.815505in}{2.346579in}}%
\pgfpathcurveto{\pgfqpoint{7.819072in}{2.350145in}}{\pgfqpoint{7.821076in}{2.354983in}}{\pgfqpoint{7.821076in}{2.360027in}}%
\pgfpathcurveto{\pgfqpoint{7.821076in}{2.365070in}}{\pgfqpoint{7.819072in}{2.369908in}}{\pgfqpoint{7.815505in}{2.373475in}}%
\pgfpathcurveto{\pgfqpoint{7.811939in}{2.377041in}}{\pgfqpoint{7.807101in}{2.379045in}}{\pgfqpoint{7.802058in}{2.379045in}}%
\pgfpathcurveto{\pgfqpoint{7.797014in}{2.379045in}}{\pgfqpoint{7.792176in}{2.377041in}}{\pgfqpoint{7.788610in}{2.373475in}}%
\pgfpathcurveto{\pgfqpoint{7.785043in}{2.369908in}}{\pgfqpoint{7.783039in}{2.365070in}}{\pgfqpoint{7.783039in}{2.360027in}}%
\pgfpathcurveto{\pgfqpoint{7.783039in}{2.354983in}}{\pgfqpoint{7.785043in}{2.350145in}}{\pgfqpoint{7.788610in}{2.346579in}}%
\pgfpathcurveto{\pgfqpoint{7.792176in}{2.343012in}}{\pgfqpoint{7.797014in}{2.341009in}}{\pgfqpoint{7.802058in}{2.341009in}}%
\pgfpathclose%
\pgfusepath{fill}%
\end{pgfscope}%
\begin{pgfscope}%
\pgfpathrectangle{\pgfqpoint{6.572727in}{0.474100in}}{\pgfqpoint{4.227273in}{3.318700in}}%
\pgfusepath{clip}%
\pgfsetbuttcap%
\pgfsetroundjoin%
\definecolor{currentfill}{rgb}{0.127568,0.566949,0.550556}%
\pgfsetfillcolor{currentfill}%
\pgfsetfillopacity{0.700000}%
\pgfsetlinewidth{0.000000pt}%
\definecolor{currentstroke}{rgb}{0.000000,0.000000,0.000000}%
\pgfsetstrokecolor{currentstroke}%
\pgfsetstrokeopacity{0.700000}%
\pgfsetdash{}{0pt}%
\pgfpathmoveto{\pgfqpoint{9.261556in}{1.255216in}}%
\pgfpathcurveto{\pgfqpoint{9.266599in}{1.255216in}}{\pgfqpoint{9.271437in}{1.257220in}}{\pgfqpoint{9.275003in}{1.260786in}}%
\pgfpathcurveto{\pgfqpoint{9.278570in}{1.264353in}}{\pgfqpoint{9.280574in}{1.269190in}}{\pgfqpoint{9.280574in}{1.274234in}}%
\pgfpathcurveto{\pgfqpoint{9.280574in}{1.279278in}}{\pgfqpoint{9.278570in}{1.284115in}}{\pgfqpoint{9.275003in}{1.287682in}}%
\pgfpathcurveto{\pgfqpoint{9.271437in}{1.291248in}}{\pgfqpoint{9.266599in}{1.293252in}}{\pgfqpoint{9.261556in}{1.293252in}}%
\pgfpathcurveto{\pgfqpoint{9.256512in}{1.293252in}}{\pgfqpoint{9.251674in}{1.291248in}}{\pgfqpoint{9.248108in}{1.287682in}}%
\pgfpathcurveto{\pgfqpoint{9.244541in}{1.284115in}}{\pgfqpoint{9.242537in}{1.279278in}}{\pgfqpoint{9.242537in}{1.274234in}}%
\pgfpathcurveto{\pgfqpoint{9.242537in}{1.269190in}}{\pgfqpoint{9.244541in}{1.264353in}}{\pgfqpoint{9.248108in}{1.260786in}}%
\pgfpathcurveto{\pgfqpoint{9.251674in}{1.257220in}}{\pgfqpoint{9.256512in}{1.255216in}}{\pgfqpoint{9.261556in}{1.255216in}}%
\pgfpathclose%
\pgfusepath{fill}%
\end{pgfscope}%
\begin{pgfscope}%
\pgfpathrectangle{\pgfqpoint{6.572727in}{0.474100in}}{\pgfqpoint{4.227273in}{3.318700in}}%
\pgfusepath{clip}%
\pgfsetbuttcap%
\pgfsetroundjoin%
\definecolor{currentfill}{rgb}{0.993248,0.906157,0.143936}%
\pgfsetfillcolor{currentfill}%
\pgfsetfillopacity{0.700000}%
\pgfsetlinewidth{0.000000pt}%
\definecolor{currentstroke}{rgb}{0.000000,0.000000,0.000000}%
\pgfsetstrokecolor{currentstroke}%
\pgfsetstrokeopacity{0.700000}%
\pgfsetdash{}{0pt}%
\pgfpathmoveto{\pgfqpoint{8.208631in}{2.591468in}}%
\pgfpathcurveto{\pgfqpoint{8.213675in}{2.591468in}}{\pgfqpoint{8.218513in}{2.593472in}}{\pgfqpoint{8.222079in}{2.597039in}}%
\pgfpathcurveto{\pgfqpoint{8.225645in}{2.600605in}}{\pgfqpoint{8.227649in}{2.605443in}}{\pgfqpoint{8.227649in}{2.610486in}}%
\pgfpathcurveto{\pgfqpoint{8.227649in}{2.615530in}}{\pgfqpoint{8.225645in}{2.620368in}}{\pgfqpoint{8.222079in}{2.623934in}}%
\pgfpathcurveto{\pgfqpoint{8.218513in}{2.627501in}}{\pgfqpoint{8.213675in}{2.629505in}}{\pgfqpoint{8.208631in}{2.629505in}}%
\pgfpathcurveto{\pgfqpoint{8.203588in}{2.629505in}}{\pgfqpoint{8.198750in}{2.627501in}}{\pgfqpoint{8.195183in}{2.623934in}}%
\pgfpathcurveto{\pgfqpoint{8.191617in}{2.620368in}}{\pgfqpoint{8.189613in}{2.615530in}}{\pgfqpoint{8.189613in}{2.610486in}}%
\pgfpathcurveto{\pgfqpoint{8.189613in}{2.605443in}}{\pgfqpoint{8.191617in}{2.600605in}}{\pgfqpoint{8.195183in}{2.597039in}}%
\pgfpathcurveto{\pgfqpoint{8.198750in}{2.593472in}}{\pgfqpoint{8.203588in}{2.591468in}}{\pgfqpoint{8.208631in}{2.591468in}}%
\pgfpathclose%
\pgfusepath{fill}%
\end{pgfscope}%
\begin{pgfscope}%
\pgfpathrectangle{\pgfqpoint{6.572727in}{0.474100in}}{\pgfqpoint{4.227273in}{3.318700in}}%
\pgfusepath{clip}%
\pgfsetbuttcap%
\pgfsetroundjoin%
\definecolor{currentfill}{rgb}{0.267004,0.004874,0.329415}%
\pgfsetfillcolor{currentfill}%
\pgfsetfillopacity{0.700000}%
\pgfsetlinewidth{0.000000pt}%
\definecolor{currentstroke}{rgb}{0.000000,0.000000,0.000000}%
\pgfsetstrokecolor{currentstroke}%
\pgfsetstrokeopacity{0.700000}%
\pgfsetdash{}{0pt}%
\pgfpathmoveto{\pgfqpoint{7.589720in}{1.922219in}}%
\pgfpathcurveto{\pgfqpoint{7.594764in}{1.922219in}}{\pgfqpoint{7.599601in}{1.924223in}}{\pgfqpoint{7.603168in}{1.927789in}}%
\pgfpathcurveto{\pgfqpoint{7.606734in}{1.931355in}}{\pgfqpoint{7.608738in}{1.936193in}}{\pgfqpoint{7.608738in}{1.941237in}}%
\pgfpathcurveto{\pgfqpoint{7.608738in}{1.946280in}}{\pgfqpoint{7.606734in}{1.951118in}}{\pgfqpoint{7.603168in}{1.954685in}}%
\pgfpathcurveto{\pgfqpoint{7.599601in}{1.958251in}}{\pgfqpoint{7.594764in}{1.960255in}}{\pgfqpoint{7.589720in}{1.960255in}}%
\pgfpathcurveto{\pgfqpoint{7.584676in}{1.960255in}}{\pgfqpoint{7.579838in}{1.958251in}}{\pgfqpoint{7.576272in}{1.954685in}}%
\pgfpathcurveto{\pgfqpoint{7.572706in}{1.951118in}}{\pgfqpoint{7.570702in}{1.946280in}}{\pgfqpoint{7.570702in}{1.941237in}}%
\pgfpathcurveto{\pgfqpoint{7.570702in}{1.936193in}}{\pgfqpoint{7.572706in}{1.931355in}}{\pgfqpoint{7.576272in}{1.927789in}}%
\pgfpathcurveto{\pgfqpoint{7.579838in}{1.924223in}}{\pgfqpoint{7.584676in}{1.922219in}}{\pgfqpoint{7.589720in}{1.922219in}}%
\pgfpathclose%
\pgfusepath{fill}%
\end{pgfscope}%
\begin{pgfscope}%
\pgfpathrectangle{\pgfqpoint{6.572727in}{0.474100in}}{\pgfqpoint{4.227273in}{3.318700in}}%
\pgfusepath{clip}%
\pgfsetbuttcap%
\pgfsetroundjoin%
\definecolor{currentfill}{rgb}{0.267004,0.004874,0.329415}%
\pgfsetfillcolor{currentfill}%
\pgfsetfillopacity{0.700000}%
\pgfsetlinewidth{0.000000pt}%
\definecolor{currentstroke}{rgb}{0.000000,0.000000,0.000000}%
\pgfsetstrokecolor{currentstroke}%
\pgfsetstrokeopacity{0.700000}%
\pgfsetdash{}{0pt}%
\pgfpathmoveto{\pgfqpoint{7.919911in}{1.991658in}}%
\pgfpathcurveto{\pgfqpoint{7.924954in}{1.991658in}}{\pgfqpoint{7.929792in}{1.993662in}}{\pgfqpoint{7.933358in}{1.997228in}}%
\pgfpathcurveto{\pgfqpoint{7.936925in}{2.000795in}}{\pgfqpoint{7.938929in}{2.005633in}}{\pgfqpoint{7.938929in}{2.010676in}}%
\pgfpathcurveto{\pgfqpoint{7.938929in}{2.015720in}}{\pgfqpoint{7.936925in}{2.020558in}}{\pgfqpoint{7.933358in}{2.024124in}}%
\pgfpathcurveto{\pgfqpoint{7.929792in}{2.027690in}}{\pgfqpoint{7.924954in}{2.029694in}}{\pgfqpoint{7.919911in}{2.029694in}}%
\pgfpathcurveto{\pgfqpoint{7.914867in}{2.029694in}}{\pgfqpoint{7.910029in}{2.027690in}}{\pgfqpoint{7.906463in}{2.024124in}}%
\pgfpathcurveto{\pgfqpoint{7.902896in}{2.020558in}}{\pgfqpoint{7.900892in}{2.015720in}}{\pgfqpoint{7.900892in}{2.010676in}}%
\pgfpathcurveto{\pgfqpoint{7.900892in}{2.005633in}}{\pgfqpoint{7.902896in}{2.000795in}}{\pgfqpoint{7.906463in}{1.997228in}}%
\pgfpathcurveto{\pgfqpoint{7.910029in}{1.993662in}}{\pgfqpoint{7.914867in}{1.991658in}}{\pgfqpoint{7.919911in}{1.991658in}}%
\pgfpathclose%
\pgfusepath{fill}%
\end{pgfscope}%
\begin{pgfscope}%
\pgfpathrectangle{\pgfqpoint{6.572727in}{0.474100in}}{\pgfqpoint{4.227273in}{3.318700in}}%
\pgfusepath{clip}%
\pgfsetbuttcap%
\pgfsetroundjoin%
\definecolor{currentfill}{rgb}{0.127568,0.566949,0.550556}%
\pgfsetfillcolor{currentfill}%
\pgfsetfillopacity{0.700000}%
\pgfsetlinewidth{0.000000pt}%
\definecolor{currentstroke}{rgb}{0.000000,0.000000,0.000000}%
\pgfsetstrokecolor{currentstroke}%
\pgfsetstrokeopacity{0.700000}%
\pgfsetdash{}{0pt}%
\pgfpathmoveto{\pgfqpoint{9.923362in}{1.636330in}}%
\pgfpathcurveto{\pgfqpoint{9.928405in}{1.636330in}}{\pgfqpoint{9.933243in}{1.638334in}}{\pgfqpoint{9.936809in}{1.641900in}}%
\pgfpathcurveto{\pgfqpoint{9.940376in}{1.645467in}}{\pgfqpoint{9.942380in}{1.650305in}}{\pgfqpoint{9.942380in}{1.655348in}}%
\pgfpathcurveto{\pgfqpoint{9.942380in}{1.660392in}}{\pgfqpoint{9.940376in}{1.665230in}}{\pgfqpoint{9.936809in}{1.668796in}}%
\pgfpathcurveto{\pgfqpoint{9.933243in}{1.672362in}}{\pgfqpoint{9.928405in}{1.674366in}}{\pgfqpoint{9.923362in}{1.674366in}}%
\pgfpathcurveto{\pgfqpoint{9.918318in}{1.674366in}}{\pgfqpoint{9.913480in}{1.672362in}}{\pgfqpoint{9.909914in}{1.668796in}}%
\pgfpathcurveto{\pgfqpoint{9.906347in}{1.665230in}}{\pgfqpoint{9.904343in}{1.660392in}}{\pgfqpoint{9.904343in}{1.655348in}}%
\pgfpathcurveto{\pgfqpoint{9.904343in}{1.650305in}}{\pgfqpoint{9.906347in}{1.645467in}}{\pgfqpoint{9.909914in}{1.641900in}}%
\pgfpathcurveto{\pgfqpoint{9.913480in}{1.638334in}}{\pgfqpoint{9.918318in}{1.636330in}}{\pgfqpoint{9.923362in}{1.636330in}}%
\pgfpathclose%
\pgfusepath{fill}%
\end{pgfscope}%
\begin{pgfscope}%
\pgfpathrectangle{\pgfqpoint{6.572727in}{0.474100in}}{\pgfqpoint{4.227273in}{3.318700in}}%
\pgfusepath{clip}%
\pgfsetbuttcap%
\pgfsetroundjoin%
\definecolor{currentfill}{rgb}{0.267004,0.004874,0.329415}%
\pgfsetfillcolor{currentfill}%
\pgfsetfillopacity{0.700000}%
\pgfsetlinewidth{0.000000pt}%
\definecolor{currentstroke}{rgb}{0.000000,0.000000,0.000000}%
\pgfsetstrokecolor{currentstroke}%
\pgfsetstrokeopacity{0.700000}%
\pgfsetdash{}{0pt}%
\pgfpathmoveto{\pgfqpoint{8.561788in}{1.923436in}}%
\pgfpathcurveto{\pgfqpoint{8.566831in}{1.923436in}}{\pgfqpoint{8.571669in}{1.925440in}}{\pgfqpoint{8.575236in}{1.929007in}}%
\pgfpathcurveto{\pgfqpoint{8.578802in}{1.932573in}}{\pgfqpoint{8.580806in}{1.937411in}}{\pgfqpoint{8.580806in}{1.942454in}}%
\pgfpathcurveto{\pgfqpoint{8.580806in}{1.947498in}}{\pgfqpoint{8.578802in}{1.952336in}}{\pgfqpoint{8.575236in}{1.955902in}}%
\pgfpathcurveto{\pgfqpoint{8.571669in}{1.959469in}}{\pgfqpoint{8.566831in}{1.961473in}}{\pgfqpoint{8.561788in}{1.961473in}}%
\pgfpathcurveto{\pgfqpoint{8.556744in}{1.961473in}}{\pgfqpoint{8.551906in}{1.959469in}}{\pgfqpoint{8.548340in}{1.955902in}}%
\pgfpathcurveto{\pgfqpoint{8.544773in}{1.952336in}}{\pgfqpoint{8.542770in}{1.947498in}}{\pgfqpoint{8.542770in}{1.942454in}}%
\pgfpathcurveto{\pgfqpoint{8.542770in}{1.937411in}}{\pgfqpoint{8.544773in}{1.932573in}}{\pgfqpoint{8.548340in}{1.929007in}}%
\pgfpathcurveto{\pgfqpoint{8.551906in}{1.925440in}}{\pgfqpoint{8.556744in}{1.923436in}}{\pgfqpoint{8.561788in}{1.923436in}}%
\pgfpathclose%
\pgfusepath{fill}%
\end{pgfscope}%
\begin{pgfscope}%
\pgfpathrectangle{\pgfqpoint{6.572727in}{0.474100in}}{\pgfqpoint{4.227273in}{3.318700in}}%
\pgfusepath{clip}%
\pgfsetbuttcap%
\pgfsetroundjoin%
\definecolor{currentfill}{rgb}{0.993248,0.906157,0.143936}%
\pgfsetfillcolor{currentfill}%
\pgfsetfillopacity{0.700000}%
\pgfsetlinewidth{0.000000pt}%
\definecolor{currentstroke}{rgb}{0.000000,0.000000,0.000000}%
\pgfsetstrokecolor{currentstroke}%
\pgfsetstrokeopacity{0.700000}%
\pgfsetdash{}{0pt}%
\pgfpathmoveto{\pgfqpoint{8.647684in}{2.735414in}}%
\pgfpathcurveto{\pgfqpoint{8.652728in}{2.735414in}}{\pgfqpoint{8.657566in}{2.737418in}}{\pgfqpoint{8.661132in}{2.740985in}}%
\pgfpathcurveto{\pgfqpoint{8.664698in}{2.744551in}}{\pgfqpoint{8.666702in}{2.749389in}}{\pgfqpoint{8.666702in}{2.754433in}}%
\pgfpathcurveto{\pgfqpoint{8.666702in}{2.759476in}}{\pgfqpoint{8.664698in}{2.764314in}}{\pgfqpoint{8.661132in}{2.767880in}}%
\pgfpathcurveto{\pgfqpoint{8.657566in}{2.771447in}}{\pgfqpoint{8.652728in}{2.773451in}}{\pgfqpoint{8.647684in}{2.773451in}}%
\pgfpathcurveto{\pgfqpoint{8.642640in}{2.773451in}}{\pgfqpoint{8.637803in}{2.771447in}}{\pgfqpoint{8.634236in}{2.767880in}}%
\pgfpathcurveto{\pgfqpoint{8.630670in}{2.764314in}}{\pgfqpoint{8.628666in}{2.759476in}}{\pgfqpoint{8.628666in}{2.754433in}}%
\pgfpathcurveto{\pgfqpoint{8.628666in}{2.749389in}}{\pgfqpoint{8.630670in}{2.744551in}}{\pgfqpoint{8.634236in}{2.740985in}}%
\pgfpathcurveto{\pgfqpoint{8.637803in}{2.737418in}}{\pgfqpoint{8.642640in}{2.735414in}}{\pgfqpoint{8.647684in}{2.735414in}}%
\pgfpathclose%
\pgfusepath{fill}%
\end{pgfscope}%
\begin{pgfscope}%
\pgfpathrectangle{\pgfqpoint{6.572727in}{0.474100in}}{\pgfqpoint{4.227273in}{3.318700in}}%
\pgfusepath{clip}%
\pgfsetbuttcap%
\pgfsetroundjoin%
\definecolor{currentfill}{rgb}{0.993248,0.906157,0.143936}%
\pgfsetfillcolor{currentfill}%
\pgfsetfillopacity{0.700000}%
\pgfsetlinewidth{0.000000pt}%
\definecolor{currentstroke}{rgb}{0.000000,0.000000,0.000000}%
\pgfsetstrokecolor{currentstroke}%
\pgfsetstrokeopacity{0.700000}%
\pgfsetdash{}{0pt}%
\pgfpathmoveto{\pgfqpoint{8.316492in}{2.216537in}}%
\pgfpathcurveto{\pgfqpoint{8.321535in}{2.216537in}}{\pgfqpoint{8.326373in}{2.218541in}}{\pgfqpoint{8.329940in}{2.222107in}}%
\pgfpathcurveto{\pgfqpoint{8.333506in}{2.225673in}}{\pgfqpoint{8.335510in}{2.230511in}}{\pgfqpoint{8.335510in}{2.235555in}}%
\pgfpathcurveto{\pgfqpoint{8.335510in}{2.240599in}}{\pgfqpoint{8.333506in}{2.245436in}}{\pgfqpoint{8.329940in}{2.249003in}}%
\pgfpathcurveto{\pgfqpoint{8.326373in}{2.252569in}}{\pgfqpoint{8.321535in}{2.254573in}}{\pgfqpoint{8.316492in}{2.254573in}}%
\pgfpathcurveto{\pgfqpoint{8.311448in}{2.254573in}}{\pgfqpoint{8.306610in}{2.252569in}}{\pgfqpoint{8.303044in}{2.249003in}}%
\pgfpathcurveto{\pgfqpoint{8.299478in}{2.245436in}}{\pgfqpoint{8.297474in}{2.240599in}}{\pgfqpoint{8.297474in}{2.235555in}}%
\pgfpathcurveto{\pgfqpoint{8.297474in}{2.230511in}}{\pgfqpoint{8.299478in}{2.225673in}}{\pgfqpoint{8.303044in}{2.222107in}}%
\pgfpathcurveto{\pgfqpoint{8.306610in}{2.218541in}}{\pgfqpoint{8.311448in}{2.216537in}}{\pgfqpoint{8.316492in}{2.216537in}}%
\pgfpathclose%
\pgfusepath{fill}%
\end{pgfscope}%
\begin{pgfscope}%
\pgfpathrectangle{\pgfqpoint{6.572727in}{0.474100in}}{\pgfqpoint{4.227273in}{3.318700in}}%
\pgfusepath{clip}%
\pgfsetbuttcap%
\pgfsetroundjoin%
\definecolor{currentfill}{rgb}{0.993248,0.906157,0.143936}%
\pgfsetfillcolor{currentfill}%
\pgfsetfillopacity{0.700000}%
\pgfsetlinewidth{0.000000pt}%
\definecolor{currentstroke}{rgb}{0.000000,0.000000,0.000000}%
\pgfsetstrokecolor{currentstroke}%
\pgfsetstrokeopacity{0.700000}%
\pgfsetdash{}{0pt}%
\pgfpathmoveto{\pgfqpoint{8.881533in}{3.043491in}}%
\pgfpathcurveto{\pgfqpoint{8.886576in}{3.043491in}}{\pgfqpoint{8.891414in}{3.045495in}}{\pgfqpoint{8.894980in}{3.049061in}}%
\pgfpathcurveto{\pgfqpoint{8.898547in}{3.052628in}}{\pgfqpoint{8.900551in}{3.057466in}}{\pgfqpoint{8.900551in}{3.062509in}}%
\pgfpathcurveto{\pgfqpoint{8.900551in}{3.067553in}}{\pgfqpoint{8.898547in}{3.072391in}}{\pgfqpoint{8.894980in}{3.075957in}}%
\pgfpathcurveto{\pgfqpoint{8.891414in}{3.079524in}}{\pgfqpoint{8.886576in}{3.081527in}}{\pgfqpoint{8.881533in}{3.081527in}}%
\pgfpathcurveto{\pgfqpoint{8.876489in}{3.081527in}}{\pgfqpoint{8.871651in}{3.079524in}}{\pgfqpoint{8.868085in}{3.075957in}}%
\pgfpathcurveto{\pgfqpoint{8.864518in}{3.072391in}}{\pgfqpoint{8.862514in}{3.067553in}}{\pgfqpoint{8.862514in}{3.062509in}}%
\pgfpathcurveto{\pgfqpoint{8.862514in}{3.057466in}}{\pgfqpoint{8.864518in}{3.052628in}}{\pgfqpoint{8.868085in}{3.049061in}}%
\pgfpathcurveto{\pgfqpoint{8.871651in}{3.045495in}}{\pgfqpoint{8.876489in}{3.043491in}}{\pgfqpoint{8.881533in}{3.043491in}}%
\pgfpathclose%
\pgfusepath{fill}%
\end{pgfscope}%
\begin{pgfscope}%
\pgfpathrectangle{\pgfqpoint{6.572727in}{0.474100in}}{\pgfqpoint{4.227273in}{3.318700in}}%
\pgfusepath{clip}%
\pgfsetbuttcap%
\pgfsetroundjoin%
\definecolor{currentfill}{rgb}{0.267004,0.004874,0.329415}%
\pgfsetfillcolor{currentfill}%
\pgfsetfillopacity{0.700000}%
\pgfsetlinewidth{0.000000pt}%
\definecolor{currentstroke}{rgb}{0.000000,0.000000,0.000000}%
\pgfsetstrokecolor{currentstroke}%
\pgfsetstrokeopacity{0.700000}%
\pgfsetdash{}{0pt}%
\pgfpathmoveto{\pgfqpoint{7.455704in}{1.996569in}}%
\pgfpathcurveto{\pgfqpoint{7.460748in}{1.996569in}}{\pgfqpoint{7.465586in}{1.998573in}}{\pgfqpoint{7.469152in}{2.002139in}}%
\pgfpathcurveto{\pgfqpoint{7.472718in}{2.005706in}}{\pgfqpoint{7.474722in}{2.010544in}}{\pgfqpoint{7.474722in}{2.015587in}}%
\pgfpathcurveto{\pgfqpoint{7.474722in}{2.020631in}}{\pgfqpoint{7.472718in}{2.025469in}}{\pgfqpoint{7.469152in}{2.029035in}}%
\pgfpathcurveto{\pgfqpoint{7.465586in}{2.032601in}}{\pgfqpoint{7.460748in}{2.034605in}}{\pgfqpoint{7.455704in}{2.034605in}}%
\pgfpathcurveto{\pgfqpoint{7.450661in}{2.034605in}}{\pgfqpoint{7.445823in}{2.032601in}}{\pgfqpoint{7.442256in}{2.029035in}}%
\pgfpathcurveto{\pgfqpoint{7.438690in}{2.025469in}}{\pgfqpoint{7.436686in}{2.020631in}}{\pgfqpoint{7.436686in}{2.015587in}}%
\pgfpathcurveto{\pgfqpoint{7.436686in}{2.010544in}}{\pgfqpoint{7.438690in}{2.005706in}}{\pgfqpoint{7.442256in}{2.002139in}}%
\pgfpathcurveto{\pgfqpoint{7.445823in}{1.998573in}}{\pgfqpoint{7.450661in}{1.996569in}}{\pgfqpoint{7.455704in}{1.996569in}}%
\pgfpathclose%
\pgfusepath{fill}%
\end{pgfscope}%
\begin{pgfscope}%
\pgfpathrectangle{\pgfqpoint{6.572727in}{0.474100in}}{\pgfqpoint{4.227273in}{3.318700in}}%
\pgfusepath{clip}%
\pgfsetbuttcap%
\pgfsetroundjoin%
\definecolor{currentfill}{rgb}{0.267004,0.004874,0.329415}%
\pgfsetfillcolor{currentfill}%
\pgfsetfillopacity{0.700000}%
\pgfsetlinewidth{0.000000pt}%
\definecolor{currentstroke}{rgb}{0.000000,0.000000,0.000000}%
\pgfsetstrokecolor{currentstroke}%
\pgfsetstrokeopacity{0.700000}%
\pgfsetdash{}{0pt}%
\pgfpathmoveto{\pgfqpoint{8.261768in}{1.763173in}}%
\pgfpathcurveto{\pgfqpoint{8.266812in}{1.763173in}}{\pgfqpoint{8.271650in}{1.765177in}}{\pgfqpoint{8.275216in}{1.768743in}}%
\pgfpathcurveto{\pgfqpoint{8.278783in}{1.772310in}}{\pgfqpoint{8.280786in}{1.777148in}}{\pgfqpoint{8.280786in}{1.782191in}}%
\pgfpathcurveto{\pgfqpoint{8.280786in}{1.787235in}}{\pgfqpoint{8.278783in}{1.792073in}}{\pgfqpoint{8.275216in}{1.795639in}}%
\pgfpathcurveto{\pgfqpoint{8.271650in}{1.799205in}}{\pgfqpoint{8.266812in}{1.801209in}}{\pgfqpoint{8.261768in}{1.801209in}}%
\pgfpathcurveto{\pgfqpoint{8.256725in}{1.801209in}}{\pgfqpoint{8.251887in}{1.799205in}}{\pgfqpoint{8.248320in}{1.795639in}}%
\pgfpathcurveto{\pgfqpoint{8.244754in}{1.792073in}}{\pgfqpoint{8.242750in}{1.787235in}}{\pgfqpoint{8.242750in}{1.782191in}}%
\pgfpathcurveto{\pgfqpoint{8.242750in}{1.777148in}}{\pgfqpoint{8.244754in}{1.772310in}}{\pgfqpoint{8.248320in}{1.768743in}}%
\pgfpathcurveto{\pgfqpoint{8.251887in}{1.765177in}}{\pgfqpoint{8.256725in}{1.763173in}}{\pgfqpoint{8.261768in}{1.763173in}}%
\pgfpathclose%
\pgfusepath{fill}%
\end{pgfscope}%
\begin{pgfscope}%
\pgfpathrectangle{\pgfqpoint{6.572727in}{0.474100in}}{\pgfqpoint{4.227273in}{3.318700in}}%
\pgfusepath{clip}%
\pgfsetbuttcap%
\pgfsetroundjoin%
\definecolor{currentfill}{rgb}{0.127568,0.566949,0.550556}%
\pgfsetfillcolor{currentfill}%
\pgfsetfillopacity{0.700000}%
\pgfsetlinewidth{0.000000pt}%
\definecolor{currentstroke}{rgb}{0.000000,0.000000,0.000000}%
\pgfsetstrokecolor{currentstroke}%
\pgfsetstrokeopacity{0.700000}%
\pgfsetdash{}{0pt}%
\pgfpathmoveto{\pgfqpoint{9.843072in}{1.470502in}}%
\pgfpathcurveto{\pgfqpoint{9.848116in}{1.470502in}}{\pgfqpoint{9.852954in}{1.472506in}}{\pgfqpoint{9.856520in}{1.476072in}}%
\pgfpathcurveto{\pgfqpoint{9.860086in}{1.479639in}}{\pgfqpoint{9.862090in}{1.484476in}}{\pgfqpoint{9.862090in}{1.489520in}}%
\pgfpathcurveto{\pgfqpoint{9.862090in}{1.494564in}}{\pgfqpoint{9.860086in}{1.499401in}}{\pgfqpoint{9.856520in}{1.502968in}}%
\pgfpathcurveto{\pgfqpoint{9.852954in}{1.506534in}}{\pgfqpoint{9.848116in}{1.508538in}}{\pgfqpoint{9.843072in}{1.508538in}}%
\pgfpathcurveto{\pgfqpoint{9.838029in}{1.508538in}}{\pgfqpoint{9.833191in}{1.506534in}}{\pgfqpoint{9.829624in}{1.502968in}}%
\pgfpathcurveto{\pgfqpoint{9.826058in}{1.499401in}}{\pgfqpoint{9.824054in}{1.494564in}}{\pgfqpoint{9.824054in}{1.489520in}}%
\pgfpathcurveto{\pgfqpoint{9.824054in}{1.484476in}}{\pgfqpoint{9.826058in}{1.479639in}}{\pgfqpoint{9.829624in}{1.476072in}}%
\pgfpathcurveto{\pgfqpoint{9.833191in}{1.472506in}}{\pgfqpoint{9.838029in}{1.470502in}}{\pgfqpoint{9.843072in}{1.470502in}}%
\pgfpathclose%
\pgfusepath{fill}%
\end{pgfscope}%
\begin{pgfscope}%
\pgfpathrectangle{\pgfqpoint{6.572727in}{0.474100in}}{\pgfqpoint{4.227273in}{3.318700in}}%
\pgfusepath{clip}%
\pgfsetbuttcap%
\pgfsetroundjoin%
\definecolor{currentfill}{rgb}{0.993248,0.906157,0.143936}%
\pgfsetfillcolor{currentfill}%
\pgfsetfillopacity{0.700000}%
\pgfsetlinewidth{0.000000pt}%
\definecolor{currentstroke}{rgb}{0.000000,0.000000,0.000000}%
\pgfsetstrokecolor{currentstroke}%
\pgfsetstrokeopacity{0.700000}%
\pgfsetdash{}{0pt}%
\pgfpathmoveto{\pgfqpoint{8.493389in}{2.685854in}}%
\pgfpathcurveto{\pgfqpoint{8.498432in}{2.685854in}}{\pgfqpoint{8.503270in}{2.687858in}}{\pgfqpoint{8.506836in}{2.691425in}}%
\pgfpathcurveto{\pgfqpoint{8.510403in}{2.694991in}}{\pgfqpoint{8.512407in}{2.699829in}}{\pgfqpoint{8.512407in}{2.704873in}}%
\pgfpathcurveto{\pgfqpoint{8.512407in}{2.709916in}}{\pgfqpoint{8.510403in}{2.714754in}}{\pgfqpoint{8.506836in}{2.718320in}}%
\pgfpathcurveto{\pgfqpoint{8.503270in}{2.721887in}}{\pgfqpoint{8.498432in}{2.723891in}}{\pgfqpoint{8.493389in}{2.723891in}}%
\pgfpathcurveto{\pgfqpoint{8.488345in}{2.723891in}}{\pgfqpoint{8.483507in}{2.721887in}}{\pgfqpoint{8.479941in}{2.718320in}}%
\pgfpathcurveto{\pgfqpoint{8.476374in}{2.714754in}}{\pgfqpoint{8.474370in}{2.709916in}}{\pgfqpoint{8.474370in}{2.704873in}}%
\pgfpathcurveto{\pgfqpoint{8.474370in}{2.699829in}}{\pgfqpoint{8.476374in}{2.694991in}}{\pgfqpoint{8.479941in}{2.691425in}}%
\pgfpathcurveto{\pgfqpoint{8.483507in}{2.687858in}}{\pgfqpoint{8.488345in}{2.685854in}}{\pgfqpoint{8.493389in}{2.685854in}}%
\pgfpathclose%
\pgfusepath{fill}%
\end{pgfscope}%
\begin{pgfscope}%
\pgfpathrectangle{\pgfqpoint{6.572727in}{0.474100in}}{\pgfqpoint{4.227273in}{3.318700in}}%
\pgfusepath{clip}%
\pgfsetbuttcap%
\pgfsetroundjoin%
\definecolor{currentfill}{rgb}{0.267004,0.004874,0.329415}%
\pgfsetfillcolor{currentfill}%
\pgfsetfillopacity{0.700000}%
\pgfsetlinewidth{0.000000pt}%
\definecolor{currentstroke}{rgb}{0.000000,0.000000,0.000000}%
\pgfsetstrokecolor{currentstroke}%
\pgfsetstrokeopacity{0.700000}%
\pgfsetdash{}{0pt}%
\pgfpathmoveto{\pgfqpoint{7.884330in}{1.732476in}}%
\pgfpathcurveto{\pgfqpoint{7.889374in}{1.732476in}}{\pgfqpoint{7.894212in}{1.734480in}}{\pgfqpoint{7.897778in}{1.738046in}}%
\pgfpathcurveto{\pgfqpoint{7.901345in}{1.741613in}}{\pgfqpoint{7.903349in}{1.746451in}}{\pgfqpoint{7.903349in}{1.751494in}}%
\pgfpathcurveto{\pgfqpoint{7.903349in}{1.756538in}}{\pgfqpoint{7.901345in}{1.761376in}}{\pgfqpoint{7.897778in}{1.764942in}}%
\pgfpathcurveto{\pgfqpoint{7.894212in}{1.768509in}}{\pgfqpoint{7.889374in}{1.770512in}}{\pgfqpoint{7.884330in}{1.770512in}}%
\pgfpathcurveto{\pgfqpoint{7.879287in}{1.770512in}}{\pgfqpoint{7.874449in}{1.768509in}}{\pgfqpoint{7.870883in}{1.764942in}}%
\pgfpathcurveto{\pgfqpoint{7.867316in}{1.761376in}}{\pgfqpoint{7.865312in}{1.756538in}}{\pgfqpoint{7.865312in}{1.751494in}}%
\pgfpathcurveto{\pgfqpoint{7.865312in}{1.746451in}}{\pgfqpoint{7.867316in}{1.741613in}}{\pgfqpoint{7.870883in}{1.738046in}}%
\pgfpathcurveto{\pgfqpoint{7.874449in}{1.734480in}}{\pgfqpoint{7.879287in}{1.732476in}}{\pgfqpoint{7.884330in}{1.732476in}}%
\pgfpathclose%
\pgfusepath{fill}%
\end{pgfscope}%
\begin{pgfscope}%
\pgfpathrectangle{\pgfqpoint{6.572727in}{0.474100in}}{\pgfqpoint{4.227273in}{3.318700in}}%
\pgfusepath{clip}%
\pgfsetbuttcap%
\pgfsetroundjoin%
\definecolor{currentfill}{rgb}{0.993248,0.906157,0.143936}%
\pgfsetfillcolor{currentfill}%
\pgfsetfillopacity{0.700000}%
\pgfsetlinewidth{0.000000pt}%
\definecolor{currentstroke}{rgb}{0.000000,0.000000,0.000000}%
\pgfsetstrokecolor{currentstroke}%
\pgfsetstrokeopacity{0.700000}%
\pgfsetdash{}{0pt}%
\pgfpathmoveto{\pgfqpoint{7.794499in}{3.097098in}}%
\pgfpathcurveto{\pgfqpoint{7.799542in}{3.097098in}}{\pgfqpoint{7.804380in}{3.099102in}}{\pgfqpoint{7.807947in}{3.102668in}}%
\pgfpathcurveto{\pgfqpoint{7.811513in}{3.106235in}}{\pgfqpoint{7.813517in}{3.111073in}}{\pgfqpoint{7.813517in}{3.116116in}}%
\pgfpathcurveto{\pgfqpoint{7.813517in}{3.121160in}}{\pgfqpoint{7.811513in}{3.125998in}}{\pgfqpoint{7.807947in}{3.129564in}}%
\pgfpathcurveto{\pgfqpoint{7.804380in}{3.133131in}}{\pgfqpoint{7.799542in}{3.135134in}}{\pgfqpoint{7.794499in}{3.135134in}}%
\pgfpathcurveto{\pgfqpoint{7.789455in}{3.135134in}}{\pgfqpoint{7.784617in}{3.133131in}}{\pgfqpoint{7.781051in}{3.129564in}}%
\pgfpathcurveto{\pgfqpoint{7.777485in}{3.125998in}}{\pgfqpoint{7.775481in}{3.121160in}}{\pgfqpoint{7.775481in}{3.116116in}}%
\pgfpathcurveto{\pgfqpoint{7.775481in}{3.111073in}}{\pgfqpoint{7.777485in}{3.106235in}}{\pgfqpoint{7.781051in}{3.102668in}}%
\pgfpathcurveto{\pgfqpoint{7.784617in}{3.099102in}}{\pgfqpoint{7.789455in}{3.097098in}}{\pgfqpoint{7.794499in}{3.097098in}}%
\pgfpathclose%
\pgfusepath{fill}%
\end{pgfscope}%
\begin{pgfscope}%
\pgfpathrectangle{\pgfqpoint{6.572727in}{0.474100in}}{\pgfqpoint{4.227273in}{3.318700in}}%
\pgfusepath{clip}%
\pgfsetbuttcap%
\pgfsetroundjoin%
\definecolor{currentfill}{rgb}{0.127568,0.566949,0.550556}%
\pgfsetfillcolor{currentfill}%
\pgfsetfillopacity{0.700000}%
\pgfsetlinewidth{0.000000pt}%
\definecolor{currentstroke}{rgb}{0.000000,0.000000,0.000000}%
\pgfsetstrokecolor{currentstroke}%
\pgfsetstrokeopacity{0.700000}%
\pgfsetdash{}{0pt}%
\pgfpathmoveto{\pgfqpoint{9.464640in}{1.941879in}}%
\pgfpathcurveto{\pgfqpoint{9.469684in}{1.941879in}}{\pgfqpoint{9.474522in}{1.943883in}}{\pgfqpoint{9.478088in}{1.947449in}}%
\pgfpathcurveto{\pgfqpoint{9.481654in}{1.951016in}}{\pgfqpoint{9.483658in}{1.955854in}}{\pgfqpoint{9.483658in}{1.960897in}}%
\pgfpathcurveto{\pgfqpoint{9.483658in}{1.965941in}}{\pgfqpoint{9.481654in}{1.970779in}}{\pgfqpoint{9.478088in}{1.974345in}}%
\pgfpathcurveto{\pgfqpoint{9.474522in}{1.977912in}}{\pgfqpoint{9.469684in}{1.979915in}}{\pgfqpoint{9.464640in}{1.979915in}}%
\pgfpathcurveto{\pgfqpoint{9.459596in}{1.979915in}}{\pgfqpoint{9.454759in}{1.977912in}}{\pgfqpoint{9.451192in}{1.974345in}}%
\pgfpathcurveto{\pgfqpoint{9.447626in}{1.970779in}}{\pgfqpoint{9.445622in}{1.965941in}}{\pgfqpoint{9.445622in}{1.960897in}}%
\pgfpathcurveto{\pgfqpoint{9.445622in}{1.955854in}}{\pgfqpoint{9.447626in}{1.951016in}}{\pgfqpoint{9.451192in}{1.947449in}}%
\pgfpathcurveto{\pgfqpoint{9.454759in}{1.943883in}}{\pgfqpoint{9.459596in}{1.941879in}}{\pgfqpoint{9.464640in}{1.941879in}}%
\pgfpathclose%
\pgfusepath{fill}%
\end{pgfscope}%
\begin{pgfscope}%
\pgfpathrectangle{\pgfqpoint{6.572727in}{0.474100in}}{\pgfqpoint{4.227273in}{3.318700in}}%
\pgfusepath{clip}%
\pgfsetbuttcap%
\pgfsetroundjoin%
\definecolor{currentfill}{rgb}{0.267004,0.004874,0.329415}%
\pgfsetfillcolor{currentfill}%
\pgfsetfillopacity{0.700000}%
\pgfsetlinewidth{0.000000pt}%
\definecolor{currentstroke}{rgb}{0.000000,0.000000,0.000000}%
\pgfsetstrokecolor{currentstroke}%
\pgfsetstrokeopacity{0.700000}%
\pgfsetdash{}{0pt}%
\pgfpathmoveto{\pgfqpoint{8.272907in}{1.232988in}}%
\pgfpathcurveto{\pgfqpoint{8.277951in}{1.232988in}}{\pgfqpoint{8.282788in}{1.234992in}}{\pgfqpoint{8.286355in}{1.238559in}}%
\pgfpathcurveto{\pgfqpoint{8.289921in}{1.242125in}}{\pgfqpoint{8.291925in}{1.246963in}}{\pgfqpoint{8.291925in}{1.252007in}}%
\pgfpathcurveto{\pgfqpoint{8.291925in}{1.257050in}}{\pgfqpoint{8.289921in}{1.261888in}}{\pgfqpoint{8.286355in}{1.265454in}}%
\pgfpathcurveto{\pgfqpoint{8.282788in}{1.269021in}}{\pgfqpoint{8.277951in}{1.271025in}}{\pgfqpoint{8.272907in}{1.271025in}}%
\pgfpathcurveto{\pgfqpoint{8.267863in}{1.271025in}}{\pgfqpoint{8.263025in}{1.269021in}}{\pgfqpoint{8.259459in}{1.265454in}}%
\pgfpathcurveto{\pgfqpoint{8.255893in}{1.261888in}}{\pgfqpoint{8.253889in}{1.257050in}}{\pgfqpoint{8.253889in}{1.252007in}}%
\pgfpathcurveto{\pgfqpoint{8.253889in}{1.246963in}}{\pgfqpoint{8.255893in}{1.242125in}}{\pgfqpoint{8.259459in}{1.238559in}}%
\pgfpathcurveto{\pgfqpoint{8.263025in}{1.234992in}}{\pgfqpoint{8.267863in}{1.232988in}}{\pgfqpoint{8.272907in}{1.232988in}}%
\pgfpathclose%
\pgfusepath{fill}%
\end{pgfscope}%
\begin{pgfscope}%
\pgfpathrectangle{\pgfqpoint{6.572727in}{0.474100in}}{\pgfqpoint{4.227273in}{3.318700in}}%
\pgfusepath{clip}%
\pgfsetbuttcap%
\pgfsetroundjoin%
\definecolor{currentfill}{rgb}{0.267004,0.004874,0.329415}%
\pgfsetfillcolor{currentfill}%
\pgfsetfillopacity{0.700000}%
\pgfsetlinewidth{0.000000pt}%
\definecolor{currentstroke}{rgb}{0.000000,0.000000,0.000000}%
\pgfsetstrokecolor{currentstroke}%
\pgfsetstrokeopacity{0.700000}%
\pgfsetdash{}{0pt}%
\pgfpathmoveto{\pgfqpoint{8.250350in}{1.627376in}}%
\pgfpathcurveto{\pgfqpoint{8.255394in}{1.627376in}}{\pgfqpoint{8.260232in}{1.629379in}}{\pgfqpoint{8.263798in}{1.632946in}}%
\pgfpathcurveto{\pgfqpoint{8.267364in}{1.636512in}}{\pgfqpoint{8.269368in}{1.641350in}}{\pgfqpoint{8.269368in}{1.646394in}}%
\pgfpathcurveto{\pgfqpoint{8.269368in}{1.651437in}}{\pgfqpoint{8.267364in}{1.656275in}}{\pgfqpoint{8.263798in}{1.659842in}}%
\pgfpathcurveto{\pgfqpoint{8.260232in}{1.663408in}}{\pgfqpoint{8.255394in}{1.665412in}}{\pgfqpoint{8.250350in}{1.665412in}}%
\pgfpathcurveto{\pgfqpoint{8.245307in}{1.665412in}}{\pgfqpoint{8.240469in}{1.663408in}}{\pgfqpoint{8.236902in}{1.659842in}}%
\pgfpathcurveto{\pgfqpoint{8.233336in}{1.656275in}}{\pgfqpoint{8.231332in}{1.651437in}}{\pgfqpoint{8.231332in}{1.646394in}}%
\pgfpathcurveto{\pgfqpoint{8.231332in}{1.641350in}}{\pgfqpoint{8.233336in}{1.636512in}}{\pgfqpoint{8.236902in}{1.632946in}}%
\pgfpathcurveto{\pgfqpoint{8.240469in}{1.629379in}}{\pgfqpoint{8.245307in}{1.627376in}}{\pgfqpoint{8.250350in}{1.627376in}}%
\pgfpathclose%
\pgfusepath{fill}%
\end{pgfscope}%
\begin{pgfscope}%
\pgfpathrectangle{\pgfqpoint{6.572727in}{0.474100in}}{\pgfqpoint{4.227273in}{3.318700in}}%
\pgfusepath{clip}%
\pgfsetbuttcap%
\pgfsetroundjoin%
\definecolor{currentfill}{rgb}{0.127568,0.566949,0.550556}%
\pgfsetfillcolor{currentfill}%
\pgfsetfillopacity{0.700000}%
\pgfsetlinewidth{0.000000pt}%
\definecolor{currentstroke}{rgb}{0.000000,0.000000,0.000000}%
\pgfsetstrokecolor{currentstroke}%
\pgfsetstrokeopacity{0.700000}%
\pgfsetdash{}{0pt}%
\pgfpathmoveto{\pgfqpoint{9.945971in}{1.769533in}}%
\pgfpathcurveto{\pgfqpoint{9.951015in}{1.769533in}}{\pgfqpoint{9.955853in}{1.771537in}}{\pgfqpoint{9.959419in}{1.775103in}}%
\pgfpathcurveto{\pgfqpoint{9.962986in}{1.778669in}}{\pgfqpoint{9.964989in}{1.783507in}}{\pgfqpoint{9.964989in}{1.788551in}}%
\pgfpathcurveto{\pgfqpoint{9.964989in}{1.793595in}}{\pgfqpoint{9.962986in}{1.798432in}}{\pgfqpoint{9.959419in}{1.801999in}}%
\pgfpathcurveto{\pgfqpoint{9.955853in}{1.805565in}}{\pgfqpoint{9.951015in}{1.807569in}}{\pgfqpoint{9.945971in}{1.807569in}}%
\pgfpathcurveto{\pgfqpoint{9.940928in}{1.807569in}}{\pgfqpoint{9.936090in}{1.805565in}}{\pgfqpoint{9.932523in}{1.801999in}}%
\pgfpathcurveto{\pgfqpoint{9.928957in}{1.798432in}}{\pgfqpoint{9.926953in}{1.793595in}}{\pgfqpoint{9.926953in}{1.788551in}}%
\pgfpathcurveto{\pgfqpoint{9.926953in}{1.783507in}}{\pgfqpoint{9.928957in}{1.778669in}}{\pgfqpoint{9.932523in}{1.775103in}}%
\pgfpathcurveto{\pgfqpoint{9.936090in}{1.771537in}}{\pgfqpoint{9.940928in}{1.769533in}}{\pgfqpoint{9.945971in}{1.769533in}}%
\pgfpathclose%
\pgfusepath{fill}%
\end{pgfscope}%
\begin{pgfscope}%
\pgfpathrectangle{\pgfqpoint{6.572727in}{0.474100in}}{\pgfqpoint{4.227273in}{3.318700in}}%
\pgfusepath{clip}%
\pgfsetbuttcap%
\pgfsetroundjoin%
\definecolor{currentfill}{rgb}{0.267004,0.004874,0.329415}%
\pgfsetfillcolor{currentfill}%
\pgfsetfillopacity{0.700000}%
\pgfsetlinewidth{0.000000pt}%
\definecolor{currentstroke}{rgb}{0.000000,0.000000,0.000000}%
\pgfsetstrokecolor{currentstroke}%
\pgfsetstrokeopacity{0.700000}%
\pgfsetdash{}{0pt}%
\pgfpathmoveto{\pgfqpoint{7.906949in}{1.874768in}}%
\pgfpathcurveto{\pgfqpoint{7.911993in}{1.874768in}}{\pgfqpoint{7.916831in}{1.876772in}}{\pgfqpoint{7.920397in}{1.880338in}}%
\pgfpathcurveto{\pgfqpoint{7.923963in}{1.883905in}}{\pgfqpoint{7.925967in}{1.888742in}}{\pgfqpoint{7.925967in}{1.893786in}}%
\pgfpathcurveto{\pgfqpoint{7.925967in}{1.898830in}}{\pgfqpoint{7.923963in}{1.903667in}}{\pgfqpoint{7.920397in}{1.907234in}}%
\pgfpathcurveto{\pgfqpoint{7.916831in}{1.910800in}}{\pgfqpoint{7.911993in}{1.912804in}}{\pgfqpoint{7.906949in}{1.912804in}}%
\pgfpathcurveto{\pgfqpoint{7.901905in}{1.912804in}}{\pgfqpoint{7.897068in}{1.910800in}}{\pgfqpoint{7.893501in}{1.907234in}}%
\pgfpathcurveto{\pgfqpoint{7.889935in}{1.903667in}}{\pgfqpoint{7.887931in}{1.898830in}}{\pgfqpoint{7.887931in}{1.893786in}}%
\pgfpathcurveto{\pgfqpoint{7.887931in}{1.888742in}}{\pgfqpoint{7.889935in}{1.883905in}}{\pgfqpoint{7.893501in}{1.880338in}}%
\pgfpathcurveto{\pgfqpoint{7.897068in}{1.876772in}}{\pgfqpoint{7.901905in}{1.874768in}}{\pgfqpoint{7.906949in}{1.874768in}}%
\pgfpathclose%
\pgfusepath{fill}%
\end{pgfscope}%
\begin{pgfscope}%
\pgfpathrectangle{\pgfqpoint{6.572727in}{0.474100in}}{\pgfqpoint{4.227273in}{3.318700in}}%
\pgfusepath{clip}%
\pgfsetbuttcap%
\pgfsetroundjoin%
\definecolor{currentfill}{rgb}{0.127568,0.566949,0.550556}%
\pgfsetfillcolor{currentfill}%
\pgfsetfillopacity{0.700000}%
\pgfsetlinewidth{0.000000pt}%
\definecolor{currentstroke}{rgb}{0.000000,0.000000,0.000000}%
\pgfsetstrokecolor{currentstroke}%
\pgfsetstrokeopacity{0.700000}%
\pgfsetdash{}{0pt}%
\pgfpathmoveto{\pgfqpoint{9.251025in}{1.584946in}}%
\pgfpathcurveto{\pgfqpoint{9.256069in}{1.584946in}}{\pgfqpoint{9.260907in}{1.586950in}}{\pgfqpoint{9.264473in}{1.590516in}}%
\pgfpathcurveto{\pgfqpoint{9.268040in}{1.594083in}}{\pgfqpoint{9.270044in}{1.598921in}}{\pgfqpoint{9.270044in}{1.603964in}}%
\pgfpathcurveto{\pgfqpoint{9.270044in}{1.609008in}}{\pgfqpoint{9.268040in}{1.613846in}}{\pgfqpoint{9.264473in}{1.617412in}}%
\pgfpathcurveto{\pgfqpoint{9.260907in}{1.620979in}}{\pgfqpoint{9.256069in}{1.622982in}}{\pgfqpoint{9.251025in}{1.622982in}}%
\pgfpathcurveto{\pgfqpoint{9.245982in}{1.622982in}}{\pgfqpoint{9.241144in}{1.620979in}}{\pgfqpoint{9.237578in}{1.617412in}}%
\pgfpathcurveto{\pgfqpoint{9.234011in}{1.613846in}}{\pgfqpoint{9.232007in}{1.609008in}}{\pgfqpoint{9.232007in}{1.603964in}}%
\pgfpathcurveto{\pgfqpoint{9.232007in}{1.598921in}}{\pgfqpoint{9.234011in}{1.594083in}}{\pgfqpoint{9.237578in}{1.590516in}}%
\pgfpathcurveto{\pgfqpoint{9.241144in}{1.586950in}}{\pgfqpoint{9.245982in}{1.584946in}}{\pgfqpoint{9.251025in}{1.584946in}}%
\pgfpathclose%
\pgfusepath{fill}%
\end{pgfscope}%
\begin{pgfscope}%
\pgfpathrectangle{\pgfqpoint{6.572727in}{0.474100in}}{\pgfqpoint{4.227273in}{3.318700in}}%
\pgfusepath{clip}%
\pgfsetbuttcap%
\pgfsetroundjoin%
\definecolor{currentfill}{rgb}{0.267004,0.004874,0.329415}%
\pgfsetfillcolor{currentfill}%
\pgfsetfillopacity{0.700000}%
\pgfsetlinewidth{0.000000pt}%
\definecolor{currentstroke}{rgb}{0.000000,0.000000,0.000000}%
\pgfsetstrokecolor{currentstroke}%
\pgfsetstrokeopacity{0.700000}%
\pgfsetdash{}{0pt}%
\pgfpathmoveto{\pgfqpoint{7.804424in}{1.847281in}}%
\pgfpathcurveto{\pgfqpoint{7.809468in}{1.847281in}}{\pgfqpoint{7.814306in}{1.849284in}}{\pgfqpoint{7.817872in}{1.852851in}}%
\pgfpathcurveto{\pgfqpoint{7.821439in}{1.856417in}}{\pgfqpoint{7.823442in}{1.861255in}}{\pgfqpoint{7.823442in}{1.866299in}}%
\pgfpathcurveto{\pgfqpoint{7.823442in}{1.871342in}}{\pgfqpoint{7.821439in}{1.876180in}}{\pgfqpoint{7.817872in}{1.879747in}}%
\pgfpathcurveto{\pgfqpoint{7.814306in}{1.883313in}}{\pgfqpoint{7.809468in}{1.885317in}}{\pgfqpoint{7.804424in}{1.885317in}}%
\pgfpathcurveto{\pgfqpoint{7.799381in}{1.885317in}}{\pgfqpoint{7.794543in}{1.883313in}}{\pgfqpoint{7.790976in}{1.879747in}}%
\pgfpathcurveto{\pgfqpoint{7.787410in}{1.876180in}}{\pgfqpoint{7.785406in}{1.871342in}}{\pgfqpoint{7.785406in}{1.866299in}}%
\pgfpathcurveto{\pgfqpoint{7.785406in}{1.861255in}}{\pgfqpoint{7.787410in}{1.856417in}}{\pgfqpoint{7.790976in}{1.852851in}}%
\pgfpathcurveto{\pgfqpoint{7.794543in}{1.849284in}}{\pgfqpoint{7.799381in}{1.847281in}}{\pgfqpoint{7.804424in}{1.847281in}}%
\pgfpathclose%
\pgfusepath{fill}%
\end{pgfscope}%
\begin{pgfscope}%
\pgfpathrectangle{\pgfqpoint{6.572727in}{0.474100in}}{\pgfqpoint{4.227273in}{3.318700in}}%
\pgfusepath{clip}%
\pgfsetbuttcap%
\pgfsetroundjoin%
\definecolor{currentfill}{rgb}{0.993248,0.906157,0.143936}%
\pgfsetfillcolor{currentfill}%
\pgfsetfillopacity{0.700000}%
\pgfsetlinewidth{0.000000pt}%
\definecolor{currentstroke}{rgb}{0.000000,0.000000,0.000000}%
\pgfsetstrokecolor{currentstroke}%
\pgfsetstrokeopacity{0.700000}%
\pgfsetdash{}{0pt}%
\pgfpathmoveto{\pgfqpoint{8.593939in}{2.763200in}}%
\pgfpathcurveto{\pgfqpoint{8.598982in}{2.763200in}}{\pgfqpoint{8.603820in}{2.765204in}}{\pgfqpoint{8.607386in}{2.768770in}}%
\pgfpathcurveto{\pgfqpoint{8.610953in}{2.772336in}}{\pgfqpoint{8.612957in}{2.777174in}}{\pgfqpoint{8.612957in}{2.782218in}}%
\pgfpathcurveto{\pgfqpoint{8.612957in}{2.787261in}}{\pgfqpoint{8.610953in}{2.792099in}}{\pgfqpoint{8.607386in}{2.795666in}}%
\pgfpathcurveto{\pgfqpoint{8.603820in}{2.799232in}}{\pgfqpoint{8.598982in}{2.801236in}}{\pgfqpoint{8.593939in}{2.801236in}}%
\pgfpathcurveto{\pgfqpoint{8.588895in}{2.801236in}}{\pgfqpoint{8.584057in}{2.799232in}}{\pgfqpoint{8.580491in}{2.795666in}}%
\pgfpathcurveto{\pgfqpoint{8.576924in}{2.792099in}}{\pgfqpoint{8.574920in}{2.787261in}}{\pgfqpoint{8.574920in}{2.782218in}}%
\pgfpathcurveto{\pgfqpoint{8.574920in}{2.777174in}}{\pgfqpoint{8.576924in}{2.772336in}}{\pgfqpoint{8.580491in}{2.768770in}}%
\pgfpathcurveto{\pgfqpoint{8.584057in}{2.765204in}}{\pgfqpoint{8.588895in}{2.763200in}}{\pgfqpoint{8.593939in}{2.763200in}}%
\pgfpathclose%
\pgfusepath{fill}%
\end{pgfscope}%
\begin{pgfscope}%
\pgfpathrectangle{\pgfqpoint{6.572727in}{0.474100in}}{\pgfqpoint{4.227273in}{3.318700in}}%
\pgfusepath{clip}%
\pgfsetbuttcap%
\pgfsetroundjoin%
\definecolor{currentfill}{rgb}{0.127568,0.566949,0.550556}%
\pgfsetfillcolor{currentfill}%
\pgfsetfillopacity{0.700000}%
\pgfsetlinewidth{0.000000pt}%
\definecolor{currentstroke}{rgb}{0.000000,0.000000,0.000000}%
\pgfsetstrokecolor{currentstroke}%
\pgfsetstrokeopacity{0.700000}%
\pgfsetdash{}{0pt}%
\pgfpathmoveto{\pgfqpoint{9.787090in}{1.591905in}}%
\pgfpathcurveto{\pgfqpoint{9.792134in}{1.591905in}}{\pgfqpoint{9.796971in}{1.593909in}}{\pgfqpoint{9.800538in}{1.597475in}}%
\pgfpathcurveto{\pgfqpoint{9.804104in}{1.601042in}}{\pgfqpoint{9.806108in}{1.605879in}}{\pgfqpoint{9.806108in}{1.610923in}}%
\pgfpathcurveto{\pgfqpoint{9.806108in}{1.615967in}}{\pgfqpoint{9.804104in}{1.620805in}}{\pgfqpoint{9.800538in}{1.624371in}}%
\pgfpathcurveto{\pgfqpoint{9.796971in}{1.627937in}}{\pgfqpoint{9.792134in}{1.629941in}}{\pgfqpoint{9.787090in}{1.629941in}}%
\pgfpathcurveto{\pgfqpoint{9.782046in}{1.629941in}}{\pgfqpoint{9.777208in}{1.627937in}}{\pgfqpoint{9.773642in}{1.624371in}}%
\pgfpathcurveto{\pgfqpoint{9.770076in}{1.620805in}}{\pgfqpoint{9.768072in}{1.615967in}}{\pgfqpoint{9.768072in}{1.610923in}}%
\pgfpathcurveto{\pgfqpoint{9.768072in}{1.605879in}}{\pgfqpoint{9.770076in}{1.601042in}}{\pgfqpoint{9.773642in}{1.597475in}}%
\pgfpathcurveto{\pgfqpoint{9.777208in}{1.593909in}}{\pgfqpoint{9.782046in}{1.591905in}}{\pgfqpoint{9.787090in}{1.591905in}}%
\pgfpathclose%
\pgfusepath{fill}%
\end{pgfscope}%
\begin{pgfscope}%
\pgfpathrectangle{\pgfqpoint{6.572727in}{0.474100in}}{\pgfqpoint{4.227273in}{3.318700in}}%
\pgfusepath{clip}%
\pgfsetbuttcap%
\pgfsetroundjoin%
\definecolor{currentfill}{rgb}{0.993248,0.906157,0.143936}%
\pgfsetfillcolor{currentfill}%
\pgfsetfillopacity{0.700000}%
\pgfsetlinewidth{0.000000pt}%
\definecolor{currentstroke}{rgb}{0.000000,0.000000,0.000000}%
\pgfsetstrokecolor{currentstroke}%
\pgfsetstrokeopacity{0.700000}%
\pgfsetdash{}{0pt}%
\pgfpathmoveto{\pgfqpoint{8.258408in}{3.071458in}}%
\pgfpathcurveto{\pgfqpoint{8.263452in}{3.071458in}}{\pgfqpoint{8.268290in}{3.073462in}}{\pgfqpoint{8.271856in}{3.077029in}}%
\pgfpathcurveto{\pgfqpoint{8.275422in}{3.080595in}}{\pgfqpoint{8.277426in}{3.085433in}}{\pgfqpoint{8.277426in}{3.090477in}}%
\pgfpathcurveto{\pgfqpoint{8.277426in}{3.095520in}}{\pgfqpoint{8.275422in}{3.100358in}}{\pgfqpoint{8.271856in}{3.103924in}}%
\pgfpathcurveto{\pgfqpoint{8.268290in}{3.107491in}}{\pgfqpoint{8.263452in}{3.109495in}}{\pgfqpoint{8.258408in}{3.109495in}}%
\pgfpathcurveto{\pgfqpoint{8.253364in}{3.109495in}}{\pgfqpoint{8.248527in}{3.107491in}}{\pgfqpoint{8.244960in}{3.103924in}}%
\pgfpathcurveto{\pgfqpoint{8.241394in}{3.100358in}}{\pgfqpoint{8.239390in}{3.095520in}}{\pgfqpoint{8.239390in}{3.090477in}}%
\pgfpathcurveto{\pgfqpoint{8.239390in}{3.085433in}}{\pgfqpoint{8.241394in}{3.080595in}}{\pgfqpoint{8.244960in}{3.077029in}}%
\pgfpathcurveto{\pgfqpoint{8.248527in}{3.073462in}}{\pgfqpoint{8.253364in}{3.071458in}}{\pgfqpoint{8.258408in}{3.071458in}}%
\pgfpathclose%
\pgfusepath{fill}%
\end{pgfscope}%
\begin{pgfscope}%
\pgfpathrectangle{\pgfqpoint{6.572727in}{0.474100in}}{\pgfqpoint{4.227273in}{3.318700in}}%
\pgfusepath{clip}%
\pgfsetbuttcap%
\pgfsetroundjoin%
\definecolor{currentfill}{rgb}{0.267004,0.004874,0.329415}%
\pgfsetfillcolor{currentfill}%
\pgfsetfillopacity{0.700000}%
\pgfsetlinewidth{0.000000pt}%
\definecolor{currentstroke}{rgb}{0.000000,0.000000,0.000000}%
\pgfsetstrokecolor{currentstroke}%
\pgfsetstrokeopacity{0.700000}%
\pgfsetdash{}{0pt}%
\pgfpathmoveto{\pgfqpoint{8.370768in}{1.469745in}}%
\pgfpathcurveto{\pgfqpoint{8.375811in}{1.469745in}}{\pgfqpoint{8.380649in}{1.471749in}}{\pgfqpoint{8.384215in}{1.475316in}}%
\pgfpathcurveto{\pgfqpoint{8.387782in}{1.478882in}}{\pgfqpoint{8.389786in}{1.483720in}}{\pgfqpoint{8.389786in}{1.488764in}}%
\pgfpathcurveto{\pgfqpoint{8.389786in}{1.493807in}}{\pgfqpoint{8.387782in}{1.498645in}}{\pgfqpoint{8.384215in}{1.502211in}}%
\pgfpathcurveto{\pgfqpoint{8.380649in}{1.505778in}}{\pgfqpoint{8.375811in}{1.507782in}}{\pgfqpoint{8.370768in}{1.507782in}}%
\pgfpathcurveto{\pgfqpoint{8.365724in}{1.507782in}}{\pgfqpoint{8.360886in}{1.505778in}}{\pgfqpoint{8.357320in}{1.502211in}}%
\pgfpathcurveto{\pgfqpoint{8.353753in}{1.498645in}}{\pgfqpoint{8.351749in}{1.493807in}}{\pgfqpoint{8.351749in}{1.488764in}}%
\pgfpathcurveto{\pgfqpoint{8.351749in}{1.483720in}}{\pgfqpoint{8.353753in}{1.478882in}}{\pgfqpoint{8.357320in}{1.475316in}}%
\pgfpathcurveto{\pgfqpoint{8.360886in}{1.471749in}}{\pgfqpoint{8.365724in}{1.469745in}}{\pgfqpoint{8.370768in}{1.469745in}}%
\pgfpathclose%
\pgfusepath{fill}%
\end{pgfscope}%
\begin{pgfscope}%
\pgfpathrectangle{\pgfqpoint{6.572727in}{0.474100in}}{\pgfqpoint{4.227273in}{3.318700in}}%
\pgfusepath{clip}%
\pgfsetbuttcap%
\pgfsetroundjoin%
\definecolor{currentfill}{rgb}{0.267004,0.004874,0.329415}%
\pgfsetfillcolor{currentfill}%
\pgfsetfillopacity{0.700000}%
\pgfsetlinewidth{0.000000pt}%
\definecolor{currentstroke}{rgb}{0.000000,0.000000,0.000000}%
\pgfsetstrokecolor{currentstroke}%
\pgfsetstrokeopacity{0.700000}%
\pgfsetdash{}{0pt}%
\pgfpathmoveto{\pgfqpoint{7.401273in}{1.192665in}}%
\pgfpathcurveto{\pgfqpoint{7.406317in}{1.192665in}}{\pgfqpoint{7.411155in}{1.194669in}}{\pgfqpoint{7.414721in}{1.198235in}}%
\pgfpathcurveto{\pgfqpoint{7.418288in}{1.201801in}}{\pgfqpoint{7.420291in}{1.206639in}}{\pgfqpoint{7.420291in}{1.211683in}}%
\pgfpathcurveto{\pgfqpoint{7.420291in}{1.216727in}}{\pgfqpoint{7.418288in}{1.221564in}}{\pgfqpoint{7.414721in}{1.225131in}}%
\pgfpathcurveto{\pgfqpoint{7.411155in}{1.228697in}}{\pgfqpoint{7.406317in}{1.230701in}}{\pgfqpoint{7.401273in}{1.230701in}}%
\pgfpathcurveto{\pgfqpoint{7.396230in}{1.230701in}}{\pgfqpoint{7.391392in}{1.228697in}}{\pgfqpoint{7.387825in}{1.225131in}}%
\pgfpathcurveto{\pgfqpoint{7.384259in}{1.221564in}}{\pgfqpoint{7.382255in}{1.216727in}}{\pgfqpoint{7.382255in}{1.211683in}}%
\pgfpathcurveto{\pgfqpoint{7.382255in}{1.206639in}}{\pgfqpoint{7.384259in}{1.201801in}}{\pgfqpoint{7.387825in}{1.198235in}}%
\pgfpathcurveto{\pgfqpoint{7.391392in}{1.194669in}}{\pgfqpoint{7.396230in}{1.192665in}}{\pgfqpoint{7.401273in}{1.192665in}}%
\pgfpathclose%
\pgfusepath{fill}%
\end{pgfscope}%
\begin{pgfscope}%
\pgfpathrectangle{\pgfqpoint{6.572727in}{0.474100in}}{\pgfqpoint{4.227273in}{3.318700in}}%
\pgfusepath{clip}%
\pgfsetbuttcap%
\pgfsetroundjoin%
\definecolor{currentfill}{rgb}{0.993248,0.906157,0.143936}%
\pgfsetfillcolor{currentfill}%
\pgfsetfillopacity{0.700000}%
\pgfsetlinewidth{0.000000pt}%
\definecolor{currentstroke}{rgb}{0.000000,0.000000,0.000000}%
\pgfsetstrokecolor{currentstroke}%
\pgfsetstrokeopacity{0.700000}%
\pgfsetdash{}{0pt}%
\pgfpathmoveto{\pgfqpoint{8.392658in}{2.584315in}}%
\pgfpathcurveto{\pgfqpoint{8.397702in}{2.584315in}}{\pgfqpoint{8.402539in}{2.586319in}}{\pgfqpoint{8.406106in}{2.589885in}}%
\pgfpathcurveto{\pgfqpoint{8.409672in}{2.593452in}}{\pgfqpoint{8.411676in}{2.598289in}}{\pgfqpoint{8.411676in}{2.603333in}}%
\pgfpathcurveto{\pgfqpoint{8.411676in}{2.608377in}}{\pgfqpoint{8.409672in}{2.613215in}}{\pgfqpoint{8.406106in}{2.616781in}}%
\pgfpathcurveto{\pgfqpoint{8.402539in}{2.620347in}}{\pgfqpoint{8.397702in}{2.622351in}}{\pgfqpoint{8.392658in}{2.622351in}}%
\pgfpathcurveto{\pgfqpoint{8.387614in}{2.622351in}}{\pgfqpoint{8.382777in}{2.620347in}}{\pgfqpoint{8.379210in}{2.616781in}}%
\pgfpathcurveto{\pgfqpoint{8.375644in}{2.613215in}}{\pgfqpoint{8.373640in}{2.608377in}}{\pgfqpoint{8.373640in}{2.603333in}}%
\pgfpathcurveto{\pgfqpoint{8.373640in}{2.598289in}}{\pgfqpoint{8.375644in}{2.593452in}}{\pgfqpoint{8.379210in}{2.589885in}}%
\pgfpathcurveto{\pgfqpoint{8.382777in}{2.586319in}}{\pgfqpoint{8.387614in}{2.584315in}}{\pgfqpoint{8.392658in}{2.584315in}}%
\pgfpathclose%
\pgfusepath{fill}%
\end{pgfscope}%
\begin{pgfscope}%
\pgfpathrectangle{\pgfqpoint{6.572727in}{0.474100in}}{\pgfqpoint{4.227273in}{3.318700in}}%
\pgfusepath{clip}%
\pgfsetbuttcap%
\pgfsetroundjoin%
\definecolor{currentfill}{rgb}{0.993248,0.906157,0.143936}%
\pgfsetfillcolor{currentfill}%
\pgfsetfillopacity{0.700000}%
\pgfsetlinewidth{0.000000pt}%
\definecolor{currentstroke}{rgb}{0.000000,0.000000,0.000000}%
\pgfsetstrokecolor{currentstroke}%
\pgfsetstrokeopacity{0.700000}%
\pgfsetdash{}{0pt}%
\pgfpathmoveto{\pgfqpoint{8.593693in}{2.843314in}}%
\pgfpathcurveto{\pgfqpoint{8.598737in}{2.843314in}}{\pgfqpoint{8.603575in}{2.845318in}}{\pgfqpoint{8.607141in}{2.848884in}}%
\pgfpathcurveto{\pgfqpoint{8.610707in}{2.852450in}}{\pgfqpoint{8.612711in}{2.857288in}}{\pgfqpoint{8.612711in}{2.862332in}}%
\pgfpathcurveto{\pgfqpoint{8.612711in}{2.867376in}}{\pgfqpoint{8.610707in}{2.872213in}}{\pgfqpoint{8.607141in}{2.875780in}}%
\pgfpathcurveto{\pgfqpoint{8.603575in}{2.879346in}}{\pgfqpoint{8.598737in}{2.881350in}}{\pgfqpoint{8.593693in}{2.881350in}}%
\pgfpathcurveto{\pgfqpoint{8.588650in}{2.881350in}}{\pgfqpoint{8.583812in}{2.879346in}}{\pgfqpoint{8.580245in}{2.875780in}}%
\pgfpathcurveto{\pgfqpoint{8.576679in}{2.872213in}}{\pgfqpoint{8.574675in}{2.867376in}}{\pgfqpoint{8.574675in}{2.862332in}}%
\pgfpathcurveto{\pgfqpoint{8.574675in}{2.857288in}}{\pgfqpoint{8.576679in}{2.852450in}}{\pgfqpoint{8.580245in}{2.848884in}}%
\pgfpathcurveto{\pgfqpoint{8.583812in}{2.845318in}}{\pgfqpoint{8.588650in}{2.843314in}}{\pgfqpoint{8.593693in}{2.843314in}}%
\pgfpathclose%
\pgfusepath{fill}%
\end{pgfscope}%
\begin{pgfscope}%
\pgfpathrectangle{\pgfqpoint{6.572727in}{0.474100in}}{\pgfqpoint{4.227273in}{3.318700in}}%
\pgfusepath{clip}%
\pgfsetbuttcap%
\pgfsetroundjoin%
\definecolor{currentfill}{rgb}{0.127568,0.566949,0.550556}%
\pgfsetfillcolor{currentfill}%
\pgfsetfillopacity{0.700000}%
\pgfsetlinewidth{0.000000pt}%
\definecolor{currentstroke}{rgb}{0.000000,0.000000,0.000000}%
\pgfsetstrokecolor{currentstroke}%
\pgfsetstrokeopacity{0.700000}%
\pgfsetdash{}{0pt}%
\pgfpathmoveto{\pgfqpoint{8.962311in}{1.199849in}}%
\pgfpathcurveto{\pgfqpoint{8.967355in}{1.199849in}}{\pgfqpoint{8.972192in}{1.201852in}}{\pgfqpoint{8.975759in}{1.205419in}}%
\pgfpathcurveto{\pgfqpoint{8.979325in}{1.208985in}}{\pgfqpoint{8.981329in}{1.213823in}}{\pgfqpoint{8.981329in}{1.218867in}}%
\pgfpathcurveto{\pgfqpoint{8.981329in}{1.223910in}}{\pgfqpoint{8.979325in}{1.228748in}}{\pgfqpoint{8.975759in}{1.232315in}}%
\pgfpathcurveto{\pgfqpoint{8.972192in}{1.235881in}}{\pgfqpoint{8.967355in}{1.237885in}}{\pgfqpoint{8.962311in}{1.237885in}}%
\pgfpathcurveto{\pgfqpoint{8.957267in}{1.237885in}}{\pgfqpoint{8.952429in}{1.235881in}}{\pgfqpoint{8.948863in}{1.232315in}}%
\pgfpathcurveto{\pgfqpoint{8.945297in}{1.228748in}}{\pgfqpoint{8.943293in}{1.223910in}}{\pgfqpoint{8.943293in}{1.218867in}}%
\pgfpathcurveto{\pgfqpoint{8.943293in}{1.213823in}}{\pgfqpoint{8.945297in}{1.208985in}}{\pgfqpoint{8.948863in}{1.205419in}}%
\pgfpathcurveto{\pgfqpoint{8.952429in}{1.201852in}}{\pgfqpoint{8.957267in}{1.199849in}}{\pgfqpoint{8.962311in}{1.199849in}}%
\pgfpathclose%
\pgfusepath{fill}%
\end{pgfscope}%
\begin{pgfscope}%
\pgfpathrectangle{\pgfqpoint{6.572727in}{0.474100in}}{\pgfqpoint{4.227273in}{3.318700in}}%
\pgfusepath{clip}%
\pgfsetbuttcap%
\pgfsetroundjoin%
\definecolor{currentfill}{rgb}{0.993248,0.906157,0.143936}%
\pgfsetfillcolor{currentfill}%
\pgfsetfillopacity{0.700000}%
\pgfsetlinewidth{0.000000pt}%
\definecolor{currentstroke}{rgb}{0.000000,0.000000,0.000000}%
\pgfsetstrokecolor{currentstroke}%
\pgfsetstrokeopacity{0.700000}%
\pgfsetdash{}{0pt}%
\pgfpathmoveto{\pgfqpoint{8.247980in}{2.990677in}}%
\pgfpathcurveto{\pgfqpoint{8.253024in}{2.990677in}}{\pgfqpoint{8.257862in}{2.992681in}}{\pgfqpoint{8.261428in}{2.996247in}}%
\pgfpathcurveto{\pgfqpoint{8.264994in}{2.999814in}}{\pgfqpoint{8.266998in}{3.004652in}}{\pgfqpoint{8.266998in}{3.009695in}}%
\pgfpathcurveto{\pgfqpoint{8.266998in}{3.014739in}}{\pgfqpoint{8.264994in}{3.019577in}}{\pgfqpoint{8.261428in}{3.023143in}}%
\pgfpathcurveto{\pgfqpoint{8.257862in}{3.026710in}}{\pgfqpoint{8.253024in}{3.028713in}}{\pgfqpoint{8.247980in}{3.028713in}}%
\pgfpathcurveto{\pgfqpoint{8.242936in}{3.028713in}}{\pgfqpoint{8.238099in}{3.026710in}}{\pgfqpoint{8.234532in}{3.023143in}}%
\pgfpathcurveto{\pgfqpoint{8.230966in}{3.019577in}}{\pgfqpoint{8.228962in}{3.014739in}}{\pgfqpoint{8.228962in}{3.009695in}}%
\pgfpathcurveto{\pgfqpoint{8.228962in}{3.004652in}}{\pgfqpoint{8.230966in}{2.999814in}}{\pgfqpoint{8.234532in}{2.996247in}}%
\pgfpathcurveto{\pgfqpoint{8.238099in}{2.992681in}}{\pgfqpoint{8.242936in}{2.990677in}}{\pgfqpoint{8.247980in}{2.990677in}}%
\pgfpathclose%
\pgfusepath{fill}%
\end{pgfscope}%
\begin{pgfscope}%
\pgfpathrectangle{\pgfqpoint{6.572727in}{0.474100in}}{\pgfqpoint{4.227273in}{3.318700in}}%
\pgfusepath{clip}%
\pgfsetbuttcap%
\pgfsetroundjoin%
\definecolor{currentfill}{rgb}{0.993248,0.906157,0.143936}%
\pgfsetfillcolor{currentfill}%
\pgfsetfillopacity{0.700000}%
\pgfsetlinewidth{0.000000pt}%
\definecolor{currentstroke}{rgb}{0.000000,0.000000,0.000000}%
\pgfsetstrokecolor{currentstroke}%
\pgfsetstrokeopacity{0.700000}%
\pgfsetdash{}{0pt}%
\pgfpathmoveto{\pgfqpoint{8.512728in}{2.857207in}}%
\pgfpathcurveto{\pgfqpoint{8.517772in}{2.857207in}}{\pgfqpoint{8.522610in}{2.859211in}}{\pgfqpoint{8.526176in}{2.862777in}}%
\pgfpathcurveto{\pgfqpoint{8.529743in}{2.866344in}}{\pgfqpoint{8.531747in}{2.871182in}}{\pgfqpoint{8.531747in}{2.876225in}}%
\pgfpathcurveto{\pgfqpoint{8.531747in}{2.881269in}}{\pgfqpoint{8.529743in}{2.886107in}}{\pgfqpoint{8.526176in}{2.889673in}}%
\pgfpathcurveto{\pgfqpoint{8.522610in}{2.893240in}}{\pgfqpoint{8.517772in}{2.895243in}}{\pgfqpoint{8.512728in}{2.895243in}}%
\pgfpathcurveto{\pgfqpoint{8.507685in}{2.895243in}}{\pgfqpoint{8.502847in}{2.893240in}}{\pgfqpoint{8.499281in}{2.889673in}}%
\pgfpathcurveto{\pgfqpoint{8.495714in}{2.886107in}}{\pgfqpoint{8.493710in}{2.881269in}}{\pgfqpoint{8.493710in}{2.876225in}}%
\pgfpathcurveto{\pgfqpoint{8.493710in}{2.871182in}}{\pgfqpoint{8.495714in}{2.866344in}}{\pgfqpoint{8.499281in}{2.862777in}}%
\pgfpathcurveto{\pgfqpoint{8.502847in}{2.859211in}}{\pgfqpoint{8.507685in}{2.857207in}}{\pgfqpoint{8.512728in}{2.857207in}}%
\pgfpathclose%
\pgfusepath{fill}%
\end{pgfscope}%
\begin{pgfscope}%
\pgfpathrectangle{\pgfqpoint{6.572727in}{0.474100in}}{\pgfqpoint{4.227273in}{3.318700in}}%
\pgfusepath{clip}%
\pgfsetbuttcap%
\pgfsetroundjoin%
\definecolor{currentfill}{rgb}{0.993248,0.906157,0.143936}%
\pgfsetfillcolor{currentfill}%
\pgfsetfillopacity{0.700000}%
\pgfsetlinewidth{0.000000pt}%
\definecolor{currentstroke}{rgb}{0.000000,0.000000,0.000000}%
\pgfsetstrokecolor{currentstroke}%
\pgfsetstrokeopacity{0.700000}%
\pgfsetdash{}{0pt}%
\pgfpathmoveto{\pgfqpoint{7.567537in}{2.653674in}}%
\pgfpathcurveto{\pgfqpoint{7.572581in}{2.653674in}}{\pgfqpoint{7.577418in}{2.655678in}}{\pgfqpoint{7.580985in}{2.659244in}}%
\pgfpathcurveto{\pgfqpoint{7.584551in}{2.662811in}}{\pgfqpoint{7.586555in}{2.667649in}}{\pgfqpoint{7.586555in}{2.672692in}}%
\pgfpathcurveto{\pgfqpoint{7.586555in}{2.677736in}}{\pgfqpoint{7.584551in}{2.682574in}}{\pgfqpoint{7.580985in}{2.686140in}}%
\pgfpathcurveto{\pgfqpoint{7.577418in}{2.689706in}}{\pgfqpoint{7.572581in}{2.691710in}}{\pgfqpoint{7.567537in}{2.691710in}}%
\pgfpathcurveto{\pgfqpoint{7.562493in}{2.691710in}}{\pgfqpoint{7.557655in}{2.689706in}}{\pgfqpoint{7.554089in}{2.686140in}}%
\pgfpathcurveto{\pgfqpoint{7.550523in}{2.682574in}}{\pgfqpoint{7.548519in}{2.677736in}}{\pgfqpoint{7.548519in}{2.672692in}}%
\pgfpathcurveto{\pgfqpoint{7.548519in}{2.667649in}}{\pgfqpoint{7.550523in}{2.662811in}}{\pgfqpoint{7.554089in}{2.659244in}}%
\pgfpathcurveto{\pgfqpoint{7.557655in}{2.655678in}}{\pgfqpoint{7.562493in}{2.653674in}}{\pgfqpoint{7.567537in}{2.653674in}}%
\pgfpathclose%
\pgfusepath{fill}%
\end{pgfscope}%
\begin{pgfscope}%
\pgfpathrectangle{\pgfqpoint{6.572727in}{0.474100in}}{\pgfqpoint{4.227273in}{3.318700in}}%
\pgfusepath{clip}%
\pgfsetbuttcap%
\pgfsetroundjoin%
\definecolor{currentfill}{rgb}{0.267004,0.004874,0.329415}%
\pgfsetfillcolor{currentfill}%
\pgfsetfillopacity{0.700000}%
\pgfsetlinewidth{0.000000pt}%
\definecolor{currentstroke}{rgb}{0.000000,0.000000,0.000000}%
\pgfsetstrokecolor{currentstroke}%
\pgfsetstrokeopacity{0.700000}%
\pgfsetdash{}{0pt}%
\pgfpathmoveto{\pgfqpoint{8.304035in}{1.819960in}}%
\pgfpathcurveto{\pgfqpoint{8.309079in}{1.819960in}}{\pgfqpoint{8.313917in}{1.821964in}}{\pgfqpoint{8.317483in}{1.825530in}}%
\pgfpathcurveto{\pgfqpoint{8.321050in}{1.829097in}}{\pgfqpoint{8.323054in}{1.833934in}}{\pgfqpoint{8.323054in}{1.838978in}}%
\pgfpathcurveto{\pgfqpoint{8.323054in}{1.844022in}}{\pgfqpoint{8.321050in}{1.848860in}}{\pgfqpoint{8.317483in}{1.852426in}}%
\pgfpathcurveto{\pgfqpoint{8.313917in}{1.855992in}}{\pgfqpoint{8.309079in}{1.857996in}}{\pgfqpoint{8.304035in}{1.857996in}}%
\pgfpathcurveto{\pgfqpoint{8.298992in}{1.857996in}}{\pgfqpoint{8.294154in}{1.855992in}}{\pgfqpoint{8.290588in}{1.852426in}}%
\pgfpathcurveto{\pgfqpoint{8.287021in}{1.848860in}}{\pgfqpoint{8.285017in}{1.844022in}}{\pgfqpoint{8.285017in}{1.838978in}}%
\pgfpathcurveto{\pgfqpoint{8.285017in}{1.833934in}}{\pgfqpoint{8.287021in}{1.829097in}}{\pgfqpoint{8.290588in}{1.825530in}}%
\pgfpathcurveto{\pgfqpoint{8.294154in}{1.821964in}}{\pgfqpoint{8.298992in}{1.819960in}}{\pgfqpoint{8.304035in}{1.819960in}}%
\pgfpathclose%
\pgfusepath{fill}%
\end{pgfscope}%
\begin{pgfscope}%
\pgfpathrectangle{\pgfqpoint{6.572727in}{0.474100in}}{\pgfqpoint{4.227273in}{3.318700in}}%
\pgfusepath{clip}%
\pgfsetbuttcap%
\pgfsetroundjoin%
\definecolor{currentfill}{rgb}{0.267004,0.004874,0.329415}%
\pgfsetfillcolor{currentfill}%
\pgfsetfillopacity{0.700000}%
\pgfsetlinewidth{0.000000pt}%
\definecolor{currentstroke}{rgb}{0.000000,0.000000,0.000000}%
\pgfsetstrokecolor{currentstroke}%
\pgfsetstrokeopacity{0.700000}%
\pgfsetdash{}{0pt}%
\pgfpathmoveto{\pgfqpoint{7.945251in}{1.737565in}}%
\pgfpathcurveto{\pgfqpoint{7.950295in}{1.737565in}}{\pgfqpoint{7.955132in}{1.739569in}}{\pgfqpoint{7.958699in}{1.743136in}}%
\pgfpathcurveto{\pgfqpoint{7.962265in}{1.746702in}}{\pgfqpoint{7.964269in}{1.751540in}}{\pgfqpoint{7.964269in}{1.756584in}}%
\pgfpathcurveto{\pgfqpoint{7.964269in}{1.761627in}}{\pgfqpoint{7.962265in}{1.766465in}}{\pgfqpoint{7.958699in}{1.770031in}}%
\pgfpathcurveto{\pgfqpoint{7.955132in}{1.773598in}}{\pgfqpoint{7.950295in}{1.775602in}}{\pgfqpoint{7.945251in}{1.775602in}}%
\pgfpathcurveto{\pgfqpoint{7.940207in}{1.775602in}}{\pgfqpoint{7.935370in}{1.773598in}}{\pgfqpoint{7.931803in}{1.770031in}}%
\pgfpathcurveto{\pgfqpoint{7.928237in}{1.766465in}}{\pgfqpoint{7.926233in}{1.761627in}}{\pgfqpoint{7.926233in}{1.756584in}}%
\pgfpathcurveto{\pgfqpoint{7.926233in}{1.751540in}}{\pgfqpoint{7.928237in}{1.746702in}}{\pgfqpoint{7.931803in}{1.743136in}}%
\pgfpathcurveto{\pgfqpoint{7.935370in}{1.739569in}}{\pgfqpoint{7.940207in}{1.737565in}}{\pgfqpoint{7.945251in}{1.737565in}}%
\pgfpathclose%
\pgfusepath{fill}%
\end{pgfscope}%
\begin{pgfscope}%
\pgfpathrectangle{\pgfqpoint{6.572727in}{0.474100in}}{\pgfqpoint{4.227273in}{3.318700in}}%
\pgfusepath{clip}%
\pgfsetbuttcap%
\pgfsetroundjoin%
\definecolor{currentfill}{rgb}{0.993248,0.906157,0.143936}%
\pgfsetfillcolor{currentfill}%
\pgfsetfillopacity{0.700000}%
\pgfsetlinewidth{0.000000pt}%
\definecolor{currentstroke}{rgb}{0.000000,0.000000,0.000000}%
\pgfsetstrokecolor{currentstroke}%
\pgfsetstrokeopacity{0.700000}%
\pgfsetdash{}{0pt}%
\pgfpathmoveto{\pgfqpoint{8.374078in}{2.941125in}}%
\pgfpathcurveto{\pgfqpoint{8.379122in}{2.941125in}}{\pgfqpoint{8.383960in}{2.943129in}}{\pgfqpoint{8.387526in}{2.946696in}}%
\pgfpathcurveto{\pgfqpoint{8.391092in}{2.950262in}}{\pgfqpoint{8.393096in}{2.955100in}}{\pgfqpoint{8.393096in}{2.960143in}}%
\pgfpathcurveto{\pgfqpoint{8.393096in}{2.965187in}}{\pgfqpoint{8.391092in}{2.970025in}}{\pgfqpoint{8.387526in}{2.973591in}}%
\pgfpathcurveto{\pgfqpoint{8.383960in}{2.977158in}}{\pgfqpoint{8.379122in}{2.979162in}}{\pgfqpoint{8.374078in}{2.979162in}}%
\pgfpathcurveto{\pgfqpoint{8.369035in}{2.979162in}}{\pgfqpoint{8.364197in}{2.977158in}}{\pgfqpoint{8.360630in}{2.973591in}}%
\pgfpathcurveto{\pgfqpoint{8.357064in}{2.970025in}}{\pgfqpoint{8.355060in}{2.965187in}}{\pgfqpoint{8.355060in}{2.960143in}}%
\pgfpathcurveto{\pgfqpoint{8.355060in}{2.955100in}}{\pgfqpoint{8.357064in}{2.950262in}}{\pgfqpoint{8.360630in}{2.946696in}}%
\pgfpathcurveto{\pgfqpoint{8.364197in}{2.943129in}}{\pgfqpoint{8.369035in}{2.941125in}}{\pgfqpoint{8.374078in}{2.941125in}}%
\pgfpathclose%
\pgfusepath{fill}%
\end{pgfscope}%
\begin{pgfscope}%
\pgfpathrectangle{\pgfqpoint{6.572727in}{0.474100in}}{\pgfqpoint{4.227273in}{3.318700in}}%
\pgfusepath{clip}%
\pgfsetbuttcap%
\pgfsetroundjoin%
\definecolor{currentfill}{rgb}{0.267004,0.004874,0.329415}%
\pgfsetfillcolor{currentfill}%
\pgfsetfillopacity{0.700000}%
\pgfsetlinewidth{0.000000pt}%
\definecolor{currentstroke}{rgb}{0.000000,0.000000,0.000000}%
\pgfsetstrokecolor{currentstroke}%
\pgfsetstrokeopacity{0.700000}%
\pgfsetdash{}{0pt}%
\pgfpathmoveto{\pgfqpoint{8.391447in}{1.884211in}}%
\pgfpathcurveto{\pgfqpoint{8.396491in}{1.884211in}}{\pgfqpoint{8.401329in}{1.886215in}}{\pgfqpoint{8.404895in}{1.889781in}}%
\pgfpathcurveto{\pgfqpoint{8.408461in}{1.893348in}}{\pgfqpoint{8.410465in}{1.898186in}}{\pgfqpoint{8.410465in}{1.903229in}}%
\pgfpathcurveto{\pgfqpoint{8.410465in}{1.908273in}}{\pgfqpoint{8.408461in}{1.913111in}}{\pgfqpoint{8.404895in}{1.916677in}}%
\pgfpathcurveto{\pgfqpoint{8.401329in}{1.920243in}}{\pgfqpoint{8.396491in}{1.922247in}}{\pgfqpoint{8.391447in}{1.922247in}}%
\pgfpathcurveto{\pgfqpoint{8.386403in}{1.922247in}}{\pgfqpoint{8.381566in}{1.920243in}}{\pgfqpoint{8.377999in}{1.916677in}}%
\pgfpathcurveto{\pgfqpoint{8.374433in}{1.913111in}}{\pgfqpoint{8.372429in}{1.908273in}}{\pgfqpoint{8.372429in}{1.903229in}}%
\pgfpathcurveto{\pgfqpoint{8.372429in}{1.898186in}}{\pgfqpoint{8.374433in}{1.893348in}}{\pgfqpoint{8.377999in}{1.889781in}}%
\pgfpathcurveto{\pgfqpoint{8.381566in}{1.886215in}}{\pgfqpoint{8.386403in}{1.884211in}}{\pgfqpoint{8.391447in}{1.884211in}}%
\pgfpathclose%
\pgfusepath{fill}%
\end{pgfscope}%
\begin{pgfscope}%
\pgfpathrectangle{\pgfqpoint{6.572727in}{0.474100in}}{\pgfqpoint{4.227273in}{3.318700in}}%
\pgfusepath{clip}%
\pgfsetbuttcap%
\pgfsetroundjoin%
\definecolor{currentfill}{rgb}{0.127568,0.566949,0.550556}%
\pgfsetfillcolor{currentfill}%
\pgfsetfillopacity{0.700000}%
\pgfsetlinewidth{0.000000pt}%
\definecolor{currentstroke}{rgb}{0.000000,0.000000,0.000000}%
\pgfsetstrokecolor{currentstroke}%
\pgfsetstrokeopacity{0.700000}%
\pgfsetdash{}{0pt}%
\pgfpathmoveto{\pgfqpoint{9.050304in}{1.386428in}}%
\pgfpathcurveto{\pgfqpoint{9.055348in}{1.386428in}}{\pgfqpoint{9.060186in}{1.388432in}}{\pgfqpoint{9.063752in}{1.391998in}}%
\pgfpathcurveto{\pgfqpoint{9.067319in}{1.395564in}}{\pgfqpoint{9.069322in}{1.400402in}}{\pgfqpoint{9.069322in}{1.405446in}}%
\pgfpathcurveto{\pgfqpoint{9.069322in}{1.410489in}}{\pgfqpoint{9.067319in}{1.415327in}}{\pgfqpoint{9.063752in}{1.418894in}}%
\pgfpathcurveto{\pgfqpoint{9.060186in}{1.422460in}}{\pgfqpoint{9.055348in}{1.424464in}}{\pgfqpoint{9.050304in}{1.424464in}}%
\pgfpathcurveto{\pgfqpoint{9.045261in}{1.424464in}}{\pgfqpoint{9.040423in}{1.422460in}}{\pgfqpoint{9.036856in}{1.418894in}}%
\pgfpathcurveto{\pgfqpoint{9.033290in}{1.415327in}}{\pgfqpoint{9.031286in}{1.410489in}}{\pgfqpoint{9.031286in}{1.405446in}}%
\pgfpathcurveto{\pgfqpoint{9.031286in}{1.400402in}}{\pgfqpoint{9.033290in}{1.395564in}}{\pgfqpoint{9.036856in}{1.391998in}}%
\pgfpathcurveto{\pgfqpoint{9.040423in}{1.388432in}}{\pgfqpoint{9.045261in}{1.386428in}}{\pgfqpoint{9.050304in}{1.386428in}}%
\pgfpathclose%
\pgfusepath{fill}%
\end{pgfscope}%
\begin{pgfscope}%
\pgfpathrectangle{\pgfqpoint{6.572727in}{0.474100in}}{\pgfqpoint{4.227273in}{3.318700in}}%
\pgfusepath{clip}%
\pgfsetbuttcap%
\pgfsetroundjoin%
\definecolor{currentfill}{rgb}{0.267004,0.004874,0.329415}%
\pgfsetfillcolor{currentfill}%
\pgfsetfillopacity{0.700000}%
\pgfsetlinewidth{0.000000pt}%
\definecolor{currentstroke}{rgb}{0.000000,0.000000,0.000000}%
\pgfsetstrokecolor{currentstroke}%
\pgfsetstrokeopacity{0.700000}%
\pgfsetdash{}{0pt}%
\pgfpathmoveto{\pgfqpoint{8.129471in}{1.783937in}}%
\pgfpathcurveto{\pgfqpoint{8.134515in}{1.783937in}}{\pgfqpoint{8.139353in}{1.785941in}}{\pgfqpoint{8.142919in}{1.789507in}}%
\pgfpathcurveto{\pgfqpoint{8.146486in}{1.793074in}}{\pgfqpoint{8.148490in}{1.797912in}}{\pgfqpoint{8.148490in}{1.802955in}}%
\pgfpathcurveto{\pgfqpoint{8.148490in}{1.807999in}}{\pgfqpoint{8.146486in}{1.812837in}}{\pgfqpoint{8.142919in}{1.816403in}}%
\pgfpathcurveto{\pgfqpoint{8.139353in}{1.819970in}}{\pgfqpoint{8.134515in}{1.821973in}}{\pgfqpoint{8.129471in}{1.821973in}}%
\pgfpathcurveto{\pgfqpoint{8.124428in}{1.821973in}}{\pgfqpoint{8.119590in}{1.819970in}}{\pgfqpoint{8.116024in}{1.816403in}}%
\pgfpathcurveto{\pgfqpoint{8.112457in}{1.812837in}}{\pgfqpoint{8.110453in}{1.807999in}}{\pgfqpoint{8.110453in}{1.802955in}}%
\pgfpathcurveto{\pgfqpoint{8.110453in}{1.797912in}}{\pgfqpoint{8.112457in}{1.793074in}}{\pgfqpoint{8.116024in}{1.789507in}}%
\pgfpathcurveto{\pgfqpoint{8.119590in}{1.785941in}}{\pgfqpoint{8.124428in}{1.783937in}}{\pgfqpoint{8.129471in}{1.783937in}}%
\pgfpathclose%
\pgfusepath{fill}%
\end{pgfscope}%
\begin{pgfscope}%
\pgfpathrectangle{\pgfqpoint{6.572727in}{0.474100in}}{\pgfqpoint{4.227273in}{3.318700in}}%
\pgfusepath{clip}%
\pgfsetbuttcap%
\pgfsetroundjoin%
\definecolor{currentfill}{rgb}{0.993248,0.906157,0.143936}%
\pgfsetfillcolor{currentfill}%
\pgfsetfillopacity{0.700000}%
\pgfsetlinewidth{0.000000pt}%
\definecolor{currentstroke}{rgb}{0.000000,0.000000,0.000000}%
\pgfsetstrokecolor{currentstroke}%
\pgfsetstrokeopacity{0.700000}%
\pgfsetdash{}{0pt}%
\pgfpathmoveto{\pgfqpoint{7.962267in}{2.481758in}}%
\pgfpathcurveto{\pgfqpoint{7.967311in}{2.481758in}}{\pgfqpoint{7.972149in}{2.483762in}}{\pgfqpoint{7.975715in}{2.487329in}}%
\pgfpathcurveto{\pgfqpoint{7.979282in}{2.490895in}}{\pgfqpoint{7.981286in}{2.495733in}}{\pgfqpoint{7.981286in}{2.500776in}}%
\pgfpathcurveto{\pgfqpoint{7.981286in}{2.505820in}}{\pgfqpoint{7.979282in}{2.510658in}}{\pgfqpoint{7.975715in}{2.514224in}}%
\pgfpathcurveto{\pgfqpoint{7.972149in}{2.517791in}}{\pgfqpoint{7.967311in}{2.519795in}}{\pgfqpoint{7.962267in}{2.519795in}}%
\pgfpathcurveto{\pgfqpoint{7.957224in}{2.519795in}}{\pgfqpoint{7.952386in}{2.517791in}}{\pgfqpoint{7.948820in}{2.514224in}}%
\pgfpathcurveto{\pgfqpoint{7.945253in}{2.510658in}}{\pgfqpoint{7.943249in}{2.505820in}}{\pgfqpoint{7.943249in}{2.500776in}}%
\pgfpathcurveto{\pgfqpoint{7.943249in}{2.495733in}}{\pgfqpoint{7.945253in}{2.490895in}}{\pgfqpoint{7.948820in}{2.487329in}}%
\pgfpathcurveto{\pgfqpoint{7.952386in}{2.483762in}}{\pgfqpoint{7.957224in}{2.481758in}}{\pgfqpoint{7.962267in}{2.481758in}}%
\pgfpathclose%
\pgfusepath{fill}%
\end{pgfscope}%
\begin{pgfscope}%
\pgfpathrectangle{\pgfqpoint{6.572727in}{0.474100in}}{\pgfqpoint{4.227273in}{3.318700in}}%
\pgfusepath{clip}%
\pgfsetbuttcap%
\pgfsetroundjoin%
\definecolor{currentfill}{rgb}{0.127568,0.566949,0.550556}%
\pgfsetfillcolor{currentfill}%
\pgfsetfillopacity{0.700000}%
\pgfsetlinewidth{0.000000pt}%
\definecolor{currentstroke}{rgb}{0.000000,0.000000,0.000000}%
\pgfsetstrokecolor{currentstroke}%
\pgfsetstrokeopacity{0.700000}%
\pgfsetdash{}{0pt}%
\pgfpathmoveto{\pgfqpoint{9.766269in}{2.089141in}}%
\pgfpathcurveto{\pgfqpoint{9.771313in}{2.089141in}}{\pgfqpoint{9.776151in}{2.091145in}}{\pgfqpoint{9.779717in}{2.094712in}}%
\pgfpathcurveto{\pgfqpoint{9.783284in}{2.098278in}}{\pgfqpoint{9.785287in}{2.103116in}}{\pgfqpoint{9.785287in}{2.108160in}}%
\pgfpathcurveto{\pgfqpoint{9.785287in}{2.113203in}}{\pgfqpoint{9.783284in}{2.118041in}}{\pgfqpoint{9.779717in}{2.121607in}}%
\pgfpathcurveto{\pgfqpoint{9.776151in}{2.125174in}}{\pgfqpoint{9.771313in}{2.127178in}}{\pgfqpoint{9.766269in}{2.127178in}}%
\pgfpathcurveto{\pgfqpoint{9.761226in}{2.127178in}}{\pgfqpoint{9.756388in}{2.125174in}}{\pgfqpoint{9.752821in}{2.121607in}}%
\pgfpathcurveto{\pgfqpoint{9.749255in}{2.118041in}}{\pgfqpoint{9.747251in}{2.113203in}}{\pgfqpoint{9.747251in}{2.108160in}}%
\pgfpathcurveto{\pgfqpoint{9.747251in}{2.103116in}}{\pgfqpoint{9.749255in}{2.098278in}}{\pgfqpoint{9.752821in}{2.094712in}}%
\pgfpathcurveto{\pgfqpoint{9.756388in}{2.091145in}}{\pgfqpoint{9.761226in}{2.089141in}}{\pgfqpoint{9.766269in}{2.089141in}}%
\pgfpathclose%
\pgfusepath{fill}%
\end{pgfscope}%
\begin{pgfscope}%
\pgfpathrectangle{\pgfqpoint{6.572727in}{0.474100in}}{\pgfqpoint{4.227273in}{3.318700in}}%
\pgfusepath{clip}%
\pgfsetbuttcap%
\pgfsetroundjoin%
\definecolor{currentfill}{rgb}{0.267004,0.004874,0.329415}%
\pgfsetfillcolor{currentfill}%
\pgfsetfillopacity{0.700000}%
\pgfsetlinewidth{0.000000pt}%
\definecolor{currentstroke}{rgb}{0.000000,0.000000,0.000000}%
\pgfsetstrokecolor{currentstroke}%
\pgfsetstrokeopacity{0.700000}%
\pgfsetdash{}{0pt}%
\pgfpathmoveto{\pgfqpoint{7.789275in}{1.797905in}}%
\pgfpathcurveto{\pgfqpoint{7.794319in}{1.797905in}}{\pgfqpoint{7.799157in}{1.799909in}}{\pgfqpoint{7.802723in}{1.803475in}}%
\pgfpathcurveto{\pgfqpoint{7.806289in}{1.807042in}}{\pgfqpoint{7.808293in}{1.811880in}}{\pgfqpoint{7.808293in}{1.816923in}}%
\pgfpathcurveto{\pgfqpoint{7.808293in}{1.821967in}}{\pgfqpoint{7.806289in}{1.826805in}}{\pgfqpoint{7.802723in}{1.830371in}}%
\pgfpathcurveto{\pgfqpoint{7.799157in}{1.833937in}}{\pgfqpoint{7.794319in}{1.835941in}}{\pgfqpoint{7.789275in}{1.835941in}}%
\pgfpathcurveto{\pgfqpoint{7.784232in}{1.835941in}}{\pgfqpoint{7.779394in}{1.833937in}}{\pgfqpoint{7.775827in}{1.830371in}}%
\pgfpathcurveto{\pgfqpoint{7.772261in}{1.826805in}}{\pgfqpoint{7.770257in}{1.821967in}}{\pgfqpoint{7.770257in}{1.816923in}}%
\pgfpathcurveto{\pgfqpoint{7.770257in}{1.811880in}}{\pgfqpoint{7.772261in}{1.807042in}}{\pgfqpoint{7.775827in}{1.803475in}}%
\pgfpathcurveto{\pgfqpoint{7.779394in}{1.799909in}}{\pgfqpoint{7.784232in}{1.797905in}}{\pgfqpoint{7.789275in}{1.797905in}}%
\pgfpathclose%
\pgfusepath{fill}%
\end{pgfscope}%
\begin{pgfscope}%
\pgfpathrectangle{\pgfqpoint{6.572727in}{0.474100in}}{\pgfqpoint{4.227273in}{3.318700in}}%
\pgfusepath{clip}%
\pgfsetbuttcap%
\pgfsetroundjoin%
\definecolor{currentfill}{rgb}{0.127568,0.566949,0.550556}%
\pgfsetfillcolor{currentfill}%
\pgfsetfillopacity{0.700000}%
\pgfsetlinewidth{0.000000pt}%
\definecolor{currentstroke}{rgb}{0.000000,0.000000,0.000000}%
\pgfsetstrokecolor{currentstroke}%
\pgfsetstrokeopacity{0.700000}%
\pgfsetdash{}{0pt}%
\pgfpathmoveto{\pgfqpoint{10.206277in}{1.340748in}}%
\pgfpathcurveto{\pgfqpoint{10.211320in}{1.340748in}}{\pgfqpoint{10.216158in}{1.342751in}}{\pgfqpoint{10.219724in}{1.346318in}}%
\pgfpathcurveto{\pgfqpoint{10.223291in}{1.349884in}}{\pgfqpoint{10.225295in}{1.354722in}}{\pgfqpoint{10.225295in}{1.359766in}}%
\pgfpathcurveto{\pgfqpoint{10.225295in}{1.364809in}}{\pgfqpoint{10.223291in}{1.369647in}}{\pgfqpoint{10.219724in}{1.373214in}}%
\pgfpathcurveto{\pgfqpoint{10.216158in}{1.376780in}}{\pgfqpoint{10.211320in}{1.378784in}}{\pgfqpoint{10.206277in}{1.378784in}}%
\pgfpathcurveto{\pgfqpoint{10.201233in}{1.378784in}}{\pgfqpoint{10.196395in}{1.376780in}}{\pgfqpoint{10.192829in}{1.373214in}}%
\pgfpathcurveto{\pgfqpoint{10.189262in}{1.369647in}}{\pgfqpoint{10.187258in}{1.364809in}}{\pgfqpoint{10.187258in}{1.359766in}}%
\pgfpathcurveto{\pgfqpoint{10.187258in}{1.354722in}}{\pgfqpoint{10.189262in}{1.349884in}}{\pgfqpoint{10.192829in}{1.346318in}}%
\pgfpathcurveto{\pgfqpoint{10.196395in}{1.342751in}}{\pgfqpoint{10.201233in}{1.340748in}}{\pgfqpoint{10.206277in}{1.340748in}}%
\pgfpathclose%
\pgfusepath{fill}%
\end{pgfscope}%
\begin{pgfscope}%
\pgfpathrectangle{\pgfqpoint{6.572727in}{0.474100in}}{\pgfqpoint{4.227273in}{3.318700in}}%
\pgfusepath{clip}%
\pgfsetbuttcap%
\pgfsetroundjoin%
\definecolor{currentfill}{rgb}{0.267004,0.004874,0.329415}%
\pgfsetfillcolor{currentfill}%
\pgfsetfillopacity{0.700000}%
\pgfsetlinewidth{0.000000pt}%
\definecolor{currentstroke}{rgb}{0.000000,0.000000,0.000000}%
\pgfsetstrokecolor{currentstroke}%
\pgfsetstrokeopacity{0.700000}%
\pgfsetdash{}{0pt}%
\pgfpathmoveto{\pgfqpoint{7.979880in}{1.698175in}}%
\pgfpathcurveto{\pgfqpoint{7.984924in}{1.698175in}}{\pgfqpoint{7.989762in}{1.700179in}}{\pgfqpoint{7.993328in}{1.703745in}}%
\pgfpathcurveto{\pgfqpoint{7.996895in}{1.707312in}}{\pgfqpoint{7.998898in}{1.712150in}}{\pgfqpoint{7.998898in}{1.717193in}}%
\pgfpathcurveto{\pgfqpoint{7.998898in}{1.722237in}}{\pgfqpoint{7.996895in}{1.727075in}}{\pgfqpoint{7.993328in}{1.730641in}}%
\pgfpathcurveto{\pgfqpoint{7.989762in}{1.734208in}}{\pgfqpoint{7.984924in}{1.736211in}}{\pgfqpoint{7.979880in}{1.736211in}}%
\pgfpathcurveto{\pgfqpoint{7.974837in}{1.736211in}}{\pgfqpoint{7.969999in}{1.734208in}}{\pgfqpoint{7.966432in}{1.730641in}}%
\pgfpathcurveto{\pgfqpoint{7.962866in}{1.727075in}}{\pgfqpoint{7.960862in}{1.722237in}}{\pgfqpoint{7.960862in}{1.717193in}}%
\pgfpathcurveto{\pgfqpoint{7.960862in}{1.712150in}}{\pgfqpoint{7.962866in}{1.707312in}}{\pgfqpoint{7.966432in}{1.703745in}}%
\pgfpathcurveto{\pgfqpoint{7.969999in}{1.700179in}}{\pgfqpoint{7.974837in}{1.698175in}}{\pgfqpoint{7.979880in}{1.698175in}}%
\pgfpathclose%
\pgfusepath{fill}%
\end{pgfscope}%
\begin{pgfscope}%
\pgfpathrectangle{\pgfqpoint{6.572727in}{0.474100in}}{\pgfqpoint{4.227273in}{3.318700in}}%
\pgfusepath{clip}%
\pgfsetbuttcap%
\pgfsetroundjoin%
\definecolor{currentfill}{rgb}{0.993248,0.906157,0.143936}%
\pgfsetfillcolor{currentfill}%
\pgfsetfillopacity{0.700000}%
\pgfsetlinewidth{0.000000pt}%
\definecolor{currentstroke}{rgb}{0.000000,0.000000,0.000000}%
\pgfsetstrokecolor{currentstroke}%
\pgfsetstrokeopacity{0.700000}%
\pgfsetdash{}{0pt}%
\pgfpathmoveto{\pgfqpoint{7.675073in}{2.953896in}}%
\pgfpathcurveto{\pgfqpoint{7.680117in}{2.953896in}}{\pgfqpoint{7.684955in}{2.955900in}}{\pgfqpoint{7.688521in}{2.959466in}}%
\pgfpathcurveto{\pgfqpoint{7.692088in}{2.963033in}}{\pgfqpoint{7.694091in}{2.967870in}}{\pgfqpoint{7.694091in}{2.972914in}}%
\pgfpathcurveto{\pgfqpoint{7.694091in}{2.977958in}}{\pgfqpoint{7.692088in}{2.982796in}}{\pgfqpoint{7.688521in}{2.986362in}}%
\pgfpathcurveto{\pgfqpoint{7.684955in}{2.989928in}}{\pgfqpoint{7.680117in}{2.991932in}}{\pgfqpoint{7.675073in}{2.991932in}}%
\pgfpathcurveto{\pgfqpoint{7.670030in}{2.991932in}}{\pgfqpoint{7.665192in}{2.989928in}}{\pgfqpoint{7.661625in}{2.986362in}}%
\pgfpathcurveto{\pgfqpoint{7.658059in}{2.982796in}}{\pgfqpoint{7.656055in}{2.977958in}}{\pgfqpoint{7.656055in}{2.972914in}}%
\pgfpathcurveto{\pgfqpoint{7.656055in}{2.967870in}}{\pgfqpoint{7.658059in}{2.963033in}}{\pgfqpoint{7.661625in}{2.959466in}}%
\pgfpathcurveto{\pgfqpoint{7.665192in}{2.955900in}}{\pgfqpoint{7.670030in}{2.953896in}}{\pgfqpoint{7.675073in}{2.953896in}}%
\pgfpathclose%
\pgfusepath{fill}%
\end{pgfscope}%
\begin{pgfscope}%
\pgfpathrectangle{\pgfqpoint{6.572727in}{0.474100in}}{\pgfqpoint{4.227273in}{3.318700in}}%
\pgfusepath{clip}%
\pgfsetbuttcap%
\pgfsetroundjoin%
\definecolor{currentfill}{rgb}{0.993248,0.906157,0.143936}%
\pgfsetfillcolor{currentfill}%
\pgfsetfillopacity{0.700000}%
\pgfsetlinewidth{0.000000pt}%
\definecolor{currentstroke}{rgb}{0.000000,0.000000,0.000000}%
\pgfsetstrokecolor{currentstroke}%
\pgfsetstrokeopacity{0.700000}%
\pgfsetdash{}{0pt}%
\pgfpathmoveto{\pgfqpoint{8.025314in}{2.734172in}}%
\pgfpathcurveto{\pgfqpoint{8.030358in}{2.734172in}}{\pgfqpoint{8.035195in}{2.736176in}}{\pgfqpoint{8.038762in}{2.739742in}}%
\pgfpathcurveto{\pgfqpoint{8.042328in}{2.743308in}}{\pgfqpoint{8.044332in}{2.748146in}}{\pgfqpoint{8.044332in}{2.753190in}}%
\pgfpathcurveto{\pgfqpoint{8.044332in}{2.758233in}}{\pgfqpoint{8.042328in}{2.763071in}}{\pgfqpoint{8.038762in}{2.766638in}}%
\pgfpathcurveto{\pgfqpoint{8.035195in}{2.770204in}}{\pgfqpoint{8.030358in}{2.772208in}}{\pgfqpoint{8.025314in}{2.772208in}}%
\pgfpathcurveto{\pgfqpoint{8.020270in}{2.772208in}}{\pgfqpoint{8.015432in}{2.770204in}}{\pgfqpoint{8.011866in}{2.766638in}}%
\pgfpathcurveto{\pgfqpoint{8.008300in}{2.763071in}}{\pgfqpoint{8.006296in}{2.758233in}}{\pgfqpoint{8.006296in}{2.753190in}}%
\pgfpathcurveto{\pgfqpoint{8.006296in}{2.748146in}}{\pgfqpoint{8.008300in}{2.743308in}}{\pgfqpoint{8.011866in}{2.739742in}}%
\pgfpathcurveto{\pgfqpoint{8.015432in}{2.736176in}}{\pgfqpoint{8.020270in}{2.734172in}}{\pgfqpoint{8.025314in}{2.734172in}}%
\pgfpathclose%
\pgfusepath{fill}%
\end{pgfscope}%
\begin{pgfscope}%
\pgfpathrectangle{\pgfqpoint{6.572727in}{0.474100in}}{\pgfqpoint{4.227273in}{3.318700in}}%
\pgfusepath{clip}%
\pgfsetbuttcap%
\pgfsetroundjoin%
\definecolor{currentfill}{rgb}{0.267004,0.004874,0.329415}%
\pgfsetfillcolor{currentfill}%
\pgfsetfillopacity{0.700000}%
\pgfsetlinewidth{0.000000pt}%
\definecolor{currentstroke}{rgb}{0.000000,0.000000,0.000000}%
\pgfsetstrokecolor{currentstroke}%
\pgfsetstrokeopacity{0.700000}%
\pgfsetdash{}{0pt}%
\pgfpathmoveto{\pgfqpoint{7.433749in}{1.104281in}}%
\pgfpathcurveto{\pgfqpoint{7.438793in}{1.104281in}}{\pgfqpoint{7.443631in}{1.106285in}}{\pgfqpoint{7.447197in}{1.109852in}}%
\pgfpathcurveto{\pgfqpoint{7.450763in}{1.113418in}}{\pgfqpoint{7.452767in}{1.118256in}}{\pgfqpoint{7.452767in}{1.123300in}}%
\pgfpathcurveto{\pgfqpoint{7.452767in}{1.128343in}}{\pgfqpoint{7.450763in}{1.133181in}}{\pgfqpoint{7.447197in}{1.136747in}}%
\pgfpathcurveto{\pgfqpoint{7.443631in}{1.140314in}}{\pgfqpoint{7.438793in}{1.142318in}}{\pgfqpoint{7.433749in}{1.142318in}}%
\pgfpathcurveto{\pgfqpoint{7.428705in}{1.142318in}}{\pgfqpoint{7.423868in}{1.140314in}}{\pgfqpoint{7.420301in}{1.136747in}}%
\pgfpathcurveto{\pgfqpoint{7.416735in}{1.133181in}}{\pgfqpoint{7.414731in}{1.128343in}}{\pgfqpoint{7.414731in}{1.123300in}}%
\pgfpathcurveto{\pgfqpoint{7.414731in}{1.118256in}}{\pgfqpoint{7.416735in}{1.113418in}}{\pgfqpoint{7.420301in}{1.109852in}}%
\pgfpathcurveto{\pgfqpoint{7.423868in}{1.106285in}}{\pgfqpoint{7.428705in}{1.104281in}}{\pgfqpoint{7.433749in}{1.104281in}}%
\pgfpathclose%
\pgfusepath{fill}%
\end{pgfscope}%
\begin{pgfscope}%
\pgfpathrectangle{\pgfqpoint{6.572727in}{0.474100in}}{\pgfqpoint{4.227273in}{3.318700in}}%
\pgfusepath{clip}%
\pgfsetbuttcap%
\pgfsetroundjoin%
\definecolor{currentfill}{rgb}{0.267004,0.004874,0.329415}%
\pgfsetfillcolor{currentfill}%
\pgfsetfillopacity{0.700000}%
\pgfsetlinewidth{0.000000pt}%
\definecolor{currentstroke}{rgb}{0.000000,0.000000,0.000000}%
\pgfsetstrokecolor{currentstroke}%
\pgfsetstrokeopacity{0.700000}%
\pgfsetdash{}{0pt}%
\pgfpathmoveto{\pgfqpoint{7.968491in}{1.547670in}}%
\pgfpathcurveto{\pgfqpoint{7.973535in}{1.547670in}}{\pgfqpoint{7.978373in}{1.549674in}}{\pgfqpoint{7.981939in}{1.553241in}}%
\pgfpathcurveto{\pgfqpoint{7.985506in}{1.556807in}}{\pgfqpoint{7.987510in}{1.561645in}}{\pgfqpoint{7.987510in}{1.566688in}}%
\pgfpathcurveto{\pgfqpoint{7.987510in}{1.571732in}}{\pgfqpoint{7.985506in}{1.576570in}}{\pgfqpoint{7.981939in}{1.580136in}}%
\pgfpathcurveto{\pgfqpoint{7.978373in}{1.583703in}}{\pgfqpoint{7.973535in}{1.585707in}}{\pgfqpoint{7.968491in}{1.585707in}}%
\pgfpathcurveto{\pgfqpoint{7.963448in}{1.585707in}}{\pgfqpoint{7.958610in}{1.583703in}}{\pgfqpoint{7.955044in}{1.580136in}}%
\pgfpathcurveto{\pgfqpoint{7.951477in}{1.576570in}}{\pgfqpoint{7.949473in}{1.571732in}}{\pgfqpoint{7.949473in}{1.566688in}}%
\pgfpathcurveto{\pgfqpoint{7.949473in}{1.561645in}}{\pgfqpoint{7.951477in}{1.556807in}}{\pgfqpoint{7.955044in}{1.553241in}}%
\pgfpathcurveto{\pgfqpoint{7.958610in}{1.549674in}}{\pgfqpoint{7.963448in}{1.547670in}}{\pgfqpoint{7.968491in}{1.547670in}}%
\pgfpathclose%
\pgfusepath{fill}%
\end{pgfscope}%
\begin{pgfscope}%
\pgfpathrectangle{\pgfqpoint{6.572727in}{0.474100in}}{\pgfqpoint{4.227273in}{3.318700in}}%
\pgfusepath{clip}%
\pgfsetbuttcap%
\pgfsetroundjoin%
\definecolor{currentfill}{rgb}{0.267004,0.004874,0.329415}%
\pgfsetfillcolor{currentfill}%
\pgfsetfillopacity{0.700000}%
\pgfsetlinewidth{0.000000pt}%
\definecolor{currentstroke}{rgb}{0.000000,0.000000,0.000000}%
\pgfsetstrokecolor{currentstroke}%
\pgfsetstrokeopacity{0.700000}%
\pgfsetdash{}{0pt}%
\pgfpathmoveto{\pgfqpoint{7.977755in}{1.376184in}}%
\pgfpathcurveto{\pgfqpoint{7.982799in}{1.376184in}}{\pgfqpoint{7.987637in}{1.378188in}}{\pgfqpoint{7.991203in}{1.381754in}}%
\pgfpathcurveto{\pgfqpoint{7.994769in}{1.385320in}}{\pgfqpoint{7.996773in}{1.390158in}}{\pgfqpoint{7.996773in}{1.395202in}}%
\pgfpathcurveto{\pgfqpoint{7.996773in}{1.400245in}}{\pgfqpoint{7.994769in}{1.405083in}}{\pgfqpoint{7.991203in}{1.408650in}}%
\pgfpathcurveto{\pgfqpoint{7.987637in}{1.412216in}}{\pgfqpoint{7.982799in}{1.414220in}}{\pgfqpoint{7.977755in}{1.414220in}}%
\pgfpathcurveto{\pgfqpoint{7.972711in}{1.414220in}}{\pgfqpoint{7.967874in}{1.412216in}}{\pgfqpoint{7.964307in}{1.408650in}}%
\pgfpathcurveto{\pgfqpoint{7.960741in}{1.405083in}}{\pgfqpoint{7.958737in}{1.400245in}}{\pgfqpoint{7.958737in}{1.395202in}}%
\pgfpathcurveto{\pgfqpoint{7.958737in}{1.390158in}}{\pgfqpoint{7.960741in}{1.385320in}}{\pgfqpoint{7.964307in}{1.381754in}}%
\pgfpathcurveto{\pgfqpoint{7.967874in}{1.378188in}}{\pgfqpoint{7.972711in}{1.376184in}}{\pgfqpoint{7.977755in}{1.376184in}}%
\pgfpathclose%
\pgfusepath{fill}%
\end{pgfscope}%
\begin{pgfscope}%
\pgfpathrectangle{\pgfqpoint{6.572727in}{0.474100in}}{\pgfqpoint{4.227273in}{3.318700in}}%
\pgfusepath{clip}%
\pgfsetbuttcap%
\pgfsetroundjoin%
\definecolor{currentfill}{rgb}{0.993248,0.906157,0.143936}%
\pgfsetfillcolor{currentfill}%
\pgfsetfillopacity{0.700000}%
\pgfsetlinewidth{0.000000pt}%
\definecolor{currentstroke}{rgb}{0.000000,0.000000,0.000000}%
\pgfsetstrokecolor{currentstroke}%
\pgfsetstrokeopacity{0.700000}%
\pgfsetdash{}{0pt}%
\pgfpathmoveto{\pgfqpoint{8.078050in}{2.949183in}}%
\pgfpathcurveto{\pgfqpoint{8.083094in}{2.949183in}}{\pgfqpoint{8.087932in}{2.951187in}}{\pgfqpoint{8.091498in}{2.954753in}}%
\pgfpathcurveto{\pgfqpoint{8.095064in}{2.958320in}}{\pgfqpoint{8.097068in}{2.963158in}}{\pgfqpoint{8.097068in}{2.968201in}}%
\pgfpathcurveto{\pgfqpoint{8.097068in}{2.973245in}}{\pgfqpoint{8.095064in}{2.978083in}}{\pgfqpoint{8.091498in}{2.981649in}}%
\pgfpathcurveto{\pgfqpoint{8.087932in}{2.985215in}}{\pgfqpoint{8.083094in}{2.987219in}}{\pgfqpoint{8.078050in}{2.987219in}}%
\pgfpathcurveto{\pgfqpoint{8.073007in}{2.987219in}}{\pgfqpoint{8.068169in}{2.985215in}}{\pgfqpoint{8.064602in}{2.981649in}}%
\pgfpathcurveto{\pgfqpoint{8.061036in}{2.978083in}}{\pgfqpoint{8.059032in}{2.973245in}}{\pgfqpoint{8.059032in}{2.968201in}}%
\pgfpathcurveto{\pgfqpoint{8.059032in}{2.963158in}}{\pgfqpoint{8.061036in}{2.958320in}}{\pgfqpoint{8.064602in}{2.954753in}}%
\pgfpathcurveto{\pgfqpoint{8.068169in}{2.951187in}}{\pgfqpoint{8.073007in}{2.949183in}}{\pgfqpoint{8.078050in}{2.949183in}}%
\pgfpathclose%
\pgfusepath{fill}%
\end{pgfscope}%
\begin{pgfscope}%
\pgfpathrectangle{\pgfqpoint{6.572727in}{0.474100in}}{\pgfqpoint{4.227273in}{3.318700in}}%
\pgfusepath{clip}%
\pgfsetbuttcap%
\pgfsetroundjoin%
\definecolor{currentfill}{rgb}{0.267004,0.004874,0.329415}%
\pgfsetfillcolor{currentfill}%
\pgfsetfillopacity{0.700000}%
\pgfsetlinewidth{0.000000pt}%
\definecolor{currentstroke}{rgb}{0.000000,0.000000,0.000000}%
\pgfsetstrokecolor{currentstroke}%
\pgfsetstrokeopacity{0.700000}%
\pgfsetdash{}{0pt}%
\pgfpathmoveto{\pgfqpoint{8.105524in}{1.262040in}}%
\pgfpathcurveto{\pgfqpoint{8.110568in}{1.262040in}}{\pgfqpoint{8.115406in}{1.264044in}}{\pgfqpoint{8.118972in}{1.267611in}}%
\pgfpathcurveto{\pgfqpoint{8.122539in}{1.271177in}}{\pgfqpoint{8.124543in}{1.276015in}}{\pgfqpoint{8.124543in}{1.281059in}}%
\pgfpathcurveto{\pgfqpoint{8.124543in}{1.286102in}}{\pgfqpoint{8.122539in}{1.290940in}}{\pgfqpoint{8.118972in}{1.294506in}}%
\pgfpathcurveto{\pgfqpoint{8.115406in}{1.298073in}}{\pgfqpoint{8.110568in}{1.300077in}}{\pgfqpoint{8.105524in}{1.300077in}}%
\pgfpathcurveto{\pgfqpoint{8.100481in}{1.300077in}}{\pgfqpoint{8.095643in}{1.298073in}}{\pgfqpoint{8.092077in}{1.294506in}}%
\pgfpathcurveto{\pgfqpoint{8.088510in}{1.290940in}}{\pgfqpoint{8.086506in}{1.286102in}}{\pgfqpoint{8.086506in}{1.281059in}}%
\pgfpathcurveto{\pgfqpoint{8.086506in}{1.276015in}}{\pgfqpoint{8.088510in}{1.271177in}}{\pgfqpoint{8.092077in}{1.267611in}}%
\pgfpathcurveto{\pgfqpoint{8.095643in}{1.264044in}}{\pgfqpoint{8.100481in}{1.262040in}}{\pgfqpoint{8.105524in}{1.262040in}}%
\pgfpathclose%
\pgfusepath{fill}%
\end{pgfscope}%
\begin{pgfscope}%
\pgfpathrectangle{\pgfqpoint{6.572727in}{0.474100in}}{\pgfqpoint{4.227273in}{3.318700in}}%
\pgfusepath{clip}%
\pgfsetbuttcap%
\pgfsetroundjoin%
\definecolor{currentfill}{rgb}{0.267004,0.004874,0.329415}%
\pgfsetfillcolor{currentfill}%
\pgfsetfillopacity{0.700000}%
\pgfsetlinewidth{0.000000pt}%
\definecolor{currentstroke}{rgb}{0.000000,0.000000,0.000000}%
\pgfsetstrokecolor{currentstroke}%
\pgfsetstrokeopacity{0.700000}%
\pgfsetdash{}{0pt}%
\pgfpathmoveto{\pgfqpoint{7.643309in}{1.385335in}}%
\pgfpathcurveto{\pgfqpoint{7.648352in}{1.385335in}}{\pgfqpoint{7.653190in}{1.387339in}}{\pgfqpoint{7.656757in}{1.390905in}}%
\pgfpathcurveto{\pgfqpoint{7.660323in}{1.394472in}}{\pgfqpoint{7.662327in}{1.399310in}}{\pgfqpoint{7.662327in}{1.404353in}}%
\pgfpathcurveto{\pgfqpoint{7.662327in}{1.409397in}}{\pgfqpoint{7.660323in}{1.414235in}}{\pgfqpoint{7.656757in}{1.417801in}}%
\pgfpathcurveto{\pgfqpoint{7.653190in}{1.421368in}}{\pgfqpoint{7.648352in}{1.423371in}}{\pgfqpoint{7.643309in}{1.423371in}}%
\pgfpathcurveto{\pgfqpoint{7.638265in}{1.423371in}}{\pgfqpoint{7.633427in}{1.421368in}}{\pgfqpoint{7.629861in}{1.417801in}}%
\pgfpathcurveto{\pgfqpoint{7.626294in}{1.414235in}}{\pgfqpoint{7.624291in}{1.409397in}}{\pgfqpoint{7.624291in}{1.404353in}}%
\pgfpathcurveto{\pgfqpoint{7.624291in}{1.399310in}}{\pgfqpoint{7.626294in}{1.394472in}}{\pgfqpoint{7.629861in}{1.390905in}}%
\pgfpathcurveto{\pgfqpoint{7.633427in}{1.387339in}}{\pgfqpoint{7.638265in}{1.385335in}}{\pgfqpoint{7.643309in}{1.385335in}}%
\pgfpathclose%
\pgfusepath{fill}%
\end{pgfscope}%
\begin{pgfscope}%
\pgfpathrectangle{\pgfqpoint{6.572727in}{0.474100in}}{\pgfqpoint{4.227273in}{3.318700in}}%
\pgfusepath{clip}%
\pgfsetbuttcap%
\pgfsetroundjoin%
\definecolor{currentfill}{rgb}{0.993248,0.906157,0.143936}%
\pgfsetfillcolor{currentfill}%
\pgfsetfillopacity{0.700000}%
\pgfsetlinewidth{0.000000pt}%
\definecolor{currentstroke}{rgb}{0.000000,0.000000,0.000000}%
\pgfsetstrokecolor{currentstroke}%
\pgfsetstrokeopacity{0.700000}%
\pgfsetdash{}{0pt}%
\pgfpathmoveto{\pgfqpoint{7.772328in}{2.354798in}}%
\pgfpathcurveto{\pgfqpoint{7.777372in}{2.354798in}}{\pgfqpoint{7.782210in}{2.356801in}}{\pgfqpoint{7.785776in}{2.360368in}}%
\pgfpathcurveto{\pgfqpoint{7.789343in}{2.363934in}}{\pgfqpoint{7.791347in}{2.368772in}}{\pgfqpoint{7.791347in}{2.373816in}}%
\pgfpathcurveto{\pgfqpoint{7.791347in}{2.378859in}}{\pgfqpoint{7.789343in}{2.383697in}}{\pgfqpoint{7.785776in}{2.387264in}}%
\pgfpathcurveto{\pgfqpoint{7.782210in}{2.390830in}}{\pgfqpoint{7.777372in}{2.392834in}}{\pgfqpoint{7.772328in}{2.392834in}}%
\pgfpathcurveto{\pgfqpoint{7.767285in}{2.392834in}}{\pgfqpoint{7.762447in}{2.390830in}}{\pgfqpoint{7.758881in}{2.387264in}}%
\pgfpathcurveto{\pgfqpoint{7.755314in}{2.383697in}}{\pgfqpoint{7.753310in}{2.378859in}}{\pgfqpoint{7.753310in}{2.373816in}}%
\pgfpathcurveto{\pgfqpoint{7.753310in}{2.368772in}}{\pgfqpoint{7.755314in}{2.363934in}}{\pgfqpoint{7.758881in}{2.360368in}}%
\pgfpathcurveto{\pgfqpoint{7.762447in}{2.356801in}}{\pgfqpoint{7.767285in}{2.354798in}}{\pgfqpoint{7.772328in}{2.354798in}}%
\pgfpathclose%
\pgfusepath{fill}%
\end{pgfscope}%
\begin{pgfscope}%
\pgfpathrectangle{\pgfqpoint{6.572727in}{0.474100in}}{\pgfqpoint{4.227273in}{3.318700in}}%
\pgfusepath{clip}%
\pgfsetbuttcap%
\pgfsetroundjoin%
\definecolor{currentfill}{rgb}{0.993248,0.906157,0.143936}%
\pgfsetfillcolor{currentfill}%
\pgfsetfillopacity{0.700000}%
\pgfsetlinewidth{0.000000pt}%
\definecolor{currentstroke}{rgb}{0.000000,0.000000,0.000000}%
\pgfsetstrokecolor{currentstroke}%
\pgfsetstrokeopacity{0.700000}%
\pgfsetdash{}{0pt}%
\pgfpathmoveto{\pgfqpoint{8.983886in}{2.799000in}}%
\pgfpathcurveto{\pgfqpoint{8.988930in}{2.799000in}}{\pgfqpoint{8.993767in}{2.801004in}}{\pgfqpoint{8.997334in}{2.804571in}}%
\pgfpathcurveto{\pgfqpoint{9.000900in}{2.808137in}}{\pgfqpoint{9.002904in}{2.812975in}}{\pgfqpoint{9.002904in}{2.818019in}}%
\pgfpathcurveto{\pgfqpoint{9.002904in}{2.823062in}}{\pgfqpoint{9.000900in}{2.827900in}}{\pgfqpoint{8.997334in}{2.831467in}}%
\pgfpathcurveto{\pgfqpoint{8.993767in}{2.835033in}}{\pgfqpoint{8.988930in}{2.837037in}}{\pgfqpoint{8.983886in}{2.837037in}}%
\pgfpathcurveto{\pgfqpoint{8.978842in}{2.837037in}}{\pgfqpoint{8.974005in}{2.835033in}}{\pgfqpoint{8.970438in}{2.831467in}}%
\pgfpathcurveto{\pgfqpoint{8.966872in}{2.827900in}}{\pgfqpoint{8.964868in}{2.823062in}}{\pgfqpoint{8.964868in}{2.818019in}}%
\pgfpathcurveto{\pgfqpoint{8.964868in}{2.812975in}}{\pgfqpoint{8.966872in}{2.808137in}}{\pgfqpoint{8.970438in}{2.804571in}}%
\pgfpathcurveto{\pgfqpoint{8.974005in}{2.801004in}}{\pgfqpoint{8.978842in}{2.799000in}}{\pgfqpoint{8.983886in}{2.799000in}}%
\pgfpathclose%
\pgfusepath{fill}%
\end{pgfscope}%
\begin{pgfscope}%
\pgfpathrectangle{\pgfqpoint{6.572727in}{0.474100in}}{\pgfqpoint{4.227273in}{3.318700in}}%
\pgfusepath{clip}%
\pgfsetbuttcap%
\pgfsetroundjoin%
\definecolor{currentfill}{rgb}{0.267004,0.004874,0.329415}%
\pgfsetfillcolor{currentfill}%
\pgfsetfillopacity{0.700000}%
\pgfsetlinewidth{0.000000pt}%
\definecolor{currentstroke}{rgb}{0.000000,0.000000,0.000000}%
\pgfsetstrokecolor{currentstroke}%
\pgfsetstrokeopacity{0.700000}%
\pgfsetdash{}{0pt}%
\pgfpathmoveto{\pgfqpoint{8.129588in}{1.544685in}}%
\pgfpathcurveto{\pgfqpoint{8.134631in}{1.544685in}}{\pgfqpoint{8.139469in}{1.546689in}}{\pgfqpoint{8.143036in}{1.550255in}}%
\pgfpathcurveto{\pgfqpoint{8.146602in}{1.553822in}}{\pgfqpoint{8.148606in}{1.558659in}}{\pgfqpoint{8.148606in}{1.563703in}}%
\pgfpathcurveto{\pgfqpoint{8.148606in}{1.568747in}}{\pgfqpoint{8.146602in}{1.573584in}}{\pgfqpoint{8.143036in}{1.577151in}}%
\pgfpathcurveto{\pgfqpoint{8.139469in}{1.580717in}}{\pgfqpoint{8.134631in}{1.582721in}}{\pgfqpoint{8.129588in}{1.582721in}}%
\pgfpathcurveto{\pgfqpoint{8.124544in}{1.582721in}}{\pgfqpoint{8.119706in}{1.580717in}}{\pgfqpoint{8.116140in}{1.577151in}}%
\pgfpathcurveto{\pgfqpoint{8.112574in}{1.573584in}}{\pgfqpoint{8.110570in}{1.568747in}}{\pgfqpoint{8.110570in}{1.563703in}}%
\pgfpathcurveto{\pgfqpoint{8.110570in}{1.558659in}}{\pgfqpoint{8.112574in}{1.553822in}}{\pgfqpoint{8.116140in}{1.550255in}}%
\pgfpathcurveto{\pgfqpoint{8.119706in}{1.546689in}}{\pgfqpoint{8.124544in}{1.544685in}}{\pgfqpoint{8.129588in}{1.544685in}}%
\pgfpathclose%
\pgfusepath{fill}%
\end{pgfscope}%
\begin{pgfscope}%
\pgfpathrectangle{\pgfqpoint{6.572727in}{0.474100in}}{\pgfqpoint{4.227273in}{3.318700in}}%
\pgfusepath{clip}%
\pgfsetbuttcap%
\pgfsetroundjoin%
\definecolor{currentfill}{rgb}{0.993248,0.906157,0.143936}%
\pgfsetfillcolor{currentfill}%
\pgfsetfillopacity{0.700000}%
\pgfsetlinewidth{0.000000pt}%
\definecolor{currentstroke}{rgb}{0.000000,0.000000,0.000000}%
\pgfsetstrokecolor{currentstroke}%
\pgfsetstrokeopacity{0.700000}%
\pgfsetdash{}{0pt}%
\pgfpathmoveto{\pgfqpoint{8.301748in}{2.940765in}}%
\pgfpathcurveto{\pgfqpoint{8.306792in}{2.940765in}}{\pgfqpoint{8.311630in}{2.942769in}}{\pgfqpoint{8.315196in}{2.946336in}}%
\pgfpathcurveto{\pgfqpoint{8.318762in}{2.949902in}}{\pgfqpoint{8.320766in}{2.954740in}}{\pgfqpoint{8.320766in}{2.959784in}}%
\pgfpathcurveto{\pgfqpoint{8.320766in}{2.964827in}}{\pgfqpoint{8.318762in}{2.969665in}}{\pgfqpoint{8.315196in}{2.973231in}}%
\pgfpathcurveto{\pgfqpoint{8.311630in}{2.976798in}}{\pgfqpoint{8.306792in}{2.978802in}}{\pgfqpoint{8.301748in}{2.978802in}}%
\pgfpathcurveto{\pgfqpoint{8.296704in}{2.978802in}}{\pgfqpoint{8.291867in}{2.976798in}}{\pgfqpoint{8.288300in}{2.973231in}}%
\pgfpathcurveto{\pgfqpoint{8.284734in}{2.969665in}}{\pgfqpoint{8.282730in}{2.964827in}}{\pgfqpoint{8.282730in}{2.959784in}}%
\pgfpathcurveto{\pgfqpoint{8.282730in}{2.954740in}}{\pgfqpoint{8.284734in}{2.949902in}}{\pgfqpoint{8.288300in}{2.946336in}}%
\pgfpathcurveto{\pgfqpoint{8.291867in}{2.942769in}}{\pgfqpoint{8.296704in}{2.940765in}}{\pgfqpoint{8.301748in}{2.940765in}}%
\pgfpathclose%
\pgfusepath{fill}%
\end{pgfscope}%
\begin{pgfscope}%
\pgfpathrectangle{\pgfqpoint{6.572727in}{0.474100in}}{\pgfqpoint{4.227273in}{3.318700in}}%
\pgfusepath{clip}%
\pgfsetbuttcap%
\pgfsetroundjoin%
\definecolor{currentfill}{rgb}{0.127568,0.566949,0.550556}%
\pgfsetfillcolor{currentfill}%
\pgfsetfillopacity{0.700000}%
\pgfsetlinewidth{0.000000pt}%
\definecolor{currentstroke}{rgb}{0.000000,0.000000,0.000000}%
\pgfsetstrokecolor{currentstroke}%
\pgfsetstrokeopacity{0.700000}%
\pgfsetdash{}{0pt}%
\pgfpathmoveto{\pgfqpoint{9.998091in}{1.794103in}}%
\pgfpathcurveto{\pgfqpoint{10.003135in}{1.794103in}}{\pgfqpoint{10.007973in}{1.796107in}}{\pgfqpoint{10.011539in}{1.799673in}}%
\pgfpathcurveto{\pgfqpoint{10.015106in}{1.803240in}}{\pgfqpoint{10.017110in}{1.808077in}}{\pgfqpoint{10.017110in}{1.813121in}}%
\pgfpathcurveto{\pgfqpoint{10.017110in}{1.818165in}}{\pgfqpoint{10.015106in}{1.823002in}}{\pgfqpoint{10.011539in}{1.826569in}}%
\pgfpathcurveto{\pgfqpoint{10.007973in}{1.830135in}}{\pgfqpoint{10.003135in}{1.832139in}}{\pgfqpoint{9.998091in}{1.832139in}}%
\pgfpathcurveto{\pgfqpoint{9.993048in}{1.832139in}}{\pgfqpoint{9.988210in}{1.830135in}}{\pgfqpoint{9.984644in}{1.826569in}}%
\pgfpathcurveto{\pgfqpoint{9.981077in}{1.823002in}}{\pgfqpoint{9.979073in}{1.818165in}}{\pgfqpoint{9.979073in}{1.813121in}}%
\pgfpathcurveto{\pgfqpoint{9.979073in}{1.808077in}}{\pgfqpoint{9.981077in}{1.803240in}}{\pgfqpoint{9.984644in}{1.799673in}}%
\pgfpathcurveto{\pgfqpoint{9.988210in}{1.796107in}}{\pgfqpoint{9.993048in}{1.794103in}}{\pgfqpoint{9.998091in}{1.794103in}}%
\pgfpathclose%
\pgfusepath{fill}%
\end{pgfscope}%
\begin{pgfscope}%
\pgfpathrectangle{\pgfqpoint{6.572727in}{0.474100in}}{\pgfqpoint{4.227273in}{3.318700in}}%
\pgfusepath{clip}%
\pgfsetbuttcap%
\pgfsetroundjoin%
\definecolor{currentfill}{rgb}{0.127568,0.566949,0.550556}%
\pgfsetfillcolor{currentfill}%
\pgfsetfillopacity{0.700000}%
\pgfsetlinewidth{0.000000pt}%
\definecolor{currentstroke}{rgb}{0.000000,0.000000,0.000000}%
\pgfsetstrokecolor{currentstroke}%
\pgfsetstrokeopacity{0.700000}%
\pgfsetdash{}{0pt}%
\pgfpathmoveto{\pgfqpoint{9.226196in}{1.518150in}}%
\pgfpathcurveto{\pgfqpoint{9.231239in}{1.518150in}}{\pgfqpoint{9.236077in}{1.520154in}}{\pgfqpoint{9.239644in}{1.523720in}}%
\pgfpathcurveto{\pgfqpoint{9.243210in}{1.527287in}}{\pgfqpoint{9.245214in}{1.532124in}}{\pgfqpoint{9.245214in}{1.537168in}}%
\pgfpathcurveto{\pgfqpoint{9.245214in}{1.542212in}}{\pgfqpoint{9.243210in}{1.547049in}}{\pgfqpoint{9.239644in}{1.550616in}}%
\pgfpathcurveto{\pgfqpoint{9.236077in}{1.554182in}}{\pgfqpoint{9.231239in}{1.556186in}}{\pgfqpoint{9.226196in}{1.556186in}}%
\pgfpathcurveto{\pgfqpoint{9.221152in}{1.556186in}}{\pgfqpoint{9.216314in}{1.554182in}}{\pgfqpoint{9.212748in}{1.550616in}}%
\pgfpathcurveto{\pgfqpoint{9.209181in}{1.547049in}}{\pgfqpoint{9.207178in}{1.542212in}}{\pgfqpoint{9.207178in}{1.537168in}}%
\pgfpathcurveto{\pgfqpoint{9.207178in}{1.532124in}}{\pgfqpoint{9.209181in}{1.527287in}}{\pgfqpoint{9.212748in}{1.523720in}}%
\pgfpathcurveto{\pgfqpoint{9.216314in}{1.520154in}}{\pgfqpoint{9.221152in}{1.518150in}}{\pgfqpoint{9.226196in}{1.518150in}}%
\pgfpathclose%
\pgfusepath{fill}%
\end{pgfscope}%
\begin{pgfscope}%
\pgfpathrectangle{\pgfqpoint{6.572727in}{0.474100in}}{\pgfqpoint{4.227273in}{3.318700in}}%
\pgfusepath{clip}%
\pgfsetbuttcap%
\pgfsetroundjoin%
\definecolor{currentfill}{rgb}{0.267004,0.004874,0.329415}%
\pgfsetfillcolor{currentfill}%
\pgfsetfillopacity{0.700000}%
\pgfsetlinewidth{0.000000pt}%
\definecolor{currentstroke}{rgb}{0.000000,0.000000,0.000000}%
\pgfsetstrokecolor{currentstroke}%
\pgfsetstrokeopacity{0.700000}%
\pgfsetdash{}{0pt}%
\pgfpathmoveto{\pgfqpoint{7.568149in}{1.933509in}}%
\pgfpathcurveto{\pgfqpoint{7.573193in}{1.933509in}}{\pgfqpoint{7.578031in}{1.935513in}}{\pgfqpoint{7.581597in}{1.939079in}}%
\pgfpathcurveto{\pgfqpoint{7.585164in}{1.942646in}}{\pgfqpoint{7.587168in}{1.947484in}}{\pgfqpoint{7.587168in}{1.952527in}}%
\pgfpathcurveto{\pgfqpoint{7.587168in}{1.957571in}}{\pgfqpoint{7.585164in}{1.962409in}}{\pgfqpoint{7.581597in}{1.965975in}}%
\pgfpathcurveto{\pgfqpoint{7.578031in}{1.969542in}}{\pgfqpoint{7.573193in}{1.971545in}}{\pgfqpoint{7.568149in}{1.971545in}}%
\pgfpathcurveto{\pgfqpoint{7.563106in}{1.971545in}}{\pgfqpoint{7.558268in}{1.969542in}}{\pgfqpoint{7.554702in}{1.965975in}}%
\pgfpathcurveto{\pgfqpoint{7.551135in}{1.962409in}}{\pgfqpoint{7.549131in}{1.957571in}}{\pgfqpoint{7.549131in}{1.952527in}}%
\pgfpathcurveto{\pgfqpoint{7.549131in}{1.947484in}}{\pgfqpoint{7.551135in}{1.942646in}}{\pgfqpoint{7.554702in}{1.939079in}}%
\pgfpathcurveto{\pgfqpoint{7.558268in}{1.935513in}}{\pgfqpoint{7.563106in}{1.933509in}}{\pgfqpoint{7.568149in}{1.933509in}}%
\pgfpathclose%
\pgfusepath{fill}%
\end{pgfscope}%
\begin{pgfscope}%
\pgfpathrectangle{\pgfqpoint{6.572727in}{0.474100in}}{\pgfqpoint{4.227273in}{3.318700in}}%
\pgfusepath{clip}%
\pgfsetbuttcap%
\pgfsetroundjoin%
\definecolor{currentfill}{rgb}{0.267004,0.004874,0.329415}%
\pgfsetfillcolor{currentfill}%
\pgfsetfillopacity{0.700000}%
\pgfsetlinewidth{0.000000pt}%
\definecolor{currentstroke}{rgb}{0.000000,0.000000,0.000000}%
\pgfsetstrokecolor{currentstroke}%
\pgfsetstrokeopacity{0.700000}%
\pgfsetdash{}{0pt}%
\pgfpathmoveto{\pgfqpoint{7.540108in}{2.120459in}}%
\pgfpathcurveto{\pgfqpoint{7.545151in}{2.120459in}}{\pgfqpoint{7.549989in}{2.122462in}}{\pgfqpoint{7.553556in}{2.126029in}}%
\pgfpathcurveto{\pgfqpoint{7.557122in}{2.129595in}}{\pgfqpoint{7.559126in}{2.134433in}}{\pgfqpoint{7.559126in}{2.139477in}}%
\pgfpathcurveto{\pgfqpoint{7.559126in}{2.144520in}}{\pgfqpoint{7.557122in}{2.149358in}}{\pgfqpoint{7.553556in}{2.152925in}}%
\pgfpathcurveto{\pgfqpoint{7.549989in}{2.156491in}}{\pgfqpoint{7.545151in}{2.158495in}}{\pgfqpoint{7.540108in}{2.158495in}}%
\pgfpathcurveto{\pgfqpoint{7.535064in}{2.158495in}}{\pgfqpoint{7.530226in}{2.156491in}}{\pgfqpoint{7.526660in}{2.152925in}}%
\pgfpathcurveto{\pgfqpoint{7.523093in}{2.149358in}}{\pgfqpoint{7.521090in}{2.144520in}}{\pgfqpoint{7.521090in}{2.139477in}}%
\pgfpathcurveto{\pgfqpoint{7.521090in}{2.134433in}}{\pgfqpoint{7.523093in}{2.129595in}}{\pgfqpoint{7.526660in}{2.126029in}}%
\pgfpathcurveto{\pgfqpoint{7.530226in}{2.122462in}}{\pgfqpoint{7.535064in}{2.120459in}}{\pgfqpoint{7.540108in}{2.120459in}}%
\pgfpathclose%
\pgfusepath{fill}%
\end{pgfscope}%
\begin{pgfscope}%
\pgfpathrectangle{\pgfqpoint{6.572727in}{0.474100in}}{\pgfqpoint{4.227273in}{3.318700in}}%
\pgfusepath{clip}%
\pgfsetbuttcap%
\pgfsetroundjoin%
\definecolor{currentfill}{rgb}{0.127568,0.566949,0.550556}%
\pgfsetfillcolor{currentfill}%
\pgfsetfillopacity{0.700000}%
\pgfsetlinewidth{0.000000pt}%
\definecolor{currentstroke}{rgb}{0.000000,0.000000,0.000000}%
\pgfsetstrokecolor{currentstroke}%
\pgfsetstrokeopacity{0.700000}%
\pgfsetdash{}{0pt}%
\pgfpathmoveto{\pgfqpoint{9.661492in}{1.525717in}}%
\pgfpathcurveto{\pgfqpoint{9.666535in}{1.525717in}}{\pgfqpoint{9.671373in}{1.527721in}}{\pgfqpoint{9.674939in}{1.531287in}}%
\pgfpathcurveto{\pgfqpoint{9.678506in}{1.534854in}}{\pgfqpoint{9.680510in}{1.539691in}}{\pgfqpoint{9.680510in}{1.544735in}}%
\pgfpathcurveto{\pgfqpoint{9.680510in}{1.549779in}}{\pgfqpoint{9.678506in}{1.554617in}}{\pgfqpoint{9.674939in}{1.558183in}}%
\pgfpathcurveto{\pgfqpoint{9.671373in}{1.561749in}}{\pgfqpoint{9.666535in}{1.563753in}}{\pgfqpoint{9.661492in}{1.563753in}}%
\pgfpathcurveto{\pgfqpoint{9.656448in}{1.563753in}}{\pgfqpoint{9.651610in}{1.561749in}}{\pgfqpoint{9.648044in}{1.558183in}}%
\pgfpathcurveto{\pgfqpoint{9.644477in}{1.554617in}}{\pgfqpoint{9.642473in}{1.549779in}}{\pgfqpoint{9.642473in}{1.544735in}}%
\pgfpathcurveto{\pgfqpoint{9.642473in}{1.539691in}}{\pgfqpoint{9.644477in}{1.534854in}}{\pgfqpoint{9.648044in}{1.531287in}}%
\pgfpathcurveto{\pgfqpoint{9.651610in}{1.527721in}}{\pgfqpoint{9.656448in}{1.525717in}}{\pgfqpoint{9.661492in}{1.525717in}}%
\pgfpathclose%
\pgfusepath{fill}%
\end{pgfscope}%
\begin{pgfscope}%
\pgfpathrectangle{\pgfqpoint{6.572727in}{0.474100in}}{\pgfqpoint{4.227273in}{3.318700in}}%
\pgfusepath{clip}%
\pgfsetbuttcap%
\pgfsetroundjoin%
\definecolor{currentfill}{rgb}{0.127568,0.566949,0.550556}%
\pgfsetfillcolor{currentfill}%
\pgfsetfillopacity{0.700000}%
\pgfsetlinewidth{0.000000pt}%
\definecolor{currentstroke}{rgb}{0.000000,0.000000,0.000000}%
\pgfsetstrokecolor{currentstroke}%
\pgfsetstrokeopacity{0.700000}%
\pgfsetdash{}{0pt}%
\pgfpathmoveto{\pgfqpoint{9.978806in}{1.311540in}}%
\pgfpathcurveto{\pgfqpoint{9.983850in}{1.311540in}}{\pgfqpoint{9.988688in}{1.313544in}}{\pgfqpoint{9.992254in}{1.317110in}}%
\pgfpathcurveto{\pgfqpoint{9.995821in}{1.320677in}}{\pgfqpoint{9.997825in}{1.325515in}}{\pgfqpoint{9.997825in}{1.330558in}}%
\pgfpathcurveto{\pgfqpoint{9.997825in}{1.335602in}}{\pgfqpoint{9.995821in}{1.340440in}}{\pgfqpoint{9.992254in}{1.344006in}}%
\pgfpathcurveto{\pgfqpoint{9.988688in}{1.347573in}}{\pgfqpoint{9.983850in}{1.349576in}}{\pgfqpoint{9.978806in}{1.349576in}}%
\pgfpathcurveto{\pgfqpoint{9.973763in}{1.349576in}}{\pgfqpoint{9.968925in}{1.347573in}}{\pgfqpoint{9.965359in}{1.344006in}}%
\pgfpathcurveto{\pgfqpoint{9.961792in}{1.340440in}}{\pgfqpoint{9.959788in}{1.335602in}}{\pgfqpoint{9.959788in}{1.330558in}}%
\pgfpathcurveto{\pgfqpoint{9.959788in}{1.325515in}}{\pgfqpoint{9.961792in}{1.320677in}}{\pgfqpoint{9.965359in}{1.317110in}}%
\pgfpathcurveto{\pgfqpoint{9.968925in}{1.313544in}}{\pgfqpoint{9.973763in}{1.311540in}}{\pgfqpoint{9.978806in}{1.311540in}}%
\pgfpathclose%
\pgfusepath{fill}%
\end{pgfscope}%
\begin{pgfscope}%
\pgfpathrectangle{\pgfqpoint{6.572727in}{0.474100in}}{\pgfqpoint{4.227273in}{3.318700in}}%
\pgfusepath{clip}%
\pgfsetbuttcap%
\pgfsetroundjoin%
\definecolor{currentfill}{rgb}{0.993248,0.906157,0.143936}%
\pgfsetfillcolor{currentfill}%
\pgfsetfillopacity{0.700000}%
\pgfsetlinewidth{0.000000pt}%
\definecolor{currentstroke}{rgb}{0.000000,0.000000,0.000000}%
\pgfsetstrokecolor{currentstroke}%
\pgfsetstrokeopacity{0.700000}%
\pgfsetdash{}{0pt}%
\pgfpathmoveto{\pgfqpoint{8.092817in}{2.563308in}}%
\pgfpathcurveto{\pgfqpoint{8.097861in}{2.563308in}}{\pgfqpoint{8.102698in}{2.565312in}}{\pgfqpoint{8.106265in}{2.568878in}}%
\pgfpathcurveto{\pgfqpoint{8.109831in}{2.572445in}}{\pgfqpoint{8.111835in}{2.577283in}}{\pgfqpoint{8.111835in}{2.582326in}}%
\pgfpathcurveto{\pgfqpoint{8.111835in}{2.587370in}}{\pgfqpoint{8.109831in}{2.592208in}}{\pgfqpoint{8.106265in}{2.595774in}}%
\pgfpathcurveto{\pgfqpoint{8.102698in}{2.599340in}}{\pgfqpoint{8.097861in}{2.601344in}}{\pgfqpoint{8.092817in}{2.601344in}}%
\pgfpathcurveto{\pgfqpoint{8.087773in}{2.601344in}}{\pgfqpoint{8.082936in}{2.599340in}}{\pgfqpoint{8.079369in}{2.595774in}}%
\pgfpathcurveto{\pgfqpoint{8.075803in}{2.592208in}}{\pgfqpoint{8.073799in}{2.587370in}}{\pgfqpoint{8.073799in}{2.582326in}}%
\pgfpathcurveto{\pgfqpoint{8.073799in}{2.577283in}}{\pgfqpoint{8.075803in}{2.572445in}}{\pgfqpoint{8.079369in}{2.568878in}}%
\pgfpathcurveto{\pgfqpoint{8.082936in}{2.565312in}}{\pgfqpoint{8.087773in}{2.563308in}}{\pgfqpoint{8.092817in}{2.563308in}}%
\pgfpathclose%
\pgfusepath{fill}%
\end{pgfscope}%
\begin{pgfscope}%
\pgfpathrectangle{\pgfqpoint{6.572727in}{0.474100in}}{\pgfqpoint{4.227273in}{3.318700in}}%
\pgfusepath{clip}%
\pgfsetbuttcap%
\pgfsetroundjoin%
\definecolor{currentfill}{rgb}{0.267004,0.004874,0.329415}%
\pgfsetfillcolor{currentfill}%
\pgfsetfillopacity{0.700000}%
\pgfsetlinewidth{0.000000pt}%
\definecolor{currentstroke}{rgb}{0.000000,0.000000,0.000000}%
\pgfsetstrokecolor{currentstroke}%
\pgfsetstrokeopacity{0.700000}%
\pgfsetdash{}{0pt}%
\pgfpathmoveto{\pgfqpoint{7.165098in}{1.518740in}}%
\pgfpathcurveto{\pgfqpoint{7.170142in}{1.518740in}}{\pgfqpoint{7.174980in}{1.520744in}}{\pgfqpoint{7.178546in}{1.524310in}}%
\pgfpathcurveto{\pgfqpoint{7.182113in}{1.527877in}}{\pgfqpoint{7.184116in}{1.532714in}}{\pgfqpoint{7.184116in}{1.537758in}}%
\pgfpathcurveto{\pgfqpoint{7.184116in}{1.542802in}}{\pgfqpoint{7.182113in}{1.547639in}}{\pgfqpoint{7.178546in}{1.551206in}}%
\pgfpathcurveto{\pgfqpoint{7.174980in}{1.554772in}}{\pgfqpoint{7.170142in}{1.556776in}}{\pgfqpoint{7.165098in}{1.556776in}}%
\pgfpathcurveto{\pgfqpoint{7.160055in}{1.556776in}}{\pgfqpoint{7.155217in}{1.554772in}}{\pgfqpoint{7.151650in}{1.551206in}}%
\pgfpathcurveto{\pgfqpoint{7.148084in}{1.547639in}}{\pgfqpoint{7.146080in}{1.542802in}}{\pgfqpoint{7.146080in}{1.537758in}}%
\pgfpathcurveto{\pgfqpoint{7.146080in}{1.532714in}}{\pgfqpoint{7.148084in}{1.527877in}}{\pgfqpoint{7.151650in}{1.524310in}}%
\pgfpathcurveto{\pgfqpoint{7.155217in}{1.520744in}}{\pgfqpoint{7.160055in}{1.518740in}}{\pgfqpoint{7.165098in}{1.518740in}}%
\pgfpathclose%
\pgfusepath{fill}%
\end{pgfscope}%
\begin{pgfscope}%
\pgfpathrectangle{\pgfqpoint{6.572727in}{0.474100in}}{\pgfqpoint{4.227273in}{3.318700in}}%
\pgfusepath{clip}%
\pgfsetbuttcap%
\pgfsetroundjoin%
\definecolor{currentfill}{rgb}{0.267004,0.004874,0.329415}%
\pgfsetfillcolor{currentfill}%
\pgfsetfillopacity{0.700000}%
\pgfsetlinewidth{0.000000pt}%
\definecolor{currentstroke}{rgb}{0.000000,0.000000,0.000000}%
\pgfsetstrokecolor{currentstroke}%
\pgfsetstrokeopacity{0.700000}%
\pgfsetdash{}{0pt}%
\pgfpathmoveto{\pgfqpoint{7.917579in}{1.507975in}}%
\pgfpathcurveto{\pgfqpoint{7.922623in}{1.507975in}}{\pgfqpoint{7.927461in}{1.509979in}}{\pgfqpoint{7.931027in}{1.513546in}}%
\pgfpathcurveto{\pgfqpoint{7.934593in}{1.517112in}}{\pgfqpoint{7.936597in}{1.521950in}}{\pgfqpoint{7.936597in}{1.526994in}}%
\pgfpathcurveto{\pgfqpoint{7.936597in}{1.532037in}}{\pgfqpoint{7.934593in}{1.536875in}}{\pgfqpoint{7.931027in}{1.540441in}}%
\pgfpathcurveto{\pgfqpoint{7.927461in}{1.544008in}}{\pgfqpoint{7.922623in}{1.546012in}}{\pgfqpoint{7.917579in}{1.546012in}}%
\pgfpathcurveto{\pgfqpoint{7.912536in}{1.546012in}}{\pgfqpoint{7.907698in}{1.544008in}}{\pgfqpoint{7.904131in}{1.540441in}}%
\pgfpathcurveto{\pgfqpoint{7.900565in}{1.536875in}}{\pgfqpoint{7.898561in}{1.532037in}}{\pgfqpoint{7.898561in}{1.526994in}}%
\pgfpathcurveto{\pgfqpoint{7.898561in}{1.521950in}}{\pgfqpoint{7.900565in}{1.517112in}}{\pgfqpoint{7.904131in}{1.513546in}}%
\pgfpathcurveto{\pgfqpoint{7.907698in}{1.509979in}}{\pgfqpoint{7.912536in}{1.507975in}}{\pgfqpoint{7.917579in}{1.507975in}}%
\pgfpathclose%
\pgfusepath{fill}%
\end{pgfscope}%
\begin{pgfscope}%
\pgfpathrectangle{\pgfqpoint{6.572727in}{0.474100in}}{\pgfqpoint{4.227273in}{3.318700in}}%
\pgfusepath{clip}%
\pgfsetbuttcap%
\pgfsetroundjoin%
\definecolor{currentfill}{rgb}{0.267004,0.004874,0.329415}%
\pgfsetfillcolor{currentfill}%
\pgfsetfillopacity{0.700000}%
\pgfsetlinewidth{0.000000pt}%
\definecolor{currentstroke}{rgb}{0.000000,0.000000,0.000000}%
\pgfsetstrokecolor{currentstroke}%
\pgfsetstrokeopacity{0.700000}%
\pgfsetdash{}{0pt}%
\pgfpathmoveto{\pgfqpoint{7.869032in}{1.215507in}}%
\pgfpathcurveto{\pgfqpoint{7.874076in}{1.215507in}}{\pgfqpoint{7.878914in}{1.217511in}}{\pgfqpoint{7.882480in}{1.221077in}}%
\pgfpathcurveto{\pgfqpoint{7.886046in}{1.224643in}}{\pgfqpoint{7.888050in}{1.229481in}}{\pgfqpoint{7.888050in}{1.234525in}}%
\pgfpathcurveto{\pgfqpoint{7.888050in}{1.239569in}}{\pgfqpoint{7.886046in}{1.244406in}}{\pgfqpoint{7.882480in}{1.247973in}}%
\pgfpathcurveto{\pgfqpoint{7.878914in}{1.251539in}}{\pgfqpoint{7.874076in}{1.253543in}}{\pgfqpoint{7.869032in}{1.253543in}}%
\pgfpathcurveto{\pgfqpoint{7.863988in}{1.253543in}}{\pgfqpoint{7.859151in}{1.251539in}}{\pgfqpoint{7.855584in}{1.247973in}}%
\pgfpathcurveto{\pgfqpoint{7.852018in}{1.244406in}}{\pgfqpoint{7.850014in}{1.239569in}}{\pgfqpoint{7.850014in}{1.234525in}}%
\pgfpathcurveto{\pgfqpoint{7.850014in}{1.229481in}}{\pgfqpoint{7.852018in}{1.224643in}}{\pgfqpoint{7.855584in}{1.221077in}}%
\pgfpathcurveto{\pgfqpoint{7.859151in}{1.217511in}}{\pgfqpoint{7.863988in}{1.215507in}}{\pgfqpoint{7.869032in}{1.215507in}}%
\pgfpathclose%
\pgfusepath{fill}%
\end{pgfscope}%
\begin{pgfscope}%
\pgfpathrectangle{\pgfqpoint{6.572727in}{0.474100in}}{\pgfqpoint{4.227273in}{3.318700in}}%
\pgfusepath{clip}%
\pgfsetbuttcap%
\pgfsetroundjoin%
\definecolor{currentfill}{rgb}{0.127568,0.566949,0.550556}%
\pgfsetfillcolor{currentfill}%
\pgfsetfillopacity{0.700000}%
\pgfsetlinewidth{0.000000pt}%
\definecolor{currentstroke}{rgb}{0.000000,0.000000,0.000000}%
\pgfsetstrokecolor{currentstroke}%
\pgfsetstrokeopacity{0.700000}%
\pgfsetdash{}{0pt}%
\pgfpathmoveto{\pgfqpoint{9.679658in}{1.905726in}}%
\pgfpathcurveto{\pgfqpoint{9.684702in}{1.905726in}}{\pgfqpoint{9.689540in}{1.907730in}}{\pgfqpoint{9.693106in}{1.911297in}}%
\pgfpathcurveto{\pgfqpoint{9.696672in}{1.914863in}}{\pgfqpoint{9.698676in}{1.919701in}}{\pgfqpoint{9.698676in}{1.924745in}}%
\pgfpathcurveto{\pgfqpoint{9.698676in}{1.929788in}}{\pgfqpoint{9.696672in}{1.934626in}}{\pgfqpoint{9.693106in}{1.938192in}}%
\pgfpathcurveto{\pgfqpoint{9.689540in}{1.941759in}}{\pgfqpoint{9.684702in}{1.943763in}}{\pgfqpoint{9.679658in}{1.943763in}}%
\pgfpathcurveto{\pgfqpoint{9.674614in}{1.943763in}}{\pgfqpoint{9.669777in}{1.941759in}}{\pgfqpoint{9.666210in}{1.938192in}}%
\pgfpathcurveto{\pgfqpoint{9.662644in}{1.934626in}}{\pgfqpoint{9.660640in}{1.929788in}}{\pgfqpoint{9.660640in}{1.924745in}}%
\pgfpathcurveto{\pgfqpoint{9.660640in}{1.919701in}}{\pgfqpoint{9.662644in}{1.914863in}}{\pgfqpoint{9.666210in}{1.911297in}}%
\pgfpathcurveto{\pgfqpoint{9.669777in}{1.907730in}}{\pgfqpoint{9.674614in}{1.905726in}}{\pgfqpoint{9.679658in}{1.905726in}}%
\pgfpathclose%
\pgfusepath{fill}%
\end{pgfscope}%
\begin{pgfscope}%
\pgfpathrectangle{\pgfqpoint{6.572727in}{0.474100in}}{\pgfqpoint{4.227273in}{3.318700in}}%
\pgfusepath{clip}%
\pgfsetbuttcap%
\pgfsetroundjoin%
\definecolor{currentfill}{rgb}{0.993248,0.906157,0.143936}%
\pgfsetfillcolor{currentfill}%
\pgfsetfillopacity{0.700000}%
\pgfsetlinewidth{0.000000pt}%
\definecolor{currentstroke}{rgb}{0.000000,0.000000,0.000000}%
\pgfsetstrokecolor{currentstroke}%
\pgfsetstrokeopacity{0.700000}%
\pgfsetdash{}{0pt}%
\pgfpathmoveto{\pgfqpoint{7.533285in}{2.877259in}}%
\pgfpathcurveto{\pgfqpoint{7.538329in}{2.877259in}}{\pgfqpoint{7.543166in}{2.879263in}}{\pgfqpoint{7.546733in}{2.882829in}}%
\pgfpathcurveto{\pgfqpoint{7.550299in}{2.886395in}}{\pgfqpoint{7.552303in}{2.891233in}}{\pgfqpoint{7.552303in}{2.896277in}}%
\pgfpathcurveto{\pgfqpoint{7.552303in}{2.901320in}}{\pgfqpoint{7.550299in}{2.906158in}}{\pgfqpoint{7.546733in}{2.909725in}}%
\pgfpathcurveto{\pgfqpoint{7.543166in}{2.913291in}}{\pgfqpoint{7.538329in}{2.915295in}}{\pgfqpoint{7.533285in}{2.915295in}}%
\pgfpathcurveto{\pgfqpoint{7.528241in}{2.915295in}}{\pgfqpoint{7.523403in}{2.913291in}}{\pgfqpoint{7.519837in}{2.909725in}}%
\pgfpathcurveto{\pgfqpoint{7.516271in}{2.906158in}}{\pgfqpoint{7.514267in}{2.901320in}}{\pgfqpoint{7.514267in}{2.896277in}}%
\pgfpathcurveto{\pgfqpoint{7.514267in}{2.891233in}}{\pgfqpoint{7.516271in}{2.886395in}}{\pgfqpoint{7.519837in}{2.882829in}}%
\pgfpathcurveto{\pgfqpoint{7.523403in}{2.879263in}}{\pgfqpoint{7.528241in}{2.877259in}}{\pgfqpoint{7.533285in}{2.877259in}}%
\pgfpathclose%
\pgfusepath{fill}%
\end{pgfscope}%
\begin{pgfscope}%
\pgfpathrectangle{\pgfqpoint{6.572727in}{0.474100in}}{\pgfqpoint{4.227273in}{3.318700in}}%
\pgfusepath{clip}%
\pgfsetbuttcap%
\pgfsetroundjoin%
\definecolor{currentfill}{rgb}{0.267004,0.004874,0.329415}%
\pgfsetfillcolor{currentfill}%
\pgfsetfillopacity{0.700000}%
\pgfsetlinewidth{0.000000pt}%
\definecolor{currentstroke}{rgb}{0.000000,0.000000,0.000000}%
\pgfsetstrokecolor{currentstroke}%
\pgfsetstrokeopacity{0.700000}%
\pgfsetdash{}{0pt}%
\pgfpathmoveto{\pgfqpoint{7.858219in}{1.919263in}}%
\pgfpathcurveto{\pgfqpoint{7.863263in}{1.919263in}}{\pgfqpoint{7.868101in}{1.921267in}}{\pgfqpoint{7.871667in}{1.924833in}}%
\pgfpathcurveto{\pgfqpoint{7.875233in}{1.928400in}}{\pgfqpoint{7.877237in}{1.933237in}}{\pgfqpoint{7.877237in}{1.938281in}}%
\pgfpathcurveto{\pgfqpoint{7.877237in}{1.943325in}}{\pgfqpoint{7.875233in}{1.948162in}}{\pgfqpoint{7.871667in}{1.951729in}}%
\pgfpathcurveto{\pgfqpoint{7.868101in}{1.955295in}}{\pgfqpoint{7.863263in}{1.957299in}}{\pgfqpoint{7.858219in}{1.957299in}}%
\pgfpathcurveto{\pgfqpoint{7.853175in}{1.957299in}}{\pgfqpoint{7.848338in}{1.955295in}}{\pgfqpoint{7.844771in}{1.951729in}}%
\pgfpathcurveto{\pgfqpoint{7.841205in}{1.948162in}}{\pgfqpoint{7.839201in}{1.943325in}}{\pgfqpoint{7.839201in}{1.938281in}}%
\pgfpathcurveto{\pgfqpoint{7.839201in}{1.933237in}}{\pgfqpoint{7.841205in}{1.928400in}}{\pgfqpoint{7.844771in}{1.924833in}}%
\pgfpathcurveto{\pgfqpoint{7.848338in}{1.921267in}}{\pgfqpoint{7.853175in}{1.919263in}}{\pgfqpoint{7.858219in}{1.919263in}}%
\pgfpathclose%
\pgfusepath{fill}%
\end{pgfscope}%
\begin{pgfscope}%
\pgfpathrectangle{\pgfqpoint{6.572727in}{0.474100in}}{\pgfqpoint{4.227273in}{3.318700in}}%
\pgfusepath{clip}%
\pgfsetbuttcap%
\pgfsetroundjoin%
\definecolor{currentfill}{rgb}{0.993248,0.906157,0.143936}%
\pgfsetfillcolor{currentfill}%
\pgfsetfillopacity{0.700000}%
\pgfsetlinewidth{0.000000pt}%
\definecolor{currentstroke}{rgb}{0.000000,0.000000,0.000000}%
\pgfsetstrokecolor{currentstroke}%
\pgfsetstrokeopacity{0.700000}%
\pgfsetdash{}{0pt}%
\pgfpathmoveto{\pgfqpoint{8.060040in}{3.622932in}}%
\pgfpathcurveto{\pgfqpoint{8.065083in}{3.622932in}}{\pgfqpoint{8.069921in}{3.624936in}}{\pgfqpoint{8.073487in}{3.628502in}}%
\pgfpathcurveto{\pgfqpoint{8.077054in}{3.632069in}}{\pgfqpoint{8.079058in}{3.636906in}}{\pgfqpoint{8.079058in}{3.641950in}}%
\pgfpathcurveto{\pgfqpoint{8.079058in}{3.646994in}}{\pgfqpoint{8.077054in}{3.651831in}}{\pgfqpoint{8.073487in}{3.655398in}}%
\pgfpathcurveto{\pgfqpoint{8.069921in}{3.658964in}}{\pgfqpoint{8.065083in}{3.660968in}}{\pgfqpoint{8.060040in}{3.660968in}}%
\pgfpathcurveto{\pgfqpoint{8.054996in}{3.660968in}}{\pgfqpoint{8.050158in}{3.658964in}}{\pgfqpoint{8.046592in}{3.655398in}}%
\pgfpathcurveto{\pgfqpoint{8.043025in}{3.651831in}}{\pgfqpoint{8.041021in}{3.646994in}}{\pgfqpoint{8.041021in}{3.641950in}}%
\pgfpathcurveto{\pgfqpoint{8.041021in}{3.636906in}}{\pgfqpoint{8.043025in}{3.632069in}}{\pgfqpoint{8.046592in}{3.628502in}}%
\pgfpathcurveto{\pgfqpoint{8.050158in}{3.624936in}}{\pgfqpoint{8.054996in}{3.622932in}}{\pgfqpoint{8.060040in}{3.622932in}}%
\pgfpathclose%
\pgfusepath{fill}%
\end{pgfscope}%
\begin{pgfscope}%
\pgfpathrectangle{\pgfqpoint{6.572727in}{0.474100in}}{\pgfqpoint{4.227273in}{3.318700in}}%
\pgfusepath{clip}%
\pgfsetbuttcap%
\pgfsetroundjoin%
\definecolor{currentfill}{rgb}{0.267004,0.004874,0.329415}%
\pgfsetfillcolor{currentfill}%
\pgfsetfillopacity{0.700000}%
\pgfsetlinewidth{0.000000pt}%
\definecolor{currentstroke}{rgb}{0.000000,0.000000,0.000000}%
\pgfsetstrokecolor{currentstroke}%
\pgfsetstrokeopacity{0.700000}%
\pgfsetdash{}{0pt}%
\pgfpathmoveto{\pgfqpoint{7.588776in}{1.564973in}}%
\pgfpathcurveto{\pgfqpoint{7.593820in}{1.564973in}}{\pgfqpoint{7.598658in}{1.566977in}}{\pgfqpoint{7.602224in}{1.570543in}}%
\pgfpathcurveto{\pgfqpoint{7.605791in}{1.574109in}}{\pgfqpoint{7.607795in}{1.578947in}}{\pgfqpoint{7.607795in}{1.583991in}}%
\pgfpathcurveto{\pgfqpoint{7.607795in}{1.589035in}}{\pgfqpoint{7.605791in}{1.593872in}}{\pgfqpoint{7.602224in}{1.597439in}}%
\pgfpathcurveto{\pgfqpoint{7.598658in}{1.601005in}}{\pgfqpoint{7.593820in}{1.603009in}}{\pgfqpoint{7.588776in}{1.603009in}}%
\pgfpathcurveto{\pgfqpoint{7.583733in}{1.603009in}}{\pgfqpoint{7.578895in}{1.601005in}}{\pgfqpoint{7.575329in}{1.597439in}}%
\pgfpathcurveto{\pgfqpoint{7.571762in}{1.593872in}}{\pgfqpoint{7.569758in}{1.589035in}}{\pgfqpoint{7.569758in}{1.583991in}}%
\pgfpathcurveto{\pgfqpoint{7.569758in}{1.578947in}}{\pgfqpoint{7.571762in}{1.574109in}}{\pgfqpoint{7.575329in}{1.570543in}}%
\pgfpathcurveto{\pgfqpoint{7.578895in}{1.566977in}}{\pgfqpoint{7.583733in}{1.564973in}}{\pgfqpoint{7.588776in}{1.564973in}}%
\pgfpathclose%
\pgfusepath{fill}%
\end{pgfscope}%
\begin{pgfscope}%
\pgfpathrectangle{\pgfqpoint{6.572727in}{0.474100in}}{\pgfqpoint{4.227273in}{3.318700in}}%
\pgfusepath{clip}%
\pgfsetbuttcap%
\pgfsetroundjoin%
\definecolor{currentfill}{rgb}{0.993248,0.906157,0.143936}%
\pgfsetfillcolor{currentfill}%
\pgfsetfillopacity{0.700000}%
\pgfsetlinewidth{0.000000pt}%
\definecolor{currentstroke}{rgb}{0.000000,0.000000,0.000000}%
\pgfsetstrokecolor{currentstroke}%
\pgfsetstrokeopacity{0.700000}%
\pgfsetdash{}{0pt}%
\pgfpathmoveto{\pgfqpoint{8.196337in}{2.949685in}}%
\pgfpathcurveto{\pgfqpoint{8.201381in}{2.949685in}}{\pgfqpoint{8.206219in}{2.951689in}}{\pgfqpoint{8.209785in}{2.955255in}}%
\pgfpathcurveto{\pgfqpoint{8.213351in}{2.958822in}}{\pgfqpoint{8.215355in}{2.963660in}}{\pgfqpoint{8.215355in}{2.968703in}}%
\pgfpathcurveto{\pgfqpoint{8.215355in}{2.973747in}}{\pgfqpoint{8.213351in}{2.978585in}}{\pgfqpoint{8.209785in}{2.982151in}}%
\pgfpathcurveto{\pgfqpoint{8.206219in}{2.985718in}}{\pgfqpoint{8.201381in}{2.987721in}}{\pgfqpoint{8.196337in}{2.987721in}}%
\pgfpathcurveto{\pgfqpoint{8.191293in}{2.987721in}}{\pgfqpoint{8.186456in}{2.985718in}}{\pgfqpoint{8.182889in}{2.982151in}}%
\pgfpathcurveto{\pgfqpoint{8.179323in}{2.978585in}}{\pgfqpoint{8.177319in}{2.973747in}}{\pgfqpoint{8.177319in}{2.968703in}}%
\pgfpathcurveto{\pgfqpoint{8.177319in}{2.963660in}}{\pgfqpoint{8.179323in}{2.958822in}}{\pgfqpoint{8.182889in}{2.955255in}}%
\pgfpathcurveto{\pgfqpoint{8.186456in}{2.951689in}}{\pgfqpoint{8.191293in}{2.949685in}}{\pgfqpoint{8.196337in}{2.949685in}}%
\pgfpathclose%
\pgfusepath{fill}%
\end{pgfscope}%
\begin{pgfscope}%
\pgfpathrectangle{\pgfqpoint{6.572727in}{0.474100in}}{\pgfqpoint{4.227273in}{3.318700in}}%
\pgfusepath{clip}%
\pgfsetbuttcap%
\pgfsetroundjoin%
\definecolor{currentfill}{rgb}{0.127568,0.566949,0.550556}%
\pgfsetfillcolor{currentfill}%
\pgfsetfillopacity{0.700000}%
\pgfsetlinewidth{0.000000pt}%
\definecolor{currentstroke}{rgb}{0.000000,0.000000,0.000000}%
\pgfsetstrokecolor{currentstroke}%
\pgfsetstrokeopacity{0.700000}%
\pgfsetdash{}{0pt}%
\pgfpathmoveto{\pgfqpoint{9.253400in}{1.590459in}}%
\pgfpathcurveto{\pgfqpoint{9.258444in}{1.590459in}}{\pgfqpoint{9.263281in}{1.592463in}}{\pgfqpoint{9.266848in}{1.596029in}}%
\pgfpathcurveto{\pgfqpoint{9.270414in}{1.599595in}}{\pgfqpoint{9.272418in}{1.604433in}}{\pgfqpoint{9.272418in}{1.609477in}}%
\pgfpathcurveto{\pgfqpoint{9.272418in}{1.614521in}}{\pgfqpoint{9.270414in}{1.619358in}}{\pgfqpoint{9.266848in}{1.622925in}}%
\pgfpathcurveto{\pgfqpoint{9.263281in}{1.626491in}}{\pgfqpoint{9.258444in}{1.628495in}}{\pgfqpoint{9.253400in}{1.628495in}}%
\pgfpathcurveto{\pgfqpoint{9.248356in}{1.628495in}}{\pgfqpoint{9.243519in}{1.626491in}}{\pgfqpoint{9.239952in}{1.622925in}}%
\pgfpathcurveto{\pgfqpoint{9.236386in}{1.619358in}}{\pgfqpoint{9.234382in}{1.614521in}}{\pgfqpoint{9.234382in}{1.609477in}}%
\pgfpathcurveto{\pgfqpoint{9.234382in}{1.604433in}}{\pgfqpoint{9.236386in}{1.599595in}}{\pgfqpoint{9.239952in}{1.596029in}}%
\pgfpathcurveto{\pgfqpoint{9.243519in}{1.592463in}}{\pgfqpoint{9.248356in}{1.590459in}}{\pgfqpoint{9.253400in}{1.590459in}}%
\pgfpathclose%
\pgfusepath{fill}%
\end{pgfscope}%
\begin{pgfscope}%
\pgfpathrectangle{\pgfqpoint{6.572727in}{0.474100in}}{\pgfqpoint{4.227273in}{3.318700in}}%
\pgfusepath{clip}%
\pgfsetbuttcap%
\pgfsetroundjoin%
\definecolor{currentfill}{rgb}{0.127568,0.566949,0.550556}%
\pgfsetfillcolor{currentfill}%
\pgfsetfillopacity{0.700000}%
\pgfsetlinewidth{0.000000pt}%
\definecolor{currentstroke}{rgb}{0.000000,0.000000,0.000000}%
\pgfsetstrokecolor{currentstroke}%
\pgfsetstrokeopacity{0.700000}%
\pgfsetdash{}{0pt}%
\pgfpathmoveto{\pgfqpoint{9.654639in}{2.032378in}}%
\pgfpathcurveto{\pgfqpoint{9.659682in}{2.032378in}}{\pgfqpoint{9.664520in}{2.034381in}}{\pgfqpoint{9.668087in}{2.037948in}}%
\pgfpathcurveto{\pgfqpoint{9.671653in}{2.041514in}}{\pgfqpoint{9.673657in}{2.046352in}}{\pgfqpoint{9.673657in}{2.051396in}}%
\pgfpathcurveto{\pgfqpoint{9.673657in}{2.056439in}}{\pgfqpoint{9.671653in}{2.061277in}}{\pgfqpoint{9.668087in}{2.064844in}}%
\pgfpathcurveto{\pgfqpoint{9.664520in}{2.068410in}}{\pgfqpoint{9.659682in}{2.070414in}}{\pgfqpoint{9.654639in}{2.070414in}}%
\pgfpathcurveto{\pgfqpoint{9.649595in}{2.070414in}}{\pgfqpoint{9.644757in}{2.068410in}}{\pgfqpoint{9.641191in}{2.064844in}}%
\pgfpathcurveto{\pgfqpoint{9.637624in}{2.061277in}}{\pgfqpoint{9.635621in}{2.056439in}}{\pgfqpoint{9.635621in}{2.051396in}}%
\pgfpathcurveto{\pgfqpoint{9.635621in}{2.046352in}}{\pgfqpoint{9.637624in}{2.041514in}}{\pgfqpoint{9.641191in}{2.037948in}}%
\pgfpathcurveto{\pgfqpoint{9.644757in}{2.034381in}}{\pgfqpoint{9.649595in}{2.032378in}}{\pgfqpoint{9.654639in}{2.032378in}}%
\pgfpathclose%
\pgfusepath{fill}%
\end{pgfscope}%
\begin{pgfscope}%
\pgfpathrectangle{\pgfqpoint{6.572727in}{0.474100in}}{\pgfqpoint{4.227273in}{3.318700in}}%
\pgfusepath{clip}%
\pgfsetbuttcap%
\pgfsetroundjoin%
\definecolor{currentfill}{rgb}{0.267004,0.004874,0.329415}%
\pgfsetfillcolor{currentfill}%
\pgfsetfillopacity{0.700000}%
\pgfsetlinewidth{0.000000pt}%
\definecolor{currentstroke}{rgb}{0.000000,0.000000,0.000000}%
\pgfsetstrokecolor{currentstroke}%
\pgfsetstrokeopacity{0.700000}%
\pgfsetdash{}{0pt}%
\pgfpathmoveto{\pgfqpoint{7.898775in}{1.094336in}}%
\pgfpathcurveto{\pgfqpoint{7.903818in}{1.094336in}}{\pgfqpoint{7.908656in}{1.096340in}}{\pgfqpoint{7.912222in}{1.099906in}}%
\pgfpathcurveto{\pgfqpoint{7.915789in}{1.103472in}}{\pgfqpoint{7.917793in}{1.108310in}}{\pgfqpoint{7.917793in}{1.113354in}}%
\pgfpathcurveto{\pgfqpoint{7.917793in}{1.118397in}}{\pgfqpoint{7.915789in}{1.123235in}}{\pgfqpoint{7.912222in}{1.126802in}}%
\pgfpathcurveto{\pgfqpoint{7.908656in}{1.130368in}}{\pgfqpoint{7.903818in}{1.132372in}}{\pgfqpoint{7.898775in}{1.132372in}}%
\pgfpathcurveto{\pgfqpoint{7.893731in}{1.132372in}}{\pgfqpoint{7.888893in}{1.130368in}}{\pgfqpoint{7.885327in}{1.126802in}}%
\pgfpathcurveto{\pgfqpoint{7.881760in}{1.123235in}}{\pgfqpoint{7.879756in}{1.118397in}}{\pgfqpoint{7.879756in}{1.113354in}}%
\pgfpathcurveto{\pgfqpoint{7.879756in}{1.108310in}}{\pgfqpoint{7.881760in}{1.103472in}}{\pgfqpoint{7.885327in}{1.099906in}}%
\pgfpathcurveto{\pgfqpoint{7.888893in}{1.096340in}}{\pgfqpoint{7.893731in}{1.094336in}}{\pgfqpoint{7.898775in}{1.094336in}}%
\pgfpathclose%
\pgfusepath{fill}%
\end{pgfscope}%
\begin{pgfscope}%
\pgfpathrectangle{\pgfqpoint{6.572727in}{0.474100in}}{\pgfqpoint{4.227273in}{3.318700in}}%
\pgfusepath{clip}%
\pgfsetbuttcap%
\pgfsetroundjoin%
\definecolor{currentfill}{rgb}{0.267004,0.004874,0.329415}%
\pgfsetfillcolor{currentfill}%
\pgfsetfillopacity{0.700000}%
\pgfsetlinewidth{0.000000pt}%
\definecolor{currentstroke}{rgb}{0.000000,0.000000,0.000000}%
\pgfsetstrokecolor{currentstroke}%
\pgfsetstrokeopacity{0.700000}%
\pgfsetdash{}{0pt}%
\pgfpathmoveto{\pgfqpoint{7.254500in}{1.150518in}}%
\pgfpathcurveto{\pgfqpoint{7.259544in}{1.150518in}}{\pgfqpoint{7.264381in}{1.152522in}}{\pgfqpoint{7.267948in}{1.156088in}}%
\pgfpathcurveto{\pgfqpoint{7.271514in}{1.159654in}}{\pgfqpoint{7.273518in}{1.164492in}}{\pgfqpoint{7.273518in}{1.169536in}}%
\pgfpathcurveto{\pgfqpoint{7.273518in}{1.174579in}}{\pgfqpoint{7.271514in}{1.179417in}}{\pgfqpoint{7.267948in}{1.182984in}}%
\pgfpathcurveto{\pgfqpoint{7.264381in}{1.186550in}}{\pgfqpoint{7.259544in}{1.188554in}}{\pgfqpoint{7.254500in}{1.188554in}}%
\pgfpathcurveto{\pgfqpoint{7.249456in}{1.188554in}}{\pgfqpoint{7.244619in}{1.186550in}}{\pgfqpoint{7.241052in}{1.182984in}}%
\pgfpathcurveto{\pgfqpoint{7.237486in}{1.179417in}}{\pgfqpoint{7.235482in}{1.174579in}}{\pgfqpoint{7.235482in}{1.169536in}}%
\pgfpathcurveto{\pgfqpoint{7.235482in}{1.164492in}}{\pgfqpoint{7.237486in}{1.159654in}}{\pgfqpoint{7.241052in}{1.156088in}}%
\pgfpathcurveto{\pgfqpoint{7.244619in}{1.152522in}}{\pgfqpoint{7.249456in}{1.150518in}}{\pgfqpoint{7.254500in}{1.150518in}}%
\pgfpathclose%
\pgfusepath{fill}%
\end{pgfscope}%
\begin{pgfscope}%
\pgfpathrectangle{\pgfqpoint{6.572727in}{0.474100in}}{\pgfqpoint{4.227273in}{3.318700in}}%
\pgfusepath{clip}%
\pgfsetbuttcap%
\pgfsetroundjoin%
\definecolor{currentfill}{rgb}{0.993248,0.906157,0.143936}%
\pgfsetfillcolor{currentfill}%
\pgfsetfillopacity{0.700000}%
\pgfsetlinewidth{0.000000pt}%
\definecolor{currentstroke}{rgb}{0.000000,0.000000,0.000000}%
\pgfsetstrokecolor{currentstroke}%
\pgfsetstrokeopacity{0.700000}%
\pgfsetdash{}{0pt}%
\pgfpathmoveto{\pgfqpoint{7.777944in}{3.139026in}}%
\pgfpathcurveto{\pgfqpoint{7.782988in}{3.139026in}}{\pgfqpoint{7.787826in}{3.141030in}}{\pgfqpoint{7.791392in}{3.144597in}}%
\pgfpathcurveto{\pgfqpoint{7.794959in}{3.148163in}}{\pgfqpoint{7.796963in}{3.153001in}}{\pgfqpoint{7.796963in}{3.158044in}}%
\pgfpathcurveto{\pgfqpoint{7.796963in}{3.163088in}}{\pgfqpoint{7.794959in}{3.167926in}}{\pgfqpoint{7.791392in}{3.171492in}}%
\pgfpathcurveto{\pgfqpoint{7.787826in}{3.175059in}}{\pgfqpoint{7.782988in}{3.177063in}}{\pgfqpoint{7.777944in}{3.177063in}}%
\pgfpathcurveto{\pgfqpoint{7.772901in}{3.177063in}}{\pgfqpoint{7.768063in}{3.175059in}}{\pgfqpoint{7.764497in}{3.171492in}}%
\pgfpathcurveto{\pgfqpoint{7.760930in}{3.167926in}}{\pgfqpoint{7.758926in}{3.163088in}}{\pgfqpoint{7.758926in}{3.158044in}}%
\pgfpathcurveto{\pgfqpoint{7.758926in}{3.153001in}}{\pgfqpoint{7.760930in}{3.148163in}}{\pgfqpoint{7.764497in}{3.144597in}}%
\pgfpathcurveto{\pgfqpoint{7.768063in}{3.141030in}}{\pgfqpoint{7.772901in}{3.139026in}}{\pgfqpoint{7.777944in}{3.139026in}}%
\pgfpathclose%
\pgfusepath{fill}%
\end{pgfscope}%
\begin{pgfscope}%
\pgfpathrectangle{\pgfqpoint{6.572727in}{0.474100in}}{\pgfqpoint{4.227273in}{3.318700in}}%
\pgfusepath{clip}%
\pgfsetbuttcap%
\pgfsetroundjoin%
\definecolor{currentfill}{rgb}{0.127568,0.566949,0.550556}%
\pgfsetfillcolor{currentfill}%
\pgfsetfillopacity{0.700000}%
\pgfsetlinewidth{0.000000pt}%
\definecolor{currentstroke}{rgb}{0.000000,0.000000,0.000000}%
\pgfsetstrokecolor{currentstroke}%
\pgfsetstrokeopacity{0.700000}%
\pgfsetdash{}{0pt}%
\pgfpathmoveto{\pgfqpoint{9.963304in}{1.743597in}}%
\pgfpathcurveto{\pgfqpoint{9.968348in}{1.743597in}}{\pgfqpoint{9.973186in}{1.745601in}}{\pgfqpoint{9.976752in}{1.749167in}}%
\pgfpathcurveto{\pgfqpoint{9.980319in}{1.752734in}}{\pgfqpoint{9.982322in}{1.757572in}}{\pgfqpoint{9.982322in}{1.762615in}}%
\pgfpathcurveto{\pgfqpoint{9.982322in}{1.767659in}}{\pgfqpoint{9.980319in}{1.772497in}}{\pgfqpoint{9.976752in}{1.776063in}}%
\pgfpathcurveto{\pgfqpoint{9.973186in}{1.779630in}}{\pgfqpoint{9.968348in}{1.781633in}}{\pgfqpoint{9.963304in}{1.781633in}}%
\pgfpathcurveto{\pgfqpoint{9.958261in}{1.781633in}}{\pgfqpoint{9.953423in}{1.779630in}}{\pgfqpoint{9.949856in}{1.776063in}}%
\pgfpathcurveto{\pgfqpoint{9.946290in}{1.772497in}}{\pgfqpoint{9.944286in}{1.767659in}}{\pgfqpoint{9.944286in}{1.762615in}}%
\pgfpathcurveto{\pgfqpoint{9.944286in}{1.757572in}}{\pgfqpoint{9.946290in}{1.752734in}}{\pgfqpoint{9.949856in}{1.749167in}}%
\pgfpathcurveto{\pgfqpoint{9.953423in}{1.745601in}}{\pgfqpoint{9.958261in}{1.743597in}}{\pgfqpoint{9.963304in}{1.743597in}}%
\pgfpathclose%
\pgfusepath{fill}%
\end{pgfscope}%
\begin{pgfscope}%
\pgfpathrectangle{\pgfqpoint{6.572727in}{0.474100in}}{\pgfqpoint{4.227273in}{3.318700in}}%
\pgfusepath{clip}%
\pgfsetbuttcap%
\pgfsetroundjoin%
\definecolor{currentfill}{rgb}{0.127568,0.566949,0.550556}%
\pgfsetfillcolor{currentfill}%
\pgfsetfillopacity{0.700000}%
\pgfsetlinewidth{0.000000pt}%
\definecolor{currentstroke}{rgb}{0.000000,0.000000,0.000000}%
\pgfsetstrokecolor{currentstroke}%
\pgfsetstrokeopacity{0.700000}%
\pgfsetdash{}{0pt}%
\pgfpathmoveto{\pgfqpoint{9.563522in}{2.158193in}}%
\pgfpathcurveto{\pgfqpoint{9.568566in}{2.158193in}}{\pgfqpoint{9.573403in}{2.160197in}}{\pgfqpoint{9.576970in}{2.163763in}}%
\pgfpathcurveto{\pgfqpoint{9.580536in}{2.167329in}}{\pgfqpoint{9.582540in}{2.172167in}}{\pgfqpoint{9.582540in}{2.177211in}}%
\pgfpathcurveto{\pgfqpoint{9.582540in}{2.182254in}}{\pgfqpoint{9.580536in}{2.187092in}}{\pgfqpoint{9.576970in}{2.190659in}}%
\pgfpathcurveto{\pgfqpoint{9.573403in}{2.194225in}}{\pgfqpoint{9.568566in}{2.196229in}}{\pgfqpoint{9.563522in}{2.196229in}}%
\pgfpathcurveto{\pgfqpoint{9.558478in}{2.196229in}}{\pgfqpoint{9.553641in}{2.194225in}}{\pgfqpoint{9.550074in}{2.190659in}}%
\pgfpathcurveto{\pgfqpoint{9.546508in}{2.187092in}}{\pgfqpoint{9.544504in}{2.182254in}}{\pgfqpoint{9.544504in}{2.177211in}}%
\pgfpathcurveto{\pgfqpoint{9.544504in}{2.172167in}}{\pgfqpoint{9.546508in}{2.167329in}}{\pgfqpoint{9.550074in}{2.163763in}}%
\pgfpathcurveto{\pgfqpoint{9.553641in}{2.160197in}}{\pgfqpoint{9.558478in}{2.158193in}}{\pgfqpoint{9.563522in}{2.158193in}}%
\pgfpathclose%
\pgfusepath{fill}%
\end{pgfscope}%
\begin{pgfscope}%
\pgfpathrectangle{\pgfqpoint{6.572727in}{0.474100in}}{\pgfqpoint{4.227273in}{3.318700in}}%
\pgfusepath{clip}%
\pgfsetbuttcap%
\pgfsetroundjoin%
\definecolor{currentfill}{rgb}{0.993248,0.906157,0.143936}%
\pgfsetfillcolor{currentfill}%
\pgfsetfillopacity{0.700000}%
\pgfsetlinewidth{0.000000pt}%
\definecolor{currentstroke}{rgb}{0.000000,0.000000,0.000000}%
\pgfsetstrokecolor{currentstroke}%
\pgfsetstrokeopacity{0.700000}%
\pgfsetdash{}{0pt}%
\pgfpathmoveto{\pgfqpoint{8.235870in}{2.566468in}}%
\pgfpathcurveto{\pgfqpoint{8.240914in}{2.566468in}}{\pgfqpoint{8.245752in}{2.568472in}}{\pgfqpoint{8.249318in}{2.572039in}}%
\pgfpathcurveto{\pgfqpoint{8.252885in}{2.575605in}}{\pgfqpoint{8.254889in}{2.580443in}}{\pgfqpoint{8.254889in}{2.585486in}}%
\pgfpathcurveto{\pgfqpoint{8.254889in}{2.590530in}}{\pgfqpoint{8.252885in}{2.595368in}}{\pgfqpoint{8.249318in}{2.598934in}}%
\pgfpathcurveto{\pgfqpoint{8.245752in}{2.602501in}}{\pgfqpoint{8.240914in}{2.604505in}}{\pgfqpoint{8.235870in}{2.604505in}}%
\pgfpathcurveto{\pgfqpoint{8.230827in}{2.604505in}}{\pgfqpoint{8.225989in}{2.602501in}}{\pgfqpoint{8.222423in}{2.598934in}}%
\pgfpathcurveto{\pgfqpoint{8.218856in}{2.595368in}}{\pgfqpoint{8.216852in}{2.590530in}}{\pgfqpoint{8.216852in}{2.585486in}}%
\pgfpathcurveto{\pgfqpoint{8.216852in}{2.580443in}}{\pgfqpoint{8.218856in}{2.575605in}}{\pgfqpoint{8.222423in}{2.572039in}}%
\pgfpathcurveto{\pgfqpoint{8.225989in}{2.568472in}}{\pgfqpoint{8.230827in}{2.566468in}}{\pgfqpoint{8.235870in}{2.566468in}}%
\pgfpathclose%
\pgfusepath{fill}%
\end{pgfscope}%
\begin{pgfscope}%
\pgfpathrectangle{\pgfqpoint{6.572727in}{0.474100in}}{\pgfqpoint{4.227273in}{3.318700in}}%
\pgfusepath{clip}%
\pgfsetbuttcap%
\pgfsetroundjoin%
\definecolor{currentfill}{rgb}{0.267004,0.004874,0.329415}%
\pgfsetfillcolor{currentfill}%
\pgfsetfillopacity{0.700000}%
\pgfsetlinewidth{0.000000pt}%
\definecolor{currentstroke}{rgb}{0.000000,0.000000,0.000000}%
\pgfsetstrokecolor{currentstroke}%
\pgfsetstrokeopacity{0.700000}%
\pgfsetdash{}{0pt}%
\pgfpathmoveto{\pgfqpoint{7.442700in}{1.973380in}}%
\pgfpathcurveto{\pgfqpoint{7.447744in}{1.973380in}}{\pgfqpoint{7.452582in}{1.975384in}}{\pgfqpoint{7.456148in}{1.978951in}}%
\pgfpathcurveto{\pgfqpoint{7.459714in}{1.982517in}}{\pgfqpoint{7.461718in}{1.987355in}}{\pgfqpoint{7.461718in}{1.992399in}}%
\pgfpathcurveto{\pgfqpoint{7.461718in}{1.997442in}}{\pgfqpoint{7.459714in}{2.002280in}}{\pgfqpoint{7.456148in}{2.005846in}}%
\pgfpathcurveto{\pgfqpoint{7.452582in}{2.009413in}}{\pgfqpoint{7.447744in}{2.011417in}}{\pgfqpoint{7.442700in}{2.011417in}}%
\pgfpathcurveto{\pgfqpoint{7.437656in}{2.011417in}}{\pgfqpoint{7.432819in}{2.009413in}}{\pgfqpoint{7.429252in}{2.005846in}}%
\pgfpathcurveto{\pgfqpoint{7.425686in}{2.002280in}}{\pgfqpoint{7.423682in}{1.997442in}}{\pgfqpoint{7.423682in}{1.992399in}}%
\pgfpathcurveto{\pgfqpoint{7.423682in}{1.987355in}}{\pgfqpoint{7.425686in}{1.982517in}}{\pgfqpoint{7.429252in}{1.978951in}}%
\pgfpathcurveto{\pgfqpoint{7.432819in}{1.975384in}}{\pgfqpoint{7.437656in}{1.973380in}}{\pgfqpoint{7.442700in}{1.973380in}}%
\pgfpathclose%
\pgfusepath{fill}%
\end{pgfscope}%
\begin{pgfscope}%
\pgfpathrectangle{\pgfqpoint{6.572727in}{0.474100in}}{\pgfqpoint{4.227273in}{3.318700in}}%
\pgfusepath{clip}%
\pgfsetbuttcap%
\pgfsetroundjoin%
\definecolor{currentfill}{rgb}{0.127568,0.566949,0.550556}%
\pgfsetfillcolor{currentfill}%
\pgfsetfillopacity{0.700000}%
\pgfsetlinewidth{0.000000pt}%
\definecolor{currentstroke}{rgb}{0.000000,0.000000,0.000000}%
\pgfsetstrokecolor{currentstroke}%
\pgfsetstrokeopacity{0.700000}%
\pgfsetdash{}{0pt}%
\pgfpathmoveto{\pgfqpoint{9.637547in}{1.446546in}}%
\pgfpathcurveto{\pgfqpoint{9.642591in}{1.446546in}}{\pgfqpoint{9.647428in}{1.448550in}}{\pgfqpoint{9.650995in}{1.452117in}}%
\pgfpathcurveto{\pgfqpoint{9.654561in}{1.455683in}}{\pgfqpoint{9.656565in}{1.460521in}}{\pgfqpoint{9.656565in}{1.465564in}}%
\pgfpathcurveto{\pgfqpoint{9.656565in}{1.470608in}}{\pgfqpoint{9.654561in}{1.475446in}}{\pgfqpoint{9.650995in}{1.479012in}}%
\pgfpathcurveto{\pgfqpoint{9.647428in}{1.482579in}}{\pgfqpoint{9.642591in}{1.484583in}}{\pgfqpoint{9.637547in}{1.484583in}}%
\pgfpathcurveto{\pgfqpoint{9.632503in}{1.484583in}}{\pgfqpoint{9.627666in}{1.482579in}}{\pgfqpoint{9.624099in}{1.479012in}}%
\pgfpathcurveto{\pgfqpoint{9.620533in}{1.475446in}}{\pgfqpoint{9.618529in}{1.470608in}}{\pgfqpoint{9.618529in}{1.465564in}}%
\pgfpathcurveto{\pgfqpoint{9.618529in}{1.460521in}}{\pgfqpoint{9.620533in}{1.455683in}}{\pgfqpoint{9.624099in}{1.452117in}}%
\pgfpathcurveto{\pgfqpoint{9.627666in}{1.448550in}}{\pgfqpoint{9.632503in}{1.446546in}}{\pgfqpoint{9.637547in}{1.446546in}}%
\pgfpathclose%
\pgfusepath{fill}%
\end{pgfscope}%
\begin{pgfscope}%
\pgfpathrectangle{\pgfqpoint{6.572727in}{0.474100in}}{\pgfqpoint{4.227273in}{3.318700in}}%
\pgfusepath{clip}%
\pgfsetbuttcap%
\pgfsetroundjoin%
\definecolor{currentfill}{rgb}{0.127568,0.566949,0.550556}%
\pgfsetfillcolor{currentfill}%
\pgfsetfillopacity{0.700000}%
\pgfsetlinewidth{0.000000pt}%
\definecolor{currentstroke}{rgb}{0.000000,0.000000,0.000000}%
\pgfsetstrokecolor{currentstroke}%
\pgfsetstrokeopacity{0.700000}%
\pgfsetdash{}{0pt}%
\pgfpathmoveto{\pgfqpoint{9.799364in}{1.231788in}}%
\pgfpathcurveto{\pgfqpoint{9.804407in}{1.231788in}}{\pgfqpoint{9.809245in}{1.233792in}}{\pgfqpoint{9.812811in}{1.237358in}}%
\pgfpathcurveto{\pgfqpoint{9.816378in}{1.240925in}}{\pgfqpoint{9.818382in}{1.245763in}}{\pgfqpoint{9.818382in}{1.250806in}}%
\pgfpathcurveto{\pgfqpoint{9.818382in}{1.255850in}}{\pgfqpoint{9.816378in}{1.260688in}}{\pgfqpoint{9.812811in}{1.264254in}}%
\pgfpathcurveto{\pgfqpoint{9.809245in}{1.267820in}}{\pgfqpoint{9.804407in}{1.269824in}}{\pgfqpoint{9.799364in}{1.269824in}}%
\pgfpathcurveto{\pgfqpoint{9.794320in}{1.269824in}}{\pgfqpoint{9.789482in}{1.267820in}}{\pgfqpoint{9.785916in}{1.264254in}}%
\pgfpathcurveto{\pgfqpoint{9.782349in}{1.260688in}}{\pgfqpoint{9.780345in}{1.255850in}}{\pgfqpoint{9.780345in}{1.250806in}}%
\pgfpathcurveto{\pgfqpoint{9.780345in}{1.245763in}}{\pgfqpoint{9.782349in}{1.240925in}}{\pgfqpoint{9.785916in}{1.237358in}}%
\pgfpathcurveto{\pgfqpoint{9.789482in}{1.233792in}}{\pgfqpoint{9.794320in}{1.231788in}}{\pgfqpoint{9.799364in}{1.231788in}}%
\pgfpathclose%
\pgfusepath{fill}%
\end{pgfscope}%
\begin{pgfscope}%
\pgfpathrectangle{\pgfqpoint{6.572727in}{0.474100in}}{\pgfqpoint{4.227273in}{3.318700in}}%
\pgfusepath{clip}%
\pgfsetbuttcap%
\pgfsetroundjoin%
\definecolor{currentfill}{rgb}{0.993248,0.906157,0.143936}%
\pgfsetfillcolor{currentfill}%
\pgfsetfillopacity{0.700000}%
\pgfsetlinewidth{0.000000pt}%
\definecolor{currentstroke}{rgb}{0.000000,0.000000,0.000000}%
\pgfsetstrokecolor{currentstroke}%
\pgfsetstrokeopacity{0.700000}%
\pgfsetdash{}{0pt}%
\pgfpathmoveto{\pgfqpoint{8.002259in}{2.872913in}}%
\pgfpathcurveto{\pgfqpoint{8.007302in}{2.872913in}}{\pgfqpoint{8.012140in}{2.874917in}}{\pgfqpoint{8.015707in}{2.878483in}}%
\pgfpathcurveto{\pgfqpoint{8.019273in}{2.882050in}}{\pgfqpoint{8.021277in}{2.886887in}}{\pgfqpoint{8.021277in}{2.891931in}}%
\pgfpathcurveto{\pgfqpoint{8.021277in}{2.896975in}}{\pgfqpoint{8.019273in}{2.901812in}}{\pgfqpoint{8.015707in}{2.905379in}}%
\pgfpathcurveto{\pgfqpoint{8.012140in}{2.908945in}}{\pgfqpoint{8.007302in}{2.910949in}}{\pgfqpoint{8.002259in}{2.910949in}}%
\pgfpathcurveto{\pgfqpoint{7.997215in}{2.910949in}}{\pgfqpoint{7.992377in}{2.908945in}}{\pgfqpoint{7.988811in}{2.905379in}}%
\pgfpathcurveto{\pgfqpoint{7.985244in}{2.901812in}}{\pgfqpoint{7.983241in}{2.896975in}}{\pgfqpoint{7.983241in}{2.891931in}}%
\pgfpathcurveto{\pgfqpoint{7.983241in}{2.886887in}}{\pgfqpoint{7.985244in}{2.882050in}}{\pgfqpoint{7.988811in}{2.878483in}}%
\pgfpathcurveto{\pgfqpoint{7.992377in}{2.874917in}}{\pgfqpoint{7.997215in}{2.872913in}}{\pgfqpoint{8.002259in}{2.872913in}}%
\pgfpathclose%
\pgfusepath{fill}%
\end{pgfscope}%
\begin{pgfscope}%
\pgfpathrectangle{\pgfqpoint{6.572727in}{0.474100in}}{\pgfqpoint{4.227273in}{3.318700in}}%
\pgfusepath{clip}%
\pgfsetbuttcap%
\pgfsetroundjoin%
\definecolor{currentfill}{rgb}{0.267004,0.004874,0.329415}%
\pgfsetfillcolor{currentfill}%
\pgfsetfillopacity{0.700000}%
\pgfsetlinewidth{0.000000pt}%
\definecolor{currentstroke}{rgb}{0.000000,0.000000,0.000000}%
\pgfsetstrokecolor{currentstroke}%
\pgfsetstrokeopacity{0.700000}%
\pgfsetdash{}{0pt}%
\pgfpathmoveto{\pgfqpoint{7.437299in}{1.638424in}}%
\pgfpathcurveto{\pgfqpoint{7.442342in}{1.638424in}}{\pgfqpoint{7.447180in}{1.640427in}}{\pgfqpoint{7.450746in}{1.643994in}}%
\pgfpathcurveto{\pgfqpoint{7.454313in}{1.647560in}}{\pgfqpoint{7.456317in}{1.652398in}}{\pgfqpoint{7.456317in}{1.657442in}}%
\pgfpathcurveto{\pgfqpoint{7.456317in}{1.662485in}}{\pgfqpoint{7.454313in}{1.667323in}}{\pgfqpoint{7.450746in}{1.670890in}}%
\pgfpathcurveto{\pgfqpoint{7.447180in}{1.674456in}}{\pgfqpoint{7.442342in}{1.676460in}}{\pgfqpoint{7.437299in}{1.676460in}}%
\pgfpathcurveto{\pgfqpoint{7.432255in}{1.676460in}}{\pgfqpoint{7.427417in}{1.674456in}}{\pgfqpoint{7.423851in}{1.670890in}}%
\pgfpathcurveto{\pgfqpoint{7.420284in}{1.667323in}}{\pgfqpoint{7.418280in}{1.662485in}}{\pgfqpoint{7.418280in}{1.657442in}}%
\pgfpathcurveto{\pgfqpoint{7.418280in}{1.652398in}}{\pgfqpoint{7.420284in}{1.647560in}}{\pgfqpoint{7.423851in}{1.643994in}}%
\pgfpathcurveto{\pgfqpoint{7.427417in}{1.640427in}}{\pgfqpoint{7.432255in}{1.638424in}}{\pgfqpoint{7.437299in}{1.638424in}}%
\pgfpathclose%
\pgfusepath{fill}%
\end{pgfscope}%
\begin{pgfscope}%
\pgfpathrectangle{\pgfqpoint{6.572727in}{0.474100in}}{\pgfqpoint{4.227273in}{3.318700in}}%
\pgfusepath{clip}%
\pgfsetbuttcap%
\pgfsetroundjoin%
\definecolor{currentfill}{rgb}{0.127568,0.566949,0.550556}%
\pgfsetfillcolor{currentfill}%
\pgfsetfillopacity{0.700000}%
\pgfsetlinewidth{0.000000pt}%
\definecolor{currentstroke}{rgb}{0.000000,0.000000,0.000000}%
\pgfsetstrokecolor{currentstroke}%
\pgfsetstrokeopacity{0.700000}%
\pgfsetdash{}{0pt}%
\pgfpathmoveto{\pgfqpoint{9.747416in}{1.358730in}}%
\pgfpathcurveto{\pgfqpoint{9.752460in}{1.358730in}}{\pgfqpoint{9.757297in}{1.360734in}}{\pgfqpoint{9.760864in}{1.364300in}}%
\pgfpathcurveto{\pgfqpoint{9.764430in}{1.367867in}}{\pgfqpoint{9.766434in}{1.372704in}}{\pgfqpoint{9.766434in}{1.377748in}}%
\pgfpathcurveto{\pgfqpoint{9.766434in}{1.382792in}}{\pgfqpoint{9.764430in}{1.387630in}}{\pgfqpoint{9.760864in}{1.391196in}}%
\pgfpathcurveto{\pgfqpoint{9.757297in}{1.394762in}}{\pgfqpoint{9.752460in}{1.396766in}}{\pgfqpoint{9.747416in}{1.396766in}}%
\pgfpathcurveto{\pgfqpoint{9.742372in}{1.396766in}}{\pgfqpoint{9.737535in}{1.394762in}}{\pgfqpoint{9.733968in}{1.391196in}}%
\pgfpathcurveto{\pgfqpoint{9.730402in}{1.387630in}}{\pgfqpoint{9.728398in}{1.382792in}}{\pgfqpoint{9.728398in}{1.377748in}}%
\pgfpathcurveto{\pgfqpoint{9.728398in}{1.372704in}}{\pgfqpoint{9.730402in}{1.367867in}}{\pgfqpoint{9.733968in}{1.364300in}}%
\pgfpathcurveto{\pgfqpoint{9.737535in}{1.360734in}}{\pgfqpoint{9.742372in}{1.358730in}}{\pgfqpoint{9.747416in}{1.358730in}}%
\pgfpathclose%
\pgfusepath{fill}%
\end{pgfscope}%
\begin{pgfscope}%
\pgfpathrectangle{\pgfqpoint{6.572727in}{0.474100in}}{\pgfqpoint{4.227273in}{3.318700in}}%
\pgfusepath{clip}%
\pgfsetbuttcap%
\pgfsetroundjoin%
\definecolor{currentfill}{rgb}{0.267004,0.004874,0.329415}%
\pgfsetfillcolor{currentfill}%
\pgfsetfillopacity{0.700000}%
\pgfsetlinewidth{0.000000pt}%
\definecolor{currentstroke}{rgb}{0.000000,0.000000,0.000000}%
\pgfsetstrokecolor{currentstroke}%
\pgfsetstrokeopacity{0.700000}%
\pgfsetdash{}{0pt}%
\pgfpathmoveto{\pgfqpoint{8.350238in}{2.035781in}}%
\pgfpathcurveto{\pgfqpoint{8.355282in}{2.035781in}}{\pgfqpoint{8.360120in}{2.037785in}}{\pgfqpoint{8.363686in}{2.041352in}}%
\pgfpathcurveto{\pgfqpoint{8.367253in}{2.044918in}}{\pgfqpoint{8.369257in}{2.049756in}}{\pgfqpoint{8.369257in}{2.054800in}}%
\pgfpathcurveto{\pgfqpoint{8.369257in}{2.059843in}}{\pgfqpoint{8.367253in}{2.064681in}}{\pgfqpoint{8.363686in}{2.068247in}}%
\pgfpathcurveto{\pgfqpoint{8.360120in}{2.071814in}}{\pgfqpoint{8.355282in}{2.073818in}}{\pgfqpoint{8.350238in}{2.073818in}}%
\pgfpathcurveto{\pgfqpoint{8.345195in}{2.073818in}}{\pgfqpoint{8.340357in}{2.071814in}}{\pgfqpoint{8.336790in}{2.068247in}}%
\pgfpathcurveto{\pgfqpoint{8.333224in}{2.064681in}}{\pgfqpoint{8.331220in}{2.059843in}}{\pgfqpoint{8.331220in}{2.054800in}}%
\pgfpathcurveto{\pgfqpoint{8.331220in}{2.049756in}}{\pgfqpoint{8.333224in}{2.044918in}}{\pgfqpoint{8.336790in}{2.041352in}}%
\pgfpathcurveto{\pgfqpoint{8.340357in}{2.037785in}}{\pgfqpoint{8.345195in}{2.035781in}}{\pgfqpoint{8.350238in}{2.035781in}}%
\pgfpathclose%
\pgfusepath{fill}%
\end{pgfscope}%
\begin{pgfscope}%
\pgfpathrectangle{\pgfqpoint{6.572727in}{0.474100in}}{\pgfqpoint{4.227273in}{3.318700in}}%
\pgfusepath{clip}%
\pgfsetbuttcap%
\pgfsetroundjoin%
\definecolor{currentfill}{rgb}{0.127568,0.566949,0.550556}%
\pgfsetfillcolor{currentfill}%
\pgfsetfillopacity{0.700000}%
\pgfsetlinewidth{0.000000pt}%
\definecolor{currentstroke}{rgb}{0.000000,0.000000,0.000000}%
\pgfsetstrokecolor{currentstroke}%
\pgfsetstrokeopacity{0.700000}%
\pgfsetdash{}{0pt}%
\pgfpathmoveto{\pgfqpoint{10.537250in}{2.125169in}}%
\pgfpathcurveto{\pgfqpoint{10.542293in}{2.125169in}}{\pgfqpoint{10.547131in}{2.127173in}}{\pgfqpoint{10.550697in}{2.130740in}}%
\pgfpathcurveto{\pgfqpoint{10.554264in}{2.134306in}}{\pgfqpoint{10.556268in}{2.139144in}}{\pgfqpoint{10.556268in}{2.144187in}}%
\pgfpathcurveto{\pgfqpoint{10.556268in}{2.149231in}}{\pgfqpoint{10.554264in}{2.154069in}}{\pgfqpoint{10.550697in}{2.157635in}}%
\pgfpathcurveto{\pgfqpoint{10.547131in}{2.161202in}}{\pgfqpoint{10.542293in}{2.163206in}}{\pgfqpoint{10.537250in}{2.163206in}}%
\pgfpathcurveto{\pgfqpoint{10.532206in}{2.163206in}}{\pgfqpoint{10.527368in}{2.161202in}}{\pgfqpoint{10.523802in}{2.157635in}}%
\pgfpathcurveto{\pgfqpoint{10.520235in}{2.154069in}}{\pgfqpoint{10.518231in}{2.149231in}}{\pgfqpoint{10.518231in}{2.144187in}}%
\pgfpathcurveto{\pgfqpoint{10.518231in}{2.139144in}}{\pgfqpoint{10.520235in}{2.134306in}}{\pgfqpoint{10.523802in}{2.130740in}}%
\pgfpathcurveto{\pgfqpoint{10.527368in}{2.127173in}}{\pgfqpoint{10.532206in}{2.125169in}}{\pgfqpoint{10.537250in}{2.125169in}}%
\pgfpathclose%
\pgfusepath{fill}%
\end{pgfscope}%
\begin{pgfscope}%
\pgfpathrectangle{\pgfqpoint{6.572727in}{0.474100in}}{\pgfqpoint{4.227273in}{3.318700in}}%
\pgfusepath{clip}%
\pgfsetbuttcap%
\pgfsetroundjoin%
\definecolor{currentfill}{rgb}{0.127568,0.566949,0.550556}%
\pgfsetfillcolor{currentfill}%
\pgfsetfillopacity{0.700000}%
\pgfsetlinewidth{0.000000pt}%
\definecolor{currentstroke}{rgb}{0.000000,0.000000,0.000000}%
\pgfsetstrokecolor{currentstroke}%
\pgfsetstrokeopacity{0.700000}%
\pgfsetdash{}{0pt}%
\pgfpathmoveto{\pgfqpoint{9.727474in}{1.452709in}}%
\pgfpathcurveto{\pgfqpoint{9.732518in}{1.452709in}}{\pgfqpoint{9.737355in}{1.454713in}}{\pgfqpoint{9.740922in}{1.458279in}}%
\pgfpathcurveto{\pgfqpoint{9.744488in}{1.461845in}}{\pgfqpoint{9.746492in}{1.466683in}}{\pgfqpoint{9.746492in}{1.471727in}}%
\pgfpathcurveto{\pgfqpoint{9.746492in}{1.476771in}}{\pgfqpoint{9.744488in}{1.481608in}}{\pgfqpoint{9.740922in}{1.485175in}}%
\pgfpathcurveto{\pgfqpoint{9.737355in}{1.488741in}}{\pgfqpoint{9.732518in}{1.490745in}}{\pgfqpoint{9.727474in}{1.490745in}}%
\pgfpathcurveto{\pgfqpoint{9.722430in}{1.490745in}}{\pgfqpoint{9.717593in}{1.488741in}}{\pgfqpoint{9.714026in}{1.485175in}}%
\pgfpathcurveto{\pgfqpoint{9.710460in}{1.481608in}}{\pgfqpoint{9.708456in}{1.476771in}}{\pgfqpoint{9.708456in}{1.471727in}}%
\pgfpathcurveto{\pgfqpoint{9.708456in}{1.466683in}}{\pgfqpoint{9.710460in}{1.461845in}}{\pgfqpoint{9.714026in}{1.458279in}}%
\pgfpathcurveto{\pgfqpoint{9.717593in}{1.454713in}}{\pgfqpoint{9.722430in}{1.452709in}}{\pgfqpoint{9.727474in}{1.452709in}}%
\pgfpathclose%
\pgfusepath{fill}%
\end{pgfscope}%
\begin{pgfscope}%
\pgfpathrectangle{\pgfqpoint{6.572727in}{0.474100in}}{\pgfqpoint{4.227273in}{3.318700in}}%
\pgfusepath{clip}%
\pgfsetbuttcap%
\pgfsetroundjoin%
\definecolor{currentfill}{rgb}{0.993248,0.906157,0.143936}%
\pgfsetfillcolor{currentfill}%
\pgfsetfillopacity{0.700000}%
\pgfsetlinewidth{0.000000pt}%
\definecolor{currentstroke}{rgb}{0.000000,0.000000,0.000000}%
\pgfsetstrokecolor{currentstroke}%
\pgfsetstrokeopacity{0.700000}%
\pgfsetdash{}{0pt}%
\pgfpathmoveto{\pgfqpoint{8.011155in}{3.040786in}}%
\pgfpathcurveto{\pgfqpoint{8.016199in}{3.040786in}}{\pgfqpoint{8.021037in}{3.042790in}}{\pgfqpoint{8.024603in}{3.046356in}}%
\pgfpathcurveto{\pgfqpoint{8.028170in}{3.049922in}}{\pgfqpoint{8.030173in}{3.054760in}}{\pgfqpoint{8.030173in}{3.059804in}}%
\pgfpathcurveto{\pgfqpoint{8.030173in}{3.064848in}}{\pgfqpoint{8.028170in}{3.069685in}}{\pgfqpoint{8.024603in}{3.073252in}}%
\pgfpathcurveto{\pgfqpoint{8.021037in}{3.076818in}}{\pgfqpoint{8.016199in}{3.078822in}}{\pgfqpoint{8.011155in}{3.078822in}}%
\pgfpathcurveto{\pgfqpoint{8.006112in}{3.078822in}}{\pgfqpoint{8.001274in}{3.076818in}}{\pgfqpoint{7.997707in}{3.073252in}}%
\pgfpathcurveto{\pgfqpoint{7.994141in}{3.069685in}}{\pgfqpoint{7.992137in}{3.064848in}}{\pgfqpoint{7.992137in}{3.059804in}}%
\pgfpathcurveto{\pgfqpoint{7.992137in}{3.054760in}}{\pgfqpoint{7.994141in}{3.049922in}}{\pgfqpoint{7.997707in}{3.046356in}}%
\pgfpathcurveto{\pgfqpoint{8.001274in}{3.042790in}}{\pgfqpoint{8.006112in}{3.040786in}}{\pgfqpoint{8.011155in}{3.040786in}}%
\pgfpathclose%
\pgfusepath{fill}%
\end{pgfscope}%
\begin{pgfscope}%
\pgfpathrectangle{\pgfqpoint{6.572727in}{0.474100in}}{\pgfqpoint{4.227273in}{3.318700in}}%
\pgfusepath{clip}%
\pgfsetbuttcap%
\pgfsetroundjoin%
\definecolor{currentfill}{rgb}{0.127568,0.566949,0.550556}%
\pgfsetfillcolor{currentfill}%
\pgfsetfillopacity{0.700000}%
\pgfsetlinewidth{0.000000pt}%
\definecolor{currentstroke}{rgb}{0.000000,0.000000,0.000000}%
\pgfsetstrokecolor{currentstroke}%
\pgfsetstrokeopacity{0.700000}%
\pgfsetdash{}{0pt}%
\pgfpathmoveto{\pgfqpoint{10.145355in}{1.709488in}}%
\pgfpathcurveto{\pgfqpoint{10.150399in}{1.709488in}}{\pgfqpoint{10.155237in}{1.711491in}}{\pgfqpoint{10.158803in}{1.715058in}}%
\pgfpathcurveto{\pgfqpoint{10.162370in}{1.718624in}}{\pgfqpoint{10.164373in}{1.723462in}}{\pgfqpoint{10.164373in}{1.728506in}}%
\pgfpathcurveto{\pgfqpoint{10.164373in}{1.733549in}}{\pgfqpoint{10.162370in}{1.738387in}}{\pgfqpoint{10.158803in}{1.741954in}}%
\pgfpathcurveto{\pgfqpoint{10.155237in}{1.745520in}}{\pgfqpoint{10.150399in}{1.747524in}}{\pgfqpoint{10.145355in}{1.747524in}}%
\pgfpathcurveto{\pgfqpoint{10.140312in}{1.747524in}}{\pgfqpoint{10.135474in}{1.745520in}}{\pgfqpoint{10.131907in}{1.741954in}}%
\pgfpathcurveto{\pgfqpoint{10.128341in}{1.738387in}}{\pgfqpoint{10.126337in}{1.733549in}}{\pgfqpoint{10.126337in}{1.728506in}}%
\pgfpathcurveto{\pgfqpoint{10.126337in}{1.723462in}}{\pgfqpoint{10.128341in}{1.718624in}}{\pgfqpoint{10.131907in}{1.715058in}}%
\pgfpathcurveto{\pgfqpoint{10.135474in}{1.711491in}}{\pgfqpoint{10.140312in}{1.709488in}}{\pgfqpoint{10.145355in}{1.709488in}}%
\pgfpathclose%
\pgfusepath{fill}%
\end{pgfscope}%
\begin{pgfscope}%
\pgfpathrectangle{\pgfqpoint{6.572727in}{0.474100in}}{\pgfqpoint{4.227273in}{3.318700in}}%
\pgfusepath{clip}%
\pgfsetbuttcap%
\pgfsetroundjoin%
\definecolor{currentfill}{rgb}{0.993248,0.906157,0.143936}%
\pgfsetfillcolor{currentfill}%
\pgfsetfillopacity{0.700000}%
\pgfsetlinewidth{0.000000pt}%
\definecolor{currentstroke}{rgb}{0.000000,0.000000,0.000000}%
\pgfsetstrokecolor{currentstroke}%
\pgfsetstrokeopacity{0.700000}%
\pgfsetdash{}{0pt}%
\pgfpathmoveto{\pgfqpoint{7.498594in}{3.506249in}}%
\pgfpathcurveto{\pgfqpoint{7.503638in}{3.506249in}}{\pgfqpoint{7.508475in}{3.508253in}}{\pgfqpoint{7.512042in}{3.511819in}}%
\pgfpathcurveto{\pgfqpoint{7.515608in}{3.515386in}}{\pgfqpoint{7.517612in}{3.520223in}}{\pgfqpoint{7.517612in}{3.525267in}}%
\pgfpathcurveto{\pgfqpoint{7.517612in}{3.530311in}}{\pgfqpoint{7.515608in}{3.535148in}}{\pgfqpoint{7.512042in}{3.538715in}}%
\pgfpathcurveto{\pgfqpoint{7.508475in}{3.542281in}}{\pgfqpoint{7.503638in}{3.544285in}}{\pgfqpoint{7.498594in}{3.544285in}}%
\pgfpathcurveto{\pgfqpoint{7.493550in}{3.544285in}}{\pgfqpoint{7.488713in}{3.542281in}}{\pgfqpoint{7.485146in}{3.538715in}}%
\pgfpathcurveto{\pgfqpoint{7.481580in}{3.535148in}}{\pgfqpoint{7.479576in}{3.530311in}}{\pgfqpoint{7.479576in}{3.525267in}}%
\pgfpathcurveto{\pgfqpoint{7.479576in}{3.520223in}}{\pgfqpoint{7.481580in}{3.515386in}}{\pgfqpoint{7.485146in}{3.511819in}}%
\pgfpathcurveto{\pgfqpoint{7.488713in}{3.508253in}}{\pgfqpoint{7.493550in}{3.506249in}}{\pgfqpoint{7.498594in}{3.506249in}}%
\pgfpathclose%
\pgfusepath{fill}%
\end{pgfscope}%
\begin{pgfscope}%
\pgfpathrectangle{\pgfqpoint{6.572727in}{0.474100in}}{\pgfqpoint{4.227273in}{3.318700in}}%
\pgfusepath{clip}%
\pgfsetbuttcap%
\pgfsetroundjoin%
\definecolor{currentfill}{rgb}{0.127568,0.566949,0.550556}%
\pgfsetfillcolor{currentfill}%
\pgfsetfillopacity{0.700000}%
\pgfsetlinewidth{0.000000pt}%
\definecolor{currentstroke}{rgb}{0.000000,0.000000,0.000000}%
\pgfsetstrokecolor{currentstroke}%
\pgfsetstrokeopacity{0.700000}%
\pgfsetdash{}{0pt}%
\pgfpathmoveto{\pgfqpoint{9.475706in}{1.062734in}}%
\pgfpathcurveto{\pgfqpoint{9.480749in}{1.062734in}}{\pgfqpoint{9.485587in}{1.064738in}}{\pgfqpoint{9.489154in}{1.068304in}}%
\pgfpathcurveto{\pgfqpoint{9.492720in}{1.071871in}}{\pgfqpoint{9.494724in}{1.076708in}}{\pgfqpoint{9.494724in}{1.081752in}}%
\pgfpathcurveto{\pgfqpoint{9.494724in}{1.086796in}}{\pgfqpoint{9.492720in}{1.091634in}}{\pgfqpoint{9.489154in}{1.095200in}}%
\pgfpathcurveto{\pgfqpoint{9.485587in}{1.098766in}}{\pgfqpoint{9.480749in}{1.100770in}}{\pgfqpoint{9.475706in}{1.100770in}}%
\pgfpathcurveto{\pgfqpoint{9.470662in}{1.100770in}}{\pgfqpoint{9.465824in}{1.098766in}}{\pgfqpoint{9.462258in}{1.095200in}}%
\pgfpathcurveto{\pgfqpoint{9.458692in}{1.091634in}}{\pgfqpoint{9.456688in}{1.086796in}}{\pgfqpoint{9.456688in}{1.081752in}}%
\pgfpathcurveto{\pgfqpoint{9.456688in}{1.076708in}}{\pgfqpoint{9.458692in}{1.071871in}}{\pgfqpoint{9.462258in}{1.068304in}}%
\pgfpathcurveto{\pgfqpoint{9.465824in}{1.064738in}}{\pgfqpoint{9.470662in}{1.062734in}}{\pgfqpoint{9.475706in}{1.062734in}}%
\pgfpathclose%
\pgfusepath{fill}%
\end{pgfscope}%
\begin{pgfscope}%
\pgfpathrectangle{\pgfqpoint{6.572727in}{0.474100in}}{\pgfqpoint{4.227273in}{3.318700in}}%
\pgfusepath{clip}%
\pgfsetbuttcap%
\pgfsetroundjoin%
\definecolor{currentfill}{rgb}{0.127568,0.566949,0.550556}%
\pgfsetfillcolor{currentfill}%
\pgfsetfillopacity{0.700000}%
\pgfsetlinewidth{0.000000pt}%
\definecolor{currentstroke}{rgb}{0.000000,0.000000,0.000000}%
\pgfsetstrokecolor{currentstroke}%
\pgfsetstrokeopacity{0.700000}%
\pgfsetdash{}{0pt}%
\pgfpathmoveto{\pgfqpoint{8.989587in}{2.159229in}}%
\pgfpathcurveto{\pgfqpoint{8.994631in}{2.159229in}}{\pgfqpoint{8.999469in}{2.161233in}}{\pgfqpoint{9.003035in}{2.164799in}}%
\pgfpathcurveto{\pgfqpoint{9.006602in}{2.168366in}}{\pgfqpoint{9.008605in}{2.173203in}}{\pgfqpoint{9.008605in}{2.178247in}}%
\pgfpathcurveto{\pgfqpoint{9.008605in}{2.183291in}}{\pgfqpoint{9.006602in}{2.188129in}}{\pgfqpoint{9.003035in}{2.191695in}}%
\pgfpathcurveto{\pgfqpoint{8.999469in}{2.195261in}}{\pgfqpoint{8.994631in}{2.197265in}}{\pgfqpoint{8.989587in}{2.197265in}}%
\pgfpathcurveto{\pgfqpoint{8.984544in}{2.197265in}}{\pgfqpoint{8.979706in}{2.195261in}}{\pgfqpoint{8.976139in}{2.191695in}}%
\pgfpathcurveto{\pgfqpoint{8.972573in}{2.188129in}}{\pgfqpoint{8.970569in}{2.183291in}}{\pgfqpoint{8.970569in}{2.178247in}}%
\pgfpathcurveto{\pgfqpoint{8.970569in}{2.173203in}}{\pgfqpoint{8.972573in}{2.168366in}}{\pgfqpoint{8.976139in}{2.164799in}}%
\pgfpathcurveto{\pgfqpoint{8.979706in}{2.161233in}}{\pgfqpoint{8.984544in}{2.159229in}}{\pgfqpoint{8.989587in}{2.159229in}}%
\pgfpathclose%
\pgfusepath{fill}%
\end{pgfscope}%
\begin{pgfscope}%
\pgfpathrectangle{\pgfqpoint{6.572727in}{0.474100in}}{\pgfqpoint{4.227273in}{3.318700in}}%
\pgfusepath{clip}%
\pgfsetbuttcap%
\pgfsetroundjoin%
\definecolor{currentfill}{rgb}{0.267004,0.004874,0.329415}%
\pgfsetfillcolor{currentfill}%
\pgfsetfillopacity{0.700000}%
\pgfsetlinewidth{0.000000pt}%
\definecolor{currentstroke}{rgb}{0.000000,0.000000,0.000000}%
\pgfsetstrokecolor{currentstroke}%
\pgfsetstrokeopacity{0.700000}%
\pgfsetdash{}{0pt}%
\pgfpathmoveto{\pgfqpoint{7.544241in}{0.605932in}}%
\pgfpathcurveto{\pgfqpoint{7.549285in}{0.605932in}}{\pgfqpoint{7.554123in}{0.607936in}}{\pgfqpoint{7.557689in}{0.611502in}}%
\pgfpathcurveto{\pgfqpoint{7.561256in}{0.615069in}}{\pgfqpoint{7.563260in}{0.619906in}}{\pgfqpoint{7.563260in}{0.624950in}}%
\pgfpathcurveto{\pgfqpoint{7.563260in}{0.629994in}}{\pgfqpoint{7.561256in}{0.634831in}}{\pgfqpoint{7.557689in}{0.638398in}}%
\pgfpathcurveto{\pgfqpoint{7.554123in}{0.641964in}}{\pgfqpoint{7.549285in}{0.643968in}}{\pgfqpoint{7.544241in}{0.643968in}}%
\pgfpathcurveto{\pgfqpoint{7.539198in}{0.643968in}}{\pgfqpoint{7.534360in}{0.641964in}}{\pgfqpoint{7.530794in}{0.638398in}}%
\pgfpathcurveto{\pgfqpoint{7.527227in}{0.634831in}}{\pgfqpoint{7.525223in}{0.629994in}}{\pgfqpoint{7.525223in}{0.624950in}}%
\pgfpathcurveto{\pgfqpoint{7.525223in}{0.619906in}}{\pgfqpoint{7.527227in}{0.615069in}}{\pgfqpoint{7.530794in}{0.611502in}}%
\pgfpathcurveto{\pgfqpoint{7.534360in}{0.607936in}}{\pgfqpoint{7.539198in}{0.605932in}}{\pgfqpoint{7.544241in}{0.605932in}}%
\pgfpathclose%
\pgfusepath{fill}%
\end{pgfscope}%
\begin{pgfscope}%
\pgfpathrectangle{\pgfqpoint{6.572727in}{0.474100in}}{\pgfqpoint{4.227273in}{3.318700in}}%
\pgfusepath{clip}%
\pgfsetbuttcap%
\pgfsetroundjoin%
\definecolor{currentfill}{rgb}{0.267004,0.004874,0.329415}%
\pgfsetfillcolor{currentfill}%
\pgfsetfillopacity{0.700000}%
\pgfsetlinewidth{0.000000pt}%
\definecolor{currentstroke}{rgb}{0.000000,0.000000,0.000000}%
\pgfsetstrokecolor{currentstroke}%
\pgfsetstrokeopacity{0.700000}%
\pgfsetdash{}{0pt}%
\pgfpathmoveto{\pgfqpoint{8.273392in}{1.522490in}}%
\pgfpathcurveto{\pgfqpoint{8.278436in}{1.522490in}}{\pgfqpoint{8.283274in}{1.524494in}}{\pgfqpoint{8.286840in}{1.528061in}}%
\pgfpathcurveto{\pgfqpoint{8.290407in}{1.531627in}}{\pgfqpoint{8.292411in}{1.536465in}}{\pgfqpoint{8.292411in}{1.541509in}}%
\pgfpathcurveto{\pgfqpoint{8.292411in}{1.546552in}}{\pgfqpoint{8.290407in}{1.551390in}}{\pgfqpoint{8.286840in}{1.554956in}}%
\pgfpathcurveto{\pgfqpoint{8.283274in}{1.558523in}}{\pgfqpoint{8.278436in}{1.560527in}}{\pgfqpoint{8.273392in}{1.560527in}}%
\pgfpathcurveto{\pgfqpoint{8.268349in}{1.560527in}}{\pgfqpoint{8.263511in}{1.558523in}}{\pgfqpoint{8.259945in}{1.554956in}}%
\pgfpathcurveto{\pgfqpoint{8.256378in}{1.551390in}}{\pgfqpoint{8.254374in}{1.546552in}}{\pgfqpoint{8.254374in}{1.541509in}}%
\pgfpathcurveto{\pgfqpoint{8.254374in}{1.536465in}}{\pgfqpoint{8.256378in}{1.531627in}}{\pgfqpoint{8.259945in}{1.528061in}}%
\pgfpathcurveto{\pgfqpoint{8.263511in}{1.524494in}}{\pgfqpoint{8.268349in}{1.522490in}}{\pgfqpoint{8.273392in}{1.522490in}}%
\pgfpathclose%
\pgfusepath{fill}%
\end{pgfscope}%
\begin{pgfscope}%
\pgfpathrectangle{\pgfqpoint{6.572727in}{0.474100in}}{\pgfqpoint{4.227273in}{3.318700in}}%
\pgfusepath{clip}%
\pgfsetbuttcap%
\pgfsetroundjoin%
\definecolor{currentfill}{rgb}{0.267004,0.004874,0.329415}%
\pgfsetfillcolor{currentfill}%
\pgfsetfillopacity{0.700000}%
\pgfsetlinewidth{0.000000pt}%
\definecolor{currentstroke}{rgb}{0.000000,0.000000,0.000000}%
\pgfsetstrokecolor{currentstroke}%
\pgfsetstrokeopacity{0.700000}%
\pgfsetdash{}{0pt}%
\pgfpathmoveto{\pgfqpoint{7.494979in}{1.252205in}}%
\pgfpathcurveto{\pgfqpoint{7.500022in}{1.252205in}}{\pgfqpoint{7.504860in}{1.254209in}}{\pgfqpoint{7.508427in}{1.257775in}}%
\pgfpathcurveto{\pgfqpoint{7.511993in}{1.261342in}}{\pgfqpoint{7.513997in}{1.266180in}}{\pgfqpoint{7.513997in}{1.271223in}}%
\pgfpathcurveto{\pgfqpoint{7.513997in}{1.276267in}}{\pgfqpoint{7.511993in}{1.281105in}}{\pgfqpoint{7.508427in}{1.284671in}}%
\pgfpathcurveto{\pgfqpoint{7.504860in}{1.288238in}}{\pgfqpoint{7.500022in}{1.290241in}}{\pgfqpoint{7.494979in}{1.290241in}}%
\pgfpathcurveto{\pgfqpoint{7.489935in}{1.290241in}}{\pgfqpoint{7.485097in}{1.288238in}}{\pgfqpoint{7.481531in}{1.284671in}}%
\pgfpathcurveto{\pgfqpoint{7.477964in}{1.281105in}}{\pgfqpoint{7.475960in}{1.276267in}}{\pgfqpoint{7.475960in}{1.271223in}}%
\pgfpathcurveto{\pgfqpoint{7.475960in}{1.266180in}}{\pgfqpoint{7.477964in}{1.261342in}}{\pgfqpoint{7.481531in}{1.257775in}}%
\pgfpathcurveto{\pgfqpoint{7.485097in}{1.254209in}}{\pgfqpoint{7.489935in}{1.252205in}}{\pgfqpoint{7.494979in}{1.252205in}}%
\pgfpathclose%
\pgfusepath{fill}%
\end{pgfscope}%
\begin{pgfscope}%
\pgfpathrectangle{\pgfqpoint{6.572727in}{0.474100in}}{\pgfqpoint{4.227273in}{3.318700in}}%
\pgfusepath{clip}%
\pgfsetbuttcap%
\pgfsetroundjoin%
\definecolor{currentfill}{rgb}{0.993248,0.906157,0.143936}%
\pgfsetfillcolor{currentfill}%
\pgfsetfillopacity{0.700000}%
\pgfsetlinewidth{0.000000pt}%
\definecolor{currentstroke}{rgb}{0.000000,0.000000,0.000000}%
\pgfsetstrokecolor{currentstroke}%
\pgfsetstrokeopacity{0.700000}%
\pgfsetdash{}{0pt}%
\pgfpathmoveto{\pgfqpoint{7.871762in}{2.271436in}}%
\pgfpathcurveto{\pgfqpoint{7.876806in}{2.271436in}}{\pgfqpoint{7.881643in}{2.273440in}}{\pgfqpoint{7.885210in}{2.277007in}}%
\pgfpathcurveto{\pgfqpoint{7.888776in}{2.280573in}}{\pgfqpoint{7.890780in}{2.285411in}}{\pgfqpoint{7.890780in}{2.290455in}}%
\pgfpathcurveto{\pgfqpoint{7.890780in}{2.295498in}}{\pgfqpoint{7.888776in}{2.300336in}}{\pgfqpoint{7.885210in}{2.303902in}}%
\pgfpathcurveto{\pgfqpoint{7.881643in}{2.307469in}}{\pgfqpoint{7.876806in}{2.309473in}}{\pgfqpoint{7.871762in}{2.309473in}}%
\pgfpathcurveto{\pgfqpoint{7.866718in}{2.309473in}}{\pgfqpoint{7.861881in}{2.307469in}}{\pgfqpoint{7.858314in}{2.303902in}}%
\pgfpathcurveto{\pgfqpoint{7.854748in}{2.300336in}}{\pgfqpoint{7.852744in}{2.295498in}}{\pgfqpoint{7.852744in}{2.290455in}}%
\pgfpathcurveto{\pgfqpoint{7.852744in}{2.285411in}}{\pgfqpoint{7.854748in}{2.280573in}}{\pgfqpoint{7.858314in}{2.277007in}}%
\pgfpathcurveto{\pgfqpoint{7.861881in}{2.273440in}}{\pgfqpoint{7.866718in}{2.271436in}}{\pgfqpoint{7.871762in}{2.271436in}}%
\pgfpathclose%
\pgfusepath{fill}%
\end{pgfscope}%
\begin{pgfscope}%
\pgfpathrectangle{\pgfqpoint{6.572727in}{0.474100in}}{\pgfqpoint{4.227273in}{3.318700in}}%
\pgfusepath{clip}%
\pgfsetbuttcap%
\pgfsetroundjoin%
\definecolor{currentfill}{rgb}{0.127568,0.566949,0.550556}%
\pgfsetfillcolor{currentfill}%
\pgfsetfillopacity{0.700000}%
\pgfsetlinewidth{0.000000pt}%
\definecolor{currentstroke}{rgb}{0.000000,0.000000,0.000000}%
\pgfsetstrokecolor{currentstroke}%
\pgfsetstrokeopacity{0.700000}%
\pgfsetdash{}{0pt}%
\pgfpathmoveto{\pgfqpoint{9.449838in}{1.647931in}}%
\pgfpathcurveto{\pgfqpoint{9.454882in}{1.647931in}}{\pgfqpoint{9.459720in}{1.649935in}}{\pgfqpoint{9.463286in}{1.653502in}}%
\pgfpathcurveto{\pgfqpoint{9.466852in}{1.657068in}}{\pgfqpoint{9.468856in}{1.661906in}}{\pgfqpoint{9.468856in}{1.666949in}}%
\pgfpathcurveto{\pgfqpoint{9.468856in}{1.671993in}}{\pgfqpoint{9.466852in}{1.676831in}}{\pgfqpoint{9.463286in}{1.680397in}}%
\pgfpathcurveto{\pgfqpoint{9.459720in}{1.683964in}}{\pgfqpoint{9.454882in}{1.685968in}}{\pgfqpoint{9.449838in}{1.685968in}}%
\pgfpathcurveto{\pgfqpoint{9.444795in}{1.685968in}}{\pgfqpoint{9.439957in}{1.683964in}}{\pgfqpoint{9.436390in}{1.680397in}}%
\pgfpathcurveto{\pgfqpoint{9.432824in}{1.676831in}}{\pgfqpoint{9.430820in}{1.671993in}}{\pgfqpoint{9.430820in}{1.666949in}}%
\pgfpathcurveto{\pgfqpoint{9.430820in}{1.661906in}}{\pgfqpoint{9.432824in}{1.657068in}}{\pgfqpoint{9.436390in}{1.653502in}}%
\pgfpathcurveto{\pgfqpoint{9.439957in}{1.649935in}}{\pgfqpoint{9.444795in}{1.647931in}}{\pgfqpoint{9.449838in}{1.647931in}}%
\pgfpathclose%
\pgfusepath{fill}%
\end{pgfscope}%
\begin{pgfscope}%
\pgfpathrectangle{\pgfqpoint{6.572727in}{0.474100in}}{\pgfqpoint{4.227273in}{3.318700in}}%
\pgfusepath{clip}%
\pgfsetbuttcap%
\pgfsetroundjoin%
\definecolor{currentfill}{rgb}{0.127568,0.566949,0.550556}%
\pgfsetfillcolor{currentfill}%
\pgfsetfillopacity{0.700000}%
\pgfsetlinewidth{0.000000pt}%
\definecolor{currentstroke}{rgb}{0.000000,0.000000,0.000000}%
\pgfsetstrokecolor{currentstroke}%
\pgfsetstrokeopacity{0.700000}%
\pgfsetdash{}{0pt}%
\pgfpathmoveto{\pgfqpoint{9.454653in}{1.681134in}}%
\pgfpathcurveto{\pgfqpoint{9.459696in}{1.681134in}}{\pgfqpoint{9.464534in}{1.683138in}}{\pgfqpoint{9.468101in}{1.686705in}}%
\pgfpathcurveto{\pgfqpoint{9.471667in}{1.690271in}}{\pgfqpoint{9.473671in}{1.695109in}}{\pgfqpoint{9.473671in}{1.700152in}}%
\pgfpathcurveto{\pgfqpoint{9.473671in}{1.705196in}}{\pgfqpoint{9.471667in}{1.710034in}}{\pgfqpoint{9.468101in}{1.713600in}}%
\pgfpathcurveto{\pgfqpoint{9.464534in}{1.717167in}}{\pgfqpoint{9.459696in}{1.719171in}}{\pgfqpoint{9.454653in}{1.719171in}}%
\pgfpathcurveto{\pgfqpoint{9.449609in}{1.719171in}}{\pgfqpoint{9.444771in}{1.717167in}}{\pgfqpoint{9.441205in}{1.713600in}}%
\pgfpathcurveto{\pgfqpoint{9.437639in}{1.710034in}}{\pgfqpoint{9.435635in}{1.705196in}}{\pgfqpoint{9.435635in}{1.700152in}}%
\pgfpathcurveto{\pgfqpoint{9.435635in}{1.695109in}}{\pgfqpoint{9.437639in}{1.690271in}}{\pgfqpoint{9.441205in}{1.686705in}}%
\pgfpathcurveto{\pgfqpoint{9.444771in}{1.683138in}}{\pgfqpoint{9.449609in}{1.681134in}}{\pgfqpoint{9.454653in}{1.681134in}}%
\pgfpathclose%
\pgfusepath{fill}%
\end{pgfscope}%
\begin{pgfscope}%
\pgfpathrectangle{\pgfqpoint{6.572727in}{0.474100in}}{\pgfqpoint{4.227273in}{3.318700in}}%
\pgfusepath{clip}%
\pgfsetbuttcap%
\pgfsetroundjoin%
\definecolor{currentfill}{rgb}{0.993248,0.906157,0.143936}%
\pgfsetfillcolor{currentfill}%
\pgfsetfillopacity{0.700000}%
\pgfsetlinewidth{0.000000pt}%
\definecolor{currentstroke}{rgb}{0.000000,0.000000,0.000000}%
\pgfsetstrokecolor{currentstroke}%
\pgfsetstrokeopacity{0.700000}%
\pgfsetdash{}{0pt}%
\pgfpathmoveto{\pgfqpoint{8.102479in}{2.712583in}}%
\pgfpathcurveto{\pgfqpoint{8.107522in}{2.712583in}}{\pgfqpoint{8.112360in}{2.714586in}}{\pgfqpoint{8.115926in}{2.718153in}}%
\pgfpathcurveto{\pgfqpoint{8.119493in}{2.721719in}}{\pgfqpoint{8.121497in}{2.726557in}}{\pgfqpoint{8.121497in}{2.731601in}}%
\pgfpathcurveto{\pgfqpoint{8.121497in}{2.736644in}}{\pgfqpoint{8.119493in}{2.741482in}}{\pgfqpoint{8.115926in}{2.745049in}}%
\pgfpathcurveto{\pgfqpoint{8.112360in}{2.748615in}}{\pgfqpoint{8.107522in}{2.750619in}}{\pgfqpoint{8.102479in}{2.750619in}}%
\pgfpathcurveto{\pgfqpoint{8.097435in}{2.750619in}}{\pgfqpoint{8.092597in}{2.748615in}}{\pgfqpoint{8.089031in}{2.745049in}}%
\pgfpathcurveto{\pgfqpoint{8.085464in}{2.741482in}}{\pgfqpoint{8.083460in}{2.736644in}}{\pgfqpoint{8.083460in}{2.731601in}}%
\pgfpathcurveto{\pgfqpoint{8.083460in}{2.726557in}}{\pgfqpoint{8.085464in}{2.721719in}}{\pgfqpoint{8.089031in}{2.718153in}}%
\pgfpathcurveto{\pgfqpoint{8.092597in}{2.714586in}}{\pgfqpoint{8.097435in}{2.712583in}}{\pgfqpoint{8.102479in}{2.712583in}}%
\pgfpathclose%
\pgfusepath{fill}%
\end{pgfscope}%
\begin{pgfscope}%
\pgfpathrectangle{\pgfqpoint{6.572727in}{0.474100in}}{\pgfqpoint{4.227273in}{3.318700in}}%
\pgfusepath{clip}%
\pgfsetbuttcap%
\pgfsetroundjoin%
\definecolor{currentfill}{rgb}{0.993248,0.906157,0.143936}%
\pgfsetfillcolor{currentfill}%
\pgfsetfillopacity{0.700000}%
\pgfsetlinewidth{0.000000pt}%
\definecolor{currentstroke}{rgb}{0.000000,0.000000,0.000000}%
\pgfsetstrokecolor{currentstroke}%
\pgfsetstrokeopacity{0.700000}%
\pgfsetdash{}{0pt}%
\pgfpathmoveto{\pgfqpoint{8.361357in}{2.988342in}}%
\pgfpathcurveto{\pgfqpoint{8.366400in}{2.988342in}}{\pgfqpoint{8.371238in}{2.990346in}}{\pgfqpoint{8.374804in}{2.993912in}}%
\pgfpathcurveto{\pgfqpoint{8.378371in}{2.997479in}}{\pgfqpoint{8.380375in}{3.002316in}}{\pgfqpoint{8.380375in}{3.007360in}}%
\pgfpathcurveto{\pgfqpoint{8.380375in}{3.012404in}}{\pgfqpoint{8.378371in}{3.017242in}}{\pgfqpoint{8.374804in}{3.020808in}}%
\pgfpathcurveto{\pgfqpoint{8.371238in}{3.024374in}}{\pgfqpoint{8.366400in}{3.026378in}}{\pgfqpoint{8.361357in}{3.026378in}}%
\pgfpathcurveto{\pgfqpoint{8.356313in}{3.026378in}}{\pgfqpoint{8.351475in}{3.024374in}}{\pgfqpoint{8.347909in}{3.020808in}}%
\pgfpathcurveto{\pgfqpoint{8.344342in}{3.017242in}}{\pgfqpoint{8.342338in}{3.012404in}}{\pgfqpoint{8.342338in}{3.007360in}}%
\pgfpathcurveto{\pgfqpoint{8.342338in}{3.002316in}}{\pgfqpoint{8.344342in}{2.997479in}}{\pgfqpoint{8.347909in}{2.993912in}}%
\pgfpathcurveto{\pgfqpoint{8.351475in}{2.990346in}}{\pgfqpoint{8.356313in}{2.988342in}}{\pgfqpoint{8.361357in}{2.988342in}}%
\pgfpathclose%
\pgfusepath{fill}%
\end{pgfscope}%
\begin{pgfscope}%
\pgfpathrectangle{\pgfqpoint{6.572727in}{0.474100in}}{\pgfqpoint{4.227273in}{3.318700in}}%
\pgfusepath{clip}%
\pgfsetbuttcap%
\pgfsetroundjoin%
\definecolor{currentfill}{rgb}{0.127568,0.566949,0.550556}%
\pgfsetfillcolor{currentfill}%
\pgfsetfillopacity{0.700000}%
\pgfsetlinewidth{0.000000pt}%
\definecolor{currentstroke}{rgb}{0.000000,0.000000,0.000000}%
\pgfsetstrokecolor{currentstroke}%
\pgfsetstrokeopacity{0.700000}%
\pgfsetdash{}{0pt}%
\pgfpathmoveto{\pgfqpoint{9.025197in}{1.511243in}}%
\pgfpathcurveto{\pgfqpoint{9.030241in}{1.511243in}}{\pgfqpoint{9.035079in}{1.513247in}}{\pgfqpoint{9.038645in}{1.516813in}}%
\pgfpathcurveto{\pgfqpoint{9.042212in}{1.520380in}}{\pgfqpoint{9.044216in}{1.525217in}}{\pgfqpoint{9.044216in}{1.530261in}}%
\pgfpathcurveto{\pgfqpoint{9.044216in}{1.535305in}}{\pgfqpoint{9.042212in}{1.540143in}}{\pgfqpoint{9.038645in}{1.543709in}}%
\pgfpathcurveto{\pgfqpoint{9.035079in}{1.547275in}}{\pgfqpoint{9.030241in}{1.549279in}}{\pgfqpoint{9.025197in}{1.549279in}}%
\pgfpathcurveto{\pgfqpoint{9.020154in}{1.549279in}}{\pgfqpoint{9.015316in}{1.547275in}}{\pgfqpoint{9.011750in}{1.543709in}}%
\pgfpathcurveto{\pgfqpoint{9.008183in}{1.540143in}}{\pgfqpoint{9.006179in}{1.535305in}}{\pgfqpoint{9.006179in}{1.530261in}}%
\pgfpathcurveto{\pgfqpoint{9.006179in}{1.525217in}}{\pgfqpoint{9.008183in}{1.520380in}}{\pgfqpoint{9.011750in}{1.516813in}}%
\pgfpathcurveto{\pgfqpoint{9.015316in}{1.513247in}}{\pgfqpoint{9.020154in}{1.511243in}}{\pgfqpoint{9.025197in}{1.511243in}}%
\pgfpathclose%
\pgfusepath{fill}%
\end{pgfscope}%
\begin{pgfscope}%
\pgfpathrectangle{\pgfqpoint{6.572727in}{0.474100in}}{\pgfqpoint{4.227273in}{3.318700in}}%
\pgfusepath{clip}%
\pgfsetbuttcap%
\pgfsetroundjoin%
\definecolor{currentfill}{rgb}{0.993248,0.906157,0.143936}%
\pgfsetfillcolor{currentfill}%
\pgfsetfillopacity{0.700000}%
\pgfsetlinewidth{0.000000pt}%
\definecolor{currentstroke}{rgb}{0.000000,0.000000,0.000000}%
\pgfsetstrokecolor{currentstroke}%
\pgfsetstrokeopacity{0.700000}%
\pgfsetdash{}{0pt}%
\pgfpathmoveto{\pgfqpoint{8.583198in}{2.920835in}}%
\pgfpathcurveto{\pgfqpoint{8.588241in}{2.920835in}}{\pgfqpoint{8.593079in}{2.922839in}}{\pgfqpoint{8.596645in}{2.926405in}}%
\pgfpathcurveto{\pgfqpoint{8.600212in}{2.929972in}}{\pgfqpoint{8.602216in}{2.934809in}}{\pgfqpoint{8.602216in}{2.939853in}}%
\pgfpathcurveto{\pgfqpoint{8.602216in}{2.944897in}}{\pgfqpoint{8.600212in}{2.949735in}}{\pgfqpoint{8.596645in}{2.953301in}}%
\pgfpathcurveto{\pgfqpoint{8.593079in}{2.956867in}}{\pgfqpoint{8.588241in}{2.958871in}}{\pgfqpoint{8.583198in}{2.958871in}}%
\pgfpathcurveto{\pgfqpoint{8.578154in}{2.958871in}}{\pgfqpoint{8.573316in}{2.956867in}}{\pgfqpoint{8.569750in}{2.953301in}}%
\pgfpathcurveto{\pgfqpoint{8.566183in}{2.949735in}}{\pgfqpoint{8.564179in}{2.944897in}}{\pgfqpoint{8.564179in}{2.939853in}}%
\pgfpathcurveto{\pgfqpoint{8.564179in}{2.934809in}}{\pgfqpoint{8.566183in}{2.929972in}}{\pgfqpoint{8.569750in}{2.926405in}}%
\pgfpathcurveto{\pgfqpoint{8.573316in}{2.922839in}}{\pgfqpoint{8.578154in}{2.920835in}}{\pgfqpoint{8.583198in}{2.920835in}}%
\pgfpathclose%
\pgfusepath{fill}%
\end{pgfscope}%
\begin{pgfscope}%
\pgfpathrectangle{\pgfqpoint{6.572727in}{0.474100in}}{\pgfqpoint{4.227273in}{3.318700in}}%
\pgfusepath{clip}%
\pgfsetbuttcap%
\pgfsetroundjoin%
\definecolor{currentfill}{rgb}{0.993248,0.906157,0.143936}%
\pgfsetfillcolor{currentfill}%
\pgfsetfillopacity{0.700000}%
\pgfsetlinewidth{0.000000pt}%
\definecolor{currentstroke}{rgb}{0.000000,0.000000,0.000000}%
\pgfsetstrokecolor{currentstroke}%
\pgfsetstrokeopacity{0.700000}%
\pgfsetdash{}{0pt}%
\pgfpathmoveto{\pgfqpoint{8.106954in}{2.402770in}}%
\pgfpathcurveto{\pgfqpoint{8.111998in}{2.402770in}}{\pgfqpoint{8.116836in}{2.404774in}}{\pgfqpoint{8.120402in}{2.408340in}}%
\pgfpathcurveto{\pgfqpoint{8.123969in}{2.411907in}}{\pgfqpoint{8.125972in}{2.416744in}}{\pgfqpoint{8.125972in}{2.421788in}}%
\pgfpathcurveto{\pgfqpoint{8.125972in}{2.426832in}}{\pgfqpoint{8.123969in}{2.431670in}}{\pgfqpoint{8.120402in}{2.435236in}}%
\pgfpathcurveto{\pgfqpoint{8.116836in}{2.438802in}}{\pgfqpoint{8.111998in}{2.440806in}}{\pgfqpoint{8.106954in}{2.440806in}}%
\pgfpathcurveto{\pgfqpoint{8.101911in}{2.440806in}}{\pgfqpoint{8.097073in}{2.438802in}}{\pgfqpoint{8.093506in}{2.435236in}}%
\pgfpathcurveto{\pgfqpoint{8.089940in}{2.431670in}}{\pgfqpoint{8.087936in}{2.426832in}}{\pgfqpoint{8.087936in}{2.421788in}}%
\pgfpathcurveto{\pgfqpoint{8.087936in}{2.416744in}}{\pgfqpoint{8.089940in}{2.411907in}}{\pgfqpoint{8.093506in}{2.408340in}}%
\pgfpathcurveto{\pgfqpoint{8.097073in}{2.404774in}}{\pgfqpoint{8.101911in}{2.402770in}}{\pgfqpoint{8.106954in}{2.402770in}}%
\pgfpathclose%
\pgfusepath{fill}%
\end{pgfscope}%
\begin{pgfscope}%
\pgfpathrectangle{\pgfqpoint{6.572727in}{0.474100in}}{\pgfqpoint{4.227273in}{3.318700in}}%
\pgfusepath{clip}%
\pgfsetbuttcap%
\pgfsetroundjoin%
\definecolor{currentfill}{rgb}{0.993248,0.906157,0.143936}%
\pgfsetfillcolor{currentfill}%
\pgfsetfillopacity{0.700000}%
\pgfsetlinewidth{0.000000pt}%
\definecolor{currentstroke}{rgb}{0.000000,0.000000,0.000000}%
\pgfsetstrokecolor{currentstroke}%
\pgfsetstrokeopacity{0.700000}%
\pgfsetdash{}{0pt}%
\pgfpathmoveto{\pgfqpoint{8.002811in}{2.641515in}}%
\pgfpathcurveto{\pgfqpoint{8.007855in}{2.641515in}}{\pgfqpoint{8.012693in}{2.643519in}}{\pgfqpoint{8.016259in}{2.647085in}}%
\pgfpathcurveto{\pgfqpoint{8.019825in}{2.650652in}}{\pgfqpoint{8.021829in}{2.655489in}}{\pgfqpoint{8.021829in}{2.660533in}}%
\pgfpathcurveto{\pgfqpoint{8.021829in}{2.665577in}}{\pgfqpoint{8.019825in}{2.670414in}}{\pgfqpoint{8.016259in}{2.673981in}}%
\pgfpathcurveto{\pgfqpoint{8.012693in}{2.677547in}}{\pgfqpoint{8.007855in}{2.679551in}}{\pgfqpoint{8.002811in}{2.679551in}}%
\pgfpathcurveto{\pgfqpoint{7.997767in}{2.679551in}}{\pgfqpoint{7.992930in}{2.677547in}}{\pgfqpoint{7.989363in}{2.673981in}}%
\pgfpathcurveto{\pgfqpoint{7.985797in}{2.670414in}}{\pgfqpoint{7.983793in}{2.665577in}}{\pgfqpoint{7.983793in}{2.660533in}}%
\pgfpathcurveto{\pgfqpoint{7.983793in}{2.655489in}}{\pgfqpoint{7.985797in}{2.650652in}}{\pgfqpoint{7.989363in}{2.647085in}}%
\pgfpathcurveto{\pgfqpoint{7.992930in}{2.643519in}}{\pgfqpoint{7.997767in}{2.641515in}}{\pgfqpoint{8.002811in}{2.641515in}}%
\pgfpathclose%
\pgfusepath{fill}%
\end{pgfscope}%
\begin{pgfscope}%
\pgfpathrectangle{\pgfqpoint{6.572727in}{0.474100in}}{\pgfqpoint{4.227273in}{3.318700in}}%
\pgfusepath{clip}%
\pgfsetbuttcap%
\pgfsetroundjoin%
\definecolor{currentfill}{rgb}{0.267004,0.004874,0.329415}%
\pgfsetfillcolor{currentfill}%
\pgfsetfillopacity{0.700000}%
\pgfsetlinewidth{0.000000pt}%
\definecolor{currentstroke}{rgb}{0.000000,0.000000,0.000000}%
\pgfsetstrokecolor{currentstroke}%
\pgfsetstrokeopacity{0.700000}%
\pgfsetdash{}{0pt}%
\pgfpathmoveto{\pgfqpoint{7.933425in}{0.960938in}}%
\pgfpathcurveto{\pgfqpoint{7.938469in}{0.960938in}}{\pgfqpoint{7.943306in}{0.962942in}}{\pgfqpoint{7.946873in}{0.966509in}}%
\pgfpathcurveto{\pgfqpoint{7.950439in}{0.970075in}}{\pgfqpoint{7.952443in}{0.974913in}}{\pgfqpoint{7.952443in}{0.979957in}}%
\pgfpathcurveto{\pgfqpoint{7.952443in}{0.985000in}}{\pgfqpoint{7.950439in}{0.989838in}}{\pgfqpoint{7.946873in}{0.993404in}}%
\pgfpathcurveto{\pgfqpoint{7.943306in}{0.996971in}}{\pgfqpoint{7.938469in}{0.998975in}}{\pgfqpoint{7.933425in}{0.998975in}}%
\pgfpathcurveto{\pgfqpoint{7.928381in}{0.998975in}}{\pgfqpoint{7.923543in}{0.996971in}}{\pgfqpoint{7.919977in}{0.993404in}}%
\pgfpathcurveto{\pgfqpoint{7.916411in}{0.989838in}}{\pgfqpoint{7.914407in}{0.985000in}}{\pgfqpoint{7.914407in}{0.979957in}}%
\pgfpathcurveto{\pgfqpoint{7.914407in}{0.974913in}}{\pgfqpoint{7.916411in}{0.970075in}}{\pgfqpoint{7.919977in}{0.966509in}}%
\pgfpathcurveto{\pgfqpoint{7.923543in}{0.962942in}}{\pgfqpoint{7.928381in}{0.960938in}}{\pgfqpoint{7.933425in}{0.960938in}}%
\pgfpathclose%
\pgfusepath{fill}%
\end{pgfscope}%
\begin{pgfscope}%
\pgfpathrectangle{\pgfqpoint{6.572727in}{0.474100in}}{\pgfqpoint{4.227273in}{3.318700in}}%
\pgfusepath{clip}%
\pgfsetbuttcap%
\pgfsetroundjoin%
\definecolor{currentfill}{rgb}{0.993248,0.906157,0.143936}%
\pgfsetfillcolor{currentfill}%
\pgfsetfillopacity{0.700000}%
\pgfsetlinewidth{0.000000pt}%
\definecolor{currentstroke}{rgb}{0.000000,0.000000,0.000000}%
\pgfsetstrokecolor{currentstroke}%
\pgfsetstrokeopacity{0.700000}%
\pgfsetdash{}{0pt}%
\pgfpathmoveto{\pgfqpoint{8.204208in}{2.257066in}}%
\pgfpathcurveto{\pgfqpoint{8.209252in}{2.257066in}}{\pgfqpoint{8.214089in}{2.259070in}}{\pgfqpoint{8.217656in}{2.262637in}}%
\pgfpathcurveto{\pgfqpoint{8.221222in}{2.266203in}}{\pgfqpoint{8.223226in}{2.271041in}}{\pgfqpoint{8.223226in}{2.276084in}}%
\pgfpathcurveto{\pgfqpoint{8.223226in}{2.281128in}}{\pgfqpoint{8.221222in}{2.285966in}}{\pgfqpoint{8.217656in}{2.289532in}}%
\pgfpathcurveto{\pgfqpoint{8.214089in}{2.293099in}}{\pgfqpoint{8.209252in}{2.295103in}}{\pgfqpoint{8.204208in}{2.295103in}}%
\pgfpathcurveto{\pgfqpoint{8.199164in}{2.295103in}}{\pgfqpoint{8.194327in}{2.293099in}}{\pgfqpoint{8.190760in}{2.289532in}}%
\pgfpathcurveto{\pgfqpoint{8.187194in}{2.285966in}}{\pgfqpoint{8.185190in}{2.281128in}}{\pgfqpoint{8.185190in}{2.276084in}}%
\pgfpathcurveto{\pgfqpoint{8.185190in}{2.271041in}}{\pgfqpoint{8.187194in}{2.266203in}}{\pgfqpoint{8.190760in}{2.262637in}}%
\pgfpathcurveto{\pgfqpoint{8.194327in}{2.259070in}}{\pgfqpoint{8.199164in}{2.257066in}}{\pgfqpoint{8.204208in}{2.257066in}}%
\pgfpathclose%
\pgfusepath{fill}%
\end{pgfscope}%
\begin{pgfscope}%
\pgfpathrectangle{\pgfqpoint{6.572727in}{0.474100in}}{\pgfqpoint{4.227273in}{3.318700in}}%
\pgfusepath{clip}%
\pgfsetbuttcap%
\pgfsetroundjoin%
\definecolor{currentfill}{rgb}{0.267004,0.004874,0.329415}%
\pgfsetfillcolor{currentfill}%
\pgfsetfillopacity{0.700000}%
\pgfsetlinewidth{0.000000pt}%
\definecolor{currentstroke}{rgb}{0.000000,0.000000,0.000000}%
\pgfsetstrokecolor{currentstroke}%
\pgfsetstrokeopacity{0.700000}%
\pgfsetdash{}{0pt}%
\pgfpathmoveto{\pgfqpoint{7.617970in}{2.034693in}}%
\pgfpathcurveto{\pgfqpoint{7.623014in}{2.034693in}}{\pgfqpoint{7.627851in}{2.036696in}}{\pgfqpoint{7.631418in}{2.040263in}}%
\pgfpathcurveto{\pgfqpoint{7.634984in}{2.043829in}}{\pgfqpoint{7.636988in}{2.048667in}}{\pgfqpoint{7.636988in}{2.053711in}}%
\pgfpathcurveto{\pgfqpoint{7.636988in}{2.058754in}}{\pgfqpoint{7.634984in}{2.063592in}}{\pgfqpoint{7.631418in}{2.067159in}}%
\pgfpathcurveto{\pgfqpoint{7.627851in}{2.070725in}}{\pgfqpoint{7.623014in}{2.072729in}}{\pgfqpoint{7.617970in}{2.072729in}}%
\pgfpathcurveto{\pgfqpoint{7.612926in}{2.072729in}}{\pgfqpoint{7.608089in}{2.070725in}}{\pgfqpoint{7.604522in}{2.067159in}}%
\pgfpathcurveto{\pgfqpoint{7.600956in}{2.063592in}}{\pgfqpoint{7.598952in}{2.058754in}}{\pgfqpoint{7.598952in}{2.053711in}}%
\pgfpathcurveto{\pgfqpoint{7.598952in}{2.048667in}}{\pgfqpoint{7.600956in}{2.043829in}}{\pgfqpoint{7.604522in}{2.040263in}}%
\pgfpathcurveto{\pgfqpoint{7.608089in}{2.036696in}}{\pgfqpoint{7.612926in}{2.034693in}}{\pgfqpoint{7.617970in}{2.034693in}}%
\pgfpathclose%
\pgfusepath{fill}%
\end{pgfscope}%
\begin{pgfscope}%
\pgfpathrectangle{\pgfqpoint{6.572727in}{0.474100in}}{\pgfqpoint{4.227273in}{3.318700in}}%
\pgfusepath{clip}%
\pgfsetbuttcap%
\pgfsetroundjoin%
\definecolor{currentfill}{rgb}{0.127568,0.566949,0.550556}%
\pgfsetfillcolor{currentfill}%
\pgfsetfillopacity{0.700000}%
\pgfsetlinewidth{0.000000pt}%
\definecolor{currentstroke}{rgb}{0.000000,0.000000,0.000000}%
\pgfsetstrokecolor{currentstroke}%
\pgfsetstrokeopacity{0.700000}%
\pgfsetdash{}{0pt}%
\pgfpathmoveto{\pgfqpoint{9.077907in}{1.117363in}}%
\pgfpathcurveto{\pgfqpoint{9.082951in}{1.117363in}}{\pgfqpoint{9.087788in}{1.119367in}}{\pgfqpoint{9.091355in}{1.122934in}}%
\pgfpathcurveto{\pgfqpoint{9.094921in}{1.126500in}}{\pgfqpoint{9.096925in}{1.131338in}}{\pgfqpoint{9.096925in}{1.136382in}}%
\pgfpathcurveto{\pgfqpoint{9.096925in}{1.141425in}}{\pgfqpoint{9.094921in}{1.146263in}}{\pgfqpoint{9.091355in}{1.149829in}}%
\pgfpathcurveto{\pgfqpoint{9.087788in}{1.153396in}}{\pgfqpoint{9.082951in}{1.155400in}}{\pgfqpoint{9.077907in}{1.155400in}}%
\pgfpathcurveto{\pgfqpoint{9.072863in}{1.155400in}}{\pgfqpoint{9.068025in}{1.153396in}}{\pgfqpoint{9.064459in}{1.149829in}}%
\pgfpathcurveto{\pgfqpoint{9.060893in}{1.146263in}}{\pgfqpoint{9.058889in}{1.141425in}}{\pgfqpoint{9.058889in}{1.136382in}}%
\pgfpathcurveto{\pgfqpoint{9.058889in}{1.131338in}}{\pgfqpoint{9.060893in}{1.126500in}}{\pgfqpoint{9.064459in}{1.122934in}}%
\pgfpathcurveto{\pgfqpoint{9.068025in}{1.119367in}}{\pgfqpoint{9.072863in}{1.117363in}}{\pgfqpoint{9.077907in}{1.117363in}}%
\pgfpathclose%
\pgfusepath{fill}%
\end{pgfscope}%
\begin{pgfscope}%
\pgfpathrectangle{\pgfqpoint{6.572727in}{0.474100in}}{\pgfqpoint{4.227273in}{3.318700in}}%
\pgfusepath{clip}%
\pgfsetbuttcap%
\pgfsetroundjoin%
\definecolor{currentfill}{rgb}{0.993248,0.906157,0.143936}%
\pgfsetfillcolor{currentfill}%
\pgfsetfillopacity{0.700000}%
\pgfsetlinewidth{0.000000pt}%
\definecolor{currentstroke}{rgb}{0.000000,0.000000,0.000000}%
\pgfsetstrokecolor{currentstroke}%
\pgfsetstrokeopacity{0.700000}%
\pgfsetdash{}{0pt}%
\pgfpathmoveto{\pgfqpoint{8.162401in}{2.712600in}}%
\pgfpathcurveto{\pgfqpoint{8.167445in}{2.712600in}}{\pgfqpoint{8.172283in}{2.714604in}}{\pgfqpoint{8.175849in}{2.718171in}}%
\pgfpathcurveto{\pgfqpoint{8.179415in}{2.721737in}}{\pgfqpoint{8.181419in}{2.726575in}}{\pgfqpoint{8.181419in}{2.731618in}}%
\pgfpathcurveto{\pgfqpoint{8.181419in}{2.736662in}}{\pgfqpoint{8.179415in}{2.741500in}}{\pgfqpoint{8.175849in}{2.745066in}}%
\pgfpathcurveto{\pgfqpoint{8.172283in}{2.748633in}}{\pgfqpoint{8.167445in}{2.750637in}}{\pgfqpoint{8.162401in}{2.750637in}}%
\pgfpathcurveto{\pgfqpoint{8.157357in}{2.750637in}}{\pgfqpoint{8.152520in}{2.748633in}}{\pgfqpoint{8.148953in}{2.745066in}}%
\pgfpathcurveto{\pgfqpoint{8.145387in}{2.741500in}}{\pgfqpoint{8.143383in}{2.736662in}}{\pgfqpoint{8.143383in}{2.731618in}}%
\pgfpathcurveto{\pgfqpoint{8.143383in}{2.726575in}}{\pgfqpoint{8.145387in}{2.721737in}}{\pgfqpoint{8.148953in}{2.718171in}}%
\pgfpathcurveto{\pgfqpoint{8.152520in}{2.714604in}}{\pgfqpoint{8.157357in}{2.712600in}}{\pgfqpoint{8.162401in}{2.712600in}}%
\pgfpathclose%
\pgfusepath{fill}%
\end{pgfscope}%
\begin{pgfscope}%
\pgfpathrectangle{\pgfqpoint{6.572727in}{0.474100in}}{\pgfqpoint{4.227273in}{3.318700in}}%
\pgfusepath{clip}%
\pgfsetbuttcap%
\pgfsetroundjoin%
\definecolor{currentfill}{rgb}{0.267004,0.004874,0.329415}%
\pgfsetfillcolor{currentfill}%
\pgfsetfillopacity{0.700000}%
\pgfsetlinewidth{0.000000pt}%
\definecolor{currentstroke}{rgb}{0.000000,0.000000,0.000000}%
\pgfsetstrokecolor{currentstroke}%
\pgfsetstrokeopacity{0.700000}%
\pgfsetdash{}{0pt}%
\pgfpathmoveto{\pgfqpoint{7.568948in}{1.406068in}}%
\pgfpathcurveto{\pgfqpoint{7.573992in}{1.406068in}}{\pgfqpoint{7.578830in}{1.408072in}}{\pgfqpoint{7.582396in}{1.411638in}}%
\pgfpathcurveto{\pgfqpoint{7.585962in}{1.415205in}}{\pgfqpoint{7.587966in}{1.420043in}}{\pgfqpoint{7.587966in}{1.425086in}}%
\pgfpathcurveto{\pgfqpoint{7.587966in}{1.430130in}}{\pgfqpoint{7.585962in}{1.434968in}}{\pgfqpoint{7.582396in}{1.438534in}}%
\pgfpathcurveto{\pgfqpoint{7.578830in}{1.442100in}}{\pgfqpoint{7.573992in}{1.444104in}}{\pgfqpoint{7.568948in}{1.444104in}}%
\pgfpathcurveto{\pgfqpoint{7.563904in}{1.444104in}}{\pgfqpoint{7.559067in}{1.442100in}}{\pgfqpoint{7.555500in}{1.438534in}}%
\pgfpathcurveto{\pgfqpoint{7.551934in}{1.434968in}}{\pgfqpoint{7.549930in}{1.430130in}}{\pgfqpoint{7.549930in}{1.425086in}}%
\pgfpathcurveto{\pgfqpoint{7.549930in}{1.420043in}}{\pgfqpoint{7.551934in}{1.415205in}}{\pgfqpoint{7.555500in}{1.411638in}}%
\pgfpathcurveto{\pgfqpoint{7.559067in}{1.408072in}}{\pgfqpoint{7.563904in}{1.406068in}}{\pgfqpoint{7.568948in}{1.406068in}}%
\pgfpathclose%
\pgfusepath{fill}%
\end{pgfscope}%
\begin{pgfscope}%
\pgfpathrectangle{\pgfqpoint{6.572727in}{0.474100in}}{\pgfqpoint{4.227273in}{3.318700in}}%
\pgfusepath{clip}%
\pgfsetbuttcap%
\pgfsetroundjoin%
\definecolor{currentfill}{rgb}{0.127568,0.566949,0.550556}%
\pgfsetfillcolor{currentfill}%
\pgfsetfillopacity{0.700000}%
\pgfsetlinewidth{0.000000pt}%
\definecolor{currentstroke}{rgb}{0.000000,0.000000,0.000000}%
\pgfsetstrokecolor{currentstroke}%
\pgfsetstrokeopacity{0.700000}%
\pgfsetdash{}{0pt}%
\pgfpathmoveto{\pgfqpoint{9.632150in}{1.761138in}}%
\pgfpathcurveto{\pgfqpoint{9.637194in}{1.761138in}}{\pgfqpoint{9.642032in}{1.763142in}}{\pgfqpoint{9.645598in}{1.766709in}}%
\pgfpathcurveto{\pgfqpoint{9.649165in}{1.770275in}}{\pgfqpoint{9.651169in}{1.775113in}}{\pgfqpoint{9.651169in}{1.780157in}}%
\pgfpathcurveto{\pgfqpoint{9.651169in}{1.785200in}}{\pgfqpoint{9.649165in}{1.790038in}}{\pgfqpoint{9.645598in}{1.793604in}}%
\pgfpathcurveto{\pgfqpoint{9.642032in}{1.797171in}}{\pgfqpoint{9.637194in}{1.799175in}}{\pgfqpoint{9.632150in}{1.799175in}}%
\pgfpathcurveto{\pgfqpoint{9.627107in}{1.799175in}}{\pgfqpoint{9.622269in}{1.797171in}}{\pgfqpoint{9.618703in}{1.793604in}}%
\pgfpathcurveto{\pgfqpoint{9.615136in}{1.790038in}}{\pgfqpoint{9.613132in}{1.785200in}}{\pgfqpoint{9.613132in}{1.780157in}}%
\pgfpathcurveto{\pgfqpoint{9.613132in}{1.775113in}}{\pgfqpoint{9.615136in}{1.770275in}}{\pgfqpoint{9.618703in}{1.766709in}}%
\pgfpathcurveto{\pgfqpoint{9.622269in}{1.763142in}}{\pgfqpoint{9.627107in}{1.761138in}}{\pgfqpoint{9.632150in}{1.761138in}}%
\pgfpathclose%
\pgfusepath{fill}%
\end{pgfscope}%
\begin{pgfscope}%
\pgfpathrectangle{\pgfqpoint{6.572727in}{0.474100in}}{\pgfqpoint{4.227273in}{3.318700in}}%
\pgfusepath{clip}%
\pgfsetbuttcap%
\pgfsetroundjoin%
\definecolor{currentfill}{rgb}{0.993248,0.906157,0.143936}%
\pgfsetfillcolor{currentfill}%
\pgfsetfillopacity{0.700000}%
\pgfsetlinewidth{0.000000pt}%
\definecolor{currentstroke}{rgb}{0.000000,0.000000,0.000000}%
\pgfsetstrokecolor{currentstroke}%
\pgfsetstrokeopacity{0.700000}%
\pgfsetdash{}{0pt}%
\pgfpathmoveto{\pgfqpoint{7.644865in}{2.288230in}}%
\pgfpathcurveto{\pgfqpoint{7.649909in}{2.288230in}}{\pgfqpoint{7.654747in}{2.290233in}}{\pgfqpoint{7.658313in}{2.293800in}}%
\pgfpathcurveto{\pgfqpoint{7.661880in}{2.297366in}}{\pgfqpoint{7.663884in}{2.302204in}}{\pgfqpoint{7.663884in}{2.307248in}}%
\pgfpathcurveto{\pgfqpoint{7.663884in}{2.312291in}}{\pgfqpoint{7.661880in}{2.317129in}}{\pgfqpoint{7.658313in}{2.320696in}}%
\pgfpathcurveto{\pgfqpoint{7.654747in}{2.324262in}}{\pgfqpoint{7.649909in}{2.326266in}}{\pgfqpoint{7.644865in}{2.326266in}}%
\pgfpathcurveto{\pgfqpoint{7.639822in}{2.326266in}}{\pgfqpoint{7.634984in}{2.324262in}}{\pgfqpoint{7.631418in}{2.320696in}}%
\pgfpathcurveto{\pgfqpoint{7.627851in}{2.317129in}}{\pgfqpoint{7.625847in}{2.312291in}}{\pgfqpoint{7.625847in}{2.307248in}}%
\pgfpathcurveto{\pgfqpoint{7.625847in}{2.302204in}}{\pgfqpoint{7.627851in}{2.297366in}}{\pgfqpoint{7.631418in}{2.293800in}}%
\pgfpathcurveto{\pgfqpoint{7.634984in}{2.290233in}}{\pgfqpoint{7.639822in}{2.288230in}}{\pgfqpoint{7.644865in}{2.288230in}}%
\pgfpathclose%
\pgfusepath{fill}%
\end{pgfscope}%
\begin{pgfscope}%
\pgfpathrectangle{\pgfqpoint{6.572727in}{0.474100in}}{\pgfqpoint{4.227273in}{3.318700in}}%
\pgfusepath{clip}%
\pgfsetbuttcap%
\pgfsetroundjoin%
\definecolor{currentfill}{rgb}{0.127568,0.566949,0.550556}%
\pgfsetfillcolor{currentfill}%
\pgfsetfillopacity{0.700000}%
\pgfsetlinewidth{0.000000pt}%
\definecolor{currentstroke}{rgb}{0.000000,0.000000,0.000000}%
\pgfsetstrokecolor{currentstroke}%
\pgfsetstrokeopacity{0.700000}%
\pgfsetdash{}{0pt}%
\pgfpathmoveto{\pgfqpoint{9.867545in}{0.706532in}}%
\pgfpathcurveto{\pgfqpoint{9.872589in}{0.706532in}}{\pgfqpoint{9.877426in}{0.708536in}}{\pgfqpoint{9.880993in}{0.712102in}}%
\pgfpathcurveto{\pgfqpoint{9.884559in}{0.715669in}}{\pgfqpoint{9.886563in}{0.720506in}}{\pgfqpoint{9.886563in}{0.725550in}}%
\pgfpathcurveto{\pgfqpoint{9.886563in}{0.730594in}}{\pgfqpoint{9.884559in}{0.735432in}}{\pgfqpoint{9.880993in}{0.738998in}}%
\pgfpathcurveto{\pgfqpoint{9.877426in}{0.742564in}}{\pgfqpoint{9.872589in}{0.744568in}}{\pgfqpoint{9.867545in}{0.744568in}}%
\pgfpathcurveto{\pgfqpoint{9.862501in}{0.744568in}}{\pgfqpoint{9.857663in}{0.742564in}}{\pgfqpoint{9.854097in}{0.738998in}}%
\pgfpathcurveto{\pgfqpoint{9.850531in}{0.735432in}}{\pgfqpoint{9.848527in}{0.730594in}}{\pgfqpoint{9.848527in}{0.725550in}}%
\pgfpathcurveto{\pgfqpoint{9.848527in}{0.720506in}}{\pgfqpoint{9.850531in}{0.715669in}}{\pgfqpoint{9.854097in}{0.712102in}}%
\pgfpathcurveto{\pgfqpoint{9.857663in}{0.708536in}}{\pgfqpoint{9.862501in}{0.706532in}}{\pgfqpoint{9.867545in}{0.706532in}}%
\pgfpathclose%
\pgfusepath{fill}%
\end{pgfscope}%
\begin{pgfscope}%
\pgfpathrectangle{\pgfqpoint{6.572727in}{0.474100in}}{\pgfqpoint{4.227273in}{3.318700in}}%
\pgfusepath{clip}%
\pgfsetbuttcap%
\pgfsetroundjoin%
\definecolor{currentfill}{rgb}{0.993248,0.906157,0.143936}%
\pgfsetfillcolor{currentfill}%
\pgfsetfillopacity{0.700000}%
\pgfsetlinewidth{0.000000pt}%
\definecolor{currentstroke}{rgb}{0.000000,0.000000,0.000000}%
\pgfsetstrokecolor{currentstroke}%
\pgfsetstrokeopacity{0.700000}%
\pgfsetdash{}{0pt}%
\pgfpathmoveto{\pgfqpoint{8.457399in}{2.931206in}}%
\pgfpathcurveto{\pgfqpoint{8.462443in}{2.931206in}}{\pgfqpoint{8.467280in}{2.933210in}}{\pgfqpoint{8.470847in}{2.936776in}}%
\pgfpathcurveto{\pgfqpoint{8.474413in}{2.940342in}}{\pgfqpoint{8.476417in}{2.945180in}}{\pgfqpoint{8.476417in}{2.950224in}}%
\pgfpathcurveto{\pgfqpoint{8.476417in}{2.955268in}}{\pgfqpoint{8.474413in}{2.960105in}}{\pgfqpoint{8.470847in}{2.963672in}}%
\pgfpathcurveto{\pgfqpoint{8.467280in}{2.967238in}}{\pgfqpoint{8.462443in}{2.969242in}}{\pgfqpoint{8.457399in}{2.969242in}}%
\pgfpathcurveto{\pgfqpoint{8.452355in}{2.969242in}}{\pgfqpoint{8.447517in}{2.967238in}}{\pgfqpoint{8.443951in}{2.963672in}}%
\pgfpathcurveto{\pgfqpoint{8.440385in}{2.960105in}}{\pgfqpoint{8.438381in}{2.955268in}}{\pgfqpoint{8.438381in}{2.950224in}}%
\pgfpathcurveto{\pgfqpoint{8.438381in}{2.945180in}}{\pgfqpoint{8.440385in}{2.940342in}}{\pgfqpoint{8.443951in}{2.936776in}}%
\pgfpathcurveto{\pgfqpoint{8.447517in}{2.933210in}}{\pgfqpoint{8.452355in}{2.931206in}}{\pgfqpoint{8.457399in}{2.931206in}}%
\pgfpathclose%
\pgfusepath{fill}%
\end{pgfscope}%
\begin{pgfscope}%
\pgfpathrectangle{\pgfqpoint{6.572727in}{0.474100in}}{\pgfqpoint{4.227273in}{3.318700in}}%
\pgfusepath{clip}%
\pgfsetbuttcap%
\pgfsetroundjoin%
\definecolor{currentfill}{rgb}{0.127568,0.566949,0.550556}%
\pgfsetfillcolor{currentfill}%
\pgfsetfillopacity{0.700000}%
\pgfsetlinewidth{0.000000pt}%
\definecolor{currentstroke}{rgb}{0.000000,0.000000,0.000000}%
\pgfsetstrokecolor{currentstroke}%
\pgfsetstrokeopacity{0.700000}%
\pgfsetdash{}{0pt}%
\pgfpathmoveto{\pgfqpoint{9.993451in}{1.733375in}}%
\pgfpathcurveto{\pgfqpoint{9.998495in}{1.733375in}}{\pgfqpoint{10.003333in}{1.735379in}}{\pgfqpoint{10.006899in}{1.738945in}}%
\pgfpathcurveto{\pgfqpoint{10.010465in}{1.742511in}}{\pgfqpoint{10.012469in}{1.747349in}}{\pgfqpoint{10.012469in}{1.752393in}}%
\pgfpathcurveto{\pgfqpoint{10.012469in}{1.757436in}}{\pgfqpoint{10.010465in}{1.762274in}}{\pgfqpoint{10.006899in}{1.765841in}}%
\pgfpathcurveto{\pgfqpoint{10.003333in}{1.769407in}}{\pgfqpoint{9.998495in}{1.771411in}}{\pgfqpoint{9.993451in}{1.771411in}}%
\pgfpathcurveto{\pgfqpoint{9.988407in}{1.771411in}}{\pgfqpoint{9.983570in}{1.769407in}}{\pgfqpoint{9.980003in}{1.765841in}}%
\pgfpathcurveto{\pgfqpoint{9.976437in}{1.762274in}}{\pgfqpoint{9.974433in}{1.757436in}}{\pgfqpoint{9.974433in}{1.752393in}}%
\pgfpathcurveto{\pgfqpoint{9.974433in}{1.747349in}}{\pgfqpoint{9.976437in}{1.742511in}}{\pgfqpoint{9.980003in}{1.738945in}}%
\pgfpathcurveto{\pgfqpoint{9.983570in}{1.735379in}}{\pgfqpoint{9.988407in}{1.733375in}}{\pgfqpoint{9.993451in}{1.733375in}}%
\pgfpathclose%
\pgfusepath{fill}%
\end{pgfscope}%
\begin{pgfscope}%
\pgfpathrectangle{\pgfqpoint{6.572727in}{0.474100in}}{\pgfqpoint{4.227273in}{3.318700in}}%
\pgfusepath{clip}%
\pgfsetbuttcap%
\pgfsetroundjoin%
\definecolor{currentfill}{rgb}{0.267004,0.004874,0.329415}%
\pgfsetfillcolor{currentfill}%
\pgfsetfillopacity{0.700000}%
\pgfsetlinewidth{0.000000pt}%
\definecolor{currentstroke}{rgb}{0.000000,0.000000,0.000000}%
\pgfsetstrokecolor{currentstroke}%
\pgfsetstrokeopacity{0.700000}%
\pgfsetdash{}{0pt}%
\pgfpathmoveto{\pgfqpoint{8.398872in}{1.657721in}}%
\pgfpathcurveto{\pgfqpoint{8.403915in}{1.657721in}}{\pgfqpoint{8.408753in}{1.659725in}}{\pgfqpoint{8.412320in}{1.663291in}}%
\pgfpathcurveto{\pgfqpoint{8.415886in}{1.666857in}}{\pgfqpoint{8.417890in}{1.671695in}}{\pgfqpoint{8.417890in}{1.676739in}}%
\pgfpathcurveto{\pgfqpoint{8.417890in}{1.681782in}}{\pgfqpoint{8.415886in}{1.686620in}}{\pgfqpoint{8.412320in}{1.690187in}}%
\pgfpathcurveto{\pgfqpoint{8.408753in}{1.693753in}}{\pgfqpoint{8.403915in}{1.695757in}}{\pgfqpoint{8.398872in}{1.695757in}}%
\pgfpathcurveto{\pgfqpoint{8.393828in}{1.695757in}}{\pgfqpoint{8.388990in}{1.693753in}}{\pgfqpoint{8.385424in}{1.690187in}}%
\pgfpathcurveto{\pgfqpoint{8.381857in}{1.686620in}}{\pgfqpoint{8.379854in}{1.681782in}}{\pgfqpoint{8.379854in}{1.676739in}}%
\pgfpathcurveto{\pgfqpoint{8.379854in}{1.671695in}}{\pgfqpoint{8.381857in}{1.666857in}}{\pgfqpoint{8.385424in}{1.663291in}}%
\pgfpathcurveto{\pgfqpoint{8.388990in}{1.659725in}}{\pgfqpoint{8.393828in}{1.657721in}}{\pgfqpoint{8.398872in}{1.657721in}}%
\pgfpathclose%
\pgfusepath{fill}%
\end{pgfscope}%
\begin{pgfscope}%
\pgfpathrectangle{\pgfqpoint{6.572727in}{0.474100in}}{\pgfqpoint{4.227273in}{3.318700in}}%
\pgfusepath{clip}%
\pgfsetbuttcap%
\pgfsetroundjoin%
\definecolor{currentfill}{rgb}{0.127568,0.566949,0.550556}%
\pgfsetfillcolor{currentfill}%
\pgfsetfillopacity{0.700000}%
\pgfsetlinewidth{0.000000pt}%
\definecolor{currentstroke}{rgb}{0.000000,0.000000,0.000000}%
\pgfsetstrokecolor{currentstroke}%
\pgfsetstrokeopacity{0.700000}%
\pgfsetdash{}{0pt}%
\pgfpathmoveto{\pgfqpoint{10.287256in}{1.702873in}}%
\pgfpathcurveto{\pgfqpoint{10.292300in}{1.702873in}}{\pgfqpoint{10.297138in}{1.704877in}}{\pgfqpoint{10.300704in}{1.708443in}}%
\pgfpathcurveto{\pgfqpoint{10.304270in}{1.712010in}}{\pgfqpoint{10.306274in}{1.716848in}}{\pgfqpoint{10.306274in}{1.721891in}}%
\pgfpathcurveto{\pgfqpoint{10.306274in}{1.726935in}}{\pgfqpoint{10.304270in}{1.731773in}}{\pgfqpoint{10.300704in}{1.735339in}}%
\pgfpathcurveto{\pgfqpoint{10.297138in}{1.738906in}}{\pgfqpoint{10.292300in}{1.740909in}}{\pgfqpoint{10.287256in}{1.740909in}}%
\pgfpathcurveto{\pgfqpoint{10.282212in}{1.740909in}}{\pgfqpoint{10.277375in}{1.738906in}}{\pgfqpoint{10.273808in}{1.735339in}}%
\pgfpathcurveto{\pgfqpoint{10.270242in}{1.731773in}}{\pgfqpoint{10.268238in}{1.726935in}}{\pgfqpoint{10.268238in}{1.721891in}}%
\pgfpathcurveto{\pgfqpoint{10.268238in}{1.716848in}}{\pgfqpoint{10.270242in}{1.712010in}}{\pgfqpoint{10.273808in}{1.708443in}}%
\pgfpathcurveto{\pgfqpoint{10.277375in}{1.704877in}}{\pgfqpoint{10.282212in}{1.702873in}}{\pgfqpoint{10.287256in}{1.702873in}}%
\pgfpathclose%
\pgfusepath{fill}%
\end{pgfscope}%
\begin{pgfscope}%
\pgfpathrectangle{\pgfqpoint{6.572727in}{0.474100in}}{\pgfqpoint{4.227273in}{3.318700in}}%
\pgfusepath{clip}%
\pgfsetbuttcap%
\pgfsetroundjoin%
\definecolor{currentfill}{rgb}{0.993248,0.906157,0.143936}%
\pgfsetfillcolor{currentfill}%
\pgfsetfillopacity{0.700000}%
\pgfsetlinewidth{0.000000pt}%
\definecolor{currentstroke}{rgb}{0.000000,0.000000,0.000000}%
\pgfsetstrokecolor{currentstroke}%
\pgfsetstrokeopacity{0.700000}%
\pgfsetdash{}{0pt}%
\pgfpathmoveto{\pgfqpoint{7.906750in}{2.302888in}}%
\pgfpathcurveto{\pgfqpoint{7.911794in}{2.302888in}}{\pgfqpoint{7.916632in}{2.304892in}}{\pgfqpoint{7.920198in}{2.308458in}}%
\pgfpathcurveto{\pgfqpoint{7.923764in}{2.312024in}}{\pgfqpoint{7.925768in}{2.316862in}}{\pgfqpoint{7.925768in}{2.321906in}}%
\pgfpathcurveto{\pgfqpoint{7.925768in}{2.326949in}}{\pgfqpoint{7.923764in}{2.331787in}}{\pgfqpoint{7.920198in}{2.335354in}}%
\pgfpathcurveto{\pgfqpoint{7.916632in}{2.338920in}}{\pgfqpoint{7.911794in}{2.340924in}}{\pgfqpoint{7.906750in}{2.340924in}}%
\pgfpathcurveto{\pgfqpoint{7.901706in}{2.340924in}}{\pgfqpoint{7.896869in}{2.338920in}}{\pgfqpoint{7.893302in}{2.335354in}}%
\pgfpathcurveto{\pgfqpoint{7.889736in}{2.331787in}}{\pgfqpoint{7.887732in}{2.326949in}}{\pgfqpoint{7.887732in}{2.321906in}}%
\pgfpathcurveto{\pgfqpoint{7.887732in}{2.316862in}}{\pgfqpoint{7.889736in}{2.312024in}}{\pgfqpoint{7.893302in}{2.308458in}}%
\pgfpathcurveto{\pgfqpoint{7.896869in}{2.304892in}}{\pgfqpoint{7.901706in}{2.302888in}}{\pgfqpoint{7.906750in}{2.302888in}}%
\pgfpathclose%
\pgfusepath{fill}%
\end{pgfscope}%
\begin{pgfscope}%
\pgfpathrectangle{\pgfqpoint{6.572727in}{0.474100in}}{\pgfqpoint{4.227273in}{3.318700in}}%
\pgfusepath{clip}%
\pgfsetbuttcap%
\pgfsetroundjoin%
\definecolor{currentfill}{rgb}{0.993248,0.906157,0.143936}%
\pgfsetfillcolor{currentfill}%
\pgfsetfillopacity{0.700000}%
\pgfsetlinewidth{0.000000pt}%
\definecolor{currentstroke}{rgb}{0.000000,0.000000,0.000000}%
\pgfsetstrokecolor{currentstroke}%
\pgfsetstrokeopacity{0.700000}%
\pgfsetdash{}{0pt}%
\pgfpathmoveto{\pgfqpoint{7.907470in}{2.898625in}}%
\pgfpathcurveto{\pgfqpoint{7.912514in}{2.898625in}}{\pgfqpoint{7.917352in}{2.900629in}}{\pgfqpoint{7.920918in}{2.904196in}}%
\pgfpathcurveto{\pgfqpoint{7.924485in}{2.907762in}}{\pgfqpoint{7.926489in}{2.912600in}}{\pgfqpoint{7.926489in}{2.917644in}}%
\pgfpathcurveto{\pgfqpoint{7.926489in}{2.922687in}}{\pgfqpoint{7.924485in}{2.927525in}}{\pgfqpoint{7.920918in}{2.931091in}}%
\pgfpathcurveto{\pgfqpoint{7.917352in}{2.934658in}}{\pgfqpoint{7.912514in}{2.936662in}}{\pgfqpoint{7.907470in}{2.936662in}}%
\pgfpathcurveto{\pgfqpoint{7.902427in}{2.936662in}}{\pgfqpoint{7.897589in}{2.934658in}}{\pgfqpoint{7.894023in}{2.931091in}}%
\pgfpathcurveto{\pgfqpoint{7.890456in}{2.927525in}}{\pgfqpoint{7.888452in}{2.922687in}}{\pgfqpoint{7.888452in}{2.917644in}}%
\pgfpathcurveto{\pgfqpoint{7.888452in}{2.912600in}}{\pgfqpoint{7.890456in}{2.907762in}}{\pgfqpoint{7.894023in}{2.904196in}}%
\pgfpathcurveto{\pgfqpoint{7.897589in}{2.900629in}}{\pgfqpoint{7.902427in}{2.898625in}}{\pgfqpoint{7.907470in}{2.898625in}}%
\pgfpathclose%
\pgfusepath{fill}%
\end{pgfscope}%
\begin{pgfscope}%
\pgfpathrectangle{\pgfqpoint{6.572727in}{0.474100in}}{\pgfqpoint{4.227273in}{3.318700in}}%
\pgfusepath{clip}%
\pgfsetbuttcap%
\pgfsetroundjoin%
\definecolor{currentfill}{rgb}{0.267004,0.004874,0.329415}%
\pgfsetfillcolor{currentfill}%
\pgfsetfillopacity{0.700000}%
\pgfsetlinewidth{0.000000pt}%
\definecolor{currentstroke}{rgb}{0.000000,0.000000,0.000000}%
\pgfsetstrokecolor{currentstroke}%
\pgfsetstrokeopacity{0.700000}%
\pgfsetdash{}{0pt}%
\pgfpathmoveto{\pgfqpoint{7.630284in}{1.591568in}}%
\pgfpathcurveto{\pgfqpoint{7.635328in}{1.591568in}}{\pgfqpoint{7.640166in}{1.593572in}}{\pgfqpoint{7.643732in}{1.597138in}}%
\pgfpathcurveto{\pgfqpoint{7.647298in}{1.600704in}}{\pgfqpoint{7.649302in}{1.605542in}}{\pgfqpoint{7.649302in}{1.610586in}}%
\pgfpathcurveto{\pgfqpoint{7.649302in}{1.615629in}}{\pgfqpoint{7.647298in}{1.620467in}}{\pgfqpoint{7.643732in}{1.624034in}}%
\pgfpathcurveto{\pgfqpoint{7.640166in}{1.627600in}}{\pgfqpoint{7.635328in}{1.629604in}}{\pgfqpoint{7.630284in}{1.629604in}}%
\pgfpathcurveto{\pgfqpoint{7.625240in}{1.629604in}}{\pgfqpoint{7.620403in}{1.627600in}}{\pgfqpoint{7.616836in}{1.624034in}}%
\pgfpathcurveto{\pgfqpoint{7.613270in}{1.620467in}}{\pgfqpoint{7.611266in}{1.615629in}}{\pgfqpoint{7.611266in}{1.610586in}}%
\pgfpathcurveto{\pgfqpoint{7.611266in}{1.605542in}}{\pgfqpoint{7.613270in}{1.600704in}}{\pgfqpoint{7.616836in}{1.597138in}}%
\pgfpathcurveto{\pgfqpoint{7.620403in}{1.593572in}}{\pgfqpoint{7.625240in}{1.591568in}}{\pgfqpoint{7.630284in}{1.591568in}}%
\pgfpathclose%
\pgfusepath{fill}%
\end{pgfscope}%
\begin{pgfscope}%
\pgfpathrectangle{\pgfqpoint{6.572727in}{0.474100in}}{\pgfqpoint{4.227273in}{3.318700in}}%
\pgfusepath{clip}%
\pgfsetbuttcap%
\pgfsetroundjoin%
\definecolor{currentfill}{rgb}{0.993248,0.906157,0.143936}%
\pgfsetfillcolor{currentfill}%
\pgfsetfillopacity{0.700000}%
\pgfsetlinewidth{0.000000pt}%
\definecolor{currentstroke}{rgb}{0.000000,0.000000,0.000000}%
\pgfsetstrokecolor{currentstroke}%
\pgfsetstrokeopacity{0.700000}%
\pgfsetdash{}{0pt}%
\pgfpathmoveto{\pgfqpoint{8.717370in}{2.308608in}}%
\pgfpathcurveto{\pgfqpoint{8.722414in}{2.308608in}}{\pgfqpoint{8.727252in}{2.310611in}}{\pgfqpoint{8.730818in}{2.314178in}}%
\pgfpathcurveto{\pgfqpoint{8.734384in}{2.317744in}}{\pgfqpoint{8.736388in}{2.322582in}}{\pgfqpoint{8.736388in}{2.327626in}}%
\pgfpathcurveto{\pgfqpoint{8.736388in}{2.332669in}}{\pgfqpoint{8.734384in}{2.337507in}}{\pgfqpoint{8.730818in}{2.341074in}}%
\pgfpathcurveto{\pgfqpoint{8.727252in}{2.344640in}}{\pgfqpoint{8.722414in}{2.346644in}}{\pgfqpoint{8.717370in}{2.346644in}}%
\pgfpathcurveto{\pgfqpoint{8.712326in}{2.346644in}}{\pgfqpoint{8.707489in}{2.344640in}}{\pgfqpoint{8.703922in}{2.341074in}}%
\pgfpathcurveto{\pgfqpoint{8.700356in}{2.337507in}}{\pgfqpoint{8.698352in}{2.332669in}}{\pgfqpoint{8.698352in}{2.327626in}}%
\pgfpathcurveto{\pgfqpoint{8.698352in}{2.322582in}}{\pgfqpoint{8.700356in}{2.317744in}}{\pgfqpoint{8.703922in}{2.314178in}}%
\pgfpathcurveto{\pgfqpoint{8.707489in}{2.310611in}}{\pgfqpoint{8.712326in}{2.308608in}}{\pgfqpoint{8.717370in}{2.308608in}}%
\pgfpathclose%
\pgfusepath{fill}%
\end{pgfscope}%
\begin{pgfscope}%
\pgfpathrectangle{\pgfqpoint{6.572727in}{0.474100in}}{\pgfqpoint{4.227273in}{3.318700in}}%
\pgfusepath{clip}%
\pgfsetbuttcap%
\pgfsetroundjoin%
\definecolor{currentfill}{rgb}{0.993248,0.906157,0.143936}%
\pgfsetfillcolor{currentfill}%
\pgfsetfillopacity{0.700000}%
\pgfsetlinewidth{0.000000pt}%
\definecolor{currentstroke}{rgb}{0.000000,0.000000,0.000000}%
\pgfsetstrokecolor{currentstroke}%
\pgfsetstrokeopacity{0.700000}%
\pgfsetdash{}{0pt}%
\pgfpathmoveto{\pgfqpoint{8.369347in}{3.491679in}}%
\pgfpathcurveto{\pgfqpoint{8.374390in}{3.491679in}}{\pgfqpoint{8.379228in}{3.493683in}}{\pgfqpoint{8.382795in}{3.497249in}}%
\pgfpathcurveto{\pgfqpoint{8.386361in}{3.500816in}}{\pgfqpoint{8.388365in}{3.505653in}}{\pgfqpoint{8.388365in}{3.510697in}}%
\pgfpathcurveto{\pgfqpoint{8.388365in}{3.515741in}}{\pgfqpoint{8.386361in}{3.520578in}}{\pgfqpoint{8.382795in}{3.524145in}}%
\pgfpathcurveto{\pgfqpoint{8.379228in}{3.527711in}}{\pgfqpoint{8.374390in}{3.529715in}}{\pgfqpoint{8.369347in}{3.529715in}}%
\pgfpathcurveto{\pgfqpoint{8.364303in}{3.529715in}}{\pgfqpoint{8.359465in}{3.527711in}}{\pgfqpoint{8.355899in}{3.524145in}}%
\pgfpathcurveto{\pgfqpoint{8.352332in}{3.520578in}}{\pgfqpoint{8.350329in}{3.515741in}}{\pgfqpoint{8.350329in}{3.510697in}}%
\pgfpathcurveto{\pgfqpoint{8.350329in}{3.505653in}}{\pgfqpoint{8.352332in}{3.500816in}}{\pgfqpoint{8.355899in}{3.497249in}}%
\pgfpathcurveto{\pgfqpoint{8.359465in}{3.493683in}}{\pgfqpoint{8.364303in}{3.491679in}}{\pgfqpoint{8.369347in}{3.491679in}}%
\pgfpathclose%
\pgfusepath{fill}%
\end{pgfscope}%
\begin{pgfscope}%
\pgfpathrectangle{\pgfqpoint{6.572727in}{0.474100in}}{\pgfqpoint{4.227273in}{3.318700in}}%
\pgfusepath{clip}%
\pgfsetbuttcap%
\pgfsetroundjoin%
\definecolor{currentfill}{rgb}{0.993248,0.906157,0.143936}%
\pgfsetfillcolor{currentfill}%
\pgfsetfillopacity{0.700000}%
\pgfsetlinewidth{0.000000pt}%
\definecolor{currentstroke}{rgb}{0.000000,0.000000,0.000000}%
\pgfsetstrokecolor{currentstroke}%
\pgfsetstrokeopacity{0.700000}%
\pgfsetdash{}{0pt}%
\pgfpathmoveto{\pgfqpoint{7.475106in}{2.855443in}}%
\pgfpathcurveto{\pgfqpoint{7.480150in}{2.855443in}}{\pgfqpoint{7.484987in}{2.857447in}}{\pgfqpoint{7.488554in}{2.861013in}}%
\pgfpathcurveto{\pgfqpoint{7.492120in}{2.864580in}}{\pgfqpoint{7.494124in}{2.869417in}}{\pgfqpoint{7.494124in}{2.874461in}}%
\pgfpathcurveto{\pgfqpoint{7.494124in}{2.879505in}}{\pgfqpoint{7.492120in}{2.884343in}}{\pgfqpoint{7.488554in}{2.887909in}}%
\pgfpathcurveto{\pgfqpoint{7.484987in}{2.891475in}}{\pgfqpoint{7.480150in}{2.893479in}}{\pgfqpoint{7.475106in}{2.893479in}}%
\pgfpathcurveto{\pgfqpoint{7.470062in}{2.893479in}}{\pgfqpoint{7.465224in}{2.891475in}}{\pgfqpoint{7.461658in}{2.887909in}}%
\pgfpathcurveto{\pgfqpoint{7.458092in}{2.884343in}}{\pgfqpoint{7.456088in}{2.879505in}}{\pgfqpoint{7.456088in}{2.874461in}}%
\pgfpathcurveto{\pgfqpoint{7.456088in}{2.869417in}}{\pgfqpoint{7.458092in}{2.864580in}}{\pgfqpoint{7.461658in}{2.861013in}}%
\pgfpathcurveto{\pgfqpoint{7.465224in}{2.857447in}}{\pgfqpoint{7.470062in}{2.855443in}}{\pgfqpoint{7.475106in}{2.855443in}}%
\pgfpathclose%
\pgfusepath{fill}%
\end{pgfscope}%
\begin{pgfscope}%
\pgfpathrectangle{\pgfqpoint{6.572727in}{0.474100in}}{\pgfqpoint{4.227273in}{3.318700in}}%
\pgfusepath{clip}%
\pgfsetbuttcap%
\pgfsetroundjoin%
\definecolor{currentfill}{rgb}{0.267004,0.004874,0.329415}%
\pgfsetfillcolor{currentfill}%
\pgfsetfillopacity{0.700000}%
\pgfsetlinewidth{0.000000pt}%
\definecolor{currentstroke}{rgb}{0.000000,0.000000,0.000000}%
\pgfsetstrokecolor{currentstroke}%
\pgfsetstrokeopacity{0.700000}%
\pgfsetdash{}{0pt}%
\pgfpathmoveto{\pgfqpoint{8.010337in}{1.106604in}}%
\pgfpathcurveto{\pgfqpoint{8.015381in}{1.106604in}}{\pgfqpoint{8.020218in}{1.108608in}}{\pgfqpoint{8.023785in}{1.112175in}}%
\pgfpathcurveto{\pgfqpoint{8.027351in}{1.115741in}}{\pgfqpoint{8.029355in}{1.120579in}}{\pgfqpoint{8.029355in}{1.125623in}}%
\pgfpathcurveto{\pgfqpoint{8.029355in}{1.130666in}}{\pgfqpoint{8.027351in}{1.135504in}}{\pgfqpoint{8.023785in}{1.139070in}}%
\pgfpathcurveto{\pgfqpoint{8.020218in}{1.142637in}}{\pgfqpoint{8.015381in}{1.144641in}}{\pgfqpoint{8.010337in}{1.144641in}}%
\pgfpathcurveto{\pgfqpoint{8.005293in}{1.144641in}}{\pgfqpoint{8.000456in}{1.142637in}}{\pgfqpoint{7.996889in}{1.139070in}}%
\pgfpathcurveto{\pgfqpoint{7.993323in}{1.135504in}}{\pgfqpoint{7.991319in}{1.130666in}}{\pgfqpoint{7.991319in}{1.125623in}}%
\pgfpathcurveto{\pgfqpoint{7.991319in}{1.120579in}}{\pgfqpoint{7.993323in}{1.115741in}}{\pgfqpoint{7.996889in}{1.112175in}}%
\pgfpathcurveto{\pgfqpoint{8.000456in}{1.108608in}}{\pgfqpoint{8.005293in}{1.106604in}}{\pgfqpoint{8.010337in}{1.106604in}}%
\pgfpathclose%
\pgfusepath{fill}%
\end{pgfscope}%
\begin{pgfscope}%
\pgfpathrectangle{\pgfqpoint{6.572727in}{0.474100in}}{\pgfqpoint{4.227273in}{3.318700in}}%
\pgfusepath{clip}%
\pgfsetbuttcap%
\pgfsetroundjoin%
\definecolor{currentfill}{rgb}{0.127568,0.566949,0.550556}%
\pgfsetfillcolor{currentfill}%
\pgfsetfillopacity{0.700000}%
\pgfsetlinewidth{0.000000pt}%
\definecolor{currentstroke}{rgb}{0.000000,0.000000,0.000000}%
\pgfsetstrokecolor{currentstroke}%
\pgfsetstrokeopacity{0.700000}%
\pgfsetdash{}{0pt}%
\pgfpathmoveto{\pgfqpoint{9.609688in}{1.669131in}}%
\pgfpathcurveto{\pgfqpoint{9.614731in}{1.669131in}}{\pgfqpoint{9.619569in}{1.671135in}}{\pgfqpoint{9.623135in}{1.674701in}}%
\pgfpathcurveto{\pgfqpoint{9.626702in}{1.678268in}}{\pgfqpoint{9.628706in}{1.683105in}}{\pgfqpoint{9.628706in}{1.688149in}}%
\pgfpathcurveto{\pgfqpoint{9.628706in}{1.693193in}}{\pgfqpoint{9.626702in}{1.698030in}}{\pgfqpoint{9.623135in}{1.701597in}}%
\pgfpathcurveto{\pgfqpoint{9.619569in}{1.705163in}}{\pgfqpoint{9.614731in}{1.707167in}}{\pgfqpoint{9.609688in}{1.707167in}}%
\pgfpathcurveto{\pgfqpoint{9.604644in}{1.707167in}}{\pgfqpoint{9.599806in}{1.705163in}}{\pgfqpoint{9.596240in}{1.701597in}}%
\pgfpathcurveto{\pgfqpoint{9.592673in}{1.698030in}}{\pgfqpoint{9.590669in}{1.693193in}}{\pgfqpoint{9.590669in}{1.688149in}}%
\pgfpathcurveto{\pgfqpoint{9.590669in}{1.683105in}}{\pgfqpoint{9.592673in}{1.678268in}}{\pgfqpoint{9.596240in}{1.674701in}}%
\pgfpathcurveto{\pgfqpoint{9.599806in}{1.671135in}}{\pgfqpoint{9.604644in}{1.669131in}}{\pgfqpoint{9.609688in}{1.669131in}}%
\pgfpathclose%
\pgfusepath{fill}%
\end{pgfscope}%
\begin{pgfscope}%
\pgfpathrectangle{\pgfqpoint{6.572727in}{0.474100in}}{\pgfqpoint{4.227273in}{3.318700in}}%
\pgfusepath{clip}%
\pgfsetbuttcap%
\pgfsetroundjoin%
\definecolor{currentfill}{rgb}{0.127568,0.566949,0.550556}%
\pgfsetfillcolor{currentfill}%
\pgfsetfillopacity{0.700000}%
\pgfsetlinewidth{0.000000pt}%
\definecolor{currentstroke}{rgb}{0.000000,0.000000,0.000000}%
\pgfsetstrokecolor{currentstroke}%
\pgfsetstrokeopacity{0.700000}%
\pgfsetdash{}{0pt}%
\pgfpathmoveto{\pgfqpoint{9.953565in}{2.160806in}}%
\pgfpathcurveto{\pgfqpoint{9.958609in}{2.160806in}}{\pgfqpoint{9.963446in}{2.162810in}}{\pgfqpoint{9.967013in}{2.166376in}}%
\pgfpathcurveto{\pgfqpoint{9.970579in}{2.169943in}}{\pgfqpoint{9.972583in}{2.174781in}}{\pgfqpoint{9.972583in}{2.179824in}}%
\pgfpathcurveto{\pgfqpoint{9.972583in}{2.184868in}}{\pgfqpoint{9.970579in}{2.189706in}}{\pgfqpoint{9.967013in}{2.193272in}}%
\pgfpathcurveto{\pgfqpoint{9.963446in}{2.196838in}}{\pgfqpoint{9.958609in}{2.198842in}}{\pgfqpoint{9.953565in}{2.198842in}}%
\pgfpathcurveto{\pgfqpoint{9.948521in}{2.198842in}}{\pgfqpoint{9.943683in}{2.196838in}}{\pgfqpoint{9.940117in}{2.193272in}}%
\pgfpathcurveto{\pgfqpoint{9.936551in}{2.189706in}}{\pgfqpoint{9.934547in}{2.184868in}}{\pgfqpoint{9.934547in}{2.179824in}}%
\pgfpathcurveto{\pgfqpoint{9.934547in}{2.174781in}}{\pgfqpoint{9.936551in}{2.169943in}}{\pgfqpoint{9.940117in}{2.166376in}}%
\pgfpathcurveto{\pgfqpoint{9.943683in}{2.162810in}}{\pgfqpoint{9.948521in}{2.160806in}}{\pgfqpoint{9.953565in}{2.160806in}}%
\pgfpathclose%
\pgfusepath{fill}%
\end{pgfscope}%
\begin{pgfscope}%
\pgfpathrectangle{\pgfqpoint{6.572727in}{0.474100in}}{\pgfqpoint{4.227273in}{3.318700in}}%
\pgfusepath{clip}%
\pgfsetbuttcap%
\pgfsetroundjoin%
\definecolor{currentfill}{rgb}{0.267004,0.004874,0.329415}%
\pgfsetfillcolor{currentfill}%
\pgfsetfillopacity{0.700000}%
\pgfsetlinewidth{0.000000pt}%
\definecolor{currentstroke}{rgb}{0.000000,0.000000,0.000000}%
\pgfsetstrokecolor{currentstroke}%
\pgfsetstrokeopacity{0.700000}%
\pgfsetdash{}{0pt}%
\pgfpathmoveto{\pgfqpoint{7.889949in}{1.695751in}}%
\pgfpathcurveto{\pgfqpoint{7.894993in}{1.695751in}}{\pgfqpoint{7.899831in}{1.697755in}}{\pgfqpoint{7.903397in}{1.701322in}}%
\pgfpathcurveto{\pgfqpoint{7.906963in}{1.704888in}}{\pgfqpoint{7.908967in}{1.709726in}}{\pgfqpoint{7.908967in}{1.714769in}}%
\pgfpathcurveto{\pgfqpoint{7.908967in}{1.719813in}}{\pgfqpoint{7.906963in}{1.724651in}}{\pgfqpoint{7.903397in}{1.728217in}}%
\pgfpathcurveto{\pgfqpoint{7.899831in}{1.731784in}}{\pgfqpoint{7.894993in}{1.733788in}}{\pgfqpoint{7.889949in}{1.733788in}}%
\pgfpathcurveto{\pgfqpoint{7.884905in}{1.733788in}}{\pgfqpoint{7.880068in}{1.731784in}}{\pgfqpoint{7.876501in}{1.728217in}}%
\pgfpathcurveto{\pgfqpoint{7.872935in}{1.724651in}}{\pgfqpoint{7.870931in}{1.719813in}}{\pgfqpoint{7.870931in}{1.714769in}}%
\pgfpathcurveto{\pgfqpoint{7.870931in}{1.709726in}}{\pgfqpoint{7.872935in}{1.704888in}}{\pgfqpoint{7.876501in}{1.701322in}}%
\pgfpathcurveto{\pgfqpoint{7.880068in}{1.697755in}}{\pgfqpoint{7.884905in}{1.695751in}}{\pgfqpoint{7.889949in}{1.695751in}}%
\pgfpathclose%
\pgfusepath{fill}%
\end{pgfscope}%
\begin{pgfscope}%
\pgfpathrectangle{\pgfqpoint{6.572727in}{0.474100in}}{\pgfqpoint{4.227273in}{3.318700in}}%
\pgfusepath{clip}%
\pgfsetbuttcap%
\pgfsetroundjoin%
\definecolor{currentfill}{rgb}{0.267004,0.004874,0.329415}%
\pgfsetfillcolor{currentfill}%
\pgfsetfillopacity{0.700000}%
\pgfsetlinewidth{0.000000pt}%
\definecolor{currentstroke}{rgb}{0.000000,0.000000,0.000000}%
\pgfsetstrokecolor{currentstroke}%
\pgfsetstrokeopacity{0.700000}%
\pgfsetdash{}{0pt}%
\pgfpathmoveto{\pgfqpoint{7.507862in}{1.153650in}}%
\pgfpathcurveto{\pgfqpoint{7.512906in}{1.153650in}}{\pgfqpoint{7.517744in}{1.155654in}}{\pgfqpoint{7.521310in}{1.159220in}}%
\pgfpathcurveto{\pgfqpoint{7.524877in}{1.162787in}}{\pgfqpoint{7.526880in}{1.167625in}}{\pgfqpoint{7.526880in}{1.172668in}}%
\pgfpathcurveto{\pgfqpoint{7.526880in}{1.177712in}}{\pgfqpoint{7.524877in}{1.182550in}}{\pgfqpoint{7.521310in}{1.186116in}}%
\pgfpathcurveto{\pgfqpoint{7.517744in}{1.189683in}}{\pgfqpoint{7.512906in}{1.191686in}}{\pgfqpoint{7.507862in}{1.191686in}}%
\pgfpathcurveto{\pgfqpoint{7.502819in}{1.191686in}}{\pgfqpoint{7.497981in}{1.189683in}}{\pgfqpoint{7.494414in}{1.186116in}}%
\pgfpathcurveto{\pgfqpoint{7.490848in}{1.182550in}}{\pgfqpoint{7.488844in}{1.177712in}}{\pgfqpoint{7.488844in}{1.172668in}}%
\pgfpathcurveto{\pgfqpoint{7.488844in}{1.167625in}}{\pgfqpoint{7.490848in}{1.162787in}}{\pgfqpoint{7.494414in}{1.159220in}}%
\pgfpathcurveto{\pgfqpoint{7.497981in}{1.155654in}}{\pgfqpoint{7.502819in}{1.153650in}}{\pgfqpoint{7.507862in}{1.153650in}}%
\pgfpathclose%
\pgfusepath{fill}%
\end{pgfscope}%
\begin{pgfscope}%
\pgfpathrectangle{\pgfqpoint{6.572727in}{0.474100in}}{\pgfqpoint{4.227273in}{3.318700in}}%
\pgfusepath{clip}%
\pgfsetbuttcap%
\pgfsetroundjoin%
\definecolor{currentfill}{rgb}{0.993248,0.906157,0.143936}%
\pgfsetfillcolor{currentfill}%
\pgfsetfillopacity{0.700000}%
\pgfsetlinewidth{0.000000pt}%
\definecolor{currentstroke}{rgb}{0.000000,0.000000,0.000000}%
\pgfsetstrokecolor{currentstroke}%
\pgfsetstrokeopacity{0.700000}%
\pgfsetdash{}{0pt}%
\pgfpathmoveto{\pgfqpoint{7.851340in}{3.167548in}}%
\pgfpathcurveto{\pgfqpoint{7.856384in}{3.167548in}}{\pgfqpoint{7.861222in}{3.169552in}}{\pgfqpoint{7.864788in}{3.173118in}}%
\pgfpathcurveto{\pgfqpoint{7.868354in}{3.176685in}}{\pgfqpoint{7.870358in}{3.181523in}}{\pgfqpoint{7.870358in}{3.186566in}}%
\pgfpathcurveto{\pgfqpoint{7.870358in}{3.191610in}}{\pgfqpoint{7.868354in}{3.196448in}}{\pgfqpoint{7.864788in}{3.200014in}}%
\pgfpathcurveto{\pgfqpoint{7.861222in}{3.203580in}}{\pgfqpoint{7.856384in}{3.205584in}}{\pgfqpoint{7.851340in}{3.205584in}}%
\pgfpathcurveto{\pgfqpoint{7.846296in}{3.205584in}}{\pgfqpoint{7.841459in}{3.203580in}}{\pgfqpoint{7.837892in}{3.200014in}}%
\pgfpathcurveto{\pgfqpoint{7.834326in}{3.196448in}}{\pgfqpoint{7.832322in}{3.191610in}}{\pgfqpoint{7.832322in}{3.186566in}}%
\pgfpathcurveto{\pgfqpoint{7.832322in}{3.181523in}}{\pgfqpoint{7.834326in}{3.176685in}}{\pgfqpoint{7.837892in}{3.173118in}}%
\pgfpathcurveto{\pgfqpoint{7.841459in}{3.169552in}}{\pgfqpoint{7.846296in}{3.167548in}}{\pgfqpoint{7.851340in}{3.167548in}}%
\pgfpathclose%
\pgfusepath{fill}%
\end{pgfscope}%
\begin{pgfscope}%
\pgfpathrectangle{\pgfqpoint{6.572727in}{0.474100in}}{\pgfqpoint{4.227273in}{3.318700in}}%
\pgfusepath{clip}%
\pgfsetbuttcap%
\pgfsetroundjoin%
\definecolor{currentfill}{rgb}{0.267004,0.004874,0.329415}%
\pgfsetfillcolor{currentfill}%
\pgfsetfillopacity{0.700000}%
\pgfsetlinewidth{0.000000pt}%
\definecolor{currentstroke}{rgb}{0.000000,0.000000,0.000000}%
\pgfsetstrokecolor{currentstroke}%
\pgfsetstrokeopacity{0.700000}%
\pgfsetdash{}{0pt}%
\pgfpathmoveto{\pgfqpoint{7.773228in}{1.043123in}}%
\pgfpathcurveto{\pgfqpoint{7.778272in}{1.043123in}}{\pgfqpoint{7.783110in}{1.045127in}}{\pgfqpoint{7.786676in}{1.048694in}}%
\pgfpathcurveto{\pgfqpoint{7.790243in}{1.052260in}}{\pgfqpoint{7.792246in}{1.057098in}}{\pgfqpoint{7.792246in}{1.062142in}}%
\pgfpathcurveto{\pgfqpoint{7.792246in}{1.067185in}}{\pgfqpoint{7.790243in}{1.072023in}}{\pgfqpoint{7.786676in}{1.075590in}}%
\pgfpathcurveto{\pgfqpoint{7.783110in}{1.079156in}}{\pgfqpoint{7.778272in}{1.081160in}}{\pgfqpoint{7.773228in}{1.081160in}}%
\pgfpathcurveto{\pgfqpoint{7.768185in}{1.081160in}}{\pgfqpoint{7.763347in}{1.079156in}}{\pgfqpoint{7.759780in}{1.075590in}}%
\pgfpathcurveto{\pgfqpoint{7.756214in}{1.072023in}}{\pgfqpoint{7.754210in}{1.067185in}}{\pgfqpoint{7.754210in}{1.062142in}}%
\pgfpathcurveto{\pgfqpoint{7.754210in}{1.057098in}}{\pgfqpoint{7.756214in}{1.052260in}}{\pgfqpoint{7.759780in}{1.048694in}}%
\pgfpathcurveto{\pgfqpoint{7.763347in}{1.045127in}}{\pgfqpoint{7.768185in}{1.043123in}}{\pgfqpoint{7.773228in}{1.043123in}}%
\pgfpathclose%
\pgfusepath{fill}%
\end{pgfscope}%
\begin{pgfscope}%
\pgfpathrectangle{\pgfqpoint{6.572727in}{0.474100in}}{\pgfqpoint{4.227273in}{3.318700in}}%
\pgfusepath{clip}%
\pgfsetbuttcap%
\pgfsetroundjoin%
\definecolor{currentfill}{rgb}{0.267004,0.004874,0.329415}%
\pgfsetfillcolor{currentfill}%
\pgfsetfillopacity{0.700000}%
\pgfsetlinewidth{0.000000pt}%
\definecolor{currentstroke}{rgb}{0.000000,0.000000,0.000000}%
\pgfsetstrokecolor{currentstroke}%
\pgfsetstrokeopacity{0.700000}%
\pgfsetdash{}{0pt}%
\pgfpathmoveto{\pgfqpoint{7.470716in}{1.039507in}}%
\pgfpathcurveto{\pgfqpoint{7.475760in}{1.039507in}}{\pgfqpoint{7.480597in}{1.041511in}}{\pgfqpoint{7.484164in}{1.045078in}}%
\pgfpathcurveto{\pgfqpoint{7.487730in}{1.048644in}}{\pgfqpoint{7.489734in}{1.053482in}}{\pgfqpoint{7.489734in}{1.058525in}}%
\pgfpathcurveto{\pgfqpoint{7.489734in}{1.063569in}}{\pgfqpoint{7.487730in}{1.068407in}}{\pgfqpoint{7.484164in}{1.071973in}}%
\pgfpathcurveto{\pgfqpoint{7.480597in}{1.075540in}}{\pgfqpoint{7.475760in}{1.077544in}}{\pgfqpoint{7.470716in}{1.077544in}}%
\pgfpathcurveto{\pgfqpoint{7.465672in}{1.077544in}}{\pgfqpoint{7.460835in}{1.075540in}}{\pgfqpoint{7.457268in}{1.071973in}}%
\pgfpathcurveto{\pgfqpoint{7.453702in}{1.068407in}}{\pgfqpoint{7.451698in}{1.063569in}}{\pgfqpoint{7.451698in}{1.058525in}}%
\pgfpathcurveto{\pgfqpoint{7.451698in}{1.053482in}}{\pgfqpoint{7.453702in}{1.048644in}}{\pgfqpoint{7.457268in}{1.045078in}}%
\pgfpathcurveto{\pgfqpoint{7.460835in}{1.041511in}}{\pgfqpoint{7.465672in}{1.039507in}}{\pgfqpoint{7.470716in}{1.039507in}}%
\pgfpathclose%
\pgfusepath{fill}%
\end{pgfscope}%
\begin{pgfscope}%
\pgfpathrectangle{\pgfqpoint{6.572727in}{0.474100in}}{\pgfqpoint{4.227273in}{3.318700in}}%
\pgfusepath{clip}%
\pgfsetbuttcap%
\pgfsetroundjoin%
\definecolor{currentfill}{rgb}{0.267004,0.004874,0.329415}%
\pgfsetfillcolor{currentfill}%
\pgfsetfillopacity{0.700000}%
\pgfsetlinewidth{0.000000pt}%
\definecolor{currentstroke}{rgb}{0.000000,0.000000,0.000000}%
\pgfsetstrokecolor{currentstroke}%
\pgfsetstrokeopacity{0.700000}%
\pgfsetdash{}{0pt}%
\pgfpathmoveto{\pgfqpoint{7.406007in}{1.409997in}}%
\pgfpathcurveto{\pgfqpoint{7.411050in}{1.409997in}}{\pgfqpoint{7.415888in}{1.412001in}}{\pgfqpoint{7.419455in}{1.415567in}}%
\pgfpathcurveto{\pgfqpoint{7.423021in}{1.419134in}}{\pgfqpoint{7.425025in}{1.423972in}}{\pgfqpoint{7.425025in}{1.429015in}}%
\pgfpathcurveto{\pgfqpoint{7.425025in}{1.434059in}}{\pgfqpoint{7.423021in}{1.438897in}}{\pgfqpoint{7.419455in}{1.442463in}}%
\pgfpathcurveto{\pgfqpoint{7.415888in}{1.446029in}}{\pgfqpoint{7.411050in}{1.448033in}}{\pgfqpoint{7.406007in}{1.448033in}}%
\pgfpathcurveto{\pgfqpoint{7.400963in}{1.448033in}}{\pgfqpoint{7.396125in}{1.446029in}}{\pgfqpoint{7.392559in}{1.442463in}}%
\pgfpathcurveto{\pgfqpoint{7.388993in}{1.438897in}}{\pgfqpoint{7.386989in}{1.434059in}}{\pgfqpoint{7.386989in}{1.429015in}}%
\pgfpathcurveto{\pgfqpoint{7.386989in}{1.423972in}}{\pgfqpoint{7.388993in}{1.419134in}}{\pgfqpoint{7.392559in}{1.415567in}}%
\pgfpathcurveto{\pgfqpoint{7.396125in}{1.412001in}}{\pgfqpoint{7.400963in}{1.409997in}}{\pgfqpoint{7.406007in}{1.409997in}}%
\pgfpathclose%
\pgfusepath{fill}%
\end{pgfscope}%
\begin{pgfscope}%
\pgfpathrectangle{\pgfqpoint{6.572727in}{0.474100in}}{\pgfqpoint{4.227273in}{3.318700in}}%
\pgfusepath{clip}%
\pgfsetbuttcap%
\pgfsetroundjoin%
\definecolor{currentfill}{rgb}{0.267004,0.004874,0.329415}%
\pgfsetfillcolor{currentfill}%
\pgfsetfillopacity{0.700000}%
\pgfsetlinewidth{0.000000pt}%
\definecolor{currentstroke}{rgb}{0.000000,0.000000,0.000000}%
\pgfsetstrokecolor{currentstroke}%
\pgfsetstrokeopacity{0.700000}%
\pgfsetdash{}{0pt}%
\pgfpathmoveto{\pgfqpoint{7.992728in}{1.125700in}}%
\pgfpathcurveto{\pgfqpoint{7.997772in}{1.125700in}}{\pgfqpoint{8.002610in}{1.127704in}}{\pgfqpoint{8.006176in}{1.131270in}}%
\pgfpathcurveto{\pgfqpoint{8.009743in}{1.134837in}}{\pgfqpoint{8.011746in}{1.139674in}}{\pgfqpoint{8.011746in}{1.144718in}}%
\pgfpathcurveto{\pgfqpoint{8.011746in}{1.149762in}}{\pgfqpoint{8.009743in}{1.154599in}}{\pgfqpoint{8.006176in}{1.158166in}}%
\pgfpathcurveto{\pgfqpoint{8.002610in}{1.161732in}}{\pgfqpoint{7.997772in}{1.163736in}}{\pgfqpoint{7.992728in}{1.163736in}}%
\pgfpathcurveto{\pgfqpoint{7.987685in}{1.163736in}}{\pgfqpoint{7.982847in}{1.161732in}}{\pgfqpoint{7.979280in}{1.158166in}}%
\pgfpathcurveto{\pgfqpoint{7.975714in}{1.154599in}}{\pgfqpoint{7.973710in}{1.149762in}}{\pgfqpoint{7.973710in}{1.144718in}}%
\pgfpathcurveto{\pgfqpoint{7.973710in}{1.139674in}}{\pgfqpoint{7.975714in}{1.134837in}}{\pgfqpoint{7.979280in}{1.131270in}}%
\pgfpathcurveto{\pgfqpoint{7.982847in}{1.127704in}}{\pgfqpoint{7.987685in}{1.125700in}}{\pgfqpoint{7.992728in}{1.125700in}}%
\pgfpathclose%
\pgfusepath{fill}%
\end{pgfscope}%
\begin{pgfscope}%
\pgfpathrectangle{\pgfqpoint{6.572727in}{0.474100in}}{\pgfqpoint{4.227273in}{3.318700in}}%
\pgfusepath{clip}%
\pgfsetbuttcap%
\pgfsetroundjoin%
\definecolor{currentfill}{rgb}{0.267004,0.004874,0.329415}%
\pgfsetfillcolor{currentfill}%
\pgfsetfillopacity{0.700000}%
\pgfsetlinewidth{0.000000pt}%
\definecolor{currentstroke}{rgb}{0.000000,0.000000,0.000000}%
\pgfsetstrokecolor{currentstroke}%
\pgfsetstrokeopacity{0.700000}%
\pgfsetdash{}{0pt}%
\pgfpathmoveto{\pgfqpoint{8.638296in}{1.595718in}}%
\pgfpathcurveto{\pgfqpoint{8.643340in}{1.595718in}}{\pgfqpoint{8.648178in}{1.597722in}}{\pgfqpoint{8.651744in}{1.601288in}}%
\pgfpathcurveto{\pgfqpoint{8.655311in}{1.604855in}}{\pgfqpoint{8.657315in}{1.609692in}}{\pgfqpoint{8.657315in}{1.614736in}}%
\pgfpathcurveto{\pgfqpoint{8.657315in}{1.619780in}}{\pgfqpoint{8.655311in}{1.624618in}}{\pgfqpoint{8.651744in}{1.628184in}}%
\pgfpathcurveto{\pgfqpoint{8.648178in}{1.631750in}}{\pgfqpoint{8.643340in}{1.633754in}}{\pgfqpoint{8.638296in}{1.633754in}}%
\pgfpathcurveto{\pgfqpoint{8.633253in}{1.633754in}}{\pgfqpoint{8.628415in}{1.631750in}}{\pgfqpoint{8.624848in}{1.628184in}}%
\pgfpathcurveto{\pgfqpoint{8.621282in}{1.624618in}}{\pgfqpoint{8.619278in}{1.619780in}}{\pgfqpoint{8.619278in}{1.614736in}}%
\pgfpathcurveto{\pgfqpoint{8.619278in}{1.609692in}}{\pgfqpoint{8.621282in}{1.604855in}}{\pgfqpoint{8.624848in}{1.601288in}}%
\pgfpathcurveto{\pgfqpoint{8.628415in}{1.597722in}}{\pgfqpoint{8.633253in}{1.595718in}}{\pgfqpoint{8.638296in}{1.595718in}}%
\pgfpathclose%
\pgfusepath{fill}%
\end{pgfscope}%
\begin{pgfscope}%
\pgfpathrectangle{\pgfqpoint{6.572727in}{0.474100in}}{\pgfqpoint{4.227273in}{3.318700in}}%
\pgfusepath{clip}%
\pgfsetbuttcap%
\pgfsetroundjoin%
\definecolor{currentfill}{rgb}{0.267004,0.004874,0.329415}%
\pgfsetfillcolor{currentfill}%
\pgfsetfillopacity{0.700000}%
\pgfsetlinewidth{0.000000pt}%
\definecolor{currentstroke}{rgb}{0.000000,0.000000,0.000000}%
\pgfsetstrokecolor{currentstroke}%
\pgfsetstrokeopacity{0.700000}%
\pgfsetdash{}{0pt}%
\pgfpathmoveto{\pgfqpoint{7.577322in}{1.453274in}}%
\pgfpathcurveto{\pgfqpoint{7.582366in}{1.453274in}}{\pgfqpoint{7.587204in}{1.455278in}}{\pgfqpoint{7.590770in}{1.458845in}}%
\pgfpathcurveto{\pgfqpoint{7.594337in}{1.462411in}}{\pgfqpoint{7.596341in}{1.467249in}}{\pgfqpoint{7.596341in}{1.472292in}}%
\pgfpathcurveto{\pgfqpoint{7.596341in}{1.477336in}}{\pgfqpoint{7.594337in}{1.482174in}}{\pgfqpoint{7.590770in}{1.485740in}}%
\pgfpathcurveto{\pgfqpoint{7.587204in}{1.489307in}}{\pgfqpoint{7.582366in}{1.491311in}}{\pgfqpoint{7.577322in}{1.491311in}}%
\pgfpathcurveto{\pgfqpoint{7.572279in}{1.491311in}}{\pgfqpoint{7.567441in}{1.489307in}}{\pgfqpoint{7.563875in}{1.485740in}}%
\pgfpathcurveto{\pgfqpoint{7.560308in}{1.482174in}}{\pgfqpoint{7.558304in}{1.477336in}}{\pgfqpoint{7.558304in}{1.472292in}}%
\pgfpathcurveto{\pgfqpoint{7.558304in}{1.467249in}}{\pgfqpoint{7.560308in}{1.462411in}}{\pgfqpoint{7.563875in}{1.458845in}}%
\pgfpathcurveto{\pgfqpoint{7.567441in}{1.455278in}}{\pgfqpoint{7.572279in}{1.453274in}}{\pgfqpoint{7.577322in}{1.453274in}}%
\pgfpathclose%
\pgfusepath{fill}%
\end{pgfscope}%
\begin{pgfscope}%
\pgfpathrectangle{\pgfqpoint{6.572727in}{0.474100in}}{\pgfqpoint{4.227273in}{3.318700in}}%
\pgfusepath{clip}%
\pgfsetbuttcap%
\pgfsetroundjoin%
\definecolor{currentfill}{rgb}{0.267004,0.004874,0.329415}%
\pgfsetfillcolor{currentfill}%
\pgfsetfillopacity{0.700000}%
\pgfsetlinewidth{0.000000pt}%
\definecolor{currentstroke}{rgb}{0.000000,0.000000,0.000000}%
\pgfsetstrokecolor{currentstroke}%
\pgfsetstrokeopacity{0.700000}%
\pgfsetdash{}{0pt}%
\pgfpathmoveto{\pgfqpoint{7.253898in}{1.025726in}}%
\pgfpathcurveto{\pgfqpoint{7.258942in}{1.025726in}}{\pgfqpoint{7.263780in}{1.027730in}}{\pgfqpoint{7.267346in}{1.031296in}}%
\pgfpathcurveto{\pgfqpoint{7.270912in}{1.034862in}}{\pgfqpoint{7.272916in}{1.039700in}}{\pgfqpoint{7.272916in}{1.044744in}}%
\pgfpathcurveto{\pgfqpoint{7.272916in}{1.049787in}}{\pgfqpoint{7.270912in}{1.054625in}}{\pgfqpoint{7.267346in}{1.058192in}}%
\pgfpathcurveto{\pgfqpoint{7.263780in}{1.061758in}}{\pgfqpoint{7.258942in}{1.063762in}}{\pgfqpoint{7.253898in}{1.063762in}}%
\pgfpathcurveto{\pgfqpoint{7.248855in}{1.063762in}}{\pgfqpoint{7.244017in}{1.061758in}}{\pgfqpoint{7.240450in}{1.058192in}}%
\pgfpathcurveto{\pgfqpoint{7.236884in}{1.054625in}}{\pgfqpoint{7.234880in}{1.049787in}}{\pgfqpoint{7.234880in}{1.044744in}}%
\pgfpathcurveto{\pgfqpoint{7.234880in}{1.039700in}}{\pgfqpoint{7.236884in}{1.034862in}}{\pgfqpoint{7.240450in}{1.031296in}}%
\pgfpathcurveto{\pgfqpoint{7.244017in}{1.027730in}}{\pgfqpoint{7.248855in}{1.025726in}}{\pgfqpoint{7.253898in}{1.025726in}}%
\pgfpathclose%
\pgfusepath{fill}%
\end{pgfscope}%
\begin{pgfscope}%
\pgfpathrectangle{\pgfqpoint{6.572727in}{0.474100in}}{\pgfqpoint{4.227273in}{3.318700in}}%
\pgfusepath{clip}%
\pgfsetbuttcap%
\pgfsetroundjoin%
\definecolor{currentfill}{rgb}{0.127568,0.566949,0.550556}%
\pgfsetfillcolor{currentfill}%
\pgfsetfillopacity{0.700000}%
\pgfsetlinewidth{0.000000pt}%
\definecolor{currentstroke}{rgb}{0.000000,0.000000,0.000000}%
\pgfsetstrokecolor{currentstroke}%
\pgfsetstrokeopacity{0.700000}%
\pgfsetdash{}{0pt}%
\pgfpathmoveto{\pgfqpoint{10.270034in}{1.809327in}}%
\pgfpathcurveto{\pgfqpoint{10.275078in}{1.809327in}}{\pgfqpoint{10.279916in}{1.811331in}}{\pgfqpoint{10.283482in}{1.814897in}}%
\pgfpathcurveto{\pgfqpoint{10.287049in}{1.818464in}}{\pgfqpoint{10.289053in}{1.823302in}}{\pgfqpoint{10.289053in}{1.828345in}}%
\pgfpathcurveto{\pgfqpoint{10.289053in}{1.833389in}}{\pgfqpoint{10.287049in}{1.838227in}}{\pgfqpoint{10.283482in}{1.841793in}}%
\pgfpathcurveto{\pgfqpoint{10.279916in}{1.845360in}}{\pgfqpoint{10.275078in}{1.847363in}}{\pgfqpoint{10.270034in}{1.847363in}}%
\pgfpathcurveto{\pgfqpoint{10.264991in}{1.847363in}}{\pgfqpoint{10.260153in}{1.845360in}}{\pgfqpoint{10.256587in}{1.841793in}}%
\pgfpathcurveto{\pgfqpoint{10.253020in}{1.838227in}}{\pgfqpoint{10.251016in}{1.833389in}}{\pgfqpoint{10.251016in}{1.828345in}}%
\pgfpathcurveto{\pgfqpoint{10.251016in}{1.823302in}}{\pgfqpoint{10.253020in}{1.818464in}}{\pgfqpoint{10.256587in}{1.814897in}}%
\pgfpathcurveto{\pgfqpoint{10.260153in}{1.811331in}}{\pgfqpoint{10.264991in}{1.809327in}}{\pgfqpoint{10.270034in}{1.809327in}}%
\pgfpathclose%
\pgfusepath{fill}%
\end{pgfscope}%
\begin{pgfscope}%
\pgfpathrectangle{\pgfqpoint{6.572727in}{0.474100in}}{\pgfqpoint{4.227273in}{3.318700in}}%
\pgfusepath{clip}%
\pgfsetbuttcap%
\pgfsetroundjoin%
\definecolor{currentfill}{rgb}{0.267004,0.004874,0.329415}%
\pgfsetfillcolor{currentfill}%
\pgfsetfillopacity{0.700000}%
\pgfsetlinewidth{0.000000pt}%
\definecolor{currentstroke}{rgb}{0.000000,0.000000,0.000000}%
\pgfsetstrokecolor{currentstroke}%
\pgfsetstrokeopacity{0.700000}%
\pgfsetdash{}{0pt}%
\pgfpathmoveto{\pgfqpoint{7.981775in}{1.663051in}}%
\pgfpathcurveto{\pgfqpoint{7.986819in}{1.663051in}}{\pgfqpoint{7.991656in}{1.665055in}}{\pgfqpoint{7.995223in}{1.668621in}}%
\pgfpathcurveto{\pgfqpoint{7.998789in}{1.672188in}}{\pgfqpoint{8.000793in}{1.677026in}}{\pgfqpoint{8.000793in}{1.682069in}}%
\pgfpathcurveto{\pgfqpoint{8.000793in}{1.687113in}}{\pgfqpoint{7.998789in}{1.691951in}}{\pgfqpoint{7.995223in}{1.695517in}}%
\pgfpathcurveto{\pgfqpoint{7.991656in}{1.699083in}}{\pgfqpoint{7.986819in}{1.701087in}}{\pgfqpoint{7.981775in}{1.701087in}}%
\pgfpathcurveto{\pgfqpoint{7.976731in}{1.701087in}}{\pgfqpoint{7.971894in}{1.699083in}}{\pgfqpoint{7.968327in}{1.695517in}}%
\pgfpathcurveto{\pgfqpoint{7.964761in}{1.691951in}}{\pgfqpoint{7.962757in}{1.687113in}}{\pgfqpoint{7.962757in}{1.682069in}}%
\pgfpathcurveto{\pgfqpoint{7.962757in}{1.677026in}}{\pgfqpoint{7.964761in}{1.672188in}}{\pgfqpoint{7.968327in}{1.668621in}}%
\pgfpathcurveto{\pgfqpoint{7.971894in}{1.665055in}}{\pgfqpoint{7.976731in}{1.663051in}}{\pgfqpoint{7.981775in}{1.663051in}}%
\pgfpathclose%
\pgfusepath{fill}%
\end{pgfscope}%
\begin{pgfscope}%
\pgfpathrectangle{\pgfqpoint{6.572727in}{0.474100in}}{\pgfqpoint{4.227273in}{3.318700in}}%
\pgfusepath{clip}%
\pgfsetbuttcap%
\pgfsetroundjoin%
\definecolor{currentfill}{rgb}{0.127568,0.566949,0.550556}%
\pgfsetfillcolor{currentfill}%
\pgfsetfillopacity{0.700000}%
\pgfsetlinewidth{0.000000pt}%
\definecolor{currentstroke}{rgb}{0.000000,0.000000,0.000000}%
\pgfsetstrokecolor{currentstroke}%
\pgfsetstrokeopacity{0.700000}%
\pgfsetdash{}{0pt}%
\pgfpathmoveto{\pgfqpoint{9.923404in}{1.462924in}}%
\pgfpathcurveto{\pgfqpoint{9.928448in}{1.462924in}}{\pgfqpoint{9.933286in}{1.464928in}}{\pgfqpoint{9.936852in}{1.468494in}}%
\pgfpathcurveto{\pgfqpoint{9.940418in}{1.472061in}}{\pgfqpoint{9.942422in}{1.476899in}}{\pgfqpoint{9.942422in}{1.481942in}}%
\pgfpathcurveto{\pgfqpoint{9.942422in}{1.486986in}}{\pgfqpoint{9.940418in}{1.491824in}}{\pgfqpoint{9.936852in}{1.495390in}}%
\pgfpathcurveto{\pgfqpoint{9.933286in}{1.498957in}}{\pgfqpoint{9.928448in}{1.500960in}}{\pgfqpoint{9.923404in}{1.500960in}}%
\pgfpathcurveto{\pgfqpoint{9.918360in}{1.500960in}}{\pgfqpoint{9.913523in}{1.498957in}}{\pgfqpoint{9.909956in}{1.495390in}}%
\pgfpathcurveto{\pgfqpoint{9.906390in}{1.491824in}}{\pgfqpoint{9.904386in}{1.486986in}}{\pgfqpoint{9.904386in}{1.481942in}}%
\pgfpathcurveto{\pgfqpoint{9.904386in}{1.476899in}}{\pgfqpoint{9.906390in}{1.472061in}}{\pgfqpoint{9.909956in}{1.468494in}}%
\pgfpathcurveto{\pgfqpoint{9.913523in}{1.464928in}}{\pgfqpoint{9.918360in}{1.462924in}}{\pgfqpoint{9.923404in}{1.462924in}}%
\pgfpathclose%
\pgfusepath{fill}%
\end{pgfscope}%
\begin{pgfscope}%
\pgfpathrectangle{\pgfqpoint{6.572727in}{0.474100in}}{\pgfqpoint{4.227273in}{3.318700in}}%
\pgfusepath{clip}%
\pgfsetbuttcap%
\pgfsetroundjoin%
\definecolor{currentfill}{rgb}{0.267004,0.004874,0.329415}%
\pgfsetfillcolor{currentfill}%
\pgfsetfillopacity{0.700000}%
\pgfsetlinewidth{0.000000pt}%
\definecolor{currentstroke}{rgb}{0.000000,0.000000,0.000000}%
\pgfsetstrokecolor{currentstroke}%
\pgfsetstrokeopacity{0.700000}%
\pgfsetdash{}{0pt}%
\pgfpathmoveto{\pgfqpoint{7.723658in}{1.683179in}}%
\pgfpathcurveto{\pgfqpoint{7.728702in}{1.683179in}}{\pgfqpoint{7.733540in}{1.685183in}}{\pgfqpoint{7.737106in}{1.688750in}}%
\pgfpathcurveto{\pgfqpoint{7.740673in}{1.692316in}}{\pgfqpoint{7.742677in}{1.697154in}}{\pgfqpoint{7.742677in}{1.702198in}}%
\pgfpathcurveto{\pgfqpoint{7.742677in}{1.707241in}}{\pgfqpoint{7.740673in}{1.712079in}}{\pgfqpoint{7.737106in}{1.715645in}}%
\pgfpathcurveto{\pgfqpoint{7.733540in}{1.719212in}}{\pgfqpoint{7.728702in}{1.721216in}}{\pgfqpoint{7.723658in}{1.721216in}}%
\pgfpathcurveto{\pgfqpoint{7.718615in}{1.721216in}}{\pgfqpoint{7.713777in}{1.719212in}}{\pgfqpoint{7.710211in}{1.715645in}}%
\pgfpathcurveto{\pgfqpoint{7.706644in}{1.712079in}}{\pgfqpoint{7.704640in}{1.707241in}}{\pgfqpoint{7.704640in}{1.702198in}}%
\pgfpathcurveto{\pgfqpoint{7.704640in}{1.697154in}}{\pgfqpoint{7.706644in}{1.692316in}}{\pgfqpoint{7.710211in}{1.688750in}}%
\pgfpathcurveto{\pgfqpoint{7.713777in}{1.685183in}}{\pgfqpoint{7.718615in}{1.683179in}}{\pgfqpoint{7.723658in}{1.683179in}}%
\pgfpathclose%
\pgfusepath{fill}%
\end{pgfscope}%
\begin{pgfscope}%
\pgfpathrectangle{\pgfqpoint{6.572727in}{0.474100in}}{\pgfqpoint{4.227273in}{3.318700in}}%
\pgfusepath{clip}%
\pgfsetbuttcap%
\pgfsetroundjoin%
\definecolor{currentfill}{rgb}{0.127568,0.566949,0.550556}%
\pgfsetfillcolor{currentfill}%
\pgfsetfillopacity{0.700000}%
\pgfsetlinewidth{0.000000pt}%
\definecolor{currentstroke}{rgb}{0.000000,0.000000,0.000000}%
\pgfsetstrokecolor{currentstroke}%
\pgfsetstrokeopacity{0.700000}%
\pgfsetdash{}{0pt}%
\pgfpathmoveto{\pgfqpoint{10.607851in}{1.489116in}}%
\pgfpathcurveto{\pgfqpoint{10.612895in}{1.489116in}}{\pgfqpoint{10.617733in}{1.491120in}}{\pgfqpoint{10.621299in}{1.494686in}}%
\pgfpathcurveto{\pgfqpoint{10.624866in}{1.498252in}}{\pgfqpoint{10.626869in}{1.503090in}}{\pgfqpoint{10.626869in}{1.508134in}}%
\pgfpathcurveto{\pgfqpoint{10.626869in}{1.513177in}}{\pgfqpoint{10.624866in}{1.518015in}}{\pgfqpoint{10.621299in}{1.521582in}}%
\pgfpathcurveto{\pgfqpoint{10.617733in}{1.525148in}}{\pgfqpoint{10.612895in}{1.527152in}}{\pgfqpoint{10.607851in}{1.527152in}}%
\pgfpathcurveto{\pgfqpoint{10.602808in}{1.527152in}}{\pgfqpoint{10.597970in}{1.525148in}}{\pgfqpoint{10.594403in}{1.521582in}}%
\pgfpathcurveto{\pgfqpoint{10.590837in}{1.518015in}}{\pgfqpoint{10.588833in}{1.513177in}}{\pgfqpoint{10.588833in}{1.508134in}}%
\pgfpathcurveto{\pgfqpoint{10.588833in}{1.503090in}}{\pgfqpoint{10.590837in}{1.498252in}}{\pgfqpoint{10.594403in}{1.494686in}}%
\pgfpathcurveto{\pgfqpoint{10.597970in}{1.491120in}}{\pgfqpoint{10.602808in}{1.489116in}}{\pgfqpoint{10.607851in}{1.489116in}}%
\pgfpathclose%
\pgfusepath{fill}%
\end{pgfscope}%
\begin{pgfscope}%
\pgfpathrectangle{\pgfqpoint{6.572727in}{0.474100in}}{\pgfqpoint{4.227273in}{3.318700in}}%
\pgfusepath{clip}%
\pgfsetbuttcap%
\pgfsetroundjoin%
\definecolor{currentfill}{rgb}{0.127568,0.566949,0.550556}%
\pgfsetfillcolor{currentfill}%
\pgfsetfillopacity{0.700000}%
\pgfsetlinewidth{0.000000pt}%
\definecolor{currentstroke}{rgb}{0.000000,0.000000,0.000000}%
\pgfsetstrokecolor{currentstroke}%
\pgfsetstrokeopacity{0.700000}%
\pgfsetdash{}{0pt}%
\pgfpathmoveto{\pgfqpoint{10.345549in}{1.490650in}}%
\pgfpathcurveto{\pgfqpoint{10.350592in}{1.490650in}}{\pgfqpoint{10.355430in}{1.492654in}}{\pgfqpoint{10.358997in}{1.496220in}}%
\pgfpathcurveto{\pgfqpoint{10.362563in}{1.499786in}}{\pgfqpoint{10.364567in}{1.504624in}}{\pgfqpoint{10.364567in}{1.509668in}}%
\pgfpathcurveto{\pgfqpoint{10.364567in}{1.514711in}}{\pgfqpoint{10.362563in}{1.519549in}}{\pgfqpoint{10.358997in}{1.523116in}}%
\pgfpathcurveto{\pgfqpoint{10.355430in}{1.526682in}}{\pgfqpoint{10.350592in}{1.528686in}}{\pgfqpoint{10.345549in}{1.528686in}}%
\pgfpathcurveto{\pgfqpoint{10.340505in}{1.528686in}}{\pgfqpoint{10.335667in}{1.526682in}}{\pgfqpoint{10.332101in}{1.523116in}}%
\pgfpathcurveto{\pgfqpoint{10.328534in}{1.519549in}}{\pgfqpoint{10.326531in}{1.514711in}}{\pgfqpoint{10.326531in}{1.509668in}}%
\pgfpathcurveto{\pgfqpoint{10.326531in}{1.504624in}}{\pgfqpoint{10.328534in}{1.499786in}}{\pgfqpoint{10.332101in}{1.496220in}}%
\pgfpathcurveto{\pgfqpoint{10.335667in}{1.492654in}}{\pgfqpoint{10.340505in}{1.490650in}}{\pgfqpoint{10.345549in}{1.490650in}}%
\pgfpathclose%
\pgfusepath{fill}%
\end{pgfscope}%
\begin{pgfscope}%
\pgfpathrectangle{\pgfqpoint{6.572727in}{0.474100in}}{\pgfqpoint{4.227273in}{3.318700in}}%
\pgfusepath{clip}%
\pgfsetbuttcap%
\pgfsetroundjoin%
\definecolor{currentfill}{rgb}{0.267004,0.004874,0.329415}%
\pgfsetfillcolor{currentfill}%
\pgfsetfillopacity{0.700000}%
\pgfsetlinewidth{0.000000pt}%
\definecolor{currentstroke}{rgb}{0.000000,0.000000,0.000000}%
\pgfsetstrokecolor{currentstroke}%
\pgfsetstrokeopacity{0.700000}%
\pgfsetdash{}{0pt}%
\pgfpathmoveto{\pgfqpoint{7.657607in}{1.102617in}}%
\pgfpathcurveto{\pgfqpoint{7.662650in}{1.102617in}}{\pgfqpoint{7.667488in}{1.104620in}}{\pgfqpoint{7.671054in}{1.108187in}}%
\pgfpathcurveto{\pgfqpoint{7.674621in}{1.111753in}}{\pgfqpoint{7.676625in}{1.116591in}}{\pgfqpoint{7.676625in}{1.121635in}}%
\pgfpathcurveto{\pgfqpoint{7.676625in}{1.126678in}}{\pgfqpoint{7.674621in}{1.131516in}}{\pgfqpoint{7.671054in}{1.135083in}}%
\pgfpathcurveto{\pgfqpoint{7.667488in}{1.138649in}}{\pgfqpoint{7.662650in}{1.140653in}}{\pgfqpoint{7.657607in}{1.140653in}}%
\pgfpathcurveto{\pgfqpoint{7.652563in}{1.140653in}}{\pgfqpoint{7.647725in}{1.138649in}}{\pgfqpoint{7.644159in}{1.135083in}}%
\pgfpathcurveto{\pgfqpoint{7.640592in}{1.131516in}}{\pgfqpoint{7.638588in}{1.126678in}}{\pgfqpoint{7.638588in}{1.121635in}}%
\pgfpathcurveto{\pgfqpoint{7.638588in}{1.116591in}}{\pgfqpoint{7.640592in}{1.111753in}}{\pgfqpoint{7.644159in}{1.108187in}}%
\pgfpathcurveto{\pgfqpoint{7.647725in}{1.104620in}}{\pgfqpoint{7.652563in}{1.102617in}}{\pgfqpoint{7.657607in}{1.102617in}}%
\pgfpathclose%
\pgfusepath{fill}%
\end{pgfscope}%
\begin{pgfscope}%
\pgfpathrectangle{\pgfqpoint{6.572727in}{0.474100in}}{\pgfqpoint{4.227273in}{3.318700in}}%
\pgfusepath{clip}%
\pgfsetbuttcap%
\pgfsetroundjoin%
\definecolor{currentfill}{rgb}{0.993248,0.906157,0.143936}%
\pgfsetfillcolor{currentfill}%
\pgfsetfillopacity{0.700000}%
\pgfsetlinewidth{0.000000pt}%
\definecolor{currentstroke}{rgb}{0.000000,0.000000,0.000000}%
\pgfsetstrokecolor{currentstroke}%
\pgfsetstrokeopacity{0.700000}%
\pgfsetdash{}{0pt}%
\pgfpathmoveto{\pgfqpoint{7.594925in}{2.933160in}}%
\pgfpathcurveto{\pgfqpoint{7.599968in}{2.933160in}}{\pgfqpoint{7.604806in}{2.935164in}}{\pgfqpoint{7.608373in}{2.938730in}}%
\pgfpathcurveto{\pgfqpoint{7.611939in}{2.942297in}}{\pgfqpoint{7.613943in}{2.947134in}}{\pgfqpoint{7.613943in}{2.952178in}}%
\pgfpathcurveto{\pgfqpoint{7.613943in}{2.957222in}}{\pgfqpoint{7.611939in}{2.962059in}}{\pgfqpoint{7.608373in}{2.965626in}}%
\pgfpathcurveto{\pgfqpoint{7.604806in}{2.969192in}}{\pgfqpoint{7.599968in}{2.971196in}}{\pgfqpoint{7.594925in}{2.971196in}}%
\pgfpathcurveto{\pgfqpoint{7.589881in}{2.971196in}}{\pgfqpoint{7.585043in}{2.969192in}}{\pgfqpoint{7.581477in}{2.965626in}}%
\pgfpathcurveto{\pgfqpoint{7.577911in}{2.962059in}}{\pgfqpoint{7.575907in}{2.957222in}}{\pgfqpoint{7.575907in}{2.952178in}}%
\pgfpathcurveto{\pgfqpoint{7.575907in}{2.947134in}}{\pgfqpoint{7.577911in}{2.942297in}}{\pgfqpoint{7.581477in}{2.938730in}}%
\pgfpathcurveto{\pgfqpoint{7.585043in}{2.935164in}}{\pgfqpoint{7.589881in}{2.933160in}}{\pgfqpoint{7.594925in}{2.933160in}}%
\pgfpathclose%
\pgfusepath{fill}%
\end{pgfscope}%
\begin{pgfscope}%
\pgfpathrectangle{\pgfqpoint{6.572727in}{0.474100in}}{\pgfqpoint{4.227273in}{3.318700in}}%
\pgfusepath{clip}%
\pgfsetbuttcap%
\pgfsetroundjoin%
\definecolor{currentfill}{rgb}{0.267004,0.004874,0.329415}%
\pgfsetfillcolor{currentfill}%
\pgfsetfillopacity{0.700000}%
\pgfsetlinewidth{0.000000pt}%
\definecolor{currentstroke}{rgb}{0.000000,0.000000,0.000000}%
\pgfsetstrokecolor{currentstroke}%
\pgfsetstrokeopacity{0.700000}%
\pgfsetdash{}{0pt}%
\pgfpathmoveto{\pgfqpoint{8.084105in}{1.753381in}}%
\pgfpathcurveto{\pgfqpoint{8.089149in}{1.753381in}}{\pgfqpoint{8.093987in}{1.755385in}}{\pgfqpoint{8.097553in}{1.758952in}}%
\pgfpathcurveto{\pgfqpoint{8.101120in}{1.762518in}}{\pgfqpoint{8.103124in}{1.767356in}}{\pgfqpoint{8.103124in}{1.772399in}}%
\pgfpathcurveto{\pgfqpoint{8.103124in}{1.777443in}}{\pgfqpoint{8.101120in}{1.782281in}}{\pgfqpoint{8.097553in}{1.785847in}}%
\pgfpathcurveto{\pgfqpoint{8.093987in}{1.789414in}}{\pgfqpoint{8.089149in}{1.791418in}}{\pgfqpoint{8.084105in}{1.791418in}}%
\pgfpathcurveto{\pgfqpoint{8.079062in}{1.791418in}}{\pgfqpoint{8.074224in}{1.789414in}}{\pgfqpoint{8.070658in}{1.785847in}}%
\pgfpathcurveto{\pgfqpoint{8.067091in}{1.782281in}}{\pgfqpoint{8.065087in}{1.777443in}}{\pgfqpoint{8.065087in}{1.772399in}}%
\pgfpathcurveto{\pgfqpoint{8.065087in}{1.767356in}}{\pgfqpoint{8.067091in}{1.762518in}}{\pgfqpoint{8.070658in}{1.758952in}}%
\pgfpathcurveto{\pgfqpoint{8.074224in}{1.755385in}}{\pgfqpoint{8.079062in}{1.753381in}}{\pgfqpoint{8.084105in}{1.753381in}}%
\pgfpathclose%
\pgfusepath{fill}%
\end{pgfscope}%
\begin{pgfscope}%
\pgfpathrectangle{\pgfqpoint{6.572727in}{0.474100in}}{\pgfqpoint{4.227273in}{3.318700in}}%
\pgfusepath{clip}%
\pgfsetbuttcap%
\pgfsetroundjoin%
\definecolor{currentfill}{rgb}{0.127568,0.566949,0.550556}%
\pgfsetfillcolor{currentfill}%
\pgfsetfillopacity{0.700000}%
\pgfsetlinewidth{0.000000pt}%
\definecolor{currentstroke}{rgb}{0.000000,0.000000,0.000000}%
\pgfsetstrokecolor{currentstroke}%
\pgfsetstrokeopacity{0.700000}%
\pgfsetdash{}{0pt}%
\pgfpathmoveto{\pgfqpoint{9.382759in}{1.582782in}}%
\pgfpathcurveto{\pgfqpoint{9.387803in}{1.582782in}}{\pgfqpoint{9.392641in}{1.584786in}}{\pgfqpoint{9.396207in}{1.588352in}}%
\pgfpathcurveto{\pgfqpoint{9.399774in}{1.591919in}}{\pgfqpoint{9.401778in}{1.596756in}}{\pgfqpoint{9.401778in}{1.601800in}}%
\pgfpathcurveto{\pgfqpoint{9.401778in}{1.606844in}}{\pgfqpoint{9.399774in}{1.611681in}}{\pgfqpoint{9.396207in}{1.615248in}}%
\pgfpathcurveto{\pgfqpoint{9.392641in}{1.618814in}}{\pgfqpoint{9.387803in}{1.620818in}}{\pgfqpoint{9.382759in}{1.620818in}}%
\pgfpathcurveto{\pgfqpoint{9.377716in}{1.620818in}}{\pgfqpoint{9.372878in}{1.618814in}}{\pgfqpoint{9.369312in}{1.615248in}}%
\pgfpathcurveto{\pgfqpoint{9.365745in}{1.611681in}}{\pgfqpoint{9.363741in}{1.606844in}}{\pgfqpoint{9.363741in}{1.601800in}}%
\pgfpathcurveto{\pgfqpoint{9.363741in}{1.596756in}}{\pgfqpoint{9.365745in}{1.591919in}}{\pgfqpoint{9.369312in}{1.588352in}}%
\pgfpathcurveto{\pgfqpoint{9.372878in}{1.584786in}}{\pgfqpoint{9.377716in}{1.582782in}}{\pgfqpoint{9.382759in}{1.582782in}}%
\pgfpathclose%
\pgfusepath{fill}%
\end{pgfscope}%
\begin{pgfscope}%
\pgfpathrectangle{\pgfqpoint{6.572727in}{0.474100in}}{\pgfqpoint{4.227273in}{3.318700in}}%
\pgfusepath{clip}%
\pgfsetbuttcap%
\pgfsetroundjoin%
\definecolor{currentfill}{rgb}{0.993248,0.906157,0.143936}%
\pgfsetfillcolor{currentfill}%
\pgfsetfillopacity{0.700000}%
\pgfsetlinewidth{0.000000pt}%
\definecolor{currentstroke}{rgb}{0.000000,0.000000,0.000000}%
\pgfsetstrokecolor{currentstroke}%
\pgfsetstrokeopacity{0.700000}%
\pgfsetdash{}{0pt}%
\pgfpathmoveto{\pgfqpoint{8.072087in}{2.741277in}}%
\pgfpathcurveto{\pgfqpoint{8.077130in}{2.741277in}}{\pgfqpoint{8.081968in}{2.743281in}}{\pgfqpoint{8.085534in}{2.746848in}}%
\pgfpathcurveto{\pgfqpoint{8.089101in}{2.750414in}}{\pgfqpoint{8.091105in}{2.755252in}}{\pgfqpoint{8.091105in}{2.760296in}}%
\pgfpathcurveto{\pgfqpoint{8.091105in}{2.765339in}}{\pgfqpoint{8.089101in}{2.770177in}}{\pgfqpoint{8.085534in}{2.773743in}}%
\pgfpathcurveto{\pgfqpoint{8.081968in}{2.777310in}}{\pgfqpoint{8.077130in}{2.779314in}}{\pgfqpoint{8.072087in}{2.779314in}}%
\pgfpathcurveto{\pgfqpoint{8.067043in}{2.779314in}}{\pgfqpoint{8.062205in}{2.777310in}}{\pgfqpoint{8.058639in}{2.773743in}}%
\pgfpathcurveto{\pgfqpoint{8.055072in}{2.770177in}}{\pgfqpoint{8.053068in}{2.765339in}}{\pgfqpoint{8.053068in}{2.760296in}}%
\pgfpathcurveto{\pgfqpoint{8.053068in}{2.755252in}}{\pgfqpoint{8.055072in}{2.750414in}}{\pgfqpoint{8.058639in}{2.746848in}}%
\pgfpathcurveto{\pgfqpoint{8.062205in}{2.743281in}}{\pgfqpoint{8.067043in}{2.741277in}}{\pgfqpoint{8.072087in}{2.741277in}}%
\pgfpathclose%
\pgfusepath{fill}%
\end{pgfscope}%
\begin{pgfscope}%
\pgfpathrectangle{\pgfqpoint{6.572727in}{0.474100in}}{\pgfqpoint{4.227273in}{3.318700in}}%
\pgfusepath{clip}%
\pgfsetbuttcap%
\pgfsetroundjoin%
\definecolor{currentfill}{rgb}{0.127568,0.566949,0.550556}%
\pgfsetfillcolor{currentfill}%
\pgfsetfillopacity{0.700000}%
\pgfsetlinewidth{0.000000pt}%
\definecolor{currentstroke}{rgb}{0.000000,0.000000,0.000000}%
\pgfsetstrokecolor{currentstroke}%
\pgfsetstrokeopacity{0.700000}%
\pgfsetdash{}{0pt}%
\pgfpathmoveto{\pgfqpoint{9.928319in}{1.720087in}}%
\pgfpathcurveto{\pgfqpoint{9.933362in}{1.720087in}}{\pgfqpoint{9.938200in}{1.722091in}}{\pgfqpoint{9.941766in}{1.725657in}}%
\pgfpathcurveto{\pgfqpoint{9.945333in}{1.729224in}}{\pgfqpoint{9.947337in}{1.734062in}}{\pgfqpoint{9.947337in}{1.739105in}}%
\pgfpathcurveto{\pgfqpoint{9.947337in}{1.744149in}}{\pgfqpoint{9.945333in}{1.748987in}}{\pgfqpoint{9.941766in}{1.752553in}}%
\pgfpathcurveto{\pgfqpoint{9.938200in}{1.756120in}}{\pgfqpoint{9.933362in}{1.758123in}}{\pgfqpoint{9.928319in}{1.758123in}}%
\pgfpathcurveto{\pgfqpoint{9.923275in}{1.758123in}}{\pgfqpoint{9.918437in}{1.756120in}}{\pgfqpoint{9.914871in}{1.752553in}}%
\pgfpathcurveto{\pgfqpoint{9.911304in}{1.748987in}}{\pgfqpoint{9.909300in}{1.744149in}}{\pgfqpoint{9.909300in}{1.739105in}}%
\pgfpathcurveto{\pgfqpoint{9.909300in}{1.734062in}}{\pgfqpoint{9.911304in}{1.729224in}}{\pgfqpoint{9.914871in}{1.725657in}}%
\pgfpathcurveto{\pgfqpoint{9.918437in}{1.722091in}}{\pgfqpoint{9.923275in}{1.720087in}}{\pgfqpoint{9.928319in}{1.720087in}}%
\pgfpathclose%
\pgfusepath{fill}%
\end{pgfscope}%
\begin{pgfscope}%
\pgfpathrectangle{\pgfqpoint{6.572727in}{0.474100in}}{\pgfqpoint{4.227273in}{3.318700in}}%
\pgfusepath{clip}%
\pgfsetbuttcap%
\pgfsetroundjoin%
\definecolor{currentfill}{rgb}{0.993248,0.906157,0.143936}%
\pgfsetfillcolor{currentfill}%
\pgfsetfillopacity{0.700000}%
\pgfsetlinewidth{0.000000pt}%
\definecolor{currentstroke}{rgb}{0.000000,0.000000,0.000000}%
\pgfsetstrokecolor{currentstroke}%
\pgfsetstrokeopacity{0.700000}%
\pgfsetdash{}{0pt}%
\pgfpathmoveto{\pgfqpoint{7.964010in}{2.331815in}}%
\pgfpathcurveto{\pgfqpoint{7.969053in}{2.331815in}}{\pgfqpoint{7.973891in}{2.333819in}}{\pgfqpoint{7.977457in}{2.337385in}}%
\pgfpathcurveto{\pgfqpoint{7.981024in}{2.340952in}}{\pgfqpoint{7.983028in}{2.345789in}}{\pgfqpoint{7.983028in}{2.350833in}}%
\pgfpathcurveto{\pgfqpoint{7.983028in}{2.355877in}}{\pgfqpoint{7.981024in}{2.360715in}}{\pgfqpoint{7.977457in}{2.364281in}}%
\pgfpathcurveto{\pgfqpoint{7.973891in}{2.367847in}}{\pgfqpoint{7.969053in}{2.369851in}}{\pgfqpoint{7.964010in}{2.369851in}}%
\pgfpathcurveto{\pgfqpoint{7.958966in}{2.369851in}}{\pgfqpoint{7.954128in}{2.367847in}}{\pgfqpoint{7.950562in}{2.364281in}}%
\pgfpathcurveto{\pgfqpoint{7.946995in}{2.360715in}}{\pgfqpoint{7.944991in}{2.355877in}}{\pgfqpoint{7.944991in}{2.350833in}}%
\pgfpathcurveto{\pgfqpoint{7.944991in}{2.345789in}}{\pgfqpoint{7.946995in}{2.340952in}}{\pgfqpoint{7.950562in}{2.337385in}}%
\pgfpathcurveto{\pgfqpoint{7.954128in}{2.333819in}}{\pgfqpoint{7.958966in}{2.331815in}}{\pgfqpoint{7.964010in}{2.331815in}}%
\pgfpathclose%
\pgfusepath{fill}%
\end{pgfscope}%
\begin{pgfscope}%
\pgfpathrectangle{\pgfqpoint{6.572727in}{0.474100in}}{\pgfqpoint{4.227273in}{3.318700in}}%
\pgfusepath{clip}%
\pgfsetbuttcap%
\pgfsetroundjoin%
\definecolor{currentfill}{rgb}{0.127568,0.566949,0.550556}%
\pgfsetfillcolor{currentfill}%
\pgfsetfillopacity{0.700000}%
\pgfsetlinewidth{0.000000pt}%
\definecolor{currentstroke}{rgb}{0.000000,0.000000,0.000000}%
\pgfsetstrokecolor{currentstroke}%
\pgfsetstrokeopacity{0.700000}%
\pgfsetdash{}{0pt}%
\pgfpathmoveto{\pgfqpoint{9.534888in}{1.473695in}}%
\pgfpathcurveto{\pgfqpoint{9.539931in}{1.473695in}}{\pgfqpoint{9.544769in}{1.475699in}}{\pgfqpoint{9.548336in}{1.479266in}}%
\pgfpathcurveto{\pgfqpoint{9.551902in}{1.482832in}}{\pgfqpoint{9.553906in}{1.487670in}}{\pgfqpoint{9.553906in}{1.492713in}}%
\pgfpathcurveto{\pgfqpoint{9.553906in}{1.497757in}}{\pgfqpoint{9.551902in}{1.502595in}}{\pgfqpoint{9.548336in}{1.506161in}}%
\pgfpathcurveto{\pgfqpoint{9.544769in}{1.509728in}}{\pgfqpoint{9.539931in}{1.511732in}}{\pgfqpoint{9.534888in}{1.511732in}}%
\pgfpathcurveto{\pgfqpoint{9.529844in}{1.511732in}}{\pgfqpoint{9.525006in}{1.509728in}}{\pgfqpoint{9.521440in}{1.506161in}}%
\pgfpathcurveto{\pgfqpoint{9.517873in}{1.502595in}}{\pgfqpoint{9.515870in}{1.497757in}}{\pgfqpoint{9.515870in}{1.492713in}}%
\pgfpathcurveto{\pgfqpoint{9.515870in}{1.487670in}}{\pgfqpoint{9.517873in}{1.482832in}}{\pgfqpoint{9.521440in}{1.479266in}}%
\pgfpathcurveto{\pgfqpoint{9.525006in}{1.475699in}}{\pgfqpoint{9.529844in}{1.473695in}}{\pgfqpoint{9.534888in}{1.473695in}}%
\pgfpathclose%
\pgfusepath{fill}%
\end{pgfscope}%
\begin{pgfscope}%
\pgfpathrectangle{\pgfqpoint{6.572727in}{0.474100in}}{\pgfqpoint{4.227273in}{3.318700in}}%
\pgfusepath{clip}%
\pgfsetbuttcap%
\pgfsetroundjoin%
\definecolor{currentfill}{rgb}{0.993248,0.906157,0.143936}%
\pgfsetfillcolor{currentfill}%
\pgfsetfillopacity{0.700000}%
\pgfsetlinewidth{0.000000pt}%
\definecolor{currentstroke}{rgb}{0.000000,0.000000,0.000000}%
\pgfsetstrokecolor{currentstroke}%
\pgfsetstrokeopacity{0.700000}%
\pgfsetdash{}{0pt}%
\pgfpathmoveto{\pgfqpoint{7.980377in}{2.541331in}}%
\pgfpathcurveto{\pgfqpoint{7.985420in}{2.541331in}}{\pgfqpoint{7.990258in}{2.543335in}}{\pgfqpoint{7.993824in}{2.546901in}}%
\pgfpathcurveto{\pgfqpoint{7.997391in}{2.550468in}}{\pgfqpoint{7.999395in}{2.555305in}}{\pgfqpoint{7.999395in}{2.560349in}}%
\pgfpathcurveto{\pgfqpoint{7.999395in}{2.565393in}}{\pgfqpoint{7.997391in}{2.570230in}}{\pgfqpoint{7.993824in}{2.573797in}}%
\pgfpathcurveto{\pgfqpoint{7.990258in}{2.577363in}}{\pgfqpoint{7.985420in}{2.579367in}}{\pgfqpoint{7.980377in}{2.579367in}}%
\pgfpathcurveto{\pgfqpoint{7.975333in}{2.579367in}}{\pgfqpoint{7.970495in}{2.577363in}}{\pgfqpoint{7.966929in}{2.573797in}}%
\pgfpathcurveto{\pgfqpoint{7.963362in}{2.570230in}}{\pgfqpoint{7.961358in}{2.565393in}}{\pgfqpoint{7.961358in}{2.560349in}}%
\pgfpathcurveto{\pgfqpoint{7.961358in}{2.555305in}}{\pgfqpoint{7.963362in}{2.550468in}}{\pgfqpoint{7.966929in}{2.546901in}}%
\pgfpathcurveto{\pgfqpoint{7.970495in}{2.543335in}}{\pgfqpoint{7.975333in}{2.541331in}}{\pgfqpoint{7.980377in}{2.541331in}}%
\pgfpathclose%
\pgfusepath{fill}%
\end{pgfscope}%
\begin{pgfscope}%
\pgfpathrectangle{\pgfqpoint{6.572727in}{0.474100in}}{\pgfqpoint{4.227273in}{3.318700in}}%
\pgfusepath{clip}%
\pgfsetbuttcap%
\pgfsetroundjoin%
\definecolor{currentfill}{rgb}{0.127568,0.566949,0.550556}%
\pgfsetfillcolor{currentfill}%
\pgfsetfillopacity{0.700000}%
\pgfsetlinewidth{0.000000pt}%
\definecolor{currentstroke}{rgb}{0.000000,0.000000,0.000000}%
\pgfsetstrokecolor{currentstroke}%
\pgfsetstrokeopacity{0.700000}%
\pgfsetdash{}{0pt}%
\pgfpathmoveto{\pgfqpoint{9.356800in}{1.181004in}}%
\pgfpathcurveto{\pgfqpoint{9.361844in}{1.181004in}}{\pgfqpoint{9.366681in}{1.183008in}}{\pgfqpoint{9.370248in}{1.186575in}}%
\pgfpathcurveto{\pgfqpoint{9.373814in}{1.190141in}}{\pgfqpoint{9.375818in}{1.194979in}}{\pgfqpoint{9.375818in}{1.200023in}}%
\pgfpathcurveto{\pgfqpoint{9.375818in}{1.205066in}}{\pgfqpoint{9.373814in}{1.209904in}}{\pgfqpoint{9.370248in}{1.213470in}}%
\pgfpathcurveto{\pgfqpoint{9.366681in}{1.217037in}}{\pgfqpoint{9.361844in}{1.219041in}}{\pgfqpoint{9.356800in}{1.219041in}}%
\pgfpathcurveto{\pgfqpoint{9.351756in}{1.219041in}}{\pgfqpoint{9.346918in}{1.217037in}}{\pgfqpoint{9.343352in}{1.213470in}}%
\pgfpathcurveto{\pgfqpoint{9.339786in}{1.209904in}}{\pgfqpoint{9.337782in}{1.205066in}}{\pgfqpoint{9.337782in}{1.200023in}}%
\pgfpathcurveto{\pgfqpoint{9.337782in}{1.194979in}}{\pgfqpoint{9.339786in}{1.190141in}}{\pgfqpoint{9.343352in}{1.186575in}}%
\pgfpathcurveto{\pgfqpoint{9.346918in}{1.183008in}}{\pgfqpoint{9.351756in}{1.181004in}}{\pgfqpoint{9.356800in}{1.181004in}}%
\pgfpathclose%
\pgfusepath{fill}%
\end{pgfscope}%
\begin{pgfscope}%
\pgfpathrectangle{\pgfqpoint{6.572727in}{0.474100in}}{\pgfqpoint{4.227273in}{3.318700in}}%
\pgfusepath{clip}%
\pgfsetbuttcap%
\pgfsetroundjoin%
\definecolor{currentfill}{rgb}{0.267004,0.004874,0.329415}%
\pgfsetfillcolor{currentfill}%
\pgfsetfillopacity{0.700000}%
\pgfsetlinewidth{0.000000pt}%
\definecolor{currentstroke}{rgb}{0.000000,0.000000,0.000000}%
\pgfsetstrokecolor{currentstroke}%
\pgfsetstrokeopacity{0.700000}%
\pgfsetdash{}{0pt}%
\pgfpathmoveto{\pgfqpoint{7.915122in}{1.505223in}}%
\pgfpathcurveto{\pgfqpoint{7.920165in}{1.505223in}}{\pgfqpoint{7.925003in}{1.507227in}}{\pgfqpoint{7.928569in}{1.510793in}}%
\pgfpathcurveto{\pgfqpoint{7.932136in}{1.514360in}}{\pgfqpoint{7.934140in}{1.519198in}}{\pgfqpoint{7.934140in}{1.524241in}}%
\pgfpathcurveto{\pgfqpoint{7.934140in}{1.529285in}}{\pgfqpoint{7.932136in}{1.534123in}}{\pgfqpoint{7.928569in}{1.537689in}}%
\pgfpathcurveto{\pgfqpoint{7.925003in}{1.541255in}}{\pgfqpoint{7.920165in}{1.543259in}}{\pgfqpoint{7.915122in}{1.543259in}}%
\pgfpathcurveto{\pgfqpoint{7.910078in}{1.543259in}}{\pgfqpoint{7.905240in}{1.541255in}}{\pgfqpoint{7.901674in}{1.537689in}}%
\pgfpathcurveto{\pgfqpoint{7.898107in}{1.534123in}}{\pgfqpoint{7.896103in}{1.529285in}}{\pgfqpoint{7.896103in}{1.524241in}}%
\pgfpathcurveto{\pgfqpoint{7.896103in}{1.519198in}}{\pgfqpoint{7.898107in}{1.514360in}}{\pgfqpoint{7.901674in}{1.510793in}}%
\pgfpathcurveto{\pgfqpoint{7.905240in}{1.507227in}}{\pgfqpoint{7.910078in}{1.505223in}}{\pgfqpoint{7.915122in}{1.505223in}}%
\pgfpathclose%
\pgfusepath{fill}%
\end{pgfscope}%
\begin{pgfscope}%
\pgfpathrectangle{\pgfqpoint{6.572727in}{0.474100in}}{\pgfqpoint{4.227273in}{3.318700in}}%
\pgfusepath{clip}%
\pgfsetbuttcap%
\pgfsetroundjoin%
\definecolor{currentfill}{rgb}{0.127568,0.566949,0.550556}%
\pgfsetfillcolor{currentfill}%
\pgfsetfillopacity{0.700000}%
\pgfsetlinewidth{0.000000pt}%
\definecolor{currentstroke}{rgb}{0.000000,0.000000,0.000000}%
\pgfsetstrokecolor{currentstroke}%
\pgfsetstrokeopacity{0.700000}%
\pgfsetdash{}{0pt}%
\pgfpathmoveto{\pgfqpoint{9.906081in}{1.637996in}}%
\pgfpathcurveto{\pgfqpoint{9.911125in}{1.637996in}}{\pgfqpoint{9.915963in}{1.640000in}}{\pgfqpoint{9.919529in}{1.643566in}}%
\pgfpathcurveto{\pgfqpoint{9.923096in}{1.647133in}}{\pgfqpoint{9.925100in}{1.651970in}}{\pgfqpoint{9.925100in}{1.657014in}}%
\pgfpathcurveto{\pgfqpoint{9.925100in}{1.662058in}}{\pgfqpoint{9.923096in}{1.666896in}}{\pgfqpoint{9.919529in}{1.670462in}}%
\pgfpathcurveto{\pgfqpoint{9.915963in}{1.674028in}}{\pgfqpoint{9.911125in}{1.676032in}}{\pgfqpoint{9.906081in}{1.676032in}}%
\pgfpathcurveto{\pgfqpoint{9.901038in}{1.676032in}}{\pgfqpoint{9.896200in}{1.674028in}}{\pgfqpoint{9.892634in}{1.670462in}}%
\pgfpathcurveto{\pgfqpoint{9.889067in}{1.666896in}}{\pgfqpoint{9.887063in}{1.662058in}}{\pgfqpoint{9.887063in}{1.657014in}}%
\pgfpathcurveto{\pgfqpoint{9.887063in}{1.651970in}}{\pgfqpoint{9.889067in}{1.647133in}}{\pgfqpoint{9.892634in}{1.643566in}}%
\pgfpathcurveto{\pgfqpoint{9.896200in}{1.640000in}}{\pgfqpoint{9.901038in}{1.637996in}}{\pgfqpoint{9.906081in}{1.637996in}}%
\pgfpathclose%
\pgfusepath{fill}%
\end{pgfscope}%
\begin{pgfscope}%
\pgfpathrectangle{\pgfqpoint{6.572727in}{0.474100in}}{\pgfqpoint{4.227273in}{3.318700in}}%
\pgfusepath{clip}%
\pgfsetbuttcap%
\pgfsetroundjoin%
\definecolor{currentfill}{rgb}{0.993248,0.906157,0.143936}%
\pgfsetfillcolor{currentfill}%
\pgfsetfillopacity{0.700000}%
\pgfsetlinewidth{0.000000pt}%
\definecolor{currentstroke}{rgb}{0.000000,0.000000,0.000000}%
\pgfsetstrokecolor{currentstroke}%
\pgfsetstrokeopacity{0.700000}%
\pgfsetdash{}{0pt}%
\pgfpathmoveto{\pgfqpoint{7.620807in}{2.943823in}}%
\pgfpathcurveto{\pgfqpoint{7.625850in}{2.943823in}}{\pgfqpoint{7.630688in}{2.945827in}}{\pgfqpoint{7.634255in}{2.949393in}}%
\pgfpathcurveto{\pgfqpoint{7.637821in}{2.952960in}}{\pgfqpoint{7.639825in}{2.957798in}}{\pgfqpoint{7.639825in}{2.962841in}}%
\pgfpathcurveto{\pgfqpoint{7.639825in}{2.967885in}}{\pgfqpoint{7.637821in}{2.972723in}}{\pgfqpoint{7.634255in}{2.976289in}}%
\pgfpathcurveto{\pgfqpoint{7.630688in}{2.979855in}}{\pgfqpoint{7.625850in}{2.981859in}}{\pgfqpoint{7.620807in}{2.981859in}}%
\pgfpathcurveto{\pgfqpoint{7.615763in}{2.981859in}}{\pgfqpoint{7.610925in}{2.979855in}}{\pgfqpoint{7.607359in}{2.976289in}}%
\pgfpathcurveto{\pgfqpoint{7.603792in}{2.972723in}}{\pgfqpoint{7.601789in}{2.967885in}}{\pgfqpoint{7.601789in}{2.962841in}}%
\pgfpathcurveto{\pgfqpoint{7.601789in}{2.957798in}}{\pgfqpoint{7.603792in}{2.952960in}}{\pgfqpoint{7.607359in}{2.949393in}}%
\pgfpathcurveto{\pgfqpoint{7.610925in}{2.945827in}}{\pgfqpoint{7.615763in}{2.943823in}}{\pgfqpoint{7.620807in}{2.943823in}}%
\pgfpathclose%
\pgfusepath{fill}%
\end{pgfscope}%
\begin{pgfscope}%
\pgfpathrectangle{\pgfqpoint{6.572727in}{0.474100in}}{\pgfqpoint{4.227273in}{3.318700in}}%
\pgfusepath{clip}%
\pgfsetbuttcap%
\pgfsetroundjoin%
\definecolor{currentfill}{rgb}{0.127568,0.566949,0.550556}%
\pgfsetfillcolor{currentfill}%
\pgfsetfillopacity{0.700000}%
\pgfsetlinewidth{0.000000pt}%
\definecolor{currentstroke}{rgb}{0.000000,0.000000,0.000000}%
\pgfsetstrokecolor{currentstroke}%
\pgfsetstrokeopacity{0.700000}%
\pgfsetdash{}{0pt}%
\pgfpathmoveto{\pgfqpoint{8.829797in}{1.998608in}}%
\pgfpathcurveto{\pgfqpoint{8.834840in}{1.998608in}}{\pgfqpoint{8.839678in}{2.000612in}}{\pgfqpoint{8.843244in}{2.004178in}}%
\pgfpathcurveto{\pgfqpoint{8.846811in}{2.007745in}}{\pgfqpoint{8.848815in}{2.012583in}}{\pgfqpoint{8.848815in}{2.017626in}}%
\pgfpathcurveto{\pgfqpoint{8.848815in}{2.022670in}}{\pgfqpoint{8.846811in}{2.027508in}}{\pgfqpoint{8.843244in}{2.031074in}}%
\pgfpathcurveto{\pgfqpoint{8.839678in}{2.034640in}}{\pgfqpoint{8.834840in}{2.036644in}}{\pgfqpoint{8.829797in}{2.036644in}}%
\pgfpathcurveto{\pgfqpoint{8.824753in}{2.036644in}}{\pgfqpoint{8.819915in}{2.034640in}}{\pgfqpoint{8.816349in}{2.031074in}}%
\pgfpathcurveto{\pgfqpoint{8.812782in}{2.027508in}}{\pgfqpoint{8.810778in}{2.022670in}}{\pgfqpoint{8.810778in}{2.017626in}}%
\pgfpathcurveto{\pgfqpoint{8.810778in}{2.012583in}}{\pgfqpoint{8.812782in}{2.007745in}}{\pgfqpoint{8.816349in}{2.004178in}}%
\pgfpathcurveto{\pgfqpoint{8.819915in}{2.000612in}}{\pgfqpoint{8.824753in}{1.998608in}}{\pgfqpoint{8.829797in}{1.998608in}}%
\pgfpathclose%
\pgfusepath{fill}%
\end{pgfscope}%
\begin{pgfscope}%
\pgfpathrectangle{\pgfqpoint{6.572727in}{0.474100in}}{\pgfqpoint{4.227273in}{3.318700in}}%
\pgfusepath{clip}%
\pgfsetbuttcap%
\pgfsetroundjoin%
\definecolor{currentfill}{rgb}{0.993248,0.906157,0.143936}%
\pgfsetfillcolor{currentfill}%
\pgfsetfillopacity{0.700000}%
\pgfsetlinewidth{0.000000pt}%
\definecolor{currentstroke}{rgb}{0.000000,0.000000,0.000000}%
\pgfsetstrokecolor{currentstroke}%
\pgfsetstrokeopacity{0.700000}%
\pgfsetdash{}{0pt}%
\pgfpathmoveto{\pgfqpoint{8.213233in}{2.879108in}}%
\pgfpathcurveto{\pgfqpoint{8.218276in}{2.879108in}}{\pgfqpoint{8.223114in}{2.881112in}}{\pgfqpoint{8.226680in}{2.884678in}}%
\pgfpathcurveto{\pgfqpoint{8.230247in}{2.888245in}}{\pgfqpoint{8.232251in}{2.893082in}}{\pgfqpoint{8.232251in}{2.898126in}}%
\pgfpathcurveto{\pgfqpoint{8.232251in}{2.903170in}}{\pgfqpoint{8.230247in}{2.908007in}}{\pgfqpoint{8.226680in}{2.911574in}}%
\pgfpathcurveto{\pgfqpoint{8.223114in}{2.915140in}}{\pgfqpoint{8.218276in}{2.917144in}}{\pgfqpoint{8.213233in}{2.917144in}}%
\pgfpathcurveto{\pgfqpoint{8.208189in}{2.917144in}}{\pgfqpoint{8.203351in}{2.915140in}}{\pgfqpoint{8.199785in}{2.911574in}}%
\pgfpathcurveto{\pgfqpoint{8.196218in}{2.908007in}}{\pgfqpoint{8.194214in}{2.903170in}}{\pgfqpoint{8.194214in}{2.898126in}}%
\pgfpathcurveto{\pgfqpoint{8.194214in}{2.893082in}}{\pgfqpoint{8.196218in}{2.888245in}}{\pgfqpoint{8.199785in}{2.884678in}}%
\pgfpathcurveto{\pgfqpoint{8.203351in}{2.881112in}}{\pgfqpoint{8.208189in}{2.879108in}}{\pgfqpoint{8.213233in}{2.879108in}}%
\pgfpathclose%
\pgfusepath{fill}%
\end{pgfscope}%
\begin{pgfscope}%
\pgfpathrectangle{\pgfqpoint{6.572727in}{0.474100in}}{\pgfqpoint{4.227273in}{3.318700in}}%
\pgfusepath{clip}%
\pgfsetbuttcap%
\pgfsetroundjoin%
\definecolor{currentfill}{rgb}{0.127568,0.566949,0.550556}%
\pgfsetfillcolor{currentfill}%
\pgfsetfillopacity{0.700000}%
\pgfsetlinewidth{0.000000pt}%
\definecolor{currentstroke}{rgb}{0.000000,0.000000,0.000000}%
\pgfsetstrokecolor{currentstroke}%
\pgfsetstrokeopacity{0.700000}%
\pgfsetdash{}{0pt}%
\pgfpathmoveto{\pgfqpoint{9.332851in}{1.556872in}}%
\pgfpathcurveto{\pgfqpoint{9.337894in}{1.556872in}}{\pgfqpoint{9.342732in}{1.558876in}}{\pgfqpoint{9.346298in}{1.562442in}}%
\pgfpathcurveto{\pgfqpoint{9.349865in}{1.566009in}}{\pgfqpoint{9.351869in}{1.570847in}}{\pgfqpoint{9.351869in}{1.575890in}}%
\pgfpathcurveto{\pgfqpoint{9.351869in}{1.580934in}}{\pgfqpoint{9.349865in}{1.585772in}}{\pgfqpoint{9.346298in}{1.589338in}}%
\pgfpathcurveto{\pgfqpoint{9.342732in}{1.592905in}}{\pgfqpoint{9.337894in}{1.594908in}}{\pgfqpoint{9.332851in}{1.594908in}}%
\pgfpathcurveto{\pgfqpoint{9.327807in}{1.594908in}}{\pgfqpoint{9.322969in}{1.592905in}}{\pgfqpoint{9.319403in}{1.589338in}}%
\pgfpathcurveto{\pgfqpoint{9.315836in}{1.585772in}}{\pgfqpoint{9.313832in}{1.580934in}}{\pgfqpoint{9.313832in}{1.575890in}}%
\pgfpathcurveto{\pgfqpoint{9.313832in}{1.570847in}}{\pgfqpoint{9.315836in}{1.566009in}}{\pgfqpoint{9.319403in}{1.562442in}}%
\pgfpathcurveto{\pgfqpoint{9.322969in}{1.558876in}}{\pgfqpoint{9.327807in}{1.556872in}}{\pgfqpoint{9.332851in}{1.556872in}}%
\pgfpathclose%
\pgfusepath{fill}%
\end{pgfscope}%
\begin{pgfscope}%
\pgfpathrectangle{\pgfqpoint{6.572727in}{0.474100in}}{\pgfqpoint{4.227273in}{3.318700in}}%
\pgfusepath{clip}%
\pgfsetbuttcap%
\pgfsetroundjoin%
\definecolor{currentfill}{rgb}{0.127568,0.566949,0.550556}%
\pgfsetfillcolor{currentfill}%
\pgfsetfillopacity{0.700000}%
\pgfsetlinewidth{0.000000pt}%
\definecolor{currentstroke}{rgb}{0.000000,0.000000,0.000000}%
\pgfsetstrokecolor{currentstroke}%
\pgfsetstrokeopacity{0.700000}%
\pgfsetdash{}{0pt}%
\pgfpathmoveto{\pgfqpoint{9.693822in}{1.991496in}}%
\pgfpathcurveto{\pgfqpoint{9.698866in}{1.991496in}}{\pgfqpoint{9.703704in}{1.993500in}}{\pgfqpoint{9.707270in}{1.997066in}}%
\pgfpathcurveto{\pgfqpoint{9.710836in}{2.000633in}}{\pgfqpoint{9.712840in}{2.005470in}}{\pgfqpoint{9.712840in}{2.010514in}}%
\pgfpathcurveto{\pgfqpoint{9.712840in}{2.015558in}}{\pgfqpoint{9.710836in}{2.020396in}}{\pgfqpoint{9.707270in}{2.023962in}}%
\pgfpathcurveto{\pgfqpoint{9.703704in}{2.027528in}}{\pgfqpoint{9.698866in}{2.029532in}}{\pgfqpoint{9.693822in}{2.029532in}}%
\pgfpathcurveto{\pgfqpoint{9.688778in}{2.029532in}}{\pgfqpoint{9.683941in}{2.027528in}}{\pgfqpoint{9.680374in}{2.023962in}}%
\pgfpathcurveto{\pgfqpoint{9.676808in}{2.020396in}}{\pgfqpoint{9.674804in}{2.015558in}}{\pgfqpoint{9.674804in}{2.010514in}}%
\pgfpathcurveto{\pgfqpoint{9.674804in}{2.005470in}}{\pgfqpoint{9.676808in}{2.000633in}}{\pgfqpoint{9.680374in}{1.997066in}}%
\pgfpathcurveto{\pgfqpoint{9.683941in}{1.993500in}}{\pgfqpoint{9.688778in}{1.991496in}}{\pgfqpoint{9.693822in}{1.991496in}}%
\pgfpathclose%
\pgfusepath{fill}%
\end{pgfscope}%
\begin{pgfscope}%
\pgfpathrectangle{\pgfqpoint{6.572727in}{0.474100in}}{\pgfqpoint{4.227273in}{3.318700in}}%
\pgfusepath{clip}%
\pgfsetbuttcap%
\pgfsetroundjoin%
\definecolor{currentfill}{rgb}{0.127568,0.566949,0.550556}%
\pgfsetfillcolor{currentfill}%
\pgfsetfillopacity{0.700000}%
\pgfsetlinewidth{0.000000pt}%
\definecolor{currentstroke}{rgb}{0.000000,0.000000,0.000000}%
\pgfsetstrokecolor{currentstroke}%
\pgfsetstrokeopacity{0.700000}%
\pgfsetdash{}{0pt}%
\pgfpathmoveto{\pgfqpoint{9.584245in}{2.014994in}}%
\pgfpathcurveto{\pgfqpoint{9.589289in}{2.014994in}}{\pgfqpoint{9.594127in}{2.016998in}}{\pgfqpoint{9.597693in}{2.020564in}}%
\pgfpathcurveto{\pgfqpoint{9.601259in}{2.024131in}}{\pgfqpoint{9.603263in}{2.028968in}}{\pgfqpoint{9.603263in}{2.034012in}}%
\pgfpathcurveto{\pgfqpoint{9.603263in}{2.039056in}}{\pgfqpoint{9.601259in}{2.043893in}}{\pgfqpoint{9.597693in}{2.047460in}}%
\pgfpathcurveto{\pgfqpoint{9.594127in}{2.051026in}}{\pgfqpoint{9.589289in}{2.053030in}}{\pgfqpoint{9.584245in}{2.053030in}}%
\pgfpathcurveto{\pgfqpoint{9.579201in}{2.053030in}}{\pgfqpoint{9.574364in}{2.051026in}}{\pgfqpoint{9.570797in}{2.047460in}}%
\pgfpathcurveto{\pgfqpoint{9.567231in}{2.043893in}}{\pgfqpoint{9.565227in}{2.039056in}}{\pgfqpoint{9.565227in}{2.034012in}}%
\pgfpathcurveto{\pgfqpoint{9.565227in}{2.028968in}}{\pgfqpoint{9.567231in}{2.024131in}}{\pgfqpoint{9.570797in}{2.020564in}}%
\pgfpathcurveto{\pgfqpoint{9.574364in}{2.016998in}}{\pgfqpoint{9.579201in}{2.014994in}}{\pgfqpoint{9.584245in}{2.014994in}}%
\pgfpathclose%
\pgfusepath{fill}%
\end{pgfscope}%
\begin{pgfscope}%
\pgfpathrectangle{\pgfqpoint{6.572727in}{0.474100in}}{\pgfqpoint{4.227273in}{3.318700in}}%
\pgfusepath{clip}%
\pgfsetbuttcap%
\pgfsetroundjoin%
\definecolor{currentfill}{rgb}{0.267004,0.004874,0.329415}%
\pgfsetfillcolor{currentfill}%
\pgfsetfillopacity{0.700000}%
\pgfsetlinewidth{0.000000pt}%
\definecolor{currentstroke}{rgb}{0.000000,0.000000,0.000000}%
\pgfsetstrokecolor{currentstroke}%
\pgfsetstrokeopacity{0.700000}%
\pgfsetdash{}{0pt}%
\pgfpathmoveto{\pgfqpoint{7.763943in}{1.449971in}}%
\pgfpathcurveto{\pgfqpoint{7.768987in}{1.449971in}}{\pgfqpoint{7.773825in}{1.451974in}}{\pgfqpoint{7.777391in}{1.455541in}}%
\pgfpathcurveto{\pgfqpoint{7.780958in}{1.459107in}}{\pgfqpoint{7.782961in}{1.463945in}}{\pgfqpoint{7.782961in}{1.468989in}}%
\pgfpathcurveto{\pgfqpoint{7.782961in}{1.474032in}}{\pgfqpoint{7.780958in}{1.478870in}}{\pgfqpoint{7.777391in}{1.482437in}}%
\pgfpathcurveto{\pgfqpoint{7.773825in}{1.486003in}}{\pgfqpoint{7.768987in}{1.488007in}}{\pgfqpoint{7.763943in}{1.488007in}}%
\pgfpathcurveto{\pgfqpoint{7.758900in}{1.488007in}}{\pgfqpoint{7.754062in}{1.486003in}}{\pgfqpoint{7.750495in}{1.482437in}}%
\pgfpathcurveto{\pgfqpoint{7.746929in}{1.478870in}}{\pgfqpoint{7.744925in}{1.474032in}}{\pgfqpoint{7.744925in}{1.468989in}}%
\pgfpathcurveto{\pgfqpoint{7.744925in}{1.463945in}}{\pgfqpoint{7.746929in}{1.459107in}}{\pgfqpoint{7.750495in}{1.455541in}}%
\pgfpathcurveto{\pgfqpoint{7.754062in}{1.451974in}}{\pgfqpoint{7.758900in}{1.449971in}}{\pgfqpoint{7.763943in}{1.449971in}}%
\pgfpathclose%
\pgfusepath{fill}%
\end{pgfscope}%
\begin{pgfscope}%
\pgfpathrectangle{\pgfqpoint{6.572727in}{0.474100in}}{\pgfqpoint{4.227273in}{3.318700in}}%
\pgfusepath{clip}%
\pgfsetbuttcap%
\pgfsetroundjoin%
\definecolor{currentfill}{rgb}{0.127568,0.566949,0.550556}%
\pgfsetfillcolor{currentfill}%
\pgfsetfillopacity{0.700000}%
\pgfsetlinewidth{0.000000pt}%
\definecolor{currentstroke}{rgb}{0.000000,0.000000,0.000000}%
\pgfsetstrokecolor{currentstroke}%
\pgfsetstrokeopacity{0.700000}%
\pgfsetdash{}{0pt}%
\pgfpathmoveto{\pgfqpoint{9.068067in}{1.463435in}}%
\pgfpathcurveto{\pgfqpoint{9.073110in}{1.463435in}}{\pgfqpoint{9.077948in}{1.465439in}}{\pgfqpoint{9.081515in}{1.469005in}}%
\pgfpathcurveto{\pgfqpoint{9.085081in}{1.472572in}}{\pgfqpoint{9.087085in}{1.477409in}}{\pgfqpoint{9.087085in}{1.482453in}}%
\pgfpathcurveto{\pgfqpoint{9.087085in}{1.487497in}}{\pgfqpoint{9.085081in}{1.492334in}}{\pgfqpoint{9.081515in}{1.495901in}}%
\pgfpathcurveto{\pgfqpoint{9.077948in}{1.499467in}}{\pgfqpoint{9.073110in}{1.501471in}}{\pgfqpoint{9.068067in}{1.501471in}}%
\pgfpathcurveto{\pgfqpoint{9.063023in}{1.501471in}}{\pgfqpoint{9.058185in}{1.499467in}}{\pgfqpoint{9.054619in}{1.495901in}}%
\pgfpathcurveto{\pgfqpoint{9.051052in}{1.492334in}}{\pgfqpoint{9.049049in}{1.487497in}}{\pgfqpoint{9.049049in}{1.482453in}}%
\pgfpathcurveto{\pgfqpoint{9.049049in}{1.477409in}}{\pgfqpoint{9.051052in}{1.472572in}}{\pgfqpoint{9.054619in}{1.469005in}}%
\pgfpathcurveto{\pgfqpoint{9.058185in}{1.465439in}}{\pgfqpoint{9.063023in}{1.463435in}}{\pgfqpoint{9.068067in}{1.463435in}}%
\pgfpathclose%
\pgfusepath{fill}%
\end{pgfscope}%
\begin{pgfscope}%
\pgfpathrectangle{\pgfqpoint{6.572727in}{0.474100in}}{\pgfqpoint{4.227273in}{3.318700in}}%
\pgfusepath{clip}%
\pgfsetbuttcap%
\pgfsetroundjoin%
\definecolor{currentfill}{rgb}{0.267004,0.004874,0.329415}%
\pgfsetfillcolor{currentfill}%
\pgfsetfillopacity{0.700000}%
\pgfsetlinewidth{0.000000pt}%
\definecolor{currentstroke}{rgb}{0.000000,0.000000,0.000000}%
\pgfsetstrokecolor{currentstroke}%
\pgfsetstrokeopacity{0.700000}%
\pgfsetdash{}{0pt}%
\pgfpathmoveto{\pgfqpoint{7.934818in}{1.491675in}}%
\pgfpathcurveto{\pgfqpoint{7.939861in}{1.491675in}}{\pgfqpoint{7.944699in}{1.493679in}}{\pgfqpoint{7.948265in}{1.497245in}}%
\pgfpathcurveto{\pgfqpoint{7.951832in}{1.500812in}}{\pgfqpoint{7.953836in}{1.505649in}}{\pgfqpoint{7.953836in}{1.510693in}}%
\pgfpathcurveto{\pgfqpoint{7.953836in}{1.515737in}}{\pgfqpoint{7.951832in}{1.520574in}}{\pgfqpoint{7.948265in}{1.524141in}}%
\pgfpathcurveto{\pgfqpoint{7.944699in}{1.527707in}}{\pgfqpoint{7.939861in}{1.529711in}}{\pgfqpoint{7.934818in}{1.529711in}}%
\pgfpathcurveto{\pgfqpoint{7.929774in}{1.529711in}}{\pgfqpoint{7.924936in}{1.527707in}}{\pgfqpoint{7.921370in}{1.524141in}}%
\pgfpathcurveto{\pgfqpoint{7.917803in}{1.520574in}}{\pgfqpoint{7.915799in}{1.515737in}}{\pgfqpoint{7.915799in}{1.510693in}}%
\pgfpathcurveto{\pgfqpoint{7.915799in}{1.505649in}}{\pgfqpoint{7.917803in}{1.500812in}}{\pgfqpoint{7.921370in}{1.497245in}}%
\pgfpathcurveto{\pgfqpoint{7.924936in}{1.493679in}}{\pgfqpoint{7.929774in}{1.491675in}}{\pgfqpoint{7.934818in}{1.491675in}}%
\pgfpathclose%
\pgfusepath{fill}%
\end{pgfscope}%
\begin{pgfscope}%
\pgfpathrectangle{\pgfqpoint{6.572727in}{0.474100in}}{\pgfqpoint{4.227273in}{3.318700in}}%
\pgfusepath{clip}%
\pgfsetbuttcap%
\pgfsetroundjoin%
\definecolor{currentfill}{rgb}{0.267004,0.004874,0.329415}%
\pgfsetfillcolor{currentfill}%
\pgfsetfillopacity{0.700000}%
\pgfsetlinewidth{0.000000pt}%
\definecolor{currentstroke}{rgb}{0.000000,0.000000,0.000000}%
\pgfsetstrokecolor{currentstroke}%
\pgfsetstrokeopacity{0.700000}%
\pgfsetdash{}{0pt}%
\pgfpathmoveto{\pgfqpoint{8.053854in}{1.926856in}}%
\pgfpathcurveto{\pgfqpoint{8.058898in}{1.926856in}}{\pgfqpoint{8.063736in}{1.928860in}}{\pgfqpoint{8.067302in}{1.932426in}}%
\pgfpathcurveto{\pgfqpoint{8.070869in}{1.935993in}}{\pgfqpoint{8.072873in}{1.940830in}}{\pgfqpoint{8.072873in}{1.945874in}}%
\pgfpathcurveto{\pgfqpoint{8.072873in}{1.950918in}}{\pgfqpoint{8.070869in}{1.955755in}}{\pgfqpoint{8.067302in}{1.959322in}}%
\pgfpathcurveto{\pgfqpoint{8.063736in}{1.962888in}}{\pgfqpoint{8.058898in}{1.964892in}}{\pgfqpoint{8.053854in}{1.964892in}}%
\pgfpathcurveto{\pgfqpoint{8.048811in}{1.964892in}}{\pgfqpoint{8.043973in}{1.962888in}}{\pgfqpoint{8.040407in}{1.959322in}}%
\pgfpathcurveto{\pgfqpoint{8.036840in}{1.955755in}}{\pgfqpoint{8.034836in}{1.950918in}}{\pgfqpoint{8.034836in}{1.945874in}}%
\pgfpathcurveto{\pgfqpoint{8.034836in}{1.940830in}}{\pgfqpoint{8.036840in}{1.935993in}}{\pgfqpoint{8.040407in}{1.932426in}}%
\pgfpathcurveto{\pgfqpoint{8.043973in}{1.928860in}}{\pgfqpoint{8.048811in}{1.926856in}}{\pgfqpoint{8.053854in}{1.926856in}}%
\pgfpathclose%
\pgfusepath{fill}%
\end{pgfscope}%
\begin{pgfscope}%
\pgfpathrectangle{\pgfqpoint{6.572727in}{0.474100in}}{\pgfqpoint{4.227273in}{3.318700in}}%
\pgfusepath{clip}%
\pgfsetbuttcap%
\pgfsetroundjoin%
\definecolor{currentfill}{rgb}{0.267004,0.004874,0.329415}%
\pgfsetfillcolor{currentfill}%
\pgfsetfillopacity{0.700000}%
\pgfsetlinewidth{0.000000pt}%
\definecolor{currentstroke}{rgb}{0.000000,0.000000,0.000000}%
\pgfsetstrokecolor{currentstroke}%
\pgfsetstrokeopacity{0.700000}%
\pgfsetdash{}{0pt}%
\pgfpathmoveto{\pgfqpoint{7.732711in}{1.428347in}}%
\pgfpathcurveto{\pgfqpoint{7.737755in}{1.428347in}}{\pgfqpoint{7.742592in}{1.430351in}}{\pgfqpoint{7.746159in}{1.433918in}}%
\pgfpathcurveto{\pgfqpoint{7.749725in}{1.437484in}}{\pgfqpoint{7.751729in}{1.442322in}}{\pgfqpoint{7.751729in}{1.447366in}}%
\pgfpathcurveto{\pgfqpoint{7.751729in}{1.452409in}}{\pgfqpoint{7.749725in}{1.457247in}}{\pgfqpoint{7.746159in}{1.460813in}}%
\pgfpathcurveto{\pgfqpoint{7.742592in}{1.464380in}}{\pgfqpoint{7.737755in}{1.466384in}}{\pgfqpoint{7.732711in}{1.466384in}}%
\pgfpathcurveto{\pgfqpoint{7.727667in}{1.466384in}}{\pgfqpoint{7.722830in}{1.464380in}}{\pgfqpoint{7.719263in}{1.460813in}}%
\pgfpathcurveto{\pgfqpoint{7.715697in}{1.457247in}}{\pgfqpoint{7.713693in}{1.452409in}}{\pgfqpoint{7.713693in}{1.447366in}}%
\pgfpathcurveto{\pgfqpoint{7.713693in}{1.442322in}}{\pgfqpoint{7.715697in}{1.437484in}}{\pgfqpoint{7.719263in}{1.433918in}}%
\pgfpathcurveto{\pgfqpoint{7.722830in}{1.430351in}}{\pgfqpoint{7.727667in}{1.428347in}}{\pgfqpoint{7.732711in}{1.428347in}}%
\pgfpathclose%
\pgfusepath{fill}%
\end{pgfscope}%
\begin{pgfscope}%
\pgfpathrectangle{\pgfqpoint{6.572727in}{0.474100in}}{\pgfqpoint{4.227273in}{3.318700in}}%
\pgfusepath{clip}%
\pgfsetbuttcap%
\pgfsetroundjoin%
\definecolor{currentfill}{rgb}{0.267004,0.004874,0.329415}%
\pgfsetfillcolor{currentfill}%
\pgfsetfillopacity{0.700000}%
\pgfsetlinewidth{0.000000pt}%
\definecolor{currentstroke}{rgb}{0.000000,0.000000,0.000000}%
\pgfsetstrokecolor{currentstroke}%
\pgfsetstrokeopacity{0.700000}%
\pgfsetdash{}{0pt}%
\pgfpathmoveto{\pgfqpoint{8.593097in}{1.157935in}}%
\pgfpathcurveto{\pgfqpoint{8.598140in}{1.157935in}}{\pgfqpoint{8.602978in}{1.159939in}}{\pgfqpoint{8.606544in}{1.163505in}}%
\pgfpathcurveto{\pgfqpoint{8.610111in}{1.167071in}}{\pgfqpoint{8.612115in}{1.171909in}}{\pgfqpoint{8.612115in}{1.176953in}}%
\pgfpathcurveto{\pgfqpoint{8.612115in}{1.181997in}}{\pgfqpoint{8.610111in}{1.186834in}}{\pgfqpoint{8.606544in}{1.190401in}}%
\pgfpathcurveto{\pgfqpoint{8.602978in}{1.193967in}}{\pgfqpoint{8.598140in}{1.195971in}}{\pgfqpoint{8.593097in}{1.195971in}}%
\pgfpathcurveto{\pgfqpoint{8.588053in}{1.195971in}}{\pgfqpoint{8.583215in}{1.193967in}}{\pgfqpoint{8.579649in}{1.190401in}}%
\pgfpathcurveto{\pgfqpoint{8.576082in}{1.186834in}}{\pgfqpoint{8.574078in}{1.181997in}}{\pgfqpoint{8.574078in}{1.176953in}}%
\pgfpathcurveto{\pgfqpoint{8.574078in}{1.171909in}}{\pgfqpoint{8.576082in}{1.167071in}}{\pgfqpoint{8.579649in}{1.163505in}}%
\pgfpathcurveto{\pgfqpoint{8.583215in}{1.159939in}}{\pgfqpoint{8.588053in}{1.157935in}}{\pgfqpoint{8.593097in}{1.157935in}}%
\pgfpathclose%
\pgfusepath{fill}%
\end{pgfscope}%
\begin{pgfscope}%
\pgfpathrectangle{\pgfqpoint{6.572727in}{0.474100in}}{\pgfqpoint{4.227273in}{3.318700in}}%
\pgfusepath{clip}%
\pgfsetbuttcap%
\pgfsetroundjoin%
\definecolor{currentfill}{rgb}{0.127568,0.566949,0.550556}%
\pgfsetfillcolor{currentfill}%
\pgfsetfillopacity{0.700000}%
\pgfsetlinewidth{0.000000pt}%
\definecolor{currentstroke}{rgb}{0.000000,0.000000,0.000000}%
\pgfsetstrokecolor{currentstroke}%
\pgfsetstrokeopacity{0.700000}%
\pgfsetdash{}{0pt}%
\pgfpathmoveto{\pgfqpoint{9.435072in}{1.816246in}}%
\pgfpathcurveto{\pgfqpoint{9.440116in}{1.816246in}}{\pgfqpoint{9.444953in}{1.818250in}}{\pgfqpoint{9.448520in}{1.821816in}}%
\pgfpathcurveto{\pgfqpoint{9.452086in}{1.825383in}}{\pgfqpoint{9.454090in}{1.830220in}}{\pgfqpoint{9.454090in}{1.835264in}}%
\pgfpathcurveto{\pgfqpoint{9.454090in}{1.840308in}}{\pgfqpoint{9.452086in}{1.845146in}}{\pgfqpoint{9.448520in}{1.848712in}}%
\pgfpathcurveto{\pgfqpoint{9.444953in}{1.852278in}}{\pgfqpoint{9.440116in}{1.854282in}}{\pgfqpoint{9.435072in}{1.854282in}}%
\pgfpathcurveto{\pgfqpoint{9.430028in}{1.854282in}}{\pgfqpoint{9.425190in}{1.852278in}}{\pgfqpoint{9.421624in}{1.848712in}}%
\pgfpathcurveto{\pgfqpoint{9.418058in}{1.845146in}}{\pgfqpoint{9.416054in}{1.840308in}}{\pgfqpoint{9.416054in}{1.835264in}}%
\pgfpathcurveto{\pgfqpoint{9.416054in}{1.830220in}}{\pgfqpoint{9.418058in}{1.825383in}}{\pgfqpoint{9.421624in}{1.821816in}}%
\pgfpathcurveto{\pgfqpoint{9.425190in}{1.818250in}}{\pgfqpoint{9.430028in}{1.816246in}}{\pgfqpoint{9.435072in}{1.816246in}}%
\pgfpathclose%
\pgfusepath{fill}%
\end{pgfscope}%
\begin{pgfscope}%
\pgfpathrectangle{\pgfqpoint{6.572727in}{0.474100in}}{\pgfqpoint{4.227273in}{3.318700in}}%
\pgfusepath{clip}%
\pgfsetbuttcap%
\pgfsetroundjoin%
\definecolor{currentfill}{rgb}{0.267004,0.004874,0.329415}%
\pgfsetfillcolor{currentfill}%
\pgfsetfillopacity{0.700000}%
\pgfsetlinewidth{0.000000pt}%
\definecolor{currentstroke}{rgb}{0.000000,0.000000,0.000000}%
\pgfsetstrokecolor{currentstroke}%
\pgfsetstrokeopacity{0.700000}%
\pgfsetdash{}{0pt}%
\pgfpathmoveto{\pgfqpoint{7.853059in}{1.427186in}}%
\pgfpathcurveto{\pgfqpoint{7.858103in}{1.427186in}}{\pgfqpoint{7.862941in}{1.429190in}}{\pgfqpoint{7.866507in}{1.432756in}}%
\pgfpathcurveto{\pgfqpoint{7.870073in}{1.436322in}}{\pgfqpoint{7.872077in}{1.441160in}}{\pgfqpoint{7.872077in}{1.446204in}}%
\pgfpathcurveto{\pgfqpoint{7.872077in}{1.451248in}}{\pgfqpoint{7.870073in}{1.456085in}}{\pgfqpoint{7.866507in}{1.459652in}}%
\pgfpathcurveto{\pgfqpoint{7.862941in}{1.463218in}}{\pgfqpoint{7.858103in}{1.465222in}}{\pgfqpoint{7.853059in}{1.465222in}}%
\pgfpathcurveto{\pgfqpoint{7.848016in}{1.465222in}}{\pgfqpoint{7.843178in}{1.463218in}}{\pgfqpoint{7.839611in}{1.459652in}}%
\pgfpathcurveto{\pgfqpoint{7.836045in}{1.456085in}}{\pgfqpoint{7.834041in}{1.451248in}}{\pgfqpoint{7.834041in}{1.446204in}}%
\pgfpathcurveto{\pgfqpoint{7.834041in}{1.441160in}}{\pgfqpoint{7.836045in}{1.436322in}}{\pgfqpoint{7.839611in}{1.432756in}}%
\pgfpathcurveto{\pgfqpoint{7.843178in}{1.429190in}}{\pgfqpoint{7.848016in}{1.427186in}}{\pgfqpoint{7.853059in}{1.427186in}}%
\pgfpathclose%
\pgfusepath{fill}%
\end{pgfscope}%
\begin{pgfscope}%
\pgfpathrectangle{\pgfqpoint{6.572727in}{0.474100in}}{\pgfqpoint{4.227273in}{3.318700in}}%
\pgfusepath{clip}%
\pgfsetbuttcap%
\pgfsetroundjoin%
\definecolor{currentfill}{rgb}{0.267004,0.004874,0.329415}%
\pgfsetfillcolor{currentfill}%
\pgfsetfillopacity{0.700000}%
\pgfsetlinewidth{0.000000pt}%
\definecolor{currentstroke}{rgb}{0.000000,0.000000,0.000000}%
\pgfsetstrokecolor{currentstroke}%
\pgfsetstrokeopacity{0.700000}%
\pgfsetdash{}{0pt}%
\pgfpathmoveto{\pgfqpoint{7.529593in}{1.459074in}}%
\pgfpathcurveto{\pgfqpoint{7.534637in}{1.459074in}}{\pgfqpoint{7.539474in}{1.461077in}}{\pgfqpoint{7.543041in}{1.464644in}}%
\pgfpathcurveto{\pgfqpoint{7.546607in}{1.468210in}}{\pgfqpoint{7.548611in}{1.473048in}}{\pgfqpoint{7.548611in}{1.478092in}}%
\pgfpathcurveto{\pgfqpoint{7.548611in}{1.483135in}}{\pgfqpoint{7.546607in}{1.487973in}}{\pgfqpoint{7.543041in}{1.491540in}}%
\pgfpathcurveto{\pgfqpoint{7.539474in}{1.495106in}}{\pgfqpoint{7.534637in}{1.497110in}}{\pgfqpoint{7.529593in}{1.497110in}}%
\pgfpathcurveto{\pgfqpoint{7.524549in}{1.497110in}}{\pgfqpoint{7.519712in}{1.495106in}}{\pgfqpoint{7.516145in}{1.491540in}}%
\pgfpathcurveto{\pgfqpoint{7.512579in}{1.487973in}}{\pgfqpoint{7.510575in}{1.483135in}}{\pgfqpoint{7.510575in}{1.478092in}}%
\pgfpathcurveto{\pgfqpoint{7.510575in}{1.473048in}}{\pgfqpoint{7.512579in}{1.468210in}}{\pgfqpoint{7.516145in}{1.464644in}}%
\pgfpathcurveto{\pgfqpoint{7.519712in}{1.461077in}}{\pgfqpoint{7.524549in}{1.459074in}}{\pgfqpoint{7.529593in}{1.459074in}}%
\pgfpathclose%
\pgfusepath{fill}%
\end{pgfscope}%
\begin{pgfscope}%
\pgfpathrectangle{\pgfqpoint{6.572727in}{0.474100in}}{\pgfqpoint{4.227273in}{3.318700in}}%
\pgfusepath{clip}%
\pgfsetbuttcap%
\pgfsetroundjoin%
\definecolor{currentfill}{rgb}{0.993248,0.906157,0.143936}%
\pgfsetfillcolor{currentfill}%
\pgfsetfillopacity{0.700000}%
\pgfsetlinewidth{0.000000pt}%
\definecolor{currentstroke}{rgb}{0.000000,0.000000,0.000000}%
\pgfsetstrokecolor{currentstroke}%
\pgfsetstrokeopacity{0.700000}%
\pgfsetdash{}{0pt}%
\pgfpathmoveto{\pgfqpoint{8.803362in}{3.390591in}}%
\pgfpathcurveto{\pgfqpoint{8.808406in}{3.390591in}}{\pgfqpoint{8.813243in}{3.392595in}}{\pgfqpoint{8.816810in}{3.396161in}}%
\pgfpathcurveto{\pgfqpoint{8.820376in}{3.399728in}}{\pgfqpoint{8.822380in}{3.404566in}}{\pgfqpoint{8.822380in}{3.409609in}}%
\pgfpathcurveto{\pgfqpoint{8.822380in}{3.414653in}}{\pgfqpoint{8.820376in}{3.419491in}}{\pgfqpoint{8.816810in}{3.423057in}}%
\pgfpathcurveto{\pgfqpoint{8.813243in}{3.426624in}}{\pgfqpoint{8.808406in}{3.428627in}}{\pgfqpoint{8.803362in}{3.428627in}}%
\pgfpathcurveto{\pgfqpoint{8.798318in}{3.428627in}}{\pgfqpoint{8.793481in}{3.426624in}}{\pgfqpoint{8.789914in}{3.423057in}}%
\pgfpathcurveto{\pgfqpoint{8.786348in}{3.419491in}}{\pgfqpoint{8.784344in}{3.414653in}}{\pgfqpoint{8.784344in}{3.409609in}}%
\pgfpathcurveto{\pgfqpoint{8.784344in}{3.404566in}}{\pgfqpoint{8.786348in}{3.399728in}}{\pgfqpoint{8.789914in}{3.396161in}}%
\pgfpathcurveto{\pgfqpoint{8.793481in}{3.392595in}}{\pgfqpoint{8.798318in}{3.390591in}}{\pgfqpoint{8.803362in}{3.390591in}}%
\pgfpathclose%
\pgfusepath{fill}%
\end{pgfscope}%
\begin{pgfscope}%
\pgfpathrectangle{\pgfqpoint{6.572727in}{0.474100in}}{\pgfqpoint{4.227273in}{3.318700in}}%
\pgfusepath{clip}%
\pgfsetbuttcap%
\pgfsetroundjoin%
\definecolor{currentfill}{rgb}{0.993248,0.906157,0.143936}%
\pgfsetfillcolor{currentfill}%
\pgfsetfillopacity{0.700000}%
\pgfsetlinewidth{0.000000pt}%
\definecolor{currentstroke}{rgb}{0.000000,0.000000,0.000000}%
\pgfsetstrokecolor{currentstroke}%
\pgfsetstrokeopacity{0.700000}%
\pgfsetdash{}{0pt}%
\pgfpathmoveto{\pgfqpoint{7.813722in}{3.465717in}}%
\pgfpathcurveto{\pgfqpoint{7.818766in}{3.465717in}}{\pgfqpoint{7.823603in}{3.467720in}}{\pgfqpoint{7.827170in}{3.471287in}}%
\pgfpathcurveto{\pgfqpoint{7.830736in}{3.474853in}}{\pgfqpoint{7.832740in}{3.479691in}}{\pgfqpoint{7.832740in}{3.484735in}}%
\pgfpathcurveto{\pgfqpoint{7.832740in}{3.489778in}}{\pgfqpoint{7.830736in}{3.494616in}}{\pgfqpoint{7.827170in}{3.498183in}}%
\pgfpathcurveto{\pgfqpoint{7.823603in}{3.501749in}}{\pgfqpoint{7.818766in}{3.503753in}}{\pgfqpoint{7.813722in}{3.503753in}}%
\pgfpathcurveto{\pgfqpoint{7.808678in}{3.503753in}}{\pgfqpoint{7.803841in}{3.501749in}}{\pgfqpoint{7.800274in}{3.498183in}}%
\pgfpathcurveto{\pgfqpoint{7.796708in}{3.494616in}}{\pgfqpoint{7.794704in}{3.489778in}}{\pgfqpoint{7.794704in}{3.484735in}}%
\pgfpathcurveto{\pgfqpoint{7.794704in}{3.479691in}}{\pgfqpoint{7.796708in}{3.474853in}}{\pgfqpoint{7.800274in}{3.471287in}}%
\pgfpathcurveto{\pgfqpoint{7.803841in}{3.467720in}}{\pgfqpoint{7.808678in}{3.465717in}}{\pgfqpoint{7.813722in}{3.465717in}}%
\pgfpathclose%
\pgfusepath{fill}%
\end{pgfscope}%
\begin{pgfscope}%
\pgfpathrectangle{\pgfqpoint{6.572727in}{0.474100in}}{\pgfqpoint{4.227273in}{3.318700in}}%
\pgfusepath{clip}%
\pgfsetbuttcap%
\pgfsetroundjoin%
\definecolor{currentfill}{rgb}{0.267004,0.004874,0.329415}%
\pgfsetfillcolor{currentfill}%
\pgfsetfillopacity{0.700000}%
\pgfsetlinewidth{0.000000pt}%
\definecolor{currentstroke}{rgb}{0.000000,0.000000,0.000000}%
\pgfsetstrokecolor{currentstroke}%
\pgfsetstrokeopacity{0.700000}%
\pgfsetdash{}{0pt}%
\pgfpathmoveto{\pgfqpoint{7.181732in}{1.420284in}}%
\pgfpathcurveto{\pgfqpoint{7.186775in}{1.420284in}}{\pgfqpoint{7.191613in}{1.422288in}}{\pgfqpoint{7.195180in}{1.425854in}}%
\pgfpathcurveto{\pgfqpoint{7.198746in}{1.429421in}}{\pgfqpoint{7.200750in}{1.434258in}}{\pgfqpoint{7.200750in}{1.439302in}}%
\pgfpathcurveto{\pgfqpoint{7.200750in}{1.444346in}}{\pgfqpoint{7.198746in}{1.449184in}}{\pgfqpoint{7.195180in}{1.452750in}}%
\pgfpathcurveto{\pgfqpoint{7.191613in}{1.456316in}}{\pgfqpoint{7.186775in}{1.458320in}}{\pgfqpoint{7.181732in}{1.458320in}}%
\pgfpathcurveto{\pgfqpoint{7.176688in}{1.458320in}}{\pgfqpoint{7.171850in}{1.456316in}}{\pgfqpoint{7.168284in}{1.452750in}}%
\pgfpathcurveto{\pgfqpoint{7.164717in}{1.449184in}}{\pgfqpoint{7.162714in}{1.444346in}}{\pgfqpoint{7.162714in}{1.439302in}}%
\pgfpathcurveto{\pgfqpoint{7.162714in}{1.434258in}}{\pgfqpoint{7.164717in}{1.429421in}}{\pgfqpoint{7.168284in}{1.425854in}}%
\pgfpathcurveto{\pgfqpoint{7.171850in}{1.422288in}}{\pgfqpoint{7.176688in}{1.420284in}}{\pgfqpoint{7.181732in}{1.420284in}}%
\pgfpathclose%
\pgfusepath{fill}%
\end{pgfscope}%
\begin{pgfscope}%
\pgfpathrectangle{\pgfqpoint{6.572727in}{0.474100in}}{\pgfqpoint{4.227273in}{3.318700in}}%
\pgfusepath{clip}%
\pgfsetbuttcap%
\pgfsetroundjoin%
\definecolor{currentfill}{rgb}{0.267004,0.004874,0.329415}%
\pgfsetfillcolor{currentfill}%
\pgfsetfillopacity{0.700000}%
\pgfsetlinewidth{0.000000pt}%
\definecolor{currentstroke}{rgb}{0.000000,0.000000,0.000000}%
\pgfsetstrokecolor{currentstroke}%
\pgfsetstrokeopacity{0.700000}%
\pgfsetdash{}{0pt}%
\pgfpathmoveto{\pgfqpoint{8.454971in}{1.531085in}}%
\pgfpathcurveto{\pgfqpoint{8.460015in}{1.531085in}}{\pgfqpoint{8.464853in}{1.533089in}}{\pgfqpoint{8.468419in}{1.536656in}}%
\pgfpathcurveto{\pgfqpoint{8.471985in}{1.540222in}}{\pgfqpoint{8.473989in}{1.545060in}}{\pgfqpoint{8.473989in}{1.550103in}}%
\pgfpathcurveto{\pgfqpoint{8.473989in}{1.555147in}}{\pgfqpoint{8.471985in}{1.559985in}}{\pgfqpoint{8.468419in}{1.563551in}}%
\pgfpathcurveto{\pgfqpoint{8.464853in}{1.567118in}}{\pgfqpoint{8.460015in}{1.569122in}}{\pgfqpoint{8.454971in}{1.569122in}}%
\pgfpathcurveto{\pgfqpoint{8.449928in}{1.569122in}}{\pgfqpoint{8.445090in}{1.567118in}}{\pgfqpoint{8.441523in}{1.563551in}}%
\pgfpathcurveto{\pgfqpoint{8.437957in}{1.559985in}}{\pgfqpoint{8.435953in}{1.555147in}}{\pgfqpoint{8.435953in}{1.550103in}}%
\pgfpathcurveto{\pgfqpoint{8.435953in}{1.545060in}}{\pgfqpoint{8.437957in}{1.540222in}}{\pgfqpoint{8.441523in}{1.536656in}}%
\pgfpathcurveto{\pgfqpoint{8.445090in}{1.533089in}}{\pgfqpoint{8.449928in}{1.531085in}}{\pgfqpoint{8.454971in}{1.531085in}}%
\pgfpathclose%
\pgfusepath{fill}%
\end{pgfscope}%
\begin{pgfscope}%
\pgfpathrectangle{\pgfqpoint{6.572727in}{0.474100in}}{\pgfqpoint{4.227273in}{3.318700in}}%
\pgfusepath{clip}%
\pgfsetbuttcap%
\pgfsetroundjoin%
\definecolor{currentfill}{rgb}{0.127568,0.566949,0.550556}%
\pgfsetfillcolor{currentfill}%
\pgfsetfillopacity{0.700000}%
\pgfsetlinewidth{0.000000pt}%
\definecolor{currentstroke}{rgb}{0.000000,0.000000,0.000000}%
\pgfsetstrokecolor{currentstroke}%
\pgfsetstrokeopacity{0.700000}%
\pgfsetdash{}{0pt}%
\pgfpathmoveto{\pgfqpoint{9.700941in}{2.075225in}}%
\pgfpathcurveto{\pgfqpoint{9.705985in}{2.075225in}}{\pgfqpoint{9.710823in}{2.077229in}}{\pgfqpoint{9.714389in}{2.080796in}}%
\pgfpathcurveto{\pgfqpoint{9.717956in}{2.084362in}}{\pgfqpoint{9.719959in}{2.089200in}}{\pgfqpoint{9.719959in}{2.094243in}}%
\pgfpathcurveto{\pgfqpoint{9.719959in}{2.099287in}}{\pgfqpoint{9.717956in}{2.104125in}}{\pgfqpoint{9.714389in}{2.107691in}}%
\pgfpathcurveto{\pgfqpoint{9.710823in}{2.111258in}}{\pgfqpoint{9.705985in}{2.113262in}}{\pgfqpoint{9.700941in}{2.113262in}}%
\pgfpathcurveto{\pgfqpoint{9.695898in}{2.113262in}}{\pgfqpoint{9.691060in}{2.111258in}}{\pgfqpoint{9.687493in}{2.107691in}}%
\pgfpathcurveto{\pgfqpoint{9.683927in}{2.104125in}}{\pgfqpoint{9.681923in}{2.099287in}}{\pgfqpoint{9.681923in}{2.094243in}}%
\pgfpathcurveto{\pgfqpoint{9.681923in}{2.089200in}}{\pgfqpoint{9.683927in}{2.084362in}}{\pgfqpoint{9.687493in}{2.080796in}}%
\pgfpathcurveto{\pgfqpoint{9.691060in}{2.077229in}}{\pgfqpoint{9.695898in}{2.075225in}}{\pgfqpoint{9.700941in}{2.075225in}}%
\pgfpathclose%
\pgfusepath{fill}%
\end{pgfscope}%
\begin{pgfscope}%
\pgfpathrectangle{\pgfqpoint{6.572727in}{0.474100in}}{\pgfqpoint{4.227273in}{3.318700in}}%
\pgfusepath{clip}%
\pgfsetbuttcap%
\pgfsetroundjoin%
\definecolor{currentfill}{rgb}{0.993248,0.906157,0.143936}%
\pgfsetfillcolor{currentfill}%
\pgfsetfillopacity{0.700000}%
\pgfsetlinewidth{0.000000pt}%
\definecolor{currentstroke}{rgb}{0.000000,0.000000,0.000000}%
\pgfsetstrokecolor{currentstroke}%
\pgfsetstrokeopacity{0.700000}%
\pgfsetdash{}{0pt}%
\pgfpathmoveto{\pgfqpoint{7.978174in}{2.350066in}}%
\pgfpathcurveto{\pgfqpoint{7.983218in}{2.350066in}}{\pgfqpoint{7.988056in}{2.352070in}}{\pgfqpoint{7.991622in}{2.355637in}}%
\pgfpathcurveto{\pgfqpoint{7.995189in}{2.359203in}}{\pgfqpoint{7.997192in}{2.364041in}}{\pgfqpoint{7.997192in}{2.369084in}}%
\pgfpathcurveto{\pgfqpoint{7.997192in}{2.374128in}}{\pgfqpoint{7.995189in}{2.378966in}}{\pgfqpoint{7.991622in}{2.382532in}}%
\pgfpathcurveto{\pgfqpoint{7.988056in}{2.386099in}}{\pgfqpoint{7.983218in}{2.388103in}}{\pgfqpoint{7.978174in}{2.388103in}}%
\pgfpathcurveto{\pgfqpoint{7.973131in}{2.388103in}}{\pgfqpoint{7.968293in}{2.386099in}}{\pgfqpoint{7.964726in}{2.382532in}}%
\pgfpathcurveto{\pgfqpoint{7.961160in}{2.378966in}}{\pgfqpoint{7.959156in}{2.374128in}}{\pgfqpoint{7.959156in}{2.369084in}}%
\pgfpathcurveto{\pgfqpoint{7.959156in}{2.364041in}}{\pgfqpoint{7.961160in}{2.359203in}}{\pgfqpoint{7.964726in}{2.355637in}}%
\pgfpathcurveto{\pgfqpoint{7.968293in}{2.352070in}}{\pgfqpoint{7.973131in}{2.350066in}}{\pgfqpoint{7.978174in}{2.350066in}}%
\pgfpathclose%
\pgfusepath{fill}%
\end{pgfscope}%
\begin{pgfscope}%
\pgfpathrectangle{\pgfqpoint{6.572727in}{0.474100in}}{\pgfqpoint{4.227273in}{3.318700in}}%
\pgfusepath{clip}%
\pgfsetbuttcap%
\pgfsetroundjoin%
\definecolor{currentfill}{rgb}{0.993248,0.906157,0.143936}%
\pgfsetfillcolor{currentfill}%
\pgfsetfillopacity{0.700000}%
\pgfsetlinewidth{0.000000pt}%
\definecolor{currentstroke}{rgb}{0.000000,0.000000,0.000000}%
\pgfsetstrokecolor{currentstroke}%
\pgfsetstrokeopacity{0.700000}%
\pgfsetdash{}{0pt}%
\pgfpathmoveto{\pgfqpoint{9.235591in}{2.789165in}}%
\pgfpathcurveto{\pgfqpoint{9.240635in}{2.789165in}}{\pgfqpoint{9.245472in}{2.791169in}}{\pgfqpoint{9.249039in}{2.794736in}}%
\pgfpathcurveto{\pgfqpoint{9.252605in}{2.798302in}}{\pgfqpoint{9.254609in}{2.803140in}}{\pgfqpoint{9.254609in}{2.808184in}}%
\pgfpathcurveto{\pgfqpoint{9.254609in}{2.813227in}}{\pgfqpoint{9.252605in}{2.818065in}}{\pgfqpoint{9.249039in}{2.821631in}}%
\pgfpathcurveto{\pgfqpoint{9.245472in}{2.825198in}}{\pgfqpoint{9.240635in}{2.827202in}}{\pgfqpoint{9.235591in}{2.827202in}}%
\pgfpathcurveto{\pgfqpoint{9.230547in}{2.827202in}}{\pgfqpoint{9.225709in}{2.825198in}}{\pgfqpoint{9.222143in}{2.821631in}}%
\pgfpathcurveto{\pgfqpoint{9.218577in}{2.818065in}}{\pgfqpoint{9.216573in}{2.813227in}}{\pgfqpoint{9.216573in}{2.808184in}}%
\pgfpathcurveto{\pgfqpoint{9.216573in}{2.803140in}}{\pgfqpoint{9.218577in}{2.798302in}}{\pgfqpoint{9.222143in}{2.794736in}}%
\pgfpathcurveto{\pgfqpoint{9.225709in}{2.791169in}}{\pgfqpoint{9.230547in}{2.789165in}}{\pgfqpoint{9.235591in}{2.789165in}}%
\pgfpathclose%
\pgfusepath{fill}%
\end{pgfscope}%
\begin{pgfscope}%
\pgfpathrectangle{\pgfqpoint{6.572727in}{0.474100in}}{\pgfqpoint{4.227273in}{3.318700in}}%
\pgfusepath{clip}%
\pgfsetbuttcap%
\pgfsetroundjoin%
\definecolor{currentfill}{rgb}{0.267004,0.004874,0.329415}%
\pgfsetfillcolor{currentfill}%
\pgfsetfillopacity{0.700000}%
\pgfsetlinewidth{0.000000pt}%
\definecolor{currentstroke}{rgb}{0.000000,0.000000,0.000000}%
\pgfsetstrokecolor{currentstroke}%
\pgfsetstrokeopacity{0.700000}%
\pgfsetdash{}{0pt}%
\pgfpathmoveto{\pgfqpoint{8.167309in}{1.717110in}}%
\pgfpathcurveto{\pgfqpoint{8.172353in}{1.717110in}}{\pgfqpoint{8.177190in}{1.719114in}}{\pgfqpoint{8.180757in}{1.722681in}}%
\pgfpathcurveto{\pgfqpoint{8.184323in}{1.726247in}}{\pgfqpoint{8.186327in}{1.731085in}}{\pgfqpoint{8.186327in}{1.736128in}}%
\pgfpathcurveto{\pgfqpoint{8.186327in}{1.741172in}}{\pgfqpoint{8.184323in}{1.746010in}}{\pgfqpoint{8.180757in}{1.749576in}}%
\pgfpathcurveto{\pgfqpoint{8.177190in}{1.753143in}}{\pgfqpoint{8.172353in}{1.755147in}}{\pgfqpoint{8.167309in}{1.755147in}}%
\pgfpathcurveto{\pgfqpoint{8.162265in}{1.755147in}}{\pgfqpoint{8.157427in}{1.753143in}}{\pgfqpoint{8.153861in}{1.749576in}}%
\pgfpathcurveto{\pgfqpoint{8.150295in}{1.746010in}}{\pgfqpoint{8.148291in}{1.741172in}}{\pgfqpoint{8.148291in}{1.736128in}}%
\pgfpathcurveto{\pgfqpoint{8.148291in}{1.731085in}}{\pgfqpoint{8.150295in}{1.726247in}}{\pgfqpoint{8.153861in}{1.722681in}}%
\pgfpathcurveto{\pgfqpoint{8.157427in}{1.719114in}}{\pgfqpoint{8.162265in}{1.717110in}}{\pgfqpoint{8.167309in}{1.717110in}}%
\pgfpathclose%
\pgfusepath{fill}%
\end{pgfscope}%
\begin{pgfscope}%
\pgfpathrectangle{\pgfqpoint{6.572727in}{0.474100in}}{\pgfqpoint{4.227273in}{3.318700in}}%
\pgfusepath{clip}%
\pgfsetbuttcap%
\pgfsetroundjoin%
\definecolor{currentfill}{rgb}{0.127568,0.566949,0.550556}%
\pgfsetfillcolor{currentfill}%
\pgfsetfillopacity{0.700000}%
\pgfsetlinewidth{0.000000pt}%
\definecolor{currentstroke}{rgb}{0.000000,0.000000,0.000000}%
\pgfsetstrokecolor{currentstroke}%
\pgfsetstrokeopacity{0.700000}%
\pgfsetdash{}{0pt}%
\pgfpathmoveto{\pgfqpoint{9.014533in}{1.607976in}}%
\pgfpathcurveto{\pgfqpoint{9.019576in}{1.607976in}}{\pgfqpoint{9.024414in}{1.609980in}}{\pgfqpoint{9.027981in}{1.613546in}}%
\pgfpathcurveto{\pgfqpoint{9.031547in}{1.617112in}}{\pgfqpoint{9.033551in}{1.621950in}}{\pgfqpoint{9.033551in}{1.626994in}}%
\pgfpathcurveto{\pgfqpoint{9.033551in}{1.632037in}}{\pgfqpoint{9.031547in}{1.636875in}}{\pgfqpoint{9.027981in}{1.640442in}}%
\pgfpathcurveto{\pgfqpoint{9.024414in}{1.644008in}}{\pgfqpoint{9.019576in}{1.646012in}}{\pgfqpoint{9.014533in}{1.646012in}}%
\pgfpathcurveto{\pgfqpoint{9.009489in}{1.646012in}}{\pgfqpoint{9.004651in}{1.644008in}}{\pgfqpoint{9.001085in}{1.640442in}}%
\pgfpathcurveto{\pgfqpoint{8.997518in}{1.636875in}}{\pgfqpoint{8.995515in}{1.632037in}}{\pgfqpoint{8.995515in}{1.626994in}}%
\pgfpathcurveto{\pgfqpoint{8.995515in}{1.621950in}}{\pgfqpoint{8.997518in}{1.617112in}}{\pgfqpoint{9.001085in}{1.613546in}}%
\pgfpathcurveto{\pgfqpoint{9.004651in}{1.609980in}}{\pgfqpoint{9.009489in}{1.607976in}}{\pgfqpoint{9.014533in}{1.607976in}}%
\pgfpathclose%
\pgfusepath{fill}%
\end{pgfscope}%
\begin{pgfscope}%
\pgfpathrectangle{\pgfqpoint{6.572727in}{0.474100in}}{\pgfqpoint{4.227273in}{3.318700in}}%
\pgfusepath{clip}%
\pgfsetbuttcap%
\pgfsetroundjoin%
\definecolor{currentfill}{rgb}{0.267004,0.004874,0.329415}%
\pgfsetfillcolor{currentfill}%
\pgfsetfillopacity{0.700000}%
\pgfsetlinewidth{0.000000pt}%
\definecolor{currentstroke}{rgb}{0.000000,0.000000,0.000000}%
\pgfsetstrokecolor{currentstroke}%
\pgfsetstrokeopacity{0.700000}%
\pgfsetdash{}{0pt}%
\pgfpathmoveto{\pgfqpoint{7.703718in}{1.268429in}}%
\pgfpathcurveto{\pgfqpoint{7.708762in}{1.268429in}}{\pgfqpoint{7.713600in}{1.270433in}}{\pgfqpoint{7.717166in}{1.273999in}}%
\pgfpathcurveto{\pgfqpoint{7.720733in}{1.277565in}}{\pgfqpoint{7.722737in}{1.282403in}}{\pgfqpoint{7.722737in}{1.287447in}}%
\pgfpathcurveto{\pgfqpoint{7.722737in}{1.292491in}}{\pgfqpoint{7.720733in}{1.297328in}}{\pgfqpoint{7.717166in}{1.300895in}}%
\pgfpathcurveto{\pgfqpoint{7.713600in}{1.304461in}}{\pgfqpoint{7.708762in}{1.306465in}}{\pgfqpoint{7.703718in}{1.306465in}}%
\pgfpathcurveto{\pgfqpoint{7.698675in}{1.306465in}}{\pgfqpoint{7.693837in}{1.304461in}}{\pgfqpoint{7.690271in}{1.300895in}}%
\pgfpathcurveto{\pgfqpoint{7.686704in}{1.297328in}}{\pgfqpoint{7.684700in}{1.292491in}}{\pgfqpoint{7.684700in}{1.287447in}}%
\pgfpathcurveto{\pgfqpoint{7.684700in}{1.282403in}}{\pgfqpoint{7.686704in}{1.277565in}}{\pgfqpoint{7.690271in}{1.273999in}}%
\pgfpathcurveto{\pgfqpoint{7.693837in}{1.270433in}}{\pgfqpoint{7.698675in}{1.268429in}}{\pgfqpoint{7.703718in}{1.268429in}}%
\pgfpathclose%
\pgfusepath{fill}%
\end{pgfscope}%
\begin{pgfscope}%
\pgfpathrectangle{\pgfqpoint{6.572727in}{0.474100in}}{\pgfqpoint{4.227273in}{3.318700in}}%
\pgfusepath{clip}%
\pgfsetbuttcap%
\pgfsetroundjoin%
\definecolor{currentfill}{rgb}{0.993248,0.906157,0.143936}%
\pgfsetfillcolor{currentfill}%
\pgfsetfillopacity{0.700000}%
\pgfsetlinewidth{0.000000pt}%
\definecolor{currentstroke}{rgb}{0.000000,0.000000,0.000000}%
\pgfsetstrokecolor{currentstroke}%
\pgfsetstrokeopacity{0.700000}%
\pgfsetdash{}{0pt}%
\pgfpathmoveto{\pgfqpoint{8.528957in}{2.416960in}}%
\pgfpathcurveto{\pgfqpoint{8.534000in}{2.416960in}}{\pgfqpoint{8.538838in}{2.418964in}}{\pgfqpoint{8.542405in}{2.422530in}}%
\pgfpathcurveto{\pgfqpoint{8.545971in}{2.426096in}}{\pgfqpoint{8.547975in}{2.430934in}}{\pgfqpoint{8.547975in}{2.435978in}}%
\pgfpathcurveto{\pgfqpoint{8.547975in}{2.441022in}}{\pgfqpoint{8.545971in}{2.445859in}}{\pgfqpoint{8.542405in}{2.449426in}}%
\pgfpathcurveto{\pgfqpoint{8.538838in}{2.452992in}}{\pgfqpoint{8.534000in}{2.454996in}}{\pgfqpoint{8.528957in}{2.454996in}}%
\pgfpathcurveto{\pgfqpoint{8.523913in}{2.454996in}}{\pgfqpoint{8.519075in}{2.452992in}}{\pgfqpoint{8.515509in}{2.449426in}}%
\pgfpathcurveto{\pgfqpoint{8.511943in}{2.445859in}}{\pgfqpoint{8.509939in}{2.441022in}}{\pgfqpoint{8.509939in}{2.435978in}}%
\pgfpathcurveto{\pgfqpoint{8.509939in}{2.430934in}}{\pgfqpoint{8.511943in}{2.426096in}}{\pgfqpoint{8.515509in}{2.422530in}}%
\pgfpathcurveto{\pgfqpoint{8.519075in}{2.418964in}}{\pgfqpoint{8.523913in}{2.416960in}}{\pgfqpoint{8.528957in}{2.416960in}}%
\pgfpathclose%
\pgfusepath{fill}%
\end{pgfscope}%
\begin{pgfscope}%
\pgfpathrectangle{\pgfqpoint{6.572727in}{0.474100in}}{\pgfqpoint{4.227273in}{3.318700in}}%
\pgfusepath{clip}%
\pgfsetbuttcap%
\pgfsetroundjoin%
\definecolor{currentfill}{rgb}{0.127568,0.566949,0.550556}%
\pgfsetfillcolor{currentfill}%
\pgfsetfillopacity{0.700000}%
\pgfsetlinewidth{0.000000pt}%
\definecolor{currentstroke}{rgb}{0.000000,0.000000,0.000000}%
\pgfsetstrokecolor{currentstroke}%
\pgfsetstrokeopacity{0.700000}%
\pgfsetdash{}{0pt}%
\pgfpathmoveto{\pgfqpoint{10.242300in}{1.003808in}}%
\pgfpathcurveto{\pgfqpoint{10.247344in}{1.003808in}}{\pgfqpoint{10.252181in}{1.005812in}}{\pgfqpoint{10.255748in}{1.009378in}}%
\pgfpathcurveto{\pgfqpoint{10.259314in}{1.012945in}}{\pgfqpoint{10.261318in}{1.017783in}}{\pgfqpoint{10.261318in}{1.022826in}}%
\pgfpathcurveto{\pgfqpoint{10.261318in}{1.027870in}}{\pgfqpoint{10.259314in}{1.032708in}}{\pgfqpoint{10.255748in}{1.036274in}}%
\pgfpathcurveto{\pgfqpoint{10.252181in}{1.039840in}}{\pgfqpoint{10.247344in}{1.041844in}}{\pgfqpoint{10.242300in}{1.041844in}}%
\pgfpathcurveto{\pgfqpoint{10.237256in}{1.041844in}}{\pgfqpoint{10.232419in}{1.039840in}}{\pgfqpoint{10.228852in}{1.036274in}}%
\pgfpathcurveto{\pgfqpoint{10.225286in}{1.032708in}}{\pgfqpoint{10.223282in}{1.027870in}}{\pgfqpoint{10.223282in}{1.022826in}}%
\pgfpathcurveto{\pgfqpoint{10.223282in}{1.017783in}}{\pgfqpoint{10.225286in}{1.012945in}}{\pgfqpoint{10.228852in}{1.009378in}}%
\pgfpathcurveto{\pgfqpoint{10.232419in}{1.005812in}}{\pgfqpoint{10.237256in}{1.003808in}}{\pgfqpoint{10.242300in}{1.003808in}}%
\pgfpathclose%
\pgfusepath{fill}%
\end{pgfscope}%
\begin{pgfscope}%
\pgfpathrectangle{\pgfqpoint{6.572727in}{0.474100in}}{\pgfqpoint{4.227273in}{3.318700in}}%
\pgfusepath{clip}%
\pgfsetbuttcap%
\pgfsetroundjoin%
\definecolor{currentfill}{rgb}{0.267004,0.004874,0.329415}%
\pgfsetfillcolor{currentfill}%
\pgfsetfillopacity{0.700000}%
\pgfsetlinewidth{0.000000pt}%
\definecolor{currentstroke}{rgb}{0.000000,0.000000,0.000000}%
\pgfsetstrokecolor{currentstroke}%
\pgfsetstrokeopacity{0.700000}%
\pgfsetdash{}{0pt}%
\pgfpathmoveto{\pgfqpoint{8.063184in}{1.375664in}}%
\pgfpathcurveto{\pgfqpoint{8.068228in}{1.375664in}}{\pgfqpoint{8.073065in}{1.377668in}}{\pgfqpoint{8.076632in}{1.381234in}}%
\pgfpathcurveto{\pgfqpoint{8.080198in}{1.384801in}}{\pgfqpoint{8.082202in}{1.389639in}}{\pgfqpoint{8.082202in}{1.394682in}}%
\pgfpathcurveto{\pgfqpoint{8.082202in}{1.399726in}}{\pgfqpoint{8.080198in}{1.404564in}}{\pgfqpoint{8.076632in}{1.408130in}}%
\pgfpathcurveto{\pgfqpoint{8.073065in}{1.411697in}}{\pgfqpoint{8.068228in}{1.413700in}}{\pgfqpoint{8.063184in}{1.413700in}}%
\pgfpathcurveto{\pgfqpoint{8.058140in}{1.413700in}}{\pgfqpoint{8.053303in}{1.411697in}}{\pgfqpoint{8.049736in}{1.408130in}}%
\pgfpathcurveto{\pgfqpoint{8.046170in}{1.404564in}}{\pgfqpoint{8.044166in}{1.399726in}}{\pgfqpoint{8.044166in}{1.394682in}}%
\pgfpathcurveto{\pgfqpoint{8.044166in}{1.389639in}}{\pgfqpoint{8.046170in}{1.384801in}}{\pgfqpoint{8.049736in}{1.381234in}}%
\pgfpathcurveto{\pgfqpoint{8.053303in}{1.377668in}}{\pgfqpoint{8.058140in}{1.375664in}}{\pgfqpoint{8.063184in}{1.375664in}}%
\pgfpathclose%
\pgfusepath{fill}%
\end{pgfscope}%
\begin{pgfscope}%
\pgfpathrectangle{\pgfqpoint{6.572727in}{0.474100in}}{\pgfqpoint{4.227273in}{3.318700in}}%
\pgfusepath{clip}%
\pgfsetbuttcap%
\pgfsetroundjoin%
\definecolor{currentfill}{rgb}{0.127568,0.566949,0.550556}%
\pgfsetfillcolor{currentfill}%
\pgfsetfillopacity{0.700000}%
\pgfsetlinewidth{0.000000pt}%
\definecolor{currentstroke}{rgb}{0.000000,0.000000,0.000000}%
\pgfsetstrokecolor{currentstroke}%
\pgfsetstrokeopacity{0.700000}%
\pgfsetdash{}{0pt}%
\pgfpathmoveto{\pgfqpoint{9.215122in}{1.372852in}}%
\pgfpathcurveto{\pgfqpoint{9.220166in}{1.372852in}}{\pgfqpoint{9.225004in}{1.374856in}}{\pgfqpoint{9.228570in}{1.378423in}}%
\pgfpathcurveto{\pgfqpoint{9.232137in}{1.381989in}}{\pgfqpoint{9.234140in}{1.386827in}}{\pgfqpoint{9.234140in}{1.391870in}}%
\pgfpathcurveto{\pgfqpoint{9.234140in}{1.396914in}}{\pgfqpoint{9.232137in}{1.401752in}}{\pgfqpoint{9.228570in}{1.405318in}}%
\pgfpathcurveto{\pgfqpoint{9.225004in}{1.408885in}}{\pgfqpoint{9.220166in}{1.410889in}}{\pgfqpoint{9.215122in}{1.410889in}}%
\pgfpathcurveto{\pgfqpoint{9.210079in}{1.410889in}}{\pgfqpoint{9.205241in}{1.408885in}}{\pgfqpoint{9.201674in}{1.405318in}}%
\pgfpathcurveto{\pgfqpoint{9.198108in}{1.401752in}}{\pgfqpoint{9.196104in}{1.396914in}}{\pgfqpoint{9.196104in}{1.391870in}}%
\pgfpathcurveto{\pgfqpoint{9.196104in}{1.386827in}}{\pgfqpoint{9.198108in}{1.381989in}}{\pgfqpoint{9.201674in}{1.378423in}}%
\pgfpathcurveto{\pgfqpoint{9.205241in}{1.374856in}}{\pgfqpoint{9.210079in}{1.372852in}}{\pgfqpoint{9.215122in}{1.372852in}}%
\pgfpathclose%
\pgfusepath{fill}%
\end{pgfscope}%
\begin{pgfscope}%
\pgfpathrectangle{\pgfqpoint{6.572727in}{0.474100in}}{\pgfqpoint{4.227273in}{3.318700in}}%
\pgfusepath{clip}%
\pgfsetbuttcap%
\pgfsetroundjoin%
\definecolor{currentfill}{rgb}{0.127568,0.566949,0.550556}%
\pgfsetfillcolor{currentfill}%
\pgfsetfillopacity{0.700000}%
\pgfsetlinewidth{0.000000pt}%
\definecolor{currentstroke}{rgb}{0.000000,0.000000,0.000000}%
\pgfsetstrokecolor{currentstroke}%
\pgfsetstrokeopacity{0.700000}%
\pgfsetdash{}{0pt}%
\pgfpathmoveto{\pgfqpoint{10.168488in}{0.844242in}}%
\pgfpathcurveto{\pgfqpoint{10.173531in}{0.844242in}}{\pgfqpoint{10.178369in}{0.846245in}}{\pgfqpoint{10.181935in}{0.849812in}}%
\pgfpathcurveto{\pgfqpoint{10.185502in}{0.853378in}}{\pgfqpoint{10.187506in}{0.858216in}}{\pgfqpoint{10.187506in}{0.863260in}}%
\pgfpathcurveto{\pgfqpoint{10.187506in}{0.868303in}}{\pgfqpoint{10.185502in}{0.873141in}}{\pgfqpoint{10.181935in}{0.876708in}}%
\pgfpathcurveto{\pgfqpoint{10.178369in}{0.880274in}}{\pgfqpoint{10.173531in}{0.882278in}}{\pgfqpoint{10.168488in}{0.882278in}}%
\pgfpathcurveto{\pgfqpoint{10.163444in}{0.882278in}}{\pgfqpoint{10.158606in}{0.880274in}}{\pgfqpoint{10.155040in}{0.876708in}}%
\pgfpathcurveto{\pgfqpoint{10.151473in}{0.873141in}}{\pgfqpoint{10.149469in}{0.868303in}}{\pgfqpoint{10.149469in}{0.863260in}}%
\pgfpathcurveto{\pgfqpoint{10.149469in}{0.858216in}}{\pgfqpoint{10.151473in}{0.853378in}}{\pgfqpoint{10.155040in}{0.849812in}}%
\pgfpathcurveto{\pgfqpoint{10.158606in}{0.846245in}}{\pgfqpoint{10.163444in}{0.844242in}}{\pgfqpoint{10.168488in}{0.844242in}}%
\pgfpathclose%
\pgfusepath{fill}%
\end{pgfscope}%
\begin{pgfscope}%
\pgfpathrectangle{\pgfqpoint{6.572727in}{0.474100in}}{\pgfqpoint{4.227273in}{3.318700in}}%
\pgfusepath{clip}%
\pgfsetbuttcap%
\pgfsetroundjoin%
\definecolor{currentfill}{rgb}{0.993248,0.906157,0.143936}%
\pgfsetfillcolor{currentfill}%
\pgfsetfillopacity{0.700000}%
\pgfsetlinewidth{0.000000pt}%
\definecolor{currentstroke}{rgb}{0.000000,0.000000,0.000000}%
\pgfsetstrokecolor{currentstroke}%
\pgfsetstrokeopacity{0.700000}%
\pgfsetdash{}{0pt}%
\pgfpathmoveto{\pgfqpoint{8.381077in}{2.861593in}}%
\pgfpathcurveto{\pgfqpoint{8.386120in}{2.861593in}}{\pgfqpoint{8.390958in}{2.863597in}}{\pgfqpoint{8.394524in}{2.867163in}}%
\pgfpathcurveto{\pgfqpoint{8.398091in}{2.870730in}}{\pgfqpoint{8.400095in}{2.875567in}}{\pgfqpoint{8.400095in}{2.880611in}}%
\pgfpathcurveto{\pgfqpoint{8.400095in}{2.885655in}}{\pgfqpoint{8.398091in}{2.890493in}}{\pgfqpoint{8.394524in}{2.894059in}}%
\pgfpathcurveto{\pgfqpoint{8.390958in}{2.897625in}}{\pgfqpoint{8.386120in}{2.899629in}}{\pgfqpoint{8.381077in}{2.899629in}}%
\pgfpathcurveto{\pgfqpoint{8.376033in}{2.899629in}}{\pgfqpoint{8.371195in}{2.897625in}}{\pgfqpoint{8.367629in}{2.894059in}}%
\pgfpathcurveto{\pgfqpoint{8.364062in}{2.890493in}}{\pgfqpoint{8.362058in}{2.885655in}}{\pgfqpoint{8.362058in}{2.880611in}}%
\pgfpathcurveto{\pgfqpoint{8.362058in}{2.875567in}}{\pgfqpoint{8.364062in}{2.870730in}}{\pgfqpoint{8.367629in}{2.867163in}}%
\pgfpathcurveto{\pgfqpoint{8.371195in}{2.863597in}}{\pgfqpoint{8.376033in}{2.861593in}}{\pgfqpoint{8.381077in}{2.861593in}}%
\pgfpathclose%
\pgfusepath{fill}%
\end{pgfscope}%
\begin{pgfscope}%
\pgfpathrectangle{\pgfqpoint{6.572727in}{0.474100in}}{\pgfqpoint{4.227273in}{3.318700in}}%
\pgfusepath{clip}%
\pgfsetbuttcap%
\pgfsetroundjoin%
\definecolor{currentfill}{rgb}{0.993248,0.906157,0.143936}%
\pgfsetfillcolor{currentfill}%
\pgfsetfillopacity{0.700000}%
\pgfsetlinewidth{0.000000pt}%
\definecolor{currentstroke}{rgb}{0.000000,0.000000,0.000000}%
\pgfsetstrokecolor{currentstroke}%
\pgfsetstrokeopacity{0.700000}%
\pgfsetdash{}{0pt}%
\pgfpathmoveto{\pgfqpoint{8.544121in}{2.833715in}}%
\pgfpathcurveto{\pgfqpoint{8.549165in}{2.833715in}}{\pgfqpoint{8.554003in}{2.835718in}}{\pgfqpoint{8.557569in}{2.839285in}}%
\pgfpathcurveto{\pgfqpoint{8.561135in}{2.842851in}}{\pgfqpoint{8.563139in}{2.847689in}}{\pgfqpoint{8.563139in}{2.852733in}}%
\pgfpathcurveto{\pgfqpoint{8.563139in}{2.857776in}}{\pgfqpoint{8.561135in}{2.862614in}}{\pgfqpoint{8.557569in}{2.866181in}}%
\pgfpathcurveto{\pgfqpoint{8.554003in}{2.869747in}}{\pgfqpoint{8.549165in}{2.871751in}}{\pgfqpoint{8.544121in}{2.871751in}}%
\pgfpathcurveto{\pgfqpoint{8.539077in}{2.871751in}}{\pgfqpoint{8.534240in}{2.869747in}}{\pgfqpoint{8.530673in}{2.866181in}}%
\pgfpathcurveto{\pgfqpoint{8.527107in}{2.862614in}}{\pgfqpoint{8.525103in}{2.857776in}}{\pgfqpoint{8.525103in}{2.852733in}}%
\pgfpathcurveto{\pgfqpoint{8.525103in}{2.847689in}}{\pgfqpoint{8.527107in}{2.842851in}}{\pgfqpoint{8.530673in}{2.839285in}}%
\pgfpathcurveto{\pgfqpoint{8.534240in}{2.835718in}}{\pgfqpoint{8.539077in}{2.833715in}}{\pgfqpoint{8.544121in}{2.833715in}}%
\pgfpathclose%
\pgfusepath{fill}%
\end{pgfscope}%
\begin{pgfscope}%
\pgfpathrectangle{\pgfqpoint{6.572727in}{0.474100in}}{\pgfqpoint{4.227273in}{3.318700in}}%
\pgfusepath{clip}%
\pgfsetbuttcap%
\pgfsetroundjoin%
\definecolor{currentfill}{rgb}{0.993248,0.906157,0.143936}%
\pgfsetfillcolor{currentfill}%
\pgfsetfillopacity{0.700000}%
\pgfsetlinewidth{0.000000pt}%
\definecolor{currentstroke}{rgb}{0.000000,0.000000,0.000000}%
\pgfsetstrokecolor{currentstroke}%
\pgfsetstrokeopacity{0.700000}%
\pgfsetdash{}{0pt}%
\pgfpathmoveto{\pgfqpoint{8.456689in}{2.744105in}}%
\pgfpathcurveto{\pgfqpoint{8.461733in}{2.744105in}}{\pgfqpoint{8.466571in}{2.746109in}}{\pgfqpoint{8.470137in}{2.749676in}}%
\pgfpathcurveto{\pgfqpoint{8.473704in}{2.753242in}}{\pgfqpoint{8.475708in}{2.758080in}}{\pgfqpoint{8.475708in}{2.763123in}}%
\pgfpathcurveto{\pgfqpoint{8.475708in}{2.768167in}}{\pgfqpoint{8.473704in}{2.773005in}}{\pgfqpoint{8.470137in}{2.776571in}}%
\pgfpathcurveto{\pgfqpoint{8.466571in}{2.780138in}}{\pgfqpoint{8.461733in}{2.782142in}}{\pgfqpoint{8.456689in}{2.782142in}}%
\pgfpathcurveto{\pgfqpoint{8.451646in}{2.782142in}}{\pgfqpoint{8.446808in}{2.780138in}}{\pgfqpoint{8.443242in}{2.776571in}}%
\pgfpathcurveto{\pgfqpoint{8.439675in}{2.773005in}}{\pgfqpoint{8.437671in}{2.768167in}}{\pgfqpoint{8.437671in}{2.763123in}}%
\pgfpathcurveto{\pgfqpoint{8.437671in}{2.758080in}}{\pgfqpoint{8.439675in}{2.753242in}}{\pgfqpoint{8.443242in}{2.749676in}}%
\pgfpathcurveto{\pgfqpoint{8.446808in}{2.746109in}}{\pgfqpoint{8.451646in}{2.744105in}}{\pgfqpoint{8.456689in}{2.744105in}}%
\pgfpathclose%
\pgfusepath{fill}%
\end{pgfscope}%
\begin{pgfscope}%
\pgfpathrectangle{\pgfqpoint{6.572727in}{0.474100in}}{\pgfqpoint{4.227273in}{3.318700in}}%
\pgfusepath{clip}%
\pgfsetbuttcap%
\pgfsetroundjoin%
\definecolor{currentfill}{rgb}{0.993248,0.906157,0.143936}%
\pgfsetfillcolor{currentfill}%
\pgfsetfillopacity{0.700000}%
\pgfsetlinewidth{0.000000pt}%
\definecolor{currentstroke}{rgb}{0.000000,0.000000,0.000000}%
\pgfsetstrokecolor{currentstroke}%
\pgfsetstrokeopacity{0.700000}%
\pgfsetdash{}{0pt}%
\pgfpathmoveto{\pgfqpoint{8.560071in}{3.196605in}}%
\pgfpathcurveto{\pgfqpoint{8.565115in}{3.196605in}}{\pgfqpoint{8.569952in}{3.198609in}}{\pgfqpoint{8.573519in}{3.202175in}}%
\pgfpathcurveto{\pgfqpoint{8.577085in}{3.205741in}}{\pgfqpoint{8.579089in}{3.210579in}}{\pgfqpoint{8.579089in}{3.215623in}}%
\pgfpathcurveto{\pgfqpoint{8.579089in}{3.220667in}}{\pgfqpoint{8.577085in}{3.225504in}}{\pgfqpoint{8.573519in}{3.229071in}}%
\pgfpathcurveto{\pgfqpoint{8.569952in}{3.232637in}}{\pgfqpoint{8.565115in}{3.234641in}}{\pgfqpoint{8.560071in}{3.234641in}}%
\pgfpathcurveto{\pgfqpoint{8.555027in}{3.234641in}}{\pgfqpoint{8.550189in}{3.232637in}}{\pgfqpoint{8.546623in}{3.229071in}}%
\pgfpathcurveto{\pgfqpoint{8.543057in}{3.225504in}}{\pgfqpoint{8.541053in}{3.220667in}}{\pgfqpoint{8.541053in}{3.215623in}}%
\pgfpathcurveto{\pgfqpoint{8.541053in}{3.210579in}}{\pgfqpoint{8.543057in}{3.205741in}}{\pgfqpoint{8.546623in}{3.202175in}}%
\pgfpathcurveto{\pgfqpoint{8.550189in}{3.198609in}}{\pgfqpoint{8.555027in}{3.196605in}}{\pgfqpoint{8.560071in}{3.196605in}}%
\pgfpathclose%
\pgfusepath{fill}%
\end{pgfscope}%
\begin{pgfscope}%
\pgfpathrectangle{\pgfqpoint{6.572727in}{0.474100in}}{\pgfqpoint{4.227273in}{3.318700in}}%
\pgfusepath{clip}%
\pgfsetbuttcap%
\pgfsetroundjoin%
\definecolor{currentfill}{rgb}{0.267004,0.004874,0.329415}%
\pgfsetfillcolor{currentfill}%
\pgfsetfillopacity{0.700000}%
\pgfsetlinewidth{0.000000pt}%
\definecolor{currentstroke}{rgb}{0.000000,0.000000,0.000000}%
\pgfsetstrokecolor{currentstroke}%
\pgfsetstrokeopacity{0.700000}%
\pgfsetdash{}{0pt}%
\pgfpathmoveto{\pgfqpoint{7.861425in}{1.734530in}}%
\pgfpathcurveto{\pgfqpoint{7.866468in}{1.734530in}}{\pgfqpoint{7.871306in}{1.736534in}}{\pgfqpoint{7.874873in}{1.740100in}}%
\pgfpathcurveto{\pgfqpoint{7.878439in}{1.743667in}}{\pgfqpoint{7.880443in}{1.748504in}}{\pgfqpoint{7.880443in}{1.753548in}}%
\pgfpathcurveto{\pgfqpoint{7.880443in}{1.758592in}}{\pgfqpoint{7.878439in}{1.763430in}}{\pgfqpoint{7.874873in}{1.766996in}}%
\pgfpathcurveto{\pgfqpoint{7.871306in}{1.770562in}}{\pgfqpoint{7.866468in}{1.772566in}}{\pgfqpoint{7.861425in}{1.772566in}}%
\pgfpathcurveto{\pgfqpoint{7.856381in}{1.772566in}}{\pgfqpoint{7.851543in}{1.770562in}}{\pgfqpoint{7.847977in}{1.766996in}}%
\pgfpathcurveto{\pgfqpoint{7.844410in}{1.763430in}}{\pgfqpoint{7.842407in}{1.758592in}}{\pgfqpoint{7.842407in}{1.753548in}}%
\pgfpathcurveto{\pgfqpoint{7.842407in}{1.748504in}}{\pgfqpoint{7.844410in}{1.743667in}}{\pgfqpoint{7.847977in}{1.740100in}}%
\pgfpathcurveto{\pgfqpoint{7.851543in}{1.736534in}}{\pgfqpoint{7.856381in}{1.734530in}}{\pgfqpoint{7.861425in}{1.734530in}}%
\pgfpathclose%
\pgfusepath{fill}%
\end{pgfscope}%
\begin{pgfscope}%
\pgfpathrectangle{\pgfqpoint{6.572727in}{0.474100in}}{\pgfqpoint{4.227273in}{3.318700in}}%
\pgfusepath{clip}%
\pgfsetbuttcap%
\pgfsetroundjoin%
\definecolor{currentfill}{rgb}{0.127568,0.566949,0.550556}%
\pgfsetfillcolor{currentfill}%
\pgfsetfillopacity{0.700000}%
\pgfsetlinewidth{0.000000pt}%
\definecolor{currentstroke}{rgb}{0.000000,0.000000,0.000000}%
\pgfsetstrokecolor{currentstroke}%
\pgfsetstrokeopacity{0.700000}%
\pgfsetdash{}{0pt}%
\pgfpathmoveto{\pgfqpoint{8.986997in}{1.753879in}}%
\pgfpathcurveto{\pgfqpoint{8.992041in}{1.753879in}}{\pgfqpoint{8.996879in}{1.755883in}}{\pgfqpoint{9.000445in}{1.759449in}}%
\pgfpathcurveto{\pgfqpoint{9.004012in}{1.763015in}}{\pgfqpoint{9.006016in}{1.767853in}}{\pgfqpoint{9.006016in}{1.772897in}}%
\pgfpathcurveto{\pgfqpoint{9.006016in}{1.777940in}}{\pgfqpoint{9.004012in}{1.782778in}}{\pgfqpoint{9.000445in}{1.786345in}}%
\pgfpathcurveto{\pgfqpoint{8.996879in}{1.789911in}}{\pgfqpoint{8.992041in}{1.791915in}}{\pgfqpoint{8.986997in}{1.791915in}}%
\pgfpathcurveto{\pgfqpoint{8.981954in}{1.791915in}}{\pgfqpoint{8.977116in}{1.789911in}}{\pgfqpoint{8.973550in}{1.786345in}}%
\pgfpathcurveto{\pgfqpoint{8.969983in}{1.782778in}}{\pgfqpoint{8.967979in}{1.777940in}}{\pgfqpoint{8.967979in}{1.772897in}}%
\pgfpathcurveto{\pgfqpoint{8.967979in}{1.767853in}}{\pgfqpoint{8.969983in}{1.763015in}}{\pgfqpoint{8.973550in}{1.759449in}}%
\pgfpathcurveto{\pgfqpoint{8.977116in}{1.755883in}}{\pgfqpoint{8.981954in}{1.753879in}}{\pgfqpoint{8.986997in}{1.753879in}}%
\pgfpathclose%
\pgfusepath{fill}%
\end{pgfscope}%
\begin{pgfscope}%
\pgfpathrectangle{\pgfqpoint{6.572727in}{0.474100in}}{\pgfqpoint{4.227273in}{3.318700in}}%
\pgfusepath{clip}%
\pgfsetbuttcap%
\pgfsetroundjoin%
\definecolor{currentfill}{rgb}{0.993248,0.906157,0.143936}%
\pgfsetfillcolor{currentfill}%
\pgfsetfillopacity{0.700000}%
\pgfsetlinewidth{0.000000pt}%
\definecolor{currentstroke}{rgb}{0.000000,0.000000,0.000000}%
\pgfsetstrokecolor{currentstroke}%
\pgfsetstrokeopacity{0.700000}%
\pgfsetdash{}{0pt}%
\pgfpathmoveto{\pgfqpoint{7.945923in}{2.322213in}}%
\pgfpathcurveto{\pgfqpoint{7.950966in}{2.322213in}}{\pgfqpoint{7.955804in}{2.324217in}}{\pgfqpoint{7.959370in}{2.327784in}}%
\pgfpathcurveto{\pgfqpoint{7.962937in}{2.331350in}}{\pgfqpoint{7.964941in}{2.336188in}}{\pgfqpoint{7.964941in}{2.341232in}}%
\pgfpathcurveto{\pgfqpoint{7.964941in}{2.346275in}}{\pgfqpoint{7.962937in}{2.351113in}}{\pgfqpoint{7.959370in}{2.354679in}}%
\pgfpathcurveto{\pgfqpoint{7.955804in}{2.358246in}}{\pgfqpoint{7.950966in}{2.360250in}}{\pgfqpoint{7.945923in}{2.360250in}}%
\pgfpathcurveto{\pgfqpoint{7.940879in}{2.360250in}}{\pgfqpoint{7.936041in}{2.358246in}}{\pgfqpoint{7.932475in}{2.354679in}}%
\pgfpathcurveto{\pgfqpoint{7.928908in}{2.351113in}}{\pgfqpoint{7.926904in}{2.346275in}}{\pgfqpoint{7.926904in}{2.341232in}}%
\pgfpathcurveto{\pgfqpoint{7.926904in}{2.336188in}}{\pgfqpoint{7.928908in}{2.331350in}}{\pgfqpoint{7.932475in}{2.327784in}}%
\pgfpathcurveto{\pgfqpoint{7.936041in}{2.324217in}}{\pgfqpoint{7.940879in}{2.322213in}}{\pgfqpoint{7.945923in}{2.322213in}}%
\pgfpathclose%
\pgfusepath{fill}%
\end{pgfscope}%
\begin{pgfscope}%
\pgfpathrectangle{\pgfqpoint{6.572727in}{0.474100in}}{\pgfqpoint{4.227273in}{3.318700in}}%
\pgfusepath{clip}%
\pgfsetbuttcap%
\pgfsetroundjoin%
\definecolor{currentfill}{rgb}{0.993248,0.906157,0.143936}%
\pgfsetfillcolor{currentfill}%
\pgfsetfillopacity{0.700000}%
\pgfsetlinewidth{0.000000pt}%
\definecolor{currentstroke}{rgb}{0.000000,0.000000,0.000000}%
\pgfsetstrokecolor{currentstroke}%
\pgfsetstrokeopacity{0.700000}%
\pgfsetdash{}{0pt}%
\pgfpathmoveto{\pgfqpoint{7.784445in}{3.074674in}}%
\pgfpathcurveto{\pgfqpoint{7.789489in}{3.074674in}}{\pgfqpoint{7.794326in}{3.076677in}}{\pgfqpoint{7.797893in}{3.080244in}}%
\pgfpathcurveto{\pgfqpoint{7.801459in}{3.083810in}}{\pgfqpoint{7.803463in}{3.088648in}}{\pgfqpoint{7.803463in}{3.093692in}}%
\pgfpathcurveto{\pgfqpoint{7.803463in}{3.098735in}}{\pgfqpoint{7.801459in}{3.103573in}}{\pgfqpoint{7.797893in}{3.107140in}}%
\pgfpathcurveto{\pgfqpoint{7.794326in}{3.110706in}}{\pgfqpoint{7.789489in}{3.112710in}}{\pgfqpoint{7.784445in}{3.112710in}}%
\pgfpathcurveto{\pgfqpoint{7.779401in}{3.112710in}}{\pgfqpoint{7.774563in}{3.110706in}}{\pgfqpoint{7.770997in}{3.107140in}}%
\pgfpathcurveto{\pgfqpoint{7.767431in}{3.103573in}}{\pgfqpoint{7.765427in}{3.098735in}}{\pgfqpoint{7.765427in}{3.093692in}}%
\pgfpathcurveto{\pgfqpoint{7.765427in}{3.088648in}}{\pgfqpoint{7.767431in}{3.083810in}}{\pgfqpoint{7.770997in}{3.080244in}}%
\pgfpathcurveto{\pgfqpoint{7.774563in}{3.076677in}}{\pgfqpoint{7.779401in}{3.074674in}}{\pgfqpoint{7.784445in}{3.074674in}}%
\pgfpathclose%
\pgfusepath{fill}%
\end{pgfscope}%
\begin{pgfscope}%
\pgfpathrectangle{\pgfqpoint{6.572727in}{0.474100in}}{\pgfqpoint{4.227273in}{3.318700in}}%
\pgfusepath{clip}%
\pgfsetbuttcap%
\pgfsetroundjoin%
\definecolor{currentfill}{rgb}{0.267004,0.004874,0.329415}%
\pgfsetfillcolor{currentfill}%
\pgfsetfillopacity{0.700000}%
\pgfsetlinewidth{0.000000pt}%
\definecolor{currentstroke}{rgb}{0.000000,0.000000,0.000000}%
\pgfsetstrokecolor{currentstroke}%
\pgfsetstrokeopacity{0.700000}%
\pgfsetdash{}{0pt}%
\pgfpathmoveto{\pgfqpoint{7.712566in}{1.588039in}}%
\pgfpathcurveto{\pgfqpoint{7.717610in}{1.588039in}}{\pgfqpoint{7.722448in}{1.590043in}}{\pgfqpoint{7.726014in}{1.593610in}}%
\pgfpathcurveto{\pgfqpoint{7.729581in}{1.597176in}}{\pgfqpoint{7.731585in}{1.602014in}}{\pgfqpoint{7.731585in}{1.607057in}}%
\pgfpathcurveto{\pgfqpoint{7.731585in}{1.612101in}}{\pgfqpoint{7.729581in}{1.616939in}}{\pgfqpoint{7.726014in}{1.620505in}}%
\pgfpathcurveto{\pgfqpoint{7.722448in}{1.624072in}}{\pgfqpoint{7.717610in}{1.626076in}}{\pgfqpoint{7.712566in}{1.626076in}}%
\pgfpathcurveto{\pgfqpoint{7.707523in}{1.626076in}}{\pgfqpoint{7.702685in}{1.624072in}}{\pgfqpoint{7.699119in}{1.620505in}}%
\pgfpathcurveto{\pgfqpoint{7.695552in}{1.616939in}}{\pgfqpoint{7.693548in}{1.612101in}}{\pgfqpoint{7.693548in}{1.607057in}}%
\pgfpathcurveto{\pgfqpoint{7.693548in}{1.602014in}}{\pgfqpoint{7.695552in}{1.597176in}}{\pgfqpoint{7.699119in}{1.593610in}}%
\pgfpathcurveto{\pgfqpoint{7.702685in}{1.590043in}}{\pgfqpoint{7.707523in}{1.588039in}}{\pgfqpoint{7.712566in}{1.588039in}}%
\pgfpathclose%
\pgfusepath{fill}%
\end{pgfscope}%
\begin{pgfscope}%
\pgfpathrectangle{\pgfqpoint{6.572727in}{0.474100in}}{\pgfqpoint{4.227273in}{3.318700in}}%
\pgfusepath{clip}%
\pgfsetbuttcap%
\pgfsetroundjoin%
\definecolor{currentfill}{rgb}{0.267004,0.004874,0.329415}%
\pgfsetfillcolor{currentfill}%
\pgfsetfillopacity{0.700000}%
\pgfsetlinewidth{0.000000pt}%
\definecolor{currentstroke}{rgb}{0.000000,0.000000,0.000000}%
\pgfsetstrokecolor{currentstroke}%
\pgfsetstrokeopacity{0.700000}%
\pgfsetdash{}{0pt}%
\pgfpathmoveto{\pgfqpoint{8.301600in}{1.605560in}}%
\pgfpathcurveto{\pgfqpoint{8.306644in}{1.605560in}}{\pgfqpoint{8.311482in}{1.607564in}}{\pgfqpoint{8.315048in}{1.611130in}}%
\pgfpathcurveto{\pgfqpoint{8.318615in}{1.614697in}}{\pgfqpoint{8.320619in}{1.619534in}}{\pgfqpoint{8.320619in}{1.624578in}}%
\pgfpathcurveto{\pgfqpoint{8.320619in}{1.629622in}}{\pgfqpoint{8.318615in}{1.634460in}}{\pgfqpoint{8.315048in}{1.638026in}}%
\pgfpathcurveto{\pgfqpoint{8.311482in}{1.641592in}}{\pgfqpoint{8.306644in}{1.643596in}}{\pgfqpoint{8.301600in}{1.643596in}}%
\pgfpathcurveto{\pgfqpoint{8.296557in}{1.643596in}}{\pgfqpoint{8.291719in}{1.641592in}}{\pgfqpoint{8.288153in}{1.638026in}}%
\pgfpathcurveto{\pgfqpoint{8.284586in}{1.634460in}}{\pgfqpoint{8.282582in}{1.629622in}}{\pgfqpoint{8.282582in}{1.624578in}}%
\pgfpathcurveto{\pgfqpoint{8.282582in}{1.619534in}}{\pgfqpoint{8.284586in}{1.614697in}}{\pgfqpoint{8.288153in}{1.611130in}}%
\pgfpathcurveto{\pgfqpoint{8.291719in}{1.607564in}}{\pgfqpoint{8.296557in}{1.605560in}}{\pgfqpoint{8.301600in}{1.605560in}}%
\pgfpathclose%
\pgfusepath{fill}%
\end{pgfscope}%
\begin{pgfscope}%
\pgfpathrectangle{\pgfqpoint{6.572727in}{0.474100in}}{\pgfqpoint{4.227273in}{3.318700in}}%
\pgfusepath{clip}%
\pgfsetbuttcap%
\pgfsetroundjoin%
\definecolor{currentfill}{rgb}{0.993248,0.906157,0.143936}%
\pgfsetfillcolor{currentfill}%
\pgfsetfillopacity{0.700000}%
\pgfsetlinewidth{0.000000pt}%
\definecolor{currentstroke}{rgb}{0.000000,0.000000,0.000000}%
\pgfsetstrokecolor{currentstroke}%
\pgfsetstrokeopacity{0.700000}%
\pgfsetdash{}{0pt}%
\pgfpathmoveto{\pgfqpoint{8.559669in}{2.595644in}}%
\pgfpathcurveto{\pgfqpoint{8.564713in}{2.595644in}}{\pgfqpoint{8.569551in}{2.597648in}}{\pgfqpoint{8.573117in}{2.601214in}}%
\pgfpathcurveto{\pgfqpoint{8.576683in}{2.604781in}}{\pgfqpoint{8.578687in}{2.609619in}}{\pgfqpoint{8.578687in}{2.614662in}}%
\pgfpathcurveto{\pgfqpoint{8.578687in}{2.619706in}}{\pgfqpoint{8.576683in}{2.624544in}}{\pgfqpoint{8.573117in}{2.628110in}}%
\pgfpathcurveto{\pgfqpoint{8.569551in}{2.631676in}}{\pgfqpoint{8.564713in}{2.633680in}}{\pgfqpoint{8.559669in}{2.633680in}}%
\pgfpathcurveto{\pgfqpoint{8.554625in}{2.633680in}}{\pgfqpoint{8.549788in}{2.631676in}}{\pgfqpoint{8.546221in}{2.628110in}}%
\pgfpathcurveto{\pgfqpoint{8.542655in}{2.624544in}}{\pgfqpoint{8.540651in}{2.619706in}}{\pgfqpoint{8.540651in}{2.614662in}}%
\pgfpathcurveto{\pgfqpoint{8.540651in}{2.609619in}}{\pgfqpoint{8.542655in}{2.604781in}}{\pgfqpoint{8.546221in}{2.601214in}}%
\pgfpathcurveto{\pgfqpoint{8.549788in}{2.597648in}}{\pgfqpoint{8.554625in}{2.595644in}}{\pgfqpoint{8.559669in}{2.595644in}}%
\pgfpathclose%
\pgfusepath{fill}%
\end{pgfscope}%
\begin{pgfscope}%
\pgfpathrectangle{\pgfqpoint{6.572727in}{0.474100in}}{\pgfqpoint{4.227273in}{3.318700in}}%
\pgfusepath{clip}%
\pgfsetbuttcap%
\pgfsetroundjoin%
\definecolor{currentfill}{rgb}{0.267004,0.004874,0.329415}%
\pgfsetfillcolor{currentfill}%
\pgfsetfillopacity{0.700000}%
\pgfsetlinewidth{0.000000pt}%
\definecolor{currentstroke}{rgb}{0.000000,0.000000,0.000000}%
\pgfsetstrokecolor{currentstroke}%
\pgfsetstrokeopacity{0.700000}%
\pgfsetdash{}{0pt}%
\pgfpathmoveto{\pgfqpoint{7.904852in}{1.890821in}}%
\pgfpathcurveto{\pgfqpoint{7.909896in}{1.890821in}}{\pgfqpoint{7.914733in}{1.892825in}}{\pgfqpoint{7.918300in}{1.896391in}}%
\pgfpathcurveto{\pgfqpoint{7.921866in}{1.899957in}}{\pgfqpoint{7.923870in}{1.904795in}}{\pgfqpoint{7.923870in}{1.909839in}}%
\pgfpathcurveto{\pgfqpoint{7.923870in}{1.914883in}}{\pgfqpoint{7.921866in}{1.919720in}}{\pgfqpoint{7.918300in}{1.923287in}}%
\pgfpathcurveto{\pgfqpoint{7.914733in}{1.926853in}}{\pgfqpoint{7.909896in}{1.928857in}}{\pgfqpoint{7.904852in}{1.928857in}}%
\pgfpathcurveto{\pgfqpoint{7.899808in}{1.928857in}}{\pgfqpoint{7.894970in}{1.926853in}}{\pgfqpoint{7.891404in}{1.923287in}}%
\pgfpathcurveto{\pgfqpoint{7.887838in}{1.919720in}}{\pgfqpoint{7.885834in}{1.914883in}}{\pgfqpoint{7.885834in}{1.909839in}}%
\pgfpathcurveto{\pgfqpoint{7.885834in}{1.904795in}}{\pgfqpoint{7.887838in}{1.899957in}}{\pgfqpoint{7.891404in}{1.896391in}}%
\pgfpathcurveto{\pgfqpoint{7.894970in}{1.892825in}}{\pgfqpoint{7.899808in}{1.890821in}}{\pgfqpoint{7.904852in}{1.890821in}}%
\pgfpathclose%
\pgfusepath{fill}%
\end{pgfscope}%
\begin{pgfscope}%
\pgfpathrectangle{\pgfqpoint{6.572727in}{0.474100in}}{\pgfqpoint{4.227273in}{3.318700in}}%
\pgfusepath{clip}%
\pgfsetbuttcap%
\pgfsetroundjoin%
\definecolor{currentfill}{rgb}{0.127568,0.566949,0.550556}%
\pgfsetfillcolor{currentfill}%
\pgfsetfillopacity{0.700000}%
\pgfsetlinewidth{0.000000pt}%
\definecolor{currentstroke}{rgb}{0.000000,0.000000,0.000000}%
\pgfsetstrokecolor{currentstroke}%
\pgfsetstrokeopacity{0.700000}%
\pgfsetdash{}{0pt}%
\pgfpathmoveto{\pgfqpoint{10.120906in}{1.422941in}}%
\pgfpathcurveto{\pgfqpoint{10.125950in}{1.422941in}}{\pgfqpoint{10.130788in}{1.424945in}}{\pgfqpoint{10.134354in}{1.428511in}}%
\pgfpathcurveto{\pgfqpoint{10.137921in}{1.432078in}}{\pgfqpoint{10.139925in}{1.436915in}}{\pgfqpoint{10.139925in}{1.441959in}}%
\pgfpathcurveto{\pgfqpoint{10.139925in}{1.447003in}}{\pgfqpoint{10.137921in}{1.451841in}}{\pgfqpoint{10.134354in}{1.455407in}}%
\pgfpathcurveto{\pgfqpoint{10.130788in}{1.458973in}}{\pgfqpoint{10.125950in}{1.460977in}}{\pgfqpoint{10.120906in}{1.460977in}}%
\pgfpathcurveto{\pgfqpoint{10.115863in}{1.460977in}}{\pgfqpoint{10.111025in}{1.458973in}}{\pgfqpoint{10.107459in}{1.455407in}}%
\pgfpathcurveto{\pgfqpoint{10.103892in}{1.451841in}}{\pgfqpoint{10.101888in}{1.447003in}}{\pgfqpoint{10.101888in}{1.441959in}}%
\pgfpathcurveto{\pgfqpoint{10.101888in}{1.436915in}}{\pgfqpoint{10.103892in}{1.432078in}}{\pgfqpoint{10.107459in}{1.428511in}}%
\pgfpathcurveto{\pgfqpoint{10.111025in}{1.424945in}}{\pgfqpoint{10.115863in}{1.422941in}}{\pgfqpoint{10.120906in}{1.422941in}}%
\pgfpathclose%
\pgfusepath{fill}%
\end{pgfscope}%
\begin{pgfscope}%
\pgfpathrectangle{\pgfqpoint{6.572727in}{0.474100in}}{\pgfqpoint{4.227273in}{3.318700in}}%
\pgfusepath{clip}%
\pgfsetbuttcap%
\pgfsetroundjoin%
\definecolor{currentfill}{rgb}{0.127568,0.566949,0.550556}%
\pgfsetfillcolor{currentfill}%
\pgfsetfillopacity{0.700000}%
\pgfsetlinewidth{0.000000pt}%
\definecolor{currentstroke}{rgb}{0.000000,0.000000,0.000000}%
\pgfsetstrokecolor{currentstroke}%
\pgfsetstrokeopacity{0.700000}%
\pgfsetdash{}{0pt}%
\pgfpathmoveto{\pgfqpoint{9.162532in}{1.359841in}}%
\pgfpathcurveto{\pgfqpoint{9.167575in}{1.359841in}}{\pgfqpoint{9.172413in}{1.361845in}}{\pgfqpoint{9.175979in}{1.365411in}}%
\pgfpathcurveto{\pgfqpoint{9.179546in}{1.368977in}}{\pgfqpoint{9.181550in}{1.373815in}}{\pgfqpoint{9.181550in}{1.378859in}}%
\pgfpathcurveto{\pgfqpoint{9.181550in}{1.383902in}}{\pgfqpoint{9.179546in}{1.388740in}}{\pgfqpoint{9.175979in}{1.392307in}}%
\pgfpathcurveto{\pgfqpoint{9.172413in}{1.395873in}}{\pgfqpoint{9.167575in}{1.397877in}}{\pgfqpoint{9.162532in}{1.397877in}}%
\pgfpathcurveto{\pgfqpoint{9.157488in}{1.397877in}}{\pgfqpoint{9.152650in}{1.395873in}}{\pgfqpoint{9.149084in}{1.392307in}}%
\pgfpathcurveto{\pgfqpoint{9.145517in}{1.388740in}}{\pgfqpoint{9.143513in}{1.383902in}}{\pgfqpoint{9.143513in}{1.378859in}}%
\pgfpathcurveto{\pgfqpoint{9.143513in}{1.373815in}}{\pgfqpoint{9.145517in}{1.368977in}}{\pgfqpoint{9.149084in}{1.365411in}}%
\pgfpathcurveto{\pgfqpoint{9.152650in}{1.361845in}}{\pgfqpoint{9.157488in}{1.359841in}}{\pgfqpoint{9.162532in}{1.359841in}}%
\pgfpathclose%
\pgfusepath{fill}%
\end{pgfscope}%
\begin{pgfscope}%
\pgfpathrectangle{\pgfqpoint{6.572727in}{0.474100in}}{\pgfqpoint{4.227273in}{3.318700in}}%
\pgfusepath{clip}%
\pgfsetbuttcap%
\pgfsetroundjoin%
\definecolor{currentfill}{rgb}{0.993248,0.906157,0.143936}%
\pgfsetfillcolor{currentfill}%
\pgfsetfillopacity{0.700000}%
\pgfsetlinewidth{0.000000pt}%
\definecolor{currentstroke}{rgb}{0.000000,0.000000,0.000000}%
\pgfsetstrokecolor{currentstroke}%
\pgfsetstrokeopacity{0.700000}%
\pgfsetdash{}{0pt}%
\pgfpathmoveto{\pgfqpoint{8.793988in}{2.119819in}}%
\pgfpathcurveto{\pgfqpoint{8.799031in}{2.119819in}}{\pgfqpoint{8.803869in}{2.121823in}}{\pgfqpoint{8.807436in}{2.125389in}}%
\pgfpathcurveto{\pgfqpoint{8.811002in}{2.128956in}}{\pgfqpoint{8.813006in}{2.133794in}}{\pgfqpoint{8.813006in}{2.138837in}}%
\pgfpathcurveto{\pgfqpoint{8.813006in}{2.143881in}}{\pgfqpoint{8.811002in}{2.148719in}}{\pgfqpoint{8.807436in}{2.152285in}}%
\pgfpathcurveto{\pgfqpoint{8.803869in}{2.155852in}}{\pgfqpoint{8.799031in}{2.157855in}}{\pgfqpoint{8.793988in}{2.157855in}}%
\pgfpathcurveto{\pgfqpoint{8.788944in}{2.157855in}}{\pgfqpoint{8.784106in}{2.155852in}}{\pgfqpoint{8.780540in}{2.152285in}}%
\pgfpathcurveto{\pgfqpoint{8.776973in}{2.148719in}}{\pgfqpoint{8.774970in}{2.143881in}}{\pgfqpoint{8.774970in}{2.138837in}}%
\pgfpathcurveto{\pgfqpoint{8.774970in}{2.133794in}}{\pgfqpoint{8.776973in}{2.128956in}}{\pgfqpoint{8.780540in}{2.125389in}}%
\pgfpathcurveto{\pgfqpoint{8.784106in}{2.121823in}}{\pgfqpoint{8.788944in}{2.119819in}}{\pgfqpoint{8.793988in}{2.119819in}}%
\pgfpathclose%
\pgfusepath{fill}%
\end{pgfscope}%
\begin{pgfscope}%
\pgfpathrectangle{\pgfqpoint{6.572727in}{0.474100in}}{\pgfqpoint{4.227273in}{3.318700in}}%
\pgfusepath{clip}%
\pgfsetbuttcap%
\pgfsetroundjoin%
\definecolor{currentfill}{rgb}{0.127568,0.566949,0.550556}%
\pgfsetfillcolor{currentfill}%
\pgfsetfillopacity{0.700000}%
\pgfsetlinewidth{0.000000pt}%
\definecolor{currentstroke}{rgb}{0.000000,0.000000,0.000000}%
\pgfsetstrokecolor{currentstroke}%
\pgfsetstrokeopacity{0.700000}%
\pgfsetdash{}{0pt}%
\pgfpathmoveto{\pgfqpoint{9.683772in}{2.002159in}}%
\pgfpathcurveto{\pgfqpoint{9.688815in}{2.002159in}}{\pgfqpoint{9.693653in}{2.004163in}}{\pgfqpoint{9.697220in}{2.007730in}}%
\pgfpathcurveto{\pgfqpoint{9.700786in}{2.011296in}}{\pgfqpoint{9.702790in}{2.016134in}}{\pgfqpoint{9.702790in}{2.021177in}}%
\pgfpathcurveto{\pgfqpoint{9.702790in}{2.026221in}}{\pgfqpoint{9.700786in}{2.031059in}}{\pgfqpoint{9.697220in}{2.034625in}}%
\pgfpathcurveto{\pgfqpoint{9.693653in}{2.038192in}}{\pgfqpoint{9.688815in}{2.040196in}}{\pgfqpoint{9.683772in}{2.040196in}}%
\pgfpathcurveto{\pgfqpoint{9.678728in}{2.040196in}}{\pgfqpoint{9.673890in}{2.038192in}}{\pgfqpoint{9.670324in}{2.034625in}}%
\pgfpathcurveto{\pgfqpoint{9.666758in}{2.031059in}}{\pgfqpoint{9.664754in}{2.026221in}}{\pgfqpoint{9.664754in}{2.021177in}}%
\pgfpathcurveto{\pgfqpoint{9.664754in}{2.016134in}}{\pgfqpoint{9.666758in}{2.011296in}}{\pgfqpoint{9.670324in}{2.007730in}}%
\pgfpathcurveto{\pgfqpoint{9.673890in}{2.004163in}}{\pgfqpoint{9.678728in}{2.002159in}}{\pgfqpoint{9.683772in}{2.002159in}}%
\pgfpathclose%
\pgfusepath{fill}%
\end{pgfscope}%
\begin{pgfscope}%
\pgfpathrectangle{\pgfqpoint{6.572727in}{0.474100in}}{\pgfqpoint{4.227273in}{3.318700in}}%
\pgfusepath{clip}%
\pgfsetbuttcap%
\pgfsetroundjoin%
\definecolor{currentfill}{rgb}{0.993248,0.906157,0.143936}%
\pgfsetfillcolor{currentfill}%
\pgfsetfillopacity{0.700000}%
\pgfsetlinewidth{0.000000pt}%
\definecolor{currentstroke}{rgb}{0.000000,0.000000,0.000000}%
\pgfsetstrokecolor{currentstroke}%
\pgfsetstrokeopacity{0.700000}%
\pgfsetdash{}{0pt}%
\pgfpathmoveto{\pgfqpoint{8.239047in}{2.867919in}}%
\pgfpathcurveto{\pgfqpoint{8.244091in}{2.867919in}}{\pgfqpoint{8.248928in}{2.869923in}}{\pgfqpoint{8.252495in}{2.873490in}}%
\pgfpathcurveto{\pgfqpoint{8.256061in}{2.877056in}}{\pgfqpoint{8.258065in}{2.881894in}}{\pgfqpoint{8.258065in}{2.886938in}}%
\pgfpathcurveto{\pgfqpoint{8.258065in}{2.891981in}}{\pgfqpoint{8.256061in}{2.896819in}}{\pgfqpoint{8.252495in}{2.900385in}}%
\pgfpathcurveto{\pgfqpoint{8.248928in}{2.903952in}}{\pgfqpoint{8.244091in}{2.905956in}}{\pgfqpoint{8.239047in}{2.905956in}}%
\pgfpathcurveto{\pgfqpoint{8.234003in}{2.905956in}}{\pgfqpoint{8.229165in}{2.903952in}}{\pgfqpoint{8.225599in}{2.900385in}}%
\pgfpathcurveto{\pgfqpoint{8.222033in}{2.896819in}}{\pgfqpoint{8.220029in}{2.891981in}}{\pgfqpoint{8.220029in}{2.886938in}}%
\pgfpathcurveto{\pgfqpoint{8.220029in}{2.881894in}}{\pgfqpoint{8.222033in}{2.877056in}}{\pgfqpoint{8.225599in}{2.873490in}}%
\pgfpathcurveto{\pgfqpoint{8.229165in}{2.869923in}}{\pgfqpoint{8.234003in}{2.867919in}}{\pgfqpoint{8.239047in}{2.867919in}}%
\pgfpathclose%
\pgfusepath{fill}%
\end{pgfscope}%
\begin{pgfscope}%
\pgfpathrectangle{\pgfqpoint{6.572727in}{0.474100in}}{\pgfqpoint{4.227273in}{3.318700in}}%
\pgfusepath{clip}%
\pgfsetbuttcap%
\pgfsetroundjoin%
\definecolor{currentfill}{rgb}{0.267004,0.004874,0.329415}%
\pgfsetfillcolor{currentfill}%
\pgfsetfillopacity{0.700000}%
\pgfsetlinewidth{0.000000pt}%
\definecolor{currentstroke}{rgb}{0.000000,0.000000,0.000000}%
\pgfsetstrokecolor{currentstroke}%
\pgfsetstrokeopacity{0.700000}%
\pgfsetdash{}{0pt}%
\pgfpathmoveto{\pgfqpoint{8.050045in}{1.186612in}}%
\pgfpathcurveto{\pgfqpoint{8.055088in}{1.186612in}}{\pgfqpoint{8.059926in}{1.188616in}}{\pgfqpoint{8.063492in}{1.192182in}}%
\pgfpathcurveto{\pgfqpoint{8.067059in}{1.195749in}}{\pgfqpoint{8.069063in}{1.200586in}}{\pgfqpoint{8.069063in}{1.205630in}}%
\pgfpathcurveto{\pgfqpoint{8.069063in}{1.210674in}}{\pgfqpoint{8.067059in}{1.215511in}}{\pgfqpoint{8.063492in}{1.219078in}}%
\pgfpathcurveto{\pgfqpoint{8.059926in}{1.222644in}}{\pgfqpoint{8.055088in}{1.224648in}}{\pgfqpoint{8.050045in}{1.224648in}}%
\pgfpathcurveto{\pgfqpoint{8.045001in}{1.224648in}}{\pgfqpoint{8.040163in}{1.222644in}}{\pgfqpoint{8.036597in}{1.219078in}}%
\pgfpathcurveto{\pgfqpoint{8.033030in}{1.215511in}}{\pgfqpoint{8.031026in}{1.210674in}}{\pgfqpoint{8.031026in}{1.205630in}}%
\pgfpathcurveto{\pgfqpoint{8.031026in}{1.200586in}}{\pgfqpoint{8.033030in}{1.195749in}}{\pgfqpoint{8.036597in}{1.192182in}}%
\pgfpathcurveto{\pgfqpoint{8.040163in}{1.188616in}}{\pgfqpoint{8.045001in}{1.186612in}}{\pgfqpoint{8.050045in}{1.186612in}}%
\pgfpathclose%
\pgfusepath{fill}%
\end{pgfscope}%
\begin{pgfscope}%
\pgfpathrectangle{\pgfqpoint{6.572727in}{0.474100in}}{\pgfqpoint{4.227273in}{3.318700in}}%
\pgfusepath{clip}%
\pgfsetbuttcap%
\pgfsetroundjoin%
\definecolor{currentfill}{rgb}{0.127568,0.566949,0.550556}%
\pgfsetfillcolor{currentfill}%
\pgfsetfillopacity{0.700000}%
\pgfsetlinewidth{0.000000pt}%
\definecolor{currentstroke}{rgb}{0.000000,0.000000,0.000000}%
\pgfsetstrokecolor{currentstroke}%
\pgfsetstrokeopacity{0.700000}%
\pgfsetdash{}{0pt}%
\pgfpathmoveto{\pgfqpoint{9.711457in}{1.934637in}}%
\pgfpathcurveto{\pgfqpoint{9.716501in}{1.934637in}}{\pgfqpoint{9.721339in}{1.936641in}}{\pgfqpoint{9.724905in}{1.940208in}}%
\pgfpathcurveto{\pgfqpoint{9.728471in}{1.943774in}}{\pgfqpoint{9.730475in}{1.948612in}}{\pgfqpoint{9.730475in}{1.953655in}}%
\pgfpathcurveto{\pgfqpoint{9.730475in}{1.958699in}}{\pgfqpoint{9.728471in}{1.963537in}}{\pgfqpoint{9.724905in}{1.967103in}}%
\pgfpathcurveto{\pgfqpoint{9.721339in}{1.970670in}}{\pgfqpoint{9.716501in}{1.972674in}}{\pgfqpoint{9.711457in}{1.972674in}}%
\pgfpathcurveto{\pgfqpoint{9.706413in}{1.972674in}}{\pgfqpoint{9.701576in}{1.970670in}}{\pgfqpoint{9.698009in}{1.967103in}}%
\pgfpathcurveto{\pgfqpoint{9.694443in}{1.963537in}}{\pgfqpoint{9.692439in}{1.958699in}}{\pgfqpoint{9.692439in}{1.953655in}}%
\pgfpathcurveto{\pgfqpoint{9.692439in}{1.948612in}}{\pgfqpoint{9.694443in}{1.943774in}}{\pgfqpoint{9.698009in}{1.940208in}}%
\pgfpathcurveto{\pgfqpoint{9.701576in}{1.936641in}}{\pgfqpoint{9.706413in}{1.934637in}}{\pgfqpoint{9.711457in}{1.934637in}}%
\pgfpathclose%
\pgfusepath{fill}%
\end{pgfscope}%
\begin{pgfscope}%
\pgfpathrectangle{\pgfqpoint{6.572727in}{0.474100in}}{\pgfqpoint{4.227273in}{3.318700in}}%
\pgfusepath{clip}%
\pgfsetbuttcap%
\pgfsetroundjoin%
\definecolor{currentfill}{rgb}{0.267004,0.004874,0.329415}%
\pgfsetfillcolor{currentfill}%
\pgfsetfillopacity{0.700000}%
\pgfsetlinewidth{0.000000pt}%
\definecolor{currentstroke}{rgb}{0.000000,0.000000,0.000000}%
\pgfsetstrokecolor{currentstroke}%
\pgfsetstrokeopacity{0.700000}%
\pgfsetdash{}{0pt}%
\pgfpathmoveto{\pgfqpoint{7.683891in}{1.469409in}}%
\pgfpathcurveto{\pgfqpoint{7.688935in}{1.469409in}}{\pgfqpoint{7.693773in}{1.471412in}}{\pgfqpoint{7.697339in}{1.474979in}}%
\pgfpathcurveto{\pgfqpoint{7.700906in}{1.478545in}}{\pgfqpoint{7.702909in}{1.483383in}}{\pgfqpoint{7.702909in}{1.488427in}}%
\pgfpathcurveto{\pgfqpoint{7.702909in}{1.493470in}}{\pgfqpoint{7.700906in}{1.498308in}}{\pgfqpoint{7.697339in}{1.501875in}}%
\pgfpathcurveto{\pgfqpoint{7.693773in}{1.505441in}}{\pgfqpoint{7.688935in}{1.507445in}}{\pgfqpoint{7.683891in}{1.507445in}}%
\pgfpathcurveto{\pgfqpoint{7.678848in}{1.507445in}}{\pgfqpoint{7.674010in}{1.505441in}}{\pgfqpoint{7.670443in}{1.501875in}}%
\pgfpathcurveto{\pgfqpoint{7.666877in}{1.498308in}}{\pgfqpoint{7.664873in}{1.493470in}}{\pgfqpoint{7.664873in}{1.488427in}}%
\pgfpathcurveto{\pgfqpoint{7.664873in}{1.483383in}}{\pgfqpoint{7.666877in}{1.478545in}}{\pgfqpoint{7.670443in}{1.474979in}}%
\pgfpathcurveto{\pgfqpoint{7.674010in}{1.471412in}}{\pgfqpoint{7.678848in}{1.469409in}}{\pgfqpoint{7.683891in}{1.469409in}}%
\pgfpathclose%
\pgfusepath{fill}%
\end{pgfscope}%
\begin{pgfscope}%
\pgfpathrectangle{\pgfqpoint{6.572727in}{0.474100in}}{\pgfqpoint{4.227273in}{3.318700in}}%
\pgfusepath{clip}%
\pgfsetbuttcap%
\pgfsetroundjoin%
\definecolor{currentfill}{rgb}{0.993248,0.906157,0.143936}%
\pgfsetfillcolor{currentfill}%
\pgfsetfillopacity{0.700000}%
\pgfsetlinewidth{0.000000pt}%
\definecolor{currentstroke}{rgb}{0.000000,0.000000,0.000000}%
\pgfsetstrokecolor{currentstroke}%
\pgfsetstrokeopacity{0.700000}%
\pgfsetdash{}{0pt}%
\pgfpathmoveto{\pgfqpoint{8.415927in}{2.298801in}}%
\pgfpathcurveto{\pgfqpoint{8.420970in}{2.298801in}}{\pgfqpoint{8.425808in}{2.300804in}}{\pgfqpoint{8.429374in}{2.304371in}}%
\pgfpathcurveto{\pgfqpoint{8.432941in}{2.307937in}}{\pgfqpoint{8.434945in}{2.312775in}}{\pgfqpoint{8.434945in}{2.317819in}}%
\pgfpathcurveto{\pgfqpoint{8.434945in}{2.322862in}}{\pgfqpoint{8.432941in}{2.327700in}}{\pgfqpoint{8.429374in}{2.331267in}}%
\pgfpathcurveto{\pgfqpoint{8.425808in}{2.334833in}}{\pgfqpoint{8.420970in}{2.336837in}}{\pgfqpoint{8.415927in}{2.336837in}}%
\pgfpathcurveto{\pgfqpoint{8.410883in}{2.336837in}}{\pgfqpoint{8.406045in}{2.334833in}}{\pgfqpoint{8.402479in}{2.331267in}}%
\pgfpathcurveto{\pgfqpoint{8.398912in}{2.327700in}}{\pgfqpoint{8.396908in}{2.322862in}}{\pgfqpoint{8.396908in}{2.317819in}}%
\pgfpathcurveto{\pgfqpoint{8.396908in}{2.312775in}}{\pgfqpoint{8.398912in}{2.307937in}}{\pgfqpoint{8.402479in}{2.304371in}}%
\pgfpathcurveto{\pgfqpoint{8.406045in}{2.300804in}}{\pgfqpoint{8.410883in}{2.298801in}}{\pgfqpoint{8.415927in}{2.298801in}}%
\pgfpathclose%
\pgfusepath{fill}%
\end{pgfscope}%
\begin{pgfscope}%
\pgfpathrectangle{\pgfqpoint{6.572727in}{0.474100in}}{\pgfqpoint{4.227273in}{3.318700in}}%
\pgfusepath{clip}%
\pgfsetbuttcap%
\pgfsetroundjoin%
\definecolor{currentfill}{rgb}{0.993248,0.906157,0.143936}%
\pgfsetfillcolor{currentfill}%
\pgfsetfillopacity{0.700000}%
\pgfsetlinewidth{0.000000pt}%
\definecolor{currentstroke}{rgb}{0.000000,0.000000,0.000000}%
\pgfsetstrokecolor{currentstroke}%
\pgfsetstrokeopacity{0.700000}%
\pgfsetdash{}{0pt}%
\pgfpathmoveto{\pgfqpoint{7.992050in}{2.406834in}}%
\pgfpathcurveto{\pgfqpoint{7.997094in}{2.406834in}}{\pgfqpoint{8.001931in}{2.408838in}}{\pgfqpoint{8.005498in}{2.412404in}}%
\pgfpathcurveto{\pgfqpoint{8.009064in}{2.415970in}}{\pgfqpoint{8.011068in}{2.420808in}}{\pgfqpoint{8.011068in}{2.425852in}}%
\pgfpathcurveto{\pgfqpoint{8.011068in}{2.430896in}}{\pgfqpoint{8.009064in}{2.435733in}}{\pgfqpoint{8.005498in}{2.439300in}}%
\pgfpathcurveto{\pgfqpoint{8.001931in}{2.442866in}}{\pgfqpoint{7.997094in}{2.444870in}}{\pgfqpoint{7.992050in}{2.444870in}}%
\pgfpathcurveto{\pgfqpoint{7.987006in}{2.444870in}}{\pgfqpoint{7.982168in}{2.442866in}}{\pgfqpoint{7.978602in}{2.439300in}}%
\pgfpathcurveto{\pgfqpoint{7.975036in}{2.435733in}}{\pgfqpoint{7.973032in}{2.430896in}}{\pgfqpoint{7.973032in}{2.425852in}}%
\pgfpathcurveto{\pgfqpoint{7.973032in}{2.420808in}}{\pgfqpoint{7.975036in}{2.415970in}}{\pgfqpoint{7.978602in}{2.412404in}}%
\pgfpathcurveto{\pgfqpoint{7.982168in}{2.408838in}}{\pgfqpoint{7.987006in}{2.406834in}}{\pgfqpoint{7.992050in}{2.406834in}}%
\pgfpathclose%
\pgfusepath{fill}%
\end{pgfscope}%
\begin{pgfscope}%
\pgfpathrectangle{\pgfqpoint{6.572727in}{0.474100in}}{\pgfqpoint{4.227273in}{3.318700in}}%
\pgfusepath{clip}%
\pgfsetbuttcap%
\pgfsetroundjoin%
\definecolor{currentfill}{rgb}{0.127568,0.566949,0.550556}%
\pgfsetfillcolor{currentfill}%
\pgfsetfillopacity{0.700000}%
\pgfsetlinewidth{0.000000pt}%
\definecolor{currentstroke}{rgb}{0.000000,0.000000,0.000000}%
\pgfsetstrokecolor{currentstroke}%
\pgfsetstrokeopacity{0.700000}%
\pgfsetdash{}{0pt}%
\pgfpathmoveto{\pgfqpoint{9.851579in}{1.344794in}}%
\pgfpathcurveto{\pgfqpoint{9.856623in}{1.344794in}}{\pgfqpoint{9.861460in}{1.346797in}}{\pgfqpoint{9.865027in}{1.350364in}}%
\pgfpathcurveto{\pgfqpoint{9.868593in}{1.353930in}}{\pgfqpoint{9.870597in}{1.358768in}}{\pgfqpoint{9.870597in}{1.363812in}}%
\pgfpathcurveto{\pgfqpoint{9.870597in}{1.368855in}}{\pgfqpoint{9.868593in}{1.373693in}}{\pgfqpoint{9.865027in}{1.377260in}}%
\pgfpathcurveto{\pgfqpoint{9.861460in}{1.380826in}}{\pgfqpoint{9.856623in}{1.382830in}}{\pgfqpoint{9.851579in}{1.382830in}}%
\pgfpathcurveto{\pgfqpoint{9.846535in}{1.382830in}}{\pgfqpoint{9.841698in}{1.380826in}}{\pgfqpoint{9.838131in}{1.377260in}}%
\pgfpathcurveto{\pgfqpoint{9.834565in}{1.373693in}}{\pgfqpoint{9.832561in}{1.368855in}}{\pgfqpoint{9.832561in}{1.363812in}}%
\pgfpathcurveto{\pgfqpoint{9.832561in}{1.358768in}}{\pgfqpoint{9.834565in}{1.353930in}}{\pgfqpoint{9.838131in}{1.350364in}}%
\pgfpathcurveto{\pgfqpoint{9.841698in}{1.346797in}}{\pgfqpoint{9.846535in}{1.344794in}}{\pgfqpoint{9.851579in}{1.344794in}}%
\pgfpathclose%
\pgfusepath{fill}%
\end{pgfscope}%
\begin{pgfscope}%
\pgfpathrectangle{\pgfqpoint{6.572727in}{0.474100in}}{\pgfqpoint{4.227273in}{3.318700in}}%
\pgfusepath{clip}%
\pgfsetbuttcap%
\pgfsetroundjoin%
\definecolor{currentfill}{rgb}{0.267004,0.004874,0.329415}%
\pgfsetfillcolor{currentfill}%
\pgfsetfillopacity{0.700000}%
\pgfsetlinewidth{0.000000pt}%
\definecolor{currentstroke}{rgb}{0.000000,0.000000,0.000000}%
\pgfsetstrokecolor{currentstroke}%
\pgfsetstrokeopacity{0.700000}%
\pgfsetdash{}{0pt}%
\pgfpathmoveto{\pgfqpoint{8.376556in}{1.368137in}}%
\pgfpathcurveto{\pgfqpoint{8.381599in}{1.368137in}}{\pgfqpoint{8.386437in}{1.370141in}}{\pgfqpoint{8.390004in}{1.373708in}}%
\pgfpathcurveto{\pgfqpoint{8.393570in}{1.377274in}}{\pgfqpoint{8.395574in}{1.382112in}}{\pgfqpoint{8.395574in}{1.387155in}}%
\pgfpathcurveto{\pgfqpoint{8.395574in}{1.392199in}}{\pgfqpoint{8.393570in}{1.397037in}}{\pgfqpoint{8.390004in}{1.400603in}}%
\pgfpathcurveto{\pgfqpoint{8.386437in}{1.404170in}}{\pgfqpoint{8.381599in}{1.406174in}}{\pgfqpoint{8.376556in}{1.406174in}}%
\pgfpathcurveto{\pgfqpoint{8.371512in}{1.406174in}}{\pgfqpoint{8.366674in}{1.404170in}}{\pgfqpoint{8.363108in}{1.400603in}}%
\pgfpathcurveto{\pgfqpoint{8.359541in}{1.397037in}}{\pgfqpoint{8.357538in}{1.392199in}}{\pgfqpoint{8.357538in}{1.387155in}}%
\pgfpathcurveto{\pgfqpoint{8.357538in}{1.382112in}}{\pgfqpoint{8.359541in}{1.377274in}}{\pgfqpoint{8.363108in}{1.373708in}}%
\pgfpathcurveto{\pgfqpoint{8.366674in}{1.370141in}}{\pgfqpoint{8.371512in}{1.368137in}}{\pgfqpoint{8.376556in}{1.368137in}}%
\pgfpathclose%
\pgfusepath{fill}%
\end{pgfscope}%
\begin{pgfscope}%
\pgfpathrectangle{\pgfqpoint{6.572727in}{0.474100in}}{\pgfqpoint{4.227273in}{3.318700in}}%
\pgfusepath{clip}%
\pgfsetbuttcap%
\pgfsetroundjoin%
\definecolor{currentfill}{rgb}{0.993248,0.906157,0.143936}%
\pgfsetfillcolor{currentfill}%
\pgfsetfillopacity{0.700000}%
\pgfsetlinewidth{0.000000pt}%
\definecolor{currentstroke}{rgb}{0.000000,0.000000,0.000000}%
\pgfsetstrokecolor{currentstroke}%
\pgfsetstrokeopacity{0.700000}%
\pgfsetdash{}{0pt}%
\pgfpathmoveto{\pgfqpoint{8.663505in}{2.399876in}}%
\pgfpathcurveto{\pgfqpoint{8.668549in}{2.399876in}}{\pgfqpoint{8.673387in}{2.401880in}}{\pgfqpoint{8.676953in}{2.405446in}}%
\pgfpathcurveto{\pgfqpoint{8.680519in}{2.409013in}}{\pgfqpoint{8.682523in}{2.413851in}}{\pgfqpoint{8.682523in}{2.418894in}}%
\pgfpathcurveto{\pgfqpoint{8.682523in}{2.423938in}}{\pgfqpoint{8.680519in}{2.428776in}}{\pgfqpoint{8.676953in}{2.432342in}}%
\pgfpathcurveto{\pgfqpoint{8.673387in}{2.435909in}}{\pgfqpoint{8.668549in}{2.437912in}}{\pgfqpoint{8.663505in}{2.437912in}}%
\pgfpathcurveto{\pgfqpoint{8.658461in}{2.437912in}}{\pgfqpoint{8.653624in}{2.435909in}}{\pgfqpoint{8.650057in}{2.432342in}}%
\pgfpathcurveto{\pgfqpoint{8.646491in}{2.428776in}}{\pgfqpoint{8.644487in}{2.423938in}}{\pgfqpoint{8.644487in}{2.418894in}}%
\pgfpathcurveto{\pgfqpoint{8.644487in}{2.413851in}}{\pgfqpoint{8.646491in}{2.409013in}}{\pgfqpoint{8.650057in}{2.405446in}}%
\pgfpathcurveto{\pgfqpoint{8.653624in}{2.401880in}}{\pgfqpoint{8.658461in}{2.399876in}}{\pgfqpoint{8.663505in}{2.399876in}}%
\pgfpathclose%
\pgfusepath{fill}%
\end{pgfscope}%
\begin{pgfscope}%
\pgfpathrectangle{\pgfqpoint{6.572727in}{0.474100in}}{\pgfqpoint{4.227273in}{3.318700in}}%
\pgfusepath{clip}%
\pgfsetbuttcap%
\pgfsetroundjoin%
\definecolor{currentfill}{rgb}{0.127568,0.566949,0.550556}%
\pgfsetfillcolor{currentfill}%
\pgfsetfillopacity{0.700000}%
\pgfsetlinewidth{0.000000pt}%
\definecolor{currentstroke}{rgb}{0.000000,0.000000,0.000000}%
\pgfsetstrokecolor{currentstroke}%
\pgfsetstrokeopacity{0.700000}%
\pgfsetdash{}{0pt}%
\pgfpathmoveto{\pgfqpoint{9.500784in}{1.314533in}}%
\pgfpathcurveto{\pgfqpoint{9.505828in}{1.314533in}}{\pgfqpoint{9.510666in}{1.316537in}}{\pgfqpoint{9.514232in}{1.320104in}}%
\pgfpathcurveto{\pgfqpoint{9.517799in}{1.323670in}}{\pgfqpoint{9.519802in}{1.328508in}}{\pgfqpoint{9.519802in}{1.333551in}}%
\pgfpathcurveto{\pgfqpoint{9.519802in}{1.338595in}}{\pgfqpoint{9.517799in}{1.343433in}}{\pgfqpoint{9.514232in}{1.346999in}}%
\pgfpathcurveto{\pgfqpoint{9.510666in}{1.350566in}}{\pgfqpoint{9.505828in}{1.352570in}}{\pgfqpoint{9.500784in}{1.352570in}}%
\pgfpathcurveto{\pgfqpoint{9.495741in}{1.352570in}}{\pgfqpoint{9.490903in}{1.350566in}}{\pgfqpoint{9.487336in}{1.346999in}}%
\pgfpathcurveto{\pgfqpoint{9.483770in}{1.343433in}}{\pgfqpoint{9.481766in}{1.338595in}}{\pgfqpoint{9.481766in}{1.333551in}}%
\pgfpathcurveto{\pgfqpoint{9.481766in}{1.328508in}}{\pgfqpoint{9.483770in}{1.323670in}}{\pgfqpoint{9.487336in}{1.320104in}}%
\pgfpathcurveto{\pgfqpoint{9.490903in}{1.316537in}}{\pgfqpoint{9.495741in}{1.314533in}}{\pgfqpoint{9.500784in}{1.314533in}}%
\pgfpathclose%
\pgfusepath{fill}%
\end{pgfscope}%
\begin{pgfscope}%
\pgfpathrectangle{\pgfqpoint{6.572727in}{0.474100in}}{\pgfqpoint{4.227273in}{3.318700in}}%
\pgfusepath{clip}%
\pgfsetbuttcap%
\pgfsetroundjoin%
\definecolor{currentfill}{rgb}{0.267004,0.004874,0.329415}%
\pgfsetfillcolor{currentfill}%
\pgfsetfillopacity{0.700000}%
\pgfsetlinewidth{0.000000pt}%
\definecolor{currentstroke}{rgb}{0.000000,0.000000,0.000000}%
\pgfsetstrokecolor{currentstroke}%
\pgfsetstrokeopacity{0.700000}%
\pgfsetdash{}{0pt}%
\pgfpathmoveto{\pgfqpoint{7.629421in}{1.278138in}}%
\pgfpathcurveto{\pgfqpoint{7.634465in}{1.278138in}}{\pgfqpoint{7.639303in}{1.280142in}}{\pgfqpoint{7.642869in}{1.283709in}}%
\pgfpathcurveto{\pgfqpoint{7.646435in}{1.287275in}}{\pgfqpoint{7.648439in}{1.292113in}}{\pgfqpoint{7.648439in}{1.297157in}}%
\pgfpathcurveto{\pgfqpoint{7.648439in}{1.302200in}}{\pgfqpoint{7.646435in}{1.307038in}}{\pgfqpoint{7.642869in}{1.310604in}}%
\pgfpathcurveto{\pgfqpoint{7.639303in}{1.314171in}}{\pgfqpoint{7.634465in}{1.316175in}}{\pgfqpoint{7.629421in}{1.316175in}}%
\pgfpathcurveto{\pgfqpoint{7.624377in}{1.316175in}}{\pgfqpoint{7.619540in}{1.314171in}}{\pgfqpoint{7.615973in}{1.310604in}}%
\pgfpathcurveto{\pgfqpoint{7.612407in}{1.307038in}}{\pgfqpoint{7.610403in}{1.302200in}}{\pgfqpoint{7.610403in}{1.297157in}}%
\pgfpathcurveto{\pgfqpoint{7.610403in}{1.292113in}}{\pgfqpoint{7.612407in}{1.287275in}}{\pgfqpoint{7.615973in}{1.283709in}}%
\pgfpathcurveto{\pgfqpoint{7.619540in}{1.280142in}}{\pgfqpoint{7.624377in}{1.278138in}}{\pgfqpoint{7.629421in}{1.278138in}}%
\pgfpathclose%
\pgfusepath{fill}%
\end{pgfscope}%
\begin{pgfscope}%
\pgfpathrectangle{\pgfqpoint{6.572727in}{0.474100in}}{\pgfqpoint{4.227273in}{3.318700in}}%
\pgfusepath{clip}%
\pgfsetbuttcap%
\pgfsetroundjoin%
\definecolor{currentfill}{rgb}{0.993248,0.906157,0.143936}%
\pgfsetfillcolor{currentfill}%
\pgfsetfillopacity{0.700000}%
\pgfsetlinewidth{0.000000pt}%
\definecolor{currentstroke}{rgb}{0.000000,0.000000,0.000000}%
\pgfsetstrokecolor{currentstroke}%
\pgfsetstrokeopacity{0.700000}%
\pgfsetdash{}{0pt}%
\pgfpathmoveto{\pgfqpoint{8.269253in}{2.915384in}}%
\pgfpathcurveto{\pgfqpoint{8.274296in}{2.915384in}}{\pgfqpoint{8.279134in}{2.917388in}}{\pgfqpoint{8.282701in}{2.920954in}}%
\pgfpathcurveto{\pgfqpoint{8.286267in}{2.924521in}}{\pgfqpoint{8.288271in}{2.929358in}}{\pgfqpoint{8.288271in}{2.934402in}}%
\pgfpathcurveto{\pgfqpoint{8.288271in}{2.939446in}}{\pgfqpoint{8.286267in}{2.944283in}}{\pgfqpoint{8.282701in}{2.947850in}}%
\pgfpathcurveto{\pgfqpoint{8.279134in}{2.951416in}}{\pgfqpoint{8.274296in}{2.953420in}}{\pgfqpoint{8.269253in}{2.953420in}}%
\pgfpathcurveto{\pgfqpoint{8.264209in}{2.953420in}}{\pgfqpoint{8.259371in}{2.951416in}}{\pgfqpoint{8.255805in}{2.947850in}}%
\pgfpathcurveto{\pgfqpoint{8.252238in}{2.944283in}}{\pgfqpoint{8.250235in}{2.939446in}}{\pgfqpoint{8.250235in}{2.934402in}}%
\pgfpathcurveto{\pgfqpoint{8.250235in}{2.929358in}}{\pgfqpoint{8.252238in}{2.924521in}}{\pgfqpoint{8.255805in}{2.920954in}}%
\pgfpathcurveto{\pgfqpoint{8.259371in}{2.917388in}}{\pgfqpoint{8.264209in}{2.915384in}}{\pgfqpoint{8.269253in}{2.915384in}}%
\pgfpathclose%
\pgfusepath{fill}%
\end{pgfscope}%
\begin{pgfscope}%
\pgfpathrectangle{\pgfqpoint{6.572727in}{0.474100in}}{\pgfqpoint{4.227273in}{3.318700in}}%
\pgfusepath{clip}%
\pgfsetbuttcap%
\pgfsetroundjoin%
\definecolor{currentfill}{rgb}{0.267004,0.004874,0.329415}%
\pgfsetfillcolor{currentfill}%
\pgfsetfillopacity{0.700000}%
\pgfsetlinewidth{0.000000pt}%
\definecolor{currentstroke}{rgb}{0.000000,0.000000,0.000000}%
\pgfsetstrokecolor{currentstroke}%
\pgfsetstrokeopacity{0.700000}%
\pgfsetdash{}{0pt}%
\pgfpathmoveto{\pgfqpoint{7.947538in}{1.616409in}}%
\pgfpathcurveto{\pgfqpoint{7.952582in}{1.616409in}}{\pgfqpoint{7.957420in}{1.618413in}}{\pgfqpoint{7.960986in}{1.621979in}}%
\pgfpathcurveto{\pgfqpoint{7.964553in}{1.625545in}}{\pgfqpoint{7.966557in}{1.630383in}}{\pgfqpoint{7.966557in}{1.635427in}}%
\pgfpathcurveto{\pgfqpoint{7.966557in}{1.640471in}}{\pgfqpoint{7.964553in}{1.645308in}}{\pgfqpoint{7.960986in}{1.648875in}}%
\pgfpathcurveto{\pgfqpoint{7.957420in}{1.652441in}}{\pgfqpoint{7.952582in}{1.654445in}}{\pgfqpoint{7.947538in}{1.654445in}}%
\pgfpathcurveto{\pgfqpoint{7.942495in}{1.654445in}}{\pgfqpoint{7.937657in}{1.652441in}}{\pgfqpoint{7.934091in}{1.648875in}}%
\pgfpathcurveto{\pgfqpoint{7.930524in}{1.645308in}}{\pgfqpoint{7.928520in}{1.640471in}}{\pgfqpoint{7.928520in}{1.635427in}}%
\pgfpathcurveto{\pgfqpoint{7.928520in}{1.630383in}}{\pgfqpoint{7.930524in}{1.625545in}}{\pgfqpoint{7.934091in}{1.621979in}}%
\pgfpathcurveto{\pgfqpoint{7.937657in}{1.618413in}}{\pgfqpoint{7.942495in}{1.616409in}}{\pgfqpoint{7.947538in}{1.616409in}}%
\pgfpathclose%
\pgfusepath{fill}%
\end{pgfscope}%
\begin{pgfscope}%
\pgfpathrectangle{\pgfqpoint{6.572727in}{0.474100in}}{\pgfqpoint{4.227273in}{3.318700in}}%
\pgfusepath{clip}%
\pgfsetbuttcap%
\pgfsetroundjoin%
\definecolor{currentfill}{rgb}{0.267004,0.004874,0.329415}%
\pgfsetfillcolor{currentfill}%
\pgfsetfillopacity{0.700000}%
\pgfsetlinewidth{0.000000pt}%
\definecolor{currentstroke}{rgb}{0.000000,0.000000,0.000000}%
\pgfsetstrokecolor{currentstroke}%
\pgfsetstrokeopacity{0.700000}%
\pgfsetdash{}{0pt}%
\pgfpathmoveto{\pgfqpoint{8.072516in}{2.075737in}}%
\pgfpathcurveto{\pgfqpoint{8.077560in}{2.075737in}}{\pgfqpoint{8.082398in}{2.077741in}}{\pgfqpoint{8.085964in}{2.081308in}}%
\pgfpathcurveto{\pgfqpoint{8.089530in}{2.084874in}}{\pgfqpoint{8.091534in}{2.089712in}}{\pgfqpoint{8.091534in}{2.094756in}}%
\pgfpathcurveto{\pgfqpoint{8.091534in}{2.099799in}}{\pgfqpoint{8.089530in}{2.104637in}}{\pgfqpoint{8.085964in}{2.108203in}}%
\pgfpathcurveto{\pgfqpoint{8.082398in}{2.111770in}}{\pgfqpoint{8.077560in}{2.113774in}}{\pgfqpoint{8.072516in}{2.113774in}}%
\pgfpathcurveto{\pgfqpoint{8.067472in}{2.113774in}}{\pgfqpoint{8.062635in}{2.111770in}}{\pgfqpoint{8.059068in}{2.108203in}}%
\pgfpathcurveto{\pgfqpoint{8.055502in}{2.104637in}}{\pgfqpoint{8.053498in}{2.099799in}}{\pgfqpoint{8.053498in}{2.094756in}}%
\pgfpathcurveto{\pgfqpoint{8.053498in}{2.089712in}}{\pgfqpoint{8.055502in}{2.084874in}}{\pgfqpoint{8.059068in}{2.081308in}}%
\pgfpathcurveto{\pgfqpoint{8.062635in}{2.077741in}}{\pgfqpoint{8.067472in}{2.075737in}}{\pgfqpoint{8.072516in}{2.075737in}}%
\pgfpathclose%
\pgfusepath{fill}%
\end{pgfscope}%
\begin{pgfscope}%
\pgfpathrectangle{\pgfqpoint{6.572727in}{0.474100in}}{\pgfqpoint{4.227273in}{3.318700in}}%
\pgfusepath{clip}%
\pgfsetbuttcap%
\pgfsetroundjoin%
\definecolor{currentfill}{rgb}{0.993248,0.906157,0.143936}%
\pgfsetfillcolor{currentfill}%
\pgfsetfillopacity{0.700000}%
\pgfsetlinewidth{0.000000pt}%
\definecolor{currentstroke}{rgb}{0.000000,0.000000,0.000000}%
\pgfsetstrokecolor{currentstroke}%
\pgfsetstrokeopacity{0.700000}%
\pgfsetdash{}{0pt}%
\pgfpathmoveto{\pgfqpoint{7.909375in}{3.188528in}}%
\pgfpathcurveto{\pgfqpoint{7.914418in}{3.188528in}}{\pgfqpoint{7.919256in}{3.190532in}}{\pgfqpoint{7.922823in}{3.194098in}}%
\pgfpathcurveto{\pgfqpoint{7.926389in}{3.197665in}}{\pgfqpoint{7.928393in}{3.202503in}}{\pgfqpoint{7.928393in}{3.207546in}}%
\pgfpathcurveto{\pgfqpoint{7.928393in}{3.212590in}}{\pgfqpoint{7.926389in}{3.217428in}}{\pgfqpoint{7.922823in}{3.220994in}}%
\pgfpathcurveto{\pgfqpoint{7.919256in}{3.224560in}}{\pgfqpoint{7.914418in}{3.226564in}}{\pgfqpoint{7.909375in}{3.226564in}}%
\pgfpathcurveto{\pgfqpoint{7.904331in}{3.226564in}}{\pgfqpoint{7.899493in}{3.224560in}}{\pgfqpoint{7.895927in}{3.220994in}}%
\pgfpathcurveto{\pgfqpoint{7.892361in}{3.217428in}}{\pgfqpoint{7.890357in}{3.212590in}}{\pgfqpoint{7.890357in}{3.207546in}}%
\pgfpathcurveto{\pgfqpoint{7.890357in}{3.202503in}}{\pgfqpoint{7.892361in}{3.197665in}}{\pgfqpoint{7.895927in}{3.194098in}}%
\pgfpathcurveto{\pgfqpoint{7.899493in}{3.190532in}}{\pgfqpoint{7.904331in}{3.188528in}}{\pgfqpoint{7.909375in}{3.188528in}}%
\pgfpathclose%
\pgfusepath{fill}%
\end{pgfscope}%
\begin{pgfscope}%
\pgfpathrectangle{\pgfqpoint{6.572727in}{0.474100in}}{\pgfqpoint{4.227273in}{3.318700in}}%
\pgfusepath{clip}%
\pgfsetbuttcap%
\pgfsetroundjoin%
\definecolor{currentfill}{rgb}{0.993248,0.906157,0.143936}%
\pgfsetfillcolor{currentfill}%
\pgfsetfillopacity{0.700000}%
\pgfsetlinewidth{0.000000pt}%
\definecolor{currentstroke}{rgb}{0.000000,0.000000,0.000000}%
\pgfsetstrokecolor{currentstroke}%
\pgfsetstrokeopacity{0.700000}%
\pgfsetdash{}{0pt}%
\pgfpathmoveto{\pgfqpoint{8.131219in}{2.735666in}}%
\pgfpathcurveto{\pgfqpoint{8.136263in}{2.735666in}}{\pgfqpoint{8.141101in}{2.737669in}}{\pgfqpoint{8.144667in}{2.741236in}}%
\pgfpathcurveto{\pgfqpoint{8.148234in}{2.744802in}}{\pgfqpoint{8.150237in}{2.749640in}}{\pgfqpoint{8.150237in}{2.754684in}}%
\pgfpathcurveto{\pgfqpoint{8.150237in}{2.759727in}}{\pgfqpoint{8.148234in}{2.764565in}}{\pgfqpoint{8.144667in}{2.768132in}}%
\pgfpathcurveto{\pgfqpoint{8.141101in}{2.771698in}}{\pgfqpoint{8.136263in}{2.773702in}}{\pgfqpoint{8.131219in}{2.773702in}}%
\pgfpathcurveto{\pgfqpoint{8.126176in}{2.773702in}}{\pgfqpoint{8.121338in}{2.771698in}}{\pgfqpoint{8.117771in}{2.768132in}}%
\pgfpathcurveto{\pgfqpoint{8.114205in}{2.764565in}}{\pgfqpoint{8.112201in}{2.759727in}}{\pgfqpoint{8.112201in}{2.754684in}}%
\pgfpathcurveto{\pgfqpoint{8.112201in}{2.749640in}}{\pgfqpoint{8.114205in}{2.744802in}}{\pgfqpoint{8.117771in}{2.741236in}}%
\pgfpathcurveto{\pgfqpoint{8.121338in}{2.737669in}}{\pgfqpoint{8.126176in}{2.735666in}}{\pgfqpoint{8.131219in}{2.735666in}}%
\pgfpathclose%
\pgfusepath{fill}%
\end{pgfscope}%
\begin{pgfscope}%
\pgfpathrectangle{\pgfqpoint{6.572727in}{0.474100in}}{\pgfqpoint{4.227273in}{3.318700in}}%
\pgfusepath{clip}%
\pgfsetbuttcap%
\pgfsetroundjoin%
\definecolor{currentfill}{rgb}{0.993248,0.906157,0.143936}%
\pgfsetfillcolor{currentfill}%
\pgfsetfillopacity{0.700000}%
\pgfsetlinewidth{0.000000pt}%
\definecolor{currentstroke}{rgb}{0.000000,0.000000,0.000000}%
\pgfsetstrokecolor{currentstroke}%
\pgfsetstrokeopacity{0.700000}%
\pgfsetdash{}{0pt}%
\pgfpathmoveto{\pgfqpoint{7.992248in}{2.774562in}}%
\pgfpathcurveto{\pgfqpoint{7.997291in}{2.774562in}}{\pgfqpoint{8.002129in}{2.776566in}}{\pgfqpoint{8.005696in}{2.780132in}}%
\pgfpathcurveto{\pgfqpoint{8.009262in}{2.783698in}}{\pgfqpoint{8.011266in}{2.788536in}}{\pgfqpoint{8.011266in}{2.793580in}}%
\pgfpathcurveto{\pgfqpoint{8.011266in}{2.798623in}}{\pgfqpoint{8.009262in}{2.803461in}}{\pgfqpoint{8.005696in}{2.807028in}}%
\pgfpathcurveto{\pgfqpoint{8.002129in}{2.810594in}}{\pgfqpoint{7.997291in}{2.812598in}}{\pgfqpoint{7.992248in}{2.812598in}}%
\pgfpathcurveto{\pgfqpoint{7.987204in}{2.812598in}}{\pgfqpoint{7.982366in}{2.810594in}}{\pgfqpoint{7.978800in}{2.807028in}}%
\pgfpathcurveto{\pgfqpoint{7.975234in}{2.803461in}}{\pgfqpoint{7.973230in}{2.798623in}}{\pgfqpoint{7.973230in}{2.793580in}}%
\pgfpathcurveto{\pgfqpoint{7.973230in}{2.788536in}}{\pgfqpoint{7.975234in}{2.783698in}}{\pgfqpoint{7.978800in}{2.780132in}}%
\pgfpathcurveto{\pgfqpoint{7.982366in}{2.776566in}}{\pgfqpoint{7.987204in}{2.774562in}}{\pgfqpoint{7.992248in}{2.774562in}}%
\pgfpathclose%
\pgfusepath{fill}%
\end{pgfscope}%
\begin{pgfscope}%
\pgfpathrectangle{\pgfqpoint{6.572727in}{0.474100in}}{\pgfqpoint{4.227273in}{3.318700in}}%
\pgfusepath{clip}%
\pgfsetbuttcap%
\pgfsetroundjoin%
\definecolor{currentfill}{rgb}{0.267004,0.004874,0.329415}%
\pgfsetfillcolor{currentfill}%
\pgfsetfillopacity{0.700000}%
\pgfsetlinewidth{0.000000pt}%
\definecolor{currentstroke}{rgb}{0.000000,0.000000,0.000000}%
\pgfsetstrokecolor{currentstroke}%
\pgfsetstrokeopacity{0.700000}%
\pgfsetdash{}{0pt}%
\pgfpathmoveto{\pgfqpoint{8.465141in}{1.484636in}}%
\pgfpathcurveto{\pgfqpoint{8.470185in}{1.484636in}}{\pgfqpoint{8.475023in}{1.486640in}}{\pgfqpoint{8.478589in}{1.490206in}}%
\pgfpathcurveto{\pgfqpoint{8.482156in}{1.493773in}}{\pgfqpoint{8.484159in}{1.498611in}}{\pgfqpoint{8.484159in}{1.503654in}}%
\pgfpathcurveto{\pgfqpoint{8.484159in}{1.508698in}}{\pgfqpoint{8.482156in}{1.513536in}}{\pgfqpoint{8.478589in}{1.517102in}}%
\pgfpathcurveto{\pgfqpoint{8.475023in}{1.520669in}}{\pgfqpoint{8.470185in}{1.522672in}}{\pgfqpoint{8.465141in}{1.522672in}}%
\pgfpathcurveto{\pgfqpoint{8.460098in}{1.522672in}}{\pgfqpoint{8.455260in}{1.520669in}}{\pgfqpoint{8.451693in}{1.517102in}}%
\pgfpathcurveto{\pgfqpoint{8.448127in}{1.513536in}}{\pgfqpoint{8.446123in}{1.508698in}}{\pgfqpoint{8.446123in}{1.503654in}}%
\pgfpathcurveto{\pgfqpoint{8.446123in}{1.498611in}}{\pgfqpoint{8.448127in}{1.493773in}}{\pgfqpoint{8.451693in}{1.490206in}}%
\pgfpathcurveto{\pgfqpoint{8.455260in}{1.486640in}}{\pgfqpoint{8.460098in}{1.484636in}}{\pgfqpoint{8.465141in}{1.484636in}}%
\pgfpathclose%
\pgfusepath{fill}%
\end{pgfscope}%
\begin{pgfscope}%
\pgfpathrectangle{\pgfqpoint{6.572727in}{0.474100in}}{\pgfqpoint{4.227273in}{3.318700in}}%
\pgfusepath{clip}%
\pgfsetbuttcap%
\pgfsetroundjoin%
\definecolor{currentfill}{rgb}{0.993248,0.906157,0.143936}%
\pgfsetfillcolor{currentfill}%
\pgfsetfillopacity{0.700000}%
\pgfsetlinewidth{0.000000pt}%
\definecolor{currentstroke}{rgb}{0.000000,0.000000,0.000000}%
\pgfsetstrokecolor{currentstroke}%
\pgfsetstrokeopacity{0.700000}%
\pgfsetdash{}{0pt}%
\pgfpathmoveto{\pgfqpoint{7.924818in}{2.619847in}}%
\pgfpathcurveto{\pgfqpoint{7.929861in}{2.619847in}}{\pgfqpoint{7.934699in}{2.621851in}}{\pgfqpoint{7.938266in}{2.625417in}}%
\pgfpathcurveto{\pgfqpoint{7.941832in}{2.628984in}}{\pgfqpoint{7.943836in}{2.633822in}}{\pgfqpoint{7.943836in}{2.638865in}}%
\pgfpathcurveto{\pgfqpoint{7.943836in}{2.643909in}}{\pgfqpoint{7.941832in}{2.648747in}}{\pgfqpoint{7.938266in}{2.652313in}}%
\pgfpathcurveto{\pgfqpoint{7.934699in}{2.655879in}}{\pgfqpoint{7.929861in}{2.657883in}}{\pgfqpoint{7.924818in}{2.657883in}}%
\pgfpathcurveto{\pgfqpoint{7.919774in}{2.657883in}}{\pgfqpoint{7.914936in}{2.655879in}}{\pgfqpoint{7.911370in}{2.652313in}}%
\pgfpathcurveto{\pgfqpoint{7.907803in}{2.648747in}}{\pgfqpoint{7.905800in}{2.643909in}}{\pgfqpoint{7.905800in}{2.638865in}}%
\pgfpathcurveto{\pgfqpoint{7.905800in}{2.633822in}}{\pgfqpoint{7.907803in}{2.628984in}}{\pgfqpoint{7.911370in}{2.625417in}}%
\pgfpathcurveto{\pgfqpoint{7.914936in}{2.621851in}}{\pgfqpoint{7.919774in}{2.619847in}}{\pgfqpoint{7.924818in}{2.619847in}}%
\pgfpathclose%
\pgfusepath{fill}%
\end{pgfscope}%
\begin{pgfscope}%
\pgfpathrectangle{\pgfqpoint{6.572727in}{0.474100in}}{\pgfqpoint{4.227273in}{3.318700in}}%
\pgfusepath{clip}%
\pgfsetbuttcap%
\pgfsetroundjoin%
\definecolor{currentfill}{rgb}{0.127568,0.566949,0.550556}%
\pgfsetfillcolor{currentfill}%
\pgfsetfillopacity{0.700000}%
\pgfsetlinewidth{0.000000pt}%
\definecolor{currentstroke}{rgb}{0.000000,0.000000,0.000000}%
\pgfsetstrokecolor{currentstroke}%
\pgfsetstrokeopacity{0.700000}%
\pgfsetdash{}{0pt}%
\pgfpathmoveto{\pgfqpoint{9.627687in}{1.515561in}}%
\pgfpathcurveto{\pgfqpoint{9.632731in}{1.515561in}}{\pgfqpoint{9.637569in}{1.517565in}}{\pgfqpoint{9.641135in}{1.521132in}}%
\pgfpathcurveto{\pgfqpoint{9.644701in}{1.524698in}}{\pgfqpoint{9.646705in}{1.529536in}}{\pgfqpoint{9.646705in}{1.534579in}}%
\pgfpathcurveto{\pgfqpoint{9.646705in}{1.539623in}}{\pgfqpoint{9.644701in}{1.544461in}}{\pgfqpoint{9.641135in}{1.548027in}}%
\pgfpathcurveto{\pgfqpoint{9.637569in}{1.551594in}}{\pgfqpoint{9.632731in}{1.553598in}}{\pgfqpoint{9.627687in}{1.553598in}}%
\pgfpathcurveto{\pgfqpoint{9.622644in}{1.553598in}}{\pgfqpoint{9.617806in}{1.551594in}}{\pgfqpoint{9.614239in}{1.548027in}}%
\pgfpathcurveto{\pgfqpoint{9.610673in}{1.544461in}}{\pgfqpoint{9.608669in}{1.539623in}}{\pgfqpoint{9.608669in}{1.534579in}}%
\pgfpathcurveto{\pgfqpoint{9.608669in}{1.529536in}}{\pgfqpoint{9.610673in}{1.524698in}}{\pgfqpoint{9.614239in}{1.521132in}}%
\pgfpathcurveto{\pgfqpoint{9.617806in}{1.517565in}}{\pgfqpoint{9.622644in}{1.515561in}}{\pgfqpoint{9.627687in}{1.515561in}}%
\pgfpathclose%
\pgfusepath{fill}%
\end{pgfscope}%
\begin{pgfscope}%
\pgfpathrectangle{\pgfqpoint{6.572727in}{0.474100in}}{\pgfqpoint{4.227273in}{3.318700in}}%
\pgfusepath{clip}%
\pgfsetbuttcap%
\pgfsetroundjoin%
\definecolor{currentfill}{rgb}{0.267004,0.004874,0.329415}%
\pgfsetfillcolor{currentfill}%
\pgfsetfillopacity{0.700000}%
\pgfsetlinewidth{0.000000pt}%
\definecolor{currentstroke}{rgb}{0.000000,0.000000,0.000000}%
\pgfsetstrokecolor{currentstroke}%
\pgfsetstrokeopacity{0.700000}%
\pgfsetdash{}{0pt}%
\pgfpathmoveto{\pgfqpoint{7.472487in}{2.167674in}}%
\pgfpathcurveto{\pgfqpoint{7.477531in}{2.167674in}}{\pgfqpoint{7.482369in}{2.169678in}}{\pgfqpoint{7.485935in}{2.173244in}}%
\pgfpathcurveto{\pgfqpoint{7.489501in}{2.176811in}}{\pgfqpoint{7.491505in}{2.181649in}}{\pgfqpoint{7.491505in}{2.186692in}}%
\pgfpathcurveto{\pgfqpoint{7.491505in}{2.191736in}}{\pgfqpoint{7.489501in}{2.196574in}}{\pgfqpoint{7.485935in}{2.200140in}}%
\pgfpathcurveto{\pgfqpoint{7.482369in}{2.203707in}}{\pgfqpoint{7.477531in}{2.205710in}}{\pgfqpoint{7.472487in}{2.205710in}}%
\pgfpathcurveto{\pgfqpoint{7.467444in}{2.205710in}}{\pgfqpoint{7.462606in}{2.203707in}}{\pgfqpoint{7.459039in}{2.200140in}}%
\pgfpathcurveto{\pgfqpoint{7.455473in}{2.196574in}}{\pgfqpoint{7.453469in}{2.191736in}}{\pgfqpoint{7.453469in}{2.186692in}}%
\pgfpathcurveto{\pgfqpoint{7.453469in}{2.181649in}}{\pgfqpoint{7.455473in}{2.176811in}}{\pgfqpoint{7.459039in}{2.173244in}}%
\pgfpathcurveto{\pgfqpoint{7.462606in}{2.169678in}}{\pgfqpoint{7.467444in}{2.167674in}}{\pgfqpoint{7.472487in}{2.167674in}}%
\pgfpathclose%
\pgfusepath{fill}%
\end{pgfscope}%
\begin{pgfscope}%
\pgfpathrectangle{\pgfqpoint{6.572727in}{0.474100in}}{\pgfqpoint{4.227273in}{3.318700in}}%
\pgfusepath{clip}%
\pgfsetbuttcap%
\pgfsetroundjoin%
\definecolor{currentfill}{rgb}{0.993248,0.906157,0.143936}%
\pgfsetfillcolor{currentfill}%
\pgfsetfillopacity{0.700000}%
\pgfsetlinewidth{0.000000pt}%
\definecolor{currentstroke}{rgb}{0.000000,0.000000,0.000000}%
\pgfsetstrokecolor{currentstroke}%
\pgfsetstrokeopacity{0.700000}%
\pgfsetdash{}{0pt}%
\pgfpathmoveto{\pgfqpoint{8.528278in}{3.159160in}}%
\pgfpathcurveto{\pgfqpoint{8.533322in}{3.159160in}}{\pgfqpoint{8.538160in}{3.161164in}}{\pgfqpoint{8.541726in}{3.164731in}}%
\pgfpathcurveto{\pgfqpoint{8.545292in}{3.168297in}}{\pgfqpoint{8.547296in}{3.173135in}}{\pgfqpoint{8.547296in}{3.178179in}}%
\pgfpathcurveto{\pgfqpoint{8.547296in}{3.183222in}}{\pgfqpoint{8.545292in}{3.188060in}}{\pgfqpoint{8.541726in}{3.191626in}}%
\pgfpathcurveto{\pgfqpoint{8.538160in}{3.195193in}}{\pgfqpoint{8.533322in}{3.197197in}}{\pgfqpoint{8.528278in}{3.197197in}}%
\pgfpathcurveto{\pgfqpoint{8.523234in}{3.197197in}}{\pgfqpoint{8.518397in}{3.195193in}}{\pgfqpoint{8.514830in}{3.191626in}}%
\pgfpathcurveto{\pgfqpoint{8.511264in}{3.188060in}}{\pgfqpoint{8.509260in}{3.183222in}}{\pgfqpoint{8.509260in}{3.178179in}}%
\pgfpathcurveto{\pgfqpoint{8.509260in}{3.173135in}}{\pgfqpoint{8.511264in}{3.168297in}}{\pgfqpoint{8.514830in}{3.164731in}}%
\pgfpathcurveto{\pgfqpoint{8.518397in}{3.161164in}}{\pgfqpoint{8.523234in}{3.159160in}}{\pgfqpoint{8.528278in}{3.159160in}}%
\pgfpathclose%
\pgfusepath{fill}%
\end{pgfscope}%
\begin{pgfscope}%
\pgfpathrectangle{\pgfqpoint{6.572727in}{0.474100in}}{\pgfqpoint{4.227273in}{3.318700in}}%
\pgfusepath{clip}%
\pgfsetbuttcap%
\pgfsetroundjoin%
\definecolor{currentfill}{rgb}{0.267004,0.004874,0.329415}%
\pgfsetfillcolor{currentfill}%
\pgfsetfillopacity{0.700000}%
\pgfsetlinewidth{0.000000pt}%
\definecolor{currentstroke}{rgb}{0.000000,0.000000,0.000000}%
\pgfsetstrokecolor{currentstroke}%
\pgfsetstrokeopacity{0.700000}%
\pgfsetdash{}{0pt}%
\pgfpathmoveto{\pgfqpoint{8.062541in}{1.913690in}}%
\pgfpathcurveto{\pgfqpoint{8.067585in}{1.913690in}}{\pgfqpoint{8.072423in}{1.915694in}}{\pgfqpoint{8.075989in}{1.919260in}}%
\pgfpathcurveto{\pgfqpoint{8.079556in}{1.922826in}}{\pgfqpoint{8.081559in}{1.927664in}}{\pgfqpoint{8.081559in}{1.932708in}}%
\pgfpathcurveto{\pgfqpoint{8.081559in}{1.937751in}}{\pgfqpoint{8.079556in}{1.942589in}}{\pgfqpoint{8.075989in}{1.946156in}}%
\pgfpathcurveto{\pgfqpoint{8.072423in}{1.949722in}}{\pgfqpoint{8.067585in}{1.951726in}}{\pgfqpoint{8.062541in}{1.951726in}}%
\pgfpathcurveto{\pgfqpoint{8.057498in}{1.951726in}}{\pgfqpoint{8.052660in}{1.949722in}}{\pgfqpoint{8.049093in}{1.946156in}}%
\pgfpathcurveto{\pgfqpoint{8.045527in}{1.942589in}}{\pgfqpoint{8.043523in}{1.937751in}}{\pgfqpoint{8.043523in}{1.932708in}}%
\pgfpathcurveto{\pgfqpoint{8.043523in}{1.927664in}}{\pgfqpoint{8.045527in}{1.922826in}}{\pgfqpoint{8.049093in}{1.919260in}}%
\pgfpathcurveto{\pgfqpoint{8.052660in}{1.915694in}}{\pgfqpoint{8.057498in}{1.913690in}}{\pgfqpoint{8.062541in}{1.913690in}}%
\pgfpathclose%
\pgfusepath{fill}%
\end{pgfscope}%
\begin{pgfscope}%
\pgfpathrectangle{\pgfqpoint{6.572727in}{0.474100in}}{\pgfqpoint{4.227273in}{3.318700in}}%
\pgfusepath{clip}%
\pgfsetbuttcap%
\pgfsetroundjoin%
\definecolor{currentfill}{rgb}{0.127568,0.566949,0.550556}%
\pgfsetfillcolor{currentfill}%
\pgfsetfillopacity{0.700000}%
\pgfsetlinewidth{0.000000pt}%
\definecolor{currentstroke}{rgb}{0.000000,0.000000,0.000000}%
\pgfsetstrokecolor{currentstroke}%
\pgfsetstrokeopacity{0.700000}%
\pgfsetdash{}{0pt}%
\pgfpathmoveto{\pgfqpoint{9.647632in}{1.817210in}}%
\pgfpathcurveto{\pgfqpoint{9.652676in}{1.817210in}}{\pgfqpoint{9.657514in}{1.819214in}}{\pgfqpoint{9.661080in}{1.822781in}}%
\pgfpathcurveto{\pgfqpoint{9.664647in}{1.826347in}}{\pgfqpoint{9.666651in}{1.831185in}}{\pgfqpoint{9.666651in}{1.836229in}}%
\pgfpathcurveto{\pgfqpoint{9.666651in}{1.841272in}}{\pgfqpoint{9.664647in}{1.846110in}}{\pgfqpoint{9.661080in}{1.849676in}}%
\pgfpathcurveto{\pgfqpoint{9.657514in}{1.853243in}}{\pgfqpoint{9.652676in}{1.855247in}}{\pgfqpoint{9.647632in}{1.855247in}}%
\pgfpathcurveto{\pgfqpoint{9.642589in}{1.855247in}}{\pgfqpoint{9.637751in}{1.853243in}}{\pgfqpoint{9.634185in}{1.849676in}}%
\pgfpathcurveto{\pgfqpoint{9.630618in}{1.846110in}}{\pgfqpoint{9.628614in}{1.841272in}}{\pgfqpoint{9.628614in}{1.836229in}}%
\pgfpathcurveto{\pgfqpoint{9.628614in}{1.831185in}}{\pgfqpoint{9.630618in}{1.826347in}}{\pgfqpoint{9.634185in}{1.822781in}}%
\pgfpathcurveto{\pgfqpoint{9.637751in}{1.819214in}}{\pgfqpoint{9.642589in}{1.817210in}}{\pgfqpoint{9.647632in}{1.817210in}}%
\pgfpathclose%
\pgfusepath{fill}%
\end{pgfscope}%
\begin{pgfscope}%
\pgfpathrectangle{\pgfqpoint{6.572727in}{0.474100in}}{\pgfqpoint{4.227273in}{3.318700in}}%
\pgfusepath{clip}%
\pgfsetbuttcap%
\pgfsetroundjoin%
\definecolor{currentfill}{rgb}{0.267004,0.004874,0.329415}%
\pgfsetfillcolor{currentfill}%
\pgfsetfillopacity{0.700000}%
\pgfsetlinewidth{0.000000pt}%
\definecolor{currentstroke}{rgb}{0.000000,0.000000,0.000000}%
\pgfsetstrokecolor{currentstroke}%
\pgfsetstrokeopacity{0.700000}%
\pgfsetdash{}{0pt}%
\pgfpathmoveto{\pgfqpoint{8.206322in}{1.672645in}}%
\pgfpathcurveto{\pgfqpoint{8.211365in}{1.672645in}}{\pgfqpoint{8.216203in}{1.674649in}}{\pgfqpoint{8.219769in}{1.678215in}}%
\pgfpathcurveto{\pgfqpoint{8.223336in}{1.681781in}}{\pgfqpoint{8.225340in}{1.686619in}}{\pgfqpoint{8.225340in}{1.691663in}}%
\pgfpathcurveto{\pgfqpoint{8.225340in}{1.696707in}}{\pgfqpoint{8.223336in}{1.701544in}}{\pgfqpoint{8.219769in}{1.705111in}}%
\pgfpathcurveto{\pgfqpoint{8.216203in}{1.708677in}}{\pgfqpoint{8.211365in}{1.710681in}}{\pgfqpoint{8.206322in}{1.710681in}}%
\pgfpathcurveto{\pgfqpoint{8.201278in}{1.710681in}}{\pgfqpoint{8.196440in}{1.708677in}}{\pgfqpoint{8.192874in}{1.705111in}}%
\pgfpathcurveto{\pgfqpoint{8.189307in}{1.701544in}}{\pgfqpoint{8.187303in}{1.696707in}}{\pgfqpoint{8.187303in}{1.691663in}}%
\pgfpathcurveto{\pgfqpoint{8.187303in}{1.686619in}}{\pgfqpoint{8.189307in}{1.681781in}}{\pgfqpoint{8.192874in}{1.678215in}}%
\pgfpathcurveto{\pgfqpoint{8.196440in}{1.674649in}}{\pgfqpoint{8.201278in}{1.672645in}}{\pgfqpoint{8.206322in}{1.672645in}}%
\pgfpathclose%
\pgfusepath{fill}%
\end{pgfscope}%
\begin{pgfscope}%
\pgfpathrectangle{\pgfqpoint{6.572727in}{0.474100in}}{\pgfqpoint{4.227273in}{3.318700in}}%
\pgfusepath{clip}%
\pgfsetbuttcap%
\pgfsetroundjoin%
\definecolor{currentfill}{rgb}{0.267004,0.004874,0.329415}%
\pgfsetfillcolor{currentfill}%
\pgfsetfillopacity{0.700000}%
\pgfsetlinewidth{0.000000pt}%
\definecolor{currentstroke}{rgb}{0.000000,0.000000,0.000000}%
\pgfsetstrokecolor{currentstroke}%
\pgfsetstrokeopacity{0.700000}%
\pgfsetdash{}{0pt}%
\pgfpathmoveto{\pgfqpoint{7.334725in}{1.880871in}}%
\pgfpathcurveto{\pgfqpoint{7.339769in}{1.880871in}}{\pgfqpoint{7.344607in}{1.882875in}}{\pgfqpoint{7.348173in}{1.886441in}}%
\pgfpathcurveto{\pgfqpoint{7.351740in}{1.890007in}}{\pgfqpoint{7.353744in}{1.894845in}}{\pgfqpoint{7.353744in}{1.899889in}}%
\pgfpathcurveto{\pgfqpoint{7.353744in}{1.904933in}}{\pgfqpoint{7.351740in}{1.909770in}}{\pgfqpoint{7.348173in}{1.913337in}}%
\pgfpathcurveto{\pgfqpoint{7.344607in}{1.916903in}}{\pgfqpoint{7.339769in}{1.918907in}}{\pgfqpoint{7.334725in}{1.918907in}}%
\pgfpathcurveto{\pgfqpoint{7.329682in}{1.918907in}}{\pgfqpoint{7.324844in}{1.916903in}}{\pgfqpoint{7.321278in}{1.913337in}}%
\pgfpathcurveto{\pgfqpoint{7.317711in}{1.909770in}}{\pgfqpoint{7.315707in}{1.904933in}}{\pgfqpoint{7.315707in}{1.899889in}}%
\pgfpathcurveto{\pgfqpoint{7.315707in}{1.894845in}}{\pgfqpoint{7.317711in}{1.890007in}}{\pgfqpoint{7.321278in}{1.886441in}}%
\pgfpathcurveto{\pgfqpoint{7.324844in}{1.882875in}}{\pgfqpoint{7.329682in}{1.880871in}}{\pgfqpoint{7.334725in}{1.880871in}}%
\pgfpathclose%
\pgfusepath{fill}%
\end{pgfscope}%
\begin{pgfscope}%
\pgfpathrectangle{\pgfqpoint{6.572727in}{0.474100in}}{\pgfqpoint{4.227273in}{3.318700in}}%
\pgfusepath{clip}%
\pgfsetbuttcap%
\pgfsetroundjoin%
\definecolor{currentfill}{rgb}{0.267004,0.004874,0.329415}%
\pgfsetfillcolor{currentfill}%
\pgfsetfillopacity{0.700000}%
\pgfsetlinewidth{0.000000pt}%
\definecolor{currentstroke}{rgb}{0.000000,0.000000,0.000000}%
\pgfsetstrokecolor{currentstroke}%
\pgfsetstrokeopacity{0.700000}%
\pgfsetdash{}{0pt}%
\pgfpathmoveto{\pgfqpoint{7.490342in}{1.400745in}}%
\pgfpathcurveto{\pgfqpoint{7.495385in}{1.400745in}}{\pgfqpoint{7.500223in}{1.402749in}}{\pgfqpoint{7.503790in}{1.406316in}}%
\pgfpathcurveto{\pgfqpoint{7.507356in}{1.409882in}}{\pgfqpoint{7.509360in}{1.414720in}}{\pgfqpoint{7.509360in}{1.419764in}}%
\pgfpathcurveto{\pgfqpoint{7.509360in}{1.424807in}}{\pgfqpoint{7.507356in}{1.429645in}}{\pgfqpoint{7.503790in}{1.433211in}}%
\pgfpathcurveto{\pgfqpoint{7.500223in}{1.436778in}}{\pgfqpoint{7.495385in}{1.438782in}}{\pgfqpoint{7.490342in}{1.438782in}}%
\pgfpathcurveto{\pgfqpoint{7.485298in}{1.438782in}}{\pgfqpoint{7.480460in}{1.436778in}}{\pgfqpoint{7.476894in}{1.433211in}}%
\pgfpathcurveto{\pgfqpoint{7.473327in}{1.429645in}}{\pgfqpoint{7.471324in}{1.424807in}}{\pgfqpoint{7.471324in}{1.419764in}}%
\pgfpathcurveto{\pgfqpoint{7.471324in}{1.414720in}}{\pgfqpoint{7.473327in}{1.409882in}}{\pgfqpoint{7.476894in}{1.406316in}}%
\pgfpathcurveto{\pgfqpoint{7.480460in}{1.402749in}}{\pgfqpoint{7.485298in}{1.400745in}}{\pgfqpoint{7.490342in}{1.400745in}}%
\pgfpathclose%
\pgfusepath{fill}%
\end{pgfscope}%
\begin{pgfscope}%
\pgfpathrectangle{\pgfqpoint{6.572727in}{0.474100in}}{\pgfqpoint{4.227273in}{3.318700in}}%
\pgfusepath{clip}%
\pgfsetbuttcap%
\pgfsetroundjoin%
\definecolor{currentfill}{rgb}{0.267004,0.004874,0.329415}%
\pgfsetfillcolor{currentfill}%
\pgfsetfillopacity{0.700000}%
\pgfsetlinewidth{0.000000pt}%
\definecolor{currentstroke}{rgb}{0.000000,0.000000,0.000000}%
\pgfsetstrokecolor{currentstroke}%
\pgfsetstrokeopacity{0.700000}%
\pgfsetdash{}{0pt}%
\pgfpathmoveto{\pgfqpoint{7.807482in}{1.485187in}}%
\pgfpathcurveto{\pgfqpoint{7.812525in}{1.485187in}}{\pgfqpoint{7.817363in}{1.487191in}}{\pgfqpoint{7.820929in}{1.490758in}}%
\pgfpathcurveto{\pgfqpoint{7.824496in}{1.494324in}}{\pgfqpoint{7.826500in}{1.499162in}}{\pgfqpoint{7.826500in}{1.504205in}}%
\pgfpathcurveto{\pgfqpoint{7.826500in}{1.509249in}}{\pgfqpoint{7.824496in}{1.514087in}}{\pgfqpoint{7.820929in}{1.517653in}}%
\pgfpathcurveto{\pgfqpoint{7.817363in}{1.521220in}}{\pgfqpoint{7.812525in}{1.523224in}}{\pgfqpoint{7.807482in}{1.523224in}}%
\pgfpathcurveto{\pgfqpoint{7.802438in}{1.523224in}}{\pgfqpoint{7.797600in}{1.521220in}}{\pgfqpoint{7.794034in}{1.517653in}}%
\pgfpathcurveto{\pgfqpoint{7.790467in}{1.514087in}}{\pgfqpoint{7.788463in}{1.509249in}}{\pgfqpoint{7.788463in}{1.504205in}}%
\pgfpathcurveto{\pgfqpoint{7.788463in}{1.499162in}}{\pgfqpoint{7.790467in}{1.494324in}}{\pgfqpoint{7.794034in}{1.490758in}}%
\pgfpathcurveto{\pgfqpoint{7.797600in}{1.487191in}}{\pgfqpoint{7.802438in}{1.485187in}}{\pgfqpoint{7.807482in}{1.485187in}}%
\pgfpathclose%
\pgfusepath{fill}%
\end{pgfscope}%
\begin{pgfscope}%
\pgfpathrectangle{\pgfqpoint{6.572727in}{0.474100in}}{\pgfqpoint{4.227273in}{3.318700in}}%
\pgfusepath{clip}%
\pgfsetbuttcap%
\pgfsetroundjoin%
\definecolor{currentfill}{rgb}{0.267004,0.004874,0.329415}%
\pgfsetfillcolor{currentfill}%
\pgfsetfillopacity{0.700000}%
\pgfsetlinewidth{0.000000pt}%
\definecolor{currentstroke}{rgb}{0.000000,0.000000,0.000000}%
\pgfsetstrokecolor{currentstroke}%
\pgfsetstrokeopacity{0.700000}%
\pgfsetdash{}{0pt}%
\pgfpathmoveto{\pgfqpoint{7.499513in}{2.138853in}}%
\pgfpathcurveto{\pgfqpoint{7.504557in}{2.138853in}}{\pgfqpoint{7.509394in}{2.140857in}}{\pgfqpoint{7.512961in}{2.144423in}}%
\pgfpathcurveto{\pgfqpoint{7.516527in}{2.147989in}}{\pgfqpoint{7.518531in}{2.152827in}}{\pgfqpoint{7.518531in}{2.157871in}}%
\pgfpathcurveto{\pgfqpoint{7.518531in}{2.162914in}}{\pgfqpoint{7.516527in}{2.167752in}}{\pgfqpoint{7.512961in}{2.171319in}}%
\pgfpathcurveto{\pgfqpoint{7.509394in}{2.174885in}}{\pgfqpoint{7.504557in}{2.176889in}}{\pgfqpoint{7.499513in}{2.176889in}}%
\pgfpathcurveto{\pgfqpoint{7.494469in}{2.176889in}}{\pgfqpoint{7.489632in}{2.174885in}}{\pgfqpoint{7.486065in}{2.171319in}}%
\pgfpathcurveto{\pgfqpoint{7.482499in}{2.167752in}}{\pgfqpoint{7.480495in}{2.162914in}}{\pgfqpoint{7.480495in}{2.157871in}}%
\pgfpathcurveto{\pgfqpoint{7.480495in}{2.152827in}}{\pgfqpoint{7.482499in}{2.147989in}}{\pgfqpoint{7.486065in}{2.144423in}}%
\pgfpathcurveto{\pgfqpoint{7.489632in}{2.140857in}}{\pgfqpoint{7.494469in}{2.138853in}}{\pgfqpoint{7.499513in}{2.138853in}}%
\pgfpathclose%
\pgfusepath{fill}%
\end{pgfscope}%
\begin{pgfscope}%
\pgfpathrectangle{\pgfqpoint{6.572727in}{0.474100in}}{\pgfqpoint{4.227273in}{3.318700in}}%
\pgfusepath{clip}%
\pgfsetbuttcap%
\pgfsetroundjoin%
\definecolor{currentfill}{rgb}{0.993248,0.906157,0.143936}%
\pgfsetfillcolor{currentfill}%
\pgfsetfillopacity{0.700000}%
\pgfsetlinewidth{0.000000pt}%
\definecolor{currentstroke}{rgb}{0.000000,0.000000,0.000000}%
\pgfsetstrokecolor{currentstroke}%
\pgfsetstrokeopacity{0.700000}%
\pgfsetdash{}{0pt}%
\pgfpathmoveto{\pgfqpoint{7.911405in}{2.831374in}}%
\pgfpathcurveto{\pgfqpoint{7.916448in}{2.831374in}}{\pgfqpoint{7.921286in}{2.833378in}}{\pgfqpoint{7.924853in}{2.836944in}}%
\pgfpathcurveto{\pgfqpoint{7.928419in}{2.840511in}}{\pgfqpoint{7.930423in}{2.845348in}}{\pgfqpoint{7.930423in}{2.850392in}}%
\pgfpathcurveto{\pgfqpoint{7.930423in}{2.855436in}}{\pgfqpoint{7.928419in}{2.860273in}}{\pgfqpoint{7.924853in}{2.863840in}}%
\pgfpathcurveto{\pgfqpoint{7.921286in}{2.867406in}}{\pgfqpoint{7.916448in}{2.869410in}}{\pgfqpoint{7.911405in}{2.869410in}}%
\pgfpathcurveto{\pgfqpoint{7.906361in}{2.869410in}}{\pgfqpoint{7.901523in}{2.867406in}}{\pgfqpoint{7.897957in}{2.863840in}}%
\pgfpathcurveto{\pgfqpoint{7.894390in}{2.860273in}}{\pgfqpoint{7.892387in}{2.855436in}}{\pgfqpoint{7.892387in}{2.850392in}}%
\pgfpathcurveto{\pgfqpoint{7.892387in}{2.845348in}}{\pgfqpoint{7.894390in}{2.840511in}}{\pgfqpoint{7.897957in}{2.836944in}}%
\pgfpathcurveto{\pgfqpoint{7.901523in}{2.833378in}}{\pgfqpoint{7.906361in}{2.831374in}}{\pgfqpoint{7.911405in}{2.831374in}}%
\pgfpathclose%
\pgfusepath{fill}%
\end{pgfscope}%
\begin{pgfscope}%
\pgfpathrectangle{\pgfqpoint{6.572727in}{0.474100in}}{\pgfqpoint{4.227273in}{3.318700in}}%
\pgfusepath{clip}%
\pgfsetbuttcap%
\pgfsetroundjoin%
\definecolor{currentfill}{rgb}{0.993248,0.906157,0.143936}%
\pgfsetfillcolor{currentfill}%
\pgfsetfillopacity{0.700000}%
\pgfsetlinewidth{0.000000pt}%
\definecolor{currentstroke}{rgb}{0.000000,0.000000,0.000000}%
\pgfsetstrokecolor{currentstroke}%
\pgfsetstrokeopacity{0.700000}%
\pgfsetdash{}{0pt}%
\pgfpathmoveto{\pgfqpoint{7.651751in}{2.742616in}}%
\pgfpathcurveto{\pgfqpoint{7.656794in}{2.742616in}}{\pgfqpoint{7.661632in}{2.744620in}}{\pgfqpoint{7.665198in}{2.748186in}}%
\pgfpathcurveto{\pgfqpoint{7.668765in}{2.751753in}}{\pgfqpoint{7.670769in}{2.756590in}}{\pgfqpoint{7.670769in}{2.761634in}}%
\pgfpathcurveto{\pgfqpoint{7.670769in}{2.766678in}}{\pgfqpoint{7.668765in}{2.771515in}}{\pgfqpoint{7.665198in}{2.775082in}}%
\pgfpathcurveto{\pgfqpoint{7.661632in}{2.778648in}}{\pgfqpoint{7.656794in}{2.780652in}}{\pgfqpoint{7.651751in}{2.780652in}}%
\pgfpathcurveto{\pgfqpoint{7.646707in}{2.780652in}}{\pgfqpoint{7.641869in}{2.778648in}}{\pgfqpoint{7.638303in}{2.775082in}}%
\pgfpathcurveto{\pgfqpoint{7.634736in}{2.771515in}}{\pgfqpoint{7.632732in}{2.766678in}}{\pgfqpoint{7.632732in}{2.761634in}}%
\pgfpathcurveto{\pgfqpoint{7.632732in}{2.756590in}}{\pgfqpoint{7.634736in}{2.751753in}}{\pgfqpoint{7.638303in}{2.748186in}}%
\pgfpathcurveto{\pgfqpoint{7.641869in}{2.744620in}}{\pgfqpoint{7.646707in}{2.742616in}}{\pgfqpoint{7.651751in}{2.742616in}}%
\pgfpathclose%
\pgfusepath{fill}%
\end{pgfscope}%
\begin{pgfscope}%
\pgfpathrectangle{\pgfqpoint{6.572727in}{0.474100in}}{\pgfqpoint{4.227273in}{3.318700in}}%
\pgfusepath{clip}%
\pgfsetbuttcap%
\pgfsetroundjoin%
\definecolor{currentfill}{rgb}{0.127568,0.566949,0.550556}%
\pgfsetfillcolor{currentfill}%
\pgfsetfillopacity{0.700000}%
\pgfsetlinewidth{0.000000pt}%
\definecolor{currentstroke}{rgb}{0.000000,0.000000,0.000000}%
\pgfsetstrokecolor{currentstroke}%
\pgfsetstrokeopacity{0.700000}%
\pgfsetdash{}{0pt}%
\pgfpathmoveto{\pgfqpoint{9.465287in}{1.342564in}}%
\pgfpathcurveto{\pgfqpoint{9.470330in}{1.342564in}}{\pgfqpoint{9.475168in}{1.344568in}}{\pgfqpoint{9.478735in}{1.348134in}}%
\pgfpathcurveto{\pgfqpoint{9.482301in}{1.351701in}}{\pgfqpoint{9.484305in}{1.356539in}}{\pgfqpoint{9.484305in}{1.361582in}}%
\pgfpathcurveto{\pgfqpoint{9.484305in}{1.366626in}}{\pgfqpoint{9.482301in}{1.371464in}}{\pgfqpoint{9.478735in}{1.375030in}}%
\pgfpathcurveto{\pgfqpoint{9.475168in}{1.378596in}}{\pgfqpoint{9.470330in}{1.380600in}}{\pgfqpoint{9.465287in}{1.380600in}}%
\pgfpathcurveto{\pgfqpoint{9.460243in}{1.380600in}}{\pgfqpoint{9.455405in}{1.378596in}}{\pgfqpoint{9.451839in}{1.375030in}}%
\pgfpathcurveto{\pgfqpoint{9.448273in}{1.371464in}}{\pgfqpoint{9.446269in}{1.366626in}}{\pgfqpoint{9.446269in}{1.361582in}}%
\pgfpathcurveto{\pgfqpoint{9.446269in}{1.356539in}}{\pgfqpoint{9.448273in}{1.351701in}}{\pgfqpoint{9.451839in}{1.348134in}}%
\pgfpathcurveto{\pgfqpoint{9.455405in}{1.344568in}}{\pgfqpoint{9.460243in}{1.342564in}}{\pgfqpoint{9.465287in}{1.342564in}}%
\pgfpathclose%
\pgfusepath{fill}%
\end{pgfscope}%
\begin{pgfscope}%
\pgfpathrectangle{\pgfqpoint{6.572727in}{0.474100in}}{\pgfqpoint{4.227273in}{3.318700in}}%
\pgfusepath{clip}%
\pgfsetbuttcap%
\pgfsetroundjoin%
\definecolor{currentfill}{rgb}{0.267004,0.004874,0.329415}%
\pgfsetfillcolor{currentfill}%
\pgfsetfillopacity{0.700000}%
\pgfsetlinewidth{0.000000pt}%
\definecolor{currentstroke}{rgb}{0.000000,0.000000,0.000000}%
\pgfsetstrokecolor{currentstroke}%
\pgfsetstrokeopacity{0.700000}%
\pgfsetdash{}{0pt}%
\pgfpathmoveto{\pgfqpoint{8.266369in}{1.547802in}}%
\pgfpathcurveto{\pgfqpoint{8.271412in}{1.547802in}}{\pgfqpoint{8.276250in}{1.549806in}}{\pgfqpoint{8.279816in}{1.553373in}}%
\pgfpathcurveto{\pgfqpoint{8.283383in}{1.556939in}}{\pgfqpoint{8.285387in}{1.561777in}}{\pgfqpoint{8.285387in}{1.566821in}}%
\pgfpathcurveto{\pgfqpoint{8.285387in}{1.571864in}}{\pgfqpoint{8.283383in}{1.576702in}}{\pgfqpoint{8.279816in}{1.580268in}}%
\pgfpathcurveto{\pgfqpoint{8.276250in}{1.583835in}}{\pgfqpoint{8.271412in}{1.585839in}}{\pgfqpoint{8.266369in}{1.585839in}}%
\pgfpathcurveto{\pgfqpoint{8.261325in}{1.585839in}}{\pgfqpoint{8.256487in}{1.583835in}}{\pgfqpoint{8.252921in}{1.580268in}}%
\pgfpathcurveto{\pgfqpoint{8.249354in}{1.576702in}}{\pgfqpoint{8.247350in}{1.571864in}}{\pgfqpoint{8.247350in}{1.566821in}}%
\pgfpathcurveto{\pgfqpoint{8.247350in}{1.561777in}}{\pgfqpoint{8.249354in}{1.556939in}}{\pgfqpoint{8.252921in}{1.553373in}}%
\pgfpathcurveto{\pgfqpoint{8.256487in}{1.549806in}}{\pgfqpoint{8.261325in}{1.547802in}}{\pgfqpoint{8.266369in}{1.547802in}}%
\pgfpathclose%
\pgfusepath{fill}%
\end{pgfscope}%
\begin{pgfscope}%
\pgfpathrectangle{\pgfqpoint{6.572727in}{0.474100in}}{\pgfqpoint{4.227273in}{3.318700in}}%
\pgfusepath{clip}%
\pgfsetbuttcap%
\pgfsetroundjoin%
\definecolor{currentfill}{rgb}{0.267004,0.004874,0.329415}%
\pgfsetfillcolor{currentfill}%
\pgfsetfillopacity{0.700000}%
\pgfsetlinewidth{0.000000pt}%
\definecolor{currentstroke}{rgb}{0.000000,0.000000,0.000000}%
\pgfsetstrokecolor{currentstroke}%
\pgfsetstrokeopacity{0.700000}%
\pgfsetdash{}{0pt}%
\pgfpathmoveto{\pgfqpoint{7.877706in}{1.910149in}}%
\pgfpathcurveto{\pgfqpoint{7.882750in}{1.910149in}}{\pgfqpoint{7.887588in}{1.912153in}}{\pgfqpoint{7.891154in}{1.915720in}}%
\pgfpathcurveto{\pgfqpoint{7.894721in}{1.919286in}}{\pgfqpoint{7.896725in}{1.924124in}}{\pgfqpoint{7.896725in}{1.929168in}}%
\pgfpathcurveto{\pgfqpoint{7.896725in}{1.934211in}}{\pgfqpoint{7.894721in}{1.939049in}}{\pgfqpoint{7.891154in}{1.942615in}}%
\pgfpathcurveto{\pgfqpoint{7.887588in}{1.946182in}}{\pgfqpoint{7.882750in}{1.948186in}}{\pgfqpoint{7.877706in}{1.948186in}}%
\pgfpathcurveto{\pgfqpoint{7.872663in}{1.948186in}}{\pgfqpoint{7.867825in}{1.946182in}}{\pgfqpoint{7.864259in}{1.942615in}}%
\pgfpathcurveto{\pgfqpoint{7.860692in}{1.939049in}}{\pgfqpoint{7.858688in}{1.934211in}}{\pgfqpoint{7.858688in}{1.929168in}}%
\pgfpathcurveto{\pgfqpoint{7.858688in}{1.924124in}}{\pgfqpoint{7.860692in}{1.919286in}}{\pgfqpoint{7.864259in}{1.915720in}}%
\pgfpathcurveto{\pgfqpoint{7.867825in}{1.912153in}}{\pgfqpoint{7.872663in}{1.910149in}}{\pgfqpoint{7.877706in}{1.910149in}}%
\pgfpathclose%
\pgfusepath{fill}%
\end{pgfscope}%
\begin{pgfscope}%
\pgfpathrectangle{\pgfqpoint{6.572727in}{0.474100in}}{\pgfqpoint{4.227273in}{3.318700in}}%
\pgfusepath{clip}%
\pgfsetbuttcap%
\pgfsetroundjoin%
\definecolor{currentfill}{rgb}{0.267004,0.004874,0.329415}%
\pgfsetfillcolor{currentfill}%
\pgfsetfillopacity{0.700000}%
\pgfsetlinewidth{0.000000pt}%
\definecolor{currentstroke}{rgb}{0.000000,0.000000,0.000000}%
\pgfsetstrokecolor{currentstroke}%
\pgfsetstrokeopacity{0.700000}%
\pgfsetdash{}{0pt}%
\pgfpathmoveto{\pgfqpoint{8.063057in}{1.847324in}}%
\pgfpathcurveto{\pgfqpoint{8.068101in}{1.847324in}}{\pgfqpoint{8.072939in}{1.849328in}}{\pgfqpoint{8.076505in}{1.852895in}}%
\pgfpathcurveto{\pgfqpoint{8.080072in}{1.856461in}}{\pgfqpoint{8.082075in}{1.861299in}}{\pgfqpoint{8.082075in}{1.866343in}}%
\pgfpathcurveto{\pgfqpoint{8.082075in}{1.871386in}}{\pgfqpoint{8.080072in}{1.876224in}}{\pgfqpoint{8.076505in}{1.879790in}}%
\pgfpathcurveto{\pgfqpoint{8.072939in}{1.883357in}}{\pgfqpoint{8.068101in}{1.885361in}}{\pgfqpoint{8.063057in}{1.885361in}}%
\pgfpathcurveto{\pgfqpoint{8.058014in}{1.885361in}}{\pgfqpoint{8.053176in}{1.883357in}}{\pgfqpoint{8.049609in}{1.879790in}}%
\pgfpathcurveto{\pgfqpoint{8.046043in}{1.876224in}}{\pgfqpoint{8.044039in}{1.871386in}}{\pgfqpoint{8.044039in}{1.866343in}}%
\pgfpathcurveto{\pgfqpoint{8.044039in}{1.861299in}}{\pgfqpoint{8.046043in}{1.856461in}}{\pgfqpoint{8.049609in}{1.852895in}}%
\pgfpathcurveto{\pgfqpoint{8.053176in}{1.849328in}}{\pgfqpoint{8.058014in}{1.847324in}}{\pgfqpoint{8.063057in}{1.847324in}}%
\pgfpathclose%
\pgfusepath{fill}%
\end{pgfscope}%
\begin{pgfscope}%
\pgfpathrectangle{\pgfqpoint{6.572727in}{0.474100in}}{\pgfqpoint{4.227273in}{3.318700in}}%
\pgfusepath{clip}%
\pgfsetbuttcap%
\pgfsetroundjoin%
\definecolor{currentfill}{rgb}{0.993248,0.906157,0.143936}%
\pgfsetfillcolor{currentfill}%
\pgfsetfillopacity{0.700000}%
\pgfsetlinewidth{0.000000pt}%
\definecolor{currentstroke}{rgb}{0.000000,0.000000,0.000000}%
\pgfsetstrokecolor{currentstroke}%
\pgfsetstrokeopacity{0.700000}%
\pgfsetdash{}{0pt}%
\pgfpathmoveto{\pgfqpoint{7.997701in}{2.693385in}}%
\pgfpathcurveto{\pgfqpoint{8.002744in}{2.693385in}}{\pgfqpoint{8.007582in}{2.695388in}}{\pgfqpoint{8.011148in}{2.698955in}}%
\pgfpathcurveto{\pgfqpoint{8.014715in}{2.702521in}}{\pgfqpoint{8.016719in}{2.707359in}}{\pgfqpoint{8.016719in}{2.712403in}}%
\pgfpathcurveto{\pgfqpoint{8.016719in}{2.717446in}}{\pgfqpoint{8.014715in}{2.722284in}}{\pgfqpoint{8.011148in}{2.725851in}}%
\pgfpathcurveto{\pgfqpoint{8.007582in}{2.729417in}}{\pgfqpoint{8.002744in}{2.731421in}}{\pgfqpoint{7.997701in}{2.731421in}}%
\pgfpathcurveto{\pgfqpoint{7.992657in}{2.731421in}}{\pgfqpoint{7.987819in}{2.729417in}}{\pgfqpoint{7.984253in}{2.725851in}}%
\pgfpathcurveto{\pgfqpoint{7.980686in}{2.722284in}}{\pgfqpoint{7.978682in}{2.717446in}}{\pgfqpoint{7.978682in}{2.712403in}}%
\pgfpathcurveto{\pgfqpoint{7.978682in}{2.707359in}}{\pgfqpoint{7.980686in}{2.702521in}}{\pgfqpoint{7.984253in}{2.698955in}}%
\pgfpathcurveto{\pgfqpoint{7.987819in}{2.695388in}}{\pgfqpoint{7.992657in}{2.693385in}}{\pgfqpoint{7.997701in}{2.693385in}}%
\pgfpathclose%
\pgfusepath{fill}%
\end{pgfscope}%
\begin{pgfscope}%
\pgfpathrectangle{\pgfqpoint{6.572727in}{0.474100in}}{\pgfqpoint{4.227273in}{3.318700in}}%
\pgfusepath{clip}%
\pgfsetbuttcap%
\pgfsetroundjoin%
\definecolor{currentfill}{rgb}{0.127568,0.566949,0.550556}%
\pgfsetfillcolor{currentfill}%
\pgfsetfillopacity{0.700000}%
\pgfsetlinewidth{0.000000pt}%
\definecolor{currentstroke}{rgb}{0.000000,0.000000,0.000000}%
\pgfsetstrokecolor{currentstroke}%
\pgfsetstrokeopacity{0.700000}%
\pgfsetdash{}{0pt}%
\pgfpathmoveto{\pgfqpoint{9.206261in}{1.890701in}}%
\pgfpathcurveto{\pgfqpoint{9.211304in}{1.890701in}}{\pgfqpoint{9.216142in}{1.892704in}}{\pgfqpoint{9.219708in}{1.896271in}}%
\pgfpathcurveto{\pgfqpoint{9.223275in}{1.899837in}}{\pgfqpoint{9.225279in}{1.904675in}}{\pgfqpoint{9.225279in}{1.909719in}}%
\pgfpathcurveto{\pgfqpoint{9.225279in}{1.914762in}}{\pgfqpoint{9.223275in}{1.919600in}}{\pgfqpoint{9.219708in}{1.923167in}}%
\pgfpathcurveto{\pgfqpoint{9.216142in}{1.926733in}}{\pgfqpoint{9.211304in}{1.928737in}}{\pgfqpoint{9.206261in}{1.928737in}}%
\pgfpathcurveto{\pgfqpoint{9.201217in}{1.928737in}}{\pgfqpoint{9.196379in}{1.926733in}}{\pgfqpoint{9.192813in}{1.923167in}}%
\pgfpathcurveto{\pgfqpoint{9.189246in}{1.919600in}}{\pgfqpoint{9.187242in}{1.914762in}}{\pgfqpoint{9.187242in}{1.909719in}}%
\pgfpathcurveto{\pgfqpoint{9.187242in}{1.904675in}}{\pgfqpoint{9.189246in}{1.899837in}}{\pgfqpoint{9.192813in}{1.896271in}}%
\pgfpathcurveto{\pgfqpoint{9.196379in}{1.892704in}}{\pgfqpoint{9.201217in}{1.890701in}}{\pgfqpoint{9.206261in}{1.890701in}}%
\pgfpathclose%
\pgfusepath{fill}%
\end{pgfscope}%
\begin{pgfscope}%
\pgfpathrectangle{\pgfqpoint{6.572727in}{0.474100in}}{\pgfqpoint{4.227273in}{3.318700in}}%
\pgfusepath{clip}%
\pgfsetbuttcap%
\pgfsetroundjoin%
\definecolor{currentfill}{rgb}{0.267004,0.004874,0.329415}%
\pgfsetfillcolor{currentfill}%
\pgfsetfillopacity{0.700000}%
\pgfsetlinewidth{0.000000pt}%
\definecolor{currentstroke}{rgb}{0.000000,0.000000,0.000000}%
\pgfsetstrokecolor{currentstroke}%
\pgfsetstrokeopacity{0.700000}%
\pgfsetdash{}{0pt}%
\pgfpathmoveto{\pgfqpoint{7.603169in}{1.531369in}}%
\pgfpathcurveto{\pgfqpoint{7.608213in}{1.531369in}}{\pgfqpoint{7.613051in}{1.533373in}}{\pgfqpoint{7.616617in}{1.536939in}}%
\pgfpathcurveto{\pgfqpoint{7.620184in}{1.540506in}}{\pgfqpoint{7.622188in}{1.545343in}}{\pgfqpoint{7.622188in}{1.550387in}}%
\pgfpathcurveto{\pgfqpoint{7.622188in}{1.555431in}}{\pgfqpoint{7.620184in}{1.560268in}}{\pgfqpoint{7.616617in}{1.563835in}}%
\pgfpathcurveto{\pgfqpoint{7.613051in}{1.567401in}}{\pgfqpoint{7.608213in}{1.569405in}}{\pgfqpoint{7.603169in}{1.569405in}}%
\pgfpathcurveto{\pgfqpoint{7.598126in}{1.569405in}}{\pgfqpoint{7.593288in}{1.567401in}}{\pgfqpoint{7.589722in}{1.563835in}}%
\pgfpathcurveto{\pgfqpoint{7.586155in}{1.560268in}}{\pgfqpoint{7.584151in}{1.555431in}}{\pgfqpoint{7.584151in}{1.550387in}}%
\pgfpathcurveto{\pgfqpoint{7.584151in}{1.545343in}}{\pgfqpoint{7.586155in}{1.540506in}}{\pgfqpoint{7.589722in}{1.536939in}}%
\pgfpathcurveto{\pgfqpoint{7.593288in}{1.533373in}}{\pgfqpoint{7.598126in}{1.531369in}}{\pgfqpoint{7.603169in}{1.531369in}}%
\pgfpathclose%
\pgfusepath{fill}%
\end{pgfscope}%
\begin{pgfscope}%
\pgfpathrectangle{\pgfqpoint{6.572727in}{0.474100in}}{\pgfqpoint{4.227273in}{3.318700in}}%
\pgfusepath{clip}%
\pgfsetbuttcap%
\pgfsetroundjoin%
\definecolor{currentfill}{rgb}{0.993248,0.906157,0.143936}%
\pgfsetfillcolor{currentfill}%
\pgfsetfillopacity{0.700000}%
\pgfsetlinewidth{0.000000pt}%
\definecolor{currentstroke}{rgb}{0.000000,0.000000,0.000000}%
\pgfsetstrokecolor{currentstroke}%
\pgfsetstrokeopacity{0.700000}%
\pgfsetdash{}{0pt}%
\pgfpathmoveto{\pgfqpoint{8.624924in}{2.479891in}}%
\pgfpathcurveto{\pgfqpoint{8.629968in}{2.479891in}}{\pgfqpoint{8.634806in}{2.481895in}}{\pgfqpoint{8.638372in}{2.485461in}}%
\pgfpathcurveto{\pgfqpoint{8.641938in}{2.489028in}}{\pgfqpoint{8.643942in}{2.493866in}}{\pgfqpoint{8.643942in}{2.498909in}}%
\pgfpathcurveto{\pgfqpoint{8.643942in}{2.503953in}}{\pgfqpoint{8.641938in}{2.508791in}}{\pgfqpoint{8.638372in}{2.512357in}}%
\pgfpathcurveto{\pgfqpoint{8.634806in}{2.515924in}}{\pgfqpoint{8.629968in}{2.517927in}}{\pgfqpoint{8.624924in}{2.517927in}}%
\pgfpathcurveto{\pgfqpoint{8.619880in}{2.517927in}}{\pgfqpoint{8.615043in}{2.515924in}}{\pgfqpoint{8.611476in}{2.512357in}}%
\pgfpathcurveto{\pgfqpoint{8.607910in}{2.508791in}}{\pgfqpoint{8.605906in}{2.503953in}}{\pgfqpoint{8.605906in}{2.498909in}}%
\pgfpathcurveto{\pgfqpoint{8.605906in}{2.493866in}}{\pgfqpoint{8.607910in}{2.489028in}}{\pgfqpoint{8.611476in}{2.485461in}}%
\pgfpathcurveto{\pgfqpoint{8.615043in}{2.481895in}}{\pgfqpoint{8.619880in}{2.479891in}}{\pgfqpoint{8.624924in}{2.479891in}}%
\pgfpathclose%
\pgfusepath{fill}%
\end{pgfscope}%
\begin{pgfscope}%
\pgfpathrectangle{\pgfqpoint{6.572727in}{0.474100in}}{\pgfqpoint{4.227273in}{3.318700in}}%
\pgfusepath{clip}%
\pgfsetbuttcap%
\pgfsetroundjoin%
\definecolor{currentfill}{rgb}{0.127568,0.566949,0.550556}%
\pgfsetfillcolor{currentfill}%
\pgfsetfillopacity{0.700000}%
\pgfsetlinewidth{0.000000pt}%
\definecolor{currentstroke}{rgb}{0.000000,0.000000,0.000000}%
\pgfsetstrokecolor{currentstroke}%
\pgfsetstrokeopacity{0.700000}%
\pgfsetdash{}{0pt}%
\pgfpathmoveto{\pgfqpoint{9.678243in}{1.470786in}}%
\pgfpathcurveto{\pgfqpoint{9.683287in}{1.470786in}}{\pgfqpoint{9.688125in}{1.472789in}}{\pgfqpoint{9.691691in}{1.476356in}}%
\pgfpathcurveto{\pgfqpoint{9.695257in}{1.479922in}}{\pgfqpoint{9.697261in}{1.484760in}}{\pgfqpoint{9.697261in}{1.489804in}}%
\pgfpathcurveto{\pgfqpoint{9.697261in}{1.494847in}}{\pgfqpoint{9.695257in}{1.499685in}}{\pgfqpoint{9.691691in}{1.503252in}}%
\pgfpathcurveto{\pgfqpoint{9.688125in}{1.506818in}}{\pgfqpoint{9.683287in}{1.508822in}}{\pgfqpoint{9.678243in}{1.508822in}}%
\pgfpathcurveto{\pgfqpoint{9.673199in}{1.508822in}}{\pgfqpoint{9.668362in}{1.506818in}}{\pgfqpoint{9.664795in}{1.503252in}}%
\pgfpathcurveto{\pgfqpoint{9.661229in}{1.499685in}}{\pgfqpoint{9.659225in}{1.494847in}}{\pgfqpoint{9.659225in}{1.489804in}}%
\pgfpathcurveto{\pgfqpoint{9.659225in}{1.484760in}}{\pgfqpoint{9.661229in}{1.479922in}}{\pgfqpoint{9.664795in}{1.476356in}}%
\pgfpathcurveto{\pgfqpoint{9.668362in}{1.472789in}}{\pgfqpoint{9.673199in}{1.470786in}}{\pgfqpoint{9.678243in}{1.470786in}}%
\pgfpathclose%
\pgfusepath{fill}%
\end{pgfscope}%
\begin{pgfscope}%
\pgfpathrectangle{\pgfqpoint{6.572727in}{0.474100in}}{\pgfqpoint{4.227273in}{3.318700in}}%
\pgfusepath{clip}%
\pgfsetbuttcap%
\pgfsetroundjoin%
\definecolor{currentfill}{rgb}{0.127568,0.566949,0.550556}%
\pgfsetfillcolor{currentfill}%
\pgfsetfillopacity{0.700000}%
\pgfsetlinewidth{0.000000pt}%
\definecolor{currentstroke}{rgb}{0.000000,0.000000,0.000000}%
\pgfsetstrokecolor{currentstroke}%
\pgfsetstrokeopacity{0.700000}%
\pgfsetdash{}{0pt}%
\pgfpathmoveto{\pgfqpoint{10.026416in}{1.510231in}}%
\pgfpathcurveto{\pgfqpoint{10.031459in}{1.510231in}}{\pgfqpoint{10.036297in}{1.512235in}}{\pgfqpoint{10.039863in}{1.515802in}}%
\pgfpathcurveto{\pgfqpoint{10.043430in}{1.519368in}}{\pgfqpoint{10.045434in}{1.524206in}}{\pgfqpoint{10.045434in}{1.529250in}}%
\pgfpathcurveto{\pgfqpoint{10.045434in}{1.534293in}}{\pgfqpoint{10.043430in}{1.539131in}}{\pgfqpoint{10.039863in}{1.542697in}}%
\pgfpathcurveto{\pgfqpoint{10.036297in}{1.546264in}}{\pgfqpoint{10.031459in}{1.548268in}}{\pgfqpoint{10.026416in}{1.548268in}}%
\pgfpathcurveto{\pgfqpoint{10.021372in}{1.548268in}}{\pgfqpoint{10.016534in}{1.546264in}}{\pgfqpoint{10.012968in}{1.542697in}}%
\pgfpathcurveto{\pgfqpoint{10.009401in}{1.539131in}}{\pgfqpoint{10.007397in}{1.534293in}}{\pgfqpoint{10.007397in}{1.529250in}}%
\pgfpathcurveto{\pgfqpoint{10.007397in}{1.524206in}}{\pgfqpoint{10.009401in}{1.519368in}}{\pgfqpoint{10.012968in}{1.515802in}}%
\pgfpathcurveto{\pgfqpoint{10.016534in}{1.512235in}}{\pgfqpoint{10.021372in}{1.510231in}}{\pgfqpoint{10.026416in}{1.510231in}}%
\pgfpathclose%
\pgfusepath{fill}%
\end{pgfscope}%
\begin{pgfscope}%
\pgfpathrectangle{\pgfqpoint{6.572727in}{0.474100in}}{\pgfqpoint{4.227273in}{3.318700in}}%
\pgfusepath{clip}%
\pgfsetbuttcap%
\pgfsetroundjoin%
\definecolor{currentfill}{rgb}{0.127568,0.566949,0.550556}%
\pgfsetfillcolor{currentfill}%
\pgfsetfillopacity{0.700000}%
\pgfsetlinewidth{0.000000pt}%
\definecolor{currentstroke}{rgb}{0.000000,0.000000,0.000000}%
\pgfsetstrokecolor{currentstroke}%
\pgfsetstrokeopacity{0.700000}%
\pgfsetdash{}{0pt}%
\pgfpathmoveto{\pgfqpoint{9.879733in}{1.456269in}}%
\pgfpathcurveto{\pgfqpoint{9.884777in}{1.456269in}}{\pgfqpoint{9.889615in}{1.458273in}}{\pgfqpoint{9.893181in}{1.461840in}}%
\pgfpathcurveto{\pgfqpoint{9.896748in}{1.465406in}}{\pgfqpoint{9.898752in}{1.470244in}}{\pgfqpoint{9.898752in}{1.475287in}}%
\pgfpathcurveto{\pgfqpoint{9.898752in}{1.480331in}}{\pgfqpoint{9.896748in}{1.485169in}}{\pgfqpoint{9.893181in}{1.488735in}}%
\pgfpathcurveto{\pgfqpoint{9.889615in}{1.492302in}}{\pgfqpoint{9.884777in}{1.494306in}}{\pgfqpoint{9.879733in}{1.494306in}}%
\pgfpathcurveto{\pgfqpoint{9.874690in}{1.494306in}}{\pgfqpoint{9.869852in}{1.492302in}}{\pgfqpoint{9.866286in}{1.488735in}}%
\pgfpathcurveto{\pgfqpoint{9.862719in}{1.485169in}}{\pgfqpoint{9.860715in}{1.480331in}}{\pgfqpoint{9.860715in}{1.475287in}}%
\pgfpathcurveto{\pgfqpoint{9.860715in}{1.470244in}}{\pgfqpoint{9.862719in}{1.465406in}}{\pgfqpoint{9.866286in}{1.461840in}}%
\pgfpathcurveto{\pgfqpoint{9.869852in}{1.458273in}}{\pgfqpoint{9.874690in}{1.456269in}}{\pgfqpoint{9.879733in}{1.456269in}}%
\pgfpathclose%
\pgfusepath{fill}%
\end{pgfscope}%
\begin{pgfscope}%
\pgfpathrectangle{\pgfqpoint{6.572727in}{0.474100in}}{\pgfqpoint{4.227273in}{3.318700in}}%
\pgfusepath{clip}%
\pgfsetbuttcap%
\pgfsetroundjoin%
\definecolor{currentfill}{rgb}{0.267004,0.004874,0.329415}%
\pgfsetfillcolor{currentfill}%
\pgfsetfillopacity{0.700000}%
\pgfsetlinewidth{0.000000pt}%
\definecolor{currentstroke}{rgb}{0.000000,0.000000,0.000000}%
\pgfsetstrokecolor{currentstroke}%
\pgfsetstrokeopacity{0.700000}%
\pgfsetdash{}{0pt}%
\pgfpathmoveto{\pgfqpoint{8.054616in}{1.430629in}}%
\pgfpathcurveto{\pgfqpoint{8.059660in}{1.430629in}}{\pgfqpoint{8.064497in}{1.432633in}}{\pgfqpoint{8.068064in}{1.436199in}}%
\pgfpathcurveto{\pgfqpoint{8.071630in}{1.439765in}}{\pgfqpoint{8.073634in}{1.444603in}}{\pgfqpoint{8.073634in}{1.449647in}}%
\pgfpathcurveto{\pgfqpoint{8.073634in}{1.454691in}}{\pgfqpoint{8.071630in}{1.459528in}}{\pgfqpoint{8.068064in}{1.463095in}}%
\pgfpathcurveto{\pgfqpoint{8.064497in}{1.466661in}}{\pgfqpoint{8.059660in}{1.468665in}}{\pgfqpoint{8.054616in}{1.468665in}}%
\pgfpathcurveto{\pgfqpoint{8.049572in}{1.468665in}}{\pgfqpoint{8.044734in}{1.466661in}}{\pgfqpoint{8.041168in}{1.463095in}}%
\pgfpathcurveto{\pgfqpoint{8.037602in}{1.459528in}}{\pgfqpoint{8.035598in}{1.454691in}}{\pgfqpoint{8.035598in}{1.449647in}}%
\pgfpathcurveto{\pgfqpoint{8.035598in}{1.444603in}}{\pgfqpoint{8.037602in}{1.439765in}}{\pgfqpoint{8.041168in}{1.436199in}}%
\pgfpathcurveto{\pgfqpoint{8.044734in}{1.432633in}}{\pgfqpoint{8.049572in}{1.430629in}}{\pgfqpoint{8.054616in}{1.430629in}}%
\pgfpathclose%
\pgfusepath{fill}%
\end{pgfscope}%
\begin{pgfscope}%
\pgfpathrectangle{\pgfqpoint{6.572727in}{0.474100in}}{\pgfqpoint{4.227273in}{3.318700in}}%
\pgfusepath{clip}%
\pgfsetbuttcap%
\pgfsetroundjoin%
\definecolor{currentfill}{rgb}{0.993248,0.906157,0.143936}%
\pgfsetfillcolor{currentfill}%
\pgfsetfillopacity{0.700000}%
\pgfsetlinewidth{0.000000pt}%
\definecolor{currentstroke}{rgb}{0.000000,0.000000,0.000000}%
\pgfsetstrokecolor{currentstroke}%
\pgfsetstrokeopacity{0.700000}%
\pgfsetdash{}{0pt}%
\pgfpathmoveto{\pgfqpoint{8.951433in}{3.310421in}}%
\pgfpathcurveto{\pgfqpoint{8.956477in}{3.310421in}}{\pgfqpoint{8.961314in}{3.312425in}}{\pgfqpoint{8.964881in}{3.315992in}}%
\pgfpathcurveto{\pgfqpoint{8.968447in}{3.319558in}}{\pgfqpoint{8.970451in}{3.324396in}}{\pgfqpoint{8.970451in}{3.329439in}}%
\pgfpathcurveto{\pgfqpoint{8.970451in}{3.334483in}}{\pgfqpoint{8.968447in}{3.339321in}}{\pgfqpoint{8.964881in}{3.342887in}}%
\pgfpathcurveto{\pgfqpoint{8.961314in}{3.346454in}}{\pgfqpoint{8.956477in}{3.348458in}}{\pgfqpoint{8.951433in}{3.348458in}}%
\pgfpathcurveto{\pgfqpoint{8.946389in}{3.348458in}}{\pgfqpoint{8.941552in}{3.346454in}}{\pgfqpoint{8.937985in}{3.342887in}}%
\pgfpathcurveto{\pgfqpoint{8.934419in}{3.339321in}}{\pgfqpoint{8.932415in}{3.334483in}}{\pgfqpoint{8.932415in}{3.329439in}}%
\pgfpathcurveto{\pgfqpoint{8.932415in}{3.324396in}}{\pgfqpoint{8.934419in}{3.319558in}}{\pgfqpoint{8.937985in}{3.315992in}}%
\pgfpathcurveto{\pgfqpoint{8.941552in}{3.312425in}}{\pgfqpoint{8.946389in}{3.310421in}}{\pgfqpoint{8.951433in}{3.310421in}}%
\pgfpathclose%
\pgfusepath{fill}%
\end{pgfscope}%
\begin{pgfscope}%
\pgfpathrectangle{\pgfqpoint{6.572727in}{0.474100in}}{\pgfqpoint{4.227273in}{3.318700in}}%
\pgfusepath{clip}%
\pgfsetbuttcap%
\pgfsetroundjoin%
\definecolor{currentfill}{rgb}{0.267004,0.004874,0.329415}%
\pgfsetfillcolor{currentfill}%
\pgfsetfillopacity{0.700000}%
\pgfsetlinewidth{0.000000pt}%
\definecolor{currentstroke}{rgb}{0.000000,0.000000,0.000000}%
\pgfsetstrokecolor{currentstroke}%
\pgfsetstrokeopacity{0.700000}%
\pgfsetdash{}{0pt}%
\pgfpathmoveto{\pgfqpoint{8.088472in}{1.734575in}}%
\pgfpathcurveto{\pgfqpoint{8.093516in}{1.734575in}}{\pgfqpoint{8.098354in}{1.736579in}}{\pgfqpoint{8.101920in}{1.740146in}}%
\pgfpathcurveto{\pgfqpoint{8.105486in}{1.743712in}}{\pgfqpoint{8.107490in}{1.748550in}}{\pgfqpoint{8.107490in}{1.753594in}}%
\pgfpathcurveto{\pgfqpoint{8.107490in}{1.758637in}}{\pgfqpoint{8.105486in}{1.763475in}}{\pgfqpoint{8.101920in}{1.767041in}}%
\pgfpathcurveto{\pgfqpoint{8.098354in}{1.770608in}}{\pgfqpoint{8.093516in}{1.772612in}}{\pgfqpoint{8.088472in}{1.772612in}}%
\pgfpathcurveto{\pgfqpoint{8.083428in}{1.772612in}}{\pgfqpoint{8.078591in}{1.770608in}}{\pgfqpoint{8.075024in}{1.767041in}}%
\pgfpathcurveto{\pgfqpoint{8.071458in}{1.763475in}}{\pgfqpoint{8.069454in}{1.758637in}}{\pgfqpoint{8.069454in}{1.753594in}}%
\pgfpathcurveto{\pgfqpoint{8.069454in}{1.748550in}}{\pgfqpoint{8.071458in}{1.743712in}}{\pgfqpoint{8.075024in}{1.740146in}}%
\pgfpathcurveto{\pgfqpoint{8.078591in}{1.736579in}}{\pgfqpoint{8.083428in}{1.734575in}}{\pgfqpoint{8.088472in}{1.734575in}}%
\pgfpathclose%
\pgfusepath{fill}%
\end{pgfscope}%
\begin{pgfscope}%
\pgfpathrectangle{\pgfqpoint{6.572727in}{0.474100in}}{\pgfqpoint{4.227273in}{3.318700in}}%
\pgfusepath{clip}%
\pgfsetbuttcap%
\pgfsetroundjoin%
\definecolor{currentfill}{rgb}{0.127568,0.566949,0.550556}%
\pgfsetfillcolor{currentfill}%
\pgfsetfillopacity{0.700000}%
\pgfsetlinewidth{0.000000pt}%
\definecolor{currentstroke}{rgb}{0.000000,0.000000,0.000000}%
\pgfsetstrokecolor{currentstroke}%
\pgfsetstrokeopacity{0.700000}%
\pgfsetdash{}{0pt}%
\pgfpathmoveto{\pgfqpoint{9.027756in}{1.385084in}}%
\pgfpathcurveto{\pgfqpoint{9.032799in}{1.385084in}}{\pgfqpoint{9.037637in}{1.387088in}}{\pgfqpoint{9.041204in}{1.390655in}}%
\pgfpathcurveto{\pgfqpoint{9.044770in}{1.394221in}}{\pgfqpoint{9.046774in}{1.399059in}}{\pgfqpoint{9.046774in}{1.404103in}}%
\pgfpathcurveto{\pgfqpoint{9.046774in}{1.409146in}}{\pgfqpoint{9.044770in}{1.413984in}}{\pgfqpoint{9.041204in}{1.417550in}}%
\pgfpathcurveto{\pgfqpoint{9.037637in}{1.421117in}}{\pgfqpoint{9.032799in}{1.423121in}}{\pgfqpoint{9.027756in}{1.423121in}}%
\pgfpathcurveto{\pgfqpoint{9.022712in}{1.423121in}}{\pgfqpoint{9.017874in}{1.421117in}}{\pgfqpoint{9.014308in}{1.417550in}}%
\pgfpathcurveto{\pgfqpoint{9.010742in}{1.413984in}}{\pgfqpoint{9.008738in}{1.409146in}}{\pgfqpoint{9.008738in}{1.404103in}}%
\pgfpathcurveto{\pgfqpoint{9.008738in}{1.399059in}}{\pgfqpoint{9.010742in}{1.394221in}}{\pgfqpoint{9.014308in}{1.390655in}}%
\pgfpathcurveto{\pgfqpoint{9.017874in}{1.387088in}}{\pgfqpoint{9.022712in}{1.385084in}}{\pgfqpoint{9.027756in}{1.385084in}}%
\pgfpathclose%
\pgfusepath{fill}%
\end{pgfscope}%
\begin{pgfscope}%
\pgfpathrectangle{\pgfqpoint{6.572727in}{0.474100in}}{\pgfqpoint{4.227273in}{3.318700in}}%
\pgfusepath{clip}%
\pgfsetbuttcap%
\pgfsetroundjoin%
\definecolor{currentfill}{rgb}{0.127568,0.566949,0.550556}%
\pgfsetfillcolor{currentfill}%
\pgfsetfillopacity{0.700000}%
\pgfsetlinewidth{0.000000pt}%
\definecolor{currentstroke}{rgb}{0.000000,0.000000,0.000000}%
\pgfsetstrokecolor{currentstroke}%
\pgfsetstrokeopacity{0.700000}%
\pgfsetdash{}{0pt}%
\pgfpathmoveto{\pgfqpoint{9.303541in}{1.776563in}}%
\pgfpathcurveto{\pgfqpoint{9.308585in}{1.776563in}}{\pgfqpoint{9.313423in}{1.778567in}}{\pgfqpoint{9.316989in}{1.782134in}}%
\pgfpathcurveto{\pgfqpoint{9.320555in}{1.785700in}}{\pgfqpoint{9.322559in}{1.790538in}}{\pgfqpoint{9.322559in}{1.795582in}}%
\pgfpathcurveto{\pgfqpoint{9.322559in}{1.800625in}}{\pgfqpoint{9.320555in}{1.805463in}}{\pgfqpoint{9.316989in}{1.809029in}}%
\pgfpathcurveto{\pgfqpoint{9.313423in}{1.812596in}}{\pgfqpoint{9.308585in}{1.814600in}}{\pgfqpoint{9.303541in}{1.814600in}}%
\pgfpathcurveto{\pgfqpoint{9.298497in}{1.814600in}}{\pgfqpoint{9.293660in}{1.812596in}}{\pgfqpoint{9.290093in}{1.809029in}}%
\pgfpathcurveto{\pgfqpoint{9.286527in}{1.805463in}}{\pgfqpoint{9.284523in}{1.800625in}}{\pgfqpoint{9.284523in}{1.795582in}}%
\pgfpathcurveto{\pgfqpoint{9.284523in}{1.790538in}}{\pgfqpoint{9.286527in}{1.785700in}}{\pgfqpoint{9.290093in}{1.782134in}}%
\pgfpathcurveto{\pgfqpoint{9.293660in}{1.778567in}}{\pgfqpoint{9.298497in}{1.776563in}}{\pgfqpoint{9.303541in}{1.776563in}}%
\pgfpathclose%
\pgfusepath{fill}%
\end{pgfscope}%
\begin{pgfscope}%
\pgfpathrectangle{\pgfqpoint{6.572727in}{0.474100in}}{\pgfqpoint{4.227273in}{3.318700in}}%
\pgfusepath{clip}%
\pgfsetbuttcap%
\pgfsetroundjoin%
\definecolor{currentfill}{rgb}{0.127568,0.566949,0.550556}%
\pgfsetfillcolor{currentfill}%
\pgfsetfillopacity{0.700000}%
\pgfsetlinewidth{0.000000pt}%
\definecolor{currentstroke}{rgb}{0.000000,0.000000,0.000000}%
\pgfsetstrokecolor{currentstroke}%
\pgfsetstrokeopacity{0.700000}%
\pgfsetdash{}{0pt}%
\pgfpathmoveto{\pgfqpoint{9.887278in}{1.473150in}}%
\pgfpathcurveto{\pgfqpoint{9.892322in}{1.473150in}}{\pgfqpoint{9.897160in}{1.475154in}}{\pgfqpoint{9.900726in}{1.478720in}}%
\pgfpathcurveto{\pgfqpoint{9.904293in}{1.482287in}}{\pgfqpoint{9.906297in}{1.487125in}}{\pgfqpoint{9.906297in}{1.492168in}}%
\pgfpathcurveto{\pgfqpoint{9.906297in}{1.497212in}}{\pgfqpoint{9.904293in}{1.502050in}}{\pgfqpoint{9.900726in}{1.505616in}}%
\pgfpathcurveto{\pgfqpoint{9.897160in}{1.509183in}}{\pgfqpoint{9.892322in}{1.511186in}}{\pgfqpoint{9.887278in}{1.511186in}}%
\pgfpathcurveto{\pgfqpoint{9.882235in}{1.511186in}}{\pgfqpoint{9.877397in}{1.509183in}}{\pgfqpoint{9.873831in}{1.505616in}}%
\pgfpathcurveto{\pgfqpoint{9.870264in}{1.502050in}}{\pgfqpoint{9.868260in}{1.497212in}}{\pgfqpoint{9.868260in}{1.492168in}}%
\pgfpathcurveto{\pgfqpoint{9.868260in}{1.487125in}}{\pgfqpoint{9.870264in}{1.482287in}}{\pgfqpoint{9.873831in}{1.478720in}}%
\pgfpathcurveto{\pgfqpoint{9.877397in}{1.475154in}}{\pgfqpoint{9.882235in}{1.473150in}}{\pgfqpoint{9.887278in}{1.473150in}}%
\pgfpathclose%
\pgfusepath{fill}%
\end{pgfscope}%
\begin{pgfscope}%
\pgfpathrectangle{\pgfqpoint{6.572727in}{0.474100in}}{\pgfqpoint{4.227273in}{3.318700in}}%
\pgfusepath{clip}%
\pgfsetbuttcap%
\pgfsetroundjoin%
\definecolor{currentfill}{rgb}{0.127568,0.566949,0.550556}%
\pgfsetfillcolor{currentfill}%
\pgfsetfillopacity{0.700000}%
\pgfsetlinewidth{0.000000pt}%
\definecolor{currentstroke}{rgb}{0.000000,0.000000,0.000000}%
\pgfsetstrokecolor{currentstroke}%
\pgfsetstrokeopacity{0.700000}%
\pgfsetdash{}{0pt}%
\pgfpathmoveto{\pgfqpoint{9.792150in}{1.421989in}}%
\pgfpathcurveto{\pgfqpoint{9.797194in}{1.421989in}}{\pgfqpoint{9.802031in}{1.423993in}}{\pgfqpoint{9.805598in}{1.427560in}}%
\pgfpathcurveto{\pgfqpoint{9.809164in}{1.431126in}}{\pgfqpoint{9.811168in}{1.435964in}}{\pgfqpoint{9.811168in}{1.441007in}}%
\pgfpathcurveto{\pgfqpoint{9.811168in}{1.446051in}}{\pgfqpoint{9.809164in}{1.450889in}}{\pgfqpoint{9.805598in}{1.454455in}}%
\pgfpathcurveto{\pgfqpoint{9.802031in}{1.458022in}}{\pgfqpoint{9.797194in}{1.460026in}}{\pgfqpoint{9.792150in}{1.460026in}}%
\pgfpathcurveto{\pgfqpoint{9.787106in}{1.460026in}}{\pgfqpoint{9.782269in}{1.458022in}}{\pgfqpoint{9.778702in}{1.454455in}}%
\pgfpathcurveto{\pgfqpoint{9.775136in}{1.450889in}}{\pgfqpoint{9.773132in}{1.446051in}}{\pgfqpoint{9.773132in}{1.441007in}}%
\pgfpathcurveto{\pgfqpoint{9.773132in}{1.435964in}}{\pgfqpoint{9.775136in}{1.431126in}}{\pgfqpoint{9.778702in}{1.427560in}}%
\pgfpathcurveto{\pgfqpoint{9.782269in}{1.423993in}}{\pgfqpoint{9.787106in}{1.421989in}}{\pgfqpoint{9.792150in}{1.421989in}}%
\pgfpathclose%
\pgfusepath{fill}%
\end{pgfscope}%
\begin{pgfscope}%
\pgfpathrectangle{\pgfqpoint{6.572727in}{0.474100in}}{\pgfqpoint{4.227273in}{3.318700in}}%
\pgfusepath{clip}%
\pgfsetbuttcap%
\pgfsetroundjoin%
\definecolor{currentfill}{rgb}{0.993248,0.906157,0.143936}%
\pgfsetfillcolor{currentfill}%
\pgfsetfillopacity{0.700000}%
\pgfsetlinewidth{0.000000pt}%
\definecolor{currentstroke}{rgb}{0.000000,0.000000,0.000000}%
\pgfsetstrokecolor{currentstroke}%
\pgfsetstrokeopacity{0.700000}%
\pgfsetdash{}{0pt}%
\pgfpathmoveto{\pgfqpoint{8.204189in}{3.156053in}}%
\pgfpathcurveto{\pgfqpoint{8.209233in}{3.156053in}}{\pgfqpoint{8.214071in}{3.158057in}}{\pgfqpoint{8.217637in}{3.161623in}}%
\pgfpathcurveto{\pgfqpoint{8.221203in}{3.165190in}}{\pgfqpoint{8.223207in}{3.170028in}}{\pgfqpoint{8.223207in}{3.175071in}}%
\pgfpathcurveto{\pgfqpoint{8.223207in}{3.180115in}}{\pgfqpoint{8.221203in}{3.184953in}}{\pgfqpoint{8.217637in}{3.188519in}}%
\pgfpathcurveto{\pgfqpoint{8.214071in}{3.192086in}}{\pgfqpoint{8.209233in}{3.194089in}}{\pgfqpoint{8.204189in}{3.194089in}}%
\pgfpathcurveto{\pgfqpoint{8.199146in}{3.194089in}}{\pgfqpoint{8.194308in}{3.192086in}}{\pgfqpoint{8.190741in}{3.188519in}}%
\pgfpathcurveto{\pgfqpoint{8.187175in}{3.184953in}}{\pgfqpoint{8.185171in}{3.180115in}}{\pgfqpoint{8.185171in}{3.175071in}}%
\pgfpathcurveto{\pgfqpoint{8.185171in}{3.170028in}}{\pgfqpoint{8.187175in}{3.165190in}}{\pgfqpoint{8.190741in}{3.161623in}}%
\pgfpathcurveto{\pgfqpoint{8.194308in}{3.158057in}}{\pgfqpoint{8.199146in}{3.156053in}}{\pgfqpoint{8.204189in}{3.156053in}}%
\pgfpathclose%
\pgfusepath{fill}%
\end{pgfscope}%
\begin{pgfscope}%
\pgfpathrectangle{\pgfqpoint{6.572727in}{0.474100in}}{\pgfqpoint{4.227273in}{3.318700in}}%
\pgfusepath{clip}%
\pgfsetbuttcap%
\pgfsetroundjoin%
\definecolor{currentfill}{rgb}{0.127568,0.566949,0.550556}%
\pgfsetfillcolor{currentfill}%
\pgfsetfillopacity{0.700000}%
\pgfsetlinewidth{0.000000pt}%
\definecolor{currentstroke}{rgb}{0.000000,0.000000,0.000000}%
\pgfsetstrokecolor{currentstroke}%
\pgfsetstrokeopacity{0.700000}%
\pgfsetdash{}{0pt}%
\pgfpathmoveto{\pgfqpoint{9.905322in}{1.415168in}}%
\pgfpathcurveto{\pgfqpoint{9.910366in}{1.415168in}}{\pgfqpoint{9.915204in}{1.417172in}}{\pgfqpoint{9.918770in}{1.420738in}}%
\pgfpathcurveto{\pgfqpoint{9.922336in}{1.424305in}}{\pgfqpoint{9.924340in}{1.429143in}}{\pgfqpoint{9.924340in}{1.434186in}}%
\pgfpathcurveto{\pgfqpoint{9.924340in}{1.439230in}}{\pgfqpoint{9.922336in}{1.444068in}}{\pgfqpoint{9.918770in}{1.447634in}}%
\pgfpathcurveto{\pgfqpoint{9.915204in}{1.451201in}}{\pgfqpoint{9.910366in}{1.453204in}}{\pgfqpoint{9.905322in}{1.453204in}}%
\pgfpathcurveto{\pgfqpoint{9.900278in}{1.453204in}}{\pgfqpoint{9.895441in}{1.451201in}}{\pgfqpoint{9.891874in}{1.447634in}}%
\pgfpathcurveto{\pgfqpoint{9.888308in}{1.444068in}}{\pgfqpoint{9.886304in}{1.439230in}}{\pgfqpoint{9.886304in}{1.434186in}}%
\pgfpathcurveto{\pgfqpoint{9.886304in}{1.429143in}}{\pgfqpoint{9.888308in}{1.424305in}}{\pgfqpoint{9.891874in}{1.420738in}}%
\pgfpathcurveto{\pgfqpoint{9.895441in}{1.417172in}}{\pgfqpoint{9.900278in}{1.415168in}}{\pgfqpoint{9.905322in}{1.415168in}}%
\pgfpathclose%
\pgfusepath{fill}%
\end{pgfscope}%
\begin{pgfscope}%
\pgfpathrectangle{\pgfqpoint{6.572727in}{0.474100in}}{\pgfqpoint{4.227273in}{3.318700in}}%
\pgfusepath{clip}%
\pgfsetbuttcap%
\pgfsetroundjoin%
\definecolor{currentfill}{rgb}{0.267004,0.004874,0.329415}%
\pgfsetfillcolor{currentfill}%
\pgfsetfillopacity{0.700000}%
\pgfsetlinewidth{0.000000pt}%
\definecolor{currentstroke}{rgb}{0.000000,0.000000,0.000000}%
\pgfsetstrokecolor{currentstroke}%
\pgfsetstrokeopacity{0.700000}%
\pgfsetdash{}{0pt}%
\pgfpathmoveto{\pgfqpoint{8.158699in}{1.791536in}}%
\pgfpathcurveto{\pgfqpoint{8.163742in}{1.791536in}}{\pgfqpoint{8.168580in}{1.793540in}}{\pgfqpoint{8.172146in}{1.797107in}}%
\pgfpathcurveto{\pgfqpoint{8.175713in}{1.800673in}}{\pgfqpoint{8.177717in}{1.805511in}}{\pgfqpoint{8.177717in}{1.810554in}}%
\pgfpathcurveto{\pgfqpoint{8.177717in}{1.815598in}}{\pgfqpoint{8.175713in}{1.820436in}}{\pgfqpoint{8.172146in}{1.824002in}}%
\pgfpathcurveto{\pgfqpoint{8.168580in}{1.827569in}}{\pgfqpoint{8.163742in}{1.829573in}}{\pgfqpoint{8.158699in}{1.829573in}}%
\pgfpathcurveto{\pgfqpoint{8.153655in}{1.829573in}}{\pgfqpoint{8.148817in}{1.827569in}}{\pgfqpoint{8.145251in}{1.824002in}}%
\pgfpathcurveto{\pgfqpoint{8.141684in}{1.820436in}}{\pgfqpoint{8.139680in}{1.815598in}}{\pgfqpoint{8.139680in}{1.810554in}}%
\pgfpathcurveto{\pgfqpoint{8.139680in}{1.805511in}}{\pgfqpoint{8.141684in}{1.800673in}}{\pgfqpoint{8.145251in}{1.797107in}}%
\pgfpathcurveto{\pgfqpoint{8.148817in}{1.793540in}}{\pgfqpoint{8.153655in}{1.791536in}}{\pgfqpoint{8.158699in}{1.791536in}}%
\pgfpathclose%
\pgfusepath{fill}%
\end{pgfscope}%
\begin{pgfscope}%
\pgfpathrectangle{\pgfqpoint{6.572727in}{0.474100in}}{\pgfqpoint{4.227273in}{3.318700in}}%
\pgfusepath{clip}%
\pgfsetbuttcap%
\pgfsetroundjoin%
\definecolor{currentfill}{rgb}{0.993248,0.906157,0.143936}%
\pgfsetfillcolor{currentfill}%
\pgfsetfillopacity{0.700000}%
\pgfsetlinewidth{0.000000pt}%
\definecolor{currentstroke}{rgb}{0.000000,0.000000,0.000000}%
\pgfsetstrokecolor{currentstroke}%
\pgfsetstrokeopacity{0.700000}%
\pgfsetdash{}{0pt}%
\pgfpathmoveto{\pgfqpoint{7.893528in}{2.697936in}}%
\pgfpathcurveto{\pgfqpoint{7.898572in}{2.697936in}}{\pgfqpoint{7.903410in}{2.699939in}}{\pgfqpoint{7.906976in}{2.703506in}}%
\pgfpathcurveto{\pgfqpoint{7.910543in}{2.707072in}}{\pgfqpoint{7.912547in}{2.711910in}}{\pgfqpoint{7.912547in}{2.716954in}}%
\pgfpathcurveto{\pgfqpoint{7.912547in}{2.721997in}}{\pgfqpoint{7.910543in}{2.726835in}}{\pgfqpoint{7.906976in}{2.730402in}}%
\pgfpathcurveto{\pgfqpoint{7.903410in}{2.733968in}}{\pgfqpoint{7.898572in}{2.735972in}}{\pgfqpoint{7.893528in}{2.735972in}}%
\pgfpathcurveto{\pgfqpoint{7.888485in}{2.735972in}}{\pgfqpoint{7.883647in}{2.733968in}}{\pgfqpoint{7.880081in}{2.730402in}}%
\pgfpathcurveto{\pgfqpoint{7.876514in}{2.726835in}}{\pgfqpoint{7.874510in}{2.721997in}}{\pgfqpoint{7.874510in}{2.716954in}}%
\pgfpathcurveto{\pgfqpoint{7.874510in}{2.711910in}}{\pgfqpoint{7.876514in}{2.707072in}}{\pgfqpoint{7.880081in}{2.703506in}}%
\pgfpathcurveto{\pgfqpoint{7.883647in}{2.699939in}}{\pgfqpoint{7.888485in}{2.697936in}}{\pgfqpoint{7.893528in}{2.697936in}}%
\pgfpathclose%
\pgfusepath{fill}%
\end{pgfscope}%
\begin{pgfscope}%
\pgfpathrectangle{\pgfqpoint{6.572727in}{0.474100in}}{\pgfqpoint{4.227273in}{3.318700in}}%
\pgfusepath{clip}%
\pgfsetbuttcap%
\pgfsetroundjoin%
\definecolor{currentfill}{rgb}{0.127568,0.566949,0.550556}%
\pgfsetfillcolor{currentfill}%
\pgfsetfillopacity{0.700000}%
\pgfsetlinewidth{0.000000pt}%
\definecolor{currentstroke}{rgb}{0.000000,0.000000,0.000000}%
\pgfsetstrokecolor{currentstroke}%
\pgfsetstrokeopacity{0.700000}%
\pgfsetdash{}{0pt}%
\pgfpathmoveto{\pgfqpoint{9.565851in}{1.578606in}}%
\pgfpathcurveto{\pgfqpoint{9.570895in}{1.578606in}}{\pgfqpoint{9.575732in}{1.580610in}}{\pgfqpoint{9.579299in}{1.584176in}}%
\pgfpathcurveto{\pgfqpoint{9.582865in}{1.587742in}}{\pgfqpoint{9.584869in}{1.592580in}}{\pgfqpoint{9.584869in}{1.597624in}}%
\pgfpathcurveto{\pgfqpoint{9.584869in}{1.602668in}}{\pgfqpoint{9.582865in}{1.607505in}}{\pgfqpoint{9.579299in}{1.611072in}}%
\pgfpathcurveto{\pgfqpoint{9.575732in}{1.614638in}}{\pgfqpoint{9.570895in}{1.616642in}}{\pgfqpoint{9.565851in}{1.616642in}}%
\pgfpathcurveto{\pgfqpoint{9.560807in}{1.616642in}}{\pgfqpoint{9.555970in}{1.614638in}}{\pgfqpoint{9.552403in}{1.611072in}}%
\pgfpathcurveto{\pgfqpoint{9.548837in}{1.607505in}}{\pgfqpoint{9.546833in}{1.602668in}}{\pgfqpoint{9.546833in}{1.597624in}}%
\pgfpathcurveto{\pgfqpoint{9.546833in}{1.592580in}}{\pgfqpoint{9.548837in}{1.587742in}}{\pgfqpoint{9.552403in}{1.584176in}}%
\pgfpathcurveto{\pgfqpoint{9.555970in}{1.580610in}}{\pgfqpoint{9.560807in}{1.578606in}}{\pgfqpoint{9.565851in}{1.578606in}}%
\pgfpathclose%
\pgfusepath{fill}%
\end{pgfscope}%
\begin{pgfscope}%
\pgfpathrectangle{\pgfqpoint{6.572727in}{0.474100in}}{\pgfqpoint{4.227273in}{3.318700in}}%
\pgfusepath{clip}%
\pgfsetbuttcap%
\pgfsetroundjoin%
\definecolor{currentfill}{rgb}{0.267004,0.004874,0.329415}%
\pgfsetfillcolor{currentfill}%
\pgfsetfillopacity{0.700000}%
\pgfsetlinewidth{0.000000pt}%
\definecolor{currentstroke}{rgb}{0.000000,0.000000,0.000000}%
\pgfsetstrokecolor{currentstroke}%
\pgfsetstrokeopacity{0.700000}%
\pgfsetdash{}{0pt}%
\pgfpathmoveto{\pgfqpoint{7.945522in}{1.355513in}}%
\pgfpathcurveto{\pgfqpoint{7.950566in}{1.355513in}}{\pgfqpoint{7.955404in}{1.357517in}}{\pgfqpoint{7.958970in}{1.361084in}}%
\pgfpathcurveto{\pgfqpoint{7.962537in}{1.364650in}}{\pgfqpoint{7.964540in}{1.369488in}}{\pgfqpoint{7.964540in}{1.374532in}}%
\pgfpathcurveto{\pgfqpoint{7.964540in}{1.379575in}}{\pgfqpoint{7.962537in}{1.384413in}}{\pgfqpoint{7.958970in}{1.387979in}}%
\pgfpathcurveto{\pgfqpoint{7.955404in}{1.391546in}}{\pgfqpoint{7.950566in}{1.393550in}}{\pgfqpoint{7.945522in}{1.393550in}}%
\pgfpathcurveto{\pgfqpoint{7.940479in}{1.393550in}}{\pgfqpoint{7.935641in}{1.391546in}}{\pgfqpoint{7.932074in}{1.387979in}}%
\pgfpathcurveto{\pgfqpoint{7.928508in}{1.384413in}}{\pgfqpoint{7.926504in}{1.379575in}}{\pgfqpoint{7.926504in}{1.374532in}}%
\pgfpathcurveto{\pgfqpoint{7.926504in}{1.369488in}}{\pgfqpoint{7.928508in}{1.364650in}}{\pgfqpoint{7.932074in}{1.361084in}}%
\pgfpathcurveto{\pgfqpoint{7.935641in}{1.357517in}}{\pgfqpoint{7.940479in}{1.355513in}}{\pgfqpoint{7.945522in}{1.355513in}}%
\pgfpathclose%
\pgfusepath{fill}%
\end{pgfscope}%
\begin{pgfscope}%
\pgfpathrectangle{\pgfqpoint{6.572727in}{0.474100in}}{\pgfqpoint{4.227273in}{3.318700in}}%
\pgfusepath{clip}%
\pgfsetbuttcap%
\pgfsetroundjoin%
\definecolor{currentfill}{rgb}{0.267004,0.004874,0.329415}%
\pgfsetfillcolor{currentfill}%
\pgfsetfillopacity{0.700000}%
\pgfsetlinewidth{0.000000pt}%
\definecolor{currentstroke}{rgb}{0.000000,0.000000,0.000000}%
\pgfsetstrokecolor{currentstroke}%
\pgfsetstrokeopacity{0.700000}%
\pgfsetdash{}{0pt}%
\pgfpathmoveto{\pgfqpoint{7.508508in}{1.602626in}}%
\pgfpathcurveto{\pgfqpoint{7.513552in}{1.602626in}}{\pgfqpoint{7.518390in}{1.604629in}}{\pgfqpoint{7.521956in}{1.608196in}}%
\pgfpathcurveto{\pgfqpoint{7.525523in}{1.611762in}}{\pgfqpoint{7.527526in}{1.616600in}}{\pgfqpoint{7.527526in}{1.621644in}}%
\pgfpathcurveto{\pgfqpoint{7.527526in}{1.626687in}}{\pgfqpoint{7.525523in}{1.631525in}}{\pgfqpoint{7.521956in}{1.635092in}}%
\pgfpathcurveto{\pgfqpoint{7.518390in}{1.638658in}}{\pgfqpoint{7.513552in}{1.640662in}}{\pgfqpoint{7.508508in}{1.640662in}}%
\pgfpathcurveto{\pgfqpoint{7.503465in}{1.640662in}}{\pgfqpoint{7.498627in}{1.638658in}}{\pgfqpoint{7.495060in}{1.635092in}}%
\pgfpathcurveto{\pgfqpoint{7.491494in}{1.631525in}}{\pgfqpoint{7.489490in}{1.626687in}}{\pgfqpoint{7.489490in}{1.621644in}}%
\pgfpathcurveto{\pgfqpoint{7.489490in}{1.616600in}}{\pgfqpoint{7.491494in}{1.611762in}}{\pgfqpoint{7.495060in}{1.608196in}}%
\pgfpathcurveto{\pgfqpoint{7.498627in}{1.604629in}}{\pgfqpoint{7.503465in}{1.602626in}}{\pgfqpoint{7.508508in}{1.602626in}}%
\pgfpathclose%
\pgfusepath{fill}%
\end{pgfscope}%
\begin{pgfscope}%
\pgfpathrectangle{\pgfqpoint{6.572727in}{0.474100in}}{\pgfqpoint{4.227273in}{3.318700in}}%
\pgfusepath{clip}%
\pgfsetbuttcap%
\pgfsetroundjoin%
\definecolor{currentfill}{rgb}{0.267004,0.004874,0.329415}%
\pgfsetfillcolor{currentfill}%
\pgfsetfillopacity{0.700000}%
\pgfsetlinewidth{0.000000pt}%
\definecolor{currentstroke}{rgb}{0.000000,0.000000,0.000000}%
\pgfsetstrokecolor{currentstroke}%
\pgfsetstrokeopacity{0.700000}%
\pgfsetdash{}{0pt}%
\pgfpathmoveto{\pgfqpoint{8.140730in}{1.726826in}}%
\pgfpathcurveto{\pgfqpoint{8.145774in}{1.726826in}}{\pgfqpoint{8.150612in}{1.728830in}}{\pgfqpoint{8.154178in}{1.732396in}}%
\pgfpathcurveto{\pgfqpoint{8.157745in}{1.735962in}}{\pgfqpoint{8.159748in}{1.740800in}}{\pgfqpoint{8.159748in}{1.745844in}}%
\pgfpathcurveto{\pgfqpoint{8.159748in}{1.750888in}}{\pgfqpoint{8.157745in}{1.755725in}}{\pgfqpoint{8.154178in}{1.759292in}}%
\pgfpathcurveto{\pgfqpoint{8.150612in}{1.762858in}}{\pgfqpoint{8.145774in}{1.764862in}}{\pgfqpoint{8.140730in}{1.764862in}}%
\pgfpathcurveto{\pgfqpoint{8.135687in}{1.764862in}}{\pgfqpoint{8.130849in}{1.762858in}}{\pgfqpoint{8.127282in}{1.759292in}}%
\pgfpathcurveto{\pgfqpoint{8.123716in}{1.755725in}}{\pgfqpoint{8.121712in}{1.750888in}}{\pgfqpoint{8.121712in}{1.745844in}}%
\pgfpathcurveto{\pgfqpoint{8.121712in}{1.740800in}}{\pgfqpoint{8.123716in}{1.735962in}}{\pgfqpoint{8.127282in}{1.732396in}}%
\pgfpathcurveto{\pgfqpoint{8.130849in}{1.728830in}}{\pgfqpoint{8.135687in}{1.726826in}}{\pgfqpoint{8.140730in}{1.726826in}}%
\pgfpathclose%
\pgfusepath{fill}%
\end{pgfscope}%
\begin{pgfscope}%
\pgfpathrectangle{\pgfqpoint{6.572727in}{0.474100in}}{\pgfqpoint{4.227273in}{3.318700in}}%
\pgfusepath{clip}%
\pgfsetbuttcap%
\pgfsetroundjoin%
\definecolor{currentfill}{rgb}{0.267004,0.004874,0.329415}%
\pgfsetfillcolor{currentfill}%
\pgfsetfillopacity{0.700000}%
\pgfsetlinewidth{0.000000pt}%
\definecolor{currentstroke}{rgb}{0.000000,0.000000,0.000000}%
\pgfsetstrokecolor{currentstroke}%
\pgfsetstrokeopacity{0.700000}%
\pgfsetdash{}{0pt}%
\pgfpathmoveto{\pgfqpoint{7.789967in}{2.082679in}}%
\pgfpathcurveto{\pgfqpoint{7.795011in}{2.082679in}}{\pgfqpoint{7.799848in}{2.084683in}}{\pgfqpoint{7.803415in}{2.088249in}}%
\pgfpathcurveto{\pgfqpoint{7.806981in}{2.091816in}}{\pgfqpoint{7.808985in}{2.096654in}}{\pgfqpoint{7.808985in}{2.101697in}}%
\pgfpathcurveto{\pgfqpoint{7.808985in}{2.106741in}}{\pgfqpoint{7.806981in}{2.111579in}}{\pgfqpoint{7.803415in}{2.115145in}}%
\pgfpathcurveto{\pgfqpoint{7.799848in}{2.118712in}}{\pgfqpoint{7.795011in}{2.120715in}}{\pgfqpoint{7.789967in}{2.120715in}}%
\pgfpathcurveto{\pgfqpoint{7.784923in}{2.120715in}}{\pgfqpoint{7.780086in}{2.118712in}}{\pgfqpoint{7.776519in}{2.115145in}}%
\pgfpathcurveto{\pgfqpoint{7.772953in}{2.111579in}}{\pgfqpoint{7.770949in}{2.106741in}}{\pgfqpoint{7.770949in}{2.101697in}}%
\pgfpathcurveto{\pgfqpoint{7.770949in}{2.096654in}}{\pgfqpoint{7.772953in}{2.091816in}}{\pgfqpoint{7.776519in}{2.088249in}}%
\pgfpathcurveto{\pgfqpoint{7.780086in}{2.084683in}}{\pgfqpoint{7.784923in}{2.082679in}}{\pgfqpoint{7.789967in}{2.082679in}}%
\pgfpathclose%
\pgfusepath{fill}%
\end{pgfscope}%
\begin{pgfscope}%
\pgfpathrectangle{\pgfqpoint{6.572727in}{0.474100in}}{\pgfqpoint{4.227273in}{3.318700in}}%
\pgfusepath{clip}%
\pgfsetbuttcap%
\pgfsetroundjoin%
\definecolor{currentfill}{rgb}{0.267004,0.004874,0.329415}%
\pgfsetfillcolor{currentfill}%
\pgfsetfillopacity{0.700000}%
\pgfsetlinewidth{0.000000pt}%
\definecolor{currentstroke}{rgb}{0.000000,0.000000,0.000000}%
\pgfsetstrokecolor{currentstroke}%
\pgfsetstrokeopacity{0.700000}%
\pgfsetdash{}{0pt}%
\pgfpathmoveto{\pgfqpoint{7.241159in}{1.129722in}}%
\pgfpathcurveto{\pgfqpoint{7.246202in}{1.129722in}}{\pgfqpoint{7.251040in}{1.131726in}}{\pgfqpoint{7.254607in}{1.135292in}}%
\pgfpathcurveto{\pgfqpoint{7.258173in}{1.138859in}}{\pgfqpoint{7.260177in}{1.143696in}}{\pgfqpoint{7.260177in}{1.148740in}}%
\pgfpathcurveto{\pgfqpoint{7.260177in}{1.153784in}}{\pgfqpoint{7.258173in}{1.158621in}}{\pgfqpoint{7.254607in}{1.162188in}}%
\pgfpathcurveto{\pgfqpoint{7.251040in}{1.165754in}}{\pgfqpoint{7.246202in}{1.167758in}}{\pgfqpoint{7.241159in}{1.167758in}}%
\pgfpathcurveto{\pgfqpoint{7.236115in}{1.167758in}}{\pgfqpoint{7.231277in}{1.165754in}}{\pgfqpoint{7.227711in}{1.162188in}}%
\pgfpathcurveto{\pgfqpoint{7.224144in}{1.158621in}}{\pgfqpoint{7.222141in}{1.153784in}}{\pgfqpoint{7.222141in}{1.148740in}}%
\pgfpathcurveto{\pgfqpoint{7.222141in}{1.143696in}}{\pgfqpoint{7.224144in}{1.138859in}}{\pgfqpoint{7.227711in}{1.135292in}}%
\pgfpathcurveto{\pgfqpoint{7.231277in}{1.131726in}}{\pgfqpoint{7.236115in}{1.129722in}}{\pgfqpoint{7.241159in}{1.129722in}}%
\pgfpathclose%
\pgfusepath{fill}%
\end{pgfscope}%
\begin{pgfscope}%
\pgfpathrectangle{\pgfqpoint{6.572727in}{0.474100in}}{\pgfqpoint{4.227273in}{3.318700in}}%
\pgfusepath{clip}%
\pgfsetbuttcap%
\pgfsetroundjoin%
\definecolor{currentfill}{rgb}{0.267004,0.004874,0.329415}%
\pgfsetfillcolor{currentfill}%
\pgfsetfillopacity{0.700000}%
\pgfsetlinewidth{0.000000pt}%
\definecolor{currentstroke}{rgb}{0.000000,0.000000,0.000000}%
\pgfsetstrokecolor{currentstroke}%
\pgfsetstrokeopacity{0.700000}%
\pgfsetdash{}{0pt}%
\pgfpathmoveto{\pgfqpoint{7.509983in}{1.567455in}}%
\pgfpathcurveto{\pgfqpoint{7.515027in}{1.567455in}}{\pgfqpoint{7.519865in}{1.569459in}}{\pgfqpoint{7.523431in}{1.573026in}}%
\pgfpathcurveto{\pgfqpoint{7.526998in}{1.576592in}}{\pgfqpoint{7.529002in}{1.581430in}}{\pgfqpoint{7.529002in}{1.586474in}}%
\pgfpathcurveto{\pgfqpoint{7.529002in}{1.591517in}}{\pgfqpoint{7.526998in}{1.596355in}}{\pgfqpoint{7.523431in}{1.599921in}}%
\pgfpathcurveto{\pgfqpoint{7.519865in}{1.603488in}}{\pgfqpoint{7.515027in}{1.605492in}}{\pgfqpoint{7.509983in}{1.605492in}}%
\pgfpathcurveto{\pgfqpoint{7.504940in}{1.605492in}}{\pgfqpoint{7.500102in}{1.603488in}}{\pgfqpoint{7.496536in}{1.599921in}}%
\pgfpathcurveto{\pgfqpoint{7.492969in}{1.596355in}}{\pgfqpoint{7.490965in}{1.591517in}}{\pgfqpoint{7.490965in}{1.586474in}}%
\pgfpathcurveto{\pgfqpoint{7.490965in}{1.581430in}}{\pgfqpoint{7.492969in}{1.576592in}}{\pgfqpoint{7.496536in}{1.573026in}}%
\pgfpathcurveto{\pgfqpoint{7.500102in}{1.569459in}}{\pgfqpoint{7.504940in}{1.567455in}}{\pgfqpoint{7.509983in}{1.567455in}}%
\pgfpathclose%
\pgfusepath{fill}%
\end{pgfscope}%
\begin{pgfscope}%
\pgfpathrectangle{\pgfqpoint{6.572727in}{0.474100in}}{\pgfqpoint{4.227273in}{3.318700in}}%
\pgfusepath{clip}%
\pgfsetbuttcap%
\pgfsetroundjoin%
\definecolor{currentfill}{rgb}{0.993248,0.906157,0.143936}%
\pgfsetfillcolor{currentfill}%
\pgfsetfillopacity{0.700000}%
\pgfsetlinewidth{0.000000pt}%
\definecolor{currentstroke}{rgb}{0.000000,0.000000,0.000000}%
\pgfsetstrokecolor{currentstroke}%
\pgfsetstrokeopacity{0.700000}%
\pgfsetdash{}{0pt}%
\pgfpathmoveto{\pgfqpoint{8.349274in}{3.398330in}}%
\pgfpathcurveto{\pgfqpoint{8.354317in}{3.398330in}}{\pgfqpoint{8.359155in}{3.400334in}}{\pgfqpoint{8.362722in}{3.403901in}}%
\pgfpathcurveto{\pgfqpoint{8.366288in}{3.407467in}}{\pgfqpoint{8.368292in}{3.412305in}}{\pgfqpoint{8.368292in}{3.417349in}}%
\pgfpathcurveto{\pgfqpoint{8.368292in}{3.422392in}}{\pgfqpoint{8.366288in}{3.427230in}}{\pgfqpoint{8.362722in}{3.430796in}}%
\pgfpathcurveto{\pgfqpoint{8.359155in}{3.434363in}}{\pgfqpoint{8.354317in}{3.436367in}}{\pgfqpoint{8.349274in}{3.436367in}}%
\pgfpathcurveto{\pgfqpoint{8.344230in}{3.436367in}}{\pgfqpoint{8.339392in}{3.434363in}}{\pgfqpoint{8.335826in}{3.430796in}}%
\pgfpathcurveto{\pgfqpoint{8.332260in}{3.427230in}}{\pgfqpoint{8.330256in}{3.422392in}}{\pgfqpoint{8.330256in}{3.417349in}}%
\pgfpathcurveto{\pgfqpoint{8.330256in}{3.412305in}}{\pgfqpoint{8.332260in}{3.407467in}}{\pgfqpoint{8.335826in}{3.403901in}}%
\pgfpathcurveto{\pgfqpoint{8.339392in}{3.400334in}}{\pgfqpoint{8.344230in}{3.398330in}}{\pgfqpoint{8.349274in}{3.398330in}}%
\pgfpathclose%
\pgfusepath{fill}%
\end{pgfscope}%
\begin{pgfscope}%
\pgfpathrectangle{\pgfqpoint{6.572727in}{0.474100in}}{\pgfqpoint{4.227273in}{3.318700in}}%
\pgfusepath{clip}%
\pgfsetbuttcap%
\pgfsetroundjoin%
\definecolor{currentfill}{rgb}{0.267004,0.004874,0.329415}%
\pgfsetfillcolor{currentfill}%
\pgfsetfillopacity{0.700000}%
\pgfsetlinewidth{0.000000pt}%
\definecolor{currentstroke}{rgb}{0.000000,0.000000,0.000000}%
\pgfsetstrokecolor{currentstroke}%
\pgfsetstrokeopacity{0.700000}%
\pgfsetdash{}{0pt}%
\pgfpathmoveto{\pgfqpoint{7.844840in}{2.080716in}}%
\pgfpathcurveto{\pgfqpoint{7.849884in}{2.080716in}}{\pgfqpoint{7.854722in}{2.082720in}}{\pgfqpoint{7.858288in}{2.086286in}}%
\pgfpathcurveto{\pgfqpoint{7.861855in}{2.089852in}}{\pgfqpoint{7.863858in}{2.094690in}}{\pgfqpoint{7.863858in}{2.099734in}}%
\pgfpathcurveto{\pgfqpoint{7.863858in}{2.104778in}}{\pgfqpoint{7.861855in}{2.109615in}}{\pgfqpoint{7.858288in}{2.113182in}}%
\pgfpathcurveto{\pgfqpoint{7.854722in}{2.116748in}}{\pgfqpoint{7.849884in}{2.118752in}}{\pgfqpoint{7.844840in}{2.118752in}}%
\pgfpathcurveto{\pgfqpoint{7.839797in}{2.118752in}}{\pgfqpoint{7.834959in}{2.116748in}}{\pgfqpoint{7.831392in}{2.113182in}}%
\pgfpathcurveto{\pgfqpoint{7.827826in}{2.109615in}}{\pgfqpoint{7.825822in}{2.104778in}}{\pgfqpoint{7.825822in}{2.099734in}}%
\pgfpathcurveto{\pgfqpoint{7.825822in}{2.094690in}}{\pgfqpoint{7.827826in}{2.089852in}}{\pgfqpoint{7.831392in}{2.086286in}}%
\pgfpathcurveto{\pgfqpoint{7.834959in}{2.082720in}}{\pgfqpoint{7.839797in}{2.080716in}}{\pgfqpoint{7.844840in}{2.080716in}}%
\pgfpathclose%
\pgfusepath{fill}%
\end{pgfscope}%
\begin{pgfscope}%
\pgfpathrectangle{\pgfqpoint{6.572727in}{0.474100in}}{\pgfqpoint{4.227273in}{3.318700in}}%
\pgfusepath{clip}%
\pgfsetbuttcap%
\pgfsetroundjoin%
\definecolor{currentfill}{rgb}{0.993248,0.906157,0.143936}%
\pgfsetfillcolor{currentfill}%
\pgfsetfillopacity{0.700000}%
\pgfsetlinewidth{0.000000pt}%
\definecolor{currentstroke}{rgb}{0.000000,0.000000,0.000000}%
\pgfsetstrokecolor{currentstroke}%
\pgfsetstrokeopacity{0.700000}%
\pgfsetdash{}{0pt}%
\pgfpathmoveto{\pgfqpoint{8.142850in}{2.272011in}}%
\pgfpathcurveto{\pgfqpoint{8.147893in}{2.272011in}}{\pgfqpoint{8.152731in}{2.274015in}}{\pgfqpoint{8.156298in}{2.277582in}}%
\pgfpathcurveto{\pgfqpoint{8.159864in}{2.281148in}}{\pgfqpoint{8.161868in}{2.285986in}}{\pgfqpoint{8.161868in}{2.291030in}}%
\pgfpathcurveto{\pgfqpoint{8.161868in}{2.296073in}}{\pgfqpoint{8.159864in}{2.300911in}}{\pgfqpoint{8.156298in}{2.304477in}}%
\pgfpathcurveto{\pgfqpoint{8.152731in}{2.308044in}}{\pgfqpoint{8.147893in}{2.310048in}}{\pgfqpoint{8.142850in}{2.310048in}}%
\pgfpathcurveto{\pgfqpoint{8.137806in}{2.310048in}}{\pgfqpoint{8.132968in}{2.308044in}}{\pgfqpoint{8.129402in}{2.304477in}}%
\pgfpathcurveto{\pgfqpoint{8.125836in}{2.300911in}}{\pgfqpoint{8.123832in}{2.296073in}}{\pgfqpoint{8.123832in}{2.291030in}}%
\pgfpathcurveto{\pgfqpoint{8.123832in}{2.285986in}}{\pgfqpoint{8.125836in}{2.281148in}}{\pgfqpoint{8.129402in}{2.277582in}}%
\pgfpathcurveto{\pgfqpoint{8.132968in}{2.274015in}}{\pgfqpoint{8.137806in}{2.272011in}}{\pgfqpoint{8.142850in}{2.272011in}}%
\pgfpathclose%
\pgfusepath{fill}%
\end{pgfscope}%
\begin{pgfscope}%
\pgfpathrectangle{\pgfqpoint{6.572727in}{0.474100in}}{\pgfqpoint{4.227273in}{3.318700in}}%
\pgfusepath{clip}%
\pgfsetbuttcap%
\pgfsetroundjoin%
\definecolor{currentfill}{rgb}{0.267004,0.004874,0.329415}%
\pgfsetfillcolor{currentfill}%
\pgfsetfillopacity{0.700000}%
\pgfsetlinewidth{0.000000pt}%
\definecolor{currentstroke}{rgb}{0.000000,0.000000,0.000000}%
\pgfsetstrokecolor{currentstroke}%
\pgfsetstrokeopacity{0.700000}%
\pgfsetdash{}{0pt}%
\pgfpathmoveto{\pgfqpoint{8.365658in}{1.289695in}}%
\pgfpathcurveto{\pgfqpoint{8.370701in}{1.289695in}}{\pgfqpoint{8.375539in}{1.291699in}}{\pgfqpoint{8.379106in}{1.295266in}}%
\pgfpathcurveto{\pgfqpoint{8.382672in}{1.298832in}}{\pgfqpoint{8.384676in}{1.303670in}}{\pgfqpoint{8.384676in}{1.308713in}}%
\pgfpathcurveto{\pgfqpoint{8.384676in}{1.313757in}}{\pgfqpoint{8.382672in}{1.318595in}}{\pgfqpoint{8.379106in}{1.322161in}}%
\pgfpathcurveto{\pgfqpoint{8.375539in}{1.325728in}}{\pgfqpoint{8.370701in}{1.327732in}}{\pgfqpoint{8.365658in}{1.327732in}}%
\pgfpathcurveto{\pgfqpoint{8.360614in}{1.327732in}}{\pgfqpoint{8.355776in}{1.325728in}}{\pgfqpoint{8.352210in}{1.322161in}}%
\pgfpathcurveto{\pgfqpoint{8.348643in}{1.318595in}}{\pgfqpoint{8.346640in}{1.313757in}}{\pgfqpoint{8.346640in}{1.308713in}}%
\pgfpathcurveto{\pgfqpoint{8.346640in}{1.303670in}}{\pgfqpoint{8.348643in}{1.298832in}}{\pgfqpoint{8.352210in}{1.295266in}}%
\pgfpathcurveto{\pgfqpoint{8.355776in}{1.291699in}}{\pgfqpoint{8.360614in}{1.289695in}}{\pgfqpoint{8.365658in}{1.289695in}}%
\pgfpathclose%
\pgfusepath{fill}%
\end{pgfscope}%
\begin{pgfscope}%
\pgfpathrectangle{\pgfqpoint{6.572727in}{0.474100in}}{\pgfqpoint{4.227273in}{3.318700in}}%
\pgfusepath{clip}%
\pgfsetbuttcap%
\pgfsetroundjoin%
\definecolor{currentfill}{rgb}{0.267004,0.004874,0.329415}%
\pgfsetfillcolor{currentfill}%
\pgfsetfillopacity{0.700000}%
\pgfsetlinewidth{0.000000pt}%
\definecolor{currentstroke}{rgb}{0.000000,0.000000,0.000000}%
\pgfsetstrokecolor{currentstroke}%
\pgfsetstrokeopacity{0.700000}%
\pgfsetdash{}{0pt}%
\pgfpathmoveto{\pgfqpoint{7.651325in}{0.955761in}}%
\pgfpathcurveto{\pgfqpoint{7.656369in}{0.955761in}}{\pgfqpoint{7.661207in}{0.957765in}}{\pgfqpoint{7.664773in}{0.961331in}}%
\pgfpathcurveto{\pgfqpoint{7.668340in}{0.964898in}}{\pgfqpoint{7.670343in}{0.969735in}}{\pgfqpoint{7.670343in}{0.974779in}}%
\pgfpathcurveto{\pgfqpoint{7.670343in}{0.979823in}}{\pgfqpoint{7.668340in}{0.984661in}}{\pgfqpoint{7.664773in}{0.988227in}}%
\pgfpathcurveto{\pgfqpoint{7.661207in}{0.991793in}}{\pgfqpoint{7.656369in}{0.993797in}}{\pgfqpoint{7.651325in}{0.993797in}}%
\pgfpathcurveto{\pgfqpoint{7.646282in}{0.993797in}}{\pgfqpoint{7.641444in}{0.991793in}}{\pgfqpoint{7.637877in}{0.988227in}}%
\pgfpathcurveto{\pgfqpoint{7.634311in}{0.984661in}}{\pgfqpoint{7.632307in}{0.979823in}}{\pgfqpoint{7.632307in}{0.974779in}}%
\pgfpathcurveto{\pgfqpoint{7.632307in}{0.969735in}}{\pgfqpoint{7.634311in}{0.964898in}}{\pgfqpoint{7.637877in}{0.961331in}}%
\pgfpathcurveto{\pgfqpoint{7.641444in}{0.957765in}}{\pgfqpoint{7.646282in}{0.955761in}}{\pgfqpoint{7.651325in}{0.955761in}}%
\pgfpathclose%
\pgfusepath{fill}%
\end{pgfscope}%
\begin{pgfscope}%
\pgfpathrectangle{\pgfqpoint{6.572727in}{0.474100in}}{\pgfqpoint{4.227273in}{3.318700in}}%
\pgfusepath{clip}%
\pgfsetbuttcap%
\pgfsetroundjoin%
\definecolor{currentfill}{rgb}{0.127568,0.566949,0.550556}%
\pgfsetfillcolor{currentfill}%
\pgfsetfillopacity{0.700000}%
\pgfsetlinewidth{0.000000pt}%
\definecolor{currentstroke}{rgb}{0.000000,0.000000,0.000000}%
\pgfsetstrokecolor{currentstroke}%
\pgfsetstrokeopacity{0.700000}%
\pgfsetdash{}{0pt}%
\pgfpathmoveto{\pgfqpoint{9.634312in}{1.525565in}}%
\pgfpathcurveto{\pgfqpoint{9.639356in}{1.525565in}}{\pgfqpoint{9.644193in}{1.527569in}}{\pgfqpoint{9.647760in}{1.531135in}}%
\pgfpathcurveto{\pgfqpoint{9.651326in}{1.534702in}}{\pgfqpoint{9.653330in}{1.539540in}}{\pgfqpoint{9.653330in}{1.544583in}}%
\pgfpathcurveto{\pgfqpoint{9.653330in}{1.549627in}}{\pgfqpoint{9.651326in}{1.554465in}}{\pgfqpoint{9.647760in}{1.558031in}}%
\pgfpathcurveto{\pgfqpoint{9.644193in}{1.561598in}}{\pgfqpoint{9.639356in}{1.563601in}}{\pgfqpoint{9.634312in}{1.563601in}}%
\pgfpathcurveto{\pgfqpoint{9.629268in}{1.563601in}}{\pgfqpoint{9.624430in}{1.561598in}}{\pgfqpoint{9.620864in}{1.558031in}}%
\pgfpathcurveto{\pgfqpoint{9.617298in}{1.554465in}}{\pgfqpoint{9.615294in}{1.549627in}}{\pgfqpoint{9.615294in}{1.544583in}}%
\pgfpathcurveto{\pgfqpoint{9.615294in}{1.539540in}}{\pgfqpoint{9.617298in}{1.534702in}}{\pgfqpoint{9.620864in}{1.531135in}}%
\pgfpathcurveto{\pgfqpoint{9.624430in}{1.527569in}}{\pgfqpoint{9.629268in}{1.525565in}}{\pgfqpoint{9.634312in}{1.525565in}}%
\pgfpathclose%
\pgfusepath{fill}%
\end{pgfscope}%
\begin{pgfscope}%
\pgfpathrectangle{\pgfqpoint{6.572727in}{0.474100in}}{\pgfqpoint{4.227273in}{3.318700in}}%
\pgfusepath{clip}%
\pgfsetbuttcap%
\pgfsetroundjoin%
\definecolor{currentfill}{rgb}{0.267004,0.004874,0.329415}%
\pgfsetfillcolor{currentfill}%
\pgfsetfillopacity{0.700000}%
\pgfsetlinewidth{0.000000pt}%
\definecolor{currentstroke}{rgb}{0.000000,0.000000,0.000000}%
\pgfsetstrokecolor{currentstroke}%
\pgfsetstrokeopacity{0.700000}%
\pgfsetdash{}{0pt}%
\pgfpathmoveto{\pgfqpoint{7.958342in}{1.660892in}}%
\pgfpathcurveto{\pgfqpoint{7.963386in}{1.660892in}}{\pgfqpoint{7.968224in}{1.662896in}}{\pgfqpoint{7.971790in}{1.666462in}}%
\pgfpathcurveto{\pgfqpoint{7.975356in}{1.670029in}}{\pgfqpoint{7.977360in}{1.674867in}}{\pgfqpoint{7.977360in}{1.679910in}}%
\pgfpathcurveto{\pgfqpoint{7.977360in}{1.684954in}}{\pgfqpoint{7.975356in}{1.689792in}}{\pgfqpoint{7.971790in}{1.693358in}}%
\pgfpathcurveto{\pgfqpoint{7.968224in}{1.696925in}}{\pgfqpoint{7.963386in}{1.698928in}}{\pgfqpoint{7.958342in}{1.698928in}}%
\pgfpathcurveto{\pgfqpoint{7.953298in}{1.698928in}}{\pgfqpoint{7.948461in}{1.696925in}}{\pgfqpoint{7.944894in}{1.693358in}}%
\pgfpathcurveto{\pgfqpoint{7.941328in}{1.689792in}}{\pgfqpoint{7.939324in}{1.684954in}}{\pgfqpoint{7.939324in}{1.679910in}}%
\pgfpathcurveto{\pgfqpoint{7.939324in}{1.674867in}}{\pgfqpoint{7.941328in}{1.670029in}}{\pgfqpoint{7.944894in}{1.666462in}}%
\pgfpathcurveto{\pgfqpoint{7.948461in}{1.662896in}}{\pgfqpoint{7.953298in}{1.660892in}}{\pgfqpoint{7.958342in}{1.660892in}}%
\pgfpathclose%
\pgfusepath{fill}%
\end{pgfscope}%
\begin{pgfscope}%
\pgfpathrectangle{\pgfqpoint{6.572727in}{0.474100in}}{\pgfqpoint{4.227273in}{3.318700in}}%
\pgfusepath{clip}%
\pgfsetbuttcap%
\pgfsetroundjoin%
\definecolor{currentfill}{rgb}{0.127568,0.566949,0.550556}%
\pgfsetfillcolor{currentfill}%
\pgfsetfillopacity{0.700000}%
\pgfsetlinewidth{0.000000pt}%
\definecolor{currentstroke}{rgb}{0.000000,0.000000,0.000000}%
\pgfsetstrokecolor{currentstroke}%
\pgfsetstrokeopacity{0.700000}%
\pgfsetdash{}{0pt}%
\pgfpathmoveto{\pgfqpoint{9.283712in}{1.069771in}}%
\pgfpathcurveto{\pgfqpoint{9.288756in}{1.069771in}}{\pgfqpoint{9.293594in}{1.071775in}}{\pgfqpoint{9.297160in}{1.075341in}}%
\pgfpathcurveto{\pgfqpoint{9.300726in}{1.078907in}}{\pgfqpoint{9.302730in}{1.083745in}}{\pgfqpoint{9.302730in}{1.088789in}}%
\pgfpathcurveto{\pgfqpoint{9.302730in}{1.093833in}}{\pgfqpoint{9.300726in}{1.098670in}}{\pgfqpoint{9.297160in}{1.102237in}}%
\pgfpathcurveto{\pgfqpoint{9.293594in}{1.105803in}}{\pgfqpoint{9.288756in}{1.107807in}}{\pgfqpoint{9.283712in}{1.107807in}}%
\pgfpathcurveto{\pgfqpoint{9.278668in}{1.107807in}}{\pgfqpoint{9.273831in}{1.105803in}}{\pgfqpoint{9.270264in}{1.102237in}}%
\pgfpathcurveto{\pgfqpoint{9.266698in}{1.098670in}}{\pgfqpoint{9.264694in}{1.093833in}}{\pgfqpoint{9.264694in}{1.088789in}}%
\pgfpathcurveto{\pgfqpoint{9.264694in}{1.083745in}}{\pgfqpoint{9.266698in}{1.078907in}}{\pgfqpoint{9.270264in}{1.075341in}}%
\pgfpathcurveto{\pgfqpoint{9.273831in}{1.071775in}}{\pgfqpoint{9.278668in}{1.069771in}}{\pgfqpoint{9.283712in}{1.069771in}}%
\pgfpathclose%
\pgfusepath{fill}%
\end{pgfscope}%
\begin{pgfscope}%
\pgfpathrectangle{\pgfqpoint{6.572727in}{0.474100in}}{\pgfqpoint{4.227273in}{3.318700in}}%
\pgfusepath{clip}%
\pgfsetbuttcap%
\pgfsetroundjoin%
\definecolor{currentfill}{rgb}{0.127568,0.566949,0.550556}%
\pgfsetfillcolor{currentfill}%
\pgfsetfillopacity{0.700000}%
\pgfsetlinewidth{0.000000pt}%
\definecolor{currentstroke}{rgb}{0.000000,0.000000,0.000000}%
\pgfsetstrokecolor{currentstroke}%
\pgfsetstrokeopacity{0.700000}%
\pgfsetdash{}{0pt}%
\pgfpathmoveto{\pgfqpoint{9.519123in}{1.168069in}}%
\pgfpathcurveto{\pgfqpoint{9.524167in}{1.168069in}}{\pgfqpoint{9.529005in}{1.170073in}}{\pgfqpoint{9.532571in}{1.173639in}}%
\pgfpathcurveto{\pgfqpoint{9.536138in}{1.177206in}}{\pgfqpoint{9.538142in}{1.182043in}}{\pgfqpoint{9.538142in}{1.187087in}}%
\pgfpathcurveto{\pgfqpoint{9.538142in}{1.192131in}}{\pgfqpoint{9.536138in}{1.196969in}}{\pgfqpoint{9.532571in}{1.200535in}}%
\pgfpathcurveto{\pgfqpoint{9.529005in}{1.204101in}}{\pgfqpoint{9.524167in}{1.206105in}}{\pgfqpoint{9.519123in}{1.206105in}}%
\pgfpathcurveto{\pgfqpoint{9.514080in}{1.206105in}}{\pgfqpoint{9.509242in}{1.204101in}}{\pgfqpoint{9.505676in}{1.200535in}}%
\pgfpathcurveto{\pgfqpoint{9.502109in}{1.196969in}}{\pgfqpoint{9.500105in}{1.192131in}}{\pgfqpoint{9.500105in}{1.187087in}}%
\pgfpathcurveto{\pgfqpoint{9.500105in}{1.182043in}}{\pgfqpoint{9.502109in}{1.177206in}}{\pgfqpoint{9.505676in}{1.173639in}}%
\pgfpathcurveto{\pgfqpoint{9.509242in}{1.170073in}}{\pgfqpoint{9.514080in}{1.168069in}}{\pgfqpoint{9.519123in}{1.168069in}}%
\pgfpathclose%
\pgfusepath{fill}%
\end{pgfscope}%
\begin{pgfscope}%
\pgfpathrectangle{\pgfqpoint{6.572727in}{0.474100in}}{\pgfqpoint{4.227273in}{3.318700in}}%
\pgfusepath{clip}%
\pgfsetbuttcap%
\pgfsetroundjoin%
\definecolor{currentfill}{rgb}{0.127568,0.566949,0.550556}%
\pgfsetfillcolor{currentfill}%
\pgfsetfillopacity{0.700000}%
\pgfsetlinewidth{0.000000pt}%
\definecolor{currentstroke}{rgb}{0.000000,0.000000,0.000000}%
\pgfsetstrokecolor{currentstroke}%
\pgfsetstrokeopacity{0.700000}%
\pgfsetdash{}{0pt}%
\pgfpathmoveto{\pgfqpoint{9.659079in}{1.622575in}}%
\pgfpathcurveto{\pgfqpoint{9.664123in}{1.622575in}}{\pgfqpoint{9.668961in}{1.624579in}}{\pgfqpoint{9.672527in}{1.628145in}}%
\pgfpathcurveto{\pgfqpoint{9.676094in}{1.631712in}}{\pgfqpoint{9.678097in}{1.636549in}}{\pgfqpoint{9.678097in}{1.641593in}}%
\pgfpathcurveto{\pgfqpoint{9.678097in}{1.646637in}}{\pgfqpoint{9.676094in}{1.651475in}}{\pgfqpoint{9.672527in}{1.655041in}}%
\pgfpathcurveto{\pgfqpoint{9.668961in}{1.658607in}}{\pgfqpoint{9.664123in}{1.660611in}}{\pgfqpoint{9.659079in}{1.660611in}}%
\pgfpathcurveto{\pgfqpoint{9.654036in}{1.660611in}}{\pgfqpoint{9.649198in}{1.658607in}}{\pgfqpoint{9.645631in}{1.655041in}}%
\pgfpathcurveto{\pgfqpoint{9.642065in}{1.651475in}}{\pgfqpoint{9.640061in}{1.646637in}}{\pgfqpoint{9.640061in}{1.641593in}}%
\pgfpathcurveto{\pgfqpoint{9.640061in}{1.636549in}}{\pgfqpoint{9.642065in}{1.631712in}}{\pgfqpoint{9.645631in}{1.628145in}}%
\pgfpathcurveto{\pgfqpoint{9.649198in}{1.624579in}}{\pgfqpoint{9.654036in}{1.622575in}}{\pgfqpoint{9.659079in}{1.622575in}}%
\pgfpathclose%
\pgfusepath{fill}%
\end{pgfscope}%
\begin{pgfscope}%
\pgfpathrectangle{\pgfqpoint{6.572727in}{0.474100in}}{\pgfqpoint{4.227273in}{3.318700in}}%
\pgfusepath{clip}%
\pgfsetbuttcap%
\pgfsetroundjoin%
\definecolor{currentfill}{rgb}{0.127568,0.566949,0.550556}%
\pgfsetfillcolor{currentfill}%
\pgfsetfillopacity{0.700000}%
\pgfsetlinewidth{0.000000pt}%
\definecolor{currentstroke}{rgb}{0.000000,0.000000,0.000000}%
\pgfsetstrokecolor{currentstroke}%
\pgfsetstrokeopacity{0.700000}%
\pgfsetdash{}{0pt}%
\pgfpathmoveto{\pgfqpoint{9.654489in}{2.291135in}}%
\pgfpathcurveto{\pgfqpoint{9.659533in}{2.291135in}}{\pgfqpoint{9.664371in}{2.293139in}}{\pgfqpoint{9.667937in}{2.296706in}}%
\pgfpathcurveto{\pgfqpoint{9.671504in}{2.300272in}}{\pgfqpoint{9.673508in}{2.305110in}}{\pgfqpoint{9.673508in}{2.310154in}}%
\pgfpathcurveto{\pgfqpoint{9.673508in}{2.315197in}}{\pgfqpoint{9.671504in}{2.320035in}}{\pgfqpoint{9.667937in}{2.323601in}}%
\pgfpathcurveto{\pgfqpoint{9.664371in}{2.327168in}}{\pgfqpoint{9.659533in}{2.329172in}}{\pgfqpoint{9.654489in}{2.329172in}}%
\pgfpathcurveto{\pgfqpoint{9.649446in}{2.329172in}}{\pgfqpoint{9.644608in}{2.327168in}}{\pgfqpoint{9.641041in}{2.323601in}}%
\pgfpathcurveto{\pgfqpoint{9.637475in}{2.320035in}}{\pgfqpoint{9.635471in}{2.315197in}}{\pgfqpoint{9.635471in}{2.310154in}}%
\pgfpathcurveto{\pgfqpoint{9.635471in}{2.305110in}}{\pgfqpoint{9.637475in}{2.300272in}}{\pgfqpoint{9.641041in}{2.296706in}}%
\pgfpathcurveto{\pgfqpoint{9.644608in}{2.293139in}}{\pgfqpoint{9.649446in}{2.291135in}}{\pgfqpoint{9.654489in}{2.291135in}}%
\pgfpathclose%
\pgfusepath{fill}%
\end{pgfscope}%
\begin{pgfscope}%
\pgfpathrectangle{\pgfqpoint{6.572727in}{0.474100in}}{\pgfqpoint{4.227273in}{3.318700in}}%
\pgfusepath{clip}%
\pgfsetbuttcap%
\pgfsetroundjoin%
\definecolor{currentfill}{rgb}{0.127568,0.566949,0.550556}%
\pgfsetfillcolor{currentfill}%
\pgfsetfillopacity{0.700000}%
\pgfsetlinewidth{0.000000pt}%
\definecolor{currentstroke}{rgb}{0.000000,0.000000,0.000000}%
\pgfsetstrokecolor{currentstroke}%
\pgfsetstrokeopacity{0.700000}%
\pgfsetdash{}{0pt}%
\pgfpathmoveto{\pgfqpoint{9.831714in}{1.444850in}}%
\pgfpathcurveto{\pgfqpoint{9.836757in}{1.444850in}}{\pgfqpoint{9.841595in}{1.446854in}}{\pgfqpoint{9.845161in}{1.450421in}}%
\pgfpathcurveto{\pgfqpoint{9.848728in}{1.453987in}}{\pgfqpoint{9.850732in}{1.458825in}}{\pgfqpoint{9.850732in}{1.463869in}}%
\pgfpathcurveto{\pgfqpoint{9.850732in}{1.468912in}}{\pgfqpoint{9.848728in}{1.473750in}}{\pgfqpoint{9.845161in}{1.477316in}}%
\pgfpathcurveto{\pgfqpoint{9.841595in}{1.480883in}}{\pgfqpoint{9.836757in}{1.482887in}}{\pgfqpoint{9.831714in}{1.482887in}}%
\pgfpathcurveto{\pgfqpoint{9.826670in}{1.482887in}}{\pgfqpoint{9.821832in}{1.480883in}}{\pgfqpoint{9.818266in}{1.477316in}}%
\pgfpathcurveto{\pgfqpoint{9.814699in}{1.473750in}}{\pgfqpoint{9.812695in}{1.468912in}}{\pgfqpoint{9.812695in}{1.463869in}}%
\pgfpathcurveto{\pgfqpoint{9.812695in}{1.458825in}}{\pgfqpoint{9.814699in}{1.453987in}}{\pgfqpoint{9.818266in}{1.450421in}}%
\pgfpathcurveto{\pgfqpoint{9.821832in}{1.446854in}}{\pgfqpoint{9.826670in}{1.444850in}}{\pgfqpoint{9.831714in}{1.444850in}}%
\pgfpathclose%
\pgfusepath{fill}%
\end{pgfscope}%
\begin{pgfscope}%
\pgfpathrectangle{\pgfqpoint{6.572727in}{0.474100in}}{\pgfqpoint{4.227273in}{3.318700in}}%
\pgfusepath{clip}%
\pgfsetbuttcap%
\pgfsetroundjoin%
\definecolor{currentfill}{rgb}{0.127568,0.566949,0.550556}%
\pgfsetfillcolor{currentfill}%
\pgfsetfillopacity{0.700000}%
\pgfsetlinewidth{0.000000pt}%
\definecolor{currentstroke}{rgb}{0.000000,0.000000,0.000000}%
\pgfsetstrokecolor{currentstroke}%
\pgfsetstrokeopacity{0.700000}%
\pgfsetdash{}{0pt}%
\pgfpathmoveto{\pgfqpoint{9.821301in}{2.324524in}}%
\pgfpathcurveto{\pgfqpoint{9.826345in}{2.324524in}}{\pgfqpoint{9.831182in}{2.326528in}}{\pgfqpoint{9.834749in}{2.330095in}}%
\pgfpathcurveto{\pgfqpoint{9.838315in}{2.333661in}}{\pgfqpoint{9.840319in}{2.338499in}}{\pgfqpoint{9.840319in}{2.343542in}}%
\pgfpathcurveto{\pgfqpoint{9.840319in}{2.348586in}}{\pgfqpoint{9.838315in}{2.353424in}}{\pgfqpoint{9.834749in}{2.356990in}}%
\pgfpathcurveto{\pgfqpoint{9.831182in}{2.360557in}}{\pgfqpoint{9.826345in}{2.362561in}}{\pgfqpoint{9.821301in}{2.362561in}}%
\pgfpathcurveto{\pgfqpoint{9.816257in}{2.362561in}}{\pgfqpoint{9.811420in}{2.360557in}}{\pgfqpoint{9.807853in}{2.356990in}}%
\pgfpathcurveto{\pgfqpoint{9.804287in}{2.353424in}}{\pgfqpoint{9.802283in}{2.348586in}}{\pgfqpoint{9.802283in}{2.343542in}}%
\pgfpathcurveto{\pgfqpoint{9.802283in}{2.338499in}}{\pgfqpoint{9.804287in}{2.333661in}}{\pgfqpoint{9.807853in}{2.330095in}}%
\pgfpathcurveto{\pgfqpoint{9.811420in}{2.326528in}}{\pgfqpoint{9.816257in}{2.324524in}}{\pgfqpoint{9.821301in}{2.324524in}}%
\pgfpathclose%
\pgfusepath{fill}%
\end{pgfscope}%
\begin{pgfscope}%
\pgfpathrectangle{\pgfqpoint{6.572727in}{0.474100in}}{\pgfqpoint{4.227273in}{3.318700in}}%
\pgfusepath{clip}%
\pgfsetbuttcap%
\pgfsetroundjoin%
\definecolor{currentfill}{rgb}{0.267004,0.004874,0.329415}%
\pgfsetfillcolor{currentfill}%
\pgfsetfillopacity{0.700000}%
\pgfsetlinewidth{0.000000pt}%
\definecolor{currentstroke}{rgb}{0.000000,0.000000,0.000000}%
\pgfsetstrokecolor{currentstroke}%
\pgfsetstrokeopacity{0.700000}%
\pgfsetdash{}{0pt}%
\pgfpathmoveto{\pgfqpoint{8.129570in}{1.380188in}}%
\pgfpathcurveto{\pgfqpoint{8.134614in}{1.380188in}}{\pgfqpoint{8.139452in}{1.382192in}}{\pgfqpoint{8.143018in}{1.385759in}}%
\pgfpathcurveto{\pgfqpoint{8.146585in}{1.389325in}}{\pgfqpoint{8.148589in}{1.394163in}}{\pgfqpoint{8.148589in}{1.399207in}}%
\pgfpathcurveto{\pgfqpoint{8.148589in}{1.404250in}}{\pgfqpoint{8.146585in}{1.409088in}}{\pgfqpoint{8.143018in}{1.412654in}}%
\pgfpathcurveto{\pgfqpoint{8.139452in}{1.416221in}}{\pgfqpoint{8.134614in}{1.418225in}}{\pgfqpoint{8.129570in}{1.418225in}}%
\pgfpathcurveto{\pgfqpoint{8.124527in}{1.418225in}}{\pgfqpoint{8.119689in}{1.416221in}}{\pgfqpoint{8.116123in}{1.412654in}}%
\pgfpathcurveto{\pgfqpoint{8.112556in}{1.409088in}}{\pgfqpoint{8.110552in}{1.404250in}}{\pgfqpoint{8.110552in}{1.399207in}}%
\pgfpathcurveto{\pgfqpoint{8.110552in}{1.394163in}}{\pgfqpoint{8.112556in}{1.389325in}}{\pgfqpoint{8.116123in}{1.385759in}}%
\pgfpathcurveto{\pgfqpoint{8.119689in}{1.382192in}}{\pgfqpoint{8.124527in}{1.380188in}}{\pgfqpoint{8.129570in}{1.380188in}}%
\pgfpathclose%
\pgfusepath{fill}%
\end{pgfscope}%
\begin{pgfscope}%
\pgfpathrectangle{\pgfqpoint{6.572727in}{0.474100in}}{\pgfqpoint{4.227273in}{3.318700in}}%
\pgfusepath{clip}%
\pgfsetbuttcap%
\pgfsetroundjoin%
\definecolor{currentfill}{rgb}{0.127568,0.566949,0.550556}%
\pgfsetfillcolor{currentfill}%
\pgfsetfillopacity{0.700000}%
\pgfsetlinewidth{0.000000pt}%
\definecolor{currentstroke}{rgb}{0.000000,0.000000,0.000000}%
\pgfsetstrokecolor{currentstroke}%
\pgfsetstrokeopacity{0.700000}%
\pgfsetdash{}{0pt}%
\pgfpathmoveto{\pgfqpoint{9.330031in}{2.075572in}}%
\pgfpathcurveto{\pgfqpoint{9.335075in}{2.075572in}}{\pgfqpoint{9.339912in}{2.077576in}}{\pgfqpoint{9.343479in}{2.081143in}}%
\pgfpathcurveto{\pgfqpoint{9.347045in}{2.084709in}}{\pgfqpoint{9.349049in}{2.089547in}}{\pgfqpoint{9.349049in}{2.094591in}}%
\pgfpathcurveto{\pgfqpoint{9.349049in}{2.099634in}}{\pgfqpoint{9.347045in}{2.104472in}}{\pgfqpoint{9.343479in}{2.108038in}}%
\pgfpathcurveto{\pgfqpoint{9.339912in}{2.111605in}}{\pgfqpoint{9.335075in}{2.113609in}}{\pgfqpoint{9.330031in}{2.113609in}}%
\pgfpathcurveto{\pgfqpoint{9.324987in}{2.113609in}}{\pgfqpoint{9.320150in}{2.111605in}}{\pgfqpoint{9.316583in}{2.108038in}}%
\pgfpathcurveto{\pgfqpoint{9.313017in}{2.104472in}}{\pgfqpoint{9.311013in}{2.099634in}}{\pgfqpoint{9.311013in}{2.094591in}}%
\pgfpathcurveto{\pgfqpoint{9.311013in}{2.089547in}}{\pgfqpoint{9.313017in}{2.084709in}}{\pgfqpoint{9.316583in}{2.081143in}}%
\pgfpathcurveto{\pgfqpoint{9.320150in}{2.077576in}}{\pgfqpoint{9.324987in}{2.075572in}}{\pgfqpoint{9.330031in}{2.075572in}}%
\pgfpathclose%
\pgfusepath{fill}%
\end{pgfscope}%
\begin{pgfscope}%
\pgfpathrectangle{\pgfqpoint{6.572727in}{0.474100in}}{\pgfqpoint{4.227273in}{3.318700in}}%
\pgfusepath{clip}%
\pgfsetbuttcap%
\pgfsetroundjoin%
\definecolor{currentfill}{rgb}{0.993248,0.906157,0.143936}%
\pgfsetfillcolor{currentfill}%
\pgfsetfillopacity{0.700000}%
\pgfsetlinewidth{0.000000pt}%
\definecolor{currentstroke}{rgb}{0.000000,0.000000,0.000000}%
\pgfsetstrokecolor{currentstroke}%
\pgfsetstrokeopacity{0.700000}%
\pgfsetdash{}{0pt}%
\pgfpathmoveto{\pgfqpoint{8.165087in}{2.963310in}}%
\pgfpathcurveto{\pgfqpoint{8.170131in}{2.963310in}}{\pgfqpoint{8.174969in}{2.965314in}}{\pgfqpoint{8.178535in}{2.968880in}}%
\pgfpathcurveto{\pgfqpoint{8.182101in}{2.972447in}}{\pgfqpoint{8.184105in}{2.977285in}}{\pgfqpoint{8.184105in}{2.982328in}}%
\pgfpathcurveto{\pgfqpoint{8.184105in}{2.987372in}}{\pgfqpoint{8.182101in}{2.992210in}}{\pgfqpoint{8.178535in}{2.995776in}}%
\pgfpathcurveto{\pgfqpoint{8.174969in}{2.999343in}}{\pgfqpoint{8.170131in}{3.001347in}}{\pgfqpoint{8.165087in}{3.001347in}}%
\pgfpathcurveto{\pgfqpoint{8.160043in}{3.001347in}}{\pgfqpoint{8.155206in}{2.999343in}}{\pgfqpoint{8.151639in}{2.995776in}}%
\pgfpathcurveto{\pgfqpoint{8.148073in}{2.992210in}}{\pgfqpoint{8.146069in}{2.987372in}}{\pgfqpoint{8.146069in}{2.982328in}}%
\pgfpathcurveto{\pgfqpoint{8.146069in}{2.977285in}}{\pgfqpoint{8.148073in}{2.972447in}}{\pgfqpoint{8.151639in}{2.968880in}}%
\pgfpathcurveto{\pgfqpoint{8.155206in}{2.965314in}}{\pgfqpoint{8.160043in}{2.963310in}}{\pgfqpoint{8.165087in}{2.963310in}}%
\pgfpathclose%
\pgfusepath{fill}%
\end{pgfscope}%
\begin{pgfscope}%
\pgfpathrectangle{\pgfqpoint{6.572727in}{0.474100in}}{\pgfqpoint{4.227273in}{3.318700in}}%
\pgfusepath{clip}%
\pgfsetbuttcap%
\pgfsetroundjoin%
\definecolor{currentfill}{rgb}{0.267004,0.004874,0.329415}%
\pgfsetfillcolor{currentfill}%
\pgfsetfillopacity{0.700000}%
\pgfsetlinewidth{0.000000pt}%
\definecolor{currentstroke}{rgb}{0.000000,0.000000,0.000000}%
\pgfsetstrokecolor{currentstroke}%
\pgfsetstrokeopacity{0.700000}%
\pgfsetdash{}{0pt}%
\pgfpathmoveto{\pgfqpoint{7.494331in}{1.378188in}}%
\pgfpathcurveto{\pgfqpoint{7.499374in}{1.378188in}}{\pgfqpoint{7.504212in}{1.380192in}}{\pgfqpoint{7.507778in}{1.383758in}}%
\pgfpathcurveto{\pgfqpoint{7.511345in}{1.387325in}}{\pgfqpoint{7.513349in}{1.392163in}}{\pgfqpoint{7.513349in}{1.397206in}}%
\pgfpathcurveto{\pgfqpoint{7.513349in}{1.402250in}}{\pgfqpoint{7.511345in}{1.407088in}}{\pgfqpoint{7.507778in}{1.410654in}}%
\pgfpathcurveto{\pgfqpoint{7.504212in}{1.414221in}}{\pgfqpoint{7.499374in}{1.416224in}}{\pgfqpoint{7.494331in}{1.416224in}}%
\pgfpathcurveto{\pgfqpoint{7.489287in}{1.416224in}}{\pgfqpoint{7.484449in}{1.414221in}}{\pgfqpoint{7.480883in}{1.410654in}}%
\pgfpathcurveto{\pgfqpoint{7.477316in}{1.407088in}}{\pgfqpoint{7.475312in}{1.402250in}}{\pgfqpoint{7.475312in}{1.397206in}}%
\pgfpathcurveto{\pgfqpoint{7.475312in}{1.392163in}}{\pgfqpoint{7.477316in}{1.387325in}}{\pgfqpoint{7.480883in}{1.383758in}}%
\pgfpathcurveto{\pgfqpoint{7.484449in}{1.380192in}}{\pgfqpoint{7.489287in}{1.378188in}}{\pgfqpoint{7.494331in}{1.378188in}}%
\pgfpathclose%
\pgfusepath{fill}%
\end{pgfscope}%
\begin{pgfscope}%
\pgfpathrectangle{\pgfqpoint{6.572727in}{0.474100in}}{\pgfqpoint{4.227273in}{3.318700in}}%
\pgfusepath{clip}%
\pgfsetbuttcap%
\pgfsetroundjoin%
\definecolor{currentfill}{rgb}{0.267004,0.004874,0.329415}%
\pgfsetfillcolor{currentfill}%
\pgfsetfillopacity{0.700000}%
\pgfsetlinewidth{0.000000pt}%
\definecolor{currentstroke}{rgb}{0.000000,0.000000,0.000000}%
\pgfsetstrokecolor{currentstroke}%
\pgfsetstrokeopacity{0.700000}%
\pgfsetdash{}{0pt}%
\pgfpathmoveto{\pgfqpoint{7.507483in}{1.657561in}}%
\pgfpathcurveto{\pgfqpoint{7.512527in}{1.657561in}}{\pgfqpoint{7.517365in}{1.659565in}}{\pgfqpoint{7.520931in}{1.663132in}}%
\pgfpathcurveto{\pgfqpoint{7.524498in}{1.666698in}}{\pgfqpoint{7.526502in}{1.671536in}}{\pgfqpoint{7.526502in}{1.676579in}}%
\pgfpathcurveto{\pgfqpoint{7.526502in}{1.681623in}}{\pgfqpoint{7.524498in}{1.686461in}}{\pgfqpoint{7.520931in}{1.690027in}}%
\pgfpathcurveto{\pgfqpoint{7.517365in}{1.693594in}}{\pgfqpoint{7.512527in}{1.695598in}}{\pgfqpoint{7.507483in}{1.695598in}}%
\pgfpathcurveto{\pgfqpoint{7.502440in}{1.695598in}}{\pgfqpoint{7.497602in}{1.693594in}}{\pgfqpoint{7.494036in}{1.690027in}}%
\pgfpathcurveto{\pgfqpoint{7.490469in}{1.686461in}}{\pgfqpoint{7.488465in}{1.681623in}}{\pgfqpoint{7.488465in}{1.676579in}}%
\pgfpathcurveto{\pgfqpoint{7.488465in}{1.671536in}}{\pgfqpoint{7.490469in}{1.666698in}}{\pgfqpoint{7.494036in}{1.663132in}}%
\pgfpathcurveto{\pgfqpoint{7.497602in}{1.659565in}}{\pgfqpoint{7.502440in}{1.657561in}}{\pgfqpoint{7.507483in}{1.657561in}}%
\pgfpathclose%
\pgfusepath{fill}%
\end{pgfscope}%
\begin{pgfscope}%
\pgfpathrectangle{\pgfqpoint{6.572727in}{0.474100in}}{\pgfqpoint{4.227273in}{3.318700in}}%
\pgfusepath{clip}%
\pgfsetbuttcap%
\pgfsetroundjoin%
\definecolor{currentfill}{rgb}{0.127568,0.566949,0.550556}%
\pgfsetfillcolor{currentfill}%
\pgfsetfillopacity{0.700000}%
\pgfsetlinewidth{0.000000pt}%
\definecolor{currentstroke}{rgb}{0.000000,0.000000,0.000000}%
\pgfsetstrokecolor{currentstroke}%
\pgfsetstrokeopacity{0.700000}%
\pgfsetdash{}{0pt}%
\pgfpathmoveto{\pgfqpoint{10.420660in}{1.778469in}}%
\pgfpathcurveto{\pgfqpoint{10.425704in}{1.778469in}}{\pgfqpoint{10.430541in}{1.780473in}}{\pgfqpoint{10.434108in}{1.784040in}}%
\pgfpathcurveto{\pgfqpoint{10.437674in}{1.787606in}}{\pgfqpoint{10.439678in}{1.792444in}}{\pgfqpoint{10.439678in}{1.797487in}}%
\pgfpathcurveto{\pgfqpoint{10.439678in}{1.802531in}}{\pgfqpoint{10.437674in}{1.807369in}}{\pgfqpoint{10.434108in}{1.810935in}}%
\pgfpathcurveto{\pgfqpoint{10.430541in}{1.814502in}}{\pgfqpoint{10.425704in}{1.816506in}}{\pgfqpoint{10.420660in}{1.816506in}}%
\pgfpathcurveto{\pgfqpoint{10.415616in}{1.816506in}}{\pgfqpoint{10.410778in}{1.814502in}}{\pgfqpoint{10.407212in}{1.810935in}}%
\pgfpathcurveto{\pgfqpoint{10.403646in}{1.807369in}}{\pgfqpoint{10.401642in}{1.802531in}}{\pgfqpoint{10.401642in}{1.797487in}}%
\pgfpathcurveto{\pgfqpoint{10.401642in}{1.792444in}}{\pgfqpoint{10.403646in}{1.787606in}}{\pgfqpoint{10.407212in}{1.784040in}}%
\pgfpathcurveto{\pgfqpoint{10.410778in}{1.780473in}}{\pgfqpoint{10.415616in}{1.778469in}}{\pgfqpoint{10.420660in}{1.778469in}}%
\pgfpathclose%
\pgfusepath{fill}%
\end{pgfscope}%
\begin{pgfscope}%
\pgfpathrectangle{\pgfqpoint{6.572727in}{0.474100in}}{\pgfqpoint{4.227273in}{3.318700in}}%
\pgfusepath{clip}%
\pgfsetbuttcap%
\pgfsetroundjoin%
\definecolor{currentfill}{rgb}{0.127568,0.566949,0.550556}%
\pgfsetfillcolor{currentfill}%
\pgfsetfillopacity{0.700000}%
\pgfsetlinewidth{0.000000pt}%
\definecolor{currentstroke}{rgb}{0.000000,0.000000,0.000000}%
\pgfsetstrokecolor{currentstroke}%
\pgfsetstrokeopacity{0.700000}%
\pgfsetdash{}{0pt}%
\pgfpathmoveto{\pgfqpoint{9.917552in}{0.937249in}}%
\pgfpathcurveto{\pgfqpoint{9.922596in}{0.937249in}}{\pgfqpoint{9.927433in}{0.939253in}}{\pgfqpoint{9.931000in}{0.942819in}}%
\pgfpathcurveto{\pgfqpoint{9.934566in}{0.946386in}}{\pgfqpoint{9.936570in}{0.951224in}}{\pgfqpoint{9.936570in}{0.956267in}}%
\pgfpathcurveto{\pgfqpoint{9.936570in}{0.961311in}}{\pgfqpoint{9.934566in}{0.966149in}}{\pgfqpoint{9.931000in}{0.969715in}}%
\pgfpathcurveto{\pgfqpoint{9.927433in}{0.973282in}}{\pgfqpoint{9.922596in}{0.975285in}}{\pgfqpoint{9.917552in}{0.975285in}}%
\pgfpathcurveto{\pgfqpoint{9.912508in}{0.975285in}}{\pgfqpoint{9.907670in}{0.973282in}}{\pgfqpoint{9.904104in}{0.969715in}}%
\pgfpathcurveto{\pgfqpoint{9.900538in}{0.966149in}}{\pgfqpoint{9.898534in}{0.961311in}}{\pgfqpoint{9.898534in}{0.956267in}}%
\pgfpathcurveto{\pgfqpoint{9.898534in}{0.951224in}}{\pgfqpoint{9.900538in}{0.946386in}}{\pgfqpoint{9.904104in}{0.942819in}}%
\pgfpathcurveto{\pgfqpoint{9.907670in}{0.939253in}}{\pgfqpoint{9.912508in}{0.937249in}}{\pgfqpoint{9.917552in}{0.937249in}}%
\pgfpathclose%
\pgfusepath{fill}%
\end{pgfscope}%
\begin{pgfscope}%
\pgfpathrectangle{\pgfqpoint{6.572727in}{0.474100in}}{\pgfqpoint{4.227273in}{3.318700in}}%
\pgfusepath{clip}%
\pgfsetbuttcap%
\pgfsetroundjoin%
\definecolor{currentfill}{rgb}{0.127568,0.566949,0.550556}%
\pgfsetfillcolor{currentfill}%
\pgfsetfillopacity{0.700000}%
\pgfsetlinewidth{0.000000pt}%
\definecolor{currentstroke}{rgb}{0.000000,0.000000,0.000000}%
\pgfsetstrokecolor{currentstroke}%
\pgfsetstrokeopacity{0.700000}%
\pgfsetdash{}{0pt}%
\pgfpathmoveto{\pgfqpoint{9.715546in}{1.425406in}}%
\pgfpathcurveto{\pgfqpoint{9.720590in}{1.425406in}}{\pgfqpoint{9.725428in}{1.427410in}}{\pgfqpoint{9.728994in}{1.430976in}}%
\pgfpathcurveto{\pgfqpoint{9.732560in}{1.434542in}}{\pgfqpoint{9.734564in}{1.439380in}}{\pgfqpoint{9.734564in}{1.444424in}}%
\pgfpathcurveto{\pgfqpoint{9.734564in}{1.449468in}}{\pgfqpoint{9.732560in}{1.454305in}}{\pgfqpoint{9.728994in}{1.457872in}}%
\pgfpathcurveto{\pgfqpoint{9.725428in}{1.461438in}}{\pgfqpoint{9.720590in}{1.463442in}}{\pgfqpoint{9.715546in}{1.463442in}}%
\pgfpathcurveto{\pgfqpoint{9.710503in}{1.463442in}}{\pgfqpoint{9.705665in}{1.461438in}}{\pgfqpoint{9.702098in}{1.457872in}}%
\pgfpathcurveto{\pgfqpoint{9.698532in}{1.454305in}}{\pgfqpoint{9.696528in}{1.449468in}}{\pgfqpoint{9.696528in}{1.444424in}}%
\pgfpathcurveto{\pgfqpoint{9.696528in}{1.439380in}}{\pgfqpoint{9.698532in}{1.434542in}}{\pgfqpoint{9.702098in}{1.430976in}}%
\pgfpathcurveto{\pgfqpoint{9.705665in}{1.427410in}}{\pgfqpoint{9.710503in}{1.425406in}}{\pgfqpoint{9.715546in}{1.425406in}}%
\pgfpathclose%
\pgfusepath{fill}%
\end{pgfscope}%
\begin{pgfscope}%
\pgfpathrectangle{\pgfqpoint{6.572727in}{0.474100in}}{\pgfqpoint{4.227273in}{3.318700in}}%
\pgfusepath{clip}%
\pgfsetbuttcap%
\pgfsetroundjoin%
\definecolor{currentfill}{rgb}{0.127568,0.566949,0.550556}%
\pgfsetfillcolor{currentfill}%
\pgfsetfillopacity{0.700000}%
\pgfsetlinewidth{0.000000pt}%
\definecolor{currentstroke}{rgb}{0.000000,0.000000,0.000000}%
\pgfsetstrokecolor{currentstroke}%
\pgfsetstrokeopacity{0.700000}%
\pgfsetdash{}{0pt}%
\pgfpathmoveto{\pgfqpoint{9.740088in}{1.659490in}}%
\pgfpathcurveto{\pgfqpoint{9.745132in}{1.659490in}}{\pgfqpoint{9.749970in}{1.661494in}}{\pgfqpoint{9.753536in}{1.665060in}}%
\pgfpathcurveto{\pgfqpoint{9.757103in}{1.668626in}}{\pgfqpoint{9.759107in}{1.673464in}}{\pgfqpoint{9.759107in}{1.678508in}}%
\pgfpathcurveto{\pgfqpoint{9.759107in}{1.683552in}}{\pgfqpoint{9.757103in}{1.688389in}}{\pgfqpoint{9.753536in}{1.691956in}}%
\pgfpathcurveto{\pgfqpoint{9.749970in}{1.695522in}}{\pgfqpoint{9.745132in}{1.697526in}}{\pgfqpoint{9.740088in}{1.697526in}}%
\pgfpathcurveto{\pgfqpoint{9.735045in}{1.697526in}}{\pgfqpoint{9.730207in}{1.695522in}}{\pgfqpoint{9.726641in}{1.691956in}}%
\pgfpathcurveto{\pgfqpoint{9.723074in}{1.688389in}}{\pgfqpoint{9.721070in}{1.683552in}}{\pgfqpoint{9.721070in}{1.678508in}}%
\pgfpathcurveto{\pgfqpoint{9.721070in}{1.673464in}}{\pgfqpoint{9.723074in}{1.668626in}}{\pgfqpoint{9.726641in}{1.665060in}}%
\pgfpathcurveto{\pgfqpoint{9.730207in}{1.661494in}}{\pgfqpoint{9.735045in}{1.659490in}}{\pgfqpoint{9.740088in}{1.659490in}}%
\pgfpathclose%
\pgfusepath{fill}%
\end{pgfscope}%
\begin{pgfscope}%
\pgfpathrectangle{\pgfqpoint{6.572727in}{0.474100in}}{\pgfqpoint{4.227273in}{3.318700in}}%
\pgfusepath{clip}%
\pgfsetbuttcap%
\pgfsetroundjoin%
\definecolor{currentfill}{rgb}{0.267004,0.004874,0.329415}%
\pgfsetfillcolor{currentfill}%
\pgfsetfillopacity{0.700000}%
\pgfsetlinewidth{0.000000pt}%
\definecolor{currentstroke}{rgb}{0.000000,0.000000,0.000000}%
\pgfsetstrokecolor{currentstroke}%
\pgfsetstrokeopacity{0.700000}%
\pgfsetdash{}{0pt}%
\pgfpathmoveto{\pgfqpoint{7.619632in}{1.645548in}}%
\pgfpathcurveto{\pgfqpoint{7.624675in}{1.645548in}}{\pgfqpoint{7.629513in}{1.647551in}}{\pgfqpoint{7.633080in}{1.651118in}}%
\pgfpathcurveto{\pgfqpoint{7.636646in}{1.654684in}}{\pgfqpoint{7.638650in}{1.659522in}}{\pgfqpoint{7.638650in}{1.664566in}}%
\pgfpathcurveto{\pgfqpoint{7.638650in}{1.669609in}}{\pgfqpoint{7.636646in}{1.674447in}}{\pgfqpoint{7.633080in}{1.678014in}}%
\pgfpathcurveto{\pgfqpoint{7.629513in}{1.681580in}}{\pgfqpoint{7.624675in}{1.683584in}}{\pgfqpoint{7.619632in}{1.683584in}}%
\pgfpathcurveto{\pgfqpoint{7.614588in}{1.683584in}}{\pgfqpoint{7.609750in}{1.681580in}}{\pgfqpoint{7.606184in}{1.678014in}}%
\pgfpathcurveto{\pgfqpoint{7.602618in}{1.674447in}}{\pgfqpoint{7.600614in}{1.669609in}}{\pgfqpoint{7.600614in}{1.664566in}}%
\pgfpathcurveto{\pgfqpoint{7.600614in}{1.659522in}}{\pgfqpoint{7.602618in}{1.654684in}}{\pgfqpoint{7.606184in}{1.651118in}}%
\pgfpathcurveto{\pgfqpoint{7.609750in}{1.647551in}}{\pgfqpoint{7.614588in}{1.645548in}}{\pgfqpoint{7.619632in}{1.645548in}}%
\pgfpathclose%
\pgfusepath{fill}%
\end{pgfscope}%
\begin{pgfscope}%
\pgfpathrectangle{\pgfqpoint{6.572727in}{0.474100in}}{\pgfqpoint{4.227273in}{3.318700in}}%
\pgfusepath{clip}%
\pgfsetbuttcap%
\pgfsetroundjoin%
\definecolor{currentfill}{rgb}{0.127568,0.566949,0.550556}%
\pgfsetfillcolor{currentfill}%
\pgfsetfillopacity{0.700000}%
\pgfsetlinewidth{0.000000pt}%
\definecolor{currentstroke}{rgb}{0.000000,0.000000,0.000000}%
\pgfsetstrokecolor{currentstroke}%
\pgfsetstrokeopacity{0.700000}%
\pgfsetdash{}{0pt}%
\pgfpathmoveto{\pgfqpoint{9.385049in}{1.613877in}}%
\pgfpathcurveto{\pgfqpoint{9.390093in}{1.613877in}}{\pgfqpoint{9.394931in}{1.615881in}}{\pgfqpoint{9.398497in}{1.619448in}}%
\pgfpathcurveto{\pgfqpoint{9.402063in}{1.623014in}}{\pgfqpoint{9.404067in}{1.627852in}}{\pgfqpoint{9.404067in}{1.632895in}}%
\pgfpathcurveto{\pgfqpoint{9.404067in}{1.637939in}}{\pgfqpoint{9.402063in}{1.642777in}}{\pgfqpoint{9.398497in}{1.646343in}}%
\pgfpathcurveto{\pgfqpoint{9.394931in}{1.649910in}}{\pgfqpoint{9.390093in}{1.651914in}}{\pgfqpoint{9.385049in}{1.651914in}}%
\pgfpathcurveto{\pgfqpoint{9.380006in}{1.651914in}}{\pgfqpoint{9.375168in}{1.649910in}}{\pgfqpoint{9.371601in}{1.646343in}}%
\pgfpathcurveto{\pgfqpoint{9.368035in}{1.642777in}}{\pgfqpoint{9.366031in}{1.637939in}}{\pgfqpoint{9.366031in}{1.632895in}}%
\pgfpathcurveto{\pgfqpoint{9.366031in}{1.627852in}}{\pgfqpoint{9.368035in}{1.623014in}}{\pgfqpoint{9.371601in}{1.619448in}}%
\pgfpathcurveto{\pgfqpoint{9.375168in}{1.615881in}}{\pgfqpoint{9.380006in}{1.613877in}}{\pgfqpoint{9.385049in}{1.613877in}}%
\pgfpathclose%
\pgfusepath{fill}%
\end{pgfscope}%
\begin{pgfscope}%
\pgfpathrectangle{\pgfqpoint{6.572727in}{0.474100in}}{\pgfqpoint{4.227273in}{3.318700in}}%
\pgfusepath{clip}%
\pgfsetbuttcap%
\pgfsetroundjoin%
\definecolor{currentfill}{rgb}{0.267004,0.004874,0.329415}%
\pgfsetfillcolor{currentfill}%
\pgfsetfillopacity{0.700000}%
\pgfsetlinewidth{0.000000pt}%
\definecolor{currentstroke}{rgb}{0.000000,0.000000,0.000000}%
\pgfsetstrokecolor{currentstroke}%
\pgfsetstrokeopacity{0.700000}%
\pgfsetdash{}{0pt}%
\pgfpathmoveto{\pgfqpoint{6.836372in}{1.169495in}}%
\pgfpathcurveto{\pgfqpoint{6.841416in}{1.169495in}}{\pgfqpoint{6.846253in}{1.171499in}}{\pgfqpoint{6.849820in}{1.175065in}}%
\pgfpathcurveto{\pgfqpoint{6.853386in}{1.178632in}}{\pgfqpoint{6.855390in}{1.183470in}}{\pgfqpoint{6.855390in}{1.188513in}}%
\pgfpathcurveto{\pgfqpoint{6.855390in}{1.193557in}}{\pgfqpoint{6.853386in}{1.198395in}}{\pgfqpoint{6.849820in}{1.201961in}}%
\pgfpathcurveto{\pgfqpoint{6.846253in}{1.205527in}}{\pgfqpoint{6.841416in}{1.207531in}}{\pgfqpoint{6.836372in}{1.207531in}}%
\pgfpathcurveto{\pgfqpoint{6.831328in}{1.207531in}}{\pgfqpoint{6.826490in}{1.205527in}}{\pgfqpoint{6.822924in}{1.201961in}}%
\pgfpathcurveto{\pgfqpoint{6.819358in}{1.198395in}}{\pgfqpoint{6.817354in}{1.193557in}}{\pgfqpoint{6.817354in}{1.188513in}}%
\pgfpathcurveto{\pgfqpoint{6.817354in}{1.183470in}}{\pgfqpoint{6.819358in}{1.178632in}}{\pgfqpoint{6.822924in}{1.175065in}}%
\pgfpathcurveto{\pgfqpoint{6.826490in}{1.171499in}}{\pgfqpoint{6.831328in}{1.169495in}}{\pgfqpoint{6.836372in}{1.169495in}}%
\pgfpathclose%
\pgfusepath{fill}%
\end{pgfscope}%
\begin{pgfscope}%
\pgfpathrectangle{\pgfqpoint{6.572727in}{0.474100in}}{\pgfqpoint{4.227273in}{3.318700in}}%
\pgfusepath{clip}%
\pgfsetbuttcap%
\pgfsetroundjoin%
\definecolor{currentfill}{rgb}{0.127568,0.566949,0.550556}%
\pgfsetfillcolor{currentfill}%
\pgfsetfillopacity{0.700000}%
\pgfsetlinewidth{0.000000pt}%
\definecolor{currentstroke}{rgb}{0.000000,0.000000,0.000000}%
\pgfsetstrokecolor{currentstroke}%
\pgfsetstrokeopacity{0.700000}%
\pgfsetdash{}{0pt}%
\pgfpathmoveto{\pgfqpoint{9.283312in}{2.029996in}}%
\pgfpathcurveto{\pgfqpoint{9.288356in}{2.029996in}}{\pgfqpoint{9.293193in}{2.031999in}}{\pgfqpoint{9.296760in}{2.035566in}}%
\pgfpathcurveto{\pgfqpoint{9.300326in}{2.039132in}}{\pgfqpoint{9.302330in}{2.043970in}}{\pgfqpoint{9.302330in}{2.049014in}}%
\pgfpathcurveto{\pgfqpoint{9.302330in}{2.054057in}}{\pgfqpoint{9.300326in}{2.058895in}}{\pgfqpoint{9.296760in}{2.062462in}}%
\pgfpathcurveto{\pgfqpoint{9.293193in}{2.066028in}}{\pgfqpoint{9.288356in}{2.068032in}}{\pgfqpoint{9.283312in}{2.068032in}}%
\pgfpathcurveto{\pgfqpoint{9.278268in}{2.068032in}}{\pgfqpoint{9.273431in}{2.066028in}}{\pgfqpoint{9.269864in}{2.062462in}}%
\pgfpathcurveto{\pgfqpoint{9.266298in}{2.058895in}}{\pgfqpoint{9.264294in}{2.054057in}}{\pgfqpoint{9.264294in}{2.049014in}}%
\pgfpathcurveto{\pgfqpoint{9.264294in}{2.043970in}}{\pgfqpoint{9.266298in}{2.039132in}}{\pgfqpoint{9.269864in}{2.035566in}}%
\pgfpathcurveto{\pgfqpoint{9.273431in}{2.031999in}}{\pgfqpoint{9.278268in}{2.029996in}}{\pgfqpoint{9.283312in}{2.029996in}}%
\pgfpathclose%
\pgfusepath{fill}%
\end{pgfscope}%
\begin{pgfscope}%
\pgfpathrectangle{\pgfqpoint{6.572727in}{0.474100in}}{\pgfqpoint{4.227273in}{3.318700in}}%
\pgfusepath{clip}%
\pgfsetbuttcap%
\pgfsetroundjoin%
\definecolor{currentfill}{rgb}{0.267004,0.004874,0.329415}%
\pgfsetfillcolor{currentfill}%
\pgfsetfillopacity{0.700000}%
\pgfsetlinewidth{0.000000pt}%
\definecolor{currentstroke}{rgb}{0.000000,0.000000,0.000000}%
\pgfsetstrokecolor{currentstroke}%
\pgfsetstrokeopacity{0.700000}%
\pgfsetdash{}{0pt}%
\pgfpathmoveto{\pgfqpoint{7.758034in}{1.643210in}}%
\pgfpathcurveto{\pgfqpoint{7.763078in}{1.643210in}}{\pgfqpoint{7.767915in}{1.645214in}}{\pgfqpoint{7.771482in}{1.648780in}}%
\pgfpathcurveto{\pgfqpoint{7.775048in}{1.652347in}}{\pgfqpoint{7.777052in}{1.657184in}}{\pgfqpoint{7.777052in}{1.662228in}}%
\pgfpathcurveto{\pgfqpoint{7.777052in}{1.667272in}}{\pgfqpoint{7.775048in}{1.672109in}}{\pgfqpoint{7.771482in}{1.675676in}}%
\pgfpathcurveto{\pgfqpoint{7.767915in}{1.679242in}}{\pgfqpoint{7.763078in}{1.681246in}}{\pgfqpoint{7.758034in}{1.681246in}}%
\pgfpathcurveto{\pgfqpoint{7.752990in}{1.681246in}}{\pgfqpoint{7.748152in}{1.679242in}}{\pgfqpoint{7.744586in}{1.675676in}}%
\pgfpathcurveto{\pgfqpoint{7.741020in}{1.672109in}}{\pgfqpoint{7.739016in}{1.667272in}}{\pgfqpoint{7.739016in}{1.662228in}}%
\pgfpathcurveto{\pgfqpoint{7.739016in}{1.657184in}}{\pgfqpoint{7.741020in}{1.652347in}}{\pgfqpoint{7.744586in}{1.648780in}}%
\pgfpathcurveto{\pgfqpoint{7.748152in}{1.645214in}}{\pgfqpoint{7.752990in}{1.643210in}}{\pgfqpoint{7.758034in}{1.643210in}}%
\pgfpathclose%
\pgfusepath{fill}%
\end{pgfscope}%
\begin{pgfscope}%
\pgfpathrectangle{\pgfqpoint{6.572727in}{0.474100in}}{\pgfqpoint{4.227273in}{3.318700in}}%
\pgfusepath{clip}%
\pgfsetbuttcap%
\pgfsetroundjoin%
\definecolor{currentfill}{rgb}{0.127568,0.566949,0.550556}%
\pgfsetfillcolor{currentfill}%
\pgfsetfillopacity{0.700000}%
\pgfsetlinewidth{0.000000pt}%
\definecolor{currentstroke}{rgb}{0.000000,0.000000,0.000000}%
\pgfsetstrokecolor{currentstroke}%
\pgfsetstrokeopacity{0.700000}%
\pgfsetdash{}{0pt}%
\pgfpathmoveto{\pgfqpoint{9.315670in}{1.240538in}}%
\pgfpathcurveto{\pgfqpoint{9.320714in}{1.240538in}}{\pgfqpoint{9.325551in}{1.242542in}}{\pgfqpoint{9.329118in}{1.246108in}}%
\pgfpathcurveto{\pgfqpoint{9.332684in}{1.249674in}}{\pgfqpoint{9.334688in}{1.254512in}}{\pgfqpoint{9.334688in}{1.259556in}}%
\pgfpathcurveto{\pgfqpoint{9.334688in}{1.264600in}}{\pgfqpoint{9.332684in}{1.269437in}}{\pgfqpoint{9.329118in}{1.273004in}}%
\pgfpathcurveto{\pgfqpoint{9.325551in}{1.276570in}}{\pgfqpoint{9.320714in}{1.278574in}}{\pgfqpoint{9.315670in}{1.278574in}}%
\pgfpathcurveto{\pgfqpoint{9.310626in}{1.278574in}}{\pgfqpoint{9.305789in}{1.276570in}}{\pgfqpoint{9.302222in}{1.273004in}}%
\pgfpathcurveto{\pgfqpoint{9.298656in}{1.269437in}}{\pgfqpoint{9.296652in}{1.264600in}}{\pgfqpoint{9.296652in}{1.259556in}}%
\pgfpathcurveto{\pgfqpoint{9.296652in}{1.254512in}}{\pgfqpoint{9.298656in}{1.249674in}}{\pgfqpoint{9.302222in}{1.246108in}}%
\pgfpathcurveto{\pgfqpoint{9.305789in}{1.242542in}}{\pgfqpoint{9.310626in}{1.240538in}}{\pgfqpoint{9.315670in}{1.240538in}}%
\pgfpathclose%
\pgfusepath{fill}%
\end{pgfscope}%
\begin{pgfscope}%
\pgfpathrectangle{\pgfqpoint{6.572727in}{0.474100in}}{\pgfqpoint{4.227273in}{3.318700in}}%
\pgfusepath{clip}%
\pgfsetbuttcap%
\pgfsetroundjoin%
\definecolor{currentfill}{rgb}{0.993248,0.906157,0.143936}%
\pgfsetfillcolor{currentfill}%
\pgfsetfillopacity{0.700000}%
\pgfsetlinewidth{0.000000pt}%
\definecolor{currentstroke}{rgb}{0.000000,0.000000,0.000000}%
\pgfsetstrokecolor{currentstroke}%
\pgfsetstrokeopacity{0.700000}%
\pgfsetdash{}{0pt}%
\pgfpathmoveto{\pgfqpoint{7.663127in}{3.018636in}}%
\pgfpathcurveto{\pgfqpoint{7.668171in}{3.018636in}}{\pgfqpoint{7.673009in}{3.020639in}}{\pgfqpoint{7.676575in}{3.024206in}}%
\pgfpathcurveto{\pgfqpoint{7.680142in}{3.027772in}}{\pgfqpoint{7.682145in}{3.032610in}}{\pgfqpoint{7.682145in}{3.037654in}}%
\pgfpathcurveto{\pgfqpoint{7.682145in}{3.042697in}}{\pgfqpoint{7.680142in}{3.047535in}}{\pgfqpoint{7.676575in}{3.051102in}}%
\pgfpathcurveto{\pgfqpoint{7.673009in}{3.054668in}}{\pgfqpoint{7.668171in}{3.056672in}}{\pgfqpoint{7.663127in}{3.056672in}}%
\pgfpathcurveto{\pgfqpoint{7.658084in}{3.056672in}}{\pgfqpoint{7.653246in}{3.054668in}}{\pgfqpoint{7.649679in}{3.051102in}}%
\pgfpathcurveto{\pgfqpoint{7.646113in}{3.047535in}}{\pgfqpoint{7.644109in}{3.042697in}}{\pgfqpoint{7.644109in}{3.037654in}}%
\pgfpathcurveto{\pgfqpoint{7.644109in}{3.032610in}}{\pgfqpoint{7.646113in}{3.027772in}}{\pgfqpoint{7.649679in}{3.024206in}}%
\pgfpathcurveto{\pgfqpoint{7.653246in}{3.020639in}}{\pgfqpoint{7.658084in}{3.018636in}}{\pgfqpoint{7.663127in}{3.018636in}}%
\pgfpathclose%
\pgfusepath{fill}%
\end{pgfscope}%
\begin{pgfscope}%
\pgfpathrectangle{\pgfqpoint{6.572727in}{0.474100in}}{\pgfqpoint{4.227273in}{3.318700in}}%
\pgfusepath{clip}%
\pgfsetbuttcap%
\pgfsetroundjoin%
\definecolor{currentfill}{rgb}{0.993248,0.906157,0.143936}%
\pgfsetfillcolor{currentfill}%
\pgfsetfillopacity{0.700000}%
\pgfsetlinewidth{0.000000pt}%
\definecolor{currentstroke}{rgb}{0.000000,0.000000,0.000000}%
\pgfsetstrokecolor{currentstroke}%
\pgfsetstrokeopacity{0.700000}%
\pgfsetdash{}{0pt}%
\pgfpathmoveto{\pgfqpoint{7.816652in}{3.116032in}}%
\pgfpathcurveto{\pgfqpoint{7.821696in}{3.116032in}}{\pgfqpoint{7.826533in}{3.118036in}}{\pgfqpoint{7.830100in}{3.121603in}}%
\pgfpathcurveto{\pgfqpoint{7.833666in}{3.125169in}}{\pgfqpoint{7.835670in}{3.130007in}}{\pgfqpoint{7.835670in}{3.135050in}}%
\pgfpathcurveto{\pgfqpoint{7.835670in}{3.140094in}}{\pgfqpoint{7.833666in}{3.144932in}}{\pgfqpoint{7.830100in}{3.148498in}}%
\pgfpathcurveto{\pgfqpoint{7.826533in}{3.152065in}}{\pgfqpoint{7.821696in}{3.154069in}}{\pgfqpoint{7.816652in}{3.154069in}}%
\pgfpathcurveto{\pgfqpoint{7.811608in}{3.154069in}}{\pgfqpoint{7.806771in}{3.152065in}}{\pgfqpoint{7.803204in}{3.148498in}}%
\pgfpathcurveto{\pgfqpoint{7.799638in}{3.144932in}}{\pgfqpoint{7.797634in}{3.140094in}}{\pgfqpoint{7.797634in}{3.135050in}}%
\pgfpathcurveto{\pgfqpoint{7.797634in}{3.130007in}}{\pgfqpoint{7.799638in}{3.125169in}}{\pgfqpoint{7.803204in}{3.121603in}}%
\pgfpathcurveto{\pgfqpoint{7.806771in}{3.118036in}}{\pgfqpoint{7.811608in}{3.116032in}}{\pgfqpoint{7.816652in}{3.116032in}}%
\pgfpathclose%
\pgfusepath{fill}%
\end{pgfscope}%
\begin{pgfscope}%
\pgfpathrectangle{\pgfqpoint{6.572727in}{0.474100in}}{\pgfqpoint{4.227273in}{3.318700in}}%
\pgfusepath{clip}%
\pgfsetbuttcap%
\pgfsetroundjoin%
\definecolor{currentfill}{rgb}{0.127568,0.566949,0.550556}%
\pgfsetfillcolor{currentfill}%
\pgfsetfillopacity{0.700000}%
\pgfsetlinewidth{0.000000pt}%
\definecolor{currentstroke}{rgb}{0.000000,0.000000,0.000000}%
\pgfsetstrokecolor{currentstroke}%
\pgfsetstrokeopacity{0.700000}%
\pgfsetdash{}{0pt}%
\pgfpathmoveto{\pgfqpoint{9.033129in}{2.045610in}}%
\pgfpathcurveto{\pgfqpoint{9.038172in}{2.045610in}}{\pgfqpoint{9.043010in}{2.047614in}}{\pgfqpoint{9.046577in}{2.051180in}}%
\pgfpathcurveto{\pgfqpoint{9.050143in}{2.054746in}}{\pgfqpoint{9.052147in}{2.059584in}}{\pgfqpoint{9.052147in}{2.064628in}}%
\pgfpathcurveto{\pgfqpoint{9.052147in}{2.069672in}}{\pgfqpoint{9.050143in}{2.074509in}}{\pgfqpoint{9.046577in}{2.078076in}}%
\pgfpathcurveto{\pgfqpoint{9.043010in}{2.081642in}}{\pgfqpoint{9.038172in}{2.083646in}}{\pgfqpoint{9.033129in}{2.083646in}}%
\pgfpathcurveto{\pgfqpoint{9.028085in}{2.083646in}}{\pgfqpoint{9.023247in}{2.081642in}}{\pgfqpoint{9.019681in}{2.078076in}}%
\pgfpathcurveto{\pgfqpoint{9.016114in}{2.074509in}}{\pgfqpoint{9.014111in}{2.069672in}}{\pgfqpoint{9.014111in}{2.064628in}}%
\pgfpathcurveto{\pgfqpoint{9.014111in}{2.059584in}}{\pgfqpoint{9.016114in}{2.054746in}}{\pgfqpoint{9.019681in}{2.051180in}}%
\pgfpathcurveto{\pgfqpoint{9.023247in}{2.047614in}}{\pgfqpoint{9.028085in}{2.045610in}}{\pgfqpoint{9.033129in}{2.045610in}}%
\pgfpathclose%
\pgfusepath{fill}%
\end{pgfscope}%
\begin{pgfscope}%
\pgfpathrectangle{\pgfqpoint{6.572727in}{0.474100in}}{\pgfqpoint{4.227273in}{3.318700in}}%
\pgfusepath{clip}%
\pgfsetbuttcap%
\pgfsetroundjoin%
\definecolor{currentfill}{rgb}{0.993248,0.906157,0.143936}%
\pgfsetfillcolor{currentfill}%
\pgfsetfillopacity{0.700000}%
\pgfsetlinewidth{0.000000pt}%
\definecolor{currentstroke}{rgb}{0.000000,0.000000,0.000000}%
\pgfsetstrokecolor{currentstroke}%
\pgfsetstrokeopacity{0.700000}%
\pgfsetdash{}{0pt}%
\pgfpathmoveto{\pgfqpoint{7.749487in}{2.558574in}}%
\pgfpathcurveto{\pgfqpoint{7.754531in}{2.558574in}}{\pgfqpoint{7.759369in}{2.560578in}}{\pgfqpoint{7.762935in}{2.564144in}}%
\pgfpathcurveto{\pgfqpoint{7.766502in}{2.567710in}}{\pgfqpoint{7.768506in}{2.572548in}}{\pgfqpoint{7.768506in}{2.577592in}}%
\pgfpathcurveto{\pgfqpoint{7.768506in}{2.582635in}}{\pgfqpoint{7.766502in}{2.587473in}}{\pgfqpoint{7.762935in}{2.591040in}}%
\pgfpathcurveto{\pgfqpoint{7.759369in}{2.594606in}}{\pgfqpoint{7.754531in}{2.596610in}}{\pgfqpoint{7.749487in}{2.596610in}}%
\pgfpathcurveto{\pgfqpoint{7.744444in}{2.596610in}}{\pgfqpoint{7.739606in}{2.594606in}}{\pgfqpoint{7.736040in}{2.591040in}}%
\pgfpathcurveto{\pgfqpoint{7.732473in}{2.587473in}}{\pgfqpoint{7.730469in}{2.582635in}}{\pgfqpoint{7.730469in}{2.577592in}}%
\pgfpathcurveto{\pgfqpoint{7.730469in}{2.572548in}}{\pgfqpoint{7.732473in}{2.567710in}}{\pgfqpoint{7.736040in}{2.564144in}}%
\pgfpathcurveto{\pgfqpoint{7.739606in}{2.560578in}}{\pgfqpoint{7.744444in}{2.558574in}}{\pgfqpoint{7.749487in}{2.558574in}}%
\pgfpathclose%
\pgfusepath{fill}%
\end{pgfscope}%
\begin{pgfscope}%
\pgfpathrectangle{\pgfqpoint{6.572727in}{0.474100in}}{\pgfqpoint{4.227273in}{3.318700in}}%
\pgfusepath{clip}%
\pgfsetbuttcap%
\pgfsetroundjoin%
\definecolor{currentfill}{rgb}{0.127568,0.566949,0.550556}%
\pgfsetfillcolor{currentfill}%
\pgfsetfillopacity{0.700000}%
\pgfsetlinewidth{0.000000pt}%
\definecolor{currentstroke}{rgb}{0.000000,0.000000,0.000000}%
\pgfsetstrokecolor{currentstroke}%
\pgfsetstrokeopacity{0.700000}%
\pgfsetdash{}{0pt}%
\pgfpathmoveto{\pgfqpoint{9.647676in}{1.401187in}}%
\pgfpathcurveto{\pgfqpoint{9.652719in}{1.401187in}}{\pgfqpoint{9.657557in}{1.403191in}}{\pgfqpoint{9.661123in}{1.406757in}}%
\pgfpathcurveto{\pgfqpoint{9.664690in}{1.410323in}}{\pgfqpoint{9.666694in}{1.415161in}}{\pgfqpoint{9.666694in}{1.420205in}}%
\pgfpathcurveto{\pgfqpoint{9.666694in}{1.425249in}}{\pgfqpoint{9.664690in}{1.430086in}}{\pgfqpoint{9.661123in}{1.433653in}}%
\pgfpathcurveto{\pgfqpoint{9.657557in}{1.437219in}}{\pgfqpoint{9.652719in}{1.439223in}}{\pgfqpoint{9.647676in}{1.439223in}}%
\pgfpathcurveto{\pgfqpoint{9.642632in}{1.439223in}}{\pgfqpoint{9.637794in}{1.437219in}}{\pgfqpoint{9.634228in}{1.433653in}}%
\pgfpathcurveto{\pgfqpoint{9.630661in}{1.430086in}}{\pgfqpoint{9.628657in}{1.425249in}}{\pgfqpoint{9.628657in}{1.420205in}}%
\pgfpathcurveto{\pgfqpoint{9.628657in}{1.415161in}}{\pgfqpoint{9.630661in}{1.410323in}}{\pgfqpoint{9.634228in}{1.406757in}}%
\pgfpathcurveto{\pgfqpoint{9.637794in}{1.403191in}}{\pgfqpoint{9.642632in}{1.401187in}}{\pgfqpoint{9.647676in}{1.401187in}}%
\pgfpathclose%
\pgfusepath{fill}%
\end{pgfscope}%
\begin{pgfscope}%
\pgfpathrectangle{\pgfqpoint{6.572727in}{0.474100in}}{\pgfqpoint{4.227273in}{3.318700in}}%
\pgfusepath{clip}%
\pgfsetbuttcap%
\pgfsetroundjoin%
\definecolor{currentfill}{rgb}{0.993248,0.906157,0.143936}%
\pgfsetfillcolor{currentfill}%
\pgfsetfillopacity{0.700000}%
\pgfsetlinewidth{0.000000pt}%
\definecolor{currentstroke}{rgb}{0.000000,0.000000,0.000000}%
\pgfsetstrokecolor{currentstroke}%
\pgfsetstrokeopacity{0.700000}%
\pgfsetdash{}{0pt}%
\pgfpathmoveto{\pgfqpoint{8.130323in}{2.832789in}}%
\pgfpathcurveto{\pgfqpoint{8.135367in}{2.832789in}}{\pgfqpoint{8.140205in}{2.834793in}}{\pgfqpoint{8.143771in}{2.838359in}}%
\pgfpathcurveto{\pgfqpoint{8.147338in}{2.841926in}}{\pgfqpoint{8.149342in}{2.846763in}}{\pgfqpoint{8.149342in}{2.851807in}}%
\pgfpathcurveto{\pgfqpoint{8.149342in}{2.856851in}}{\pgfqpoint{8.147338in}{2.861689in}}{\pgfqpoint{8.143771in}{2.865255in}}%
\pgfpathcurveto{\pgfqpoint{8.140205in}{2.868821in}}{\pgfqpoint{8.135367in}{2.870825in}}{\pgfqpoint{8.130323in}{2.870825in}}%
\pgfpathcurveto{\pgfqpoint{8.125280in}{2.870825in}}{\pgfqpoint{8.120442in}{2.868821in}}{\pgfqpoint{8.116876in}{2.865255in}}%
\pgfpathcurveto{\pgfqpoint{8.113309in}{2.861689in}}{\pgfqpoint{8.111305in}{2.856851in}}{\pgfqpoint{8.111305in}{2.851807in}}%
\pgfpathcurveto{\pgfqpoint{8.111305in}{2.846763in}}{\pgfqpoint{8.113309in}{2.841926in}}{\pgfqpoint{8.116876in}{2.838359in}}%
\pgfpathcurveto{\pgfqpoint{8.120442in}{2.834793in}}{\pgfqpoint{8.125280in}{2.832789in}}{\pgfqpoint{8.130323in}{2.832789in}}%
\pgfpathclose%
\pgfusepath{fill}%
\end{pgfscope}%
\begin{pgfscope}%
\pgfpathrectangle{\pgfqpoint{6.572727in}{0.474100in}}{\pgfqpoint{4.227273in}{3.318700in}}%
\pgfusepath{clip}%
\pgfsetbuttcap%
\pgfsetroundjoin%
\definecolor{currentfill}{rgb}{0.993248,0.906157,0.143936}%
\pgfsetfillcolor{currentfill}%
\pgfsetfillopacity{0.700000}%
\pgfsetlinewidth{0.000000pt}%
\definecolor{currentstroke}{rgb}{0.000000,0.000000,0.000000}%
\pgfsetstrokecolor{currentstroke}%
\pgfsetstrokeopacity{0.700000}%
\pgfsetdash{}{0pt}%
\pgfpathmoveto{\pgfqpoint{8.963549in}{2.944766in}}%
\pgfpathcurveto{\pgfqpoint{8.968592in}{2.944766in}}{\pgfqpoint{8.973430in}{2.946770in}}{\pgfqpoint{8.976996in}{2.950337in}}%
\pgfpathcurveto{\pgfqpoint{8.980563in}{2.953903in}}{\pgfqpoint{8.982567in}{2.958741in}}{\pgfqpoint{8.982567in}{2.963784in}}%
\pgfpathcurveto{\pgfqpoint{8.982567in}{2.968828in}}{\pgfqpoint{8.980563in}{2.973666in}}{\pgfqpoint{8.976996in}{2.977232in}}%
\pgfpathcurveto{\pgfqpoint{8.973430in}{2.980799in}}{\pgfqpoint{8.968592in}{2.982803in}}{\pgfqpoint{8.963549in}{2.982803in}}%
\pgfpathcurveto{\pgfqpoint{8.958505in}{2.982803in}}{\pgfqpoint{8.953667in}{2.980799in}}{\pgfqpoint{8.950101in}{2.977232in}}%
\pgfpathcurveto{\pgfqpoint{8.946534in}{2.973666in}}{\pgfqpoint{8.944530in}{2.968828in}}{\pgfqpoint{8.944530in}{2.963784in}}%
\pgfpathcurveto{\pgfqpoint{8.944530in}{2.958741in}}{\pgfqpoint{8.946534in}{2.953903in}}{\pgfqpoint{8.950101in}{2.950337in}}%
\pgfpathcurveto{\pgfqpoint{8.953667in}{2.946770in}}{\pgfqpoint{8.958505in}{2.944766in}}{\pgfqpoint{8.963549in}{2.944766in}}%
\pgfpathclose%
\pgfusepath{fill}%
\end{pgfscope}%
\begin{pgfscope}%
\pgfpathrectangle{\pgfqpoint{6.572727in}{0.474100in}}{\pgfqpoint{4.227273in}{3.318700in}}%
\pgfusepath{clip}%
\pgfsetbuttcap%
\pgfsetroundjoin%
\definecolor{currentfill}{rgb}{0.993248,0.906157,0.143936}%
\pgfsetfillcolor{currentfill}%
\pgfsetfillopacity{0.700000}%
\pgfsetlinewidth{0.000000pt}%
\definecolor{currentstroke}{rgb}{0.000000,0.000000,0.000000}%
\pgfsetstrokecolor{currentstroke}%
\pgfsetstrokeopacity{0.700000}%
\pgfsetdash{}{0pt}%
\pgfpathmoveto{\pgfqpoint{8.296973in}{2.706873in}}%
\pgfpathcurveto{\pgfqpoint{8.302017in}{2.706873in}}{\pgfqpoint{8.306855in}{2.708877in}}{\pgfqpoint{8.310421in}{2.712444in}}%
\pgfpathcurveto{\pgfqpoint{8.313987in}{2.716010in}}{\pgfqpoint{8.315991in}{2.720848in}}{\pgfqpoint{8.315991in}{2.725891in}}%
\pgfpathcurveto{\pgfqpoint{8.315991in}{2.730935in}}{\pgfqpoint{8.313987in}{2.735773in}}{\pgfqpoint{8.310421in}{2.739339in}}%
\pgfpathcurveto{\pgfqpoint{8.306855in}{2.742906in}}{\pgfqpoint{8.302017in}{2.744910in}}{\pgfqpoint{8.296973in}{2.744910in}}%
\pgfpathcurveto{\pgfqpoint{8.291929in}{2.744910in}}{\pgfqpoint{8.287092in}{2.742906in}}{\pgfqpoint{8.283525in}{2.739339in}}%
\pgfpathcurveto{\pgfqpoint{8.279959in}{2.735773in}}{\pgfqpoint{8.277955in}{2.730935in}}{\pgfqpoint{8.277955in}{2.725891in}}%
\pgfpathcurveto{\pgfqpoint{8.277955in}{2.720848in}}{\pgfqpoint{8.279959in}{2.716010in}}{\pgfqpoint{8.283525in}{2.712444in}}%
\pgfpathcurveto{\pgfqpoint{8.287092in}{2.708877in}}{\pgfqpoint{8.291929in}{2.706873in}}{\pgfqpoint{8.296973in}{2.706873in}}%
\pgfpathclose%
\pgfusepath{fill}%
\end{pgfscope}%
\begin{pgfscope}%
\pgfpathrectangle{\pgfqpoint{6.572727in}{0.474100in}}{\pgfqpoint{4.227273in}{3.318700in}}%
\pgfusepath{clip}%
\pgfsetbuttcap%
\pgfsetroundjoin%
\definecolor{currentfill}{rgb}{0.127568,0.566949,0.550556}%
\pgfsetfillcolor{currentfill}%
\pgfsetfillopacity{0.700000}%
\pgfsetlinewidth{0.000000pt}%
\definecolor{currentstroke}{rgb}{0.000000,0.000000,0.000000}%
\pgfsetstrokecolor{currentstroke}%
\pgfsetstrokeopacity{0.700000}%
\pgfsetdash{}{0pt}%
\pgfpathmoveto{\pgfqpoint{10.308772in}{1.810458in}}%
\pgfpathcurveto{\pgfqpoint{10.313816in}{1.810458in}}{\pgfqpoint{10.318653in}{1.812462in}}{\pgfqpoint{10.322220in}{1.816028in}}%
\pgfpathcurveto{\pgfqpoint{10.325786in}{1.819595in}}{\pgfqpoint{10.327790in}{1.824433in}}{\pgfqpoint{10.327790in}{1.829476in}}%
\pgfpathcurveto{\pgfqpoint{10.327790in}{1.834520in}}{\pgfqpoint{10.325786in}{1.839358in}}{\pgfqpoint{10.322220in}{1.842924in}}%
\pgfpathcurveto{\pgfqpoint{10.318653in}{1.846491in}}{\pgfqpoint{10.313816in}{1.848494in}}{\pgfqpoint{10.308772in}{1.848494in}}%
\pgfpathcurveto{\pgfqpoint{10.303728in}{1.848494in}}{\pgfqpoint{10.298891in}{1.846491in}}{\pgfqpoint{10.295324in}{1.842924in}}%
\pgfpathcurveto{\pgfqpoint{10.291758in}{1.839358in}}{\pgfqpoint{10.289754in}{1.834520in}}{\pgfqpoint{10.289754in}{1.829476in}}%
\pgfpathcurveto{\pgfqpoint{10.289754in}{1.824433in}}{\pgfqpoint{10.291758in}{1.819595in}}{\pgfqpoint{10.295324in}{1.816028in}}%
\pgfpathcurveto{\pgfqpoint{10.298891in}{1.812462in}}{\pgfqpoint{10.303728in}{1.810458in}}{\pgfqpoint{10.308772in}{1.810458in}}%
\pgfpathclose%
\pgfusepath{fill}%
\end{pgfscope}%
\begin{pgfscope}%
\pgfpathrectangle{\pgfqpoint{6.572727in}{0.474100in}}{\pgfqpoint{4.227273in}{3.318700in}}%
\pgfusepath{clip}%
\pgfsetbuttcap%
\pgfsetroundjoin%
\definecolor{currentfill}{rgb}{0.127568,0.566949,0.550556}%
\pgfsetfillcolor{currentfill}%
\pgfsetfillopacity{0.700000}%
\pgfsetlinewidth{0.000000pt}%
\definecolor{currentstroke}{rgb}{0.000000,0.000000,0.000000}%
\pgfsetstrokecolor{currentstroke}%
\pgfsetstrokeopacity{0.700000}%
\pgfsetdash{}{0pt}%
\pgfpathmoveto{\pgfqpoint{9.307642in}{1.601876in}}%
\pgfpathcurveto{\pgfqpoint{9.312686in}{1.601876in}}{\pgfqpoint{9.317523in}{1.603880in}}{\pgfqpoint{9.321090in}{1.607446in}}%
\pgfpathcurveto{\pgfqpoint{9.324656in}{1.611013in}}{\pgfqpoint{9.326660in}{1.615850in}}{\pgfqpoint{9.326660in}{1.620894in}}%
\pgfpathcurveto{\pgfqpoint{9.326660in}{1.625938in}}{\pgfqpoint{9.324656in}{1.630776in}}{\pgfqpoint{9.321090in}{1.634342in}}%
\pgfpathcurveto{\pgfqpoint{9.317523in}{1.637908in}}{\pgfqpoint{9.312686in}{1.639912in}}{\pgfqpoint{9.307642in}{1.639912in}}%
\pgfpathcurveto{\pgfqpoint{9.302598in}{1.639912in}}{\pgfqpoint{9.297761in}{1.637908in}}{\pgfqpoint{9.294194in}{1.634342in}}%
\pgfpathcurveto{\pgfqpoint{9.290628in}{1.630776in}}{\pgfqpoint{9.288624in}{1.625938in}}{\pgfqpoint{9.288624in}{1.620894in}}%
\pgfpathcurveto{\pgfqpoint{9.288624in}{1.615850in}}{\pgfqpoint{9.290628in}{1.611013in}}{\pgfqpoint{9.294194in}{1.607446in}}%
\pgfpathcurveto{\pgfqpoint{9.297761in}{1.603880in}}{\pgfqpoint{9.302598in}{1.601876in}}{\pgfqpoint{9.307642in}{1.601876in}}%
\pgfpathclose%
\pgfusepath{fill}%
\end{pgfscope}%
\begin{pgfscope}%
\pgfpathrectangle{\pgfqpoint{6.572727in}{0.474100in}}{\pgfqpoint{4.227273in}{3.318700in}}%
\pgfusepath{clip}%
\pgfsetbuttcap%
\pgfsetroundjoin%
\definecolor{currentfill}{rgb}{0.267004,0.004874,0.329415}%
\pgfsetfillcolor{currentfill}%
\pgfsetfillopacity{0.700000}%
\pgfsetlinewidth{0.000000pt}%
\definecolor{currentstroke}{rgb}{0.000000,0.000000,0.000000}%
\pgfsetstrokecolor{currentstroke}%
\pgfsetstrokeopacity{0.700000}%
\pgfsetdash{}{0pt}%
\pgfpathmoveto{\pgfqpoint{7.812880in}{1.195266in}}%
\pgfpathcurveto{\pgfqpoint{7.817924in}{1.195266in}}{\pgfqpoint{7.822762in}{1.197270in}}{\pgfqpoint{7.826328in}{1.200836in}}%
\pgfpathcurveto{\pgfqpoint{7.829895in}{1.204403in}}{\pgfqpoint{7.831899in}{1.209240in}}{\pgfqpoint{7.831899in}{1.214284in}}%
\pgfpathcurveto{\pgfqpoint{7.831899in}{1.219328in}}{\pgfqpoint{7.829895in}{1.224165in}}{\pgfqpoint{7.826328in}{1.227732in}}%
\pgfpathcurveto{\pgfqpoint{7.822762in}{1.231298in}}{\pgfqpoint{7.817924in}{1.233302in}}{\pgfqpoint{7.812880in}{1.233302in}}%
\pgfpathcurveto{\pgfqpoint{7.807837in}{1.233302in}}{\pgfqpoint{7.802999in}{1.231298in}}{\pgfqpoint{7.799433in}{1.227732in}}%
\pgfpathcurveto{\pgfqpoint{7.795866in}{1.224165in}}{\pgfqpoint{7.793862in}{1.219328in}}{\pgfqpoint{7.793862in}{1.214284in}}%
\pgfpathcurveto{\pgfqpoint{7.793862in}{1.209240in}}{\pgfqpoint{7.795866in}{1.204403in}}{\pgfqpoint{7.799433in}{1.200836in}}%
\pgfpathcurveto{\pgfqpoint{7.802999in}{1.197270in}}{\pgfqpoint{7.807837in}{1.195266in}}{\pgfqpoint{7.812880in}{1.195266in}}%
\pgfpathclose%
\pgfusepath{fill}%
\end{pgfscope}%
\begin{pgfscope}%
\pgfpathrectangle{\pgfqpoint{6.572727in}{0.474100in}}{\pgfqpoint{4.227273in}{3.318700in}}%
\pgfusepath{clip}%
\pgfsetbuttcap%
\pgfsetroundjoin%
\definecolor{currentfill}{rgb}{0.267004,0.004874,0.329415}%
\pgfsetfillcolor{currentfill}%
\pgfsetfillopacity{0.700000}%
\pgfsetlinewidth{0.000000pt}%
\definecolor{currentstroke}{rgb}{0.000000,0.000000,0.000000}%
\pgfsetstrokecolor{currentstroke}%
\pgfsetstrokeopacity{0.700000}%
\pgfsetdash{}{0pt}%
\pgfpathmoveto{\pgfqpoint{8.647794in}{1.127508in}}%
\pgfpathcurveto{\pgfqpoint{8.652838in}{1.127508in}}{\pgfqpoint{8.657676in}{1.129512in}}{\pgfqpoint{8.661242in}{1.133078in}}%
\pgfpathcurveto{\pgfqpoint{8.664809in}{1.136644in}}{\pgfqpoint{8.666813in}{1.141482in}}{\pgfqpoint{8.666813in}{1.146526in}}%
\pgfpathcurveto{\pgfqpoint{8.666813in}{1.151569in}}{\pgfqpoint{8.664809in}{1.156407in}}{\pgfqpoint{8.661242in}{1.159974in}}%
\pgfpathcurveto{\pgfqpoint{8.657676in}{1.163540in}}{\pgfqpoint{8.652838in}{1.165544in}}{\pgfqpoint{8.647794in}{1.165544in}}%
\pgfpathcurveto{\pgfqpoint{8.642751in}{1.165544in}}{\pgfqpoint{8.637913in}{1.163540in}}{\pgfqpoint{8.634347in}{1.159974in}}%
\pgfpathcurveto{\pgfqpoint{8.630780in}{1.156407in}}{\pgfqpoint{8.628776in}{1.151569in}}{\pgfqpoint{8.628776in}{1.146526in}}%
\pgfpathcurveto{\pgfqpoint{8.628776in}{1.141482in}}{\pgfqpoint{8.630780in}{1.136644in}}{\pgfqpoint{8.634347in}{1.133078in}}%
\pgfpathcurveto{\pgfqpoint{8.637913in}{1.129512in}}{\pgfqpoint{8.642751in}{1.127508in}}{\pgfqpoint{8.647794in}{1.127508in}}%
\pgfpathclose%
\pgfusepath{fill}%
\end{pgfscope}%
\begin{pgfscope}%
\pgfpathrectangle{\pgfqpoint{6.572727in}{0.474100in}}{\pgfqpoint{4.227273in}{3.318700in}}%
\pgfusepath{clip}%
\pgfsetbuttcap%
\pgfsetroundjoin%
\definecolor{currentfill}{rgb}{0.267004,0.004874,0.329415}%
\pgfsetfillcolor{currentfill}%
\pgfsetfillopacity{0.700000}%
\pgfsetlinewidth{0.000000pt}%
\definecolor{currentstroke}{rgb}{0.000000,0.000000,0.000000}%
\pgfsetstrokecolor{currentstroke}%
\pgfsetstrokeopacity{0.700000}%
\pgfsetdash{}{0pt}%
\pgfpathmoveto{\pgfqpoint{8.008567in}{1.204967in}}%
\pgfpathcurveto{\pgfqpoint{8.013611in}{1.204967in}}{\pgfqpoint{8.018449in}{1.206971in}}{\pgfqpoint{8.022015in}{1.210537in}}%
\pgfpathcurveto{\pgfqpoint{8.025582in}{1.214103in}}{\pgfqpoint{8.027586in}{1.218941in}}{\pgfqpoint{8.027586in}{1.223985in}}%
\pgfpathcurveto{\pgfqpoint{8.027586in}{1.229028in}}{\pgfqpoint{8.025582in}{1.233866in}}{\pgfqpoint{8.022015in}{1.237433in}}%
\pgfpathcurveto{\pgfqpoint{8.018449in}{1.240999in}}{\pgfqpoint{8.013611in}{1.243003in}}{\pgfqpoint{8.008567in}{1.243003in}}%
\pgfpathcurveto{\pgfqpoint{8.003524in}{1.243003in}}{\pgfqpoint{7.998686in}{1.240999in}}{\pgfqpoint{7.995120in}{1.237433in}}%
\pgfpathcurveto{\pgfqpoint{7.991553in}{1.233866in}}{\pgfqpoint{7.989549in}{1.229028in}}{\pgfqpoint{7.989549in}{1.223985in}}%
\pgfpathcurveto{\pgfqpoint{7.989549in}{1.218941in}}{\pgfqpoint{7.991553in}{1.214103in}}{\pgfqpoint{7.995120in}{1.210537in}}%
\pgfpathcurveto{\pgfqpoint{7.998686in}{1.206971in}}{\pgfqpoint{8.003524in}{1.204967in}}{\pgfqpoint{8.008567in}{1.204967in}}%
\pgfpathclose%
\pgfusepath{fill}%
\end{pgfscope}%
\begin{pgfscope}%
\pgfpathrectangle{\pgfqpoint{6.572727in}{0.474100in}}{\pgfqpoint{4.227273in}{3.318700in}}%
\pgfusepath{clip}%
\pgfsetbuttcap%
\pgfsetroundjoin%
\definecolor{currentfill}{rgb}{0.267004,0.004874,0.329415}%
\pgfsetfillcolor{currentfill}%
\pgfsetfillopacity{0.700000}%
\pgfsetlinewidth{0.000000pt}%
\definecolor{currentstroke}{rgb}{0.000000,0.000000,0.000000}%
\pgfsetstrokecolor{currentstroke}%
\pgfsetstrokeopacity{0.700000}%
\pgfsetdash{}{0pt}%
\pgfpathmoveto{\pgfqpoint{8.385035in}{1.683129in}}%
\pgfpathcurveto{\pgfqpoint{8.390079in}{1.683129in}}{\pgfqpoint{8.394917in}{1.685133in}}{\pgfqpoint{8.398483in}{1.688699in}}%
\pgfpathcurveto{\pgfqpoint{8.402049in}{1.692266in}}{\pgfqpoint{8.404053in}{1.697103in}}{\pgfqpoint{8.404053in}{1.702147in}}%
\pgfpathcurveto{\pgfqpoint{8.404053in}{1.707191in}}{\pgfqpoint{8.402049in}{1.712028in}}{\pgfqpoint{8.398483in}{1.715595in}}%
\pgfpathcurveto{\pgfqpoint{8.394917in}{1.719161in}}{\pgfqpoint{8.390079in}{1.721165in}}{\pgfqpoint{8.385035in}{1.721165in}}%
\pgfpathcurveto{\pgfqpoint{8.379991in}{1.721165in}}{\pgfqpoint{8.375154in}{1.719161in}}{\pgfqpoint{8.371587in}{1.715595in}}%
\pgfpathcurveto{\pgfqpoint{8.368021in}{1.712028in}}{\pgfqpoint{8.366017in}{1.707191in}}{\pgfqpoint{8.366017in}{1.702147in}}%
\pgfpathcurveto{\pgfqpoint{8.366017in}{1.697103in}}{\pgfqpoint{8.368021in}{1.692266in}}{\pgfqpoint{8.371587in}{1.688699in}}%
\pgfpathcurveto{\pgfqpoint{8.375154in}{1.685133in}}{\pgfqpoint{8.379991in}{1.683129in}}{\pgfqpoint{8.385035in}{1.683129in}}%
\pgfpathclose%
\pgfusepath{fill}%
\end{pgfscope}%
\begin{pgfscope}%
\pgfpathrectangle{\pgfqpoint{6.572727in}{0.474100in}}{\pgfqpoint{4.227273in}{3.318700in}}%
\pgfusepath{clip}%
\pgfsetbuttcap%
\pgfsetroundjoin%
\definecolor{currentfill}{rgb}{0.127568,0.566949,0.550556}%
\pgfsetfillcolor{currentfill}%
\pgfsetfillopacity{0.700000}%
\pgfsetlinewidth{0.000000pt}%
\definecolor{currentstroke}{rgb}{0.000000,0.000000,0.000000}%
\pgfsetstrokecolor{currentstroke}%
\pgfsetstrokeopacity{0.700000}%
\pgfsetdash{}{0pt}%
\pgfpathmoveto{\pgfqpoint{9.265493in}{2.007887in}}%
\pgfpathcurveto{\pgfqpoint{9.270537in}{2.007887in}}{\pgfqpoint{9.275375in}{2.009891in}}{\pgfqpoint{9.278941in}{2.013457in}}%
\pgfpathcurveto{\pgfqpoint{9.282507in}{2.017023in}}{\pgfqpoint{9.284511in}{2.021861in}}{\pgfqpoint{9.284511in}{2.026905in}}%
\pgfpathcurveto{\pgfqpoint{9.284511in}{2.031949in}}{\pgfqpoint{9.282507in}{2.036786in}}{\pgfqpoint{9.278941in}{2.040353in}}%
\pgfpathcurveto{\pgfqpoint{9.275375in}{2.043919in}}{\pgfqpoint{9.270537in}{2.045923in}}{\pgfqpoint{9.265493in}{2.045923in}}%
\pgfpathcurveto{\pgfqpoint{9.260449in}{2.045923in}}{\pgfqpoint{9.255612in}{2.043919in}}{\pgfqpoint{9.252045in}{2.040353in}}%
\pgfpathcurveto{\pgfqpoint{9.248479in}{2.036786in}}{\pgfqpoint{9.246475in}{2.031949in}}{\pgfqpoint{9.246475in}{2.026905in}}%
\pgfpathcurveto{\pgfqpoint{9.246475in}{2.021861in}}{\pgfqpoint{9.248479in}{2.017023in}}{\pgfqpoint{9.252045in}{2.013457in}}%
\pgfpathcurveto{\pgfqpoint{9.255612in}{2.009891in}}{\pgfqpoint{9.260449in}{2.007887in}}{\pgfqpoint{9.265493in}{2.007887in}}%
\pgfpathclose%
\pgfusepath{fill}%
\end{pgfscope}%
\begin{pgfscope}%
\pgfpathrectangle{\pgfqpoint{6.572727in}{0.474100in}}{\pgfqpoint{4.227273in}{3.318700in}}%
\pgfusepath{clip}%
\pgfsetbuttcap%
\pgfsetroundjoin%
\definecolor{currentfill}{rgb}{1.000000,1.000000,1.000000}%
\pgfsetfillcolor{currentfill}%
\pgfsetlinewidth{1.003750pt}%
\definecolor{currentstroke}{rgb}{0.000000,0.000000,0.000000}%
\pgfsetstrokecolor{currentstroke}%
\pgfsetdash{}{0pt}%
\pgfsys@defobject{currentmarker}{\pgfqpoint{-0.098209in}{-0.098209in}}{\pgfqpoint{0.098209in}{0.098209in}}{%
\pgfpathmoveto{\pgfqpoint{0.000000in}{-0.098209in}}%
\pgfpathcurveto{\pgfqpoint{0.026045in}{-0.098209in}}{\pgfqpoint{0.051028in}{-0.087861in}}{\pgfqpoint{0.069444in}{-0.069444in}}%
\pgfpathcurveto{\pgfqpoint{0.087861in}{-0.051028in}}{\pgfqpoint{0.098209in}{-0.026045in}}{\pgfqpoint{0.098209in}{0.000000in}}%
\pgfpathcurveto{\pgfqpoint{0.098209in}{0.026045in}}{\pgfqpoint{0.087861in}{0.051028in}}{\pgfqpoint{0.069444in}{0.069444in}}%
\pgfpathcurveto{\pgfqpoint{0.051028in}{0.087861in}}{\pgfqpoint{0.026045in}{0.098209in}}{\pgfqpoint{0.000000in}{0.098209in}}%
\pgfpathcurveto{\pgfqpoint{-0.026045in}{0.098209in}}{\pgfqpoint{-0.051028in}{0.087861in}}{\pgfqpoint{-0.069444in}{0.069444in}}%
\pgfpathcurveto{\pgfqpoint{-0.087861in}{0.051028in}}{\pgfqpoint{-0.098209in}{0.026045in}}{\pgfqpoint{-0.098209in}{0.000000in}}%
\pgfpathcurveto{\pgfqpoint{-0.098209in}{-0.026045in}}{\pgfqpoint{-0.087861in}{-0.051028in}}{\pgfqpoint{-0.069444in}{-0.069444in}}%
\pgfpathcurveto{\pgfqpoint{-0.051028in}{-0.087861in}}{\pgfqpoint{-0.026045in}{-0.098209in}}{\pgfqpoint{0.000000in}{-0.098209in}}%
\pgfpathclose%
\pgfusepath{stroke,fill}%
}%
\begin{pgfscope}%
\pgfsys@transformshift{7.817120in}{1.558345in}%
\pgfsys@useobject{currentmarker}{}%
\end{pgfscope}%
\begin{pgfscope}%
\pgfsys@transformshift{9.607314in}{1.594393in}%
\pgfsys@useobject{currentmarker}{}%
\end{pgfscope}%
\begin{pgfscope}%
\pgfsys@transformshift{8.204077in}{2.836440in}%
\pgfsys@useobject{currentmarker}{}%
\end{pgfscope}%
\end{pgfscope}%
\begin{pgfscope}%
\pgfpathrectangle{\pgfqpoint{6.572727in}{0.474100in}}{\pgfqpoint{4.227273in}{3.318700in}}%
\pgfusepath{clip}%
\pgfsetbuttcap%
\pgfsetroundjoin%
\definecolor{currentfill}{rgb}{0.121569,0.466667,0.705882}%
\pgfsetfillcolor{currentfill}%
\pgfsetlinewidth{1.003750pt}%
\definecolor{currentstroke}{rgb}{0.000000,0.000000,0.000000}%
\pgfsetstrokecolor{currentstroke}%
\pgfsetdash{}{0pt}%
\pgfsys@defobject{currentmarker}{\pgfqpoint{-0.028432in}{-0.049105in}}{\pgfqpoint{0.036993in}{0.049105in}}{%
\pgfpathmoveto{\pgfqpoint{0.004270in}{0.038961in}}%
\pgfpathquadraticcurveto{\pgfqpoint{-0.005609in}{0.038961in}}{\pgfqpoint{-0.010600in}{0.029223in}}%
\pgfpathquadraticcurveto{\pgfqpoint{-0.015570in}{0.019506in}}{\pgfqpoint{-0.015570in}{-0.000030in}}%
\pgfpathquadraticcurveto{\pgfqpoint{-0.015570in}{-0.019486in}}{\pgfqpoint{-0.010600in}{-0.029223in}}%
\pgfpathquadraticcurveto{\pgfqpoint{-0.005609in}{-0.038961in}}{\pgfqpoint{0.004270in}{-0.038961in}}%
\pgfpathquadraticcurveto{\pgfqpoint{0.014231in}{-0.038961in}}{\pgfqpoint{0.019202in}{-0.029223in}}%
\pgfpathquadraticcurveto{\pgfqpoint{0.024192in}{-0.019486in}}{\pgfqpoint{0.024192in}{-0.000030in}}%
\pgfpathquadraticcurveto{\pgfqpoint{0.024192in}{0.019506in}}{\pgfqpoint{0.019202in}{0.029223in}}%
\pgfpathquadraticcurveto{\pgfqpoint{0.014231in}{0.038961in}}{\pgfqpoint{0.004270in}{0.038961in}}%
\pgfpathclose%
\pgfpathmoveto{\pgfqpoint{0.004270in}{0.049105in}}%
\pgfpathquadraticcurveto{\pgfqpoint{0.020196in}{0.049105in}}{\pgfqpoint{0.028594in}{0.036506in}}%
\pgfpathquadraticcurveto{\pgfqpoint{0.036993in}{0.023928in}}{\pgfqpoint{0.036993in}{-0.000030in}}%
\pgfpathquadraticcurveto{\pgfqpoint{0.036993in}{-0.023928in}}{\pgfqpoint{0.028594in}{-0.036527in}}%
\pgfpathquadraticcurveto{\pgfqpoint{0.020196in}{-0.049105in}}{\pgfqpoint{0.004270in}{-0.049105in}}%
\pgfpathquadraticcurveto{\pgfqpoint{-0.011635in}{-0.049105in}}{\pgfqpoint{-0.020033in}{-0.036527in}}%
\pgfpathquadraticcurveto{\pgfqpoint{-0.028432in}{-0.023928in}}{\pgfqpoint{-0.028432in}{-0.000030in}}%
\pgfpathquadraticcurveto{\pgfqpoint{-0.028432in}{0.023928in}}{\pgfqpoint{-0.020033in}{0.036506in}}%
\pgfpathquadraticcurveto{\pgfqpoint{-0.011635in}{0.049105in}}{\pgfqpoint{0.004270in}{0.049105in}}%
\pgfpathclose%
\pgfusepath{stroke,fill}%
}%
\begin{pgfscope}%
\pgfsys@transformshift{7.817120in}{1.558345in}%
\pgfsys@useobject{currentmarker}{}%
\end{pgfscope}%
\end{pgfscope}%
\begin{pgfscope}%
\pgfpathrectangle{\pgfqpoint{6.572727in}{0.474100in}}{\pgfqpoint{4.227273in}{3.318700in}}%
\pgfusepath{clip}%
\pgfsetbuttcap%
\pgfsetroundjoin%
\definecolor{currentfill}{rgb}{1.000000,0.498039,0.054902}%
\pgfsetfillcolor{currentfill}%
\pgfsetlinewidth{1.003750pt}%
\definecolor{currentstroke}{rgb}{0.000000,0.000000,0.000000}%
\pgfsetstrokecolor{currentstroke}%
\pgfsetdash{}{0pt}%
\pgfsys@defobject{currentmarker}{\pgfqpoint{-0.021837in}{-0.049105in}}{\pgfqpoint{0.036634in}{0.049105in}}{%
\pgfpathmoveto{\pgfqpoint{-0.019922in}{-0.037928in}}%
\pgfpathlineto{\pgfqpoint{0.001779in}{-0.037928in}}%
\pgfpathlineto{\pgfqpoint{0.001779in}{0.037002in}}%
\pgfpathlineto{\pgfqpoint{-0.021837in}{0.032266in}}%
\pgfpathlineto{\pgfqpoint{-0.021837in}{0.044369in}}%
\pgfpathlineto{\pgfqpoint{0.001652in}{0.049105in}}%
\pgfpathlineto{\pgfqpoint{0.014933in}{0.049105in}}%
\pgfpathlineto{\pgfqpoint{0.014933in}{-0.037928in}}%
\pgfpathlineto{\pgfqpoint{0.036634in}{-0.037928in}}%
\pgfpathlineto{\pgfqpoint{0.036634in}{-0.049105in}}%
\pgfpathlineto{\pgfqpoint{-0.019922in}{-0.049105in}}%
\pgfpathlineto{\pgfqpoint{-0.019922in}{-0.037928in}}%
\pgfpathclose%
\pgfusepath{stroke,fill}%
}%
\begin{pgfscope}%
\pgfsys@transformshift{9.607314in}{1.594393in}%
\pgfsys@useobject{currentmarker}{}%
\end{pgfscope}%
\end{pgfscope}%
\begin{pgfscope}%
\pgfpathrectangle{\pgfqpoint{6.572727in}{0.474100in}}{\pgfqpoint{4.227273in}{3.318700in}}%
\pgfusepath{clip}%
\pgfsetbuttcap%
\pgfsetroundjoin%
\definecolor{currentfill}{rgb}{0.172549,0.627451,0.172549}%
\pgfsetfillcolor{currentfill}%
\pgfsetlinewidth{1.003750pt}%
\definecolor{currentstroke}{rgb}{0.000000,0.000000,0.000000}%
\pgfsetstrokecolor{currentstroke}%
\pgfsetdash{}{0pt}%
\pgfsys@defobject{currentmarker}{\pgfqpoint{-0.025772in}{-0.049105in}}{\pgfqpoint{0.035469in}{0.049105in}}{%
\pgfpathmoveto{\pgfqpoint{-0.010079in}{-0.038126in}}%
\pgfpathlineto{\pgfqpoint{0.035469in}{-0.038126in}}%
\pgfpathlineto{\pgfqpoint{0.035469in}{-0.049105in}}%
\pgfpathlineto{\pgfqpoint{-0.025772in}{-0.049105in}}%
\pgfpathlineto{\pgfqpoint{-0.025772in}{-0.038126in}}%
\pgfpathquadraticcurveto{\pgfqpoint{-0.018350in}{-0.030435in}}{\pgfqpoint{-0.005531in}{-0.017492in}}%
\pgfpathquadraticcurveto{\pgfqpoint{0.007309in}{-0.004528in}}{\pgfqpoint{0.010596in}{-0.000765in}}%
\pgfpathquadraticcurveto{\pgfqpoint{0.016861in}{0.006265in}}{\pgfqpoint{0.019342in}{0.011144in}}%
\pgfpathquadraticcurveto{\pgfqpoint{0.021844in}{0.016024in}}{\pgfqpoint{0.021844in}{0.020738in}}%
\pgfpathquadraticcurveto{\pgfqpoint{0.021844in}{0.028429in}}{\pgfqpoint{0.016447in}{0.033267in}}%
\pgfpathquadraticcurveto{\pgfqpoint{0.011051in}{0.038126in}}{\pgfqpoint{0.002388in}{0.038126in}}%
\pgfpathquadraticcurveto{\pgfqpoint{-0.003753in}{0.038126in}}{\pgfqpoint{-0.010576in}{0.035996in}}%
\pgfpathquadraticcurveto{\pgfqpoint{-0.017378in}{0.033867in}}{\pgfqpoint{-0.025131in}{0.029525in}}%
\pgfpathlineto{\pgfqpoint{-0.025131in}{0.042716in}}%
\pgfpathquadraticcurveto{\pgfqpoint{-0.017254in}{0.045879in}}{\pgfqpoint{-0.010410in}{0.047492in}}%
\pgfpathquadraticcurveto{\pgfqpoint{-0.003546in}{0.049105in}}{\pgfqpoint{0.002140in}{0.049105in}}%
\pgfpathquadraticcurveto{\pgfqpoint{0.017130in}{0.049105in}}{\pgfqpoint{0.026041in}{0.041599in}}%
\pgfpathquadraticcurveto{\pgfqpoint{0.034952in}{0.034115in}}{\pgfqpoint{0.034952in}{0.021585in}}%
\pgfpathquadraticcurveto{\pgfqpoint{0.034952in}{0.015631in}}{\pgfqpoint{0.032719in}{0.010296in}}%
\pgfpathquadraticcurveto{\pgfqpoint{0.030507in}{0.004983in}}{\pgfqpoint{0.024614in}{-0.002254in}}%
\pgfpathquadraticcurveto{\pgfqpoint{0.023002in}{-0.004135in}}{\pgfqpoint{0.014339in}{-0.013088in}}%
\pgfpathquadraticcurveto{\pgfqpoint{0.005696in}{-0.022040in}}{\pgfqpoint{-0.010079in}{-0.038126in}}%
\pgfpathclose%
\pgfusepath{stroke,fill}%
}%
\begin{pgfscope}%
\pgfsys@transformshift{8.204077in}{2.836440in}%
\pgfsys@useobject{currentmarker}{}%
\end{pgfscope}%
\end{pgfscope}%
\begin{pgfscope}%
\pgfpathrectangle{\pgfqpoint{6.572727in}{0.474100in}}{\pgfqpoint{4.227273in}{3.318700in}}%
\pgfusepath{clip}%
\pgfsetbuttcap%
\pgfsetroundjoin%
\definecolor{currentfill}{rgb}{1.000000,0.000000,0.000000}%
\pgfsetfillcolor{currentfill}%
\pgfsetlinewidth{1.003750pt}%
\definecolor{currentstroke}{rgb}{1.000000,0.000000,0.000000}%
\pgfsetstrokecolor{currentstroke}%
\pgfsetdash{}{0pt}%
\pgfsys@defobject{currentmarker}{\pgfqpoint{-0.031056in}{-0.031056in}}{\pgfqpoint{0.031056in}{0.031056in}}{%
\pgfpathmoveto{\pgfqpoint{0.000000in}{-0.031056in}}%
\pgfpathcurveto{\pgfqpoint{0.008236in}{-0.031056in}}{\pgfqpoint{0.016136in}{-0.027784in}}{\pgfqpoint{0.021960in}{-0.021960in}}%
\pgfpathcurveto{\pgfqpoint{0.027784in}{-0.016136in}}{\pgfqpoint{0.031056in}{-0.008236in}}{\pgfqpoint{0.031056in}{0.000000in}}%
\pgfpathcurveto{\pgfqpoint{0.031056in}{0.008236in}}{\pgfqpoint{0.027784in}{0.016136in}}{\pgfqpoint{0.021960in}{0.021960in}}%
\pgfpathcurveto{\pgfqpoint{0.016136in}{0.027784in}}{\pgfqpoint{0.008236in}{0.031056in}}{\pgfqpoint{0.000000in}{0.031056in}}%
\pgfpathcurveto{\pgfqpoint{-0.008236in}{0.031056in}}{\pgfqpoint{-0.016136in}{0.027784in}}{\pgfqpoint{-0.021960in}{0.021960in}}%
\pgfpathcurveto{\pgfqpoint{-0.027784in}{0.016136in}}{\pgfqpoint{-0.031056in}{0.008236in}}{\pgfqpoint{-0.031056in}{0.000000in}}%
\pgfpathcurveto{\pgfqpoint{-0.031056in}{-0.008236in}}{\pgfqpoint{-0.027784in}{-0.016136in}}{\pgfqpoint{-0.021960in}{-0.021960in}}%
\pgfpathcurveto{\pgfqpoint{-0.016136in}{-0.027784in}}{\pgfqpoint{-0.008236in}{-0.031056in}}{\pgfqpoint{0.000000in}{-0.031056in}}%
\pgfpathclose%
\pgfusepath{stroke,fill}%
}%
\begin{pgfscope}%
\pgfsys@transformshift{8.704147in}{2.413806in}%
\pgfsys@useobject{currentmarker}{}%
\end{pgfscope}%
\end{pgfscope}%
\begin{pgfscope}%
\pgfpathrectangle{\pgfqpoint{6.572727in}{0.474100in}}{\pgfqpoint{4.227273in}{3.318700in}}%
\pgfusepath{clip}%
\pgfsetbuttcap%
\pgfsetroundjoin%
\definecolor{currentfill}{rgb}{1.000000,0.000000,0.000000}%
\pgfsetfillcolor{currentfill}%
\pgfsetlinewidth{1.003750pt}%
\definecolor{currentstroke}{rgb}{1.000000,0.000000,0.000000}%
\pgfsetstrokecolor{currentstroke}%
\pgfsetdash{}{0pt}%
\pgfsys@defobject{currentmarker}{\pgfqpoint{-0.031056in}{-0.031056in}}{\pgfqpoint{0.031056in}{0.031056in}}{%
\pgfpathmoveto{\pgfqpoint{0.000000in}{-0.031056in}}%
\pgfpathcurveto{\pgfqpoint{0.008236in}{-0.031056in}}{\pgfqpoint{0.016136in}{-0.027784in}}{\pgfqpoint{0.021960in}{-0.021960in}}%
\pgfpathcurveto{\pgfqpoint{0.027784in}{-0.016136in}}{\pgfqpoint{0.031056in}{-0.008236in}}{\pgfqpoint{0.031056in}{0.000000in}}%
\pgfpathcurveto{\pgfqpoint{0.031056in}{0.008236in}}{\pgfqpoint{0.027784in}{0.016136in}}{\pgfqpoint{0.021960in}{0.021960in}}%
\pgfpathcurveto{\pgfqpoint{0.016136in}{0.027784in}}{\pgfqpoint{0.008236in}{0.031056in}}{\pgfqpoint{0.000000in}{0.031056in}}%
\pgfpathcurveto{\pgfqpoint{-0.008236in}{0.031056in}}{\pgfqpoint{-0.016136in}{0.027784in}}{\pgfqpoint{-0.021960in}{0.021960in}}%
\pgfpathcurveto{\pgfqpoint{-0.027784in}{0.016136in}}{\pgfqpoint{-0.031056in}{0.008236in}}{\pgfqpoint{-0.031056in}{0.000000in}}%
\pgfpathcurveto{\pgfqpoint{-0.031056in}{-0.008236in}}{\pgfqpoint{-0.027784in}{-0.016136in}}{\pgfqpoint{-0.021960in}{-0.021960in}}%
\pgfpathcurveto{\pgfqpoint{-0.016136in}{-0.027784in}}{\pgfqpoint{-0.008236in}{-0.031056in}}{\pgfqpoint{0.000000in}{-0.031056in}}%
\pgfpathclose%
\pgfusepath{stroke,fill}%
}%
\begin{pgfscope}%
\pgfsys@transformshift{8.704147in}{1.575472in}%
\pgfsys@useobject{currentmarker}{}%
\end{pgfscope}%
\end{pgfscope}%
\begin{pgfscope}%
\pgfsetbuttcap%
\pgfsetroundjoin%
\definecolor{currentfill}{rgb}{0.000000,0.000000,0.000000}%
\pgfsetfillcolor{currentfill}%
\pgfsetlinewidth{0.803000pt}%
\definecolor{currentstroke}{rgb}{0.000000,0.000000,0.000000}%
\pgfsetstrokecolor{currentstroke}%
\pgfsetdash{}{0pt}%
\pgfsys@defobject{currentmarker}{\pgfqpoint{0.000000in}{-0.048611in}}{\pgfqpoint{0.000000in}{0.000000in}}{%
\pgfpathmoveto{\pgfqpoint{0.000000in}{0.000000in}}%
\pgfpathlineto{\pgfqpoint{0.000000in}{-0.048611in}}%
\pgfusepath{stroke,fill}%
}%
\begin{pgfscope}%
\pgfsys@transformshift{10.541497in}{0.474100in}%
\pgfsys@useobject{currentmarker}{}%
\end{pgfscope}%
\end{pgfscope}%
\begin{pgfscope}%
\definecolor{textcolor}{rgb}{0.000000,0.000000,0.000000}%
\pgfsetstrokecolor{textcolor}%
\pgfsetfillcolor{textcolor}%
\pgftext[x=10.541497in,y=0.376878in,,top]{\color{textcolor}\sffamily\fontsize{10.000000}{12.000000}\selectfont 2}%
\end{pgfscope}%
\begin{pgfscope}%
\pgfsetbuttcap%
\pgfsetroundjoin%
\definecolor{currentfill}{rgb}{0.000000,0.000000,0.000000}%
\pgfsetfillcolor{currentfill}%
\pgfsetlinewidth{0.803000pt}%
\definecolor{currentstroke}{rgb}{0.000000,0.000000,0.000000}%
\pgfsetstrokecolor{currentstroke}%
\pgfsetdash{}{0pt}%
\pgfsys@defobject{currentmarker}{\pgfqpoint{-0.048611in}{0.000000in}}{\pgfqpoint{-0.000000in}{0.000000in}}{%
\pgfpathmoveto{\pgfqpoint{-0.000000in}{0.000000in}}%
\pgfpathlineto{\pgfqpoint{-0.048611in}{0.000000in}}%
\pgfusepath{stroke,fill}%
}%
\begin{pgfscope}%
\pgfsys@transformshift{6.572727in}{0.737137in}%
\pgfsys@useobject{currentmarker}{}%
\end{pgfscope}%
\end{pgfscope}%
\begin{pgfscope}%
\definecolor{textcolor}{rgb}{0.000000,0.000000,0.000000}%
\pgfsetstrokecolor{textcolor}%
\pgfsetfillcolor{textcolor}%
\pgftext[x=6.146601in, y=0.684376in, left, base]{\color{textcolor}\sffamily\fontsize{10.000000}{12.000000}\selectfont \ensuremath{-}2.0}%
\end{pgfscope}%
\begin{pgfscope}%
\pgfsetbuttcap%
\pgfsetroundjoin%
\definecolor{currentfill}{rgb}{0.000000,0.000000,0.000000}%
\pgfsetfillcolor{currentfill}%
\pgfsetlinewidth{0.803000pt}%
\definecolor{currentstroke}{rgb}{0.000000,0.000000,0.000000}%
\pgfsetstrokecolor{currentstroke}%
\pgfsetdash{}{0pt}%
\pgfsys@defobject{currentmarker}{\pgfqpoint{-0.048611in}{0.000000in}}{\pgfqpoint{-0.000000in}{0.000000in}}{%
\pgfpathmoveto{\pgfqpoint{-0.000000in}{0.000000in}}%
\pgfpathlineto{\pgfqpoint{-0.048611in}{0.000000in}}%
\pgfusepath{stroke,fill}%
}%
\begin{pgfscope}%
\pgfsys@transformshift{6.572727in}{1.156305in}%
\pgfsys@useobject{currentmarker}{}%
\end{pgfscope}%
\end{pgfscope}%
\begin{pgfscope}%
\definecolor{textcolor}{rgb}{0.000000,0.000000,0.000000}%
\pgfsetstrokecolor{textcolor}%
\pgfsetfillcolor{textcolor}%
\pgftext[x=6.146601in, y=1.103543in, left, base]{\color{textcolor}\sffamily\fontsize{10.000000}{12.000000}\selectfont \ensuremath{-}1.5}%
\end{pgfscope}%
\begin{pgfscope}%
\pgfsetbuttcap%
\pgfsetroundjoin%
\definecolor{currentfill}{rgb}{0.000000,0.000000,0.000000}%
\pgfsetfillcolor{currentfill}%
\pgfsetlinewidth{0.803000pt}%
\definecolor{currentstroke}{rgb}{0.000000,0.000000,0.000000}%
\pgfsetstrokecolor{currentstroke}%
\pgfsetdash{}{0pt}%
\pgfsys@defobject{currentmarker}{\pgfqpoint{-0.048611in}{0.000000in}}{\pgfqpoint{-0.000000in}{0.000000in}}{%
\pgfpathmoveto{\pgfqpoint{-0.000000in}{0.000000in}}%
\pgfpathlineto{\pgfqpoint{-0.048611in}{0.000000in}}%
\pgfusepath{stroke,fill}%
}%
\begin{pgfscope}%
\pgfsys@transformshift{6.572727in}{1.575472in}%
\pgfsys@useobject{currentmarker}{}%
\end{pgfscope}%
\end{pgfscope}%
\begin{pgfscope}%
\definecolor{textcolor}{rgb}{0.000000,0.000000,0.000000}%
\pgfsetstrokecolor{textcolor}%
\pgfsetfillcolor{textcolor}%
\pgftext[x=6.146601in, y=1.522710in, left, base]{\color{textcolor}\sffamily\fontsize{10.000000}{12.000000}\selectfont \ensuremath{-}1.0}%
\end{pgfscope}%
\begin{pgfscope}%
\pgfsetbuttcap%
\pgfsetroundjoin%
\definecolor{currentfill}{rgb}{0.000000,0.000000,0.000000}%
\pgfsetfillcolor{currentfill}%
\pgfsetlinewidth{0.803000pt}%
\definecolor{currentstroke}{rgb}{0.000000,0.000000,0.000000}%
\pgfsetstrokecolor{currentstroke}%
\pgfsetdash{}{0pt}%
\pgfsys@defobject{currentmarker}{\pgfqpoint{-0.048611in}{0.000000in}}{\pgfqpoint{-0.000000in}{0.000000in}}{%
\pgfpathmoveto{\pgfqpoint{-0.000000in}{0.000000in}}%
\pgfpathlineto{\pgfqpoint{-0.048611in}{0.000000in}}%
\pgfusepath{stroke,fill}%
}%
\begin{pgfscope}%
\pgfsys@transformshift{6.572727in}{1.994639in}%
\pgfsys@useobject{currentmarker}{}%
\end{pgfscope}%
\end{pgfscope}%
\begin{pgfscope}%
\definecolor{textcolor}{rgb}{0.000000,0.000000,0.000000}%
\pgfsetstrokecolor{textcolor}%
\pgfsetfillcolor{textcolor}%
\pgftext[x=6.146601in, y=1.941877in, left, base]{\color{textcolor}\sffamily\fontsize{10.000000}{12.000000}\selectfont \ensuremath{-}0.5}%
\end{pgfscope}%
\begin{pgfscope}%
\pgfsetbuttcap%
\pgfsetroundjoin%
\definecolor{currentfill}{rgb}{0.000000,0.000000,0.000000}%
\pgfsetfillcolor{currentfill}%
\pgfsetlinewidth{0.803000pt}%
\definecolor{currentstroke}{rgb}{0.000000,0.000000,0.000000}%
\pgfsetstrokecolor{currentstroke}%
\pgfsetdash{}{0pt}%
\pgfsys@defobject{currentmarker}{\pgfqpoint{-0.048611in}{0.000000in}}{\pgfqpoint{-0.000000in}{0.000000in}}{%
\pgfpathmoveto{\pgfqpoint{-0.000000in}{0.000000in}}%
\pgfpathlineto{\pgfqpoint{-0.048611in}{0.000000in}}%
\pgfusepath{stroke,fill}%
}%
\begin{pgfscope}%
\pgfsys@transformshift{6.572727in}{2.413806in}%
\pgfsys@useobject{currentmarker}{}%
\end{pgfscope}%
\end{pgfscope}%
\begin{pgfscope}%
\definecolor{textcolor}{rgb}{0.000000,0.000000,0.000000}%
\pgfsetstrokecolor{textcolor}%
\pgfsetfillcolor{textcolor}%
\pgftext[x=6.254626in, y=2.361044in, left, base]{\color{textcolor}\sffamily\fontsize{10.000000}{12.000000}\selectfont 0.0}%
\end{pgfscope}%
\begin{pgfscope}%
\pgfsetbuttcap%
\pgfsetroundjoin%
\definecolor{currentfill}{rgb}{0.000000,0.000000,0.000000}%
\pgfsetfillcolor{currentfill}%
\pgfsetlinewidth{0.803000pt}%
\definecolor{currentstroke}{rgb}{0.000000,0.000000,0.000000}%
\pgfsetstrokecolor{currentstroke}%
\pgfsetdash{}{0pt}%
\pgfsys@defobject{currentmarker}{\pgfqpoint{-0.048611in}{0.000000in}}{\pgfqpoint{-0.000000in}{0.000000in}}{%
\pgfpathmoveto{\pgfqpoint{-0.000000in}{0.000000in}}%
\pgfpathlineto{\pgfqpoint{-0.048611in}{0.000000in}}%
\pgfusepath{stroke,fill}%
}%
\begin{pgfscope}%
\pgfsys@transformshift{6.572727in}{2.832973in}%
\pgfsys@useobject{currentmarker}{}%
\end{pgfscope}%
\end{pgfscope}%
\begin{pgfscope}%
\definecolor{textcolor}{rgb}{0.000000,0.000000,0.000000}%
\pgfsetstrokecolor{textcolor}%
\pgfsetfillcolor{textcolor}%
\pgftext[x=6.254626in, y=2.780211in, left, base]{\color{textcolor}\sffamily\fontsize{10.000000}{12.000000}\selectfont 0.5}%
\end{pgfscope}%
\begin{pgfscope}%
\pgfsetbuttcap%
\pgfsetroundjoin%
\definecolor{currentfill}{rgb}{0.000000,0.000000,0.000000}%
\pgfsetfillcolor{currentfill}%
\pgfsetlinewidth{0.803000pt}%
\definecolor{currentstroke}{rgb}{0.000000,0.000000,0.000000}%
\pgfsetstrokecolor{currentstroke}%
\pgfsetdash{}{0pt}%
\pgfsys@defobject{currentmarker}{\pgfqpoint{-0.048611in}{0.000000in}}{\pgfqpoint{-0.000000in}{0.000000in}}{%
\pgfpathmoveto{\pgfqpoint{-0.000000in}{0.000000in}}%
\pgfpathlineto{\pgfqpoint{-0.048611in}{0.000000in}}%
\pgfusepath{stroke,fill}%
}%
\begin{pgfscope}%
\pgfsys@transformshift{6.572727in}{3.252140in}%
\pgfsys@useobject{currentmarker}{}%
\end{pgfscope}%
\end{pgfscope}%
\begin{pgfscope}%
\definecolor{textcolor}{rgb}{0.000000,0.000000,0.000000}%
\pgfsetstrokecolor{textcolor}%
\pgfsetfillcolor{textcolor}%
\pgftext[x=6.254626in, y=3.199379in, left, base]{\color{textcolor}\sffamily\fontsize{10.000000}{12.000000}\selectfont 1.0}%
\end{pgfscope}%
\begin{pgfscope}%
\pgfsetbuttcap%
\pgfsetroundjoin%
\definecolor{currentfill}{rgb}{0.000000,0.000000,0.000000}%
\pgfsetfillcolor{currentfill}%
\pgfsetlinewidth{0.803000pt}%
\definecolor{currentstroke}{rgb}{0.000000,0.000000,0.000000}%
\pgfsetstrokecolor{currentstroke}%
\pgfsetdash{}{0pt}%
\pgfsys@defobject{currentmarker}{\pgfqpoint{-0.048611in}{0.000000in}}{\pgfqpoint{-0.000000in}{0.000000in}}{%
\pgfpathmoveto{\pgfqpoint{-0.000000in}{0.000000in}}%
\pgfpathlineto{\pgfqpoint{-0.048611in}{0.000000in}}%
\pgfusepath{stroke,fill}%
}%
\begin{pgfscope}%
\pgfsys@transformshift{6.572727in}{3.671307in}%
\pgfsys@useobject{currentmarker}{}%
\end{pgfscope}%
\end{pgfscope}%
\begin{pgfscope}%
\definecolor{textcolor}{rgb}{0.000000,0.000000,0.000000}%
\pgfsetstrokecolor{textcolor}%
\pgfsetfillcolor{textcolor}%
\pgftext[x=6.254626in, y=3.618546in, left, base]{\color{textcolor}\sffamily\fontsize{10.000000}{12.000000}\selectfont 1.5}%
\end{pgfscope}%
\begin{pgfscope}%
\pgfpathrectangle{\pgfqpoint{6.572727in}{0.474100in}}{\pgfqpoint{4.227273in}{3.318700in}}%
\pgfusepath{clip}%
\pgfsetbuttcap%
\pgfsetroundjoin%
\pgfsetlinewidth{1.003750pt}%
\definecolor{currentstroke}{rgb}{0.000000,0.000000,0.000000}%
\pgfsetstrokecolor{currentstroke}%
\pgfsetdash{{3.700000pt}{1.600000pt}}{0.000000pt}%
\pgfpathmoveto{\pgfqpoint{8.701412in}{2.023223in}}%
\pgfpathlineto{\pgfqpoint{9.947819in}{3.195861in}}%
\pgfusepath{stroke}%
\end{pgfscope}%
\begin{pgfscope}%
\pgfpathrectangle{\pgfqpoint{6.572727in}{0.474100in}}{\pgfqpoint{4.227273in}{3.318700in}}%
\pgfusepath{clip}%
\pgfsetbuttcap%
\pgfsetroundjoin%
\pgfsetlinewidth{1.003750pt}%
\definecolor{currentstroke}{rgb}{0.000000,0.000000,0.000000}%
\pgfsetstrokecolor{currentstroke}%
\pgfsetdash{{3.700000pt}{1.600000pt}}{0.000000pt}%
\pgfpathmoveto{\pgfqpoint{8.701412in}{2.023223in}}%
\pgfpathlineto{\pgfqpoint{6.975865in}{2.458271in}}%
\pgfusepath{stroke}%
\end{pgfscope}%
\begin{pgfscope}%
\pgfpathrectangle{\pgfqpoint{6.572727in}{0.474100in}}{\pgfqpoint{4.227273in}{3.318700in}}%
\pgfusepath{clip}%
\pgfsetbuttcap%
\pgfsetroundjoin%
\pgfsetlinewidth{1.003750pt}%
\definecolor{currentstroke}{rgb}{0.000000,0.000000,0.000000}%
\pgfsetstrokecolor{currentstroke}%
\pgfsetdash{{3.700000pt}{1.600000pt}}{0.000000pt}%
\pgfpathmoveto{\pgfqpoint{8.701412in}{2.023223in}}%
\pgfpathlineto{\pgfqpoint{8.739113in}{0.464100in}}%
\pgfusepath{stroke}%
\end{pgfscope}%
\begin{pgfscope}%
\pgfsetrectcap%
\pgfsetmiterjoin%
\pgfsetlinewidth{0.803000pt}%
\definecolor{currentstroke}{rgb}{0.000000,0.000000,0.000000}%
\pgfsetstrokecolor{currentstroke}%
\pgfsetdash{}{0pt}%
\pgfpathmoveto{\pgfqpoint{6.572727in}{0.474100in}}%
\pgfpathlineto{\pgfqpoint{6.572727in}{3.792800in}}%
\pgfusepath{stroke}%
\end{pgfscope}%
\begin{pgfscope}%
\pgfsetrectcap%
\pgfsetmiterjoin%
\pgfsetlinewidth{0.803000pt}%
\definecolor{currentstroke}{rgb}{0.000000,0.000000,0.000000}%
\pgfsetstrokecolor{currentstroke}%
\pgfsetdash{}{0pt}%
\pgfpathmoveto{\pgfqpoint{10.800000in}{0.474100in}}%
\pgfpathlineto{\pgfqpoint{10.800000in}{3.792800in}}%
\pgfusepath{stroke}%
\end{pgfscope}%
\begin{pgfscope}%
\pgfsetrectcap%
\pgfsetmiterjoin%
\pgfsetlinewidth{0.803000pt}%
\definecolor{currentstroke}{rgb}{0.000000,0.000000,0.000000}%
\pgfsetstrokecolor{currentstroke}%
\pgfsetdash{}{0pt}%
\pgfpathmoveto{\pgfqpoint{6.572727in}{0.474100in}}%
\pgfpathlineto{\pgfqpoint{10.800000in}{0.474100in}}%
\pgfusepath{stroke}%
\end{pgfscope}%
\begin{pgfscope}%
\pgfsetrectcap%
\pgfsetmiterjoin%
\pgfsetlinewidth{0.803000pt}%
\definecolor{currentstroke}{rgb}{0.000000,0.000000,0.000000}%
\pgfsetstrokecolor{currentstroke}%
\pgfsetdash{}{0pt}%
\pgfpathmoveto{\pgfqpoint{6.572727in}{3.792800in}}%
\pgfpathlineto{\pgfqpoint{10.800000in}{3.792800in}}%
\pgfusepath{stroke}%
\end{pgfscope}%
\begin{pgfscope}%
\definecolor{textcolor}{rgb}{0.000000,0.000000,0.000000}%
\pgfsetstrokecolor{textcolor}%
\pgfsetfillcolor{textcolor}%
\pgftext[x=8.686364in,y=3.876133in,,base]{\color{textcolor}\sffamily\fontsize{12.000000}{14.400000}\selectfont n\_clusters = 3}%
\end{pgfscope}%
\begin{pgfscope}%
\definecolor{textcolor}{rgb}{0.000000,0.000000,0.000000}%
\pgfsetstrokecolor{textcolor}%
\pgfsetfillcolor{textcolor}%
%\pgftext[x=6.000000in,y=4.223800in,,top]{\color{textcolor}\sffamily\fontsize{12.000000}{14.400000}\bfseries\selectfont Kmeans}%
\end{pgfscope}%
\end{pgfpicture}%
\makeatother%
\endgroup%
}}
\end{center}

\subsection{Clasificación con algoritmo DBSCAN}

Al aplicar el algoritmo \textit{DBSCAN} he fijado el número mínimo de elementos
en $n_0=10$ y he considerado los coeficientes de \textit{Silhouette} correspondientes a
los valores del umbral de distancia $\varepsilon\in(0.1, 0.4)$, tanto para la
métrica euclideana como para la de Manhattan.

En las gŕaficas de la izquierda se puede observar que ahora los valores del
umbral de distancia con coeficiente de \textit{Silhouette} máximo no están tan
diferenciados como en el caso del algoritmo \textit{KMeans}. Es decir, se intuye
que existe un entorno del valor óptimo $\varepsilon$ en el que el coeficiente de
\textit{Silhouette} tiene una variación acotada.

Además, para ambas métricas, los coeficientes de \textit{Silhouette} máximos son
considerablemente menores que en el caso anterior. Esto se refleja también en
las gráficas de la derecha, en las que están dibujadas las vecindades. En este
caso se han distinguido solo dos vecindades, estando agrupadas las vecindades 0
y 2 del apartado anterior en una sola. Con esto se entiende que el coeficiente
medio de \textit{Silhouette} sea menor, ya que la vecindad 0 ahora es más grande y sus 
puntos están "más separados entre sí".

\subsubsection{DBSCAN con métrica euclideana}
\begin{center}
    \vspace{-1em}
    \makebox[\textwidth][c]{\scalebox{0.65}{%% Creator: Matplotlib, PGF backend
%%
%% To include the figure in your LaTeX document, write
%%   \input{<filename>.pgf}
%%
%% Make sure the required packages are loaded in your preamble
%%   \usepackage{pgf}
%%
%% Figures using additional raster images can only be included by \input if
%% they are in the same directory as the main LaTeX file. For loading figures
%% from other directories you can use the `import` package
%%   \usepackage{import}
%%
%% and then include the figures with
%%   \import{<path to file>}{<filename>.pgf}
%%
%% Matplotlib used the following preamble
%%   \usepackage{fontspec}
%%   \setmainfont{DejaVuSerif.ttf}[Path=\detokenize{/nix/store/zl80nl46sadml2lln6v1xgbhqks16lz2-python3.8-matplotlib-3.4.3/lib/python3.8/site-packages/matplotlib/mpl-data/fonts/ttf/}]
%%   \setsansfont{DejaVuSans.ttf}[Path=\detokenize{/nix/store/zl80nl46sadml2lln6v1xgbhqks16lz2-python3.8-matplotlib-3.4.3/lib/python3.8/site-packages/matplotlib/mpl-data/fonts/ttf/}]
%%   \setmonofont{DejaVuSansMono.ttf}[Path=\detokenize{/nix/store/zl80nl46sadml2lln6v1xgbhqks16lz2-python3.8-matplotlib-3.4.3/lib/python3.8/site-packages/matplotlib/mpl-data/fonts/ttf/}]
%%
\begingroup%
\makeatletter%
\begin{pgfpicture}%
\pgfpathrectangle{\pgfpointorigin}{\pgfqpoint{12.000000in}{4.310000in}}%
\pgfusepath{use as bounding box, clip}%
\begin{pgfscope}%
\pgfsetbuttcap%
\pgfsetmiterjoin%
\definecolor{currentfill}{rgb}{1.000000,1.000000,1.000000}%
\pgfsetfillcolor{currentfill}%
\pgfsetlinewidth{0.000000pt}%
\definecolor{currentstroke}{rgb}{1.000000,1.000000,1.000000}%
\pgfsetstrokecolor{currentstroke}%
\pgfsetdash{}{0pt}%
\pgfpathmoveto{\pgfqpoint{0.000000in}{0.000000in}}%
\pgfpathlineto{\pgfqpoint{12.000000in}{0.000000in}}%
\pgfpathlineto{\pgfqpoint{12.000000in}{4.310000in}}%
\pgfpathlineto{\pgfqpoint{0.000000in}{4.310000in}}%
\pgfpathclose%
\pgfusepath{fill}%
\end{pgfscope}%
\begin{pgfscope}%
\pgfsetbuttcap%
\pgfsetmiterjoin%
\definecolor{currentfill}{rgb}{1.000000,1.000000,1.000000}%
\pgfsetfillcolor{currentfill}%
\pgfsetlinewidth{0.000000pt}%
\definecolor{currentstroke}{rgb}{0.000000,0.000000,0.000000}%
\pgfsetstrokecolor{currentstroke}%
\pgfsetstrokeopacity{0.000000}%
\pgfsetdash{}{0pt}%
\pgfpathmoveto{\pgfqpoint{1.500000in}{0.474100in}}%
\pgfpathlineto{\pgfqpoint{5.727273in}{0.474100in}}%
\pgfpathlineto{\pgfqpoint{5.727273in}{3.792800in}}%
\pgfpathlineto{\pgfqpoint{1.500000in}{3.792800in}}%
\pgfpathclose%
\pgfusepath{fill}%
\end{pgfscope}%
\begin{pgfscope}%
\pgfpathrectangle{\pgfqpoint{1.500000in}{0.474100in}}{\pgfqpoint{4.227273in}{3.318700in}}%
\pgfusepath{clip}%
\pgfsetbuttcap%
\pgfsetmiterjoin%
\definecolor{currentfill}{rgb}{0.121569,0.466667,0.705882}%
\pgfsetfillcolor{currentfill}%
\pgfsetlinewidth{0.000000pt}%
\definecolor{currentstroke}{rgb}{0.000000,0.000000,0.000000}%
\pgfsetstrokecolor{currentstroke}%
\pgfsetstrokeopacity{0.000000}%
\pgfsetdash{}{0pt}%
\pgfpathmoveto{\pgfqpoint{1.692149in}{1.840783in}}%
\pgfpathlineto{\pgfqpoint{1.804080in}{1.840783in}}%
\pgfpathlineto{\pgfqpoint{1.804080in}{0.624950in}}%
\pgfpathlineto{\pgfqpoint{1.692149in}{0.624950in}}%
\pgfpathclose%
\pgfusepath{fill}%
\end{pgfscope}%
\begin{pgfscope}%
\pgfpathrectangle{\pgfqpoint{1.500000in}{0.474100in}}{\pgfqpoint{4.227273in}{3.318700in}}%
\pgfusepath{clip}%
\pgfsetbuttcap%
\pgfsetmiterjoin%
\definecolor{currentfill}{rgb}{0.121569,0.466667,0.705882}%
\pgfsetfillcolor{currentfill}%
\pgfsetlinewidth{0.000000pt}%
\definecolor{currentstroke}{rgb}{0.000000,0.000000,0.000000}%
\pgfsetstrokecolor{currentstroke}%
\pgfsetstrokeopacity{0.000000}%
\pgfsetdash{}{0pt}%
\pgfpathmoveto{\pgfqpoint{1.816517in}{1.840783in}}%
\pgfpathlineto{\pgfqpoint{1.928448in}{1.840783in}}%
\pgfpathlineto{\pgfqpoint{1.928448in}{0.866687in}}%
\pgfpathlineto{\pgfqpoint{1.816517in}{0.866687in}}%
\pgfpathclose%
\pgfusepath{fill}%
\end{pgfscope}%
\begin{pgfscope}%
\pgfpathrectangle{\pgfqpoint{1.500000in}{0.474100in}}{\pgfqpoint{4.227273in}{3.318700in}}%
\pgfusepath{clip}%
\pgfsetbuttcap%
\pgfsetmiterjoin%
\definecolor{currentfill}{rgb}{0.121569,0.466667,0.705882}%
\pgfsetfillcolor{currentfill}%
\pgfsetlinewidth{0.000000pt}%
\definecolor{currentstroke}{rgb}{0.000000,0.000000,0.000000}%
\pgfsetstrokecolor{currentstroke}%
\pgfsetstrokeopacity{0.000000}%
\pgfsetdash{}{0pt}%
\pgfpathmoveto{\pgfqpoint{1.940885in}{1.840783in}}%
\pgfpathlineto{\pgfqpoint{2.052816in}{1.840783in}}%
\pgfpathlineto{\pgfqpoint{2.052816in}{1.874796in}}%
\pgfpathlineto{\pgfqpoint{1.940885in}{1.874796in}}%
\pgfpathclose%
\pgfusepath{fill}%
\end{pgfscope}%
\begin{pgfscope}%
\pgfpathrectangle{\pgfqpoint{1.500000in}{0.474100in}}{\pgfqpoint{4.227273in}{3.318700in}}%
\pgfusepath{clip}%
\pgfsetbuttcap%
\pgfsetmiterjoin%
\definecolor{currentfill}{rgb}{0.121569,0.466667,0.705882}%
\pgfsetfillcolor{currentfill}%
\pgfsetlinewidth{0.000000pt}%
\definecolor{currentstroke}{rgb}{0.000000,0.000000,0.000000}%
\pgfsetstrokecolor{currentstroke}%
\pgfsetstrokeopacity{0.000000}%
\pgfsetdash{}{0pt}%
\pgfpathmoveto{\pgfqpoint{2.065253in}{1.840783in}}%
\pgfpathlineto{\pgfqpoint{2.177184in}{1.840783in}}%
\pgfpathlineto{\pgfqpoint{2.177184in}{1.675965in}}%
\pgfpathlineto{\pgfqpoint{2.065253in}{1.675965in}}%
\pgfpathclose%
\pgfusepath{fill}%
\end{pgfscope}%
\begin{pgfscope}%
\pgfpathrectangle{\pgfqpoint{1.500000in}{0.474100in}}{\pgfqpoint{4.227273in}{3.318700in}}%
\pgfusepath{clip}%
\pgfsetbuttcap%
\pgfsetmiterjoin%
\definecolor{currentfill}{rgb}{0.121569,0.466667,0.705882}%
\pgfsetfillcolor{currentfill}%
\pgfsetlinewidth{0.000000pt}%
\definecolor{currentstroke}{rgb}{0.000000,0.000000,0.000000}%
\pgfsetstrokecolor{currentstroke}%
\pgfsetstrokeopacity{0.000000}%
\pgfsetdash{}{0pt}%
\pgfpathmoveto{\pgfqpoint{2.189621in}{1.840783in}}%
\pgfpathlineto{\pgfqpoint{2.301553in}{1.840783in}}%
\pgfpathlineto{\pgfqpoint{2.301553in}{1.972304in}}%
\pgfpathlineto{\pgfqpoint{2.189621in}{1.972304in}}%
\pgfpathclose%
\pgfusepath{fill}%
\end{pgfscope}%
\begin{pgfscope}%
\pgfpathrectangle{\pgfqpoint{1.500000in}{0.474100in}}{\pgfqpoint{4.227273in}{3.318700in}}%
\pgfusepath{clip}%
\pgfsetbuttcap%
\pgfsetmiterjoin%
\definecolor{currentfill}{rgb}{0.121569,0.466667,0.705882}%
\pgfsetfillcolor{currentfill}%
\pgfsetlinewidth{0.000000pt}%
\definecolor{currentstroke}{rgb}{0.000000,0.000000,0.000000}%
\pgfsetstrokecolor{currentstroke}%
\pgfsetstrokeopacity{0.000000}%
\pgfsetdash{}{0pt}%
\pgfpathmoveto{\pgfqpoint{2.313989in}{1.840783in}}%
\pgfpathlineto{\pgfqpoint{2.425921in}{1.840783in}}%
\pgfpathlineto{\pgfqpoint{2.425921in}{2.954643in}}%
\pgfpathlineto{\pgfqpoint{2.313989in}{2.954643in}}%
\pgfpathclose%
\pgfusepath{fill}%
\end{pgfscope}%
\begin{pgfscope}%
\pgfpathrectangle{\pgfqpoint{1.500000in}{0.474100in}}{\pgfqpoint{4.227273in}{3.318700in}}%
\pgfusepath{clip}%
\pgfsetbuttcap%
\pgfsetmiterjoin%
\definecolor{currentfill}{rgb}{0.121569,0.466667,0.705882}%
\pgfsetfillcolor{currentfill}%
\pgfsetlinewidth{0.000000pt}%
\definecolor{currentstroke}{rgb}{0.000000,0.000000,0.000000}%
\pgfsetstrokecolor{currentstroke}%
\pgfsetstrokeopacity{0.000000}%
\pgfsetdash{}{0pt}%
\pgfpathmoveto{\pgfqpoint{2.438358in}{1.840783in}}%
\pgfpathlineto{\pgfqpoint{2.550289in}{1.840783in}}%
\pgfpathlineto{\pgfqpoint{2.550289in}{3.089582in}}%
\pgfpathlineto{\pgfqpoint{2.438358in}{3.089582in}}%
\pgfpathclose%
\pgfusepath{fill}%
\end{pgfscope}%
\begin{pgfscope}%
\pgfpathrectangle{\pgfqpoint{1.500000in}{0.474100in}}{\pgfqpoint{4.227273in}{3.318700in}}%
\pgfusepath{clip}%
\pgfsetbuttcap%
\pgfsetmiterjoin%
\definecolor{currentfill}{rgb}{0.121569,0.466667,0.705882}%
\pgfsetfillcolor{currentfill}%
\pgfsetlinewidth{0.000000pt}%
\definecolor{currentstroke}{rgb}{0.000000,0.000000,0.000000}%
\pgfsetstrokecolor{currentstroke}%
\pgfsetstrokeopacity{0.000000}%
\pgfsetdash{}{0pt}%
\pgfpathmoveto{\pgfqpoint{2.562726in}{1.840783in}}%
\pgfpathlineto{\pgfqpoint{2.674657in}{1.840783in}}%
\pgfpathlineto{\pgfqpoint{2.674657in}{3.178798in}}%
\pgfpathlineto{\pgfqpoint{2.562726in}{3.178798in}}%
\pgfpathclose%
\pgfusepath{fill}%
\end{pgfscope}%
\begin{pgfscope}%
\pgfpathrectangle{\pgfqpoint{1.500000in}{0.474100in}}{\pgfqpoint{4.227273in}{3.318700in}}%
\pgfusepath{clip}%
\pgfsetbuttcap%
\pgfsetmiterjoin%
\definecolor{currentfill}{rgb}{0.121569,0.466667,0.705882}%
\pgfsetfillcolor{currentfill}%
\pgfsetlinewidth{0.000000pt}%
\definecolor{currentstroke}{rgb}{0.000000,0.000000,0.000000}%
\pgfsetstrokecolor{currentstroke}%
\pgfsetstrokeopacity{0.000000}%
\pgfsetdash{}{0pt}%
\pgfpathmoveto{\pgfqpoint{2.687094in}{1.840783in}}%
\pgfpathlineto{\pgfqpoint{2.799025in}{1.840783in}}%
\pgfpathlineto{\pgfqpoint{2.799025in}{3.290464in}}%
\pgfpathlineto{\pgfqpoint{2.687094in}{3.290464in}}%
\pgfpathclose%
\pgfusepath{fill}%
\end{pgfscope}%
\begin{pgfscope}%
\pgfpathrectangle{\pgfqpoint{1.500000in}{0.474100in}}{\pgfqpoint{4.227273in}{3.318700in}}%
\pgfusepath{clip}%
\pgfsetbuttcap%
\pgfsetmiterjoin%
\definecolor{currentfill}{rgb}{0.121569,0.466667,0.705882}%
\pgfsetfillcolor{currentfill}%
\pgfsetlinewidth{0.000000pt}%
\definecolor{currentstroke}{rgb}{0.000000,0.000000,0.000000}%
\pgfsetstrokecolor{currentstroke}%
\pgfsetstrokeopacity{0.000000}%
\pgfsetdash{}{0pt}%
\pgfpathmoveto{\pgfqpoint{2.811462in}{1.840783in}}%
\pgfpathlineto{\pgfqpoint{2.923393in}{1.840783in}}%
\pgfpathlineto{\pgfqpoint{2.923393in}{3.326820in}}%
\pgfpathlineto{\pgfqpoint{2.811462in}{3.326820in}}%
\pgfpathclose%
\pgfusepath{fill}%
\end{pgfscope}%
\begin{pgfscope}%
\pgfpathrectangle{\pgfqpoint{1.500000in}{0.474100in}}{\pgfqpoint{4.227273in}{3.318700in}}%
\pgfusepath{clip}%
\pgfsetbuttcap%
\pgfsetmiterjoin%
\definecolor{currentfill}{rgb}{0.121569,0.466667,0.705882}%
\pgfsetfillcolor{currentfill}%
\pgfsetlinewidth{0.000000pt}%
\definecolor{currentstroke}{rgb}{0.000000,0.000000,0.000000}%
\pgfsetstrokecolor{currentstroke}%
\pgfsetstrokeopacity{0.000000}%
\pgfsetdash{}{0pt}%
\pgfpathmoveto{\pgfqpoint{2.935830in}{1.840783in}}%
\pgfpathlineto{\pgfqpoint{3.047761in}{1.840783in}}%
\pgfpathlineto{\pgfqpoint{3.047761in}{3.353640in}}%
\pgfpathlineto{\pgfqpoint{2.935830in}{3.353640in}}%
\pgfpathclose%
\pgfusepath{fill}%
\end{pgfscope}%
\begin{pgfscope}%
\pgfpathrectangle{\pgfqpoint{1.500000in}{0.474100in}}{\pgfqpoint{4.227273in}{3.318700in}}%
\pgfusepath{clip}%
\pgfsetbuttcap%
\pgfsetmiterjoin%
\definecolor{currentfill}{rgb}{0.121569,0.466667,0.705882}%
\pgfsetfillcolor{currentfill}%
\pgfsetlinewidth{0.000000pt}%
\definecolor{currentstroke}{rgb}{0.000000,0.000000,0.000000}%
\pgfsetstrokecolor{currentstroke}%
\pgfsetstrokeopacity{0.000000}%
\pgfsetdash{}{0pt}%
\pgfpathmoveto{\pgfqpoint{3.060198in}{1.840783in}}%
\pgfpathlineto{\pgfqpoint{3.172130in}{1.840783in}}%
\pgfpathlineto{\pgfqpoint{3.172130in}{3.367294in}}%
\pgfpathlineto{\pgfqpoint{3.060198in}{3.367294in}}%
\pgfpathclose%
\pgfusepath{fill}%
\end{pgfscope}%
\begin{pgfscope}%
\pgfpathrectangle{\pgfqpoint{1.500000in}{0.474100in}}{\pgfqpoint{4.227273in}{3.318700in}}%
\pgfusepath{clip}%
\pgfsetbuttcap%
\pgfsetmiterjoin%
\definecolor{currentfill}{rgb}{0.121569,0.466667,0.705882}%
\pgfsetfillcolor{currentfill}%
\pgfsetlinewidth{0.000000pt}%
\definecolor{currentstroke}{rgb}{0.000000,0.000000,0.000000}%
\pgfsetstrokecolor{currentstroke}%
\pgfsetstrokeopacity{0.000000}%
\pgfsetdash{}{0pt}%
\pgfpathmoveto{\pgfqpoint{3.184566in}{1.840783in}}%
\pgfpathlineto{\pgfqpoint{3.296498in}{1.840783in}}%
\pgfpathlineto{\pgfqpoint{3.296498in}{3.411393in}}%
\pgfpathlineto{\pgfqpoint{3.184566in}{3.411393in}}%
\pgfpathclose%
\pgfusepath{fill}%
\end{pgfscope}%
\begin{pgfscope}%
\pgfpathrectangle{\pgfqpoint{1.500000in}{0.474100in}}{\pgfqpoint{4.227273in}{3.318700in}}%
\pgfusepath{clip}%
\pgfsetbuttcap%
\pgfsetmiterjoin%
\definecolor{currentfill}{rgb}{0.121569,0.466667,0.705882}%
\pgfsetfillcolor{currentfill}%
\pgfsetlinewidth{0.000000pt}%
\definecolor{currentstroke}{rgb}{0.000000,0.000000,0.000000}%
\pgfsetstrokecolor{currentstroke}%
\pgfsetstrokeopacity{0.000000}%
\pgfsetdash{}{0pt}%
\pgfpathmoveto{\pgfqpoint{3.308934in}{1.840783in}}%
\pgfpathlineto{\pgfqpoint{3.420866in}{1.840783in}}%
\pgfpathlineto{\pgfqpoint{3.420866in}{3.452739in}}%
\pgfpathlineto{\pgfqpoint{3.308934in}{3.452739in}}%
\pgfpathclose%
\pgfusepath{fill}%
\end{pgfscope}%
\begin{pgfscope}%
\pgfpathrectangle{\pgfqpoint{1.500000in}{0.474100in}}{\pgfqpoint{4.227273in}{3.318700in}}%
\pgfusepath{clip}%
\pgfsetbuttcap%
\pgfsetmiterjoin%
\definecolor{currentfill}{rgb}{0.121569,0.466667,0.705882}%
\pgfsetfillcolor{currentfill}%
\pgfsetlinewidth{0.000000pt}%
\definecolor{currentstroke}{rgb}{0.000000,0.000000,0.000000}%
\pgfsetstrokecolor{currentstroke}%
\pgfsetstrokeopacity{0.000000}%
\pgfsetdash{}{0pt}%
\pgfpathmoveto{\pgfqpoint{3.433303in}{1.840783in}}%
\pgfpathlineto{\pgfqpoint{3.545234in}{1.840783in}}%
\pgfpathlineto{\pgfqpoint{3.545234in}{3.475594in}}%
\pgfpathlineto{\pgfqpoint{3.433303in}{3.475594in}}%
\pgfpathclose%
\pgfusepath{fill}%
\end{pgfscope}%
\begin{pgfscope}%
\pgfpathrectangle{\pgfqpoint{1.500000in}{0.474100in}}{\pgfqpoint{4.227273in}{3.318700in}}%
\pgfusepath{clip}%
\pgfsetbuttcap%
\pgfsetmiterjoin%
\definecolor{currentfill}{rgb}{0.121569,0.466667,0.705882}%
\pgfsetfillcolor{currentfill}%
\pgfsetlinewidth{0.000000pt}%
\definecolor{currentstroke}{rgb}{0.000000,0.000000,0.000000}%
\pgfsetstrokecolor{currentstroke}%
\pgfsetstrokeopacity{0.000000}%
\pgfsetdash{}{0pt}%
\pgfpathmoveto{\pgfqpoint{3.557671in}{1.840783in}}%
\pgfpathlineto{\pgfqpoint{3.669602in}{1.840783in}}%
\pgfpathlineto{\pgfqpoint{3.669602in}{3.450853in}}%
\pgfpathlineto{\pgfqpoint{3.557671in}{3.450853in}}%
\pgfpathclose%
\pgfusepath{fill}%
\end{pgfscope}%
\begin{pgfscope}%
\pgfpathrectangle{\pgfqpoint{1.500000in}{0.474100in}}{\pgfqpoint{4.227273in}{3.318700in}}%
\pgfusepath{clip}%
\pgfsetbuttcap%
\pgfsetmiterjoin%
\definecolor{currentfill}{rgb}{0.121569,0.466667,0.705882}%
\pgfsetfillcolor{currentfill}%
\pgfsetlinewidth{0.000000pt}%
\definecolor{currentstroke}{rgb}{0.000000,0.000000,0.000000}%
\pgfsetstrokecolor{currentstroke}%
\pgfsetstrokeopacity{0.000000}%
\pgfsetdash{}{0pt}%
\pgfpathmoveto{\pgfqpoint{3.682039in}{1.840783in}}%
\pgfpathlineto{\pgfqpoint{3.793970in}{1.840783in}}%
\pgfpathlineto{\pgfqpoint{3.793970in}{3.602838in}}%
\pgfpathlineto{\pgfqpoint{3.682039in}{3.602838in}}%
\pgfpathclose%
\pgfusepath{fill}%
\end{pgfscope}%
\begin{pgfscope}%
\pgfpathrectangle{\pgfqpoint{1.500000in}{0.474100in}}{\pgfqpoint{4.227273in}{3.318700in}}%
\pgfusepath{clip}%
\pgfsetbuttcap%
\pgfsetmiterjoin%
\definecolor{currentfill}{rgb}{0.121569,0.466667,0.705882}%
\pgfsetfillcolor{currentfill}%
\pgfsetlinewidth{0.000000pt}%
\definecolor{currentstroke}{rgb}{0.000000,0.000000,0.000000}%
\pgfsetstrokecolor{currentstroke}%
\pgfsetstrokeopacity{0.000000}%
\pgfsetdash{}{0pt}%
\pgfpathmoveto{\pgfqpoint{3.806407in}{1.840783in}}%
\pgfpathlineto{\pgfqpoint{3.918338in}{1.840783in}}%
\pgfpathlineto{\pgfqpoint{3.918338in}{3.603244in}}%
\pgfpathlineto{\pgfqpoint{3.806407in}{3.603244in}}%
\pgfpathclose%
\pgfusepath{fill}%
\end{pgfscope}%
\begin{pgfscope}%
\pgfpathrectangle{\pgfqpoint{1.500000in}{0.474100in}}{\pgfqpoint{4.227273in}{3.318700in}}%
\pgfusepath{clip}%
\pgfsetbuttcap%
\pgfsetmiterjoin%
\definecolor{currentfill}{rgb}{1.000000,0.000000,0.000000}%
\pgfsetfillcolor{currentfill}%
\pgfsetlinewidth{1.003750pt}%
\definecolor{currentstroke}{rgb}{1.000000,0.000000,0.000000}%
\pgfsetstrokecolor{currentstroke}%
\pgfsetdash{}{0pt}%
\pgfpathmoveto{\pgfqpoint{3.930775in}{1.840783in}}%
\pgfpathlineto{\pgfqpoint{4.042706in}{1.840783in}}%
\pgfpathlineto{\pgfqpoint{4.042706in}{3.641950in}}%
\pgfpathlineto{\pgfqpoint{3.930775in}{3.641950in}}%
\pgfpathclose%
\pgfusepath{stroke,fill}%
\end{pgfscope}%
\begin{pgfscope}%
\pgfpathrectangle{\pgfqpoint{1.500000in}{0.474100in}}{\pgfqpoint{4.227273in}{3.318700in}}%
\pgfusepath{clip}%
\pgfsetbuttcap%
\pgfsetmiterjoin%
\definecolor{currentfill}{rgb}{0.121569,0.466667,0.705882}%
\pgfsetfillcolor{currentfill}%
\pgfsetlinewidth{0.000000pt}%
\definecolor{currentstroke}{rgb}{0.000000,0.000000,0.000000}%
\pgfsetstrokecolor{currentstroke}%
\pgfsetstrokeopacity{0.000000}%
\pgfsetdash{}{0pt}%
\pgfpathmoveto{\pgfqpoint{4.055143in}{1.840783in}}%
\pgfpathlineto{\pgfqpoint{4.167075in}{1.840783in}}%
\pgfpathlineto{\pgfqpoint{4.167075in}{3.600384in}}%
\pgfpathlineto{\pgfqpoint{4.055143in}{3.600384in}}%
\pgfpathclose%
\pgfusepath{fill}%
\end{pgfscope}%
\begin{pgfscope}%
\pgfpathrectangle{\pgfqpoint{1.500000in}{0.474100in}}{\pgfqpoint{4.227273in}{3.318700in}}%
\pgfusepath{clip}%
\pgfsetbuttcap%
\pgfsetmiterjoin%
\definecolor{currentfill}{rgb}{0.121569,0.466667,0.705882}%
\pgfsetfillcolor{currentfill}%
\pgfsetlinewidth{0.000000pt}%
\definecolor{currentstroke}{rgb}{0.000000,0.000000,0.000000}%
\pgfsetstrokecolor{currentstroke}%
\pgfsetstrokeopacity{0.000000}%
\pgfsetdash{}{0pt}%
\pgfpathmoveto{\pgfqpoint{4.179511in}{1.840783in}}%
\pgfpathlineto{\pgfqpoint{4.291443in}{1.840783in}}%
\pgfpathlineto{\pgfqpoint{4.291443in}{3.639370in}}%
\pgfpathlineto{\pgfqpoint{4.179511in}{3.639370in}}%
\pgfpathclose%
\pgfusepath{fill}%
\end{pgfscope}%
\begin{pgfscope}%
\pgfpathrectangle{\pgfqpoint{1.500000in}{0.474100in}}{\pgfqpoint{4.227273in}{3.318700in}}%
\pgfusepath{clip}%
\pgfsetbuttcap%
\pgfsetmiterjoin%
\definecolor{currentfill}{rgb}{0.121569,0.466667,0.705882}%
\pgfsetfillcolor{currentfill}%
\pgfsetlinewidth{0.000000pt}%
\definecolor{currentstroke}{rgb}{0.000000,0.000000,0.000000}%
\pgfsetstrokecolor{currentstroke}%
\pgfsetstrokeopacity{0.000000}%
\pgfsetdash{}{0pt}%
\pgfpathmoveto{\pgfqpoint{4.303879in}{1.840783in}}%
\pgfpathlineto{\pgfqpoint{4.415811in}{1.840783in}}%
\pgfpathlineto{\pgfqpoint{4.415811in}{2.723775in}}%
\pgfpathlineto{\pgfqpoint{4.303879in}{2.723775in}}%
\pgfpathclose%
\pgfusepath{fill}%
\end{pgfscope}%
\begin{pgfscope}%
\pgfpathrectangle{\pgfqpoint{1.500000in}{0.474100in}}{\pgfqpoint{4.227273in}{3.318700in}}%
\pgfusepath{clip}%
\pgfsetbuttcap%
\pgfsetmiterjoin%
\definecolor{currentfill}{rgb}{0.121569,0.466667,0.705882}%
\pgfsetfillcolor{currentfill}%
\pgfsetlinewidth{0.000000pt}%
\definecolor{currentstroke}{rgb}{0.000000,0.000000,0.000000}%
\pgfsetstrokecolor{currentstroke}%
\pgfsetstrokeopacity{0.000000}%
\pgfsetdash{}{0pt}%
\pgfpathmoveto{\pgfqpoint{4.428248in}{1.840783in}}%
\pgfpathlineto{\pgfqpoint{4.540179in}{1.840783in}}%
\pgfpathlineto{\pgfqpoint{4.540179in}{2.989963in}}%
\pgfpathlineto{\pgfqpoint{4.428248in}{2.989963in}}%
\pgfpathclose%
\pgfusepath{fill}%
\end{pgfscope}%
\begin{pgfscope}%
\pgfpathrectangle{\pgfqpoint{1.500000in}{0.474100in}}{\pgfqpoint{4.227273in}{3.318700in}}%
\pgfusepath{clip}%
\pgfsetbuttcap%
\pgfsetmiterjoin%
\definecolor{currentfill}{rgb}{0.121569,0.466667,0.705882}%
\pgfsetfillcolor{currentfill}%
\pgfsetlinewidth{0.000000pt}%
\definecolor{currentstroke}{rgb}{0.000000,0.000000,0.000000}%
\pgfsetstrokecolor{currentstroke}%
\pgfsetstrokeopacity{0.000000}%
\pgfsetdash{}{0pt}%
\pgfpathmoveto{\pgfqpoint{4.552616in}{1.840783in}}%
\pgfpathlineto{\pgfqpoint{4.664547in}{1.840783in}}%
\pgfpathlineto{\pgfqpoint{4.664547in}{2.989963in}}%
\pgfpathlineto{\pgfqpoint{4.552616in}{2.989963in}}%
\pgfpathclose%
\pgfusepath{fill}%
\end{pgfscope}%
\begin{pgfscope}%
\pgfpathrectangle{\pgfqpoint{1.500000in}{0.474100in}}{\pgfqpoint{4.227273in}{3.318700in}}%
\pgfusepath{clip}%
\pgfsetbuttcap%
\pgfsetmiterjoin%
\definecolor{currentfill}{rgb}{0.121569,0.466667,0.705882}%
\pgfsetfillcolor{currentfill}%
\pgfsetlinewidth{0.000000pt}%
\definecolor{currentstroke}{rgb}{0.000000,0.000000,0.000000}%
\pgfsetstrokecolor{currentstroke}%
\pgfsetstrokeopacity{0.000000}%
\pgfsetdash{}{0pt}%
\pgfpathmoveto{\pgfqpoint{4.676984in}{1.840783in}}%
\pgfpathlineto{\pgfqpoint{4.788915in}{1.840783in}}%
\pgfpathlineto{\pgfqpoint{4.788915in}{2.911859in}}%
\pgfpathlineto{\pgfqpoint{4.676984in}{2.911859in}}%
\pgfpathclose%
\pgfusepath{fill}%
\end{pgfscope}%
\begin{pgfscope}%
\pgfpathrectangle{\pgfqpoint{1.500000in}{0.474100in}}{\pgfqpoint{4.227273in}{3.318700in}}%
\pgfusepath{clip}%
\pgfsetbuttcap%
\pgfsetmiterjoin%
\definecolor{currentfill}{rgb}{0.121569,0.466667,0.705882}%
\pgfsetfillcolor{currentfill}%
\pgfsetlinewidth{0.000000pt}%
\definecolor{currentstroke}{rgb}{0.000000,0.000000,0.000000}%
\pgfsetstrokecolor{currentstroke}%
\pgfsetstrokeopacity{0.000000}%
\pgfsetdash{}{0pt}%
\pgfpathmoveto{\pgfqpoint{4.801352in}{1.840783in}}%
\pgfpathlineto{\pgfqpoint{4.913283in}{1.840783in}}%
\pgfpathlineto{\pgfqpoint{4.913283in}{2.820096in}}%
\pgfpathlineto{\pgfqpoint{4.801352in}{2.820096in}}%
\pgfpathclose%
\pgfusepath{fill}%
\end{pgfscope}%
\begin{pgfscope}%
\pgfpathrectangle{\pgfqpoint{1.500000in}{0.474100in}}{\pgfqpoint{4.227273in}{3.318700in}}%
\pgfusepath{clip}%
\pgfsetbuttcap%
\pgfsetmiterjoin%
\definecolor{currentfill}{rgb}{0.121569,0.466667,0.705882}%
\pgfsetfillcolor{currentfill}%
\pgfsetlinewidth{0.000000pt}%
\definecolor{currentstroke}{rgb}{0.000000,0.000000,0.000000}%
\pgfsetstrokecolor{currentstroke}%
\pgfsetstrokeopacity{0.000000}%
\pgfsetdash{}{0pt}%
\pgfpathmoveto{\pgfqpoint{4.925720in}{1.840783in}}%
\pgfpathlineto{\pgfqpoint{5.037651in}{1.840783in}}%
\pgfpathlineto{\pgfqpoint{5.037651in}{2.820096in}}%
\pgfpathlineto{\pgfqpoint{4.925720in}{2.820096in}}%
\pgfpathclose%
\pgfusepath{fill}%
\end{pgfscope}%
\begin{pgfscope}%
\pgfpathrectangle{\pgfqpoint{1.500000in}{0.474100in}}{\pgfqpoint{4.227273in}{3.318700in}}%
\pgfusepath{clip}%
\pgfsetbuttcap%
\pgfsetmiterjoin%
\definecolor{currentfill}{rgb}{0.121569,0.466667,0.705882}%
\pgfsetfillcolor{currentfill}%
\pgfsetlinewidth{0.000000pt}%
\definecolor{currentstroke}{rgb}{0.000000,0.000000,0.000000}%
\pgfsetstrokecolor{currentstroke}%
\pgfsetstrokeopacity{0.000000}%
\pgfsetdash{}{0pt}%
\pgfpathmoveto{\pgfqpoint{5.050088in}{1.840783in}}%
\pgfpathlineto{\pgfqpoint{5.162020in}{1.840783in}}%
\pgfpathlineto{\pgfqpoint{5.162020in}{2.813265in}}%
\pgfpathlineto{\pgfqpoint{5.050088in}{2.813265in}}%
\pgfpathclose%
\pgfusepath{fill}%
\end{pgfscope}%
\begin{pgfscope}%
\pgfpathrectangle{\pgfqpoint{1.500000in}{0.474100in}}{\pgfqpoint{4.227273in}{3.318700in}}%
\pgfusepath{clip}%
\pgfsetbuttcap%
\pgfsetmiterjoin%
\definecolor{currentfill}{rgb}{0.121569,0.466667,0.705882}%
\pgfsetfillcolor{currentfill}%
\pgfsetlinewidth{0.000000pt}%
\definecolor{currentstroke}{rgb}{0.000000,0.000000,0.000000}%
\pgfsetstrokecolor{currentstroke}%
\pgfsetstrokeopacity{0.000000}%
\pgfsetdash{}{0pt}%
\pgfpathmoveto{\pgfqpoint{5.174456in}{1.840783in}}%
\pgfpathlineto{\pgfqpoint{5.286388in}{1.840783in}}%
\pgfpathlineto{\pgfqpoint{5.286388in}{2.587022in}}%
\pgfpathlineto{\pgfqpoint{5.174456in}{2.587022in}}%
\pgfpathclose%
\pgfusepath{fill}%
\end{pgfscope}%
\begin{pgfscope}%
\pgfpathrectangle{\pgfqpoint{1.500000in}{0.474100in}}{\pgfqpoint{4.227273in}{3.318700in}}%
\pgfusepath{clip}%
\pgfsetbuttcap%
\pgfsetmiterjoin%
\definecolor{currentfill}{rgb}{0.121569,0.466667,0.705882}%
\pgfsetfillcolor{currentfill}%
\pgfsetlinewidth{0.000000pt}%
\definecolor{currentstroke}{rgb}{0.000000,0.000000,0.000000}%
\pgfsetstrokecolor{currentstroke}%
\pgfsetstrokeopacity{0.000000}%
\pgfsetdash{}{0pt}%
\pgfpathmoveto{\pgfqpoint{5.298825in}{1.840783in}}%
\pgfpathlineto{\pgfqpoint{5.410756in}{1.840783in}}%
\pgfpathlineto{\pgfqpoint{5.410756in}{2.587022in}}%
\pgfpathlineto{\pgfqpoint{5.298825in}{2.587022in}}%
\pgfpathclose%
\pgfusepath{fill}%
\end{pgfscope}%
\begin{pgfscope}%
\pgfpathrectangle{\pgfqpoint{1.500000in}{0.474100in}}{\pgfqpoint{4.227273in}{3.318700in}}%
\pgfusepath{clip}%
\pgfsetbuttcap%
\pgfsetmiterjoin%
\definecolor{currentfill}{rgb}{0.121569,0.466667,0.705882}%
\pgfsetfillcolor{currentfill}%
\pgfsetlinewidth{0.000000pt}%
\definecolor{currentstroke}{rgb}{0.000000,0.000000,0.000000}%
\pgfsetstrokecolor{currentstroke}%
\pgfsetstrokeopacity{0.000000}%
\pgfsetdash{}{0pt}%
\pgfpathmoveto{\pgfqpoint{5.423193in}{1.840783in}}%
\pgfpathlineto{\pgfqpoint{5.535124in}{1.840783in}}%
\pgfpathlineto{\pgfqpoint{5.535124in}{2.808911in}}%
\pgfpathlineto{\pgfqpoint{5.423193in}{2.808911in}}%
\pgfpathclose%
\pgfusepath{fill}%
\end{pgfscope}%
\begin{pgfscope}%
\pgfsetbuttcap%
\pgfsetroundjoin%
\definecolor{currentfill}{rgb}{0.000000,0.000000,0.000000}%
\pgfsetfillcolor{currentfill}%
\pgfsetlinewidth{0.803000pt}%
\definecolor{currentstroke}{rgb}{0.000000,0.000000,0.000000}%
\pgfsetstrokecolor{currentstroke}%
\pgfsetdash{}{0pt}%
\pgfsys@defobject{currentmarker}{\pgfqpoint{0.000000in}{-0.048611in}}{\pgfqpoint{0.000000in}{0.000000in}}{%
\pgfpathmoveto{\pgfqpoint{0.000000in}{0.000000in}}%
\pgfpathlineto{\pgfqpoint{0.000000in}{-0.048611in}}%
\pgfusepath{stroke,fill}%
}%
\begin{pgfscope}%
\pgfsys@transformshift{1.748114in}{0.474100in}%
\pgfsys@useobject{currentmarker}{}%
\end{pgfscope}%
\end{pgfscope}%
\begin{pgfscope}%
\definecolor{textcolor}{rgb}{0.000000,0.000000,0.000000}%
\pgfsetstrokecolor{textcolor}%
\pgfsetfillcolor{textcolor}%
\pgftext[x=1.748114in,y=0.376878in,,top]{\color{textcolor}\sffamily\fontsize{10.000000}{12.000000}\selectfont 0.10}%
\end{pgfscope}%
\begin{pgfscope}%
\pgfsetbuttcap%
\pgfsetroundjoin%
\definecolor{currentfill}{rgb}{0.000000,0.000000,0.000000}%
\pgfsetfillcolor{currentfill}%
\pgfsetlinewidth{0.803000pt}%
\definecolor{currentstroke}{rgb}{0.000000,0.000000,0.000000}%
\pgfsetstrokecolor{currentstroke}%
\pgfsetdash{}{0pt}%
\pgfsys@defobject{currentmarker}{\pgfqpoint{0.000000in}{-0.048611in}}{\pgfqpoint{0.000000in}{0.000000in}}{%
\pgfpathmoveto{\pgfqpoint{0.000000in}{0.000000in}}%
\pgfpathlineto{\pgfqpoint{0.000000in}{-0.048611in}}%
\pgfusepath{stroke,fill}%
}%
\begin{pgfscope}%
\pgfsys@transformshift{2.369955in}{0.474100in}%
\pgfsys@useobject{currentmarker}{}%
\end{pgfscope}%
\end{pgfscope}%
\begin{pgfscope}%
\definecolor{textcolor}{rgb}{0.000000,0.000000,0.000000}%
\pgfsetstrokecolor{textcolor}%
\pgfsetfillcolor{textcolor}%
\pgftext[x=2.369955in,y=0.376878in,,top]{\color{textcolor}\sffamily\fontsize{10.000000}{12.000000}\selectfont 0.15}%
\end{pgfscope}%
\begin{pgfscope}%
\pgfsetbuttcap%
\pgfsetroundjoin%
\definecolor{currentfill}{rgb}{0.000000,0.000000,0.000000}%
\pgfsetfillcolor{currentfill}%
\pgfsetlinewidth{0.803000pt}%
\definecolor{currentstroke}{rgb}{0.000000,0.000000,0.000000}%
\pgfsetstrokecolor{currentstroke}%
\pgfsetdash{}{0pt}%
\pgfsys@defobject{currentmarker}{\pgfqpoint{0.000000in}{-0.048611in}}{\pgfqpoint{0.000000in}{0.000000in}}{%
\pgfpathmoveto{\pgfqpoint{0.000000in}{0.000000in}}%
\pgfpathlineto{\pgfqpoint{0.000000in}{-0.048611in}}%
\pgfusepath{stroke,fill}%
}%
\begin{pgfscope}%
\pgfsys@transformshift{2.991796in}{0.474100in}%
\pgfsys@useobject{currentmarker}{}%
\end{pgfscope}%
\end{pgfscope}%
\begin{pgfscope}%
\definecolor{textcolor}{rgb}{0.000000,0.000000,0.000000}%
\pgfsetstrokecolor{textcolor}%
\pgfsetfillcolor{textcolor}%
\pgftext[x=2.991796in,y=0.376878in,,top]{\color{textcolor}\sffamily\fontsize{10.000000}{12.000000}\selectfont 0.20}%
\end{pgfscope}%
\begin{pgfscope}%
\pgfsetbuttcap%
\pgfsetroundjoin%
\definecolor{currentfill}{rgb}{0.000000,0.000000,0.000000}%
\pgfsetfillcolor{currentfill}%
\pgfsetlinewidth{0.803000pt}%
\definecolor{currentstroke}{rgb}{0.000000,0.000000,0.000000}%
\pgfsetstrokecolor{currentstroke}%
\pgfsetdash{}{0pt}%
\pgfsys@defobject{currentmarker}{\pgfqpoint{0.000000in}{-0.048611in}}{\pgfqpoint{0.000000in}{0.000000in}}{%
\pgfpathmoveto{\pgfqpoint{0.000000in}{0.000000in}}%
\pgfpathlineto{\pgfqpoint{0.000000in}{-0.048611in}}%
\pgfusepath{stroke,fill}%
}%
\begin{pgfscope}%
\pgfsys@transformshift{3.613636in}{0.474100in}%
\pgfsys@useobject{currentmarker}{}%
\end{pgfscope}%
\end{pgfscope}%
\begin{pgfscope}%
\definecolor{textcolor}{rgb}{0.000000,0.000000,0.000000}%
\pgfsetstrokecolor{textcolor}%
\pgfsetfillcolor{textcolor}%
\pgftext[x=3.613636in,y=0.376878in,,top]{\color{textcolor}\sffamily\fontsize{10.000000}{12.000000}\selectfont 0.25}%
\end{pgfscope}%
\begin{pgfscope}%
\pgfsetbuttcap%
\pgfsetroundjoin%
\definecolor{currentfill}{rgb}{0.000000,0.000000,0.000000}%
\pgfsetfillcolor{currentfill}%
\pgfsetlinewidth{0.803000pt}%
\definecolor{currentstroke}{rgb}{0.000000,0.000000,0.000000}%
\pgfsetstrokecolor{currentstroke}%
\pgfsetdash{}{0pt}%
\pgfsys@defobject{currentmarker}{\pgfqpoint{0.000000in}{-0.048611in}}{\pgfqpoint{0.000000in}{0.000000in}}{%
\pgfpathmoveto{\pgfqpoint{0.000000in}{0.000000in}}%
\pgfpathlineto{\pgfqpoint{0.000000in}{-0.048611in}}%
\pgfusepath{stroke,fill}%
}%
\begin{pgfscope}%
\pgfsys@transformshift{4.235477in}{0.474100in}%
\pgfsys@useobject{currentmarker}{}%
\end{pgfscope}%
\end{pgfscope}%
\begin{pgfscope}%
\definecolor{textcolor}{rgb}{0.000000,0.000000,0.000000}%
\pgfsetstrokecolor{textcolor}%
\pgfsetfillcolor{textcolor}%
\pgftext[x=4.235477in,y=0.376878in,,top]{\color{textcolor}\sffamily\fontsize{10.000000}{12.000000}\selectfont 0.30}%
\end{pgfscope}%
\begin{pgfscope}%
\pgfsetbuttcap%
\pgfsetroundjoin%
\definecolor{currentfill}{rgb}{0.000000,0.000000,0.000000}%
\pgfsetfillcolor{currentfill}%
\pgfsetlinewidth{0.803000pt}%
\definecolor{currentstroke}{rgb}{0.000000,0.000000,0.000000}%
\pgfsetstrokecolor{currentstroke}%
\pgfsetdash{}{0pt}%
\pgfsys@defobject{currentmarker}{\pgfqpoint{0.000000in}{-0.048611in}}{\pgfqpoint{0.000000in}{0.000000in}}{%
\pgfpathmoveto{\pgfqpoint{0.000000in}{0.000000in}}%
\pgfpathlineto{\pgfqpoint{0.000000in}{-0.048611in}}%
\pgfusepath{stroke,fill}%
}%
\begin{pgfscope}%
\pgfsys@transformshift{4.857318in}{0.474100in}%
\pgfsys@useobject{currentmarker}{}%
\end{pgfscope}%
\end{pgfscope}%
\begin{pgfscope}%
\definecolor{textcolor}{rgb}{0.000000,0.000000,0.000000}%
\pgfsetstrokecolor{textcolor}%
\pgfsetfillcolor{textcolor}%
\pgftext[x=4.857318in,y=0.376878in,,top]{\color{textcolor}\sffamily\fontsize{10.000000}{12.000000}\selectfont 0.35}%
\end{pgfscope}%
\begin{pgfscope}%
\pgfsetbuttcap%
\pgfsetroundjoin%
\definecolor{currentfill}{rgb}{0.000000,0.000000,0.000000}%
\pgfsetfillcolor{currentfill}%
\pgfsetlinewidth{0.803000pt}%
\definecolor{currentstroke}{rgb}{0.000000,0.000000,0.000000}%
\pgfsetstrokecolor{currentstroke}%
\pgfsetdash{}{0pt}%
\pgfsys@defobject{currentmarker}{\pgfqpoint{0.000000in}{-0.048611in}}{\pgfqpoint{0.000000in}{0.000000in}}{%
\pgfpathmoveto{\pgfqpoint{0.000000in}{0.000000in}}%
\pgfpathlineto{\pgfqpoint{0.000000in}{-0.048611in}}%
\pgfusepath{stroke,fill}%
}%
\begin{pgfscope}%
\pgfsys@transformshift{5.479158in}{0.474100in}%
\pgfsys@useobject{currentmarker}{}%
\end{pgfscope}%
\end{pgfscope}%
\begin{pgfscope}%
\definecolor{textcolor}{rgb}{0.000000,0.000000,0.000000}%
\pgfsetstrokecolor{textcolor}%
\pgfsetfillcolor{textcolor}%
\pgftext[x=5.479158in,y=0.376878in,,top]{\color{textcolor}\sffamily\fontsize{10.000000}{12.000000}\selectfont 0.40}%
\end{pgfscope}%
\begin{pgfscope}%
\definecolor{textcolor}{rgb}{0.000000,0.000000,0.000000}%
\pgfsetstrokecolor{textcolor}%
\pgfsetfillcolor{textcolor}%
\pgftext[x=3.613636in,y=0.186909in,,top]{\color{textcolor}\sffamily\fontsize{10.000000}{12.000000}\selectfont Umbral de distancia ε}%
\end{pgfscope}%
\begin{pgfscope}%
\pgfsetbuttcap%
\pgfsetroundjoin%
\definecolor{currentfill}{rgb}{0.000000,0.000000,0.000000}%
\pgfsetfillcolor{currentfill}%
\pgfsetlinewidth{0.803000pt}%
\definecolor{currentstroke}{rgb}{0.000000,0.000000,0.000000}%
\pgfsetstrokecolor{currentstroke}%
\pgfsetdash{}{0pt}%
\pgfsys@defobject{currentmarker}{\pgfqpoint{-0.048611in}{0.000000in}}{\pgfqpoint{-0.000000in}{0.000000in}}{%
\pgfpathmoveto{\pgfqpoint{-0.000000in}{0.000000in}}%
\pgfpathlineto{\pgfqpoint{-0.048611in}{0.000000in}}%
\pgfusepath{stroke,fill}%
}%
\begin{pgfscope}%
\pgfsys@transformshift{1.500000in}{0.683613in}%
\pgfsys@useobject{currentmarker}{}%
\end{pgfscope}%
\end{pgfscope}%
\begin{pgfscope}%
\definecolor{textcolor}{rgb}{0.000000,0.000000,0.000000}%
\pgfsetstrokecolor{textcolor}%
\pgfsetfillcolor{textcolor}%
\pgftext[x=1.073873in, y=0.630851in, left, base]{\color{textcolor}\sffamily\fontsize{10.000000}{12.000000}\selectfont \ensuremath{-}0.3}%
\end{pgfscope}%
\begin{pgfscope}%
\pgfsetbuttcap%
\pgfsetroundjoin%
\definecolor{currentfill}{rgb}{0.000000,0.000000,0.000000}%
\pgfsetfillcolor{currentfill}%
\pgfsetlinewidth{0.803000pt}%
\definecolor{currentstroke}{rgb}{0.000000,0.000000,0.000000}%
\pgfsetstrokecolor{currentstroke}%
\pgfsetdash{}{0pt}%
\pgfsys@defobject{currentmarker}{\pgfqpoint{-0.048611in}{0.000000in}}{\pgfqpoint{-0.000000in}{0.000000in}}{%
\pgfpathmoveto{\pgfqpoint{-0.000000in}{0.000000in}}%
\pgfpathlineto{\pgfqpoint{-0.048611in}{0.000000in}}%
\pgfusepath{stroke,fill}%
}%
\begin{pgfscope}%
\pgfsys@transformshift{1.500000in}{1.069336in}%
\pgfsys@useobject{currentmarker}{}%
\end{pgfscope}%
\end{pgfscope}%
\begin{pgfscope}%
\definecolor{textcolor}{rgb}{0.000000,0.000000,0.000000}%
\pgfsetstrokecolor{textcolor}%
\pgfsetfillcolor{textcolor}%
\pgftext[x=1.073873in, y=1.016575in, left, base]{\color{textcolor}\sffamily\fontsize{10.000000}{12.000000}\selectfont \ensuremath{-}0.2}%
\end{pgfscope}%
\begin{pgfscope}%
\pgfsetbuttcap%
\pgfsetroundjoin%
\definecolor{currentfill}{rgb}{0.000000,0.000000,0.000000}%
\pgfsetfillcolor{currentfill}%
\pgfsetlinewidth{0.803000pt}%
\definecolor{currentstroke}{rgb}{0.000000,0.000000,0.000000}%
\pgfsetstrokecolor{currentstroke}%
\pgfsetdash{}{0pt}%
\pgfsys@defobject{currentmarker}{\pgfqpoint{-0.048611in}{0.000000in}}{\pgfqpoint{-0.000000in}{0.000000in}}{%
\pgfpathmoveto{\pgfqpoint{-0.000000in}{0.000000in}}%
\pgfpathlineto{\pgfqpoint{-0.048611in}{0.000000in}}%
\pgfusepath{stroke,fill}%
}%
\begin{pgfscope}%
\pgfsys@transformshift{1.500000in}{1.455059in}%
\pgfsys@useobject{currentmarker}{}%
\end{pgfscope}%
\end{pgfscope}%
\begin{pgfscope}%
\definecolor{textcolor}{rgb}{0.000000,0.000000,0.000000}%
\pgfsetstrokecolor{textcolor}%
\pgfsetfillcolor{textcolor}%
\pgftext[x=1.073873in, y=1.402298in, left, base]{\color{textcolor}\sffamily\fontsize{10.000000}{12.000000}\selectfont \ensuremath{-}0.1}%
\end{pgfscope}%
\begin{pgfscope}%
\pgfsetbuttcap%
\pgfsetroundjoin%
\definecolor{currentfill}{rgb}{0.000000,0.000000,0.000000}%
\pgfsetfillcolor{currentfill}%
\pgfsetlinewidth{0.803000pt}%
\definecolor{currentstroke}{rgb}{0.000000,0.000000,0.000000}%
\pgfsetstrokecolor{currentstroke}%
\pgfsetdash{}{0pt}%
\pgfsys@defobject{currentmarker}{\pgfqpoint{-0.048611in}{0.000000in}}{\pgfqpoint{-0.000000in}{0.000000in}}{%
\pgfpathmoveto{\pgfqpoint{-0.000000in}{0.000000in}}%
\pgfpathlineto{\pgfqpoint{-0.048611in}{0.000000in}}%
\pgfusepath{stroke,fill}%
}%
\begin{pgfscope}%
\pgfsys@transformshift{1.500000in}{1.840783in}%
\pgfsys@useobject{currentmarker}{}%
\end{pgfscope}%
\end{pgfscope}%
\begin{pgfscope}%
\definecolor{textcolor}{rgb}{0.000000,0.000000,0.000000}%
\pgfsetstrokecolor{textcolor}%
\pgfsetfillcolor{textcolor}%
\pgftext[x=1.181898in, y=1.788021in, left, base]{\color{textcolor}\sffamily\fontsize{10.000000}{12.000000}\selectfont 0.0}%
\end{pgfscope}%
\begin{pgfscope}%
\pgfsetbuttcap%
\pgfsetroundjoin%
\definecolor{currentfill}{rgb}{0.000000,0.000000,0.000000}%
\pgfsetfillcolor{currentfill}%
\pgfsetlinewidth{0.803000pt}%
\definecolor{currentstroke}{rgb}{0.000000,0.000000,0.000000}%
\pgfsetstrokecolor{currentstroke}%
\pgfsetdash{}{0pt}%
\pgfsys@defobject{currentmarker}{\pgfqpoint{-0.048611in}{0.000000in}}{\pgfqpoint{-0.000000in}{0.000000in}}{%
\pgfpathmoveto{\pgfqpoint{-0.000000in}{0.000000in}}%
\pgfpathlineto{\pgfqpoint{-0.048611in}{0.000000in}}%
\pgfusepath{stroke,fill}%
}%
\begin{pgfscope}%
\pgfsys@transformshift{1.500000in}{2.226506in}%
\pgfsys@useobject{currentmarker}{}%
\end{pgfscope}%
\end{pgfscope}%
\begin{pgfscope}%
\definecolor{textcolor}{rgb}{0.000000,0.000000,0.000000}%
\pgfsetstrokecolor{textcolor}%
\pgfsetfillcolor{textcolor}%
\pgftext[x=1.181898in, y=2.173744in, left, base]{\color{textcolor}\sffamily\fontsize{10.000000}{12.000000}\selectfont 0.1}%
\end{pgfscope}%
\begin{pgfscope}%
\pgfsetbuttcap%
\pgfsetroundjoin%
\definecolor{currentfill}{rgb}{0.000000,0.000000,0.000000}%
\pgfsetfillcolor{currentfill}%
\pgfsetlinewidth{0.803000pt}%
\definecolor{currentstroke}{rgb}{0.000000,0.000000,0.000000}%
\pgfsetstrokecolor{currentstroke}%
\pgfsetdash{}{0pt}%
\pgfsys@defobject{currentmarker}{\pgfqpoint{-0.048611in}{0.000000in}}{\pgfqpoint{-0.000000in}{0.000000in}}{%
\pgfpathmoveto{\pgfqpoint{-0.000000in}{0.000000in}}%
\pgfpathlineto{\pgfqpoint{-0.048611in}{0.000000in}}%
\pgfusepath{stroke,fill}%
}%
\begin{pgfscope}%
\pgfsys@transformshift{1.500000in}{2.612229in}%
\pgfsys@useobject{currentmarker}{}%
\end{pgfscope}%
\end{pgfscope}%
\begin{pgfscope}%
\definecolor{textcolor}{rgb}{0.000000,0.000000,0.000000}%
\pgfsetstrokecolor{textcolor}%
\pgfsetfillcolor{textcolor}%
\pgftext[x=1.181898in, y=2.559468in, left, base]{\color{textcolor}\sffamily\fontsize{10.000000}{12.000000}\selectfont 0.2}%
\end{pgfscope}%
\begin{pgfscope}%
\pgfsetbuttcap%
\pgfsetroundjoin%
\definecolor{currentfill}{rgb}{0.000000,0.000000,0.000000}%
\pgfsetfillcolor{currentfill}%
\pgfsetlinewidth{0.803000pt}%
\definecolor{currentstroke}{rgb}{0.000000,0.000000,0.000000}%
\pgfsetstrokecolor{currentstroke}%
\pgfsetdash{}{0pt}%
\pgfsys@defobject{currentmarker}{\pgfqpoint{-0.048611in}{0.000000in}}{\pgfqpoint{-0.000000in}{0.000000in}}{%
\pgfpathmoveto{\pgfqpoint{-0.000000in}{0.000000in}}%
\pgfpathlineto{\pgfqpoint{-0.048611in}{0.000000in}}%
\pgfusepath{stroke,fill}%
}%
\begin{pgfscope}%
\pgfsys@transformshift{1.500000in}{2.997952in}%
\pgfsys@useobject{currentmarker}{}%
\end{pgfscope}%
\end{pgfscope}%
\begin{pgfscope}%
\definecolor{textcolor}{rgb}{0.000000,0.000000,0.000000}%
\pgfsetstrokecolor{textcolor}%
\pgfsetfillcolor{textcolor}%
\pgftext[x=1.181898in, y=2.945191in, left, base]{\color{textcolor}\sffamily\fontsize{10.000000}{12.000000}\selectfont 0.3}%
\end{pgfscope}%
\begin{pgfscope}%
\pgfsetbuttcap%
\pgfsetroundjoin%
\definecolor{currentfill}{rgb}{0.000000,0.000000,0.000000}%
\pgfsetfillcolor{currentfill}%
\pgfsetlinewidth{0.803000pt}%
\definecolor{currentstroke}{rgb}{0.000000,0.000000,0.000000}%
\pgfsetstrokecolor{currentstroke}%
\pgfsetdash{}{0pt}%
\pgfsys@defobject{currentmarker}{\pgfqpoint{-0.048611in}{0.000000in}}{\pgfqpoint{-0.000000in}{0.000000in}}{%
\pgfpathmoveto{\pgfqpoint{-0.000000in}{0.000000in}}%
\pgfpathlineto{\pgfqpoint{-0.048611in}{0.000000in}}%
\pgfusepath{stroke,fill}%
}%
\begin{pgfscope}%
\pgfsys@transformshift{1.500000in}{3.383675in}%
\pgfsys@useobject{currentmarker}{}%
\end{pgfscope}%
\end{pgfscope}%
\begin{pgfscope}%
\definecolor{textcolor}{rgb}{0.000000,0.000000,0.000000}%
\pgfsetstrokecolor{textcolor}%
\pgfsetfillcolor{textcolor}%
\pgftext[x=1.181898in, y=3.330914in, left, base]{\color{textcolor}\sffamily\fontsize{10.000000}{12.000000}\selectfont 0.4}%
\end{pgfscope}%
\begin{pgfscope}%
\pgfsetbuttcap%
\pgfsetroundjoin%
\definecolor{currentfill}{rgb}{0.000000,0.000000,0.000000}%
\pgfsetfillcolor{currentfill}%
\pgfsetlinewidth{0.803000pt}%
\definecolor{currentstroke}{rgb}{0.000000,0.000000,0.000000}%
\pgfsetstrokecolor{currentstroke}%
\pgfsetdash{}{0pt}%
\pgfsys@defobject{currentmarker}{\pgfqpoint{-0.048611in}{0.000000in}}{\pgfqpoint{-0.000000in}{0.000000in}}{%
\pgfpathmoveto{\pgfqpoint{-0.000000in}{0.000000in}}%
\pgfpathlineto{\pgfqpoint{-0.048611in}{0.000000in}}%
\pgfusepath{stroke,fill}%
}%
\begin{pgfscope}%
\pgfsys@transformshift{1.500000in}{3.769399in}%
\pgfsys@useobject{currentmarker}{}%
\end{pgfscope}%
\end{pgfscope}%
\begin{pgfscope}%
\definecolor{textcolor}{rgb}{0.000000,0.000000,0.000000}%
\pgfsetstrokecolor{textcolor}%
\pgfsetfillcolor{textcolor}%
\pgftext[x=1.181898in, y=3.716637in, left, base]{\color{textcolor}\sffamily\fontsize{10.000000}{12.000000}\selectfont 0.5}%
\end{pgfscope}%
\begin{pgfscope}%
\definecolor{textcolor}{rgb}{0.000000,0.000000,0.000000}%
\pgfsetstrokecolor{textcolor}%
\pgfsetfillcolor{textcolor}%
\pgftext[x=1.018318in,y=2.133450in,,bottom,rotate=90.000000]{\color{textcolor}\sffamily\fontsize{10.000000}{12.000000}\selectfont Valor medio del coeficiente de Silhouette}%
\end{pgfscope}%
\begin{pgfscope}%
\pgfsetrectcap%
\pgfsetmiterjoin%
\pgfsetlinewidth{0.803000pt}%
\definecolor{currentstroke}{rgb}{0.000000,0.000000,0.000000}%
\pgfsetstrokecolor{currentstroke}%
\pgfsetdash{}{0pt}%
\pgfpathmoveto{\pgfqpoint{1.500000in}{0.474100in}}%
\pgfpathlineto{\pgfqpoint{1.500000in}{3.792800in}}%
\pgfusepath{stroke}%
\end{pgfscope}%
\begin{pgfscope}%
\pgfsetrectcap%
\pgfsetmiterjoin%
\pgfsetlinewidth{0.803000pt}%
\definecolor{currentstroke}{rgb}{0.000000,0.000000,0.000000}%
\pgfsetstrokecolor{currentstroke}%
\pgfsetdash{}{0pt}%
\pgfpathmoveto{\pgfqpoint{5.727273in}{0.474100in}}%
\pgfpathlineto{\pgfqpoint{5.727273in}{3.792800in}}%
\pgfusepath{stroke}%
\end{pgfscope}%
\begin{pgfscope}%
\pgfsetrectcap%
\pgfsetmiterjoin%
\pgfsetlinewidth{0.803000pt}%
\definecolor{currentstroke}{rgb}{0.000000,0.000000,0.000000}%
\pgfsetstrokecolor{currentstroke}%
\pgfsetdash{}{0pt}%
\pgfpathmoveto{\pgfqpoint{1.500000in}{0.474100in}}%
\pgfpathlineto{\pgfqpoint{5.727273in}{0.474100in}}%
\pgfusepath{stroke}%
\end{pgfscope}%
\begin{pgfscope}%
\pgfsetrectcap%
\pgfsetmiterjoin%
\pgfsetlinewidth{0.803000pt}%
\definecolor{currentstroke}{rgb}{0.000000,0.000000,0.000000}%
\pgfsetstrokecolor{currentstroke}%
\pgfsetdash{}{0pt}%
\pgfpathmoveto{\pgfqpoint{1.500000in}{3.792800in}}%
\pgfpathlineto{\pgfqpoint{5.727273in}{3.792800in}}%
\pgfusepath{stroke}%
\end{pgfscope}%
\begin{pgfscope}%
\pgfsetbuttcap%
\pgfsetmiterjoin%
\definecolor{currentfill}{rgb}{1.000000,1.000000,1.000000}%
\pgfsetfillcolor{currentfill}%
\pgfsetlinewidth{0.000000pt}%
\definecolor{currentstroke}{rgb}{0.000000,0.000000,0.000000}%
\pgfsetstrokecolor{currentstroke}%
\pgfsetstrokeopacity{0.000000}%
\pgfsetdash{}{0pt}%
\pgfpathmoveto{\pgfqpoint{6.572727in}{0.474100in}}%
\pgfpathlineto{\pgfqpoint{10.800000in}{0.474100in}}%
\pgfpathlineto{\pgfqpoint{10.800000in}{3.792800in}}%
\pgfpathlineto{\pgfqpoint{6.572727in}{3.792800in}}%
\pgfpathclose%
\pgfusepath{fill}%
\end{pgfscope}%
\begin{pgfscope}%
\pgfpathrectangle{\pgfqpoint{6.572727in}{0.474100in}}{\pgfqpoint{4.227273in}{3.318700in}}%
\pgfusepath{clip}%
\pgfsetbuttcap%
\pgfsetroundjoin%
\definecolor{currentfill}{rgb}{0.127568,0.566949,0.550556}%
\pgfsetfillcolor{currentfill}%
\pgfsetfillopacity{0.700000}%
\pgfsetlinewidth{0.000000pt}%
\definecolor{currentstroke}{rgb}{0.000000,0.000000,0.000000}%
\pgfsetstrokecolor{currentstroke}%
\pgfsetstrokeopacity{0.700000}%
\pgfsetdash{}{0pt}%
\pgfpathmoveto{\pgfqpoint{7.859498in}{2.337775in}}%
\pgfpathcurveto{\pgfqpoint{7.864542in}{2.337775in}}{\pgfqpoint{7.869379in}{2.339779in}}{\pgfqpoint{7.872946in}{2.343345in}}%
\pgfpathcurveto{\pgfqpoint{7.876512in}{2.346912in}}{\pgfqpoint{7.878516in}{2.351749in}}{\pgfqpoint{7.878516in}{2.356793in}}%
\pgfpathcurveto{\pgfqpoint{7.878516in}{2.361837in}}{\pgfqpoint{7.876512in}{2.366675in}}{\pgfqpoint{7.872946in}{2.370241in}}%
\pgfpathcurveto{\pgfqpoint{7.869379in}{2.373807in}}{\pgfqpoint{7.864542in}{2.375811in}}{\pgfqpoint{7.859498in}{2.375811in}}%
\pgfpathcurveto{\pgfqpoint{7.854454in}{2.375811in}}{\pgfqpoint{7.849616in}{2.373807in}}{\pgfqpoint{7.846050in}{2.370241in}}%
\pgfpathcurveto{\pgfqpoint{7.842484in}{2.366675in}}{\pgfqpoint{7.840480in}{2.361837in}}{\pgfqpoint{7.840480in}{2.356793in}}%
\pgfpathcurveto{\pgfqpoint{7.840480in}{2.351749in}}{\pgfqpoint{7.842484in}{2.346912in}}{\pgfqpoint{7.846050in}{2.343345in}}%
\pgfpathcurveto{\pgfqpoint{7.849616in}{2.339779in}}{\pgfqpoint{7.854454in}{2.337775in}}{\pgfqpoint{7.859498in}{2.337775in}}%
\pgfpathclose%
\pgfusepath{fill}%
\end{pgfscope}%
\begin{pgfscope}%
\pgfpathrectangle{\pgfqpoint{6.572727in}{0.474100in}}{\pgfqpoint{4.227273in}{3.318700in}}%
\pgfusepath{clip}%
\pgfsetbuttcap%
\pgfsetroundjoin%
\definecolor{currentfill}{rgb}{0.127568,0.566949,0.550556}%
\pgfsetfillcolor{currentfill}%
\pgfsetfillopacity{0.700000}%
\pgfsetlinewidth{0.000000pt}%
\definecolor{currentstroke}{rgb}{0.000000,0.000000,0.000000}%
\pgfsetstrokecolor{currentstroke}%
\pgfsetstrokeopacity{0.700000}%
\pgfsetdash{}{0pt}%
\pgfpathmoveto{\pgfqpoint{8.893046in}{2.948141in}}%
\pgfpathcurveto{\pgfqpoint{8.898090in}{2.948141in}}{\pgfqpoint{8.902927in}{2.950145in}}{\pgfqpoint{8.906494in}{2.953711in}}%
\pgfpathcurveto{\pgfqpoint{8.910060in}{2.957278in}}{\pgfqpoint{8.912064in}{2.962116in}}{\pgfqpoint{8.912064in}{2.967159in}}%
\pgfpathcurveto{\pgfqpoint{8.912064in}{2.972203in}}{\pgfqpoint{8.910060in}{2.977041in}}{\pgfqpoint{8.906494in}{2.980607in}}%
\pgfpathcurveto{\pgfqpoint{8.902927in}{2.984173in}}{\pgfqpoint{8.898090in}{2.986177in}}{\pgfqpoint{8.893046in}{2.986177in}}%
\pgfpathcurveto{\pgfqpoint{8.888002in}{2.986177in}}{\pgfqpoint{8.883164in}{2.984173in}}{\pgfqpoint{8.879598in}{2.980607in}}%
\pgfpathcurveto{\pgfqpoint{8.876032in}{2.977041in}}{\pgfqpoint{8.874028in}{2.972203in}}{\pgfqpoint{8.874028in}{2.967159in}}%
\pgfpathcurveto{\pgfqpoint{8.874028in}{2.962116in}}{\pgfqpoint{8.876032in}{2.957278in}}{\pgfqpoint{8.879598in}{2.953711in}}%
\pgfpathcurveto{\pgfqpoint{8.883164in}{2.950145in}}{\pgfqpoint{8.888002in}{2.948141in}}{\pgfqpoint{8.893046in}{2.948141in}}%
\pgfpathclose%
\pgfusepath{fill}%
\end{pgfscope}%
\begin{pgfscope}%
\pgfpathrectangle{\pgfqpoint{6.572727in}{0.474100in}}{\pgfqpoint{4.227273in}{3.318700in}}%
\pgfusepath{clip}%
\pgfsetbuttcap%
\pgfsetroundjoin%
\definecolor{currentfill}{rgb}{0.993248,0.906157,0.143936}%
\pgfsetfillcolor{currentfill}%
\pgfsetfillopacity{0.700000}%
\pgfsetlinewidth{0.000000pt}%
\definecolor{currentstroke}{rgb}{0.000000,0.000000,0.000000}%
\pgfsetstrokecolor{currentstroke}%
\pgfsetstrokeopacity{0.700000}%
\pgfsetdash{}{0pt}%
\pgfpathmoveto{\pgfqpoint{9.440632in}{1.964413in}}%
\pgfpathcurveto{\pgfqpoint{9.445676in}{1.964413in}}{\pgfqpoint{9.450513in}{1.966417in}}{\pgfqpoint{9.454080in}{1.969983in}}%
\pgfpathcurveto{\pgfqpoint{9.457646in}{1.973550in}}{\pgfqpoint{9.459650in}{1.978387in}}{\pgfqpoint{9.459650in}{1.983431in}}%
\pgfpathcurveto{\pgfqpoint{9.459650in}{1.988475in}}{\pgfqpoint{9.457646in}{1.993313in}}{\pgfqpoint{9.454080in}{1.996879in}}%
\pgfpathcurveto{\pgfqpoint{9.450513in}{2.000445in}}{\pgfqpoint{9.445676in}{2.002449in}}{\pgfqpoint{9.440632in}{2.002449in}}%
\pgfpathcurveto{\pgfqpoint{9.435588in}{2.002449in}}{\pgfqpoint{9.430751in}{2.000445in}}{\pgfqpoint{9.427184in}{1.996879in}}%
\pgfpathcurveto{\pgfqpoint{9.423618in}{1.993313in}}{\pgfqpoint{9.421614in}{1.988475in}}{\pgfqpoint{9.421614in}{1.983431in}}%
\pgfpathcurveto{\pgfqpoint{9.421614in}{1.978387in}}{\pgfqpoint{9.423618in}{1.973550in}}{\pgfqpoint{9.427184in}{1.969983in}}%
\pgfpathcurveto{\pgfqpoint{9.430751in}{1.966417in}}{\pgfqpoint{9.435588in}{1.964413in}}{\pgfqpoint{9.440632in}{1.964413in}}%
\pgfpathclose%
\pgfusepath{fill}%
\end{pgfscope}%
\begin{pgfscope}%
\pgfpathrectangle{\pgfqpoint{6.572727in}{0.474100in}}{\pgfqpoint{4.227273in}{3.318700in}}%
\pgfusepath{clip}%
\pgfsetbuttcap%
\pgfsetroundjoin%
\definecolor{currentfill}{rgb}{0.267004,0.004874,0.329415}%
\pgfsetfillcolor{currentfill}%
\pgfsetfillopacity{0.700000}%
\pgfsetlinewidth{0.000000pt}%
\definecolor{currentstroke}{rgb}{0.000000,0.000000,0.000000}%
\pgfsetstrokecolor{currentstroke}%
\pgfsetstrokeopacity{0.700000}%
\pgfsetdash{}{0pt}%
\pgfpathmoveto{\pgfqpoint{8.563329in}{1.016272in}}%
\pgfpathcurveto{\pgfqpoint{8.568373in}{1.016272in}}{\pgfqpoint{8.573211in}{1.018276in}}{\pgfqpoint{8.576777in}{1.021842in}}%
\pgfpathcurveto{\pgfqpoint{8.580344in}{1.025408in}}{\pgfqpoint{8.582347in}{1.030246in}}{\pgfqpoint{8.582347in}{1.035290in}}%
\pgfpathcurveto{\pgfqpoint{8.582347in}{1.040333in}}{\pgfqpoint{8.580344in}{1.045171in}}{\pgfqpoint{8.576777in}{1.048738in}}%
\pgfpathcurveto{\pgfqpoint{8.573211in}{1.052304in}}{\pgfqpoint{8.568373in}{1.054308in}}{\pgfqpoint{8.563329in}{1.054308in}}%
\pgfpathcurveto{\pgfqpoint{8.558286in}{1.054308in}}{\pgfqpoint{8.553448in}{1.052304in}}{\pgfqpoint{8.549881in}{1.048738in}}%
\pgfpathcurveto{\pgfqpoint{8.546315in}{1.045171in}}{\pgfqpoint{8.544311in}{1.040333in}}{\pgfqpoint{8.544311in}{1.035290in}}%
\pgfpathcurveto{\pgfqpoint{8.544311in}{1.030246in}}{\pgfqpoint{8.546315in}{1.025408in}}{\pgfqpoint{8.549881in}{1.021842in}}%
\pgfpathcurveto{\pgfqpoint{8.553448in}{1.018276in}}{\pgfqpoint{8.558286in}{1.016272in}}{\pgfqpoint{8.563329in}{1.016272in}}%
\pgfpathclose%
\pgfusepath{fill}%
\end{pgfscope}%
\begin{pgfscope}%
\pgfpathrectangle{\pgfqpoint{6.572727in}{0.474100in}}{\pgfqpoint{4.227273in}{3.318700in}}%
\pgfusepath{clip}%
\pgfsetbuttcap%
\pgfsetroundjoin%
\definecolor{currentfill}{rgb}{0.993248,0.906157,0.143936}%
\pgfsetfillcolor{currentfill}%
\pgfsetfillopacity{0.700000}%
\pgfsetlinewidth{0.000000pt}%
\definecolor{currentstroke}{rgb}{0.000000,0.000000,0.000000}%
\pgfsetstrokecolor{currentstroke}%
\pgfsetstrokeopacity{0.700000}%
\pgfsetdash{}{0pt}%
\pgfpathmoveto{\pgfqpoint{9.354248in}{1.005935in}}%
\pgfpathcurveto{\pgfqpoint{9.359292in}{1.005935in}}{\pgfqpoint{9.364129in}{1.007939in}}{\pgfqpoint{9.367696in}{1.011505in}}%
\pgfpathcurveto{\pgfqpoint{9.371262in}{1.015072in}}{\pgfqpoint{9.373266in}{1.019910in}}{\pgfqpoint{9.373266in}{1.024953in}}%
\pgfpathcurveto{\pgfqpoint{9.373266in}{1.029997in}}{\pgfqpoint{9.371262in}{1.034835in}}{\pgfqpoint{9.367696in}{1.038401in}}%
\pgfpathcurveto{\pgfqpoint{9.364129in}{1.041967in}}{\pgfqpoint{9.359292in}{1.043971in}}{\pgfqpoint{9.354248in}{1.043971in}}%
\pgfpathcurveto{\pgfqpoint{9.349204in}{1.043971in}}{\pgfqpoint{9.344366in}{1.041967in}}{\pgfqpoint{9.340800in}{1.038401in}}%
\pgfpathcurveto{\pgfqpoint{9.337234in}{1.034835in}}{\pgfqpoint{9.335230in}{1.029997in}}{\pgfqpoint{9.335230in}{1.024953in}}%
\pgfpathcurveto{\pgfqpoint{9.335230in}{1.019910in}}{\pgfqpoint{9.337234in}{1.015072in}}{\pgfqpoint{9.340800in}{1.011505in}}%
\pgfpathcurveto{\pgfqpoint{9.344366in}{1.007939in}}{\pgfqpoint{9.349204in}{1.005935in}}{\pgfqpoint{9.354248in}{1.005935in}}%
\pgfpathclose%
\pgfusepath{fill}%
\end{pgfscope}%
\begin{pgfscope}%
\pgfpathrectangle{\pgfqpoint{6.572727in}{0.474100in}}{\pgfqpoint{4.227273in}{3.318700in}}%
\pgfusepath{clip}%
\pgfsetbuttcap%
\pgfsetroundjoin%
\definecolor{currentfill}{rgb}{0.127568,0.566949,0.550556}%
\pgfsetfillcolor{currentfill}%
\pgfsetfillopacity{0.700000}%
\pgfsetlinewidth{0.000000pt}%
\definecolor{currentstroke}{rgb}{0.000000,0.000000,0.000000}%
\pgfsetstrokecolor{currentstroke}%
\pgfsetstrokeopacity{0.700000}%
\pgfsetdash{}{0pt}%
\pgfpathmoveto{\pgfqpoint{8.124062in}{1.710980in}}%
\pgfpathcurveto{\pgfqpoint{8.129106in}{1.710980in}}{\pgfqpoint{8.133943in}{1.712984in}}{\pgfqpoint{8.137510in}{1.716550in}}%
\pgfpathcurveto{\pgfqpoint{8.141076in}{1.720117in}}{\pgfqpoint{8.143080in}{1.724955in}}{\pgfqpoint{8.143080in}{1.729998in}}%
\pgfpathcurveto{\pgfqpoint{8.143080in}{1.735042in}}{\pgfqpoint{8.141076in}{1.739880in}}{\pgfqpoint{8.137510in}{1.743446in}}%
\pgfpathcurveto{\pgfqpoint{8.133943in}{1.747013in}}{\pgfqpoint{8.129106in}{1.749016in}}{\pgfqpoint{8.124062in}{1.749016in}}%
\pgfpathcurveto{\pgfqpoint{8.119018in}{1.749016in}}{\pgfqpoint{8.114181in}{1.747013in}}{\pgfqpoint{8.110614in}{1.743446in}}%
\pgfpathcurveto{\pgfqpoint{8.107048in}{1.739880in}}{\pgfqpoint{8.105044in}{1.735042in}}{\pgfqpoint{8.105044in}{1.729998in}}%
\pgfpathcurveto{\pgfqpoint{8.105044in}{1.724955in}}{\pgfqpoint{8.107048in}{1.720117in}}{\pgfqpoint{8.110614in}{1.716550in}}%
\pgfpathcurveto{\pgfqpoint{8.114181in}{1.712984in}}{\pgfqpoint{8.119018in}{1.710980in}}{\pgfqpoint{8.124062in}{1.710980in}}%
\pgfpathclose%
\pgfusepath{fill}%
\end{pgfscope}%
\begin{pgfscope}%
\pgfpathrectangle{\pgfqpoint{6.572727in}{0.474100in}}{\pgfqpoint{4.227273in}{3.318700in}}%
\pgfusepath{clip}%
\pgfsetbuttcap%
\pgfsetroundjoin%
\definecolor{currentfill}{rgb}{0.127568,0.566949,0.550556}%
\pgfsetfillcolor{currentfill}%
\pgfsetfillopacity{0.700000}%
\pgfsetlinewidth{0.000000pt}%
\definecolor{currentstroke}{rgb}{0.000000,0.000000,0.000000}%
\pgfsetstrokecolor{currentstroke}%
\pgfsetstrokeopacity{0.700000}%
\pgfsetdash{}{0pt}%
\pgfpathmoveto{\pgfqpoint{9.075038in}{2.799785in}}%
\pgfpathcurveto{\pgfqpoint{9.080081in}{2.799785in}}{\pgfqpoint{9.084919in}{2.801788in}}{\pgfqpoint{9.088486in}{2.805355in}}%
\pgfpathcurveto{\pgfqpoint{9.092052in}{2.808921in}}{\pgfqpoint{9.094056in}{2.813759in}}{\pgfqpoint{9.094056in}{2.818803in}}%
\pgfpathcurveto{\pgfqpoint{9.094056in}{2.823846in}}{\pgfqpoint{9.092052in}{2.828684in}}{\pgfqpoint{9.088486in}{2.832251in}}%
\pgfpathcurveto{\pgfqpoint{9.084919in}{2.835817in}}{\pgfqpoint{9.080081in}{2.837821in}}{\pgfqpoint{9.075038in}{2.837821in}}%
\pgfpathcurveto{\pgfqpoint{9.069994in}{2.837821in}}{\pgfqpoint{9.065156in}{2.835817in}}{\pgfqpoint{9.061590in}{2.832251in}}%
\pgfpathcurveto{\pgfqpoint{9.058023in}{2.828684in}}{\pgfqpoint{9.056020in}{2.823846in}}{\pgfqpoint{9.056020in}{2.818803in}}%
\pgfpathcurveto{\pgfqpoint{9.056020in}{2.813759in}}{\pgfqpoint{9.058023in}{2.808921in}}{\pgfqpoint{9.061590in}{2.805355in}}%
\pgfpathcurveto{\pgfqpoint{9.065156in}{2.801788in}}{\pgfqpoint{9.069994in}{2.799785in}}{\pgfqpoint{9.075038in}{2.799785in}}%
\pgfpathclose%
\pgfusepath{fill}%
\end{pgfscope}%
\begin{pgfscope}%
\pgfpathrectangle{\pgfqpoint{6.572727in}{0.474100in}}{\pgfqpoint{4.227273in}{3.318700in}}%
\pgfusepath{clip}%
\pgfsetbuttcap%
\pgfsetroundjoin%
\definecolor{currentfill}{rgb}{0.127568,0.566949,0.550556}%
\pgfsetfillcolor{currentfill}%
\pgfsetfillopacity{0.700000}%
\pgfsetlinewidth{0.000000pt}%
\definecolor{currentstroke}{rgb}{0.000000,0.000000,0.000000}%
\pgfsetstrokecolor{currentstroke}%
\pgfsetstrokeopacity{0.700000}%
\pgfsetdash{}{0pt}%
\pgfpathmoveto{\pgfqpoint{7.421822in}{2.948591in}}%
\pgfpathcurveto{\pgfqpoint{7.426866in}{2.948591in}}{\pgfqpoint{7.431704in}{2.950595in}}{\pgfqpoint{7.435270in}{2.954161in}}%
\pgfpathcurveto{\pgfqpoint{7.438837in}{2.957728in}}{\pgfqpoint{7.440840in}{2.962565in}}{\pgfqpoint{7.440840in}{2.967609in}}%
\pgfpathcurveto{\pgfqpoint{7.440840in}{2.972653in}}{\pgfqpoint{7.438837in}{2.977491in}}{\pgfqpoint{7.435270in}{2.981057in}}%
\pgfpathcurveto{\pgfqpoint{7.431704in}{2.984623in}}{\pgfqpoint{7.426866in}{2.986627in}}{\pgfqpoint{7.421822in}{2.986627in}}%
\pgfpathcurveto{\pgfqpoint{7.416779in}{2.986627in}}{\pgfqpoint{7.411941in}{2.984623in}}{\pgfqpoint{7.408374in}{2.981057in}}%
\pgfpathcurveto{\pgfqpoint{7.404808in}{2.977491in}}{\pgfqpoint{7.402804in}{2.972653in}}{\pgfqpoint{7.402804in}{2.967609in}}%
\pgfpathcurveto{\pgfqpoint{7.402804in}{2.962565in}}{\pgfqpoint{7.404808in}{2.957728in}}{\pgfqpoint{7.408374in}{2.954161in}}%
\pgfpathcurveto{\pgfqpoint{7.411941in}{2.950595in}}{\pgfqpoint{7.416779in}{2.948591in}}{\pgfqpoint{7.421822in}{2.948591in}}%
\pgfpathclose%
\pgfusepath{fill}%
\end{pgfscope}%
\begin{pgfscope}%
\pgfpathrectangle{\pgfqpoint{6.572727in}{0.474100in}}{\pgfqpoint{4.227273in}{3.318700in}}%
\pgfusepath{clip}%
\pgfsetbuttcap%
\pgfsetroundjoin%
\definecolor{currentfill}{rgb}{0.127568,0.566949,0.550556}%
\pgfsetfillcolor{currentfill}%
\pgfsetfillopacity{0.700000}%
\pgfsetlinewidth{0.000000pt}%
\definecolor{currentstroke}{rgb}{0.000000,0.000000,0.000000}%
\pgfsetstrokecolor{currentstroke}%
\pgfsetstrokeopacity{0.700000}%
\pgfsetdash{}{0pt}%
\pgfpathmoveto{\pgfqpoint{8.522392in}{1.851862in}}%
\pgfpathcurveto{\pgfqpoint{8.527435in}{1.851862in}}{\pgfqpoint{8.532273in}{1.853865in}}{\pgfqpoint{8.535840in}{1.857432in}}%
\pgfpathcurveto{\pgfqpoint{8.539406in}{1.860998in}}{\pgfqpoint{8.541410in}{1.865836in}}{\pgfqpoint{8.541410in}{1.870880in}}%
\pgfpathcurveto{\pgfqpoint{8.541410in}{1.875923in}}{\pgfqpoint{8.539406in}{1.880761in}}{\pgfqpoint{8.535840in}{1.884328in}}%
\pgfpathcurveto{\pgfqpoint{8.532273in}{1.887894in}}{\pgfqpoint{8.527435in}{1.889898in}}{\pgfqpoint{8.522392in}{1.889898in}}%
\pgfpathcurveto{\pgfqpoint{8.517348in}{1.889898in}}{\pgfqpoint{8.512510in}{1.887894in}}{\pgfqpoint{8.508944in}{1.884328in}}%
\pgfpathcurveto{\pgfqpoint{8.505377in}{1.880761in}}{\pgfqpoint{8.503374in}{1.875923in}}{\pgfqpoint{8.503374in}{1.870880in}}%
\pgfpathcurveto{\pgfqpoint{8.503374in}{1.865836in}}{\pgfqpoint{8.505377in}{1.860998in}}{\pgfqpoint{8.508944in}{1.857432in}}%
\pgfpathcurveto{\pgfqpoint{8.512510in}{1.853865in}}{\pgfqpoint{8.517348in}{1.851862in}}{\pgfqpoint{8.522392in}{1.851862in}}%
\pgfpathclose%
\pgfusepath{fill}%
\end{pgfscope}%
\begin{pgfscope}%
\pgfpathrectangle{\pgfqpoint{6.572727in}{0.474100in}}{\pgfqpoint{4.227273in}{3.318700in}}%
\pgfusepath{clip}%
\pgfsetbuttcap%
\pgfsetroundjoin%
\definecolor{currentfill}{rgb}{0.127568,0.566949,0.550556}%
\pgfsetfillcolor{currentfill}%
\pgfsetfillopacity{0.700000}%
\pgfsetlinewidth{0.000000pt}%
\definecolor{currentstroke}{rgb}{0.000000,0.000000,0.000000}%
\pgfsetstrokecolor{currentstroke}%
\pgfsetstrokeopacity{0.700000}%
\pgfsetdash{}{0pt}%
\pgfpathmoveto{\pgfqpoint{7.738347in}{1.465319in}}%
\pgfpathcurveto{\pgfqpoint{7.743391in}{1.465319in}}{\pgfqpoint{7.748228in}{1.467323in}}{\pgfqpoint{7.751795in}{1.470890in}}%
\pgfpathcurveto{\pgfqpoint{7.755361in}{1.474456in}}{\pgfqpoint{7.757365in}{1.479294in}}{\pgfqpoint{7.757365in}{1.484337in}}%
\pgfpathcurveto{\pgfqpoint{7.757365in}{1.489381in}}{\pgfqpoint{7.755361in}{1.494219in}}{\pgfqpoint{7.751795in}{1.497785in}}%
\pgfpathcurveto{\pgfqpoint{7.748228in}{1.501352in}}{\pgfqpoint{7.743391in}{1.503356in}}{\pgfqpoint{7.738347in}{1.503356in}}%
\pgfpathcurveto{\pgfqpoint{7.733303in}{1.503356in}}{\pgfqpoint{7.728466in}{1.501352in}}{\pgfqpoint{7.724899in}{1.497785in}}%
\pgfpathcurveto{\pgfqpoint{7.721333in}{1.494219in}}{\pgfqpoint{7.719329in}{1.489381in}}{\pgfqpoint{7.719329in}{1.484337in}}%
\pgfpathcurveto{\pgfqpoint{7.719329in}{1.479294in}}{\pgfqpoint{7.721333in}{1.474456in}}{\pgfqpoint{7.724899in}{1.470890in}}%
\pgfpathcurveto{\pgfqpoint{7.728466in}{1.467323in}}{\pgfqpoint{7.733303in}{1.465319in}}{\pgfqpoint{7.738347in}{1.465319in}}%
\pgfpathclose%
\pgfusepath{fill}%
\end{pgfscope}%
\begin{pgfscope}%
\pgfpathrectangle{\pgfqpoint{6.572727in}{0.474100in}}{\pgfqpoint{4.227273in}{3.318700in}}%
\pgfusepath{clip}%
\pgfsetbuttcap%
\pgfsetroundjoin%
\definecolor{currentfill}{rgb}{0.127568,0.566949,0.550556}%
\pgfsetfillcolor{currentfill}%
\pgfsetfillopacity{0.700000}%
\pgfsetlinewidth{0.000000pt}%
\definecolor{currentstroke}{rgb}{0.000000,0.000000,0.000000}%
\pgfsetstrokecolor{currentstroke}%
\pgfsetstrokeopacity{0.700000}%
\pgfsetdash{}{0pt}%
\pgfpathmoveto{\pgfqpoint{8.209600in}{1.614300in}}%
\pgfpathcurveto{\pgfqpoint{8.214644in}{1.614300in}}{\pgfqpoint{8.219482in}{1.616304in}}{\pgfqpoint{8.223048in}{1.619870in}}%
\pgfpathcurveto{\pgfqpoint{8.226614in}{1.623437in}}{\pgfqpoint{8.228618in}{1.628275in}}{\pgfqpoint{8.228618in}{1.633318in}}%
\pgfpathcurveto{\pgfqpoint{8.228618in}{1.638362in}}{\pgfqpoint{8.226614in}{1.643200in}}{\pgfqpoint{8.223048in}{1.646766in}}%
\pgfpathcurveto{\pgfqpoint{8.219482in}{1.650332in}}{\pgfqpoint{8.214644in}{1.652336in}}{\pgfqpoint{8.209600in}{1.652336in}}%
\pgfpathcurveto{\pgfqpoint{8.204556in}{1.652336in}}{\pgfqpoint{8.199719in}{1.650332in}}{\pgfqpoint{8.196152in}{1.646766in}}%
\pgfpathcurveto{\pgfqpoint{8.192586in}{1.643200in}}{\pgfqpoint{8.190582in}{1.638362in}}{\pgfqpoint{8.190582in}{1.633318in}}%
\pgfpathcurveto{\pgfqpoint{8.190582in}{1.628275in}}{\pgfqpoint{8.192586in}{1.623437in}}{\pgfqpoint{8.196152in}{1.619870in}}%
\pgfpathcurveto{\pgfqpoint{8.199719in}{1.616304in}}{\pgfqpoint{8.204556in}{1.614300in}}{\pgfqpoint{8.209600in}{1.614300in}}%
\pgfpathclose%
\pgfusepath{fill}%
\end{pgfscope}%
\begin{pgfscope}%
\pgfpathrectangle{\pgfqpoint{6.572727in}{0.474100in}}{\pgfqpoint{4.227273in}{3.318700in}}%
\pgfusepath{clip}%
\pgfsetbuttcap%
\pgfsetroundjoin%
\definecolor{currentfill}{rgb}{0.127568,0.566949,0.550556}%
\pgfsetfillcolor{currentfill}%
\pgfsetfillopacity{0.700000}%
\pgfsetlinewidth{0.000000pt}%
\definecolor{currentstroke}{rgb}{0.000000,0.000000,0.000000}%
\pgfsetstrokecolor{currentstroke}%
\pgfsetstrokeopacity{0.700000}%
\pgfsetdash{}{0pt}%
\pgfpathmoveto{\pgfqpoint{8.211597in}{3.272556in}}%
\pgfpathcurveto{\pgfqpoint{8.216641in}{3.272556in}}{\pgfqpoint{8.221478in}{3.274560in}}{\pgfqpoint{8.225045in}{3.278127in}}%
\pgfpathcurveto{\pgfqpoint{8.228611in}{3.281693in}}{\pgfqpoint{8.230615in}{3.286531in}}{\pgfqpoint{8.230615in}{3.291574in}}%
\pgfpathcurveto{\pgfqpoint{8.230615in}{3.296618in}}{\pgfqpoint{8.228611in}{3.301456in}}{\pgfqpoint{8.225045in}{3.305022in}}%
\pgfpathcurveto{\pgfqpoint{8.221478in}{3.308589in}}{\pgfqpoint{8.216641in}{3.310593in}}{\pgfqpoint{8.211597in}{3.310593in}}%
\pgfpathcurveto{\pgfqpoint{8.206553in}{3.310593in}}{\pgfqpoint{8.201715in}{3.308589in}}{\pgfqpoint{8.198149in}{3.305022in}}%
\pgfpathcurveto{\pgfqpoint{8.194583in}{3.301456in}}{\pgfqpoint{8.192579in}{3.296618in}}{\pgfqpoint{8.192579in}{3.291574in}}%
\pgfpathcurveto{\pgfqpoint{8.192579in}{3.286531in}}{\pgfqpoint{8.194583in}{3.281693in}}{\pgfqpoint{8.198149in}{3.278127in}}%
\pgfpathcurveto{\pgfqpoint{8.201715in}{3.274560in}}{\pgfqpoint{8.206553in}{3.272556in}}{\pgfqpoint{8.211597in}{3.272556in}}%
\pgfpathclose%
\pgfusepath{fill}%
\end{pgfscope}%
\begin{pgfscope}%
\pgfpathrectangle{\pgfqpoint{6.572727in}{0.474100in}}{\pgfqpoint{4.227273in}{3.318700in}}%
\pgfusepath{clip}%
\pgfsetbuttcap%
\pgfsetroundjoin%
\definecolor{currentfill}{rgb}{0.993248,0.906157,0.143936}%
\pgfsetfillcolor{currentfill}%
\pgfsetfillopacity{0.700000}%
\pgfsetlinewidth{0.000000pt}%
\definecolor{currentstroke}{rgb}{0.000000,0.000000,0.000000}%
\pgfsetstrokecolor{currentstroke}%
\pgfsetstrokeopacity{0.700000}%
\pgfsetdash{}{0pt}%
\pgfpathmoveto{\pgfqpoint{10.157788in}{1.575062in}}%
\pgfpathcurveto{\pgfqpoint{10.162831in}{1.575062in}}{\pgfqpoint{10.167669in}{1.577066in}}{\pgfqpoint{10.171236in}{1.580632in}}%
\pgfpathcurveto{\pgfqpoint{10.174802in}{1.584199in}}{\pgfqpoint{10.176806in}{1.589036in}}{\pgfqpoint{10.176806in}{1.594080in}}%
\pgfpathcurveto{\pgfqpoint{10.176806in}{1.599124in}}{\pgfqpoint{10.174802in}{1.603961in}}{\pgfqpoint{10.171236in}{1.607528in}}%
\pgfpathcurveto{\pgfqpoint{10.167669in}{1.611094in}}{\pgfqpoint{10.162831in}{1.613098in}}{\pgfqpoint{10.157788in}{1.613098in}}%
\pgfpathcurveto{\pgfqpoint{10.152744in}{1.613098in}}{\pgfqpoint{10.147906in}{1.611094in}}{\pgfqpoint{10.144340in}{1.607528in}}%
\pgfpathcurveto{\pgfqpoint{10.140773in}{1.603961in}}{\pgfqpoint{10.138770in}{1.599124in}}{\pgfqpoint{10.138770in}{1.594080in}}%
\pgfpathcurveto{\pgfqpoint{10.138770in}{1.589036in}}{\pgfqpoint{10.140773in}{1.584199in}}{\pgfqpoint{10.144340in}{1.580632in}}%
\pgfpathcurveto{\pgfqpoint{10.147906in}{1.577066in}}{\pgfqpoint{10.152744in}{1.575062in}}{\pgfqpoint{10.157788in}{1.575062in}}%
\pgfpathclose%
\pgfusepath{fill}%
\end{pgfscope}%
\begin{pgfscope}%
\pgfpathrectangle{\pgfqpoint{6.572727in}{0.474100in}}{\pgfqpoint{4.227273in}{3.318700in}}%
\pgfusepath{clip}%
\pgfsetbuttcap%
\pgfsetroundjoin%
\definecolor{currentfill}{rgb}{0.127568,0.566949,0.550556}%
\pgfsetfillcolor{currentfill}%
\pgfsetfillopacity{0.700000}%
\pgfsetlinewidth{0.000000pt}%
\definecolor{currentstroke}{rgb}{0.000000,0.000000,0.000000}%
\pgfsetstrokecolor{currentstroke}%
\pgfsetstrokeopacity{0.700000}%
\pgfsetdash{}{0pt}%
\pgfpathmoveto{\pgfqpoint{8.554084in}{2.730303in}}%
\pgfpathcurveto{\pgfqpoint{8.559127in}{2.730303in}}{\pgfqpoint{8.563965in}{2.732307in}}{\pgfqpoint{8.567532in}{2.735873in}}%
\pgfpathcurveto{\pgfqpoint{8.571098in}{2.739440in}}{\pgfqpoint{8.573102in}{2.744277in}}{\pgfqpoint{8.573102in}{2.749321in}}%
\pgfpathcurveto{\pgfqpoint{8.573102in}{2.754365in}}{\pgfqpoint{8.571098in}{2.759203in}}{\pgfqpoint{8.567532in}{2.762769in}}%
\pgfpathcurveto{\pgfqpoint{8.563965in}{2.766335in}}{\pgfqpoint{8.559127in}{2.768339in}}{\pgfqpoint{8.554084in}{2.768339in}}%
\pgfpathcurveto{\pgfqpoint{8.549040in}{2.768339in}}{\pgfqpoint{8.544202in}{2.766335in}}{\pgfqpoint{8.540636in}{2.762769in}}%
\pgfpathcurveto{\pgfqpoint{8.537070in}{2.759203in}}{\pgfqpoint{8.535066in}{2.754365in}}{\pgfqpoint{8.535066in}{2.749321in}}%
\pgfpathcurveto{\pgfqpoint{8.535066in}{2.744277in}}{\pgfqpoint{8.537070in}{2.739440in}}{\pgfqpoint{8.540636in}{2.735873in}}%
\pgfpathcurveto{\pgfqpoint{8.544202in}{2.732307in}}{\pgfqpoint{8.549040in}{2.730303in}}{\pgfqpoint{8.554084in}{2.730303in}}%
\pgfpathclose%
\pgfusepath{fill}%
\end{pgfscope}%
\begin{pgfscope}%
\pgfpathrectangle{\pgfqpoint{6.572727in}{0.474100in}}{\pgfqpoint{4.227273in}{3.318700in}}%
\pgfusepath{clip}%
\pgfsetbuttcap%
\pgfsetroundjoin%
\definecolor{currentfill}{rgb}{0.993248,0.906157,0.143936}%
\pgfsetfillcolor{currentfill}%
\pgfsetfillopacity{0.700000}%
\pgfsetlinewidth{0.000000pt}%
\definecolor{currentstroke}{rgb}{0.000000,0.000000,0.000000}%
\pgfsetstrokecolor{currentstroke}%
\pgfsetstrokeopacity{0.700000}%
\pgfsetdash{}{0pt}%
\pgfpathmoveto{\pgfqpoint{9.476601in}{1.818879in}}%
\pgfpathcurveto{\pgfqpoint{9.481645in}{1.818879in}}{\pgfqpoint{9.486483in}{1.820882in}}{\pgfqpoint{9.490049in}{1.824449in}}%
\pgfpathcurveto{\pgfqpoint{9.493615in}{1.828015in}}{\pgfqpoint{9.495619in}{1.832853in}}{\pgfqpoint{9.495619in}{1.837897in}}%
\pgfpathcurveto{\pgfqpoint{9.495619in}{1.842940in}}{\pgfqpoint{9.493615in}{1.847778in}}{\pgfqpoint{9.490049in}{1.851345in}}%
\pgfpathcurveto{\pgfqpoint{9.486483in}{1.854911in}}{\pgfqpoint{9.481645in}{1.856915in}}{\pgfqpoint{9.476601in}{1.856915in}}%
\pgfpathcurveto{\pgfqpoint{9.471558in}{1.856915in}}{\pgfqpoint{9.466720in}{1.854911in}}{\pgfqpoint{9.463153in}{1.851345in}}%
\pgfpathcurveto{\pgfqpoint{9.459587in}{1.847778in}}{\pgfqpoint{9.457583in}{1.842940in}}{\pgfqpoint{9.457583in}{1.837897in}}%
\pgfpathcurveto{\pgfqpoint{9.457583in}{1.832853in}}{\pgfqpoint{9.459587in}{1.828015in}}{\pgfqpoint{9.463153in}{1.824449in}}%
\pgfpathcurveto{\pgfqpoint{9.466720in}{1.820882in}}{\pgfqpoint{9.471558in}{1.818879in}}{\pgfqpoint{9.476601in}{1.818879in}}%
\pgfpathclose%
\pgfusepath{fill}%
\end{pgfscope}%
\begin{pgfscope}%
\pgfpathrectangle{\pgfqpoint{6.572727in}{0.474100in}}{\pgfqpoint{4.227273in}{3.318700in}}%
\pgfusepath{clip}%
\pgfsetbuttcap%
\pgfsetroundjoin%
\definecolor{currentfill}{rgb}{0.127568,0.566949,0.550556}%
\pgfsetfillcolor{currentfill}%
\pgfsetfillopacity{0.700000}%
\pgfsetlinewidth{0.000000pt}%
\definecolor{currentstroke}{rgb}{0.000000,0.000000,0.000000}%
\pgfsetstrokecolor{currentstroke}%
\pgfsetstrokeopacity{0.700000}%
\pgfsetdash{}{0pt}%
\pgfpathmoveto{\pgfqpoint{8.696904in}{3.217153in}}%
\pgfpathcurveto{\pgfqpoint{8.701948in}{3.217153in}}{\pgfqpoint{8.706786in}{3.219157in}}{\pgfqpoint{8.710352in}{3.222724in}}%
\pgfpathcurveto{\pgfqpoint{8.713919in}{3.226290in}}{\pgfqpoint{8.715923in}{3.231128in}}{\pgfqpoint{8.715923in}{3.236171in}}%
\pgfpathcurveto{\pgfqpoint{8.715923in}{3.241215in}}{\pgfqpoint{8.713919in}{3.246053in}}{\pgfqpoint{8.710352in}{3.249619in}}%
\pgfpathcurveto{\pgfqpoint{8.706786in}{3.253186in}}{\pgfqpoint{8.701948in}{3.255190in}}{\pgfqpoint{8.696904in}{3.255190in}}%
\pgfpathcurveto{\pgfqpoint{8.691861in}{3.255190in}}{\pgfqpoint{8.687023in}{3.253186in}}{\pgfqpoint{8.683457in}{3.249619in}}%
\pgfpathcurveto{\pgfqpoint{8.679890in}{3.246053in}}{\pgfqpoint{8.677886in}{3.241215in}}{\pgfqpoint{8.677886in}{3.236171in}}%
\pgfpathcurveto{\pgfqpoint{8.677886in}{3.231128in}}{\pgfqpoint{8.679890in}{3.226290in}}{\pgfqpoint{8.683457in}{3.222724in}}%
\pgfpathcurveto{\pgfqpoint{8.687023in}{3.219157in}}{\pgfqpoint{8.691861in}{3.217153in}}{\pgfqpoint{8.696904in}{3.217153in}}%
\pgfpathclose%
\pgfusepath{fill}%
\end{pgfscope}%
\begin{pgfscope}%
\pgfpathrectangle{\pgfqpoint{6.572727in}{0.474100in}}{\pgfqpoint{4.227273in}{3.318700in}}%
\pgfusepath{clip}%
\pgfsetbuttcap%
\pgfsetroundjoin%
\definecolor{currentfill}{rgb}{0.127568,0.566949,0.550556}%
\pgfsetfillcolor{currentfill}%
\pgfsetfillopacity{0.700000}%
\pgfsetlinewidth{0.000000pt}%
\definecolor{currentstroke}{rgb}{0.000000,0.000000,0.000000}%
\pgfsetstrokecolor{currentstroke}%
\pgfsetstrokeopacity{0.700000}%
\pgfsetdash{}{0pt}%
\pgfpathmoveto{\pgfqpoint{7.504100in}{1.742828in}}%
\pgfpathcurveto{\pgfqpoint{7.509143in}{1.742828in}}{\pgfqpoint{7.513981in}{1.744831in}}{\pgfqpoint{7.517547in}{1.748398in}}%
\pgfpathcurveto{\pgfqpoint{7.521114in}{1.751964in}}{\pgfqpoint{7.523118in}{1.756802in}}{\pgfqpoint{7.523118in}{1.761846in}}%
\pgfpathcurveto{\pgfqpoint{7.523118in}{1.766889in}}{\pgfqpoint{7.521114in}{1.771727in}}{\pgfqpoint{7.517547in}{1.775294in}}%
\pgfpathcurveto{\pgfqpoint{7.513981in}{1.778860in}}{\pgfqpoint{7.509143in}{1.780864in}}{\pgfqpoint{7.504100in}{1.780864in}}%
\pgfpathcurveto{\pgfqpoint{7.499056in}{1.780864in}}{\pgfqpoint{7.494218in}{1.778860in}}{\pgfqpoint{7.490652in}{1.775294in}}%
\pgfpathcurveto{\pgfqpoint{7.487085in}{1.771727in}}{\pgfqpoint{7.485081in}{1.766889in}}{\pgfqpoint{7.485081in}{1.761846in}}%
\pgfpathcurveto{\pgfqpoint{7.485081in}{1.756802in}}{\pgfqpoint{7.487085in}{1.751964in}}{\pgfqpoint{7.490652in}{1.748398in}}%
\pgfpathcurveto{\pgfqpoint{7.494218in}{1.744831in}}{\pgfqpoint{7.499056in}{1.742828in}}{\pgfqpoint{7.504100in}{1.742828in}}%
\pgfpathclose%
\pgfusepath{fill}%
\end{pgfscope}%
\begin{pgfscope}%
\pgfpathrectangle{\pgfqpoint{6.572727in}{0.474100in}}{\pgfqpoint{4.227273in}{3.318700in}}%
\pgfusepath{clip}%
\pgfsetbuttcap%
\pgfsetroundjoin%
\definecolor{currentfill}{rgb}{0.993248,0.906157,0.143936}%
\pgfsetfillcolor{currentfill}%
\pgfsetfillopacity{0.700000}%
\pgfsetlinewidth{0.000000pt}%
\definecolor{currentstroke}{rgb}{0.000000,0.000000,0.000000}%
\pgfsetstrokecolor{currentstroke}%
\pgfsetstrokeopacity{0.700000}%
\pgfsetdash{}{0pt}%
\pgfpathmoveto{\pgfqpoint{9.212377in}{1.544617in}}%
\pgfpathcurveto{\pgfqpoint{9.217420in}{1.544617in}}{\pgfqpoint{9.222258in}{1.546621in}}{\pgfqpoint{9.225824in}{1.550188in}}%
\pgfpathcurveto{\pgfqpoint{9.229391in}{1.553754in}}{\pgfqpoint{9.231395in}{1.558592in}}{\pgfqpoint{9.231395in}{1.563635in}}%
\pgfpathcurveto{\pgfqpoint{9.231395in}{1.568679in}}{\pgfqpoint{9.229391in}{1.573517in}}{\pgfqpoint{9.225824in}{1.577083in}}%
\pgfpathcurveto{\pgfqpoint{9.222258in}{1.580650in}}{\pgfqpoint{9.217420in}{1.582654in}}{\pgfqpoint{9.212377in}{1.582654in}}%
\pgfpathcurveto{\pgfqpoint{9.207333in}{1.582654in}}{\pgfqpoint{9.202495in}{1.580650in}}{\pgfqpoint{9.198929in}{1.577083in}}%
\pgfpathcurveto{\pgfqpoint{9.195362in}{1.573517in}}{\pgfqpoint{9.193358in}{1.568679in}}{\pgfqpoint{9.193358in}{1.563635in}}%
\pgfpathcurveto{\pgfqpoint{9.193358in}{1.558592in}}{\pgfqpoint{9.195362in}{1.553754in}}{\pgfqpoint{9.198929in}{1.550188in}}%
\pgfpathcurveto{\pgfqpoint{9.202495in}{1.546621in}}{\pgfqpoint{9.207333in}{1.544617in}}{\pgfqpoint{9.212377in}{1.544617in}}%
\pgfpathclose%
\pgfusepath{fill}%
\end{pgfscope}%
\begin{pgfscope}%
\pgfpathrectangle{\pgfqpoint{6.572727in}{0.474100in}}{\pgfqpoint{4.227273in}{3.318700in}}%
\pgfusepath{clip}%
\pgfsetbuttcap%
\pgfsetroundjoin%
\definecolor{currentfill}{rgb}{0.993248,0.906157,0.143936}%
\pgfsetfillcolor{currentfill}%
\pgfsetfillopacity{0.700000}%
\pgfsetlinewidth{0.000000pt}%
\definecolor{currentstroke}{rgb}{0.000000,0.000000,0.000000}%
\pgfsetstrokecolor{currentstroke}%
\pgfsetstrokeopacity{0.700000}%
\pgfsetdash{}{0pt}%
\pgfpathmoveto{\pgfqpoint{9.314282in}{1.097635in}}%
\pgfpathcurveto{\pgfqpoint{9.319326in}{1.097635in}}{\pgfqpoint{9.324164in}{1.099639in}}{\pgfqpoint{9.327730in}{1.103205in}}%
\pgfpathcurveto{\pgfqpoint{9.331297in}{1.106772in}}{\pgfqpoint{9.333300in}{1.111609in}}{\pgfqpoint{9.333300in}{1.116653in}}%
\pgfpathcurveto{\pgfqpoint{9.333300in}{1.121697in}}{\pgfqpoint{9.331297in}{1.126534in}}{\pgfqpoint{9.327730in}{1.130101in}}%
\pgfpathcurveto{\pgfqpoint{9.324164in}{1.133667in}}{\pgfqpoint{9.319326in}{1.135671in}}{\pgfqpoint{9.314282in}{1.135671in}}%
\pgfpathcurveto{\pgfqpoint{9.309239in}{1.135671in}}{\pgfqpoint{9.304401in}{1.133667in}}{\pgfqpoint{9.300834in}{1.130101in}}%
\pgfpathcurveto{\pgfqpoint{9.297268in}{1.126534in}}{\pgfqpoint{9.295264in}{1.121697in}}{\pgfqpoint{9.295264in}{1.116653in}}%
\pgfpathcurveto{\pgfqpoint{9.295264in}{1.111609in}}{\pgfqpoint{9.297268in}{1.106772in}}{\pgfqpoint{9.300834in}{1.103205in}}%
\pgfpathcurveto{\pgfqpoint{9.304401in}{1.099639in}}{\pgfqpoint{9.309239in}{1.097635in}}{\pgfqpoint{9.314282in}{1.097635in}}%
\pgfpathclose%
\pgfusepath{fill}%
\end{pgfscope}%
\begin{pgfscope}%
\pgfpathrectangle{\pgfqpoint{6.572727in}{0.474100in}}{\pgfqpoint{4.227273in}{3.318700in}}%
\pgfusepath{clip}%
\pgfsetbuttcap%
\pgfsetroundjoin%
\definecolor{currentfill}{rgb}{0.993248,0.906157,0.143936}%
\pgfsetfillcolor{currentfill}%
\pgfsetfillopacity{0.700000}%
\pgfsetlinewidth{0.000000pt}%
\definecolor{currentstroke}{rgb}{0.000000,0.000000,0.000000}%
\pgfsetstrokecolor{currentstroke}%
\pgfsetstrokeopacity{0.700000}%
\pgfsetdash{}{0pt}%
\pgfpathmoveto{\pgfqpoint{9.338851in}{1.852237in}}%
\pgfpathcurveto{\pgfqpoint{9.343895in}{1.852237in}}{\pgfqpoint{9.348733in}{1.854241in}}{\pgfqpoint{9.352299in}{1.857807in}}%
\pgfpathcurveto{\pgfqpoint{9.355866in}{1.861373in}}{\pgfqpoint{9.357870in}{1.866211in}}{\pgfqpoint{9.357870in}{1.871255in}}%
\pgfpathcurveto{\pgfqpoint{9.357870in}{1.876299in}}{\pgfqpoint{9.355866in}{1.881136in}}{\pgfqpoint{9.352299in}{1.884703in}}%
\pgfpathcurveto{\pgfqpoint{9.348733in}{1.888269in}}{\pgfqpoint{9.343895in}{1.890273in}}{\pgfqpoint{9.338851in}{1.890273in}}%
\pgfpathcurveto{\pgfqpoint{9.333808in}{1.890273in}}{\pgfqpoint{9.328970in}{1.888269in}}{\pgfqpoint{9.325404in}{1.884703in}}%
\pgfpathcurveto{\pgfqpoint{9.321837in}{1.881136in}}{\pgfqpoint{9.319833in}{1.876299in}}{\pgfqpoint{9.319833in}{1.871255in}}%
\pgfpathcurveto{\pgfqpoint{9.319833in}{1.866211in}}{\pgfqpoint{9.321837in}{1.861373in}}{\pgfqpoint{9.325404in}{1.857807in}}%
\pgfpathcurveto{\pgfqpoint{9.328970in}{1.854241in}}{\pgfqpoint{9.333808in}{1.852237in}}{\pgfqpoint{9.338851in}{1.852237in}}%
\pgfpathclose%
\pgfusepath{fill}%
\end{pgfscope}%
\begin{pgfscope}%
\pgfpathrectangle{\pgfqpoint{6.572727in}{0.474100in}}{\pgfqpoint{4.227273in}{3.318700in}}%
\pgfusepath{clip}%
\pgfsetbuttcap%
\pgfsetroundjoin%
\definecolor{currentfill}{rgb}{0.993248,0.906157,0.143936}%
\pgfsetfillcolor{currentfill}%
\pgfsetfillopacity{0.700000}%
\pgfsetlinewidth{0.000000pt}%
\definecolor{currentstroke}{rgb}{0.000000,0.000000,0.000000}%
\pgfsetstrokecolor{currentstroke}%
\pgfsetstrokeopacity{0.700000}%
\pgfsetdash{}{0pt}%
\pgfpathmoveto{\pgfqpoint{9.753532in}{1.396403in}}%
\pgfpathcurveto{\pgfqpoint{9.758576in}{1.396403in}}{\pgfqpoint{9.763414in}{1.398407in}}{\pgfqpoint{9.766980in}{1.401974in}}%
\pgfpathcurveto{\pgfqpoint{9.770547in}{1.405540in}}{\pgfqpoint{9.772550in}{1.410378in}}{\pgfqpoint{9.772550in}{1.415421in}}%
\pgfpathcurveto{\pgfqpoint{9.772550in}{1.420465in}}{\pgfqpoint{9.770547in}{1.425303in}}{\pgfqpoint{9.766980in}{1.428869in}}%
\pgfpathcurveto{\pgfqpoint{9.763414in}{1.432436in}}{\pgfqpoint{9.758576in}{1.434440in}}{\pgfqpoint{9.753532in}{1.434440in}}%
\pgfpathcurveto{\pgfqpoint{9.748489in}{1.434440in}}{\pgfqpoint{9.743651in}{1.432436in}}{\pgfqpoint{9.740084in}{1.428869in}}%
\pgfpathcurveto{\pgfqpoint{9.736518in}{1.425303in}}{\pgfqpoint{9.734514in}{1.420465in}}{\pgfqpoint{9.734514in}{1.415421in}}%
\pgfpathcurveto{\pgfqpoint{9.734514in}{1.410378in}}{\pgfqpoint{9.736518in}{1.405540in}}{\pgfqpoint{9.740084in}{1.401974in}}%
\pgfpathcurveto{\pgfqpoint{9.743651in}{1.398407in}}{\pgfqpoint{9.748489in}{1.396403in}}{\pgfqpoint{9.753532in}{1.396403in}}%
\pgfpathclose%
\pgfusepath{fill}%
\end{pgfscope}%
\begin{pgfscope}%
\pgfpathrectangle{\pgfqpoint{6.572727in}{0.474100in}}{\pgfqpoint{4.227273in}{3.318700in}}%
\pgfusepath{clip}%
\pgfsetbuttcap%
\pgfsetroundjoin%
\definecolor{currentfill}{rgb}{0.993248,0.906157,0.143936}%
\pgfsetfillcolor{currentfill}%
\pgfsetfillopacity{0.700000}%
\pgfsetlinewidth{0.000000pt}%
\definecolor{currentstroke}{rgb}{0.000000,0.000000,0.000000}%
\pgfsetstrokecolor{currentstroke}%
\pgfsetstrokeopacity{0.700000}%
\pgfsetdash{}{0pt}%
\pgfpathmoveto{\pgfqpoint{9.121166in}{2.209381in}}%
\pgfpathcurveto{\pgfqpoint{9.126209in}{2.209381in}}{\pgfqpoint{9.131047in}{2.211385in}}{\pgfqpoint{9.134613in}{2.214952in}}%
\pgfpathcurveto{\pgfqpoint{9.138180in}{2.218518in}}{\pgfqpoint{9.140184in}{2.223356in}}{\pgfqpoint{9.140184in}{2.228399in}}%
\pgfpathcurveto{\pgfqpoint{9.140184in}{2.233443in}}{\pgfqpoint{9.138180in}{2.238281in}}{\pgfqpoint{9.134613in}{2.241847in}}%
\pgfpathcurveto{\pgfqpoint{9.131047in}{2.245414in}}{\pgfqpoint{9.126209in}{2.247418in}}{\pgfqpoint{9.121166in}{2.247418in}}%
\pgfpathcurveto{\pgfqpoint{9.116122in}{2.247418in}}{\pgfqpoint{9.111284in}{2.245414in}}{\pgfqpoint{9.107718in}{2.241847in}}%
\pgfpathcurveto{\pgfqpoint{9.104151in}{2.238281in}}{\pgfqpoint{9.102147in}{2.233443in}}{\pgfqpoint{9.102147in}{2.228399in}}%
\pgfpathcurveto{\pgfqpoint{9.102147in}{2.223356in}}{\pgfqpoint{9.104151in}{2.218518in}}{\pgfqpoint{9.107718in}{2.214952in}}%
\pgfpathcurveto{\pgfqpoint{9.111284in}{2.211385in}}{\pgfqpoint{9.116122in}{2.209381in}}{\pgfqpoint{9.121166in}{2.209381in}}%
\pgfpathclose%
\pgfusepath{fill}%
\end{pgfscope}%
\begin{pgfscope}%
\pgfpathrectangle{\pgfqpoint{6.572727in}{0.474100in}}{\pgfqpoint{4.227273in}{3.318700in}}%
\pgfusepath{clip}%
\pgfsetbuttcap%
\pgfsetroundjoin%
\definecolor{currentfill}{rgb}{0.127568,0.566949,0.550556}%
\pgfsetfillcolor{currentfill}%
\pgfsetfillopacity{0.700000}%
\pgfsetlinewidth{0.000000pt}%
\definecolor{currentstroke}{rgb}{0.000000,0.000000,0.000000}%
\pgfsetstrokecolor{currentstroke}%
\pgfsetstrokeopacity{0.700000}%
\pgfsetdash{}{0pt}%
\pgfpathmoveto{\pgfqpoint{7.761253in}{2.659170in}}%
\pgfpathcurveto{\pgfqpoint{7.766297in}{2.659170in}}{\pgfqpoint{7.771135in}{2.661174in}}{\pgfqpoint{7.774701in}{2.664740in}}%
\pgfpathcurveto{\pgfqpoint{7.778267in}{2.668307in}}{\pgfqpoint{7.780271in}{2.673144in}}{\pgfqpoint{7.780271in}{2.678188in}}%
\pgfpathcurveto{\pgfqpoint{7.780271in}{2.683232in}}{\pgfqpoint{7.778267in}{2.688070in}}{\pgfqpoint{7.774701in}{2.691636in}}%
\pgfpathcurveto{\pgfqpoint{7.771135in}{2.695202in}}{\pgfqpoint{7.766297in}{2.697206in}}{\pgfqpoint{7.761253in}{2.697206in}}%
\pgfpathcurveto{\pgfqpoint{7.756209in}{2.697206in}}{\pgfqpoint{7.751372in}{2.695202in}}{\pgfqpoint{7.747805in}{2.691636in}}%
\pgfpathcurveto{\pgfqpoint{7.744239in}{2.688070in}}{\pgfqpoint{7.742235in}{2.683232in}}{\pgfqpoint{7.742235in}{2.678188in}}%
\pgfpathcurveto{\pgfqpoint{7.742235in}{2.673144in}}{\pgfqpoint{7.744239in}{2.668307in}}{\pgfqpoint{7.747805in}{2.664740in}}%
\pgfpathcurveto{\pgfqpoint{7.751372in}{2.661174in}}{\pgfqpoint{7.756209in}{2.659170in}}{\pgfqpoint{7.761253in}{2.659170in}}%
\pgfpathclose%
\pgfusepath{fill}%
\end{pgfscope}%
\begin{pgfscope}%
\pgfpathrectangle{\pgfqpoint{6.572727in}{0.474100in}}{\pgfqpoint{4.227273in}{3.318700in}}%
\pgfusepath{clip}%
\pgfsetbuttcap%
\pgfsetroundjoin%
\definecolor{currentfill}{rgb}{0.993248,0.906157,0.143936}%
\pgfsetfillcolor{currentfill}%
\pgfsetfillopacity{0.700000}%
\pgfsetlinewidth{0.000000pt}%
\definecolor{currentstroke}{rgb}{0.000000,0.000000,0.000000}%
\pgfsetstrokecolor{currentstroke}%
\pgfsetstrokeopacity{0.700000}%
\pgfsetdash{}{0pt}%
\pgfpathmoveto{\pgfqpoint{9.003812in}{1.826922in}}%
\pgfpathcurveto{\pgfqpoint{9.008856in}{1.826922in}}{\pgfqpoint{9.013694in}{1.828925in}}{\pgfqpoint{9.017260in}{1.832492in}}%
\pgfpathcurveto{\pgfqpoint{9.020827in}{1.836058in}}{\pgfqpoint{9.022831in}{1.840896in}}{\pgfqpoint{9.022831in}{1.845940in}}%
\pgfpathcurveto{\pgfqpoint{9.022831in}{1.850983in}}{\pgfqpoint{9.020827in}{1.855821in}}{\pgfqpoint{9.017260in}{1.859388in}}%
\pgfpathcurveto{\pgfqpoint{9.013694in}{1.862954in}}{\pgfqpoint{9.008856in}{1.864958in}}{\pgfqpoint{9.003812in}{1.864958in}}%
\pgfpathcurveto{\pgfqpoint{8.998769in}{1.864958in}}{\pgfqpoint{8.993931in}{1.862954in}}{\pgfqpoint{8.990365in}{1.859388in}}%
\pgfpathcurveto{\pgfqpoint{8.986798in}{1.855821in}}{\pgfqpoint{8.984794in}{1.850983in}}{\pgfqpoint{8.984794in}{1.845940in}}%
\pgfpathcurveto{\pgfqpoint{8.984794in}{1.840896in}}{\pgfqpoint{8.986798in}{1.836058in}}{\pgfqpoint{8.990365in}{1.832492in}}%
\pgfpathcurveto{\pgfqpoint{8.993931in}{1.828925in}}{\pgfqpoint{8.998769in}{1.826922in}}{\pgfqpoint{9.003812in}{1.826922in}}%
\pgfpathclose%
\pgfusepath{fill}%
\end{pgfscope}%
\begin{pgfscope}%
\pgfpathrectangle{\pgfqpoint{6.572727in}{0.474100in}}{\pgfqpoint{4.227273in}{3.318700in}}%
\pgfusepath{clip}%
\pgfsetbuttcap%
\pgfsetroundjoin%
\definecolor{currentfill}{rgb}{0.127568,0.566949,0.550556}%
\pgfsetfillcolor{currentfill}%
\pgfsetfillopacity{0.700000}%
\pgfsetlinewidth{0.000000pt}%
\definecolor{currentstroke}{rgb}{0.000000,0.000000,0.000000}%
\pgfsetstrokecolor{currentstroke}%
\pgfsetstrokeopacity{0.700000}%
\pgfsetdash{}{0pt}%
\pgfpathmoveto{\pgfqpoint{8.194852in}{3.195193in}}%
\pgfpathcurveto{\pgfqpoint{8.199896in}{3.195193in}}{\pgfqpoint{8.204734in}{3.197197in}}{\pgfqpoint{8.208300in}{3.200763in}}%
\pgfpathcurveto{\pgfqpoint{8.211866in}{3.204330in}}{\pgfqpoint{8.213870in}{3.209167in}}{\pgfqpoint{8.213870in}{3.214211in}}%
\pgfpathcurveto{\pgfqpoint{8.213870in}{3.219255in}}{\pgfqpoint{8.211866in}{3.224092in}}{\pgfqpoint{8.208300in}{3.227659in}}%
\pgfpathcurveto{\pgfqpoint{8.204734in}{3.231225in}}{\pgfqpoint{8.199896in}{3.233229in}}{\pgfqpoint{8.194852in}{3.233229in}}%
\pgfpathcurveto{\pgfqpoint{8.189808in}{3.233229in}}{\pgfqpoint{8.184971in}{3.231225in}}{\pgfqpoint{8.181404in}{3.227659in}}%
\pgfpathcurveto{\pgfqpoint{8.177838in}{3.224092in}}{\pgfqpoint{8.175834in}{3.219255in}}{\pgfqpoint{8.175834in}{3.214211in}}%
\pgfpathcurveto{\pgfqpoint{8.175834in}{3.209167in}}{\pgfqpoint{8.177838in}{3.204330in}}{\pgfqpoint{8.181404in}{3.200763in}}%
\pgfpathcurveto{\pgfqpoint{8.184971in}{3.197197in}}{\pgfqpoint{8.189808in}{3.195193in}}{\pgfqpoint{8.194852in}{3.195193in}}%
\pgfpathclose%
\pgfusepath{fill}%
\end{pgfscope}%
\begin{pgfscope}%
\pgfpathrectangle{\pgfqpoint{6.572727in}{0.474100in}}{\pgfqpoint{4.227273in}{3.318700in}}%
\pgfusepath{clip}%
\pgfsetbuttcap%
\pgfsetroundjoin%
\definecolor{currentfill}{rgb}{0.127568,0.566949,0.550556}%
\pgfsetfillcolor{currentfill}%
\pgfsetfillopacity{0.700000}%
\pgfsetlinewidth{0.000000pt}%
\definecolor{currentstroke}{rgb}{0.000000,0.000000,0.000000}%
\pgfsetstrokecolor{currentstroke}%
\pgfsetstrokeopacity{0.700000}%
\pgfsetdash{}{0pt}%
\pgfpathmoveto{\pgfqpoint{7.894513in}{1.834776in}}%
\pgfpathcurveto{\pgfqpoint{7.899556in}{1.834776in}}{\pgfqpoint{7.904394in}{1.836780in}}{\pgfqpoint{7.907961in}{1.840346in}}%
\pgfpathcurveto{\pgfqpoint{7.911527in}{1.843913in}}{\pgfqpoint{7.913531in}{1.848750in}}{\pgfqpoint{7.913531in}{1.853794in}}%
\pgfpathcurveto{\pgfqpoint{7.913531in}{1.858838in}}{\pgfqpoint{7.911527in}{1.863675in}}{\pgfqpoint{7.907961in}{1.867242in}}%
\pgfpathcurveto{\pgfqpoint{7.904394in}{1.870808in}}{\pgfqpoint{7.899556in}{1.872812in}}{\pgfqpoint{7.894513in}{1.872812in}}%
\pgfpathcurveto{\pgfqpoint{7.889469in}{1.872812in}}{\pgfqpoint{7.884631in}{1.870808in}}{\pgfqpoint{7.881065in}{1.867242in}}%
\pgfpathcurveto{\pgfqpoint{7.877498in}{1.863675in}}{\pgfqpoint{7.875494in}{1.858838in}}{\pgfqpoint{7.875494in}{1.853794in}}%
\pgfpathcurveto{\pgfqpoint{7.875494in}{1.848750in}}{\pgfqpoint{7.877498in}{1.843913in}}{\pgfqpoint{7.881065in}{1.840346in}}%
\pgfpathcurveto{\pgfqpoint{7.884631in}{1.836780in}}{\pgfqpoint{7.889469in}{1.834776in}}{\pgfqpoint{7.894513in}{1.834776in}}%
\pgfpathclose%
\pgfusepath{fill}%
\end{pgfscope}%
\begin{pgfscope}%
\pgfpathrectangle{\pgfqpoint{6.572727in}{0.474100in}}{\pgfqpoint{4.227273in}{3.318700in}}%
\pgfusepath{clip}%
\pgfsetbuttcap%
\pgfsetroundjoin%
\definecolor{currentfill}{rgb}{0.127568,0.566949,0.550556}%
\pgfsetfillcolor{currentfill}%
\pgfsetfillopacity{0.700000}%
\pgfsetlinewidth{0.000000pt}%
\definecolor{currentstroke}{rgb}{0.000000,0.000000,0.000000}%
\pgfsetstrokecolor{currentstroke}%
\pgfsetstrokeopacity{0.700000}%
\pgfsetdash{}{0pt}%
\pgfpathmoveto{\pgfqpoint{7.919408in}{2.049837in}}%
\pgfpathcurveto{\pgfqpoint{7.924452in}{2.049837in}}{\pgfqpoint{7.929289in}{2.051841in}}{\pgfqpoint{7.932856in}{2.055408in}}%
\pgfpathcurveto{\pgfqpoint{7.936422in}{2.058974in}}{\pgfqpoint{7.938426in}{2.063812in}}{\pgfqpoint{7.938426in}{2.068855in}}%
\pgfpathcurveto{\pgfqpoint{7.938426in}{2.073899in}}{\pgfqpoint{7.936422in}{2.078737in}}{\pgfqpoint{7.932856in}{2.082303in}}%
\pgfpathcurveto{\pgfqpoint{7.929289in}{2.085870in}}{\pgfqpoint{7.924452in}{2.087874in}}{\pgfqpoint{7.919408in}{2.087874in}}%
\pgfpathcurveto{\pgfqpoint{7.914364in}{2.087874in}}{\pgfqpoint{7.909527in}{2.085870in}}{\pgfqpoint{7.905960in}{2.082303in}}%
\pgfpathcurveto{\pgfqpoint{7.902394in}{2.078737in}}{\pgfqpoint{7.900390in}{2.073899in}}{\pgfqpoint{7.900390in}{2.068855in}}%
\pgfpathcurveto{\pgfqpoint{7.900390in}{2.063812in}}{\pgfqpoint{7.902394in}{2.058974in}}{\pgfqpoint{7.905960in}{2.055408in}}%
\pgfpathcurveto{\pgfqpoint{7.909527in}{2.051841in}}{\pgfqpoint{7.914364in}{2.049837in}}{\pgfqpoint{7.919408in}{2.049837in}}%
\pgfpathclose%
\pgfusepath{fill}%
\end{pgfscope}%
\begin{pgfscope}%
\pgfpathrectangle{\pgfqpoint{6.572727in}{0.474100in}}{\pgfqpoint{4.227273in}{3.318700in}}%
\pgfusepath{clip}%
\pgfsetbuttcap%
\pgfsetroundjoin%
\definecolor{currentfill}{rgb}{0.127568,0.566949,0.550556}%
\pgfsetfillcolor{currentfill}%
\pgfsetfillopacity{0.700000}%
\pgfsetlinewidth{0.000000pt}%
\definecolor{currentstroke}{rgb}{0.000000,0.000000,0.000000}%
\pgfsetstrokecolor{currentstroke}%
\pgfsetstrokeopacity{0.700000}%
\pgfsetdash{}{0pt}%
\pgfpathmoveto{\pgfqpoint{7.527580in}{2.592671in}}%
\pgfpathcurveto{\pgfqpoint{7.532624in}{2.592671in}}{\pgfqpoint{7.537462in}{2.594675in}}{\pgfqpoint{7.541028in}{2.598241in}}%
\pgfpathcurveto{\pgfqpoint{7.544594in}{2.601808in}}{\pgfqpoint{7.546598in}{2.606645in}}{\pgfqpoint{7.546598in}{2.611689in}}%
\pgfpathcurveto{\pgfqpoint{7.546598in}{2.616733in}}{\pgfqpoint{7.544594in}{2.621571in}}{\pgfqpoint{7.541028in}{2.625137in}}%
\pgfpathcurveto{\pgfqpoint{7.537462in}{2.628703in}}{\pgfqpoint{7.532624in}{2.630707in}}{\pgfqpoint{7.527580in}{2.630707in}}%
\pgfpathcurveto{\pgfqpoint{7.522537in}{2.630707in}}{\pgfqpoint{7.517699in}{2.628703in}}{\pgfqpoint{7.514132in}{2.625137in}}%
\pgfpathcurveto{\pgfqpoint{7.510566in}{2.621571in}}{\pgfqpoint{7.508562in}{2.616733in}}{\pgfqpoint{7.508562in}{2.611689in}}%
\pgfpathcurveto{\pgfqpoint{7.508562in}{2.606645in}}{\pgfqpoint{7.510566in}{2.601808in}}{\pgfqpoint{7.514132in}{2.598241in}}%
\pgfpathcurveto{\pgfqpoint{7.517699in}{2.594675in}}{\pgfqpoint{7.522537in}{2.592671in}}{\pgfqpoint{7.527580in}{2.592671in}}%
\pgfpathclose%
\pgfusepath{fill}%
\end{pgfscope}%
\begin{pgfscope}%
\pgfpathrectangle{\pgfqpoint{6.572727in}{0.474100in}}{\pgfqpoint{4.227273in}{3.318700in}}%
\pgfusepath{clip}%
\pgfsetbuttcap%
\pgfsetroundjoin%
\definecolor{currentfill}{rgb}{0.127568,0.566949,0.550556}%
\pgfsetfillcolor{currentfill}%
\pgfsetfillopacity{0.700000}%
\pgfsetlinewidth{0.000000pt}%
\definecolor{currentstroke}{rgb}{0.000000,0.000000,0.000000}%
\pgfsetstrokecolor{currentstroke}%
\pgfsetstrokeopacity{0.700000}%
\pgfsetdash{}{0pt}%
\pgfpathmoveto{\pgfqpoint{8.538993in}{2.840198in}}%
\pgfpathcurveto{\pgfqpoint{8.544037in}{2.840198in}}{\pgfqpoint{8.548875in}{2.842202in}}{\pgfqpoint{8.552441in}{2.845768in}}%
\pgfpathcurveto{\pgfqpoint{8.556007in}{2.849335in}}{\pgfqpoint{8.558011in}{2.854173in}}{\pgfqpoint{8.558011in}{2.859216in}}%
\pgfpathcurveto{\pgfqpoint{8.558011in}{2.864260in}}{\pgfqpoint{8.556007in}{2.869098in}}{\pgfqpoint{8.552441in}{2.872664in}}%
\pgfpathcurveto{\pgfqpoint{8.548875in}{2.876231in}}{\pgfqpoint{8.544037in}{2.878234in}}{\pgfqpoint{8.538993in}{2.878234in}}%
\pgfpathcurveto{\pgfqpoint{8.533949in}{2.878234in}}{\pgfqpoint{8.529112in}{2.876231in}}{\pgfqpoint{8.525545in}{2.872664in}}%
\pgfpathcurveto{\pgfqpoint{8.521979in}{2.869098in}}{\pgfqpoint{8.519975in}{2.864260in}}{\pgfqpoint{8.519975in}{2.859216in}}%
\pgfpathcurveto{\pgfqpoint{8.519975in}{2.854173in}}{\pgfqpoint{8.521979in}{2.849335in}}{\pgfqpoint{8.525545in}{2.845768in}}%
\pgfpathcurveto{\pgfqpoint{8.529112in}{2.842202in}}{\pgfqpoint{8.533949in}{2.840198in}}{\pgfqpoint{8.538993in}{2.840198in}}%
\pgfpathclose%
\pgfusepath{fill}%
\end{pgfscope}%
\begin{pgfscope}%
\pgfpathrectangle{\pgfqpoint{6.572727in}{0.474100in}}{\pgfqpoint{4.227273in}{3.318700in}}%
\pgfusepath{clip}%
\pgfsetbuttcap%
\pgfsetroundjoin%
\definecolor{currentfill}{rgb}{0.127568,0.566949,0.550556}%
\pgfsetfillcolor{currentfill}%
\pgfsetfillopacity{0.700000}%
\pgfsetlinewidth{0.000000pt}%
\definecolor{currentstroke}{rgb}{0.000000,0.000000,0.000000}%
\pgfsetstrokecolor{currentstroke}%
\pgfsetstrokeopacity{0.700000}%
\pgfsetdash{}{0pt}%
\pgfpathmoveto{\pgfqpoint{7.185606in}{1.533941in}}%
\pgfpathcurveto{\pgfqpoint{7.190649in}{1.533941in}}{\pgfqpoint{7.195487in}{1.535945in}}{\pgfqpoint{7.199053in}{1.539511in}}%
\pgfpathcurveto{\pgfqpoint{7.202620in}{1.543078in}}{\pgfqpoint{7.204624in}{1.547916in}}{\pgfqpoint{7.204624in}{1.552959in}}%
\pgfpathcurveto{\pgfqpoint{7.204624in}{1.558003in}}{\pgfqpoint{7.202620in}{1.562841in}}{\pgfqpoint{7.199053in}{1.566407in}}%
\pgfpathcurveto{\pgfqpoint{7.195487in}{1.569974in}}{\pgfqpoint{7.190649in}{1.571977in}}{\pgfqpoint{7.185606in}{1.571977in}}%
\pgfpathcurveto{\pgfqpoint{7.180562in}{1.571977in}}{\pgfqpoint{7.175724in}{1.569974in}}{\pgfqpoint{7.172158in}{1.566407in}}%
\pgfpathcurveto{\pgfqpoint{7.168591in}{1.562841in}}{\pgfqpoint{7.166587in}{1.558003in}}{\pgfqpoint{7.166587in}{1.552959in}}%
\pgfpathcurveto{\pgfqpoint{7.166587in}{1.547916in}}{\pgfqpoint{7.168591in}{1.543078in}}{\pgfqpoint{7.172158in}{1.539511in}}%
\pgfpathcurveto{\pgfqpoint{7.175724in}{1.535945in}}{\pgfqpoint{7.180562in}{1.533941in}}{\pgfqpoint{7.185606in}{1.533941in}}%
\pgfpathclose%
\pgfusepath{fill}%
\end{pgfscope}%
\begin{pgfscope}%
\pgfpathrectangle{\pgfqpoint{6.572727in}{0.474100in}}{\pgfqpoint{4.227273in}{3.318700in}}%
\pgfusepath{clip}%
\pgfsetbuttcap%
\pgfsetroundjoin%
\definecolor{currentfill}{rgb}{0.127568,0.566949,0.550556}%
\pgfsetfillcolor{currentfill}%
\pgfsetfillopacity{0.700000}%
\pgfsetlinewidth{0.000000pt}%
\definecolor{currentstroke}{rgb}{0.000000,0.000000,0.000000}%
\pgfsetstrokecolor{currentstroke}%
\pgfsetstrokeopacity{0.700000}%
\pgfsetdash{}{0pt}%
\pgfpathmoveto{\pgfqpoint{8.430377in}{1.511858in}}%
\pgfpathcurveto{\pgfqpoint{8.435420in}{1.511858in}}{\pgfqpoint{8.440258in}{1.513862in}}{\pgfqpoint{8.443825in}{1.517428in}}%
\pgfpathcurveto{\pgfqpoint{8.447391in}{1.520995in}}{\pgfqpoint{8.449395in}{1.525832in}}{\pgfqpoint{8.449395in}{1.530876in}}%
\pgfpathcurveto{\pgfqpoint{8.449395in}{1.535920in}}{\pgfqpoint{8.447391in}{1.540758in}}{\pgfqpoint{8.443825in}{1.544324in}}%
\pgfpathcurveto{\pgfqpoint{8.440258in}{1.547890in}}{\pgfqpoint{8.435420in}{1.549894in}}{\pgfqpoint{8.430377in}{1.549894in}}%
\pgfpathcurveto{\pgfqpoint{8.425333in}{1.549894in}}{\pgfqpoint{8.420495in}{1.547890in}}{\pgfqpoint{8.416929in}{1.544324in}}%
\pgfpathcurveto{\pgfqpoint{8.413363in}{1.540758in}}{\pgfqpoint{8.411359in}{1.535920in}}{\pgfqpoint{8.411359in}{1.530876in}}%
\pgfpathcurveto{\pgfqpoint{8.411359in}{1.525832in}}{\pgfqpoint{8.413363in}{1.520995in}}{\pgfqpoint{8.416929in}{1.517428in}}%
\pgfpathcurveto{\pgfqpoint{8.420495in}{1.513862in}}{\pgfqpoint{8.425333in}{1.511858in}}{\pgfqpoint{8.430377in}{1.511858in}}%
\pgfpathclose%
\pgfusepath{fill}%
\end{pgfscope}%
\begin{pgfscope}%
\pgfpathrectangle{\pgfqpoint{6.572727in}{0.474100in}}{\pgfqpoint{4.227273in}{3.318700in}}%
\pgfusepath{clip}%
\pgfsetbuttcap%
\pgfsetroundjoin%
\definecolor{currentfill}{rgb}{0.993248,0.906157,0.143936}%
\pgfsetfillcolor{currentfill}%
\pgfsetfillopacity{0.700000}%
\pgfsetlinewidth{0.000000pt}%
\definecolor{currentstroke}{rgb}{0.000000,0.000000,0.000000}%
\pgfsetstrokecolor{currentstroke}%
\pgfsetstrokeopacity{0.700000}%
\pgfsetdash{}{0pt}%
\pgfpathmoveto{\pgfqpoint{9.990781in}{1.756719in}}%
\pgfpathcurveto{\pgfqpoint{9.995825in}{1.756719in}}{\pgfqpoint{10.000662in}{1.758723in}}{\pgfqpoint{10.004229in}{1.762289in}}%
\pgfpathcurveto{\pgfqpoint{10.007795in}{1.765856in}}{\pgfqpoint{10.009799in}{1.770693in}}{\pgfqpoint{10.009799in}{1.775737in}}%
\pgfpathcurveto{\pgfqpoint{10.009799in}{1.780781in}}{\pgfqpoint{10.007795in}{1.785619in}}{\pgfqpoint{10.004229in}{1.789185in}}%
\pgfpathcurveto{\pgfqpoint{10.000662in}{1.792751in}}{\pgfqpoint{9.995825in}{1.794755in}}{\pgfqpoint{9.990781in}{1.794755in}}%
\pgfpathcurveto{\pgfqpoint{9.985737in}{1.794755in}}{\pgfqpoint{9.980900in}{1.792751in}}{\pgfqpoint{9.977333in}{1.789185in}}%
\pgfpathcurveto{\pgfqpoint{9.973767in}{1.785619in}}{\pgfqpoint{9.971763in}{1.780781in}}{\pgfqpoint{9.971763in}{1.775737in}}%
\pgfpathcurveto{\pgfqpoint{9.971763in}{1.770693in}}{\pgfqpoint{9.973767in}{1.765856in}}{\pgfqpoint{9.977333in}{1.762289in}}%
\pgfpathcurveto{\pgfqpoint{9.980900in}{1.758723in}}{\pgfqpoint{9.985737in}{1.756719in}}{\pgfqpoint{9.990781in}{1.756719in}}%
\pgfpathclose%
\pgfusepath{fill}%
\end{pgfscope}%
\begin{pgfscope}%
\pgfpathrectangle{\pgfqpoint{6.572727in}{0.474100in}}{\pgfqpoint{4.227273in}{3.318700in}}%
\pgfusepath{clip}%
\pgfsetbuttcap%
\pgfsetroundjoin%
\definecolor{currentfill}{rgb}{0.993248,0.906157,0.143936}%
\pgfsetfillcolor{currentfill}%
\pgfsetfillopacity{0.700000}%
\pgfsetlinewidth{0.000000pt}%
\definecolor{currentstroke}{rgb}{0.000000,0.000000,0.000000}%
\pgfsetstrokecolor{currentstroke}%
\pgfsetstrokeopacity{0.700000}%
\pgfsetdash{}{0pt}%
\pgfpathmoveto{\pgfqpoint{9.653179in}{1.280965in}}%
\pgfpathcurveto{\pgfqpoint{9.658223in}{1.280965in}}{\pgfqpoint{9.663060in}{1.282969in}}{\pgfqpoint{9.666627in}{1.286536in}}%
\pgfpathcurveto{\pgfqpoint{9.670193in}{1.290102in}}{\pgfqpoint{9.672197in}{1.294940in}}{\pgfqpoint{9.672197in}{1.299983in}}%
\pgfpathcurveto{\pgfqpoint{9.672197in}{1.305027in}}{\pgfqpoint{9.670193in}{1.309865in}}{\pgfqpoint{9.666627in}{1.313431in}}%
\pgfpathcurveto{\pgfqpoint{9.663060in}{1.316998in}}{\pgfqpoint{9.658223in}{1.319002in}}{\pgfqpoint{9.653179in}{1.319002in}}%
\pgfpathcurveto{\pgfqpoint{9.648135in}{1.319002in}}{\pgfqpoint{9.643298in}{1.316998in}}{\pgfqpoint{9.639731in}{1.313431in}}%
\pgfpathcurveto{\pgfqpoint{9.636165in}{1.309865in}}{\pgfqpoint{9.634161in}{1.305027in}}{\pgfqpoint{9.634161in}{1.299983in}}%
\pgfpathcurveto{\pgfqpoint{9.634161in}{1.294940in}}{\pgfqpoint{9.636165in}{1.290102in}}{\pgfqpoint{9.639731in}{1.286536in}}%
\pgfpathcurveto{\pgfqpoint{9.643298in}{1.282969in}}{\pgfqpoint{9.648135in}{1.280965in}}{\pgfqpoint{9.653179in}{1.280965in}}%
\pgfpathclose%
\pgfusepath{fill}%
\end{pgfscope}%
\begin{pgfscope}%
\pgfpathrectangle{\pgfqpoint{6.572727in}{0.474100in}}{\pgfqpoint{4.227273in}{3.318700in}}%
\pgfusepath{clip}%
\pgfsetbuttcap%
\pgfsetroundjoin%
\definecolor{currentfill}{rgb}{0.993248,0.906157,0.143936}%
\pgfsetfillcolor{currentfill}%
\pgfsetfillopacity{0.700000}%
\pgfsetlinewidth{0.000000pt}%
\definecolor{currentstroke}{rgb}{0.000000,0.000000,0.000000}%
\pgfsetstrokecolor{currentstroke}%
\pgfsetstrokeopacity{0.700000}%
\pgfsetdash{}{0pt}%
\pgfpathmoveto{\pgfqpoint{9.630854in}{1.570718in}}%
\pgfpathcurveto{\pgfqpoint{9.635898in}{1.570718in}}{\pgfqpoint{9.640735in}{1.572722in}}{\pgfqpoint{9.644302in}{1.576289in}}%
\pgfpathcurveto{\pgfqpoint{9.647868in}{1.579855in}}{\pgfqpoint{9.649872in}{1.584693in}}{\pgfqpoint{9.649872in}{1.589736in}}%
\pgfpathcurveto{\pgfqpoint{9.649872in}{1.594780in}}{\pgfqpoint{9.647868in}{1.599618in}}{\pgfqpoint{9.644302in}{1.603184in}}%
\pgfpathcurveto{\pgfqpoint{9.640735in}{1.606751in}}{\pgfqpoint{9.635898in}{1.608755in}}{\pgfqpoint{9.630854in}{1.608755in}}%
\pgfpathcurveto{\pgfqpoint{9.625810in}{1.608755in}}{\pgfqpoint{9.620972in}{1.606751in}}{\pgfqpoint{9.617406in}{1.603184in}}%
\pgfpathcurveto{\pgfqpoint{9.613840in}{1.599618in}}{\pgfqpoint{9.611836in}{1.594780in}}{\pgfqpoint{9.611836in}{1.589736in}}%
\pgfpathcurveto{\pgfqpoint{9.611836in}{1.584693in}}{\pgfqpoint{9.613840in}{1.579855in}}{\pgfqpoint{9.617406in}{1.576289in}}%
\pgfpathcurveto{\pgfqpoint{9.620972in}{1.572722in}}{\pgfqpoint{9.625810in}{1.570718in}}{\pgfqpoint{9.630854in}{1.570718in}}%
\pgfpathclose%
\pgfusepath{fill}%
\end{pgfscope}%
\begin{pgfscope}%
\pgfpathrectangle{\pgfqpoint{6.572727in}{0.474100in}}{\pgfqpoint{4.227273in}{3.318700in}}%
\pgfusepath{clip}%
\pgfsetbuttcap%
\pgfsetroundjoin%
\definecolor{currentfill}{rgb}{0.127568,0.566949,0.550556}%
\pgfsetfillcolor{currentfill}%
\pgfsetfillopacity{0.700000}%
\pgfsetlinewidth{0.000000pt}%
\definecolor{currentstroke}{rgb}{0.000000,0.000000,0.000000}%
\pgfsetstrokecolor{currentstroke}%
\pgfsetstrokeopacity{0.700000}%
\pgfsetdash{}{0pt}%
\pgfpathmoveto{\pgfqpoint{7.593807in}{1.159188in}}%
\pgfpathcurveto{\pgfqpoint{7.598851in}{1.159188in}}{\pgfqpoint{7.603689in}{1.161192in}}{\pgfqpoint{7.607255in}{1.164758in}}%
\pgfpathcurveto{\pgfqpoint{7.610822in}{1.168325in}}{\pgfqpoint{7.612825in}{1.173163in}}{\pgfqpoint{7.612825in}{1.178206in}}%
\pgfpathcurveto{\pgfqpoint{7.612825in}{1.183250in}}{\pgfqpoint{7.610822in}{1.188088in}}{\pgfqpoint{7.607255in}{1.191654in}}%
\pgfpathcurveto{\pgfqpoint{7.603689in}{1.195221in}}{\pgfqpoint{7.598851in}{1.197224in}}{\pgfqpoint{7.593807in}{1.197224in}}%
\pgfpathcurveto{\pgfqpoint{7.588764in}{1.197224in}}{\pgfqpoint{7.583926in}{1.195221in}}{\pgfqpoint{7.580359in}{1.191654in}}%
\pgfpathcurveto{\pgfqpoint{7.576793in}{1.188088in}}{\pgfqpoint{7.574789in}{1.183250in}}{\pgfqpoint{7.574789in}{1.178206in}}%
\pgfpathcurveto{\pgfqpoint{7.574789in}{1.173163in}}{\pgfqpoint{7.576793in}{1.168325in}}{\pgfqpoint{7.580359in}{1.164758in}}%
\pgfpathcurveto{\pgfqpoint{7.583926in}{1.161192in}}{\pgfqpoint{7.588764in}{1.159188in}}{\pgfqpoint{7.593807in}{1.159188in}}%
\pgfpathclose%
\pgfusepath{fill}%
\end{pgfscope}%
\begin{pgfscope}%
\pgfpathrectangle{\pgfqpoint{6.572727in}{0.474100in}}{\pgfqpoint{4.227273in}{3.318700in}}%
\pgfusepath{clip}%
\pgfsetbuttcap%
\pgfsetroundjoin%
\definecolor{currentfill}{rgb}{0.127568,0.566949,0.550556}%
\pgfsetfillcolor{currentfill}%
\pgfsetfillopacity{0.700000}%
\pgfsetlinewidth{0.000000pt}%
\definecolor{currentstroke}{rgb}{0.000000,0.000000,0.000000}%
\pgfsetstrokecolor{currentstroke}%
\pgfsetstrokeopacity{0.700000}%
\pgfsetdash{}{0pt}%
\pgfpathmoveto{\pgfqpoint{7.796717in}{1.533100in}}%
\pgfpathcurveto{\pgfqpoint{7.801761in}{1.533100in}}{\pgfqpoint{7.806599in}{1.535104in}}{\pgfqpoint{7.810165in}{1.538671in}}%
\pgfpathcurveto{\pgfqpoint{7.813732in}{1.542237in}}{\pgfqpoint{7.815736in}{1.547075in}}{\pgfqpoint{7.815736in}{1.552118in}}%
\pgfpathcurveto{\pgfqpoint{7.815736in}{1.557162in}}{\pgfqpoint{7.813732in}{1.562000in}}{\pgfqpoint{7.810165in}{1.565566in}}%
\pgfpathcurveto{\pgfqpoint{7.806599in}{1.569133in}}{\pgfqpoint{7.801761in}{1.571137in}}{\pgfqpoint{7.796717in}{1.571137in}}%
\pgfpathcurveto{\pgfqpoint{7.791674in}{1.571137in}}{\pgfqpoint{7.786836in}{1.569133in}}{\pgfqpoint{7.783270in}{1.565566in}}%
\pgfpathcurveto{\pgfqpoint{7.779703in}{1.562000in}}{\pgfqpoint{7.777699in}{1.557162in}}{\pgfqpoint{7.777699in}{1.552118in}}%
\pgfpathcurveto{\pgfqpoint{7.777699in}{1.547075in}}{\pgfqpoint{7.779703in}{1.542237in}}{\pgfqpoint{7.783270in}{1.538671in}}%
\pgfpathcurveto{\pgfqpoint{7.786836in}{1.535104in}}{\pgfqpoint{7.791674in}{1.533100in}}{\pgfqpoint{7.796717in}{1.533100in}}%
\pgfpathclose%
\pgfusepath{fill}%
\end{pgfscope}%
\begin{pgfscope}%
\pgfpathrectangle{\pgfqpoint{6.572727in}{0.474100in}}{\pgfqpoint{4.227273in}{3.318700in}}%
\pgfusepath{clip}%
\pgfsetbuttcap%
\pgfsetroundjoin%
\definecolor{currentfill}{rgb}{0.993248,0.906157,0.143936}%
\pgfsetfillcolor{currentfill}%
\pgfsetfillopacity{0.700000}%
\pgfsetlinewidth{0.000000pt}%
\definecolor{currentstroke}{rgb}{0.000000,0.000000,0.000000}%
\pgfsetstrokecolor{currentstroke}%
\pgfsetstrokeopacity{0.700000}%
\pgfsetdash{}{0pt}%
\pgfpathmoveto{\pgfqpoint{9.555610in}{1.856996in}}%
\pgfpathcurveto{\pgfqpoint{9.560654in}{1.856996in}}{\pgfqpoint{9.565491in}{1.859000in}}{\pgfqpoint{9.569058in}{1.862566in}}%
\pgfpathcurveto{\pgfqpoint{9.572624in}{1.866132in}}{\pgfqpoint{9.574628in}{1.870970in}}{\pgfqpoint{9.574628in}{1.876014in}}%
\pgfpathcurveto{\pgfqpoint{9.574628in}{1.881058in}}{\pgfqpoint{9.572624in}{1.885895in}}{\pgfqpoint{9.569058in}{1.889462in}}%
\pgfpathcurveto{\pgfqpoint{9.565491in}{1.893028in}}{\pgfqpoint{9.560654in}{1.895032in}}{\pgfqpoint{9.555610in}{1.895032in}}%
\pgfpathcurveto{\pgfqpoint{9.550566in}{1.895032in}}{\pgfqpoint{9.545728in}{1.893028in}}{\pgfqpoint{9.542162in}{1.889462in}}%
\pgfpathcurveto{\pgfqpoint{9.538596in}{1.885895in}}{\pgfqpoint{9.536592in}{1.881058in}}{\pgfqpoint{9.536592in}{1.876014in}}%
\pgfpathcurveto{\pgfqpoint{9.536592in}{1.870970in}}{\pgfqpoint{9.538596in}{1.866132in}}{\pgfqpoint{9.542162in}{1.862566in}}%
\pgfpathcurveto{\pgfqpoint{9.545728in}{1.859000in}}{\pgfqpoint{9.550566in}{1.856996in}}{\pgfqpoint{9.555610in}{1.856996in}}%
\pgfpathclose%
\pgfusepath{fill}%
\end{pgfscope}%
\begin{pgfscope}%
\pgfpathrectangle{\pgfqpoint{6.572727in}{0.474100in}}{\pgfqpoint{4.227273in}{3.318700in}}%
\pgfusepath{clip}%
\pgfsetbuttcap%
\pgfsetroundjoin%
\definecolor{currentfill}{rgb}{0.127568,0.566949,0.550556}%
\pgfsetfillcolor{currentfill}%
\pgfsetfillopacity{0.700000}%
\pgfsetlinewidth{0.000000pt}%
\definecolor{currentstroke}{rgb}{0.000000,0.000000,0.000000}%
\pgfsetstrokecolor{currentstroke}%
\pgfsetstrokeopacity{0.700000}%
\pgfsetdash{}{0pt}%
\pgfpathmoveto{\pgfqpoint{7.679271in}{0.886242in}}%
\pgfpathcurveto{\pgfqpoint{7.684315in}{0.886242in}}{\pgfqpoint{7.689153in}{0.888246in}}{\pgfqpoint{7.692719in}{0.891813in}}%
\pgfpathcurveto{\pgfqpoint{7.696286in}{0.895379in}}{\pgfqpoint{7.698290in}{0.900217in}}{\pgfqpoint{7.698290in}{0.905260in}}%
\pgfpathcurveto{\pgfqpoint{7.698290in}{0.910304in}}{\pgfqpoint{7.696286in}{0.915142in}}{\pgfqpoint{7.692719in}{0.918708in}}%
\pgfpathcurveto{\pgfqpoint{7.689153in}{0.922275in}}{\pgfqpoint{7.684315in}{0.924279in}}{\pgfqpoint{7.679271in}{0.924279in}}%
\pgfpathcurveto{\pgfqpoint{7.674228in}{0.924279in}}{\pgfqpoint{7.669390in}{0.922275in}}{\pgfqpoint{7.665824in}{0.918708in}}%
\pgfpathcurveto{\pgfqpoint{7.662257in}{0.915142in}}{\pgfqpoint{7.660253in}{0.910304in}}{\pgfqpoint{7.660253in}{0.905260in}}%
\pgfpathcurveto{\pgfqpoint{7.660253in}{0.900217in}}{\pgfqpoint{7.662257in}{0.895379in}}{\pgfqpoint{7.665824in}{0.891813in}}%
\pgfpathcurveto{\pgfqpoint{7.669390in}{0.888246in}}{\pgfqpoint{7.674228in}{0.886242in}}{\pgfqpoint{7.679271in}{0.886242in}}%
\pgfpathclose%
\pgfusepath{fill}%
\end{pgfscope}%
\begin{pgfscope}%
\pgfpathrectangle{\pgfqpoint{6.572727in}{0.474100in}}{\pgfqpoint{4.227273in}{3.318700in}}%
\pgfusepath{clip}%
\pgfsetbuttcap%
\pgfsetroundjoin%
\definecolor{currentfill}{rgb}{0.127568,0.566949,0.550556}%
\pgfsetfillcolor{currentfill}%
\pgfsetfillopacity{0.700000}%
\pgfsetlinewidth{0.000000pt}%
\definecolor{currentstroke}{rgb}{0.000000,0.000000,0.000000}%
\pgfsetstrokecolor{currentstroke}%
\pgfsetstrokeopacity{0.700000}%
\pgfsetdash{}{0pt}%
\pgfpathmoveto{\pgfqpoint{8.637455in}{2.661919in}}%
\pgfpathcurveto{\pgfqpoint{8.642498in}{2.661919in}}{\pgfqpoint{8.647336in}{2.663923in}}{\pgfqpoint{8.650903in}{2.667490in}}%
\pgfpathcurveto{\pgfqpoint{8.654469in}{2.671056in}}{\pgfqpoint{8.656473in}{2.675894in}}{\pgfqpoint{8.656473in}{2.680937in}}%
\pgfpathcurveto{\pgfqpoint{8.656473in}{2.685981in}}{\pgfqpoint{8.654469in}{2.690819in}}{\pgfqpoint{8.650903in}{2.694385in}}%
\pgfpathcurveto{\pgfqpoint{8.647336in}{2.697952in}}{\pgfqpoint{8.642498in}{2.699956in}}{\pgfqpoint{8.637455in}{2.699956in}}%
\pgfpathcurveto{\pgfqpoint{8.632411in}{2.699956in}}{\pgfqpoint{8.627573in}{2.697952in}}{\pgfqpoint{8.624007in}{2.694385in}}%
\pgfpathcurveto{\pgfqpoint{8.620440in}{2.690819in}}{\pgfqpoint{8.618437in}{2.685981in}}{\pgfqpoint{8.618437in}{2.680937in}}%
\pgfpathcurveto{\pgfqpoint{8.618437in}{2.675894in}}{\pgfqpoint{8.620440in}{2.671056in}}{\pgfqpoint{8.624007in}{2.667490in}}%
\pgfpathcurveto{\pgfqpoint{8.627573in}{2.663923in}}{\pgfqpoint{8.632411in}{2.661919in}}{\pgfqpoint{8.637455in}{2.661919in}}%
\pgfpathclose%
\pgfusepath{fill}%
\end{pgfscope}%
\begin{pgfscope}%
\pgfpathrectangle{\pgfqpoint{6.572727in}{0.474100in}}{\pgfqpoint{4.227273in}{3.318700in}}%
\pgfusepath{clip}%
\pgfsetbuttcap%
\pgfsetroundjoin%
\definecolor{currentfill}{rgb}{0.993248,0.906157,0.143936}%
\pgfsetfillcolor{currentfill}%
\pgfsetfillopacity{0.700000}%
\pgfsetlinewidth{0.000000pt}%
\definecolor{currentstroke}{rgb}{0.000000,0.000000,0.000000}%
\pgfsetstrokecolor{currentstroke}%
\pgfsetstrokeopacity{0.700000}%
\pgfsetdash{}{0pt}%
\pgfpathmoveto{\pgfqpoint{9.117483in}{1.724453in}}%
\pgfpathcurveto{\pgfqpoint{9.122527in}{1.724453in}}{\pgfqpoint{9.127364in}{1.726457in}}{\pgfqpoint{9.130931in}{1.730023in}}%
\pgfpathcurveto{\pgfqpoint{9.134497in}{1.733590in}}{\pgfqpoint{9.136501in}{1.738428in}}{\pgfqpoint{9.136501in}{1.743471in}}%
\pgfpathcurveto{\pgfqpoint{9.136501in}{1.748515in}}{\pgfqpoint{9.134497in}{1.753353in}}{\pgfqpoint{9.130931in}{1.756919in}}%
\pgfpathcurveto{\pgfqpoint{9.127364in}{1.760485in}}{\pgfqpoint{9.122527in}{1.762489in}}{\pgfqpoint{9.117483in}{1.762489in}}%
\pgfpathcurveto{\pgfqpoint{9.112439in}{1.762489in}}{\pgfqpoint{9.107602in}{1.760485in}}{\pgfqpoint{9.104035in}{1.756919in}}%
\pgfpathcurveto{\pgfqpoint{9.100469in}{1.753353in}}{\pgfqpoint{9.098465in}{1.748515in}}{\pgfqpoint{9.098465in}{1.743471in}}%
\pgfpathcurveto{\pgfqpoint{9.098465in}{1.738428in}}{\pgfqpoint{9.100469in}{1.733590in}}{\pgfqpoint{9.104035in}{1.730023in}}%
\pgfpathcurveto{\pgfqpoint{9.107602in}{1.726457in}}{\pgfqpoint{9.112439in}{1.724453in}}{\pgfqpoint{9.117483in}{1.724453in}}%
\pgfpathclose%
\pgfusepath{fill}%
\end{pgfscope}%
\begin{pgfscope}%
\pgfpathrectangle{\pgfqpoint{6.572727in}{0.474100in}}{\pgfqpoint{4.227273in}{3.318700in}}%
\pgfusepath{clip}%
\pgfsetbuttcap%
\pgfsetroundjoin%
\definecolor{currentfill}{rgb}{0.993248,0.906157,0.143936}%
\pgfsetfillcolor{currentfill}%
\pgfsetfillopacity{0.700000}%
\pgfsetlinewidth{0.000000pt}%
\definecolor{currentstroke}{rgb}{0.000000,0.000000,0.000000}%
\pgfsetstrokecolor{currentstroke}%
\pgfsetstrokeopacity{0.700000}%
\pgfsetdash{}{0pt}%
\pgfpathmoveto{\pgfqpoint{9.693644in}{1.789993in}}%
\pgfpathcurveto{\pgfqpoint{9.698688in}{1.789993in}}{\pgfqpoint{9.703526in}{1.791997in}}{\pgfqpoint{9.707092in}{1.795563in}}%
\pgfpathcurveto{\pgfqpoint{9.710658in}{1.799130in}}{\pgfqpoint{9.712662in}{1.803967in}}{\pgfqpoint{9.712662in}{1.809011in}}%
\pgfpathcurveto{\pgfqpoint{9.712662in}{1.814055in}}{\pgfqpoint{9.710658in}{1.818892in}}{\pgfqpoint{9.707092in}{1.822459in}}%
\pgfpathcurveto{\pgfqpoint{9.703526in}{1.826025in}}{\pgfqpoint{9.698688in}{1.828029in}}{\pgfqpoint{9.693644in}{1.828029in}}%
\pgfpathcurveto{\pgfqpoint{9.688600in}{1.828029in}}{\pgfqpoint{9.683763in}{1.826025in}}{\pgfqpoint{9.680196in}{1.822459in}}%
\pgfpathcurveto{\pgfqpoint{9.676630in}{1.818892in}}{\pgfqpoint{9.674626in}{1.814055in}}{\pgfqpoint{9.674626in}{1.809011in}}%
\pgfpathcurveto{\pgfqpoint{9.674626in}{1.803967in}}{\pgfqpoint{9.676630in}{1.799130in}}{\pgfqpoint{9.680196in}{1.795563in}}%
\pgfpathcurveto{\pgfqpoint{9.683763in}{1.791997in}}{\pgfqpoint{9.688600in}{1.789993in}}{\pgfqpoint{9.693644in}{1.789993in}}%
\pgfpathclose%
\pgfusepath{fill}%
\end{pgfscope}%
\begin{pgfscope}%
\pgfpathrectangle{\pgfqpoint{6.572727in}{0.474100in}}{\pgfqpoint{4.227273in}{3.318700in}}%
\pgfusepath{clip}%
\pgfsetbuttcap%
\pgfsetroundjoin%
\definecolor{currentfill}{rgb}{0.993248,0.906157,0.143936}%
\pgfsetfillcolor{currentfill}%
\pgfsetfillopacity{0.700000}%
\pgfsetlinewidth{0.000000pt}%
\definecolor{currentstroke}{rgb}{0.000000,0.000000,0.000000}%
\pgfsetstrokecolor{currentstroke}%
\pgfsetstrokeopacity{0.700000}%
\pgfsetdash{}{0pt}%
\pgfpathmoveto{\pgfqpoint{9.619127in}{1.971599in}}%
\pgfpathcurveto{\pgfqpoint{9.624170in}{1.971599in}}{\pgfqpoint{9.629008in}{1.973603in}}{\pgfqpoint{9.632574in}{1.977169in}}%
\pgfpathcurveto{\pgfqpoint{9.636141in}{1.980736in}}{\pgfqpoint{9.638145in}{1.985573in}}{\pgfqpoint{9.638145in}{1.990617in}}%
\pgfpathcurveto{\pgfqpoint{9.638145in}{1.995661in}}{\pgfqpoint{9.636141in}{2.000499in}}{\pgfqpoint{9.632574in}{2.004065in}}%
\pgfpathcurveto{\pgfqpoint{9.629008in}{2.007631in}}{\pgfqpoint{9.624170in}{2.009635in}}{\pgfqpoint{9.619127in}{2.009635in}}%
\pgfpathcurveto{\pgfqpoint{9.614083in}{2.009635in}}{\pgfqpoint{9.609245in}{2.007631in}}{\pgfqpoint{9.605679in}{2.004065in}}%
\pgfpathcurveto{\pgfqpoint{9.602112in}{2.000499in}}{\pgfqpoint{9.600108in}{1.995661in}}{\pgfqpoint{9.600108in}{1.990617in}}%
\pgfpathcurveto{\pgfqpoint{9.600108in}{1.985573in}}{\pgfqpoint{9.602112in}{1.980736in}}{\pgfqpoint{9.605679in}{1.977169in}}%
\pgfpathcurveto{\pgfqpoint{9.609245in}{1.973603in}}{\pgfqpoint{9.614083in}{1.971599in}}{\pgfqpoint{9.619127in}{1.971599in}}%
\pgfpathclose%
\pgfusepath{fill}%
\end{pgfscope}%
\begin{pgfscope}%
\pgfpathrectangle{\pgfqpoint{6.572727in}{0.474100in}}{\pgfqpoint{4.227273in}{3.318700in}}%
\pgfusepath{clip}%
\pgfsetbuttcap%
\pgfsetroundjoin%
\definecolor{currentfill}{rgb}{0.127568,0.566949,0.550556}%
\pgfsetfillcolor{currentfill}%
\pgfsetfillopacity{0.700000}%
\pgfsetlinewidth{0.000000pt}%
\definecolor{currentstroke}{rgb}{0.000000,0.000000,0.000000}%
\pgfsetstrokecolor{currentstroke}%
\pgfsetstrokeopacity{0.700000}%
\pgfsetdash{}{0pt}%
\pgfpathmoveto{\pgfqpoint{7.627012in}{2.515853in}}%
\pgfpathcurveto{\pgfqpoint{7.632056in}{2.515853in}}{\pgfqpoint{7.636894in}{2.517857in}}{\pgfqpoint{7.640460in}{2.521423in}}%
\pgfpathcurveto{\pgfqpoint{7.644027in}{2.524989in}}{\pgfqpoint{7.646031in}{2.529827in}}{\pgfqpoint{7.646031in}{2.534871in}}%
\pgfpathcurveto{\pgfqpoint{7.646031in}{2.539915in}}{\pgfqpoint{7.644027in}{2.544752in}}{\pgfqpoint{7.640460in}{2.548319in}}%
\pgfpathcurveto{\pgfqpoint{7.636894in}{2.551885in}}{\pgfqpoint{7.632056in}{2.553889in}}{\pgfqpoint{7.627012in}{2.553889in}}%
\pgfpathcurveto{\pgfqpoint{7.621969in}{2.553889in}}{\pgfqpoint{7.617131in}{2.551885in}}{\pgfqpoint{7.613565in}{2.548319in}}%
\pgfpathcurveto{\pgfqpoint{7.609998in}{2.544752in}}{\pgfqpoint{7.607994in}{2.539915in}}{\pgfqpoint{7.607994in}{2.534871in}}%
\pgfpathcurveto{\pgfqpoint{7.607994in}{2.529827in}}{\pgfqpoint{7.609998in}{2.524989in}}{\pgfqpoint{7.613565in}{2.521423in}}%
\pgfpathcurveto{\pgfqpoint{7.617131in}{2.517857in}}{\pgfqpoint{7.621969in}{2.515853in}}{\pgfqpoint{7.627012in}{2.515853in}}%
\pgfpathclose%
\pgfusepath{fill}%
\end{pgfscope}%
\begin{pgfscope}%
\pgfpathrectangle{\pgfqpoint{6.572727in}{0.474100in}}{\pgfqpoint{4.227273in}{3.318700in}}%
\pgfusepath{clip}%
\pgfsetbuttcap%
\pgfsetroundjoin%
\definecolor{currentfill}{rgb}{0.127568,0.566949,0.550556}%
\pgfsetfillcolor{currentfill}%
\pgfsetfillopacity{0.700000}%
\pgfsetlinewidth{0.000000pt}%
\definecolor{currentstroke}{rgb}{0.000000,0.000000,0.000000}%
\pgfsetstrokecolor{currentstroke}%
\pgfsetstrokeopacity{0.700000}%
\pgfsetdash{}{0pt}%
\pgfpathmoveto{\pgfqpoint{7.374058in}{3.103701in}}%
\pgfpathcurveto{\pgfqpoint{7.379101in}{3.103701in}}{\pgfqpoint{7.383939in}{3.105705in}}{\pgfqpoint{7.387505in}{3.109271in}}%
\pgfpathcurveto{\pgfqpoint{7.391072in}{3.112837in}}{\pgfqpoint{7.393076in}{3.117675in}}{\pgfqpoint{7.393076in}{3.122719in}}%
\pgfpathcurveto{\pgfqpoint{7.393076in}{3.127763in}}{\pgfqpoint{7.391072in}{3.132600in}}{\pgfqpoint{7.387505in}{3.136167in}}%
\pgfpathcurveto{\pgfqpoint{7.383939in}{3.139733in}}{\pgfqpoint{7.379101in}{3.141737in}}{\pgfqpoint{7.374058in}{3.141737in}}%
\pgfpathcurveto{\pgfqpoint{7.369014in}{3.141737in}}{\pgfqpoint{7.364176in}{3.139733in}}{\pgfqpoint{7.360610in}{3.136167in}}%
\pgfpathcurveto{\pgfqpoint{7.357043in}{3.132600in}}{\pgfqpoint{7.355039in}{3.127763in}}{\pgfqpoint{7.355039in}{3.122719in}}%
\pgfpathcurveto{\pgfqpoint{7.355039in}{3.117675in}}{\pgfqpoint{7.357043in}{3.112837in}}{\pgfqpoint{7.360610in}{3.109271in}}%
\pgfpathcurveto{\pgfqpoint{7.364176in}{3.105705in}}{\pgfqpoint{7.369014in}{3.103701in}}{\pgfqpoint{7.374058in}{3.103701in}}%
\pgfpathclose%
\pgfusepath{fill}%
\end{pgfscope}%
\begin{pgfscope}%
\pgfpathrectangle{\pgfqpoint{6.572727in}{0.474100in}}{\pgfqpoint{4.227273in}{3.318700in}}%
\pgfusepath{clip}%
\pgfsetbuttcap%
\pgfsetroundjoin%
\definecolor{currentfill}{rgb}{0.993248,0.906157,0.143936}%
\pgfsetfillcolor{currentfill}%
\pgfsetfillopacity{0.700000}%
\pgfsetlinewidth{0.000000pt}%
\definecolor{currentstroke}{rgb}{0.000000,0.000000,0.000000}%
\pgfsetstrokecolor{currentstroke}%
\pgfsetstrokeopacity{0.700000}%
\pgfsetdash{}{0pt}%
\pgfpathmoveto{\pgfqpoint{9.408185in}{1.668826in}}%
\pgfpathcurveto{\pgfqpoint{9.413229in}{1.668826in}}{\pgfqpoint{9.418067in}{1.670830in}}{\pgfqpoint{9.421633in}{1.674396in}}%
\pgfpathcurveto{\pgfqpoint{9.425199in}{1.677962in}}{\pgfqpoint{9.427203in}{1.682800in}}{\pgfqpoint{9.427203in}{1.687844in}}%
\pgfpathcurveto{\pgfqpoint{9.427203in}{1.692888in}}{\pgfqpoint{9.425199in}{1.697725in}}{\pgfqpoint{9.421633in}{1.701292in}}%
\pgfpathcurveto{\pgfqpoint{9.418067in}{1.704858in}}{\pgfqpoint{9.413229in}{1.706862in}}{\pgfqpoint{9.408185in}{1.706862in}}%
\pgfpathcurveto{\pgfqpoint{9.403141in}{1.706862in}}{\pgfqpoint{9.398304in}{1.704858in}}{\pgfqpoint{9.394737in}{1.701292in}}%
\pgfpathcurveto{\pgfqpoint{9.391171in}{1.697725in}}{\pgfqpoint{9.389167in}{1.692888in}}{\pgfqpoint{9.389167in}{1.687844in}}%
\pgfpathcurveto{\pgfqpoint{9.389167in}{1.682800in}}{\pgfqpoint{9.391171in}{1.677962in}}{\pgfqpoint{9.394737in}{1.674396in}}%
\pgfpathcurveto{\pgfqpoint{9.398304in}{1.670830in}}{\pgfqpoint{9.403141in}{1.668826in}}{\pgfqpoint{9.408185in}{1.668826in}}%
\pgfpathclose%
\pgfusepath{fill}%
\end{pgfscope}%
\begin{pgfscope}%
\pgfpathrectangle{\pgfqpoint{6.572727in}{0.474100in}}{\pgfqpoint{4.227273in}{3.318700in}}%
\pgfusepath{clip}%
\pgfsetbuttcap%
\pgfsetroundjoin%
\definecolor{currentfill}{rgb}{0.993248,0.906157,0.143936}%
\pgfsetfillcolor{currentfill}%
\pgfsetfillopacity{0.700000}%
\pgfsetlinewidth{0.000000pt}%
\definecolor{currentstroke}{rgb}{0.000000,0.000000,0.000000}%
\pgfsetstrokecolor{currentstroke}%
\pgfsetstrokeopacity{0.700000}%
\pgfsetdash{}{0pt}%
\pgfpathmoveto{\pgfqpoint{9.627874in}{0.937863in}}%
\pgfpathcurveto{\pgfqpoint{9.632918in}{0.937863in}}{\pgfqpoint{9.637756in}{0.939867in}}{\pgfqpoint{9.641322in}{0.943433in}}%
\pgfpathcurveto{\pgfqpoint{9.644888in}{0.947000in}}{\pgfqpoint{9.646892in}{0.951838in}}{\pgfqpoint{9.646892in}{0.956881in}}%
\pgfpathcurveto{\pgfqpoint{9.646892in}{0.961925in}}{\pgfqpoint{9.644888in}{0.966763in}}{\pgfqpoint{9.641322in}{0.970329in}}%
\pgfpathcurveto{\pgfqpoint{9.637756in}{0.973895in}}{\pgfqpoint{9.632918in}{0.975899in}}{\pgfqpoint{9.627874in}{0.975899in}}%
\pgfpathcurveto{\pgfqpoint{9.622830in}{0.975899in}}{\pgfqpoint{9.617993in}{0.973895in}}{\pgfqpoint{9.614426in}{0.970329in}}%
\pgfpathcurveto{\pgfqpoint{9.610860in}{0.966763in}}{\pgfqpoint{9.608856in}{0.961925in}}{\pgfqpoint{9.608856in}{0.956881in}}%
\pgfpathcurveto{\pgfqpoint{9.608856in}{0.951838in}}{\pgfqpoint{9.610860in}{0.947000in}}{\pgfqpoint{9.614426in}{0.943433in}}%
\pgfpathcurveto{\pgfqpoint{9.617993in}{0.939867in}}{\pgfqpoint{9.622830in}{0.937863in}}{\pgfqpoint{9.627874in}{0.937863in}}%
\pgfpathclose%
\pgfusepath{fill}%
\end{pgfscope}%
\begin{pgfscope}%
\pgfpathrectangle{\pgfqpoint{6.572727in}{0.474100in}}{\pgfqpoint{4.227273in}{3.318700in}}%
\pgfusepath{clip}%
\pgfsetbuttcap%
\pgfsetroundjoin%
\definecolor{currentfill}{rgb}{0.993248,0.906157,0.143936}%
\pgfsetfillcolor{currentfill}%
\pgfsetfillopacity{0.700000}%
\pgfsetlinewidth{0.000000pt}%
\definecolor{currentstroke}{rgb}{0.000000,0.000000,0.000000}%
\pgfsetstrokecolor{currentstroke}%
\pgfsetstrokeopacity{0.700000}%
\pgfsetdash{}{0pt}%
\pgfpathmoveto{\pgfqpoint{9.445242in}{1.557316in}}%
\pgfpathcurveto{\pgfqpoint{9.450286in}{1.557316in}}{\pgfqpoint{9.455124in}{1.559320in}}{\pgfqpoint{9.458690in}{1.562886in}}%
\pgfpathcurveto{\pgfqpoint{9.462257in}{1.566452in}}{\pgfqpoint{9.464260in}{1.571290in}}{\pgfqpoint{9.464260in}{1.576334in}}%
\pgfpathcurveto{\pgfqpoint{9.464260in}{1.581378in}}{\pgfqpoint{9.462257in}{1.586215in}}{\pgfqpoint{9.458690in}{1.589782in}}%
\pgfpathcurveto{\pgfqpoint{9.455124in}{1.593348in}}{\pgfqpoint{9.450286in}{1.595352in}}{\pgfqpoint{9.445242in}{1.595352in}}%
\pgfpathcurveto{\pgfqpoint{9.440199in}{1.595352in}}{\pgfqpoint{9.435361in}{1.593348in}}{\pgfqpoint{9.431794in}{1.589782in}}%
\pgfpathcurveto{\pgfqpoint{9.428228in}{1.586215in}}{\pgfqpoint{9.426224in}{1.581378in}}{\pgfqpoint{9.426224in}{1.576334in}}%
\pgfpathcurveto{\pgfqpoint{9.426224in}{1.571290in}}{\pgfqpoint{9.428228in}{1.566452in}}{\pgfqpoint{9.431794in}{1.562886in}}%
\pgfpathcurveto{\pgfqpoint{9.435361in}{1.559320in}}{\pgfqpoint{9.440199in}{1.557316in}}{\pgfqpoint{9.445242in}{1.557316in}}%
\pgfpathclose%
\pgfusepath{fill}%
\end{pgfscope}%
\begin{pgfscope}%
\pgfpathrectangle{\pgfqpoint{6.572727in}{0.474100in}}{\pgfqpoint{4.227273in}{3.318700in}}%
\pgfusepath{clip}%
\pgfsetbuttcap%
\pgfsetroundjoin%
\definecolor{currentfill}{rgb}{0.127568,0.566949,0.550556}%
\pgfsetfillcolor{currentfill}%
\pgfsetfillopacity{0.700000}%
\pgfsetlinewidth{0.000000pt}%
\definecolor{currentstroke}{rgb}{0.000000,0.000000,0.000000}%
\pgfsetstrokecolor{currentstroke}%
\pgfsetstrokeopacity{0.700000}%
\pgfsetdash{}{0pt}%
\pgfpathmoveto{\pgfqpoint{8.011715in}{2.692316in}}%
\pgfpathcurveto{\pgfqpoint{8.016759in}{2.692316in}}{\pgfqpoint{8.021597in}{2.694319in}}{\pgfqpoint{8.025163in}{2.697886in}}%
\pgfpathcurveto{\pgfqpoint{8.028730in}{2.701452in}}{\pgfqpoint{8.030733in}{2.706290in}}{\pgfqpoint{8.030733in}{2.711334in}}%
\pgfpathcurveto{\pgfqpoint{8.030733in}{2.716377in}}{\pgfqpoint{8.028730in}{2.721215in}}{\pgfqpoint{8.025163in}{2.724782in}}%
\pgfpathcurveto{\pgfqpoint{8.021597in}{2.728348in}}{\pgfqpoint{8.016759in}{2.730352in}}{\pgfqpoint{8.011715in}{2.730352in}}%
\pgfpathcurveto{\pgfqpoint{8.006672in}{2.730352in}}{\pgfqpoint{8.001834in}{2.728348in}}{\pgfqpoint{7.998267in}{2.724782in}}%
\pgfpathcurveto{\pgfqpoint{7.994701in}{2.721215in}}{\pgfqpoint{7.992697in}{2.716377in}}{\pgfqpoint{7.992697in}{2.711334in}}%
\pgfpathcurveto{\pgfqpoint{7.992697in}{2.706290in}}{\pgfqpoint{7.994701in}{2.701452in}}{\pgfqpoint{7.998267in}{2.697886in}}%
\pgfpathcurveto{\pgfqpoint{8.001834in}{2.694319in}}{\pgfqpoint{8.006672in}{2.692316in}}{\pgfqpoint{8.011715in}{2.692316in}}%
\pgfpathclose%
\pgfusepath{fill}%
\end{pgfscope}%
\begin{pgfscope}%
\pgfpathrectangle{\pgfqpoint{6.572727in}{0.474100in}}{\pgfqpoint{4.227273in}{3.318700in}}%
\pgfusepath{clip}%
\pgfsetbuttcap%
\pgfsetroundjoin%
\definecolor{currentfill}{rgb}{0.127568,0.566949,0.550556}%
\pgfsetfillcolor{currentfill}%
\pgfsetfillopacity{0.700000}%
\pgfsetlinewidth{0.000000pt}%
\definecolor{currentstroke}{rgb}{0.000000,0.000000,0.000000}%
\pgfsetstrokecolor{currentstroke}%
\pgfsetstrokeopacity{0.700000}%
\pgfsetdash{}{0pt}%
\pgfpathmoveto{\pgfqpoint{8.222164in}{2.777972in}}%
\pgfpathcurveto{\pgfqpoint{8.227207in}{2.777972in}}{\pgfqpoint{8.232045in}{2.779976in}}{\pgfqpoint{8.235612in}{2.783542in}}%
\pgfpathcurveto{\pgfqpoint{8.239178in}{2.787108in}}{\pgfqpoint{8.241182in}{2.791946in}}{\pgfqpoint{8.241182in}{2.796990in}}%
\pgfpathcurveto{\pgfqpoint{8.241182in}{2.802034in}}{\pgfqpoint{8.239178in}{2.806871in}}{\pgfqpoint{8.235612in}{2.810438in}}%
\pgfpathcurveto{\pgfqpoint{8.232045in}{2.814004in}}{\pgfqpoint{8.227207in}{2.816008in}}{\pgfqpoint{8.222164in}{2.816008in}}%
\pgfpathcurveto{\pgfqpoint{8.217120in}{2.816008in}}{\pgfqpoint{8.212282in}{2.814004in}}{\pgfqpoint{8.208716in}{2.810438in}}%
\pgfpathcurveto{\pgfqpoint{8.205149in}{2.806871in}}{\pgfqpoint{8.203146in}{2.802034in}}{\pgfqpoint{8.203146in}{2.796990in}}%
\pgfpathcurveto{\pgfqpoint{8.203146in}{2.791946in}}{\pgfqpoint{8.205149in}{2.787108in}}{\pgfqpoint{8.208716in}{2.783542in}}%
\pgfpathcurveto{\pgfqpoint{8.212282in}{2.779976in}}{\pgfqpoint{8.217120in}{2.777972in}}{\pgfqpoint{8.222164in}{2.777972in}}%
\pgfpathclose%
\pgfusepath{fill}%
\end{pgfscope}%
\begin{pgfscope}%
\pgfpathrectangle{\pgfqpoint{6.572727in}{0.474100in}}{\pgfqpoint{4.227273in}{3.318700in}}%
\pgfusepath{clip}%
\pgfsetbuttcap%
\pgfsetroundjoin%
\definecolor{currentfill}{rgb}{0.127568,0.566949,0.550556}%
\pgfsetfillcolor{currentfill}%
\pgfsetfillopacity{0.700000}%
\pgfsetlinewidth{0.000000pt}%
\definecolor{currentstroke}{rgb}{0.000000,0.000000,0.000000}%
\pgfsetstrokecolor{currentstroke}%
\pgfsetstrokeopacity{0.700000}%
\pgfsetdash{}{0pt}%
\pgfpathmoveto{\pgfqpoint{7.855825in}{1.851719in}}%
\pgfpathcurveto{\pgfqpoint{7.860868in}{1.851719in}}{\pgfqpoint{7.865706in}{1.853722in}}{\pgfqpoint{7.869272in}{1.857289in}}%
\pgfpathcurveto{\pgfqpoint{7.872839in}{1.860855in}}{\pgfqpoint{7.874843in}{1.865693in}}{\pgfqpoint{7.874843in}{1.870737in}}%
\pgfpathcurveto{\pgfqpoint{7.874843in}{1.875780in}}{\pgfqpoint{7.872839in}{1.880618in}}{\pgfqpoint{7.869272in}{1.884185in}}%
\pgfpathcurveto{\pgfqpoint{7.865706in}{1.887751in}}{\pgfqpoint{7.860868in}{1.889755in}}{\pgfqpoint{7.855825in}{1.889755in}}%
\pgfpathcurveto{\pgfqpoint{7.850781in}{1.889755in}}{\pgfqpoint{7.845943in}{1.887751in}}{\pgfqpoint{7.842377in}{1.884185in}}%
\pgfpathcurveto{\pgfqpoint{7.838810in}{1.880618in}}{\pgfqpoint{7.836806in}{1.875780in}}{\pgfqpoint{7.836806in}{1.870737in}}%
\pgfpathcurveto{\pgfqpoint{7.836806in}{1.865693in}}{\pgfqpoint{7.838810in}{1.860855in}}{\pgfqpoint{7.842377in}{1.857289in}}%
\pgfpathcurveto{\pgfqpoint{7.845943in}{1.853722in}}{\pgfqpoint{7.850781in}{1.851719in}}{\pgfqpoint{7.855825in}{1.851719in}}%
\pgfpathclose%
\pgfusepath{fill}%
\end{pgfscope}%
\begin{pgfscope}%
\pgfpathrectangle{\pgfqpoint{6.572727in}{0.474100in}}{\pgfqpoint{4.227273in}{3.318700in}}%
\pgfusepath{clip}%
\pgfsetbuttcap%
\pgfsetroundjoin%
\definecolor{currentfill}{rgb}{0.127568,0.566949,0.550556}%
\pgfsetfillcolor{currentfill}%
\pgfsetfillopacity{0.700000}%
\pgfsetlinewidth{0.000000pt}%
\definecolor{currentstroke}{rgb}{0.000000,0.000000,0.000000}%
\pgfsetstrokecolor{currentstroke}%
\pgfsetstrokeopacity{0.700000}%
\pgfsetdash{}{0pt}%
\pgfpathmoveto{\pgfqpoint{8.054335in}{1.534493in}}%
\pgfpathcurveto{\pgfqpoint{8.059378in}{1.534493in}}{\pgfqpoint{8.064216in}{1.536497in}}{\pgfqpoint{8.067783in}{1.540063in}}%
\pgfpathcurveto{\pgfqpoint{8.071349in}{1.543630in}}{\pgfqpoint{8.073353in}{1.548468in}}{\pgfqpoint{8.073353in}{1.553511in}}%
\pgfpathcurveto{\pgfqpoint{8.073353in}{1.558555in}}{\pgfqpoint{8.071349in}{1.563393in}}{\pgfqpoint{8.067783in}{1.566959in}}%
\pgfpathcurveto{\pgfqpoint{8.064216in}{1.570525in}}{\pgfqpoint{8.059378in}{1.572529in}}{\pgfqpoint{8.054335in}{1.572529in}}%
\pgfpathcurveto{\pgfqpoint{8.049291in}{1.572529in}}{\pgfqpoint{8.044453in}{1.570525in}}{\pgfqpoint{8.040887in}{1.566959in}}%
\pgfpathcurveto{\pgfqpoint{8.037321in}{1.563393in}}{\pgfqpoint{8.035317in}{1.558555in}}{\pgfqpoint{8.035317in}{1.553511in}}%
\pgfpathcurveto{\pgfqpoint{8.035317in}{1.548468in}}{\pgfqpoint{8.037321in}{1.543630in}}{\pgfqpoint{8.040887in}{1.540063in}}%
\pgfpathcurveto{\pgfqpoint{8.044453in}{1.536497in}}{\pgfqpoint{8.049291in}{1.534493in}}{\pgfqpoint{8.054335in}{1.534493in}}%
\pgfpathclose%
\pgfusepath{fill}%
\end{pgfscope}%
\begin{pgfscope}%
\pgfpathrectangle{\pgfqpoint{6.572727in}{0.474100in}}{\pgfqpoint{4.227273in}{3.318700in}}%
\pgfusepath{clip}%
\pgfsetbuttcap%
\pgfsetroundjoin%
\definecolor{currentfill}{rgb}{0.127568,0.566949,0.550556}%
\pgfsetfillcolor{currentfill}%
\pgfsetfillopacity{0.700000}%
\pgfsetlinewidth{0.000000pt}%
\definecolor{currentstroke}{rgb}{0.000000,0.000000,0.000000}%
\pgfsetstrokecolor{currentstroke}%
\pgfsetstrokeopacity{0.700000}%
\pgfsetdash{}{0pt}%
\pgfpathmoveto{\pgfqpoint{8.437098in}{2.756430in}}%
\pgfpathcurveto{\pgfqpoint{8.442142in}{2.756430in}}{\pgfqpoint{8.446979in}{2.758433in}}{\pgfqpoint{8.450546in}{2.762000in}}%
\pgfpathcurveto{\pgfqpoint{8.454112in}{2.765566in}}{\pgfqpoint{8.456116in}{2.770404in}}{\pgfqpoint{8.456116in}{2.775448in}}%
\pgfpathcurveto{\pgfqpoint{8.456116in}{2.780491in}}{\pgfqpoint{8.454112in}{2.785329in}}{\pgfqpoint{8.450546in}{2.788896in}}%
\pgfpathcurveto{\pgfqpoint{8.446979in}{2.792462in}}{\pgfqpoint{8.442142in}{2.794466in}}{\pgfqpoint{8.437098in}{2.794466in}}%
\pgfpathcurveto{\pgfqpoint{8.432054in}{2.794466in}}{\pgfqpoint{8.427217in}{2.792462in}}{\pgfqpoint{8.423650in}{2.788896in}}%
\pgfpathcurveto{\pgfqpoint{8.420084in}{2.785329in}}{\pgfqpoint{8.418080in}{2.780491in}}{\pgfqpoint{8.418080in}{2.775448in}}%
\pgfpathcurveto{\pgfqpoint{8.418080in}{2.770404in}}{\pgfqpoint{8.420084in}{2.765566in}}{\pgfqpoint{8.423650in}{2.762000in}}%
\pgfpathcurveto{\pgfqpoint{8.427217in}{2.758433in}}{\pgfqpoint{8.432054in}{2.756430in}}{\pgfqpoint{8.437098in}{2.756430in}}%
\pgfpathclose%
\pgfusepath{fill}%
\end{pgfscope}%
\begin{pgfscope}%
\pgfpathrectangle{\pgfqpoint{6.572727in}{0.474100in}}{\pgfqpoint{4.227273in}{3.318700in}}%
\pgfusepath{clip}%
\pgfsetbuttcap%
\pgfsetroundjoin%
\definecolor{currentfill}{rgb}{0.127568,0.566949,0.550556}%
\pgfsetfillcolor{currentfill}%
\pgfsetfillopacity{0.700000}%
\pgfsetlinewidth{0.000000pt}%
\definecolor{currentstroke}{rgb}{0.000000,0.000000,0.000000}%
\pgfsetstrokecolor{currentstroke}%
\pgfsetstrokeopacity{0.700000}%
\pgfsetdash{}{0pt}%
\pgfpathmoveto{\pgfqpoint{7.403810in}{2.898286in}}%
\pgfpathcurveto{\pgfqpoint{7.408854in}{2.898286in}}{\pgfqpoint{7.413692in}{2.900290in}}{\pgfqpoint{7.417258in}{2.903856in}}%
\pgfpathcurveto{\pgfqpoint{7.420825in}{2.907423in}}{\pgfqpoint{7.422829in}{2.912261in}}{\pgfqpoint{7.422829in}{2.917304in}}%
\pgfpathcurveto{\pgfqpoint{7.422829in}{2.922348in}}{\pgfqpoint{7.420825in}{2.927186in}}{\pgfqpoint{7.417258in}{2.930752in}}%
\pgfpathcurveto{\pgfqpoint{7.413692in}{2.934318in}}{\pgfqpoint{7.408854in}{2.936322in}}{\pgfqpoint{7.403810in}{2.936322in}}%
\pgfpathcurveto{\pgfqpoint{7.398767in}{2.936322in}}{\pgfqpoint{7.393929in}{2.934318in}}{\pgfqpoint{7.390363in}{2.930752in}}%
\pgfpathcurveto{\pgfqpoint{7.386796in}{2.927186in}}{\pgfqpoint{7.384792in}{2.922348in}}{\pgfqpoint{7.384792in}{2.917304in}}%
\pgfpathcurveto{\pgfqpoint{7.384792in}{2.912261in}}{\pgfqpoint{7.386796in}{2.907423in}}{\pgfqpoint{7.390363in}{2.903856in}}%
\pgfpathcurveto{\pgfqpoint{7.393929in}{2.900290in}}{\pgfqpoint{7.398767in}{2.898286in}}{\pgfqpoint{7.403810in}{2.898286in}}%
\pgfpathclose%
\pgfusepath{fill}%
\end{pgfscope}%
\begin{pgfscope}%
\pgfpathrectangle{\pgfqpoint{6.572727in}{0.474100in}}{\pgfqpoint{4.227273in}{3.318700in}}%
\pgfusepath{clip}%
\pgfsetbuttcap%
\pgfsetroundjoin%
\definecolor{currentfill}{rgb}{0.127568,0.566949,0.550556}%
\pgfsetfillcolor{currentfill}%
\pgfsetfillopacity{0.700000}%
\pgfsetlinewidth{0.000000pt}%
\definecolor{currentstroke}{rgb}{0.000000,0.000000,0.000000}%
\pgfsetstrokecolor{currentstroke}%
\pgfsetstrokeopacity{0.700000}%
\pgfsetdash{}{0pt}%
\pgfpathmoveto{\pgfqpoint{7.807939in}{2.644002in}}%
\pgfpathcurveto{\pgfqpoint{7.812983in}{2.644002in}}{\pgfqpoint{7.817821in}{2.646006in}}{\pgfqpoint{7.821387in}{2.649573in}}%
\pgfpathcurveto{\pgfqpoint{7.824954in}{2.653139in}}{\pgfqpoint{7.826958in}{2.657977in}}{\pgfqpoint{7.826958in}{2.663020in}}%
\pgfpathcurveto{\pgfqpoint{7.826958in}{2.668064in}}{\pgfqpoint{7.824954in}{2.672902in}}{\pgfqpoint{7.821387in}{2.676468in}}%
\pgfpathcurveto{\pgfqpoint{7.817821in}{2.680035in}}{\pgfqpoint{7.812983in}{2.682039in}}{\pgfqpoint{7.807939in}{2.682039in}}%
\pgfpathcurveto{\pgfqpoint{7.802896in}{2.682039in}}{\pgfqpoint{7.798058in}{2.680035in}}{\pgfqpoint{7.794492in}{2.676468in}}%
\pgfpathcurveto{\pgfqpoint{7.790925in}{2.672902in}}{\pgfqpoint{7.788921in}{2.668064in}}{\pgfqpoint{7.788921in}{2.663020in}}%
\pgfpathcurveto{\pgfqpoint{7.788921in}{2.657977in}}{\pgfqpoint{7.790925in}{2.653139in}}{\pgfqpoint{7.794492in}{2.649573in}}%
\pgfpathcurveto{\pgfqpoint{7.798058in}{2.646006in}}{\pgfqpoint{7.802896in}{2.644002in}}{\pgfqpoint{7.807939in}{2.644002in}}%
\pgfpathclose%
\pgfusepath{fill}%
\end{pgfscope}%
\begin{pgfscope}%
\pgfpathrectangle{\pgfqpoint{6.572727in}{0.474100in}}{\pgfqpoint{4.227273in}{3.318700in}}%
\pgfusepath{clip}%
\pgfsetbuttcap%
\pgfsetroundjoin%
\definecolor{currentfill}{rgb}{0.127568,0.566949,0.550556}%
\pgfsetfillcolor{currentfill}%
\pgfsetfillopacity{0.700000}%
\pgfsetlinewidth{0.000000pt}%
\definecolor{currentstroke}{rgb}{0.000000,0.000000,0.000000}%
\pgfsetstrokecolor{currentstroke}%
\pgfsetstrokeopacity{0.700000}%
\pgfsetdash{}{0pt}%
\pgfpathmoveto{\pgfqpoint{8.068671in}{1.512713in}}%
\pgfpathcurveto{\pgfqpoint{8.073715in}{1.512713in}}{\pgfqpoint{8.078553in}{1.514717in}}{\pgfqpoint{8.082119in}{1.518283in}}%
\pgfpathcurveto{\pgfqpoint{8.085686in}{1.521849in}}{\pgfqpoint{8.087689in}{1.526687in}}{\pgfqpoint{8.087689in}{1.531731in}}%
\pgfpathcurveto{\pgfqpoint{8.087689in}{1.536774in}}{\pgfqpoint{8.085686in}{1.541612in}}{\pgfqpoint{8.082119in}{1.545179in}}%
\pgfpathcurveto{\pgfqpoint{8.078553in}{1.548745in}}{\pgfqpoint{8.073715in}{1.550749in}}{\pgfqpoint{8.068671in}{1.550749in}}%
\pgfpathcurveto{\pgfqpoint{8.063628in}{1.550749in}}{\pgfqpoint{8.058790in}{1.548745in}}{\pgfqpoint{8.055223in}{1.545179in}}%
\pgfpathcurveto{\pgfqpoint{8.051657in}{1.541612in}}{\pgfqpoint{8.049653in}{1.536774in}}{\pgfqpoint{8.049653in}{1.531731in}}%
\pgfpathcurveto{\pgfqpoint{8.049653in}{1.526687in}}{\pgfqpoint{8.051657in}{1.521849in}}{\pgfqpoint{8.055223in}{1.518283in}}%
\pgfpathcurveto{\pgfqpoint{8.058790in}{1.514717in}}{\pgfqpoint{8.063628in}{1.512713in}}{\pgfqpoint{8.068671in}{1.512713in}}%
\pgfpathclose%
\pgfusepath{fill}%
\end{pgfscope}%
\begin{pgfscope}%
\pgfpathrectangle{\pgfqpoint{6.572727in}{0.474100in}}{\pgfqpoint{4.227273in}{3.318700in}}%
\pgfusepath{clip}%
\pgfsetbuttcap%
\pgfsetroundjoin%
\definecolor{currentfill}{rgb}{0.993248,0.906157,0.143936}%
\pgfsetfillcolor{currentfill}%
\pgfsetfillopacity{0.700000}%
\pgfsetlinewidth{0.000000pt}%
\definecolor{currentstroke}{rgb}{0.000000,0.000000,0.000000}%
\pgfsetstrokecolor{currentstroke}%
\pgfsetstrokeopacity{0.700000}%
\pgfsetdash{}{0pt}%
\pgfpathmoveto{\pgfqpoint{9.265946in}{2.034746in}}%
\pgfpathcurveto{\pgfqpoint{9.270990in}{2.034746in}}{\pgfqpoint{9.275827in}{2.036749in}}{\pgfqpoint{9.279394in}{2.040316in}}%
\pgfpathcurveto{\pgfqpoint{9.282960in}{2.043882in}}{\pgfqpoint{9.284964in}{2.048720in}}{\pgfqpoint{9.284964in}{2.053764in}}%
\pgfpathcurveto{\pgfqpoint{9.284964in}{2.058807in}}{\pgfqpoint{9.282960in}{2.063645in}}{\pgfqpoint{9.279394in}{2.067212in}}%
\pgfpathcurveto{\pgfqpoint{9.275827in}{2.070778in}}{\pgfqpoint{9.270990in}{2.072782in}}{\pgfqpoint{9.265946in}{2.072782in}}%
\pgfpathcurveto{\pgfqpoint{9.260902in}{2.072782in}}{\pgfqpoint{9.256064in}{2.070778in}}{\pgfqpoint{9.252498in}{2.067212in}}%
\pgfpathcurveto{\pgfqpoint{9.248932in}{2.063645in}}{\pgfqpoint{9.246928in}{2.058807in}}{\pgfqpoint{9.246928in}{2.053764in}}%
\pgfpathcurveto{\pgfqpoint{9.246928in}{2.048720in}}{\pgfqpoint{9.248932in}{2.043882in}}{\pgfqpoint{9.252498in}{2.040316in}}%
\pgfpathcurveto{\pgfqpoint{9.256064in}{2.036749in}}{\pgfqpoint{9.260902in}{2.034746in}}{\pgfqpoint{9.265946in}{2.034746in}}%
\pgfpathclose%
\pgfusepath{fill}%
\end{pgfscope}%
\begin{pgfscope}%
\pgfpathrectangle{\pgfqpoint{6.572727in}{0.474100in}}{\pgfqpoint{4.227273in}{3.318700in}}%
\pgfusepath{clip}%
\pgfsetbuttcap%
\pgfsetroundjoin%
\definecolor{currentfill}{rgb}{0.127568,0.566949,0.550556}%
\pgfsetfillcolor{currentfill}%
\pgfsetfillopacity{0.700000}%
\pgfsetlinewidth{0.000000pt}%
\definecolor{currentstroke}{rgb}{0.000000,0.000000,0.000000}%
\pgfsetstrokecolor{currentstroke}%
\pgfsetstrokeopacity{0.700000}%
\pgfsetdash{}{0pt}%
\pgfpathmoveto{\pgfqpoint{7.660278in}{1.132869in}}%
\pgfpathcurveto{\pgfqpoint{7.665322in}{1.132869in}}{\pgfqpoint{7.670159in}{1.134873in}}{\pgfqpoint{7.673726in}{1.138439in}}%
\pgfpathcurveto{\pgfqpoint{7.677292in}{1.142006in}}{\pgfqpoint{7.679296in}{1.146844in}}{\pgfqpoint{7.679296in}{1.151887in}}%
\pgfpathcurveto{\pgfqpoint{7.679296in}{1.156931in}}{\pgfqpoint{7.677292in}{1.161769in}}{\pgfqpoint{7.673726in}{1.165335in}}%
\pgfpathcurveto{\pgfqpoint{7.670159in}{1.168901in}}{\pgfqpoint{7.665322in}{1.170905in}}{\pgfqpoint{7.660278in}{1.170905in}}%
\pgfpathcurveto{\pgfqpoint{7.655234in}{1.170905in}}{\pgfqpoint{7.650396in}{1.168901in}}{\pgfqpoint{7.646830in}{1.165335in}}%
\pgfpathcurveto{\pgfqpoint{7.643264in}{1.161769in}}{\pgfqpoint{7.641260in}{1.156931in}}{\pgfqpoint{7.641260in}{1.151887in}}%
\pgfpathcurveto{\pgfqpoint{7.641260in}{1.146844in}}{\pgfqpoint{7.643264in}{1.142006in}}{\pgfqpoint{7.646830in}{1.138439in}}%
\pgfpathcurveto{\pgfqpoint{7.650396in}{1.134873in}}{\pgfqpoint{7.655234in}{1.132869in}}{\pgfqpoint{7.660278in}{1.132869in}}%
\pgfpathclose%
\pgfusepath{fill}%
\end{pgfscope}%
\begin{pgfscope}%
\pgfpathrectangle{\pgfqpoint{6.572727in}{0.474100in}}{\pgfqpoint{4.227273in}{3.318700in}}%
\pgfusepath{clip}%
\pgfsetbuttcap%
\pgfsetroundjoin%
\definecolor{currentfill}{rgb}{0.993248,0.906157,0.143936}%
\pgfsetfillcolor{currentfill}%
\pgfsetfillopacity{0.700000}%
\pgfsetlinewidth{0.000000pt}%
\definecolor{currentstroke}{rgb}{0.000000,0.000000,0.000000}%
\pgfsetstrokecolor{currentstroke}%
\pgfsetstrokeopacity{0.700000}%
\pgfsetdash{}{0pt}%
\pgfpathmoveto{\pgfqpoint{9.000927in}{1.801028in}}%
\pgfpathcurveto{\pgfqpoint{9.005970in}{1.801028in}}{\pgfqpoint{9.010808in}{1.803032in}}{\pgfqpoint{9.014375in}{1.806598in}}%
\pgfpathcurveto{\pgfqpoint{9.017941in}{1.810165in}}{\pgfqpoint{9.019945in}{1.815002in}}{\pgfqpoint{9.019945in}{1.820046in}}%
\pgfpathcurveto{\pgfqpoint{9.019945in}{1.825090in}}{\pgfqpoint{9.017941in}{1.829927in}}{\pgfqpoint{9.014375in}{1.833494in}}%
\pgfpathcurveto{\pgfqpoint{9.010808in}{1.837060in}}{\pgfqpoint{9.005970in}{1.839064in}}{\pgfqpoint{9.000927in}{1.839064in}}%
\pgfpathcurveto{\pgfqpoint{8.995883in}{1.839064in}}{\pgfqpoint{8.991045in}{1.837060in}}{\pgfqpoint{8.987479in}{1.833494in}}%
\pgfpathcurveto{\pgfqpoint{8.983912in}{1.829927in}}{\pgfqpoint{8.981908in}{1.825090in}}{\pgfqpoint{8.981908in}{1.820046in}}%
\pgfpathcurveto{\pgfqpoint{8.981908in}{1.815002in}}{\pgfqpoint{8.983912in}{1.810165in}}{\pgfqpoint{8.987479in}{1.806598in}}%
\pgfpathcurveto{\pgfqpoint{8.991045in}{1.803032in}}{\pgfqpoint{8.995883in}{1.801028in}}{\pgfqpoint{9.000927in}{1.801028in}}%
\pgfpathclose%
\pgfusepath{fill}%
\end{pgfscope}%
\begin{pgfscope}%
\pgfpathrectangle{\pgfqpoint{6.572727in}{0.474100in}}{\pgfqpoint{4.227273in}{3.318700in}}%
\pgfusepath{clip}%
\pgfsetbuttcap%
\pgfsetroundjoin%
\definecolor{currentfill}{rgb}{0.127568,0.566949,0.550556}%
\pgfsetfillcolor{currentfill}%
\pgfsetfillopacity{0.700000}%
\pgfsetlinewidth{0.000000pt}%
\definecolor{currentstroke}{rgb}{0.000000,0.000000,0.000000}%
\pgfsetstrokecolor{currentstroke}%
\pgfsetstrokeopacity{0.700000}%
\pgfsetdash{}{0pt}%
\pgfpathmoveto{\pgfqpoint{8.412896in}{2.490488in}}%
\pgfpathcurveto{\pgfqpoint{8.417940in}{2.490488in}}{\pgfqpoint{8.422778in}{2.492492in}}{\pgfqpoint{8.426344in}{2.496058in}}%
\pgfpathcurveto{\pgfqpoint{8.429910in}{2.499625in}}{\pgfqpoint{8.431914in}{2.504462in}}{\pgfqpoint{8.431914in}{2.509506in}}%
\pgfpathcurveto{\pgfqpoint{8.431914in}{2.514550in}}{\pgfqpoint{8.429910in}{2.519388in}}{\pgfqpoint{8.426344in}{2.522954in}}%
\pgfpathcurveto{\pgfqpoint{8.422778in}{2.526520in}}{\pgfqpoint{8.417940in}{2.528524in}}{\pgfqpoint{8.412896in}{2.528524in}}%
\pgfpathcurveto{\pgfqpoint{8.407852in}{2.528524in}}{\pgfqpoint{8.403015in}{2.526520in}}{\pgfqpoint{8.399448in}{2.522954in}}%
\pgfpathcurveto{\pgfqpoint{8.395882in}{2.519388in}}{\pgfqpoint{8.393878in}{2.514550in}}{\pgfqpoint{8.393878in}{2.509506in}}%
\pgfpathcurveto{\pgfqpoint{8.393878in}{2.504462in}}{\pgfqpoint{8.395882in}{2.499625in}}{\pgfqpoint{8.399448in}{2.496058in}}%
\pgfpathcurveto{\pgfqpoint{8.403015in}{2.492492in}}{\pgfqpoint{8.407852in}{2.490488in}}{\pgfqpoint{8.412896in}{2.490488in}}%
\pgfpathclose%
\pgfusepath{fill}%
\end{pgfscope}%
\begin{pgfscope}%
\pgfpathrectangle{\pgfqpoint{6.572727in}{0.474100in}}{\pgfqpoint{4.227273in}{3.318700in}}%
\pgfusepath{clip}%
\pgfsetbuttcap%
\pgfsetroundjoin%
\definecolor{currentfill}{rgb}{0.127568,0.566949,0.550556}%
\pgfsetfillcolor{currentfill}%
\pgfsetfillopacity{0.700000}%
\pgfsetlinewidth{0.000000pt}%
\definecolor{currentstroke}{rgb}{0.000000,0.000000,0.000000}%
\pgfsetstrokecolor{currentstroke}%
\pgfsetstrokeopacity{0.700000}%
\pgfsetdash{}{0pt}%
\pgfpathmoveto{\pgfqpoint{8.091446in}{1.666130in}}%
\pgfpathcurveto{\pgfqpoint{8.096490in}{1.666130in}}{\pgfqpoint{8.101328in}{1.668134in}}{\pgfqpoint{8.104894in}{1.671700in}}%
\pgfpathcurveto{\pgfqpoint{8.108461in}{1.675267in}}{\pgfqpoint{8.110464in}{1.680104in}}{\pgfqpoint{8.110464in}{1.685148in}}%
\pgfpathcurveto{\pgfqpoint{8.110464in}{1.690192in}}{\pgfqpoint{8.108461in}{1.695029in}}{\pgfqpoint{8.104894in}{1.698596in}}%
\pgfpathcurveto{\pgfqpoint{8.101328in}{1.702162in}}{\pgfqpoint{8.096490in}{1.704166in}}{\pgfqpoint{8.091446in}{1.704166in}}%
\pgfpathcurveto{\pgfqpoint{8.086403in}{1.704166in}}{\pgfqpoint{8.081565in}{1.702162in}}{\pgfqpoint{8.077998in}{1.698596in}}%
\pgfpathcurveto{\pgfqpoint{8.074432in}{1.695029in}}{\pgfqpoint{8.072428in}{1.690192in}}{\pgfqpoint{8.072428in}{1.685148in}}%
\pgfpathcurveto{\pgfqpoint{8.072428in}{1.680104in}}{\pgfqpoint{8.074432in}{1.675267in}}{\pgfqpoint{8.077998in}{1.671700in}}%
\pgfpathcurveto{\pgfqpoint{8.081565in}{1.668134in}}{\pgfqpoint{8.086403in}{1.666130in}}{\pgfqpoint{8.091446in}{1.666130in}}%
\pgfpathclose%
\pgfusepath{fill}%
\end{pgfscope}%
\begin{pgfscope}%
\pgfpathrectangle{\pgfqpoint{6.572727in}{0.474100in}}{\pgfqpoint{4.227273in}{3.318700in}}%
\pgfusepath{clip}%
\pgfsetbuttcap%
\pgfsetroundjoin%
\definecolor{currentfill}{rgb}{0.127568,0.566949,0.550556}%
\pgfsetfillcolor{currentfill}%
\pgfsetfillopacity{0.700000}%
\pgfsetlinewidth{0.000000pt}%
\definecolor{currentstroke}{rgb}{0.000000,0.000000,0.000000}%
\pgfsetstrokecolor{currentstroke}%
\pgfsetstrokeopacity{0.700000}%
\pgfsetdash{}{0pt}%
\pgfpathmoveto{\pgfqpoint{8.080122in}{2.596210in}}%
\pgfpathcurveto{\pgfqpoint{8.085166in}{2.596210in}}{\pgfqpoint{8.090004in}{2.598214in}}{\pgfqpoint{8.093570in}{2.601780in}}%
\pgfpathcurveto{\pgfqpoint{8.097137in}{2.605347in}}{\pgfqpoint{8.099140in}{2.610184in}}{\pgfqpoint{8.099140in}{2.615228in}}%
\pgfpathcurveto{\pgfqpoint{8.099140in}{2.620272in}}{\pgfqpoint{8.097137in}{2.625110in}}{\pgfqpoint{8.093570in}{2.628676in}}%
\pgfpathcurveto{\pgfqpoint{8.090004in}{2.632242in}}{\pgfqpoint{8.085166in}{2.634246in}}{\pgfqpoint{8.080122in}{2.634246in}}%
\pgfpathcurveto{\pgfqpoint{8.075079in}{2.634246in}}{\pgfqpoint{8.070241in}{2.632242in}}{\pgfqpoint{8.066674in}{2.628676in}}%
\pgfpathcurveto{\pgfqpoint{8.063108in}{2.625110in}}{\pgfqpoint{8.061104in}{2.620272in}}{\pgfqpoint{8.061104in}{2.615228in}}%
\pgfpathcurveto{\pgfqpoint{8.061104in}{2.610184in}}{\pgfqpoint{8.063108in}{2.605347in}}{\pgfqpoint{8.066674in}{2.601780in}}%
\pgfpathcurveto{\pgfqpoint{8.070241in}{2.598214in}}{\pgfqpoint{8.075079in}{2.596210in}}{\pgfqpoint{8.080122in}{2.596210in}}%
\pgfpathclose%
\pgfusepath{fill}%
\end{pgfscope}%
\begin{pgfscope}%
\pgfpathrectangle{\pgfqpoint{6.572727in}{0.474100in}}{\pgfqpoint{4.227273in}{3.318700in}}%
\pgfusepath{clip}%
\pgfsetbuttcap%
\pgfsetroundjoin%
\definecolor{currentfill}{rgb}{0.127568,0.566949,0.550556}%
\pgfsetfillcolor{currentfill}%
\pgfsetfillopacity{0.700000}%
\pgfsetlinewidth{0.000000pt}%
\definecolor{currentstroke}{rgb}{0.000000,0.000000,0.000000}%
\pgfsetstrokecolor{currentstroke}%
\pgfsetstrokeopacity{0.700000}%
\pgfsetdash{}{0pt}%
\pgfpathmoveto{\pgfqpoint{7.656938in}{2.234992in}}%
\pgfpathcurveto{\pgfqpoint{7.661981in}{2.234992in}}{\pgfqpoint{7.666819in}{2.236996in}}{\pgfqpoint{7.670386in}{2.240562in}}%
\pgfpathcurveto{\pgfqpoint{7.673952in}{2.244129in}}{\pgfqpoint{7.675956in}{2.248966in}}{\pgfqpoint{7.675956in}{2.254010in}}%
\pgfpathcurveto{\pgfqpoint{7.675956in}{2.259054in}}{\pgfqpoint{7.673952in}{2.263891in}}{\pgfqpoint{7.670386in}{2.267458in}}%
\pgfpathcurveto{\pgfqpoint{7.666819in}{2.271024in}}{\pgfqpoint{7.661981in}{2.273028in}}{\pgfqpoint{7.656938in}{2.273028in}}%
\pgfpathcurveto{\pgfqpoint{7.651894in}{2.273028in}}{\pgfqpoint{7.647056in}{2.271024in}}{\pgfqpoint{7.643490in}{2.267458in}}%
\pgfpathcurveto{\pgfqpoint{7.639923in}{2.263891in}}{\pgfqpoint{7.637920in}{2.259054in}}{\pgfqpoint{7.637920in}{2.254010in}}%
\pgfpathcurveto{\pgfqpoint{7.637920in}{2.248966in}}{\pgfqpoint{7.639923in}{2.244129in}}{\pgfqpoint{7.643490in}{2.240562in}}%
\pgfpathcurveto{\pgfqpoint{7.647056in}{2.236996in}}{\pgfqpoint{7.651894in}{2.234992in}}{\pgfqpoint{7.656938in}{2.234992in}}%
\pgfpathclose%
\pgfusepath{fill}%
\end{pgfscope}%
\begin{pgfscope}%
\pgfpathrectangle{\pgfqpoint{6.572727in}{0.474100in}}{\pgfqpoint{4.227273in}{3.318700in}}%
\pgfusepath{clip}%
\pgfsetbuttcap%
\pgfsetroundjoin%
\definecolor{currentfill}{rgb}{0.127568,0.566949,0.550556}%
\pgfsetfillcolor{currentfill}%
\pgfsetfillopacity{0.700000}%
\pgfsetlinewidth{0.000000pt}%
\definecolor{currentstroke}{rgb}{0.000000,0.000000,0.000000}%
\pgfsetstrokecolor{currentstroke}%
\pgfsetstrokeopacity{0.700000}%
\pgfsetdash{}{0pt}%
\pgfpathmoveto{\pgfqpoint{7.769723in}{2.903506in}}%
\pgfpathcurveto{\pgfqpoint{7.774767in}{2.903506in}}{\pgfqpoint{7.779605in}{2.905510in}}{\pgfqpoint{7.783171in}{2.909076in}}%
\pgfpathcurveto{\pgfqpoint{7.786738in}{2.912643in}}{\pgfqpoint{7.788742in}{2.917480in}}{\pgfqpoint{7.788742in}{2.922524in}}%
\pgfpathcurveto{\pgfqpoint{7.788742in}{2.927568in}}{\pgfqpoint{7.786738in}{2.932406in}}{\pgfqpoint{7.783171in}{2.935972in}}%
\pgfpathcurveto{\pgfqpoint{7.779605in}{2.939538in}}{\pgfqpoint{7.774767in}{2.941542in}}{\pgfqpoint{7.769723in}{2.941542in}}%
\pgfpathcurveto{\pgfqpoint{7.764680in}{2.941542in}}{\pgfqpoint{7.759842in}{2.939538in}}{\pgfqpoint{7.756276in}{2.935972in}}%
\pgfpathcurveto{\pgfqpoint{7.752709in}{2.932406in}}{\pgfqpoint{7.750705in}{2.927568in}}{\pgfqpoint{7.750705in}{2.922524in}}%
\pgfpathcurveto{\pgfqpoint{7.750705in}{2.917480in}}{\pgfqpoint{7.752709in}{2.912643in}}{\pgfqpoint{7.756276in}{2.909076in}}%
\pgfpathcurveto{\pgfqpoint{7.759842in}{2.905510in}}{\pgfqpoint{7.764680in}{2.903506in}}{\pgfqpoint{7.769723in}{2.903506in}}%
\pgfpathclose%
\pgfusepath{fill}%
\end{pgfscope}%
\begin{pgfscope}%
\pgfpathrectangle{\pgfqpoint{6.572727in}{0.474100in}}{\pgfqpoint{4.227273in}{3.318700in}}%
\pgfusepath{clip}%
\pgfsetbuttcap%
\pgfsetroundjoin%
\definecolor{currentfill}{rgb}{0.127568,0.566949,0.550556}%
\pgfsetfillcolor{currentfill}%
\pgfsetfillopacity{0.700000}%
\pgfsetlinewidth{0.000000pt}%
\definecolor{currentstroke}{rgb}{0.000000,0.000000,0.000000}%
\pgfsetstrokecolor{currentstroke}%
\pgfsetstrokeopacity{0.700000}%
\pgfsetdash{}{0pt}%
\pgfpathmoveto{\pgfqpoint{8.519379in}{2.415262in}}%
\pgfpathcurveto{\pgfqpoint{8.524423in}{2.415262in}}{\pgfqpoint{8.529260in}{2.417265in}}{\pgfqpoint{8.532827in}{2.420832in}}%
\pgfpathcurveto{\pgfqpoint{8.536393in}{2.424398in}}{\pgfqpoint{8.538397in}{2.429236in}}{\pgfqpoint{8.538397in}{2.434280in}}%
\pgfpathcurveto{\pgfqpoint{8.538397in}{2.439323in}}{\pgfqpoint{8.536393in}{2.444161in}}{\pgfqpoint{8.532827in}{2.447728in}}%
\pgfpathcurveto{\pgfqpoint{8.529260in}{2.451294in}}{\pgfqpoint{8.524423in}{2.453298in}}{\pgfqpoint{8.519379in}{2.453298in}}%
\pgfpathcurveto{\pgfqpoint{8.514335in}{2.453298in}}{\pgfqpoint{8.509497in}{2.451294in}}{\pgfqpoint{8.505931in}{2.447728in}}%
\pgfpathcurveto{\pgfqpoint{8.502365in}{2.444161in}}{\pgfqpoint{8.500361in}{2.439323in}}{\pgfqpoint{8.500361in}{2.434280in}}%
\pgfpathcurveto{\pgfqpoint{8.500361in}{2.429236in}}{\pgfqpoint{8.502365in}{2.424398in}}{\pgfqpoint{8.505931in}{2.420832in}}%
\pgfpathcurveto{\pgfqpoint{8.509497in}{2.417265in}}{\pgfqpoint{8.514335in}{2.415262in}}{\pgfqpoint{8.519379in}{2.415262in}}%
\pgfpathclose%
\pgfusepath{fill}%
\end{pgfscope}%
\begin{pgfscope}%
\pgfpathrectangle{\pgfqpoint{6.572727in}{0.474100in}}{\pgfqpoint{4.227273in}{3.318700in}}%
\pgfusepath{clip}%
\pgfsetbuttcap%
\pgfsetroundjoin%
\definecolor{currentfill}{rgb}{0.127568,0.566949,0.550556}%
\pgfsetfillcolor{currentfill}%
\pgfsetfillopacity{0.700000}%
\pgfsetlinewidth{0.000000pt}%
\definecolor{currentstroke}{rgb}{0.000000,0.000000,0.000000}%
\pgfsetstrokecolor{currentstroke}%
\pgfsetstrokeopacity{0.700000}%
\pgfsetdash{}{0pt}%
\pgfpathmoveto{\pgfqpoint{8.308376in}{1.689275in}}%
\pgfpathcurveto{\pgfqpoint{8.313419in}{1.689275in}}{\pgfqpoint{8.318257in}{1.691279in}}{\pgfqpoint{8.321824in}{1.694845in}}%
\pgfpathcurveto{\pgfqpoint{8.325390in}{1.698411in}}{\pgfqpoint{8.327394in}{1.703249in}}{\pgfqpoint{8.327394in}{1.708293in}}%
\pgfpathcurveto{\pgfqpoint{8.327394in}{1.713336in}}{\pgfqpoint{8.325390in}{1.718174in}}{\pgfqpoint{8.321824in}{1.721741in}}%
\pgfpathcurveto{\pgfqpoint{8.318257in}{1.725307in}}{\pgfqpoint{8.313419in}{1.727311in}}{\pgfqpoint{8.308376in}{1.727311in}}%
\pgfpathcurveto{\pgfqpoint{8.303332in}{1.727311in}}{\pgfqpoint{8.298494in}{1.725307in}}{\pgfqpoint{8.294928in}{1.721741in}}%
\pgfpathcurveto{\pgfqpoint{8.291362in}{1.718174in}}{\pgfqpoint{8.289358in}{1.713336in}}{\pgfqpoint{8.289358in}{1.708293in}}%
\pgfpathcurveto{\pgfqpoint{8.289358in}{1.703249in}}{\pgfqpoint{8.291362in}{1.698411in}}{\pgfqpoint{8.294928in}{1.694845in}}%
\pgfpathcurveto{\pgfqpoint{8.298494in}{1.691279in}}{\pgfqpoint{8.303332in}{1.689275in}}{\pgfqpoint{8.308376in}{1.689275in}}%
\pgfpathclose%
\pgfusepath{fill}%
\end{pgfscope}%
\begin{pgfscope}%
\pgfpathrectangle{\pgfqpoint{6.572727in}{0.474100in}}{\pgfqpoint{4.227273in}{3.318700in}}%
\pgfusepath{clip}%
\pgfsetbuttcap%
\pgfsetroundjoin%
\definecolor{currentfill}{rgb}{0.993248,0.906157,0.143936}%
\pgfsetfillcolor{currentfill}%
\pgfsetfillopacity{0.700000}%
\pgfsetlinewidth{0.000000pt}%
\definecolor{currentstroke}{rgb}{0.000000,0.000000,0.000000}%
\pgfsetstrokecolor{currentstroke}%
\pgfsetstrokeopacity{0.700000}%
\pgfsetdash{}{0pt}%
\pgfpathmoveto{\pgfqpoint{8.950710in}{1.619999in}}%
\pgfpathcurveto{\pgfqpoint{8.955754in}{1.619999in}}{\pgfqpoint{8.960592in}{1.622003in}}{\pgfqpoint{8.964158in}{1.625569in}}%
\pgfpathcurveto{\pgfqpoint{8.967724in}{1.629136in}}{\pgfqpoint{8.969728in}{1.633974in}}{\pgfqpoint{8.969728in}{1.639017in}}%
\pgfpathcurveto{\pgfqpoint{8.969728in}{1.644061in}}{\pgfqpoint{8.967724in}{1.648899in}}{\pgfqpoint{8.964158in}{1.652465in}}%
\pgfpathcurveto{\pgfqpoint{8.960592in}{1.656032in}}{\pgfqpoint{8.955754in}{1.658035in}}{\pgfqpoint{8.950710in}{1.658035in}}%
\pgfpathcurveto{\pgfqpoint{8.945666in}{1.658035in}}{\pgfqpoint{8.940829in}{1.656032in}}{\pgfqpoint{8.937262in}{1.652465in}}%
\pgfpathcurveto{\pgfqpoint{8.933696in}{1.648899in}}{\pgfqpoint{8.931692in}{1.644061in}}{\pgfqpoint{8.931692in}{1.639017in}}%
\pgfpathcurveto{\pgfqpoint{8.931692in}{1.633974in}}{\pgfqpoint{8.933696in}{1.629136in}}{\pgfqpoint{8.937262in}{1.625569in}}%
\pgfpathcurveto{\pgfqpoint{8.940829in}{1.622003in}}{\pgfqpoint{8.945666in}{1.619999in}}{\pgfqpoint{8.950710in}{1.619999in}}%
\pgfpathclose%
\pgfusepath{fill}%
\end{pgfscope}%
\begin{pgfscope}%
\pgfpathrectangle{\pgfqpoint{6.572727in}{0.474100in}}{\pgfqpoint{4.227273in}{3.318700in}}%
\pgfusepath{clip}%
\pgfsetbuttcap%
\pgfsetroundjoin%
\definecolor{currentfill}{rgb}{0.127568,0.566949,0.550556}%
\pgfsetfillcolor{currentfill}%
\pgfsetfillopacity{0.700000}%
\pgfsetlinewidth{0.000000pt}%
\definecolor{currentstroke}{rgb}{0.000000,0.000000,0.000000}%
\pgfsetstrokecolor{currentstroke}%
\pgfsetstrokeopacity{0.700000}%
\pgfsetdash{}{0pt}%
\pgfpathmoveto{\pgfqpoint{8.641015in}{1.955774in}}%
\pgfpathcurveto{\pgfqpoint{8.646058in}{1.955774in}}{\pgfqpoint{8.650896in}{1.957778in}}{\pgfqpoint{8.654463in}{1.961344in}}%
\pgfpathcurveto{\pgfqpoint{8.658029in}{1.964911in}}{\pgfqpoint{8.660033in}{1.969748in}}{\pgfqpoint{8.660033in}{1.974792in}}%
\pgfpathcurveto{\pgfqpoint{8.660033in}{1.979836in}}{\pgfqpoint{8.658029in}{1.984674in}}{\pgfqpoint{8.654463in}{1.988240in}}%
\pgfpathcurveto{\pgfqpoint{8.650896in}{1.991806in}}{\pgfqpoint{8.646058in}{1.993810in}}{\pgfqpoint{8.641015in}{1.993810in}}%
\pgfpathcurveto{\pgfqpoint{8.635971in}{1.993810in}}{\pgfqpoint{8.631133in}{1.991806in}}{\pgfqpoint{8.627567in}{1.988240in}}%
\pgfpathcurveto{\pgfqpoint{8.624000in}{1.984674in}}{\pgfqpoint{8.621997in}{1.979836in}}{\pgfqpoint{8.621997in}{1.974792in}}%
\pgfpathcurveto{\pgfqpoint{8.621997in}{1.969748in}}{\pgfqpoint{8.624000in}{1.964911in}}{\pgfqpoint{8.627567in}{1.961344in}}%
\pgfpathcurveto{\pgfqpoint{8.631133in}{1.957778in}}{\pgfqpoint{8.635971in}{1.955774in}}{\pgfqpoint{8.641015in}{1.955774in}}%
\pgfpathclose%
\pgfusepath{fill}%
\end{pgfscope}%
\begin{pgfscope}%
\pgfpathrectangle{\pgfqpoint{6.572727in}{0.474100in}}{\pgfqpoint{4.227273in}{3.318700in}}%
\pgfusepath{clip}%
\pgfsetbuttcap%
\pgfsetroundjoin%
\definecolor{currentfill}{rgb}{0.993248,0.906157,0.143936}%
\pgfsetfillcolor{currentfill}%
\pgfsetfillopacity{0.700000}%
\pgfsetlinewidth{0.000000pt}%
\definecolor{currentstroke}{rgb}{0.000000,0.000000,0.000000}%
\pgfsetstrokecolor{currentstroke}%
\pgfsetstrokeopacity{0.700000}%
\pgfsetdash{}{0pt}%
\pgfpathmoveto{\pgfqpoint{9.948102in}{2.050656in}}%
\pgfpathcurveto{\pgfqpoint{9.953145in}{2.050656in}}{\pgfqpoint{9.957983in}{2.052660in}}{\pgfqpoint{9.961550in}{2.056227in}}%
\pgfpathcurveto{\pgfqpoint{9.965116in}{2.059793in}}{\pgfqpoint{9.967120in}{2.064631in}}{\pgfqpoint{9.967120in}{2.069675in}}%
\pgfpathcurveto{\pgfqpoint{9.967120in}{2.074718in}}{\pgfqpoint{9.965116in}{2.079556in}}{\pgfqpoint{9.961550in}{2.083122in}}%
\pgfpathcurveto{\pgfqpoint{9.957983in}{2.086689in}}{\pgfqpoint{9.953145in}{2.088693in}}{\pgfqpoint{9.948102in}{2.088693in}}%
\pgfpathcurveto{\pgfqpoint{9.943058in}{2.088693in}}{\pgfqpoint{9.938220in}{2.086689in}}{\pgfqpoint{9.934654in}{2.083122in}}%
\pgfpathcurveto{\pgfqpoint{9.931088in}{2.079556in}}{\pgfqpoint{9.929084in}{2.074718in}}{\pgfqpoint{9.929084in}{2.069675in}}%
\pgfpathcurveto{\pgfqpoint{9.929084in}{2.064631in}}{\pgfqpoint{9.931088in}{2.059793in}}{\pgfqpoint{9.934654in}{2.056227in}}%
\pgfpathcurveto{\pgfqpoint{9.938220in}{2.052660in}}{\pgfqpoint{9.943058in}{2.050656in}}{\pgfqpoint{9.948102in}{2.050656in}}%
\pgfpathclose%
\pgfusepath{fill}%
\end{pgfscope}%
\begin{pgfscope}%
\pgfpathrectangle{\pgfqpoint{6.572727in}{0.474100in}}{\pgfqpoint{4.227273in}{3.318700in}}%
\pgfusepath{clip}%
\pgfsetbuttcap%
\pgfsetroundjoin%
\definecolor{currentfill}{rgb}{0.127568,0.566949,0.550556}%
\pgfsetfillcolor{currentfill}%
\pgfsetfillopacity{0.700000}%
\pgfsetlinewidth{0.000000pt}%
\definecolor{currentstroke}{rgb}{0.000000,0.000000,0.000000}%
\pgfsetstrokecolor{currentstroke}%
\pgfsetstrokeopacity{0.700000}%
\pgfsetdash{}{0pt}%
\pgfpathmoveto{\pgfqpoint{8.229313in}{2.717752in}}%
\pgfpathcurveto{\pgfqpoint{8.234357in}{2.717752in}}{\pgfqpoint{8.239194in}{2.719756in}}{\pgfqpoint{8.242761in}{2.723322in}}%
\pgfpathcurveto{\pgfqpoint{8.246327in}{2.726889in}}{\pgfqpoint{8.248331in}{2.731726in}}{\pgfqpoint{8.248331in}{2.736770in}}%
\pgfpathcurveto{\pgfqpoint{8.248331in}{2.741814in}}{\pgfqpoint{8.246327in}{2.746651in}}{\pgfqpoint{8.242761in}{2.750218in}}%
\pgfpathcurveto{\pgfqpoint{8.239194in}{2.753784in}}{\pgfqpoint{8.234357in}{2.755788in}}{\pgfqpoint{8.229313in}{2.755788in}}%
\pgfpathcurveto{\pgfqpoint{8.224269in}{2.755788in}}{\pgfqpoint{8.219431in}{2.753784in}}{\pgfqpoint{8.215865in}{2.750218in}}%
\pgfpathcurveto{\pgfqpoint{8.212299in}{2.746651in}}{\pgfqpoint{8.210295in}{2.741814in}}{\pgfqpoint{8.210295in}{2.736770in}}%
\pgfpathcurveto{\pgfqpoint{8.210295in}{2.731726in}}{\pgfqpoint{8.212299in}{2.726889in}}{\pgfqpoint{8.215865in}{2.723322in}}%
\pgfpathcurveto{\pgfqpoint{8.219431in}{2.719756in}}{\pgfqpoint{8.224269in}{2.717752in}}{\pgfqpoint{8.229313in}{2.717752in}}%
\pgfpathclose%
\pgfusepath{fill}%
\end{pgfscope}%
\begin{pgfscope}%
\pgfpathrectangle{\pgfqpoint{6.572727in}{0.474100in}}{\pgfqpoint{4.227273in}{3.318700in}}%
\pgfusepath{clip}%
\pgfsetbuttcap%
\pgfsetroundjoin%
\definecolor{currentfill}{rgb}{0.127568,0.566949,0.550556}%
\pgfsetfillcolor{currentfill}%
\pgfsetfillopacity{0.700000}%
\pgfsetlinewidth{0.000000pt}%
\definecolor{currentstroke}{rgb}{0.000000,0.000000,0.000000}%
\pgfsetstrokecolor{currentstroke}%
\pgfsetstrokeopacity{0.700000}%
\pgfsetdash{}{0pt}%
\pgfpathmoveto{\pgfqpoint{8.259005in}{2.258402in}}%
\pgfpathcurveto{\pgfqpoint{8.264049in}{2.258402in}}{\pgfqpoint{8.268887in}{2.260406in}}{\pgfqpoint{8.272453in}{2.263973in}}%
\pgfpathcurveto{\pgfqpoint{8.276019in}{2.267539in}}{\pgfqpoint{8.278023in}{2.272377in}}{\pgfqpoint{8.278023in}{2.277421in}}%
\pgfpathcurveto{\pgfqpoint{8.278023in}{2.282464in}}{\pgfqpoint{8.276019in}{2.287302in}}{\pgfqpoint{8.272453in}{2.290868in}}%
\pgfpathcurveto{\pgfqpoint{8.268887in}{2.294435in}}{\pgfqpoint{8.264049in}{2.296439in}}{\pgfqpoint{8.259005in}{2.296439in}}%
\pgfpathcurveto{\pgfqpoint{8.253961in}{2.296439in}}{\pgfqpoint{8.249124in}{2.294435in}}{\pgfqpoint{8.245557in}{2.290868in}}%
\pgfpathcurveto{\pgfqpoint{8.241991in}{2.287302in}}{\pgfqpoint{8.239987in}{2.282464in}}{\pgfqpoint{8.239987in}{2.277421in}}%
\pgfpathcurveto{\pgfqpoint{8.239987in}{2.272377in}}{\pgfqpoint{8.241991in}{2.267539in}}{\pgfqpoint{8.245557in}{2.263973in}}%
\pgfpathcurveto{\pgfqpoint{8.249124in}{2.260406in}}{\pgfqpoint{8.253961in}{2.258402in}}{\pgfqpoint{8.259005in}{2.258402in}}%
\pgfpathclose%
\pgfusepath{fill}%
\end{pgfscope}%
\begin{pgfscope}%
\pgfpathrectangle{\pgfqpoint{6.572727in}{0.474100in}}{\pgfqpoint{4.227273in}{3.318700in}}%
\pgfusepath{clip}%
\pgfsetbuttcap%
\pgfsetroundjoin%
\definecolor{currentfill}{rgb}{0.127568,0.566949,0.550556}%
\pgfsetfillcolor{currentfill}%
\pgfsetfillopacity{0.700000}%
\pgfsetlinewidth{0.000000pt}%
\definecolor{currentstroke}{rgb}{0.000000,0.000000,0.000000}%
\pgfsetstrokecolor{currentstroke}%
\pgfsetstrokeopacity{0.700000}%
\pgfsetdash{}{0pt}%
\pgfpathmoveto{\pgfqpoint{8.883793in}{3.147409in}}%
\pgfpathcurveto{\pgfqpoint{8.888837in}{3.147409in}}{\pgfqpoint{8.893675in}{3.149413in}}{\pgfqpoint{8.897241in}{3.152979in}}%
\pgfpathcurveto{\pgfqpoint{8.900807in}{3.156545in}}{\pgfqpoint{8.902811in}{3.161383in}}{\pgfqpoint{8.902811in}{3.166427in}}%
\pgfpathcurveto{\pgfqpoint{8.902811in}{3.171471in}}{\pgfqpoint{8.900807in}{3.176308in}}{\pgfqpoint{8.897241in}{3.179875in}}%
\pgfpathcurveto{\pgfqpoint{8.893675in}{3.183441in}}{\pgfqpoint{8.888837in}{3.185445in}}{\pgfqpoint{8.883793in}{3.185445in}}%
\pgfpathcurveto{\pgfqpoint{8.878749in}{3.185445in}}{\pgfqpoint{8.873912in}{3.183441in}}{\pgfqpoint{8.870345in}{3.179875in}}%
\pgfpathcurveto{\pgfqpoint{8.866779in}{3.176308in}}{\pgfqpoint{8.864775in}{3.171471in}}{\pgfqpoint{8.864775in}{3.166427in}}%
\pgfpathcurveto{\pgfqpoint{8.864775in}{3.161383in}}{\pgfqpoint{8.866779in}{3.156545in}}{\pgfqpoint{8.870345in}{3.152979in}}%
\pgfpathcurveto{\pgfqpoint{8.873912in}{3.149413in}}{\pgfqpoint{8.878749in}{3.147409in}}{\pgfqpoint{8.883793in}{3.147409in}}%
\pgfpathclose%
\pgfusepath{fill}%
\end{pgfscope}%
\begin{pgfscope}%
\pgfpathrectangle{\pgfqpoint{6.572727in}{0.474100in}}{\pgfqpoint{4.227273in}{3.318700in}}%
\pgfusepath{clip}%
\pgfsetbuttcap%
\pgfsetroundjoin%
\definecolor{currentfill}{rgb}{0.127568,0.566949,0.550556}%
\pgfsetfillcolor{currentfill}%
\pgfsetfillopacity{0.700000}%
\pgfsetlinewidth{0.000000pt}%
\definecolor{currentstroke}{rgb}{0.000000,0.000000,0.000000}%
\pgfsetstrokecolor{currentstroke}%
\pgfsetstrokeopacity{0.700000}%
\pgfsetdash{}{0pt}%
\pgfpathmoveto{\pgfqpoint{7.810273in}{1.561932in}}%
\pgfpathcurveto{\pgfqpoint{7.815317in}{1.561932in}}{\pgfqpoint{7.820155in}{1.563936in}}{\pgfqpoint{7.823721in}{1.567502in}}%
\pgfpathcurveto{\pgfqpoint{7.827288in}{1.571069in}}{\pgfqpoint{7.829292in}{1.575907in}}{\pgfqpoint{7.829292in}{1.580950in}}%
\pgfpathcurveto{\pgfqpoint{7.829292in}{1.585994in}}{\pgfqpoint{7.827288in}{1.590832in}}{\pgfqpoint{7.823721in}{1.594398in}}%
\pgfpathcurveto{\pgfqpoint{7.820155in}{1.597965in}}{\pgfqpoint{7.815317in}{1.599968in}}{\pgfqpoint{7.810273in}{1.599968in}}%
\pgfpathcurveto{\pgfqpoint{7.805230in}{1.599968in}}{\pgfqpoint{7.800392in}{1.597965in}}{\pgfqpoint{7.796826in}{1.594398in}}%
\pgfpathcurveto{\pgfqpoint{7.793259in}{1.590832in}}{\pgfqpoint{7.791255in}{1.585994in}}{\pgfqpoint{7.791255in}{1.580950in}}%
\pgfpathcurveto{\pgfqpoint{7.791255in}{1.575907in}}{\pgfqpoint{7.793259in}{1.571069in}}{\pgfqpoint{7.796826in}{1.567502in}}%
\pgfpathcurveto{\pgfqpoint{7.800392in}{1.563936in}}{\pgfqpoint{7.805230in}{1.561932in}}{\pgfqpoint{7.810273in}{1.561932in}}%
\pgfpathclose%
\pgfusepath{fill}%
\end{pgfscope}%
\begin{pgfscope}%
\pgfpathrectangle{\pgfqpoint{6.572727in}{0.474100in}}{\pgfqpoint{4.227273in}{3.318700in}}%
\pgfusepath{clip}%
\pgfsetbuttcap%
\pgfsetroundjoin%
\definecolor{currentfill}{rgb}{0.127568,0.566949,0.550556}%
\pgfsetfillcolor{currentfill}%
\pgfsetfillopacity{0.700000}%
\pgfsetlinewidth{0.000000pt}%
\definecolor{currentstroke}{rgb}{0.000000,0.000000,0.000000}%
\pgfsetstrokecolor{currentstroke}%
\pgfsetstrokeopacity{0.700000}%
\pgfsetdash{}{0pt}%
\pgfpathmoveto{\pgfqpoint{8.219733in}{3.388496in}}%
\pgfpathcurveto{\pgfqpoint{8.224777in}{3.388496in}}{\pgfqpoint{8.229614in}{3.390500in}}{\pgfqpoint{8.233181in}{3.394067in}}%
\pgfpathcurveto{\pgfqpoint{8.236747in}{3.397633in}}{\pgfqpoint{8.238751in}{3.402471in}}{\pgfqpoint{8.238751in}{3.407515in}}%
\pgfpathcurveto{\pgfqpoint{8.238751in}{3.412558in}}{\pgfqpoint{8.236747in}{3.417396in}}{\pgfqpoint{8.233181in}{3.420962in}}%
\pgfpathcurveto{\pgfqpoint{8.229614in}{3.424529in}}{\pgfqpoint{8.224777in}{3.426533in}}{\pgfqpoint{8.219733in}{3.426533in}}%
\pgfpathcurveto{\pgfqpoint{8.214689in}{3.426533in}}{\pgfqpoint{8.209851in}{3.424529in}}{\pgfqpoint{8.206285in}{3.420962in}}%
\pgfpathcurveto{\pgfqpoint{8.202719in}{3.417396in}}{\pgfqpoint{8.200715in}{3.412558in}}{\pgfqpoint{8.200715in}{3.407515in}}%
\pgfpathcurveto{\pgfqpoint{8.200715in}{3.402471in}}{\pgfqpoint{8.202719in}{3.397633in}}{\pgfqpoint{8.206285in}{3.394067in}}%
\pgfpathcurveto{\pgfqpoint{8.209851in}{3.390500in}}{\pgfqpoint{8.214689in}{3.388496in}}{\pgfqpoint{8.219733in}{3.388496in}}%
\pgfpathclose%
\pgfusepath{fill}%
\end{pgfscope}%
\begin{pgfscope}%
\pgfpathrectangle{\pgfqpoint{6.572727in}{0.474100in}}{\pgfqpoint{4.227273in}{3.318700in}}%
\pgfusepath{clip}%
\pgfsetbuttcap%
\pgfsetroundjoin%
\definecolor{currentfill}{rgb}{0.127568,0.566949,0.550556}%
\pgfsetfillcolor{currentfill}%
\pgfsetfillopacity{0.700000}%
\pgfsetlinewidth{0.000000pt}%
\definecolor{currentstroke}{rgb}{0.000000,0.000000,0.000000}%
\pgfsetstrokecolor{currentstroke}%
\pgfsetstrokeopacity{0.700000}%
\pgfsetdash{}{0pt}%
\pgfpathmoveto{\pgfqpoint{8.327065in}{2.924489in}}%
\pgfpathcurveto{\pgfqpoint{8.332109in}{2.924489in}}{\pgfqpoint{8.336947in}{2.926492in}}{\pgfqpoint{8.340513in}{2.930059in}}%
\pgfpathcurveto{\pgfqpoint{8.344080in}{2.933625in}}{\pgfqpoint{8.346084in}{2.938463in}}{\pgfqpoint{8.346084in}{2.943507in}}%
\pgfpathcurveto{\pgfqpoint{8.346084in}{2.948550in}}{\pgfqpoint{8.344080in}{2.953388in}}{\pgfqpoint{8.340513in}{2.956955in}}%
\pgfpathcurveto{\pgfqpoint{8.336947in}{2.960521in}}{\pgfqpoint{8.332109in}{2.962525in}}{\pgfqpoint{8.327065in}{2.962525in}}%
\pgfpathcurveto{\pgfqpoint{8.322022in}{2.962525in}}{\pgfqpoint{8.317184in}{2.960521in}}{\pgfqpoint{8.313618in}{2.956955in}}%
\pgfpathcurveto{\pgfqpoint{8.310051in}{2.953388in}}{\pgfqpoint{8.308047in}{2.948550in}}{\pgfqpoint{8.308047in}{2.943507in}}%
\pgfpathcurveto{\pgfqpoint{8.308047in}{2.938463in}}{\pgfqpoint{8.310051in}{2.933625in}}{\pgfqpoint{8.313618in}{2.930059in}}%
\pgfpathcurveto{\pgfqpoint{8.317184in}{2.926492in}}{\pgfqpoint{8.322022in}{2.924489in}}{\pgfqpoint{8.327065in}{2.924489in}}%
\pgfpathclose%
\pgfusepath{fill}%
\end{pgfscope}%
\begin{pgfscope}%
\pgfpathrectangle{\pgfqpoint{6.572727in}{0.474100in}}{\pgfqpoint{4.227273in}{3.318700in}}%
\pgfusepath{clip}%
\pgfsetbuttcap%
\pgfsetroundjoin%
\definecolor{currentfill}{rgb}{0.127568,0.566949,0.550556}%
\pgfsetfillcolor{currentfill}%
\pgfsetfillopacity{0.700000}%
\pgfsetlinewidth{0.000000pt}%
\definecolor{currentstroke}{rgb}{0.000000,0.000000,0.000000}%
\pgfsetstrokecolor{currentstroke}%
\pgfsetstrokeopacity{0.700000}%
\pgfsetdash{}{0pt}%
\pgfpathmoveto{\pgfqpoint{7.352729in}{1.174107in}}%
\pgfpathcurveto{\pgfqpoint{7.357773in}{1.174107in}}{\pgfqpoint{7.362610in}{1.176111in}}{\pgfqpoint{7.366177in}{1.179678in}}%
\pgfpathcurveto{\pgfqpoint{7.369743in}{1.183244in}}{\pgfqpoint{7.371747in}{1.188082in}}{\pgfqpoint{7.371747in}{1.193125in}}%
\pgfpathcurveto{\pgfqpoint{7.371747in}{1.198169in}}{\pgfqpoint{7.369743in}{1.203007in}}{\pgfqpoint{7.366177in}{1.206573in}}%
\pgfpathcurveto{\pgfqpoint{7.362610in}{1.210140in}}{\pgfqpoint{7.357773in}{1.212144in}}{\pgfqpoint{7.352729in}{1.212144in}}%
\pgfpathcurveto{\pgfqpoint{7.347685in}{1.212144in}}{\pgfqpoint{7.342847in}{1.210140in}}{\pgfqpoint{7.339281in}{1.206573in}}%
\pgfpathcurveto{\pgfqpoint{7.335715in}{1.203007in}}{\pgfqpoint{7.333711in}{1.198169in}}{\pgfqpoint{7.333711in}{1.193125in}}%
\pgfpathcurveto{\pgfqpoint{7.333711in}{1.188082in}}{\pgfqpoint{7.335715in}{1.183244in}}{\pgfqpoint{7.339281in}{1.179678in}}%
\pgfpathcurveto{\pgfqpoint{7.342847in}{1.176111in}}{\pgfqpoint{7.347685in}{1.174107in}}{\pgfqpoint{7.352729in}{1.174107in}}%
\pgfpathclose%
\pgfusepath{fill}%
\end{pgfscope}%
\begin{pgfscope}%
\pgfpathrectangle{\pgfqpoint{6.572727in}{0.474100in}}{\pgfqpoint{4.227273in}{3.318700in}}%
\pgfusepath{clip}%
\pgfsetbuttcap%
\pgfsetroundjoin%
\definecolor{currentfill}{rgb}{0.127568,0.566949,0.550556}%
\pgfsetfillcolor{currentfill}%
\pgfsetfillopacity{0.700000}%
\pgfsetlinewidth{0.000000pt}%
\definecolor{currentstroke}{rgb}{0.000000,0.000000,0.000000}%
\pgfsetstrokecolor{currentstroke}%
\pgfsetstrokeopacity{0.700000}%
\pgfsetdash{}{0pt}%
\pgfpathmoveto{\pgfqpoint{8.524468in}{2.854757in}}%
\pgfpathcurveto{\pgfqpoint{8.529512in}{2.854757in}}{\pgfqpoint{8.534350in}{2.856760in}}{\pgfqpoint{8.537916in}{2.860327in}}%
\pgfpathcurveto{\pgfqpoint{8.541482in}{2.863893in}}{\pgfqpoint{8.543486in}{2.868731in}}{\pgfqpoint{8.543486in}{2.873775in}}%
\pgfpathcurveto{\pgfqpoint{8.543486in}{2.878818in}}{\pgfqpoint{8.541482in}{2.883656in}}{\pgfqpoint{8.537916in}{2.887223in}}%
\pgfpathcurveto{\pgfqpoint{8.534350in}{2.890789in}}{\pgfqpoint{8.529512in}{2.892793in}}{\pgfqpoint{8.524468in}{2.892793in}}%
\pgfpathcurveto{\pgfqpoint{8.519424in}{2.892793in}}{\pgfqpoint{8.514587in}{2.890789in}}{\pgfqpoint{8.511020in}{2.887223in}}%
\pgfpathcurveto{\pgfqpoint{8.507454in}{2.883656in}}{\pgfqpoint{8.505450in}{2.878818in}}{\pgfqpoint{8.505450in}{2.873775in}}%
\pgfpathcurveto{\pgfqpoint{8.505450in}{2.868731in}}{\pgfqpoint{8.507454in}{2.863893in}}{\pgfqpoint{8.511020in}{2.860327in}}%
\pgfpathcurveto{\pgfqpoint{8.514587in}{2.856760in}}{\pgfqpoint{8.519424in}{2.854757in}}{\pgfqpoint{8.524468in}{2.854757in}}%
\pgfpathclose%
\pgfusepath{fill}%
\end{pgfscope}%
\begin{pgfscope}%
\pgfpathrectangle{\pgfqpoint{6.572727in}{0.474100in}}{\pgfqpoint{4.227273in}{3.318700in}}%
\pgfusepath{clip}%
\pgfsetbuttcap%
\pgfsetroundjoin%
\definecolor{currentfill}{rgb}{0.127568,0.566949,0.550556}%
\pgfsetfillcolor{currentfill}%
\pgfsetfillopacity{0.700000}%
\pgfsetlinewidth{0.000000pt}%
\definecolor{currentstroke}{rgb}{0.000000,0.000000,0.000000}%
\pgfsetstrokecolor{currentstroke}%
\pgfsetstrokeopacity{0.700000}%
\pgfsetdash{}{0pt}%
\pgfpathmoveto{\pgfqpoint{8.186872in}{3.570710in}}%
\pgfpathcurveto{\pgfqpoint{8.191915in}{3.570710in}}{\pgfqpoint{8.196753in}{3.572714in}}{\pgfqpoint{8.200320in}{3.576281in}}%
\pgfpathcurveto{\pgfqpoint{8.203886in}{3.579847in}}{\pgfqpoint{8.205890in}{3.584685in}}{\pgfqpoint{8.205890in}{3.589728in}}%
\pgfpathcurveto{\pgfqpoint{8.205890in}{3.594772in}}{\pgfqpoint{8.203886in}{3.599610in}}{\pgfqpoint{8.200320in}{3.603176in}}%
\pgfpathcurveto{\pgfqpoint{8.196753in}{3.606743in}}{\pgfqpoint{8.191915in}{3.608747in}}{\pgfqpoint{8.186872in}{3.608747in}}%
\pgfpathcurveto{\pgfqpoint{8.181828in}{3.608747in}}{\pgfqpoint{8.176990in}{3.606743in}}{\pgfqpoint{8.173424in}{3.603176in}}%
\pgfpathcurveto{\pgfqpoint{8.169857in}{3.599610in}}{\pgfqpoint{8.167854in}{3.594772in}}{\pgfqpoint{8.167854in}{3.589728in}}%
\pgfpathcurveto{\pgfqpoint{8.167854in}{3.584685in}}{\pgfqpoint{8.169857in}{3.579847in}}{\pgfqpoint{8.173424in}{3.576281in}}%
\pgfpathcurveto{\pgfqpoint{8.176990in}{3.572714in}}{\pgfqpoint{8.181828in}{3.570710in}}{\pgfqpoint{8.186872in}{3.570710in}}%
\pgfpathclose%
\pgfusepath{fill}%
\end{pgfscope}%
\begin{pgfscope}%
\pgfpathrectangle{\pgfqpoint{6.572727in}{0.474100in}}{\pgfqpoint{4.227273in}{3.318700in}}%
\pgfusepath{clip}%
\pgfsetbuttcap%
\pgfsetroundjoin%
\definecolor{currentfill}{rgb}{0.993248,0.906157,0.143936}%
\pgfsetfillcolor{currentfill}%
\pgfsetfillopacity{0.700000}%
\pgfsetlinewidth{0.000000pt}%
\definecolor{currentstroke}{rgb}{0.000000,0.000000,0.000000}%
\pgfsetstrokecolor{currentstroke}%
\pgfsetstrokeopacity{0.700000}%
\pgfsetdash{}{0pt}%
\pgfpathmoveto{\pgfqpoint{9.933332in}{1.329904in}}%
\pgfpathcurveto{\pgfqpoint{9.938376in}{1.329904in}}{\pgfqpoint{9.943214in}{1.331908in}}{\pgfqpoint{9.946780in}{1.335474in}}%
\pgfpathcurveto{\pgfqpoint{9.950347in}{1.339041in}}{\pgfqpoint{9.952351in}{1.343879in}}{\pgfqpoint{9.952351in}{1.348922in}}%
\pgfpathcurveto{\pgfqpoint{9.952351in}{1.353966in}}{\pgfqpoint{9.950347in}{1.358804in}}{\pgfqpoint{9.946780in}{1.362370in}}%
\pgfpathcurveto{\pgfqpoint{9.943214in}{1.365937in}}{\pgfqpoint{9.938376in}{1.367940in}}{\pgfqpoint{9.933332in}{1.367940in}}%
\pgfpathcurveto{\pgfqpoint{9.928289in}{1.367940in}}{\pgfqpoint{9.923451in}{1.365937in}}{\pgfqpoint{9.919885in}{1.362370in}}%
\pgfpathcurveto{\pgfqpoint{9.916318in}{1.358804in}}{\pgfqpoint{9.914314in}{1.353966in}}{\pgfqpoint{9.914314in}{1.348922in}}%
\pgfpathcurveto{\pgfqpoint{9.914314in}{1.343879in}}{\pgfqpoint{9.916318in}{1.339041in}}{\pgfqpoint{9.919885in}{1.335474in}}%
\pgfpathcurveto{\pgfqpoint{9.923451in}{1.331908in}}{\pgfqpoint{9.928289in}{1.329904in}}{\pgfqpoint{9.933332in}{1.329904in}}%
\pgfpathclose%
\pgfusepath{fill}%
\end{pgfscope}%
\begin{pgfscope}%
\pgfpathrectangle{\pgfqpoint{6.572727in}{0.474100in}}{\pgfqpoint{4.227273in}{3.318700in}}%
\pgfusepath{clip}%
\pgfsetbuttcap%
\pgfsetroundjoin%
\definecolor{currentfill}{rgb}{0.127568,0.566949,0.550556}%
\pgfsetfillcolor{currentfill}%
\pgfsetfillopacity{0.700000}%
\pgfsetlinewidth{0.000000pt}%
\definecolor{currentstroke}{rgb}{0.000000,0.000000,0.000000}%
\pgfsetstrokecolor{currentstroke}%
\pgfsetstrokeopacity{0.700000}%
\pgfsetdash{}{0pt}%
\pgfpathmoveto{\pgfqpoint{8.230374in}{2.422254in}}%
\pgfpathcurveto{\pgfqpoint{8.235418in}{2.422254in}}{\pgfqpoint{8.240256in}{2.424258in}}{\pgfqpoint{8.243822in}{2.427824in}}%
\pgfpathcurveto{\pgfqpoint{8.247389in}{2.431390in}}{\pgfqpoint{8.249392in}{2.436228in}}{\pgfqpoint{8.249392in}{2.441272in}}%
\pgfpathcurveto{\pgfqpoint{8.249392in}{2.446316in}}{\pgfqpoint{8.247389in}{2.451153in}}{\pgfqpoint{8.243822in}{2.454720in}}%
\pgfpathcurveto{\pgfqpoint{8.240256in}{2.458286in}}{\pgfqpoint{8.235418in}{2.460290in}}{\pgfqpoint{8.230374in}{2.460290in}}%
\pgfpathcurveto{\pgfqpoint{8.225331in}{2.460290in}}{\pgfqpoint{8.220493in}{2.458286in}}{\pgfqpoint{8.216926in}{2.454720in}}%
\pgfpathcurveto{\pgfqpoint{8.213360in}{2.451153in}}{\pgfqpoint{8.211356in}{2.446316in}}{\pgfqpoint{8.211356in}{2.441272in}}%
\pgfpathcurveto{\pgfqpoint{8.211356in}{2.436228in}}{\pgfqpoint{8.213360in}{2.431390in}}{\pgfqpoint{8.216926in}{2.427824in}}%
\pgfpathcurveto{\pgfqpoint{8.220493in}{2.424258in}}{\pgfqpoint{8.225331in}{2.422254in}}{\pgfqpoint{8.230374in}{2.422254in}}%
\pgfpathclose%
\pgfusepath{fill}%
\end{pgfscope}%
\begin{pgfscope}%
\pgfpathrectangle{\pgfqpoint{6.572727in}{0.474100in}}{\pgfqpoint{4.227273in}{3.318700in}}%
\pgfusepath{clip}%
\pgfsetbuttcap%
\pgfsetroundjoin%
\definecolor{currentfill}{rgb}{0.127568,0.566949,0.550556}%
\pgfsetfillcolor{currentfill}%
\pgfsetfillopacity{0.700000}%
\pgfsetlinewidth{0.000000pt}%
\definecolor{currentstroke}{rgb}{0.000000,0.000000,0.000000}%
\pgfsetstrokecolor{currentstroke}%
\pgfsetstrokeopacity{0.700000}%
\pgfsetdash{}{0pt}%
\pgfpathmoveto{\pgfqpoint{8.234453in}{2.957589in}}%
\pgfpathcurveto{\pgfqpoint{8.239497in}{2.957589in}}{\pgfqpoint{8.244335in}{2.959593in}}{\pgfqpoint{8.247901in}{2.963159in}}%
\pgfpathcurveto{\pgfqpoint{8.251468in}{2.966726in}}{\pgfqpoint{8.253472in}{2.971563in}}{\pgfqpoint{8.253472in}{2.976607in}}%
\pgfpathcurveto{\pgfqpoint{8.253472in}{2.981651in}}{\pgfqpoint{8.251468in}{2.986489in}}{\pgfqpoint{8.247901in}{2.990055in}}%
\pgfpathcurveto{\pgfqpoint{8.244335in}{2.993621in}}{\pgfqpoint{8.239497in}{2.995625in}}{\pgfqpoint{8.234453in}{2.995625in}}%
\pgfpathcurveto{\pgfqpoint{8.229410in}{2.995625in}}{\pgfqpoint{8.224572in}{2.993621in}}{\pgfqpoint{8.221006in}{2.990055in}}%
\pgfpathcurveto{\pgfqpoint{8.217439in}{2.986489in}}{\pgfqpoint{8.215435in}{2.981651in}}{\pgfqpoint{8.215435in}{2.976607in}}%
\pgfpathcurveto{\pgfqpoint{8.215435in}{2.971563in}}{\pgfqpoint{8.217439in}{2.966726in}}{\pgfqpoint{8.221006in}{2.963159in}}%
\pgfpathcurveto{\pgfqpoint{8.224572in}{2.959593in}}{\pgfqpoint{8.229410in}{2.957589in}}{\pgfqpoint{8.234453in}{2.957589in}}%
\pgfpathclose%
\pgfusepath{fill}%
\end{pgfscope}%
\begin{pgfscope}%
\pgfpathrectangle{\pgfqpoint{6.572727in}{0.474100in}}{\pgfqpoint{4.227273in}{3.318700in}}%
\pgfusepath{clip}%
\pgfsetbuttcap%
\pgfsetroundjoin%
\definecolor{currentfill}{rgb}{0.127568,0.566949,0.550556}%
\pgfsetfillcolor{currentfill}%
\pgfsetfillopacity{0.700000}%
\pgfsetlinewidth{0.000000pt}%
\definecolor{currentstroke}{rgb}{0.000000,0.000000,0.000000}%
\pgfsetstrokecolor{currentstroke}%
\pgfsetstrokeopacity{0.700000}%
\pgfsetdash{}{0pt}%
\pgfpathmoveto{\pgfqpoint{8.201388in}{2.585868in}}%
\pgfpathcurveto{\pgfqpoint{8.206432in}{2.585868in}}{\pgfqpoint{8.211269in}{2.587872in}}{\pgfqpoint{8.214836in}{2.591438in}}%
\pgfpathcurveto{\pgfqpoint{8.218402in}{2.595005in}}{\pgfqpoint{8.220406in}{2.599843in}}{\pgfqpoint{8.220406in}{2.604886in}}%
\pgfpathcurveto{\pgfqpoint{8.220406in}{2.609930in}}{\pgfqpoint{8.218402in}{2.614768in}}{\pgfqpoint{8.214836in}{2.618334in}}%
\pgfpathcurveto{\pgfqpoint{8.211269in}{2.621901in}}{\pgfqpoint{8.206432in}{2.623904in}}{\pgfqpoint{8.201388in}{2.623904in}}%
\pgfpathcurveto{\pgfqpoint{8.196344in}{2.623904in}}{\pgfqpoint{8.191506in}{2.621901in}}{\pgfqpoint{8.187940in}{2.618334in}}%
\pgfpathcurveto{\pgfqpoint{8.184374in}{2.614768in}}{\pgfqpoint{8.182370in}{2.609930in}}{\pgfqpoint{8.182370in}{2.604886in}}%
\pgfpathcurveto{\pgfqpoint{8.182370in}{2.599843in}}{\pgfqpoint{8.184374in}{2.595005in}}{\pgfqpoint{8.187940in}{2.591438in}}%
\pgfpathcurveto{\pgfqpoint{8.191506in}{2.587872in}}{\pgfqpoint{8.196344in}{2.585868in}}{\pgfqpoint{8.201388in}{2.585868in}}%
\pgfpathclose%
\pgfusepath{fill}%
\end{pgfscope}%
\begin{pgfscope}%
\pgfpathrectangle{\pgfqpoint{6.572727in}{0.474100in}}{\pgfqpoint{4.227273in}{3.318700in}}%
\pgfusepath{clip}%
\pgfsetbuttcap%
\pgfsetroundjoin%
\definecolor{currentfill}{rgb}{0.127568,0.566949,0.550556}%
\pgfsetfillcolor{currentfill}%
\pgfsetfillopacity{0.700000}%
\pgfsetlinewidth{0.000000pt}%
\definecolor{currentstroke}{rgb}{0.000000,0.000000,0.000000}%
\pgfsetstrokecolor{currentstroke}%
\pgfsetstrokeopacity{0.700000}%
\pgfsetdash{}{0pt}%
\pgfpathmoveto{\pgfqpoint{7.506977in}{1.539659in}}%
\pgfpathcurveto{\pgfqpoint{7.512021in}{1.539659in}}{\pgfqpoint{7.516859in}{1.541662in}}{\pgfqpoint{7.520425in}{1.545229in}}%
\pgfpathcurveto{\pgfqpoint{7.523992in}{1.548795in}}{\pgfqpoint{7.525995in}{1.553633in}}{\pgfqpoint{7.525995in}{1.558677in}}%
\pgfpathcurveto{\pgfqpoint{7.525995in}{1.563720in}}{\pgfqpoint{7.523992in}{1.568558in}}{\pgfqpoint{7.520425in}{1.572125in}}%
\pgfpathcurveto{\pgfqpoint{7.516859in}{1.575691in}}{\pgfqpoint{7.512021in}{1.577695in}}{\pgfqpoint{7.506977in}{1.577695in}}%
\pgfpathcurveto{\pgfqpoint{7.501934in}{1.577695in}}{\pgfqpoint{7.497096in}{1.575691in}}{\pgfqpoint{7.493529in}{1.572125in}}%
\pgfpathcurveto{\pgfqpoint{7.489963in}{1.568558in}}{\pgfqpoint{7.487959in}{1.563720in}}{\pgfqpoint{7.487959in}{1.558677in}}%
\pgfpathcurveto{\pgfqpoint{7.487959in}{1.553633in}}{\pgfqpoint{7.489963in}{1.548795in}}{\pgfqpoint{7.493529in}{1.545229in}}%
\pgfpathcurveto{\pgfqpoint{7.497096in}{1.541662in}}{\pgfqpoint{7.501934in}{1.539659in}}{\pgfqpoint{7.506977in}{1.539659in}}%
\pgfpathclose%
\pgfusepath{fill}%
\end{pgfscope}%
\begin{pgfscope}%
\pgfpathrectangle{\pgfqpoint{6.572727in}{0.474100in}}{\pgfqpoint{4.227273in}{3.318700in}}%
\pgfusepath{clip}%
\pgfsetbuttcap%
\pgfsetroundjoin%
\definecolor{currentfill}{rgb}{0.993248,0.906157,0.143936}%
\pgfsetfillcolor{currentfill}%
\pgfsetfillopacity{0.700000}%
\pgfsetlinewidth{0.000000pt}%
\definecolor{currentstroke}{rgb}{0.000000,0.000000,0.000000}%
\pgfsetstrokecolor{currentstroke}%
\pgfsetstrokeopacity{0.700000}%
\pgfsetdash{}{0pt}%
\pgfpathmoveto{\pgfqpoint{9.828348in}{1.199761in}}%
\pgfpathcurveto{\pgfqpoint{9.833391in}{1.199761in}}{\pgfqpoint{9.838229in}{1.201764in}}{\pgfqpoint{9.841795in}{1.205331in}}%
\pgfpathcurveto{\pgfqpoint{9.845362in}{1.208897in}}{\pgfqpoint{9.847366in}{1.213735in}}{\pgfqpoint{9.847366in}{1.218779in}}%
\pgfpathcurveto{\pgfqpoint{9.847366in}{1.223822in}}{\pgfqpoint{9.845362in}{1.228660in}}{\pgfqpoint{9.841795in}{1.232227in}}%
\pgfpathcurveto{\pgfqpoint{9.838229in}{1.235793in}}{\pgfqpoint{9.833391in}{1.237797in}}{\pgfqpoint{9.828348in}{1.237797in}}%
\pgfpathcurveto{\pgfqpoint{9.823304in}{1.237797in}}{\pgfqpoint{9.818466in}{1.235793in}}{\pgfqpoint{9.814900in}{1.232227in}}%
\pgfpathcurveto{\pgfqpoint{9.811333in}{1.228660in}}{\pgfqpoint{9.809329in}{1.223822in}}{\pgfqpoint{9.809329in}{1.218779in}}%
\pgfpathcurveto{\pgfqpoint{9.809329in}{1.213735in}}{\pgfqpoint{9.811333in}{1.208897in}}{\pgfqpoint{9.814900in}{1.205331in}}%
\pgfpathcurveto{\pgfqpoint{9.818466in}{1.201764in}}{\pgfqpoint{9.823304in}{1.199761in}}{\pgfqpoint{9.828348in}{1.199761in}}%
\pgfpathclose%
\pgfusepath{fill}%
\end{pgfscope}%
\begin{pgfscope}%
\pgfpathrectangle{\pgfqpoint{6.572727in}{0.474100in}}{\pgfqpoint{4.227273in}{3.318700in}}%
\pgfusepath{clip}%
\pgfsetbuttcap%
\pgfsetroundjoin%
\definecolor{currentfill}{rgb}{0.127568,0.566949,0.550556}%
\pgfsetfillcolor{currentfill}%
\pgfsetfillopacity{0.700000}%
\pgfsetlinewidth{0.000000pt}%
\definecolor{currentstroke}{rgb}{0.000000,0.000000,0.000000}%
\pgfsetstrokecolor{currentstroke}%
\pgfsetstrokeopacity{0.700000}%
\pgfsetdash{}{0pt}%
\pgfpathmoveto{\pgfqpoint{8.108587in}{2.496858in}}%
\pgfpathcurveto{\pgfqpoint{8.113631in}{2.496858in}}{\pgfqpoint{8.118468in}{2.498862in}}{\pgfqpoint{8.122035in}{2.502428in}}%
\pgfpathcurveto{\pgfqpoint{8.125601in}{2.505995in}}{\pgfqpoint{8.127605in}{2.510832in}}{\pgfqpoint{8.127605in}{2.515876in}}%
\pgfpathcurveto{\pgfqpoint{8.127605in}{2.520920in}}{\pgfqpoint{8.125601in}{2.525758in}}{\pgfqpoint{8.122035in}{2.529324in}}%
\pgfpathcurveto{\pgfqpoint{8.118468in}{2.532890in}}{\pgfqpoint{8.113631in}{2.534894in}}{\pgfqpoint{8.108587in}{2.534894in}}%
\pgfpathcurveto{\pgfqpoint{8.103543in}{2.534894in}}{\pgfqpoint{8.098706in}{2.532890in}}{\pgfqpoint{8.095139in}{2.529324in}}%
\pgfpathcurveto{\pgfqpoint{8.091573in}{2.525758in}}{\pgfqpoint{8.089569in}{2.520920in}}{\pgfqpoint{8.089569in}{2.515876in}}%
\pgfpathcurveto{\pgfqpoint{8.089569in}{2.510832in}}{\pgfqpoint{8.091573in}{2.505995in}}{\pgfqpoint{8.095139in}{2.502428in}}%
\pgfpathcurveto{\pgfqpoint{8.098706in}{2.498862in}}{\pgfqpoint{8.103543in}{2.496858in}}{\pgfqpoint{8.108587in}{2.496858in}}%
\pgfpathclose%
\pgfusepath{fill}%
\end{pgfscope}%
\begin{pgfscope}%
\pgfpathrectangle{\pgfqpoint{6.572727in}{0.474100in}}{\pgfqpoint{4.227273in}{3.318700in}}%
\pgfusepath{clip}%
\pgfsetbuttcap%
\pgfsetroundjoin%
\definecolor{currentfill}{rgb}{0.993248,0.906157,0.143936}%
\pgfsetfillcolor{currentfill}%
\pgfsetfillopacity{0.700000}%
\pgfsetlinewidth{0.000000pt}%
\definecolor{currentstroke}{rgb}{0.000000,0.000000,0.000000}%
\pgfsetstrokecolor{currentstroke}%
\pgfsetstrokeopacity{0.700000}%
\pgfsetdash{}{0pt}%
\pgfpathmoveto{\pgfqpoint{10.210872in}{1.493131in}}%
\pgfpathcurveto{\pgfqpoint{10.215916in}{1.493131in}}{\pgfqpoint{10.220754in}{1.495135in}}{\pgfqpoint{10.224320in}{1.498701in}}%
\pgfpathcurveto{\pgfqpoint{10.227887in}{1.502268in}}{\pgfqpoint{10.229890in}{1.507105in}}{\pgfqpoint{10.229890in}{1.512149in}}%
\pgfpathcurveto{\pgfqpoint{10.229890in}{1.517193in}}{\pgfqpoint{10.227887in}{1.522030in}}{\pgfqpoint{10.224320in}{1.525597in}}%
\pgfpathcurveto{\pgfqpoint{10.220754in}{1.529163in}}{\pgfqpoint{10.215916in}{1.531167in}}{\pgfqpoint{10.210872in}{1.531167in}}%
\pgfpathcurveto{\pgfqpoint{10.205829in}{1.531167in}}{\pgfqpoint{10.200991in}{1.529163in}}{\pgfqpoint{10.197424in}{1.525597in}}%
\pgfpathcurveto{\pgfqpoint{10.193858in}{1.522030in}}{\pgfqpoint{10.191854in}{1.517193in}}{\pgfqpoint{10.191854in}{1.512149in}}%
\pgfpathcurveto{\pgfqpoint{10.191854in}{1.507105in}}{\pgfqpoint{10.193858in}{1.502268in}}{\pgfqpoint{10.197424in}{1.498701in}}%
\pgfpathcurveto{\pgfqpoint{10.200991in}{1.495135in}}{\pgfqpoint{10.205829in}{1.493131in}}{\pgfqpoint{10.210872in}{1.493131in}}%
\pgfpathclose%
\pgfusepath{fill}%
\end{pgfscope}%
\begin{pgfscope}%
\pgfpathrectangle{\pgfqpoint{6.572727in}{0.474100in}}{\pgfqpoint{4.227273in}{3.318700in}}%
\pgfusepath{clip}%
\pgfsetbuttcap%
\pgfsetroundjoin%
\definecolor{currentfill}{rgb}{0.127568,0.566949,0.550556}%
\pgfsetfillcolor{currentfill}%
\pgfsetfillopacity{0.700000}%
\pgfsetlinewidth{0.000000pt}%
\definecolor{currentstroke}{rgb}{0.000000,0.000000,0.000000}%
\pgfsetstrokecolor{currentstroke}%
\pgfsetstrokeopacity{0.700000}%
\pgfsetdash{}{0pt}%
\pgfpathmoveto{\pgfqpoint{7.602098in}{1.563613in}}%
\pgfpathcurveto{\pgfqpoint{7.607142in}{1.563613in}}{\pgfqpoint{7.611980in}{1.565617in}}{\pgfqpoint{7.615546in}{1.569184in}}%
\pgfpathcurveto{\pgfqpoint{7.619113in}{1.572750in}}{\pgfqpoint{7.621117in}{1.577588in}}{\pgfqpoint{7.621117in}{1.582631in}}%
\pgfpathcurveto{\pgfqpoint{7.621117in}{1.587675in}}{\pgfqpoint{7.619113in}{1.592513in}}{\pgfqpoint{7.615546in}{1.596079in}}%
\pgfpathcurveto{\pgfqpoint{7.611980in}{1.599646in}}{\pgfqpoint{7.607142in}{1.601650in}}{\pgfqpoint{7.602098in}{1.601650in}}%
\pgfpathcurveto{\pgfqpoint{7.597055in}{1.601650in}}{\pgfqpoint{7.592217in}{1.599646in}}{\pgfqpoint{7.588651in}{1.596079in}}%
\pgfpathcurveto{\pgfqpoint{7.585084in}{1.592513in}}{\pgfqpoint{7.583080in}{1.587675in}}{\pgfqpoint{7.583080in}{1.582631in}}%
\pgfpathcurveto{\pgfqpoint{7.583080in}{1.577588in}}{\pgfqpoint{7.585084in}{1.572750in}}{\pgfqpoint{7.588651in}{1.569184in}}%
\pgfpathcurveto{\pgfqpoint{7.592217in}{1.565617in}}{\pgfqpoint{7.597055in}{1.563613in}}{\pgfqpoint{7.602098in}{1.563613in}}%
\pgfpathclose%
\pgfusepath{fill}%
\end{pgfscope}%
\begin{pgfscope}%
\pgfpathrectangle{\pgfqpoint{6.572727in}{0.474100in}}{\pgfqpoint{4.227273in}{3.318700in}}%
\pgfusepath{clip}%
\pgfsetbuttcap%
\pgfsetroundjoin%
\definecolor{currentfill}{rgb}{0.127568,0.566949,0.550556}%
\pgfsetfillcolor{currentfill}%
\pgfsetfillopacity{0.700000}%
\pgfsetlinewidth{0.000000pt}%
\definecolor{currentstroke}{rgb}{0.000000,0.000000,0.000000}%
\pgfsetstrokecolor{currentstroke}%
\pgfsetstrokeopacity{0.700000}%
\pgfsetdash{}{0pt}%
\pgfpathmoveto{\pgfqpoint{8.272134in}{1.498709in}}%
\pgfpathcurveto{\pgfqpoint{8.277178in}{1.498709in}}{\pgfqpoint{8.282016in}{1.500713in}}{\pgfqpoint{8.285582in}{1.504279in}}%
\pgfpathcurveto{\pgfqpoint{8.289148in}{1.507845in}}{\pgfqpoint{8.291152in}{1.512683in}}{\pgfqpoint{8.291152in}{1.517727in}}%
\pgfpathcurveto{\pgfqpoint{8.291152in}{1.522771in}}{\pgfqpoint{8.289148in}{1.527608in}}{\pgfqpoint{8.285582in}{1.531175in}}%
\pgfpathcurveto{\pgfqpoint{8.282016in}{1.534741in}}{\pgfqpoint{8.277178in}{1.536745in}}{\pgfqpoint{8.272134in}{1.536745in}}%
\pgfpathcurveto{\pgfqpoint{8.267091in}{1.536745in}}{\pgfqpoint{8.262253in}{1.534741in}}{\pgfqpoint{8.258686in}{1.531175in}}%
\pgfpathcurveto{\pgfqpoint{8.255120in}{1.527608in}}{\pgfqpoint{8.253116in}{1.522771in}}{\pgfqpoint{8.253116in}{1.517727in}}%
\pgfpathcurveto{\pgfqpoint{8.253116in}{1.512683in}}{\pgfqpoint{8.255120in}{1.507845in}}{\pgfqpoint{8.258686in}{1.504279in}}%
\pgfpathcurveto{\pgfqpoint{8.262253in}{1.500713in}}{\pgfqpoint{8.267091in}{1.498709in}}{\pgfqpoint{8.272134in}{1.498709in}}%
\pgfpathclose%
\pgfusepath{fill}%
\end{pgfscope}%
\begin{pgfscope}%
\pgfpathrectangle{\pgfqpoint{6.572727in}{0.474100in}}{\pgfqpoint{4.227273in}{3.318700in}}%
\pgfusepath{clip}%
\pgfsetbuttcap%
\pgfsetroundjoin%
\definecolor{currentfill}{rgb}{0.127568,0.566949,0.550556}%
\pgfsetfillcolor{currentfill}%
\pgfsetfillopacity{0.700000}%
\pgfsetlinewidth{0.000000pt}%
\definecolor{currentstroke}{rgb}{0.000000,0.000000,0.000000}%
\pgfsetstrokecolor{currentstroke}%
\pgfsetstrokeopacity{0.700000}%
\pgfsetdash{}{0pt}%
\pgfpathmoveto{\pgfqpoint{8.312181in}{3.374278in}}%
\pgfpathcurveto{\pgfqpoint{8.317225in}{3.374278in}}{\pgfqpoint{8.322063in}{3.376282in}}{\pgfqpoint{8.325629in}{3.379849in}}%
\pgfpathcurveto{\pgfqpoint{8.329195in}{3.383415in}}{\pgfqpoint{8.331199in}{3.388253in}}{\pgfqpoint{8.331199in}{3.393296in}}%
\pgfpathcurveto{\pgfqpoint{8.331199in}{3.398340in}}{\pgfqpoint{8.329195in}{3.403178in}}{\pgfqpoint{8.325629in}{3.406744in}}%
\pgfpathcurveto{\pgfqpoint{8.322063in}{3.410311in}}{\pgfqpoint{8.317225in}{3.412315in}}{\pgfqpoint{8.312181in}{3.412315in}}%
\pgfpathcurveto{\pgfqpoint{8.307138in}{3.412315in}}{\pgfqpoint{8.302300in}{3.410311in}}{\pgfqpoint{8.298733in}{3.406744in}}%
\pgfpathcurveto{\pgfqpoint{8.295167in}{3.403178in}}{\pgfqpoint{8.293163in}{3.398340in}}{\pgfqpoint{8.293163in}{3.393296in}}%
\pgfpathcurveto{\pgfqpoint{8.293163in}{3.388253in}}{\pgfqpoint{8.295167in}{3.383415in}}{\pgfqpoint{8.298733in}{3.379849in}}%
\pgfpathcurveto{\pgfqpoint{8.302300in}{3.376282in}}{\pgfqpoint{8.307138in}{3.374278in}}{\pgfqpoint{8.312181in}{3.374278in}}%
\pgfpathclose%
\pgfusepath{fill}%
\end{pgfscope}%
\begin{pgfscope}%
\pgfpathrectangle{\pgfqpoint{6.572727in}{0.474100in}}{\pgfqpoint{4.227273in}{3.318700in}}%
\pgfusepath{clip}%
\pgfsetbuttcap%
\pgfsetroundjoin%
\definecolor{currentfill}{rgb}{0.127568,0.566949,0.550556}%
\pgfsetfillcolor{currentfill}%
\pgfsetfillopacity{0.700000}%
\pgfsetlinewidth{0.000000pt}%
\definecolor{currentstroke}{rgb}{0.000000,0.000000,0.000000}%
\pgfsetstrokecolor{currentstroke}%
\pgfsetstrokeopacity{0.700000}%
\pgfsetdash{}{0pt}%
\pgfpathmoveto{\pgfqpoint{7.728138in}{1.908250in}}%
\pgfpathcurveto{\pgfqpoint{7.733182in}{1.908250in}}{\pgfqpoint{7.738019in}{1.910254in}}{\pgfqpoint{7.741586in}{1.913820in}}%
\pgfpathcurveto{\pgfqpoint{7.745152in}{1.917386in}}{\pgfqpoint{7.747156in}{1.922224in}}{\pgfqpoint{7.747156in}{1.927268in}}%
\pgfpathcurveto{\pgfqpoint{7.747156in}{1.932312in}}{\pgfqpoint{7.745152in}{1.937149in}}{\pgfqpoint{7.741586in}{1.940716in}}%
\pgfpathcurveto{\pgfqpoint{7.738019in}{1.944282in}}{\pgfqpoint{7.733182in}{1.946286in}}{\pgfqpoint{7.728138in}{1.946286in}}%
\pgfpathcurveto{\pgfqpoint{7.723094in}{1.946286in}}{\pgfqpoint{7.718257in}{1.944282in}}{\pgfqpoint{7.714690in}{1.940716in}}%
\pgfpathcurveto{\pgfqpoint{7.711124in}{1.937149in}}{\pgfqpoint{7.709120in}{1.932312in}}{\pgfqpoint{7.709120in}{1.927268in}}%
\pgfpathcurveto{\pgfqpoint{7.709120in}{1.922224in}}{\pgfqpoint{7.711124in}{1.917386in}}{\pgfqpoint{7.714690in}{1.913820in}}%
\pgfpathcurveto{\pgfqpoint{7.718257in}{1.910254in}}{\pgfqpoint{7.723094in}{1.908250in}}{\pgfqpoint{7.728138in}{1.908250in}}%
\pgfpathclose%
\pgfusepath{fill}%
\end{pgfscope}%
\begin{pgfscope}%
\pgfpathrectangle{\pgfqpoint{6.572727in}{0.474100in}}{\pgfqpoint{4.227273in}{3.318700in}}%
\pgfusepath{clip}%
\pgfsetbuttcap%
\pgfsetroundjoin%
\definecolor{currentfill}{rgb}{0.127568,0.566949,0.550556}%
\pgfsetfillcolor{currentfill}%
\pgfsetfillopacity{0.700000}%
\pgfsetlinewidth{0.000000pt}%
\definecolor{currentstroke}{rgb}{0.000000,0.000000,0.000000}%
\pgfsetstrokecolor{currentstroke}%
\pgfsetstrokeopacity{0.700000}%
\pgfsetdash{}{0pt}%
\pgfpathmoveto{\pgfqpoint{8.401447in}{3.040945in}}%
\pgfpathcurveto{\pgfqpoint{8.406491in}{3.040945in}}{\pgfqpoint{8.411328in}{3.042949in}}{\pgfqpoint{8.414895in}{3.046515in}}%
\pgfpathcurveto{\pgfqpoint{8.418461in}{3.050082in}}{\pgfqpoint{8.420465in}{3.054919in}}{\pgfqpoint{8.420465in}{3.059963in}}%
\pgfpathcurveto{\pgfqpoint{8.420465in}{3.065007in}}{\pgfqpoint{8.418461in}{3.069845in}}{\pgfqpoint{8.414895in}{3.073411in}}%
\pgfpathcurveto{\pgfqpoint{8.411328in}{3.076977in}}{\pgfqpoint{8.406491in}{3.078981in}}{\pgfqpoint{8.401447in}{3.078981in}}%
\pgfpathcurveto{\pgfqpoint{8.396403in}{3.078981in}}{\pgfqpoint{8.391565in}{3.076977in}}{\pgfqpoint{8.387999in}{3.073411in}}%
\pgfpathcurveto{\pgfqpoint{8.384433in}{3.069845in}}{\pgfqpoint{8.382429in}{3.065007in}}{\pgfqpoint{8.382429in}{3.059963in}}%
\pgfpathcurveto{\pgfqpoint{8.382429in}{3.054919in}}{\pgfqpoint{8.384433in}{3.050082in}}{\pgfqpoint{8.387999in}{3.046515in}}%
\pgfpathcurveto{\pgfqpoint{8.391565in}{3.042949in}}{\pgfqpoint{8.396403in}{3.040945in}}{\pgfqpoint{8.401447in}{3.040945in}}%
\pgfpathclose%
\pgfusepath{fill}%
\end{pgfscope}%
\begin{pgfscope}%
\pgfpathrectangle{\pgfqpoint{6.572727in}{0.474100in}}{\pgfqpoint{4.227273in}{3.318700in}}%
\pgfusepath{clip}%
\pgfsetbuttcap%
\pgfsetroundjoin%
\definecolor{currentfill}{rgb}{0.993248,0.906157,0.143936}%
\pgfsetfillcolor{currentfill}%
\pgfsetfillopacity{0.700000}%
\pgfsetlinewidth{0.000000pt}%
\definecolor{currentstroke}{rgb}{0.000000,0.000000,0.000000}%
\pgfsetstrokecolor{currentstroke}%
\pgfsetstrokeopacity{0.700000}%
\pgfsetdash{}{0pt}%
\pgfpathmoveto{\pgfqpoint{9.217785in}{1.548221in}}%
\pgfpathcurveto{\pgfqpoint{9.222829in}{1.548221in}}{\pgfqpoint{9.227667in}{1.550225in}}{\pgfqpoint{9.231233in}{1.553792in}}%
\pgfpathcurveto{\pgfqpoint{9.234800in}{1.557358in}}{\pgfqpoint{9.236804in}{1.562196in}}{\pgfqpoint{9.236804in}{1.567239in}}%
\pgfpathcurveto{\pgfqpoint{9.236804in}{1.572283in}}{\pgfqpoint{9.234800in}{1.577121in}}{\pgfqpoint{9.231233in}{1.580687in}}%
\pgfpathcurveto{\pgfqpoint{9.227667in}{1.584254in}}{\pgfqpoint{9.222829in}{1.586258in}}{\pgfqpoint{9.217785in}{1.586258in}}%
\pgfpathcurveto{\pgfqpoint{9.212742in}{1.586258in}}{\pgfqpoint{9.207904in}{1.584254in}}{\pgfqpoint{9.204338in}{1.580687in}}%
\pgfpathcurveto{\pgfqpoint{9.200771in}{1.577121in}}{\pgfqpoint{9.198767in}{1.572283in}}{\pgfqpoint{9.198767in}{1.567239in}}%
\pgfpathcurveto{\pgfqpoint{9.198767in}{1.562196in}}{\pgfqpoint{9.200771in}{1.557358in}}{\pgfqpoint{9.204338in}{1.553792in}}%
\pgfpathcurveto{\pgfqpoint{9.207904in}{1.550225in}}{\pgfqpoint{9.212742in}{1.548221in}}{\pgfqpoint{9.217785in}{1.548221in}}%
\pgfpathclose%
\pgfusepath{fill}%
\end{pgfscope}%
\begin{pgfscope}%
\pgfpathrectangle{\pgfqpoint{6.572727in}{0.474100in}}{\pgfqpoint{4.227273in}{3.318700in}}%
\pgfusepath{clip}%
\pgfsetbuttcap%
\pgfsetroundjoin%
\definecolor{currentfill}{rgb}{0.267004,0.004874,0.329415}%
\pgfsetfillcolor{currentfill}%
\pgfsetfillopacity{0.700000}%
\pgfsetlinewidth{0.000000pt}%
\definecolor{currentstroke}{rgb}{0.000000,0.000000,0.000000}%
\pgfsetstrokecolor{currentstroke}%
\pgfsetstrokeopacity{0.700000}%
\pgfsetdash{}{0pt}%
\pgfpathmoveto{\pgfqpoint{6.764876in}{1.942669in}}%
\pgfpathcurveto{\pgfqpoint{6.769920in}{1.942669in}}{\pgfqpoint{6.774757in}{1.944673in}}{\pgfqpoint{6.778324in}{1.948239in}}%
\pgfpathcurveto{\pgfqpoint{6.781890in}{1.951805in}}{\pgfqpoint{6.783894in}{1.956643in}}{\pgfqpoint{6.783894in}{1.961687in}}%
\pgfpathcurveto{\pgfqpoint{6.783894in}{1.966731in}}{\pgfqpoint{6.781890in}{1.971568in}}{\pgfqpoint{6.778324in}{1.975135in}}%
\pgfpathcurveto{\pgfqpoint{6.774757in}{1.978701in}}{\pgfqpoint{6.769920in}{1.980705in}}{\pgfqpoint{6.764876in}{1.980705in}}%
\pgfpathcurveto{\pgfqpoint{6.759832in}{1.980705in}}{\pgfqpoint{6.754995in}{1.978701in}}{\pgfqpoint{6.751428in}{1.975135in}}%
\pgfpathcurveto{\pgfqpoint{6.747862in}{1.971568in}}{\pgfqpoint{6.745858in}{1.966731in}}{\pgfqpoint{6.745858in}{1.961687in}}%
\pgfpathcurveto{\pgfqpoint{6.745858in}{1.956643in}}{\pgfqpoint{6.747862in}{1.951805in}}{\pgfqpoint{6.751428in}{1.948239in}}%
\pgfpathcurveto{\pgfqpoint{6.754995in}{1.944673in}}{\pgfqpoint{6.759832in}{1.942669in}}{\pgfqpoint{6.764876in}{1.942669in}}%
\pgfpathclose%
\pgfusepath{fill}%
\end{pgfscope}%
\begin{pgfscope}%
\pgfpathrectangle{\pgfqpoint{6.572727in}{0.474100in}}{\pgfqpoint{4.227273in}{3.318700in}}%
\pgfusepath{clip}%
\pgfsetbuttcap%
\pgfsetroundjoin%
\definecolor{currentfill}{rgb}{0.127568,0.566949,0.550556}%
\pgfsetfillcolor{currentfill}%
\pgfsetfillopacity{0.700000}%
\pgfsetlinewidth{0.000000pt}%
\definecolor{currentstroke}{rgb}{0.000000,0.000000,0.000000}%
\pgfsetstrokecolor{currentstroke}%
\pgfsetstrokeopacity{0.700000}%
\pgfsetdash{}{0pt}%
\pgfpathmoveto{\pgfqpoint{7.804196in}{0.954788in}}%
\pgfpathcurveto{\pgfqpoint{7.809240in}{0.954788in}}{\pgfqpoint{7.814078in}{0.956792in}}{\pgfqpoint{7.817644in}{0.960358in}}%
\pgfpathcurveto{\pgfqpoint{7.821211in}{0.963925in}}{\pgfqpoint{7.823214in}{0.968762in}}{\pgfqpoint{7.823214in}{0.973806in}}%
\pgfpathcurveto{\pgfqpoint{7.823214in}{0.978850in}}{\pgfqpoint{7.821211in}{0.983688in}}{\pgfqpoint{7.817644in}{0.987254in}}%
\pgfpathcurveto{\pgfqpoint{7.814078in}{0.990820in}}{\pgfqpoint{7.809240in}{0.992824in}}{\pgfqpoint{7.804196in}{0.992824in}}%
\pgfpathcurveto{\pgfqpoint{7.799153in}{0.992824in}}{\pgfqpoint{7.794315in}{0.990820in}}{\pgfqpoint{7.790748in}{0.987254in}}%
\pgfpathcurveto{\pgfqpoint{7.787182in}{0.983688in}}{\pgfqpoint{7.785178in}{0.978850in}}{\pgfqpoint{7.785178in}{0.973806in}}%
\pgfpathcurveto{\pgfqpoint{7.785178in}{0.968762in}}{\pgfqpoint{7.787182in}{0.963925in}}{\pgfqpoint{7.790748in}{0.960358in}}%
\pgfpathcurveto{\pgfqpoint{7.794315in}{0.956792in}}{\pgfqpoint{7.799153in}{0.954788in}}{\pgfqpoint{7.804196in}{0.954788in}}%
\pgfpathclose%
\pgfusepath{fill}%
\end{pgfscope}%
\begin{pgfscope}%
\pgfpathrectangle{\pgfqpoint{6.572727in}{0.474100in}}{\pgfqpoint{4.227273in}{3.318700in}}%
\pgfusepath{clip}%
\pgfsetbuttcap%
\pgfsetroundjoin%
\definecolor{currentfill}{rgb}{0.127568,0.566949,0.550556}%
\pgfsetfillcolor{currentfill}%
\pgfsetfillopacity{0.700000}%
\pgfsetlinewidth{0.000000pt}%
\definecolor{currentstroke}{rgb}{0.000000,0.000000,0.000000}%
\pgfsetstrokecolor{currentstroke}%
\pgfsetstrokeopacity{0.700000}%
\pgfsetdash{}{0pt}%
\pgfpathmoveto{\pgfqpoint{8.931081in}{3.117782in}}%
\pgfpathcurveto{\pgfqpoint{8.936125in}{3.117782in}}{\pgfqpoint{8.940963in}{3.119786in}}{\pgfqpoint{8.944529in}{3.123352in}}%
\pgfpathcurveto{\pgfqpoint{8.948096in}{3.126919in}}{\pgfqpoint{8.950100in}{3.131757in}}{\pgfqpoint{8.950100in}{3.136800in}}%
\pgfpathcurveto{\pgfqpoint{8.950100in}{3.141844in}}{\pgfqpoint{8.948096in}{3.146682in}}{\pgfqpoint{8.944529in}{3.150248in}}%
\pgfpathcurveto{\pgfqpoint{8.940963in}{3.153815in}}{\pgfqpoint{8.936125in}{3.155818in}}{\pgfqpoint{8.931081in}{3.155818in}}%
\pgfpathcurveto{\pgfqpoint{8.926038in}{3.155818in}}{\pgfqpoint{8.921200in}{3.153815in}}{\pgfqpoint{8.917634in}{3.150248in}}%
\pgfpathcurveto{\pgfqpoint{8.914067in}{3.146682in}}{\pgfqpoint{8.912063in}{3.141844in}}{\pgfqpoint{8.912063in}{3.136800in}}%
\pgfpathcurveto{\pgfqpoint{8.912063in}{3.131757in}}{\pgfqpoint{8.914067in}{3.126919in}}{\pgfqpoint{8.917634in}{3.123352in}}%
\pgfpathcurveto{\pgfqpoint{8.921200in}{3.119786in}}{\pgfqpoint{8.926038in}{3.117782in}}{\pgfqpoint{8.931081in}{3.117782in}}%
\pgfpathclose%
\pgfusepath{fill}%
\end{pgfscope}%
\begin{pgfscope}%
\pgfpathrectangle{\pgfqpoint{6.572727in}{0.474100in}}{\pgfqpoint{4.227273in}{3.318700in}}%
\pgfusepath{clip}%
\pgfsetbuttcap%
\pgfsetroundjoin%
\definecolor{currentfill}{rgb}{0.127568,0.566949,0.550556}%
\pgfsetfillcolor{currentfill}%
\pgfsetfillopacity{0.700000}%
\pgfsetlinewidth{0.000000pt}%
\definecolor{currentstroke}{rgb}{0.000000,0.000000,0.000000}%
\pgfsetstrokecolor{currentstroke}%
\pgfsetstrokeopacity{0.700000}%
\pgfsetdash{}{0pt}%
\pgfpathmoveto{\pgfqpoint{8.546253in}{3.269991in}}%
\pgfpathcurveto{\pgfqpoint{8.551297in}{3.269991in}}{\pgfqpoint{8.556135in}{3.271995in}}{\pgfqpoint{8.559701in}{3.275562in}}%
\pgfpathcurveto{\pgfqpoint{8.563267in}{3.279128in}}{\pgfqpoint{8.565271in}{3.283966in}}{\pgfqpoint{8.565271in}{3.289010in}}%
\pgfpathcurveto{\pgfqpoint{8.565271in}{3.294053in}}{\pgfqpoint{8.563267in}{3.298891in}}{\pgfqpoint{8.559701in}{3.302457in}}%
\pgfpathcurveto{\pgfqpoint{8.556135in}{3.306024in}}{\pgfqpoint{8.551297in}{3.308028in}}{\pgfqpoint{8.546253in}{3.308028in}}%
\pgfpathcurveto{\pgfqpoint{8.541210in}{3.308028in}}{\pgfqpoint{8.536372in}{3.306024in}}{\pgfqpoint{8.532805in}{3.302457in}}%
\pgfpathcurveto{\pgfqpoint{8.529239in}{3.298891in}}{\pgfqpoint{8.527235in}{3.294053in}}{\pgfqpoint{8.527235in}{3.289010in}}%
\pgfpathcurveto{\pgfqpoint{8.527235in}{3.283966in}}{\pgfqpoint{8.529239in}{3.279128in}}{\pgfqpoint{8.532805in}{3.275562in}}%
\pgfpathcurveto{\pgfqpoint{8.536372in}{3.271995in}}{\pgfqpoint{8.541210in}{3.269991in}}{\pgfqpoint{8.546253in}{3.269991in}}%
\pgfpathclose%
\pgfusepath{fill}%
\end{pgfscope}%
\begin{pgfscope}%
\pgfpathrectangle{\pgfqpoint{6.572727in}{0.474100in}}{\pgfqpoint{4.227273in}{3.318700in}}%
\pgfusepath{clip}%
\pgfsetbuttcap%
\pgfsetroundjoin%
\definecolor{currentfill}{rgb}{0.127568,0.566949,0.550556}%
\pgfsetfillcolor{currentfill}%
\pgfsetfillopacity{0.700000}%
\pgfsetlinewidth{0.000000pt}%
\definecolor{currentstroke}{rgb}{0.000000,0.000000,0.000000}%
\pgfsetstrokecolor{currentstroke}%
\pgfsetstrokeopacity{0.700000}%
\pgfsetdash{}{0pt}%
\pgfpathmoveto{\pgfqpoint{8.474563in}{2.276731in}}%
\pgfpathcurveto{\pgfqpoint{8.479607in}{2.276731in}}{\pgfqpoint{8.484445in}{2.278735in}}{\pgfqpoint{8.488011in}{2.282301in}}%
\pgfpathcurveto{\pgfqpoint{8.491578in}{2.285868in}}{\pgfqpoint{8.493581in}{2.290705in}}{\pgfqpoint{8.493581in}{2.295749in}}%
\pgfpathcurveto{\pgfqpoint{8.493581in}{2.300793in}}{\pgfqpoint{8.491578in}{2.305631in}}{\pgfqpoint{8.488011in}{2.309197in}}%
\pgfpathcurveto{\pgfqpoint{8.484445in}{2.312763in}}{\pgfqpoint{8.479607in}{2.314767in}}{\pgfqpoint{8.474563in}{2.314767in}}%
\pgfpathcurveto{\pgfqpoint{8.469520in}{2.314767in}}{\pgfqpoint{8.464682in}{2.312763in}}{\pgfqpoint{8.461115in}{2.309197in}}%
\pgfpathcurveto{\pgfqpoint{8.457549in}{2.305631in}}{\pgfqpoint{8.455545in}{2.300793in}}{\pgfqpoint{8.455545in}{2.295749in}}%
\pgfpathcurveto{\pgfqpoint{8.455545in}{2.290705in}}{\pgfqpoint{8.457549in}{2.285868in}}{\pgfqpoint{8.461115in}{2.282301in}}%
\pgfpathcurveto{\pgfqpoint{8.464682in}{2.278735in}}{\pgfqpoint{8.469520in}{2.276731in}}{\pgfqpoint{8.474563in}{2.276731in}}%
\pgfpathclose%
\pgfusepath{fill}%
\end{pgfscope}%
\begin{pgfscope}%
\pgfpathrectangle{\pgfqpoint{6.572727in}{0.474100in}}{\pgfqpoint{4.227273in}{3.318700in}}%
\pgfusepath{clip}%
\pgfsetbuttcap%
\pgfsetroundjoin%
\definecolor{currentfill}{rgb}{0.993248,0.906157,0.143936}%
\pgfsetfillcolor{currentfill}%
\pgfsetfillopacity{0.700000}%
\pgfsetlinewidth{0.000000pt}%
\definecolor{currentstroke}{rgb}{0.000000,0.000000,0.000000}%
\pgfsetstrokecolor{currentstroke}%
\pgfsetstrokeopacity{0.700000}%
\pgfsetdash{}{0pt}%
\pgfpathmoveto{\pgfqpoint{9.510394in}{1.409418in}}%
\pgfpathcurveto{\pgfqpoint{9.515437in}{1.409418in}}{\pgfqpoint{9.520275in}{1.411422in}}{\pgfqpoint{9.523841in}{1.414989in}}%
\pgfpathcurveto{\pgfqpoint{9.527408in}{1.418555in}}{\pgfqpoint{9.529412in}{1.423393in}}{\pgfqpoint{9.529412in}{1.428436in}}%
\pgfpathcurveto{\pgfqpoint{9.529412in}{1.433480in}}{\pgfqpoint{9.527408in}{1.438318in}}{\pgfqpoint{9.523841in}{1.441884in}}%
\pgfpathcurveto{\pgfqpoint{9.520275in}{1.445451in}}{\pgfqpoint{9.515437in}{1.447455in}}{\pgfqpoint{9.510394in}{1.447455in}}%
\pgfpathcurveto{\pgfqpoint{9.505350in}{1.447455in}}{\pgfqpoint{9.500512in}{1.445451in}}{\pgfqpoint{9.496946in}{1.441884in}}%
\pgfpathcurveto{\pgfqpoint{9.493379in}{1.438318in}}{\pgfqpoint{9.491375in}{1.433480in}}{\pgfqpoint{9.491375in}{1.428436in}}%
\pgfpathcurveto{\pgfqpoint{9.491375in}{1.423393in}}{\pgfqpoint{9.493379in}{1.418555in}}{\pgfqpoint{9.496946in}{1.414989in}}%
\pgfpathcurveto{\pgfqpoint{9.500512in}{1.411422in}}{\pgfqpoint{9.505350in}{1.409418in}}{\pgfqpoint{9.510394in}{1.409418in}}%
\pgfpathclose%
\pgfusepath{fill}%
\end{pgfscope}%
\begin{pgfscope}%
\pgfpathrectangle{\pgfqpoint{6.572727in}{0.474100in}}{\pgfqpoint{4.227273in}{3.318700in}}%
\pgfusepath{clip}%
\pgfsetbuttcap%
\pgfsetroundjoin%
\definecolor{currentfill}{rgb}{0.127568,0.566949,0.550556}%
\pgfsetfillcolor{currentfill}%
\pgfsetfillopacity{0.700000}%
\pgfsetlinewidth{0.000000pt}%
\definecolor{currentstroke}{rgb}{0.000000,0.000000,0.000000}%
\pgfsetstrokecolor{currentstroke}%
\pgfsetstrokeopacity{0.700000}%
\pgfsetdash{}{0pt}%
\pgfpathmoveto{\pgfqpoint{7.520916in}{1.256808in}}%
\pgfpathcurveto{\pgfqpoint{7.525959in}{1.256808in}}{\pgfqpoint{7.530797in}{1.258812in}}{\pgfqpoint{7.534363in}{1.262378in}}%
\pgfpathcurveto{\pgfqpoint{7.537930in}{1.265945in}}{\pgfqpoint{7.539934in}{1.270782in}}{\pgfqpoint{7.539934in}{1.275826in}}%
\pgfpathcurveto{\pgfqpoint{7.539934in}{1.280870in}}{\pgfqpoint{7.537930in}{1.285707in}}{\pgfqpoint{7.534363in}{1.289274in}}%
\pgfpathcurveto{\pgfqpoint{7.530797in}{1.292840in}}{\pgfqpoint{7.525959in}{1.294844in}}{\pgfqpoint{7.520916in}{1.294844in}}%
\pgfpathcurveto{\pgfqpoint{7.515872in}{1.294844in}}{\pgfqpoint{7.511034in}{1.292840in}}{\pgfqpoint{7.507468in}{1.289274in}}%
\pgfpathcurveto{\pgfqpoint{7.503901in}{1.285707in}}{\pgfqpoint{7.501897in}{1.280870in}}{\pgfqpoint{7.501897in}{1.275826in}}%
\pgfpathcurveto{\pgfqpoint{7.501897in}{1.270782in}}{\pgfqpoint{7.503901in}{1.265945in}}{\pgfqpoint{7.507468in}{1.262378in}}%
\pgfpathcurveto{\pgfqpoint{7.511034in}{1.258812in}}{\pgfqpoint{7.515872in}{1.256808in}}{\pgfqpoint{7.520916in}{1.256808in}}%
\pgfpathclose%
\pgfusepath{fill}%
\end{pgfscope}%
\begin{pgfscope}%
\pgfpathrectangle{\pgfqpoint{6.572727in}{0.474100in}}{\pgfqpoint{4.227273in}{3.318700in}}%
\pgfusepath{clip}%
\pgfsetbuttcap%
\pgfsetroundjoin%
\definecolor{currentfill}{rgb}{0.127568,0.566949,0.550556}%
\pgfsetfillcolor{currentfill}%
\pgfsetfillopacity{0.700000}%
\pgfsetlinewidth{0.000000pt}%
\definecolor{currentstroke}{rgb}{0.000000,0.000000,0.000000}%
\pgfsetstrokecolor{currentstroke}%
\pgfsetstrokeopacity{0.700000}%
\pgfsetdash{}{0pt}%
\pgfpathmoveto{\pgfqpoint{7.748101in}{1.562919in}}%
\pgfpathcurveto{\pgfqpoint{7.753145in}{1.562919in}}{\pgfqpoint{7.757983in}{1.564922in}}{\pgfqpoint{7.761549in}{1.568489in}}%
\pgfpathcurveto{\pgfqpoint{7.765116in}{1.572055in}}{\pgfqpoint{7.767120in}{1.576893in}}{\pgfqpoint{7.767120in}{1.581937in}}%
\pgfpathcurveto{\pgfqpoint{7.767120in}{1.586980in}}{\pgfqpoint{7.765116in}{1.591818in}}{\pgfqpoint{7.761549in}{1.595385in}}%
\pgfpathcurveto{\pgfqpoint{7.757983in}{1.598951in}}{\pgfqpoint{7.753145in}{1.600955in}}{\pgfqpoint{7.748101in}{1.600955in}}%
\pgfpathcurveto{\pgfqpoint{7.743058in}{1.600955in}}{\pgfqpoint{7.738220in}{1.598951in}}{\pgfqpoint{7.734654in}{1.595385in}}%
\pgfpathcurveto{\pgfqpoint{7.731087in}{1.591818in}}{\pgfqpoint{7.729083in}{1.586980in}}{\pgfqpoint{7.729083in}{1.581937in}}%
\pgfpathcurveto{\pgfqpoint{7.729083in}{1.576893in}}{\pgfqpoint{7.731087in}{1.572055in}}{\pgfqpoint{7.734654in}{1.568489in}}%
\pgfpathcurveto{\pgfqpoint{7.738220in}{1.564922in}}{\pgfqpoint{7.743058in}{1.562919in}}{\pgfqpoint{7.748101in}{1.562919in}}%
\pgfpathclose%
\pgfusepath{fill}%
\end{pgfscope}%
\begin{pgfscope}%
\pgfpathrectangle{\pgfqpoint{6.572727in}{0.474100in}}{\pgfqpoint{4.227273in}{3.318700in}}%
\pgfusepath{clip}%
\pgfsetbuttcap%
\pgfsetroundjoin%
\definecolor{currentfill}{rgb}{0.127568,0.566949,0.550556}%
\pgfsetfillcolor{currentfill}%
\pgfsetfillopacity{0.700000}%
\pgfsetlinewidth{0.000000pt}%
\definecolor{currentstroke}{rgb}{0.000000,0.000000,0.000000}%
\pgfsetstrokecolor{currentstroke}%
\pgfsetstrokeopacity{0.700000}%
\pgfsetdash{}{0pt}%
\pgfpathmoveto{\pgfqpoint{8.681375in}{2.920236in}}%
\pgfpathcurveto{\pgfqpoint{8.686419in}{2.920236in}}{\pgfqpoint{8.691256in}{2.922240in}}{\pgfqpoint{8.694823in}{2.925807in}}%
\pgfpathcurveto{\pgfqpoint{8.698389in}{2.929373in}}{\pgfqpoint{8.700393in}{2.934211in}}{\pgfqpoint{8.700393in}{2.939255in}}%
\pgfpathcurveto{\pgfqpoint{8.700393in}{2.944298in}}{\pgfqpoint{8.698389in}{2.949136in}}{\pgfqpoint{8.694823in}{2.952702in}}%
\pgfpathcurveto{\pgfqpoint{8.691256in}{2.956269in}}{\pgfqpoint{8.686419in}{2.958273in}}{\pgfqpoint{8.681375in}{2.958273in}}%
\pgfpathcurveto{\pgfqpoint{8.676331in}{2.958273in}}{\pgfqpoint{8.671493in}{2.956269in}}{\pgfqpoint{8.667927in}{2.952702in}}%
\pgfpathcurveto{\pgfqpoint{8.664361in}{2.949136in}}{\pgfqpoint{8.662357in}{2.944298in}}{\pgfqpoint{8.662357in}{2.939255in}}%
\pgfpathcurveto{\pgfqpoint{8.662357in}{2.934211in}}{\pgfqpoint{8.664361in}{2.929373in}}{\pgfqpoint{8.667927in}{2.925807in}}%
\pgfpathcurveto{\pgfqpoint{8.671493in}{2.922240in}}{\pgfqpoint{8.676331in}{2.920236in}}{\pgfqpoint{8.681375in}{2.920236in}}%
\pgfpathclose%
\pgfusepath{fill}%
\end{pgfscope}%
\begin{pgfscope}%
\pgfpathrectangle{\pgfqpoint{6.572727in}{0.474100in}}{\pgfqpoint{4.227273in}{3.318700in}}%
\pgfusepath{clip}%
\pgfsetbuttcap%
\pgfsetroundjoin%
\definecolor{currentfill}{rgb}{0.993248,0.906157,0.143936}%
\pgfsetfillcolor{currentfill}%
\pgfsetfillopacity{0.700000}%
\pgfsetlinewidth{0.000000pt}%
\definecolor{currentstroke}{rgb}{0.000000,0.000000,0.000000}%
\pgfsetstrokecolor{currentstroke}%
\pgfsetstrokeopacity{0.700000}%
\pgfsetdash{}{0pt}%
\pgfpathmoveto{\pgfqpoint{9.373092in}{1.367219in}}%
\pgfpathcurveto{\pgfqpoint{9.378136in}{1.367219in}}{\pgfqpoint{9.382973in}{1.369223in}}{\pgfqpoint{9.386540in}{1.372789in}}%
\pgfpathcurveto{\pgfqpoint{9.390106in}{1.376355in}}{\pgfqpoint{9.392110in}{1.381193in}}{\pgfqpoint{9.392110in}{1.386237in}}%
\pgfpathcurveto{\pgfqpoint{9.392110in}{1.391280in}}{\pgfqpoint{9.390106in}{1.396118in}}{\pgfqpoint{9.386540in}{1.399685in}}%
\pgfpathcurveto{\pgfqpoint{9.382973in}{1.403251in}}{\pgfqpoint{9.378136in}{1.405255in}}{\pgfqpoint{9.373092in}{1.405255in}}%
\pgfpathcurveto{\pgfqpoint{9.368048in}{1.405255in}}{\pgfqpoint{9.363210in}{1.403251in}}{\pgfqpoint{9.359644in}{1.399685in}}%
\pgfpathcurveto{\pgfqpoint{9.356078in}{1.396118in}}{\pgfqpoint{9.354074in}{1.391280in}}{\pgfqpoint{9.354074in}{1.386237in}}%
\pgfpathcurveto{\pgfqpoint{9.354074in}{1.381193in}}{\pgfqpoint{9.356078in}{1.376355in}}{\pgfqpoint{9.359644in}{1.372789in}}%
\pgfpathcurveto{\pgfqpoint{9.363210in}{1.369223in}}{\pgfqpoint{9.368048in}{1.367219in}}{\pgfqpoint{9.373092in}{1.367219in}}%
\pgfpathclose%
\pgfusepath{fill}%
\end{pgfscope}%
\begin{pgfscope}%
\pgfpathrectangle{\pgfqpoint{6.572727in}{0.474100in}}{\pgfqpoint{4.227273in}{3.318700in}}%
\pgfusepath{clip}%
\pgfsetbuttcap%
\pgfsetroundjoin%
\definecolor{currentfill}{rgb}{0.127568,0.566949,0.550556}%
\pgfsetfillcolor{currentfill}%
\pgfsetfillopacity{0.700000}%
\pgfsetlinewidth{0.000000pt}%
\definecolor{currentstroke}{rgb}{0.000000,0.000000,0.000000}%
\pgfsetstrokecolor{currentstroke}%
\pgfsetstrokeopacity{0.700000}%
\pgfsetdash{}{0pt}%
\pgfpathmoveto{\pgfqpoint{8.099575in}{2.425162in}}%
\pgfpathcurveto{\pgfqpoint{8.104618in}{2.425162in}}{\pgfqpoint{8.109456in}{2.427166in}}{\pgfqpoint{8.113023in}{2.430732in}}%
\pgfpathcurveto{\pgfqpoint{8.116589in}{2.434299in}}{\pgfqpoint{8.118593in}{2.439137in}}{\pgfqpoint{8.118593in}{2.444180in}}%
\pgfpathcurveto{\pgfqpoint{8.118593in}{2.449224in}}{\pgfqpoint{8.116589in}{2.454062in}}{\pgfqpoint{8.113023in}{2.457628in}}%
\pgfpathcurveto{\pgfqpoint{8.109456in}{2.461194in}}{\pgfqpoint{8.104618in}{2.463198in}}{\pgfqpoint{8.099575in}{2.463198in}}%
\pgfpathcurveto{\pgfqpoint{8.094531in}{2.463198in}}{\pgfqpoint{8.089693in}{2.461194in}}{\pgfqpoint{8.086127in}{2.457628in}}%
\pgfpathcurveto{\pgfqpoint{8.082561in}{2.454062in}}{\pgfqpoint{8.080557in}{2.449224in}}{\pgfqpoint{8.080557in}{2.444180in}}%
\pgfpathcurveto{\pgfqpoint{8.080557in}{2.439137in}}{\pgfqpoint{8.082561in}{2.434299in}}{\pgfqpoint{8.086127in}{2.430732in}}%
\pgfpathcurveto{\pgfqpoint{8.089693in}{2.427166in}}{\pgfqpoint{8.094531in}{2.425162in}}{\pgfqpoint{8.099575in}{2.425162in}}%
\pgfpathclose%
\pgfusepath{fill}%
\end{pgfscope}%
\begin{pgfscope}%
\pgfpathrectangle{\pgfqpoint{6.572727in}{0.474100in}}{\pgfqpoint{4.227273in}{3.318700in}}%
\pgfusepath{clip}%
\pgfsetbuttcap%
\pgfsetroundjoin%
\definecolor{currentfill}{rgb}{0.127568,0.566949,0.550556}%
\pgfsetfillcolor{currentfill}%
\pgfsetfillopacity{0.700000}%
\pgfsetlinewidth{0.000000pt}%
\definecolor{currentstroke}{rgb}{0.000000,0.000000,0.000000}%
\pgfsetstrokecolor{currentstroke}%
\pgfsetstrokeopacity{0.700000}%
\pgfsetdash{}{0pt}%
\pgfpathmoveto{\pgfqpoint{7.860710in}{3.220093in}}%
\pgfpathcurveto{\pgfqpoint{7.865754in}{3.220093in}}{\pgfqpoint{7.870592in}{3.222097in}}{\pgfqpoint{7.874158in}{3.225663in}}%
\pgfpathcurveto{\pgfqpoint{7.877725in}{3.229229in}}{\pgfqpoint{7.879728in}{3.234067in}}{\pgfqpoint{7.879728in}{3.239111in}}%
\pgfpathcurveto{\pgfqpoint{7.879728in}{3.244154in}}{\pgfqpoint{7.877725in}{3.248992in}}{\pgfqpoint{7.874158in}{3.252559in}}%
\pgfpathcurveto{\pgfqpoint{7.870592in}{3.256125in}}{\pgfqpoint{7.865754in}{3.258129in}}{\pgfqpoint{7.860710in}{3.258129in}}%
\pgfpathcurveto{\pgfqpoint{7.855667in}{3.258129in}}{\pgfqpoint{7.850829in}{3.256125in}}{\pgfqpoint{7.847262in}{3.252559in}}%
\pgfpathcurveto{\pgfqpoint{7.843696in}{3.248992in}}{\pgfqpoint{7.841692in}{3.244154in}}{\pgfqpoint{7.841692in}{3.239111in}}%
\pgfpathcurveto{\pgfqpoint{7.841692in}{3.234067in}}{\pgfqpoint{7.843696in}{3.229229in}}{\pgfqpoint{7.847262in}{3.225663in}}%
\pgfpathcurveto{\pgfqpoint{7.850829in}{3.222097in}}{\pgfqpoint{7.855667in}{3.220093in}}{\pgfqpoint{7.860710in}{3.220093in}}%
\pgfpathclose%
\pgfusepath{fill}%
\end{pgfscope}%
\begin{pgfscope}%
\pgfpathrectangle{\pgfqpoint{6.572727in}{0.474100in}}{\pgfqpoint{4.227273in}{3.318700in}}%
\pgfusepath{clip}%
\pgfsetbuttcap%
\pgfsetroundjoin%
\definecolor{currentfill}{rgb}{0.127568,0.566949,0.550556}%
\pgfsetfillcolor{currentfill}%
\pgfsetfillopacity{0.700000}%
\pgfsetlinewidth{0.000000pt}%
\definecolor{currentstroke}{rgb}{0.000000,0.000000,0.000000}%
\pgfsetstrokecolor{currentstroke}%
\pgfsetstrokeopacity{0.700000}%
\pgfsetdash{}{0pt}%
\pgfpathmoveto{\pgfqpoint{7.780012in}{1.831899in}}%
\pgfpathcurveto{\pgfqpoint{7.785056in}{1.831899in}}{\pgfqpoint{7.789894in}{1.833902in}}{\pgfqpoint{7.793460in}{1.837469in}}%
\pgfpathcurveto{\pgfqpoint{7.797027in}{1.841035in}}{\pgfqpoint{7.799030in}{1.845873in}}{\pgfqpoint{7.799030in}{1.850917in}}%
\pgfpathcurveto{\pgfqpoint{7.799030in}{1.855960in}}{\pgfqpoint{7.797027in}{1.860798in}}{\pgfqpoint{7.793460in}{1.864365in}}%
\pgfpathcurveto{\pgfqpoint{7.789894in}{1.867931in}}{\pgfqpoint{7.785056in}{1.869935in}}{\pgfqpoint{7.780012in}{1.869935in}}%
\pgfpathcurveto{\pgfqpoint{7.774969in}{1.869935in}}{\pgfqpoint{7.770131in}{1.867931in}}{\pgfqpoint{7.766564in}{1.864365in}}%
\pgfpathcurveto{\pgfqpoint{7.762998in}{1.860798in}}{\pgfqpoint{7.760994in}{1.855960in}}{\pgfqpoint{7.760994in}{1.850917in}}%
\pgfpathcurveto{\pgfqpoint{7.760994in}{1.845873in}}{\pgfqpoint{7.762998in}{1.841035in}}{\pgfqpoint{7.766564in}{1.837469in}}%
\pgfpathcurveto{\pgfqpoint{7.770131in}{1.833902in}}{\pgfqpoint{7.774969in}{1.831899in}}{\pgfqpoint{7.780012in}{1.831899in}}%
\pgfpathclose%
\pgfusepath{fill}%
\end{pgfscope}%
\begin{pgfscope}%
\pgfpathrectangle{\pgfqpoint{6.572727in}{0.474100in}}{\pgfqpoint{4.227273in}{3.318700in}}%
\pgfusepath{clip}%
\pgfsetbuttcap%
\pgfsetroundjoin%
\definecolor{currentfill}{rgb}{0.993248,0.906157,0.143936}%
\pgfsetfillcolor{currentfill}%
\pgfsetfillopacity{0.700000}%
\pgfsetlinewidth{0.000000pt}%
\definecolor{currentstroke}{rgb}{0.000000,0.000000,0.000000}%
\pgfsetstrokecolor{currentstroke}%
\pgfsetstrokeopacity{0.700000}%
\pgfsetdash{}{0pt}%
\pgfpathmoveto{\pgfqpoint{9.546283in}{2.084393in}}%
\pgfpathcurveto{\pgfqpoint{9.551327in}{2.084393in}}{\pgfqpoint{9.556165in}{2.086397in}}{\pgfqpoint{9.559731in}{2.089964in}}%
\pgfpathcurveto{\pgfqpoint{9.563298in}{2.093530in}}{\pgfqpoint{9.565302in}{2.098368in}}{\pgfqpoint{9.565302in}{2.103411in}}%
\pgfpathcurveto{\pgfqpoint{9.565302in}{2.108455in}}{\pgfqpoint{9.563298in}{2.113293in}}{\pgfqpoint{9.559731in}{2.116859in}}%
\pgfpathcurveto{\pgfqpoint{9.556165in}{2.120426in}}{\pgfqpoint{9.551327in}{2.122430in}}{\pgfqpoint{9.546283in}{2.122430in}}%
\pgfpathcurveto{\pgfqpoint{9.541240in}{2.122430in}}{\pgfqpoint{9.536402in}{2.120426in}}{\pgfqpoint{9.532836in}{2.116859in}}%
\pgfpathcurveto{\pgfqpoint{9.529269in}{2.113293in}}{\pgfqpoint{9.527265in}{2.108455in}}{\pgfqpoint{9.527265in}{2.103411in}}%
\pgfpathcurveto{\pgfqpoint{9.527265in}{2.098368in}}{\pgfqpoint{9.529269in}{2.093530in}}{\pgfqpoint{9.532836in}{2.089964in}}%
\pgfpathcurveto{\pgfqpoint{9.536402in}{2.086397in}}{\pgfqpoint{9.541240in}{2.084393in}}{\pgfqpoint{9.546283in}{2.084393in}}%
\pgfpathclose%
\pgfusepath{fill}%
\end{pgfscope}%
\begin{pgfscope}%
\pgfpathrectangle{\pgfqpoint{6.572727in}{0.474100in}}{\pgfqpoint{4.227273in}{3.318700in}}%
\pgfusepath{clip}%
\pgfsetbuttcap%
\pgfsetroundjoin%
\definecolor{currentfill}{rgb}{0.127568,0.566949,0.550556}%
\pgfsetfillcolor{currentfill}%
\pgfsetfillopacity{0.700000}%
\pgfsetlinewidth{0.000000pt}%
\definecolor{currentstroke}{rgb}{0.000000,0.000000,0.000000}%
\pgfsetstrokecolor{currentstroke}%
\pgfsetstrokeopacity{0.700000}%
\pgfsetdash{}{0pt}%
\pgfpathmoveto{\pgfqpoint{8.109146in}{2.733683in}}%
\pgfpathcurveto{\pgfqpoint{8.114190in}{2.733683in}}{\pgfqpoint{8.119028in}{2.735687in}}{\pgfqpoint{8.122594in}{2.739253in}}%
\pgfpathcurveto{\pgfqpoint{8.126161in}{2.742820in}}{\pgfqpoint{8.128164in}{2.747657in}}{\pgfqpoint{8.128164in}{2.752701in}}%
\pgfpathcurveto{\pgfqpoint{8.128164in}{2.757745in}}{\pgfqpoint{8.126161in}{2.762583in}}{\pgfqpoint{8.122594in}{2.766149in}}%
\pgfpathcurveto{\pgfqpoint{8.119028in}{2.769715in}}{\pgfqpoint{8.114190in}{2.771719in}}{\pgfqpoint{8.109146in}{2.771719in}}%
\pgfpathcurveto{\pgfqpoint{8.104103in}{2.771719in}}{\pgfqpoint{8.099265in}{2.769715in}}{\pgfqpoint{8.095698in}{2.766149in}}%
\pgfpathcurveto{\pgfqpoint{8.092132in}{2.762583in}}{\pgfqpoint{8.090128in}{2.757745in}}{\pgfqpoint{8.090128in}{2.752701in}}%
\pgfpathcurveto{\pgfqpoint{8.090128in}{2.747657in}}{\pgfqpoint{8.092132in}{2.742820in}}{\pgfqpoint{8.095698in}{2.739253in}}%
\pgfpathcurveto{\pgfqpoint{8.099265in}{2.735687in}}{\pgfqpoint{8.104103in}{2.733683in}}{\pgfqpoint{8.109146in}{2.733683in}}%
\pgfpathclose%
\pgfusepath{fill}%
\end{pgfscope}%
\begin{pgfscope}%
\pgfpathrectangle{\pgfqpoint{6.572727in}{0.474100in}}{\pgfqpoint{4.227273in}{3.318700in}}%
\pgfusepath{clip}%
\pgfsetbuttcap%
\pgfsetroundjoin%
\definecolor{currentfill}{rgb}{0.127568,0.566949,0.550556}%
\pgfsetfillcolor{currentfill}%
\pgfsetfillopacity{0.700000}%
\pgfsetlinewidth{0.000000pt}%
\definecolor{currentstroke}{rgb}{0.000000,0.000000,0.000000}%
\pgfsetstrokecolor{currentstroke}%
\pgfsetstrokeopacity{0.700000}%
\pgfsetdash{}{0pt}%
\pgfpathmoveto{\pgfqpoint{7.675055in}{1.437400in}}%
\pgfpathcurveto{\pgfqpoint{7.680099in}{1.437400in}}{\pgfqpoint{7.684937in}{1.439404in}}{\pgfqpoint{7.688503in}{1.442971in}}%
\pgfpathcurveto{\pgfqpoint{7.692069in}{1.446537in}}{\pgfqpoint{7.694073in}{1.451375in}}{\pgfqpoint{7.694073in}{1.456419in}}%
\pgfpathcurveto{\pgfqpoint{7.694073in}{1.461462in}}{\pgfqpoint{7.692069in}{1.466300in}}{\pgfqpoint{7.688503in}{1.469866in}}%
\pgfpathcurveto{\pgfqpoint{7.684937in}{1.473433in}}{\pgfqpoint{7.680099in}{1.475437in}}{\pgfqpoint{7.675055in}{1.475437in}}%
\pgfpathcurveto{\pgfqpoint{7.670012in}{1.475437in}}{\pgfqpoint{7.665174in}{1.473433in}}{\pgfqpoint{7.661607in}{1.469866in}}%
\pgfpathcurveto{\pgfqpoint{7.658041in}{1.466300in}}{\pgfqpoint{7.656037in}{1.461462in}}{\pgfqpoint{7.656037in}{1.456419in}}%
\pgfpathcurveto{\pgfqpoint{7.656037in}{1.451375in}}{\pgfqpoint{7.658041in}{1.446537in}}{\pgfqpoint{7.661607in}{1.442971in}}%
\pgfpathcurveto{\pgfqpoint{7.665174in}{1.439404in}}{\pgfqpoint{7.670012in}{1.437400in}}{\pgfqpoint{7.675055in}{1.437400in}}%
\pgfpathclose%
\pgfusepath{fill}%
\end{pgfscope}%
\begin{pgfscope}%
\pgfpathrectangle{\pgfqpoint{6.572727in}{0.474100in}}{\pgfqpoint{4.227273in}{3.318700in}}%
\pgfusepath{clip}%
\pgfsetbuttcap%
\pgfsetroundjoin%
\definecolor{currentfill}{rgb}{0.993248,0.906157,0.143936}%
\pgfsetfillcolor{currentfill}%
\pgfsetfillopacity{0.700000}%
\pgfsetlinewidth{0.000000pt}%
\definecolor{currentstroke}{rgb}{0.000000,0.000000,0.000000}%
\pgfsetstrokecolor{currentstroke}%
\pgfsetstrokeopacity{0.700000}%
\pgfsetdash{}{0pt}%
\pgfpathmoveto{\pgfqpoint{9.871804in}{1.994796in}}%
\pgfpathcurveto{\pgfqpoint{9.876847in}{1.994796in}}{\pgfqpoint{9.881685in}{1.996800in}}{\pgfqpoint{9.885252in}{2.000367in}}%
\pgfpathcurveto{\pgfqpoint{9.888818in}{2.003933in}}{\pgfqpoint{9.890822in}{2.008771in}}{\pgfqpoint{9.890822in}{2.013814in}}%
\pgfpathcurveto{\pgfqpoint{9.890822in}{2.018858in}}{\pgfqpoint{9.888818in}{2.023696in}}{\pgfqpoint{9.885252in}{2.027262in}}%
\pgfpathcurveto{\pgfqpoint{9.881685in}{2.030829in}}{\pgfqpoint{9.876847in}{2.032833in}}{\pgfqpoint{9.871804in}{2.032833in}}%
\pgfpathcurveto{\pgfqpoint{9.866760in}{2.032833in}}{\pgfqpoint{9.861922in}{2.030829in}}{\pgfqpoint{9.858356in}{2.027262in}}%
\pgfpathcurveto{\pgfqpoint{9.854790in}{2.023696in}}{\pgfqpoint{9.852786in}{2.018858in}}{\pgfqpoint{9.852786in}{2.013814in}}%
\pgfpathcurveto{\pgfqpoint{9.852786in}{2.008771in}}{\pgfqpoint{9.854790in}{2.003933in}}{\pgfqpoint{9.858356in}{2.000367in}}%
\pgfpathcurveto{\pgfqpoint{9.861922in}{1.996800in}}{\pgfqpoint{9.866760in}{1.994796in}}{\pgfqpoint{9.871804in}{1.994796in}}%
\pgfpathclose%
\pgfusepath{fill}%
\end{pgfscope}%
\begin{pgfscope}%
\pgfpathrectangle{\pgfqpoint{6.572727in}{0.474100in}}{\pgfqpoint{4.227273in}{3.318700in}}%
\pgfusepath{clip}%
\pgfsetbuttcap%
\pgfsetroundjoin%
\definecolor{currentfill}{rgb}{0.127568,0.566949,0.550556}%
\pgfsetfillcolor{currentfill}%
\pgfsetfillopacity{0.700000}%
\pgfsetlinewidth{0.000000pt}%
\definecolor{currentstroke}{rgb}{0.000000,0.000000,0.000000}%
\pgfsetstrokecolor{currentstroke}%
\pgfsetstrokeopacity{0.700000}%
\pgfsetdash{}{0pt}%
\pgfpathmoveto{\pgfqpoint{8.783733in}{3.099844in}}%
\pgfpathcurveto{\pgfqpoint{8.788777in}{3.099844in}}{\pgfqpoint{8.793615in}{3.101848in}}{\pgfqpoint{8.797181in}{3.105415in}}%
\pgfpathcurveto{\pgfqpoint{8.800747in}{3.108981in}}{\pgfqpoint{8.802751in}{3.113819in}}{\pgfqpoint{8.802751in}{3.118862in}}%
\pgfpathcurveto{\pgfqpoint{8.802751in}{3.123906in}}{\pgfqpoint{8.800747in}{3.128744in}}{\pgfqpoint{8.797181in}{3.132310in}}%
\pgfpathcurveto{\pgfqpoint{8.793615in}{3.135877in}}{\pgfqpoint{8.788777in}{3.137881in}}{\pgfqpoint{8.783733in}{3.137881in}}%
\pgfpathcurveto{\pgfqpoint{8.778690in}{3.137881in}}{\pgfqpoint{8.773852in}{3.135877in}}{\pgfqpoint{8.770285in}{3.132310in}}%
\pgfpathcurveto{\pgfqpoint{8.766719in}{3.128744in}}{\pgfqpoint{8.764715in}{3.123906in}}{\pgfqpoint{8.764715in}{3.118862in}}%
\pgfpathcurveto{\pgfqpoint{8.764715in}{3.113819in}}{\pgfqpoint{8.766719in}{3.108981in}}{\pgfqpoint{8.770285in}{3.105415in}}%
\pgfpathcurveto{\pgfqpoint{8.773852in}{3.101848in}}{\pgfqpoint{8.778690in}{3.099844in}}{\pgfqpoint{8.783733in}{3.099844in}}%
\pgfpathclose%
\pgfusepath{fill}%
\end{pgfscope}%
\begin{pgfscope}%
\pgfpathrectangle{\pgfqpoint{6.572727in}{0.474100in}}{\pgfqpoint{4.227273in}{3.318700in}}%
\pgfusepath{clip}%
\pgfsetbuttcap%
\pgfsetroundjoin%
\definecolor{currentfill}{rgb}{0.127568,0.566949,0.550556}%
\pgfsetfillcolor{currentfill}%
\pgfsetfillopacity{0.700000}%
\pgfsetlinewidth{0.000000pt}%
\definecolor{currentstroke}{rgb}{0.000000,0.000000,0.000000}%
\pgfsetstrokecolor{currentstroke}%
\pgfsetstrokeopacity{0.700000}%
\pgfsetdash{}{0pt}%
\pgfpathmoveto{\pgfqpoint{7.678613in}{2.835169in}}%
\pgfpathcurveto{\pgfqpoint{7.683657in}{2.835169in}}{\pgfqpoint{7.688494in}{2.837173in}}{\pgfqpoint{7.692061in}{2.840739in}}%
\pgfpathcurveto{\pgfqpoint{7.695627in}{2.844305in}}{\pgfqpoint{7.697631in}{2.849143in}}{\pgfqpoint{7.697631in}{2.854187in}}%
\pgfpathcurveto{\pgfqpoint{7.697631in}{2.859231in}}{\pgfqpoint{7.695627in}{2.864068in}}{\pgfqpoint{7.692061in}{2.867635in}}%
\pgfpathcurveto{\pgfqpoint{7.688494in}{2.871201in}}{\pgfqpoint{7.683657in}{2.873205in}}{\pgfqpoint{7.678613in}{2.873205in}}%
\pgfpathcurveto{\pgfqpoint{7.673569in}{2.873205in}}{\pgfqpoint{7.668731in}{2.871201in}}{\pgfqpoint{7.665165in}{2.867635in}}%
\pgfpathcurveto{\pgfqpoint{7.661599in}{2.864068in}}{\pgfqpoint{7.659595in}{2.859231in}}{\pgfqpoint{7.659595in}{2.854187in}}%
\pgfpathcurveto{\pgfqpoint{7.659595in}{2.849143in}}{\pgfqpoint{7.661599in}{2.844305in}}{\pgfqpoint{7.665165in}{2.840739in}}%
\pgfpathcurveto{\pgfqpoint{7.668731in}{2.837173in}}{\pgfqpoint{7.673569in}{2.835169in}}{\pgfqpoint{7.678613in}{2.835169in}}%
\pgfpathclose%
\pgfusepath{fill}%
\end{pgfscope}%
\begin{pgfscope}%
\pgfpathrectangle{\pgfqpoint{6.572727in}{0.474100in}}{\pgfqpoint{4.227273in}{3.318700in}}%
\pgfusepath{clip}%
\pgfsetbuttcap%
\pgfsetroundjoin%
\definecolor{currentfill}{rgb}{0.993248,0.906157,0.143936}%
\pgfsetfillcolor{currentfill}%
\pgfsetfillopacity{0.700000}%
\pgfsetlinewidth{0.000000pt}%
\definecolor{currentstroke}{rgb}{0.000000,0.000000,0.000000}%
\pgfsetstrokecolor{currentstroke}%
\pgfsetstrokeopacity{0.700000}%
\pgfsetdash{}{0pt}%
\pgfpathmoveto{\pgfqpoint{9.317019in}{1.614245in}}%
\pgfpathcurveto{\pgfqpoint{9.322062in}{1.614245in}}{\pgfqpoint{9.326900in}{1.616248in}}{\pgfqpoint{9.330466in}{1.619815in}}%
\pgfpathcurveto{\pgfqpoint{9.334033in}{1.623381in}}{\pgfqpoint{9.336037in}{1.628219in}}{\pgfqpoint{9.336037in}{1.633263in}}%
\pgfpathcurveto{\pgfqpoint{9.336037in}{1.638306in}}{\pgfqpoint{9.334033in}{1.643144in}}{\pgfqpoint{9.330466in}{1.646711in}}%
\pgfpathcurveto{\pgfqpoint{9.326900in}{1.650277in}}{\pgfqpoint{9.322062in}{1.652281in}}{\pgfqpoint{9.317019in}{1.652281in}}%
\pgfpathcurveto{\pgfqpoint{9.311975in}{1.652281in}}{\pgfqpoint{9.307137in}{1.650277in}}{\pgfqpoint{9.303571in}{1.646711in}}%
\pgfpathcurveto{\pgfqpoint{9.300004in}{1.643144in}}{\pgfqpoint{9.298000in}{1.638306in}}{\pgfqpoint{9.298000in}{1.633263in}}%
\pgfpathcurveto{\pgfqpoint{9.298000in}{1.628219in}}{\pgfqpoint{9.300004in}{1.623381in}}{\pgfqpoint{9.303571in}{1.619815in}}%
\pgfpathcurveto{\pgfqpoint{9.307137in}{1.616248in}}{\pgfqpoint{9.311975in}{1.614245in}}{\pgfqpoint{9.317019in}{1.614245in}}%
\pgfpathclose%
\pgfusepath{fill}%
\end{pgfscope}%
\begin{pgfscope}%
\pgfpathrectangle{\pgfqpoint{6.572727in}{0.474100in}}{\pgfqpoint{4.227273in}{3.318700in}}%
\pgfusepath{clip}%
\pgfsetbuttcap%
\pgfsetroundjoin%
\definecolor{currentfill}{rgb}{0.127568,0.566949,0.550556}%
\pgfsetfillcolor{currentfill}%
\pgfsetfillopacity{0.700000}%
\pgfsetlinewidth{0.000000pt}%
\definecolor{currentstroke}{rgb}{0.000000,0.000000,0.000000}%
\pgfsetstrokecolor{currentstroke}%
\pgfsetstrokeopacity{0.700000}%
\pgfsetdash{}{0pt}%
\pgfpathmoveto{\pgfqpoint{8.354403in}{3.258737in}}%
\pgfpathcurveto{\pgfqpoint{8.359447in}{3.258737in}}{\pgfqpoint{8.364285in}{3.260741in}}{\pgfqpoint{8.367851in}{3.264307in}}%
\pgfpathcurveto{\pgfqpoint{8.371417in}{3.267873in}}{\pgfqpoint{8.373421in}{3.272711in}}{\pgfqpoint{8.373421in}{3.277755in}}%
\pgfpathcurveto{\pgfqpoint{8.373421in}{3.282799in}}{\pgfqpoint{8.371417in}{3.287636in}}{\pgfqpoint{8.367851in}{3.291203in}}%
\pgfpathcurveto{\pgfqpoint{8.364285in}{3.294769in}}{\pgfqpoint{8.359447in}{3.296773in}}{\pgfqpoint{8.354403in}{3.296773in}}%
\pgfpathcurveto{\pgfqpoint{8.349359in}{3.296773in}}{\pgfqpoint{8.344522in}{3.294769in}}{\pgfqpoint{8.340955in}{3.291203in}}%
\pgfpathcurveto{\pgfqpoint{8.337389in}{3.287636in}}{\pgfqpoint{8.335385in}{3.282799in}}{\pgfqpoint{8.335385in}{3.277755in}}%
\pgfpathcurveto{\pgfqpoint{8.335385in}{3.272711in}}{\pgfqpoint{8.337389in}{3.267873in}}{\pgfqpoint{8.340955in}{3.264307in}}%
\pgfpathcurveto{\pgfqpoint{8.344522in}{3.260741in}}{\pgfqpoint{8.349359in}{3.258737in}}{\pgfqpoint{8.354403in}{3.258737in}}%
\pgfpathclose%
\pgfusepath{fill}%
\end{pgfscope}%
\begin{pgfscope}%
\pgfpathrectangle{\pgfqpoint{6.572727in}{0.474100in}}{\pgfqpoint{4.227273in}{3.318700in}}%
\pgfusepath{clip}%
\pgfsetbuttcap%
\pgfsetroundjoin%
\definecolor{currentfill}{rgb}{0.993248,0.906157,0.143936}%
\pgfsetfillcolor{currentfill}%
\pgfsetfillopacity{0.700000}%
\pgfsetlinewidth{0.000000pt}%
\definecolor{currentstroke}{rgb}{0.000000,0.000000,0.000000}%
\pgfsetstrokecolor{currentstroke}%
\pgfsetstrokeopacity{0.700000}%
\pgfsetdash{}{0pt}%
\pgfpathmoveto{\pgfqpoint{9.600668in}{1.649703in}}%
\pgfpathcurveto{\pgfqpoint{9.605712in}{1.649703in}}{\pgfqpoint{9.610550in}{1.651707in}}{\pgfqpoint{9.614116in}{1.655274in}}%
\pgfpathcurveto{\pgfqpoint{9.617682in}{1.658840in}}{\pgfqpoint{9.619686in}{1.663678in}}{\pgfqpoint{9.619686in}{1.668722in}}%
\pgfpathcurveto{\pgfqpoint{9.619686in}{1.673765in}}{\pgfqpoint{9.617682in}{1.678603in}}{\pgfqpoint{9.614116in}{1.682169in}}%
\pgfpathcurveto{\pgfqpoint{9.610550in}{1.685736in}}{\pgfqpoint{9.605712in}{1.687740in}}{\pgfqpoint{9.600668in}{1.687740in}}%
\pgfpathcurveto{\pgfqpoint{9.595624in}{1.687740in}}{\pgfqpoint{9.590787in}{1.685736in}}{\pgfqpoint{9.587220in}{1.682169in}}%
\pgfpathcurveto{\pgfqpoint{9.583654in}{1.678603in}}{\pgfqpoint{9.581650in}{1.673765in}}{\pgfqpoint{9.581650in}{1.668722in}}%
\pgfpathcurveto{\pgfqpoint{9.581650in}{1.663678in}}{\pgfqpoint{9.583654in}{1.658840in}}{\pgfqpoint{9.587220in}{1.655274in}}%
\pgfpathcurveto{\pgfqpoint{9.590787in}{1.651707in}}{\pgfqpoint{9.595624in}{1.649703in}}{\pgfqpoint{9.600668in}{1.649703in}}%
\pgfpathclose%
\pgfusepath{fill}%
\end{pgfscope}%
\begin{pgfscope}%
\pgfpathrectangle{\pgfqpoint{6.572727in}{0.474100in}}{\pgfqpoint{4.227273in}{3.318700in}}%
\pgfusepath{clip}%
\pgfsetbuttcap%
\pgfsetroundjoin%
\definecolor{currentfill}{rgb}{0.127568,0.566949,0.550556}%
\pgfsetfillcolor{currentfill}%
\pgfsetfillopacity{0.700000}%
\pgfsetlinewidth{0.000000pt}%
\definecolor{currentstroke}{rgb}{0.000000,0.000000,0.000000}%
\pgfsetstrokecolor{currentstroke}%
\pgfsetstrokeopacity{0.700000}%
\pgfsetdash{}{0pt}%
\pgfpathmoveto{\pgfqpoint{8.470465in}{3.123202in}}%
\pgfpathcurveto{\pgfqpoint{8.475509in}{3.123202in}}{\pgfqpoint{8.480347in}{3.125206in}}{\pgfqpoint{8.483913in}{3.128772in}}%
\pgfpathcurveto{\pgfqpoint{8.487480in}{3.132339in}}{\pgfqpoint{8.489483in}{3.137176in}}{\pgfqpoint{8.489483in}{3.142220in}}%
\pgfpathcurveto{\pgfqpoint{8.489483in}{3.147264in}}{\pgfqpoint{8.487480in}{3.152101in}}{\pgfqpoint{8.483913in}{3.155668in}}%
\pgfpathcurveto{\pgfqpoint{8.480347in}{3.159234in}}{\pgfqpoint{8.475509in}{3.161238in}}{\pgfqpoint{8.470465in}{3.161238in}}%
\pgfpathcurveto{\pgfqpoint{8.465422in}{3.161238in}}{\pgfqpoint{8.460584in}{3.159234in}}{\pgfqpoint{8.457017in}{3.155668in}}%
\pgfpathcurveto{\pgfqpoint{8.453451in}{3.152101in}}{\pgfqpoint{8.451447in}{3.147264in}}{\pgfqpoint{8.451447in}{3.142220in}}%
\pgfpathcurveto{\pgfqpoint{8.451447in}{3.137176in}}{\pgfqpoint{8.453451in}{3.132339in}}{\pgfqpoint{8.457017in}{3.128772in}}%
\pgfpathcurveto{\pgfqpoint{8.460584in}{3.125206in}}{\pgfqpoint{8.465422in}{3.123202in}}{\pgfqpoint{8.470465in}{3.123202in}}%
\pgfpathclose%
\pgfusepath{fill}%
\end{pgfscope}%
\begin{pgfscope}%
\pgfpathrectangle{\pgfqpoint{6.572727in}{0.474100in}}{\pgfqpoint{4.227273in}{3.318700in}}%
\pgfusepath{clip}%
\pgfsetbuttcap%
\pgfsetroundjoin%
\definecolor{currentfill}{rgb}{0.993248,0.906157,0.143936}%
\pgfsetfillcolor{currentfill}%
\pgfsetfillopacity{0.700000}%
\pgfsetlinewidth{0.000000pt}%
\definecolor{currentstroke}{rgb}{0.000000,0.000000,0.000000}%
\pgfsetstrokecolor{currentstroke}%
\pgfsetstrokeopacity{0.700000}%
\pgfsetdash{}{0pt}%
\pgfpathmoveto{\pgfqpoint{9.776351in}{1.140567in}}%
\pgfpathcurveto{\pgfqpoint{9.781395in}{1.140567in}}{\pgfqpoint{9.786233in}{1.142571in}}{\pgfqpoint{9.789799in}{1.146137in}}%
\pgfpathcurveto{\pgfqpoint{9.793365in}{1.149704in}}{\pgfqpoint{9.795369in}{1.154542in}}{\pgfqpoint{9.795369in}{1.159585in}}%
\pgfpathcurveto{\pgfqpoint{9.795369in}{1.164629in}}{\pgfqpoint{9.793365in}{1.169467in}}{\pgfqpoint{9.789799in}{1.173033in}}%
\pgfpathcurveto{\pgfqpoint{9.786233in}{1.176600in}}{\pgfqpoint{9.781395in}{1.178603in}}{\pgfqpoint{9.776351in}{1.178603in}}%
\pgfpathcurveto{\pgfqpoint{9.771307in}{1.178603in}}{\pgfqpoint{9.766470in}{1.176600in}}{\pgfqpoint{9.762903in}{1.173033in}}%
\pgfpathcurveto{\pgfqpoint{9.759337in}{1.169467in}}{\pgfqpoint{9.757333in}{1.164629in}}{\pgfqpoint{9.757333in}{1.159585in}}%
\pgfpathcurveto{\pgfqpoint{9.757333in}{1.154542in}}{\pgfqpoint{9.759337in}{1.149704in}}{\pgfqpoint{9.762903in}{1.146137in}}%
\pgfpathcurveto{\pgfqpoint{9.766470in}{1.142571in}}{\pgfqpoint{9.771307in}{1.140567in}}{\pgfqpoint{9.776351in}{1.140567in}}%
\pgfpathclose%
\pgfusepath{fill}%
\end{pgfscope}%
\begin{pgfscope}%
\pgfpathrectangle{\pgfqpoint{6.572727in}{0.474100in}}{\pgfqpoint{4.227273in}{3.318700in}}%
\pgfusepath{clip}%
\pgfsetbuttcap%
\pgfsetroundjoin%
\definecolor{currentfill}{rgb}{0.993248,0.906157,0.143936}%
\pgfsetfillcolor{currentfill}%
\pgfsetfillopacity{0.700000}%
\pgfsetlinewidth{0.000000pt}%
\definecolor{currentstroke}{rgb}{0.000000,0.000000,0.000000}%
\pgfsetstrokecolor{currentstroke}%
\pgfsetstrokeopacity{0.700000}%
\pgfsetdash{}{0pt}%
\pgfpathmoveto{\pgfqpoint{9.637120in}{1.766562in}}%
\pgfpathcurveto{\pgfqpoint{9.642163in}{1.766562in}}{\pgfqpoint{9.647001in}{1.768565in}}{\pgfqpoint{9.650568in}{1.772132in}}%
\pgfpathcurveto{\pgfqpoint{9.654134in}{1.775698in}}{\pgfqpoint{9.656138in}{1.780536in}}{\pgfqpoint{9.656138in}{1.785580in}}%
\pgfpathcurveto{\pgfqpoint{9.656138in}{1.790623in}}{\pgfqpoint{9.654134in}{1.795461in}}{\pgfqpoint{9.650568in}{1.799028in}}%
\pgfpathcurveto{\pgfqpoint{9.647001in}{1.802594in}}{\pgfqpoint{9.642163in}{1.804598in}}{\pgfqpoint{9.637120in}{1.804598in}}%
\pgfpathcurveto{\pgfqpoint{9.632076in}{1.804598in}}{\pgfqpoint{9.627238in}{1.802594in}}{\pgfqpoint{9.623672in}{1.799028in}}%
\pgfpathcurveto{\pgfqpoint{9.620105in}{1.795461in}}{\pgfqpoint{9.618102in}{1.790623in}}{\pgfqpoint{9.618102in}{1.785580in}}%
\pgfpathcurveto{\pgfqpoint{9.618102in}{1.780536in}}{\pgfqpoint{9.620105in}{1.775698in}}{\pgfqpoint{9.623672in}{1.772132in}}%
\pgfpathcurveto{\pgfqpoint{9.627238in}{1.768565in}}{\pgfqpoint{9.632076in}{1.766562in}}{\pgfqpoint{9.637120in}{1.766562in}}%
\pgfpathclose%
\pgfusepath{fill}%
\end{pgfscope}%
\begin{pgfscope}%
\pgfpathrectangle{\pgfqpoint{6.572727in}{0.474100in}}{\pgfqpoint{4.227273in}{3.318700in}}%
\pgfusepath{clip}%
\pgfsetbuttcap%
\pgfsetroundjoin%
\definecolor{currentfill}{rgb}{0.127568,0.566949,0.550556}%
\pgfsetfillcolor{currentfill}%
\pgfsetfillopacity{0.700000}%
\pgfsetlinewidth{0.000000pt}%
\definecolor{currentstroke}{rgb}{0.000000,0.000000,0.000000}%
\pgfsetstrokecolor{currentstroke}%
\pgfsetstrokeopacity{0.700000}%
\pgfsetdash{}{0pt}%
\pgfpathmoveto{\pgfqpoint{7.540105in}{1.666095in}}%
\pgfpathcurveto{\pgfqpoint{7.545149in}{1.666095in}}{\pgfqpoint{7.549986in}{1.668099in}}{\pgfqpoint{7.553553in}{1.671665in}}%
\pgfpathcurveto{\pgfqpoint{7.557119in}{1.675232in}}{\pgfqpoint{7.559123in}{1.680070in}}{\pgfqpoint{7.559123in}{1.685113in}}%
\pgfpathcurveto{\pgfqpoint{7.559123in}{1.690157in}}{\pgfqpoint{7.557119in}{1.694995in}}{\pgfqpoint{7.553553in}{1.698561in}}%
\pgfpathcurveto{\pgfqpoint{7.549986in}{1.702127in}}{\pgfqpoint{7.545149in}{1.704131in}}{\pgfqpoint{7.540105in}{1.704131in}}%
\pgfpathcurveto{\pgfqpoint{7.535061in}{1.704131in}}{\pgfqpoint{7.530224in}{1.702127in}}{\pgfqpoint{7.526657in}{1.698561in}}%
\pgfpathcurveto{\pgfqpoint{7.523091in}{1.694995in}}{\pgfqpoint{7.521087in}{1.690157in}}{\pgfqpoint{7.521087in}{1.685113in}}%
\pgfpathcurveto{\pgfqpoint{7.521087in}{1.680070in}}{\pgfqpoint{7.523091in}{1.675232in}}{\pgfqpoint{7.526657in}{1.671665in}}%
\pgfpathcurveto{\pgfqpoint{7.530224in}{1.668099in}}{\pgfqpoint{7.535061in}{1.666095in}}{\pgfqpoint{7.540105in}{1.666095in}}%
\pgfpathclose%
\pgfusepath{fill}%
\end{pgfscope}%
\begin{pgfscope}%
\pgfpathrectangle{\pgfqpoint{6.572727in}{0.474100in}}{\pgfqpoint{4.227273in}{3.318700in}}%
\pgfusepath{clip}%
\pgfsetbuttcap%
\pgfsetroundjoin%
\definecolor{currentfill}{rgb}{0.127568,0.566949,0.550556}%
\pgfsetfillcolor{currentfill}%
\pgfsetfillopacity{0.700000}%
\pgfsetlinewidth{0.000000pt}%
\definecolor{currentstroke}{rgb}{0.000000,0.000000,0.000000}%
\pgfsetstrokecolor{currentstroke}%
\pgfsetstrokeopacity{0.700000}%
\pgfsetdash{}{0pt}%
\pgfpathmoveto{\pgfqpoint{7.233386in}{0.960342in}}%
\pgfpathcurveto{\pgfqpoint{7.238430in}{0.960342in}}{\pgfqpoint{7.243268in}{0.962346in}}{\pgfqpoint{7.246834in}{0.965912in}}%
\pgfpathcurveto{\pgfqpoint{7.250401in}{0.969479in}}{\pgfqpoint{7.252405in}{0.974316in}}{\pgfqpoint{7.252405in}{0.979360in}}%
\pgfpathcurveto{\pgfqpoint{7.252405in}{0.984404in}}{\pgfqpoint{7.250401in}{0.989241in}}{\pgfqpoint{7.246834in}{0.992808in}}%
\pgfpathcurveto{\pgfqpoint{7.243268in}{0.996374in}}{\pgfqpoint{7.238430in}{0.998378in}}{\pgfqpoint{7.233386in}{0.998378in}}%
\pgfpathcurveto{\pgfqpoint{7.228343in}{0.998378in}}{\pgfqpoint{7.223505in}{0.996374in}}{\pgfqpoint{7.219939in}{0.992808in}}%
\pgfpathcurveto{\pgfqpoint{7.216372in}{0.989241in}}{\pgfqpoint{7.214368in}{0.984404in}}{\pgfqpoint{7.214368in}{0.979360in}}%
\pgfpathcurveto{\pgfqpoint{7.214368in}{0.974316in}}{\pgfqpoint{7.216372in}{0.969479in}}{\pgfqpoint{7.219939in}{0.965912in}}%
\pgfpathcurveto{\pgfqpoint{7.223505in}{0.962346in}}{\pgfqpoint{7.228343in}{0.960342in}}{\pgfqpoint{7.233386in}{0.960342in}}%
\pgfpathclose%
\pgfusepath{fill}%
\end{pgfscope}%
\begin{pgfscope}%
\pgfpathrectangle{\pgfqpoint{6.572727in}{0.474100in}}{\pgfqpoint{4.227273in}{3.318700in}}%
\pgfusepath{clip}%
\pgfsetbuttcap%
\pgfsetroundjoin%
\definecolor{currentfill}{rgb}{0.127568,0.566949,0.550556}%
\pgfsetfillcolor{currentfill}%
\pgfsetfillopacity{0.700000}%
\pgfsetlinewidth{0.000000pt}%
\definecolor{currentstroke}{rgb}{0.000000,0.000000,0.000000}%
\pgfsetstrokecolor{currentstroke}%
\pgfsetstrokeopacity{0.700000}%
\pgfsetdash{}{0pt}%
\pgfpathmoveto{\pgfqpoint{8.024822in}{2.439757in}}%
\pgfpathcurveto{\pgfqpoint{8.029866in}{2.439757in}}{\pgfqpoint{8.034704in}{2.441761in}}{\pgfqpoint{8.038270in}{2.445327in}}%
\pgfpathcurveto{\pgfqpoint{8.041837in}{2.448894in}}{\pgfqpoint{8.043840in}{2.453731in}}{\pgfqpoint{8.043840in}{2.458775in}}%
\pgfpathcurveto{\pgfqpoint{8.043840in}{2.463819in}}{\pgfqpoint{8.041837in}{2.468657in}}{\pgfqpoint{8.038270in}{2.472223in}}%
\pgfpathcurveto{\pgfqpoint{8.034704in}{2.475789in}}{\pgfqpoint{8.029866in}{2.477793in}}{\pgfqpoint{8.024822in}{2.477793in}}%
\pgfpathcurveto{\pgfqpoint{8.019779in}{2.477793in}}{\pgfqpoint{8.014941in}{2.475789in}}{\pgfqpoint{8.011374in}{2.472223in}}%
\pgfpathcurveto{\pgfqpoint{8.007808in}{2.468657in}}{\pgfqpoint{8.005804in}{2.463819in}}{\pgfqpoint{8.005804in}{2.458775in}}%
\pgfpathcurveto{\pgfqpoint{8.005804in}{2.453731in}}{\pgfqpoint{8.007808in}{2.448894in}}{\pgfqpoint{8.011374in}{2.445327in}}%
\pgfpathcurveto{\pgfqpoint{8.014941in}{2.441761in}}{\pgfqpoint{8.019779in}{2.439757in}}{\pgfqpoint{8.024822in}{2.439757in}}%
\pgfpathclose%
\pgfusepath{fill}%
\end{pgfscope}%
\begin{pgfscope}%
\pgfpathrectangle{\pgfqpoint{6.572727in}{0.474100in}}{\pgfqpoint{4.227273in}{3.318700in}}%
\pgfusepath{clip}%
\pgfsetbuttcap%
\pgfsetroundjoin%
\definecolor{currentfill}{rgb}{0.127568,0.566949,0.550556}%
\pgfsetfillcolor{currentfill}%
\pgfsetfillopacity{0.700000}%
\pgfsetlinewidth{0.000000pt}%
\definecolor{currentstroke}{rgb}{0.000000,0.000000,0.000000}%
\pgfsetstrokecolor{currentstroke}%
\pgfsetstrokeopacity{0.700000}%
\pgfsetdash{}{0pt}%
\pgfpathmoveto{\pgfqpoint{8.562464in}{2.565082in}}%
\pgfpathcurveto{\pgfqpoint{8.567508in}{2.565082in}}{\pgfqpoint{8.572345in}{2.567086in}}{\pgfqpoint{8.575912in}{2.570652in}}%
\pgfpathcurveto{\pgfqpoint{8.579478in}{2.574219in}}{\pgfqpoint{8.581482in}{2.579056in}}{\pgfqpoint{8.581482in}{2.584100in}}%
\pgfpathcurveto{\pgfqpoint{8.581482in}{2.589144in}}{\pgfqpoint{8.579478in}{2.593981in}}{\pgfqpoint{8.575912in}{2.597548in}}%
\pgfpathcurveto{\pgfqpoint{8.572345in}{2.601114in}}{\pgfqpoint{8.567508in}{2.603118in}}{\pgfqpoint{8.562464in}{2.603118in}}%
\pgfpathcurveto{\pgfqpoint{8.557420in}{2.603118in}}{\pgfqpoint{8.552582in}{2.601114in}}{\pgfqpoint{8.549016in}{2.597548in}}%
\pgfpathcurveto{\pgfqpoint{8.545450in}{2.593981in}}{\pgfqpoint{8.543446in}{2.589144in}}{\pgfqpoint{8.543446in}{2.584100in}}%
\pgfpathcurveto{\pgfqpoint{8.543446in}{2.579056in}}{\pgfqpoint{8.545450in}{2.574219in}}{\pgfqpoint{8.549016in}{2.570652in}}%
\pgfpathcurveto{\pgfqpoint{8.552582in}{2.567086in}}{\pgfqpoint{8.557420in}{2.565082in}}{\pgfqpoint{8.562464in}{2.565082in}}%
\pgfpathclose%
\pgfusepath{fill}%
\end{pgfscope}%
\begin{pgfscope}%
\pgfpathrectangle{\pgfqpoint{6.572727in}{0.474100in}}{\pgfqpoint{4.227273in}{3.318700in}}%
\pgfusepath{clip}%
\pgfsetbuttcap%
\pgfsetroundjoin%
\definecolor{currentfill}{rgb}{0.993248,0.906157,0.143936}%
\pgfsetfillcolor{currentfill}%
\pgfsetfillopacity{0.700000}%
\pgfsetlinewidth{0.000000pt}%
\definecolor{currentstroke}{rgb}{0.000000,0.000000,0.000000}%
\pgfsetstrokecolor{currentstroke}%
\pgfsetstrokeopacity{0.700000}%
\pgfsetdash{}{0pt}%
\pgfpathmoveto{\pgfqpoint{9.774988in}{1.386361in}}%
\pgfpathcurveto{\pgfqpoint{9.780032in}{1.386361in}}{\pgfqpoint{9.784870in}{1.388365in}}{\pgfqpoint{9.788436in}{1.391931in}}%
\pgfpathcurveto{\pgfqpoint{9.792002in}{1.395497in}}{\pgfqpoint{9.794006in}{1.400335in}}{\pgfqpoint{9.794006in}{1.405379in}}%
\pgfpathcurveto{\pgfqpoint{9.794006in}{1.410423in}}{\pgfqpoint{9.792002in}{1.415260in}}{\pgfqpoint{9.788436in}{1.418827in}}%
\pgfpathcurveto{\pgfqpoint{9.784870in}{1.422393in}}{\pgfqpoint{9.780032in}{1.424397in}}{\pgfqpoint{9.774988in}{1.424397in}}%
\pgfpathcurveto{\pgfqpoint{9.769944in}{1.424397in}}{\pgfqpoint{9.765107in}{1.422393in}}{\pgfqpoint{9.761540in}{1.418827in}}%
\pgfpathcurveto{\pgfqpoint{9.757974in}{1.415260in}}{\pgfqpoint{9.755970in}{1.410423in}}{\pgfqpoint{9.755970in}{1.405379in}}%
\pgfpathcurveto{\pgfqpoint{9.755970in}{1.400335in}}{\pgfqpoint{9.757974in}{1.395497in}}{\pgfqpoint{9.761540in}{1.391931in}}%
\pgfpathcurveto{\pgfqpoint{9.765107in}{1.388365in}}{\pgfqpoint{9.769944in}{1.386361in}}{\pgfqpoint{9.774988in}{1.386361in}}%
\pgfpathclose%
\pgfusepath{fill}%
\end{pgfscope}%
\begin{pgfscope}%
\pgfpathrectangle{\pgfqpoint{6.572727in}{0.474100in}}{\pgfqpoint{4.227273in}{3.318700in}}%
\pgfusepath{clip}%
\pgfsetbuttcap%
\pgfsetroundjoin%
\definecolor{currentfill}{rgb}{0.993248,0.906157,0.143936}%
\pgfsetfillcolor{currentfill}%
\pgfsetfillopacity{0.700000}%
\pgfsetlinewidth{0.000000pt}%
\definecolor{currentstroke}{rgb}{0.000000,0.000000,0.000000}%
\pgfsetstrokecolor{currentstroke}%
\pgfsetstrokeopacity{0.700000}%
\pgfsetdash{}{0pt}%
\pgfpathmoveto{\pgfqpoint{9.605057in}{1.352537in}}%
\pgfpathcurveto{\pgfqpoint{9.610100in}{1.352537in}}{\pgfqpoint{9.614938in}{1.354541in}}{\pgfqpoint{9.618505in}{1.358107in}}%
\pgfpathcurveto{\pgfqpoint{9.622071in}{1.361673in}}{\pgfqpoint{9.624075in}{1.366511in}}{\pgfqpoint{9.624075in}{1.371555in}}%
\pgfpathcurveto{\pgfqpoint{9.624075in}{1.376598in}}{\pgfqpoint{9.622071in}{1.381436in}}{\pgfqpoint{9.618505in}{1.385003in}}%
\pgfpathcurveto{\pgfqpoint{9.614938in}{1.388569in}}{\pgfqpoint{9.610100in}{1.390573in}}{\pgfqpoint{9.605057in}{1.390573in}}%
\pgfpathcurveto{\pgfqpoint{9.600013in}{1.390573in}}{\pgfqpoint{9.595175in}{1.388569in}}{\pgfqpoint{9.591609in}{1.385003in}}%
\pgfpathcurveto{\pgfqpoint{9.588042in}{1.381436in}}{\pgfqpoint{9.586039in}{1.376598in}}{\pgfqpoint{9.586039in}{1.371555in}}%
\pgfpathcurveto{\pgfqpoint{9.586039in}{1.366511in}}{\pgfqpoint{9.588042in}{1.361673in}}{\pgfqpoint{9.591609in}{1.358107in}}%
\pgfpathcurveto{\pgfqpoint{9.595175in}{1.354541in}}{\pgfqpoint{9.600013in}{1.352537in}}{\pgfqpoint{9.605057in}{1.352537in}}%
\pgfpathclose%
\pgfusepath{fill}%
\end{pgfscope}%
\begin{pgfscope}%
\pgfpathrectangle{\pgfqpoint{6.572727in}{0.474100in}}{\pgfqpoint{4.227273in}{3.318700in}}%
\pgfusepath{clip}%
\pgfsetbuttcap%
\pgfsetroundjoin%
\definecolor{currentfill}{rgb}{0.127568,0.566949,0.550556}%
\pgfsetfillcolor{currentfill}%
\pgfsetfillopacity{0.700000}%
\pgfsetlinewidth{0.000000pt}%
\definecolor{currentstroke}{rgb}{0.000000,0.000000,0.000000}%
\pgfsetstrokecolor{currentstroke}%
\pgfsetstrokeopacity{0.700000}%
\pgfsetdash{}{0pt}%
\pgfpathmoveto{\pgfqpoint{7.976908in}{2.879867in}}%
\pgfpathcurveto{\pgfqpoint{7.981951in}{2.879867in}}{\pgfqpoint{7.986789in}{2.881871in}}{\pgfqpoint{7.990355in}{2.885437in}}%
\pgfpathcurveto{\pgfqpoint{7.993922in}{2.889004in}}{\pgfqpoint{7.995926in}{2.893842in}}{\pgfqpoint{7.995926in}{2.898885in}}%
\pgfpathcurveto{\pgfqpoint{7.995926in}{2.903929in}}{\pgfqpoint{7.993922in}{2.908767in}}{\pgfqpoint{7.990355in}{2.912333in}}%
\pgfpathcurveto{\pgfqpoint{7.986789in}{2.915900in}}{\pgfqpoint{7.981951in}{2.917903in}}{\pgfqpoint{7.976908in}{2.917903in}}%
\pgfpathcurveto{\pgfqpoint{7.971864in}{2.917903in}}{\pgfqpoint{7.967026in}{2.915900in}}{\pgfqpoint{7.963460in}{2.912333in}}%
\pgfpathcurveto{\pgfqpoint{7.959893in}{2.908767in}}{\pgfqpoint{7.957889in}{2.903929in}}{\pgfqpoint{7.957889in}{2.898885in}}%
\pgfpathcurveto{\pgfqpoint{7.957889in}{2.893842in}}{\pgfqpoint{7.959893in}{2.889004in}}{\pgfqpoint{7.963460in}{2.885437in}}%
\pgfpathcurveto{\pgfqpoint{7.967026in}{2.881871in}}{\pgfqpoint{7.971864in}{2.879867in}}{\pgfqpoint{7.976908in}{2.879867in}}%
\pgfpathclose%
\pgfusepath{fill}%
\end{pgfscope}%
\begin{pgfscope}%
\pgfpathrectangle{\pgfqpoint{6.572727in}{0.474100in}}{\pgfqpoint{4.227273in}{3.318700in}}%
\pgfusepath{clip}%
\pgfsetbuttcap%
\pgfsetroundjoin%
\definecolor{currentfill}{rgb}{0.127568,0.566949,0.550556}%
\pgfsetfillcolor{currentfill}%
\pgfsetfillopacity{0.700000}%
\pgfsetlinewidth{0.000000pt}%
\definecolor{currentstroke}{rgb}{0.000000,0.000000,0.000000}%
\pgfsetstrokecolor{currentstroke}%
\pgfsetstrokeopacity{0.700000}%
\pgfsetdash{}{0pt}%
\pgfpathmoveto{\pgfqpoint{8.196999in}{2.845461in}}%
\pgfpathcurveto{\pgfqpoint{8.202043in}{2.845461in}}{\pgfqpoint{8.206881in}{2.847464in}}{\pgfqpoint{8.210447in}{2.851031in}}%
\pgfpathcurveto{\pgfqpoint{8.214013in}{2.854597in}}{\pgfqpoint{8.216017in}{2.859435in}}{\pgfqpoint{8.216017in}{2.864479in}}%
\pgfpathcurveto{\pgfqpoint{8.216017in}{2.869522in}}{\pgfqpoint{8.214013in}{2.874360in}}{\pgfqpoint{8.210447in}{2.877927in}}%
\pgfpathcurveto{\pgfqpoint{8.206881in}{2.881493in}}{\pgfqpoint{8.202043in}{2.883497in}}{\pgfqpoint{8.196999in}{2.883497in}}%
\pgfpathcurveto{\pgfqpoint{8.191956in}{2.883497in}}{\pgfqpoint{8.187118in}{2.881493in}}{\pgfqpoint{8.183551in}{2.877927in}}%
\pgfpathcurveto{\pgfqpoint{8.179985in}{2.874360in}}{\pgfqpoint{8.177981in}{2.869522in}}{\pgfqpoint{8.177981in}{2.864479in}}%
\pgfpathcurveto{\pgfqpoint{8.177981in}{2.859435in}}{\pgfqpoint{8.179985in}{2.854597in}}{\pgfqpoint{8.183551in}{2.851031in}}%
\pgfpathcurveto{\pgfqpoint{8.187118in}{2.847464in}}{\pgfqpoint{8.191956in}{2.845461in}}{\pgfqpoint{8.196999in}{2.845461in}}%
\pgfpathclose%
\pgfusepath{fill}%
\end{pgfscope}%
\begin{pgfscope}%
\pgfpathrectangle{\pgfqpoint{6.572727in}{0.474100in}}{\pgfqpoint{4.227273in}{3.318700in}}%
\pgfusepath{clip}%
\pgfsetbuttcap%
\pgfsetroundjoin%
\definecolor{currentfill}{rgb}{0.127568,0.566949,0.550556}%
\pgfsetfillcolor{currentfill}%
\pgfsetfillopacity{0.700000}%
\pgfsetlinewidth{0.000000pt}%
\definecolor{currentstroke}{rgb}{0.000000,0.000000,0.000000}%
\pgfsetstrokecolor{currentstroke}%
\pgfsetstrokeopacity{0.700000}%
\pgfsetdash{}{0pt}%
\pgfpathmoveto{\pgfqpoint{7.797632in}{1.270024in}}%
\pgfpathcurveto{\pgfqpoint{7.802675in}{1.270024in}}{\pgfqpoint{7.807513in}{1.272028in}}{\pgfqpoint{7.811079in}{1.275595in}}%
\pgfpathcurveto{\pgfqpoint{7.814646in}{1.279161in}}{\pgfqpoint{7.816650in}{1.283999in}}{\pgfqpoint{7.816650in}{1.289043in}}%
\pgfpathcurveto{\pgfqpoint{7.816650in}{1.294086in}}{\pgfqpoint{7.814646in}{1.298924in}}{\pgfqpoint{7.811079in}{1.302490in}}%
\pgfpathcurveto{\pgfqpoint{7.807513in}{1.306057in}}{\pgfqpoint{7.802675in}{1.308061in}}{\pgfqpoint{7.797632in}{1.308061in}}%
\pgfpathcurveto{\pgfqpoint{7.792588in}{1.308061in}}{\pgfqpoint{7.787750in}{1.306057in}}{\pgfqpoint{7.784184in}{1.302490in}}%
\pgfpathcurveto{\pgfqpoint{7.780617in}{1.298924in}}{\pgfqpoint{7.778613in}{1.294086in}}{\pgfqpoint{7.778613in}{1.289043in}}%
\pgfpathcurveto{\pgfqpoint{7.778613in}{1.283999in}}{\pgfqpoint{7.780617in}{1.279161in}}{\pgfqpoint{7.784184in}{1.275595in}}%
\pgfpathcurveto{\pgfqpoint{7.787750in}{1.272028in}}{\pgfqpoint{7.792588in}{1.270024in}}{\pgfqpoint{7.797632in}{1.270024in}}%
\pgfpathclose%
\pgfusepath{fill}%
\end{pgfscope}%
\begin{pgfscope}%
\pgfpathrectangle{\pgfqpoint{6.572727in}{0.474100in}}{\pgfqpoint{4.227273in}{3.318700in}}%
\pgfusepath{clip}%
\pgfsetbuttcap%
\pgfsetroundjoin%
\definecolor{currentfill}{rgb}{0.127568,0.566949,0.550556}%
\pgfsetfillcolor{currentfill}%
\pgfsetfillopacity{0.700000}%
\pgfsetlinewidth{0.000000pt}%
\definecolor{currentstroke}{rgb}{0.000000,0.000000,0.000000}%
\pgfsetstrokecolor{currentstroke}%
\pgfsetstrokeopacity{0.700000}%
\pgfsetdash{}{0pt}%
\pgfpathmoveto{\pgfqpoint{8.678931in}{2.602056in}}%
\pgfpathcurveto{\pgfqpoint{8.683974in}{2.602056in}}{\pgfqpoint{8.688812in}{2.604060in}}{\pgfqpoint{8.692379in}{2.607627in}}%
\pgfpathcurveto{\pgfqpoint{8.695945in}{2.611193in}}{\pgfqpoint{8.697949in}{2.616031in}}{\pgfqpoint{8.697949in}{2.621074in}}%
\pgfpathcurveto{\pgfqpoint{8.697949in}{2.626118in}}{\pgfqpoint{8.695945in}{2.630956in}}{\pgfqpoint{8.692379in}{2.634522in}}%
\pgfpathcurveto{\pgfqpoint{8.688812in}{2.638089in}}{\pgfqpoint{8.683974in}{2.640093in}}{\pgfqpoint{8.678931in}{2.640093in}}%
\pgfpathcurveto{\pgfqpoint{8.673887in}{2.640093in}}{\pgfqpoint{8.669049in}{2.638089in}}{\pgfqpoint{8.665483in}{2.634522in}}%
\pgfpathcurveto{\pgfqpoint{8.661916in}{2.630956in}}{\pgfqpoint{8.659913in}{2.626118in}}{\pgfqpoint{8.659913in}{2.621074in}}%
\pgfpathcurveto{\pgfqpoint{8.659913in}{2.616031in}}{\pgfqpoint{8.661916in}{2.611193in}}{\pgfqpoint{8.665483in}{2.607627in}}%
\pgfpathcurveto{\pgfqpoint{8.669049in}{2.604060in}}{\pgfqpoint{8.673887in}{2.602056in}}{\pgfqpoint{8.678931in}{2.602056in}}%
\pgfpathclose%
\pgfusepath{fill}%
\end{pgfscope}%
\begin{pgfscope}%
\pgfpathrectangle{\pgfqpoint{6.572727in}{0.474100in}}{\pgfqpoint{4.227273in}{3.318700in}}%
\pgfusepath{clip}%
\pgfsetbuttcap%
\pgfsetroundjoin%
\definecolor{currentfill}{rgb}{0.127568,0.566949,0.550556}%
\pgfsetfillcolor{currentfill}%
\pgfsetfillopacity{0.700000}%
\pgfsetlinewidth{0.000000pt}%
\definecolor{currentstroke}{rgb}{0.000000,0.000000,0.000000}%
\pgfsetstrokecolor{currentstroke}%
\pgfsetstrokeopacity{0.700000}%
\pgfsetdash{}{0pt}%
\pgfpathmoveto{\pgfqpoint{8.480039in}{2.271735in}}%
\pgfpathcurveto{\pgfqpoint{8.485082in}{2.271735in}}{\pgfqpoint{8.489920in}{2.273739in}}{\pgfqpoint{8.493487in}{2.277305in}}%
\pgfpathcurveto{\pgfqpoint{8.497053in}{2.280872in}}{\pgfqpoint{8.499057in}{2.285709in}}{\pgfqpoint{8.499057in}{2.290753in}}%
\pgfpathcurveto{\pgfqpoint{8.499057in}{2.295797in}}{\pgfqpoint{8.497053in}{2.300635in}}{\pgfqpoint{8.493487in}{2.304201in}}%
\pgfpathcurveto{\pgfqpoint{8.489920in}{2.307767in}}{\pgfqpoint{8.485082in}{2.309771in}}{\pgfqpoint{8.480039in}{2.309771in}}%
\pgfpathcurveto{\pgfqpoint{8.474995in}{2.309771in}}{\pgfqpoint{8.470157in}{2.307767in}}{\pgfqpoint{8.466591in}{2.304201in}}%
\pgfpathcurveto{\pgfqpoint{8.463024in}{2.300635in}}{\pgfqpoint{8.461021in}{2.295797in}}{\pgfqpoint{8.461021in}{2.290753in}}%
\pgfpathcurveto{\pgfqpoint{8.461021in}{2.285709in}}{\pgfqpoint{8.463024in}{2.280872in}}{\pgfqpoint{8.466591in}{2.277305in}}%
\pgfpathcurveto{\pgfqpoint{8.470157in}{2.273739in}}{\pgfqpoint{8.474995in}{2.271735in}}{\pgfqpoint{8.480039in}{2.271735in}}%
\pgfpathclose%
\pgfusepath{fill}%
\end{pgfscope}%
\begin{pgfscope}%
\pgfpathrectangle{\pgfqpoint{6.572727in}{0.474100in}}{\pgfqpoint{4.227273in}{3.318700in}}%
\pgfusepath{clip}%
\pgfsetbuttcap%
\pgfsetroundjoin%
\definecolor{currentfill}{rgb}{0.127568,0.566949,0.550556}%
\pgfsetfillcolor{currentfill}%
\pgfsetfillopacity{0.700000}%
\pgfsetlinewidth{0.000000pt}%
\definecolor{currentstroke}{rgb}{0.000000,0.000000,0.000000}%
\pgfsetstrokecolor{currentstroke}%
\pgfsetstrokeopacity{0.700000}%
\pgfsetdash{}{0pt}%
\pgfpathmoveto{\pgfqpoint{8.791698in}{3.449710in}}%
\pgfpathcurveto{\pgfqpoint{8.796741in}{3.449710in}}{\pgfqpoint{8.801579in}{3.451714in}}{\pgfqpoint{8.805145in}{3.455281in}}%
\pgfpathcurveto{\pgfqpoint{8.808712in}{3.458847in}}{\pgfqpoint{8.810716in}{3.463685in}}{\pgfqpoint{8.810716in}{3.468729in}}%
\pgfpathcurveto{\pgfqpoint{8.810716in}{3.473772in}}{\pgfqpoint{8.808712in}{3.478610in}}{\pgfqpoint{8.805145in}{3.482176in}}%
\pgfpathcurveto{\pgfqpoint{8.801579in}{3.485743in}}{\pgfqpoint{8.796741in}{3.487747in}}{\pgfqpoint{8.791698in}{3.487747in}}%
\pgfpathcurveto{\pgfqpoint{8.786654in}{3.487747in}}{\pgfqpoint{8.781816in}{3.485743in}}{\pgfqpoint{8.778250in}{3.482176in}}%
\pgfpathcurveto{\pgfqpoint{8.774683in}{3.478610in}}{\pgfqpoint{8.772679in}{3.473772in}}{\pgfqpoint{8.772679in}{3.468729in}}%
\pgfpathcurveto{\pgfqpoint{8.772679in}{3.463685in}}{\pgfqpoint{8.774683in}{3.458847in}}{\pgfqpoint{8.778250in}{3.455281in}}%
\pgfpathcurveto{\pgfqpoint{8.781816in}{3.451714in}}{\pgfqpoint{8.786654in}{3.449710in}}{\pgfqpoint{8.791698in}{3.449710in}}%
\pgfpathclose%
\pgfusepath{fill}%
\end{pgfscope}%
\begin{pgfscope}%
\pgfpathrectangle{\pgfqpoint{6.572727in}{0.474100in}}{\pgfqpoint{4.227273in}{3.318700in}}%
\pgfusepath{clip}%
\pgfsetbuttcap%
\pgfsetroundjoin%
\definecolor{currentfill}{rgb}{0.127568,0.566949,0.550556}%
\pgfsetfillcolor{currentfill}%
\pgfsetfillopacity{0.700000}%
\pgfsetlinewidth{0.000000pt}%
\definecolor{currentstroke}{rgb}{0.000000,0.000000,0.000000}%
\pgfsetstrokecolor{currentstroke}%
\pgfsetstrokeopacity{0.700000}%
\pgfsetdash{}{0pt}%
\pgfpathmoveto{\pgfqpoint{8.713837in}{2.149968in}}%
\pgfpathcurveto{\pgfqpoint{8.718880in}{2.149968in}}{\pgfqpoint{8.723718in}{2.151972in}}{\pgfqpoint{8.727284in}{2.155538in}}%
\pgfpathcurveto{\pgfqpoint{8.730851in}{2.159104in}}{\pgfqpoint{8.732855in}{2.163942in}}{\pgfqpoint{8.732855in}{2.168986in}}%
\pgfpathcurveto{\pgfqpoint{8.732855in}{2.174030in}}{\pgfqpoint{8.730851in}{2.178867in}}{\pgfqpoint{8.727284in}{2.182434in}}%
\pgfpathcurveto{\pgfqpoint{8.723718in}{2.186000in}}{\pgfqpoint{8.718880in}{2.188004in}}{\pgfqpoint{8.713837in}{2.188004in}}%
\pgfpathcurveto{\pgfqpoint{8.708793in}{2.188004in}}{\pgfqpoint{8.703955in}{2.186000in}}{\pgfqpoint{8.700389in}{2.182434in}}%
\pgfpathcurveto{\pgfqpoint{8.696822in}{2.178867in}}{\pgfqpoint{8.694818in}{2.174030in}}{\pgfqpoint{8.694818in}{2.168986in}}%
\pgfpathcurveto{\pgfqpoint{8.694818in}{2.163942in}}{\pgfqpoint{8.696822in}{2.159104in}}{\pgfqpoint{8.700389in}{2.155538in}}%
\pgfpathcurveto{\pgfqpoint{8.703955in}{2.151972in}}{\pgfqpoint{8.708793in}{2.149968in}}{\pgfqpoint{8.713837in}{2.149968in}}%
\pgfpathclose%
\pgfusepath{fill}%
\end{pgfscope}%
\begin{pgfscope}%
\pgfpathrectangle{\pgfqpoint{6.572727in}{0.474100in}}{\pgfqpoint{4.227273in}{3.318700in}}%
\pgfusepath{clip}%
\pgfsetbuttcap%
\pgfsetroundjoin%
\definecolor{currentfill}{rgb}{0.127568,0.566949,0.550556}%
\pgfsetfillcolor{currentfill}%
\pgfsetfillopacity{0.700000}%
\pgfsetlinewidth{0.000000pt}%
\definecolor{currentstroke}{rgb}{0.000000,0.000000,0.000000}%
\pgfsetstrokecolor{currentstroke}%
\pgfsetstrokeopacity{0.700000}%
\pgfsetdash{}{0pt}%
\pgfpathmoveto{\pgfqpoint{8.528419in}{3.090104in}}%
\pgfpathcurveto{\pgfqpoint{8.533463in}{3.090104in}}{\pgfqpoint{8.538301in}{3.092107in}}{\pgfqpoint{8.541867in}{3.095674in}}%
\pgfpathcurveto{\pgfqpoint{8.545434in}{3.099240in}}{\pgfqpoint{8.547438in}{3.104078in}}{\pgfqpoint{8.547438in}{3.109122in}}%
\pgfpathcurveto{\pgfqpoint{8.547438in}{3.114165in}}{\pgfqpoint{8.545434in}{3.119003in}}{\pgfqpoint{8.541867in}{3.122570in}}%
\pgfpathcurveto{\pgfqpoint{8.538301in}{3.126136in}}{\pgfqpoint{8.533463in}{3.128140in}}{\pgfqpoint{8.528419in}{3.128140in}}%
\pgfpathcurveto{\pgfqpoint{8.523376in}{3.128140in}}{\pgfqpoint{8.518538in}{3.126136in}}{\pgfqpoint{8.514972in}{3.122570in}}%
\pgfpathcurveto{\pgfqpoint{8.511405in}{3.119003in}}{\pgfqpoint{8.509401in}{3.114165in}}{\pgfqpoint{8.509401in}{3.109122in}}%
\pgfpathcurveto{\pgfqpoint{8.509401in}{3.104078in}}{\pgfqpoint{8.511405in}{3.099240in}}{\pgfqpoint{8.514972in}{3.095674in}}%
\pgfpathcurveto{\pgfqpoint{8.518538in}{3.092107in}}{\pgfqpoint{8.523376in}{3.090104in}}{\pgfqpoint{8.528419in}{3.090104in}}%
\pgfpathclose%
\pgfusepath{fill}%
\end{pgfscope}%
\begin{pgfscope}%
\pgfpathrectangle{\pgfqpoint{6.572727in}{0.474100in}}{\pgfqpoint{4.227273in}{3.318700in}}%
\pgfusepath{clip}%
\pgfsetbuttcap%
\pgfsetroundjoin%
\definecolor{currentfill}{rgb}{0.127568,0.566949,0.550556}%
\pgfsetfillcolor{currentfill}%
\pgfsetfillopacity{0.700000}%
\pgfsetlinewidth{0.000000pt}%
\definecolor{currentstroke}{rgb}{0.000000,0.000000,0.000000}%
\pgfsetstrokecolor{currentstroke}%
\pgfsetstrokeopacity{0.700000}%
\pgfsetdash{}{0pt}%
\pgfpathmoveto{\pgfqpoint{7.823149in}{2.667139in}}%
\pgfpathcurveto{\pgfqpoint{7.828193in}{2.667139in}}{\pgfqpoint{7.833031in}{2.669143in}}{\pgfqpoint{7.836597in}{2.672709in}}%
\pgfpathcurveto{\pgfqpoint{7.840164in}{2.676276in}}{\pgfqpoint{7.842167in}{2.681114in}}{\pgfqpoint{7.842167in}{2.686157in}}%
\pgfpathcurveto{\pgfqpoint{7.842167in}{2.691201in}}{\pgfqpoint{7.840164in}{2.696039in}}{\pgfqpoint{7.836597in}{2.699605in}}%
\pgfpathcurveto{\pgfqpoint{7.833031in}{2.703171in}}{\pgfqpoint{7.828193in}{2.705175in}}{\pgfqpoint{7.823149in}{2.705175in}}%
\pgfpathcurveto{\pgfqpoint{7.818106in}{2.705175in}}{\pgfqpoint{7.813268in}{2.703171in}}{\pgfqpoint{7.809701in}{2.699605in}}%
\pgfpathcurveto{\pgfqpoint{7.806135in}{2.696039in}}{\pgfqpoint{7.804131in}{2.691201in}}{\pgfqpoint{7.804131in}{2.686157in}}%
\pgfpathcurveto{\pgfqpoint{7.804131in}{2.681114in}}{\pgfqpoint{7.806135in}{2.676276in}}{\pgfqpoint{7.809701in}{2.672709in}}%
\pgfpathcurveto{\pgfqpoint{7.813268in}{2.669143in}}{\pgfqpoint{7.818106in}{2.667139in}}{\pgfqpoint{7.823149in}{2.667139in}}%
\pgfpathclose%
\pgfusepath{fill}%
\end{pgfscope}%
\begin{pgfscope}%
\pgfpathrectangle{\pgfqpoint{6.572727in}{0.474100in}}{\pgfqpoint{4.227273in}{3.318700in}}%
\pgfusepath{clip}%
\pgfsetbuttcap%
\pgfsetroundjoin%
\definecolor{currentfill}{rgb}{0.127568,0.566949,0.550556}%
\pgfsetfillcolor{currentfill}%
\pgfsetfillopacity{0.700000}%
\pgfsetlinewidth{0.000000pt}%
\definecolor{currentstroke}{rgb}{0.000000,0.000000,0.000000}%
\pgfsetstrokecolor{currentstroke}%
\pgfsetstrokeopacity{0.700000}%
\pgfsetdash{}{0pt}%
\pgfpathmoveto{\pgfqpoint{8.131231in}{1.692332in}}%
\pgfpathcurveto{\pgfqpoint{8.136275in}{1.692332in}}{\pgfqpoint{8.141113in}{1.694336in}}{\pgfqpoint{8.144679in}{1.697902in}}%
\pgfpathcurveto{\pgfqpoint{8.148245in}{1.701469in}}{\pgfqpoint{8.150249in}{1.706307in}}{\pgfqpoint{8.150249in}{1.711350in}}%
\pgfpathcurveto{\pgfqpoint{8.150249in}{1.716394in}}{\pgfqpoint{8.148245in}{1.721232in}}{\pgfqpoint{8.144679in}{1.724798in}}%
\pgfpathcurveto{\pgfqpoint{8.141113in}{1.728364in}}{\pgfqpoint{8.136275in}{1.730368in}}{\pgfqpoint{8.131231in}{1.730368in}}%
\pgfpathcurveto{\pgfqpoint{8.126187in}{1.730368in}}{\pgfqpoint{8.121350in}{1.728364in}}{\pgfqpoint{8.117783in}{1.724798in}}%
\pgfpathcurveto{\pgfqpoint{8.114217in}{1.721232in}}{\pgfqpoint{8.112213in}{1.716394in}}{\pgfqpoint{8.112213in}{1.711350in}}%
\pgfpathcurveto{\pgfqpoint{8.112213in}{1.706307in}}{\pgfqpoint{8.114217in}{1.701469in}}{\pgfqpoint{8.117783in}{1.697902in}}%
\pgfpathcurveto{\pgfqpoint{8.121350in}{1.694336in}}{\pgfqpoint{8.126187in}{1.692332in}}{\pgfqpoint{8.131231in}{1.692332in}}%
\pgfpathclose%
\pgfusepath{fill}%
\end{pgfscope}%
\begin{pgfscope}%
\pgfpathrectangle{\pgfqpoint{6.572727in}{0.474100in}}{\pgfqpoint{4.227273in}{3.318700in}}%
\pgfusepath{clip}%
\pgfsetbuttcap%
\pgfsetroundjoin%
\definecolor{currentfill}{rgb}{0.993248,0.906157,0.143936}%
\pgfsetfillcolor{currentfill}%
\pgfsetfillopacity{0.700000}%
\pgfsetlinewidth{0.000000pt}%
\definecolor{currentstroke}{rgb}{0.000000,0.000000,0.000000}%
\pgfsetstrokecolor{currentstroke}%
\pgfsetstrokeopacity{0.700000}%
\pgfsetdash{}{0pt}%
\pgfpathmoveto{\pgfqpoint{9.491682in}{1.739431in}}%
\pgfpathcurveto{\pgfqpoint{9.496726in}{1.739431in}}{\pgfqpoint{9.501564in}{1.741435in}}{\pgfqpoint{9.505130in}{1.745001in}}%
\pgfpathcurveto{\pgfqpoint{9.508697in}{1.748567in}}{\pgfqpoint{9.510701in}{1.753405in}}{\pgfqpoint{9.510701in}{1.758449in}}%
\pgfpathcurveto{\pgfqpoint{9.510701in}{1.763493in}}{\pgfqpoint{9.508697in}{1.768330in}}{\pgfqpoint{9.505130in}{1.771897in}}%
\pgfpathcurveto{\pgfqpoint{9.501564in}{1.775463in}}{\pgfqpoint{9.496726in}{1.777467in}}{\pgfqpoint{9.491682in}{1.777467in}}%
\pgfpathcurveto{\pgfqpoint{9.486639in}{1.777467in}}{\pgfqpoint{9.481801in}{1.775463in}}{\pgfqpoint{9.478235in}{1.771897in}}%
\pgfpathcurveto{\pgfqpoint{9.474668in}{1.768330in}}{\pgfqpoint{9.472664in}{1.763493in}}{\pgfqpoint{9.472664in}{1.758449in}}%
\pgfpathcurveto{\pgfqpoint{9.472664in}{1.753405in}}{\pgfqpoint{9.474668in}{1.748567in}}{\pgfqpoint{9.478235in}{1.745001in}}%
\pgfpathcurveto{\pgfqpoint{9.481801in}{1.741435in}}{\pgfqpoint{9.486639in}{1.739431in}}{\pgfqpoint{9.491682in}{1.739431in}}%
\pgfpathclose%
\pgfusepath{fill}%
\end{pgfscope}%
\begin{pgfscope}%
\pgfpathrectangle{\pgfqpoint{6.572727in}{0.474100in}}{\pgfqpoint{4.227273in}{3.318700in}}%
\pgfusepath{clip}%
\pgfsetbuttcap%
\pgfsetroundjoin%
\definecolor{currentfill}{rgb}{0.993248,0.906157,0.143936}%
\pgfsetfillcolor{currentfill}%
\pgfsetfillopacity{0.700000}%
\pgfsetlinewidth{0.000000pt}%
\definecolor{currentstroke}{rgb}{0.000000,0.000000,0.000000}%
\pgfsetstrokecolor{currentstroke}%
\pgfsetstrokeopacity{0.700000}%
\pgfsetdash{}{0pt}%
\pgfpathmoveto{\pgfqpoint{8.950341in}{1.323117in}}%
\pgfpathcurveto{\pgfqpoint{8.955385in}{1.323117in}}{\pgfqpoint{8.960223in}{1.325120in}}{\pgfqpoint{8.963789in}{1.328687in}}%
\pgfpathcurveto{\pgfqpoint{8.967356in}{1.332253in}}{\pgfqpoint{8.969360in}{1.337091in}}{\pgfqpoint{8.969360in}{1.342135in}}%
\pgfpathcurveto{\pgfqpoint{8.969360in}{1.347178in}}{\pgfqpoint{8.967356in}{1.352016in}}{\pgfqpoint{8.963789in}{1.355583in}}%
\pgfpathcurveto{\pgfqpoint{8.960223in}{1.359149in}}{\pgfqpoint{8.955385in}{1.361153in}}{\pgfqpoint{8.950341in}{1.361153in}}%
\pgfpathcurveto{\pgfqpoint{8.945298in}{1.361153in}}{\pgfqpoint{8.940460in}{1.359149in}}{\pgfqpoint{8.936894in}{1.355583in}}%
\pgfpathcurveto{\pgfqpoint{8.933327in}{1.352016in}}{\pgfqpoint{8.931323in}{1.347178in}}{\pgfqpoint{8.931323in}{1.342135in}}%
\pgfpathcurveto{\pgfqpoint{8.931323in}{1.337091in}}{\pgfqpoint{8.933327in}{1.332253in}}{\pgfqpoint{8.936894in}{1.328687in}}%
\pgfpathcurveto{\pgfqpoint{8.940460in}{1.325120in}}{\pgfqpoint{8.945298in}{1.323117in}}{\pgfqpoint{8.950341in}{1.323117in}}%
\pgfpathclose%
\pgfusepath{fill}%
\end{pgfscope}%
\begin{pgfscope}%
\pgfpathrectangle{\pgfqpoint{6.572727in}{0.474100in}}{\pgfqpoint{4.227273in}{3.318700in}}%
\pgfusepath{clip}%
\pgfsetbuttcap%
\pgfsetroundjoin%
\definecolor{currentfill}{rgb}{0.127568,0.566949,0.550556}%
\pgfsetfillcolor{currentfill}%
\pgfsetfillopacity{0.700000}%
\pgfsetlinewidth{0.000000pt}%
\definecolor{currentstroke}{rgb}{0.000000,0.000000,0.000000}%
\pgfsetstrokecolor{currentstroke}%
\pgfsetstrokeopacity{0.700000}%
\pgfsetdash{}{0pt}%
\pgfpathmoveto{\pgfqpoint{8.587236in}{2.927957in}}%
\pgfpathcurveto{\pgfqpoint{8.592280in}{2.927957in}}{\pgfqpoint{8.597117in}{2.929960in}}{\pgfqpoint{8.600684in}{2.933527in}}%
\pgfpathcurveto{\pgfqpoint{8.604250in}{2.937093in}}{\pgfqpoint{8.606254in}{2.941931in}}{\pgfqpoint{8.606254in}{2.946975in}}%
\pgfpathcurveto{\pgfqpoint{8.606254in}{2.952018in}}{\pgfqpoint{8.604250in}{2.956856in}}{\pgfqpoint{8.600684in}{2.960423in}}%
\pgfpathcurveto{\pgfqpoint{8.597117in}{2.963989in}}{\pgfqpoint{8.592280in}{2.965993in}}{\pgfqpoint{8.587236in}{2.965993in}}%
\pgfpathcurveto{\pgfqpoint{8.582192in}{2.965993in}}{\pgfqpoint{8.577354in}{2.963989in}}{\pgfqpoint{8.573788in}{2.960423in}}%
\pgfpathcurveto{\pgfqpoint{8.570222in}{2.956856in}}{\pgfqpoint{8.568218in}{2.952018in}}{\pgfqpoint{8.568218in}{2.946975in}}%
\pgfpathcurveto{\pgfqpoint{8.568218in}{2.941931in}}{\pgfqpoint{8.570222in}{2.937093in}}{\pgfqpoint{8.573788in}{2.933527in}}%
\pgfpathcurveto{\pgfqpoint{8.577354in}{2.929960in}}{\pgfqpoint{8.582192in}{2.927957in}}{\pgfqpoint{8.587236in}{2.927957in}}%
\pgfpathclose%
\pgfusepath{fill}%
\end{pgfscope}%
\begin{pgfscope}%
\pgfpathrectangle{\pgfqpoint{6.572727in}{0.474100in}}{\pgfqpoint{4.227273in}{3.318700in}}%
\pgfusepath{clip}%
\pgfsetbuttcap%
\pgfsetroundjoin%
\definecolor{currentfill}{rgb}{0.267004,0.004874,0.329415}%
\pgfsetfillcolor{currentfill}%
\pgfsetfillopacity{0.700000}%
\pgfsetlinewidth{0.000000pt}%
\definecolor{currentstroke}{rgb}{0.000000,0.000000,0.000000}%
\pgfsetstrokecolor{currentstroke}%
\pgfsetstrokeopacity{0.700000}%
\pgfsetdash{}{0pt}%
\pgfpathmoveto{\pgfqpoint{10.585804in}{1.305801in}}%
\pgfpathcurveto{\pgfqpoint{10.590848in}{1.305801in}}{\pgfqpoint{10.595686in}{1.307804in}}{\pgfqpoint{10.599252in}{1.311371in}}%
\pgfpathcurveto{\pgfqpoint{10.602818in}{1.314937in}}{\pgfqpoint{10.604822in}{1.319775in}}{\pgfqpoint{10.604822in}{1.324819in}}%
\pgfpathcurveto{\pgfqpoint{10.604822in}{1.329862in}}{\pgfqpoint{10.602818in}{1.334700in}}{\pgfqpoint{10.599252in}{1.338267in}}%
\pgfpathcurveto{\pgfqpoint{10.595686in}{1.341833in}}{\pgfqpoint{10.590848in}{1.343837in}}{\pgfqpoint{10.585804in}{1.343837in}}%
\pgfpathcurveto{\pgfqpoint{10.580760in}{1.343837in}}{\pgfqpoint{10.575923in}{1.341833in}}{\pgfqpoint{10.572356in}{1.338267in}}%
\pgfpathcurveto{\pgfqpoint{10.568790in}{1.334700in}}{\pgfqpoint{10.566786in}{1.329862in}}{\pgfqpoint{10.566786in}{1.324819in}}%
\pgfpathcurveto{\pgfqpoint{10.566786in}{1.319775in}}{\pgfqpoint{10.568790in}{1.314937in}}{\pgfqpoint{10.572356in}{1.311371in}}%
\pgfpathcurveto{\pgfqpoint{10.575923in}{1.307804in}}{\pgfqpoint{10.580760in}{1.305801in}}{\pgfqpoint{10.585804in}{1.305801in}}%
\pgfpathclose%
\pgfusepath{fill}%
\end{pgfscope}%
\begin{pgfscope}%
\pgfpathrectangle{\pgfqpoint{6.572727in}{0.474100in}}{\pgfqpoint{4.227273in}{3.318700in}}%
\pgfusepath{clip}%
\pgfsetbuttcap%
\pgfsetroundjoin%
\definecolor{currentfill}{rgb}{0.127568,0.566949,0.550556}%
\pgfsetfillcolor{currentfill}%
\pgfsetfillopacity{0.700000}%
\pgfsetlinewidth{0.000000pt}%
\definecolor{currentstroke}{rgb}{0.000000,0.000000,0.000000}%
\pgfsetstrokecolor{currentstroke}%
\pgfsetstrokeopacity{0.700000}%
\pgfsetdash{}{0pt}%
\pgfpathmoveto{\pgfqpoint{7.754195in}{1.367224in}}%
\pgfpathcurveto{\pgfqpoint{7.759238in}{1.367224in}}{\pgfqpoint{7.764076in}{1.369228in}}{\pgfqpoint{7.767643in}{1.372795in}}%
\pgfpathcurveto{\pgfqpoint{7.771209in}{1.376361in}}{\pgfqpoint{7.773213in}{1.381199in}}{\pgfqpoint{7.773213in}{1.386242in}}%
\pgfpathcurveto{\pgfqpoint{7.773213in}{1.391286in}}{\pgfqpoint{7.771209in}{1.396124in}}{\pgfqpoint{7.767643in}{1.399690in}}%
\pgfpathcurveto{\pgfqpoint{7.764076in}{1.403257in}}{\pgfqpoint{7.759238in}{1.405261in}}{\pgfqpoint{7.754195in}{1.405261in}}%
\pgfpathcurveto{\pgfqpoint{7.749151in}{1.405261in}}{\pgfqpoint{7.744313in}{1.403257in}}{\pgfqpoint{7.740747in}{1.399690in}}%
\pgfpathcurveto{\pgfqpoint{7.737180in}{1.396124in}}{\pgfqpoint{7.735177in}{1.391286in}}{\pgfqpoint{7.735177in}{1.386242in}}%
\pgfpathcurveto{\pgfqpoint{7.735177in}{1.381199in}}{\pgfqpoint{7.737180in}{1.376361in}}{\pgfqpoint{7.740747in}{1.372795in}}%
\pgfpathcurveto{\pgfqpoint{7.744313in}{1.369228in}}{\pgfqpoint{7.749151in}{1.367224in}}{\pgfqpoint{7.754195in}{1.367224in}}%
\pgfpathclose%
\pgfusepath{fill}%
\end{pgfscope}%
\begin{pgfscope}%
\pgfpathrectangle{\pgfqpoint{6.572727in}{0.474100in}}{\pgfqpoint{4.227273in}{3.318700in}}%
\pgfusepath{clip}%
\pgfsetbuttcap%
\pgfsetroundjoin%
\definecolor{currentfill}{rgb}{0.127568,0.566949,0.550556}%
\pgfsetfillcolor{currentfill}%
\pgfsetfillopacity{0.700000}%
\pgfsetlinewidth{0.000000pt}%
\definecolor{currentstroke}{rgb}{0.000000,0.000000,0.000000}%
\pgfsetstrokecolor{currentstroke}%
\pgfsetstrokeopacity{0.700000}%
\pgfsetdash{}{0pt}%
\pgfpathmoveto{\pgfqpoint{8.103378in}{2.845594in}}%
\pgfpathcurveto{\pgfqpoint{8.108421in}{2.845594in}}{\pgfqpoint{8.113259in}{2.847598in}}{\pgfqpoint{8.116826in}{2.851164in}}%
\pgfpathcurveto{\pgfqpoint{8.120392in}{2.854731in}}{\pgfqpoint{8.122396in}{2.859569in}}{\pgfqpoint{8.122396in}{2.864612in}}%
\pgfpathcurveto{\pgfqpoint{8.122396in}{2.869656in}}{\pgfqpoint{8.120392in}{2.874494in}}{\pgfqpoint{8.116826in}{2.878060in}}%
\pgfpathcurveto{\pgfqpoint{8.113259in}{2.881627in}}{\pgfqpoint{8.108421in}{2.883630in}}{\pgfqpoint{8.103378in}{2.883630in}}%
\pgfpathcurveto{\pgfqpoint{8.098334in}{2.883630in}}{\pgfqpoint{8.093496in}{2.881627in}}{\pgfqpoint{8.089930in}{2.878060in}}%
\pgfpathcurveto{\pgfqpoint{8.086364in}{2.874494in}}{\pgfqpoint{8.084360in}{2.869656in}}{\pgfqpoint{8.084360in}{2.864612in}}%
\pgfpathcurveto{\pgfqpoint{8.084360in}{2.859569in}}{\pgfqpoint{8.086364in}{2.854731in}}{\pgfqpoint{8.089930in}{2.851164in}}%
\pgfpathcurveto{\pgfqpoint{8.093496in}{2.847598in}}{\pgfqpoint{8.098334in}{2.845594in}}{\pgfqpoint{8.103378in}{2.845594in}}%
\pgfpathclose%
\pgfusepath{fill}%
\end{pgfscope}%
\begin{pgfscope}%
\pgfpathrectangle{\pgfqpoint{6.572727in}{0.474100in}}{\pgfqpoint{4.227273in}{3.318700in}}%
\pgfusepath{clip}%
\pgfsetbuttcap%
\pgfsetroundjoin%
\definecolor{currentfill}{rgb}{0.993248,0.906157,0.143936}%
\pgfsetfillcolor{currentfill}%
\pgfsetfillopacity{0.700000}%
\pgfsetlinewidth{0.000000pt}%
\definecolor{currentstroke}{rgb}{0.000000,0.000000,0.000000}%
\pgfsetstrokecolor{currentstroke}%
\pgfsetstrokeopacity{0.700000}%
\pgfsetdash{}{0pt}%
\pgfpathmoveto{\pgfqpoint{10.005022in}{1.611664in}}%
\pgfpathcurveto{\pgfqpoint{10.010066in}{1.611664in}}{\pgfqpoint{10.014904in}{1.613668in}}{\pgfqpoint{10.018470in}{1.617234in}}%
\pgfpathcurveto{\pgfqpoint{10.022037in}{1.620801in}}{\pgfqpoint{10.024041in}{1.625639in}}{\pgfqpoint{10.024041in}{1.630682in}}%
\pgfpathcurveto{\pgfqpoint{10.024041in}{1.635726in}}{\pgfqpoint{10.022037in}{1.640564in}}{\pgfqpoint{10.018470in}{1.644130in}}%
\pgfpathcurveto{\pgfqpoint{10.014904in}{1.647697in}}{\pgfqpoint{10.010066in}{1.649700in}}{\pgfqpoint{10.005022in}{1.649700in}}%
\pgfpathcurveto{\pgfqpoint{9.999979in}{1.649700in}}{\pgfqpoint{9.995141in}{1.647697in}}{\pgfqpoint{9.991575in}{1.644130in}}%
\pgfpathcurveto{\pgfqpoint{9.988008in}{1.640564in}}{\pgfqpoint{9.986004in}{1.635726in}}{\pgfqpoint{9.986004in}{1.630682in}}%
\pgfpathcurveto{\pgfqpoint{9.986004in}{1.625639in}}{\pgfqpoint{9.988008in}{1.620801in}}{\pgfqpoint{9.991575in}{1.617234in}}%
\pgfpathcurveto{\pgfqpoint{9.995141in}{1.613668in}}{\pgfqpoint{9.999979in}{1.611664in}}{\pgfqpoint{10.005022in}{1.611664in}}%
\pgfpathclose%
\pgfusepath{fill}%
\end{pgfscope}%
\begin{pgfscope}%
\pgfpathrectangle{\pgfqpoint{6.572727in}{0.474100in}}{\pgfqpoint{4.227273in}{3.318700in}}%
\pgfusepath{clip}%
\pgfsetbuttcap%
\pgfsetroundjoin%
\definecolor{currentfill}{rgb}{0.127568,0.566949,0.550556}%
\pgfsetfillcolor{currentfill}%
\pgfsetfillopacity{0.700000}%
\pgfsetlinewidth{0.000000pt}%
\definecolor{currentstroke}{rgb}{0.000000,0.000000,0.000000}%
\pgfsetstrokecolor{currentstroke}%
\pgfsetstrokeopacity{0.700000}%
\pgfsetdash{}{0pt}%
\pgfpathmoveto{\pgfqpoint{9.120544in}{3.130671in}}%
\pgfpathcurveto{\pgfqpoint{9.125587in}{3.130671in}}{\pgfqpoint{9.130425in}{3.132675in}}{\pgfqpoint{9.133992in}{3.136241in}}%
\pgfpathcurveto{\pgfqpoint{9.137558in}{3.139807in}}{\pgfqpoint{9.139562in}{3.144645in}}{\pgfqpoint{9.139562in}{3.149689in}}%
\pgfpathcurveto{\pgfqpoint{9.139562in}{3.154732in}}{\pgfqpoint{9.137558in}{3.159570in}}{\pgfqpoint{9.133992in}{3.163137in}}%
\pgfpathcurveto{\pgfqpoint{9.130425in}{3.166703in}}{\pgfqpoint{9.125587in}{3.168707in}}{\pgfqpoint{9.120544in}{3.168707in}}%
\pgfpathcurveto{\pgfqpoint{9.115500in}{3.168707in}}{\pgfqpoint{9.110662in}{3.166703in}}{\pgfqpoint{9.107096in}{3.163137in}}%
\pgfpathcurveto{\pgfqpoint{9.103529in}{3.159570in}}{\pgfqpoint{9.101526in}{3.154732in}}{\pgfqpoint{9.101526in}{3.149689in}}%
\pgfpathcurveto{\pgfqpoint{9.101526in}{3.144645in}}{\pgfqpoint{9.103529in}{3.139807in}}{\pgfqpoint{9.107096in}{3.136241in}}%
\pgfpathcurveto{\pgfqpoint{9.110662in}{3.132675in}}{\pgfqpoint{9.115500in}{3.130671in}}{\pgfqpoint{9.120544in}{3.130671in}}%
\pgfpathclose%
\pgfusepath{fill}%
\end{pgfscope}%
\begin{pgfscope}%
\pgfpathrectangle{\pgfqpoint{6.572727in}{0.474100in}}{\pgfqpoint{4.227273in}{3.318700in}}%
\pgfusepath{clip}%
\pgfsetbuttcap%
\pgfsetroundjoin%
\definecolor{currentfill}{rgb}{0.127568,0.566949,0.550556}%
\pgfsetfillcolor{currentfill}%
\pgfsetfillopacity{0.700000}%
\pgfsetlinewidth{0.000000pt}%
\definecolor{currentstroke}{rgb}{0.000000,0.000000,0.000000}%
\pgfsetstrokecolor{currentstroke}%
\pgfsetstrokeopacity{0.700000}%
\pgfsetdash{}{0pt}%
\pgfpathmoveto{\pgfqpoint{7.971187in}{1.676922in}}%
\pgfpathcurveto{\pgfqpoint{7.976230in}{1.676922in}}{\pgfqpoint{7.981068in}{1.678926in}}{\pgfqpoint{7.984634in}{1.682492in}}%
\pgfpathcurveto{\pgfqpoint{7.988201in}{1.686059in}}{\pgfqpoint{7.990205in}{1.690896in}}{\pgfqpoint{7.990205in}{1.695940in}}%
\pgfpathcurveto{\pgfqpoint{7.990205in}{1.700984in}}{\pgfqpoint{7.988201in}{1.705821in}}{\pgfqpoint{7.984634in}{1.709388in}}%
\pgfpathcurveto{\pgfqpoint{7.981068in}{1.712954in}}{\pgfqpoint{7.976230in}{1.714958in}}{\pgfqpoint{7.971187in}{1.714958in}}%
\pgfpathcurveto{\pgfqpoint{7.966143in}{1.714958in}}{\pgfqpoint{7.961305in}{1.712954in}}{\pgfqpoint{7.957739in}{1.709388in}}%
\pgfpathcurveto{\pgfqpoint{7.954172in}{1.705821in}}{\pgfqpoint{7.952168in}{1.700984in}}{\pgfqpoint{7.952168in}{1.695940in}}%
\pgfpathcurveto{\pgfqpoint{7.952168in}{1.690896in}}{\pgfqpoint{7.954172in}{1.686059in}}{\pgfqpoint{7.957739in}{1.682492in}}%
\pgfpathcurveto{\pgfqpoint{7.961305in}{1.678926in}}{\pgfqpoint{7.966143in}{1.676922in}}{\pgfqpoint{7.971187in}{1.676922in}}%
\pgfpathclose%
\pgfusepath{fill}%
\end{pgfscope}%
\begin{pgfscope}%
\pgfpathrectangle{\pgfqpoint{6.572727in}{0.474100in}}{\pgfqpoint{4.227273in}{3.318700in}}%
\pgfusepath{clip}%
\pgfsetbuttcap%
\pgfsetroundjoin%
\definecolor{currentfill}{rgb}{0.993248,0.906157,0.143936}%
\pgfsetfillcolor{currentfill}%
\pgfsetfillopacity{0.700000}%
\pgfsetlinewidth{0.000000pt}%
\definecolor{currentstroke}{rgb}{0.000000,0.000000,0.000000}%
\pgfsetstrokecolor{currentstroke}%
\pgfsetstrokeopacity{0.700000}%
\pgfsetdash{}{0pt}%
\pgfpathmoveto{\pgfqpoint{9.790444in}{1.436545in}}%
\pgfpathcurveto{\pgfqpoint{9.795488in}{1.436545in}}{\pgfqpoint{9.800326in}{1.438549in}}{\pgfqpoint{9.803892in}{1.442115in}}%
\pgfpathcurveto{\pgfqpoint{9.807459in}{1.445682in}}{\pgfqpoint{9.809463in}{1.450520in}}{\pgfqpoint{9.809463in}{1.455563in}}%
\pgfpathcurveto{\pgfqpoint{9.809463in}{1.460607in}}{\pgfqpoint{9.807459in}{1.465445in}}{\pgfqpoint{9.803892in}{1.469011in}}%
\pgfpathcurveto{\pgfqpoint{9.800326in}{1.472578in}}{\pgfqpoint{9.795488in}{1.474581in}}{\pgfqpoint{9.790444in}{1.474581in}}%
\pgfpathcurveto{\pgfqpoint{9.785401in}{1.474581in}}{\pgfqpoint{9.780563in}{1.472578in}}{\pgfqpoint{9.776997in}{1.469011in}}%
\pgfpathcurveto{\pgfqpoint{9.773430in}{1.465445in}}{\pgfqpoint{9.771426in}{1.460607in}}{\pgfqpoint{9.771426in}{1.455563in}}%
\pgfpathcurveto{\pgfqpoint{9.771426in}{1.450520in}}{\pgfqpoint{9.773430in}{1.445682in}}{\pgfqpoint{9.776997in}{1.442115in}}%
\pgfpathcurveto{\pgfqpoint{9.780563in}{1.438549in}}{\pgfqpoint{9.785401in}{1.436545in}}{\pgfqpoint{9.790444in}{1.436545in}}%
\pgfpathclose%
\pgfusepath{fill}%
\end{pgfscope}%
\begin{pgfscope}%
\pgfpathrectangle{\pgfqpoint{6.572727in}{0.474100in}}{\pgfqpoint{4.227273in}{3.318700in}}%
\pgfusepath{clip}%
\pgfsetbuttcap%
\pgfsetroundjoin%
\definecolor{currentfill}{rgb}{0.993248,0.906157,0.143936}%
\pgfsetfillcolor{currentfill}%
\pgfsetfillopacity{0.700000}%
\pgfsetlinewidth{0.000000pt}%
\definecolor{currentstroke}{rgb}{0.000000,0.000000,0.000000}%
\pgfsetstrokecolor{currentstroke}%
\pgfsetstrokeopacity{0.700000}%
\pgfsetdash{}{0pt}%
\pgfpathmoveto{\pgfqpoint{9.639419in}{2.179955in}}%
\pgfpathcurveto{\pgfqpoint{9.644463in}{2.179955in}}{\pgfqpoint{9.649300in}{2.181959in}}{\pgfqpoint{9.652867in}{2.185525in}}%
\pgfpathcurveto{\pgfqpoint{9.656433in}{2.189092in}}{\pgfqpoint{9.658437in}{2.193929in}}{\pgfqpoint{9.658437in}{2.198973in}}%
\pgfpathcurveto{\pgfqpoint{9.658437in}{2.204017in}}{\pgfqpoint{9.656433in}{2.208855in}}{\pgfqpoint{9.652867in}{2.212421in}}%
\pgfpathcurveto{\pgfqpoint{9.649300in}{2.215987in}}{\pgfqpoint{9.644463in}{2.217991in}}{\pgfqpoint{9.639419in}{2.217991in}}%
\pgfpathcurveto{\pgfqpoint{9.634375in}{2.217991in}}{\pgfqpoint{9.629538in}{2.215987in}}{\pgfqpoint{9.625971in}{2.212421in}}%
\pgfpathcurveto{\pgfqpoint{9.622405in}{2.208855in}}{\pgfqpoint{9.620401in}{2.204017in}}{\pgfqpoint{9.620401in}{2.198973in}}%
\pgfpathcurveto{\pgfqpoint{9.620401in}{2.193929in}}{\pgfqpoint{9.622405in}{2.189092in}}{\pgfqpoint{9.625971in}{2.185525in}}%
\pgfpathcurveto{\pgfqpoint{9.629538in}{2.181959in}}{\pgfqpoint{9.634375in}{2.179955in}}{\pgfqpoint{9.639419in}{2.179955in}}%
\pgfpathclose%
\pgfusepath{fill}%
\end{pgfscope}%
\begin{pgfscope}%
\pgfpathrectangle{\pgfqpoint{6.572727in}{0.474100in}}{\pgfqpoint{4.227273in}{3.318700in}}%
\pgfusepath{clip}%
\pgfsetbuttcap%
\pgfsetroundjoin%
\definecolor{currentfill}{rgb}{0.127568,0.566949,0.550556}%
\pgfsetfillcolor{currentfill}%
\pgfsetfillopacity{0.700000}%
\pgfsetlinewidth{0.000000pt}%
\definecolor{currentstroke}{rgb}{0.000000,0.000000,0.000000}%
\pgfsetstrokecolor{currentstroke}%
\pgfsetstrokeopacity{0.700000}%
\pgfsetdash{}{0pt}%
\pgfpathmoveto{\pgfqpoint{7.990502in}{3.329154in}}%
\pgfpathcurveto{\pgfqpoint{7.995546in}{3.329154in}}{\pgfqpoint{8.000383in}{3.331158in}}{\pgfqpoint{8.003950in}{3.334724in}}%
\pgfpathcurveto{\pgfqpoint{8.007516in}{3.338291in}}{\pgfqpoint{8.009520in}{3.343128in}}{\pgfqpoint{8.009520in}{3.348172in}}%
\pgfpathcurveto{\pgfqpoint{8.009520in}{3.353216in}}{\pgfqpoint{8.007516in}{3.358053in}}{\pgfqpoint{8.003950in}{3.361620in}}%
\pgfpathcurveto{\pgfqpoint{8.000383in}{3.365186in}}{\pgfqpoint{7.995546in}{3.367190in}}{\pgfqpoint{7.990502in}{3.367190in}}%
\pgfpathcurveto{\pgfqpoint{7.985458in}{3.367190in}}{\pgfqpoint{7.980621in}{3.365186in}}{\pgfqpoint{7.977054in}{3.361620in}}%
\pgfpathcurveto{\pgfqpoint{7.973488in}{3.358053in}}{\pgfqpoint{7.971484in}{3.353216in}}{\pgfqpoint{7.971484in}{3.348172in}}%
\pgfpathcurveto{\pgfqpoint{7.971484in}{3.343128in}}{\pgfqpoint{7.973488in}{3.338291in}}{\pgfqpoint{7.977054in}{3.334724in}}%
\pgfpathcurveto{\pgfqpoint{7.980621in}{3.331158in}}{\pgfqpoint{7.985458in}{3.329154in}}{\pgfqpoint{7.990502in}{3.329154in}}%
\pgfpathclose%
\pgfusepath{fill}%
\end{pgfscope}%
\begin{pgfscope}%
\pgfpathrectangle{\pgfqpoint{6.572727in}{0.474100in}}{\pgfqpoint{4.227273in}{3.318700in}}%
\pgfusepath{clip}%
\pgfsetbuttcap%
\pgfsetroundjoin%
\definecolor{currentfill}{rgb}{0.993248,0.906157,0.143936}%
\pgfsetfillcolor{currentfill}%
\pgfsetfillopacity{0.700000}%
\pgfsetlinewidth{0.000000pt}%
\definecolor{currentstroke}{rgb}{0.000000,0.000000,0.000000}%
\pgfsetstrokecolor{currentstroke}%
\pgfsetstrokeopacity{0.700000}%
\pgfsetdash{}{0pt}%
\pgfpathmoveto{\pgfqpoint{9.177129in}{1.651611in}}%
\pgfpathcurveto{\pgfqpoint{9.182173in}{1.651611in}}{\pgfqpoint{9.187011in}{1.653615in}}{\pgfqpoint{9.190577in}{1.657181in}}%
\pgfpathcurveto{\pgfqpoint{9.194144in}{1.660748in}}{\pgfqpoint{9.196148in}{1.665585in}}{\pgfqpoint{9.196148in}{1.670629in}}%
\pgfpathcurveto{\pgfqpoint{9.196148in}{1.675673in}}{\pgfqpoint{9.194144in}{1.680511in}}{\pgfqpoint{9.190577in}{1.684077in}}%
\pgfpathcurveto{\pgfqpoint{9.187011in}{1.687643in}}{\pgfqpoint{9.182173in}{1.689647in}}{\pgfqpoint{9.177129in}{1.689647in}}%
\pgfpathcurveto{\pgfqpoint{9.172086in}{1.689647in}}{\pgfqpoint{9.167248in}{1.687643in}}{\pgfqpoint{9.163682in}{1.684077in}}%
\pgfpathcurveto{\pgfqpoint{9.160115in}{1.680511in}}{\pgfqpoint{9.158111in}{1.675673in}}{\pgfqpoint{9.158111in}{1.670629in}}%
\pgfpathcurveto{\pgfqpoint{9.158111in}{1.665585in}}{\pgfqpoint{9.160115in}{1.660748in}}{\pgfqpoint{9.163682in}{1.657181in}}%
\pgfpathcurveto{\pgfqpoint{9.167248in}{1.653615in}}{\pgfqpoint{9.172086in}{1.651611in}}{\pgfqpoint{9.177129in}{1.651611in}}%
\pgfpathclose%
\pgfusepath{fill}%
\end{pgfscope}%
\begin{pgfscope}%
\pgfpathrectangle{\pgfqpoint{6.572727in}{0.474100in}}{\pgfqpoint{4.227273in}{3.318700in}}%
\pgfusepath{clip}%
\pgfsetbuttcap%
\pgfsetroundjoin%
\definecolor{currentfill}{rgb}{0.993248,0.906157,0.143936}%
\pgfsetfillcolor{currentfill}%
\pgfsetfillopacity{0.700000}%
\pgfsetlinewidth{0.000000pt}%
\definecolor{currentstroke}{rgb}{0.000000,0.000000,0.000000}%
\pgfsetstrokecolor{currentstroke}%
\pgfsetstrokeopacity{0.700000}%
\pgfsetdash{}{0pt}%
\pgfpathmoveto{\pgfqpoint{9.986951in}{1.493844in}}%
\pgfpathcurveto{\pgfqpoint{9.991995in}{1.493844in}}{\pgfqpoint{9.996833in}{1.495848in}}{\pgfqpoint{10.000399in}{1.499414in}}%
\pgfpathcurveto{\pgfqpoint{10.003966in}{1.502980in}}{\pgfqpoint{10.005969in}{1.507818in}}{\pgfqpoint{10.005969in}{1.512862in}}%
\pgfpathcurveto{\pgfqpoint{10.005969in}{1.517905in}}{\pgfqpoint{10.003966in}{1.522743in}}{\pgfqpoint{10.000399in}{1.526310in}}%
\pgfpathcurveto{\pgfqpoint{9.996833in}{1.529876in}}{\pgfqpoint{9.991995in}{1.531880in}}{\pgfqpoint{9.986951in}{1.531880in}}%
\pgfpathcurveto{\pgfqpoint{9.981908in}{1.531880in}}{\pgfqpoint{9.977070in}{1.529876in}}{\pgfqpoint{9.973503in}{1.526310in}}%
\pgfpathcurveto{\pgfqpoint{9.969937in}{1.522743in}}{\pgfqpoint{9.967933in}{1.517905in}}{\pgfqpoint{9.967933in}{1.512862in}}%
\pgfpathcurveto{\pgfqpoint{9.967933in}{1.507818in}}{\pgfqpoint{9.969937in}{1.502980in}}{\pgfqpoint{9.973503in}{1.499414in}}%
\pgfpathcurveto{\pgfqpoint{9.977070in}{1.495848in}}{\pgfqpoint{9.981908in}{1.493844in}}{\pgfqpoint{9.986951in}{1.493844in}}%
\pgfpathclose%
\pgfusepath{fill}%
\end{pgfscope}%
\begin{pgfscope}%
\pgfpathrectangle{\pgfqpoint{6.572727in}{0.474100in}}{\pgfqpoint{4.227273in}{3.318700in}}%
\pgfusepath{clip}%
\pgfsetbuttcap%
\pgfsetroundjoin%
\definecolor{currentfill}{rgb}{0.993248,0.906157,0.143936}%
\pgfsetfillcolor{currentfill}%
\pgfsetfillopacity{0.700000}%
\pgfsetlinewidth{0.000000pt}%
\definecolor{currentstroke}{rgb}{0.000000,0.000000,0.000000}%
\pgfsetstrokecolor{currentstroke}%
\pgfsetstrokeopacity{0.700000}%
\pgfsetdash{}{0pt}%
\pgfpathmoveto{\pgfqpoint{9.387374in}{1.176081in}}%
\pgfpathcurveto{\pgfqpoint{9.392418in}{1.176081in}}{\pgfqpoint{9.397256in}{1.178084in}}{\pgfqpoint{9.400822in}{1.181651in}}%
\pgfpathcurveto{\pgfqpoint{9.404389in}{1.185217in}}{\pgfqpoint{9.406393in}{1.190055in}}{\pgfqpoint{9.406393in}{1.195099in}}%
\pgfpathcurveto{\pgfqpoint{9.406393in}{1.200142in}}{\pgfqpoint{9.404389in}{1.204980in}}{\pgfqpoint{9.400822in}{1.208547in}}%
\pgfpathcurveto{\pgfqpoint{9.397256in}{1.212113in}}{\pgfqpoint{9.392418in}{1.214117in}}{\pgfqpoint{9.387374in}{1.214117in}}%
\pgfpathcurveto{\pgfqpoint{9.382331in}{1.214117in}}{\pgfqpoint{9.377493in}{1.212113in}}{\pgfqpoint{9.373927in}{1.208547in}}%
\pgfpathcurveto{\pgfqpoint{9.370360in}{1.204980in}}{\pgfqpoint{9.368356in}{1.200142in}}{\pgfqpoint{9.368356in}{1.195099in}}%
\pgfpathcurveto{\pgfqpoint{9.368356in}{1.190055in}}{\pgfqpoint{9.370360in}{1.185217in}}{\pgfqpoint{9.373927in}{1.181651in}}%
\pgfpathcurveto{\pgfqpoint{9.377493in}{1.178084in}}{\pgfqpoint{9.382331in}{1.176081in}}{\pgfqpoint{9.387374in}{1.176081in}}%
\pgfpathclose%
\pgfusepath{fill}%
\end{pgfscope}%
\begin{pgfscope}%
\pgfpathrectangle{\pgfqpoint{6.572727in}{0.474100in}}{\pgfqpoint{4.227273in}{3.318700in}}%
\pgfusepath{clip}%
\pgfsetbuttcap%
\pgfsetroundjoin%
\definecolor{currentfill}{rgb}{0.127568,0.566949,0.550556}%
\pgfsetfillcolor{currentfill}%
\pgfsetfillopacity{0.700000}%
\pgfsetlinewidth{0.000000pt}%
\definecolor{currentstroke}{rgb}{0.000000,0.000000,0.000000}%
\pgfsetstrokecolor{currentstroke}%
\pgfsetstrokeopacity{0.700000}%
\pgfsetdash{}{0pt}%
\pgfpathmoveto{\pgfqpoint{8.347673in}{2.781042in}}%
\pgfpathcurveto{\pgfqpoint{8.352717in}{2.781042in}}{\pgfqpoint{8.357555in}{2.783046in}}{\pgfqpoint{8.361121in}{2.786612in}}%
\pgfpathcurveto{\pgfqpoint{8.364688in}{2.790178in}}{\pgfqpoint{8.366692in}{2.795016in}}{\pgfqpoint{8.366692in}{2.800060in}}%
\pgfpathcurveto{\pgfqpoint{8.366692in}{2.805104in}}{\pgfqpoint{8.364688in}{2.809941in}}{\pgfqpoint{8.361121in}{2.813508in}}%
\pgfpathcurveto{\pgfqpoint{8.357555in}{2.817074in}}{\pgfqpoint{8.352717in}{2.819078in}}{\pgfqpoint{8.347673in}{2.819078in}}%
\pgfpathcurveto{\pgfqpoint{8.342630in}{2.819078in}}{\pgfqpoint{8.337792in}{2.817074in}}{\pgfqpoint{8.334226in}{2.813508in}}%
\pgfpathcurveto{\pgfqpoint{8.330659in}{2.809941in}}{\pgfqpoint{8.328655in}{2.805104in}}{\pgfqpoint{8.328655in}{2.800060in}}%
\pgfpathcurveto{\pgfqpoint{8.328655in}{2.795016in}}{\pgfqpoint{8.330659in}{2.790178in}}{\pgfqpoint{8.334226in}{2.786612in}}%
\pgfpathcurveto{\pgfqpoint{8.337792in}{2.783046in}}{\pgfqpoint{8.342630in}{2.781042in}}{\pgfqpoint{8.347673in}{2.781042in}}%
\pgfpathclose%
\pgfusepath{fill}%
\end{pgfscope}%
\begin{pgfscope}%
\pgfpathrectangle{\pgfqpoint{6.572727in}{0.474100in}}{\pgfqpoint{4.227273in}{3.318700in}}%
\pgfusepath{clip}%
\pgfsetbuttcap%
\pgfsetroundjoin%
\definecolor{currentfill}{rgb}{0.993248,0.906157,0.143936}%
\pgfsetfillcolor{currentfill}%
\pgfsetfillopacity{0.700000}%
\pgfsetlinewidth{0.000000pt}%
\definecolor{currentstroke}{rgb}{0.000000,0.000000,0.000000}%
\pgfsetstrokecolor{currentstroke}%
\pgfsetstrokeopacity{0.700000}%
\pgfsetdash{}{0pt}%
\pgfpathmoveto{\pgfqpoint{9.489269in}{1.817696in}}%
\pgfpathcurveto{\pgfqpoint{9.494313in}{1.817696in}}{\pgfqpoint{9.499151in}{1.819699in}}{\pgfqpoint{9.502717in}{1.823266in}}%
\pgfpathcurveto{\pgfqpoint{9.506284in}{1.826832in}}{\pgfqpoint{9.508288in}{1.831670in}}{\pgfqpoint{9.508288in}{1.836714in}}%
\pgfpathcurveto{\pgfqpoint{9.508288in}{1.841757in}}{\pgfqpoint{9.506284in}{1.846595in}}{\pgfqpoint{9.502717in}{1.850162in}}%
\pgfpathcurveto{\pgfqpoint{9.499151in}{1.853728in}}{\pgfqpoint{9.494313in}{1.855732in}}{\pgfqpoint{9.489269in}{1.855732in}}%
\pgfpathcurveto{\pgfqpoint{9.484226in}{1.855732in}}{\pgfqpoint{9.479388in}{1.853728in}}{\pgfqpoint{9.475822in}{1.850162in}}%
\pgfpathcurveto{\pgfqpoint{9.472255in}{1.846595in}}{\pgfqpoint{9.470251in}{1.841757in}}{\pgfqpoint{9.470251in}{1.836714in}}%
\pgfpathcurveto{\pgfqpoint{9.470251in}{1.831670in}}{\pgfqpoint{9.472255in}{1.826832in}}{\pgfqpoint{9.475822in}{1.823266in}}%
\pgfpathcurveto{\pgfqpoint{9.479388in}{1.819699in}}{\pgfqpoint{9.484226in}{1.817696in}}{\pgfqpoint{9.489269in}{1.817696in}}%
\pgfpathclose%
\pgfusepath{fill}%
\end{pgfscope}%
\begin{pgfscope}%
\pgfpathrectangle{\pgfqpoint{6.572727in}{0.474100in}}{\pgfqpoint{4.227273in}{3.318700in}}%
\pgfusepath{clip}%
\pgfsetbuttcap%
\pgfsetroundjoin%
\definecolor{currentfill}{rgb}{0.127568,0.566949,0.550556}%
\pgfsetfillcolor{currentfill}%
\pgfsetfillopacity{0.700000}%
\pgfsetlinewidth{0.000000pt}%
\definecolor{currentstroke}{rgb}{0.000000,0.000000,0.000000}%
\pgfsetstrokecolor{currentstroke}%
\pgfsetstrokeopacity{0.700000}%
\pgfsetdash{}{0pt}%
\pgfpathmoveto{\pgfqpoint{7.928335in}{3.454464in}}%
\pgfpathcurveto{\pgfqpoint{7.933379in}{3.454464in}}{\pgfqpoint{7.938216in}{3.456468in}}{\pgfqpoint{7.941783in}{3.460034in}}%
\pgfpathcurveto{\pgfqpoint{7.945349in}{3.463601in}}{\pgfqpoint{7.947353in}{3.468438in}}{\pgfqpoint{7.947353in}{3.473482in}}%
\pgfpathcurveto{\pgfqpoint{7.947353in}{3.478526in}}{\pgfqpoint{7.945349in}{3.483364in}}{\pgfqpoint{7.941783in}{3.486930in}}%
\pgfpathcurveto{\pgfqpoint{7.938216in}{3.490496in}}{\pgfqpoint{7.933379in}{3.492500in}}{\pgfqpoint{7.928335in}{3.492500in}}%
\pgfpathcurveto{\pgfqpoint{7.923291in}{3.492500in}}{\pgfqpoint{7.918454in}{3.490496in}}{\pgfqpoint{7.914887in}{3.486930in}}%
\pgfpathcurveto{\pgfqpoint{7.911321in}{3.483364in}}{\pgfqpoint{7.909317in}{3.478526in}}{\pgfqpoint{7.909317in}{3.473482in}}%
\pgfpathcurveto{\pgfqpoint{7.909317in}{3.468438in}}{\pgfqpoint{7.911321in}{3.463601in}}{\pgfqpoint{7.914887in}{3.460034in}}%
\pgfpathcurveto{\pgfqpoint{7.918454in}{3.456468in}}{\pgfqpoint{7.923291in}{3.454464in}}{\pgfqpoint{7.928335in}{3.454464in}}%
\pgfpathclose%
\pgfusepath{fill}%
\end{pgfscope}%
\begin{pgfscope}%
\pgfpathrectangle{\pgfqpoint{6.572727in}{0.474100in}}{\pgfqpoint{4.227273in}{3.318700in}}%
\pgfusepath{clip}%
\pgfsetbuttcap%
\pgfsetroundjoin%
\definecolor{currentfill}{rgb}{0.127568,0.566949,0.550556}%
\pgfsetfillcolor{currentfill}%
\pgfsetfillopacity{0.700000}%
\pgfsetlinewidth{0.000000pt}%
\definecolor{currentstroke}{rgb}{0.000000,0.000000,0.000000}%
\pgfsetstrokecolor{currentstroke}%
\pgfsetstrokeopacity{0.700000}%
\pgfsetdash{}{0pt}%
\pgfpathmoveto{\pgfqpoint{8.035604in}{2.904463in}}%
\pgfpathcurveto{\pgfqpoint{8.040648in}{2.904463in}}{\pgfqpoint{8.045486in}{2.906467in}}{\pgfqpoint{8.049052in}{2.910033in}}%
\pgfpathcurveto{\pgfqpoint{8.052619in}{2.913600in}}{\pgfqpoint{8.054623in}{2.918437in}}{\pgfqpoint{8.054623in}{2.923481in}}%
\pgfpathcurveto{\pgfqpoint{8.054623in}{2.928525in}}{\pgfqpoint{8.052619in}{2.933362in}}{\pgfqpoint{8.049052in}{2.936929in}}%
\pgfpathcurveto{\pgfqpoint{8.045486in}{2.940495in}}{\pgfqpoint{8.040648in}{2.942499in}}{\pgfqpoint{8.035604in}{2.942499in}}%
\pgfpathcurveto{\pgfqpoint{8.030561in}{2.942499in}}{\pgfqpoint{8.025723in}{2.940495in}}{\pgfqpoint{8.022157in}{2.936929in}}%
\pgfpathcurveto{\pgfqpoint{8.018590in}{2.933362in}}{\pgfqpoint{8.016586in}{2.928525in}}{\pgfqpoint{8.016586in}{2.923481in}}%
\pgfpathcurveto{\pgfqpoint{8.016586in}{2.918437in}}{\pgfqpoint{8.018590in}{2.913600in}}{\pgfqpoint{8.022157in}{2.910033in}}%
\pgfpathcurveto{\pgfqpoint{8.025723in}{2.906467in}}{\pgfqpoint{8.030561in}{2.904463in}}{\pgfqpoint{8.035604in}{2.904463in}}%
\pgfpathclose%
\pgfusepath{fill}%
\end{pgfscope}%
\begin{pgfscope}%
\pgfpathrectangle{\pgfqpoint{6.572727in}{0.474100in}}{\pgfqpoint{4.227273in}{3.318700in}}%
\pgfusepath{clip}%
\pgfsetbuttcap%
\pgfsetroundjoin%
\definecolor{currentfill}{rgb}{0.127568,0.566949,0.550556}%
\pgfsetfillcolor{currentfill}%
\pgfsetfillopacity{0.700000}%
\pgfsetlinewidth{0.000000pt}%
\definecolor{currentstroke}{rgb}{0.000000,0.000000,0.000000}%
\pgfsetstrokecolor{currentstroke}%
\pgfsetstrokeopacity{0.700000}%
\pgfsetdash{}{0pt}%
\pgfpathmoveto{\pgfqpoint{7.868367in}{2.022281in}}%
\pgfpathcurveto{\pgfqpoint{7.873410in}{2.022281in}}{\pgfqpoint{7.878248in}{2.024285in}}{\pgfqpoint{7.881814in}{2.027851in}}%
\pgfpathcurveto{\pgfqpoint{7.885381in}{2.031417in}}{\pgfqpoint{7.887385in}{2.036255in}}{\pgfqpoint{7.887385in}{2.041299in}}%
\pgfpathcurveto{\pgfqpoint{7.887385in}{2.046343in}}{\pgfqpoint{7.885381in}{2.051180in}}{\pgfqpoint{7.881814in}{2.054747in}}%
\pgfpathcurveto{\pgfqpoint{7.878248in}{2.058313in}}{\pgfqpoint{7.873410in}{2.060317in}}{\pgfqpoint{7.868367in}{2.060317in}}%
\pgfpathcurveto{\pgfqpoint{7.863323in}{2.060317in}}{\pgfqpoint{7.858485in}{2.058313in}}{\pgfqpoint{7.854919in}{2.054747in}}%
\pgfpathcurveto{\pgfqpoint{7.851352in}{2.051180in}}{\pgfqpoint{7.849348in}{2.046343in}}{\pgfqpoint{7.849348in}{2.041299in}}%
\pgfpathcurveto{\pgfqpoint{7.849348in}{2.036255in}}{\pgfqpoint{7.851352in}{2.031417in}}{\pgfqpoint{7.854919in}{2.027851in}}%
\pgfpathcurveto{\pgfqpoint{7.858485in}{2.024285in}}{\pgfqpoint{7.863323in}{2.022281in}}{\pgfqpoint{7.868367in}{2.022281in}}%
\pgfpathclose%
\pgfusepath{fill}%
\end{pgfscope}%
\begin{pgfscope}%
\pgfpathrectangle{\pgfqpoint{6.572727in}{0.474100in}}{\pgfqpoint{4.227273in}{3.318700in}}%
\pgfusepath{clip}%
\pgfsetbuttcap%
\pgfsetroundjoin%
\definecolor{currentfill}{rgb}{0.127568,0.566949,0.550556}%
\pgfsetfillcolor{currentfill}%
\pgfsetfillopacity{0.700000}%
\pgfsetlinewidth{0.000000pt}%
\definecolor{currentstroke}{rgb}{0.000000,0.000000,0.000000}%
\pgfsetstrokecolor{currentstroke}%
\pgfsetstrokeopacity{0.700000}%
\pgfsetdash{}{0pt}%
\pgfpathmoveto{\pgfqpoint{7.754510in}{2.362355in}}%
\pgfpathcurveto{\pgfqpoint{7.759553in}{2.362355in}}{\pgfqpoint{7.764391in}{2.364359in}}{\pgfqpoint{7.767958in}{2.367925in}}%
\pgfpathcurveto{\pgfqpoint{7.771524in}{2.371492in}}{\pgfqpoint{7.773528in}{2.376330in}}{\pgfqpoint{7.773528in}{2.381373in}}%
\pgfpathcurveto{\pgfqpoint{7.773528in}{2.386417in}}{\pgfqpoint{7.771524in}{2.391255in}}{\pgfqpoint{7.767958in}{2.394821in}}%
\pgfpathcurveto{\pgfqpoint{7.764391in}{2.398388in}}{\pgfqpoint{7.759553in}{2.400391in}}{\pgfqpoint{7.754510in}{2.400391in}}%
\pgfpathcurveto{\pgfqpoint{7.749466in}{2.400391in}}{\pgfqpoint{7.744628in}{2.398388in}}{\pgfqpoint{7.741062in}{2.394821in}}%
\pgfpathcurveto{\pgfqpoint{7.737495in}{2.391255in}}{\pgfqpoint{7.735492in}{2.386417in}}{\pgfqpoint{7.735492in}{2.381373in}}%
\pgfpathcurveto{\pgfqpoint{7.735492in}{2.376330in}}{\pgfqpoint{7.737495in}{2.371492in}}{\pgfqpoint{7.741062in}{2.367925in}}%
\pgfpathcurveto{\pgfqpoint{7.744628in}{2.364359in}}{\pgfqpoint{7.749466in}{2.362355in}}{\pgfqpoint{7.754510in}{2.362355in}}%
\pgfpathclose%
\pgfusepath{fill}%
\end{pgfscope}%
\begin{pgfscope}%
\pgfpathrectangle{\pgfqpoint{6.572727in}{0.474100in}}{\pgfqpoint{4.227273in}{3.318700in}}%
\pgfusepath{clip}%
\pgfsetbuttcap%
\pgfsetroundjoin%
\definecolor{currentfill}{rgb}{0.127568,0.566949,0.550556}%
\pgfsetfillcolor{currentfill}%
\pgfsetfillopacity{0.700000}%
\pgfsetlinewidth{0.000000pt}%
\definecolor{currentstroke}{rgb}{0.000000,0.000000,0.000000}%
\pgfsetstrokecolor{currentstroke}%
\pgfsetstrokeopacity{0.700000}%
\pgfsetdash{}{0pt}%
\pgfpathmoveto{\pgfqpoint{8.084904in}{3.434075in}}%
\pgfpathcurveto{\pgfqpoint{8.089947in}{3.434075in}}{\pgfqpoint{8.094785in}{3.436079in}}{\pgfqpoint{8.098352in}{3.439646in}}%
\pgfpathcurveto{\pgfqpoint{8.101918in}{3.443212in}}{\pgfqpoint{8.103922in}{3.448050in}}{\pgfqpoint{8.103922in}{3.453093in}}%
\pgfpathcurveto{\pgfqpoint{8.103922in}{3.458137in}}{\pgfqpoint{8.101918in}{3.462975in}}{\pgfqpoint{8.098352in}{3.466541in}}%
\pgfpathcurveto{\pgfqpoint{8.094785in}{3.470108in}}{\pgfqpoint{8.089947in}{3.472112in}}{\pgfqpoint{8.084904in}{3.472112in}}%
\pgfpathcurveto{\pgfqpoint{8.079860in}{3.472112in}}{\pgfqpoint{8.075022in}{3.470108in}}{\pgfqpoint{8.071456in}{3.466541in}}%
\pgfpathcurveto{\pgfqpoint{8.067889in}{3.462975in}}{\pgfqpoint{8.065886in}{3.458137in}}{\pgfqpoint{8.065886in}{3.453093in}}%
\pgfpathcurveto{\pgfqpoint{8.065886in}{3.448050in}}{\pgfqpoint{8.067889in}{3.443212in}}{\pgfqpoint{8.071456in}{3.439646in}}%
\pgfpathcurveto{\pgfqpoint{8.075022in}{3.436079in}}{\pgfqpoint{8.079860in}{3.434075in}}{\pgfqpoint{8.084904in}{3.434075in}}%
\pgfpathclose%
\pgfusepath{fill}%
\end{pgfscope}%
\begin{pgfscope}%
\pgfpathrectangle{\pgfqpoint{6.572727in}{0.474100in}}{\pgfqpoint{4.227273in}{3.318700in}}%
\pgfusepath{clip}%
\pgfsetbuttcap%
\pgfsetroundjoin%
\definecolor{currentfill}{rgb}{0.127568,0.566949,0.550556}%
\pgfsetfillcolor{currentfill}%
\pgfsetfillopacity{0.700000}%
\pgfsetlinewidth{0.000000pt}%
\definecolor{currentstroke}{rgb}{0.000000,0.000000,0.000000}%
\pgfsetstrokecolor{currentstroke}%
\pgfsetstrokeopacity{0.700000}%
\pgfsetdash{}{0pt}%
\pgfpathmoveto{\pgfqpoint{7.821623in}{1.684355in}}%
\pgfpathcurveto{\pgfqpoint{7.826667in}{1.684355in}}{\pgfqpoint{7.831504in}{1.686359in}}{\pgfqpoint{7.835071in}{1.689925in}}%
\pgfpathcurveto{\pgfqpoint{7.838637in}{1.693492in}}{\pgfqpoint{7.840641in}{1.698330in}}{\pgfqpoint{7.840641in}{1.703373in}}%
\pgfpathcurveto{\pgfqpoint{7.840641in}{1.708417in}}{\pgfqpoint{7.838637in}{1.713255in}}{\pgfqpoint{7.835071in}{1.716821in}}%
\pgfpathcurveto{\pgfqpoint{7.831504in}{1.720388in}}{\pgfqpoint{7.826667in}{1.722392in}}{\pgfqpoint{7.821623in}{1.722392in}}%
\pgfpathcurveto{\pgfqpoint{7.816579in}{1.722392in}}{\pgfqpoint{7.811742in}{1.720388in}}{\pgfqpoint{7.808175in}{1.716821in}}%
\pgfpathcurveto{\pgfqpoint{7.804609in}{1.713255in}}{\pgfqpoint{7.802605in}{1.708417in}}{\pgfqpoint{7.802605in}{1.703373in}}%
\pgfpathcurveto{\pgfqpoint{7.802605in}{1.698330in}}{\pgfqpoint{7.804609in}{1.693492in}}{\pgfqpoint{7.808175in}{1.689925in}}%
\pgfpathcurveto{\pgfqpoint{7.811742in}{1.686359in}}{\pgfqpoint{7.816579in}{1.684355in}}{\pgfqpoint{7.821623in}{1.684355in}}%
\pgfpathclose%
\pgfusepath{fill}%
\end{pgfscope}%
\begin{pgfscope}%
\pgfpathrectangle{\pgfqpoint{6.572727in}{0.474100in}}{\pgfqpoint{4.227273in}{3.318700in}}%
\pgfusepath{clip}%
\pgfsetbuttcap%
\pgfsetroundjoin%
\definecolor{currentfill}{rgb}{0.127568,0.566949,0.550556}%
\pgfsetfillcolor{currentfill}%
\pgfsetfillopacity{0.700000}%
\pgfsetlinewidth{0.000000pt}%
\definecolor{currentstroke}{rgb}{0.000000,0.000000,0.000000}%
\pgfsetstrokecolor{currentstroke}%
\pgfsetstrokeopacity{0.700000}%
\pgfsetdash{}{0pt}%
\pgfpathmoveto{\pgfqpoint{7.861638in}{1.659647in}}%
\pgfpathcurveto{\pgfqpoint{7.866682in}{1.659647in}}{\pgfqpoint{7.871519in}{1.661650in}}{\pgfqpoint{7.875086in}{1.665217in}}%
\pgfpathcurveto{\pgfqpoint{7.878652in}{1.668783in}}{\pgfqpoint{7.880656in}{1.673621in}}{\pgfqpoint{7.880656in}{1.678665in}}%
\pgfpathcurveto{\pgfqpoint{7.880656in}{1.683708in}}{\pgfqpoint{7.878652in}{1.688546in}}{\pgfqpoint{7.875086in}{1.692113in}}%
\pgfpathcurveto{\pgfqpoint{7.871519in}{1.695679in}}{\pgfqpoint{7.866682in}{1.697683in}}{\pgfqpoint{7.861638in}{1.697683in}}%
\pgfpathcurveto{\pgfqpoint{7.856594in}{1.697683in}}{\pgfqpoint{7.851756in}{1.695679in}}{\pgfqpoint{7.848190in}{1.692113in}}%
\pgfpathcurveto{\pgfqpoint{7.844624in}{1.688546in}}{\pgfqpoint{7.842620in}{1.683708in}}{\pgfqpoint{7.842620in}{1.678665in}}%
\pgfpathcurveto{\pgfqpoint{7.842620in}{1.673621in}}{\pgfqpoint{7.844624in}{1.668783in}}{\pgfqpoint{7.848190in}{1.665217in}}%
\pgfpathcurveto{\pgfqpoint{7.851756in}{1.661650in}}{\pgfqpoint{7.856594in}{1.659647in}}{\pgfqpoint{7.861638in}{1.659647in}}%
\pgfpathclose%
\pgfusepath{fill}%
\end{pgfscope}%
\begin{pgfscope}%
\pgfpathrectangle{\pgfqpoint{6.572727in}{0.474100in}}{\pgfqpoint{4.227273in}{3.318700in}}%
\pgfusepath{clip}%
\pgfsetbuttcap%
\pgfsetroundjoin%
\definecolor{currentfill}{rgb}{0.993248,0.906157,0.143936}%
\pgfsetfillcolor{currentfill}%
\pgfsetfillopacity{0.700000}%
\pgfsetlinewidth{0.000000pt}%
\definecolor{currentstroke}{rgb}{0.000000,0.000000,0.000000}%
\pgfsetstrokecolor{currentstroke}%
\pgfsetstrokeopacity{0.700000}%
\pgfsetdash{}{0pt}%
\pgfpathmoveto{\pgfqpoint{9.457914in}{0.971569in}}%
\pgfpathcurveto{\pgfqpoint{9.462957in}{0.971569in}}{\pgfqpoint{9.467795in}{0.973572in}}{\pgfqpoint{9.471361in}{0.977139in}}%
\pgfpathcurveto{\pgfqpoint{9.474928in}{0.980705in}}{\pgfqpoint{9.476932in}{0.985543in}}{\pgfqpoint{9.476932in}{0.990587in}}%
\pgfpathcurveto{\pgfqpoint{9.476932in}{0.995630in}}{\pgfqpoint{9.474928in}{1.000468in}}{\pgfqpoint{9.471361in}{1.004035in}}%
\pgfpathcurveto{\pgfqpoint{9.467795in}{1.007601in}}{\pgfqpoint{9.462957in}{1.009605in}}{\pgfqpoint{9.457914in}{1.009605in}}%
\pgfpathcurveto{\pgfqpoint{9.452870in}{1.009605in}}{\pgfqpoint{9.448032in}{1.007601in}}{\pgfqpoint{9.444466in}{1.004035in}}%
\pgfpathcurveto{\pgfqpoint{9.440899in}{1.000468in}}{\pgfqpoint{9.438895in}{0.995630in}}{\pgfqpoint{9.438895in}{0.990587in}}%
\pgfpathcurveto{\pgfqpoint{9.438895in}{0.985543in}}{\pgfqpoint{9.440899in}{0.980705in}}{\pgfqpoint{9.444466in}{0.977139in}}%
\pgfpathcurveto{\pgfqpoint{9.448032in}{0.973572in}}{\pgfqpoint{9.452870in}{0.971569in}}{\pgfqpoint{9.457914in}{0.971569in}}%
\pgfpathclose%
\pgfusepath{fill}%
\end{pgfscope}%
\begin{pgfscope}%
\pgfpathrectangle{\pgfqpoint{6.572727in}{0.474100in}}{\pgfqpoint{4.227273in}{3.318700in}}%
\pgfusepath{clip}%
\pgfsetbuttcap%
\pgfsetroundjoin%
\definecolor{currentfill}{rgb}{0.127568,0.566949,0.550556}%
\pgfsetfillcolor{currentfill}%
\pgfsetfillopacity{0.700000}%
\pgfsetlinewidth{0.000000pt}%
\definecolor{currentstroke}{rgb}{0.000000,0.000000,0.000000}%
\pgfsetstrokecolor{currentstroke}%
\pgfsetstrokeopacity{0.700000}%
\pgfsetdash{}{0pt}%
\pgfpathmoveto{\pgfqpoint{8.546718in}{2.815730in}}%
\pgfpathcurveto{\pgfqpoint{8.551761in}{2.815730in}}{\pgfqpoint{8.556599in}{2.817734in}}{\pgfqpoint{8.560166in}{2.821300in}}%
\pgfpathcurveto{\pgfqpoint{8.563732in}{2.824866in}}{\pgfqpoint{8.565736in}{2.829704in}}{\pgfqpoint{8.565736in}{2.834748in}}%
\pgfpathcurveto{\pgfqpoint{8.565736in}{2.839791in}}{\pgfqpoint{8.563732in}{2.844629in}}{\pgfqpoint{8.560166in}{2.848196in}}%
\pgfpathcurveto{\pgfqpoint{8.556599in}{2.851762in}}{\pgfqpoint{8.551761in}{2.853766in}}{\pgfqpoint{8.546718in}{2.853766in}}%
\pgfpathcurveto{\pgfqpoint{8.541674in}{2.853766in}}{\pgfqpoint{8.536836in}{2.851762in}}{\pgfqpoint{8.533270in}{2.848196in}}%
\pgfpathcurveto{\pgfqpoint{8.529703in}{2.844629in}}{\pgfqpoint{8.527700in}{2.839791in}}{\pgfqpoint{8.527700in}{2.834748in}}%
\pgfpathcurveto{\pgfqpoint{8.527700in}{2.829704in}}{\pgfqpoint{8.529703in}{2.824866in}}{\pgfqpoint{8.533270in}{2.821300in}}%
\pgfpathcurveto{\pgfqpoint{8.536836in}{2.817734in}}{\pgfqpoint{8.541674in}{2.815730in}}{\pgfqpoint{8.546718in}{2.815730in}}%
\pgfpathclose%
\pgfusepath{fill}%
\end{pgfscope}%
\begin{pgfscope}%
\pgfpathrectangle{\pgfqpoint{6.572727in}{0.474100in}}{\pgfqpoint{4.227273in}{3.318700in}}%
\pgfusepath{clip}%
\pgfsetbuttcap%
\pgfsetroundjoin%
\definecolor{currentfill}{rgb}{0.127568,0.566949,0.550556}%
\pgfsetfillcolor{currentfill}%
\pgfsetfillopacity{0.700000}%
\pgfsetlinewidth{0.000000pt}%
\definecolor{currentstroke}{rgb}{0.000000,0.000000,0.000000}%
\pgfsetstrokecolor{currentstroke}%
\pgfsetstrokeopacity{0.700000}%
\pgfsetdash{}{0pt}%
\pgfpathmoveto{\pgfqpoint{8.045324in}{2.953470in}}%
\pgfpathcurveto{\pgfqpoint{8.050368in}{2.953470in}}{\pgfqpoint{8.055206in}{2.955474in}}{\pgfqpoint{8.058772in}{2.959041in}}%
\pgfpathcurveto{\pgfqpoint{8.062339in}{2.962607in}}{\pgfqpoint{8.064342in}{2.967445in}}{\pgfqpoint{8.064342in}{2.972489in}}%
\pgfpathcurveto{\pgfqpoint{8.064342in}{2.977532in}}{\pgfqpoint{8.062339in}{2.982370in}}{\pgfqpoint{8.058772in}{2.985936in}}%
\pgfpathcurveto{\pgfqpoint{8.055206in}{2.989503in}}{\pgfqpoint{8.050368in}{2.991507in}}{\pgfqpoint{8.045324in}{2.991507in}}%
\pgfpathcurveto{\pgfqpoint{8.040281in}{2.991507in}}{\pgfqpoint{8.035443in}{2.989503in}}{\pgfqpoint{8.031876in}{2.985936in}}%
\pgfpathcurveto{\pgfqpoint{8.028310in}{2.982370in}}{\pgfqpoint{8.026306in}{2.977532in}}{\pgfqpoint{8.026306in}{2.972489in}}%
\pgfpathcurveto{\pgfqpoint{8.026306in}{2.967445in}}{\pgfqpoint{8.028310in}{2.962607in}}{\pgfqpoint{8.031876in}{2.959041in}}%
\pgfpathcurveto{\pgfqpoint{8.035443in}{2.955474in}}{\pgfqpoint{8.040281in}{2.953470in}}{\pgfqpoint{8.045324in}{2.953470in}}%
\pgfpathclose%
\pgfusepath{fill}%
\end{pgfscope}%
\begin{pgfscope}%
\pgfpathrectangle{\pgfqpoint{6.572727in}{0.474100in}}{\pgfqpoint{4.227273in}{3.318700in}}%
\pgfusepath{clip}%
\pgfsetbuttcap%
\pgfsetroundjoin%
\definecolor{currentfill}{rgb}{0.127568,0.566949,0.550556}%
\pgfsetfillcolor{currentfill}%
\pgfsetfillopacity{0.700000}%
\pgfsetlinewidth{0.000000pt}%
\definecolor{currentstroke}{rgb}{0.000000,0.000000,0.000000}%
\pgfsetstrokecolor{currentstroke}%
\pgfsetstrokeopacity{0.700000}%
\pgfsetdash{}{0pt}%
\pgfpathmoveto{\pgfqpoint{8.499750in}{2.760444in}}%
\pgfpathcurveto{\pgfqpoint{8.504794in}{2.760444in}}{\pgfqpoint{8.509632in}{2.762448in}}{\pgfqpoint{8.513198in}{2.766015in}}%
\pgfpathcurveto{\pgfqpoint{8.516765in}{2.769581in}}{\pgfqpoint{8.518769in}{2.774419in}}{\pgfqpoint{8.518769in}{2.779463in}}%
\pgfpathcurveto{\pgfqpoint{8.518769in}{2.784506in}}{\pgfqpoint{8.516765in}{2.789344in}}{\pgfqpoint{8.513198in}{2.792911in}}%
\pgfpathcurveto{\pgfqpoint{8.509632in}{2.796477in}}{\pgfqpoint{8.504794in}{2.798481in}}{\pgfqpoint{8.499750in}{2.798481in}}%
\pgfpathcurveto{\pgfqpoint{8.494707in}{2.798481in}}{\pgfqpoint{8.489869in}{2.796477in}}{\pgfqpoint{8.486303in}{2.792911in}}%
\pgfpathcurveto{\pgfqpoint{8.482736in}{2.789344in}}{\pgfqpoint{8.480732in}{2.784506in}}{\pgfqpoint{8.480732in}{2.779463in}}%
\pgfpathcurveto{\pgfqpoint{8.480732in}{2.774419in}}{\pgfqpoint{8.482736in}{2.769581in}}{\pgfqpoint{8.486303in}{2.766015in}}%
\pgfpathcurveto{\pgfqpoint{8.489869in}{2.762448in}}{\pgfqpoint{8.494707in}{2.760444in}}{\pgfqpoint{8.499750in}{2.760444in}}%
\pgfpathclose%
\pgfusepath{fill}%
\end{pgfscope}%
\begin{pgfscope}%
\pgfpathrectangle{\pgfqpoint{6.572727in}{0.474100in}}{\pgfqpoint{4.227273in}{3.318700in}}%
\pgfusepath{clip}%
\pgfsetbuttcap%
\pgfsetroundjoin%
\definecolor{currentfill}{rgb}{0.127568,0.566949,0.550556}%
\pgfsetfillcolor{currentfill}%
\pgfsetfillopacity{0.700000}%
\pgfsetlinewidth{0.000000pt}%
\definecolor{currentstroke}{rgb}{0.000000,0.000000,0.000000}%
\pgfsetstrokecolor{currentstroke}%
\pgfsetstrokeopacity{0.700000}%
\pgfsetdash{}{0pt}%
\pgfpathmoveto{\pgfqpoint{8.114727in}{2.352887in}}%
\pgfpathcurveto{\pgfqpoint{8.119771in}{2.352887in}}{\pgfqpoint{8.124609in}{2.354891in}}{\pgfqpoint{8.128175in}{2.358458in}}%
\pgfpathcurveto{\pgfqpoint{8.131742in}{2.362024in}}{\pgfqpoint{8.133746in}{2.366862in}}{\pgfqpoint{8.133746in}{2.371905in}}%
\pgfpathcurveto{\pgfqpoint{8.133746in}{2.376949in}}{\pgfqpoint{8.131742in}{2.381787in}}{\pgfqpoint{8.128175in}{2.385353in}}%
\pgfpathcurveto{\pgfqpoint{8.124609in}{2.388920in}}{\pgfqpoint{8.119771in}{2.390924in}}{\pgfqpoint{8.114727in}{2.390924in}}%
\pgfpathcurveto{\pgfqpoint{8.109684in}{2.390924in}}{\pgfqpoint{8.104846in}{2.388920in}}{\pgfqpoint{8.101280in}{2.385353in}}%
\pgfpathcurveto{\pgfqpoint{8.097713in}{2.381787in}}{\pgfqpoint{8.095709in}{2.376949in}}{\pgfqpoint{8.095709in}{2.371905in}}%
\pgfpathcurveto{\pgfqpoint{8.095709in}{2.366862in}}{\pgfqpoint{8.097713in}{2.362024in}}{\pgfqpoint{8.101280in}{2.358458in}}%
\pgfpathcurveto{\pgfqpoint{8.104846in}{2.354891in}}{\pgfqpoint{8.109684in}{2.352887in}}{\pgfqpoint{8.114727in}{2.352887in}}%
\pgfpathclose%
\pgfusepath{fill}%
\end{pgfscope}%
\begin{pgfscope}%
\pgfpathrectangle{\pgfqpoint{6.572727in}{0.474100in}}{\pgfqpoint{4.227273in}{3.318700in}}%
\pgfusepath{clip}%
\pgfsetbuttcap%
\pgfsetroundjoin%
\definecolor{currentfill}{rgb}{0.127568,0.566949,0.550556}%
\pgfsetfillcolor{currentfill}%
\pgfsetfillopacity{0.700000}%
\pgfsetlinewidth{0.000000pt}%
\definecolor{currentstroke}{rgb}{0.000000,0.000000,0.000000}%
\pgfsetstrokecolor{currentstroke}%
\pgfsetstrokeopacity{0.700000}%
\pgfsetdash{}{0pt}%
\pgfpathmoveto{\pgfqpoint{7.721510in}{2.648528in}}%
\pgfpathcurveto{\pgfqpoint{7.726554in}{2.648528in}}{\pgfqpoint{7.731391in}{2.650532in}}{\pgfqpoint{7.734958in}{2.654098in}}%
\pgfpathcurveto{\pgfqpoint{7.738524in}{2.657665in}}{\pgfqpoint{7.740528in}{2.662503in}}{\pgfqpoint{7.740528in}{2.667546in}}%
\pgfpathcurveto{\pgfqpoint{7.740528in}{2.672590in}}{\pgfqpoint{7.738524in}{2.677428in}}{\pgfqpoint{7.734958in}{2.680994in}}%
\pgfpathcurveto{\pgfqpoint{7.731391in}{2.684560in}}{\pgfqpoint{7.726554in}{2.686564in}}{\pgfqpoint{7.721510in}{2.686564in}}%
\pgfpathcurveto{\pgfqpoint{7.716466in}{2.686564in}}{\pgfqpoint{7.711629in}{2.684560in}}{\pgfqpoint{7.708062in}{2.680994in}}%
\pgfpathcurveto{\pgfqpoint{7.704496in}{2.677428in}}{\pgfqpoint{7.702492in}{2.672590in}}{\pgfqpoint{7.702492in}{2.667546in}}%
\pgfpathcurveto{\pgfqpoint{7.702492in}{2.662503in}}{\pgfqpoint{7.704496in}{2.657665in}}{\pgfqpoint{7.708062in}{2.654098in}}%
\pgfpathcurveto{\pgfqpoint{7.711629in}{2.650532in}}{\pgfqpoint{7.716466in}{2.648528in}}{\pgfqpoint{7.721510in}{2.648528in}}%
\pgfpathclose%
\pgfusepath{fill}%
\end{pgfscope}%
\begin{pgfscope}%
\pgfpathrectangle{\pgfqpoint{6.572727in}{0.474100in}}{\pgfqpoint{4.227273in}{3.318700in}}%
\pgfusepath{clip}%
\pgfsetbuttcap%
\pgfsetroundjoin%
\definecolor{currentfill}{rgb}{0.993248,0.906157,0.143936}%
\pgfsetfillcolor{currentfill}%
\pgfsetfillopacity{0.700000}%
\pgfsetlinewidth{0.000000pt}%
\definecolor{currentstroke}{rgb}{0.000000,0.000000,0.000000}%
\pgfsetstrokecolor{currentstroke}%
\pgfsetstrokeopacity{0.700000}%
\pgfsetdash{}{0pt}%
\pgfpathmoveto{\pgfqpoint{9.075194in}{2.058160in}}%
\pgfpathcurveto{\pgfqpoint{9.080238in}{2.058160in}}{\pgfqpoint{9.085076in}{2.060163in}}{\pgfqpoint{9.088642in}{2.063730in}}%
\pgfpathcurveto{\pgfqpoint{9.092208in}{2.067296in}}{\pgfqpoint{9.094212in}{2.072134in}}{\pgfqpoint{9.094212in}{2.077178in}}%
\pgfpathcurveto{\pgfqpoint{9.094212in}{2.082221in}}{\pgfqpoint{9.092208in}{2.087059in}}{\pgfqpoint{9.088642in}{2.090626in}}%
\pgfpathcurveto{\pgfqpoint{9.085076in}{2.094192in}}{\pgfqpoint{9.080238in}{2.096196in}}{\pgfqpoint{9.075194in}{2.096196in}}%
\pgfpathcurveto{\pgfqpoint{9.070151in}{2.096196in}}{\pgfqpoint{9.065313in}{2.094192in}}{\pgfqpoint{9.061746in}{2.090626in}}%
\pgfpathcurveto{\pgfqpoint{9.058180in}{2.087059in}}{\pgfqpoint{9.056176in}{2.082221in}}{\pgfqpoint{9.056176in}{2.077178in}}%
\pgfpathcurveto{\pgfqpoint{9.056176in}{2.072134in}}{\pgfqpoint{9.058180in}{2.067296in}}{\pgfqpoint{9.061746in}{2.063730in}}%
\pgfpathcurveto{\pgfqpoint{9.065313in}{2.060163in}}{\pgfqpoint{9.070151in}{2.058160in}}{\pgfqpoint{9.075194in}{2.058160in}}%
\pgfpathclose%
\pgfusepath{fill}%
\end{pgfscope}%
\begin{pgfscope}%
\pgfpathrectangle{\pgfqpoint{6.572727in}{0.474100in}}{\pgfqpoint{4.227273in}{3.318700in}}%
\pgfusepath{clip}%
\pgfsetbuttcap%
\pgfsetroundjoin%
\definecolor{currentfill}{rgb}{0.993248,0.906157,0.143936}%
\pgfsetfillcolor{currentfill}%
\pgfsetfillopacity{0.700000}%
\pgfsetlinewidth{0.000000pt}%
\definecolor{currentstroke}{rgb}{0.000000,0.000000,0.000000}%
\pgfsetstrokecolor{currentstroke}%
\pgfsetstrokeopacity{0.700000}%
\pgfsetdash{}{0pt}%
\pgfpathmoveto{\pgfqpoint{10.148880in}{2.046783in}}%
\pgfpathcurveto{\pgfqpoint{10.153924in}{2.046783in}}{\pgfqpoint{10.158762in}{2.048787in}}{\pgfqpoint{10.162328in}{2.052353in}}%
\pgfpathcurveto{\pgfqpoint{10.165894in}{2.055920in}}{\pgfqpoint{10.167898in}{2.060758in}}{\pgfqpoint{10.167898in}{2.065801in}}%
\pgfpathcurveto{\pgfqpoint{10.167898in}{2.070845in}}{\pgfqpoint{10.165894in}{2.075683in}}{\pgfqpoint{10.162328in}{2.079249in}}%
\pgfpathcurveto{\pgfqpoint{10.158762in}{2.082816in}}{\pgfqpoint{10.153924in}{2.084819in}}{\pgfqpoint{10.148880in}{2.084819in}}%
\pgfpathcurveto{\pgfqpoint{10.143836in}{2.084819in}}{\pgfqpoint{10.138999in}{2.082816in}}{\pgfqpoint{10.135432in}{2.079249in}}%
\pgfpathcurveto{\pgfqpoint{10.131866in}{2.075683in}}{\pgfqpoint{10.129862in}{2.070845in}}{\pgfqpoint{10.129862in}{2.065801in}}%
\pgfpathcurveto{\pgfqpoint{10.129862in}{2.060758in}}{\pgfqpoint{10.131866in}{2.055920in}}{\pgfqpoint{10.135432in}{2.052353in}}%
\pgfpathcurveto{\pgfqpoint{10.138999in}{2.048787in}}{\pgfqpoint{10.143836in}{2.046783in}}{\pgfqpoint{10.148880in}{2.046783in}}%
\pgfpathclose%
\pgfusepath{fill}%
\end{pgfscope}%
\begin{pgfscope}%
\pgfpathrectangle{\pgfqpoint{6.572727in}{0.474100in}}{\pgfqpoint{4.227273in}{3.318700in}}%
\pgfusepath{clip}%
\pgfsetbuttcap%
\pgfsetroundjoin%
\definecolor{currentfill}{rgb}{0.127568,0.566949,0.550556}%
\pgfsetfillcolor{currentfill}%
\pgfsetfillopacity{0.700000}%
\pgfsetlinewidth{0.000000pt}%
\definecolor{currentstroke}{rgb}{0.000000,0.000000,0.000000}%
\pgfsetstrokecolor{currentstroke}%
\pgfsetstrokeopacity{0.700000}%
\pgfsetdash{}{0pt}%
\pgfpathmoveto{\pgfqpoint{8.426245in}{2.775021in}}%
\pgfpathcurveto{\pgfqpoint{8.431289in}{2.775021in}}{\pgfqpoint{8.436126in}{2.777025in}}{\pgfqpoint{8.439693in}{2.780592in}}%
\pgfpathcurveto{\pgfqpoint{8.443259in}{2.784158in}}{\pgfqpoint{8.445263in}{2.788996in}}{\pgfqpoint{8.445263in}{2.794039in}}%
\pgfpathcurveto{\pgfqpoint{8.445263in}{2.799083in}}{\pgfqpoint{8.443259in}{2.803921in}}{\pgfqpoint{8.439693in}{2.807487in}}%
\pgfpathcurveto{\pgfqpoint{8.436126in}{2.811054in}}{\pgfqpoint{8.431289in}{2.813058in}}{\pgfqpoint{8.426245in}{2.813058in}}%
\pgfpathcurveto{\pgfqpoint{8.421201in}{2.813058in}}{\pgfqpoint{8.416363in}{2.811054in}}{\pgfqpoint{8.412797in}{2.807487in}}%
\pgfpathcurveto{\pgfqpoint{8.409231in}{2.803921in}}{\pgfqpoint{8.407227in}{2.799083in}}{\pgfqpoint{8.407227in}{2.794039in}}%
\pgfpathcurveto{\pgfqpoint{8.407227in}{2.788996in}}{\pgfqpoint{8.409231in}{2.784158in}}{\pgfqpoint{8.412797in}{2.780592in}}%
\pgfpathcurveto{\pgfqpoint{8.416363in}{2.777025in}}{\pgfqpoint{8.421201in}{2.775021in}}{\pgfqpoint{8.426245in}{2.775021in}}%
\pgfpathclose%
\pgfusepath{fill}%
\end{pgfscope}%
\begin{pgfscope}%
\pgfpathrectangle{\pgfqpoint{6.572727in}{0.474100in}}{\pgfqpoint{4.227273in}{3.318700in}}%
\pgfusepath{clip}%
\pgfsetbuttcap%
\pgfsetroundjoin%
\definecolor{currentfill}{rgb}{0.127568,0.566949,0.550556}%
\pgfsetfillcolor{currentfill}%
\pgfsetfillopacity{0.700000}%
\pgfsetlinewidth{0.000000pt}%
\definecolor{currentstroke}{rgb}{0.000000,0.000000,0.000000}%
\pgfsetstrokecolor{currentstroke}%
\pgfsetstrokeopacity{0.700000}%
\pgfsetdash{}{0pt}%
\pgfpathmoveto{\pgfqpoint{8.220088in}{2.371852in}}%
\pgfpathcurveto{\pgfqpoint{8.225131in}{2.371852in}}{\pgfqpoint{8.229969in}{2.373856in}}{\pgfqpoint{8.233536in}{2.377423in}}%
\pgfpathcurveto{\pgfqpoint{8.237102in}{2.380989in}}{\pgfqpoint{8.239106in}{2.385827in}}{\pgfqpoint{8.239106in}{2.390870in}}%
\pgfpathcurveto{\pgfqpoint{8.239106in}{2.395914in}}{\pgfqpoint{8.237102in}{2.400752in}}{\pgfqpoint{8.233536in}{2.404318in}}%
\pgfpathcurveto{\pgfqpoint{8.229969in}{2.407885in}}{\pgfqpoint{8.225131in}{2.409889in}}{\pgfqpoint{8.220088in}{2.409889in}}%
\pgfpathcurveto{\pgfqpoint{8.215044in}{2.409889in}}{\pgfqpoint{8.210206in}{2.407885in}}{\pgfqpoint{8.206640in}{2.404318in}}%
\pgfpathcurveto{\pgfqpoint{8.203074in}{2.400752in}}{\pgfqpoint{8.201070in}{2.395914in}}{\pgfqpoint{8.201070in}{2.390870in}}%
\pgfpathcurveto{\pgfqpoint{8.201070in}{2.385827in}}{\pgfqpoint{8.203074in}{2.380989in}}{\pgfqpoint{8.206640in}{2.377423in}}%
\pgfpathcurveto{\pgfqpoint{8.210206in}{2.373856in}}{\pgfqpoint{8.215044in}{2.371852in}}{\pgfqpoint{8.220088in}{2.371852in}}%
\pgfpathclose%
\pgfusepath{fill}%
\end{pgfscope}%
\begin{pgfscope}%
\pgfpathrectangle{\pgfqpoint{6.572727in}{0.474100in}}{\pgfqpoint{4.227273in}{3.318700in}}%
\pgfusepath{clip}%
\pgfsetbuttcap%
\pgfsetroundjoin%
\definecolor{currentfill}{rgb}{0.127568,0.566949,0.550556}%
\pgfsetfillcolor{currentfill}%
\pgfsetfillopacity{0.700000}%
\pgfsetlinewidth{0.000000pt}%
\definecolor{currentstroke}{rgb}{0.000000,0.000000,0.000000}%
\pgfsetstrokecolor{currentstroke}%
\pgfsetstrokeopacity{0.700000}%
\pgfsetdash{}{0pt}%
\pgfpathmoveto{\pgfqpoint{7.902854in}{1.510251in}}%
\pgfpathcurveto{\pgfqpoint{7.907897in}{1.510251in}}{\pgfqpoint{7.912735in}{1.512255in}}{\pgfqpoint{7.916302in}{1.515822in}}%
\pgfpathcurveto{\pgfqpoint{7.919868in}{1.519388in}}{\pgfqpoint{7.921872in}{1.524226in}}{\pgfqpoint{7.921872in}{1.529270in}}%
\pgfpathcurveto{\pgfqpoint{7.921872in}{1.534313in}}{\pgfqpoint{7.919868in}{1.539151in}}{\pgfqpoint{7.916302in}{1.542717in}}%
\pgfpathcurveto{\pgfqpoint{7.912735in}{1.546284in}}{\pgfqpoint{7.907897in}{1.548288in}}{\pgfqpoint{7.902854in}{1.548288in}}%
\pgfpathcurveto{\pgfqpoint{7.897810in}{1.548288in}}{\pgfqpoint{7.892972in}{1.546284in}}{\pgfqpoint{7.889406in}{1.542717in}}%
\pgfpathcurveto{\pgfqpoint{7.885839in}{1.539151in}}{\pgfqpoint{7.883836in}{1.534313in}}{\pgfqpoint{7.883836in}{1.529270in}}%
\pgfpathcurveto{\pgfqpoint{7.883836in}{1.524226in}}{\pgfqpoint{7.885839in}{1.519388in}}{\pgfqpoint{7.889406in}{1.515822in}}%
\pgfpathcurveto{\pgfqpoint{7.892972in}{1.512255in}}{\pgfqpoint{7.897810in}{1.510251in}}{\pgfqpoint{7.902854in}{1.510251in}}%
\pgfpathclose%
\pgfusepath{fill}%
\end{pgfscope}%
\begin{pgfscope}%
\pgfpathrectangle{\pgfqpoint{6.572727in}{0.474100in}}{\pgfqpoint{4.227273in}{3.318700in}}%
\pgfusepath{clip}%
\pgfsetbuttcap%
\pgfsetroundjoin%
\definecolor{currentfill}{rgb}{0.127568,0.566949,0.550556}%
\pgfsetfillcolor{currentfill}%
\pgfsetfillopacity{0.700000}%
\pgfsetlinewidth{0.000000pt}%
\definecolor{currentstroke}{rgb}{0.000000,0.000000,0.000000}%
\pgfsetstrokecolor{currentstroke}%
\pgfsetstrokeopacity{0.700000}%
\pgfsetdash{}{0pt}%
\pgfpathmoveto{\pgfqpoint{8.066402in}{1.466459in}}%
\pgfpathcurveto{\pgfqpoint{8.071446in}{1.466459in}}{\pgfqpoint{8.076283in}{1.468463in}}{\pgfqpoint{8.079850in}{1.472029in}}%
\pgfpathcurveto{\pgfqpoint{8.083416in}{1.475596in}}{\pgfqpoint{8.085420in}{1.480434in}}{\pgfqpoint{8.085420in}{1.485477in}}%
\pgfpathcurveto{\pgfqpoint{8.085420in}{1.490521in}}{\pgfqpoint{8.083416in}{1.495359in}}{\pgfqpoint{8.079850in}{1.498925in}}%
\pgfpathcurveto{\pgfqpoint{8.076283in}{1.502491in}}{\pgfqpoint{8.071446in}{1.504495in}}{\pgfqpoint{8.066402in}{1.504495in}}%
\pgfpathcurveto{\pgfqpoint{8.061358in}{1.504495in}}{\pgfqpoint{8.056521in}{1.502491in}}{\pgfqpoint{8.052954in}{1.498925in}}%
\pgfpathcurveto{\pgfqpoint{8.049388in}{1.495359in}}{\pgfqpoint{8.047384in}{1.490521in}}{\pgfqpoint{8.047384in}{1.485477in}}%
\pgfpathcurveto{\pgfqpoint{8.047384in}{1.480434in}}{\pgfqpoint{8.049388in}{1.475596in}}{\pgfqpoint{8.052954in}{1.472029in}}%
\pgfpathcurveto{\pgfqpoint{8.056521in}{1.468463in}}{\pgfqpoint{8.061358in}{1.466459in}}{\pgfqpoint{8.066402in}{1.466459in}}%
\pgfpathclose%
\pgfusepath{fill}%
\end{pgfscope}%
\begin{pgfscope}%
\pgfpathrectangle{\pgfqpoint{6.572727in}{0.474100in}}{\pgfqpoint{4.227273in}{3.318700in}}%
\pgfusepath{clip}%
\pgfsetbuttcap%
\pgfsetroundjoin%
\definecolor{currentfill}{rgb}{0.127568,0.566949,0.550556}%
\pgfsetfillcolor{currentfill}%
\pgfsetfillopacity{0.700000}%
\pgfsetlinewidth{0.000000pt}%
\definecolor{currentstroke}{rgb}{0.000000,0.000000,0.000000}%
\pgfsetstrokecolor{currentstroke}%
\pgfsetstrokeopacity{0.700000}%
\pgfsetdash{}{0pt}%
\pgfpathmoveto{\pgfqpoint{8.305048in}{2.846255in}}%
\pgfpathcurveto{\pgfqpoint{8.310092in}{2.846255in}}{\pgfqpoint{8.314930in}{2.848259in}}{\pgfqpoint{8.318496in}{2.851825in}}%
\pgfpathcurveto{\pgfqpoint{8.322063in}{2.855391in}}{\pgfqpoint{8.324067in}{2.860229in}}{\pgfqpoint{8.324067in}{2.865273in}}%
\pgfpathcurveto{\pgfqpoint{8.324067in}{2.870316in}}{\pgfqpoint{8.322063in}{2.875154in}}{\pgfqpoint{8.318496in}{2.878721in}}%
\pgfpathcurveto{\pgfqpoint{8.314930in}{2.882287in}}{\pgfqpoint{8.310092in}{2.884291in}}{\pgfqpoint{8.305048in}{2.884291in}}%
\pgfpathcurveto{\pgfqpoint{8.300005in}{2.884291in}}{\pgfqpoint{8.295167in}{2.882287in}}{\pgfqpoint{8.291601in}{2.878721in}}%
\pgfpathcurveto{\pgfqpoint{8.288034in}{2.875154in}}{\pgfqpoint{8.286030in}{2.870316in}}{\pgfqpoint{8.286030in}{2.865273in}}%
\pgfpathcurveto{\pgfqpoint{8.286030in}{2.860229in}}{\pgfqpoint{8.288034in}{2.855391in}}{\pgfqpoint{8.291601in}{2.851825in}}%
\pgfpathcurveto{\pgfqpoint{8.295167in}{2.848259in}}{\pgfqpoint{8.300005in}{2.846255in}}{\pgfqpoint{8.305048in}{2.846255in}}%
\pgfpathclose%
\pgfusepath{fill}%
\end{pgfscope}%
\begin{pgfscope}%
\pgfpathrectangle{\pgfqpoint{6.572727in}{0.474100in}}{\pgfqpoint{4.227273in}{3.318700in}}%
\pgfusepath{clip}%
\pgfsetbuttcap%
\pgfsetroundjoin%
\definecolor{currentfill}{rgb}{0.127568,0.566949,0.550556}%
\pgfsetfillcolor{currentfill}%
\pgfsetfillopacity{0.700000}%
\pgfsetlinewidth{0.000000pt}%
\definecolor{currentstroke}{rgb}{0.000000,0.000000,0.000000}%
\pgfsetstrokecolor{currentstroke}%
\pgfsetstrokeopacity{0.700000}%
\pgfsetdash{}{0pt}%
\pgfpathmoveto{\pgfqpoint{7.743296in}{1.659988in}}%
\pgfpathcurveto{\pgfqpoint{7.748339in}{1.659988in}}{\pgfqpoint{7.753177in}{1.661992in}}{\pgfqpoint{7.756743in}{1.665558in}}%
\pgfpathcurveto{\pgfqpoint{7.760310in}{1.669125in}}{\pgfqpoint{7.762314in}{1.673963in}}{\pgfqpoint{7.762314in}{1.679006in}}%
\pgfpathcurveto{\pgfqpoint{7.762314in}{1.684050in}}{\pgfqpoint{7.760310in}{1.688888in}}{\pgfqpoint{7.756743in}{1.692454in}}%
\pgfpathcurveto{\pgfqpoint{7.753177in}{1.696021in}}{\pgfqpoint{7.748339in}{1.698024in}}{\pgfqpoint{7.743296in}{1.698024in}}%
\pgfpathcurveto{\pgfqpoint{7.738252in}{1.698024in}}{\pgfqpoint{7.733414in}{1.696021in}}{\pgfqpoint{7.729848in}{1.692454in}}%
\pgfpathcurveto{\pgfqpoint{7.726281in}{1.688888in}}{\pgfqpoint{7.724277in}{1.684050in}}{\pgfqpoint{7.724277in}{1.679006in}}%
\pgfpathcurveto{\pgfqpoint{7.724277in}{1.673963in}}{\pgfqpoint{7.726281in}{1.669125in}}{\pgfqpoint{7.729848in}{1.665558in}}%
\pgfpathcurveto{\pgfqpoint{7.733414in}{1.661992in}}{\pgfqpoint{7.738252in}{1.659988in}}{\pgfqpoint{7.743296in}{1.659988in}}%
\pgfpathclose%
\pgfusepath{fill}%
\end{pgfscope}%
\begin{pgfscope}%
\pgfpathrectangle{\pgfqpoint{6.572727in}{0.474100in}}{\pgfqpoint{4.227273in}{3.318700in}}%
\pgfusepath{clip}%
\pgfsetbuttcap%
\pgfsetroundjoin%
\definecolor{currentfill}{rgb}{0.993248,0.906157,0.143936}%
\pgfsetfillcolor{currentfill}%
\pgfsetfillopacity{0.700000}%
\pgfsetlinewidth{0.000000pt}%
\definecolor{currentstroke}{rgb}{0.000000,0.000000,0.000000}%
\pgfsetstrokecolor{currentstroke}%
\pgfsetstrokeopacity{0.700000}%
\pgfsetdash{}{0pt}%
\pgfpathmoveto{\pgfqpoint{9.271095in}{1.896473in}}%
\pgfpathcurveto{\pgfqpoint{9.276139in}{1.896473in}}{\pgfqpoint{9.280977in}{1.898476in}}{\pgfqpoint{9.284543in}{1.902043in}}%
\pgfpathcurveto{\pgfqpoint{9.288109in}{1.905609in}}{\pgfqpoint{9.290113in}{1.910447in}}{\pgfqpoint{9.290113in}{1.915491in}}%
\pgfpathcurveto{\pgfqpoint{9.290113in}{1.920534in}}{\pgfqpoint{9.288109in}{1.925372in}}{\pgfqpoint{9.284543in}{1.928939in}}%
\pgfpathcurveto{\pgfqpoint{9.280977in}{1.932505in}}{\pgfqpoint{9.276139in}{1.934509in}}{\pgfqpoint{9.271095in}{1.934509in}}%
\pgfpathcurveto{\pgfqpoint{9.266051in}{1.934509in}}{\pgfqpoint{9.261214in}{1.932505in}}{\pgfqpoint{9.257647in}{1.928939in}}%
\pgfpathcurveto{\pgfqpoint{9.254081in}{1.925372in}}{\pgfqpoint{9.252077in}{1.920534in}}{\pgfqpoint{9.252077in}{1.915491in}}%
\pgfpathcurveto{\pgfqpoint{9.252077in}{1.910447in}}{\pgfqpoint{9.254081in}{1.905609in}}{\pgfqpoint{9.257647in}{1.902043in}}%
\pgfpathcurveto{\pgfqpoint{9.261214in}{1.898476in}}{\pgfqpoint{9.266051in}{1.896473in}}{\pgfqpoint{9.271095in}{1.896473in}}%
\pgfpathclose%
\pgfusepath{fill}%
\end{pgfscope}%
\begin{pgfscope}%
\pgfpathrectangle{\pgfqpoint{6.572727in}{0.474100in}}{\pgfqpoint{4.227273in}{3.318700in}}%
\pgfusepath{clip}%
\pgfsetbuttcap%
\pgfsetroundjoin%
\definecolor{currentfill}{rgb}{0.993248,0.906157,0.143936}%
\pgfsetfillcolor{currentfill}%
\pgfsetfillopacity{0.700000}%
\pgfsetlinewidth{0.000000pt}%
\definecolor{currentstroke}{rgb}{0.000000,0.000000,0.000000}%
\pgfsetstrokecolor{currentstroke}%
\pgfsetstrokeopacity{0.700000}%
\pgfsetdash{}{0pt}%
\pgfpathmoveto{\pgfqpoint{9.844824in}{1.009999in}}%
\pgfpathcurveto{\pgfqpoint{9.849868in}{1.009999in}}{\pgfqpoint{9.854706in}{1.012003in}}{\pgfqpoint{9.858272in}{1.015569in}}%
\pgfpathcurveto{\pgfqpoint{9.861838in}{1.019136in}}{\pgfqpoint{9.863842in}{1.023974in}}{\pgfqpoint{9.863842in}{1.029017in}}%
\pgfpathcurveto{\pgfqpoint{9.863842in}{1.034061in}}{\pgfqpoint{9.861838in}{1.038899in}}{\pgfqpoint{9.858272in}{1.042465in}}%
\pgfpathcurveto{\pgfqpoint{9.854706in}{1.046032in}}{\pgfqpoint{9.849868in}{1.048035in}}{\pgfqpoint{9.844824in}{1.048035in}}%
\pgfpathcurveto{\pgfqpoint{9.839780in}{1.048035in}}{\pgfqpoint{9.834943in}{1.046032in}}{\pgfqpoint{9.831376in}{1.042465in}}%
\pgfpathcurveto{\pgfqpoint{9.827810in}{1.038899in}}{\pgfqpoint{9.825806in}{1.034061in}}{\pgfqpoint{9.825806in}{1.029017in}}%
\pgfpathcurveto{\pgfqpoint{9.825806in}{1.023974in}}{\pgfqpoint{9.827810in}{1.019136in}}{\pgfqpoint{9.831376in}{1.015569in}}%
\pgfpathcurveto{\pgfqpoint{9.834943in}{1.012003in}}{\pgfqpoint{9.839780in}{1.009999in}}{\pgfqpoint{9.844824in}{1.009999in}}%
\pgfpathclose%
\pgfusepath{fill}%
\end{pgfscope}%
\begin{pgfscope}%
\pgfpathrectangle{\pgfqpoint{6.572727in}{0.474100in}}{\pgfqpoint{4.227273in}{3.318700in}}%
\pgfusepath{clip}%
\pgfsetbuttcap%
\pgfsetroundjoin%
\definecolor{currentfill}{rgb}{0.127568,0.566949,0.550556}%
\pgfsetfillcolor{currentfill}%
\pgfsetfillopacity{0.700000}%
\pgfsetlinewidth{0.000000pt}%
\definecolor{currentstroke}{rgb}{0.000000,0.000000,0.000000}%
\pgfsetstrokecolor{currentstroke}%
\pgfsetstrokeopacity{0.700000}%
\pgfsetdash{}{0pt}%
\pgfpathmoveto{\pgfqpoint{8.068550in}{3.021982in}}%
\pgfpathcurveto{\pgfqpoint{8.073594in}{3.021982in}}{\pgfqpoint{8.078432in}{3.023986in}}{\pgfqpoint{8.081998in}{3.027552in}}%
\pgfpathcurveto{\pgfqpoint{8.085565in}{3.031119in}}{\pgfqpoint{8.087569in}{3.035956in}}{\pgfqpoint{8.087569in}{3.041000in}}%
\pgfpathcurveto{\pgfqpoint{8.087569in}{3.046044in}}{\pgfqpoint{8.085565in}{3.050881in}}{\pgfqpoint{8.081998in}{3.054448in}}%
\pgfpathcurveto{\pgfqpoint{8.078432in}{3.058014in}}{\pgfqpoint{8.073594in}{3.060018in}}{\pgfqpoint{8.068550in}{3.060018in}}%
\pgfpathcurveto{\pgfqpoint{8.063507in}{3.060018in}}{\pgfqpoint{8.058669in}{3.058014in}}{\pgfqpoint{8.055103in}{3.054448in}}%
\pgfpathcurveto{\pgfqpoint{8.051536in}{3.050881in}}{\pgfqpoint{8.049532in}{3.046044in}}{\pgfqpoint{8.049532in}{3.041000in}}%
\pgfpathcurveto{\pgfqpoint{8.049532in}{3.035956in}}{\pgfqpoint{8.051536in}{3.031119in}}{\pgfqpoint{8.055103in}{3.027552in}}%
\pgfpathcurveto{\pgfqpoint{8.058669in}{3.023986in}}{\pgfqpoint{8.063507in}{3.021982in}}{\pgfqpoint{8.068550in}{3.021982in}}%
\pgfpathclose%
\pgfusepath{fill}%
\end{pgfscope}%
\begin{pgfscope}%
\pgfpathrectangle{\pgfqpoint{6.572727in}{0.474100in}}{\pgfqpoint{4.227273in}{3.318700in}}%
\pgfusepath{clip}%
\pgfsetbuttcap%
\pgfsetroundjoin%
\definecolor{currentfill}{rgb}{0.993248,0.906157,0.143936}%
\pgfsetfillcolor{currentfill}%
\pgfsetfillopacity{0.700000}%
\pgfsetlinewidth{0.000000pt}%
\definecolor{currentstroke}{rgb}{0.000000,0.000000,0.000000}%
\pgfsetstrokecolor{currentstroke}%
\pgfsetstrokeopacity{0.700000}%
\pgfsetdash{}{0pt}%
\pgfpathmoveto{\pgfqpoint{9.180485in}{1.696302in}}%
\pgfpathcurveto{\pgfqpoint{9.185529in}{1.696302in}}{\pgfqpoint{9.190367in}{1.698306in}}{\pgfqpoint{9.193933in}{1.701873in}}%
\pgfpathcurveto{\pgfqpoint{9.197500in}{1.705439in}}{\pgfqpoint{9.199504in}{1.710277in}}{\pgfqpoint{9.199504in}{1.715320in}}%
\pgfpathcurveto{\pgfqpoint{9.199504in}{1.720364in}}{\pgfqpoint{9.197500in}{1.725202in}}{\pgfqpoint{9.193933in}{1.728768in}}%
\pgfpathcurveto{\pgfqpoint{9.190367in}{1.732335in}}{\pgfqpoint{9.185529in}{1.734339in}}{\pgfqpoint{9.180485in}{1.734339in}}%
\pgfpathcurveto{\pgfqpoint{9.175442in}{1.734339in}}{\pgfqpoint{9.170604in}{1.732335in}}{\pgfqpoint{9.167038in}{1.728768in}}%
\pgfpathcurveto{\pgfqpoint{9.163471in}{1.725202in}}{\pgfqpoint{9.161467in}{1.720364in}}{\pgfqpoint{9.161467in}{1.715320in}}%
\pgfpathcurveto{\pgfqpoint{9.161467in}{1.710277in}}{\pgfqpoint{9.163471in}{1.705439in}}{\pgfqpoint{9.167038in}{1.701873in}}%
\pgfpathcurveto{\pgfqpoint{9.170604in}{1.698306in}}{\pgfqpoint{9.175442in}{1.696302in}}{\pgfqpoint{9.180485in}{1.696302in}}%
\pgfpathclose%
\pgfusepath{fill}%
\end{pgfscope}%
\begin{pgfscope}%
\pgfpathrectangle{\pgfqpoint{6.572727in}{0.474100in}}{\pgfqpoint{4.227273in}{3.318700in}}%
\pgfusepath{clip}%
\pgfsetbuttcap%
\pgfsetroundjoin%
\definecolor{currentfill}{rgb}{0.127568,0.566949,0.550556}%
\pgfsetfillcolor{currentfill}%
\pgfsetfillopacity{0.700000}%
\pgfsetlinewidth{0.000000pt}%
\definecolor{currentstroke}{rgb}{0.000000,0.000000,0.000000}%
\pgfsetstrokecolor{currentstroke}%
\pgfsetstrokeopacity{0.700000}%
\pgfsetdash{}{0pt}%
\pgfpathmoveto{\pgfqpoint{7.364351in}{1.572236in}}%
\pgfpathcurveto{\pgfqpoint{7.369395in}{1.572236in}}{\pgfqpoint{7.374233in}{1.574240in}}{\pgfqpoint{7.377799in}{1.577807in}}%
\pgfpathcurveto{\pgfqpoint{7.381366in}{1.581373in}}{\pgfqpoint{7.383370in}{1.586211in}}{\pgfqpoint{7.383370in}{1.591254in}}%
\pgfpathcurveto{\pgfqpoint{7.383370in}{1.596298in}}{\pgfqpoint{7.381366in}{1.601136in}}{\pgfqpoint{7.377799in}{1.604702in}}%
\pgfpathcurveto{\pgfqpoint{7.374233in}{1.608269in}}{\pgfqpoint{7.369395in}{1.610273in}}{\pgfqpoint{7.364351in}{1.610273in}}%
\pgfpathcurveto{\pgfqpoint{7.359308in}{1.610273in}}{\pgfqpoint{7.354470in}{1.608269in}}{\pgfqpoint{7.350904in}{1.604702in}}%
\pgfpathcurveto{\pgfqpoint{7.347337in}{1.601136in}}{\pgfqpoint{7.345333in}{1.596298in}}{\pgfqpoint{7.345333in}{1.591254in}}%
\pgfpathcurveto{\pgfqpoint{7.345333in}{1.586211in}}{\pgfqpoint{7.347337in}{1.581373in}}{\pgfqpoint{7.350904in}{1.577807in}}%
\pgfpathcurveto{\pgfqpoint{7.354470in}{1.574240in}}{\pgfqpoint{7.359308in}{1.572236in}}{\pgfqpoint{7.364351in}{1.572236in}}%
\pgfpathclose%
\pgfusepath{fill}%
\end{pgfscope}%
\begin{pgfscope}%
\pgfpathrectangle{\pgfqpoint{6.572727in}{0.474100in}}{\pgfqpoint{4.227273in}{3.318700in}}%
\pgfusepath{clip}%
\pgfsetbuttcap%
\pgfsetroundjoin%
\definecolor{currentfill}{rgb}{0.127568,0.566949,0.550556}%
\pgfsetfillcolor{currentfill}%
\pgfsetfillopacity{0.700000}%
\pgfsetlinewidth{0.000000pt}%
\definecolor{currentstroke}{rgb}{0.000000,0.000000,0.000000}%
\pgfsetstrokecolor{currentstroke}%
\pgfsetstrokeopacity{0.700000}%
\pgfsetdash{}{0pt}%
\pgfpathmoveto{\pgfqpoint{8.498929in}{3.046928in}}%
\pgfpathcurveto{\pgfqpoint{8.503973in}{3.046928in}}{\pgfqpoint{8.508811in}{3.048932in}}{\pgfqpoint{8.512377in}{3.052498in}}%
\pgfpathcurveto{\pgfqpoint{8.515944in}{3.056064in}}{\pgfqpoint{8.517947in}{3.060902in}}{\pgfqpoint{8.517947in}{3.065946in}}%
\pgfpathcurveto{\pgfqpoint{8.517947in}{3.070989in}}{\pgfqpoint{8.515944in}{3.075827in}}{\pgfqpoint{8.512377in}{3.079394in}}%
\pgfpathcurveto{\pgfqpoint{8.508811in}{3.082960in}}{\pgfqpoint{8.503973in}{3.084964in}}{\pgfqpoint{8.498929in}{3.084964in}}%
\pgfpathcurveto{\pgfqpoint{8.493886in}{3.084964in}}{\pgfqpoint{8.489048in}{3.082960in}}{\pgfqpoint{8.485481in}{3.079394in}}%
\pgfpathcurveto{\pgfqpoint{8.481915in}{3.075827in}}{\pgfqpoint{8.479911in}{3.070989in}}{\pgfqpoint{8.479911in}{3.065946in}}%
\pgfpathcurveto{\pgfqpoint{8.479911in}{3.060902in}}{\pgfqpoint{8.481915in}{3.056064in}}{\pgfqpoint{8.485481in}{3.052498in}}%
\pgfpathcurveto{\pgfqpoint{8.489048in}{3.048932in}}{\pgfqpoint{8.493886in}{3.046928in}}{\pgfqpoint{8.498929in}{3.046928in}}%
\pgfpathclose%
\pgfusepath{fill}%
\end{pgfscope}%
\begin{pgfscope}%
\pgfpathrectangle{\pgfqpoint{6.572727in}{0.474100in}}{\pgfqpoint{4.227273in}{3.318700in}}%
\pgfusepath{clip}%
\pgfsetbuttcap%
\pgfsetroundjoin%
\definecolor{currentfill}{rgb}{0.127568,0.566949,0.550556}%
\pgfsetfillcolor{currentfill}%
\pgfsetfillopacity{0.700000}%
\pgfsetlinewidth{0.000000pt}%
\definecolor{currentstroke}{rgb}{0.000000,0.000000,0.000000}%
\pgfsetstrokecolor{currentstroke}%
\pgfsetstrokeopacity{0.700000}%
\pgfsetdash{}{0pt}%
\pgfpathmoveto{\pgfqpoint{7.473740in}{1.447246in}}%
\pgfpathcurveto{\pgfqpoint{7.478783in}{1.447246in}}{\pgfqpoint{7.483621in}{1.449249in}}{\pgfqpoint{7.487188in}{1.452816in}}%
\pgfpathcurveto{\pgfqpoint{7.490754in}{1.456382in}}{\pgfqpoint{7.492758in}{1.461220in}}{\pgfqpoint{7.492758in}{1.466264in}}%
\pgfpathcurveto{\pgfqpoint{7.492758in}{1.471307in}}{\pgfqpoint{7.490754in}{1.476145in}}{\pgfqpoint{7.487188in}{1.479712in}}%
\pgfpathcurveto{\pgfqpoint{7.483621in}{1.483278in}}{\pgfqpoint{7.478783in}{1.485282in}}{\pgfqpoint{7.473740in}{1.485282in}}%
\pgfpathcurveto{\pgfqpoint{7.468696in}{1.485282in}}{\pgfqpoint{7.463858in}{1.483278in}}{\pgfqpoint{7.460292in}{1.479712in}}%
\pgfpathcurveto{\pgfqpoint{7.456726in}{1.476145in}}{\pgfqpoint{7.454722in}{1.471307in}}{\pgfqpoint{7.454722in}{1.466264in}}%
\pgfpathcurveto{\pgfqpoint{7.454722in}{1.461220in}}{\pgfqpoint{7.456726in}{1.456382in}}{\pgfqpoint{7.460292in}{1.452816in}}%
\pgfpathcurveto{\pgfqpoint{7.463858in}{1.449249in}}{\pgfqpoint{7.468696in}{1.447246in}}{\pgfqpoint{7.473740in}{1.447246in}}%
\pgfpathclose%
\pgfusepath{fill}%
\end{pgfscope}%
\begin{pgfscope}%
\pgfpathrectangle{\pgfqpoint{6.572727in}{0.474100in}}{\pgfqpoint{4.227273in}{3.318700in}}%
\pgfusepath{clip}%
\pgfsetbuttcap%
\pgfsetroundjoin%
\definecolor{currentfill}{rgb}{0.127568,0.566949,0.550556}%
\pgfsetfillcolor{currentfill}%
\pgfsetfillopacity{0.700000}%
\pgfsetlinewidth{0.000000pt}%
\definecolor{currentstroke}{rgb}{0.000000,0.000000,0.000000}%
\pgfsetstrokecolor{currentstroke}%
\pgfsetstrokeopacity{0.700000}%
\pgfsetdash{}{0pt}%
\pgfpathmoveto{\pgfqpoint{8.014006in}{2.652650in}}%
\pgfpathcurveto{\pgfqpoint{8.019050in}{2.652650in}}{\pgfqpoint{8.023888in}{2.654654in}}{\pgfqpoint{8.027454in}{2.658221in}}%
\pgfpathcurveto{\pgfqpoint{8.031021in}{2.661787in}}{\pgfqpoint{8.033024in}{2.666625in}}{\pgfqpoint{8.033024in}{2.671668in}}%
\pgfpathcurveto{\pgfqpoint{8.033024in}{2.676712in}}{\pgfqpoint{8.031021in}{2.681550in}}{\pgfqpoint{8.027454in}{2.685116in}}%
\pgfpathcurveto{\pgfqpoint{8.023888in}{2.688683in}}{\pgfqpoint{8.019050in}{2.690687in}}{\pgfqpoint{8.014006in}{2.690687in}}%
\pgfpathcurveto{\pgfqpoint{8.008963in}{2.690687in}}{\pgfqpoint{8.004125in}{2.688683in}}{\pgfqpoint{8.000558in}{2.685116in}}%
\pgfpathcurveto{\pgfqpoint{7.996992in}{2.681550in}}{\pgfqpoint{7.994988in}{2.676712in}}{\pgfqpoint{7.994988in}{2.671668in}}%
\pgfpathcurveto{\pgfqpoint{7.994988in}{2.666625in}}{\pgfqpoint{7.996992in}{2.661787in}}{\pgfqpoint{8.000558in}{2.658221in}}%
\pgfpathcurveto{\pgfqpoint{8.004125in}{2.654654in}}{\pgfqpoint{8.008963in}{2.652650in}}{\pgfqpoint{8.014006in}{2.652650in}}%
\pgfpathclose%
\pgfusepath{fill}%
\end{pgfscope}%
\begin{pgfscope}%
\pgfpathrectangle{\pgfqpoint{6.572727in}{0.474100in}}{\pgfqpoint{4.227273in}{3.318700in}}%
\pgfusepath{clip}%
\pgfsetbuttcap%
\pgfsetroundjoin%
\definecolor{currentfill}{rgb}{0.127568,0.566949,0.550556}%
\pgfsetfillcolor{currentfill}%
\pgfsetfillopacity{0.700000}%
\pgfsetlinewidth{0.000000pt}%
\definecolor{currentstroke}{rgb}{0.000000,0.000000,0.000000}%
\pgfsetstrokecolor{currentstroke}%
\pgfsetstrokeopacity{0.700000}%
\pgfsetdash{}{0pt}%
\pgfpathmoveto{\pgfqpoint{8.168779in}{3.109108in}}%
\pgfpathcurveto{\pgfqpoint{8.173822in}{3.109108in}}{\pgfqpoint{8.178660in}{3.111112in}}{\pgfqpoint{8.182227in}{3.114679in}}%
\pgfpathcurveto{\pgfqpoint{8.185793in}{3.118245in}}{\pgfqpoint{8.187797in}{3.123083in}}{\pgfqpoint{8.187797in}{3.128127in}}%
\pgfpathcurveto{\pgfqpoint{8.187797in}{3.133170in}}{\pgfqpoint{8.185793in}{3.138008in}}{\pgfqpoint{8.182227in}{3.141574in}}%
\pgfpathcurveto{\pgfqpoint{8.178660in}{3.145141in}}{\pgfqpoint{8.173822in}{3.147145in}}{\pgfqpoint{8.168779in}{3.147145in}}%
\pgfpathcurveto{\pgfqpoint{8.163735in}{3.147145in}}{\pgfqpoint{8.158897in}{3.145141in}}{\pgfqpoint{8.155331in}{3.141574in}}%
\pgfpathcurveto{\pgfqpoint{8.151764in}{3.138008in}}{\pgfqpoint{8.149761in}{3.133170in}}{\pgfqpoint{8.149761in}{3.128127in}}%
\pgfpathcurveto{\pgfqpoint{8.149761in}{3.123083in}}{\pgfqpoint{8.151764in}{3.118245in}}{\pgfqpoint{8.155331in}{3.114679in}}%
\pgfpathcurveto{\pgfqpoint{8.158897in}{3.111112in}}{\pgfqpoint{8.163735in}{3.109108in}}{\pgfqpoint{8.168779in}{3.109108in}}%
\pgfpathclose%
\pgfusepath{fill}%
\end{pgfscope}%
\begin{pgfscope}%
\pgfpathrectangle{\pgfqpoint{6.572727in}{0.474100in}}{\pgfqpoint{4.227273in}{3.318700in}}%
\pgfusepath{clip}%
\pgfsetbuttcap%
\pgfsetroundjoin%
\definecolor{currentfill}{rgb}{0.127568,0.566949,0.550556}%
\pgfsetfillcolor{currentfill}%
\pgfsetfillopacity{0.700000}%
\pgfsetlinewidth{0.000000pt}%
\definecolor{currentstroke}{rgb}{0.000000,0.000000,0.000000}%
\pgfsetstrokecolor{currentstroke}%
\pgfsetstrokeopacity{0.700000}%
\pgfsetdash{}{0pt}%
\pgfpathmoveto{\pgfqpoint{8.098024in}{2.938047in}}%
\pgfpathcurveto{\pgfqpoint{8.103068in}{2.938047in}}{\pgfqpoint{8.107905in}{2.940051in}}{\pgfqpoint{8.111472in}{2.943617in}}%
\pgfpathcurveto{\pgfqpoint{8.115038in}{2.947184in}}{\pgfqpoint{8.117042in}{2.952022in}}{\pgfqpoint{8.117042in}{2.957065in}}%
\pgfpathcurveto{\pgfqpoint{8.117042in}{2.962109in}}{\pgfqpoint{8.115038in}{2.966947in}}{\pgfqpoint{8.111472in}{2.970513in}}%
\pgfpathcurveto{\pgfqpoint{8.107905in}{2.974079in}}{\pgfqpoint{8.103068in}{2.976083in}}{\pgfqpoint{8.098024in}{2.976083in}}%
\pgfpathcurveto{\pgfqpoint{8.092980in}{2.976083in}}{\pgfqpoint{8.088143in}{2.974079in}}{\pgfqpoint{8.084576in}{2.970513in}}%
\pgfpathcurveto{\pgfqpoint{8.081010in}{2.966947in}}{\pgfqpoint{8.079006in}{2.962109in}}{\pgfqpoint{8.079006in}{2.957065in}}%
\pgfpathcurveto{\pgfqpoint{8.079006in}{2.952022in}}{\pgfqpoint{8.081010in}{2.947184in}}{\pgfqpoint{8.084576in}{2.943617in}}%
\pgfpathcurveto{\pgfqpoint{8.088143in}{2.940051in}}{\pgfqpoint{8.092980in}{2.938047in}}{\pgfqpoint{8.098024in}{2.938047in}}%
\pgfpathclose%
\pgfusepath{fill}%
\end{pgfscope}%
\begin{pgfscope}%
\pgfpathrectangle{\pgfqpoint{6.572727in}{0.474100in}}{\pgfqpoint{4.227273in}{3.318700in}}%
\pgfusepath{clip}%
\pgfsetbuttcap%
\pgfsetroundjoin%
\definecolor{currentfill}{rgb}{0.127568,0.566949,0.550556}%
\pgfsetfillcolor{currentfill}%
\pgfsetfillopacity{0.700000}%
\pgfsetlinewidth{0.000000pt}%
\definecolor{currentstroke}{rgb}{0.000000,0.000000,0.000000}%
\pgfsetstrokecolor{currentstroke}%
\pgfsetstrokeopacity{0.700000}%
\pgfsetdash{}{0pt}%
\pgfpathmoveto{\pgfqpoint{7.048983in}{1.682833in}}%
\pgfpathcurveto{\pgfqpoint{7.054027in}{1.682833in}}{\pgfqpoint{7.058864in}{1.684837in}}{\pgfqpoint{7.062431in}{1.688403in}}%
\pgfpathcurveto{\pgfqpoint{7.065997in}{1.691970in}}{\pgfqpoint{7.068001in}{1.696807in}}{\pgfqpoint{7.068001in}{1.701851in}}%
\pgfpathcurveto{\pgfqpoint{7.068001in}{1.706895in}}{\pgfqpoint{7.065997in}{1.711732in}}{\pgfqpoint{7.062431in}{1.715299in}}%
\pgfpathcurveto{\pgfqpoint{7.058864in}{1.718865in}}{\pgfqpoint{7.054027in}{1.720869in}}{\pgfqpoint{7.048983in}{1.720869in}}%
\pgfpathcurveto{\pgfqpoint{7.043939in}{1.720869in}}{\pgfqpoint{7.039102in}{1.718865in}}{\pgfqpoint{7.035535in}{1.715299in}}%
\pgfpathcurveto{\pgfqpoint{7.031969in}{1.711732in}}{\pgfqpoint{7.029965in}{1.706895in}}{\pgfqpoint{7.029965in}{1.701851in}}%
\pgfpathcurveto{\pgfqpoint{7.029965in}{1.696807in}}{\pgfqpoint{7.031969in}{1.691970in}}{\pgfqpoint{7.035535in}{1.688403in}}%
\pgfpathcurveto{\pgfqpoint{7.039102in}{1.684837in}}{\pgfqpoint{7.043939in}{1.682833in}}{\pgfqpoint{7.048983in}{1.682833in}}%
\pgfpathclose%
\pgfusepath{fill}%
\end{pgfscope}%
\begin{pgfscope}%
\pgfpathrectangle{\pgfqpoint{6.572727in}{0.474100in}}{\pgfqpoint{4.227273in}{3.318700in}}%
\pgfusepath{clip}%
\pgfsetbuttcap%
\pgfsetroundjoin%
\definecolor{currentfill}{rgb}{0.127568,0.566949,0.550556}%
\pgfsetfillcolor{currentfill}%
\pgfsetfillopacity{0.700000}%
\pgfsetlinewidth{0.000000pt}%
\definecolor{currentstroke}{rgb}{0.000000,0.000000,0.000000}%
\pgfsetstrokecolor{currentstroke}%
\pgfsetstrokeopacity{0.700000}%
\pgfsetdash{}{0pt}%
\pgfpathmoveto{\pgfqpoint{7.721729in}{1.385423in}}%
\pgfpathcurveto{\pgfqpoint{7.726773in}{1.385423in}}{\pgfqpoint{7.731611in}{1.387427in}}{\pgfqpoint{7.735177in}{1.390994in}}%
\pgfpathcurveto{\pgfqpoint{7.738744in}{1.394560in}}{\pgfqpoint{7.740747in}{1.399398in}}{\pgfqpoint{7.740747in}{1.404442in}}%
\pgfpathcurveto{\pgfqpoint{7.740747in}{1.409485in}}{\pgfqpoint{7.738744in}{1.414323in}}{\pgfqpoint{7.735177in}{1.417889in}}%
\pgfpathcurveto{\pgfqpoint{7.731611in}{1.421456in}}{\pgfqpoint{7.726773in}{1.423460in}}{\pgfqpoint{7.721729in}{1.423460in}}%
\pgfpathcurveto{\pgfqpoint{7.716686in}{1.423460in}}{\pgfqpoint{7.711848in}{1.421456in}}{\pgfqpoint{7.708281in}{1.417889in}}%
\pgfpathcurveto{\pgfqpoint{7.704715in}{1.414323in}}{\pgfqpoint{7.702711in}{1.409485in}}{\pgfqpoint{7.702711in}{1.404442in}}%
\pgfpathcurveto{\pgfqpoint{7.702711in}{1.399398in}}{\pgfqpoint{7.704715in}{1.394560in}}{\pgfqpoint{7.708281in}{1.390994in}}%
\pgfpathcurveto{\pgfqpoint{7.711848in}{1.387427in}}{\pgfqpoint{7.716686in}{1.385423in}}{\pgfqpoint{7.721729in}{1.385423in}}%
\pgfpathclose%
\pgfusepath{fill}%
\end{pgfscope}%
\begin{pgfscope}%
\pgfpathrectangle{\pgfqpoint{6.572727in}{0.474100in}}{\pgfqpoint{4.227273in}{3.318700in}}%
\pgfusepath{clip}%
\pgfsetbuttcap%
\pgfsetroundjoin%
\definecolor{currentfill}{rgb}{0.127568,0.566949,0.550556}%
\pgfsetfillcolor{currentfill}%
\pgfsetfillopacity{0.700000}%
\pgfsetlinewidth{0.000000pt}%
\definecolor{currentstroke}{rgb}{0.000000,0.000000,0.000000}%
\pgfsetstrokecolor{currentstroke}%
\pgfsetstrokeopacity{0.700000}%
\pgfsetdash{}{0pt}%
\pgfpathmoveto{\pgfqpoint{7.758603in}{1.121398in}}%
\pgfpathcurveto{\pgfqpoint{7.763647in}{1.121398in}}{\pgfqpoint{7.768484in}{1.123402in}}{\pgfqpoint{7.772051in}{1.126969in}}%
\pgfpathcurveto{\pgfqpoint{7.775617in}{1.130535in}}{\pgfqpoint{7.777621in}{1.135373in}}{\pgfqpoint{7.777621in}{1.140417in}}%
\pgfpathcurveto{\pgfqpoint{7.777621in}{1.145460in}}{\pgfqpoint{7.775617in}{1.150298in}}{\pgfqpoint{7.772051in}{1.153864in}}%
\pgfpathcurveto{\pgfqpoint{7.768484in}{1.157431in}}{\pgfqpoint{7.763647in}{1.159435in}}{\pgfqpoint{7.758603in}{1.159435in}}%
\pgfpathcurveto{\pgfqpoint{7.753559in}{1.159435in}}{\pgfqpoint{7.748721in}{1.157431in}}{\pgfqpoint{7.745155in}{1.153864in}}%
\pgfpathcurveto{\pgfqpoint{7.741589in}{1.150298in}}{\pgfqpoint{7.739585in}{1.145460in}}{\pgfqpoint{7.739585in}{1.140417in}}%
\pgfpathcurveto{\pgfqpoint{7.739585in}{1.135373in}}{\pgfqpoint{7.741589in}{1.130535in}}{\pgfqpoint{7.745155in}{1.126969in}}%
\pgfpathcurveto{\pgfqpoint{7.748721in}{1.123402in}}{\pgfqpoint{7.753559in}{1.121398in}}{\pgfqpoint{7.758603in}{1.121398in}}%
\pgfpathclose%
\pgfusepath{fill}%
\end{pgfscope}%
\begin{pgfscope}%
\pgfpathrectangle{\pgfqpoint{6.572727in}{0.474100in}}{\pgfqpoint{4.227273in}{3.318700in}}%
\pgfusepath{clip}%
\pgfsetbuttcap%
\pgfsetroundjoin%
\definecolor{currentfill}{rgb}{0.993248,0.906157,0.143936}%
\pgfsetfillcolor{currentfill}%
\pgfsetfillopacity{0.700000}%
\pgfsetlinewidth{0.000000pt}%
\definecolor{currentstroke}{rgb}{0.000000,0.000000,0.000000}%
\pgfsetstrokecolor{currentstroke}%
\pgfsetstrokeopacity{0.700000}%
\pgfsetdash{}{0pt}%
\pgfpathmoveto{\pgfqpoint{9.794202in}{1.524801in}}%
\pgfpathcurveto{\pgfqpoint{9.799246in}{1.524801in}}{\pgfqpoint{9.804084in}{1.526804in}}{\pgfqpoint{9.807650in}{1.530371in}}%
\pgfpathcurveto{\pgfqpoint{9.811217in}{1.533937in}}{\pgfqpoint{9.813221in}{1.538775in}}{\pgfqpoint{9.813221in}{1.543819in}}%
\pgfpathcurveto{\pgfqpoint{9.813221in}{1.548862in}}{\pgfqpoint{9.811217in}{1.553700in}}{\pgfqpoint{9.807650in}{1.557267in}}%
\pgfpathcurveto{\pgfqpoint{9.804084in}{1.560833in}}{\pgfqpoint{9.799246in}{1.562837in}}{\pgfqpoint{9.794202in}{1.562837in}}%
\pgfpathcurveto{\pgfqpoint{9.789159in}{1.562837in}}{\pgfqpoint{9.784321in}{1.560833in}}{\pgfqpoint{9.780755in}{1.557267in}}%
\pgfpathcurveto{\pgfqpoint{9.777188in}{1.553700in}}{\pgfqpoint{9.775184in}{1.548862in}}{\pgfqpoint{9.775184in}{1.543819in}}%
\pgfpathcurveto{\pgfqpoint{9.775184in}{1.538775in}}{\pgfqpoint{9.777188in}{1.533937in}}{\pgfqpoint{9.780755in}{1.530371in}}%
\pgfpathcurveto{\pgfqpoint{9.784321in}{1.526804in}}{\pgfqpoint{9.789159in}{1.524801in}}{\pgfqpoint{9.794202in}{1.524801in}}%
\pgfpathclose%
\pgfusepath{fill}%
\end{pgfscope}%
\begin{pgfscope}%
\pgfpathrectangle{\pgfqpoint{6.572727in}{0.474100in}}{\pgfqpoint{4.227273in}{3.318700in}}%
\pgfusepath{clip}%
\pgfsetbuttcap%
\pgfsetroundjoin%
\definecolor{currentfill}{rgb}{0.127568,0.566949,0.550556}%
\pgfsetfillcolor{currentfill}%
\pgfsetfillopacity{0.700000}%
\pgfsetlinewidth{0.000000pt}%
\definecolor{currentstroke}{rgb}{0.000000,0.000000,0.000000}%
\pgfsetstrokecolor{currentstroke}%
\pgfsetstrokeopacity{0.700000}%
\pgfsetdash{}{0pt}%
\pgfpathmoveto{\pgfqpoint{7.971867in}{1.517478in}}%
\pgfpathcurveto{\pgfqpoint{7.976911in}{1.517478in}}{\pgfqpoint{7.981749in}{1.519482in}}{\pgfqpoint{7.985315in}{1.523048in}}%
\pgfpathcurveto{\pgfqpoint{7.988881in}{1.526614in}}{\pgfqpoint{7.990885in}{1.531452in}}{\pgfqpoint{7.990885in}{1.536496in}}%
\pgfpathcurveto{\pgfqpoint{7.990885in}{1.541540in}}{\pgfqpoint{7.988881in}{1.546377in}}{\pgfqpoint{7.985315in}{1.549944in}}%
\pgfpathcurveto{\pgfqpoint{7.981749in}{1.553510in}}{\pgfqpoint{7.976911in}{1.555514in}}{\pgfqpoint{7.971867in}{1.555514in}}%
\pgfpathcurveto{\pgfqpoint{7.966824in}{1.555514in}}{\pgfqpoint{7.961986in}{1.553510in}}{\pgfqpoint{7.958419in}{1.549944in}}%
\pgfpathcurveto{\pgfqpoint{7.954853in}{1.546377in}}{\pgfqpoint{7.952849in}{1.541540in}}{\pgfqpoint{7.952849in}{1.536496in}}%
\pgfpathcurveto{\pgfqpoint{7.952849in}{1.531452in}}{\pgfqpoint{7.954853in}{1.526614in}}{\pgfqpoint{7.958419in}{1.523048in}}%
\pgfpathcurveto{\pgfqpoint{7.961986in}{1.519482in}}{\pgfqpoint{7.966824in}{1.517478in}}{\pgfqpoint{7.971867in}{1.517478in}}%
\pgfpathclose%
\pgfusepath{fill}%
\end{pgfscope}%
\begin{pgfscope}%
\pgfpathrectangle{\pgfqpoint{6.572727in}{0.474100in}}{\pgfqpoint{4.227273in}{3.318700in}}%
\pgfusepath{clip}%
\pgfsetbuttcap%
\pgfsetroundjoin%
\definecolor{currentfill}{rgb}{0.993248,0.906157,0.143936}%
\pgfsetfillcolor{currentfill}%
\pgfsetfillopacity{0.700000}%
\pgfsetlinewidth{0.000000pt}%
\definecolor{currentstroke}{rgb}{0.000000,0.000000,0.000000}%
\pgfsetstrokecolor{currentstroke}%
\pgfsetstrokeopacity{0.700000}%
\pgfsetdash{}{0pt}%
\pgfpathmoveto{\pgfqpoint{9.284878in}{2.442719in}}%
\pgfpathcurveto{\pgfqpoint{9.289922in}{2.442719in}}{\pgfqpoint{9.294759in}{2.444723in}}{\pgfqpoint{9.298326in}{2.448289in}}%
\pgfpathcurveto{\pgfqpoint{9.301892in}{2.451856in}}{\pgfqpoint{9.303896in}{2.456693in}}{\pgfqpoint{9.303896in}{2.461737in}}%
\pgfpathcurveto{\pgfqpoint{9.303896in}{2.466781in}}{\pgfqpoint{9.301892in}{2.471618in}}{\pgfqpoint{9.298326in}{2.475185in}}%
\pgfpathcurveto{\pgfqpoint{9.294759in}{2.478751in}}{\pgfqpoint{9.289922in}{2.480755in}}{\pgfqpoint{9.284878in}{2.480755in}}%
\pgfpathcurveto{\pgfqpoint{9.279834in}{2.480755in}}{\pgfqpoint{9.274997in}{2.478751in}}{\pgfqpoint{9.271430in}{2.475185in}}%
\pgfpathcurveto{\pgfqpoint{9.267864in}{2.471618in}}{\pgfqpoint{9.265860in}{2.466781in}}{\pgfqpoint{9.265860in}{2.461737in}}%
\pgfpathcurveto{\pgfqpoint{9.265860in}{2.456693in}}{\pgfqpoint{9.267864in}{2.451856in}}{\pgfqpoint{9.271430in}{2.448289in}}%
\pgfpathcurveto{\pgfqpoint{9.274997in}{2.444723in}}{\pgfqpoint{9.279834in}{2.442719in}}{\pgfqpoint{9.284878in}{2.442719in}}%
\pgfpathclose%
\pgfusepath{fill}%
\end{pgfscope}%
\begin{pgfscope}%
\pgfpathrectangle{\pgfqpoint{6.572727in}{0.474100in}}{\pgfqpoint{4.227273in}{3.318700in}}%
\pgfusepath{clip}%
\pgfsetbuttcap%
\pgfsetroundjoin%
\definecolor{currentfill}{rgb}{0.127568,0.566949,0.550556}%
\pgfsetfillcolor{currentfill}%
\pgfsetfillopacity{0.700000}%
\pgfsetlinewidth{0.000000pt}%
\definecolor{currentstroke}{rgb}{0.000000,0.000000,0.000000}%
\pgfsetstrokecolor{currentstroke}%
\pgfsetstrokeopacity{0.700000}%
\pgfsetdash{}{0pt}%
\pgfpathmoveto{\pgfqpoint{7.469351in}{1.666544in}}%
\pgfpathcurveto{\pgfqpoint{7.474395in}{1.666544in}}{\pgfqpoint{7.479233in}{1.668548in}}{\pgfqpoint{7.482799in}{1.672114in}}%
\pgfpathcurveto{\pgfqpoint{7.486366in}{1.675681in}}{\pgfqpoint{7.488370in}{1.680518in}}{\pgfqpoint{7.488370in}{1.685562in}}%
\pgfpathcurveto{\pgfqpoint{7.488370in}{1.690606in}}{\pgfqpoint{7.486366in}{1.695444in}}{\pgfqpoint{7.482799in}{1.699010in}}%
\pgfpathcurveto{\pgfqpoint{7.479233in}{1.702576in}}{\pgfqpoint{7.474395in}{1.704580in}}{\pgfqpoint{7.469351in}{1.704580in}}%
\pgfpathcurveto{\pgfqpoint{7.464308in}{1.704580in}}{\pgfqpoint{7.459470in}{1.702576in}}{\pgfqpoint{7.455904in}{1.699010in}}%
\pgfpathcurveto{\pgfqpoint{7.452337in}{1.695444in}}{\pgfqpoint{7.450333in}{1.690606in}}{\pgfqpoint{7.450333in}{1.685562in}}%
\pgfpathcurveto{\pgfqpoint{7.450333in}{1.680518in}}{\pgfqpoint{7.452337in}{1.675681in}}{\pgfqpoint{7.455904in}{1.672114in}}%
\pgfpathcurveto{\pgfqpoint{7.459470in}{1.668548in}}{\pgfqpoint{7.464308in}{1.666544in}}{\pgfqpoint{7.469351in}{1.666544in}}%
\pgfpathclose%
\pgfusepath{fill}%
\end{pgfscope}%
\begin{pgfscope}%
\pgfpathrectangle{\pgfqpoint{6.572727in}{0.474100in}}{\pgfqpoint{4.227273in}{3.318700in}}%
\pgfusepath{clip}%
\pgfsetbuttcap%
\pgfsetroundjoin%
\definecolor{currentfill}{rgb}{0.127568,0.566949,0.550556}%
\pgfsetfillcolor{currentfill}%
\pgfsetfillopacity{0.700000}%
\pgfsetlinewidth{0.000000pt}%
\definecolor{currentstroke}{rgb}{0.000000,0.000000,0.000000}%
\pgfsetstrokecolor{currentstroke}%
\pgfsetstrokeopacity{0.700000}%
\pgfsetdash{}{0pt}%
\pgfpathmoveto{\pgfqpoint{7.865279in}{1.582680in}}%
\pgfpathcurveto{\pgfqpoint{7.870323in}{1.582680in}}{\pgfqpoint{7.875161in}{1.584684in}}{\pgfqpoint{7.878727in}{1.588251in}}%
\pgfpathcurveto{\pgfqpoint{7.882293in}{1.591817in}}{\pgfqpoint{7.884297in}{1.596655in}}{\pgfqpoint{7.884297in}{1.601698in}}%
\pgfpathcurveto{\pgfqpoint{7.884297in}{1.606742in}}{\pgfqpoint{7.882293in}{1.611580in}}{\pgfqpoint{7.878727in}{1.615146in}}%
\pgfpathcurveto{\pgfqpoint{7.875161in}{1.618713in}}{\pgfqpoint{7.870323in}{1.620717in}}{\pgfqpoint{7.865279in}{1.620717in}}%
\pgfpathcurveto{\pgfqpoint{7.860235in}{1.620717in}}{\pgfqpoint{7.855398in}{1.618713in}}{\pgfqpoint{7.851831in}{1.615146in}}%
\pgfpathcurveto{\pgfqpoint{7.848265in}{1.611580in}}{\pgfqpoint{7.846261in}{1.606742in}}{\pgfqpoint{7.846261in}{1.601698in}}%
\pgfpathcurveto{\pgfqpoint{7.846261in}{1.596655in}}{\pgfqpoint{7.848265in}{1.591817in}}{\pgfqpoint{7.851831in}{1.588251in}}%
\pgfpathcurveto{\pgfqpoint{7.855398in}{1.584684in}}{\pgfqpoint{7.860235in}{1.582680in}}{\pgfqpoint{7.865279in}{1.582680in}}%
\pgfpathclose%
\pgfusepath{fill}%
\end{pgfscope}%
\begin{pgfscope}%
\pgfpathrectangle{\pgfqpoint{6.572727in}{0.474100in}}{\pgfqpoint{4.227273in}{3.318700in}}%
\pgfusepath{clip}%
\pgfsetbuttcap%
\pgfsetroundjoin%
\definecolor{currentfill}{rgb}{0.993248,0.906157,0.143936}%
\pgfsetfillcolor{currentfill}%
\pgfsetfillopacity{0.700000}%
\pgfsetlinewidth{0.000000pt}%
\definecolor{currentstroke}{rgb}{0.000000,0.000000,0.000000}%
\pgfsetstrokecolor{currentstroke}%
\pgfsetstrokeopacity{0.700000}%
\pgfsetdash{}{0pt}%
\pgfpathmoveto{\pgfqpoint{9.115555in}{1.376743in}}%
\pgfpathcurveto{\pgfqpoint{9.120599in}{1.376743in}}{\pgfqpoint{9.125437in}{1.378747in}}{\pgfqpoint{9.129003in}{1.382314in}}%
\pgfpathcurveto{\pgfqpoint{9.132569in}{1.385880in}}{\pgfqpoint{9.134573in}{1.390718in}}{\pgfqpoint{9.134573in}{1.395761in}}%
\pgfpathcurveto{\pgfqpoint{9.134573in}{1.400805in}}{\pgfqpoint{9.132569in}{1.405643in}}{\pgfqpoint{9.129003in}{1.409209in}}%
\pgfpathcurveto{\pgfqpoint{9.125437in}{1.412776in}}{\pgfqpoint{9.120599in}{1.414780in}}{\pgfqpoint{9.115555in}{1.414780in}}%
\pgfpathcurveto{\pgfqpoint{9.110511in}{1.414780in}}{\pgfqpoint{9.105674in}{1.412776in}}{\pgfqpoint{9.102107in}{1.409209in}}%
\pgfpathcurveto{\pgfqpoint{9.098541in}{1.405643in}}{\pgfqpoint{9.096537in}{1.400805in}}{\pgfqpoint{9.096537in}{1.395761in}}%
\pgfpathcurveto{\pgfqpoint{9.096537in}{1.390718in}}{\pgfqpoint{9.098541in}{1.385880in}}{\pgfqpoint{9.102107in}{1.382314in}}%
\pgfpathcurveto{\pgfqpoint{9.105674in}{1.378747in}}{\pgfqpoint{9.110511in}{1.376743in}}{\pgfqpoint{9.115555in}{1.376743in}}%
\pgfpathclose%
\pgfusepath{fill}%
\end{pgfscope}%
\begin{pgfscope}%
\pgfpathrectangle{\pgfqpoint{6.572727in}{0.474100in}}{\pgfqpoint{4.227273in}{3.318700in}}%
\pgfusepath{clip}%
\pgfsetbuttcap%
\pgfsetroundjoin%
\definecolor{currentfill}{rgb}{0.127568,0.566949,0.550556}%
\pgfsetfillcolor{currentfill}%
\pgfsetfillopacity{0.700000}%
\pgfsetlinewidth{0.000000pt}%
\definecolor{currentstroke}{rgb}{0.000000,0.000000,0.000000}%
\pgfsetstrokecolor{currentstroke}%
\pgfsetstrokeopacity{0.700000}%
\pgfsetdash{}{0pt}%
\pgfpathmoveto{\pgfqpoint{7.285794in}{1.542570in}}%
\pgfpathcurveto{\pgfqpoint{7.290837in}{1.542570in}}{\pgfqpoint{7.295675in}{1.544574in}}{\pgfqpoint{7.299241in}{1.548141in}}%
\pgfpathcurveto{\pgfqpoint{7.302808in}{1.551707in}}{\pgfqpoint{7.304812in}{1.556545in}}{\pgfqpoint{7.304812in}{1.561589in}}%
\pgfpathcurveto{\pgfqpoint{7.304812in}{1.566632in}}{\pgfqpoint{7.302808in}{1.571470in}}{\pgfqpoint{7.299241in}{1.575036in}}%
\pgfpathcurveto{\pgfqpoint{7.295675in}{1.578603in}}{\pgfqpoint{7.290837in}{1.580607in}}{\pgfqpoint{7.285794in}{1.580607in}}%
\pgfpathcurveto{\pgfqpoint{7.280750in}{1.580607in}}{\pgfqpoint{7.275912in}{1.578603in}}{\pgfqpoint{7.272346in}{1.575036in}}%
\pgfpathcurveto{\pgfqpoint{7.268779in}{1.571470in}}{\pgfqpoint{7.266775in}{1.566632in}}{\pgfqpoint{7.266775in}{1.561589in}}%
\pgfpathcurveto{\pgfqpoint{7.266775in}{1.556545in}}{\pgfqpoint{7.268779in}{1.551707in}}{\pgfqpoint{7.272346in}{1.548141in}}%
\pgfpathcurveto{\pgfqpoint{7.275912in}{1.544574in}}{\pgfqpoint{7.280750in}{1.542570in}}{\pgfqpoint{7.285794in}{1.542570in}}%
\pgfpathclose%
\pgfusepath{fill}%
\end{pgfscope}%
\begin{pgfscope}%
\pgfpathrectangle{\pgfqpoint{6.572727in}{0.474100in}}{\pgfqpoint{4.227273in}{3.318700in}}%
\pgfusepath{clip}%
\pgfsetbuttcap%
\pgfsetroundjoin%
\definecolor{currentfill}{rgb}{0.127568,0.566949,0.550556}%
\pgfsetfillcolor{currentfill}%
\pgfsetfillopacity{0.700000}%
\pgfsetlinewidth{0.000000pt}%
\definecolor{currentstroke}{rgb}{0.000000,0.000000,0.000000}%
\pgfsetstrokecolor{currentstroke}%
\pgfsetstrokeopacity{0.700000}%
\pgfsetdash{}{0pt}%
\pgfpathmoveto{\pgfqpoint{7.825966in}{3.077178in}}%
\pgfpathcurveto{\pgfqpoint{7.831010in}{3.077178in}}{\pgfqpoint{7.835848in}{3.079181in}}{\pgfqpoint{7.839414in}{3.082748in}}%
\pgfpathcurveto{\pgfqpoint{7.842981in}{3.086314in}}{\pgfqpoint{7.844985in}{3.091152in}}{\pgfqpoint{7.844985in}{3.096196in}}%
\pgfpathcurveto{\pgfqpoint{7.844985in}{3.101239in}}{\pgfqpoint{7.842981in}{3.106077in}}{\pgfqpoint{7.839414in}{3.109644in}}%
\pgfpathcurveto{\pgfqpoint{7.835848in}{3.113210in}}{\pgfqpoint{7.831010in}{3.115214in}}{\pgfqpoint{7.825966in}{3.115214in}}%
\pgfpathcurveto{\pgfqpoint{7.820923in}{3.115214in}}{\pgfqpoint{7.816085in}{3.113210in}}{\pgfqpoint{7.812519in}{3.109644in}}%
\pgfpathcurveto{\pgfqpoint{7.808952in}{3.106077in}}{\pgfqpoint{7.806948in}{3.101239in}}{\pgfqpoint{7.806948in}{3.096196in}}%
\pgfpathcurveto{\pgfqpoint{7.806948in}{3.091152in}}{\pgfqpoint{7.808952in}{3.086314in}}{\pgfqpoint{7.812519in}{3.082748in}}%
\pgfpathcurveto{\pgfqpoint{7.816085in}{3.079181in}}{\pgfqpoint{7.820923in}{3.077178in}}{\pgfqpoint{7.825966in}{3.077178in}}%
\pgfpathclose%
\pgfusepath{fill}%
\end{pgfscope}%
\begin{pgfscope}%
\pgfpathrectangle{\pgfqpoint{6.572727in}{0.474100in}}{\pgfqpoint{4.227273in}{3.318700in}}%
\pgfusepath{clip}%
\pgfsetbuttcap%
\pgfsetroundjoin%
\definecolor{currentfill}{rgb}{0.993248,0.906157,0.143936}%
\pgfsetfillcolor{currentfill}%
\pgfsetfillopacity{0.700000}%
\pgfsetlinewidth{0.000000pt}%
\definecolor{currentstroke}{rgb}{0.000000,0.000000,0.000000}%
\pgfsetstrokecolor{currentstroke}%
\pgfsetstrokeopacity{0.700000}%
\pgfsetdash{}{0pt}%
\pgfpathmoveto{\pgfqpoint{9.959873in}{1.537325in}}%
\pgfpathcurveto{\pgfqpoint{9.964917in}{1.537325in}}{\pgfqpoint{9.969755in}{1.539329in}}{\pgfqpoint{9.973321in}{1.542895in}}%
\pgfpathcurveto{\pgfqpoint{9.976888in}{1.546462in}}{\pgfqpoint{9.978891in}{1.551300in}}{\pgfqpoint{9.978891in}{1.556343in}}%
\pgfpathcurveto{\pgfqpoint{9.978891in}{1.561387in}}{\pgfqpoint{9.976888in}{1.566225in}}{\pgfqpoint{9.973321in}{1.569791in}}%
\pgfpathcurveto{\pgfqpoint{9.969755in}{1.573358in}}{\pgfqpoint{9.964917in}{1.575361in}}{\pgfqpoint{9.959873in}{1.575361in}}%
\pgfpathcurveto{\pgfqpoint{9.954830in}{1.575361in}}{\pgfqpoint{9.949992in}{1.573358in}}{\pgfqpoint{9.946425in}{1.569791in}}%
\pgfpathcurveto{\pgfqpoint{9.942859in}{1.566225in}}{\pgfqpoint{9.940855in}{1.561387in}}{\pgfqpoint{9.940855in}{1.556343in}}%
\pgfpathcurveto{\pgfqpoint{9.940855in}{1.551300in}}{\pgfqpoint{9.942859in}{1.546462in}}{\pgfqpoint{9.946425in}{1.542895in}}%
\pgfpathcurveto{\pgfqpoint{9.949992in}{1.539329in}}{\pgfqpoint{9.954830in}{1.537325in}}{\pgfqpoint{9.959873in}{1.537325in}}%
\pgfpathclose%
\pgfusepath{fill}%
\end{pgfscope}%
\begin{pgfscope}%
\pgfpathrectangle{\pgfqpoint{6.572727in}{0.474100in}}{\pgfqpoint{4.227273in}{3.318700in}}%
\pgfusepath{clip}%
\pgfsetbuttcap%
\pgfsetroundjoin%
\definecolor{currentfill}{rgb}{0.127568,0.566949,0.550556}%
\pgfsetfillcolor{currentfill}%
\pgfsetfillopacity{0.700000}%
\pgfsetlinewidth{0.000000pt}%
\definecolor{currentstroke}{rgb}{0.000000,0.000000,0.000000}%
\pgfsetstrokecolor{currentstroke}%
\pgfsetstrokeopacity{0.700000}%
\pgfsetdash{}{0pt}%
\pgfpathmoveto{\pgfqpoint{7.498734in}{1.647367in}}%
\pgfpathcurveto{\pgfqpoint{7.503777in}{1.647367in}}{\pgfqpoint{7.508615in}{1.649371in}}{\pgfqpoint{7.512182in}{1.652937in}}%
\pgfpathcurveto{\pgfqpoint{7.515748in}{1.656503in}}{\pgfqpoint{7.517752in}{1.661341in}}{\pgfqpoint{7.517752in}{1.666385in}}%
\pgfpathcurveto{\pgfqpoint{7.517752in}{1.671429in}}{\pgfqpoint{7.515748in}{1.676266in}}{\pgfqpoint{7.512182in}{1.679833in}}%
\pgfpathcurveto{\pgfqpoint{7.508615in}{1.683399in}}{\pgfqpoint{7.503777in}{1.685403in}}{\pgfqpoint{7.498734in}{1.685403in}}%
\pgfpathcurveto{\pgfqpoint{7.493690in}{1.685403in}}{\pgfqpoint{7.488852in}{1.683399in}}{\pgfqpoint{7.485286in}{1.679833in}}%
\pgfpathcurveto{\pgfqpoint{7.481719in}{1.676266in}}{\pgfqpoint{7.479716in}{1.671429in}}{\pgfqpoint{7.479716in}{1.666385in}}%
\pgfpathcurveto{\pgfqpoint{7.479716in}{1.661341in}}{\pgfqpoint{7.481719in}{1.656503in}}{\pgfqpoint{7.485286in}{1.652937in}}%
\pgfpathcurveto{\pgfqpoint{7.488852in}{1.649371in}}{\pgfqpoint{7.493690in}{1.647367in}}{\pgfqpoint{7.498734in}{1.647367in}}%
\pgfpathclose%
\pgfusepath{fill}%
\end{pgfscope}%
\begin{pgfscope}%
\pgfpathrectangle{\pgfqpoint{6.572727in}{0.474100in}}{\pgfqpoint{4.227273in}{3.318700in}}%
\pgfusepath{clip}%
\pgfsetbuttcap%
\pgfsetroundjoin%
\definecolor{currentfill}{rgb}{0.127568,0.566949,0.550556}%
\pgfsetfillcolor{currentfill}%
\pgfsetfillopacity{0.700000}%
\pgfsetlinewidth{0.000000pt}%
\definecolor{currentstroke}{rgb}{0.000000,0.000000,0.000000}%
\pgfsetstrokecolor{currentstroke}%
\pgfsetstrokeopacity{0.700000}%
\pgfsetdash{}{0pt}%
\pgfpathmoveto{\pgfqpoint{8.024086in}{3.334474in}}%
\pgfpathcurveto{\pgfqpoint{8.029130in}{3.334474in}}{\pgfqpoint{8.033967in}{3.336478in}}{\pgfqpoint{8.037534in}{3.340045in}}%
\pgfpathcurveto{\pgfqpoint{8.041100in}{3.343611in}}{\pgfqpoint{8.043104in}{3.348449in}}{\pgfqpoint{8.043104in}{3.353492in}}%
\pgfpathcurveto{\pgfqpoint{8.043104in}{3.358536in}}{\pgfqpoint{8.041100in}{3.363374in}}{\pgfqpoint{8.037534in}{3.366940in}}%
\pgfpathcurveto{\pgfqpoint{8.033967in}{3.370507in}}{\pgfqpoint{8.029130in}{3.372511in}}{\pgfqpoint{8.024086in}{3.372511in}}%
\pgfpathcurveto{\pgfqpoint{8.019042in}{3.372511in}}{\pgfqpoint{8.014205in}{3.370507in}}{\pgfqpoint{8.010638in}{3.366940in}}%
\pgfpathcurveto{\pgfqpoint{8.007072in}{3.363374in}}{\pgfqpoint{8.005068in}{3.358536in}}{\pgfqpoint{8.005068in}{3.353492in}}%
\pgfpathcurveto{\pgfqpoint{8.005068in}{3.348449in}}{\pgfqpoint{8.007072in}{3.343611in}}{\pgfqpoint{8.010638in}{3.340045in}}%
\pgfpathcurveto{\pgfqpoint{8.014205in}{3.336478in}}{\pgfqpoint{8.019042in}{3.334474in}}{\pgfqpoint{8.024086in}{3.334474in}}%
\pgfpathclose%
\pgfusepath{fill}%
\end{pgfscope}%
\begin{pgfscope}%
\pgfpathrectangle{\pgfqpoint{6.572727in}{0.474100in}}{\pgfqpoint{4.227273in}{3.318700in}}%
\pgfusepath{clip}%
\pgfsetbuttcap%
\pgfsetroundjoin%
\definecolor{currentfill}{rgb}{0.127568,0.566949,0.550556}%
\pgfsetfillcolor{currentfill}%
\pgfsetfillopacity{0.700000}%
\pgfsetlinewidth{0.000000pt}%
\definecolor{currentstroke}{rgb}{0.000000,0.000000,0.000000}%
\pgfsetstrokecolor{currentstroke}%
\pgfsetstrokeopacity{0.700000}%
\pgfsetdash{}{0pt}%
\pgfpathmoveto{\pgfqpoint{8.437606in}{1.424616in}}%
\pgfpathcurveto{\pgfqpoint{8.442650in}{1.424616in}}{\pgfqpoint{8.447487in}{1.426620in}}{\pgfqpoint{8.451054in}{1.430186in}}%
\pgfpathcurveto{\pgfqpoint{8.454620in}{1.433753in}}{\pgfqpoint{8.456624in}{1.438590in}}{\pgfqpoint{8.456624in}{1.443634in}}%
\pgfpathcurveto{\pgfqpoint{8.456624in}{1.448678in}}{\pgfqpoint{8.454620in}{1.453516in}}{\pgfqpoint{8.451054in}{1.457082in}}%
\pgfpathcurveto{\pgfqpoint{8.447487in}{1.460648in}}{\pgfqpoint{8.442650in}{1.462652in}}{\pgfqpoint{8.437606in}{1.462652in}}%
\pgfpathcurveto{\pgfqpoint{8.432562in}{1.462652in}}{\pgfqpoint{8.427724in}{1.460648in}}{\pgfqpoint{8.424158in}{1.457082in}}%
\pgfpathcurveto{\pgfqpoint{8.420592in}{1.453516in}}{\pgfqpoint{8.418588in}{1.448678in}}{\pgfqpoint{8.418588in}{1.443634in}}%
\pgfpathcurveto{\pgfqpoint{8.418588in}{1.438590in}}{\pgfqpoint{8.420592in}{1.433753in}}{\pgfqpoint{8.424158in}{1.430186in}}%
\pgfpathcurveto{\pgfqpoint{8.427724in}{1.426620in}}{\pgfqpoint{8.432562in}{1.424616in}}{\pgfqpoint{8.437606in}{1.424616in}}%
\pgfpathclose%
\pgfusepath{fill}%
\end{pgfscope}%
\begin{pgfscope}%
\pgfpathrectangle{\pgfqpoint{6.572727in}{0.474100in}}{\pgfqpoint{4.227273in}{3.318700in}}%
\pgfusepath{clip}%
\pgfsetbuttcap%
\pgfsetroundjoin%
\definecolor{currentfill}{rgb}{0.127568,0.566949,0.550556}%
\pgfsetfillcolor{currentfill}%
\pgfsetfillopacity{0.700000}%
\pgfsetlinewidth{0.000000pt}%
\definecolor{currentstroke}{rgb}{0.000000,0.000000,0.000000}%
\pgfsetstrokecolor{currentstroke}%
\pgfsetstrokeopacity{0.700000}%
\pgfsetdash{}{0pt}%
\pgfpathmoveto{\pgfqpoint{7.658236in}{1.289985in}}%
\pgfpathcurveto{\pgfqpoint{7.663280in}{1.289985in}}{\pgfqpoint{7.668117in}{1.291989in}}{\pgfqpoint{7.671684in}{1.295555in}}%
\pgfpathcurveto{\pgfqpoint{7.675250in}{1.299122in}}{\pgfqpoint{7.677254in}{1.303960in}}{\pgfqpoint{7.677254in}{1.309003in}}%
\pgfpathcurveto{\pgfqpoint{7.677254in}{1.314047in}}{\pgfqpoint{7.675250in}{1.318885in}}{\pgfqpoint{7.671684in}{1.322451in}}%
\pgfpathcurveto{\pgfqpoint{7.668117in}{1.326018in}}{\pgfqpoint{7.663280in}{1.328021in}}{\pgfqpoint{7.658236in}{1.328021in}}%
\pgfpathcurveto{\pgfqpoint{7.653192in}{1.328021in}}{\pgfqpoint{7.648354in}{1.326018in}}{\pgfqpoint{7.644788in}{1.322451in}}%
\pgfpathcurveto{\pgfqpoint{7.641222in}{1.318885in}}{\pgfqpoint{7.639218in}{1.314047in}}{\pgfqpoint{7.639218in}{1.309003in}}%
\pgfpathcurveto{\pgfqpoint{7.639218in}{1.303960in}}{\pgfqpoint{7.641222in}{1.299122in}}{\pgfqpoint{7.644788in}{1.295555in}}%
\pgfpathcurveto{\pgfqpoint{7.648354in}{1.291989in}}{\pgfqpoint{7.653192in}{1.289985in}}{\pgfqpoint{7.658236in}{1.289985in}}%
\pgfpathclose%
\pgfusepath{fill}%
\end{pgfscope}%
\begin{pgfscope}%
\pgfpathrectangle{\pgfqpoint{6.572727in}{0.474100in}}{\pgfqpoint{4.227273in}{3.318700in}}%
\pgfusepath{clip}%
\pgfsetbuttcap%
\pgfsetroundjoin%
\definecolor{currentfill}{rgb}{0.127568,0.566949,0.550556}%
\pgfsetfillcolor{currentfill}%
\pgfsetfillopacity{0.700000}%
\pgfsetlinewidth{0.000000pt}%
\definecolor{currentstroke}{rgb}{0.000000,0.000000,0.000000}%
\pgfsetstrokecolor{currentstroke}%
\pgfsetstrokeopacity{0.700000}%
\pgfsetdash{}{0pt}%
\pgfpathmoveto{\pgfqpoint{7.115113in}{1.658262in}}%
\pgfpathcurveto{\pgfqpoint{7.120156in}{1.658262in}}{\pgfqpoint{7.124994in}{1.660266in}}{\pgfqpoint{7.128560in}{1.663832in}}%
\pgfpathcurveto{\pgfqpoint{7.132127in}{1.667399in}}{\pgfqpoint{7.134131in}{1.672237in}}{\pgfqpoint{7.134131in}{1.677280in}}%
\pgfpathcurveto{\pgfqpoint{7.134131in}{1.682324in}}{\pgfqpoint{7.132127in}{1.687162in}}{\pgfqpoint{7.128560in}{1.690728in}}%
\pgfpathcurveto{\pgfqpoint{7.124994in}{1.694295in}}{\pgfqpoint{7.120156in}{1.696298in}}{\pgfqpoint{7.115113in}{1.696298in}}%
\pgfpathcurveto{\pgfqpoint{7.110069in}{1.696298in}}{\pgfqpoint{7.105231in}{1.694295in}}{\pgfqpoint{7.101665in}{1.690728in}}%
\pgfpathcurveto{\pgfqpoint{7.098098in}{1.687162in}}{\pgfqpoint{7.096094in}{1.682324in}}{\pgfqpoint{7.096094in}{1.677280in}}%
\pgfpathcurveto{\pgfqpoint{7.096094in}{1.672237in}}{\pgfqpoint{7.098098in}{1.667399in}}{\pgfqpoint{7.101665in}{1.663832in}}%
\pgfpathcurveto{\pgfqpoint{7.105231in}{1.660266in}}{\pgfqpoint{7.110069in}{1.658262in}}{\pgfqpoint{7.115113in}{1.658262in}}%
\pgfpathclose%
\pgfusepath{fill}%
\end{pgfscope}%
\begin{pgfscope}%
\pgfpathrectangle{\pgfqpoint{6.572727in}{0.474100in}}{\pgfqpoint{4.227273in}{3.318700in}}%
\pgfusepath{clip}%
\pgfsetbuttcap%
\pgfsetroundjoin%
\definecolor{currentfill}{rgb}{0.127568,0.566949,0.550556}%
\pgfsetfillcolor{currentfill}%
\pgfsetfillopacity{0.700000}%
\pgfsetlinewidth{0.000000pt}%
\definecolor{currentstroke}{rgb}{0.000000,0.000000,0.000000}%
\pgfsetstrokecolor{currentstroke}%
\pgfsetstrokeopacity{0.700000}%
\pgfsetdash{}{0pt}%
\pgfpathmoveto{\pgfqpoint{8.668257in}{3.175987in}}%
\pgfpathcurveto{\pgfqpoint{8.673300in}{3.175987in}}{\pgfqpoint{8.678138in}{3.177991in}}{\pgfqpoint{8.681705in}{3.181558in}}%
\pgfpathcurveto{\pgfqpoint{8.685271in}{3.185124in}}{\pgfqpoint{8.687275in}{3.189962in}}{\pgfqpoint{8.687275in}{3.195005in}}%
\pgfpathcurveto{\pgfqpoint{8.687275in}{3.200049in}}{\pgfqpoint{8.685271in}{3.204887in}}{\pgfqpoint{8.681705in}{3.208453in}}%
\pgfpathcurveto{\pgfqpoint{8.678138in}{3.212020in}}{\pgfqpoint{8.673300in}{3.214024in}}{\pgfqpoint{8.668257in}{3.214024in}}%
\pgfpathcurveto{\pgfqpoint{8.663213in}{3.214024in}}{\pgfqpoint{8.658375in}{3.212020in}}{\pgfqpoint{8.654809in}{3.208453in}}%
\pgfpathcurveto{\pgfqpoint{8.651243in}{3.204887in}}{\pgfqpoint{8.649239in}{3.200049in}}{\pgfqpoint{8.649239in}{3.195005in}}%
\pgfpathcurveto{\pgfqpoint{8.649239in}{3.189962in}}{\pgfqpoint{8.651243in}{3.185124in}}{\pgfqpoint{8.654809in}{3.181558in}}%
\pgfpathcurveto{\pgfqpoint{8.658375in}{3.177991in}}{\pgfqpoint{8.663213in}{3.175987in}}{\pgfqpoint{8.668257in}{3.175987in}}%
\pgfpathclose%
\pgfusepath{fill}%
\end{pgfscope}%
\begin{pgfscope}%
\pgfpathrectangle{\pgfqpoint{6.572727in}{0.474100in}}{\pgfqpoint{4.227273in}{3.318700in}}%
\pgfusepath{clip}%
\pgfsetbuttcap%
\pgfsetroundjoin%
\definecolor{currentfill}{rgb}{0.993248,0.906157,0.143936}%
\pgfsetfillcolor{currentfill}%
\pgfsetfillopacity{0.700000}%
\pgfsetlinewidth{0.000000pt}%
\definecolor{currentstroke}{rgb}{0.000000,0.000000,0.000000}%
\pgfsetstrokecolor{currentstroke}%
\pgfsetstrokeopacity{0.700000}%
\pgfsetdash{}{0pt}%
\pgfpathmoveto{\pgfqpoint{9.451032in}{1.267230in}}%
\pgfpathcurveto{\pgfqpoint{9.456076in}{1.267230in}}{\pgfqpoint{9.460914in}{1.269234in}}{\pgfqpoint{9.464480in}{1.272801in}}%
\pgfpathcurveto{\pgfqpoint{9.468047in}{1.276367in}}{\pgfqpoint{9.470050in}{1.281205in}}{\pgfqpoint{9.470050in}{1.286249in}}%
\pgfpathcurveto{\pgfqpoint{9.470050in}{1.291292in}}{\pgfqpoint{9.468047in}{1.296130in}}{\pgfqpoint{9.464480in}{1.299696in}}%
\pgfpathcurveto{\pgfqpoint{9.460914in}{1.303263in}}{\pgfqpoint{9.456076in}{1.305267in}}{\pgfqpoint{9.451032in}{1.305267in}}%
\pgfpathcurveto{\pgfqpoint{9.445989in}{1.305267in}}{\pgfqpoint{9.441151in}{1.303263in}}{\pgfqpoint{9.437584in}{1.299696in}}%
\pgfpathcurveto{\pgfqpoint{9.434018in}{1.296130in}}{\pgfqpoint{9.432014in}{1.291292in}}{\pgfqpoint{9.432014in}{1.286249in}}%
\pgfpathcurveto{\pgfqpoint{9.432014in}{1.281205in}}{\pgfqpoint{9.434018in}{1.276367in}}{\pgfqpoint{9.437584in}{1.272801in}}%
\pgfpathcurveto{\pgfqpoint{9.441151in}{1.269234in}}{\pgfqpoint{9.445989in}{1.267230in}}{\pgfqpoint{9.451032in}{1.267230in}}%
\pgfpathclose%
\pgfusepath{fill}%
\end{pgfscope}%
\begin{pgfscope}%
\pgfpathrectangle{\pgfqpoint{6.572727in}{0.474100in}}{\pgfqpoint{4.227273in}{3.318700in}}%
\pgfusepath{clip}%
\pgfsetbuttcap%
\pgfsetroundjoin%
\definecolor{currentfill}{rgb}{0.127568,0.566949,0.550556}%
\pgfsetfillcolor{currentfill}%
\pgfsetfillopacity{0.700000}%
\pgfsetlinewidth{0.000000pt}%
\definecolor{currentstroke}{rgb}{0.000000,0.000000,0.000000}%
\pgfsetstrokecolor{currentstroke}%
\pgfsetstrokeopacity{0.700000}%
\pgfsetdash{}{0pt}%
\pgfpathmoveto{\pgfqpoint{7.725862in}{1.814469in}}%
\pgfpathcurveto{\pgfqpoint{7.730905in}{1.814469in}}{\pgfqpoint{7.735743in}{1.816473in}}{\pgfqpoint{7.739310in}{1.820040in}}%
\pgfpathcurveto{\pgfqpoint{7.742876in}{1.823606in}}{\pgfqpoint{7.744880in}{1.828444in}}{\pgfqpoint{7.744880in}{1.833488in}}%
\pgfpathcurveto{\pgfqpoint{7.744880in}{1.838531in}}{\pgfqpoint{7.742876in}{1.843369in}}{\pgfqpoint{7.739310in}{1.846935in}}%
\pgfpathcurveto{\pgfqpoint{7.735743in}{1.850502in}}{\pgfqpoint{7.730905in}{1.852506in}}{\pgfqpoint{7.725862in}{1.852506in}}%
\pgfpathcurveto{\pgfqpoint{7.720818in}{1.852506in}}{\pgfqpoint{7.715980in}{1.850502in}}{\pgfqpoint{7.712414in}{1.846935in}}%
\pgfpathcurveto{\pgfqpoint{7.708847in}{1.843369in}}{\pgfqpoint{7.706844in}{1.838531in}}{\pgfqpoint{7.706844in}{1.833488in}}%
\pgfpathcurveto{\pgfqpoint{7.706844in}{1.828444in}}{\pgfqpoint{7.708847in}{1.823606in}}{\pgfqpoint{7.712414in}{1.820040in}}%
\pgfpathcurveto{\pgfqpoint{7.715980in}{1.816473in}}{\pgfqpoint{7.720818in}{1.814469in}}{\pgfqpoint{7.725862in}{1.814469in}}%
\pgfpathclose%
\pgfusepath{fill}%
\end{pgfscope}%
\begin{pgfscope}%
\pgfpathrectangle{\pgfqpoint{6.572727in}{0.474100in}}{\pgfqpoint{4.227273in}{3.318700in}}%
\pgfusepath{clip}%
\pgfsetbuttcap%
\pgfsetroundjoin%
\definecolor{currentfill}{rgb}{0.993248,0.906157,0.143936}%
\pgfsetfillcolor{currentfill}%
\pgfsetfillopacity{0.700000}%
\pgfsetlinewidth{0.000000pt}%
\definecolor{currentstroke}{rgb}{0.000000,0.000000,0.000000}%
\pgfsetstrokecolor{currentstroke}%
\pgfsetstrokeopacity{0.700000}%
\pgfsetdash{}{0pt}%
\pgfpathmoveto{\pgfqpoint{9.831010in}{1.605514in}}%
\pgfpathcurveto{\pgfqpoint{9.836054in}{1.605514in}}{\pgfqpoint{9.840892in}{1.607518in}}{\pgfqpoint{9.844458in}{1.611085in}}%
\pgfpathcurveto{\pgfqpoint{9.848025in}{1.614651in}}{\pgfqpoint{9.850028in}{1.619489in}}{\pgfqpoint{9.850028in}{1.624532in}}%
\pgfpathcurveto{\pgfqpoint{9.850028in}{1.629576in}}{\pgfqpoint{9.848025in}{1.634414in}}{\pgfqpoint{9.844458in}{1.637980in}}%
\pgfpathcurveto{\pgfqpoint{9.840892in}{1.641547in}}{\pgfqpoint{9.836054in}{1.643551in}}{\pgfqpoint{9.831010in}{1.643551in}}%
\pgfpathcurveto{\pgfqpoint{9.825967in}{1.643551in}}{\pgfqpoint{9.821129in}{1.641547in}}{\pgfqpoint{9.817562in}{1.637980in}}%
\pgfpathcurveto{\pgfqpoint{9.813996in}{1.634414in}}{\pgfqpoint{9.811992in}{1.629576in}}{\pgfqpoint{9.811992in}{1.624532in}}%
\pgfpathcurveto{\pgfqpoint{9.811992in}{1.619489in}}{\pgfqpoint{9.813996in}{1.614651in}}{\pgfqpoint{9.817562in}{1.611085in}}%
\pgfpathcurveto{\pgfqpoint{9.821129in}{1.607518in}}{\pgfqpoint{9.825967in}{1.605514in}}{\pgfqpoint{9.831010in}{1.605514in}}%
\pgfpathclose%
\pgfusepath{fill}%
\end{pgfscope}%
\begin{pgfscope}%
\pgfpathrectangle{\pgfqpoint{6.572727in}{0.474100in}}{\pgfqpoint{4.227273in}{3.318700in}}%
\pgfusepath{clip}%
\pgfsetbuttcap%
\pgfsetroundjoin%
\definecolor{currentfill}{rgb}{0.127568,0.566949,0.550556}%
\pgfsetfillcolor{currentfill}%
\pgfsetfillopacity{0.700000}%
\pgfsetlinewidth{0.000000pt}%
\definecolor{currentstroke}{rgb}{0.000000,0.000000,0.000000}%
\pgfsetstrokecolor{currentstroke}%
\pgfsetstrokeopacity{0.700000}%
\pgfsetdash{}{0pt}%
\pgfpathmoveto{\pgfqpoint{8.677976in}{2.753620in}}%
\pgfpathcurveto{\pgfqpoint{8.683019in}{2.753620in}}{\pgfqpoint{8.687857in}{2.755624in}}{\pgfqpoint{8.691424in}{2.759190in}}%
\pgfpathcurveto{\pgfqpoint{8.694990in}{2.762757in}}{\pgfqpoint{8.696994in}{2.767594in}}{\pgfqpoint{8.696994in}{2.772638in}}%
\pgfpathcurveto{\pgfqpoint{8.696994in}{2.777682in}}{\pgfqpoint{8.694990in}{2.782520in}}{\pgfqpoint{8.691424in}{2.786086in}}%
\pgfpathcurveto{\pgfqpoint{8.687857in}{2.789652in}}{\pgfqpoint{8.683019in}{2.791656in}}{\pgfqpoint{8.677976in}{2.791656in}}%
\pgfpathcurveto{\pgfqpoint{8.672932in}{2.791656in}}{\pgfqpoint{8.668094in}{2.789652in}}{\pgfqpoint{8.664528in}{2.786086in}}%
\pgfpathcurveto{\pgfqpoint{8.660961in}{2.782520in}}{\pgfqpoint{8.658958in}{2.777682in}}{\pgfqpoint{8.658958in}{2.772638in}}%
\pgfpathcurveto{\pgfqpoint{8.658958in}{2.767594in}}{\pgfqpoint{8.660961in}{2.762757in}}{\pgfqpoint{8.664528in}{2.759190in}}%
\pgfpathcurveto{\pgfqpoint{8.668094in}{2.755624in}}{\pgfqpoint{8.672932in}{2.753620in}}{\pgfqpoint{8.677976in}{2.753620in}}%
\pgfpathclose%
\pgfusepath{fill}%
\end{pgfscope}%
\begin{pgfscope}%
\pgfpathrectangle{\pgfqpoint{6.572727in}{0.474100in}}{\pgfqpoint{4.227273in}{3.318700in}}%
\pgfusepath{clip}%
\pgfsetbuttcap%
\pgfsetroundjoin%
\definecolor{currentfill}{rgb}{0.993248,0.906157,0.143936}%
\pgfsetfillcolor{currentfill}%
\pgfsetfillopacity{0.700000}%
\pgfsetlinewidth{0.000000pt}%
\definecolor{currentstroke}{rgb}{0.000000,0.000000,0.000000}%
\pgfsetstrokecolor{currentstroke}%
\pgfsetstrokeopacity{0.700000}%
\pgfsetdash{}{0pt}%
\pgfpathmoveto{\pgfqpoint{9.400929in}{1.170969in}}%
\pgfpathcurveto{\pgfqpoint{9.405973in}{1.170969in}}{\pgfqpoint{9.410810in}{1.172973in}}{\pgfqpoint{9.414377in}{1.176540in}}%
\pgfpathcurveto{\pgfqpoint{9.417943in}{1.180106in}}{\pgfqpoint{9.419947in}{1.184944in}}{\pgfqpoint{9.419947in}{1.189987in}}%
\pgfpathcurveto{\pgfqpoint{9.419947in}{1.195031in}}{\pgfqpoint{9.417943in}{1.199869in}}{\pgfqpoint{9.414377in}{1.203435in}}%
\pgfpathcurveto{\pgfqpoint{9.410810in}{1.207002in}}{\pgfqpoint{9.405973in}{1.209006in}}{\pgfqpoint{9.400929in}{1.209006in}}%
\pgfpathcurveto{\pgfqpoint{9.395885in}{1.209006in}}{\pgfqpoint{9.391048in}{1.207002in}}{\pgfqpoint{9.387481in}{1.203435in}}%
\pgfpathcurveto{\pgfqpoint{9.383915in}{1.199869in}}{\pgfqpoint{9.381911in}{1.195031in}}{\pgfqpoint{9.381911in}{1.189987in}}%
\pgfpathcurveto{\pgfqpoint{9.381911in}{1.184944in}}{\pgfqpoint{9.383915in}{1.180106in}}{\pgfqpoint{9.387481in}{1.176540in}}%
\pgfpathcurveto{\pgfqpoint{9.391048in}{1.172973in}}{\pgfqpoint{9.395885in}{1.170969in}}{\pgfqpoint{9.400929in}{1.170969in}}%
\pgfpathclose%
\pgfusepath{fill}%
\end{pgfscope}%
\begin{pgfscope}%
\pgfpathrectangle{\pgfqpoint{6.572727in}{0.474100in}}{\pgfqpoint{4.227273in}{3.318700in}}%
\pgfusepath{clip}%
\pgfsetbuttcap%
\pgfsetroundjoin%
\definecolor{currentfill}{rgb}{0.127568,0.566949,0.550556}%
\pgfsetfillcolor{currentfill}%
\pgfsetfillopacity{0.700000}%
\pgfsetlinewidth{0.000000pt}%
\definecolor{currentstroke}{rgb}{0.000000,0.000000,0.000000}%
\pgfsetstrokecolor{currentstroke}%
\pgfsetstrokeopacity{0.700000}%
\pgfsetdash{}{0pt}%
\pgfpathmoveto{\pgfqpoint{8.138448in}{1.632185in}}%
\pgfpathcurveto{\pgfqpoint{8.143492in}{1.632185in}}{\pgfqpoint{8.148329in}{1.634189in}}{\pgfqpoint{8.151896in}{1.637756in}}%
\pgfpathcurveto{\pgfqpoint{8.155462in}{1.641322in}}{\pgfqpoint{8.157466in}{1.646160in}}{\pgfqpoint{8.157466in}{1.651204in}}%
\pgfpathcurveto{\pgfqpoint{8.157466in}{1.656247in}}{\pgfqpoint{8.155462in}{1.661085in}}{\pgfqpoint{8.151896in}{1.664651in}}%
\pgfpathcurveto{\pgfqpoint{8.148329in}{1.668218in}}{\pgfqpoint{8.143492in}{1.670222in}}{\pgfqpoint{8.138448in}{1.670222in}}%
\pgfpathcurveto{\pgfqpoint{8.133404in}{1.670222in}}{\pgfqpoint{8.128567in}{1.668218in}}{\pgfqpoint{8.125000in}{1.664651in}}%
\pgfpathcurveto{\pgfqpoint{8.121434in}{1.661085in}}{\pgfqpoint{8.119430in}{1.656247in}}{\pgfqpoint{8.119430in}{1.651204in}}%
\pgfpathcurveto{\pgfqpoint{8.119430in}{1.646160in}}{\pgfqpoint{8.121434in}{1.641322in}}{\pgfqpoint{8.125000in}{1.637756in}}%
\pgfpathcurveto{\pgfqpoint{8.128567in}{1.634189in}}{\pgfqpoint{8.133404in}{1.632185in}}{\pgfqpoint{8.138448in}{1.632185in}}%
\pgfpathclose%
\pgfusepath{fill}%
\end{pgfscope}%
\begin{pgfscope}%
\pgfpathrectangle{\pgfqpoint{6.572727in}{0.474100in}}{\pgfqpoint{4.227273in}{3.318700in}}%
\pgfusepath{clip}%
\pgfsetbuttcap%
\pgfsetroundjoin%
\definecolor{currentfill}{rgb}{0.993248,0.906157,0.143936}%
\pgfsetfillcolor{currentfill}%
\pgfsetfillopacity{0.700000}%
\pgfsetlinewidth{0.000000pt}%
\definecolor{currentstroke}{rgb}{0.000000,0.000000,0.000000}%
\pgfsetstrokecolor{currentstroke}%
\pgfsetstrokeopacity{0.700000}%
\pgfsetdash{}{0pt}%
\pgfpathmoveto{\pgfqpoint{9.633024in}{1.413755in}}%
\pgfpathcurveto{\pgfqpoint{9.638067in}{1.413755in}}{\pgfqpoint{9.642905in}{1.415759in}}{\pgfqpoint{9.646472in}{1.419325in}}%
\pgfpathcurveto{\pgfqpoint{9.650038in}{1.422892in}}{\pgfqpoint{9.652042in}{1.427729in}}{\pgfqpoint{9.652042in}{1.432773in}}%
\pgfpathcurveto{\pgfqpoint{9.652042in}{1.437817in}}{\pgfqpoint{9.650038in}{1.442654in}}{\pgfqpoint{9.646472in}{1.446221in}}%
\pgfpathcurveto{\pgfqpoint{9.642905in}{1.449787in}}{\pgfqpoint{9.638067in}{1.451791in}}{\pgfqpoint{9.633024in}{1.451791in}}%
\pgfpathcurveto{\pgfqpoint{9.627980in}{1.451791in}}{\pgfqpoint{9.623142in}{1.449787in}}{\pgfqpoint{9.619576in}{1.446221in}}%
\pgfpathcurveto{\pgfqpoint{9.616009in}{1.442654in}}{\pgfqpoint{9.614006in}{1.437817in}}{\pgfqpoint{9.614006in}{1.432773in}}%
\pgfpathcurveto{\pgfqpoint{9.614006in}{1.427729in}}{\pgfqpoint{9.616009in}{1.422892in}}{\pgfqpoint{9.619576in}{1.419325in}}%
\pgfpathcurveto{\pgfqpoint{9.623142in}{1.415759in}}{\pgfqpoint{9.627980in}{1.413755in}}{\pgfqpoint{9.633024in}{1.413755in}}%
\pgfpathclose%
\pgfusepath{fill}%
\end{pgfscope}%
\begin{pgfscope}%
\pgfpathrectangle{\pgfqpoint{6.572727in}{0.474100in}}{\pgfqpoint{4.227273in}{3.318700in}}%
\pgfusepath{clip}%
\pgfsetbuttcap%
\pgfsetroundjoin%
\definecolor{currentfill}{rgb}{0.127568,0.566949,0.550556}%
\pgfsetfillcolor{currentfill}%
\pgfsetfillopacity{0.700000}%
\pgfsetlinewidth{0.000000pt}%
\definecolor{currentstroke}{rgb}{0.000000,0.000000,0.000000}%
\pgfsetstrokecolor{currentstroke}%
\pgfsetstrokeopacity{0.700000}%
\pgfsetdash{}{0pt}%
\pgfpathmoveto{\pgfqpoint{8.248668in}{3.412817in}}%
\pgfpathcurveto{\pgfqpoint{8.253712in}{3.412817in}}{\pgfqpoint{8.258549in}{3.414821in}}{\pgfqpoint{8.262116in}{3.418388in}}%
\pgfpathcurveto{\pgfqpoint{8.265682in}{3.421954in}}{\pgfqpoint{8.267686in}{3.426792in}}{\pgfqpoint{8.267686in}{3.431836in}}%
\pgfpathcurveto{\pgfqpoint{8.267686in}{3.436879in}}{\pgfqpoint{8.265682in}{3.441717in}}{\pgfqpoint{8.262116in}{3.445283in}}%
\pgfpathcurveto{\pgfqpoint{8.258549in}{3.448850in}}{\pgfqpoint{8.253712in}{3.450854in}}{\pgfqpoint{8.248668in}{3.450854in}}%
\pgfpathcurveto{\pgfqpoint{8.243624in}{3.450854in}}{\pgfqpoint{8.238787in}{3.448850in}}{\pgfqpoint{8.235220in}{3.445283in}}%
\pgfpathcurveto{\pgfqpoint{8.231654in}{3.441717in}}{\pgfqpoint{8.229650in}{3.436879in}}{\pgfqpoint{8.229650in}{3.431836in}}%
\pgfpathcurveto{\pgfqpoint{8.229650in}{3.426792in}}{\pgfqpoint{8.231654in}{3.421954in}}{\pgfqpoint{8.235220in}{3.418388in}}%
\pgfpathcurveto{\pgfqpoint{8.238787in}{3.414821in}}{\pgfqpoint{8.243624in}{3.412817in}}{\pgfqpoint{8.248668in}{3.412817in}}%
\pgfpathclose%
\pgfusepath{fill}%
\end{pgfscope}%
\begin{pgfscope}%
\pgfpathrectangle{\pgfqpoint{6.572727in}{0.474100in}}{\pgfqpoint{4.227273in}{3.318700in}}%
\pgfusepath{clip}%
\pgfsetbuttcap%
\pgfsetroundjoin%
\definecolor{currentfill}{rgb}{0.127568,0.566949,0.550556}%
\pgfsetfillcolor{currentfill}%
\pgfsetfillopacity{0.700000}%
\pgfsetlinewidth{0.000000pt}%
\definecolor{currentstroke}{rgb}{0.000000,0.000000,0.000000}%
\pgfsetstrokecolor{currentstroke}%
\pgfsetstrokeopacity{0.700000}%
\pgfsetdash{}{0pt}%
\pgfpathmoveto{\pgfqpoint{7.897783in}{2.893799in}}%
\pgfpathcurveto{\pgfqpoint{7.902826in}{2.893799in}}{\pgfqpoint{7.907664in}{2.895803in}}{\pgfqpoint{7.911230in}{2.899369in}}%
\pgfpathcurveto{\pgfqpoint{7.914797in}{2.902936in}}{\pgfqpoint{7.916801in}{2.907773in}}{\pgfqpoint{7.916801in}{2.912817in}}%
\pgfpathcurveto{\pgfqpoint{7.916801in}{2.917861in}}{\pgfqpoint{7.914797in}{2.922698in}}{\pgfqpoint{7.911230in}{2.926265in}}%
\pgfpathcurveto{\pgfqpoint{7.907664in}{2.929831in}}{\pgfqpoint{7.902826in}{2.931835in}}{\pgfqpoint{7.897783in}{2.931835in}}%
\pgfpathcurveto{\pgfqpoint{7.892739in}{2.931835in}}{\pgfqpoint{7.887901in}{2.929831in}}{\pgfqpoint{7.884335in}{2.926265in}}%
\pgfpathcurveto{\pgfqpoint{7.880768in}{2.922698in}}{\pgfqpoint{7.878764in}{2.917861in}}{\pgfqpoint{7.878764in}{2.912817in}}%
\pgfpathcurveto{\pgfqpoint{7.878764in}{2.907773in}}{\pgfqpoint{7.880768in}{2.902936in}}{\pgfqpoint{7.884335in}{2.899369in}}%
\pgfpathcurveto{\pgfqpoint{7.887901in}{2.895803in}}{\pgfqpoint{7.892739in}{2.893799in}}{\pgfqpoint{7.897783in}{2.893799in}}%
\pgfpathclose%
\pgfusepath{fill}%
\end{pgfscope}%
\begin{pgfscope}%
\pgfpathrectangle{\pgfqpoint{6.572727in}{0.474100in}}{\pgfqpoint{4.227273in}{3.318700in}}%
\pgfusepath{clip}%
\pgfsetbuttcap%
\pgfsetroundjoin%
\definecolor{currentfill}{rgb}{0.993248,0.906157,0.143936}%
\pgfsetfillcolor{currentfill}%
\pgfsetfillopacity{0.700000}%
\pgfsetlinewidth{0.000000pt}%
\definecolor{currentstroke}{rgb}{0.000000,0.000000,0.000000}%
\pgfsetstrokecolor{currentstroke}%
\pgfsetstrokeopacity{0.700000}%
\pgfsetdash{}{0pt}%
\pgfpathmoveto{\pgfqpoint{10.106096in}{1.447987in}}%
\pgfpathcurveto{\pgfqpoint{10.111139in}{1.447987in}}{\pgfqpoint{10.115977in}{1.449991in}}{\pgfqpoint{10.119543in}{1.453558in}}%
\pgfpathcurveto{\pgfqpoint{10.123110in}{1.457124in}}{\pgfqpoint{10.125114in}{1.461962in}}{\pgfqpoint{10.125114in}{1.467005in}}%
\pgfpathcurveto{\pgfqpoint{10.125114in}{1.472049in}}{\pgfqpoint{10.123110in}{1.476887in}}{\pgfqpoint{10.119543in}{1.480453in}}%
\pgfpathcurveto{\pgfqpoint{10.115977in}{1.484020in}}{\pgfqpoint{10.111139in}{1.486024in}}{\pgfqpoint{10.106096in}{1.486024in}}%
\pgfpathcurveto{\pgfqpoint{10.101052in}{1.486024in}}{\pgfqpoint{10.096214in}{1.484020in}}{\pgfqpoint{10.092648in}{1.480453in}}%
\pgfpathcurveto{\pgfqpoint{10.089081in}{1.476887in}}{\pgfqpoint{10.087077in}{1.472049in}}{\pgfqpoint{10.087077in}{1.467005in}}%
\pgfpathcurveto{\pgfqpoint{10.087077in}{1.461962in}}{\pgfqpoint{10.089081in}{1.457124in}}{\pgfqpoint{10.092648in}{1.453558in}}%
\pgfpathcurveto{\pgfqpoint{10.096214in}{1.449991in}}{\pgfqpoint{10.101052in}{1.447987in}}{\pgfqpoint{10.106096in}{1.447987in}}%
\pgfpathclose%
\pgfusepath{fill}%
\end{pgfscope}%
\begin{pgfscope}%
\pgfpathrectangle{\pgfqpoint{6.572727in}{0.474100in}}{\pgfqpoint{4.227273in}{3.318700in}}%
\pgfusepath{clip}%
\pgfsetbuttcap%
\pgfsetroundjoin%
\definecolor{currentfill}{rgb}{0.127568,0.566949,0.550556}%
\pgfsetfillcolor{currentfill}%
\pgfsetfillopacity{0.700000}%
\pgfsetlinewidth{0.000000pt}%
\definecolor{currentstroke}{rgb}{0.000000,0.000000,0.000000}%
\pgfsetstrokecolor{currentstroke}%
\pgfsetstrokeopacity{0.700000}%
\pgfsetdash{}{0pt}%
\pgfpathmoveto{\pgfqpoint{7.713252in}{3.082322in}}%
\pgfpathcurveto{\pgfqpoint{7.718295in}{3.082322in}}{\pgfqpoint{7.723133in}{3.084326in}}{\pgfqpoint{7.726699in}{3.087892in}}%
\pgfpathcurveto{\pgfqpoint{7.730266in}{3.091458in}}{\pgfqpoint{7.732270in}{3.096296in}}{\pgfqpoint{7.732270in}{3.101340in}}%
\pgfpathcurveto{\pgfqpoint{7.732270in}{3.106384in}}{\pgfqpoint{7.730266in}{3.111221in}}{\pgfqpoint{7.726699in}{3.114788in}}%
\pgfpathcurveto{\pgfqpoint{7.723133in}{3.118354in}}{\pgfqpoint{7.718295in}{3.120358in}}{\pgfqpoint{7.713252in}{3.120358in}}%
\pgfpathcurveto{\pgfqpoint{7.708208in}{3.120358in}}{\pgfqpoint{7.703370in}{3.118354in}}{\pgfqpoint{7.699804in}{3.114788in}}%
\pgfpathcurveto{\pgfqpoint{7.696237in}{3.111221in}}{\pgfqpoint{7.694233in}{3.106384in}}{\pgfqpoint{7.694233in}{3.101340in}}%
\pgfpathcurveto{\pgfqpoint{7.694233in}{3.096296in}}{\pgfqpoint{7.696237in}{3.091458in}}{\pgfqpoint{7.699804in}{3.087892in}}%
\pgfpathcurveto{\pgfqpoint{7.703370in}{3.084326in}}{\pgfqpoint{7.708208in}{3.082322in}}{\pgfqpoint{7.713252in}{3.082322in}}%
\pgfpathclose%
\pgfusepath{fill}%
\end{pgfscope}%
\begin{pgfscope}%
\pgfpathrectangle{\pgfqpoint{6.572727in}{0.474100in}}{\pgfqpoint{4.227273in}{3.318700in}}%
\pgfusepath{clip}%
\pgfsetbuttcap%
\pgfsetroundjoin%
\definecolor{currentfill}{rgb}{0.127568,0.566949,0.550556}%
\pgfsetfillcolor{currentfill}%
\pgfsetfillopacity{0.700000}%
\pgfsetlinewidth{0.000000pt}%
\definecolor{currentstroke}{rgb}{0.000000,0.000000,0.000000}%
\pgfsetstrokecolor{currentstroke}%
\pgfsetstrokeopacity{0.700000}%
\pgfsetdash{}{0pt}%
\pgfpathmoveto{\pgfqpoint{7.902055in}{2.390405in}}%
\pgfpathcurveto{\pgfqpoint{7.907099in}{2.390405in}}{\pgfqpoint{7.911937in}{2.392409in}}{\pgfqpoint{7.915503in}{2.395976in}}%
\pgfpathcurveto{\pgfqpoint{7.919070in}{2.399542in}}{\pgfqpoint{7.921073in}{2.404380in}}{\pgfqpoint{7.921073in}{2.409424in}}%
\pgfpathcurveto{\pgfqpoint{7.921073in}{2.414467in}}{\pgfqpoint{7.919070in}{2.419305in}}{\pgfqpoint{7.915503in}{2.422871in}}%
\pgfpathcurveto{\pgfqpoint{7.911937in}{2.426438in}}{\pgfqpoint{7.907099in}{2.428442in}}{\pgfqpoint{7.902055in}{2.428442in}}%
\pgfpathcurveto{\pgfqpoint{7.897012in}{2.428442in}}{\pgfqpoint{7.892174in}{2.426438in}}{\pgfqpoint{7.888607in}{2.422871in}}%
\pgfpathcurveto{\pgfqpoint{7.885041in}{2.419305in}}{\pgfqpoint{7.883037in}{2.414467in}}{\pgfqpoint{7.883037in}{2.409424in}}%
\pgfpathcurveto{\pgfqpoint{7.883037in}{2.404380in}}{\pgfqpoint{7.885041in}{2.399542in}}{\pgfqpoint{7.888607in}{2.395976in}}%
\pgfpathcurveto{\pgfqpoint{7.892174in}{2.392409in}}{\pgfqpoint{7.897012in}{2.390405in}}{\pgfqpoint{7.902055in}{2.390405in}}%
\pgfpathclose%
\pgfusepath{fill}%
\end{pgfscope}%
\begin{pgfscope}%
\pgfpathrectangle{\pgfqpoint{6.572727in}{0.474100in}}{\pgfqpoint{4.227273in}{3.318700in}}%
\pgfusepath{clip}%
\pgfsetbuttcap%
\pgfsetroundjoin%
\definecolor{currentfill}{rgb}{0.127568,0.566949,0.550556}%
\pgfsetfillcolor{currentfill}%
\pgfsetfillopacity{0.700000}%
\pgfsetlinewidth{0.000000pt}%
\definecolor{currentstroke}{rgb}{0.000000,0.000000,0.000000}%
\pgfsetstrokecolor{currentstroke}%
\pgfsetstrokeopacity{0.700000}%
\pgfsetdash{}{0pt}%
\pgfpathmoveto{\pgfqpoint{8.729553in}{2.518050in}}%
\pgfpathcurveto{\pgfqpoint{8.734596in}{2.518050in}}{\pgfqpoint{8.739434in}{2.520054in}}{\pgfqpoint{8.743001in}{2.523620in}}%
\pgfpathcurveto{\pgfqpoint{8.746567in}{2.527187in}}{\pgfqpoint{8.748571in}{2.532025in}}{\pgfqpoint{8.748571in}{2.537068in}}%
\pgfpathcurveto{\pgfqpoint{8.748571in}{2.542112in}}{\pgfqpoint{8.746567in}{2.546950in}}{\pgfqpoint{8.743001in}{2.550516in}}%
\pgfpathcurveto{\pgfqpoint{8.739434in}{2.554083in}}{\pgfqpoint{8.734596in}{2.556086in}}{\pgfqpoint{8.729553in}{2.556086in}}%
\pgfpathcurveto{\pgfqpoint{8.724509in}{2.556086in}}{\pgfqpoint{8.719671in}{2.554083in}}{\pgfqpoint{8.716105in}{2.550516in}}%
\pgfpathcurveto{\pgfqpoint{8.712538in}{2.546950in}}{\pgfqpoint{8.710535in}{2.542112in}}{\pgfqpoint{8.710535in}{2.537068in}}%
\pgfpathcurveto{\pgfqpoint{8.710535in}{2.532025in}}{\pgfqpoint{8.712538in}{2.527187in}}{\pgfqpoint{8.716105in}{2.523620in}}%
\pgfpathcurveto{\pgfqpoint{8.719671in}{2.520054in}}{\pgfqpoint{8.724509in}{2.518050in}}{\pgfqpoint{8.729553in}{2.518050in}}%
\pgfpathclose%
\pgfusepath{fill}%
\end{pgfscope}%
\begin{pgfscope}%
\pgfpathrectangle{\pgfqpoint{6.572727in}{0.474100in}}{\pgfqpoint{4.227273in}{3.318700in}}%
\pgfusepath{clip}%
\pgfsetbuttcap%
\pgfsetroundjoin%
\definecolor{currentfill}{rgb}{0.127568,0.566949,0.550556}%
\pgfsetfillcolor{currentfill}%
\pgfsetfillopacity{0.700000}%
\pgfsetlinewidth{0.000000pt}%
\definecolor{currentstroke}{rgb}{0.000000,0.000000,0.000000}%
\pgfsetstrokecolor{currentstroke}%
\pgfsetstrokeopacity{0.700000}%
\pgfsetdash{}{0pt}%
\pgfpathmoveto{\pgfqpoint{8.533761in}{2.657549in}}%
\pgfpathcurveto{\pgfqpoint{8.538805in}{2.657549in}}{\pgfqpoint{8.543643in}{2.659553in}}{\pgfqpoint{8.547209in}{2.663119in}}%
\pgfpathcurveto{\pgfqpoint{8.550776in}{2.666686in}}{\pgfqpoint{8.552780in}{2.671523in}}{\pgfqpoint{8.552780in}{2.676567in}}%
\pgfpathcurveto{\pgfqpoint{8.552780in}{2.681611in}}{\pgfqpoint{8.550776in}{2.686448in}}{\pgfqpoint{8.547209in}{2.690015in}}%
\pgfpathcurveto{\pgfqpoint{8.543643in}{2.693581in}}{\pgfqpoint{8.538805in}{2.695585in}}{\pgfqpoint{8.533761in}{2.695585in}}%
\pgfpathcurveto{\pgfqpoint{8.528718in}{2.695585in}}{\pgfqpoint{8.523880in}{2.693581in}}{\pgfqpoint{8.520314in}{2.690015in}}%
\pgfpathcurveto{\pgfqpoint{8.516747in}{2.686448in}}{\pgfqpoint{8.514743in}{2.681611in}}{\pgfqpoint{8.514743in}{2.676567in}}%
\pgfpathcurveto{\pgfqpoint{8.514743in}{2.671523in}}{\pgfqpoint{8.516747in}{2.666686in}}{\pgfqpoint{8.520314in}{2.663119in}}%
\pgfpathcurveto{\pgfqpoint{8.523880in}{2.659553in}}{\pgfqpoint{8.528718in}{2.657549in}}{\pgfqpoint{8.533761in}{2.657549in}}%
\pgfpathclose%
\pgfusepath{fill}%
\end{pgfscope}%
\begin{pgfscope}%
\pgfpathrectangle{\pgfqpoint{6.572727in}{0.474100in}}{\pgfqpoint{4.227273in}{3.318700in}}%
\pgfusepath{clip}%
\pgfsetbuttcap%
\pgfsetroundjoin%
\definecolor{currentfill}{rgb}{0.127568,0.566949,0.550556}%
\pgfsetfillcolor{currentfill}%
\pgfsetfillopacity{0.700000}%
\pgfsetlinewidth{0.000000pt}%
\definecolor{currentstroke}{rgb}{0.000000,0.000000,0.000000}%
\pgfsetstrokecolor{currentstroke}%
\pgfsetstrokeopacity{0.700000}%
\pgfsetdash{}{0pt}%
\pgfpathmoveto{\pgfqpoint{7.503841in}{2.633058in}}%
\pgfpathcurveto{\pgfqpoint{7.508884in}{2.633058in}}{\pgfqpoint{7.513722in}{2.635061in}}{\pgfqpoint{7.517289in}{2.638628in}}%
\pgfpathcurveto{\pgfqpoint{7.520855in}{2.642194in}}{\pgfqpoint{7.522859in}{2.647032in}}{\pgfqpoint{7.522859in}{2.652076in}}%
\pgfpathcurveto{\pgfqpoint{7.522859in}{2.657119in}}{\pgfqpoint{7.520855in}{2.661957in}}{\pgfqpoint{7.517289in}{2.665524in}}%
\pgfpathcurveto{\pgfqpoint{7.513722in}{2.669090in}}{\pgfqpoint{7.508884in}{2.671094in}}{\pgfqpoint{7.503841in}{2.671094in}}%
\pgfpathcurveto{\pgfqpoint{7.498797in}{2.671094in}}{\pgfqpoint{7.493959in}{2.669090in}}{\pgfqpoint{7.490393in}{2.665524in}}%
\pgfpathcurveto{\pgfqpoint{7.486826in}{2.661957in}}{\pgfqpoint{7.484823in}{2.657119in}}{\pgfqpoint{7.484823in}{2.652076in}}%
\pgfpathcurveto{\pgfqpoint{7.484823in}{2.647032in}}{\pgfqpoint{7.486826in}{2.642194in}}{\pgfqpoint{7.490393in}{2.638628in}}%
\pgfpathcurveto{\pgfqpoint{7.493959in}{2.635061in}}{\pgfqpoint{7.498797in}{2.633058in}}{\pgfqpoint{7.503841in}{2.633058in}}%
\pgfpathclose%
\pgfusepath{fill}%
\end{pgfscope}%
\begin{pgfscope}%
\pgfpathrectangle{\pgfqpoint{6.572727in}{0.474100in}}{\pgfqpoint{4.227273in}{3.318700in}}%
\pgfusepath{clip}%
\pgfsetbuttcap%
\pgfsetroundjoin%
\definecolor{currentfill}{rgb}{0.127568,0.566949,0.550556}%
\pgfsetfillcolor{currentfill}%
\pgfsetfillopacity{0.700000}%
\pgfsetlinewidth{0.000000pt}%
\definecolor{currentstroke}{rgb}{0.000000,0.000000,0.000000}%
\pgfsetstrokecolor{currentstroke}%
\pgfsetstrokeopacity{0.700000}%
\pgfsetdash{}{0pt}%
\pgfpathmoveto{\pgfqpoint{8.262997in}{2.979555in}}%
\pgfpathcurveto{\pgfqpoint{8.268041in}{2.979555in}}{\pgfqpoint{8.272879in}{2.981559in}}{\pgfqpoint{8.276445in}{2.985125in}}%
\pgfpathcurveto{\pgfqpoint{8.280012in}{2.988692in}}{\pgfqpoint{8.282016in}{2.993529in}}{\pgfqpoint{8.282016in}{2.998573in}}%
\pgfpathcurveto{\pgfqpoint{8.282016in}{3.003617in}}{\pgfqpoint{8.280012in}{3.008455in}}{\pgfqpoint{8.276445in}{3.012021in}}%
\pgfpathcurveto{\pgfqpoint{8.272879in}{3.015587in}}{\pgfqpoint{8.268041in}{3.017591in}}{\pgfqpoint{8.262997in}{3.017591in}}%
\pgfpathcurveto{\pgfqpoint{8.257954in}{3.017591in}}{\pgfqpoint{8.253116in}{3.015587in}}{\pgfqpoint{8.249550in}{3.012021in}}%
\pgfpathcurveto{\pgfqpoint{8.245983in}{3.008455in}}{\pgfqpoint{8.243979in}{3.003617in}}{\pgfqpoint{8.243979in}{2.998573in}}%
\pgfpathcurveto{\pgfqpoint{8.243979in}{2.993529in}}{\pgfqpoint{8.245983in}{2.988692in}}{\pgfqpoint{8.249550in}{2.985125in}}%
\pgfpathcurveto{\pgfqpoint{8.253116in}{2.981559in}}{\pgfqpoint{8.257954in}{2.979555in}}{\pgfqpoint{8.262997in}{2.979555in}}%
\pgfpathclose%
\pgfusepath{fill}%
\end{pgfscope}%
\begin{pgfscope}%
\pgfpathrectangle{\pgfqpoint{6.572727in}{0.474100in}}{\pgfqpoint{4.227273in}{3.318700in}}%
\pgfusepath{clip}%
\pgfsetbuttcap%
\pgfsetroundjoin%
\definecolor{currentfill}{rgb}{0.993248,0.906157,0.143936}%
\pgfsetfillcolor{currentfill}%
\pgfsetfillopacity{0.700000}%
\pgfsetlinewidth{0.000000pt}%
\definecolor{currentstroke}{rgb}{0.000000,0.000000,0.000000}%
\pgfsetstrokecolor{currentstroke}%
\pgfsetstrokeopacity{0.700000}%
\pgfsetdash{}{0pt}%
\pgfpathmoveto{\pgfqpoint{9.564049in}{1.277470in}}%
\pgfpathcurveto{\pgfqpoint{9.569093in}{1.277470in}}{\pgfqpoint{9.573930in}{1.279474in}}{\pgfqpoint{9.577497in}{1.283040in}}%
\pgfpathcurveto{\pgfqpoint{9.581063in}{1.286607in}}{\pgfqpoint{9.583067in}{1.291445in}}{\pgfqpoint{9.583067in}{1.296488in}}%
\pgfpathcurveto{\pgfqpoint{9.583067in}{1.301532in}}{\pgfqpoint{9.581063in}{1.306370in}}{\pgfqpoint{9.577497in}{1.309936in}}%
\pgfpathcurveto{\pgfqpoint{9.573930in}{1.313503in}}{\pgfqpoint{9.569093in}{1.315506in}}{\pgfqpoint{9.564049in}{1.315506in}}%
\pgfpathcurveto{\pgfqpoint{9.559005in}{1.315506in}}{\pgfqpoint{9.554167in}{1.313503in}}{\pgfqpoint{9.550601in}{1.309936in}}%
\pgfpathcurveto{\pgfqpoint{9.547035in}{1.306370in}}{\pgfqpoint{9.545031in}{1.301532in}}{\pgfqpoint{9.545031in}{1.296488in}}%
\pgfpathcurveto{\pgfqpoint{9.545031in}{1.291445in}}{\pgfqpoint{9.547035in}{1.286607in}}{\pgfqpoint{9.550601in}{1.283040in}}%
\pgfpathcurveto{\pgfqpoint{9.554167in}{1.279474in}}{\pgfqpoint{9.559005in}{1.277470in}}{\pgfqpoint{9.564049in}{1.277470in}}%
\pgfpathclose%
\pgfusepath{fill}%
\end{pgfscope}%
\begin{pgfscope}%
\pgfpathrectangle{\pgfqpoint{6.572727in}{0.474100in}}{\pgfqpoint{4.227273in}{3.318700in}}%
\pgfusepath{clip}%
\pgfsetbuttcap%
\pgfsetroundjoin%
\definecolor{currentfill}{rgb}{0.993248,0.906157,0.143936}%
\pgfsetfillcolor{currentfill}%
\pgfsetfillopacity{0.700000}%
\pgfsetlinewidth{0.000000pt}%
\definecolor{currentstroke}{rgb}{0.000000,0.000000,0.000000}%
\pgfsetstrokecolor{currentstroke}%
\pgfsetstrokeopacity{0.700000}%
\pgfsetdash{}{0pt}%
\pgfpathmoveto{\pgfqpoint{9.713501in}{2.068158in}}%
\pgfpathcurveto{\pgfqpoint{9.718545in}{2.068158in}}{\pgfqpoint{9.723383in}{2.070162in}}{\pgfqpoint{9.726949in}{2.073729in}}%
\pgfpathcurveto{\pgfqpoint{9.730515in}{2.077295in}}{\pgfqpoint{9.732519in}{2.082133in}}{\pgfqpoint{9.732519in}{2.087177in}}%
\pgfpathcurveto{\pgfqpoint{9.732519in}{2.092220in}}{\pgfqpoint{9.730515in}{2.097058in}}{\pgfqpoint{9.726949in}{2.100624in}}%
\pgfpathcurveto{\pgfqpoint{9.723383in}{2.104191in}}{\pgfqpoint{9.718545in}{2.106195in}}{\pgfqpoint{9.713501in}{2.106195in}}%
\pgfpathcurveto{\pgfqpoint{9.708457in}{2.106195in}}{\pgfqpoint{9.703620in}{2.104191in}}{\pgfqpoint{9.700053in}{2.100624in}}%
\pgfpathcurveto{\pgfqpoint{9.696487in}{2.097058in}}{\pgfqpoint{9.694483in}{2.092220in}}{\pgfqpoint{9.694483in}{2.087177in}}%
\pgfpathcurveto{\pgfqpoint{9.694483in}{2.082133in}}{\pgfqpoint{9.696487in}{2.077295in}}{\pgfqpoint{9.700053in}{2.073729in}}%
\pgfpathcurveto{\pgfqpoint{9.703620in}{2.070162in}}{\pgfqpoint{9.708457in}{2.068158in}}{\pgfqpoint{9.713501in}{2.068158in}}%
\pgfpathclose%
\pgfusepath{fill}%
\end{pgfscope}%
\begin{pgfscope}%
\pgfpathrectangle{\pgfqpoint{6.572727in}{0.474100in}}{\pgfqpoint{4.227273in}{3.318700in}}%
\pgfusepath{clip}%
\pgfsetbuttcap%
\pgfsetroundjoin%
\definecolor{currentfill}{rgb}{0.127568,0.566949,0.550556}%
\pgfsetfillcolor{currentfill}%
\pgfsetfillopacity{0.700000}%
\pgfsetlinewidth{0.000000pt}%
\definecolor{currentstroke}{rgb}{0.000000,0.000000,0.000000}%
\pgfsetstrokecolor{currentstroke}%
\pgfsetstrokeopacity{0.700000}%
\pgfsetdash{}{0pt}%
\pgfpathmoveto{\pgfqpoint{7.960718in}{1.606234in}}%
\pgfpathcurveto{\pgfqpoint{7.965762in}{1.606234in}}{\pgfqpoint{7.970599in}{1.608238in}}{\pgfqpoint{7.974166in}{1.611804in}}%
\pgfpathcurveto{\pgfqpoint{7.977732in}{1.615370in}}{\pgfqpoint{7.979736in}{1.620208in}}{\pgfqpoint{7.979736in}{1.625252in}}%
\pgfpathcurveto{\pgfqpoint{7.979736in}{1.630295in}}{\pgfqpoint{7.977732in}{1.635133in}}{\pgfqpoint{7.974166in}{1.638700in}}%
\pgfpathcurveto{\pgfqpoint{7.970599in}{1.642266in}}{\pgfqpoint{7.965762in}{1.644270in}}{\pgfqpoint{7.960718in}{1.644270in}}%
\pgfpathcurveto{\pgfqpoint{7.955674in}{1.644270in}}{\pgfqpoint{7.950836in}{1.642266in}}{\pgfqpoint{7.947270in}{1.638700in}}%
\pgfpathcurveto{\pgfqpoint{7.943704in}{1.635133in}}{\pgfqpoint{7.941700in}{1.630295in}}{\pgfqpoint{7.941700in}{1.625252in}}%
\pgfpathcurveto{\pgfqpoint{7.941700in}{1.620208in}}{\pgfqpoint{7.943704in}{1.615370in}}{\pgfqpoint{7.947270in}{1.611804in}}%
\pgfpathcurveto{\pgfqpoint{7.950836in}{1.608238in}}{\pgfqpoint{7.955674in}{1.606234in}}{\pgfqpoint{7.960718in}{1.606234in}}%
\pgfpathclose%
\pgfusepath{fill}%
\end{pgfscope}%
\begin{pgfscope}%
\pgfpathrectangle{\pgfqpoint{6.572727in}{0.474100in}}{\pgfqpoint{4.227273in}{3.318700in}}%
\pgfusepath{clip}%
\pgfsetbuttcap%
\pgfsetroundjoin%
\definecolor{currentfill}{rgb}{0.993248,0.906157,0.143936}%
\pgfsetfillcolor{currentfill}%
\pgfsetfillopacity{0.700000}%
\pgfsetlinewidth{0.000000pt}%
\definecolor{currentstroke}{rgb}{0.000000,0.000000,0.000000}%
\pgfsetstrokecolor{currentstroke}%
\pgfsetstrokeopacity{0.700000}%
\pgfsetdash{}{0pt}%
\pgfpathmoveto{\pgfqpoint{10.033390in}{1.032024in}}%
\pgfpathcurveto{\pgfqpoint{10.038434in}{1.032024in}}{\pgfqpoint{10.043272in}{1.034027in}}{\pgfqpoint{10.046838in}{1.037594in}}%
\pgfpathcurveto{\pgfqpoint{10.050404in}{1.041160in}}{\pgfqpoint{10.052408in}{1.045998in}}{\pgfqpoint{10.052408in}{1.051042in}}%
\pgfpathcurveto{\pgfqpoint{10.052408in}{1.056085in}}{\pgfqpoint{10.050404in}{1.060923in}}{\pgfqpoint{10.046838in}{1.064490in}}%
\pgfpathcurveto{\pgfqpoint{10.043272in}{1.068056in}}{\pgfqpoint{10.038434in}{1.070060in}}{\pgfqpoint{10.033390in}{1.070060in}}%
\pgfpathcurveto{\pgfqpoint{10.028346in}{1.070060in}}{\pgfqpoint{10.023509in}{1.068056in}}{\pgfqpoint{10.019942in}{1.064490in}}%
\pgfpathcurveto{\pgfqpoint{10.016376in}{1.060923in}}{\pgfqpoint{10.014372in}{1.056085in}}{\pgfqpoint{10.014372in}{1.051042in}}%
\pgfpathcurveto{\pgfqpoint{10.014372in}{1.045998in}}{\pgfqpoint{10.016376in}{1.041160in}}{\pgfqpoint{10.019942in}{1.037594in}}%
\pgfpathcurveto{\pgfqpoint{10.023509in}{1.034027in}}{\pgfqpoint{10.028346in}{1.032024in}}{\pgfqpoint{10.033390in}{1.032024in}}%
\pgfpathclose%
\pgfusepath{fill}%
\end{pgfscope}%
\begin{pgfscope}%
\pgfpathrectangle{\pgfqpoint{6.572727in}{0.474100in}}{\pgfqpoint{4.227273in}{3.318700in}}%
\pgfusepath{clip}%
\pgfsetbuttcap%
\pgfsetroundjoin%
\definecolor{currentfill}{rgb}{0.127568,0.566949,0.550556}%
\pgfsetfillcolor{currentfill}%
\pgfsetfillopacity{0.700000}%
\pgfsetlinewidth{0.000000pt}%
\definecolor{currentstroke}{rgb}{0.000000,0.000000,0.000000}%
\pgfsetstrokecolor{currentstroke}%
\pgfsetstrokeopacity{0.700000}%
\pgfsetdash{}{0pt}%
\pgfpathmoveto{\pgfqpoint{7.636301in}{1.799359in}}%
\pgfpathcurveto{\pgfqpoint{7.641344in}{1.799359in}}{\pgfqpoint{7.646182in}{1.801363in}}{\pgfqpoint{7.649749in}{1.804929in}}%
\pgfpathcurveto{\pgfqpoint{7.653315in}{1.808495in}}{\pgfqpoint{7.655319in}{1.813333in}}{\pgfqpoint{7.655319in}{1.818377in}}%
\pgfpathcurveto{\pgfqpoint{7.655319in}{1.823420in}}{\pgfqpoint{7.653315in}{1.828258in}}{\pgfqpoint{7.649749in}{1.831825in}}%
\pgfpathcurveto{\pgfqpoint{7.646182in}{1.835391in}}{\pgfqpoint{7.641344in}{1.837395in}}{\pgfqpoint{7.636301in}{1.837395in}}%
\pgfpathcurveto{\pgfqpoint{7.631257in}{1.837395in}}{\pgfqpoint{7.626419in}{1.835391in}}{\pgfqpoint{7.622853in}{1.831825in}}%
\pgfpathcurveto{\pgfqpoint{7.619286in}{1.828258in}}{\pgfqpoint{7.617283in}{1.823420in}}{\pgfqpoint{7.617283in}{1.818377in}}%
\pgfpathcurveto{\pgfqpoint{7.617283in}{1.813333in}}{\pgfqpoint{7.619286in}{1.808495in}}{\pgfqpoint{7.622853in}{1.804929in}}%
\pgfpathcurveto{\pgfqpoint{7.626419in}{1.801363in}}{\pgfqpoint{7.631257in}{1.799359in}}{\pgfqpoint{7.636301in}{1.799359in}}%
\pgfpathclose%
\pgfusepath{fill}%
\end{pgfscope}%
\begin{pgfscope}%
\pgfpathrectangle{\pgfqpoint{6.572727in}{0.474100in}}{\pgfqpoint{4.227273in}{3.318700in}}%
\pgfusepath{clip}%
\pgfsetbuttcap%
\pgfsetroundjoin%
\definecolor{currentfill}{rgb}{0.127568,0.566949,0.550556}%
\pgfsetfillcolor{currentfill}%
\pgfsetfillopacity{0.700000}%
\pgfsetlinewidth{0.000000pt}%
\definecolor{currentstroke}{rgb}{0.000000,0.000000,0.000000}%
\pgfsetstrokecolor{currentstroke}%
\pgfsetstrokeopacity{0.700000}%
\pgfsetdash{}{0pt}%
\pgfpathmoveto{\pgfqpoint{8.489685in}{2.659460in}}%
\pgfpathcurveto{\pgfqpoint{8.494729in}{2.659460in}}{\pgfqpoint{8.499567in}{2.661464in}}{\pgfqpoint{8.503133in}{2.665030in}}%
\pgfpathcurveto{\pgfqpoint{8.506700in}{2.668597in}}{\pgfqpoint{8.508703in}{2.673434in}}{\pgfqpoint{8.508703in}{2.678478in}}%
\pgfpathcurveto{\pgfqpoint{8.508703in}{2.683522in}}{\pgfqpoint{8.506700in}{2.688360in}}{\pgfqpoint{8.503133in}{2.691926in}}%
\pgfpathcurveto{\pgfqpoint{8.499567in}{2.695492in}}{\pgfqpoint{8.494729in}{2.697496in}}{\pgfqpoint{8.489685in}{2.697496in}}%
\pgfpathcurveto{\pgfqpoint{8.484642in}{2.697496in}}{\pgfqpoint{8.479804in}{2.695492in}}{\pgfqpoint{8.476237in}{2.691926in}}%
\pgfpathcurveto{\pgfqpoint{8.472671in}{2.688360in}}{\pgfqpoint{8.470667in}{2.683522in}}{\pgfqpoint{8.470667in}{2.678478in}}%
\pgfpathcurveto{\pgfqpoint{8.470667in}{2.673434in}}{\pgfqpoint{8.472671in}{2.668597in}}{\pgfqpoint{8.476237in}{2.665030in}}%
\pgfpathcurveto{\pgfqpoint{8.479804in}{2.661464in}}{\pgfqpoint{8.484642in}{2.659460in}}{\pgfqpoint{8.489685in}{2.659460in}}%
\pgfpathclose%
\pgfusepath{fill}%
\end{pgfscope}%
\begin{pgfscope}%
\pgfpathrectangle{\pgfqpoint{6.572727in}{0.474100in}}{\pgfqpoint{4.227273in}{3.318700in}}%
\pgfusepath{clip}%
\pgfsetbuttcap%
\pgfsetroundjoin%
\definecolor{currentfill}{rgb}{0.127568,0.566949,0.550556}%
\pgfsetfillcolor{currentfill}%
\pgfsetfillopacity{0.700000}%
\pgfsetlinewidth{0.000000pt}%
\definecolor{currentstroke}{rgb}{0.000000,0.000000,0.000000}%
\pgfsetstrokecolor{currentstroke}%
\pgfsetstrokeopacity{0.700000}%
\pgfsetdash{}{0pt}%
\pgfpathmoveto{\pgfqpoint{8.525579in}{1.453585in}}%
\pgfpathcurveto{\pgfqpoint{8.530622in}{1.453585in}}{\pgfqpoint{8.535460in}{1.455588in}}{\pgfqpoint{8.539027in}{1.459155in}}%
\pgfpathcurveto{\pgfqpoint{8.542593in}{1.462721in}}{\pgfqpoint{8.544597in}{1.467559in}}{\pgfqpoint{8.544597in}{1.472603in}}%
\pgfpathcurveto{\pgfqpoint{8.544597in}{1.477646in}}{\pgfqpoint{8.542593in}{1.482484in}}{\pgfqpoint{8.539027in}{1.486051in}}%
\pgfpathcurveto{\pgfqpoint{8.535460in}{1.489617in}}{\pgfqpoint{8.530622in}{1.491621in}}{\pgfqpoint{8.525579in}{1.491621in}}%
\pgfpathcurveto{\pgfqpoint{8.520535in}{1.491621in}}{\pgfqpoint{8.515697in}{1.489617in}}{\pgfqpoint{8.512131in}{1.486051in}}%
\pgfpathcurveto{\pgfqpoint{8.508564in}{1.482484in}}{\pgfqpoint{8.506561in}{1.477646in}}{\pgfqpoint{8.506561in}{1.472603in}}%
\pgfpathcurveto{\pgfqpoint{8.506561in}{1.467559in}}{\pgfqpoint{8.508564in}{1.462721in}}{\pgfqpoint{8.512131in}{1.459155in}}%
\pgfpathcurveto{\pgfqpoint{8.515697in}{1.455588in}}{\pgfqpoint{8.520535in}{1.453585in}}{\pgfqpoint{8.525579in}{1.453585in}}%
\pgfpathclose%
\pgfusepath{fill}%
\end{pgfscope}%
\begin{pgfscope}%
\pgfpathrectangle{\pgfqpoint{6.572727in}{0.474100in}}{\pgfqpoint{4.227273in}{3.318700in}}%
\pgfusepath{clip}%
\pgfsetbuttcap%
\pgfsetroundjoin%
\definecolor{currentfill}{rgb}{0.127568,0.566949,0.550556}%
\pgfsetfillcolor{currentfill}%
\pgfsetfillopacity{0.700000}%
\pgfsetlinewidth{0.000000pt}%
\definecolor{currentstroke}{rgb}{0.000000,0.000000,0.000000}%
\pgfsetstrokecolor{currentstroke}%
\pgfsetstrokeopacity{0.700000}%
\pgfsetdash{}{0pt}%
\pgfpathmoveto{\pgfqpoint{7.617807in}{3.468116in}}%
\pgfpathcurveto{\pgfqpoint{7.622850in}{3.468116in}}{\pgfqpoint{7.627688in}{3.470119in}}{\pgfqpoint{7.631254in}{3.473686in}}%
\pgfpathcurveto{\pgfqpoint{7.634821in}{3.477252in}}{\pgfqpoint{7.636825in}{3.482090in}}{\pgfqpoint{7.636825in}{3.487134in}}%
\pgfpathcurveto{\pgfqpoint{7.636825in}{3.492177in}}{\pgfqpoint{7.634821in}{3.497015in}}{\pgfqpoint{7.631254in}{3.500582in}}%
\pgfpathcurveto{\pgfqpoint{7.627688in}{3.504148in}}{\pgfqpoint{7.622850in}{3.506152in}}{\pgfqpoint{7.617807in}{3.506152in}}%
\pgfpathcurveto{\pgfqpoint{7.612763in}{3.506152in}}{\pgfqpoint{7.607925in}{3.504148in}}{\pgfqpoint{7.604359in}{3.500582in}}%
\pgfpathcurveto{\pgfqpoint{7.600792in}{3.497015in}}{\pgfqpoint{7.598788in}{3.492177in}}{\pgfqpoint{7.598788in}{3.487134in}}%
\pgfpathcurveto{\pgfqpoint{7.598788in}{3.482090in}}{\pgfqpoint{7.600792in}{3.477252in}}{\pgfqpoint{7.604359in}{3.473686in}}%
\pgfpathcurveto{\pgfqpoint{7.607925in}{3.470119in}}{\pgfqpoint{7.612763in}{3.468116in}}{\pgfqpoint{7.617807in}{3.468116in}}%
\pgfpathclose%
\pgfusepath{fill}%
\end{pgfscope}%
\begin{pgfscope}%
\pgfpathrectangle{\pgfqpoint{6.572727in}{0.474100in}}{\pgfqpoint{4.227273in}{3.318700in}}%
\pgfusepath{clip}%
\pgfsetbuttcap%
\pgfsetroundjoin%
\definecolor{currentfill}{rgb}{0.127568,0.566949,0.550556}%
\pgfsetfillcolor{currentfill}%
\pgfsetfillopacity{0.700000}%
\pgfsetlinewidth{0.000000pt}%
\definecolor{currentstroke}{rgb}{0.000000,0.000000,0.000000}%
\pgfsetstrokecolor{currentstroke}%
\pgfsetstrokeopacity{0.700000}%
\pgfsetdash{}{0pt}%
\pgfpathmoveto{\pgfqpoint{8.793839in}{2.745158in}}%
\pgfpathcurveto{\pgfqpoint{8.798882in}{2.745158in}}{\pgfqpoint{8.803720in}{2.747162in}}{\pgfqpoint{8.807287in}{2.750729in}}%
\pgfpathcurveto{\pgfqpoint{8.810853in}{2.754295in}}{\pgfqpoint{8.812857in}{2.759133in}}{\pgfqpoint{8.812857in}{2.764177in}}%
\pgfpathcurveto{\pgfqpoint{8.812857in}{2.769220in}}{\pgfqpoint{8.810853in}{2.774058in}}{\pgfqpoint{8.807287in}{2.777624in}}%
\pgfpathcurveto{\pgfqpoint{8.803720in}{2.781191in}}{\pgfqpoint{8.798882in}{2.783195in}}{\pgfqpoint{8.793839in}{2.783195in}}%
\pgfpathcurveto{\pgfqpoint{8.788795in}{2.783195in}}{\pgfqpoint{8.783957in}{2.781191in}}{\pgfqpoint{8.780391in}{2.777624in}}%
\pgfpathcurveto{\pgfqpoint{8.776824in}{2.774058in}}{\pgfqpoint{8.774821in}{2.769220in}}{\pgfqpoint{8.774821in}{2.764177in}}%
\pgfpathcurveto{\pgfqpoint{8.774821in}{2.759133in}}{\pgfqpoint{8.776824in}{2.754295in}}{\pgfqpoint{8.780391in}{2.750729in}}%
\pgfpathcurveto{\pgfqpoint{8.783957in}{2.747162in}}{\pgfqpoint{8.788795in}{2.745158in}}{\pgfqpoint{8.793839in}{2.745158in}}%
\pgfpathclose%
\pgfusepath{fill}%
\end{pgfscope}%
\begin{pgfscope}%
\pgfpathrectangle{\pgfqpoint{6.572727in}{0.474100in}}{\pgfqpoint{4.227273in}{3.318700in}}%
\pgfusepath{clip}%
\pgfsetbuttcap%
\pgfsetroundjoin%
\definecolor{currentfill}{rgb}{0.127568,0.566949,0.550556}%
\pgfsetfillcolor{currentfill}%
\pgfsetfillopacity{0.700000}%
\pgfsetlinewidth{0.000000pt}%
\definecolor{currentstroke}{rgb}{0.000000,0.000000,0.000000}%
\pgfsetstrokecolor{currentstroke}%
\pgfsetstrokeopacity{0.700000}%
\pgfsetdash{}{0pt}%
\pgfpathmoveto{\pgfqpoint{7.728160in}{2.616089in}}%
\pgfpathcurveto{\pgfqpoint{7.733204in}{2.616089in}}{\pgfqpoint{7.738042in}{2.618093in}}{\pgfqpoint{7.741608in}{2.621659in}}%
\pgfpathcurveto{\pgfqpoint{7.745175in}{2.625225in}}{\pgfqpoint{7.747179in}{2.630063in}}{\pgfqpoint{7.747179in}{2.635107in}}%
\pgfpathcurveto{\pgfqpoint{7.747179in}{2.640150in}}{\pgfqpoint{7.745175in}{2.644988in}}{\pgfqpoint{7.741608in}{2.648555in}}%
\pgfpathcurveto{\pgfqpoint{7.738042in}{2.652121in}}{\pgfqpoint{7.733204in}{2.654125in}}{\pgfqpoint{7.728160in}{2.654125in}}%
\pgfpathcurveto{\pgfqpoint{7.723117in}{2.654125in}}{\pgfqpoint{7.718279in}{2.652121in}}{\pgfqpoint{7.714713in}{2.648555in}}%
\pgfpathcurveto{\pgfqpoint{7.711146in}{2.644988in}}{\pgfqpoint{7.709142in}{2.640150in}}{\pgfqpoint{7.709142in}{2.635107in}}%
\pgfpathcurveto{\pgfqpoint{7.709142in}{2.630063in}}{\pgfqpoint{7.711146in}{2.625225in}}{\pgfqpoint{7.714713in}{2.621659in}}%
\pgfpathcurveto{\pgfqpoint{7.718279in}{2.618093in}}{\pgfqpoint{7.723117in}{2.616089in}}{\pgfqpoint{7.728160in}{2.616089in}}%
\pgfpathclose%
\pgfusepath{fill}%
\end{pgfscope}%
\begin{pgfscope}%
\pgfpathrectangle{\pgfqpoint{6.572727in}{0.474100in}}{\pgfqpoint{4.227273in}{3.318700in}}%
\pgfusepath{clip}%
\pgfsetbuttcap%
\pgfsetroundjoin%
\definecolor{currentfill}{rgb}{0.993248,0.906157,0.143936}%
\pgfsetfillcolor{currentfill}%
\pgfsetfillopacity{0.700000}%
\pgfsetlinewidth{0.000000pt}%
\definecolor{currentstroke}{rgb}{0.000000,0.000000,0.000000}%
\pgfsetstrokecolor{currentstroke}%
\pgfsetstrokeopacity{0.700000}%
\pgfsetdash{}{0pt}%
\pgfpathmoveto{\pgfqpoint{9.415018in}{1.716083in}}%
\pgfpathcurveto{\pgfqpoint{9.420062in}{1.716083in}}{\pgfqpoint{9.424900in}{1.718087in}}{\pgfqpoint{9.428466in}{1.721653in}}%
\pgfpathcurveto{\pgfqpoint{9.432033in}{1.725220in}}{\pgfqpoint{9.434037in}{1.730057in}}{\pgfqpoint{9.434037in}{1.735101in}}%
\pgfpathcurveto{\pgfqpoint{9.434037in}{1.740145in}}{\pgfqpoint{9.432033in}{1.744982in}}{\pgfqpoint{9.428466in}{1.748549in}}%
\pgfpathcurveto{\pgfqpoint{9.424900in}{1.752115in}}{\pgfqpoint{9.420062in}{1.754119in}}{\pgfqpoint{9.415018in}{1.754119in}}%
\pgfpathcurveto{\pgfqpoint{9.409975in}{1.754119in}}{\pgfqpoint{9.405137in}{1.752115in}}{\pgfqpoint{9.401571in}{1.748549in}}%
\pgfpathcurveto{\pgfqpoint{9.398004in}{1.744982in}}{\pgfqpoint{9.396000in}{1.740145in}}{\pgfqpoint{9.396000in}{1.735101in}}%
\pgfpathcurveto{\pgfqpoint{9.396000in}{1.730057in}}{\pgfqpoint{9.398004in}{1.725220in}}{\pgfqpoint{9.401571in}{1.721653in}}%
\pgfpathcurveto{\pgfqpoint{9.405137in}{1.718087in}}{\pgfqpoint{9.409975in}{1.716083in}}{\pgfqpoint{9.415018in}{1.716083in}}%
\pgfpathclose%
\pgfusepath{fill}%
\end{pgfscope}%
\begin{pgfscope}%
\pgfpathrectangle{\pgfqpoint{6.572727in}{0.474100in}}{\pgfqpoint{4.227273in}{3.318700in}}%
\pgfusepath{clip}%
\pgfsetbuttcap%
\pgfsetroundjoin%
\definecolor{currentfill}{rgb}{0.993248,0.906157,0.143936}%
\pgfsetfillcolor{currentfill}%
\pgfsetfillopacity{0.700000}%
\pgfsetlinewidth{0.000000pt}%
\definecolor{currentstroke}{rgb}{0.000000,0.000000,0.000000}%
\pgfsetstrokecolor{currentstroke}%
\pgfsetstrokeopacity{0.700000}%
\pgfsetdash{}{0pt}%
\pgfpathmoveto{\pgfqpoint{9.629667in}{1.910004in}}%
\pgfpathcurveto{\pgfqpoint{9.634711in}{1.910004in}}{\pgfqpoint{9.639548in}{1.912008in}}{\pgfqpoint{9.643115in}{1.915574in}}%
\pgfpathcurveto{\pgfqpoint{9.646681in}{1.919141in}}{\pgfqpoint{9.648685in}{1.923979in}}{\pgfqpoint{9.648685in}{1.929022in}}%
\pgfpathcurveto{\pgfqpoint{9.648685in}{1.934066in}}{\pgfqpoint{9.646681in}{1.938904in}}{\pgfqpoint{9.643115in}{1.942470in}}%
\pgfpathcurveto{\pgfqpoint{9.639548in}{1.946037in}}{\pgfqpoint{9.634711in}{1.948040in}}{\pgfqpoint{9.629667in}{1.948040in}}%
\pgfpathcurveto{\pgfqpoint{9.624623in}{1.948040in}}{\pgfqpoint{9.619786in}{1.946037in}}{\pgfqpoint{9.616219in}{1.942470in}}%
\pgfpathcurveto{\pgfqpoint{9.612653in}{1.938904in}}{\pgfqpoint{9.610649in}{1.934066in}}{\pgfqpoint{9.610649in}{1.929022in}}%
\pgfpathcurveto{\pgfqpoint{9.610649in}{1.923979in}}{\pgfqpoint{9.612653in}{1.919141in}}{\pgfqpoint{9.616219in}{1.915574in}}%
\pgfpathcurveto{\pgfqpoint{9.619786in}{1.912008in}}{\pgfqpoint{9.624623in}{1.910004in}}{\pgfqpoint{9.629667in}{1.910004in}}%
\pgfpathclose%
\pgfusepath{fill}%
\end{pgfscope}%
\begin{pgfscope}%
\pgfpathrectangle{\pgfqpoint{6.572727in}{0.474100in}}{\pgfqpoint{4.227273in}{3.318700in}}%
\pgfusepath{clip}%
\pgfsetbuttcap%
\pgfsetroundjoin%
\definecolor{currentfill}{rgb}{0.127568,0.566949,0.550556}%
\pgfsetfillcolor{currentfill}%
\pgfsetfillopacity{0.700000}%
\pgfsetlinewidth{0.000000pt}%
\definecolor{currentstroke}{rgb}{0.000000,0.000000,0.000000}%
\pgfsetstrokecolor{currentstroke}%
\pgfsetstrokeopacity{0.700000}%
\pgfsetdash{}{0pt}%
\pgfpathmoveto{\pgfqpoint{8.727090in}{1.930325in}}%
\pgfpathcurveto{\pgfqpoint{8.732134in}{1.930325in}}{\pgfqpoint{8.736972in}{1.932329in}}{\pgfqpoint{8.740538in}{1.935896in}}%
\pgfpathcurveto{\pgfqpoint{8.744105in}{1.939462in}}{\pgfqpoint{8.746108in}{1.944300in}}{\pgfqpoint{8.746108in}{1.949343in}}%
\pgfpathcurveto{\pgfqpoint{8.746108in}{1.954387in}}{\pgfqpoint{8.744105in}{1.959225in}}{\pgfqpoint{8.740538in}{1.962791in}}%
\pgfpathcurveto{\pgfqpoint{8.736972in}{1.966358in}}{\pgfqpoint{8.732134in}{1.968362in}}{\pgfqpoint{8.727090in}{1.968362in}}%
\pgfpathcurveto{\pgfqpoint{8.722047in}{1.968362in}}{\pgfqpoint{8.717209in}{1.966358in}}{\pgfqpoint{8.713642in}{1.962791in}}%
\pgfpathcurveto{\pgfqpoint{8.710076in}{1.959225in}}{\pgfqpoint{8.708072in}{1.954387in}}{\pgfqpoint{8.708072in}{1.949343in}}%
\pgfpathcurveto{\pgfqpoint{8.708072in}{1.944300in}}{\pgfqpoint{8.710076in}{1.939462in}}{\pgfqpoint{8.713642in}{1.935896in}}%
\pgfpathcurveto{\pgfqpoint{8.717209in}{1.932329in}}{\pgfqpoint{8.722047in}{1.930325in}}{\pgfqpoint{8.727090in}{1.930325in}}%
\pgfpathclose%
\pgfusepath{fill}%
\end{pgfscope}%
\begin{pgfscope}%
\pgfpathrectangle{\pgfqpoint{6.572727in}{0.474100in}}{\pgfqpoint{4.227273in}{3.318700in}}%
\pgfusepath{clip}%
\pgfsetbuttcap%
\pgfsetroundjoin%
\definecolor{currentfill}{rgb}{0.127568,0.566949,0.550556}%
\pgfsetfillcolor{currentfill}%
\pgfsetfillopacity{0.700000}%
\pgfsetlinewidth{0.000000pt}%
\definecolor{currentstroke}{rgb}{0.000000,0.000000,0.000000}%
\pgfsetstrokecolor{currentstroke}%
\pgfsetstrokeopacity{0.700000}%
\pgfsetdash{}{0pt}%
\pgfpathmoveto{\pgfqpoint{7.412554in}{1.354471in}}%
\pgfpathcurveto{\pgfqpoint{7.417598in}{1.354471in}}{\pgfqpoint{7.422436in}{1.356475in}}{\pgfqpoint{7.426002in}{1.360042in}}%
\pgfpathcurveto{\pgfqpoint{7.429568in}{1.363608in}}{\pgfqpoint{7.431572in}{1.368446in}}{\pgfqpoint{7.431572in}{1.373489in}}%
\pgfpathcurveto{\pgfqpoint{7.431572in}{1.378533in}}{\pgfqpoint{7.429568in}{1.383371in}}{\pgfqpoint{7.426002in}{1.386937in}}%
\pgfpathcurveto{\pgfqpoint{7.422436in}{1.390504in}}{\pgfqpoint{7.417598in}{1.392508in}}{\pgfqpoint{7.412554in}{1.392508in}}%
\pgfpathcurveto{\pgfqpoint{7.407510in}{1.392508in}}{\pgfqpoint{7.402673in}{1.390504in}}{\pgfqpoint{7.399106in}{1.386937in}}%
\pgfpathcurveto{\pgfqpoint{7.395540in}{1.383371in}}{\pgfqpoint{7.393536in}{1.378533in}}{\pgfqpoint{7.393536in}{1.373489in}}%
\pgfpathcurveto{\pgfqpoint{7.393536in}{1.368446in}}{\pgfqpoint{7.395540in}{1.363608in}}{\pgfqpoint{7.399106in}{1.360042in}}%
\pgfpathcurveto{\pgfqpoint{7.402673in}{1.356475in}}{\pgfqpoint{7.407510in}{1.354471in}}{\pgfqpoint{7.412554in}{1.354471in}}%
\pgfpathclose%
\pgfusepath{fill}%
\end{pgfscope}%
\begin{pgfscope}%
\pgfpathrectangle{\pgfqpoint{6.572727in}{0.474100in}}{\pgfqpoint{4.227273in}{3.318700in}}%
\pgfusepath{clip}%
\pgfsetbuttcap%
\pgfsetroundjoin%
\definecolor{currentfill}{rgb}{0.127568,0.566949,0.550556}%
\pgfsetfillcolor{currentfill}%
\pgfsetfillopacity{0.700000}%
\pgfsetlinewidth{0.000000pt}%
\definecolor{currentstroke}{rgb}{0.000000,0.000000,0.000000}%
\pgfsetstrokecolor{currentstroke}%
\pgfsetstrokeopacity{0.700000}%
\pgfsetdash{}{0pt}%
\pgfpathmoveto{\pgfqpoint{7.819516in}{2.709233in}}%
\pgfpathcurveto{\pgfqpoint{7.824560in}{2.709233in}}{\pgfqpoint{7.829398in}{2.711237in}}{\pgfqpoint{7.832964in}{2.714803in}}%
\pgfpathcurveto{\pgfqpoint{7.836531in}{2.718369in}}{\pgfqpoint{7.838534in}{2.723207in}}{\pgfqpoint{7.838534in}{2.728251in}}%
\pgfpathcurveto{\pgfqpoint{7.838534in}{2.733295in}}{\pgfqpoint{7.836531in}{2.738132in}}{\pgfqpoint{7.832964in}{2.741699in}}%
\pgfpathcurveto{\pgfqpoint{7.829398in}{2.745265in}}{\pgfqpoint{7.824560in}{2.747269in}}{\pgfqpoint{7.819516in}{2.747269in}}%
\pgfpathcurveto{\pgfqpoint{7.814473in}{2.747269in}}{\pgfqpoint{7.809635in}{2.745265in}}{\pgfqpoint{7.806068in}{2.741699in}}%
\pgfpathcurveto{\pgfqpoint{7.802502in}{2.738132in}}{\pgfqpoint{7.800498in}{2.733295in}}{\pgfqpoint{7.800498in}{2.728251in}}%
\pgfpathcurveto{\pgfqpoint{7.800498in}{2.723207in}}{\pgfqpoint{7.802502in}{2.718369in}}{\pgfqpoint{7.806068in}{2.714803in}}%
\pgfpathcurveto{\pgfqpoint{7.809635in}{2.711237in}}{\pgfqpoint{7.814473in}{2.709233in}}{\pgfqpoint{7.819516in}{2.709233in}}%
\pgfpathclose%
\pgfusepath{fill}%
\end{pgfscope}%
\begin{pgfscope}%
\pgfpathrectangle{\pgfqpoint{6.572727in}{0.474100in}}{\pgfqpoint{4.227273in}{3.318700in}}%
\pgfusepath{clip}%
\pgfsetbuttcap%
\pgfsetroundjoin%
\definecolor{currentfill}{rgb}{0.993248,0.906157,0.143936}%
\pgfsetfillcolor{currentfill}%
\pgfsetfillopacity{0.700000}%
\pgfsetlinewidth{0.000000pt}%
\definecolor{currentstroke}{rgb}{0.000000,0.000000,0.000000}%
\pgfsetstrokecolor{currentstroke}%
\pgfsetstrokeopacity{0.700000}%
\pgfsetdash{}{0pt}%
\pgfpathmoveto{\pgfqpoint{9.634047in}{1.544085in}}%
\pgfpathcurveto{\pgfqpoint{9.639091in}{1.544085in}}{\pgfqpoint{9.643928in}{1.546089in}}{\pgfqpoint{9.647495in}{1.549655in}}%
\pgfpathcurveto{\pgfqpoint{9.651061in}{1.553222in}}{\pgfqpoint{9.653065in}{1.558060in}}{\pgfqpoint{9.653065in}{1.563103in}}%
\pgfpathcurveto{\pgfqpoint{9.653065in}{1.568147in}}{\pgfqpoint{9.651061in}{1.572985in}}{\pgfqpoint{9.647495in}{1.576551in}}%
\pgfpathcurveto{\pgfqpoint{9.643928in}{1.580118in}}{\pgfqpoint{9.639091in}{1.582121in}}{\pgfqpoint{9.634047in}{1.582121in}}%
\pgfpathcurveto{\pgfqpoint{9.629003in}{1.582121in}}{\pgfqpoint{9.624165in}{1.580118in}}{\pgfqpoint{9.620599in}{1.576551in}}%
\pgfpathcurveto{\pgfqpoint{9.617033in}{1.572985in}}{\pgfqpoint{9.615029in}{1.568147in}}{\pgfqpoint{9.615029in}{1.563103in}}%
\pgfpathcurveto{\pgfqpoint{9.615029in}{1.558060in}}{\pgfqpoint{9.617033in}{1.553222in}}{\pgfqpoint{9.620599in}{1.549655in}}%
\pgfpathcurveto{\pgfqpoint{9.624165in}{1.546089in}}{\pgfqpoint{9.629003in}{1.544085in}}{\pgfqpoint{9.634047in}{1.544085in}}%
\pgfpathclose%
\pgfusepath{fill}%
\end{pgfscope}%
\begin{pgfscope}%
\pgfpathrectangle{\pgfqpoint{6.572727in}{0.474100in}}{\pgfqpoint{4.227273in}{3.318700in}}%
\pgfusepath{clip}%
\pgfsetbuttcap%
\pgfsetroundjoin%
\definecolor{currentfill}{rgb}{0.127568,0.566949,0.550556}%
\pgfsetfillcolor{currentfill}%
\pgfsetfillopacity{0.700000}%
\pgfsetlinewidth{0.000000pt}%
\definecolor{currentstroke}{rgb}{0.000000,0.000000,0.000000}%
\pgfsetstrokecolor{currentstroke}%
\pgfsetstrokeopacity{0.700000}%
\pgfsetdash{}{0pt}%
\pgfpathmoveto{\pgfqpoint{7.652595in}{1.913445in}}%
\pgfpathcurveto{\pgfqpoint{7.657639in}{1.913445in}}{\pgfqpoint{7.662477in}{1.915449in}}{\pgfqpoint{7.666043in}{1.919015in}}%
\pgfpathcurveto{\pgfqpoint{7.669610in}{1.922581in}}{\pgfqpoint{7.671613in}{1.927419in}}{\pgfqpoint{7.671613in}{1.932463in}}%
\pgfpathcurveto{\pgfqpoint{7.671613in}{1.937507in}}{\pgfqpoint{7.669610in}{1.942344in}}{\pgfqpoint{7.666043in}{1.945911in}}%
\pgfpathcurveto{\pgfqpoint{7.662477in}{1.949477in}}{\pgfqpoint{7.657639in}{1.951481in}}{\pgfqpoint{7.652595in}{1.951481in}}%
\pgfpathcurveto{\pgfqpoint{7.647552in}{1.951481in}}{\pgfqpoint{7.642714in}{1.949477in}}{\pgfqpoint{7.639147in}{1.945911in}}%
\pgfpathcurveto{\pgfqpoint{7.635581in}{1.942344in}}{\pgfqpoint{7.633577in}{1.937507in}}{\pgfqpoint{7.633577in}{1.932463in}}%
\pgfpathcurveto{\pgfqpoint{7.633577in}{1.927419in}}{\pgfqpoint{7.635581in}{1.922581in}}{\pgfqpoint{7.639147in}{1.919015in}}%
\pgfpathcurveto{\pgfqpoint{7.642714in}{1.915449in}}{\pgfqpoint{7.647552in}{1.913445in}}{\pgfqpoint{7.652595in}{1.913445in}}%
\pgfpathclose%
\pgfusepath{fill}%
\end{pgfscope}%
\begin{pgfscope}%
\pgfpathrectangle{\pgfqpoint{6.572727in}{0.474100in}}{\pgfqpoint{4.227273in}{3.318700in}}%
\pgfusepath{clip}%
\pgfsetbuttcap%
\pgfsetroundjoin%
\definecolor{currentfill}{rgb}{0.127568,0.566949,0.550556}%
\pgfsetfillcolor{currentfill}%
\pgfsetfillopacity{0.700000}%
\pgfsetlinewidth{0.000000pt}%
\definecolor{currentstroke}{rgb}{0.000000,0.000000,0.000000}%
\pgfsetstrokecolor{currentstroke}%
\pgfsetstrokeopacity{0.700000}%
\pgfsetdash{}{0pt}%
\pgfpathmoveto{\pgfqpoint{7.366921in}{1.149224in}}%
\pgfpathcurveto{\pgfqpoint{7.371964in}{1.149224in}}{\pgfqpoint{7.376802in}{1.151228in}}{\pgfqpoint{7.380369in}{1.154794in}}%
\pgfpathcurveto{\pgfqpoint{7.383935in}{1.158360in}}{\pgfqpoint{7.385939in}{1.163198in}}{\pgfqpoint{7.385939in}{1.168242in}}%
\pgfpathcurveto{\pgfqpoint{7.385939in}{1.173286in}}{\pgfqpoint{7.383935in}{1.178123in}}{\pgfqpoint{7.380369in}{1.181690in}}%
\pgfpathcurveto{\pgfqpoint{7.376802in}{1.185256in}}{\pgfqpoint{7.371964in}{1.187260in}}{\pgfqpoint{7.366921in}{1.187260in}}%
\pgfpathcurveto{\pgfqpoint{7.361877in}{1.187260in}}{\pgfqpoint{7.357039in}{1.185256in}}{\pgfqpoint{7.353473in}{1.181690in}}%
\pgfpathcurveto{\pgfqpoint{7.349907in}{1.178123in}}{\pgfqpoint{7.347903in}{1.173286in}}{\pgfqpoint{7.347903in}{1.168242in}}%
\pgfpathcurveto{\pgfqpoint{7.347903in}{1.163198in}}{\pgfqpoint{7.349907in}{1.158360in}}{\pgfqpoint{7.353473in}{1.154794in}}%
\pgfpathcurveto{\pgfqpoint{7.357039in}{1.151228in}}{\pgfqpoint{7.361877in}{1.149224in}}{\pgfqpoint{7.366921in}{1.149224in}}%
\pgfpathclose%
\pgfusepath{fill}%
\end{pgfscope}%
\begin{pgfscope}%
\pgfpathrectangle{\pgfqpoint{6.572727in}{0.474100in}}{\pgfqpoint{4.227273in}{3.318700in}}%
\pgfusepath{clip}%
\pgfsetbuttcap%
\pgfsetroundjoin%
\definecolor{currentfill}{rgb}{0.993248,0.906157,0.143936}%
\pgfsetfillcolor{currentfill}%
\pgfsetfillopacity{0.700000}%
\pgfsetlinewidth{0.000000pt}%
\definecolor{currentstroke}{rgb}{0.000000,0.000000,0.000000}%
\pgfsetstrokecolor{currentstroke}%
\pgfsetstrokeopacity{0.700000}%
\pgfsetdash{}{0pt}%
\pgfpathmoveto{\pgfqpoint{9.702031in}{1.746925in}}%
\pgfpathcurveto{\pgfqpoint{9.707075in}{1.746925in}}{\pgfqpoint{9.711913in}{1.748929in}}{\pgfqpoint{9.715479in}{1.752496in}}%
\pgfpathcurveto{\pgfqpoint{9.719046in}{1.756062in}}{\pgfqpoint{9.721049in}{1.760900in}}{\pgfqpoint{9.721049in}{1.765944in}}%
\pgfpathcurveto{\pgfqpoint{9.721049in}{1.770987in}}{\pgfqpoint{9.719046in}{1.775825in}}{\pgfqpoint{9.715479in}{1.779392in}}%
\pgfpathcurveto{\pgfqpoint{9.711913in}{1.782958in}}{\pgfqpoint{9.707075in}{1.784962in}}{\pgfqpoint{9.702031in}{1.784962in}}%
\pgfpathcurveto{\pgfqpoint{9.696988in}{1.784962in}}{\pgfqpoint{9.692150in}{1.782958in}}{\pgfqpoint{9.688583in}{1.779392in}}%
\pgfpathcurveto{\pgfqpoint{9.685017in}{1.775825in}}{\pgfqpoint{9.683013in}{1.770987in}}{\pgfqpoint{9.683013in}{1.765944in}}%
\pgfpathcurveto{\pgfqpoint{9.683013in}{1.760900in}}{\pgfqpoint{9.685017in}{1.756062in}}{\pgfqpoint{9.688583in}{1.752496in}}%
\pgfpathcurveto{\pgfqpoint{9.692150in}{1.748929in}}{\pgfqpoint{9.696988in}{1.746925in}}{\pgfqpoint{9.702031in}{1.746925in}}%
\pgfpathclose%
\pgfusepath{fill}%
\end{pgfscope}%
\begin{pgfscope}%
\pgfpathrectangle{\pgfqpoint{6.572727in}{0.474100in}}{\pgfqpoint{4.227273in}{3.318700in}}%
\pgfusepath{clip}%
\pgfsetbuttcap%
\pgfsetroundjoin%
\definecolor{currentfill}{rgb}{0.993248,0.906157,0.143936}%
\pgfsetfillcolor{currentfill}%
\pgfsetfillopacity{0.700000}%
\pgfsetlinewidth{0.000000pt}%
\definecolor{currentstroke}{rgb}{0.000000,0.000000,0.000000}%
\pgfsetstrokecolor{currentstroke}%
\pgfsetstrokeopacity{0.700000}%
\pgfsetdash{}{0pt}%
\pgfpathmoveto{\pgfqpoint{9.625621in}{1.853849in}}%
\pgfpathcurveto{\pgfqpoint{9.630665in}{1.853849in}}{\pgfqpoint{9.635503in}{1.855853in}}{\pgfqpoint{9.639069in}{1.859419in}}%
\pgfpathcurveto{\pgfqpoint{9.642635in}{1.862986in}}{\pgfqpoint{9.644639in}{1.867824in}}{\pgfqpoint{9.644639in}{1.872867in}}%
\pgfpathcurveto{\pgfqpoint{9.644639in}{1.877911in}}{\pgfqpoint{9.642635in}{1.882749in}}{\pgfqpoint{9.639069in}{1.886315in}}%
\pgfpathcurveto{\pgfqpoint{9.635503in}{1.889882in}}{\pgfqpoint{9.630665in}{1.891885in}}{\pgfqpoint{9.625621in}{1.891885in}}%
\pgfpathcurveto{\pgfqpoint{9.620577in}{1.891885in}}{\pgfqpoint{9.615740in}{1.889882in}}{\pgfqpoint{9.612173in}{1.886315in}}%
\pgfpathcurveto{\pgfqpoint{9.608607in}{1.882749in}}{\pgfqpoint{9.606603in}{1.877911in}}{\pgfqpoint{9.606603in}{1.872867in}}%
\pgfpathcurveto{\pgfqpoint{9.606603in}{1.867824in}}{\pgfqpoint{9.608607in}{1.862986in}}{\pgfqpoint{9.612173in}{1.859419in}}%
\pgfpathcurveto{\pgfqpoint{9.615740in}{1.855853in}}{\pgfqpoint{9.620577in}{1.853849in}}{\pgfqpoint{9.625621in}{1.853849in}}%
\pgfpathclose%
\pgfusepath{fill}%
\end{pgfscope}%
\begin{pgfscope}%
\pgfpathrectangle{\pgfqpoint{6.572727in}{0.474100in}}{\pgfqpoint{4.227273in}{3.318700in}}%
\pgfusepath{clip}%
\pgfsetbuttcap%
\pgfsetroundjoin%
\definecolor{currentfill}{rgb}{0.127568,0.566949,0.550556}%
\pgfsetfillcolor{currentfill}%
\pgfsetfillopacity{0.700000}%
\pgfsetlinewidth{0.000000pt}%
\definecolor{currentstroke}{rgb}{0.000000,0.000000,0.000000}%
\pgfsetstrokecolor{currentstroke}%
\pgfsetstrokeopacity{0.700000}%
\pgfsetdash{}{0pt}%
\pgfpathmoveto{\pgfqpoint{7.811504in}{3.250368in}}%
\pgfpathcurveto{\pgfqpoint{7.816548in}{3.250368in}}{\pgfqpoint{7.821386in}{3.252371in}}{\pgfqpoint{7.824952in}{3.255938in}}%
\pgfpathcurveto{\pgfqpoint{7.828519in}{3.259504in}}{\pgfqpoint{7.830523in}{3.264342in}}{\pgfqpoint{7.830523in}{3.269386in}}%
\pgfpathcurveto{\pgfqpoint{7.830523in}{3.274429in}}{\pgfqpoint{7.828519in}{3.279267in}}{\pgfqpoint{7.824952in}{3.282834in}}%
\pgfpathcurveto{\pgfqpoint{7.821386in}{3.286400in}}{\pgfqpoint{7.816548in}{3.288404in}}{\pgfqpoint{7.811504in}{3.288404in}}%
\pgfpathcurveto{\pgfqpoint{7.806461in}{3.288404in}}{\pgfqpoint{7.801623in}{3.286400in}}{\pgfqpoint{7.798057in}{3.282834in}}%
\pgfpathcurveto{\pgfqpoint{7.794490in}{3.279267in}}{\pgfqpoint{7.792486in}{3.274429in}}{\pgfqpoint{7.792486in}{3.269386in}}%
\pgfpathcurveto{\pgfqpoint{7.792486in}{3.264342in}}{\pgfqpoint{7.794490in}{3.259504in}}{\pgfqpoint{7.798057in}{3.255938in}}%
\pgfpathcurveto{\pgfqpoint{7.801623in}{3.252371in}}{\pgfqpoint{7.806461in}{3.250368in}}{\pgfqpoint{7.811504in}{3.250368in}}%
\pgfpathclose%
\pgfusepath{fill}%
\end{pgfscope}%
\begin{pgfscope}%
\pgfpathrectangle{\pgfqpoint{6.572727in}{0.474100in}}{\pgfqpoint{4.227273in}{3.318700in}}%
\pgfusepath{clip}%
\pgfsetbuttcap%
\pgfsetroundjoin%
\definecolor{currentfill}{rgb}{0.127568,0.566949,0.550556}%
\pgfsetfillcolor{currentfill}%
\pgfsetfillopacity{0.700000}%
\pgfsetlinewidth{0.000000pt}%
\definecolor{currentstroke}{rgb}{0.000000,0.000000,0.000000}%
\pgfsetstrokecolor{currentstroke}%
\pgfsetstrokeopacity{0.700000}%
\pgfsetdash{}{0pt}%
\pgfpathmoveto{\pgfqpoint{7.262638in}{2.201074in}}%
\pgfpathcurveto{\pgfqpoint{7.267681in}{2.201074in}}{\pgfqpoint{7.272519in}{2.203078in}}{\pgfqpoint{7.276086in}{2.206644in}}%
\pgfpathcurveto{\pgfqpoint{7.279652in}{2.210210in}}{\pgfqpoint{7.281656in}{2.215048in}}{\pgfqpoint{7.281656in}{2.220092in}}%
\pgfpathcurveto{\pgfqpoint{7.281656in}{2.225136in}}{\pgfqpoint{7.279652in}{2.229973in}}{\pgfqpoint{7.276086in}{2.233540in}}%
\pgfpathcurveto{\pgfqpoint{7.272519in}{2.237106in}}{\pgfqpoint{7.267681in}{2.239110in}}{\pgfqpoint{7.262638in}{2.239110in}}%
\pgfpathcurveto{\pgfqpoint{7.257594in}{2.239110in}}{\pgfqpoint{7.252756in}{2.237106in}}{\pgfqpoint{7.249190in}{2.233540in}}%
\pgfpathcurveto{\pgfqpoint{7.245623in}{2.229973in}}{\pgfqpoint{7.243620in}{2.225136in}}{\pgfqpoint{7.243620in}{2.220092in}}%
\pgfpathcurveto{\pgfqpoint{7.243620in}{2.215048in}}{\pgfqpoint{7.245623in}{2.210210in}}{\pgfqpoint{7.249190in}{2.206644in}}%
\pgfpathcurveto{\pgfqpoint{7.252756in}{2.203078in}}{\pgfqpoint{7.257594in}{2.201074in}}{\pgfqpoint{7.262638in}{2.201074in}}%
\pgfpathclose%
\pgfusepath{fill}%
\end{pgfscope}%
\begin{pgfscope}%
\pgfpathrectangle{\pgfqpoint{6.572727in}{0.474100in}}{\pgfqpoint{4.227273in}{3.318700in}}%
\pgfusepath{clip}%
\pgfsetbuttcap%
\pgfsetroundjoin%
\definecolor{currentfill}{rgb}{0.993248,0.906157,0.143936}%
\pgfsetfillcolor{currentfill}%
\pgfsetfillopacity{0.700000}%
\pgfsetlinewidth{0.000000pt}%
\definecolor{currentstroke}{rgb}{0.000000,0.000000,0.000000}%
\pgfsetstrokecolor{currentstroke}%
\pgfsetstrokeopacity{0.700000}%
\pgfsetdash{}{0pt}%
\pgfpathmoveto{\pgfqpoint{9.632931in}{2.304924in}}%
\pgfpathcurveto{\pgfqpoint{9.637975in}{2.304924in}}{\pgfqpoint{9.642812in}{2.306928in}}{\pgfqpoint{9.646379in}{2.310494in}}%
\pgfpathcurveto{\pgfqpoint{9.649945in}{2.314060in}}{\pgfqpoint{9.651949in}{2.318898in}}{\pgfqpoint{9.651949in}{2.323942in}}%
\pgfpathcurveto{\pgfqpoint{9.651949in}{2.328986in}}{\pgfqpoint{9.649945in}{2.333823in}}{\pgfqpoint{9.646379in}{2.337390in}}%
\pgfpathcurveto{\pgfqpoint{9.642812in}{2.340956in}}{\pgfqpoint{9.637975in}{2.342960in}}{\pgfqpoint{9.632931in}{2.342960in}}%
\pgfpathcurveto{\pgfqpoint{9.627887in}{2.342960in}}{\pgfqpoint{9.623050in}{2.340956in}}{\pgfqpoint{9.619483in}{2.337390in}}%
\pgfpathcurveto{\pgfqpoint{9.615917in}{2.333823in}}{\pgfqpoint{9.613913in}{2.328986in}}{\pgfqpoint{9.613913in}{2.323942in}}%
\pgfpathcurveto{\pgfqpoint{9.613913in}{2.318898in}}{\pgfqpoint{9.615917in}{2.314060in}}{\pgfqpoint{9.619483in}{2.310494in}}%
\pgfpathcurveto{\pgfqpoint{9.623050in}{2.306928in}}{\pgfqpoint{9.627887in}{2.304924in}}{\pgfqpoint{9.632931in}{2.304924in}}%
\pgfpathclose%
\pgfusepath{fill}%
\end{pgfscope}%
\begin{pgfscope}%
\pgfpathrectangle{\pgfqpoint{6.572727in}{0.474100in}}{\pgfqpoint{4.227273in}{3.318700in}}%
\pgfusepath{clip}%
\pgfsetbuttcap%
\pgfsetroundjoin%
\definecolor{currentfill}{rgb}{0.993248,0.906157,0.143936}%
\pgfsetfillcolor{currentfill}%
\pgfsetfillopacity{0.700000}%
\pgfsetlinewidth{0.000000pt}%
\definecolor{currentstroke}{rgb}{0.000000,0.000000,0.000000}%
\pgfsetstrokecolor{currentstroke}%
\pgfsetstrokeopacity{0.700000}%
\pgfsetdash{}{0pt}%
\pgfpathmoveto{\pgfqpoint{9.032504in}{1.096022in}}%
\pgfpathcurveto{\pgfqpoint{9.037547in}{1.096022in}}{\pgfqpoint{9.042385in}{1.098026in}}{\pgfqpoint{9.045951in}{1.101593in}}%
\pgfpathcurveto{\pgfqpoint{9.049518in}{1.105159in}}{\pgfqpoint{9.051522in}{1.109997in}}{\pgfqpoint{9.051522in}{1.115041in}}%
\pgfpathcurveto{\pgfqpoint{9.051522in}{1.120084in}}{\pgfqpoint{9.049518in}{1.124922in}}{\pgfqpoint{9.045951in}{1.128488in}}%
\pgfpathcurveto{\pgfqpoint{9.042385in}{1.132055in}}{\pgfqpoint{9.037547in}{1.134059in}}{\pgfqpoint{9.032504in}{1.134059in}}%
\pgfpathcurveto{\pgfqpoint{9.027460in}{1.134059in}}{\pgfqpoint{9.022622in}{1.132055in}}{\pgfqpoint{9.019056in}{1.128488in}}%
\pgfpathcurveto{\pgfqpoint{9.015489in}{1.124922in}}{\pgfqpoint{9.013485in}{1.120084in}}{\pgfqpoint{9.013485in}{1.115041in}}%
\pgfpathcurveto{\pgfqpoint{9.013485in}{1.109997in}}{\pgfqpoint{9.015489in}{1.105159in}}{\pgfqpoint{9.019056in}{1.101593in}}%
\pgfpathcurveto{\pgfqpoint{9.022622in}{1.098026in}}{\pgfqpoint{9.027460in}{1.096022in}}{\pgfqpoint{9.032504in}{1.096022in}}%
\pgfpathclose%
\pgfusepath{fill}%
\end{pgfscope}%
\begin{pgfscope}%
\pgfpathrectangle{\pgfqpoint{6.572727in}{0.474100in}}{\pgfqpoint{4.227273in}{3.318700in}}%
\pgfusepath{clip}%
\pgfsetbuttcap%
\pgfsetroundjoin%
\definecolor{currentfill}{rgb}{0.127568,0.566949,0.550556}%
\pgfsetfillcolor{currentfill}%
\pgfsetfillopacity{0.700000}%
\pgfsetlinewidth{0.000000pt}%
\definecolor{currentstroke}{rgb}{0.000000,0.000000,0.000000}%
\pgfsetstrokecolor{currentstroke}%
\pgfsetstrokeopacity{0.700000}%
\pgfsetdash{}{0pt}%
\pgfpathmoveto{\pgfqpoint{8.096654in}{3.223882in}}%
\pgfpathcurveto{\pgfqpoint{8.101698in}{3.223882in}}{\pgfqpoint{8.106536in}{3.225886in}}{\pgfqpoint{8.110102in}{3.229452in}}%
\pgfpathcurveto{\pgfqpoint{8.113668in}{3.233019in}}{\pgfqpoint{8.115672in}{3.237856in}}{\pgfqpoint{8.115672in}{3.242900in}}%
\pgfpathcurveto{\pgfqpoint{8.115672in}{3.247944in}}{\pgfqpoint{8.113668in}{3.252781in}}{\pgfqpoint{8.110102in}{3.256348in}}%
\pgfpathcurveto{\pgfqpoint{8.106536in}{3.259914in}}{\pgfqpoint{8.101698in}{3.261918in}}{\pgfqpoint{8.096654in}{3.261918in}}%
\pgfpathcurveto{\pgfqpoint{8.091610in}{3.261918in}}{\pgfqpoint{8.086773in}{3.259914in}}{\pgfqpoint{8.083206in}{3.256348in}}%
\pgfpathcurveto{\pgfqpoint{8.079640in}{3.252781in}}{\pgfqpoint{8.077636in}{3.247944in}}{\pgfqpoint{8.077636in}{3.242900in}}%
\pgfpathcurveto{\pgfqpoint{8.077636in}{3.237856in}}{\pgfqpoint{8.079640in}{3.233019in}}{\pgfqpoint{8.083206in}{3.229452in}}%
\pgfpathcurveto{\pgfqpoint{8.086773in}{3.225886in}}{\pgfqpoint{8.091610in}{3.223882in}}{\pgfqpoint{8.096654in}{3.223882in}}%
\pgfpathclose%
\pgfusepath{fill}%
\end{pgfscope}%
\begin{pgfscope}%
\pgfpathrectangle{\pgfqpoint{6.572727in}{0.474100in}}{\pgfqpoint{4.227273in}{3.318700in}}%
\pgfusepath{clip}%
\pgfsetbuttcap%
\pgfsetroundjoin%
\definecolor{currentfill}{rgb}{0.127568,0.566949,0.550556}%
\pgfsetfillcolor{currentfill}%
\pgfsetfillopacity{0.700000}%
\pgfsetlinewidth{0.000000pt}%
\definecolor{currentstroke}{rgb}{0.000000,0.000000,0.000000}%
\pgfsetstrokecolor{currentstroke}%
\pgfsetstrokeopacity{0.700000}%
\pgfsetdash{}{0pt}%
\pgfpathmoveto{\pgfqpoint{8.467173in}{3.114265in}}%
\pgfpathcurveto{\pgfqpoint{8.472217in}{3.114265in}}{\pgfqpoint{8.477054in}{3.116269in}}{\pgfqpoint{8.480621in}{3.119835in}}%
\pgfpathcurveto{\pgfqpoint{8.484187in}{3.123402in}}{\pgfqpoint{8.486191in}{3.128239in}}{\pgfqpoint{8.486191in}{3.133283in}}%
\pgfpathcurveto{\pgfqpoint{8.486191in}{3.138327in}}{\pgfqpoint{8.484187in}{3.143165in}}{\pgfqpoint{8.480621in}{3.146731in}}%
\pgfpathcurveto{\pgfqpoint{8.477054in}{3.150297in}}{\pgfqpoint{8.472217in}{3.152301in}}{\pgfqpoint{8.467173in}{3.152301in}}%
\pgfpathcurveto{\pgfqpoint{8.462129in}{3.152301in}}{\pgfqpoint{8.457292in}{3.150297in}}{\pgfqpoint{8.453725in}{3.146731in}}%
\pgfpathcurveto{\pgfqpoint{8.450159in}{3.143165in}}{\pgfqpoint{8.448155in}{3.138327in}}{\pgfqpoint{8.448155in}{3.133283in}}%
\pgfpathcurveto{\pgfqpoint{8.448155in}{3.128239in}}{\pgfqpoint{8.450159in}{3.123402in}}{\pgfqpoint{8.453725in}{3.119835in}}%
\pgfpathcurveto{\pgfqpoint{8.457292in}{3.116269in}}{\pgfqpoint{8.462129in}{3.114265in}}{\pgfqpoint{8.467173in}{3.114265in}}%
\pgfpathclose%
\pgfusepath{fill}%
\end{pgfscope}%
\begin{pgfscope}%
\pgfpathrectangle{\pgfqpoint{6.572727in}{0.474100in}}{\pgfqpoint{4.227273in}{3.318700in}}%
\pgfusepath{clip}%
\pgfsetbuttcap%
\pgfsetroundjoin%
\definecolor{currentfill}{rgb}{0.993248,0.906157,0.143936}%
\pgfsetfillcolor{currentfill}%
\pgfsetfillopacity{0.700000}%
\pgfsetlinewidth{0.000000pt}%
\definecolor{currentstroke}{rgb}{0.000000,0.000000,0.000000}%
\pgfsetstrokecolor{currentstroke}%
\pgfsetstrokeopacity{0.700000}%
\pgfsetdash{}{0pt}%
\pgfpathmoveto{\pgfqpoint{9.584894in}{1.203887in}}%
\pgfpathcurveto{\pgfqpoint{9.589938in}{1.203887in}}{\pgfqpoint{9.594776in}{1.205891in}}{\pgfqpoint{9.598342in}{1.209457in}}%
\pgfpathcurveto{\pgfqpoint{9.601909in}{1.213024in}}{\pgfqpoint{9.603913in}{1.217861in}}{\pgfqpoint{9.603913in}{1.222905in}}%
\pgfpathcurveto{\pgfqpoint{9.603913in}{1.227949in}}{\pgfqpoint{9.601909in}{1.232786in}}{\pgfqpoint{9.598342in}{1.236353in}}%
\pgfpathcurveto{\pgfqpoint{9.594776in}{1.239919in}}{\pgfqpoint{9.589938in}{1.241923in}}{\pgfqpoint{9.584894in}{1.241923in}}%
\pgfpathcurveto{\pgfqpoint{9.579851in}{1.241923in}}{\pgfqpoint{9.575013in}{1.239919in}}{\pgfqpoint{9.571447in}{1.236353in}}%
\pgfpathcurveto{\pgfqpoint{9.567880in}{1.232786in}}{\pgfqpoint{9.565876in}{1.227949in}}{\pgfqpoint{9.565876in}{1.222905in}}%
\pgfpathcurveto{\pgfqpoint{9.565876in}{1.217861in}}{\pgfqpoint{9.567880in}{1.213024in}}{\pgfqpoint{9.571447in}{1.209457in}}%
\pgfpathcurveto{\pgfqpoint{9.575013in}{1.205891in}}{\pgfqpoint{9.579851in}{1.203887in}}{\pgfqpoint{9.584894in}{1.203887in}}%
\pgfpathclose%
\pgfusepath{fill}%
\end{pgfscope}%
\begin{pgfscope}%
\pgfpathrectangle{\pgfqpoint{6.572727in}{0.474100in}}{\pgfqpoint{4.227273in}{3.318700in}}%
\pgfusepath{clip}%
\pgfsetbuttcap%
\pgfsetroundjoin%
\definecolor{currentfill}{rgb}{0.993248,0.906157,0.143936}%
\pgfsetfillcolor{currentfill}%
\pgfsetfillopacity{0.700000}%
\pgfsetlinewidth{0.000000pt}%
\definecolor{currentstroke}{rgb}{0.000000,0.000000,0.000000}%
\pgfsetstrokecolor{currentstroke}%
\pgfsetstrokeopacity{0.700000}%
\pgfsetdash{}{0pt}%
\pgfpathmoveto{\pgfqpoint{9.776371in}{1.543034in}}%
\pgfpathcurveto{\pgfqpoint{9.781414in}{1.543034in}}{\pgfqpoint{9.786252in}{1.545038in}}{\pgfqpoint{9.789819in}{1.548604in}}%
\pgfpathcurveto{\pgfqpoint{9.793385in}{1.552171in}}{\pgfqpoint{9.795389in}{1.557008in}}{\pgfqpoint{9.795389in}{1.562052in}}%
\pgfpathcurveto{\pgfqpoint{9.795389in}{1.567096in}}{\pgfqpoint{9.793385in}{1.571933in}}{\pgfqpoint{9.789819in}{1.575500in}}%
\pgfpathcurveto{\pgfqpoint{9.786252in}{1.579066in}}{\pgfqpoint{9.781414in}{1.581070in}}{\pgfqpoint{9.776371in}{1.581070in}}%
\pgfpathcurveto{\pgfqpoint{9.771327in}{1.581070in}}{\pgfqpoint{9.766489in}{1.579066in}}{\pgfqpoint{9.762923in}{1.575500in}}%
\pgfpathcurveto{\pgfqpoint{9.759356in}{1.571933in}}{\pgfqpoint{9.757352in}{1.567096in}}{\pgfqpoint{9.757352in}{1.562052in}}%
\pgfpathcurveto{\pgfqpoint{9.757352in}{1.557008in}}{\pgfqpoint{9.759356in}{1.552171in}}{\pgfqpoint{9.762923in}{1.548604in}}%
\pgfpathcurveto{\pgfqpoint{9.766489in}{1.545038in}}{\pgfqpoint{9.771327in}{1.543034in}}{\pgfqpoint{9.776371in}{1.543034in}}%
\pgfpathclose%
\pgfusepath{fill}%
\end{pgfscope}%
\begin{pgfscope}%
\pgfpathrectangle{\pgfqpoint{6.572727in}{0.474100in}}{\pgfqpoint{4.227273in}{3.318700in}}%
\pgfusepath{clip}%
\pgfsetbuttcap%
\pgfsetroundjoin%
\definecolor{currentfill}{rgb}{0.127568,0.566949,0.550556}%
\pgfsetfillcolor{currentfill}%
\pgfsetfillopacity{0.700000}%
\pgfsetlinewidth{0.000000pt}%
\definecolor{currentstroke}{rgb}{0.000000,0.000000,0.000000}%
\pgfsetstrokecolor{currentstroke}%
\pgfsetstrokeopacity{0.700000}%
\pgfsetdash{}{0pt}%
\pgfpathmoveto{\pgfqpoint{8.333781in}{2.515937in}}%
\pgfpathcurveto{\pgfqpoint{8.338824in}{2.515937in}}{\pgfqpoint{8.343662in}{2.517941in}}{\pgfqpoint{8.347229in}{2.521507in}}%
\pgfpathcurveto{\pgfqpoint{8.350795in}{2.525074in}}{\pgfqpoint{8.352799in}{2.529911in}}{\pgfqpoint{8.352799in}{2.534955in}}%
\pgfpathcurveto{\pgfqpoint{8.352799in}{2.539999in}}{\pgfqpoint{8.350795in}{2.544837in}}{\pgfqpoint{8.347229in}{2.548403in}}%
\pgfpathcurveto{\pgfqpoint{8.343662in}{2.551969in}}{\pgfqpoint{8.338824in}{2.553973in}}{\pgfqpoint{8.333781in}{2.553973in}}%
\pgfpathcurveto{\pgfqpoint{8.328737in}{2.553973in}}{\pgfqpoint{8.323899in}{2.551969in}}{\pgfqpoint{8.320333in}{2.548403in}}%
\pgfpathcurveto{\pgfqpoint{8.316766in}{2.544837in}}{\pgfqpoint{8.314763in}{2.539999in}}{\pgfqpoint{8.314763in}{2.534955in}}%
\pgfpathcurveto{\pgfqpoint{8.314763in}{2.529911in}}{\pgfqpoint{8.316766in}{2.525074in}}{\pgfqpoint{8.320333in}{2.521507in}}%
\pgfpathcurveto{\pgfqpoint{8.323899in}{2.517941in}}{\pgfqpoint{8.328737in}{2.515937in}}{\pgfqpoint{8.333781in}{2.515937in}}%
\pgfpathclose%
\pgfusepath{fill}%
\end{pgfscope}%
\begin{pgfscope}%
\pgfpathrectangle{\pgfqpoint{6.572727in}{0.474100in}}{\pgfqpoint{4.227273in}{3.318700in}}%
\pgfusepath{clip}%
\pgfsetbuttcap%
\pgfsetroundjoin%
\definecolor{currentfill}{rgb}{0.993248,0.906157,0.143936}%
\pgfsetfillcolor{currentfill}%
\pgfsetfillopacity{0.700000}%
\pgfsetlinewidth{0.000000pt}%
\definecolor{currentstroke}{rgb}{0.000000,0.000000,0.000000}%
\pgfsetstrokecolor{currentstroke}%
\pgfsetstrokeopacity{0.700000}%
\pgfsetdash{}{0pt}%
\pgfpathmoveto{\pgfqpoint{9.943738in}{1.958765in}}%
\pgfpathcurveto{\pgfqpoint{9.948782in}{1.958765in}}{\pgfqpoint{9.953619in}{1.960769in}}{\pgfqpoint{9.957186in}{1.964336in}}%
\pgfpathcurveto{\pgfqpoint{9.960752in}{1.967902in}}{\pgfqpoint{9.962756in}{1.972740in}}{\pgfqpoint{9.962756in}{1.977784in}}%
\pgfpathcurveto{\pgfqpoint{9.962756in}{1.982827in}}{\pgfqpoint{9.960752in}{1.987665in}}{\pgfqpoint{9.957186in}{1.991231in}}%
\pgfpathcurveto{\pgfqpoint{9.953619in}{1.994798in}}{\pgfqpoint{9.948782in}{1.996802in}}{\pgfqpoint{9.943738in}{1.996802in}}%
\pgfpathcurveto{\pgfqpoint{9.938694in}{1.996802in}}{\pgfqpoint{9.933856in}{1.994798in}}{\pgfqpoint{9.930290in}{1.991231in}}%
\pgfpathcurveto{\pgfqpoint{9.926724in}{1.987665in}}{\pgfqpoint{9.924720in}{1.982827in}}{\pgfqpoint{9.924720in}{1.977784in}}%
\pgfpathcurveto{\pgfqpoint{9.924720in}{1.972740in}}{\pgfqpoint{9.926724in}{1.967902in}}{\pgfqpoint{9.930290in}{1.964336in}}%
\pgfpathcurveto{\pgfqpoint{9.933856in}{1.960769in}}{\pgfqpoint{9.938694in}{1.958765in}}{\pgfqpoint{9.943738in}{1.958765in}}%
\pgfpathclose%
\pgfusepath{fill}%
\end{pgfscope}%
\begin{pgfscope}%
\pgfpathrectangle{\pgfqpoint{6.572727in}{0.474100in}}{\pgfqpoint{4.227273in}{3.318700in}}%
\pgfusepath{clip}%
\pgfsetbuttcap%
\pgfsetroundjoin%
\definecolor{currentfill}{rgb}{0.993248,0.906157,0.143936}%
\pgfsetfillcolor{currentfill}%
\pgfsetfillopacity{0.700000}%
\pgfsetlinewidth{0.000000pt}%
\definecolor{currentstroke}{rgb}{0.000000,0.000000,0.000000}%
\pgfsetstrokecolor{currentstroke}%
\pgfsetstrokeopacity{0.700000}%
\pgfsetdash{}{0pt}%
\pgfpathmoveto{\pgfqpoint{9.140389in}{2.030908in}}%
\pgfpathcurveto{\pgfqpoint{9.145433in}{2.030908in}}{\pgfqpoint{9.150271in}{2.032912in}}{\pgfqpoint{9.153837in}{2.036479in}}%
\pgfpathcurveto{\pgfqpoint{9.157403in}{2.040045in}}{\pgfqpoint{9.159407in}{2.044883in}}{\pgfqpoint{9.159407in}{2.049927in}}%
\pgfpathcurveto{\pgfqpoint{9.159407in}{2.054970in}}{\pgfqpoint{9.157403in}{2.059808in}}{\pgfqpoint{9.153837in}{2.063374in}}%
\pgfpathcurveto{\pgfqpoint{9.150271in}{2.066941in}}{\pgfqpoint{9.145433in}{2.068945in}}{\pgfqpoint{9.140389in}{2.068945in}}%
\pgfpathcurveto{\pgfqpoint{9.135345in}{2.068945in}}{\pgfqpoint{9.130508in}{2.066941in}}{\pgfqpoint{9.126941in}{2.063374in}}%
\pgfpathcurveto{\pgfqpoint{9.123375in}{2.059808in}}{\pgfqpoint{9.121371in}{2.054970in}}{\pgfqpoint{9.121371in}{2.049927in}}%
\pgfpathcurveto{\pgfqpoint{9.121371in}{2.044883in}}{\pgfqpoint{9.123375in}{2.040045in}}{\pgfqpoint{9.126941in}{2.036479in}}%
\pgfpathcurveto{\pgfqpoint{9.130508in}{2.032912in}}{\pgfqpoint{9.135345in}{2.030908in}}{\pgfqpoint{9.140389in}{2.030908in}}%
\pgfpathclose%
\pgfusepath{fill}%
\end{pgfscope}%
\begin{pgfscope}%
\pgfpathrectangle{\pgfqpoint{6.572727in}{0.474100in}}{\pgfqpoint{4.227273in}{3.318700in}}%
\pgfusepath{clip}%
\pgfsetbuttcap%
\pgfsetroundjoin%
\definecolor{currentfill}{rgb}{0.127568,0.566949,0.550556}%
\pgfsetfillcolor{currentfill}%
\pgfsetfillopacity{0.700000}%
\pgfsetlinewidth{0.000000pt}%
\definecolor{currentstroke}{rgb}{0.000000,0.000000,0.000000}%
\pgfsetstrokecolor{currentstroke}%
\pgfsetstrokeopacity{0.700000}%
\pgfsetdash{}{0pt}%
\pgfpathmoveto{\pgfqpoint{7.638007in}{1.450268in}}%
\pgfpathcurveto{\pgfqpoint{7.643051in}{1.450268in}}{\pgfqpoint{7.647889in}{1.452272in}}{\pgfqpoint{7.651455in}{1.455839in}}%
\pgfpathcurveto{\pgfqpoint{7.655021in}{1.459405in}}{\pgfqpoint{7.657025in}{1.464243in}}{\pgfqpoint{7.657025in}{1.469286in}}%
\pgfpathcurveto{\pgfqpoint{7.657025in}{1.474330in}}{\pgfqpoint{7.655021in}{1.479168in}}{\pgfqpoint{7.651455in}{1.482734in}}%
\pgfpathcurveto{\pgfqpoint{7.647889in}{1.486301in}}{\pgfqpoint{7.643051in}{1.488305in}}{\pgfqpoint{7.638007in}{1.488305in}}%
\pgfpathcurveto{\pgfqpoint{7.632964in}{1.488305in}}{\pgfqpoint{7.628126in}{1.486301in}}{\pgfqpoint{7.624559in}{1.482734in}}%
\pgfpathcurveto{\pgfqpoint{7.620993in}{1.479168in}}{\pgfqpoint{7.618989in}{1.474330in}}{\pgfqpoint{7.618989in}{1.469286in}}%
\pgfpathcurveto{\pgfqpoint{7.618989in}{1.464243in}}{\pgfqpoint{7.620993in}{1.459405in}}{\pgfqpoint{7.624559in}{1.455839in}}%
\pgfpathcurveto{\pgfqpoint{7.628126in}{1.452272in}}{\pgfqpoint{7.632964in}{1.450268in}}{\pgfqpoint{7.638007in}{1.450268in}}%
\pgfpathclose%
\pgfusepath{fill}%
\end{pgfscope}%
\begin{pgfscope}%
\pgfpathrectangle{\pgfqpoint{6.572727in}{0.474100in}}{\pgfqpoint{4.227273in}{3.318700in}}%
\pgfusepath{clip}%
\pgfsetbuttcap%
\pgfsetroundjoin%
\definecolor{currentfill}{rgb}{0.993248,0.906157,0.143936}%
\pgfsetfillcolor{currentfill}%
\pgfsetfillopacity{0.700000}%
\pgfsetlinewidth{0.000000pt}%
\definecolor{currentstroke}{rgb}{0.000000,0.000000,0.000000}%
\pgfsetstrokecolor{currentstroke}%
\pgfsetstrokeopacity{0.700000}%
\pgfsetdash{}{0pt}%
\pgfpathmoveto{\pgfqpoint{9.578576in}{1.060389in}}%
\pgfpathcurveto{\pgfqpoint{9.583620in}{1.060389in}}{\pgfqpoint{9.588458in}{1.062393in}}{\pgfqpoint{9.592024in}{1.065959in}}%
\pgfpathcurveto{\pgfqpoint{9.595590in}{1.069526in}}{\pgfqpoint{9.597594in}{1.074363in}}{\pgfqpoint{9.597594in}{1.079407in}}%
\pgfpathcurveto{\pgfqpoint{9.597594in}{1.084451in}}{\pgfqpoint{9.595590in}{1.089289in}}{\pgfqpoint{9.592024in}{1.092855in}}%
\pgfpathcurveto{\pgfqpoint{9.588458in}{1.096421in}}{\pgfqpoint{9.583620in}{1.098425in}}{\pgfqpoint{9.578576in}{1.098425in}}%
\pgfpathcurveto{\pgfqpoint{9.573532in}{1.098425in}}{\pgfqpoint{9.568695in}{1.096421in}}{\pgfqpoint{9.565128in}{1.092855in}}%
\pgfpathcurveto{\pgfqpoint{9.561562in}{1.089289in}}{\pgfqpoint{9.559558in}{1.084451in}}{\pgfqpoint{9.559558in}{1.079407in}}%
\pgfpathcurveto{\pgfqpoint{9.559558in}{1.074363in}}{\pgfqpoint{9.561562in}{1.069526in}}{\pgfqpoint{9.565128in}{1.065959in}}%
\pgfpathcurveto{\pgfqpoint{9.568695in}{1.062393in}}{\pgfqpoint{9.573532in}{1.060389in}}{\pgfqpoint{9.578576in}{1.060389in}}%
\pgfpathclose%
\pgfusepath{fill}%
\end{pgfscope}%
\begin{pgfscope}%
\pgfpathrectangle{\pgfqpoint{6.572727in}{0.474100in}}{\pgfqpoint{4.227273in}{3.318700in}}%
\pgfusepath{clip}%
\pgfsetbuttcap%
\pgfsetroundjoin%
\definecolor{currentfill}{rgb}{0.993248,0.906157,0.143936}%
\pgfsetfillcolor{currentfill}%
\pgfsetfillopacity{0.700000}%
\pgfsetlinewidth{0.000000pt}%
\definecolor{currentstroke}{rgb}{0.000000,0.000000,0.000000}%
\pgfsetstrokecolor{currentstroke}%
\pgfsetstrokeopacity{0.700000}%
\pgfsetdash{}{0pt}%
\pgfpathmoveto{\pgfqpoint{9.907620in}{1.701599in}}%
\pgfpathcurveto{\pgfqpoint{9.912663in}{1.701599in}}{\pgfqpoint{9.917501in}{1.703602in}}{\pgfqpoint{9.921068in}{1.707169in}}%
\pgfpathcurveto{\pgfqpoint{9.924634in}{1.710735in}}{\pgfqpoint{9.926638in}{1.715573in}}{\pgfqpoint{9.926638in}{1.720617in}}%
\pgfpathcurveto{\pgfqpoint{9.926638in}{1.725660in}}{\pgfqpoint{9.924634in}{1.730498in}}{\pgfqpoint{9.921068in}{1.734065in}}%
\pgfpathcurveto{\pgfqpoint{9.917501in}{1.737631in}}{\pgfqpoint{9.912663in}{1.739635in}}{\pgfqpoint{9.907620in}{1.739635in}}%
\pgfpathcurveto{\pgfqpoint{9.902576in}{1.739635in}}{\pgfqpoint{9.897738in}{1.737631in}}{\pgfqpoint{9.894172in}{1.734065in}}%
\pgfpathcurveto{\pgfqpoint{9.890605in}{1.730498in}}{\pgfqpoint{9.888602in}{1.725660in}}{\pgfqpoint{9.888602in}{1.720617in}}%
\pgfpathcurveto{\pgfqpoint{9.888602in}{1.715573in}}{\pgfqpoint{9.890605in}{1.710735in}}{\pgfqpoint{9.894172in}{1.707169in}}%
\pgfpathcurveto{\pgfqpoint{9.897738in}{1.703602in}}{\pgfqpoint{9.902576in}{1.701599in}}{\pgfqpoint{9.907620in}{1.701599in}}%
\pgfpathclose%
\pgfusepath{fill}%
\end{pgfscope}%
\begin{pgfscope}%
\pgfpathrectangle{\pgfqpoint{6.572727in}{0.474100in}}{\pgfqpoint{4.227273in}{3.318700in}}%
\pgfusepath{clip}%
\pgfsetbuttcap%
\pgfsetroundjoin%
\definecolor{currentfill}{rgb}{0.993248,0.906157,0.143936}%
\pgfsetfillcolor{currentfill}%
\pgfsetfillopacity{0.700000}%
\pgfsetlinewidth{0.000000pt}%
\definecolor{currentstroke}{rgb}{0.000000,0.000000,0.000000}%
\pgfsetstrokecolor{currentstroke}%
\pgfsetstrokeopacity{0.700000}%
\pgfsetdash{}{0pt}%
\pgfpathmoveto{\pgfqpoint{9.089290in}{1.684619in}}%
\pgfpathcurveto{\pgfqpoint{9.094334in}{1.684619in}}{\pgfqpoint{9.099172in}{1.686623in}}{\pgfqpoint{9.102738in}{1.690189in}}%
\pgfpathcurveto{\pgfqpoint{9.106305in}{1.693756in}}{\pgfqpoint{9.108308in}{1.698593in}}{\pgfqpoint{9.108308in}{1.703637in}}%
\pgfpathcurveto{\pgfqpoint{9.108308in}{1.708681in}}{\pgfqpoint{9.106305in}{1.713519in}}{\pgfqpoint{9.102738in}{1.717085in}}%
\pgfpathcurveto{\pgfqpoint{9.099172in}{1.720651in}}{\pgfqpoint{9.094334in}{1.722655in}}{\pgfqpoint{9.089290in}{1.722655in}}%
\pgfpathcurveto{\pgfqpoint{9.084247in}{1.722655in}}{\pgfqpoint{9.079409in}{1.720651in}}{\pgfqpoint{9.075842in}{1.717085in}}%
\pgfpathcurveto{\pgfqpoint{9.072276in}{1.713519in}}{\pgfqpoint{9.070272in}{1.708681in}}{\pgfqpoint{9.070272in}{1.703637in}}%
\pgfpathcurveto{\pgfqpoint{9.070272in}{1.698593in}}{\pgfqpoint{9.072276in}{1.693756in}}{\pgfqpoint{9.075842in}{1.690189in}}%
\pgfpathcurveto{\pgfqpoint{9.079409in}{1.686623in}}{\pgfqpoint{9.084247in}{1.684619in}}{\pgfqpoint{9.089290in}{1.684619in}}%
\pgfpathclose%
\pgfusepath{fill}%
\end{pgfscope}%
\begin{pgfscope}%
\pgfpathrectangle{\pgfqpoint{6.572727in}{0.474100in}}{\pgfqpoint{4.227273in}{3.318700in}}%
\pgfusepath{clip}%
\pgfsetbuttcap%
\pgfsetroundjoin%
\definecolor{currentfill}{rgb}{0.267004,0.004874,0.329415}%
\pgfsetfillcolor{currentfill}%
\pgfsetfillopacity{0.700000}%
\pgfsetlinewidth{0.000000pt}%
\definecolor{currentstroke}{rgb}{0.000000,0.000000,0.000000}%
\pgfsetstrokecolor{currentstroke}%
\pgfsetstrokeopacity{0.700000}%
\pgfsetdash{}{0pt}%
\pgfpathmoveto{\pgfqpoint{9.091430in}{2.458496in}}%
\pgfpathcurveto{\pgfqpoint{9.096473in}{2.458496in}}{\pgfqpoint{9.101311in}{2.460500in}}{\pgfqpoint{9.104878in}{2.464066in}}%
\pgfpathcurveto{\pgfqpoint{9.108444in}{2.467633in}}{\pgfqpoint{9.110448in}{2.472470in}}{\pgfqpoint{9.110448in}{2.477514in}}%
\pgfpathcurveto{\pgfqpoint{9.110448in}{2.482558in}}{\pgfqpoint{9.108444in}{2.487395in}}{\pgfqpoint{9.104878in}{2.490962in}}%
\pgfpathcurveto{\pgfqpoint{9.101311in}{2.494528in}}{\pgfqpoint{9.096473in}{2.496532in}}{\pgfqpoint{9.091430in}{2.496532in}}%
\pgfpathcurveto{\pgfqpoint{9.086386in}{2.496532in}}{\pgfqpoint{9.081548in}{2.494528in}}{\pgfqpoint{9.077982in}{2.490962in}}%
\pgfpathcurveto{\pgfqpoint{9.074415in}{2.487395in}}{\pgfqpoint{9.072412in}{2.482558in}}{\pgfqpoint{9.072412in}{2.477514in}}%
\pgfpathcurveto{\pgfqpoint{9.072412in}{2.472470in}}{\pgfqpoint{9.074415in}{2.467633in}}{\pgfqpoint{9.077982in}{2.464066in}}%
\pgfpathcurveto{\pgfqpoint{9.081548in}{2.460500in}}{\pgfqpoint{9.086386in}{2.458496in}}{\pgfqpoint{9.091430in}{2.458496in}}%
\pgfpathclose%
\pgfusepath{fill}%
\end{pgfscope}%
\begin{pgfscope}%
\pgfpathrectangle{\pgfqpoint{6.572727in}{0.474100in}}{\pgfqpoint{4.227273in}{3.318700in}}%
\pgfusepath{clip}%
\pgfsetbuttcap%
\pgfsetroundjoin%
\definecolor{currentfill}{rgb}{0.127568,0.566949,0.550556}%
\pgfsetfillcolor{currentfill}%
\pgfsetfillopacity{0.700000}%
\pgfsetlinewidth{0.000000pt}%
\definecolor{currentstroke}{rgb}{0.000000,0.000000,0.000000}%
\pgfsetstrokecolor{currentstroke}%
\pgfsetstrokeopacity{0.700000}%
\pgfsetdash{}{0pt}%
\pgfpathmoveto{\pgfqpoint{7.440829in}{1.701689in}}%
\pgfpathcurveto{\pgfqpoint{7.445873in}{1.701689in}}{\pgfqpoint{7.450711in}{1.703693in}}{\pgfqpoint{7.454277in}{1.707259in}}%
\pgfpathcurveto{\pgfqpoint{7.457844in}{1.710826in}}{\pgfqpoint{7.459847in}{1.715664in}}{\pgfqpoint{7.459847in}{1.720707in}}%
\pgfpathcurveto{\pgfqpoint{7.459847in}{1.725751in}}{\pgfqpoint{7.457844in}{1.730589in}}{\pgfqpoint{7.454277in}{1.734155in}}%
\pgfpathcurveto{\pgfqpoint{7.450711in}{1.737722in}}{\pgfqpoint{7.445873in}{1.739725in}}{\pgfqpoint{7.440829in}{1.739725in}}%
\pgfpathcurveto{\pgfqpoint{7.435786in}{1.739725in}}{\pgfqpoint{7.430948in}{1.737722in}}{\pgfqpoint{7.427381in}{1.734155in}}%
\pgfpathcurveto{\pgfqpoint{7.423815in}{1.730589in}}{\pgfqpoint{7.421811in}{1.725751in}}{\pgfqpoint{7.421811in}{1.720707in}}%
\pgfpathcurveto{\pgfqpoint{7.421811in}{1.715664in}}{\pgfqpoint{7.423815in}{1.710826in}}{\pgfqpoint{7.427381in}{1.707259in}}%
\pgfpathcurveto{\pgfqpoint{7.430948in}{1.703693in}}{\pgfqpoint{7.435786in}{1.701689in}}{\pgfqpoint{7.440829in}{1.701689in}}%
\pgfpathclose%
\pgfusepath{fill}%
\end{pgfscope}%
\begin{pgfscope}%
\pgfpathrectangle{\pgfqpoint{6.572727in}{0.474100in}}{\pgfqpoint{4.227273in}{3.318700in}}%
\pgfusepath{clip}%
\pgfsetbuttcap%
\pgfsetroundjoin%
\definecolor{currentfill}{rgb}{0.127568,0.566949,0.550556}%
\pgfsetfillcolor{currentfill}%
\pgfsetfillopacity{0.700000}%
\pgfsetlinewidth{0.000000pt}%
\definecolor{currentstroke}{rgb}{0.000000,0.000000,0.000000}%
\pgfsetstrokecolor{currentstroke}%
\pgfsetstrokeopacity{0.700000}%
\pgfsetdash{}{0pt}%
\pgfpathmoveto{\pgfqpoint{8.116962in}{2.866384in}}%
\pgfpathcurveto{\pgfqpoint{8.122006in}{2.866384in}}{\pgfqpoint{8.126844in}{2.868388in}}{\pgfqpoint{8.130410in}{2.871954in}}%
\pgfpathcurveto{\pgfqpoint{8.133977in}{2.875521in}}{\pgfqpoint{8.135980in}{2.880358in}}{\pgfqpoint{8.135980in}{2.885402in}}%
\pgfpathcurveto{\pgfqpoint{8.135980in}{2.890446in}}{\pgfqpoint{8.133977in}{2.895284in}}{\pgfqpoint{8.130410in}{2.898850in}}%
\pgfpathcurveto{\pgfqpoint{8.126844in}{2.902416in}}{\pgfqpoint{8.122006in}{2.904420in}}{\pgfqpoint{8.116962in}{2.904420in}}%
\pgfpathcurveto{\pgfqpoint{8.111919in}{2.904420in}}{\pgfqpoint{8.107081in}{2.902416in}}{\pgfqpoint{8.103514in}{2.898850in}}%
\pgfpathcurveto{\pgfqpoint{8.099948in}{2.895284in}}{\pgfqpoint{8.097944in}{2.890446in}}{\pgfqpoint{8.097944in}{2.885402in}}%
\pgfpathcurveto{\pgfqpoint{8.097944in}{2.880358in}}{\pgfqpoint{8.099948in}{2.875521in}}{\pgfqpoint{8.103514in}{2.871954in}}%
\pgfpathcurveto{\pgfqpoint{8.107081in}{2.868388in}}{\pgfqpoint{8.111919in}{2.866384in}}{\pgfqpoint{8.116962in}{2.866384in}}%
\pgfpathclose%
\pgfusepath{fill}%
\end{pgfscope}%
\begin{pgfscope}%
\pgfpathrectangle{\pgfqpoint{6.572727in}{0.474100in}}{\pgfqpoint{4.227273in}{3.318700in}}%
\pgfusepath{clip}%
\pgfsetbuttcap%
\pgfsetroundjoin%
\definecolor{currentfill}{rgb}{0.993248,0.906157,0.143936}%
\pgfsetfillcolor{currentfill}%
\pgfsetfillopacity{0.700000}%
\pgfsetlinewidth{0.000000pt}%
\definecolor{currentstroke}{rgb}{0.000000,0.000000,0.000000}%
\pgfsetstrokecolor{currentstroke}%
\pgfsetstrokeopacity{0.700000}%
\pgfsetdash{}{0pt}%
\pgfpathmoveto{\pgfqpoint{9.851922in}{1.855126in}}%
\pgfpathcurveto{\pgfqpoint{9.856966in}{1.855126in}}{\pgfqpoint{9.861804in}{1.857130in}}{\pgfqpoint{9.865370in}{1.860696in}}%
\pgfpathcurveto{\pgfqpoint{9.868937in}{1.864262in}}{\pgfqpoint{9.870941in}{1.869100in}}{\pgfqpoint{9.870941in}{1.874144in}}%
\pgfpathcurveto{\pgfqpoint{9.870941in}{1.879188in}}{\pgfqpoint{9.868937in}{1.884025in}}{\pgfqpoint{9.865370in}{1.887592in}}%
\pgfpathcurveto{\pgfqpoint{9.861804in}{1.891158in}}{\pgfqpoint{9.856966in}{1.893162in}}{\pgfqpoint{9.851922in}{1.893162in}}%
\pgfpathcurveto{\pgfqpoint{9.846879in}{1.893162in}}{\pgfqpoint{9.842041in}{1.891158in}}{\pgfqpoint{9.838475in}{1.887592in}}%
\pgfpathcurveto{\pgfqpoint{9.834908in}{1.884025in}}{\pgfqpoint{9.832904in}{1.879188in}}{\pgfqpoint{9.832904in}{1.874144in}}%
\pgfpathcurveto{\pgfqpoint{9.832904in}{1.869100in}}{\pgfqpoint{9.834908in}{1.864262in}}{\pgfqpoint{9.838475in}{1.860696in}}%
\pgfpathcurveto{\pgfqpoint{9.842041in}{1.857130in}}{\pgfqpoint{9.846879in}{1.855126in}}{\pgfqpoint{9.851922in}{1.855126in}}%
\pgfpathclose%
\pgfusepath{fill}%
\end{pgfscope}%
\begin{pgfscope}%
\pgfpathrectangle{\pgfqpoint{6.572727in}{0.474100in}}{\pgfqpoint{4.227273in}{3.318700in}}%
\pgfusepath{clip}%
\pgfsetbuttcap%
\pgfsetroundjoin%
\definecolor{currentfill}{rgb}{0.993248,0.906157,0.143936}%
\pgfsetfillcolor{currentfill}%
\pgfsetfillopacity{0.700000}%
\pgfsetlinewidth{0.000000pt}%
\definecolor{currentstroke}{rgb}{0.000000,0.000000,0.000000}%
\pgfsetstrokecolor{currentstroke}%
\pgfsetstrokeopacity{0.700000}%
\pgfsetdash{}{0pt}%
\pgfpathmoveto{\pgfqpoint{9.680322in}{1.483906in}}%
\pgfpathcurveto{\pgfqpoint{9.685366in}{1.483906in}}{\pgfqpoint{9.690204in}{1.485910in}}{\pgfqpoint{9.693770in}{1.489476in}}%
\pgfpathcurveto{\pgfqpoint{9.697337in}{1.493043in}}{\pgfqpoint{9.699340in}{1.497881in}}{\pgfqpoint{9.699340in}{1.502924in}}%
\pgfpathcurveto{\pgfqpoint{9.699340in}{1.507968in}}{\pgfqpoint{9.697337in}{1.512806in}}{\pgfqpoint{9.693770in}{1.516372in}}%
\pgfpathcurveto{\pgfqpoint{9.690204in}{1.519938in}}{\pgfqpoint{9.685366in}{1.521942in}}{\pgfqpoint{9.680322in}{1.521942in}}%
\pgfpathcurveto{\pgfqpoint{9.675279in}{1.521942in}}{\pgfqpoint{9.670441in}{1.519938in}}{\pgfqpoint{9.666874in}{1.516372in}}%
\pgfpathcurveto{\pgfqpoint{9.663308in}{1.512806in}}{\pgfqpoint{9.661304in}{1.507968in}}{\pgfqpoint{9.661304in}{1.502924in}}%
\pgfpathcurveto{\pgfqpoint{9.661304in}{1.497881in}}{\pgfqpoint{9.663308in}{1.493043in}}{\pgfqpoint{9.666874in}{1.489476in}}%
\pgfpathcurveto{\pgfqpoint{9.670441in}{1.485910in}}{\pgfqpoint{9.675279in}{1.483906in}}{\pgfqpoint{9.680322in}{1.483906in}}%
\pgfpathclose%
\pgfusepath{fill}%
\end{pgfscope}%
\begin{pgfscope}%
\pgfpathrectangle{\pgfqpoint{6.572727in}{0.474100in}}{\pgfqpoint{4.227273in}{3.318700in}}%
\pgfusepath{clip}%
\pgfsetbuttcap%
\pgfsetroundjoin%
\definecolor{currentfill}{rgb}{0.993248,0.906157,0.143936}%
\pgfsetfillcolor{currentfill}%
\pgfsetfillopacity{0.700000}%
\pgfsetlinewidth{0.000000pt}%
\definecolor{currentstroke}{rgb}{0.000000,0.000000,0.000000}%
\pgfsetstrokecolor{currentstroke}%
\pgfsetstrokeopacity{0.700000}%
\pgfsetdash{}{0pt}%
\pgfpathmoveto{\pgfqpoint{9.586439in}{1.204134in}}%
\pgfpathcurveto{\pgfqpoint{9.591483in}{1.204134in}}{\pgfqpoint{9.596321in}{1.206138in}}{\pgfqpoint{9.599887in}{1.209704in}}%
\pgfpathcurveto{\pgfqpoint{9.603454in}{1.213271in}}{\pgfqpoint{9.605458in}{1.218108in}}{\pgfqpoint{9.605458in}{1.223152in}}%
\pgfpathcurveto{\pgfqpoint{9.605458in}{1.228196in}}{\pgfqpoint{9.603454in}{1.233033in}}{\pgfqpoint{9.599887in}{1.236600in}}%
\pgfpathcurveto{\pgfqpoint{9.596321in}{1.240166in}}{\pgfqpoint{9.591483in}{1.242170in}}{\pgfqpoint{9.586439in}{1.242170in}}%
\pgfpathcurveto{\pgfqpoint{9.581396in}{1.242170in}}{\pgfqpoint{9.576558in}{1.240166in}}{\pgfqpoint{9.572992in}{1.236600in}}%
\pgfpathcurveto{\pgfqpoint{9.569425in}{1.233033in}}{\pgfqpoint{9.567421in}{1.228196in}}{\pgfqpoint{9.567421in}{1.223152in}}%
\pgfpathcurveto{\pgfqpoint{9.567421in}{1.218108in}}{\pgfqpoint{9.569425in}{1.213271in}}{\pgfqpoint{9.572992in}{1.209704in}}%
\pgfpathcurveto{\pgfqpoint{9.576558in}{1.206138in}}{\pgfqpoint{9.581396in}{1.204134in}}{\pgfqpoint{9.586439in}{1.204134in}}%
\pgfpathclose%
\pgfusepath{fill}%
\end{pgfscope}%
\begin{pgfscope}%
\pgfpathrectangle{\pgfqpoint{6.572727in}{0.474100in}}{\pgfqpoint{4.227273in}{3.318700in}}%
\pgfusepath{clip}%
\pgfsetbuttcap%
\pgfsetroundjoin%
\definecolor{currentfill}{rgb}{0.127568,0.566949,0.550556}%
\pgfsetfillcolor{currentfill}%
\pgfsetfillopacity{0.700000}%
\pgfsetlinewidth{0.000000pt}%
\definecolor{currentstroke}{rgb}{0.000000,0.000000,0.000000}%
\pgfsetstrokecolor{currentstroke}%
\pgfsetstrokeopacity{0.700000}%
\pgfsetdash{}{0pt}%
\pgfpathmoveto{\pgfqpoint{7.819947in}{3.075917in}}%
\pgfpathcurveto{\pgfqpoint{7.824991in}{3.075917in}}{\pgfqpoint{7.829829in}{3.077921in}}{\pgfqpoint{7.833395in}{3.081487in}}%
\pgfpathcurveto{\pgfqpoint{7.836962in}{3.085054in}}{\pgfqpoint{7.838965in}{3.089891in}}{\pgfqpoint{7.838965in}{3.094935in}}%
\pgfpathcurveto{\pgfqpoint{7.838965in}{3.099979in}}{\pgfqpoint{7.836962in}{3.104816in}}{\pgfqpoint{7.833395in}{3.108383in}}%
\pgfpathcurveto{\pgfqpoint{7.829829in}{3.111949in}}{\pgfqpoint{7.824991in}{3.113953in}}{\pgfqpoint{7.819947in}{3.113953in}}%
\pgfpathcurveto{\pgfqpoint{7.814904in}{3.113953in}}{\pgfqpoint{7.810066in}{3.111949in}}{\pgfqpoint{7.806499in}{3.108383in}}%
\pgfpathcurveto{\pgfqpoint{7.802933in}{3.104816in}}{\pgfqpoint{7.800929in}{3.099979in}}{\pgfqpoint{7.800929in}{3.094935in}}%
\pgfpathcurveto{\pgfqpoint{7.800929in}{3.089891in}}{\pgfqpoint{7.802933in}{3.085054in}}{\pgfqpoint{7.806499in}{3.081487in}}%
\pgfpathcurveto{\pgfqpoint{7.810066in}{3.077921in}}{\pgfqpoint{7.814904in}{3.075917in}}{\pgfqpoint{7.819947in}{3.075917in}}%
\pgfpathclose%
\pgfusepath{fill}%
\end{pgfscope}%
\begin{pgfscope}%
\pgfpathrectangle{\pgfqpoint{6.572727in}{0.474100in}}{\pgfqpoint{4.227273in}{3.318700in}}%
\pgfusepath{clip}%
\pgfsetbuttcap%
\pgfsetroundjoin%
\definecolor{currentfill}{rgb}{0.993248,0.906157,0.143936}%
\pgfsetfillcolor{currentfill}%
\pgfsetfillopacity{0.700000}%
\pgfsetlinewidth{0.000000pt}%
\definecolor{currentstroke}{rgb}{0.000000,0.000000,0.000000}%
\pgfsetstrokecolor{currentstroke}%
\pgfsetstrokeopacity{0.700000}%
\pgfsetdash{}{0pt}%
\pgfpathmoveto{\pgfqpoint{9.253764in}{1.221726in}}%
\pgfpathcurveto{\pgfqpoint{9.258807in}{1.221726in}}{\pgfqpoint{9.263645in}{1.223730in}}{\pgfqpoint{9.267211in}{1.227296in}}%
\pgfpathcurveto{\pgfqpoint{9.270778in}{1.230863in}}{\pgfqpoint{9.272782in}{1.235701in}}{\pgfqpoint{9.272782in}{1.240744in}}%
\pgfpathcurveto{\pgfqpoint{9.272782in}{1.245788in}}{\pgfqpoint{9.270778in}{1.250626in}}{\pgfqpoint{9.267211in}{1.254192in}}%
\pgfpathcurveto{\pgfqpoint{9.263645in}{1.257759in}}{\pgfqpoint{9.258807in}{1.259762in}}{\pgfqpoint{9.253764in}{1.259762in}}%
\pgfpathcurveto{\pgfqpoint{9.248720in}{1.259762in}}{\pgfqpoint{9.243882in}{1.257759in}}{\pgfqpoint{9.240316in}{1.254192in}}%
\pgfpathcurveto{\pgfqpoint{9.236749in}{1.250626in}}{\pgfqpoint{9.234745in}{1.245788in}}{\pgfqpoint{9.234745in}{1.240744in}}%
\pgfpathcurveto{\pgfqpoint{9.234745in}{1.235701in}}{\pgfqpoint{9.236749in}{1.230863in}}{\pgfqpoint{9.240316in}{1.227296in}}%
\pgfpathcurveto{\pgfqpoint{9.243882in}{1.223730in}}{\pgfqpoint{9.248720in}{1.221726in}}{\pgfqpoint{9.253764in}{1.221726in}}%
\pgfpathclose%
\pgfusepath{fill}%
\end{pgfscope}%
\begin{pgfscope}%
\pgfpathrectangle{\pgfqpoint{6.572727in}{0.474100in}}{\pgfqpoint{4.227273in}{3.318700in}}%
\pgfusepath{clip}%
\pgfsetbuttcap%
\pgfsetroundjoin%
\definecolor{currentfill}{rgb}{0.127568,0.566949,0.550556}%
\pgfsetfillcolor{currentfill}%
\pgfsetfillopacity{0.700000}%
\pgfsetlinewidth{0.000000pt}%
\definecolor{currentstroke}{rgb}{0.000000,0.000000,0.000000}%
\pgfsetstrokecolor{currentstroke}%
\pgfsetstrokeopacity{0.700000}%
\pgfsetdash{}{0pt}%
\pgfpathmoveto{\pgfqpoint{7.869275in}{1.624017in}}%
\pgfpathcurveto{\pgfqpoint{7.874319in}{1.624017in}}{\pgfqpoint{7.879156in}{1.626020in}}{\pgfqpoint{7.882723in}{1.629587in}}%
\pgfpathcurveto{\pgfqpoint{7.886289in}{1.633153in}}{\pgfqpoint{7.888293in}{1.637991in}}{\pgfqpoint{7.888293in}{1.643035in}}%
\pgfpathcurveto{\pgfqpoint{7.888293in}{1.648078in}}{\pgfqpoint{7.886289in}{1.652916in}}{\pgfqpoint{7.882723in}{1.656483in}}%
\pgfpathcurveto{\pgfqpoint{7.879156in}{1.660049in}}{\pgfqpoint{7.874319in}{1.662053in}}{\pgfqpoint{7.869275in}{1.662053in}}%
\pgfpathcurveto{\pgfqpoint{7.864231in}{1.662053in}}{\pgfqpoint{7.859393in}{1.660049in}}{\pgfqpoint{7.855827in}{1.656483in}}%
\pgfpathcurveto{\pgfqpoint{7.852261in}{1.652916in}}{\pgfqpoint{7.850257in}{1.648078in}}{\pgfqpoint{7.850257in}{1.643035in}}%
\pgfpathcurveto{\pgfqpoint{7.850257in}{1.637991in}}{\pgfqpoint{7.852261in}{1.633153in}}{\pgfqpoint{7.855827in}{1.629587in}}%
\pgfpathcurveto{\pgfqpoint{7.859393in}{1.626020in}}{\pgfqpoint{7.864231in}{1.624017in}}{\pgfqpoint{7.869275in}{1.624017in}}%
\pgfpathclose%
\pgfusepath{fill}%
\end{pgfscope}%
\begin{pgfscope}%
\pgfpathrectangle{\pgfqpoint{6.572727in}{0.474100in}}{\pgfqpoint{4.227273in}{3.318700in}}%
\pgfusepath{clip}%
\pgfsetbuttcap%
\pgfsetroundjoin%
\definecolor{currentfill}{rgb}{0.127568,0.566949,0.550556}%
\pgfsetfillcolor{currentfill}%
\pgfsetfillopacity{0.700000}%
\pgfsetlinewidth{0.000000pt}%
\definecolor{currentstroke}{rgb}{0.000000,0.000000,0.000000}%
\pgfsetstrokecolor{currentstroke}%
\pgfsetstrokeopacity{0.700000}%
\pgfsetdash{}{0pt}%
\pgfpathmoveto{\pgfqpoint{7.867221in}{1.526818in}}%
\pgfpathcurveto{\pgfqpoint{7.872264in}{1.526818in}}{\pgfqpoint{7.877102in}{1.528822in}}{\pgfqpoint{7.880669in}{1.532388in}}%
\pgfpathcurveto{\pgfqpoint{7.884235in}{1.535955in}}{\pgfqpoint{7.886239in}{1.540793in}}{\pgfqpoint{7.886239in}{1.545836in}}%
\pgfpathcurveto{\pgfqpoint{7.886239in}{1.550880in}}{\pgfqpoint{7.884235in}{1.555718in}}{\pgfqpoint{7.880669in}{1.559284in}}%
\pgfpathcurveto{\pgfqpoint{7.877102in}{1.562851in}}{\pgfqpoint{7.872264in}{1.564854in}}{\pgfqpoint{7.867221in}{1.564854in}}%
\pgfpathcurveto{\pgfqpoint{7.862177in}{1.564854in}}{\pgfqpoint{7.857339in}{1.562851in}}{\pgfqpoint{7.853773in}{1.559284in}}%
\pgfpathcurveto{\pgfqpoint{7.850206in}{1.555718in}}{\pgfqpoint{7.848202in}{1.550880in}}{\pgfqpoint{7.848202in}{1.545836in}}%
\pgfpathcurveto{\pgfqpoint{7.848202in}{1.540793in}}{\pgfqpoint{7.850206in}{1.535955in}}{\pgfqpoint{7.853773in}{1.532388in}}%
\pgfpathcurveto{\pgfqpoint{7.857339in}{1.528822in}}{\pgfqpoint{7.862177in}{1.526818in}}{\pgfqpoint{7.867221in}{1.526818in}}%
\pgfpathclose%
\pgfusepath{fill}%
\end{pgfscope}%
\begin{pgfscope}%
\pgfpathrectangle{\pgfqpoint{6.572727in}{0.474100in}}{\pgfqpoint{4.227273in}{3.318700in}}%
\pgfusepath{clip}%
\pgfsetbuttcap%
\pgfsetroundjoin%
\definecolor{currentfill}{rgb}{0.127568,0.566949,0.550556}%
\pgfsetfillcolor{currentfill}%
\pgfsetfillopacity{0.700000}%
\pgfsetlinewidth{0.000000pt}%
\definecolor{currentstroke}{rgb}{0.000000,0.000000,0.000000}%
\pgfsetstrokecolor{currentstroke}%
\pgfsetstrokeopacity{0.700000}%
\pgfsetdash{}{0pt}%
\pgfpathmoveto{\pgfqpoint{8.437324in}{2.843606in}}%
\pgfpathcurveto{\pgfqpoint{8.442367in}{2.843606in}}{\pgfqpoint{8.447205in}{2.845610in}}{\pgfqpoint{8.450772in}{2.849176in}}%
\pgfpathcurveto{\pgfqpoint{8.454338in}{2.852742in}}{\pgfqpoint{8.456342in}{2.857580in}}{\pgfqpoint{8.456342in}{2.862624in}}%
\pgfpathcurveto{\pgfqpoint{8.456342in}{2.867668in}}{\pgfqpoint{8.454338in}{2.872505in}}{\pgfqpoint{8.450772in}{2.876072in}}%
\pgfpathcurveto{\pgfqpoint{8.447205in}{2.879638in}}{\pgfqpoint{8.442367in}{2.881642in}}{\pgfqpoint{8.437324in}{2.881642in}}%
\pgfpathcurveto{\pgfqpoint{8.432280in}{2.881642in}}{\pgfqpoint{8.427442in}{2.879638in}}{\pgfqpoint{8.423876in}{2.876072in}}%
\pgfpathcurveto{\pgfqpoint{8.420310in}{2.872505in}}{\pgfqpoint{8.418306in}{2.867668in}}{\pgfqpoint{8.418306in}{2.862624in}}%
\pgfpathcurveto{\pgfqpoint{8.418306in}{2.857580in}}{\pgfqpoint{8.420310in}{2.852742in}}{\pgfqpoint{8.423876in}{2.849176in}}%
\pgfpathcurveto{\pgfqpoint{8.427442in}{2.845610in}}{\pgfqpoint{8.432280in}{2.843606in}}{\pgfqpoint{8.437324in}{2.843606in}}%
\pgfpathclose%
\pgfusepath{fill}%
\end{pgfscope}%
\begin{pgfscope}%
\pgfpathrectangle{\pgfqpoint{6.572727in}{0.474100in}}{\pgfqpoint{4.227273in}{3.318700in}}%
\pgfusepath{clip}%
\pgfsetbuttcap%
\pgfsetroundjoin%
\definecolor{currentfill}{rgb}{0.993248,0.906157,0.143936}%
\pgfsetfillcolor{currentfill}%
\pgfsetfillopacity{0.700000}%
\pgfsetlinewidth{0.000000pt}%
\definecolor{currentstroke}{rgb}{0.000000,0.000000,0.000000}%
\pgfsetstrokecolor{currentstroke}%
\pgfsetstrokeopacity{0.700000}%
\pgfsetdash{}{0pt}%
\pgfpathmoveto{\pgfqpoint{9.695518in}{1.589233in}}%
\pgfpathcurveto{\pgfqpoint{9.700562in}{1.589233in}}{\pgfqpoint{9.705399in}{1.591237in}}{\pgfqpoint{9.708966in}{1.594803in}}%
\pgfpathcurveto{\pgfqpoint{9.712532in}{1.598369in}}{\pgfqpoint{9.714536in}{1.603207in}}{\pgfqpoint{9.714536in}{1.608251in}}%
\pgfpathcurveto{\pgfqpoint{9.714536in}{1.613294in}}{\pgfqpoint{9.712532in}{1.618132in}}{\pgfqpoint{9.708966in}{1.621699in}}%
\pgfpathcurveto{\pgfqpoint{9.705399in}{1.625265in}}{\pgfqpoint{9.700562in}{1.627269in}}{\pgfqpoint{9.695518in}{1.627269in}}%
\pgfpathcurveto{\pgfqpoint{9.690474in}{1.627269in}}{\pgfqpoint{9.685636in}{1.625265in}}{\pgfqpoint{9.682070in}{1.621699in}}%
\pgfpathcurveto{\pgfqpoint{9.678504in}{1.618132in}}{\pgfqpoint{9.676500in}{1.613294in}}{\pgfqpoint{9.676500in}{1.608251in}}%
\pgfpathcurveto{\pgfqpoint{9.676500in}{1.603207in}}{\pgfqpoint{9.678504in}{1.598369in}}{\pgfqpoint{9.682070in}{1.594803in}}%
\pgfpathcurveto{\pgfqpoint{9.685636in}{1.591237in}}{\pgfqpoint{9.690474in}{1.589233in}}{\pgfqpoint{9.695518in}{1.589233in}}%
\pgfpathclose%
\pgfusepath{fill}%
\end{pgfscope}%
\begin{pgfscope}%
\pgfpathrectangle{\pgfqpoint{6.572727in}{0.474100in}}{\pgfqpoint{4.227273in}{3.318700in}}%
\pgfusepath{clip}%
\pgfsetbuttcap%
\pgfsetroundjoin%
\definecolor{currentfill}{rgb}{0.993248,0.906157,0.143936}%
\pgfsetfillcolor{currentfill}%
\pgfsetfillopacity{0.700000}%
\pgfsetlinewidth{0.000000pt}%
\definecolor{currentstroke}{rgb}{0.000000,0.000000,0.000000}%
\pgfsetstrokecolor{currentstroke}%
\pgfsetstrokeopacity{0.700000}%
\pgfsetdash{}{0pt}%
\pgfpathmoveto{\pgfqpoint{9.966105in}{1.089835in}}%
\pgfpathcurveto{\pgfqpoint{9.971149in}{1.089835in}}{\pgfqpoint{9.975986in}{1.091839in}}{\pgfqpoint{9.979553in}{1.095405in}}%
\pgfpathcurveto{\pgfqpoint{9.983119in}{1.098972in}}{\pgfqpoint{9.985123in}{1.103809in}}{\pgfqpoint{9.985123in}{1.108853in}}%
\pgfpathcurveto{\pgfqpoint{9.985123in}{1.113897in}}{\pgfqpoint{9.983119in}{1.118734in}}{\pgfqpoint{9.979553in}{1.122301in}}%
\pgfpathcurveto{\pgfqpoint{9.975986in}{1.125867in}}{\pgfqpoint{9.971149in}{1.127871in}}{\pgfqpoint{9.966105in}{1.127871in}}%
\pgfpathcurveto{\pgfqpoint{9.961061in}{1.127871in}}{\pgfqpoint{9.956224in}{1.125867in}}{\pgfqpoint{9.952657in}{1.122301in}}%
\pgfpathcurveto{\pgfqpoint{9.949091in}{1.118734in}}{\pgfqpoint{9.947087in}{1.113897in}}{\pgfqpoint{9.947087in}{1.108853in}}%
\pgfpathcurveto{\pgfqpoint{9.947087in}{1.103809in}}{\pgfqpoint{9.949091in}{1.098972in}}{\pgfqpoint{9.952657in}{1.095405in}}%
\pgfpathcurveto{\pgfqpoint{9.956224in}{1.091839in}}{\pgfqpoint{9.961061in}{1.089835in}}{\pgfqpoint{9.966105in}{1.089835in}}%
\pgfpathclose%
\pgfusepath{fill}%
\end{pgfscope}%
\begin{pgfscope}%
\pgfpathrectangle{\pgfqpoint{6.572727in}{0.474100in}}{\pgfqpoint{4.227273in}{3.318700in}}%
\pgfusepath{clip}%
\pgfsetbuttcap%
\pgfsetroundjoin%
\definecolor{currentfill}{rgb}{0.993248,0.906157,0.143936}%
\pgfsetfillcolor{currentfill}%
\pgfsetfillopacity{0.700000}%
\pgfsetlinewidth{0.000000pt}%
\definecolor{currentstroke}{rgb}{0.000000,0.000000,0.000000}%
\pgfsetstrokecolor{currentstroke}%
\pgfsetstrokeopacity{0.700000}%
\pgfsetdash{}{0pt}%
\pgfpathmoveto{\pgfqpoint{9.973458in}{1.738924in}}%
\pgfpathcurveto{\pgfqpoint{9.978502in}{1.738924in}}{\pgfqpoint{9.983339in}{1.740927in}}{\pgfqpoint{9.986906in}{1.744494in}}%
\pgfpathcurveto{\pgfqpoint{9.990472in}{1.748060in}}{\pgfqpoint{9.992476in}{1.752898in}}{\pgfqpoint{9.992476in}{1.757942in}}%
\pgfpathcurveto{\pgfqpoint{9.992476in}{1.762985in}}{\pgfqpoint{9.990472in}{1.767823in}}{\pgfqpoint{9.986906in}{1.771390in}}%
\pgfpathcurveto{\pgfqpoint{9.983339in}{1.774956in}}{\pgfqpoint{9.978502in}{1.776960in}}{\pgfqpoint{9.973458in}{1.776960in}}%
\pgfpathcurveto{\pgfqpoint{9.968414in}{1.776960in}}{\pgfqpoint{9.963577in}{1.774956in}}{\pgfqpoint{9.960010in}{1.771390in}}%
\pgfpathcurveto{\pgfqpoint{9.956444in}{1.767823in}}{\pgfqpoint{9.954440in}{1.762985in}}{\pgfqpoint{9.954440in}{1.757942in}}%
\pgfpathcurveto{\pgfqpoint{9.954440in}{1.752898in}}{\pgfqpoint{9.956444in}{1.748060in}}{\pgfqpoint{9.960010in}{1.744494in}}%
\pgfpathcurveto{\pgfqpoint{9.963577in}{1.740927in}}{\pgfqpoint{9.968414in}{1.738924in}}{\pgfqpoint{9.973458in}{1.738924in}}%
\pgfpathclose%
\pgfusepath{fill}%
\end{pgfscope}%
\begin{pgfscope}%
\pgfpathrectangle{\pgfqpoint{6.572727in}{0.474100in}}{\pgfqpoint{4.227273in}{3.318700in}}%
\pgfusepath{clip}%
\pgfsetbuttcap%
\pgfsetroundjoin%
\definecolor{currentfill}{rgb}{0.127568,0.566949,0.550556}%
\pgfsetfillcolor{currentfill}%
\pgfsetfillopacity{0.700000}%
\pgfsetlinewidth{0.000000pt}%
\definecolor{currentstroke}{rgb}{0.000000,0.000000,0.000000}%
\pgfsetstrokecolor{currentstroke}%
\pgfsetstrokeopacity{0.700000}%
\pgfsetdash{}{0pt}%
\pgfpathmoveto{\pgfqpoint{7.477651in}{1.358210in}}%
\pgfpathcurveto{\pgfqpoint{7.482694in}{1.358210in}}{\pgfqpoint{7.487532in}{1.360214in}}{\pgfqpoint{7.491098in}{1.363780in}}%
\pgfpathcurveto{\pgfqpoint{7.494665in}{1.367347in}}{\pgfqpoint{7.496669in}{1.372184in}}{\pgfqpoint{7.496669in}{1.377228in}}%
\pgfpathcurveto{\pgfqpoint{7.496669in}{1.382272in}}{\pgfqpoint{7.494665in}{1.387109in}}{\pgfqpoint{7.491098in}{1.390676in}}%
\pgfpathcurveto{\pgfqpoint{7.487532in}{1.394242in}}{\pgfqpoint{7.482694in}{1.396246in}}{\pgfqpoint{7.477651in}{1.396246in}}%
\pgfpathcurveto{\pgfqpoint{7.472607in}{1.396246in}}{\pgfqpoint{7.467769in}{1.394242in}}{\pgfqpoint{7.464203in}{1.390676in}}%
\pgfpathcurveto{\pgfqpoint{7.460636in}{1.387109in}}{\pgfqpoint{7.458632in}{1.382272in}}{\pgfqpoint{7.458632in}{1.377228in}}%
\pgfpathcurveto{\pgfqpoint{7.458632in}{1.372184in}}{\pgfqpoint{7.460636in}{1.367347in}}{\pgfqpoint{7.464203in}{1.363780in}}%
\pgfpathcurveto{\pgfqpoint{7.467769in}{1.360214in}}{\pgfqpoint{7.472607in}{1.358210in}}{\pgfqpoint{7.477651in}{1.358210in}}%
\pgfpathclose%
\pgfusepath{fill}%
\end{pgfscope}%
\begin{pgfscope}%
\pgfpathrectangle{\pgfqpoint{6.572727in}{0.474100in}}{\pgfqpoint{4.227273in}{3.318700in}}%
\pgfusepath{clip}%
\pgfsetbuttcap%
\pgfsetroundjoin%
\definecolor{currentfill}{rgb}{0.127568,0.566949,0.550556}%
\pgfsetfillcolor{currentfill}%
\pgfsetfillopacity{0.700000}%
\pgfsetlinewidth{0.000000pt}%
\definecolor{currentstroke}{rgb}{0.000000,0.000000,0.000000}%
\pgfsetstrokecolor{currentstroke}%
\pgfsetstrokeopacity{0.700000}%
\pgfsetdash{}{0pt}%
\pgfpathmoveto{\pgfqpoint{7.599573in}{2.965168in}}%
\pgfpathcurveto{\pgfqpoint{7.604616in}{2.965168in}}{\pgfqpoint{7.609454in}{2.967172in}}{\pgfqpoint{7.613020in}{2.970739in}}%
\pgfpathcurveto{\pgfqpoint{7.616587in}{2.974305in}}{\pgfqpoint{7.618591in}{2.979143in}}{\pgfqpoint{7.618591in}{2.984187in}}%
\pgfpathcurveto{\pgfqpoint{7.618591in}{2.989230in}}{\pgfqpoint{7.616587in}{2.994068in}}{\pgfqpoint{7.613020in}{2.997634in}}%
\pgfpathcurveto{\pgfqpoint{7.609454in}{3.001201in}}{\pgfqpoint{7.604616in}{3.003205in}}{\pgfqpoint{7.599573in}{3.003205in}}%
\pgfpathcurveto{\pgfqpoint{7.594529in}{3.003205in}}{\pgfqpoint{7.589691in}{3.001201in}}{\pgfqpoint{7.586125in}{2.997634in}}%
\pgfpathcurveto{\pgfqpoint{7.582558in}{2.994068in}}{\pgfqpoint{7.580554in}{2.989230in}}{\pgfqpoint{7.580554in}{2.984187in}}%
\pgfpathcurveto{\pgfqpoint{7.580554in}{2.979143in}}{\pgfqpoint{7.582558in}{2.974305in}}{\pgfqpoint{7.586125in}{2.970739in}}%
\pgfpathcurveto{\pgfqpoint{7.589691in}{2.967172in}}{\pgfqpoint{7.594529in}{2.965168in}}{\pgfqpoint{7.599573in}{2.965168in}}%
\pgfpathclose%
\pgfusepath{fill}%
\end{pgfscope}%
\begin{pgfscope}%
\pgfpathrectangle{\pgfqpoint{6.572727in}{0.474100in}}{\pgfqpoint{4.227273in}{3.318700in}}%
\pgfusepath{clip}%
\pgfsetbuttcap%
\pgfsetroundjoin%
\definecolor{currentfill}{rgb}{0.993248,0.906157,0.143936}%
\pgfsetfillcolor{currentfill}%
\pgfsetfillopacity{0.700000}%
\pgfsetlinewidth{0.000000pt}%
\definecolor{currentstroke}{rgb}{0.000000,0.000000,0.000000}%
\pgfsetstrokecolor{currentstroke}%
\pgfsetstrokeopacity{0.700000}%
\pgfsetdash{}{0pt}%
\pgfpathmoveto{\pgfqpoint{9.608392in}{1.238244in}}%
\pgfpathcurveto{\pgfqpoint{9.613436in}{1.238244in}}{\pgfqpoint{9.618274in}{1.240248in}}{\pgfqpoint{9.621840in}{1.243815in}}%
\pgfpathcurveto{\pgfqpoint{9.625406in}{1.247381in}}{\pgfqpoint{9.627410in}{1.252219in}}{\pgfqpoint{9.627410in}{1.257262in}}%
\pgfpathcurveto{\pgfqpoint{9.627410in}{1.262306in}}{\pgfqpoint{9.625406in}{1.267144in}}{\pgfqpoint{9.621840in}{1.270710in}}%
\pgfpathcurveto{\pgfqpoint{9.618274in}{1.274277in}}{\pgfqpoint{9.613436in}{1.276281in}}{\pgfqpoint{9.608392in}{1.276281in}}%
\pgfpathcurveto{\pgfqpoint{9.603348in}{1.276281in}}{\pgfqpoint{9.598511in}{1.274277in}}{\pgfqpoint{9.594944in}{1.270710in}}%
\pgfpathcurveto{\pgfqpoint{9.591378in}{1.267144in}}{\pgfqpoint{9.589374in}{1.262306in}}{\pgfqpoint{9.589374in}{1.257262in}}%
\pgfpathcurveto{\pgfqpoint{9.589374in}{1.252219in}}{\pgfqpoint{9.591378in}{1.247381in}}{\pgfqpoint{9.594944in}{1.243815in}}%
\pgfpathcurveto{\pgfqpoint{9.598511in}{1.240248in}}{\pgfqpoint{9.603348in}{1.238244in}}{\pgfqpoint{9.608392in}{1.238244in}}%
\pgfpathclose%
\pgfusepath{fill}%
\end{pgfscope}%
\begin{pgfscope}%
\pgfpathrectangle{\pgfqpoint{6.572727in}{0.474100in}}{\pgfqpoint{4.227273in}{3.318700in}}%
\pgfusepath{clip}%
\pgfsetbuttcap%
\pgfsetroundjoin%
\definecolor{currentfill}{rgb}{0.993248,0.906157,0.143936}%
\pgfsetfillcolor{currentfill}%
\pgfsetfillopacity{0.700000}%
\pgfsetlinewidth{0.000000pt}%
\definecolor{currentstroke}{rgb}{0.000000,0.000000,0.000000}%
\pgfsetstrokecolor{currentstroke}%
\pgfsetstrokeopacity{0.700000}%
\pgfsetdash{}{0pt}%
\pgfpathmoveto{\pgfqpoint{9.267081in}{1.716430in}}%
\pgfpathcurveto{\pgfqpoint{9.272125in}{1.716430in}}{\pgfqpoint{9.276962in}{1.718434in}}{\pgfqpoint{9.280529in}{1.722001in}}%
\pgfpathcurveto{\pgfqpoint{9.284095in}{1.725567in}}{\pgfqpoint{9.286099in}{1.730405in}}{\pgfqpoint{9.286099in}{1.735449in}}%
\pgfpathcurveto{\pgfqpoint{9.286099in}{1.740492in}}{\pgfqpoint{9.284095in}{1.745330in}}{\pgfqpoint{9.280529in}{1.748897in}}%
\pgfpathcurveto{\pgfqpoint{9.276962in}{1.752463in}}{\pgfqpoint{9.272125in}{1.754467in}}{\pgfqpoint{9.267081in}{1.754467in}}%
\pgfpathcurveto{\pgfqpoint{9.262037in}{1.754467in}}{\pgfqpoint{9.257199in}{1.752463in}}{\pgfqpoint{9.253633in}{1.748897in}}%
\pgfpathcurveto{\pgfqpoint{9.250067in}{1.745330in}}{\pgfqpoint{9.248063in}{1.740492in}}{\pgfqpoint{9.248063in}{1.735449in}}%
\pgfpathcurveto{\pgfqpoint{9.248063in}{1.730405in}}{\pgfqpoint{9.250067in}{1.725567in}}{\pgfqpoint{9.253633in}{1.722001in}}%
\pgfpathcurveto{\pgfqpoint{9.257199in}{1.718434in}}{\pgfqpoint{9.262037in}{1.716430in}}{\pgfqpoint{9.267081in}{1.716430in}}%
\pgfpathclose%
\pgfusepath{fill}%
\end{pgfscope}%
\begin{pgfscope}%
\pgfpathrectangle{\pgfqpoint{6.572727in}{0.474100in}}{\pgfqpoint{4.227273in}{3.318700in}}%
\pgfusepath{clip}%
\pgfsetbuttcap%
\pgfsetroundjoin%
\definecolor{currentfill}{rgb}{0.993248,0.906157,0.143936}%
\pgfsetfillcolor{currentfill}%
\pgfsetfillopacity{0.700000}%
\pgfsetlinewidth{0.000000pt}%
\definecolor{currentstroke}{rgb}{0.000000,0.000000,0.000000}%
\pgfsetstrokecolor{currentstroke}%
\pgfsetstrokeopacity{0.700000}%
\pgfsetdash{}{0pt}%
\pgfpathmoveto{\pgfqpoint{9.427330in}{1.235788in}}%
\pgfpathcurveto{\pgfqpoint{9.432374in}{1.235788in}}{\pgfqpoint{9.437212in}{1.237792in}}{\pgfqpoint{9.440778in}{1.241358in}}%
\pgfpathcurveto{\pgfqpoint{9.444344in}{1.244925in}}{\pgfqpoint{9.446348in}{1.249762in}}{\pgfqpoint{9.446348in}{1.254806in}}%
\pgfpathcurveto{\pgfqpoint{9.446348in}{1.259850in}}{\pgfqpoint{9.444344in}{1.264687in}}{\pgfqpoint{9.440778in}{1.268254in}}%
\pgfpathcurveto{\pgfqpoint{9.437212in}{1.271820in}}{\pgfqpoint{9.432374in}{1.273824in}}{\pgfqpoint{9.427330in}{1.273824in}}%
\pgfpathcurveto{\pgfqpoint{9.422287in}{1.273824in}}{\pgfqpoint{9.417449in}{1.271820in}}{\pgfqpoint{9.413882in}{1.268254in}}%
\pgfpathcurveto{\pgfqpoint{9.410316in}{1.264687in}}{\pgfqpoint{9.408312in}{1.259850in}}{\pgfqpoint{9.408312in}{1.254806in}}%
\pgfpathcurveto{\pgfqpoint{9.408312in}{1.249762in}}{\pgfqpoint{9.410316in}{1.244925in}}{\pgfqpoint{9.413882in}{1.241358in}}%
\pgfpathcurveto{\pgfqpoint{9.417449in}{1.237792in}}{\pgfqpoint{9.422287in}{1.235788in}}{\pgfqpoint{9.427330in}{1.235788in}}%
\pgfpathclose%
\pgfusepath{fill}%
\end{pgfscope}%
\begin{pgfscope}%
\pgfpathrectangle{\pgfqpoint{6.572727in}{0.474100in}}{\pgfqpoint{4.227273in}{3.318700in}}%
\pgfusepath{clip}%
\pgfsetbuttcap%
\pgfsetroundjoin%
\definecolor{currentfill}{rgb}{0.993248,0.906157,0.143936}%
\pgfsetfillcolor{currentfill}%
\pgfsetfillopacity{0.700000}%
\pgfsetlinewidth{0.000000pt}%
\definecolor{currentstroke}{rgb}{0.000000,0.000000,0.000000}%
\pgfsetstrokecolor{currentstroke}%
\pgfsetstrokeopacity{0.700000}%
\pgfsetdash{}{0pt}%
\pgfpathmoveto{\pgfqpoint{9.562831in}{1.527462in}}%
\pgfpathcurveto{\pgfqpoint{9.567874in}{1.527462in}}{\pgfqpoint{9.572712in}{1.529466in}}{\pgfqpoint{9.576279in}{1.533032in}}%
\pgfpathcurveto{\pgfqpoint{9.579845in}{1.536599in}}{\pgfqpoint{9.581849in}{1.541437in}}{\pgfqpoint{9.581849in}{1.546480in}}%
\pgfpathcurveto{\pgfqpoint{9.581849in}{1.551524in}}{\pgfqpoint{9.579845in}{1.556362in}}{\pgfqpoint{9.576279in}{1.559928in}}%
\pgfpathcurveto{\pgfqpoint{9.572712in}{1.563495in}}{\pgfqpoint{9.567874in}{1.565498in}}{\pgfqpoint{9.562831in}{1.565498in}}%
\pgfpathcurveto{\pgfqpoint{9.557787in}{1.565498in}}{\pgfqpoint{9.552949in}{1.563495in}}{\pgfqpoint{9.549383in}{1.559928in}}%
\pgfpathcurveto{\pgfqpoint{9.545816in}{1.556362in}}{\pgfqpoint{9.543813in}{1.551524in}}{\pgfqpoint{9.543813in}{1.546480in}}%
\pgfpathcurveto{\pgfqpoint{9.543813in}{1.541437in}}{\pgfqpoint{9.545816in}{1.536599in}}{\pgfqpoint{9.549383in}{1.533032in}}%
\pgfpathcurveto{\pgfqpoint{9.552949in}{1.529466in}}{\pgfqpoint{9.557787in}{1.527462in}}{\pgfqpoint{9.562831in}{1.527462in}}%
\pgfpathclose%
\pgfusepath{fill}%
\end{pgfscope}%
\begin{pgfscope}%
\pgfpathrectangle{\pgfqpoint{6.572727in}{0.474100in}}{\pgfqpoint{4.227273in}{3.318700in}}%
\pgfusepath{clip}%
\pgfsetbuttcap%
\pgfsetroundjoin%
\definecolor{currentfill}{rgb}{0.993248,0.906157,0.143936}%
\pgfsetfillcolor{currentfill}%
\pgfsetfillopacity{0.700000}%
\pgfsetlinewidth{0.000000pt}%
\definecolor{currentstroke}{rgb}{0.000000,0.000000,0.000000}%
\pgfsetstrokecolor{currentstroke}%
\pgfsetstrokeopacity{0.700000}%
\pgfsetdash{}{0pt}%
\pgfpathmoveto{\pgfqpoint{9.380313in}{1.627885in}}%
\pgfpathcurveto{\pgfqpoint{9.385357in}{1.627885in}}{\pgfqpoint{9.390194in}{1.629889in}}{\pgfqpoint{9.393761in}{1.633456in}}%
\pgfpathcurveto{\pgfqpoint{9.397327in}{1.637022in}}{\pgfqpoint{9.399331in}{1.641860in}}{\pgfqpoint{9.399331in}{1.646903in}}%
\pgfpathcurveto{\pgfqpoint{9.399331in}{1.651947in}}{\pgfqpoint{9.397327in}{1.656785in}}{\pgfqpoint{9.393761in}{1.660351in}}%
\pgfpathcurveto{\pgfqpoint{9.390194in}{1.663918in}}{\pgfqpoint{9.385357in}{1.665922in}}{\pgfqpoint{9.380313in}{1.665922in}}%
\pgfpathcurveto{\pgfqpoint{9.375269in}{1.665922in}}{\pgfqpoint{9.370431in}{1.663918in}}{\pgfqpoint{9.366865in}{1.660351in}}%
\pgfpathcurveto{\pgfqpoint{9.363299in}{1.656785in}}{\pgfqpoint{9.361295in}{1.651947in}}{\pgfqpoint{9.361295in}{1.646903in}}%
\pgfpathcurveto{\pgfqpoint{9.361295in}{1.641860in}}{\pgfqpoint{9.363299in}{1.637022in}}{\pgfqpoint{9.366865in}{1.633456in}}%
\pgfpathcurveto{\pgfqpoint{9.370431in}{1.629889in}}{\pgfqpoint{9.375269in}{1.627885in}}{\pgfqpoint{9.380313in}{1.627885in}}%
\pgfpathclose%
\pgfusepath{fill}%
\end{pgfscope}%
\begin{pgfscope}%
\pgfpathrectangle{\pgfqpoint{6.572727in}{0.474100in}}{\pgfqpoint{4.227273in}{3.318700in}}%
\pgfusepath{clip}%
\pgfsetbuttcap%
\pgfsetroundjoin%
\definecolor{currentfill}{rgb}{0.127568,0.566949,0.550556}%
\pgfsetfillcolor{currentfill}%
\pgfsetfillopacity{0.700000}%
\pgfsetlinewidth{0.000000pt}%
\definecolor{currentstroke}{rgb}{0.000000,0.000000,0.000000}%
\pgfsetstrokecolor{currentstroke}%
\pgfsetstrokeopacity{0.700000}%
\pgfsetdash{}{0pt}%
\pgfpathmoveto{\pgfqpoint{8.069414in}{0.999264in}}%
\pgfpathcurveto{\pgfqpoint{8.074458in}{0.999264in}}{\pgfqpoint{8.079296in}{1.001267in}}{\pgfqpoint{8.082862in}{1.004834in}}%
\pgfpathcurveto{\pgfqpoint{8.086429in}{1.008400in}}{\pgfqpoint{8.088432in}{1.013238in}}{\pgfqpoint{8.088432in}{1.018282in}}%
\pgfpathcurveto{\pgfqpoint{8.088432in}{1.023325in}}{\pgfqpoint{8.086429in}{1.028163in}}{\pgfqpoint{8.082862in}{1.031730in}}%
\pgfpathcurveto{\pgfqpoint{8.079296in}{1.035296in}}{\pgfqpoint{8.074458in}{1.037300in}}{\pgfqpoint{8.069414in}{1.037300in}}%
\pgfpathcurveto{\pgfqpoint{8.064371in}{1.037300in}}{\pgfqpoint{8.059533in}{1.035296in}}{\pgfqpoint{8.055966in}{1.031730in}}%
\pgfpathcurveto{\pgfqpoint{8.052400in}{1.028163in}}{\pgfqpoint{8.050396in}{1.023325in}}{\pgfqpoint{8.050396in}{1.018282in}}%
\pgfpathcurveto{\pgfqpoint{8.050396in}{1.013238in}}{\pgfqpoint{8.052400in}{1.008400in}}{\pgfqpoint{8.055966in}{1.004834in}}%
\pgfpathcurveto{\pgfqpoint{8.059533in}{1.001267in}}{\pgfqpoint{8.064371in}{0.999264in}}{\pgfqpoint{8.069414in}{0.999264in}}%
\pgfpathclose%
\pgfusepath{fill}%
\end{pgfscope}%
\begin{pgfscope}%
\pgfpathrectangle{\pgfqpoint{6.572727in}{0.474100in}}{\pgfqpoint{4.227273in}{3.318700in}}%
\pgfusepath{clip}%
\pgfsetbuttcap%
\pgfsetroundjoin%
\definecolor{currentfill}{rgb}{0.127568,0.566949,0.550556}%
\pgfsetfillcolor{currentfill}%
\pgfsetfillopacity{0.700000}%
\pgfsetlinewidth{0.000000pt}%
\definecolor{currentstroke}{rgb}{0.000000,0.000000,0.000000}%
\pgfsetstrokecolor{currentstroke}%
\pgfsetstrokeopacity{0.700000}%
\pgfsetdash{}{0pt}%
\pgfpathmoveto{\pgfqpoint{8.367846in}{1.773739in}}%
\pgfpathcurveto{\pgfqpoint{8.372890in}{1.773739in}}{\pgfqpoint{8.377728in}{1.775743in}}{\pgfqpoint{8.381294in}{1.779309in}}%
\pgfpathcurveto{\pgfqpoint{8.384860in}{1.782876in}}{\pgfqpoint{8.386864in}{1.787713in}}{\pgfqpoint{8.386864in}{1.792757in}}%
\pgfpathcurveto{\pgfqpoint{8.386864in}{1.797801in}}{\pgfqpoint{8.384860in}{1.802639in}}{\pgfqpoint{8.381294in}{1.806205in}}%
\pgfpathcurveto{\pgfqpoint{8.377728in}{1.809771in}}{\pgfqpoint{8.372890in}{1.811775in}}{\pgfqpoint{8.367846in}{1.811775in}}%
\pgfpathcurveto{\pgfqpoint{8.362802in}{1.811775in}}{\pgfqpoint{8.357965in}{1.809771in}}{\pgfqpoint{8.354398in}{1.806205in}}%
\pgfpathcurveto{\pgfqpoint{8.350832in}{1.802639in}}{\pgfqpoint{8.348828in}{1.797801in}}{\pgfqpoint{8.348828in}{1.792757in}}%
\pgfpathcurveto{\pgfqpoint{8.348828in}{1.787713in}}{\pgfqpoint{8.350832in}{1.782876in}}{\pgfqpoint{8.354398in}{1.779309in}}%
\pgfpathcurveto{\pgfqpoint{8.357965in}{1.775743in}}{\pgfqpoint{8.362802in}{1.773739in}}{\pgfqpoint{8.367846in}{1.773739in}}%
\pgfpathclose%
\pgfusepath{fill}%
\end{pgfscope}%
\begin{pgfscope}%
\pgfpathrectangle{\pgfqpoint{6.572727in}{0.474100in}}{\pgfqpoint{4.227273in}{3.318700in}}%
\pgfusepath{clip}%
\pgfsetbuttcap%
\pgfsetroundjoin%
\definecolor{currentfill}{rgb}{0.127568,0.566949,0.550556}%
\pgfsetfillcolor{currentfill}%
\pgfsetfillopacity{0.700000}%
\pgfsetlinewidth{0.000000pt}%
\definecolor{currentstroke}{rgb}{0.000000,0.000000,0.000000}%
\pgfsetstrokecolor{currentstroke}%
\pgfsetstrokeopacity{0.700000}%
\pgfsetdash{}{0pt}%
\pgfpathmoveto{\pgfqpoint{8.586375in}{3.009194in}}%
\pgfpathcurveto{\pgfqpoint{8.591418in}{3.009194in}}{\pgfqpoint{8.596256in}{3.011198in}}{\pgfqpoint{8.599823in}{3.014765in}}%
\pgfpathcurveto{\pgfqpoint{8.603389in}{3.018331in}}{\pgfqpoint{8.605393in}{3.023169in}}{\pgfqpoint{8.605393in}{3.028213in}}%
\pgfpathcurveto{\pgfqpoint{8.605393in}{3.033256in}}{\pgfqpoint{8.603389in}{3.038094in}}{\pgfqpoint{8.599823in}{3.041660in}}%
\pgfpathcurveto{\pgfqpoint{8.596256in}{3.045227in}}{\pgfqpoint{8.591418in}{3.047231in}}{\pgfqpoint{8.586375in}{3.047231in}}%
\pgfpathcurveto{\pgfqpoint{8.581331in}{3.047231in}}{\pgfqpoint{8.576493in}{3.045227in}}{\pgfqpoint{8.572927in}{3.041660in}}%
\pgfpathcurveto{\pgfqpoint{8.569361in}{3.038094in}}{\pgfqpoint{8.567357in}{3.033256in}}{\pgfqpoint{8.567357in}{3.028213in}}%
\pgfpathcurveto{\pgfqpoint{8.567357in}{3.023169in}}{\pgfqpoint{8.569361in}{3.018331in}}{\pgfqpoint{8.572927in}{3.014765in}}%
\pgfpathcurveto{\pgfqpoint{8.576493in}{3.011198in}}{\pgfqpoint{8.581331in}{3.009194in}}{\pgfqpoint{8.586375in}{3.009194in}}%
\pgfpathclose%
\pgfusepath{fill}%
\end{pgfscope}%
\begin{pgfscope}%
\pgfpathrectangle{\pgfqpoint{6.572727in}{0.474100in}}{\pgfqpoint{4.227273in}{3.318700in}}%
\pgfusepath{clip}%
\pgfsetbuttcap%
\pgfsetroundjoin%
\definecolor{currentfill}{rgb}{0.127568,0.566949,0.550556}%
\pgfsetfillcolor{currentfill}%
\pgfsetfillopacity{0.700000}%
\pgfsetlinewidth{0.000000pt}%
\definecolor{currentstroke}{rgb}{0.000000,0.000000,0.000000}%
\pgfsetstrokecolor{currentstroke}%
\pgfsetstrokeopacity{0.700000}%
\pgfsetdash{}{0pt}%
\pgfpathmoveto{\pgfqpoint{7.440425in}{1.897495in}}%
\pgfpathcurveto{\pgfqpoint{7.445468in}{1.897495in}}{\pgfqpoint{7.450306in}{1.899499in}}{\pgfqpoint{7.453872in}{1.903065in}}%
\pgfpathcurveto{\pgfqpoint{7.457439in}{1.906632in}}{\pgfqpoint{7.459443in}{1.911469in}}{\pgfqpoint{7.459443in}{1.916513in}}%
\pgfpathcurveto{\pgfqpoint{7.459443in}{1.921557in}}{\pgfqpoint{7.457439in}{1.926394in}}{\pgfqpoint{7.453872in}{1.929961in}}%
\pgfpathcurveto{\pgfqpoint{7.450306in}{1.933527in}}{\pgfqpoint{7.445468in}{1.935531in}}{\pgfqpoint{7.440425in}{1.935531in}}%
\pgfpathcurveto{\pgfqpoint{7.435381in}{1.935531in}}{\pgfqpoint{7.430543in}{1.933527in}}{\pgfqpoint{7.426977in}{1.929961in}}%
\pgfpathcurveto{\pgfqpoint{7.423410in}{1.926394in}}{\pgfqpoint{7.421406in}{1.921557in}}{\pgfqpoint{7.421406in}{1.916513in}}%
\pgfpathcurveto{\pgfqpoint{7.421406in}{1.911469in}}{\pgfqpoint{7.423410in}{1.906632in}}{\pgfqpoint{7.426977in}{1.903065in}}%
\pgfpathcurveto{\pgfqpoint{7.430543in}{1.899499in}}{\pgfqpoint{7.435381in}{1.897495in}}{\pgfqpoint{7.440425in}{1.897495in}}%
\pgfpathclose%
\pgfusepath{fill}%
\end{pgfscope}%
\begin{pgfscope}%
\pgfpathrectangle{\pgfqpoint{6.572727in}{0.474100in}}{\pgfqpoint{4.227273in}{3.318700in}}%
\pgfusepath{clip}%
\pgfsetbuttcap%
\pgfsetroundjoin%
\definecolor{currentfill}{rgb}{0.127568,0.566949,0.550556}%
\pgfsetfillcolor{currentfill}%
\pgfsetfillopacity{0.700000}%
\pgfsetlinewidth{0.000000pt}%
\definecolor{currentstroke}{rgb}{0.000000,0.000000,0.000000}%
\pgfsetstrokecolor{currentstroke}%
\pgfsetstrokeopacity{0.700000}%
\pgfsetdash{}{0pt}%
\pgfpathmoveto{\pgfqpoint{7.878370in}{1.419632in}}%
\pgfpathcurveto{\pgfqpoint{7.883413in}{1.419632in}}{\pgfqpoint{7.888251in}{1.421636in}}{\pgfqpoint{7.891818in}{1.425203in}}%
\pgfpathcurveto{\pgfqpoint{7.895384in}{1.428769in}}{\pgfqpoint{7.897388in}{1.433607in}}{\pgfqpoint{7.897388in}{1.438651in}}%
\pgfpathcurveto{\pgfqpoint{7.897388in}{1.443694in}}{\pgfqpoint{7.895384in}{1.448532in}}{\pgfqpoint{7.891818in}{1.452098in}}%
\pgfpathcurveto{\pgfqpoint{7.888251in}{1.455665in}}{\pgfqpoint{7.883413in}{1.457669in}}{\pgfqpoint{7.878370in}{1.457669in}}%
\pgfpathcurveto{\pgfqpoint{7.873326in}{1.457669in}}{\pgfqpoint{7.868488in}{1.455665in}}{\pgfqpoint{7.864922in}{1.452098in}}%
\pgfpathcurveto{\pgfqpoint{7.861356in}{1.448532in}}{\pgfqpoint{7.859352in}{1.443694in}}{\pgfqpoint{7.859352in}{1.438651in}}%
\pgfpathcurveto{\pgfqpoint{7.859352in}{1.433607in}}{\pgfqpoint{7.861356in}{1.428769in}}{\pgfqpoint{7.864922in}{1.425203in}}%
\pgfpathcurveto{\pgfqpoint{7.868488in}{1.421636in}}{\pgfqpoint{7.873326in}{1.419632in}}{\pgfqpoint{7.878370in}{1.419632in}}%
\pgfpathclose%
\pgfusepath{fill}%
\end{pgfscope}%
\begin{pgfscope}%
\pgfpathrectangle{\pgfqpoint{6.572727in}{0.474100in}}{\pgfqpoint{4.227273in}{3.318700in}}%
\pgfusepath{clip}%
\pgfsetbuttcap%
\pgfsetroundjoin%
\definecolor{currentfill}{rgb}{0.127568,0.566949,0.550556}%
\pgfsetfillcolor{currentfill}%
\pgfsetfillopacity{0.700000}%
\pgfsetlinewidth{0.000000pt}%
\definecolor{currentstroke}{rgb}{0.000000,0.000000,0.000000}%
\pgfsetstrokecolor{currentstroke}%
\pgfsetstrokeopacity{0.700000}%
\pgfsetdash{}{0pt}%
\pgfpathmoveto{\pgfqpoint{8.461573in}{2.691941in}}%
\pgfpathcurveto{\pgfqpoint{8.466616in}{2.691941in}}{\pgfqpoint{8.471454in}{2.693945in}}{\pgfqpoint{8.475021in}{2.697511in}}%
\pgfpathcurveto{\pgfqpoint{8.478587in}{2.701077in}}{\pgfqpoint{8.480591in}{2.705915in}}{\pgfqpoint{8.480591in}{2.710959in}}%
\pgfpathcurveto{\pgfqpoint{8.480591in}{2.716003in}}{\pgfqpoint{8.478587in}{2.720840in}}{\pgfqpoint{8.475021in}{2.724407in}}%
\pgfpathcurveto{\pgfqpoint{8.471454in}{2.727973in}}{\pgfqpoint{8.466616in}{2.729977in}}{\pgfqpoint{8.461573in}{2.729977in}}%
\pgfpathcurveto{\pgfqpoint{8.456529in}{2.729977in}}{\pgfqpoint{8.451691in}{2.727973in}}{\pgfqpoint{8.448125in}{2.724407in}}%
\pgfpathcurveto{\pgfqpoint{8.444558in}{2.720840in}}{\pgfqpoint{8.442555in}{2.716003in}}{\pgfqpoint{8.442555in}{2.710959in}}%
\pgfpathcurveto{\pgfqpoint{8.442555in}{2.705915in}}{\pgfqpoint{8.444558in}{2.701077in}}{\pgfqpoint{8.448125in}{2.697511in}}%
\pgfpathcurveto{\pgfqpoint{8.451691in}{2.693945in}}{\pgfqpoint{8.456529in}{2.691941in}}{\pgfqpoint{8.461573in}{2.691941in}}%
\pgfpathclose%
\pgfusepath{fill}%
\end{pgfscope}%
\begin{pgfscope}%
\pgfpathrectangle{\pgfqpoint{6.572727in}{0.474100in}}{\pgfqpoint{4.227273in}{3.318700in}}%
\pgfusepath{clip}%
\pgfsetbuttcap%
\pgfsetroundjoin%
\definecolor{currentfill}{rgb}{0.127568,0.566949,0.550556}%
\pgfsetfillcolor{currentfill}%
\pgfsetfillopacity{0.700000}%
\pgfsetlinewidth{0.000000pt}%
\definecolor{currentstroke}{rgb}{0.000000,0.000000,0.000000}%
\pgfsetstrokecolor{currentstroke}%
\pgfsetstrokeopacity{0.700000}%
\pgfsetdash{}{0pt}%
\pgfpathmoveto{\pgfqpoint{8.746234in}{2.582759in}}%
\pgfpathcurveto{\pgfqpoint{8.751278in}{2.582759in}}{\pgfqpoint{8.756115in}{2.584763in}}{\pgfqpoint{8.759682in}{2.588330in}}%
\pgfpathcurveto{\pgfqpoint{8.763248in}{2.591896in}}{\pgfqpoint{8.765252in}{2.596734in}}{\pgfqpoint{8.765252in}{2.601777in}}%
\pgfpathcurveto{\pgfqpoint{8.765252in}{2.606821in}}{\pgfqpoint{8.763248in}{2.611659in}}{\pgfqpoint{8.759682in}{2.615225in}}%
\pgfpathcurveto{\pgfqpoint{8.756115in}{2.618792in}}{\pgfqpoint{8.751278in}{2.620796in}}{\pgfqpoint{8.746234in}{2.620796in}}%
\pgfpathcurveto{\pgfqpoint{8.741190in}{2.620796in}}{\pgfqpoint{8.736353in}{2.618792in}}{\pgfqpoint{8.732786in}{2.615225in}}%
\pgfpathcurveto{\pgfqpoint{8.729220in}{2.611659in}}{\pgfqpoint{8.727216in}{2.606821in}}{\pgfqpoint{8.727216in}{2.601777in}}%
\pgfpathcurveto{\pgfqpoint{8.727216in}{2.596734in}}{\pgfqpoint{8.729220in}{2.591896in}}{\pgfqpoint{8.732786in}{2.588330in}}%
\pgfpathcurveto{\pgfqpoint{8.736353in}{2.584763in}}{\pgfqpoint{8.741190in}{2.582759in}}{\pgfqpoint{8.746234in}{2.582759in}}%
\pgfpathclose%
\pgfusepath{fill}%
\end{pgfscope}%
\begin{pgfscope}%
\pgfpathrectangle{\pgfqpoint{6.572727in}{0.474100in}}{\pgfqpoint{4.227273in}{3.318700in}}%
\pgfusepath{clip}%
\pgfsetbuttcap%
\pgfsetroundjoin%
\definecolor{currentfill}{rgb}{0.127568,0.566949,0.550556}%
\pgfsetfillcolor{currentfill}%
\pgfsetfillopacity{0.700000}%
\pgfsetlinewidth{0.000000pt}%
\definecolor{currentstroke}{rgb}{0.000000,0.000000,0.000000}%
\pgfsetstrokecolor{currentstroke}%
\pgfsetstrokeopacity{0.700000}%
\pgfsetdash{}{0pt}%
\pgfpathmoveto{\pgfqpoint{7.289971in}{1.980645in}}%
\pgfpathcurveto{\pgfqpoint{7.295014in}{1.980645in}}{\pgfqpoint{7.299852in}{1.982649in}}{\pgfqpoint{7.303419in}{1.986215in}}%
\pgfpathcurveto{\pgfqpoint{7.306985in}{1.989782in}}{\pgfqpoint{7.308989in}{1.994619in}}{\pgfqpoint{7.308989in}{1.999663in}}%
\pgfpathcurveto{\pgfqpoint{7.308989in}{2.004707in}}{\pgfqpoint{7.306985in}{2.009545in}}{\pgfqpoint{7.303419in}{2.013111in}}%
\pgfpathcurveto{\pgfqpoint{7.299852in}{2.016677in}}{\pgfqpoint{7.295014in}{2.018681in}}{\pgfqpoint{7.289971in}{2.018681in}}%
\pgfpathcurveto{\pgfqpoint{7.284927in}{2.018681in}}{\pgfqpoint{7.280089in}{2.016677in}}{\pgfqpoint{7.276523in}{2.013111in}}%
\pgfpathcurveto{\pgfqpoint{7.272957in}{2.009545in}}{\pgfqpoint{7.270953in}{2.004707in}}{\pgfqpoint{7.270953in}{1.999663in}}%
\pgfpathcurveto{\pgfqpoint{7.270953in}{1.994619in}}{\pgfqpoint{7.272957in}{1.989782in}}{\pgfqpoint{7.276523in}{1.986215in}}%
\pgfpathcurveto{\pgfqpoint{7.280089in}{1.982649in}}{\pgfqpoint{7.284927in}{1.980645in}}{\pgfqpoint{7.289971in}{1.980645in}}%
\pgfpathclose%
\pgfusepath{fill}%
\end{pgfscope}%
\begin{pgfscope}%
\pgfpathrectangle{\pgfqpoint{6.572727in}{0.474100in}}{\pgfqpoint{4.227273in}{3.318700in}}%
\pgfusepath{clip}%
\pgfsetbuttcap%
\pgfsetroundjoin%
\definecolor{currentfill}{rgb}{0.127568,0.566949,0.550556}%
\pgfsetfillcolor{currentfill}%
\pgfsetfillopacity{0.700000}%
\pgfsetlinewidth{0.000000pt}%
\definecolor{currentstroke}{rgb}{0.000000,0.000000,0.000000}%
\pgfsetstrokecolor{currentstroke}%
\pgfsetstrokeopacity{0.700000}%
\pgfsetdash{}{0pt}%
\pgfpathmoveto{\pgfqpoint{7.497976in}{1.766077in}}%
\pgfpathcurveto{\pgfqpoint{7.503019in}{1.766077in}}{\pgfqpoint{7.507857in}{1.768081in}}{\pgfqpoint{7.511424in}{1.771647in}}%
\pgfpathcurveto{\pgfqpoint{7.514990in}{1.775214in}}{\pgfqpoint{7.516994in}{1.780051in}}{\pgfqpoint{7.516994in}{1.785095in}}%
\pgfpathcurveto{\pgfqpoint{7.516994in}{1.790139in}}{\pgfqpoint{7.514990in}{1.794976in}}{\pgfqpoint{7.511424in}{1.798543in}}%
\pgfpathcurveto{\pgfqpoint{7.507857in}{1.802109in}}{\pgfqpoint{7.503019in}{1.804113in}}{\pgfqpoint{7.497976in}{1.804113in}}%
\pgfpathcurveto{\pgfqpoint{7.492932in}{1.804113in}}{\pgfqpoint{7.488094in}{1.802109in}}{\pgfqpoint{7.484528in}{1.798543in}}%
\pgfpathcurveto{\pgfqpoint{7.480961in}{1.794976in}}{\pgfqpoint{7.478958in}{1.790139in}}{\pgfqpoint{7.478958in}{1.785095in}}%
\pgfpathcurveto{\pgfqpoint{7.478958in}{1.780051in}}{\pgfqpoint{7.480961in}{1.775214in}}{\pgfqpoint{7.484528in}{1.771647in}}%
\pgfpathcurveto{\pgfqpoint{7.488094in}{1.768081in}}{\pgfqpoint{7.492932in}{1.766077in}}{\pgfqpoint{7.497976in}{1.766077in}}%
\pgfpathclose%
\pgfusepath{fill}%
\end{pgfscope}%
\begin{pgfscope}%
\pgfpathrectangle{\pgfqpoint{6.572727in}{0.474100in}}{\pgfqpoint{4.227273in}{3.318700in}}%
\pgfusepath{clip}%
\pgfsetbuttcap%
\pgfsetroundjoin%
\definecolor{currentfill}{rgb}{0.127568,0.566949,0.550556}%
\pgfsetfillcolor{currentfill}%
\pgfsetfillopacity{0.700000}%
\pgfsetlinewidth{0.000000pt}%
\definecolor{currentstroke}{rgb}{0.000000,0.000000,0.000000}%
\pgfsetstrokecolor{currentstroke}%
\pgfsetstrokeopacity{0.700000}%
\pgfsetdash{}{0pt}%
\pgfpathmoveto{\pgfqpoint{8.130568in}{2.846616in}}%
\pgfpathcurveto{\pgfqpoint{8.135612in}{2.846616in}}{\pgfqpoint{8.140450in}{2.848620in}}{\pgfqpoint{8.144016in}{2.852187in}}%
\pgfpathcurveto{\pgfqpoint{8.147583in}{2.855753in}}{\pgfqpoint{8.149586in}{2.860591in}}{\pgfqpoint{8.149586in}{2.865635in}}%
\pgfpathcurveto{\pgfqpoint{8.149586in}{2.870678in}}{\pgfqpoint{8.147583in}{2.875516in}}{\pgfqpoint{8.144016in}{2.879082in}}%
\pgfpathcurveto{\pgfqpoint{8.140450in}{2.882649in}}{\pgfqpoint{8.135612in}{2.884653in}}{\pgfqpoint{8.130568in}{2.884653in}}%
\pgfpathcurveto{\pgfqpoint{8.125525in}{2.884653in}}{\pgfqpoint{8.120687in}{2.882649in}}{\pgfqpoint{8.117120in}{2.879082in}}%
\pgfpathcurveto{\pgfqpoint{8.113554in}{2.875516in}}{\pgfqpoint{8.111550in}{2.870678in}}{\pgfqpoint{8.111550in}{2.865635in}}%
\pgfpathcurveto{\pgfqpoint{8.111550in}{2.860591in}}{\pgfqpoint{8.113554in}{2.855753in}}{\pgfqpoint{8.117120in}{2.852187in}}%
\pgfpathcurveto{\pgfqpoint{8.120687in}{2.848620in}}{\pgfqpoint{8.125525in}{2.846616in}}{\pgfqpoint{8.130568in}{2.846616in}}%
\pgfpathclose%
\pgfusepath{fill}%
\end{pgfscope}%
\begin{pgfscope}%
\pgfpathrectangle{\pgfqpoint{6.572727in}{0.474100in}}{\pgfqpoint{4.227273in}{3.318700in}}%
\pgfusepath{clip}%
\pgfsetbuttcap%
\pgfsetroundjoin%
\definecolor{currentfill}{rgb}{0.993248,0.906157,0.143936}%
\pgfsetfillcolor{currentfill}%
\pgfsetfillopacity{0.700000}%
\pgfsetlinewidth{0.000000pt}%
\definecolor{currentstroke}{rgb}{0.000000,0.000000,0.000000}%
\pgfsetstrokecolor{currentstroke}%
\pgfsetstrokeopacity{0.700000}%
\pgfsetdash{}{0pt}%
\pgfpathmoveto{\pgfqpoint{9.747110in}{1.775018in}}%
\pgfpathcurveto{\pgfqpoint{9.752153in}{1.775018in}}{\pgfqpoint{9.756991in}{1.777022in}}{\pgfqpoint{9.760558in}{1.780588in}}%
\pgfpathcurveto{\pgfqpoint{9.764124in}{1.784154in}}{\pgfqpoint{9.766128in}{1.788992in}}{\pgfqpoint{9.766128in}{1.794036in}}%
\pgfpathcurveto{\pgfqpoint{9.766128in}{1.799080in}}{\pgfqpoint{9.764124in}{1.803917in}}{\pgfqpoint{9.760558in}{1.807484in}}%
\pgfpathcurveto{\pgfqpoint{9.756991in}{1.811050in}}{\pgfqpoint{9.752153in}{1.813054in}}{\pgfqpoint{9.747110in}{1.813054in}}%
\pgfpathcurveto{\pgfqpoint{9.742066in}{1.813054in}}{\pgfqpoint{9.737228in}{1.811050in}}{\pgfqpoint{9.733662in}{1.807484in}}%
\pgfpathcurveto{\pgfqpoint{9.730095in}{1.803917in}}{\pgfqpoint{9.728092in}{1.799080in}}{\pgfqpoint{9.728092in}{1.794036in}}%
\pgfpathcurveto{\pgfqpoint{9.728092in}{1.788992in}}{\pgfqpoint{9.730095in}{1.784154in}}{\pgfqpoint{9.733662in}{1.780588in}}%
\pgfpathcurveto{\pgfqpoint{9.737228in}{1.777022in}}{\pgfqpoint{9.742066in}{1.775018in}}{\pgfqpoint{9.747110in}{1.775018in}}%
\pgfpathclose%
\pgfusepath{fill}%
\end{pgfscope}%
\begin{pgfscope}%
\pgfpathrectangle{\pgfqpoint{6.572727in}{0.474100in}}{\pgfqpoint{4.227273in}{3.318700in}}%
\pgfusepath{clip}%
\pgfsetbuttcap%
\pgfsetroundjoin%
\definecolor{currentfill}{rgb}{0.127568,0.566949,0.550556}%
\pgfsetfillcolor{currentfill}%
\pgfsetfillopacity{0.700000}%
\pgfsetlinewidth{0.000000pt}%
\definecolor{currentstroke}{rgb}{0.000000,0.000000,0.000000}%
\pgfsetstrokecolor{currentstroke}%
\pgfsetstrokeopacity{0.700000}%
\pgfsetdash{}{0pt}%
\pgfpathmoveto{\pgfqpoint{7.462407in}{1.478957in}}%
\pgfpathcurveto{\pgfqpoint{7.467451in}{1.478957in}}{\pgfqpoint{7.472288in}{1.480961in}}{\pgfqpoint{7.475855in}{1.484528in}}%
\pgfpathcurveto{\pgfqpoint{7.479421in}{1.488094in}}{\pgfqpoint{7.481425in}{1.492932in}}{\pgfqpoint{7.481425in}{1.497975in}}%
\pgfpathcurveto{\pgfqpoint{7.481425in}{1.503019in}}{\pgfqpoint{7.479421in}{1.507857in}}{\pgfqpoint{7.475855in}{1.511423in}}%
\pgfpathcurveto{\pgfqpoint{7.472288in}{1.514990in}}{\pgfqpoint{7.467451in}{1.516994in}}{\pgfqpoint{7.462407in}{1.516994in}}%
\pgfpathcurveto{\pgfqpoint{7.457363in}{1.516994in}}{\pgfqpoint{7.452526in}{1.514990in}}{\pgfqpoint{7.448959in}{1.511423in}}%
\pgfpathcurveto{\pgfqpoint{7.445393in}{1.507857in}}{\pgfqpoint{7.443389in}{1.503019in}}{\pgfqpoint{7.443389in}{1.497975in}}%
\pgfpathcurveto{\pgfqpoint{7.443389in}{1.492932in}}{\pgfqpoint{7.445393in}{1.488094in}}{\pgfqpoint{7.448959in}{1.484528in}}%
\pgfpathcurveto{\pgfqpoint{7.452526in}{1.480961in}}{\pgfqpoint{7.457363in}{1.478957in}}{\pgfqpoint{7.462407in}{1.478957in}}%
\pgfpathclose%
\pgfusepath{fill}%
\end{pgfscope}%
\begin{pgfscope}%
\pgfpathrectangle{\pgfqpoint{6.572727in}{0.474100in}}{\pgfqpoint{4.227273in}{3.318700in}}%
\pgfusepath{clip}%
\pgfsetbuttcap%
\pgfsetroundjoin%
\definecolor{currentfill}{rgb}{0.267004,0.004874,0.329415}%
\pgfsetfillcolor{currentfill}%
\pgfsetfillopacity{0.700000}%
\pgfsetlinewidth{0.000000pt}%
\definecolor{currentstroke}{rgb}{0.000000,0.000000,0.000000}%
\pgfsetstrokecolor{currentstroke}%
\pgfsetstrokeopacity{0.700000}%
\pgfsetdash{}{0pt}%
\pgfpathmoveto{\pgfqpoint{9.811303in}{0.704290in}}%
\pgfpathcurveto{\pgfqpoint{9.816347in}{0.704290in}}{\pgfqpoint{9.821185in}{0.706294in}}{\pgfqpoint{9.824751in}{0.709861in}}%
\pgfpathcurveto{\pgfqpoint{9.828318in}{0.713427in}}{\pgfqpoint{9.830322in}{0.718265in}}{\pgfqpoint{9.830322in}{0.723309in}}%
\pgfpathcurveto{\pgfqpoint{9.830322in}{0.728352in}}{\pgfqpoint{9.828318in}{0.733190in}}{\pgfqpoint{9.824751in}{0.736756in}}%
\pgfpathcurveto{\pgfqpoint{9.821185in}{0.740323in}}{\pgfqpoint{9.816347in}{0.742327in}}{\pgfqpoint{9.811303in}{0.742327in}}%
\pgfpathcurveto{\pgfqpoint{9.806260in}{0.742327in}}{\pgfqpoint{9.801422in}{0.740323in}}{\pgfqpoint{9.797856in}{0.736756in}}%
\pgfpathcurveto{\pgfqpoint{9.794289in}{0.733190in}}{\pgfqpoint{9.792285in}{0.728352in}}{\pgfqpoint{9.792285in}{0.723309in}}%
\pgfpathcurveto{\pgfqpoint{9.792285in}{0.718265in}}{\pgfqpoint{9.794289in}{0.713427in}}{\pgfqpoint{9.797856in}{0.709861in}}%
\pgfpathcurveto{\pgfqpoint{9.801422in}{0.706294in}}{\pgfqpoint{9.806260in}{0.704290in}}{\pgfqpoint{9.811303in}{0.704290in}}%
\pgfpathclose%
\pgfusepath{fill}%
\end{pgfscope}%
\begin{pgfscope}%
\pgfpathrectangle{\pgfqpoint{6.572727in}{0.474100in}}{\pgfqpoint{4.227273in}{3.318700in}}%
\pgfusepath{clip}%
\pgfsetbuttcap%
\pgfsetroundjoin%
\definecolor{currentfill}{rgb}{0.127568,0.566949,0.550556}%
\pgfsetfillcolor{currentfill}%
\pgfsetfillopacity{0.700000}%
\pgfsetlinewidth{0.000000pt}%
\definecolor{currentstroke}{rgb}{0.000000,0.000000,0.000000}%
\pgfsetstrokecolor{currentstroke}%
\pgfsetstrokeopacity{0.700000}%
\pgfsetdash{}{0pt}%
\pgfpathmoveto{\pgfqpoint{7.547652in}{1.714814in}}%
\pgfpathcurveto{\pgfqpoint{7.552696in}{1.714814in}}{\pgfqpoint{7.557534in}{1.716818in}}{\pgfqpoint{7.561100in}{1.720384in}}%
\pgfpathcurveto{\pgfqpoint{7.564667in}{1.723951in}}{\pgfqpoint{7.566671in}{1.728788in}}{\pgfqpoint{7.566671in}{1.733832in}}%
\pgfpathcurveto{\pgfqpoint{7.566671in}{1.738876in}}{\pgfqpoint{7.564667in}{1.743713in}}{\pgfqpoint{7.561100in}{1.747280in}}%
\pgfpathcurveto{\pgfqpoint{7.557534in}{1.750846in}}{\pgfqpoint{7.552696in}{1.752850in}}{\pgfqpoint{7.547652in}{1.752850in}}%
\pgfpathcurveto{\pgfqpoint{7.542609in}{1.752850in}}{\pgfqpoint{7.537771in}{1.750846in}}{\pgfqpoint{7.534205in}{1.747280in}}%
\pgfpathcurveto{\pgfqpoint{7.530638in}{1.743713in}}{\pgfqpoint{7.528634in}{1.738876in}}{\pgfqpoint{7.528634in}{1.733832in}}%
\pgfpathcurveto{\pgfqpoint{7.528634in}{1.728788in}}{\pgfqpoint{7.530638in}{1.723951in}}{\pgfqpoint{7.534205in}{1.720384in}}%
\pgfpathcurveto{\pgfqpoint{7.537771in}{1.716818in}}{\pgfqpoint{7.542609in}{1.714814in}}{\pgfqpoint{7.547652in}{1.714814in}}%
\pgfpathclose%
\pgfusepath{fill}%
\end{pgfscope}%
\begin{pgfscope}%
\pgfpathrectangle{\pgfqpoint{6.572727in}{0.474100in}}{\pgfqpoint{4.227273in}{3.318700in}}%
\pgfusepath{clip}%
\pgfsetbuttcap%
\pgfsetroundjoin%
\definecolor{currentfill}{rgb}{0.127568,0.566949,0.550556}%
\pgfsetfillcolor{currentfill}%
\pgfsetfillopacity{0.700000}%
\pgfsetlinewidth{0.000000pt}%
\definecolor{currentstroke}{rgb}{0.000000,0.000000,0.000000}%
\pgfsetstrokecolor{currentstroke}%
\pgfsetstrokeopacity{0.700000}%
\pgfsetdash{}{0pt}%
\pgfpathmoveto{\pgfqpoint{8.850450in}{2.869026in}}%
\pgfpathcurveto{\pgfqpoint{8.855493in}{2.869026in}}{\pgfqpoint{8.860331in}{2.871030in}}{\pgfqpoint{8.863898in}{2.874596in}}%
\pgfpathcurveto{\pgfqpoint{8.867464in}{2.878163in}}{\pgfqpoint{8.869468in}{2.883000in}}{\pgfqpoint{8.869468in}{2.888044in}}%
\pgfpathcurveto{\pgfqpoint{8.869468in}{2.893088in}}{\pgfqpoint{8.867464in}{2.897926in}}{\pgfqpoint{8.863898in}{2.901492in}}%
\pgfpathcurveto{\pgfqpoint{8.860331in}{2.905058in}}{\pgfqpoint{8.855493in}{2.907062in}}{\pgfqpoint{8.850450in}{2.907062in}}%
\pgfpathcurveto{\pgfqpoint{8.845406in}{2.907062in}}{\pgfqpoint{8.840568in}{2.905058in}}{\pgfqpoint{8.837002in}{2.901492in}}%
\pgfpathcurveto{\pgfqpoint{8.833435in}{2.897926in}}{\pgfqpoint{8.831432in}{2.893088in}}{\pgfqpoint{8.831432in}{2.888044in}}%
\pgfpathcurveto{\pgfqpoint{8.831432in}{2.883000in}}{\pgfqpoint{8.833435in}{2.878163in}}{\pgfqpoint{8.837002in}{2.874596in}}%
\pgfpathcurveto{\pgfqpoint{8.840568in}{2.871030in}}{\pgfqpoint{8.845406in}{2.869026in}}{\pgfqpoint{8.850450in}{2.869026in}}%
\pgfpathclose%
\pgfusepath{fill}%
\end{pgfscope}%
\begin{pgfscope}%
\pgfpathrectangle{\pgfqpoint{6.572727in}{0.474100in}}{\pgfqpoint{4.227273in}{3.318700in}}%
\pgfusepath{clip}%
\pgfsetbuttcap%
\pgfsetroundjoin%
\definecolor{currentfill}{rgb}{0.993248,0.906157,0.143936}%
\pgfsetfillcolor{currentfill}%
\pgfsetfillopacity{0.700000}%
\pgfsetlinewidth{0.000000pt}%
\definecolor{currentstroke}{rgb}{0.000000,0.000000,0.000000}%
\pgfsetstrokecolor{currentstroke}%
\pgfsetstrokeopacity{0.700000}%
\pgfsetdash{}{0pt}%
\pgfpathmoveto{\pgfqpoint{9.659783in}{1.737252in}}%
\pgfpathcurveto{\pgfqpoint{9.664826in}{1.737252in}}{\pgfqpoint{9.669664in}{1.739256in}}{\pgfqpoint{9.673230in}{1.742823in}}%
\pgfpathcurveto{\pgfqpoint{9.676797in}{1.746389in}}{\pgfqpoint{9.678801in}{1.751227in}}{\pgfqpoint{9.678801in}{1.756271in}}%
\pgfpathcurveto{\pgfqpoint{9.678801in}{1.761314in}}{\pgfqpoint{9.676797in}{1.766152in}}{\pgfqpoint{9.673230in}{1.769718in}}%
\pgfpathcurveto{\pgfqpoint{9.669664in}{1.773285in}}{\pgfqpoint{9.664826in}{1.775289in}}{\pgfqpoint{9.659783in}{1.775289in}}%
\pgfpathcurveto{\pgfqpoint{9.654739in}{1.775289in}}{\pgfqpoint{9.649901in}{1.773285in}}{\pgfqpoint{9.646335in}{1.769718in}}%
\pgfpathcurveto{\pgfqpoint{9.642768in}{1.766152in}}{\pgfqpoint{9.640764in}{1.761314in}}{\pgfqpoint{9.640764in}{1.756271in}}%
\pgfpathcurveto{\pgfqpoint{9.640764in}{1.751227in}}{\pgfqpoint{9.642768in}{1.746389in}}{\pgfqpoint{9.646335in}{1.742823in}}%
\pgfpathcurveto{\pgfqpoint{9.649901in}{1.739256in}}{\pgfqpoint{9.654739in}{1.737252in}}{\pgfqpoint{9.659783in}{1.737252in}}%
\pgfpathclose%
\pgfusepath{fill}%
\end{pgfscope}%
\begin{pgfscope}%
\pgfpathrectangle{\pgfqpoint{6.572727in}{0.474100in}}{\pgfqpoint{4.227273in}{3.318700in}}%
\pgfusepath{clip}%
\pgfsetbuttcap%
\pgfsetroundjoin%
\definecolor{currentfill}{rgb}{0.993248,0.906157,0.143936}%
\pgfsetfillcolor{currentfill}%
\pgfsetfillopacity{0.700000}%
\pgfsetlinewidth{0.000000pt}%
\definecolor{currentstroke}{rgb}{0.000000,0.000000,0.000000}%
\pgfsetstrokecolor{currentstroke}%
\pgfsetstrokeopacity{0.700000}%
\pgfsetdash{}{0pt}%
\pgfpathmoveto{\pgfqpoint{9.671296in}{0.892303in}}%
\pgfpathcurveto{\pgfqpoint{9.676339in}{0.892303in}}{\pgfqpoint{9.681177in}{0.894307in}}{\pgfqpoint{9.684743in}{0.897873in}}%
\pgfpathcurveto{\pgfqpoint{9.688310in}{0.901440in}}{\pgfqpoint{9.690314in}{0.906278in}}{\pgfqpoint{9.690314in}{0.911321in}}%
\pgfpathcurveto{\pgfqpoint{9.690314in}{0.916365in}}{\pgfqpoint{9.688310in}{0.921203in}}{\pgfqpoint{9.684743in}{0.924769in}}%
\pgfpathcurveto{\pgfqpoint{9.681177in}{0.928336in}}{\pgfqpoint{9.676339in}{0.930339in}}{\pgfqpoint{9.671296in}{0.930339in}}%
\pgfpathcurveto{\pgfqpoint{9.666252in}{0.930339in}}{\pgfqpoint{9.661414in}{0.928336in}}{\pgfqpoint{9.657848in}{0.924769in}}%
\pgfpathcurveto{\pgfqpoint{9.654281in}{0.921203in}}{\pgfqpoint{9.652277in}{0.916365in}}{\pgfqpoint{9.652277in}{0.911321in}}%
\pgfpathcurveto{\pgfqpoint{9.652277in}{0.906278in}}{\pgfqpoint{9.654281in}{0.901440in}}{\pgfqpoint{9.657848in}{0.897873in}}%
\pgfpathcurveto{\pgfqpoint{9.661414in}{0.894307in}}{\pgfqpoint{9.666252in}{0.892303in}}{\pgfqpoint{9.671296in}{0.892303in}}%
\pgfpathclose%
\pgfusepath{fill}%
\end{pgfscope}%
\begin{pgfscope}%
\pgfpathrectangle{\pgfqpoint{6.572727in}{0.474100in}}{\pgfqpoint{4.227273in}{3.318700in}}%
\pgfusepath{clip}%
\pgfsetbuttcap%
\pgfsetroundjoin%
\definecolor{currentfill}{rgb}{0.127568,0.566949,0.550556}%
\pgfsetfillcolor{currentfill}%
\pgfsetfillopacity{0.700000}%
\pgfsetlinewidth{0.000000pt}%
\definecolor{currentstroke}{rgb}{0.000000,0.000000,0.000000}%
\pgfsetstrokecolor{currentstroke}%
\pgfsetstrokeopacity{0.700000}%
\pgfsetdash{}{0pt}%
\pgfpathmoveto{\pgfqpoint{7.262959in}{2.226435in}}%
\pgfpathcurveto{\pgfqpoint{7.268003in}{2.226435in}}{\pgfqpoint{7.272841in}{2.228439in}}{\pgfqpoint{7.276407in}{2.232006in}}%
\pgfpathcurveto{\pgfqpoint{7.279974in}{2.235572in}}{\pgfqpoint{7.281977in}{2.240410in}}{\pgfqpoint{7.281977in}{2.245454in}}%
\pgfpathcurveto{\pgfqpoint{7.281977in}{2.250497in}}{\pgfqpoint{7.279974in}{2.255335in}}{\pgfqpoint{7.276407in}{2.258901in}}%
\pgfpathcurveto{\pgfqpoint{7.272841in}{2.262468in}}{\pgfqpoint{7.268003in}{2.264472in}}{\pgfqpoint{7.262959in}{2.264472in}}%
\pgfpathcurveto{\pgfqpoint{7.257916in}{2.264472in}}{\pgfqpoint{7.253078in}{2.262468in}}{\pgfqpoint{7.249511in}{2.258901in}}%
\pgfpathcurveto{\pgfqpoint{7.245945in}{2.255335in}}{\pgfqpoint{7.243941in}{2.250497in}}{\pgfqpoint{7.243941in}{2.245454in}}%
\pgfpathcurveto{\pgfqpoint{7.243941in}{2.240410in}}{\pgfqpoint{7.245945in}{2.235572in}}{\pgfqpoint{7.249511in}{2.232006in}}%
\pgfpathcurveto{\pgfqpoint{7.253078in}{2.228439in}}{\pgfqpoint{7.257916in}{2.226435in}}{\pgfqpoint{7.262959in}{2.226435in}}%
\pgfpathclose%
\pgfusepath{fill}%
\end{pgfscope}%
\begin{pgfscope}%
\pgfpathrectangle{\pgfqpoint{6.572727in}{0.474100in}}{\pgfqpoint{4.227273in}{3.318700in}}%
\pgfusepath{clip}%
\pgfsetbuttcap%
\pgfsetroundjoin%
\definecolor{currentfill}{rgb}{0.127568,0.566949,0.550556}%
\pgfsetfillcolor{currentfill}%
\pgfsetfillopacity{0.700000}%
\pgfsetlinewidth{0.000000pt}%
\definecolor{currentstroke}{rgb}{0.000000,0.000000,0.000000}%
\pgfsetstrokecolor{currentstroke}%
\pgfsetstrokeopacity{0.700000}%
\pgfsetdash{}{0pt}%
\pgfpathmoveto{\pgfqpoint{8.289269in}{1.385486in}}%
\pgfpathcurveto{\pgfqpoint{8.294313in}{1.385486in}}{\pgfqpoint{8.299151in}{1.387490in}}{\pgfqpoint{8.302717in}{1.391056in}}%
\pgfpathcurveto{\pgfqpoint{8.306284in}{1.394623in}}{\pgfqpoint{8.308288in}{1.399460in}}{\pgfqpoint{8.308288in}{1.404504in}}%
\pgfpathcurveto{\pgfqpoint{8.308288in}{1.409548in}}{\pgfqpoint{8.306284in}{1.414385in}}{\pgfqpoint{8.302717in}{1.417952in}}%
\pgfpathcurveto{\pgfqpoint{8.299151in}{1.421518in}}{\pgfqpoint{8.294313in}{1.423522in}}{\pgfqpoint{8.289269in}{1.423522in}}%
\pgfpathcurveto{\pgfqpoint{8.284226in}{1.423522in}}{\pgfqpoint{8.279388in}{1.421518in}}{\pgfqpoint{8.275822in}{1.417952in}}%
\pgfpathcurveto{\pgfqpoint{8.272255in}{1.414385in}}{\pgfqpoint{8.270251in}{1.409548in}}{\pgfqpoint{8.270251in}{1.404504in}}%
\pgfpathcurveto{\pgfqpoint{8.270251in}{1.399460in}}{\pgfqpoint{8.272255in}{1.394623in}}{\pgfqpoint{8.275822in}{1.391056in}}%
\pgfpathcurveto{\pgfqpoint{8.279388in}{1.387490in}}{\pgfqpoint{8.284226in}{1.385486in}}{\pgfqpoint{8.289269in}{1.385486in}}%
\pgfpathclose%
\pgfusepath{fill}%
\end{pgfscope}%
\begin{pgfscope}%
\pgfpathrectangle{\pgfqpoint{6.572727in}{0.474100in}}{\pgfqpoint{4.227273in}{3.318700in}}%
\pgfusepath{clip}%
\pgfsetbuttcap%
\pgfsetroundjoin%
\definecolor{currentfill}{rgb}{0.127568,0.566949,0.550556}%
\pgfsetfillcolor{currentfill}%
\pgfsetfillopacity{0.700000}%
\pgfsetlinewidth{0.000000pt}%
\definecolor{currentstroke}{rgb}{0.000000,0.000000,0.000000}%
\pgfsetstrokecolor{currentstroke}%
\pgfsetstrokeopacity{0.700000}%
\pgfsetdash{}{0pt}%
\pgfpathmoveto{\pgfqpoint{8.163009in}{2.507769in}}%
\pgfpathcurveto{\pgfqpoint{8.168052in}{2.507769in}}{\pgfqpoint{8.172890in}{2.509772in}}{\pgfqpoint{8.176457in}{2.513339in}}%
\pgfpathcurveto{\pgfqpoint{8.180023in}{2.516905in}}{\pgfqpoint{8.182027in}{2.521743in}}{\pgfqpoint{8.182027in}{2.526787in}}%
\pgfpathcurveto{\pgfqpoint{8.182027in}{2.531830in}}{\pgfqpoint{8.180023in}{2.536668in}}{\pgfqpoint{8.176457in}{2.540235in}}%
\pgfpathcurveto{\pgfqpoint{8.172890in}{2.543801in}}{\pgfqpoint{8.168052in}{2.545805in}}{\pgfqpoint{8.163009in}{2.545805in}}%
\pgfpathcurveto{\pgfqpoint{8.157965in}{2.545805in}}{\pgfqpoint{8.153127in}{2.543801in}}{\pgfqpoint{8.149561in}{2.540235in}}%
\pgfpathcurveto{\pgfqpoint{8.145994in}{2.536668in}}{\pgfqpoint{8.143991in}{2.531830in}}{\pgfqpoint{8.143991in}{2.526787in}}%
\pgfpathcurveto{\pgfqpoint{8.143991in}{2.521743in}}{\pgfqpoint{8.145994in}{2.516905in}}{\pgfqpoint{8.149561in}{2.513339in}}%
\pgfpathcurveto{\pgfqpoint{8.153127in}{2.509772in}}{\pgfqpoint{8.157965in}{2.507769in}}{\pgfqpoint{8.163009in}{2.507769in}}%
\pgfpathclose%
\pgfusepath{fill}%
\end{pgfscope}%
\begin{pgfscope}%
\pgfpathrectangle{\pgfqpoint{6.572727in}{0.474100in}}{\pgfqpoint{4.227273in}{3.318700in}}%
\pgfusepath{clip}%
\pgfsetbuttcap%
\pgfsetroundjoin%
\definecolor{currentfill}{rgb}{0.127568,0.566949,0.550556}%
\pgfsetfillcolor{currentfill}%
\pgfsetfillopacity{0.700000}%
\pgfsetlinewidth{0.000000pt}%
\definecolor{currentstroke}{rgb}{0.000000,0.000000,0.000000}%
\pgfsetstrokecolor{currentstroke}%
\pgfsetstrokeopacity{0.700000}%
\pgfsetdash{}{0pt}%
\pgfpathmoveto{\pgfqpoint{8.005111in}{2.639183in}}%
\pgfpathcurveto{\pgfqpoint{8.010155in}{2.639183in}}{\pgfqpoint{8.014993in}{2.641186in}}{\pgfqpoint{8.018559in}{2.644753in}}%
\pgfpathcurveto{\pgfqpoint{8.022126in}{2.648319in}}{\pgfqpoint{8.024129in}{2.653157in}}{\pgfqpoint{8.024129in}{2.658201in}}%
\pgfpathcurveto{\pgfqpoint{8.024129in}{2.663244in}}{\pgfqpoint{8.022126in}{2.668082in}}{\pgfqpoint{8.018559in}{2.671649in}}%
\pgfpathcurveto{\pgfqpoint{8.014993in}{2.675215in}}{\pgfqpoint{8.010155in}{2.677219in}}{\pgfqpoint{8.005111in}{2.677219in}}%
\pgfpathcurveto{\pgfqpoint{8.000068in}{2.677219in}}{\pgfqpoint{7.995230in}{2.675215in}}{\pgfqpoint{7.991663in}{2.671649in}}%
\pgfpathcurveto{\pgfqpoint{7.988097in}{2.668082in}}{\pgfqpoint{7.986093in}{2.663244in}}{\pgfqpoint{7.986093in}{2.658201in}}%
\pgfpathcurveto{\pgfqpoint{7.986093in}{2.653157in}}{\pgfqpoint{7.988097in}{2.648319in}}{\pgfqpoint{7.991663in}{2.644753in}}%
\pgfpathcurveto{\pgfqpoint{7.995230in}{2.641186in}}{\pgfqpoint{8.000068in}{2.639183in}}{\pgfqpoint{8.005111in}{2.639183in}}%
\pgfpathclose%
\pgfusepath{fill}%
\end{pgfscope}%
\begin{pgfscope}%
\pgfpathrectangle{\pgfqpoint{6.572727in}{0.474100in}}{\pgfqpoint{4.227273in}{3.318700in}}%
\pgfusepath{clip}%
\pgfsetbuttcap%
\pgfsetroundjoin%
\definecolor{currentfill}{rgb}{0.127568,0.566949,0.550556}%
\pgfsetfillcolor{currentfill}%
\pgfsetfillopacity{0.700000}%
\pgfsetlinewidth{0.000000pt}%
\definecolor{currentstroke}{rgb}{0.000000,0.000000,0.000000}%
\pgfsetstrokecolor{currentstroke}%
\pgfsetstrokeopacity{0.700000}%
\pgfsetdash{}{0pt}%
\pgfpathmoveto{\pgfqpoint{7.308244in}{1.984365in}}%
\pgfpathcurveto{\pgfqpoint{7.313287in}{1.984365in}}{\pgfqpoint{7.318125in}{1.986368in}}{\pgfqpoint{7.321692in}{1.989935in}}%
\pgfpathcurveto{\pgfqpoint{7.325258in}{1.993501in}}{\pgfqpoint{7.327262in}{1.998339in}}{\pgfqpoint{7.327262in}{2.003383in}}%
\pgfpathcurveto{\pgfqpoint{7.327262in}{2.008426in}}{\pgfqpoint{7.325258in}{2.013264in}}{\pgfqpoint{7.321692in}{2.016831in}}%
\pgfpathcurveto{\pgfqpoint{7.318125in}{2.020397in}}{\pgfqpoint{7.313287in}{2.022401in}}{\pgfqpoint{7.308244in}{2.022401in}}%
\pgfpathcurveto{\pgfqpoint{7.303200in}{2.022401in}}{\pgfqpoint{7.298362in}{2.020397in}}{\pgfqpoint{7.294796in}{2.016831in}}%
\pgfpathcurveto{\pgfqpoint{7.291229in}{2.013264in}}{\pgfqpoint{7.289226in}{2.008426in}}{\pgfqpoint{7.289226in}{2.003383in}}%
\pgfpathcurveto{\pgfqpoint{7.289226in}{1.998339in}}{\pgfqpoint{7.291229in}{1.993501in}}{\pgfqpoint{7.294796in}{1.989935in}}%
\pgfpathcurveto{\pgfqpoint{7.298362in}{1.986368in}}{\pgfqpoint{7.303200in}{1.984365in}}{\pgfqpoint{7.308244in}{1.984365in}}%
\pgfpathclose%
\pgfusepath{fill}%
\end{pgfscope}%
\begin{pgfscope}%
\pgfpathrectangle{\pgfqpoint{6.572727in}{0.474100in}}{\pgfqpoint{4.227273in}{3.318700in}}%
\pgfusepath{clip}%
\pgfsetbuttcap%
\pgfsetroundjoin%
\definecolor{currentfill}{rgb}{0.127568,0.566949,0.550556}%
\pgfsetfillcolor{currentfill}%
\pgfsetfillopacity{0.700000}%
\pgfsetlinewidth{0.000000pt}%
\definecolor{currentstroke}{rgb}{0.000000,0.000000,0.000000}%
\pgfsetstrokecolor{currentstroke}%
\pgfsetstrokeopacity{0.700000}%
\pgfsetdash{}{0pt}%
\pgfpathmoveto{\pgfqpoint{7.376047in}{2.820774in}}%
\pgfpathcurveto{\pgfqpoint{7.381090in}{2.820774in}}{\pgfqpoint{7.385928in}{2.822777in}}{\pgfqpoint{7.389495in}{2.826344in}}%
\pgfpathcurveto{\pgfqpoint{7.393061in}{2.829910in}}{\pgfqpoint{7.395065in}{2.834748in}}{\pgfqpoint{7.395065in}{2.839792in}}%
\pgfpathcurveto{\pgfqpoint{7.395065in}{2.844835in}}{\pgfqpoint{7.393061in}{2.849673in}}{\pgfqpoint{7.389495in}{2.853240in}}%
\pgfpathcurveto{\pgfqpoint{7.385928in}{2.856806in}}{\pgfqpoint{7.381090in}{2.858810in}}{\pgfqpoint{7.376047in}{2.858810in}}%
\pgfpathcurveto{\pgfqpoint{7.371003in}{2.858810in}}{\pgfqpoint{7.366165in}{2.856806in}}{\pgfqpoint{7.362599in}{2.853240in}}%
\pgfpathcurveto{\pgfqpoint{7.359032in}{2.849673in}}{\pgfqpoint{7.357029in}{2.844835in}}{\pgfqpoint{7.357029in}{2.839792in}}%
\pgfpathcurveto{\pgfqpoint{7.357029in}{2.834748in}}{\pgfqpoint{7.359032in}{2.829910in}}{\pgfqpoint{7.362599in}{2.826344in}}%
\pgfpathcurveto{\pgfqpoint{7.366165in}{2.822777in}}{\pgfqpoint{7.371003in}{2.820774in}}{\pgfqpoint{7.376047in}{2.820774in}}%
\pgfpathclose%
\pgfusepath{fill}%
\end{pgfscope}%
\begin{pgfscope}%
\pgfpathrectangle{\pgfqpoint{6.572727in}{0.474100in}}{\pgfqpoint{4.227273in}{3.318700in}}%
\pgfusepath{clip}%
\pgfsetbuttcap%
\pgfsetroundjoin%
\definecolor{currentfill}{rgb}{0.993248,0.906157,0.143936}%
\pgfsetfillcolor{currentfill}%
\pgfsetfillopacity{0.700000}%
\pgfsetlinewidth{0.000000pt}%
\definecolor{currentstroke}{rgb}{0.000000,0.000000,0.000000}%
\pgfsetstrokecolor{currentstroke}%
\pgfsetstrokeopacity{0.700000}%
\pgfsetdash{}{0pt}%
\pgfpathmoveto{\pgfqpoint{9.555297in}{1.832976in}}%
\pgfpathcurveto{\pgfqpoint{9.560341in}{1.832976in}}{\pgfqpoint{9.565179in}{1.834980in}}{\pgfqpoint{9.568745in}{1.838546in}}%
\pgfpathcurveto{\pgfqpoint{9.572312in}{1.842112in}}{\pgfqpoint{9.574315in}{1.846950in}}{\pgfqpoint{9.574315in}{1.851994in}}%
\pgfpathcurveto{\pgfqpoint{9.574315in}{1.857038in}}{\pgfqpoint{9.572312in}{1.861875in}}{\pgfqpoint{9.568745in}{1.865442in}}%
\pgfpathcurveto{\pgfqpoint{9.565179in}{1.869008in}}{\pgfqpoint{9.560341in}{1.871012in}}{\pgfqpoint{9.555297in}{1.871012in}}%
\pgfpathcurveto{\pgfqpoint{9.550254in}{1.871012in}}{\pgfqpoint{9.545416in}{1.869008in}}{\pgfqpoint{9.541849in}{1.865442in}}%
\pgfpathcurveto{\pgfqpoint{9.538283in}{1.861875in}}{\pgfqpoint{9.536279in}{1.857038in}}{\pgfqpoint{9.536279in}{1.851994in}}%
\pgfpathcurveto{\pgfqpoint{9.536279in}{1.846950in}}{\pgfqpoint{9.538283in}{1.842112in}}{\pgfqpoint{9.541849in}{1.838546in}}%
\pgfpathcurveto{\pgfqpoint{9.545416in}{1.834980in}}{\pgfqpoint{9.550254in}{1.832976in}}{\pgfqpoint{9.555297in}{1.832976in}}%
\pgfpathclose%
\pgfusepath{fill}%
\end{pgfscope}%
\begin{pgfscope}%
\pgfpathrectangle{\pgfqpoint{6.572727in}{0.474100in}}{\pgfqpoint{4.227273in}{3.318700in}}%
\pgfusepath{clip}%
\pgfsetbuttcap%
\pgfsetroundjoin%
\definecolor{currentfill}{rgb}{0.993248,0.906157,0.143936}%
\pgfsetfillcolor{currentfill}%
\pgfsetfillopacity{0.700000}%
\pgfsetlinewidth{0.000000pt}%
\definecolor{currentstroke}{rgb}{0.000000,0.000000,0.000000}%
\pgfsetstrokecolor{currentstroke}%
\pgfsetstrokeopacity{0.700000}%
\pgfsetdash{}{0pt}%
\pgfpathmoveto{\pgfqpoint{9.942156in}{1.618216in}}%
\pgfpathcurveto{\pgfqpoint{9.947199in}{1.618216in}}{\pgfqpoint{9.952037in}{1.620220in}}{\pgfqpoint{9.955604in}{1.623786in}}%
\pgfpathcurveto{\pgfqpoint{9.959170in}{1.627352in}}{\pgfqpoint{9.961174in}{1.632190in}}{\pgfqpoint{9.961174in}{1.637234in}}%
\pgfpathcurveto{\pgfqpoint{9.961174in}{1.642277in}}{\pgfqpoint{9.959170in}{1.647115in}}{\pgfqpoint{9.955604in}{1.650682in}}%
\pgfpathcurveto{\pgfqpoint{9.952037in}{1.654248in}}{\pgfqpoint{9.947199in}{1.656252in}}{\pgfqpoint{9.942156in}{1.656252in}}%
\pgfpathcurveto{\pgfqpoint{9.937112in}{1.656252in}}{\pgfqpoint{9.932274in}{1.654248in}}{\pgfqpoint{9.928708in}{1.650682in}}%
\pgfpathcurveto{\pgfqpoint{9.925141in}{1.647115in}}{\pgfqpoint{9.923138in}{1.642277in}}{\pgfqpoint{9.923138in}{1.637234in}}%
\pgfpathcurveto{\pgfqpoint{9.923138in}{1.632190in}}{\pgfqpoint{9.925141in}{1.627352in}}{\pgfqpoint{9.928708in}{1.623786in}}%
\pgfpathcurveto{\pgfqpoint{9.932274in}{1.620220in}}{\pgfqpoint{9.937112in}{1.618216in}}{\pgfqpoint{9.942156in}{1.618216in}}%
\pgfpathclose%
\pgfusepath{fill}%
\end{pgfscope}%
\begin{pgfscope}%
\pgfpathrectangle{\pgfqpoint{6.572727in}{0.474100in}}{\pgfqpoint{4.227273in}{3.318700in}}%
\pgfusepath{clip}%
\pgfsetbuttcap%
\pgfsetroundjoin%
\definecolor{currentfill}{rgb}{0.127568,0.566949,0.550556}%
\pgfsetfillcolor{currentfill}%
\pgfsetfillopacity{0.700000}%
\pgfsetlinewidth{0.000000pt}%
\definecolor{currentstroke}{rgb}{0.000000,0.000000,0.000000}%
\pgfsetstrokecolor{currentstroke}%
\pgfsetstrokeopacity{0.700000}%
\pgfsetdash{}{0pt}%
\pgfpathmoveto{\pgfqpoint{7.526435in}{1.784472in}}%
\pgfpathcurveto{\pgfqpoint{7.531478in}{1.784472in}}{\pgfqpoint{7.536316in}{1.786476in}}{\pgfqpoint{7.539883in}{1.790042in}}%
\pgfpathcurveto{\pgfqpoint{7.543449in}{1.793609in}}{\pgfqpoint{7.545453in}{1.798446in}}{\pgfqpoint{7.545453in}{1.803490in}}%
\pgfpathcurveto{\pgfqpoint{7.545453in}{1.808534in}}{\pgfqpoint{7.543449in}{1.813372in}}{\pgfqpoint{7.539883in}{1.816938in}}%
\pgfpathcurveto{\pgfqpoint{7.536316in}{1.820504in}}{\pgfqpoint{7.531478in}{1.822508in}}{\pgfqpoint{7.526435in}{1.822508in}}%
\pgfpathcurveto{\pgfqpoint{7.521391in}{1.822508in}}{\pgfqpoint{7.516553in}{1.820504in}}{\pgfqpoint{7.512987in}{1.816938in}}%
\pgfpathcurveto{\pgfqpoint{7.509420in}{1.813372in}}{\pgfqpoint{7.507417in}{1.808534in}}{\pgfqpoint{7.507417in}{1.803490in}}%
\pgfpathcurveto{\pgfqpoint{7.507417in}{1.798446in}}{\pgfqpoint{7.509420in}{1.793609in}}{\pgfqpoint{7.512987in}{1.790042in}}%
\pgfpathcurveto{\pgfqpoint{7.516553in}{1.786476in}}{\pgfqpoint{7.521391in}{1.784472in}}{\pgfqpoint{7.526435in}{1.784472in}}%
\pgfpathclose%
\pgfusepath{fill}%
\end{pgfscope}%
\begin{pgfscope}%
\pgfpathrectangle{\pgfqpoint{6.572727in}{0.474100in}}{\pgfqpoint{4.227273in}{3.318700in}}%
\pgfusepath{clip}%
\pgfsetbuttcap%
\pgfsetroundjoin%
\definecolor{currentfill}{rgb}{0.127568,0.566949,0.550556}%
\pgfsetfillcolor{currentfill}%
\pgfsetfillopacity{0.700000}%
\pgfsetlinewidth{0.000000pt}%
\definecolor{currentstroke}{rgb}{0.000000,0.000000,0.000000}%
\pgfsetstrokecolor{currentstroke}%
\pgfsetstrokeopacity{0.700000}%
\pgfsetdash{}{0pt}%
\pgfpathmoveto{\pgfqpoint{7.469826in}{1.255008in}}%
\pgfpathcurveto{\pgfqpoint{7.474869in}{1.255008in}}{\pgfqpoint{7.479707in}{1.257012in}}{\pgfqpoint{7.483273in}{1.260578in}}%
\pgfpathcurveto{\pgfqpoint{7.486840in}{1.264145in}}{\pgfqpoint{7.488844in}{1.268982in}}{\pgfqpoint{7.488844in}{1.274026in}}%
\pgfpathcurveto{\pgfqpoint{7.488844in}{1.279070in}}{\pgfqpoint{7.486840in}{1.283907in}}{\pgfqpoint{7.483273in}{1.287474in}}%
\pgfpathcurveto{\pgfqpoint{7.479707in}{1.291040in}}{\pgfqpoint{7.474869in}{1.293044in}}{\pgfqpoint{7.469826in}{1.293044in}}%
\pgfpathcurveto{\pgfqpoint{7.464782in}{1.293044in}}{\pgfqpoint{7.459944in}{1.291040in}}{\pgfqpoint{7.456378in}{1.287474in}}%
\pgfpathcurveto{\pgfqpoint{7.452811in}{1.283907in}}{\pgfqpoint{7.450807in}{1.279070in}}{\pgfqpoint{7.450807in}{1.274026in}}%
\pgfpathcurveto{\pgfqpoint{7.450807in}{1.268982in}}{\pgfqpoint{7.452811in}{1.264145in}}{\pgfqpoint{7.456378in}{1.260578in}}%
\pgfpathcurveto{\pgfqpoint{7.459944in}{1.257012in}}{\pgfqpoint{7.464782in}{1.255008in}}{\pgfqpoint{7.469826in}{1.255008in}}%
\pgfpathclose%
\pgfusepath{fill}%
\end{pgfscope}%
\begin{pgfscope}%
\pgfpathrectangle{\pgfqpoint{6.572727in}{0.474100in}}{\pgfqpoint{4.227273in}{3.318700in}}%
\pgfusepath{clip}%
\pgfsetbuttcap%
\pgfsetroundjoin%
\definecolor{currentfill}{rgb}{0.993248,0.906157,0.143936}%
\pgfsetfillcolor{currentfill}%
\pgfsetfillopacity{0.700000}%
\pgfsetlinewidth{0.000000pt}%
\definecolor{currentstroke}{rgb}{0.000000,0.000000,0.000000}%
\pgfsetstrokecolor{currentstroke}%
\pgfsetstrokeopacity{0.700000}%
\pgfsetdash{}{0pt}%
\pgfpathmoveto{\pgfqpoint{9.474662in}{1.874821in}}%
\pgfpathcurveto{\pgfqpoint{9.479706in}{1.874821in}}{\pgfqpoint{9.484544in}{1.876825in}}{\pgfqpoint{9.488110in}{1.880391in}}%
\pgfpathcurveto{\pgfqpoint{9.491677in}{1.883958in}}{\pgfqpoint{9.493681in}{1.888796in}}{\pgfqpoint{9.493681in}{1.893839in}}%
\pgfpathcurveto{\pgfqpoint{9.493681in}{1.898883in}}{\pgfqpoint{9.491677in}{1.903721in}}{\pgfqpoint{9.488110in}{1.907287in}}%
\pgfpathcurveto{\pgfqpoint{9.484544in}{1.910854in}}{\pgfqpoint{9.479706in}{1.912857in}}{\pgfqpoint{9.474662in}{1.912857in}}%
\pgfpathcurveto{\pgfqpoint{9.469619in}{1.912857in}}{\pgfqpoint{9.464781in}{1.910854in}}{\pgfqpoint{9.461215in}{1.907287in}}%
\pgfpathcurveto{\pgfqpoint{9.457648in}{1.903721in}}{\pgfqpoint{9.455644in}{1.898883in}}{\pgfqpoint{9.455644in}{1.893839in}}%
\pgfpathcurveto{\pgfqpoint{9.455644in}{1.888796in}}{\pgfqpoint{9.457648in}{1.883958in}}{\pgfqpoint{9.461215in}{1.880391in}}%
\pgfpathcurveto{\pgfqpoint{9.464781in}{1.876825in}}{\pgfqpoint{9.469619in}{1.874821in}}{\pgfqpoint{9.474662in}{1.874821in}}%
\pgfpathclose%
\pgfusepath{fill}%
\end{pgfscope}%
\begin{pgfscope}%
\pgfpathrectangle{\pgfqpoint{6.572727in}{0.474100in}}{\pgfqpoint{4.227273in}{3.318700in}}%
\pgfusepath{clip}%
\pgfsetbuttcap%
\pgfsetroundjoin%
\definecolor{currentfill}{rgb}{0.127568,0.566949,0.550556}%
\pgfsetfillcolor{currentfill}%
\pgfsetfillopacity{0.700000}%
\pgfsetlinewidth{0.000000pt}%
\definecolor{currentstroke}{rgb}{0.000000,0.000000,0.000000}%
\pgfsetstrokecolor{currentstroke}%
\pgfsetstrokeopacity{0.700000}%
\pgfsetdash{}{0pt}%
\pgfpathmoveto{\pgfqpoint{7.870710in}{2.787848in}}%
\pgfpathcurveto{\pgfqpoint{7.875753in}{2.787848in}}{\pgfqpoint{7.880591in}{2.789851in}}{\pgfqpoint{7.884158in}{2.793418in}}%
\pgfpathcurveto{\pgfqpoint{7.887724in}{2.796984in}}{\pgfqpoint{7.889728in}{2.801822in}}{\pgfqpoint{7.889728in}{2.806866in}}%
\pgfpathcurveto{\pgfqpoint{7.889728in}{2.811909in}}{\pgfqpoint{7.887724in}{2.816747in}}{\pgfqpoint{7.884158in}{2.820314in}}%
\pgfpathcurveto{\pgfqpoint{7.880591in}{2.823880in}}{\pgfqpoint{7.875753in}{2.825884in}}{\pgfqpoint{7.870710in}{2.825884in}}%
\pgfpathcurveto{\pgfqpoint{7.865666in}{2.825884in}}{\pgfqpoint{7.860828in}{2.823880in}}{\pgfqpoint{7.857262in}{2.820314in}}%
\pgfpathcurveto{\pgfqpoint{7.853695in}{2.816747in}}{\pgfqpoint{7.851692in}{2.811909in}}{\pgfqpoint{7.851692in}{2.806866in}}%
\pgfpathcurveto{\pgfqpoint{7.851692in}{2.801822in}}{\pgfqpoint{7.853695in}{2.796984in}}{\pgfqpoint{7.857262in}{2.793418in}}%
\pgfpathcurveto{\pgfqpoint{7.860828in}{2.789851in}}{\pgfqpoint{7.865666in}{2.787848in}}{\pgfqpoint{7.870710in}{2.787848in}}%
\pgfpathclose%
\pgfusepath{fill}%
\end{pgfscope}%
\begin{pgfscope}%
\pgfpathrectangle{\pgfqpoint{6.572727in}{0.474100in}}{\pgfqpoint{4.227273in}{3.318700in}}%
\pgfusepath{clip}%
\pgfsetbuttcap%
\pgfsetroundjoin%
\definecolor{currentfill}{rgb}{0.127568,0.566949,0.550556}%
\pgfsetfillcolor{currentfill}%
\pgfsetfillopacity{0.700000}%
\pgfsetlinewidth{0.000000pt}%
\definecolor{currentstroke}{rgb}{0.000000,0.000000,0.000000}%
\pgfsetstrokecolor{currentstroke}%
\pgfsetstrokeopacity{0.700000}%
\pgfsetdash{}{0pt}%
\pgfpathmoveto{\pgfqpoint{8.498297in}{3.250508in}}%
\pgfpathcurveto{\pgfqpoint{8.503341in}{3.250508in}}{\pgfqpoint{8.508178in}{3.252512in}}{\pgfqpoint{8.511745in}{3.256078in}}%
\pgfpathcurveto{\pgfqpoint{8.515311in}{3.259644in}}{\pgfqpoint{8.517315in}{3.264482in}}{\pgfqpoint{8.517315in}{3.269526in}}%
\pgfpathcurveto{\pgfqpoint{8.517315in}{3.274570in}}{\pgfqpoint{8.515311in}{3.279407in}}{\pgfqpoint{8.511745in}{3.282974in}}%
\pgfpathcurveto{\pgfqpoint{8.508178in}{3.286540in}}{\pgfqpoint{8.503341in}{3.288544in}}{\pgfqpoint{8.498297in}{3.288544in}}%
\pgfpathcurveto{\pgfqpoint{8.493253in}{3.288544in}}{\pgfqpoint{8.488416in}{3.286540in}}{\pgfqpoint{8.484849in}{3.282974in}}%
\pgfpathcurveto{\pgfqpoint{8.481283in}{3.279407in}}{\pgfqpoint{8.479279in}{3.274570in}}{\pgfqpoint{8.479279in}{3.269526in}}%
\pgfpathcurveto{\pgfqpoint{8.479279in}{3.264482in}}{\pgfqpoint{8.481283in}{3.259644in}}{\pgfqpoint{8.484849in}{3.256078in}}%
\pgfpathcurveto{\pgfqpoint{8.488416in}{3.252512in}}{\pgfqpoint{8.493253in}{3.250508in}}{\pgfqpoint{8.498297in}{3.250508in}}%
\pgfpathclose%
\pgfusepath{fill}%
\end{pgfscope}%
\begin{pgfscope}%
\pgfpathrectangle{\pgfqpoint{6.572727in}{0.474100in}}{\pgfqpoint{4.227273in}{3.318700in}}%
\pgfusepath{clip}%
\pgfsetbuttcap%
\pgfsetroundjoin%
\definecolor{currentfill}{rgb}{0.127568,0.566949,0.550556}%
\pgfsetfillcolor{currentfill}%
\pgfsetfillopacity{0.700000}%
\pgfsetlinewidth{0.000000pt}%
\definecolor{currentstroke}{rgb}{0.000000,0.000000,0.000000}%
\pgfsetstrokecolor{currentstroke}%
\pgfsetstrokeopacity{0.700000}%
\pgfsetdash{}{0pt}%
\pgfpathmoveto{\pgfqpoint{8.251208in}{1.723952in}}%
\pgfpathcurveto{\pgfqpoint{8.256252in}{1.723952in}}{\pgfqpoint{8.261089in}{1.725956in}}{\pgfqpoint{8.264656in}{1.729522in}}%
\pgfpathcurveto{\pgfqpoint{8.268222in}{1.733089in}}{\pgfqpoint{8.270226in}{1.737927in}}{\pgfqpoint{8.270226in}{1.742970in}}%
\pgfpathcurveto{\pgfqpoint{8.270226in}{1.748014in}}{\pgfqpoint{8.268222in}{1.752852in}}{\pgfqpoint{8.264656in}{1.756418in}}%
\pgfpathcurveto{\pgfqpoint{8.261089in}{1.759984in}}{\pgfqpoint{8.256252in}{1.761988in}}{\pgfqpoint{8.251208in}{1.761988in}}%
\pgfpathcurveto{\pgfqpoint{8.246164in}{1.761988in}}{\pgfqpoint{8.241326in}{1.759984in}}{\pgfqpoint{8.237760in}{1.756418in}}%
\pgfpathcurveto{\pgfqpoint{8.234194in}{1.752852in}}{\pgfqpoint{8.232190in}{1.748014in}}{\pgfqpoint{8.232190in}{1.742970in}}%
\pgfpathcurveto{\pgfqpoint{8.232190in}{1.737927in}}{\pgfqpoint{8.234194in}{1.733089in}}{\pgfqpoint{8.237760in}{1.729522in}}%
\pgfpathcurveto{\pgfqpoint{8.241326in}{1.725956in}}{\pgfqpoint{8.246164in}{1.723952in}}{\pgfqpoint{8.251208in}{1.723952in}}%
\pgfpathclose%
\pgfusepath{fill}%
\end{pgfscope}%
\begin{pgfscope}%
\pgfpathrectangle{\pgfqpoint{6.572727in}{0.474100in}}{\pgfqpoint{4.227273in}{3.318700in}}%
\pgfusepath{clip}%
\pgfsetbuttcap%
\pgfsetroundjoin%
\definecolor{currentfill}{rgb}{0.127568,0.566949,0.550556}%
\pgfsetfillcolor{currentfill}%
\pgfsetfillopacity{0.700000}%
\pgfsetlinewidth{0.000000pt}%
\definecolor{currentstroke}{rgb}{0.000000,0.000000,0.000000}%
\pgfsetstrokecolor{currentstroke}%
\pgfsetstrokeopacity{0.700000}%
\pgfsetdash{}{0pt}%
\pgfpathmoveto{\pgfqpoint{8.206880in}{2.951642in}}%
\pgfpathcurveto{\pgfqpoint{8.211923in}{2.951642in}}{\pgfqpoint{8.216761in}{2.953646in}}{\pgfqpoint{8.220328in}{2.957213in}}%
\pgfpathcurveto{\pgfqpoint{8.223894in}{2.960779in}}{\pgfqpoint{8.225898in}{2.965617in}}{\pgfqpoint{8.225898in}{2.970661in}}%
\pgfpathcurveto{\pgfqpoint{8.225898in}{2.975704in}}{\pgfqpoint{8.223894in}{2.980542in}}{\pgfqpoint{8.220328in}{2.984108in}}%
\pgfpathcurveto{\pgfqpoint{8.216761in}{2.987675in}}{\pgfqpoint{8.211923in}{2.989679in}}{\pgfqpoint{8.206880in}{2.989679in}}%
\pgfpathcurveto{\pgfqpoint{8.201836in}{2.989679in}}{\pgfqpoint{8.196998in}{2.987675in}}{\pgfqpoint{8.193432in}{2.984108in}}%
\pgfpathcurveto{\pgfqpoint{8.189865in}{2.980542in}}{\pgfqpoint{8.187862in}{2.975704in}}{\pgfqpoint{8.187862in}{2.970661in}}%
\pgfpathcurveto{\pgfqpoint{8.187862in}{2.965617in}}{\pgfqpoint{8.189865in}{2.960779in}}{\pgfqpoint{8.193432in}{2.957213in}}%
\pgfpathcurveto{\pgfqpoint{8.196998in}{2.953646in}}{\pgfqpoint{8.201836in}{2.951642in}}{\pgfqpoint{8.206880in}{2.951642in}}%
\pgfpathclose%
\pgfusepath{fill}%
\end{pgfscope}%
\begin{pgfscope}%
\pgfpathrectangle{\pgfqpoint{6.572727in}{0.474100in}}{\pgfqpoint{4.227273in}{3.318700in}}%
\pgfusepath{clip}%
\pgfsetbuttcap%
\pgfsetroundjoin%
\definecolor{currentfill}{rgb}{0.127568,0.566949,0.550556}%
\pgfsetfillcolor{currentfill}%
\pgfsetfillopacity{0.700000}%
\pgfsetlinewidth{0.000000pt}%
\definecolor{currentstroke}{rgb}{0.000000,0.000000,0.000000}%
\pgfsetstrokecolor{currentstroke}%
\pgfsetstrokeopacity{0.700000}%
\pgfsetdash{}{0pt}%
\pgfpathmoveto{\pgfqpoint{8.291446in}{2.948755in}}%
\pgfpathcurveto{\pgfqpoint{8.296490in}{2.948755in}}{\pgfqpoint{8.301327in}{2.950759in}}{\pgfqpoint{8.304894in}{2.954326in}}%
\pgfpathcurveto{\pgfqpoint{8.308460in}{2.957892in}}{\pgfqpoint{8.310464in}{2.962730in}}{\pgfqpoint{8.310464in}{2.967774in}}%
\pgfpathcurveto{\pgfqpoint{8.310464in}{2.972817in}}{\pgfqpoint{8.308460in}{2.977655in}}{\pgfqpoint{8.304894in}{2.981221in}}%
\pgfpathcurveto{\pgfqpoint{8.301327in}{2.984788in}}{\pgfqpoint{8.296490in}{2.986792in}}{\pgfqpoint{8.291446in}{2.986792in}}%
\pgfpathcurveto{\pgfqpoint{8.286402in}{2.986792in}}{\pgfqpoint{8.281565in}{2.984788in}}{\pgfqpoint{8.277998in}{2.981221in}}%
\pgfpathcurveto{\pgfqpoint{8.274432in}{2.977655in}}{\pgfqpoint{8.272428in}{2.972817in}}{\pgfqpoint{8.272428in}{2.967774in}}%
\pgfpathcurveto{\pgfqpoint{8.272428in}{2.962730in}}{\pgfqpoint{8.274432in}{2.957892in}}{\pgfqpoint{8.277998in}{2.954326in}}%
\pgfpathcurveto{\pgfqpoint{8.281565in}{2.950759in}}{\pgfqpoint{8.286402in}{2.948755in}}{\pgfqpoint{8.291446in}{2.948755in}}%
\pgfpathclose%
\pgfusepath{fill}%
\end{pgfscope}%
\begin{pgfscope}%
\pgfpathrectangle{\pgfqpoint{6.572727in}{0.474100in}}{\pgfqpoint{4.227273in}{3.318700in}}%
\pgfusepath{clip}%
\pgfsetbuttcap%
\pgfsetroundjoin%
\definecolor{currentfill}{rgb}{0.127568,0.566949,0.550556}%
\pgfsetfillcolor{currentfill}%
\pgfsetfillopacity{0.700000}%
\pgfsetlinewidth{0.000000pt}%
\definecolor{currentstroke}{rgb}{0.000000,0.000000,0.000000}%
\pgfsetstrokecolor{currentstroke}%
\pgfsetstrokeopacity{0.700000}%
\pgfsetdash{}{0pt}%
\pgfpathmoveto{\pgfqpoint{8.238555in}{2.941097in}}%
\pgfpathcurveto{\pgfqpoint{8.243599in}{2.941097in}}{\pgfqpoint{8.248437in}{2.943101in}}{\pgfqpoint{8.252003in}{2.946667in}}%
\pgfpathcurveto{\pgfqpoint{8.255569in}{2.950234in}}{\pgfqpoint{8.257573in}{2.955072in}}{\pgfqpoint{8.257573in}{2.960115in}}%
\pgfpathcurveto{\pgfqpoint{8.257573in}{2.965159in}}{\pgfqpoint{8.255569in}{2.969997in}}{\pgfqpoint{8.252003in}{2.973563in}}%
\pgfpathcurveto{\pgfqpoint{8.248437in}{2.977130in}}{\pgfqpoint{8.243599in}{2.979133in}}{\pgfqpoint{8.238555in}{2.979133in}}%
\pgfpathcurveto{\pgfqpoint{8.233511in}{2.979133in}}{\pgfqpoint{8.228674in}{2.977130in}}{\pgfqpoint{8.225107in}{2.973563in}}%
\pgfpathcurveto{\pgfqpoint{8.221541in}{2.969997in}}{\pgfqpoint{8.219537in}{2.965159in}}{\pgfqpoint{8.219537in}{2.960115in}}%
\pgfpathcurveto{\pgfqpoint{8.219537in}{2.955072in}}{\pgfqpoint{8.221541in}{2.950234in}}{\pgfqpoint{8.225107in}{2.946667in}}%
\pgfpathcurveto{\pgfqpoint{8.228674in}{2.943101in}}{\pgfqpoint{8.233511in}{2.941097in}}{\pgfqpoint{8.238555in}{2.941097in}}%
\pgfpathclose%
\pgfusepath{fill}%
\end{pgfscope}%
\begin{pgfscope}%
\pgfpathrectangle{\pgfqpoint{6.572727in}{0.474100in}}{\pgfqpoint{4.227273in}{3.318700in}}%
\pgfusepath{clip}%
\pgfsetbuttcap%
\pgfsetroundjoin%
\definecolor{currentfill}{rgb}{0.993248,0.906157,0.143936}%
\pgfsetfillcolor{currentfill}%
\pgfsetfillopacity{0.700000}%
\pgfsetlinewidth{0.000000pt}%
\definecolor{currentstroke}{rgb}{0.000000,0.000000,0.000000}%
\pgfsetstrokecolor{currentstroke}%
\pgfsetstrokeopacity{0.700000}%
\pgfsetdash{}{0pt}%
\pgfpathmoveto{\pgfqpoint{9.926121in}{1.456416in}}%
\pgfpathcurveto{\pgfqpoint{9.931165in}{1.456416in}}{\pgfqpoint{9.936003in}{1.458420in}}{\pgfqpoint{9.939569in}{1.461986in}}%
\pgfpathcurveto{\pgfqpoint{9.943136in}{1.465552in}}{\pgfqpoint{9.945140in}{1.470390in}}{\pgfqpoint{9.945140in}{1.475434in}}%
\pgfpathcurveto{\pgfqpoint{9.945140in}{1.480477in}}{\pgfqpoint{9.943136in}{1.485315in}}{\pgfqpoint{9.939569in}{1.488882in}}%
\pgfpathcurveto{\pgfqpoint{9.936003in}{1.492448in}}{\pgfqpoint{9.931165in}{1.494452in}}{\pgfqpoint{9.926121in}{1.494452in}}%
\pgfpathcurveto{\pgfqpoint{9.921078in}{1.494452in}}{\pgfqpoint{9.916240in}{1.492448in}}{\pgfqpoint{9.912674in}{1.488882in}}%
\pgfpathcurveto{\pgfqpoint{9.909107in}{1.485315in}}{\pgfqpoint{9.907103in}{1.480477in}}{\pgfqpoint{9.907103in}{1.475434in}}%
\pgfpathcurveto{\pgfqpoint{9.907103in}{1.470390in}}{\pgfqpoint{9.909107in}{1.465552in}}{\pgfqpoint{9.912674in}{1.461986in}}%
\pgfpathcurveto{\pgfqpoint{9.916240in}{1.458420in}}{\pgfqpoint{9.921078in}{1.456416in}}{\pgfqpoint{9.926121in}{1.456416in}}%
\pgfpathclose%
\pgfusepath{fill}%
\end{pgfscope}%
\begin{pgfscope}%
\pgfpathrectangle{\pgfqpoint{6.572727in}{0.474100in}}{\pgfqpoint{4.227273in}{3.318700in}}%
\pgfusepath{clip}%
\pgfsetbuttcap%
\pgfsetroundjoin%
\definecolor{currentfill}{rgb}{0.127568,0.566949,0.550556}%
\pgfsetfillcolor{currentfill}%
\pgfsetfillopacity{0.700000}%
\pgfsetlinewidth{0.000000pt}%
\definecolor{currentstroke}{rgb}{0.000000,0.000000,0.000000}%
\pgfsetstrokecolor{currentstroke}%
\pgfsetstrokeopacity{0.700000}%
\pgfsetdash{}{0pt}%
\pgfpathmoveto{\pgfqpoint{8.287123in}{2.686814in}}%
\pgfpathcurveto{\pgfqpoint{8.292167in}{2.686814in}}{\pgfqpoint{8.297004in}{2.688818in}}{\pgfqpoint{8.300571in}{2.692384in}}%
\pgfpathcurveto{\pgfqpoint{8.304137in}{2.695951in}}{\pgfqpoint{8.306141in}{2.700788in}}{\pgfqpoint{8.306141in}{2.705832in}}%
\pgfpathcurveto{\pgfqpoint{8.306141in}{2.710876in}}{\pgfqpoint{8.304137in}{2.715713in}}{\pgfqpoint{8.300571in}{2.719280in}}%
\pgfpathcurveto{\pgfqpoint{8.297004in}{2.722846in}}{\pgfqpoint{8.292167in}{2.724850in}}{\pgfqpoint{8.287123in}{2.724850in}}%
\pgfpathcurveto{\pgfqpoint{8.282079in}{2.724850in}}{\pgfqpoint{8.277242in}{2.722846in}}{\pgfqpoint{8.273675in}{2.719280in}}%
\pgfpathcurveto{\pgfqpoint{8.270109in}{2.715713in}}{\pgfqpoint{8.268105in}{2.710876in}}{\pgfqpoint{8.268105in}{2.705832in}}%
\pgfpathcurveto{\pgfqpoint{8.268105in}{2.700788in}}{\pgfqpoint{8.270109in}{2.695951in}}{\pgfqpoint{8.273675in}{2.692384in}}%
\pgfpathcurveto{\pgfqpoint{8.277242in}{2.688818in}}{\pgfqpoint{8.282079in}{2.686814in}}{\pgfqpoint{8.287123in}{2.686814in}}%
\pgfpathclose%
\pgfusepath{fill}%
\end{pgfscope}%
\begin{pgfscope}%
\pgfpathrectangle{\pgfqpoint{6.572727in}{0.474100in}}{\pgfqpoint{4.227273in}{3.318700in}}%
\pgfusepath{clip}%
\pgfsetbuttcap%
\pgfsetroundjoin%
\definecolor{currentfill}{rgb}{0.993248,0.906157,0.143936}%
\pgfsetfillcolor{currentfill}%
\pgfsetfillopacity{0.700000}%
\pgfsetlinewidth{0.000000pt}%
\definecolor{currentstroke}{rgb}{0.000000,0.000000,0.000000}%
\pgfsetstrokecolor{currentstroke}%
\pgfsetstrokeopacity{0.700000}%
\pgfsetdash{}{0pt}%
\pgfpathmoveto{\pgfqpoint{9.393143in}{1.425542in}}%
\pgfpathcurveto{\pgfqpoint{9.398187in}{1.425542in}}{\pgfqpoint{9.403025in}{1.427546in}}{\pgfqpoint{9.406591in}{1.431112in}}%
\pgfpathcurveto{\pgfqpoint{9.410157in}{1.434678in}}{\pgfqpoint{9.412161in}{1.439516in}}{\pgfqpoint{9.412161in}{1.444560in}}%
\pgfpathcurveto{\pgfqpoint{9.412161in}{1.449603in}}{\pgfqpoint{9.410157in}{1.454441in}}{\pgfqpoint{9.406591in}{1.458008in}}%
\pgfpathcurveto{\pgfqpoint{9.403025in}{1.461574in}}{\pgfqpoint{9.398187in}{1.463578in}}{\pgfqpoint{9.393143in}{1.463578in}}%
\pgfpathcurveto{\pgfqpoint{9.388099in}{1.463578in}}{\pgfqpoint{9.383262in}{1.461574in}}{\pgfqpoint{9.379695in}{1.458008in}}%
\pgfpathcurveto{\pgfqpoint{9.376129in}{1.454441in}}{\pgfqpoint{9.374125in}{1.449603in}}{\pgfqpoint{9.374125in}{1.444560in}}%
\pgfpathcurveto{\pgfqpoint{9.374125in}{1.439516in}}{\pgfqpoint{9.376129in}{1.434678in}}{\pgfqpoint{9.379695in}{1.431112in}}%
\pgfpathcurveto{\pgfqpoint{9.383262in}{1.427546in}}{\pgfqpoint{9.388099in}{1.425542in}}{\pgfqpoint{9.393143in}{1.425542in}}%
\pgfpathclose%
\pgfusepath{fill}%
\end{pgfscope}%
\begin{pgfscope}%
\pgfpathrectangle{\pgfqpoint{6.572727in}{0.474100in}}{\pgfqpoint{4.227273in}{3.318700in}}%
\pgfusepath{clip}%
\pgfsetbuttcap%
\pgfsetroundjoin%
\definecolor{currentfill}{rgb}{0.127568,0.566949,0.550556}%
\pgfsetfillcolor{currentfill}%
\pgfsetfillopacity{0.700000}%
\pgfsetlinewidth{0.000000pt}%
\definecolor{currentstroke}{rgb}{0.000000,0.000000,0.000000}%
\pgfsetstrokecolor{currentstroke}%
\pgfsetstrokeopacity{0.700000}%
\pgfsetdash{}{0pt}%
\pgfpathmoveto{\pgfqpoint{8.808060in}{3.306680in}}%
\pgfpathcurveto{\pgfqpoint{8.813104in}{3.306680in}}{\pgfqpoint{8.817941in}{3.308684in}}{\pgfqpoint{8.821508in}{3.312251in}}%
\pgfpathcurveto{\pgfqpoint{8.825074in}{3.315817in}}{\pgfqpoint{8.827078in}{3.320655in}}{\pgfqpoint{8.827078in}{3.325698in}}%
\pgfpathcurveto{\pgfqpoint{8.827078in}{3.330742in}}{\pgfqpoint{8.825074in}{3.335580in}}{\pgfqpoint{8.821508in}{3.339146in}}%
\pgfpathcurveto{\pgfqpoint{8.817941in}{3.342713in}}{\pgfqpoint{8.813104in}{3.344717in}}{\pgfqpoint{8.808060in}{3.344717in}}%
\pgfpathcurveto{\pgfqpoint{8.803016in}{3.344717in}}{\pgfqpoint{8.798178in}{3.342713in}}{\pgfqpoint{8.794612in}{3.339146in}}%
\pgfpathcurveto{\pgfqpoint{8.791046in}{3.335580in}}{\pgfqpoint{8.789042in}{3.330742in}}{\pgfqpoint{8.789042in}{3.325698in}}%
\pgfpathcurveto{\pgfqpoint{8.789042in}{3.320655in}}{\pgfqpoint{8.791046in}{3.315817in}}{\pgfqpoint{8.794612in}{3.312251in}}%
\pgfpathcurveto{\pgfqpoint{8.798178in}{3.308684in}}{\pgfqpoint{8.803016in}{3.306680in}}{\pgfqpoint{8.808060in}{3.306680in}}%
\pgfpathclose%
\pgfusepath{fill}%
\end{pgfscope}%
\begin{pgfscope}%
\pgfpathrectangle{\pgfqpoint{6.572727in}{0.474100in}}{\pgfqpoint{4.227273in}{3.318700in}}%
\pgfusepath{clip}%
\pgfsetbuttcap%
\pgfsetroundjoin%
\definecolor{currentfill}{rgb}{0.993248,0.906157,0.143936}%
\pgfsetfillcolor{currentfill}%
\pgfsetfillopacity{0.700000}%
\pgfsetlinewidth{0.000000pt}%
\definecolor{currentstroke}{rgb}{0.000000,0.000000,0.000000}%
\pgfsetstrokecolor{currentstroke}%
\pgfsetstrokeopacity{0.700000}%
\pgfsetdash{}{0pt}%
\pgfpathmoveto{\pgfqpoint{8.843353in}{0.939826in}}%
\pgfpathcurveto{\pgfqpoint{8.848397in}{0.939826in}}{\pgfqpoint{8.853235in}{0.941830in}}{\pgfqpoint{8.856801in}{0.945397in}}%
\pgfpathcurveto{\pgfqpoint{8.860368in}{0.948963in}}{\pgfqpoint{8.862371in}{0.953801in}}{\pgfqpoint{8.862371in}{0.958844in}}%
\pgfpathcurveto{\pgfqpoint{8.862371in}{0.963888in}}{\pgfqpoint{8.860368in}{0.968726in}}{\pgfqpoint{8.856801in}{0.972292in}}%
\pgfpathcurveto{\pgfqpoint{8.853235in}{0.975859in}}{\pgfqpoint{8.848397in}{0.977863in}}{\pgfqpoint{8.843353in}{0.977863in}}%
\pgfpathcurveto{\pgfqpoint{8.838310in}{0.977863in}}{\pgfqpoint{8.833472in}{0.975859in}}{\pgfqpoint{8.829905in}{0.972292in}}%
\pgfpathcurveto{\pgfqpoint{8.826339in}{0.968726in}}{\pgfqpoint{8.824335in}{0.963888in}}{\pgfqpoint{8.824335in}{0.958844in}}%
\pgfpathcurveto{\pgfqpoint{8.824335in}{0.953801in}}{\pgfqpoint{8.826339in}{0.948963in}}{\pgfqpoint{8.829905in}{0.945397in}}%
\pgfpathcurveto{\pgfqpoint{8.833472in}{0.941830in}}{\pgfqpoint{8.838310in}{0.939826in}}{\pgfqpoint{8.843353in}{0.939826in}}%
\pgfpathclose%
\pgfusepath{fill}%
\end{pgfscope}%
\begin{pgfscope}%
\pgfpathrectangle{\pgfqpoint{6.572727in}{0.474100in}}{\pgfqpoint{4.227273in}{3.318700in}}%
\pgfusepath{clip}%
\pgfsetbuttcap%
\pgfsetroundjoin%
\definecolor{currentfill}{rgb}{0.127568,0.566949,0.550556}%
\pgfsetfillcolor{currentfill}%
\pgfsetfillopacity{0.700000}%
\pgfsetlinewidth{0.000000pt}%
\definecolor{currentstroke}{rgb}{0.000000,0.000000,0.000000}%
\pgfsetstrokecolor{currentstroke}%
\pgfsetstrokeopacity{0.700000}%
\pgfsetdash{}{0pt}%
\pgfpathmoveto{\pgfqpoint{7.805746in}{1.656711in}}%
\pgfpathcurveto{\pgfqpoint{7.810790in}{1.656711in}}{\pgfqpoint{7.815627in}{1.658715in}}{\pgfqpoint{7.819194in}{1.662281in}}%
\pgfpathcurveto{\pgfqpoint{7.822760in}{1.665847in}}{\pgfqpoint{7.824764in}{1.670685in}}{\pgfqpoint{7.824764in}{1.675729in}}%
\pgfpathcurveto{\pgfqpoint{7.824764in}{1.680773in}}{\pgfqpoint{7.822760in}{1.685610in}}{\pgfqpoint{7.819194in}{1.689177in}}%
\pgfpathcurveto{\pgfqpoint{7.815627in}{1.692743in}}{\pgfqpoint{7.810790in}{1.694747in}}{\pgfqpoint{7.805746in}{1.694747in}}%
\pgfpathcurveto{\pgfqpoint{7.800702in}{1.694747in}}{\pgfqpoint{7.795864in}{1.692743in}}{\pgfqpoint{7.792298in}{1.689177in}}%
\pgfpathcurveto{\pgfqpoint{7.788732in}{1.685610in}}{\pgfqpoint{7.786728in}{1.680773in}}{\pgfqpoint{7.786728in}{1.675729in}}%
\pgfpathcurveto{\pgfqpoint{7.786728in}{1.670685in}}{\pgfqpoint{7.788732in}{1.665847in}}{\pgfqpoint{7.792298in}{1.662281in}}%
\pgfpathcurveto{\pgfqpoint{7.795864in}{1.658715in}}{\pgfqpoint{7.800702in}{1.656711in}}{\pgfqpoint{7.805746in}{1.656711in}}%
\pgfpathclose%
\pgfusepath{fill}%
\end{pgfscope}%
\begin{pgfscope}%
\pgfpathrectangle{\pgfqpoint{6.572727in}{0.474100in}}{\pgfqpoint{4.227273in}{3.318700in}}%
\pgfusepath{clip}%
\pgfsetbuttcap%
\pgfsetroundjoin%
\definecolor{currentfill}{rgb}{0.993248,0.906157,0.143936}%
\pgfsetfillcolor{currentfill}%
\pgfsetfillopacity{0.700000}%
\pgfsetlinewidth{0.000000pt}%
\definecolor{currentstroke}{rgb}{0.000000,0.000000,0.000000}%
\pgfsetstrokecolor{currentstroke}%
\pgfsetstrokeopacity{0.700000}%
\pgfsetdash{}{0pt}%
\pgfpathmoveto{\pgfqpoint{10.089422in}{1.318415in}}%
\pgfpathcurveto{\pgfqpoint{10.094466in}{1.318415in}}{\pgfqpoint{10.099304in}{1.320419in}}{\pgfqpoint{10.102870in}{1.323986in}}%
\pgfpathcurveto{\pgfqpoint{10.106437in}{1.327552in}}{\pgfqpoint{10.108441in}{1.332390in}}{\pgfqpoint{10.108441in}{1.337434in}}%
\pgfpathcurveto{\pgfqpoint{10.108441in}{1.342477in}}{\pgfqpoint{10.106437in}{1.347315in}}{\pgfqpoint{10.102870in}{1.350881in}}%
\pgfpathcurveto{\pgfqpoint{10.099304in}{1.354448in}}{\pgfqpoint{10.094466in}{1.356452in}}{\pgfqpoint{10.089422in}{1.356452in}}%
\pgfpathcurveto{\pgfqpoint{10.084379in}{1.356452in}}{\pgfqpoint{10.079541in}{1.354448in}}{\pgfqpoint{10.075975in}{1.350881in}}%
\pgfpathcurveto{\pgfqpoint{10.072408in}{1.347315in}}{\pgfqpoint{10.070404in}{1.342477in}}{\pgfqpoint{10.070404in}{1.337434in}}%
\pgfpathcurveto{\pgfqpoint{10.070404in}{1.332390in}}{\pgfqpoint{10.072408in}{1.327552in}}{\pgfqpoint{10.075975in}{1.323986in}}%
\pgfpathcurveto{\pgfqpoint{10.079541in}{1.320419in}}{\pgfqpoint{10.084379in}{1.318415in}}{\pgfqpoint{10.089422in}{1.318415in}}%
\pgfpathclose%
\pgfusepath{fill}%
\end{pgfscope}%
\begin{pgfscope}%
\pgfpathrectangle{\pgfqpoint{6.572727in}{0.474100in}}{\pgfqpoint{4.227273in}{3.318700in}}%
\pgfusepath{clip}%
\pgfsetbuttcap%
\pgfsetroundjoin%
\definecolor{currentfill}{rgb}{0.127568,0.566949,0.550556}%
\pgfsetfillcolor{currentfill}%
\pgfsetfillopacity{0.700000}%
\pgfsetlinewidth{0.000000pt}%
\definecolor{currentstroke}{rgb}{0.000000,0.000000,0.000000}%
\pgfsetstrokecolor{currentstroke}%
\pgfsetstrokeopacity{0.700000}%
\pgfsetdash{}{0pt}%
\pgfpathmoveto{\pgfqpoint{8.485526in}{1.551498in}}%
\pgfpathcurveto{\pgfqpoint{8.490569in}{1.551498in}}{\pgfqpoint{8.495407in}{1.553502in}}{\pgfqpoint{8.498974in}{1.557068in}}%
\pgfpathcurveto{\pgfqpoint{8.502540in}{1.560635in}}{\pgfqpoint{8.504544in}{1.565473in}}{\pgfqpoint{8.504544in}{1.570516in}}%
\pgfpathcurveto{\pgfqpoint{8.504544in}{1.575560in}}{\pgfqpoint{8.502540in}{1.580398in}}{\pgfqpoint{8.498974in}{1.583964in}}%
\pgfpathcurveto{\pgfqpoint{8.495407in}{1.587531in}}{\pgfqpoint{8.490569in}{1.589534in}}{\pgfqpoint{8.485526in}{1.589534in}}%
\pgfpathcurveto{\pgfqpoint{8.480482in}{1.589534in}}{\pgfqpoint{8.475644in}{1.587531in}}{\pgfqpoint{8.472078in}{1.583964in}}%
\pgfpathcurveto{\pgfqpoint{8.468512in}{1.580398in}}{\pgfqpoint{8.466508in}{1.575560in}}{\pgfqpoint{8.466508in}{1.570516in}}%
\pgfpathcurveto{\pgfqpoint{8.466508in}{1.565473in}}{\pgfqpoint{8.468512in}{1.560635in}}{\pgfqpoint{8.472078in}{1.557068in}}%
\pgfpathcurveto{\pgfqpoint{8.475644in}{1.553502in}}{\pgfqpoint{8.480482in}{1.551498in}}{\pgfqpoint{8.485526in}{1.551498in}}%
\pgfpathclose%
\pgfusepath{fill}%
\end{pgfscope}%
\begin{pgfscope}%
\pgfpathrectangle{\pgfqpoint{6.572727in}{0.474100in}}{\pgfqpoint{4.227273in}{3.318700in}}%
\pgfusepath{clip}%
\pgfsetbuttcap%
\pgfsetroundjoin%
\definecolor{currentfill}{rgb}{0.127568,0.566949,0.550556}%
\pgfsetfillcolor{currentfill}%
\pgfsetfillopacity{0.700000}%
\pgfsetlinewidth{0.000000pt}%
\definecolor{currentstroke}{rgb}{0.000000,0.000000,0.000000}%
\pgfsetstrokecolor{currentstroke}%
\pgfsetstrokeopacity{0.700000}%
\pgfsetdash{}{0pt}%
\pgfpathmoveto{\pgfqpoint{7.997013in}{2.824629in}}%
\pgfpathcurveto{\pgfqpoint{8.002056in}{2.824629in}}{\pgfqpoint{8.006894in}{2.826633in}}{\pgfqpoint{8.010460in}{2.830199in}}%
\pgfpathcurveto{\pgfqpoint{8.014027in}{2.833765in}}{\pgfqpoint{8.016031in}{2.838603in}}{\pgfqpoint{8.016031in}{2.843647in}}%
\pgfpathcurveto{\pgfqpoint{8.016031in}{2.848691in}}{\pgfqpoint{8.014027in}{2.853528in}}{\pgfqpoint{8.010460in}{2.857095in}}%
\pgfpathcurveto{\pgfqpoint{8.006894in}{2.860661in}}{\pgfqpoint{8.002056in}{2.862665in}}{\pgfqpoint{7.997013in}{2.862665in}}%
\pgfpathcurveto{\pgfqpoint{7.991969in}{2.862665in}}{\pgfqpoint{7.987131in}{2.860661in}}{\pgfqpoint{7.983565in}{2.857095in}}%
\pgfpathcurveto{\pgfqpoint{7.979998in}{2.853528in}}{\pgfqpoint{7.977994in}{2.848691in}}{\pgfqpoint{7.977994in}{2.843647in}}%
\pgfpathcurveto{\pgfqpoint{7.977994in}{2.838603in}}{\pgfqpoint{7.979998in}{2.833765in}}{\pgfqpoint{7.983565in}{2.830199in}}%
\pgfpathcurveto{\pgfqpoint{7.987131in}{2.826633in}}{\pgfqpoint{7.991969in}{2.824629in}}{\pgfqpoint{7.997013in}{2.824629in}}%
\pgfpathclose%
\pgfusepath{fill}%
\end{pgfscope}%
\begin{pgfscope}%
\pgfpathrectangle{\pgfqpoint{6.572727in}{0.474100in}}{\pgfqpoint{4.227273in}{3.318700in}}%
\pgfusepath{clip}%
\pgfsetbuttcap%
\pgfsetroundjoin%
\definecolor{currentfill}{rgb}{0.127568,0.566949,0.550556}%
\pgfsetfillcolor{currentfill}%
\pgfsetfillopacity{0.700000}%
\pgfsetlinewidth{0.000000pt}%
\definecolor{currentstroke}{rgb}{0.000000,0.000000,0.000000}%
\pgfsetstrokecolor{currentstroke}%
\pgfsetstrokeopacity{0.700000}%
\pgfsetdash{}{0pt}%
\pgfpathmoveto{\pgfqpoint{7.491708in}{1.637198in}}%
\pgfpathcurveto{\pgfqpoint{7.496752in}{1.637198in}}{\pgfqpoint{7.501590in}{1.639202in}}{\pgfqpoint{7.505156in}{1.642768in}}%
\pgfpathcurveto{\pgfqpoint{7.508723in}{1.646334in}}{\pgfqpoint{7.510727in}{1.651172in}}{\pgfqpoint{7.510727in}{1.656216in}}%
\pgfpathcurveto{\pgfqpoint{7.510727in}{1.661259in}}{\pgfqpoint{7.508723in}{1.666097in}}{\pgfqpoint{7.505156in}{1.669664in}}%
\pgfpathcurveto{\pgfqpoint{7.501590in}{1.673230in}}{\pgfqpoint{7.496752in}{1.675234in}}{\pgfqpoint{7.491708in}{1.675234in}}%
\pgfpathcurveto{\pgfqpoint{7.486665in}{1.675234in}}{\pgfqpoint{7.481827in}{1.673230in}}{\pgfqpoint{7.478261in}{1.669664in}}%
\pgfpathcurveto{\pgfqpoint{7.474694in}{1.666097in}}{\pgfqpoint{7.472690in}{1.661259in}}{\pgfqpoint{7.472690in}{1.656216in}}%
\pgfpathcurveto{\pgfqpoint{7.472690in}{1.651172in}}{\pgfqpoint{7.474694in}{1.646334in}}{\pgfqpoint{7.478261in}{1.642768in}}%
\pgfpathcurveto{\pgfqpoint{7.481827in}{1.639202in}}{\pgfqpoint{7.486665in}{1.637198in}}{\pgfqpoint{7.491708in}{1.637198in}}%
\pgfpathclose%
\pgfusepath{fill}%
\end{pgfscope}%
\begin{pgfscope}%
\pgfpathrectangle{\pgfqpoint{6.572727in}{0.474100in}}{\pgfqpoint{4.227273in}{3.318700in}}%
\pgfusepath{clip}%
\pgfsetbuttcap%
\pgfsetroundjoin%
\definecolor{currentfill}{rgb}{0.127568,0.566949,0.550556}%
\pgfsetfillcolor{currentfill}%
\pgfsetfillopacity{0.700000}%
\pgfsetlinewidth{0.000000pt}%
\definecolor{currentstroke}{rgb}{0.000000,0.000000,0.000000}%
\pgfsetstrokecolor{currentstroke}%
\pgfsetstrokeopacity{0.700000}%
\pgfsetdash{}{0pt}%
\pgfpathmoveto{\pgfqpoint{7.995655in}{3.249166in}}%
\pgfpathcurveto{\pgfqpoint{8.000699in}{3.249166in}}{\pgfqpoint{8.005537in}{3.251170in}}{\pgfqpoint{8.009103in}{3.254737in}}%
\pgfpathcurveto{\pgfqpoint{8.012670in}{3.258303in}}{\pgfqpoint{8.014674in}{3.263141in}}{\pgfqpoint{8.014674in}{3.268184in}}%
\pgfpathcurveto{\pgfqpoint{8.014674in}{3.273228in}}{\pgfqpoint{8.012670in}{3.278066in}}{\pgfqpoint{8.009103in}{3.281632in}}%
\pgfpathcurveto{\pgfqpoint{8.005537in}{3.285199in}}{\pgfqpoint{8.000699in}{3.287203in}}{\pgfqpoint{7.995655in}{3.287203in}}%
\pgfpathcurveto{\pgfqpoint{7.990612in}{3.287203in}}{\pgfqpoint{7.985774in}{3.285199in}}{\pgfqpoint{7.982208in}{3.281632in}}%
\pgfpathcurveto{\pgfqpoint{7.978641in}{3.278066in}}{\pgfqpoint{7.976637in}{3.273228in}}{\pgfqpoint{7.976637in}{3.268184in}}%
\pgfpathcurveto{\pgfqpoint{7.976637in}{3.263141in}}{\pgfqpoint{7.978641in}{3.258303in}}{\pgfqpoint{7.982208in}{3.254737in}}%
\pgfpathcurveto{\pgfqpoint{7.985774in}{3.251170in}}{\pgfqpoint{7.990612in}{3.249166in}}{\pgfqpoint{7.995655in}{3.249166in}}%
\pgfpathclose%
\pgfusepath{fill}%
\end{pgfscope}%
\begin{pgfscope}%
\pgfpathrectangle{\pgfqpoint{6.572727in}{0.474100in}}{\pgfqpoint{4.227273in}{3.318700in}}%
\pgfusepath{clip}%
\pgfsetbuttcap%
\pgfsetroundjoin%
\definecolor{currentfill}{rgb}{0.127568,0.566949,0.550556}%
\pgfsetfillcolor{currentfill}%
\pgfsetfillopacity{0.700000}%
\pgfsetlinewidth{0.000000pt}%
\definecolor{currentstroke}{rgb}{0.000000,0.000000,0.000000}%
\pgfsetstrokecolor{currentstroke}%
\pgfsetstrokeopacity{0.700000}%
\pgfsetdash{}{0pt}%
\pgfpathmoveto{\pgfqpoint{7.800085in}{1.030987in}}%
\pgfpathcurveto{\pgfqpoint{7.805129in}{1.030987in}}{\pgfqpoint{7.809967in}{1.032991in}}{\pgfqpoint{7.813533in}{1.036558in}}%
\pgfpathcurveto{\pgfqpoint{7.817099in}{1.040124in}}{\pgfqpoint{7.819103in}{1.044962in}}{\pgfqpoint{7.819103in}{1.050006in}}%
\pgfpathcurveto{\pgfqpoint{7.819103in}{1.055049in}}{\pgfqpoint{7.817099in}{1.059887in}}{\pgfqpoint{7.813533in}{1.063453in}}%
\pgfpathcurveto{\pgfqpoint{7.809967in}{1.067020in}}{\pgfqpoint{7.805129in}{1.069024in}}{\pgfqpoint{7.800085in}{1.069024in}}%
\pgfpathcurveto{\pgfqpoint{7.795042in}{1.069024in}}{\pgfqpoint{7.790204in}{1.067020in}}{\pgfqpoint{7.786637in}{1.063453in}}%
\pgfpathcurveto{\pgfqpoint{7.783071in}{1.059887in}}{\pgfqpoint{7.781067in}{1.055049in}}{\pgfqpoint{7.781067in}{1.050006in}}%
\pgfpathcurveto{\pgfqpoint{7.781067in}{1.044962in}}{\pgfqpoint{7.783071in}{1.040124in}}{\pgfqpoint{7.786637in}{1.036558in}}%
\pgfpathcurveto{\pgfqpoint{7.790204in}{1.032991in}}{\pgfqpoint{7.795042in}{1.030987in}}{\pgfqpoint{7.800085in}{1.030987in}}%
\pgfpathclose%
\pgfusepath{fill}%
\end{pgfscope}%
\begin{pgfscope}%
\pgfpathrectangle{\pgfqpoint{6.572727in}{0.474100in}}{\pgfqpoint{4.227273in}{3.318700in}}%
\pgfusepath{clip}%
\pgfsetbuttcap%
\pgfsetroundjoin%
\definecolor{currentfill}{rgb}{0.127568,0.566949,0.550556}%
\pgfsetfillcolor{currentfill}%
\pgfsetfillopacity{0.700000}%
\pgfsetlinewidth{0.000000pt}%
\definecolor{currentstroke}{rgb}{0.000000,0.000000,0.000000}%
\pgfsetstrokecolor{currentstroke}%
\pgfsetstrokeopacity{0.700000}%
\pgfsetdash{}{0pt}%
\pgfpathmoveto{\pgfqpoint{7.781376in}{1.560309in}}%
\pgfpathcurveto{\pgfqpoint{7.786420in}{1.560309in}}{\pgfqpoint{7.791258in}{1.562313in}}{\pgfqpoint{7.794824in}{1.565880in}}%
\pgfpathcurveto{\pgfqpoint{7.798391in}{1.569446in}}{\pgfqpoint{7.800395in}{1.574284in}}{\pgfqpoint{7.800395in}{1.579328in}}%
\pgfpathcurveto{\pgfqpoint{7.800395in}{1.584371in}}{\pgfqpoint{7.798391in}{1.589209in}}{\pgfqpoint{7.794824in}{1.592775in}}%
\pgfpathcurveto{\pgfqpoint{7.791258in}{1.596342in}}{\pgfqpoint{7.786420in}{1.598346in}}{\pgfqpoint{7.781376in}{1.598346in}}%
\pgfpathcurveto{\pgfqpoint{7.776333in}{1.598346in}}{\pgfqpoint{7.771495in}{1.596342in}}{\pgfqpoint{7.767929in}{1.592775in}}%
\pgfpathcurveto{\pgfqpoint{7.764362in}{1.589209in}}{\pgfqpoint{7.762358in}{1.584371in}}{\pgfqpoint{7.762358in}{1.579328in}}%
\pgfpathcurveto{\pgfqpoint{7.762358in}{1.574284in}}{\pgfqpoint{7.764362in}{1.569446in}}{\pgfqpoint{7.767929in}{1.565880in}}%
\pgfpathcurveto{\pgfqpoint{7.771495in}{1.562313in}}{\pgfqpoint{7.776333in}{1.560309in}}{\pgfqpoint{7.781376in}{1.560309in}}%
\pgfpathclose%
\pgfusepath{fill}%
\end{pgfscope}%
\begin{pgfscope}%
\pgfpathrectangle{\pgfqpoint{6.572727in}{0.474100in}}{\pgfqpoint{4.227273in}{3.318700in}}%
\pgfusepath{clip}%
\pgfsetbuttcap%
\pgfsetroundjoin%
\definecolor{currentfill}{rgb}{0.127568,0.566949,0.550556}%
\pgfsetfillcolor{currentfill}%
\pgfsetfillopacity{0.700000}%
\pgfsetlinewidth{0.000000pt}%
\definecolor{currentstroke}{rgb}{0.000000,0.000000,0.000000}%
\pgfsetstrokecolor{currentstroke}%
\pgfsetstrokeopacity{0.700000}%
\pgfsetdash{}{0pt}%
\pgfpathmoveto{\pgfqpoint{8.579273in}{2.920329in}}%
\pgfpathcurveto{\pgfqpoint{8.584317in}{2.920329in}}{\pgfqpoint{8.589154in}{2.922333in}}{\pgfqpoint{8.592721in}{2.925899in}}%
\pgfpathcurveto{\pgfqpoint{8.596287in}{2.929466in}}{\pgfqpoint{8.598291in}{2.934303in}}{\pgfqpoint{8.598291in}{2.939347in}}%
\pgfpathcurveto{\pgfqpoint{8.598291in}{2.944391in}}{\pgfqpoint{8.596287in}{2.949228in}}{\pgfqpoint{8.592721in}{2.952795in}}%
\pgfpathcurveto{\pgfqpoint{8.589154in}{2.956361in}}{\pgfqpoint{8.584317in}{2.958365in}}{\pgfqpoint{8.579273in}{2.958365in}}%
\pgfpathcurveto{\pgfqpoint{8.574229in}{2.958365in}}{\pgfqpoint{8.569392in}{2.956361in}}{\pgfqpoint{8.565825in}{2.952795in}}%
\pgfpathcurveto{\pgfqpoint{8.562259in}{2.949228in}}{\pgfqpoint{8.560255in}{2.944391in}}{\pgfqpoint{8.560255in}{2.939347in}}%
\pgfpathcurveto{\pgfqpoint{8.560255in}{2.934303in}}{\pgfqpoint{8.562259in}{2.929466in}}{\pgfqpoint{8.565825in}{2.925899in}}%
\pgfpathcurveto{\pgfqpoint{8.569392in}{2.922333in}}{\pgfqpoint{8.574229in}{2.920329in}}{\pgfqpoint{8.579273in}{2.920329in}}%
\pgfpathclose%
\pgfusepath{fill}%
\end{pgfscope}%
\begin{pgfscope}%
\pgfpathrectangle{\pgfqpoint{6.572727in}{0.474100in}}{\pgfqpoint{4.227273in}{3.318700in}}%
\pgfusepath{clip}%
\pgfsetbuttcap%
\pgfsetroundjoin%
\definecolor{currentfill}{rgb}{0.993248,0.906157,0.143936}%
\pgfsetfillcolor{currentfill}%
\pgfsetfillopacity{0.700000}%
\pgfsetlinewidth{0.000000pt}%
\definecolor{currentstroke}{rgb}{0.000000,0.000000,0.000000}%
\pgfsetstrokecolor{currentstroke}%
\pgfsetstrokeopacity{0.700000}%
\pgfsetdash{}{0pt}%
\pgfpathmoveto{\pgfqpoint{9.428551in}{1.520236in}}%
\pgfpathcurveto{\pgfqpoint{9.433595in}{1.520236in}}{\pgfqpoint{9.438433in}{1.522240in}}{\pgfqpoint{9.441999in}{1.525806in}}%
\pgfpathcurveto{\pgfqpoint{9.445566in}{1.529373in}}{\pgfqpoint{9.447570in}{1.534211in}}{\pgfqpoint{9.447570in}{1.539254in}}%
\pgfpathcurveto{\pgfqpoint{9.447570in}{1.544298in}}{\pgfqpoint{9.445566in}{1.549136in}}{\pgfqpoint{9.441999in}{1.552702in}}%
\pgfpathcurveto{\pgfqpoint{9.438433in}{1.556269in}}{\pgfqpoint{9.433595in}{1.558272in}}{\pgfqpoint{9.428551in}{1.558272in}}%
\pgfpathcurveto{\pgfqpoint{9.423508in}{1.558272in}}{\pgfqpoint{9.418670in}{1.556269in}}{\pgfqpoint{9.415104in}{1.552702in}}%
\pgfpathcurveto{\pgfqpoint{9.411537in}{1.549136in}}{\pgfqpoint{9.409533in}{1.544298in}}{\pgfqpoint{9.409533in}{1.539254in}}%
\pgfpathcurveto{\pgfqpoint{9.409533in}{1.534211in}}{\pgfqpoint{9.411537in}{1.529373in}}{\pgfqpoint{9.415104in}{1.525806in}}%
\pgfpathcurveto{\pgfqpoint{9.418670in}{1.522240in}}{\pgfqpoint{9.423508in}{1.520236in}}{\pgfqpoint{9.428551in}{1.520236in}}%
\pgfpathclose%
\pgfusepath{fill}%
\end{pgfscope}%
\begin{pgfscope}%
\pgfpathrectangle{\pgfqpoint{6.572727in}{0.474100in}}{\pgfqpoint{4.227273in}{3.318700in}}%
\pgfusepath{clip}%
\pgfsetbuttcap%
\pgfsetroundjoin%
\definecolor{currentfill}{rgb}{0.127568,0.566949,0.550556}%
\pgfsetfillcolor{currentfill}%
\pgfsetfillopacity{0.700000}%
\pgfsetlinewidth{0.000000pt}%
\definecolor{currentstroke}{rgb}{0.000000,0.000000,0.000000}%
\pgfsetstrokecolor{currentstroke}%
\pgfsetstrokeopacity{0.700000}%
\pgfsetdash{}{0pt}%
\pgfpathmoveto{\pgfqpoint{7.940758in}{1.740042in}}%
\pgfpathcurveto{\pgfqpoint{7.945801in}{1.740042in}}{\pgfqpoint{7.950639in}{1.742046in}}{\pgfqpoint{7.954206in}{1.745612in}}%
\pgfpathcurveto{\pgfqpoint{7.957772in}{1.749179in}}{\pgfqpoint{7.959776in}{1.754017in}}{\pgfqpoint{7.959776in}{1.759060in}}%
\pgfpathcurveto{\pgfqpoint{7.959776in}{1.764104in}}{\pgfqpoint{7.957772in}{1.768942in}}{\pgfqpoint{7.954206in}{1.772508in}}%
\pgfpathcurveto{\pgfqpoint{7.950639in}{1.776075in}}{\pgfqpoint{7.945801in}{1.778078in}}{\pgfqpoint{7.940758in}{1.778078in}}%
\pgfpathcurveto{\pgfqpoint{7.935714in}{1.778078in}}{\pgfqpoint{7.930876in}{1.776075in}}{\pgfqpoint{7.927310in}{1.772508in}}%
\pgfpathcurveto{\pgfqpoint{7.923743in}{1.768942in}}{\pgfqpoint{7.921740in}{1.764104in}}{\pgfqpoint{7.921740in}{1.759060in}}%
\pgfpathcurveto{\pgfqpoint{7.921740in}{1.754017in}}{\pgfqpoint{7.923743in}{1.749179in}}{\pgfqpoint{7.927310in}{1.745612in}}%
\pgfpathcurveto{\pgfqpoint{7.930876in}{1.742046in}}{\pgfqpoint{7.935714in}{1.740042in}}{\pgfqpoint{7.940758in}{1.740042in}}%
\pgfpathclose%
\pgfusepath{fill}%
\end{pgfscope}%
\begin{pgfscope}%
\pgfpathrectangle{\pgfqpoint{6.572727in}{0.474100in}}{\pgfqpoint{4.227273in}{3.318700in}}%
\pgfusepath{clip}%
\pgfsetbuttcap%
\pgfsetroundjoin%
\definecolor{currentfill}{rgb}{0.993248,0.906157,0.143936}%
\pgfsetfillcolor{currentfill}%
\pgfsetfillopacity{0.700000}%
\pgfsetlinewidth{0.000000pt}%
\definecolor{currentstroke}{rgb}{0.000000,0.000000,0.000000}%
\pgfsetstrokecolor{currentstroke}%
\pgfsetstrokeopacity{0.700000}%
\pgfsetdash{}{0pt}%
\pgfpathmoveto{\pgfqpoint{9.011288in}{1.514011in}}%
\pgfpathcurveto{\pgfqpoint{9.016332in}{1.514011in}}{\pgfqpoint{9.021170in}{1.516014in}}{\pgfqpoint{9.024736in}{1.519581in}}%
\pgfpathcurveto{\pgfqpoint{9.028303in}{1.523147in}}{\pgfqpoint{9.030307in}{1.527985in}}{\pgfqpoint{9.030307in}{1.533029in}}%
\pgfpathcurveto{\pgfqpoint{9.030307in}{1.538072in}}{\pgfqpoint{9.028303in}{1.542910in}}{\pgfqpoint{9.024736in}{1.546477in}}%
\pgfpathcurveto{\pgfqpoint{9.021170in}{1.550043in}}{\pgfqpoint{9.016332in}{1.552047in}}{\pgfqpoint{9.011288in}{1.552047in}}%
\pgfpathcurveto{\pgfqpoint{9.006245in}{1.552047in}}{\pgfqpoint{9.001407in}{1.550043in}}{\pgfqpoint{8.997841in}{1.546477in}}%
\pgfpathcurveto{\pgfqpoint{8.994274in}{1.542910in}}{\pgfqpoint{8.992270in}{1.538072in}}{\pgfqpoint{8.992270in}{1.533029in}}%
\pgfpathcurveto{\pgfqpoint{8.992270in}{1.527985in}}{\pgfqpoint{8.994274in}{1.523147in}}{\pgfqpoint{8.997841in}{1.519581in}}%
\pgfpathcurveto{\pgfqpoint{9.001407in}{1.516014in}}{\pgfqpoint{9.006245in}{1.514011in}}{\pgfqpoint{9.011288in}{1.514011in}}%
\pgfpathclose%
\pgfusepath{fill}%
\end{pgfscope}%
\begin{pgfscope}%
\pgfpathrectangle{\pgfqpoint{6.572727in}{0.474100in}}{\pgfqpoint{4.227273in}{3.318700in}}%
\pgfusepath{clip}%
\pgfsetbuttcap%
\pgfsetroundjoin%
\definecolor{currentfill}{rgb}{0.993248,0.906157,0.143936}%
\pgfsetfillcolor{currentfill}%
\pgfsetfillopacity{0.700000}%
\pgfsetlinewidth{0.000000pt}%
\definecolor{currentstroke}{rgb}{0.000000,0.000000,0.000000}%
\pgfsetstrokecolor{currentstroke}%
\pgfsetstrokeopacity{0.700000}%
\pgfsetdash{}{0pt}%
\pgfpathmoveto{\pgfqpoint{10.085005in}{0.856734in}}%
\pgfpathcurveto{\pgfqpoint{10.090049in}{0.856734in}}{\pgfqpoint{10.094887in}{0.858738in}}{\pgfqpoint{10.098453in}{0.862305in}}%
\pgfpathcurveto{\pgfqpoint{10.102019in}{0.865871in}}{\pgfqpoint{10.104023in}{0.870709in}}{\pgfqpoint{10.104023in}{0.875753in}}%
\pgfpathcurveto{\pgfqpoint{10.104023in}{0.880796in}}{\pgfqpoint{10.102019in}{0.885634in}}{\pgfqpoint{10.098453in}{0.889200in}}%
\pgfpathcurveto{\pgfqpoint{10.094887in}{0.892767in}}{\pgfqpoint{10.090049in}{0.894771in}}{\pgfqpoint{10.085005in}{0.894771in}}%
\pgfpathcurveto{\pgfqpoint{10.079962in}{0.894771in}}{\pgfqpoint{10.075124in}{0.892767in}}{\pgfqpoint{10.071557in}{0.889200in}}%
\pgfpathcurveto{\pgfqpoint{10.067991in}{0.885634in}}{\pgfqpoint{10.065987in}{0.880796in}}{\pgfqpoint{10.065987in}{0.875753in}}%
\pgfpathcurveto{\pgfqpoint{10.065987in}{0.870709in}}{\pgfqpoint{10.067991in}{0.865871in}}{\pgfqpoint{10.071557in}{0.862305in}}%
\pgfpathcurveto{\pgfqpoint{10.075124in}{0.858738in}}{\pgfqpoint{10.079962in}{0.856734in}}{\pgfqpoint{10.085005in}{0.856734in}}%
\pgfpathclose%
\pgfusepath{fill}%
\end{pgfscope}%
\begin{pgfscope}%
\pgfpathrectangle{\pgfqpoint{6.572727in}{0.474100in}}{\pgfqpoint{4.227273in}{3.318700in}}%
\pgfusepath{clip}%
\pgfsetbuttcap%
\pgfsetroundjoin%
\definecolor{currentfill}{rgb}{0.127568,0.566949,0.550556}%
\pgfsetfillcolor{currentfill}%
\pgfsetfillopacity{0.700000}%
\pgfsetlinewidth{0.000000pt}%
\definecolor{currentstroke}{rgb}{0.000000,0.000000,0.000000}%
\pgfsetstrokecolor{currentstroke}%
\pgfsetstrokeopacity{0.700000}%
\pgfsetdash{}{0pt}%
\pgfpathmoveto{\pgfqpoint{8.170251in}{2.777107in}}%
\pgfpathcurveto{\pgfqpoint{8.175295in}{2.777107in}}{\pgfqpoint{8.180133in}{2.779111in}}{\pgfqpoint{8.183699in}{2.782678in}}%
\pgfpathcurveto{\pgfqpoint{8.187265in}{2.786244in}}{\pgfqpoint{8.189269in}{2.791082in}}{\pgfqpoint{8.189269in}{2.796126in}}%
\pgfpathcurveto{\pgfqpoint{8.189269in}{2.801169in}}{\pgfqpoint{8.187265in}{2.806007in}}{\pgfqpoint{8.183699in}{2.809573in}}%
\pgfpathcurveto{\pgfqpoint{8.180133in}{2.813140in}}{\pgfqpoint{8.175295in}{2.815144in}}{\pgfqpoint{8.170251in}{2.815144in}}%
\pgfpathcurveto{\pgfqpoint{8.165207in}{2.815144in}}{\pgfqpoint{8.160370in}{2.813140in}}{\pgfqpoint{8.156803in}{2.809573in}}%
\pgfpathcurveto{\pgfqpoint{8.153237in}{2.806007in}}{\pgfqpoint{8.151233in}{2.801169in}}{\pgfqpoint{8.151233in}{2.796126in}}%
\pgfpathcurveto{\pgfqpoint{8.151233in}{2.791082in}}{\pgfqpoint{8.153237in}{2.786244in}}{\pgfqpoint{8.156803in}{2.782678in}}%
\pgfpathcurveto{\pgfqpoint{8.160370in}{2.779111in}}{\pgfqpoint{8.165207in}{2.777107in}}{\pgfqpoint{8.170251in}{2.777107in}}%
\pgfpathclose%
\pgfusepath{fill}%
\end{pgfscope}%
\begin{pgfscope}%
\pgfpathrectangle{\pgfqpoint{6.572727in}{0.474100in}}{\pgfqpoint{4.227273in}{3.318700in}}%
\pgfusepath{clip}%
\pgfsetbuttcap%
\pgfsetroundjoin%
\definecolor{currentfill}{rgb}{0.127568,0.566949,0.550556}%
\pgfsetfillcolor{currentfill}%
\pgfsetfillopacity{0.700000}%
\pgfsetlinewidth{0.000000pt}%
\definecolor{currentstroke}{rgb}{0.000000,0.000000,0.000000}%
\pgfsetstrokecolor{currentstroke}%
\pgfsetstrokeopacity{0.700000}%
\pgfsetdash{}{0pt}%
\pgfpathmoveto{\pgfqpoint{7.906933in}{1.507738in}}%
\pgfpathcurveto{\pgfqpoint{7.911977in}{1.507738in}}{\pgfqpoint{7.916814in}{1.509742in}}{\pgfqpoint{7.920381in}{1.513308in}}%
\pgfpathcurveto{\pgfqpoint{7.923947in}{1.516875in}}{\pgfqpoint{7.925951in}{1.521713in}}{\pgfqpoint{7.925951in}{1.526756in}}%
\pgfpathcurveto{\pgfqpoint{7.925951in}{1.531800in}}{\pgfqpoint{7.923947in}{1.536638in}}{\pgfqpoint{7.920381in}{1.540204in}}%
\pgfpathcurveto{\pgfqpoint{7.916814in}{1.543770in}}{\pgfqpoint{7.911977in}{1.545774in}}{\pgfqpoint{7.906933in}{1.545774in}}%
\pgfpathcurveto{\pgfqpoint{7.901889in}{1.545774in}}{\pgfqpoint{7.897052in}{1.543770in}}{\pgfqpoint{7.893485in}{1.540204in}}%
\pgfpathcurveto{\pgfqpoint{7.889919in}{1.536638in}}{\pgfqpoint{7.887915in}{1.531800in}}{\pgfqpoint{7.887915in}{1.526756in}}%
\pgfpathcurveto{\pgfqpoint{7.887915in}{1.521713in}}{\pgfqpoint{7.889919in}{1.516875in}}{\pgfqpoint{7.893485in}{1.513308in}}%
\pgfpathcurveto{\pgfqpoint{7.897052in}{1.509742in}}{\pgfqpoint{7.901889in}{1.507738in}}{\pgfqpoint{7.906933in}{1.507738in}}%
\pgfpathclose%
\pgfusepath{fill}%
\end{pgfscope}%
\begin{pgfscope}%
\pgfpathrectangle{\pgfqpoint{6.572727in}{0.474100in}}{\pgfqpoint{4.227273in}{3.318700in}}%
\pgfusepath{clip}%
\pgfsetbuttcap%
\pgfsetroundjoin%
\definecolor{currentfill}{rgb}{0.127568,0.566949,0.550556}%
\pgfsetfillcolor{currentfill}%
\pgfsetfillopacity{0.700000}%
\pgfsetlinewidth{0.000000pt}%
\definecolor{currentstroke}{rgb}{0.000000,0.000000,0.000000}%
\pgfsetstrokecolor{currentstroke}%
\pgfsetstrokeopacity{0.700000}%
\pgfsetdash{}{0pt}%
\pgfpathmoveto{\pgfqpoint{7.456083in}{0.898283in}}%
\pgfpathcurveto{\pgfqpoint{7.461127in}{0.898283in}}{\pgfqpoint{7.465965in}{0.900287in}}{\pgfqpoint{7.469531in}{0.903854in}}%
\pgfpathcurveto{\pgfqpoint{7.473097in}{0.907420in}}{\pgfqpoint{7.475101in}{0.912258in}}{\pgfqpoint{7.475101in}{0.917301in}}%
\pgfpathcurveto{\pgfqpoint{7.475101in}{0.922345in}}{\pgfqpoint{7.473097in}{0.927183in}}{\pgfqpoint{7.469531in}{0.930749in}}%
\pgfpathcurveto{\pgfqpoint{7.465965in}{0.934316in}}{\pgfqpoint{7.461127in}{0.936320in}}{\pgfqpoint{7.456083in}{0.936320in}}%
\pgfpathcurveto{\pgfqpoint{7.451039in}{0.936320in}}{\pgfqpoint{7.446202in}{0.934316in}}{\pgfqpoint{7.442635in}{0.930749in}}%
\pgfpathcurveto{\pgfqpoint{7.439069in}{0.927183in}}{\pgfqpoint{7.437065in}{0.922345in}}{\pgfqpoint{7.437065in}{0.917301in}}%
\pgfpathcurveto{\pgfqpoint{7.437065in}{0.912258in}}{\pgfqpoint{7.439069in}{0.907420in}}{\pgfqpoint{7.442635in}{0.903854in}}%
\pgfpathcurveto{\pgfqpoint{7.446202in}{0.900287in}}{\pgfqpoint{7.451039in}{0.898283in}}{\pgfqpoint{7.456083in}{0.898283in}}%
\pgfpathclose%
\pgfusepath{fill}%
\end{pgfscope}%
\begin{pgfscope}%
\pgfpathrectangle{\pgfqpoint{6.572727in}{0.474100in}}{\pgfqpoint{4.227273in}{3.318700in}}%
\pgfusepath{clip}%
\pgfsetbuttcap%
\pgfsetroundjoin%
\definecolor{currentfill}{rgb}{0.127568,0.566949,0.550556}%
\pgfsetfillcolor{currentfill}%
\pgfsetfillopacity{0.700000}%
\pgfsetlinewidth{0.000000pt}%
\definecolor{currentstroke}{rgb}{0.000000,0.000000,0.000000}%
\pgfsetstrokecolor{currentstroke}%
\pgfsetstrokeopacity{0.700000}%
\pgfsetdash{}{0pt}%
\pgfpathmoveto{\pgfqpoint{7.844510in}{1.237000in}}%
\pgfpathcurveto{\pgfqpoint{7.849554in}{1.237000in}}{\pgfqpoint{7.854392in}{1.239003in}}{\pgfqpoint{7.857958in}{1.242570in}}%
\pgfpathcurveto{\pgfqpoint{7.861525in}{1.246136in}}{\pgfqpoint{7.863528in}{1.250974in}}{\pgfqpoint{7.863528in}{1.256018in}}%
\pgfpathcurveto{\pgfqpoint{7.863528in}{1.261061in}}{\pgfqpoint{7.861525in}{1.265899in}}{\pgfqpoint{7.857958in}{1.269466in}}%
\pgfpathcurveto{\pgfqpoint{7.854392in}{1.273032in}}{\pgfqpoint{7.849554in}{1.275036in}}{\pgfqpoint{7.844510in}{1.275036in}}%
\pgfpathcurveto{\pgfqpoint{7.839467in}{1.275036in}}{\pgfqpoint{7.834629in}{1.273032in}}{\pgfqpoint{7.831062in}{1.269466in}}%
\pgfpathcurveto{\pgfqpoint{7.827496in}{1.265899in}}{\pgfqpoint{7.825492in}{1.261061in}}{\pgfqpoint{7.825492in}{1.256018in}}%
\pgfpathcurveto{\pgfqpoint{7.825492in}{1.250974in}}{\pgfqpoint{7.827496in}{1.246136in}}{\pgfqpoint{7.831062in}{1.242570in}}%
\pgfpathcurveto{\pgfqpoint{7.834629in}{1.239003in}}{\pgfqpoint{7.839467in}{1.237000in}}{\pgfqpoint{7.844510in}{1.237000in}}%
\pgfpathclose%
\pgfusepath{fill}%
\end{pgfscope}%
\begin{pgfscope}%
\pgfpathrectangle{\pgfqpoint{6.572727in}{0.474100in}}{\pgfqpoint{4.227273in}{3.318700in}}%
\pgfusepath{clip}%
\pgfsetbuttcap%
\pgfsetroundjoin%
\definecolor{currentfill}{rgb}{0.993248,0.906157,0.143936}%
\pgfsetfillcolor{currentfill}%
\pgfsetfillopacity{0.700000}%
\pgfsetlinewidth{0.000000pt}%
\definecolor{currentstroke}{rgb}{0.000000,0.000000,0.000000}%
\pgfsetstrokecolor{currentstroke}%
\pgfsetstrokeopacity{0.700000}%
\pgfsetdash{}{0pt}%
\pgfpathmoveto{\pgfqpoint{9.944993in}{0.943829in}}%
\pgfpathcurveto{\pgfqpoint{9.950036in}{0.943829in}}{\pgfqpoint{9.954874in}{0.945832in}}{\pgfqpoint{9.958441in}{0.949399in}}%
\pgfpathcurveto{\pgfqpoint{9.962007in}{0.952965in}}{\pgfqpoint{9.964011in}{0.957803in}}{\pgfqpoint{9.964011in}{0.962847in}}%
\pgfpathcurveto{\pgfqpoint{9.964011in}{0.967890in}}{\pgfqpoint{9.962007in}{0.972728in}}{\pgfqpoint{9.958441in}{0.976295in}}%
\pgfpathcurveto{\pgfqpoint{9.954874in}{0.979861in}}{\pgfqpoint{9.950036in}{0.981865in}}{\pgfqpoint{9.944993in}{0.981865in}}%
\pgfpathcurveto{\pgfqpoint{9.939949in}{0.981865in}}{\pgfqpoint{9.935111in}{0.979861in}}{\pgfqpoint{9.931545in}{0.976295in}}%
\pgfpathcurveto{\pgfqpoint{9.927978in}{0.972728in}}{\pgfqpoint{9.925975in}{0.967890in}}{\pgfqpoint{9.925975in}{0.962847in}}%
\pgfpathcurveto{\pgfqpoint{9.925975in}{0.957803in}}{\pgfqpoint{9.927978in}{0.952965in}}{\pgfqpoint{9.931545in}{0.949399in}}%
\pgfpathcurveto{\pgfqpoint{9.935111in}{0.945832in}}{\pgfqpoint{9.939949in}{0.943829in}}{\pgfqpoint{9.944993in}{0.943829in}}%
\pgfpathclose%
\pgfusepath{fill}%
\end{pgfscope}%
\begin{pgfscope}%
\pgfpathrectangle{\pgfqpoint{6.572727in}{0.474100in}}{\pgfqpoint{4.227273in}{3.318700in}}%
\pgfusepath{clip}%
\pgfsetbuttcap%
\pgfsetroundjoin%
\definecolor{currentfill}{rgb}{0.127568,0.566949,0.550556}%
\pgfsetfillcolor{currentfill}%
\pgfsetfillopacity{0.700000}%
\pgfsetlinewidth{0.000000pt}%
\definecolor{currentstroke}{rgb}{0.000000,0.000000,0.000000}%
\pgfsetstrokecolor{currentstroke}%
\pgfsetstrokeopacity{0.700000}%
\pgfsetdash{}{0pt}%
\pgfpathmoveto{\pgfqpoint{7.891715in}{1.694903in}}%
\pgfpathcurveto{\pgfqpoint{7.896759in}{1.694903in}}{\pgfqpoint{7.901597in}{1.696907in}}{\pgfqpoint{7.905163in}{1.700473in}}%
\pgfpathcurveto{\pgfqpoint{7.908729in}{1.704040in}}{\pgfqpoint{7.910733in}{1.708877in}}{\pgfqpoint{7.910733in}{1.713921in}}%
\pgfpathcurveto{\pgfqpoint{7.910733in}{1.718965in}}{\pgfqpoint{7.908729in}{1.723803in}}{\pgfqpoint{7.905163in}{1.727369in}}%
\pgfpathcurveto{\pgfqpoint{7.901597in}{1.730935in}}{\pgfqpoint{7.896759in}{1.732939in}}{\pgfqpoint{7.891715in}{1.732939in}}%
\pgfpathcurveto{\pgfqpoint{7.886672in}{1.732939in}}{\pgfqpoint{7.881834in}{1.730935in}}{\pgfqpoint{7.878267in}{1.727369in}}%
\pgfpathcurveto{\pgfqpoint{7.874701in}{1.723803in}}{\pgfqpoint{7.872697in}{1.718965in}}{\pgfqpoint{7.872697in}{1.713921in}}%
\pgfpathcurveto{\pgfqpoint{7.872697in}{1.708877in}}{\pgfqpoint{7.874701in}{1.704040in}}{\pgfqpoint{7.878267in}{1.700473in}}%
\pgfpathcurveto{\pgfqpoint{7.881834in}{1.696907in}}{\pgfqpoint{7.886672in}{1.694903in}}{\pgfqpoint{7.891715in}{1.694903in}}%
\pgfpathclose%
\pgfusepath{fill}%
\end{pgfscope}%
\begin{pgfscope}%
\pgfpathrectangle{\pgfqpoint{6.572727in}{0.474100in}}{\pgfqpoint{4.227273in}{3.318700in}}%
\pgfusepath{clip}%
\pgfsetbuttcap%
\pgfsetroundjoin%
\definecolor{currentfill}{rgb}{0.127568,0.566949,0.550556}%
\pgfsetfillcolor{currentfill}%
\pgfsetfillopacity{0.700000}%
\pgfsetlinewidth{0.000000pt}%
\definecolor{currentstroke}{rgb}{0.000000,0.000000,0.000000}%
\pgfsetstrokecolor{currentstroke}%
\pgfsetstrokeopacity{0.700000}%
\pgfsetdash{}{0pt}%
\pgfpathmoveto{\pgfqpoint{8.383135in}{2.445289in}}%
\pgfpathcurveto{\pgfqpoint{8.388178in}{2.445289in}}{\pgfqpoint{8.393016in}{2.447293in}}{\pgfqpoint{8.396583in}{2.450859in}}%
\pgfpathcurveto{\pgfqpoint{8.400149in}{2.454425in}}{\pgfqpoint{8.402153in}{2.459263in}}{\pgfqpoint{8.402153in}{2.464307in}}%
\pgfpathcurveto{\pgfqpoint{8.402153in}{2.469351in}}{\pgfqpoint{8.400149in}{2.474188in}}{\pgfqpoint{8.396583in}{2.477755in}}%
\pgfpathcurveto{\pgfqpoint{8.393016in}{2.481321in}}{\pgfqpoint{8.388178in}{2.483325in}}{\pgfqpoint{8.383135in}{2.483325in}}%
\pgfpathcurveto{\pgfqpoint{8.378091in}{2.483325in}}{\pgfqpoint{8.373253in}{2.481321in}}{\pgfqpoint{8.369687in}{2.477755in}}%
\pgfpathcurveto{\pgfqpoint{8.366120in}{2.474188in}}{\pgfqpoint{8.364117in}{2.469351in}}{\pgfqpoint{8.364117in}{2.464307in}}%
\pgfpathcurveto{\pgfqpoint{8.364117in}{2.459263in}}{\pgfqpoint{8.366120in}{2.454425in}}{\pgfqpoint{8.369687in}{2.450859in}}%
\pgfpathcurveto{\pgfqpoint{8.373253in}{2.447293in}}{\pgfqpoint{8.378091in}{2.445289in}}{\pgfqpoint{8.383135in}{2.445289in}}%
\pgfpathclose%
\pgfusepath{fill}%
\end{pgfscope}%
\begin{pgfscope}%
\pgfpathrectangle{\pgfqpoint{6.572727in}{0.474100in}}{\pgfqpoint{4.227273in}{3.318700in}}%
\pgfusepath{clip}%
\pgfsetbuttcap%
\pgfsetroundjoin%
\definecolor{currentfill}{rgb}{0.127568,0.566949,0.550556}%
\pgfsetfillcolor{currentfill}%
\pgfsetfillopacity{0.700000}%
\pgfsetlinewidth{0.000000pt}%
\definecolor{currentstroke}{rgb}{0.000000,0.000000,0.000000}%
\pgfsetstrokecolor{currentstroke}%
\pgfsetstrokeopacity{0.700000}%
\pgfsetdash{}{0pt}%
\pgfpathmoveto{\pgfqpoint{8.008951in}{1.930338in}}%
\pgfpathcurveto{\pgfqpoint{8.013995in}{1.930338in}}{\pgfqpoint{8.018833in}{1.932342in}}{\pgfqpoint{8.022399in}{1.935908in}}%
\pgfpathcurveto{\pgfqpoint{8.025966in}{1.939475in}}{\pgfqpoint{8.027969in}{1.944312in}}{\pgfqpoint{8.027969in}{1.949356in}}%
\pgfpathcurveto{\pgfqpoint{8.027969in}{1.954400in}}{\pgfqpoint{8.025966in}{1.959238in}}{\pgfqpoint{8.022399in}{1.962804in}}%
\pgfpathcurveto{\pgfqpoint{8.018833in}{1.966370in}}{\pgfqpoint{8.013995in}{1.968374in}}{\pgfqpoint{8.008951in}{1.968374in}}%
\pgfpathcurveto{\pgfqpoint{8.003908in}{1.968374in}}{\pgfqpoint{7.999070in}{1.966370in}}{\pgfqpoint{7.995503in}{1.962804in}}%
\pgfpathcurveto{\pgfqpoint{7.991937in}{1.959238in}}{\pgfqpoint{7.989933in}{1.954400in}}{\pgfqpoint{7.989933in}{1.949356in}}%
\pgfpathcurveto{\pgfqpoint{7.989933in}{1.944312in}}{\pgfqpoint{7.991937in}{1.939475in}}{\pgfqpoint{7.995503in}{1.935908in}}%
\pgfpathcurveto{\pgfqpoint{7.999070in}{1.932342in}}{\pgfqpoint{8.003908in}{1.930338in}}{\pgfqpoint{8.008951in}{1.930338in}}%
\pgfpathclose%
\pgfusepath{fill}%
\end{pgfscope}%
\begin{pgfscope}%
\pgfpathrectangle{\pgfqpoint{6.572727in}{0.474100in}}{\pgfqpoint{4.227273in}{3.318700in}}%
\pgfusepath{clip}%
\pgfsetbuttcap%
\pgfsetroundjoin%
\definecolor{currentfill}{rgb}{0.127568,0.566949,0.550556}%
\pgfsetfillcolor{currentfill}%
\pgfsetfillopacity{0.700000}%
\pgfsetlinewidth{0.000000pt}%
\definecolor{currentstroke}{rgb}{0.000000,0.000000,0.000000}%
\pgfsetstrokecolor{currentstroke}%
\pgfsetstrokeopacity{0.700000}%
\pgfsetdash{}{0pt}%
\pgfpathmoveto{\pgfqpoint{8.280721in}{3.009439in}}%
\pgfpathcurveto{\pgfqpoint{8.285764in}{3.009439in}}{\pgfqpoint{8.290602in}{3.011443in}}{\pgfqpoint{8.294168in}{3.015009in}}%
\pgfpathcurveto{\pgfqpoint{8.297735in}{3.018576in}}{\pgfqpoint{8.299739in}{3.023413in}}{\pgfqpoint{8.299739in}{3.028457in}}%
\pgfpathcurveto{\pgfqpoint{8.299739in}{3.033501in}}{\pgfqpoint{8.297735in}{3.038338in}}{\pgfqpoint{8.294168in}{3.041905in}}%
\pgfpathcurveto{\pgfqpoint{8.290602in}{3.045471in}}{\pgfqpoint{8.285764in}{3.047475in}}{\pgfqpoint{8.280721in}{3.047475in}}%
\pgfpathcurveto{\pgfqpoint{8.275677in}{3.047475in}}{\pgfqpoint{8.270839in}{3.045471in}}{\pgfqpoint{8.267273in}{3.041905in}}%
\pgfpathcurveto{\pgfqpoint{8.263706in}{3.038338in}}{\pgfqpoint{8.261702in}{3.033501in}}{\pgfqpoint{8.261702in}{3.028457in}}%
\pgfpathcurveto{\pgfqpoint{8.261702in}{3.023413in}}{\pgfqpoint{8.263706in}{3.018576in}}{\pgfqpoint{8.267273in}{3.015009in}}%
\pgfpathcurveto{\pgfqpoint{8.270839in}{3.011443in}}{\pgfqpoint{8.275677in}{3.009439in}}{\pgfqpoint{8.280721in}{3.009439in}}%
\pgfpathclose%
\pgfusepath{fill}%
\end{pgfscope}%
\begin{pgfscope}%
\pgfpathrectangle{\pgfqpoint{6.572727in}{0.474100in}}{\pgfqpoint{4.227273in}{3.318700in}}%
\pgfusepath{clip}%
\pgfsetbuttcap%
\pgfsetroundjoin%
\definecolor{currentfill}{rgb}{0.127568,0.566949,0.550556}%
\pgfsetfillcolor{currentfill}%
\pgfsetfillopacity{0.700000}%
\pgfsetlinewidth{0.000000pt}%
\definecolor{currentstroke}{rgb}{0.000000,0.000000,0.000000}%
\pgfsetstrokecolor{currentstroke}%
\pgfsetstrokeopacity{0.700000}%
\pgfsetdash{}{0pt}%
\pgfpathmoveto{\pgfqpoint{8.272375in}{1.625308in}}%
\pgfpathcurveto{\pgfqpoint{8.277419in}{1.625308in}}{\pgfqpoint{8.282256in}{1.627312in}}{\pgfqpoint{8.285823in}{1.630879in}}%
\pgfpathcurveto{\pgfqpoint{8.289389in}{1.634445in}}{\pgfqpoint{8.291393in}{1.639283in}}{\pgfqpoint{8.291393in}{1.644327in}}%
\pgfpathcurveto{\pgfqpoint{8.291393in}{1.649370in}}{\pgfqpoint{8.289389in}{1.654208in}}{\pgfqpoint{8.285823in}{1.657774in}}%
\pgfpathcurveto{\pgfqpoint{8.282256in}{1.661341in}}{\pgfqpoint{8.277419in}{1.663345in}}{\pgfqpoint{8.272375in}{1.663345in}}%
\pgfpathcurveto{\pgfqpoint{8.267331in}{1.663345in}}{\pgfqpoint{8.262494in}{1.661341in}}{\pgfqpoint{8.258927in}{1.657774in}}%
\pgfpathcurveto{\pgfqpoint{8.255361in}{1.654208in}}{\pgfqpoint{8.253357in}{1.649370in}}{\pgfqpoint{8.253357in}{1.644327in}}%
\pgfpathcurveto{\pgfqpoint{8.253357in}{1.639283in}}{\pgfqpoint{8.255361in}{1.634445in}}{\pgfqpoint{8.258927in}{1.630879in}}%
\pgfpathcurveto{\pgfqpoint{8.262494in}{1.627312in}}{\pgfqpoint{8.267331in}{1.625308in}}{\pgfqpoint{8.272375in}{1.625308in}}%
\pgfpathclose%
\pgfusepath{fill}%
\end{pgfscope}%
\begin{pgfscope}%
\pgfpathrectangle{\pgfqpoint{6.572727in}{0.474100in}}{\pgfqpoint{4.227273in}{3.318700in}}%
\pgfusepath{clip}%
\pgfsetbuttcap%
\pgfsetroundjoin%
\definecolor{currentfill}{rgb}{0.127568,0.566949,0.550556}%
\pgfsetfillcolor{currentfill}%
\pgfsetfillopacity{0.700000}%
\pgfsetlinewidth{0.000000pt}%
\definecolor{currentstroke}{rgb}{0.000000,0.000000,0.000000}%
\pgfsetstrokecolor{currentstroke}%
\pgfsetstrokeopacity{0.700000}%
\pgfsetdash{}{0pt}%
\pgfpathmoveto{\pgfqpoint{8.306057in}{3.026902in}}%
\pgfpathcurveto{\pgfqpoint{8.311101in}{3.026902in}}{\pgfqpoint{8.315939in}{3.028906in}}{\pgfqpoint{8.319505in}{3.032473in}}%
\pgfpathcurveto{\pgfqpoint{8.323071in}{3.036039in}}{\pgfqpoint{8.325075in}{3.040877in}}{\pgfqpoint{8.325075in}{3.045920in}}%
\pgfpathcurveto{\pgfqpoint{8.325075in}{3.050964in}}{\pgfqpoint{8.323071in}{3.055802in}}{\pgfqpoint{8.319505in}{3.059368in}}%
\pgfpathcurveto{\pgfqpoint{8.315939in}{3.062935in}}{\pgfqpoint{8.311101in}{3.064939in}}{\pgfqpoint{8.306057in}{3.064939in}}%
\pgfpathcurveto{\pgfqpoint{8.301013in}{3.064939in}}{\pgfqpoint{8.296176in}{3.062935in}}{\pgfqpoint{8.292609in}{3.059368in}}%
\pgfpathcurveto{\pgfqpoint{8.289043in}{3.055802in}}{\pgfqpoint{8.287039in}{3.050964in}}{\pgfqpoint{8.287039in}{3.045920in}}%
\pgfpathcurveto{\pgfqpoint{8.287039in}{3.040877in}}{\pgfqpoint{8.289043in}{3.036039in}}{\pgfqpoint{8.292609in}{3.032473in}}%
\pgfpathcurveto{\pgfqpoint{8.296176in}{3.028906in}}{\pgfqpoint{8.301013in}{3.026902in}}{\pgfqpoint{8.306057in}{3.026902in}}%
\pgfpathclose%
\pgfusepath{fill}%
\end{pgfscope}%
\begin{pgfscope}%
\pgfpathrectangle{\pgfqpoint{6.572727in}{0.474100in}}{\pgfqpoint{4.227273in}{3.318700in}}%
\pgfusepath{clip}%
\pgfsetbuttcap%
\pgfsetroundjoin%
\definecolor{currentfill}{rgb}{0.993248,0.906157,0.143936}%
\pgfsetfillcolor{currentfill}%
\pgfsetfillopacity{0.700000}%
\pgfsetlinewidth{0.000000pt}%
\definecolor{currentstroke}{rgb}{0.000000,0.000000,0.000000}%
\pgfsetstrokecolor{currentstroke}%
\pgfsetstrokeopacity{0.700000}%
\pgfsetdash{}{0pt}%
\pgfpathmoveto{\pgfqpoint{9.770135in}{1.553215in}}%
\pgfpathcurveto{\pgfqpoint{9.775179in}{1.553215in}}{\pgfqpoint{9.780017in}{1.555219in}}{\pgfqpoint{9.783583in}{1.558785in}}%
\pgfpathcurveto{\pgfqpoint{9.787150in}{1.562352in}}{\pgfqpoint{9.789154in}{1.567190in}}{\pgfqpoint{9.789154in}{1.572233in}}%
\pgfpathcurveto{\pgfqpoint{9.789154in}{1.577277in}}{\pgfqpoint{9.787150in}{1.582115in}}{\pgfqpoint{9.783583in}{1.585681in}}%
\pgfpathcurveto{\pgfqpoint{9.780017in}{1.589248in}}{\pgfqpoint{9.775179in}{1.591251in}}{\pgfqpoint{9.770135in}{1.591251in}}%
\pgfpathcurveto{\pgfqpoint{9.765092in}{1.591251in}}{\pgfqpoint{9.760254in}{1.589248in}}{\pgfqpoint{9.756688in}{1.585681in}}%
\pgfpathcurveto{\pgfqpoint{9.753121in}{1.582115in}}{\pgfqpoint{9.751117in}{1.577277in}}{\pgfqpoint{9.751117in}{1.572233in}}%
\pgfpathcurveto{\pgfqpoint{9.751117in}{1.567190in}}{\pgfqpoint{9.753121in}{1.562352in}}{\pgfqpoint{9.756688in}{1.558785in}}%
\pgfpathcurveto{\pgfqpoint{9.760254in}{1.555219in}}{\pgfqpoint{9.765092in}{1.553215in}}{\pgfqpoint{9.770135in}{1.553215in}}%
\pgfpathclose%
\pgfusepath{fill}%
\end{pgfscope}%
\begin{pgfscope}%
\pgfpathrectangle{\pgfqpoint{6.572727in}{0.474100in}}{\pgfqpoint{4.227273in}{3.318700in}}%
\pgfusepath{clip}%
\pgfsetbuttcap%
\pgfsetroundjoin%
\definecolor{currentfill}{rgb}{0.127568,0.566949,0.550556}%
\pgfsetfillcolor{currentfill}%
\pgfsetfillopacity{0.700000}%
\pgfsetlinewidth{0.000000pt}%
\definecolor{currentstroke}{rgb}{0.000000,0.000000,0.000000}%
\pgfsetstrokecolor{currentstroke}%
\pgfsetstrokeopacity{0.700000}%
\pgfsetdash{}{0pt}%
\pgfpathmoveto{\pgfqpoint{8.649131in}{3.249225in}}%
\pgfpathcurveto{\pgfqpoint{8.654175in}{3.249225in}}{\pgfqpoint{8.659012in}{3.251229in}}{\pgfqpoint{8.662579in}{3.254796in}}%
\pgfpathcurveto{\pgfqpoint{8.666145in}{3.258362in}}{\pgfqpoint{8.668149in}{3.263200in}}{\pgfqpoint{8.668149in}{3.268243in}}%
\pgfpathcurveto{\pgfqpoint{8.668149in}{3.273287in}}{\pgfqpoint{8.666145in}{3.278125in}}{\pgfqpoint{8.662579in}{3.281691in}}%
\pgfpathcurveto{\pgfqpoint{8.659012in}{3.285258in}}{\pgfqpoint{8.654175in}{3.287262in}}{\pgfqpoint{8.649131in}{3.287262in}}%
\pgfpathcurveto{\pgfqpoint{8.644087in}{3.287262in}}{\pgfqpoint{8.639250in}{3.285258in}}{\pgfqpoint{8.635683in}{3.281691in}}%
\pgfpathcurveto{\pgfqpoint{8.632117in}{3.278125in}}{\pgfqpoint{8.630113in}{3.273287in}}{\pgfqpoint{8.630113in}{3.268243in}}%
\pgfpathcurveto{\pgfqpoint{8.630113in}{3.263200in}}{\pgfqpoint{8.632117in}{3.258362in}}{\pgfqpoint{8.635683in}{3.254796in}}%
\pgfpathcurveto{\pgfqpoint{8.639250in}{3.251229in}}{\pgfqpoint{8.644087in}{3.249225in}}{\pgfqpoint{8.649131in}{3.249225in}}%
\pgfpathclose%
\pgfusepath{fill}%
\end{pgfscope}%
\begin{pgfscope}%
\pgfpathrectangle{\pgfqpoint{6.572727in}{0.474100in}}{\pgfqpoint{4.227273in}{3.318700in}}%
\pgfusepath{clip}%
\pgfsetbuttcap%
\pgfsetroundjoin%
\definecolor{currentfill}{rgb}{0.993248,0.906157,0.143936}%
\pgfsetfillcolor{currentfill}%
\pgfsetfillopacity{0.700000}%
\pgfsetlinewidth{0.000000pt}%
\definecolor{currentstroke}{rgb}{0.000000,0.000000,0.000000}%
\pgfsetstrokecolor{currentstroke}%
\pgfsetstrokeopacity{0.700000}%
\pgfsetdash{}{0pt}%
\pgfpathmoveto{\pgfqpoint{9.342677in}{1.071246in}}%
\pgfpathcurveto{\pgfqpoint{9.347721in}{1.071246in}}{\pgfqpoint{9.352558in}{1.073250in}}{\pgfqpoint{9.356125in}{1.076816in}}%
\pgfpathcurveto{\pgfqpoint{9.359691in}{1.080382in}}{\pgfqpoint{9.361695in}{1.085220in}}{\pgfqpoint{9.361695in}{1.090264in}}%
\pgfpathcurveto{\pgfqpoint{9.361695in}{1.095307in}}{\pgfqpoint{9.359691in}{1.100145in}}{\pgfqpoint{9.356125in}{1.103712in}}%
\pgfpathcurveto{\pgfqpoint{9.352558in}{1.107278in}}{\pgfqpoint{9.347721in}{1.109282in}}{\pgfqpoint{9.342677in}{1.109282in}}%
\pgfpathcurveto{\pgfqpoint{9.337633in}{1.109282in}}{\pgfqpoint{9.332796in}{1.107278in}}{\pgfqpoint{9.329229in}{1.103712in}}%
\pgfpathcurveto{\pgfqpoint{9.325663in}{1.100145in}}{\pgfqpoint{9.323659in}{1.095307in}}{\pgfqpoint{9.323659in}{1.090264in}}%
\pgfpathcurveto{\pgfqpoint{9.323659in}{1.085220in}}{\pgfqpoint{9.325663in}{1.080382in}}{\pgfqpoint{9.329229in}{1.076816in}}%
\pgfpathcurveto{\pgfqpoint{9.332796in}{1.073250in}}{\pgfqpoint{9.337633in}{1.071246in}}{\pgfqpoint{9.342677in}{1.071246in}}%
\pgfpathclose%
\pgfusepath{fill}%
\end{pgfscope}%
\begin{pgfscope}%
\pgfpathrectangle{\pgfqpoint{6.572727in}{0.474100in}}{\pgfqpoint{4.227273in}{3.318700in}}%
\pgfusepath{clip}%
\pgfsetbuttcap%
\pgfsetroundjoin%
\definecolor{currentfill}{rgb}{0.993248,0.906157,0.143936}%
\pgfsetfillcolor{currentfill}%
\pgfsetfillopacity{0.700000}%
\pgfsetlinewidth{0.000000pt}%
\definecolor{currentstroke}{rgb}{0.000000,0.000000,0.000000}%
\pgfsetstrokecolor{currentstroke}%
\pgfsetstrokeopacity{0.700000}%
\pgfsetdash{}{0pt}%
\pgfpathmoveto{\pgfqpoint{9.502821in}{1.553700in}}%
\pgfpathcurveto{\pgfqpoint{9.507864in}{1.553700in}}{\pgfqpoint{9.512702in}{1.555704in}}{\pgfqpoint{9.516268in}{1.559271in}}%
\pgfpathcurveto{\pgfqpoint{9.519835in}{1.562837in}}{\pgfqpoint{9.521839in}{1.567675in}}{\pgfqpoint{9.521839in}{1.572719in}}%
\pgfpathcurveto{\pgfqpoint{9.521839in}{1.577762in}}{\pgfqpoint{9.519835in}{1.582600in}}{\pgfqpoint{9.516268in}{1.586166in}}%
\pgfpathcurveto{\pgfqpoint{9.512702in}{1.589733in}}{\pgfqpoint{9.507864in}{1.591737in}}{\pgfqpoint{9.502821in}{1.591737in}}%
\pgfpathcurveto{\pgfqpoint{9.497777in}{1.591737in}}{\pgfqpoint{9.492939in}{1.589733in}}{\pgfqpoint{9.489373in}{1.586166in}}%
\pgfpathcurveto{\pgfqpoint{9.485806in}{1.582600in}}{\pgfqpoint{9.483802in}{1.577762in}}{\pgfqpoint{9.483802in}{1.572719in}}%
\pgfpathcurveto{\pgfqpoint{9.483802in}{1.567675in}}{\pgfqpoint{9.485806in}{1.562837in}}{\pgfqpoint{9.489373in}{1.559271in}}%
\pgfpathcurveto{\pgfqpoint{9.492939in}{1.555704in}}{\pgfqpoint{9.497777in}{1.553700in}}{\pgfqpoint{9.502821in}{1.553700in}}%
\pgfpathclose%
\pgfusepath{fill}%
\end{pgfscope}%
\begin{pgfscope}%
\pgfpathrectangle{\pgfqpoint{6.572727in}{0.474100in}}{\pgfqpoint{4.227273in}{3.318700in}}%
\pgfusepath{clip}%
\pgfsetbuttcap%
\pgfsetroundjoin%
\definecolor{currentfill}{rgb}{0.127568,0.566949,0.550556}%
\pgfsetfillcolor{currentfill}%
\pgfsetfillopacity{0.700000}%
\pgfsetlinewidth{0.000000pt}%
\definecolor{currentstroke}{rgb}{0.000000,0.000000,0.000000}%
\pgfsetstrokecolor{currentstroke}%
\pgfsetstrokeopacity{0.700000}%
\pgfsetdash{}{0pt}%
\pgfpathmoveto{\pgfqpoint{7.253956in}{1.404765in}}%
\pgfpathcurveto{\pgfqpoint{7.258999in}{1.404765in}}{\pgfqpoint{7.263837in}{1.406769in}}{\pgfqpoint{7.267404in}{1.410335in}}%
\pgfpathcurveto{\pgfqpoint{7.270970in}{1.413902in}}{\pgfqpoint{7.272974in}{1.418740in}}{\pgfqpoint{7.272974in}{1.423783in}}%
\pgfpathcurveto{\pgfqpoint{7.272974in}{1.428827in}}{\pgfqpoint{7.270970in}{1.433665in}}{\pgfqpoint{7.267404in}{1.437231in}}%
\pgfpathcurveto{\pgfqpoint{7.263837in}{1.440798in}}{\pgfqpoint{7.258999in}{1.442801in}}{\pgfqpoint{7.253956in}{1.442801in}}%
\pgfpathcurveto{\pgfqpoint{7.248912in}{1.442801in}}{\pgfqpoint{7.244074in}{1.440798in}}{\pgfqpoint{7.240508in}{1.437231in}}%
\pgfpathcurveto{\pgfqpoint{7.236942in}{1.433665in}}{\pgfqpoint{7.234938in}{1.428827in}}{\pgfqpoint{7.234938in}{1.423783in}}%
\pgfpathcurveto{\pgfqpoint{7.234938in}{1.418740in}}{\pgfqpoint{7.236942in}{1.413902in}}{\pgfqpoint{7.240508in}{1.410335in}}%
\pgfpathcurveto{\pgfqpoint{7.244074in}{1.406769in}}{\pgfqpoint{7.248912in}{1.404765in}}{\pgfqpoint{7.253956in}{1.404765in}}%
\pgfpathclose%
\pgfusepath{fill}%
\end{pgfscope}%
\begin{pgfscope}%
\pgfpathrectangle{\pgfqpoint{6.572727in}{0.474100in}}{\pgfqpoint{4.227273in}{3.318700in}}%
\pgfusepath{clip}%
\pgfsetbuttcap%
\pgfsetroundjoin%
\definecolor{currentfill}{rgb}{0.127568,0.566949,0.550556}%
\pgfsetfillcolor{currentfill}%
\pgfsetfillopacity{0.700000}%
\pgfsetlinewidth{0.000000pt}%
\definecolor{currentstroke}{rgb}{0.000000,0.000000,0.000000}%
\pgfsetstrokecolor{currentstroke}%
\pgfsetstrokeopacity{0.700000}%
\pgfsetdash{}{0pt}%
\pgfpathmoveto{\pgfqpoint{8.312186in}{1.696756in}}%
\pgfpathcurveto{\pgfqpoint{8.317230in}{1.696756in}}{\pgfqpoint{8.322068in}{1.698760in}}{\pgfqpoint{8.325634in}{1.702327in}}%
\pgfpathcurveto{\pgfqpoint{8.329200in}{1.705893in}}{\pgfqpoint{8.331204in}{1.710731in}}{\pgfqpoint{8.331204in}{1.715775in}}%
\pgfpathcurveto{\pgfqpoint{8.331204in}{1.720818in}}{\pgfqpoint{8.329200in}{1.725656in}}{\pgfqpoint{8.325634in}{1.729222in}}%
\pgfpathcurveto{\pgfqpoint{8.322068in}{1.732789in}}{\pgfqpoint{8.317230in}{1.734793in}}{\pgfqpoint{8.312186in}{1.734793in}}%
\pgfpathcurveto{\pgfqpoint{8.307142in}{1.734793in}}{\pgfqpoint{8.302305in}{1.732789in}}{\pgfqpoint{8.298738in}{1.729222in}}%
\pgfpathcurveto{\pgfqpoint{8.295172in}{1.725656in}}{\pgfqpoint{8.293168in}{1.720818in}}{\pgfqpoint{8.293168in}{1.715775in}}%
\pgfpathcurveto{\pgfqpoint{8.293168in}{1.710731in}}{\pgfqpoint{8.295172in}{1.705893in}}{\pgfqpoint{8.298738in}{1.702327in}}%
\pgfpathcurveto{\pgfqpoint{8.302305in}{1.698760in}}{\pgfqpoint{8.307142in}{1.696756in}}{\pgfqpoint{8.312186in}{1.696756in}}%
\pgfpathclose%
\pgfusepath{fill}%
\end{pgfscope}%
\begin{pgfscope}%
\pgfpathrectangle{\pgfqpoint{6.572727in}{0.474100in}}{\pgfqpoint{4.227273in}{3.318700in}}%
\pgfusepath{clip}%
\pgfsetbuttcap%
\pgfsetroundjoin%
\definecolor{currentfill}{rgb}{0.127568,0.566949,0.550556}%
\pgfsetfillcolor{currentfill}%
\pgfsetfillopacity{0.700000}%
\pgfsetlinewidth{0.000000pt}%
\definecolor{currentstroke}{rgb}{0.000000,0.000000,0.000000}%
\pgfsetstrokecolor{currentstroke}%
\pgfsetstrokeopacity{0.700000}%
\pgfsetdash{}{0pt}%
\pgfpathmoveto{\pgfqpoint{8.155875in}{3.323079in}}%
\pgfpathcurveto{\pgfqpoint{8.160918in}{3.323079in}}{\pgfqpoint{8.165756in}{3.325083in}}{\pgfqpoint{8.169322in}{3.328649in}}%
\pgfpathcurveto{\pgfqpoint{8.172889in}{3.332216in}}{\pgfqpoint{8.174893in}{3.337053in}}{\pgfqpoint{8.174893in}{3.342097in}}%
\pgfpathcurveto{\pgfqpoint{8.174893in}{3.347141in}}{\pgfqpoint{8.172889in}{3.351979in}}{\pgfqpoint{8.169322in}{3.355545in}}%
\pgfpathcurveto{\pgfqpoint{8.165756in}{3.359111in}}{\pgfqpoint{8.160918in}{3.361115in}}{\pgfqpoint{8.155875in}{3.361115in}}%
\pgfpathcurveto{\pgfqpoint{8.150831in}{3.361115in}}{\pgfqpoint{8.145993in}{3.359111in}}{\pgfqpoint{8.142427in}{3.355545in}}%
\pgfpathcurveto{\pgfqpoint{8.138860in}{3.351979in}}{\pgfqpoint{8.136856in}{3.347141in}}{\pgfqpoint{8.136856in}{3.342097in}}%
\pgfpathcurveto{\pgfqpoint{8.136856in}{3.337053in}}{\pgfqpoint{8.138860in}{3.332216in}}{\pgfqpoint{8.142427in}{3.328649in}}%
\pgfpathcurveto{\pgfqpoint{8.145993in}{3.325083in}}{\pgfqpoint{8.150831in}{3.323079in}}{\pgfqpoint{8.155875in}{3.323079in}}%
\pgfpathclose%
\pgfusepath{fill}%
\end{pgfscope}%
\begin{pgfscope}%
\pgfpathrectangle{\pgfqpoint{6.572727in}{0.474100in}}{\pgfqpoint{4.227273in}{3.318700in}}%
\pgfusepath{clip}%
\pgfsetbuttcap%
\pgfsetroundjoin%
\definecolor{currentfill}{rgb}{0.127568,0.566949,0.550556}%
\pgfsetfillcolor{currentfill}%
\pgfsetfillopacity{0.700000}%
\pgfsetlinewidth{0.000000pt}%
\definecolor{currentstroke}{rgb}{0.000000,0.000000,0.000000}%
\pgfsetstrokecolor{currentstroke}%
\pgfsetstrokeopacity{0.700000}%
\pgfsetdash{}{0pt}%
\pgfpathmoveto{\pgfqpoint{8.160949in}{1.371168in}}%
\pgfpathcurveto{\pgfqpoint{8.165993in}{1.371168in}}{\pgfqpoint{8.170831in}{1.373172in}}{\pgfqpoint{8.174397in}{1.376738in}}%
\pgfpathcurveto{\pgfqpoint{8.177964in}{1.380305in}}{\pgfqpoint{8.179968in}{1.385142in}}{\pgfqpoint{8.179968in}{1.390186in}}%
\pgfpathcurveto{\pgfqpoint{8.179968in}{1.395230in}}{\pgfqpoint{8.177964in}{1.400068in}}{\pgfqpoint{8.174397in}{1.403634in}}%
\pgfpathcurveto{\pgfqpoint{8.170831in}{1.407200in}}{\pgfqpoint{8.165993in}{1.409204in}}{\pgfqpoint{8.160949in}{1.409204in}}%
\pgfpathcurveto{\pgfqpoint{8.155906in}{1.409204in}}{\pgfqpoint{8.151068in}{1.407200in}}{\pgfqpoint{8.147502in}{1.403634in}}%
\pgfpathcurveto{\pgfqpoint{8.143935in}{1.400068in}}{\pgfqpoint{8.141931in}{1.395230in}}{\pgfqpoint{8.141931in}{1.390186in}}%
\pgfpathcurveto{\pgfqpoint{8.141931in}{1.385142in}}{\pgfqpoint{8.143935in}{1.380305in}}{\pgfqpoint{8.147502in}{1.376738in}}%
\pgfpathcurveto{\pgfqpoint{8.151068in}{1.373172in}}{\pgfqpoint{8.155906in}{1.371168in}}{\pgfqpoint{8.160949in}{1.371168in}}%
\pgfpathclose%
\pgfusepath{fill}%
\end{pgfscope}%
\begin{pgfscope}%
\pgfpathrectangle{\pgfqpoint{6.572727in}{0.474100in}}{\pgfqpoint{4.227273in}{3.318700in}}%
\pgfusepath{clip}%
\pgfsetbuttcap%
\pgfsetroundjoin%
\definecolor{currentfill}{rgb}{0.127568,0.566949,0.550556}%
\pgfsetfillcolor{currentfill}%
\pgfsetfillopacity{0.700000}%
\pgfsetlinewidth{0.000000pt}%
\definecolor{currentstroke}{rgb}{0.000000,0.000000,0.000000}%
\pgfsetstrokecolor{currentstroke}%
\pgfsetstrokeopacity{0.700000}%
\pgfsetdash{}{0pt}%
\pgfpathmoveto{\pgfqpoint{8.148663in}{2.752842in}}%
\pgfpathcurveto{\pgfqpoint{8.153706in}{2.752842in}}{\pgfqpoint{8.158544in}{2.754846in}}{\pgfqpoint{8.162111in}{2.758412in}}%
\pgfpathcurveto{\pgfqpoint{8.165677in}{2.761979in}}{\pgfqpoint{8.167681in}{2.766817in}}{\pgfqpoint{8.167681in}{2.771860in}}%
\pgfpathcurveto{\pgfqpoint{8.167681in}{2.776904in}}{\pgfqpoint{8.165677in}{2.781742in}}{\pgfqpoint{8.162111in}{2.785308in}}%
\pgfpathcurveto{\pgfqpoint{8.158544in}{2.788874in}}{\pgfqpoint{8.153706in}{2.790878in}}{\pgfqpoint{8.148663in}{2.790878in}}%
\pgfpathcurveto{\pgfqpoint{8.143619in}{2.790878in}}{\pgfqpoint{8.138781in}{2.788874in}}{\pgfqpoint{8.135215in}{2.785308in}}%
\pgfpathcurveto{\pgfqpoint{8.131648in}{2.781742in}}{\pgfqpoint{8.129645in}{2.776904in}}{\pgfqpoint{8.129645in}{2.771860in}}%
\pgfpathcurveto{\pgfqpoint{8.129645in}{2.766817in}}{\pgfqpoint{8.131648in}{2.761979in}}{\pgfqpoint{8.135215in}{2.758412in}}%
\pgfpathcurveto{\pgfqpoint{8.138781in}{2.754846in}}{\pgfqpoint{8.143619in}{2.752842in}}{\pgfqpoint{8.148663in}{2.752842in}}%
\pgfpathclose%
\pgfusepath{fill}%
\end{pgfscope}%
\begin{pgfscope}%
\pgfpathrectangle{\pgfqpoint{6.572727in}{0.474100in}}{\pgfqpoint{4.227273in}{3.318700in}}%
\pgfusepath{clip}%
\pgfsetbuttcap%
\pgfsetroundjoin%
\definecolor{currentfill}{rgb}{0.993248,0.906157,0.143936}%
\pgfsetfillcolor{currentfill}%
\pgfsetfillopacity{0.700000}%
\pgfsetlinewidth{0.000000pt}%
\definecolor{currentstroke}{rgb}{0.000000,0.000000,0.000000}%
\pgfsetstrokecolor{currentstroke}%
\pgfsetstrokeopacity{0.700000}%
\pgfsetdash{}{0pt}%
\pgfpathmoveto{\pgfqpoint{9.442863in}{1.462796in}}%
\pgfpathcurveto{\pgfqpoint{9.447906in}{1.462796in}}{\pgfqpoint{9.452744in}{1.464800in}}{\pgfqpoint{9.456311in}{1.468366in}}%
\pgfpathcurveto{\pgfqpoint{9.459877in}{1.471933in}}{\pgfqpoint{9.461881in}{1.476770in}}{\pgfqpoint{9.461881in}{1.481814in}}%
\pgfpathcurveto{\pgfqpoint{9.461881in}{1.486858in}}{\pgfqpoint{9.459877in}{1.491695in}}{\pgfqpoint{9.456311in}{1.495262in}}%
\pgfpathcurveto{\pgfqpoint{9.452744in}{1.498828in}}{\pgfqpoint{9.447906in}{1.500832in}}{\pgfqpoint{9.442863in}{1.500832in}}%
\pgfpathcurveto{\pgfqpoint{9.437819in}{1.500832in}}{\pgfqpoint{9.432981in}{1.498828in}}{\pgfqpoint{9.429415in}{1.495262in}}%
\pgfpathcurveto{\pgfqpoint{9.425848in}{1.491695in}}{\pgfqpoint{9.423845in}{1.486858in}}{\pgfqpoint{9.423845in}{1.481814in}}%
\pgfpathcurveto{\pgfqpoint{9.423845in}{1.476770in}}{\pgfqpoint{9.425848in}{1.471933in}}{\pgfqpoint{9.429415in}{1.468366in}}%
\pgfpathcurveto{\pgfqpoint{9.432981in}{1.464800in}}{\pgfqpoint{9.437819in}{1.462796in}}{\pgfqpoint{9.442863in}{1.462796in}}%
\pgfpathclose%
\pgfusepath{fill}%
\end{pgfscope}%
\begin{pgfscope}%
\pgfpathrectangle{\pgfqpoint{6.572727in}{0.474100in}}{\pgfqpoint{4.227273in}{3.318700in}}%
\pgfusepath{clip}%
\pgfsetbuttcap%
\pgfsetroundjoin%
\definecolor{currentfill}{rgb}{0.127568,0.566949,0.550556}%
\pgfsetfillcolor{currentfill}%
\pgfsetfillopacity{0.700000}%
\pgfsetlinewidth{0.000000pt}%
\definecolor{currentstroke}{rgb}{0.000000,0.000000,0.000000}%
\pgfsetstrokecolor{currentstroke}%
\pgfsetstrokeopacity{0.700000}%
\pgfsetdash{}{0pt}%
\pgfpathmoveto{\pgfqpoint{8.100618in}{1.792480in}}%
\pgfpathcurveto{\pgfqpoint{8.105662in}{1.792480in}}{\pgfqpoint{8.110500in}{1.794484in}}{\pgfqpoint{8.114066in}{1.798051in}}%
\pgfpathcurveto{\pgfqpoint{8.117633in}{1.801617in}}{\pgfqpoint{8.119637in}{1.806455in}}{\pgfqpoint{8.119637in}{1.811499in}}%
\pgfpathcurveto{\pgfqpoint{8.119637in}{1.816542in}}{\pgfqpoint{8.117633in}{1.821380in}}{\pgfqpoint{8.114066in}{1.824946in}}%
\pgfpathcurveto{\pgfqpoint{8.110500in}{1.828513in}}{\pgfqpoint{8.105662in}{1.830517in}}{\pgfqpoint{8.100618in}{1.830517in}}%
\pgfpathcurveto{\pgfqpoint{8.095575in}{1.830517in}}{\pgfqpoint{8.090737in}{1.828513in}}{\pgfqpoint{8.087171in}{1.824946in}}%
\pgfpathcurveto{\pgfqpoint{8.083604in}{1.821380in}}{\pgfqpoint{8.081600in}{1.816542in}}{\pgfqpoint{8.081600in}{1.811499in}}%
\pgfpathcurveto{\pgfqpoint{8.081600in}{1.806455in}}{\pgfqpoint{8.083604in}{1.801617in}}{\pgfqpoint{8.087171in}{1.798051in}}%
\pgfpathcurveto{\pgfqpoint{8.090737in}{1.794484in}}{\pgfqpoint{8.095575in}{1.792480in}}{\pgfqpoint{8.100618in}{1.792480in}}%
\pgfpathclose%
\pgfusepath{fill}%
\end{pgfscope}%
\begin{pgfscope}%
\pgfpathrectangle{\pgfqpoint{6.572727in}{0.474100in}}{\pgfqpoint{4.227273in}{3.318700in}}%
\pgfusepath{clip}%
\pgfsetbuttcap%
\pgfsetroundjoin%
\definecolor{currentfill}{rgb}{0.127568,0.566949,0.550556}%
\pgfsetfillcolor{currentfill}%
\pgfsetfillopacity{0.700000}%
\pgfsetlinewidth{0.000000pt}%
\definecolor{currentstroke}{rgb}{0.000000,0.000000,0.000000}%
\pgfsetstrokecolor{currentstroke}%
\pgfsetstrokeopacity{0.700000}%
\pgfsetdash{}{0pt}%
\pgfpathmoveto{\pgfqpoint{8.936811in}{2.362006in}}%
\pgfpathcurveto{\pgfqpoint{8.941855in}{2.362006in}}{\pgfqpoint{8.946692in}{2.364010in}}{\pgfqpoint{8.950259in}{2.367576in}}%
\pgfpathcurveto{\pgfqpoint{8.953825in}{2.371143in}}{\pgfqpoint{8.955829in}{2.375980in}}{\pgfqpoint{8.955829in}{2.381024in}}%
\pgfpathcurveto{\pgfqpoint{8.955829in}{2.386068in}}{\pgfqpoint{8.953825in}{2.390905in}}{\pgfqpoint{8.950259in}{2.394472in}}%
\pgfpathcurveto{\pgfqpoint{8.946692in}{2.398038in}}{\pgfqpoint{8.941855in}{2.400042in}}{\pgfqpoint{8.936811in}{2.400042in}}%
\pgfpathcurveto{\pgfqpoint{8.931767in}{2.400042in}}{\pgfqpoint{8.926929in}{2.398038in}}{\pgfqpoint{8.923363in}{2.394472in}}%
\pgfpathcurveto{\pgfqpoint{8.919797in}{2.390905in}}{\pgfqpoint{8.917793in}{2.386068in}}{\pgfqpoint{8.917793in}{2.381024in}}%
\pgfpathcurveto{\pgfqpoint{8.917793in}{2.375980in}}{\pgfqpoint{8.919797in}{2.371143in}}{\pgfqpoint{8.923363in}{2.367576in}}%
\pgfpathcurveto{\pgfqpoint{8.926929in}{2.364010in}}{\pgfqpoint{8.931767in}{2.362006in}}{\pgfqpoint{8.936811in}{2.362006in}}%
\pgfpathclose%
\pgfusepath{fill}%
\end{pgfscope}%
\begin{pgfscope}%
\pgfpathrectangle{\pgfqpoint{6.572727in}{0.474100in}}{\pgfqpoint{4.227273in}{3.318700in}}%
\pgfusepath{clip}%
\pgfsetbuttcap%
\pgfsetroundjoin%
\definecolor{currentfill}{rgb}{0.993248,0.906157,0.143936}%
\pgfsetfillcolor{currentfill}%
\pgfsetfillopacity{0.700000}%
\pgfsetlinewidth{0.000000pt}%
\definecolor{currentstroke}{rgb}{0.000000,0.000000,0.000000}%
\pgfsetstrokecolor{currentstroke}%
\pgfsetstrokeopacity{0.700000}%
\pgfsetdash{}{0pt}%
\pgfpathmoveto{\pgfqpoint{8.822130in}{1.345310in}}%
\pgfpathcurveto{\pgfqpoint{8.827174in}{1.345310in}}{\pgfqpoint{8.832012in}{1.347314in}}{\pgfqpoint{8.835578in}{1.350881in}}%
\pgfpathcurveto{\pgfqpoint{8.839144in}{1.354447in}}{\pgfqpoint{8.841148in}{1.359285in}}{\pgfqpoint{8.841148in}{1.364329in}}%
\pgfpathcurveto{\pgfqpoint{8.841148in}{1.369372in}}{\pgfqpoint{8.839144in}{1.374210in}}{\pgfqpoint{8.835578in}{1.377776in}}%
\pgfpathcurveto{\pgfqpoint{8.832012in}{1.381343in}}{\pgfqpoint{8.827174in}{1.383347in}}{\pgfqpoint{8.822130in}{1.383347in}}%
\pgfpathcurveto{\pgfqpoint{8.817086in}{1.383347in}}{\pgfqpoint{8.812249in}{1.381343in}}{\pgfqpoint{8.808682in}{1.377776in}}%
\pgfpathcurveto{\pgfqpoint{8.805116in}{1.374210in}}{\pgfqpoint{8.803112in}{1.369372in}}{\pgfqpoint{8.803112in}{1.364329in}}%
\pgfpathcurveto{\pgfqpoint{8.803112in}{1.359285in}}{\pgfqpoint{8.805116in}{1.354447in}}{\pgfqpoint{8.808682in}{1.350881in}}%
\pgfpathcurveto{\pgfqpoint{8.812249in}{1.347314in}}{\pgfqpoint{8.817086in}{1.345310in}}{\pgfqpoint{8.822130in}{1.345310in}}%
\pgfpathclose%
\pgfusepath{fill}%
\end{pgfscope}%
\begin{pgfscope}%
\pgfpathrectangle{\pgfqpoint{6.572727in}{0.474100in}}{\pgfqpoint{4.227273in}{3.318700in}}%
\pgfusepath{clip}%
\pgfsetbuttcap%
\pgfsetroundjoin%
\definecolor{currentfill}{rgb}{0.127568,0.566949,0.550556}%
\pgfsetfillcolor{currentfill}%
\pgfsetfillopacity{0.700000}%
\pgfsetlinewidth{0.000000pt}%
\definecolor{currentstroke}{rgb}{0.000000,0.000000,0.000000}%
\pgfsetstrokecolor{currentstroke}%
\pgfsetstrokeopacity{0.700000}%
\pgfsetdash{}{0pt}%
\pgfpathmoveto{\pgfqpoint{8.157294in}{1.335820in}}%
\pgfpathcurveto{\pgfqpoint{8.162337in}{1.335820in}}{\pgfqpoint{8.167175in}{1.337824in}}{\pgfqpoint{8.170742in}{1.341391in}}%
\pgfpathcurveto{\pgfqpoint{8.174308in}{1.344957in}}{\pgfqpoint{8.176312in}{1.349795in}}{\pgfqpoint{8.176312in}{1.354838in}}%
\pgfpathcurveto{\pgfqpoint{8.176312in}{1.359882in}}{\pgfqpoint{8.174308in}{1.364720in}}{\pgfqpoint{8.170742in}{1.368286in}}%
\pgfpathcurveto{\pgfqpoint{8.167175in}{1.371853in}}{\pgfqpoint{8.162337in}{1.373857in}}{\pgfqpoint{8.157294in}{1.373857in}}%
\pgfpathcurveto{\pgfqpoint{8.152250in}{1.373857in}}{\pgfqpoint{8.147412in}{1.371853in}}{\pgfqpoint{8.143846in}{1.368286in}}%
\pgfpathcurveto{\pgfqpoint{8.140280in}{1.364720in}}{\pgfqpoint{8.138276in}{1.359882in}}{\pgfqpoint{8.138276in}{1.354838in}}%
\pgfpathcurveto{\pgfqpoint{8.138276in}{1.349795in}}{\pgfqpoint{8.140280in}{1.344957in}}{\pgfqpoint{8.143846in}{1.341391in}}%
\pgfpathcurveto{\pgfqpoint{8.147412in}{1.337824in}}{\pgfqpoint{8.152250in}{1.335820in}}{\pgfqpoint{8.157294in}{1.335820in}}%
\pgfpathclose%
\pgfusepath{fill}%
\end{pgfscope}%
\begin{pgfscope}%
\pgfpathrectangle{\pgfqpoint{6.572727in}{0.474100in}}{\pgfqpoint{4.227273in}{3.318700in}}%
\pgfusepath{clip}%
\pgfsetbuttcap%
\pgfsetroundjoin%
\definecolor{currentfill}{rgb}{0.993248,0.906157,0.143936}%
\pgfsetfillcolor{currentfill}%
\pgfsetfillopacity{0.700000}%
\pgfsetlinewidth{0.000000pt}%
\definecolor{currentstroke}{rgb}{0.000000,0.000000,0.000000}%
\pgfsetstrokecolor{currentstroke}%
\pgfsetstrokeopacity{0.700000}%
\pgfsetdash{}{0pt}%
\pgfpathmoveto{\pgfqpoint{9.732150in}{1.571216in}}%
\pgfpathcurveto{\pgfqpoint{9.737193in}{1.571216in}}{\pgfqpoint{9.742031in}{1.573220in}}{\pgfqpoint{9.745597in}{1.576786in}}%
\pgfpathcurveto{\pgfqpoint{9.749164in}{1.580352in}}{\pgfqpoint{9.751168in}{1.585190in}}{\pgfqpoint{9.751168in}{1.590234in}}%
\pgfpathcurveto{\pgfqpoint{9.751168in}{1.595277in}}{\pgfqpoint{9.749164in}{1.600115in}}{\pgfqpoint{9.745597in}{1.603682in}}%
\pgfpathcurveto{\pgfqpoint{9.742031in}{1.607248in}}{\pgfqpoint{9.737193in}{1.609252in}}{\pgfqpoint{9.732150in}{1.609252in}}%
\pgfpathcurveto{\pgfqpoint{9.727106in}{1.609252in}}{\pgfqpoint{9.722268in}{1.607248in}}{\pgfqpoint{9.718702in}{1.603682in}}%
\pgfpathcurveto{\pgfqpoint{9.715135in}{1.600115in}}{\pgfqpoint{9.713131in}{1.595277in}}{\pgfqpoint{9.713131in}{1.590234in}}%
\pgfpathcurveto{\pgfqpoint{9.713131in}{1.585190in}}{\pgfqpoint{9.715135in}{1.580352in}}{\pgfqpoint{9.718702in}{1.576786in}}%
\pgfpathcurveto{\pgfqpoint{9.722268in}{1.573220in}}{\pgfqpoint{9.727106in}{1.571216in}}{\pgfqpoint{9.732150in}{1.571216in}}%
\pgfpathclose%
\pgfusepath{fill}%
\end{pgfscope}%
\begin{pgfscope}%
\pgfpathrectangle{\pgfqpoint{6.572727in}{0.474100in}}{\pgfqpoint{4.227273in}{3.318700in}}%
\pgfusepath{clip}%
\pgfsetbuttcap%
\pgfsetroundjoin%
\definecolor{currentfill}{rgb}{0.993248,0.906157,0.143936}%
\pgfsetfillcolor{currentfill}%
\pgfsetfillopacity{0.700000}%
\pgfsetlinewidth{0.000000pt}%
\definecolor{currentstroke}{rgb}{0.000000,0.000000,0.000000}%
\pgfsetstrokecolor{currentstroke}%
\pgfsetstrokeopacity{0.700000}%
\pgfsetdash{}{0pt}%
\pgfpathmoveto{\pgfqpoint{10.044806in}{1.251015in}}%
\pgfpathcurveto{\pgfqpoint{10.049850in}{1.251015in}}{\pgfqpoint{10.054688in}{1.253019in}}{\pgfqpoint{10.058254in}{1.256585in}}%
\pgfpathcurveto{\pgfqpoint{10.061821in}{1.260151in}}{\pgfqpoint{10.063824in}{1.264989in}}{\pgfqpoint{10.063824in}{1.270033in}}%
\pgfpathcurveto{\pgfqpoint{10.063824in}{1.275076in}}{\pgfqpoint{10.061821in}{1.279914in}}{\pgfqpoint{10.058254in}{1.283481in}}%
\pgfpathcurveto{\pgfqpoint{10.054688in}{1.287047in}}{\pgfqpoint{10.049850in}{1.289051in}}{\pgfqpoint{10.044806in}{1.289051in}}%
\pgfpathcurveto{\pgfqpoint{10.039763in}{1.289051in}}{\pgfqpoint{10.034925in}{1.287047in}}{\pgfqpoint{10.031358in}{1.283481in}}%
\pgfpathcurveto{\pgfqpoint{10.027792in}{1.279914in}}{\pgfqpoint{10.025788in}{1.275076in}}{\pgfqpoint{10.025788in}{1.270033in}}%
\pgfpathcurveto{\pgfqpoint{10.025788in}{1.264989in}}{\pgfqpoint{10.027792in}{1.260151in}}{\pgfqpoint{10.031358in}{1.256585in}}%
\pgfpathcurveto{\pgfqpoint{10.034925in}{1.253019in}}{\pgfqpoint{10.039763in}{1.251015in}}{\pgfqpoint{10.044806in}{1.251015in}}%
\pgfpathclose%
\pgfusepath{fill}%
\end{pgfscope}%
\begin{pgfscope}%
\pgfpathrectangle{\pgfqpoint{6.572727in}{0.474100in}}{\pgfqpoint{4.227273in}{3.318700in}}%
\pgfusepath{clip}%
\pgfsetbuttcap%
\pgfsetroundjoin%
\definecolor{currentfill}{rgb}{0.127568,0.566949,0.550556}%
\pgfsetfillcolor{currentfill}%
\pgfsetfillopacity{0.700000}%
\pgfsetlinewidth{0.000000pt}%
\definecolor{currentstroke}{rgb}{0.000000,0.000000,0.000000}%
\pgfsetstrokecolor{currentstroke}%
\pgfsetstrokeopacity{0.700000}%
\pgfsetdash{}{0pt}%
\pgfpathmoveto{\pgfqpoint{8.453272in}{2.739284in}}%
\pgfpathcurveto{\pgfqpoint{8.458315in}{2.739284in}}{\pgfqpoint{8.463153in}{2.741288in}}{\pgfqpoint{8.466720in}{2.744855in}}%
\pgfpathcurveto{\pgfqpoint{8.470286in}{2.748421in}}{\pgfqpoint{8.472290in}{2.753259in}}{\pgfqpoint{8.472290in}{2.758303in}}%
\pgfpathcurveto{\pgfqpoint{8.472290in}{2.763346in}}{\pgfqpoint{8.470286in}{2.768184in}}{\pgfqpoint{8.466720in}{2.771750in}}%
\pgfpathcurveto{\pgfqpoint{8.463153in}{2.775317in}}{\pgfqpoint{8.458315in}{2.777321in}}{\pgfqpoint{8.453272in}{2.777321in}}%
\pgfpathcurveto{\pgfqpoint{8.448228in}{2.777321in}}{\pgfqpoint{8.443390in}{2.775317in}}{\pgfqpoint{8.439824in}{2.771750in}}%
\pgfpathcurveto{\pgfqpoint{8.436257in}{2.768184in}}{\pgfqpoint{8.434254in}{2.763346in}}{\pgfqpoint{8.434254in}{2.758303in}}%
\pgfpathcurveto{\pgfqpoint{8.434254in}{2.753259in}}{\pgfqpoint{8.436257in}{2.748421in}}{\pgfqpoint{8.439824in}{2.744855in}}%
\pgfpathcurveto{\pgfqpoint{8.443390in}{2.741288in}}{\pgfqpoint{8.448228in}{2.739284in}}{\pgfqpoint{8.453272in}{2.739284in}}%
\pgfpathclose%
\pgfusepath{fill}%
\end{pgfscope}%
\begin{pgfscope}%
\pgfpathrectangle{\pgfqpoint{6.572727in}{0.474100in}}{\pgfqpoint{4.227273in}{3.318700in}}%
\pgfusepath{clip}%
\pgfsetbuttcap%
\pgfsetroundjoin%
\definecolor{currentfill}{rgb}{0.127568,0.566949,0.550556}%
\pgfsetfillcolor{currentfill}%
\pgfsetfillopacity{0.700000}%
\pgfsetlinewidth{0.000000pt}%
\definecolor{currentstroke}{rgb}{0.000000,0.000000,0.000000}%
\pgfsetstrokecolor{currentstroke}%
\pgfsetstrokeopacity{0.700000}%
\pgfsetdash{}{0pt}%
\pgfpathmoveto{\pgfqpoint{8.931081in}{2.486241in}}%
\pgfpathcurveto{\pgfqpoint{8.936125in}{2.486241in}}{\pgfqpoint{8.940962in}{2.488245in}}{\pgfqpoint{8.944529in}{2.491811in}}%
\pgfpathcurveto{\pgfqpoint{8.948095in}{2.495377in}}{\pgfqpoint{8.950099in}{2.500215in}}{\pgfqpoint{8.950099in}{2.505259in}}%
\pgfpathcurveto{\pgfqpoint{8.950099in}{2.510302in}}{\pgfqpoint{8.948095in}{2.515140in}}{\pgfqpoint{8.944529in}{2.518707in}}%
\pgfpathcurveto{\pgfqpoint{8.940962in}{2.522273in}}{\pgfqpoint{8.936125in}{2.524277in}}{\pgfqpoint{8.931081in}{2.524277in}}%
\pgfpathcurveto{\pgfqpoint{8.926037in}{2.524277in}}{\pgfqpoint{8.921200in}{2.522273in}}{\pgfqpoint{8.917633in}{2.518707in}}%
\pgfpathcurveto{\pgfqpoint{8.914067in}{2.515140in}}{\pgfqpoint{8.912063in}{2.510302in}}{\pgfqpoint{8.912063in}{2.505259in}}%
\pgfpathcurveto{\pgfqpoint{8.912063in}{2.500215in}}{\pgfqpoint{8.914067in}{2.495377in}}{\pgfqpoint{8.917633in}{2.491811in}}%
\pgfpathcurveto{\pgfqpoint{8.921200in}{2.488245in}}{\pgfqpoint{8.926037in}{2.486241in}}{\pgfqpoint{8.931081in}{2.486241in}}%
\pgfpathclose%
\pgfusepath{fill}%
\end{pgfscope}%
\begin{pgfscope}%
\pgfpathrectangle{\pgfqpoint{6.572727in}{0.474100in}}{\pgfqpoint{4.227273in}{3.318700in}}%
\pgfusepath{clip}%
\pgfsetbuttcap%
\pgfsetroundjoin%
\definecolor{currentfill}{rgb}{0.127568,0.566949,0.550556}%
\pgfsetfillcolor{currentfill}%
\pgfsetfillopacity{0.700000}%
\pgfsetlinewidth{0.000000pt}%
\definecolor{currentstroke}{rgb}{0.000000,0.000000,0.000000}%
\pgfsetstrokecolor{currentstroke}%
\pgfsetstrokeopacity{0.700000}%
\pgfsetdash{}{0pt}%
\pgfpathmoveto{\pgfqpoint{8.350032in}{3.018121in}}%
\pgfpathcurveto{\pgfqpoint{8.355076in}{3.018121in}}{\pgfqpoint{8.359914in}{3.020125in}}{\pgfqpoint{8.363480in}{3.023691in}}%
\pgfpathcurveto{\pgfqpoint{8.367047in}{3.027257in}}{\pgfqpoint{8.369050in}{3.032095in}}{\pgfqpoint{8.369050in}{3.037139in}}%
\pgfpathcurveto{\pgfqpoint{8.369050in}{3.042183in}}{\pgfqpoint{8.367047in}{3.047020in}}{\pgfqpoint{8.363480in}{3.050587in}}%
\pgfpathcurveto{\pgfqpoint{8.359914in}{3.054153in}}{\pgfqpoint{8.355076in}{3.056157in}}{\pgfqpoint{8.350032in}{3.056157in}}%
\pgfpathcurveto{\pgfqpoint{8.344989in}{3.056157in}}{\pgfqpoint{8.340151in}{3.054153in}}{\pgfqpoint{8.336584in}{3.050587in}}%
\pgfpathcurveto{\pgfqpoint{8.333018in}{3.047020in}}{\pgfqpoint{8.331014in}{3.042183in}}{\pgfqpoint{8.331014in}{3.037139in}}%
\pgfpathcurveto{\pgfqpoint{8.331014in}{3.032095in}}{\pgfqpoint{8.333018in}{3.027257in}}{\pgfqpoint{8.336584in}{3.023691in}}%
\pgfpathcurveto{\pgfqpoint{8.340151in}{3.020125in}}{\pgfqpoint{8.344989in}{3.018121in}}{\pgfqpoint{8.350032in}{3.018121in}}%
\pgfpathclose%
\pgfusepath{fill}%
\end{pgfscope}%
\begin{pgfscope}%
\pgfpathrectangle{\pgfqpoint{6.572727in}{0.474100in}}{\pgfqpoint{4.227273in}{3.318700in}}%
\pgfusepath{clip}%
\pgfsetbuttcap%
\pgfsetroundjoin%
\definecolor{currentfill}{rgb}{0.127568,0.566949,0.550556}%
\pgfsetfillcolor{currentfill}%
\pgfsetfillopacity{0.700000}%
\pgfsetlinewidth{0.000000pt}%
\definecolor{currentstroke}{rgb}{0.000000,0.000000,0.000000}%
\pgfsetstrokecolor{currentstroke}%
\pgfsetstrokeopacity{0.700000}%
\pgfsetdash{}{0pt}%
\pgfpathmoveto{\pgfqpoint{8.659534in}{2.980281in}}%
\pgfpathcurveto{\pgfqpoint{8.664578in}{2.980281in}}{\pgfqpoint{8.669415in}{2.982285in}}{\pgfqpoint{8.672982in}{2.985851in}}%
\pgfpathcurveto{\pgfqpoint{8.676548in}{2.989417in}}{\pgfqpoint{8.678552in}{2.994255in}}{\pgfqpoint{8.678552in}{2.999299in}}%
\pgfpathcurveto{\pgfqpoint{8.678552in}{3.004342in}}{\pgfqpoint{8.676548in}{3.009180in}}{\pgfqpoint{8.672982in}{3.012747in}}%
\pgfpathcurveto{\pgfqpoint{8.669415in}{3.016313in}}{\pgfqpoint{8.664578in}{3.018317in}}{\pgfqpoint{8.659534in}{3.018317in}}%
\pgfpathcurveto{\pgfqpoint{8.654490in}{3.018317in}}{\pgfqpoint{8.649653in}{3.016313in}}{\pgfqpoint{8.646086in}{3.012747in}}%
\pgfpathcurveto{\pgfqpoint{8.642520in}{3.009180in}}{\pgfqpoint{8.640516in}{3.004342in}}{\pgfqpoint{8.640516in}{2.999299in}}%
\pgfpathcurveto{\pgfqpoint{8.640516in}{2.994255in}}{\pgfqpoint{8.642520in}{2.989417in}}{\pgfqpoint{8.646086in}{2.985851in}}%
\pgfpathcurveto{\pgfqpoint{8.649653in}{2.982285in}}{\pgfqpoint{8.654490in}{2.980281in}}{\pgfqpoint{8.659534in}{2.980281in}}%
\pgfpathclose%
\pgfusepath{fill}%
\end{pgfscope}%
\begin{pgfscope}%
\pgfpathrectangle{\pgfqpoint{6.572727in}{0.474100in}}{\pgfqpoint{4.227273in}{3.318700in}}%
\pgfusepath{clip}%
\pgfsetbuttcap%
\pgfsetroundjoin%
\definecolor{currentfill}{rgb}{0.993248,0.906157,0.143936}%
\pgfsetfillcolor{currentfill}%
\pgfsetfillopacity{0.700000}%
\pgfsetlinewidth{0.000000pt}%
\definecolor{currentstroke}{rgb}{0.000000,0.000000,0.000000}%
\pgfsetstrokecolor{currentstroke}%
\pgfsetstrokeopacity{0.700000}%
\pgfsetdash{}{0pt}%
\pgfpathmoveto{\pgfqpoint{9.611851in}{1.516827in}}%
\pgfpathcurveto{\pgfqpoint{9.616894in}{1.516827in}}{\pgfqpoint{9.621732in}{1.518831in}}{\pgfqpoint{9.625299in}{1.522397in}}%
\pgfpathcurveto{\pgfqpoint{9.628865in}{1.525964in}}{\pgfqpoint{9.630869in}{1.530801in}}{\pgfqpoint{9.630869in}{1.535845in}}%
\pgfpathcurveto{\pgfqpoint{9.630869in}{1.540889in}}{\pgfqpoint{9.628865in}{1.545726in}}{\pgfqpoint{9.625299in}{1.549293in}}%
\pgfpathcurveto{\pgfqpoint{9.621732in}{1.552859in}}{\pgfqpoint{9.616894in}{1.554863in}}{\pgfqpoint{9.611851in}{1.554863in}}%
\pgfpathcurveto{\pgfqpoint{9.606807in}{1.554863in}}{\pgfqpoint{9.601969in}{1.552859in}}{\pgfqpoint{9.598403in}{1.549293in}}%
\pgfpathcurveto{\pgfqpoint{9.594836in}{1.545726in}}{\pgfqpoint{9.592833in}{1.540889in}}{\pgfqpoint{9.592833in}{1.535845in}}%
\pgfpathcurveto{\pgfqpoint{9.592833in}{1.530801in}}{\pgfqpoint{9.594836in}{1.525964in}}{\pgfqpoint{9.598403in}{1.522397in}}%
\pgfpathcurveto{\pgfqpoint{9.601969in}{1.518831in}}{\pgfqpoint{9.606807in}{1.516827in}}{\pgfqpoint{9.611851in}{1.516827in}}%
\pgfpathclose%
\pgfusepath{fill}%
\end{pgfscope}%
\begin{pgfscope}%
\pgfpathrectangle{\pgfqpoint{6.572727in}{0.474100in}}{\pgfqpoint{4.227273in}{3.318700in}}%
\pgfusepath{clip}%
\pgfsetbuttcap%
\pgfsetroundjoin%
\definecolor{currentfill}{rgb}{0.993248,0.906157,0.143936}%
\pgfsetfillcolor{currentfill}%
\pgfsetfillopacity{0.700000}%
\pgfsetlinewidth{0.000000pt}%
\definecolor{currentstroke}{rgb}{0.000000,0.000000,0.000000}%
\pgfsetstrokecolor{currentstroke}%
\pgfsetstrokeopacity{0.700000}%
\pgfsetdash{}{0pt}%
\pgfpathmoveto{\pgfqpoint{9.500164in}{1.549704in}}%
\pgfpathcurveto{\pgfqpoint{9.505207in}{1.549704in}}{\pgfqpoint{9.510045in}{1.551707in}}{\pgfqpoint{9.513612in}{1.555274in}}%
\pgfpathcurveto{\pgfqpoint{9.517178in}{1.558840in}}{\pgfqpoint{9.519182in}{1.563678in}}{\pgfqpoint{9.519182in}{1.568722in}}%
\pgfpathcurveto{\pgfqpoint{9.519182in}{1.573765in}}{\pgfqpoint{9.517178in}{1.578603in}}{\pgfqpoint{9.513612in}{1.582170in}}%
\pgfpathcurveto{\pgfqpoint{9.510045in}{1.585736in}}{\pgfqpoint{9.505207in}{1.587740in}}{\pgfqpoint{9.500164in}{1.587740in}}%
\pgfpathcurveto{\pgfqpoint{9.495120in}{1.587740in}}{\pgfqpoint{9.490282in}{1.585736in}}{\pgfqpoint{9.486716in}{1.582170in}}%
\pgfpathcurveto{\pgfqpoint{9.483150in}{1.578603in}}{\pgfqpoint{9.481146in}{1.573765in}}{\pgfqpoint{9.481146in}{1.568722in}}%
\pgfpathcurveto{\pgfqpoint{9.481146in}{1.563678in}}{\pgfqpoint{9.483150in}{1.558840in}}{\pgfqpoint{9.486716in}{1.555274in}}%
\pgfpathcurveto{\pgfqpoint{9.490282in}{1.551707in}}{\pgfqpoint{9.495120in}{1.549704in}}{\pgfqpoint{9.500164in}{1.549704in}}%
\pgfpathclose%
\pgfusepath{fill}%
\end{pgfscope}%
\begin{pgfscope}%
\pgfpathrectangle{\pgfqpoint{6.572727in}{0.474100in}}{\pgfqpoint{4.227273in}{3.318700in}}%
\pgfusepath{clip}%
\pgfsetbuttcap%
\pgfsetroundjoin%
\definecolor{currentfill}{rgb}{0.993248,0.906157,0.143936}%
\pgfsetfillcolor{currentfill}%
\pgfsetfillopacity{0.700000}%
\pgfsetlinewidth{0.000000pt}%
\definecolor{currentstroke}{rgb}{0.000000,0.000000,0.000000}%
\pgfsetstrokecolor{currentstroke}%
\pgfsetstrokeopacity{0.700000}%
\pgfsetdash{}{0pt}%
\pgfpathmoveto{\pgfqpoint{10.086043in}{1.522788in}}%
\pgfpathcurveto{\pgfqpoint{10.091087in}{1.522788in}}{\pgfqpoint{10.095925in}{1.524792in}}{\pgfqpoint{10.099491in}{1.528358in}}%
\pgfpathcurveto{\pgfqpoint{10.103057in}{1.531925in}}{\pgfqpoint{10.105061in}{1.536763in}}{\pgfqpoint{10.105061in}{1.541806in}}%
\pgfpathcurveto{\pgfqpoint{10.105061in}{1.546850in}}{\pgfqpoint{10.103057in}{1.551688in}}{\pgfqpoint{10.099491in}{1.555254in}}%
\pgfpathcurveto{\pgfqpoint{10.095925in}{1.558821in}}{\pgfqpoint{10.091087in}{1.560824in}}{\pgfqpoint{10.086043in}{1.560824in}}%
\pgfpathcurveto{\pgfqpoint{10.080999in}{1.560824in}}{\pgfqpoint{10.076162in}{1.558821in}}{\pgfqpoint{10.072595in}{1.555254in}}%
\pgfpathcurveto{\pgfqpoint{10.069029in}{1.551688in}}{\pgfqpoint{10.067025in}{1.546850in}}{\pgfqpoint{10.067025in}{1.541806in}}%
\pgfpathcurveto{\pgfqpoint{10.067025in}{1.536763in}}{\pgfqpoint{10.069029in}{1.531925in}}{\pgfqpoint{10.072595in}{1.528358in}}%
\pgfpathcurveto{\pgfqpoint{10.076162in}{1.524792in}}{\pgfqpoint{10.080999in}{1.522788in}}{\pgfqpoint{10.086043in}{1.522788in}}%
\pgfpathclose%
\pgfusepath{fill}%
\end{pgfscope}%
\begin{pgfscope}%
\pgfpathrectangle{\pgfqpoint{6.572727in}{0.474100in}}{\pgfqpoint{4.227273in}{3.318700in}}%
\pgfusepath{clip}%
\pgfsetbuttcap%
\pgfsetroundjoin%
\definecolor{currentfill}{rgb}{0.993248,0.906157,0.143936}%
\pgfsetfillcolor{currentfill}%
\pgfsetfillopacity{0.700000}%
\pgfsetlinewidth{0.000000pt}%
\definecolor{currentstroke}{rgb}{0.000000,0.000000,0.000000}%
\pgfsetstrokecolor{currentstroke}%
\pgfsetstrokeopacity{0.700000}%
\pgfsetdash{}{0pt}%
\pgfpathmoveto{\pgfqpoint{9.325643in}{1.186204in}}%
\pgfpathcurveto{\pgfqpoint{9.330686in}{1.186204in}}{\pgfqpoint{9.335524in}{1.188207in}}{\pgfqpoint{9.339090in}{1.191774in}}%
\pgfpathcurveto{\pgfqpoint{9.342657in}{1.195340in}}{\pgfqpoint{9.344661in}{1.200178in}}{\pgfqpoint{9.344661in}{1.205222in}}%
\pgfpathcurveto{\pgfqpoint{9.344661in}{1.210265in}}{\pgfqpoint{9.342657in}{1.215103in}}{\pgfqpoint{9.339090in}{1.218670in}}%
\pgfpathcurveto{\pgfqpoint{9.335524in}{1.222236in}}{\pgfqpoint{9.330686in}{1.224240in}}{\pgfqpoint{9.325643in}{1.224240in}}%
\pgfpathcurveto{\pgfqpoint{9.320599in}{1.224240in}}{\pgfqpoint{9.315761in}{1.222236in}}{\pgfqpoint{9.312195in}{1.218670in}}%
\pgfpathcurveto{\pgfqpoint{9.308628in}{1.215103in}}{\pgfqpoint{9.306624in}{1.210265in}}{\pgfqpoint{9.306624in}{1.205222in}}%
\pgfpathcurveto{\pgfqpoint{9.306624in}{1.200178in}}{\pgfqpoint{9.308628in}{1.195340in}}{\pgfqpoint{9.312195in}{1.191774in}}%
\pgfpathcurveto{\pgfqpoint{9.315761in}{1.188207in}}{\pgfqpoint{9.320599in}{1.186204in}}{\pgfqpoint{9.325643in}{1.186204in}}%
\pgfpathclose%
\pgfusepath{fill}%
\end{pgfscope}%
\begin{pgfscope}%
\pgfpathrectangle{\pgfqpoint{6.572727in}{0.474100in}}{\pgfqpoint{4.227273in}{3.318700in}}%
\pgfusepath{clip}%
\pgfsetbuttcap%
\pgfsetroundjoin%
\definecolor{currentfill}{rgb}{0.993248,0.906157,0.143936}%
\pgfsetfillcolor{currentfill}%
\pgfsetfillopacity{0.700000}%
\pgfsetlinewidth{0.000000pt}%
\definecolor{currentstroke}{rgb}{0.000000,0.000000,0.000000}%
\pgfsetstrokecolor{currentstroke}%
\pgfsetstrokeopacity{0.700000}%
\pgfsetdash{}{0pt}%
\pgfpathmoveto{\pgfqpoint{9.240433in}{1.450621in}}%
\pgfpathcurveto{\pgfqpoint{9.245477in}{1.450621in}}{\pgfqpoint{9.250314in}{1.452625in}}{\pgfqpoint{9.253881in}{1.456191in}}%
\pgfpathcurveto{\pgfqpoint{9.257447in}{1.459758in}}{\pgfqpoint{9.259451in}{1.464596in}}{\pgfqpoint{9.259451in}{1.469639in}}%
\pgfpathcurveto{\pgfqpoint{9.259451in}{1.474683in}}{\pgfqpoint{9.257447in}{1.479521in}}{\pgfqpoint{9.253881in}{1.483087in}}%
\pgfpathcurveto{\pgfqpoint{9.250314in}{1.486654in}}{\pgfqpoint{9.245477in}{1.488657in}}{\pgfqpoint{9.240433in}{1.488657in}}%
\pgfpathcurveto{\pgfqpoint{9.235389in}{1.488657in}}{\pgfqpoint{9.230552in}{1.486654in}}{\pgfqpoint{9.226985in}{1.483087in}}%
\pgfpathcurveto{\pgfqpoint{9.223419in}{1.479521in}}{\pgfqpoint{9.221415in}{1.474683in}}{\pgfqpoint{9.221415in}{1.469639in}}%
\pgfpathcurveto{\pgfqpoint{9.221415in}{1.464596in}}{\pgfqpoint{9.223419in}{1.459758in}}{\pgfqpoint{9.226985in}{1.456191in}}%
\pgfpathcurveto{\pgfqpoint{9.230552in}{1.452625in}}{\pgfqpoint{9.235389in}{1.450621in}}{\pgfqpoint{9.240433in}{1.450621in}}%
\pgfpathclose%
\pgfusepath{fill}%
\end{pgfscope}%
\begin{pgfscope}%
\pgfpathrectangle{\pgfqpoint{6.572727in}{0.474100in}}{\pgfqpoint{4.227273in}{3.318700in}}%
\pgfusepath{clip}%
\pgfsetbuttcap%
\pgfsetroundjoin%
\definecolor{currentfill}{rgb}{0.993248,0.906157,0.143936}%
\pgfsetfillcolor{currentfill}%
\pgfsetfillopacity{0.700000}%
\pgfsetlinewidth{0.000000pt}%
\definecolor{currentstroke}{rgb}{0.000000,0.000000,0.000000}%
\pgfsetstrokecolor{currentstroke}%
\pgfsetstrokeopacity{0.700000}%
\pgfsetdash{}{0pt}%
\pgfpathmoveto{\pgfqpoint{9.772256in}{1.959423in}}%
\pgfpathcurveto{\pgfqpoint{9.777300in}{1.959423in}}{\pgfqpoint{9.782138in}{1.961427in}}{\pgfqpoint{9.785704in}{1.964994in}}%
\pgfpathcurveto{\pgfqpoint{9.789270in}{1.968560in}}{\pgfqpoint{9.791274in}{1.973398in}}{\pgfqpoint{9.791274in}{1.978441in}}%
\pgfpathcurveto{\pgfqpoint{9.791274in}{1.983485in}}{\pgfqpoint{9.789270in}{1.988323in}}{\pgfqpoint{9.785704in}{1.991889in}}%
\pgfpathcurveto{\pgfqpoint{9.782138in}{1.995456in}}{\pgfqpoint{9.777300in}{1.997460in}}{\pgfqpoint{9.772256in}{1.997460in}}%
\pgfpathcurveto{\pgfqpoint{9.767213in}{1.997460in}}{\pgfqpoint{9.762375in}{1.995456in}}{\pgfqpoint{9.758808in}{1.991889in}}%
\pgfpathcurveto{\pgfqpoint{9.755242in}{1.988323in}}{\pgfqpoint{9.753238in}{1.983485in}}{\pgfqpoint{9.753238in}{1.978441in}}%
\pgfpathcurveto{\pgfqpoint{9.753238in}{1.973398in}}{\pgfqpoint{9.755242in}{1.968560in}}{\pgfqpoint{9.758808in}{1.964994in}}%
\pgfpathcurveto{\pgfqpoint{9.762375in}{1.961427in}}{\pgfqpoint{9.767213in}{1.959423in}}{\pgfqpoint{9.772256in}{1.959423in}}%
\pgfpathclose%
\pgfusepath{fill}%
\end{pgfscope}%
\begin{pgfscope}%
\pgfpathrectangle{\pgfqpoint{6.572727in}{0.474100in}}{\pgfqpoint{4.227273in}{3.318700in}}%
\pgfusepath{clip}%
\pgfsetbuttcap%
\pgfsetroundjoin%
\definecolor{currentfill}{rgb}{0.127568,0.566949,0.550556}%
\pgfsetfillcolor{currentfill}%
\pgfsetfillopacity{0.700000}%
\pgfsetlinewidth{0.000000pt}%
\definecolor{currentstroke}{rgb}{0.000000,0.000000,0.000000}%
\pgfsetstrokecolor{currentstroke}%
\pgfsetstrokeopacity{0.700000}%
\pgfsetdash{}{0pt}%
\pgfpathmoveto{\pgfqpoint{8.436285in}{2.620874in}}%
\pgfpathcurveto{\pgfqpoint{8.441329in}{2.620874in}}{\pgfqpoint{8.446167in}{2.622878in}}{\pgfqpoint{8.449733in}{2.626444in}}%
\pgfpathcurveto{\pgfqpoint{8.453300in}{2.630010in}}{\pgfqpoint{8.455303in}{2.634848in}}{\pgfqpoint{8.455303in}{2.639892in}}%
\pgfpathcurveto{\pgfqpoint{8.455303in}{2.644936in}}{\pgfqpoint{8.453300in}{2.649773in}}{\pgfqpoint{8.449733in}{2.653340in}}%
\pgfpathcurveto{\pgfqpoint{8.446167in}{2.656906in}}{\pgfqpoint{8.441329in}{2.658910in}}{\pgfqpoint{8.436285in}{2.658910in}}%
\pgfpathcurveto{\pgfqpoint{8.431242in}{2.658910in}}{\pgfqpoint{8.426404in}{2.656906in}}{\pgfqpoint{8.422837in}{2.653340in}}%
\pgfpathcurveto{\pgfqpoint{8.419271in}{2.649773in}}{\pgfqpoint{8.417267in}{2.644936in}}{\pgfqpoint{8.417267in}{2.639892in}}%
\pgfpathcurveto{\pgfqpoint{8.417267in}{2.634848in}}{\pgfqpoint{8.419271in}{2.630010in}}{\pgfqpoint{8.422837in}{2.626444in}}%
\pgfpathcurveto{\pgfqpoint{8.426404in}{2.622878in}}{\pgfqpoint{8.431242in}{2.620874in}}{\pgfqpoint{8.436285in}{2.620874in}}%
\pgfpathclose%
\pgfusepath{fill}%
\end{pgfscope}%
\begin{pgfscope}%
\pgfpathrectangle{\pgfqpoint{6.572727in}{0.474100in}}{\pgfqpoint{4.227273in}{3.318700in}}%
\pgfusepath{clip}%
\pgfsetbuttcap%
\pgfsetroundjoin%
\definecolor{currentfill}{rgb}{0.993248,0.906157,0.143936}%
\pgfsetfillcolor{currentfill}%
\pgfsetfillopacity{0.700000}%
\pgfsetlinewidth{0.000000pt}%
\definecolor{currentstroke}{rgb}{0.000000,0.000000,0.000000}%
\pgfsetstrokecolor{currentstroke}%
\pgfsetstrokeopacity{0.700000}%
\pgfsetdash{}{0pt}%
\pgfpathmoveto{\pgfqpoint{9.664250in}{1.776857in}}%
\pgfpathcurveto{\pgfqpoint{9.669293in}{1.776857in}}{\pgfqpoint{9.674131in}{1.778861in}}{\pgfqpoint{9.677698in}{1.782428in}}%
\pgfpathcurveto{\pgfqpoint{9.681264in}{1.785994in}}{\pgfqpoint{9.683268in}{1.790832in}}{\pgfqpoint{9.683268in}{1.795876in}}%
\pgfpathcurveto{\pgfqpoint{9.683268in}{1.800919in}}{\pgfqpoint{9.681264in}{1.805757in}}{\pgfqpoint{9.677698in}{1.809323in}}%
\pgfpathcurveto{\pgfqpoint{9.674131in}{1.812890in}}{\pgfqpoint{9.669293in}{1.814894in}}{\pgfqpoint{9.664250in}{1.814894in}}%
\pgfpathcurveto{\pgfqpoint{9.659206in}{1.814894in}}{\pgfqpoint{9.654368in}{1.812890in}}{\pgfqpoint{9.650802in}{1.809323in}}%
\pgfpathcurveto{\pgfqpoint{9.647235in}{1.805757in}}{\pgfqpoint{9.645232in}{1.800919in}}{\pgfqpoint{9.645232in}{1.795876in}}%
\pgfpathcurveto{\pgfqpoint{9.645232in}{1.790832in}}{\pgfqpoint{9.647235in}{1.785994in}}{\pgfqpoint{9.650802in}{1.782428in}}%
\pgfpathcurveto{\pgfqpoint{9.654368in}{1.778861in}}{\pgfqpoint{9.659206in}{1.776857in}}{\pgfqpoint{9.664250in}{1.776857in}}%
\pgfpathclose%
\pgfusepath{fill}%
\end{pgfscope}%
\begin{pgfscope}%
\pgfpathrectangle{\pgfqpoint{6.572727in}{0.474100in}}{\pgfqpoint{4.227273in}{3.318700in}}%
\pgfusepath{clip}%
\pgfsetbuttcap%
\pgfsetroundjoin%
\definecolor{currentfill}{rgb}{0.127568,0.566949,0.550556}%
\pgfsetfillcolor{currentfill}%
\pgfsetfillopacity{0.700000}%
\pgfsetlinewidth{0.000000pt}%
\definecolor{currentstroke}{rgb}{0.000000,0.000000,0.000000}%
\pgfsetstrokecolor{currentstroke}%
\pgfsetstrokeopacity{0.700000}%
\pgfsetdash{}{0pt}%
\pgfpathmoveto{\pgfqpoint{7.774562in}{2.617640in}}%
\pgfpathcurveto{\pgfqpoint{7.779606in}{2.617640in}}{\pgfqpoint{7.784444in}{2.619644in}}{\pgfqpoint{7.788010in}{2.623210in}}%
\pgfpathcurveto{\pgfqpoint{7.791577in}{2.626777in}}{\pgfqpoint{7.793580in}{2.631615in}}{\pgfqpoint{7.793580in}{2.636658in}}%
\pgfpathcurveto{\pgfqpoint{7.793580in}{2.641702in}}{\pgfqpoint{7.791577in}{2.646540in}}{\pgfqpoint{7.788010in}{2.650106in}}%
\pgfpathcurveto{\pgfqpoint{7.784444in}{2.653672in}}{\pgfqpoint{7.779606in}{2.655676in}}{\pgfqpoint{7.774562in}{2.655676in}}%
\pgfpathcurveto{\pgfqpoint{7.769519in}{2.655676in}}{\pgfqpoint{7.764681in}{2.653672in}}{\pgfqpoint{7.761114in}{2.650106in}}%
\pgfpathcurveto{\pgfqpoint{7.757548in}{2.646540in}}{\pgfqpoint{7.755544in}{2.641702in}}{\pgfqpoint{7.755544in}{2.636658in}}%
\pgfpathcurveto{\pgfqpoint{7.755544in}{2.631615in}}{\pgfqpoint{7.757548in}{2.626777in}}{\pgfqpoint{7.761114in}{2.623210in}}%
\pgfpathcurveto{\pgfqpoint{7.764681in}{2.619644in}}{\pgfqpoint{7.769519in}{2.617640in}}{\pgfqpoint{7.774562in}{2.617640in}}%
\pgfpathclose%
\pgfusepath{fill}%
\end{pgfscope}%
\begin{pgfscope}%
\pgfpathrectangle{\pgfqpoint{6.572727in}{0.474100in}}{\pgfqpoint{4.227273in}{3.318700in}}%
\pgfusepath{clip}%
\pgfsetbuttcap%
\pgfsetroundjoin%
\definecolor{currentfill}{rgb}{0.127568,0.566949,0.550556}%
\pgfsetfillcolor{currentfill}%
\pgfsetfillopacity{0.700000}%
\pgfsetlinewidth{0.000000pt}%
\definecolor{currentstroke}{rgb}{0.000000,0.000000,0.000000}%
\pgfsetstrokecolor{currentstroke}%
\pgfsetstrokeopacity{0.700000}%
\pgfsetdash{}{0pt}%
\pgfpathmoveto{\pgfqpoint{7.723091in}{1.511474in}}%
\pgfpathcurveto{\pgfqpoint{7.728135in}{1.511474in}}{\pgfqpoint{7.732972in}{1.513478in}}{\pgfqpoint{7.736539in}{1.517045in}}%
\pgfpathcurveto{\pgfqpoint{7.740105in}{1.520611in}}{\pgfqpoint{7.742109in}{1.525449in}}{\pgfqpoint{7.742109in}{1.530492in}}%
\pgfpathcurveto{\pgfqpoint{7.742109in}{1.535536in}}{\pgfqpoint{7.740105in}{1.540374in}}{\pgfqpoint{7.736539in}{1.543940in}}%
\pgfpathcurveto{\pgfqpoint{7.732972in}{1.547507in}}{\pgfqpoint{7.728135in}{1.549511in}}{\pgfqpoint{7.723091in}{1.549511in}}%
\pgfpathcurveto{\pgfqpoint{7.718047in}{1.549511in}}{\pgfqpoint{7.713210in}{1.547507in}}{\pgfqpoint{7.709643in}{1.543940in}}%
\pgfpathcurveto{\pgfqpoint{7.706077in}{1.540374in}}{\pgfqpoint{7.704073in}{1.535536in}}{\pgfqpoint{7.704073in}{1.530492in}}%
\pgfpathcurveto{\pgfqpoint{7.704073in}{1.525449in}}{\pgfqpoint{7.706077in}{1.520611in}}{\pgfqpoint{7.709643in}{1.517045in}}%
\pgfpathcurveto{\pgfqpoint{7.713210in}{1.513478in}}{\pgfqpoint{7.718047in}{1.511474in}}{\pgfqpoint{7.723091in}{1.511474in}}%
\pgfpathclose%
\pgfusepath{fill}%
\end{pgfscope}%
\begin{pgfscope}%
\pgfpathrectangle{\pgfqpoint{6.572727in}{0.474100in}}{\pgfqpoint{4.227273in}{3.318700in}}%
\pgfusepath{clip}%
\pgfsetbuttcap%
\pgfsetroundjoin%
\definecolor{currentfill}{rgb}{0.993248,0.906157,0.143936}%
\pgfsetfillcolor{currentfill}%
\pgfsetfillopacity{0.700000}%
\pgfsetlinewidth{0.000000pt}%
\definecolor{currentstroke}{rgb}{0.000000,0.000000,0.000000}%
\pgfsetstrokecolor{currentstroke}%
\pgfsetstrokeopacity{0.700000}%
\pgfsetdash{}{0pt}%
\pgfpathmoveto{\pgfqpoint{9.771242in}{2.205898in}}%
\pgfpathcurveto{\pgfqpoint{9.776285in}{2.205898in}}{\pgfqpoint{9.781123in}{2.207902in}}{\pgfqpoint{9.784689in}{2.211469in}}%
\pgfpathcurveto{\pgfqpoint{9.788256in}{2.215035in}}{\pgfqpoint{9.790260in}{2.219873in}}{\pgfqpoint{9.790260in}{2.224917in}}%
\pgfpathcurveto{\pgfqpoint{9.790260in}{2.229960in}}{\pgfqpoint{9.788256in}{2.234798in}}{\pgfqpoint{9.784689in}{2.238364in}}%
\pgfpathcurveto{\pgfqpoint{9.781123in}{2.241931in}}{\pgfqpoint{9.776285in}{2.243935in}}{\pgfqpoint{9.771242in}{2.243935in}}%
\pgfpathcurveto{\pgfqpoint{9.766198in}{2.243935in}}{\pgfqpoint{9.761360in}{2.241931in}}{\pgfqpoint{9.757794in}{2.238364in}}%
\pgfpathcurveto{\pgfqpoint{9.754227in}{2.234798in}}{\pgfqpoint{9.752223in}{2.229960in}}{\pgfqpoint{9.752223in}{2.224917in}}%
\pgfpathcurveto{\pgfqpoint{9.752223in}{2.219873in}}{\pgfqpoint{9.754227in}{2.215035in}}{\pgfqpoint{9.757794in}{2.211469in}}%
\pgfpathcurveto{\pgfqpoint{9.761360in}{2.207902in}}{\pgfqpoint{9.766198in}{2.205898in}}{\pgfqpoint{9.771242in}{2.205898in}}%
\pgfpathclose%
\pgfusepath{fill}%
\end{pgfscope}%
\begin{pgfscope}%
\pgfpathrectangle{\pgfqpoint{6.572727in}{0.474100in}}{\pgfqpoint{4.227273in}{3.318700in}}%
\pgfusepath{clip}%
\pgfsetbuttcap%
\pgfsetroundjoin%
\definecolor{currentfill}{rgb}{0.993248,0.906157,0.143936}%
\pgfsetfillcolor{currentfill}%
\pgfsetfillopacity{0.700000}%
\pgfsetlinewidth{0.000000pt}%
\definecolor{currentstroke}{rgb}{0.000000,0.000000,0.000000}%
\pgfsetstrokecolor{currentstroke}%
\pgfsetstrokeopacity{0.700000}%
\pgfsetdash{}{0pt}%
\pgfpathmoveto{\pgfqpoint{9.498366in}{2.238778in}}%
\pgfpathcurveto{\pgfqpoint{9.503410in}{2.238778in}}{\pgfqpoint{9.508248in}{2.240782in}}{\pgfqpoint{9.511814in}{2.244348in}}%
\pgfpathcurveto{\pgfqpoint{9.515381in}{2.247915in}}{\pgfqpoint{9.517385in}{2.252752in}}{\pgfqpoint{9.517385in}{2.257796in}}%
\pgfpathcurveto{\pgfqpoint{9.517385in}{2.262840in}}{\pgfqpoint{9.515381in}{2.267678in}}{\pgfqpoint{9.511814in}{2.271244in}}%
\pgfpathcurveto{\pgfqpoint{9.508248in}{2.274810in}}{\pgfqpoint{9.503410in}{2.276814in}}{\pgfqpoint{9.498366in}{2.276814in}}%
\pgfpathcurveto{\pgfqpoint{9.493323in}{2.276814in}}{\pgfqpoint{9.488485in}{2.274810in}}{\pgfqpoint{9.484919in}{2.271244in}}%
\pgfpathcurveto{\pgfqpoint{9.481352in}{2.267678in}}{\pgfqpoint{9.479348in}{2.262840in}}{\pgfqpoint{9.479348in}{2.257796in}}%
\pgfpathcurveto{\pgfqpoint{9.479348in}{2.252752in}}{\pgfqpoint{9.481352in}{2.247915in}}{\pgfqpoint{9.484919in}{2.244348in}}%
\pgfpathcurveto{\pgfqpoint{9.488485in}{2.240782in}}{\pgfqpoint{9.493323in}{2.238778in}}{\pgfqpoint{9.498366in}{2.238778in}}%
\pgfpathclose%
\pgfusepath{fill}%
\end{pgfscope}%
\begin{pgfscope}%
\pgfpathrectangle{\pgfqpoint{6.572727in}{0.474100in}}{\pgfqpoint{4.227273in}{3.318700in}}%
\pgfusepath{clip}%
\pgfsetbuttcap%
\pgfsetroundjoin%
\definecolor{currentfill}{rgb}{0.127568,0.566949,0.550556}%
\pgfsetfillcolor{currentfill}%
\pgfsetfillopacity{0.700000}%
\pgfsetlinewidth{0.000000pt}%
\definecolor{currentstroke}{rgb}{0.000000,0.000000,0.000000}%
\pgfsetstrokecolor{currentstroke}%
\pgfsetstrokeopacity{0.700000}%
\pgfsetdash{}{0pt}%
\pgfpathmoveto{\pgfqpoint{8.061790in}{1.745243in}}%
\pgfpathcurveto{\pgfqpoint{8.066834in}{1.745243in}}{\pgfqpoint{8.071671in}{1.747247in}}{\pgfqpoint{8.075238in}{1.750813in}}%
\pgfpathcurveto{\pgfqpoint{8.078804in}{1.754380in}}{\pgfqpoint{8.080808in}{1.759217in}}{\pgfqpoint{8.080808in}{1.764261in}}%
\pgfpathcurveto{\pgfqpoint{8.080808in}{1.769305in}}{\pgfqpoint{8.078804in}{1.774143in}}{\pgfqpoint{8.075238in}{1.777709in}}%
\pgfpathcurveto{\pgfqpoint{8.071671in}{1.781275in}}{\pgfqpoint{8.066834in}{1.783279in}}{\pgfqpoint{8.061790in}{1.783279in}}%
\pgfpathcurveto{\pgfqpoint{8.056746in}{1.783279in}}{\pgfqpoint{8.051908in}{1.781275in}}{\pgfqpoint{8.048342in}{1.777709in}}%
\pgfpathcurveto{\pgfqpoint{8.044776in}{1.774143in}}{\pgfqpoint{8.042772in}{1.769305in}}{\pgfqpoint{8.042772in}{1.764261in}}%
\pgfpathcurveto{\pgfqpoint{8.042772in}{1.759217in}}{\pgfqpoint{8.044776in}{1.754380in}}{\pgfqpoint{8.048342in}{1.750813in}}%
\pgfpathcurveto{\pgfqpoint{8.051908in}{1.747247in}}{\pgfqpoint{8.056746in}{1.745243in}}{\pgfqpoint{8.061790in}{1.745243in}}%
\pgfpathclose%
\pgfusepath{fill}%
\end{pgfscope}%
\begin{pgfscope}%
\pgfpathrectangle{\pgfqpoint{6.572727in}{0.474100in}}{\pgfqpoint{4.227273in}{3.318700in}}%
\pgfusepath{clip}%
\pgfsetbuttcap%
\pgfsetroundjoin%
\definecolor{currentfill}{rgb}{0.127568,0.566949,0.550556}%
\pgfsetfillcolor{currentfill}%
\pgfsetfillopacity{0.700000}%
\pgfsetlinewidth{0.000000pt}%
\definecolor{currentstroke}{rgb}{0.000000,0.000000,0.000000}%
\pgfsetstrokecolor{currentstroke}%
\pgfsetstrokeopacity{0.700000}%
\pgfsetdash{}{0pt}%
\pgfpathmoveto{\pgfqpoint{7.891853in}{2.547996in}}%
\pgfpathcurveto{\pgfqpoint{7.896897in}{2.547996in}}{\pgfqpoint{7.901735in}{2.550000in}}{\pgfqpoint{7.905301in}{2.553566in}}%
\pgfpathcurveto{\pgfqpoint{7.908867in}{2.557133in}}{\pgfqpoint{7.910871in}{2.561970in}}{\pgfqpoint{7.910871in}{2.567014in}}%
\pgfpathcurveto{\pgfqpoint{7.910871in}{2.572058in}}{\pgfqpoint{7.908867in}{2.576895in}}{\pgfqpoint{7.905301in}{2.580462in}}%
\pgfpathcurveto{\pgfqpoint{7.901735in}{2.584028in}}{\pgfqpoint{7.896897in}{2.586032in}}{\pgfqpoint{7.891853in}{2.586032in}}%
\pgfpathcurveto{\pgfqpoint{7.886809in}{2.586032in}}{\pgfqpoint{7.881972in}{2.584028in}}{\pgfqpoint{7.878405in}{2.580462in}}%
\pgfpathcurveto{\pgfqpoint{7.874839in}{2.576895in}}{\pgfqpoint{7.872835in}{2.572058in}}{\pgfqpoint{7.872835in}{2.567014in}}%
\pgfpathcurveto{\pgfqpoint{7.872835in}{2.561970in}}{\pgfqpoint{7.874839in}{2.557133in}}{\pgfqpoint{7.878405in}{2.553566in}}%
\pgfpathcurveto{\pgfqpoint{7.881972in}{2.550000in}}{\pgfqpoint{7.886809in}{2.547996in}}{\pgfqpoint{7.891853in}{2.547996in}}%
\pgfpathclose%
\pgfusepath{fill}%
\end{pgfscope}%
\begin{pgfscope}%
\pgfpathrectangle{\pgfqpoint{6.572727in}{0.474100in}}{\pgfqpoint{4.227273in}{3.318700in}}%
\pgfusepath{clip}%
\pgfsetbuttcap%
\pgfsetroundjoin%
\definecolor{currentfill}{rgb}{0.993248,0.906157,0.143936}%
\pgfsetfillcolor{currentfill}%
\pgfsetfillopacity{0.700000}%
\pgfsetlinewidth{0.000000pt}%
\definecolor{currentstroke}{rgb}{0.000000,0.000000,0.000000}%
\pgfsetstrokecolor{currentstroke}%
\pgfsetstrokeopacity{0.700000}%
\pgfsetdash{}{0pt}%
\pgfpathmoveto{\pgfqpoint{9.984279in}{1.527744in}}%
\pgfpathcurveto{\pgfqpoint{9.989322in}{1.527744in}}{\pgfqpoint{9.994160in}{1.529748in}}{\pgfqpoint{9.997727in}{1.533314in}}%
\pgfpathcurveto{\pgfqpoint{10.001293in}{1.536881in}}{\pgfqpoint{10.003297in}{1.541718in}}{\pgfqpoint{10.003297in}{1.546762in}}%
\pgfpathcurveto{\pgfqpoint{10.003297in}{1.551806in}}{\pgfqpoint{10.001293in}{1.556644in}}{\pgfqpoint{9.997727in}{1.560210in}}%
\pgfpathcurveto{\pgfqpoint{9.994160in}{1.563776in}}{\pgfqpoint{9.989322in}{1.565780in}}{\pgfqpoint{9.984279in}{1.565780in}}%
\pgfpathcurveto{\pgfqpoint{9.979235in}{1.565780in}}{\pgfqpoint{9.974397in}{1.563776in}}{\pgfqpoint{9.970831in}{1.560210in}}%
\pgfpathcurveto{\pgfqpoint{9.967264in}{1.556644in}}{\pgfqpoint{9.965261in}{1.551806in}}{\pgfqpoint{9.965261in}{1.546762in}}%
\pgfpathcurveto{\pgfqpoint{9.965261in}{1.541718in}}{\pgfqpoint{9.967264in}{1.536881in}}{\pgfqpoint{9.970831in}{1.533314in}}%
\pgfpathcurveto{\pgfqpoint{9.974397in}{1.529748in}}{\pgfqpoint{9.979235in}{1.527744in}}{\pgfqpoint{9.984279in}{1.527744in}}%
\pgfpathclose%
\pgfusepath{fill}%
\end{pgfscope}%
\begin{pgfscope}%
\pgfpathrectangle{\pgfqpoint{6.572727in}{0.474100in}}{\pgfqpoint{4.227273in}{3.318700in}}%
\pgfusepath{clip}%
\pgfsetbuttcap%
\pgfsetroundjoin%
\definecolor{currentfill}{rgb}{0.127568,0.566949,0.550556}%
\pgfsetfillcolor{currentfill}%
\pgfsetfillopacity{0.700000}%
\pgfsetlinewidth{0.000000pt}%
\definecolor{currentstroke}{rgb}{0.000000,0.000000,0.000000}%
\pgfsetstrokecolor{currentstroke}%
\pgfsetstrokeopacity{0.700000}%
\pgfsetdash{}{0pt}%
\pgfpathmoveto{\pgfqpoint{7.950170in}{2.125240in}}%
\pgfpathcurveto{\pgfqpoint{7.955214in}{2.125240in}}{\pgfqpoint{7.960052in}{2.127244in}}{\pgfqpoint{7.963618in}{2.130811in}}%
\pgfpathcurveto{\pgfqpoint{7.967185in}{2.134377in}}{\pgfqpoint{7.969189in}{2.139215in}}{\pgfqpoint{7.969189in}{2.144258in}}%
\pgfpathcurveto{\pgfqpoint{7.969189in}{2.149302in}}{\pgfqpoint{7.967185in}{2.154140in}}{\pgfqpoint{7.963618in}{2.157706in}}%
\pgfpathcurveto{\pgfqpoint{7.960052in}{2.161273in}}{\pgfqpoint{7.955214in}{2.163277in}}{\pgfqpoint{7.950170in}{2.163277in}}%
\pgfpathcurveto{\pgfqpoint{7.945127in}{2.163277in}}{\pgfqpoint{7.940289in}{2.161273in}}{\pgfqpoint{7.936723in}{2.157706in}}%
\pgfpathcurveto{\pgfqpoint{7.933156in}{2.154140in}}{\pgfqpoint{7.931152in}{2.149302in}}{\pgfqpoint{7.931152in}{2.144258in}}%
\pgfpathcurveto{\pgfqpoint{7.931152in}{2.139215in}}{\pgfqpoint{7.933156in}{2.134377in}}{\pgfqpoint{7.936723in}{2.130811in}}%
\pgfpathcurveto{\pgfqpoint{7.940289in}{2.127244in}}{\pgfqpoint{7.945127in}{2.125240in}}{\pgfqpoint{7.950170in}{2.125240in}}%
\pgfpathclose%
\pgfusepath{fill}%
\end{pgfscope}%
\begin{pgfscope}%
\pgfpathrectangle{\pgfqpoint{6.572727in}{0.474100in}}{\pgfqpoint{4.227273in}{3.318700in}}%
\pgfusepath{clip}%
\pgfsetbuttcap%
\pgfsetroundjoin%
\definecolor{currentfill}{rgb}{0.993248,0.906157,0.143936}%
\pgfsetfillcolor{currentfill}%
\pgfsetfillopacity{0.700000}%
\pgfsetlinewidth{0.000000pt}%
\definecolor{currentstroke}{rgb}{0.000000,0.000000,0.000000}%
\pgfsetstrokecolor{currentstroke}%
\pgfsetstrokeopacity{0.700000}%
\pgfsetdash{}{0pt}%
\pgfpathmoveto{\pgfqpoint{10.138325in}{1.669074in}}%
\pgfpathcurveto{\pgfqpoint{10.143369in}{1.669074in}}{\pgfqpoint{10.148207in}{1.671077in}}{\pgfqpoint{10.151773in}{1.674644in}}%
\pgfpathcurveto{\pgfqpoint{10.155340in}{1.678210in}}{\pgfqpoint{10.157343in}{1.683048in}}{\pgfqpoint{10.157343in}{1.688092in}}%
\pgfpathcurveto{\pgfqpoint{10.157343in}{1.693135in}}{\pgfqpoint{10.155340in}{1.697973in}}{\pgfqpoint{10.151773in}{1.701540in}}%
\pgfpathcurveto{\pgfqpoint{10.148207in}{1.705106in}}{\pgfqpoint{10.143369in}{1.707110in}}{\pgfqpoint{10.138325in}{1.707110in}}%
\pgfpathcurveto{\pgfqpoint{10.133282in}{1.707110in}}{\pgfqpoint{10.128444in}{1.705106in}}{\pgfqpoint{10.124877in}{1.701540in}}%
\pgfpathcurveto{\pgfqpoint{10.121311in}{1.697973in}}{\pgfqpoint{10.119307in}{1.693135in}}{\pgfqpoint{10.119307in}{1.688092in}}%
\pgfpathcurveto{\pgfqpoint{10.119307in}{1.683048in}}{\pgfqpoint{10.121311in}{1.678210in}}{\pgfqpoint{10.124877in}{1.674644in}}%
\pgfpathcurveto{\pgfqpoint{10.128444in}{1.671077in}}{\pgfqpoint{10.133282in}{1.669074in}}{\pgfqpoint{10.138325in}{1.669074in}}%
\pgfpathclose%
\pgfusepath{fill}%
\end{pgfscope}%
\begin{pgfscope}%
\pgfpathrectangle{\pgfqpoint{6.572727in}{0.474100in}}{\pgfqpoint{4.227273in}{3.318700in}}%
\pgfusepath{clip}%
\pgfsetbuttcap%
\pgfsetroundjoin%
\definecolor{currentfill}{rgb}{0.127568,0.566949,0.550556}%
\pgfsetfillcolor{currentfill}%
\pgfsetfillopacity{0.700000}%
\pgfsetlinewidth{0.000000pt}%
\definecolor{currentstroke}{rgb}{0.000000,0.000000,0.000000}%
\pgfsetstrokecolor{currentstroke}%
\pgfsetstrokeopacity{0.700000}%
\pgfsetdash{}{0pt}%
\pgfpathmoveto{\pgfqpoint{8.195980in}{2.714096in}}%
\pgfpathcurveto{\pgfqpoint{8.201024in}{2.714096in}}{\pgfqpoint{8.205861in}{2.716099in}}{\pgfqpoint{8.209428in}{2.719666in}}%
\pgfpathcurveto{\pgfqpoint{8.212994in}{2.723232in}}{\pgfqpoint{8.214998in}{2.728070in}}{\pgfqpoint{8.214998in}{2.733114in}}%
\pgfpathcurveto{\pgfqpoint{8.214998in}{2.738157in}}{\pgfqpoint{8.212994in}{2.742995in}}{\pgfqpoint{8.209428in}{2.746562in}}%
\pgfpathcurveto{\pgfqpoint{8.205861in}{2.750128in}}{\pgfqpoint{8.201024in}{2.752132in}}{\pgfqpoint{8.195980in}{2.752132in}}%
\pgfpathcurveto{\pgfqpoint{8.190936in}{2.752132in}}{\pgfqpoint{8.186099in}{2.750128in}}{\pgfqpoint{8.182532in}{2.746562in}}%
\pgfpathcurveto{\pgfqpoint{8.178966in}{2.742995in}}{\pgfqpoint{8.176962in}{2.738157in}}{\pgfqpoint{8.176962in}{2.733114in}}%
\pgfpathcurveto{\pgfqpoint{8.176962in}{2.728070in}}{\pgfqpoint{8.178966in}{2.723232in}}{\pgfqpoint{8.182532in}{2.719666in}}%
\pgfpathcurveto{\pgfqpoint{8.186099in}{2.716099in}}{\pgfqpoint{8.190936in}{2.714096in}}{\pgfqpoint{8.195980in}{2.714096in}}%
\pgfpathclose%
\pgfusepath{fill}%
\end{pgfscope}%
\begin{pgfscope}%
\pgfpathrectangle{\pgfqpoint{6.572727in}{0.474100in}}{\pgfqpoint{4.227273in}{3.318700in}}%
\pgfusepath{clip}%
\pgfsetbuttcap%
\pgfsetroundjoin%
\definecolor{currentfill}{rgb}{0.993248,0.906157,0.143936}%
\pgfsetfillcolor{currentfill}%
\pgfsetfillopacity{0.700000}%
\pgfsetlinewidth{0.000000pt}%
\definecolor{currentstroke}{rgb}{0.000000,0.000000,0.000000}%
\pgfsetstrokecolor{currentstroke}%
\pgfsetstrokeopacity{0.700000}%
\pgfsetdash{}{0pt}%
\pgfpathmoveto{\pgfqpoint{9.428478in}{1.614705in}}%
\pgfpathcurveto{\pgfqpoint{9.433522in}{1.614705in}}{\pgfqpoint{9.438360in}{1.616709in}}{\pgfqpoint{9.441926in}{1.620275in}}%
\pgfpathcurveto{\pgfqpoint{9.445492in}{1.623842in}}{\pgfqpoint{9.447496in}{1.628679in}}{\pgfqpoint{9.447496in}{1.633723in}}%
\pgfpathcurveto{\pgfqpoint{9.447496in}{1.638767in}}{\pgfqpoint{9.445492in}{1.643605in}}{\pgfqpoint{9.441926in}{1.647171in}}%
\pgfpathcurveto{\pgfqpoint{9.438360in}{1.650737in}}{\pgfqpoint{9.433522in}{1.652741in}}{\pgfqpoint{9.428478in}{1.652741in}}%
\pgfpathcurveto{\pgfqpoint{9.423434in}{1.652741in}}{\pgfqpoint{9.418597in}{1.650737in}}{\pgfqpoint{9.415030in}{1.647171in}}%
\pgfpathcurveto{\pgfqpoint{9.411464in}{1.643605in}}{\pgfqpoint{9.409460in}{1.638767in}}{\pgfqpoint{9.409460in}{1.633723in}}%
\pgfpathcurveto{\pgfqpoint{9.409460in}{1.628679in}}{\pgfqpoint{9.411464in}{1.623842in}}{\pgfqpoint{9.415030in}{1.620275in}}%
\pgfpathcurveto{\pgfqpoint{9.418597in}{1.616709in}}{\pgfqpoint{9.423434in}{1.614705in}}{\pgfqpoint{9.428478in}{1.614705in}}%
\pgfpathclose%
\pgfusepath{fill}%
\end{pgfscope}%
\begin{pgfscope}%
\pgfpathrectangle{\pgfqpoint{6.572727in}{0.474100in}}{\pgfqpoint{4.227273in}{3.318700in}}%
\pgfusepath{clip}%
\pgfsetbuttcap%
\pgfsetroundjoin%
\definecolor{currentfill}{rgb}{0.993248,0.906157,0.143936}%
\pgfsetfillcolor{currentfill}%
\pgfsetfillopacity{0.700000}%
\pgfsetlinewidth{0.000000pt}%
\definecolor{currentstroke}{rgb}{0.000000,0.000000,0.000000}%
\pgfsetstrokecolor{currentstroke}%
\pgfsetstrokeopacity{0.700000}%
\pgfsetdash{}{0pt}%
\pgfpathmoveto{\pgfqpoint{10.116195in}{1.308328in}}%
\pgfpathcurveto{\pgfqpoint{10.121239in}{1.308328in}}{\pgfqpoint{10.126077in}{1.310332in}}{\pgfqpoint{10.129643in}{1.313898in}}%
\pgfpathcurveto{\pgfqpoint{10.133209in}{1.317465in}}{\pgfqpoint{10.135213in}{1.322303in}}{\pgfqpoint{10.135213in}{1.327346in}}%
\pgfpathcurveto{\pgfqpoint{10.135213in}{1.332390in}}{\pgfqpoint{10.133209in}{1.337228in}}{\pgfqpoint{10.129643in}{1.340794in}}%
\pgfpathcurveto{\pgfqpoint{10.126077in}{1.344360in}}{\pgfqpoint{10.121239in}{1.346364in}}{\pgfqpoint{10.116195in}{1.346364in}}%
\pgfpathcurveto{\pgfqpoint{10.111152in}{1.346364in}}{\pgfqpoint{10.106314in}{1.344360in}}{\pgfqpoint{10.102747in}{1.340794in}}%
\pgfpathcurveto{\pgfqpoint{10.099181in}{1.337228in}}{\pgfqpoint{10.097177in}{1.332390in}}{\pgfqpoint{10.097177in}{1.327346in}}%
\pgfpathcurveto{\pgfqpoint{10.097177in}{1.322303in}}{\pgfqpoint{10.099181in}{1.317465in}}{\pgfqpoint{10.102747in}{1.313898in}}%
\pgfpathcurveto{\pgfqpoint{10.106314in}{1.310332in}}{\pgfqpoint{10.111152in}{1.308328in}}{\pgfqpoint{10.116195in}{1.308328in}}%
\pgfpathclose%
\pgfusepath{fill}%
\end{pgfscope}%
\begin{pgfscope}%
\pgfpathrectangle{\pgfqpoint{6.572727in}{0.474100in}}{\pgfqpoint{4.227273in}{3.318700in}}%
\pgfusepath{clip}%
\pgfsetbuttcap%
\pgfsetroundjoin%
\definecolor{currentfill}{rgb}{0.127568,0.566949,0.550556}%
\pgfsetfillcolor{currentfill}%
\pgfsetfillopacity{0.700000}%
\pgfsetlinewidth{0.000000pt}%
\definecolor{currentstroke}{rgb}{0.000000,0.000000,0.000000}%
\pgfsetstrokecolor{currentstroke}%
\pgfsetstrokeopacity{0.700000}%
\pgfsetdash{}{0pt}%
\pgfpathmoveto{\pgfqpoint{8.103464in}{2.665122in}}%
\pgfpathcurveto{\pgfqpoint{8.108508in}{2.665122in}}{\pgfqpoint{8.113345in}{2.667126in}}{\pgfqpoint{8.116912in}{2.670692in}}%
\pgfpathcurveto{\pgfqpoint{8.120478in}{2.674258in}}{\pgfqpoint{8.122482in}{2.679096in}}{\pgfqpoint{8.122482in}{2.684140in}}%
\pgfpathcurveto{\pgfqpoint{8.122482in}{2.689183in}}{\pgfqpoint{8.120478in}{2.694021in}}{\pgfqpoint{8.116912in}{2.697588in}}%
\pgfpathcurveto{\pgfqpoint{8.113345in}{2.701154in}}{\pgfqpoint{8.108508in}{2.703158in}}{\pgfqpoint{8.103464in}{2.703158in}}%
\pgfpathcurveto{\pgfqpoint{8.098420in}{2.703158in}}{\pgfqpoint{8.093582in}{2.701154in}}{\pgfqpoint{8.090016in}{2.697588in}}%
\pgfpathcurveto{\pgfqpoint{8.086450in}{2.694021in}}{\pgfqpoint{8.084446in}{2.689183in}}{\pgfqpoint{8.084446in}{2.684140in}}%
\pgfpathcurveto{\pgfqpoint{8.084446in}{2.679096in}}{\pgfqpoint{8.086450in}{2.674258in}}{\pgfqpoint{8.090016in}{2.670692in}}%
\pgfpathcurveto{\pgfqpoint{8.093582in}{2.667126in}}{\pgfqpoint{8.098420in}{2.665122in}}{\pgfqpoint{8.103464in}{2.665122in}}%
\pgfpathclose%
\pgfusepath{fill}%
\end{pgfscope}%
\begin{pgfscope}%
\pgfpathrectangle{\pgfqpoint{6.572727in}{0.474100in}}{\pgfqpoint{4.227273in}{3.318700in}}%
\pgfusepath{clip}%
\pgfsetbuttcap%
\pgfsetroundjoin%
\definecolor{currentfill}{rgb}{0.993248,0.906157,0.143936}%
\pgfsetfillcolor{currentfill}%
\pgfsetfillopacity{0.700000}%
\pgfsetlinewidth{0.000000pt}%
\definecolor{currentstroke}{rgb}{0.000000,0.000000,0.000000}%
\pgfsetstrokecolor{currentstroke}%
\pgfsetstrokeopacity{0.700000}%
\pgfsetdash{}{0pt}%
\pgfpathmoveto{\pgfqpoint{9.523161in}{1.114760in}}%
\pgfpathcurveto{\pgfqpoint{9.528205in}{1.114760in}}{\pgfqpoint{9.533043in}{1.116764in}}{\pgfqpoint{9.536609in}{1.120331in}}%
\pgfpathcurveto{\pgfqpoint{9.540176in}{1.123897in}}{\pgfqpoint{9.542180in}{1.128735in}}{\pgfqpoint{9.542180in}{1.133779in}}%
\pgfpathcurveto{\pgfqpoint{9.542180in}{1.138822in}}{\pgfqpoint{9.540176in}{1.143660in}}{\pgfqpoint{9.536609in}{1.147226in}}%
\pgfpathcurveto{\pgfqpoint{9.533043in}{1.150793in}}{\pgfqpoint{9.528205in}{1.152797in}}{\pgfqpoint{9.523161in}{1.152797in}}%
\pgfpathcurveto{\pgfqpoint{9.518118in}{1.152797in}}{\pgfqpoint{9.513280in}{1.150793in}}{\pgfqpoint{9.509714in}{1.147226in}}%
\pgfpathcurveto{\pgfqpoint{9.506147in}{1.143660in}}{\pgfqpoint{9.504143in}{1.138822in}}{\pgfqpoint{9.504143in}{1.133779in}}%
\pgfpathcurveto{\pgfqpoint{9.504143in}{1.128735in}}{\pgfqpoint{9.506147in}{1.123897in}}{\pgfqpoint{9.509714in}{1.120331in}}%
\pgfpathcurveto{\pgfqpoint{9.513280in}{1.116764in}}{\pgfqpoint{9.518118in}{1.114760in}}{\pgfqpoint{9.523161in}{1.114760in}}%
\pgfpathclose%
\pgfusepath{fill}%
\end{pgfscope}%
\begin{pgfscope}%
\pgfpathrectangle{\pgfqpoint{6.572727in}{0.474100in}}{\pgfqpoint{4.227273in}{3.318700in}}%
\pgfusepath{clip}%
\pgfsetbuttcap%
\pgfsetroundjoin%
\definecolor{currentfill}{rgb}{0.993248,0.906157,0.143936}%
\pgfsetfillcolor{currentfill}%
\pgfsetfillopacity{0.700000}%
\pgfsetlinewidth{0.000000pt}%
\definecolor{currentstroke}{rgb}{0.000000,0.000000,0.000000}%
\pgfsetstrokecolor{currentstroke}%
\pgfsetstrokeopacity{0.700000}%
\pgfsetdash{}{0pt}%
\pgfpathmoveto{\pgfqpoint{9.822215in}{1.796532in}}%
\pgfpathcurveto{\pgfqpoint{9.827259in}{1.796532in}}{\pgfqpoint{9.832097in}{1.798536in}}{\pgfqpoint{9.835663in}{1.802102in}}%
\pgfpathcurveto{\pgfqpoint{9.839229in}{1.805669in}}{\pgfqpoint{9.841233in}{1.810506in}}{\pgfqpoint{9.841233in}{1.815550in}}%
\pgfpathcurveto{\pgfqpoint{9.841233in}{1.820594in}}{\pgfqpoint{9.839229in}{1.825432in}}{\pgfqpoint{9.835663in}{1.828998in}}%
\pgfpathcurveto{\pgfqpoint{9.832097in}{1.832564in}}{\pgfqpoint{9.827259in}{1.834568in}}{\pgfqpoint{9.822215in}{1.834568in}}%
\pgfpathcurveto{\pgfqpoint{9.817171in}{1.834568in}}{\pgfqpoint{9.812334in}{1.832564in}}{\pgfqpoint{9.808767in}{1.828998in}}%
\pgfpathcurveto{\pgfqpoint{9.805201in}{1.825432in}}{\pgfqpoint{9.803197in}{1.820594in}}{\pgfqpoint{9.803197in}{1.815550in}}%
\pgfpathcurveto{\pgfqpoint{9.803197in}{1.810506in}}{\pgfqpoint{9.805201in}{1.805669in}}{\pgfqpoint{9.808767in}{1.802102in}}%
\pgfpathcurveto{\pgfqpoint{9.812334in}{1.798536in}}{\pgfqpoint{9.817171in}{1.796532in}}{\pgfqpoint{9.822215in}{1.796532in}}%
\pgfpathclose%
\pgfusepath{fill}%
\end{pgfscope}%
\begin{pgfscope}%
\pgfpathrectangle{\pgfqpoint{6.572727in}{0.474100in}}{\pgfqpoint{4.227273in}{3.318700in}}%
\pgfusepath{clip}%
\pgfsetbuttcap%
\pgfsetroundjoin%
\definecolor{currentfill}{rgb}{0.993248,0.906157,0.143936}%
\pgfsetfillcolor{currentfill}%
\pgfsetfillopacity{0.700000}%
\pgfsetlinewidth{0.000000pt}%
\definecolor{currentstroke}{rgb}{0.000000,0.000000,0.000000}%
\pgfsetstrokecolor{currentstroke}%
\pgfsetstrokeopacity{0.700000}%
\pgfsetdash{}{0pt}%
\pgfpathmoveto{\pgfqpoint{10.196545in}{1.241149in}}%
\pgfpathcurveto{\pgfqpoint{10.201589in}{1.241149in}}{\pgfqpoint{10.206427in}{1.243153in}}{\pgfqpoint{10.209993in}{1.246719in}}%
\pgfpathcurveto{\pgfqpoint{10.213560in}{1.250286in}}{\pgfqpoint{10.215563in}{1.255124in}}{\pgfqpoint{10.215563in}{1.260167in}}%
\pgfpathcurveto{\pgfqpoint{10.215563in}{1.265211in}}{\pgfqpoint{10.213560in}{1.270049in}}{\pgfqpoint{10.209993in}{1.273615in}}%
\pgfpathcurveto{\pgfqpoint{10.206427in}{1.277182in}}{\pgfqpoint{10.201589in}{1.279185in}}{\pgfqpoint{10.196545in}{1.279185in}}%
\pgfpathcurveto{\pgfqpoint{10.191502in}{1.279185in}}{\pgfqpoint{10.186664in}{1.277182in}}{\pgfqpoint{10.183097in}{1.273615in}}%
\pgfpathcurveto{\pgfqpoint{10.179531in}{1.270049in}}{\pgfqpoint{10.177527in}{1.265211in}}{\pgfqpoint{10.177527in}{1.260167in}}%
\pgfpathcurveto{\pgfqpoint{10.177527in}{1.255124in}}{\pgfqpoint{10.179531in}{1.250286in}}{\pgfqpoint{10.183097in}{1.246719in}}%
\pgfpathcurveto{\pgfqpoint{10.186664in}{1.243153in}}{\pgfqpoint{10.191502in}{1.241149in}}{\pgfqpoint{10.196545in}{1.241149in}}%
\pgfpathclose%
\pgfusepath{fill}%
\end{pgfscope}%
\begin{pgfscope}%
\pgfpathrectangle{\pgfqpoint{6.572727in}{0.474100in}}{\pgfqpoint{4.227273in}{3.318700in}}%
\pgfusepath{clip}%
\pgfsetbuttcap%
\pgfsetroundjoin%
\definecolor{currentfill}{rgb}{0.993248,0.906157,0.143936}%
\pgfsetfillcolor{currentfill}%
\pgfsetfillopacity{0.700000}%
\pgfsetlinewidth{0.000000pt}%
\definecolor{currentstroke}{rgb}{0.000000,0.000000,0.000000}%
\pgfsetstrokecolor{currentstroke}%
\pgfsetstrokeopacity{0.700000}%
\pgfsetdash{}{0pt}%
\pgfpathmoveto{\pgfqpoint{9.040576in}{1.839627in}}%
\pgfpathcurveto{\pgfqpoint{9.045619in}{1.839627in}}{\pgfqpoint{9.050457in}{1.841631in}}{\pgfqpoint{9.054023in}{1.845198in}}%
\pgfpathcurveto{\pgfqpoint{9.057590in}{1.848764in}}{\pgfqpoint{9.059594in}{1.853602in}}{\pgfqpoint{9.059594in}{1.858646in}}%
\pgfpathcurveto{\pgfqpoint{9.059594in}{1.863689in}}{\pgfqpoint{9.057590in}{1.868527in}}{\pgfqpoint{9.054023in}{1.872093in}}%
\pgfpathcurveto{\pgfqpoint{9.050457in}{1.875660in}}{\pgfqpoint{9.045619in}{1.877664in}}{\pgfqpoint{9.040576in}{1.877664in}}%
\pgfpathcurveto{\pgfqpoint{9.035532in}{1.877664in}}{\pgfqpoint{9.030694in}{1.875660in}}{\pgfqpoint{9.027128in}{1.872093in}}%
\pgfpathcurveto{\pgfqpoint{9.023561in}{1.868527in}}{\pgfqpoint{9.021557in}{1.863689in}}{\pgfqpoint{9.021557in}{1.858646in}}%
\pgfpathcurveto{\pgfqpoint{9.021557in}{1.853602in}}{\pgfqpoint{9.023561in}{1.848764in}}{\pgfqpoint{9.027128in}{1.845198in}}%
\pgfpathcurveto{\pgfqpoint{9.030694in}{1.841631in}}{\pgfqpoint{9.035532in}{1.839627in}}{\pgfqpoint{9.040576in}{1.839627in}}%
\pgfpathclose%
\pgfusepath{fill}%
\end{pgfscope}%
\begin{pgfscope}%
\pgfpathrectangle{\pgfqpoint{6.572727in}{0.474100in}}{\pgfqpoint{4.227273in}{3.318700in}}%
\pgfusepath{clip}%
\pgfsetbuttcap%
\pgfsetroundjoin%
\definecolor{currentfill}{rgb}{0.993248,0.906157,0.143936}%
\pgfsetfillcolor{currentfill}%
\pgfsetfillopacity{0.700000}%
\pgfsetlinewidth{0.000000pt}%
\definecolor{currentstroke}{rgb}{0.000000,0.000000,0.000000}%
\pgfsetstrokecolor{currentstroke}%
\pgfsetstrokeopacity{0.700000}%
\pgfsetdash{}{0pt}%
\pgfpathmoveto{\pgfqpoint{9.724121in}{1.318908in}}%
\pgfpathcurveto{\pgfqpoint{9.729165in}{1.318908in}}{\pgfqpoint{9.734003in}{1.320912in}}{\pgfqpoint{9.737569in}{1.324478in}}%
\pgfpathcurveto{\pgfqpoint{9.741135in}{1.328045in}}{\pgfqpoint{9.743139in}{1.332883in}}{\pgfqpoint{9.743139in}{1.337926in}}%
\pgfpathcurveto{\pgfqpoint{9.743139in}{1.342970in}}{\pgfqpoint{9.741135in}{1.347808in}}{\pgfqpoint{9.737569in}{1.351374in}}%
\pgfpathcurveto{\pgfqpoint{9.734003in}{1.354941in}}{\pgfqpoint{9.729165in}{1.356944in}}{\pgfqpoint{9.724121in}{1.356944in}}%
\pgfpathcurveto{\pgfqpoint{9.719077in}{1.356944in}}{\pgfqpoint{9.714240in}{1.354941in}}{\pgfqpoint{9.710673in}{1.351374in}}%
\pgfpathcurveto{\pgfqpoint{9.707107in}{1.347808in}}{\pgfqpoint{9.705103in}{1.342970in}}{\pgfqpoint{9.705103in}{1.337926in}}%
\pgfpathcurveto{\pgfqpoint{9.705103in}{1.332883in}}{\pgfqpoint{9.707107in}{1.328045in}}{\pgfqpoint{9.710673in}{1.324478in}}%
\pgfpathcurveto{\pgfqpoint{9.714240in}{1.320912in}}{\pgfqpoint{9.719077in}{1.318908in}}{\pgfqpoint{9.724121in}{1.318908in}}%
\pgfpathclose%
\pgfusepath{fill}%
\end{pgfscope}%
\begin{pgfscope}%
\pgfpathrectangle{\pgfqpoint{6.572727in}{0.474100in}}{\pgfqpoint{4.227273in}{3.318700in}}%
\pgfusepath{clip}%
\pgfsetbuttcap%
\pgfsetroundjoin%
\definecolor{currentfill}{rgb}{0.993248,0.906157,0.143936}%
\pgfsetfillcolor{currentfill}%
\pgfsetfillopacity{0.700000}%
\pgfsetlinewidth{0.000000pt}%
\definecolor{currentstroke}{rgb}{0.000000,0.000000,0.000000}%
\pgfsetstrokecolor{currentstroke}%
\pgfsetstrokeopacity{0.700000}%
\pgfsetdash{}{0pt}%
\pgfpathmoveto{\pgfqpoint{10.414029in}{1.713492in}}%
\pgfpathcurveto{\pgfqpoint{10.419072in}{1.713492in}}{\pgfqpoint{10.423910in}{1.715496in}}{\pgfqpoint{10.427477in}{1.719062in}}%
\pgfpathcurveto{\pgfqpoint{10.431043in}{1.722629in}}{\pgfqpoint{10.433047in}{1.727466in}}{\pgfqpoint{10.433047in}{1.732510in}}%
\pgfpathcurveto{\pgfqpoint{10.433047in}{1.737554in}}{\pgfqpoint{10.431043in}{1.742391in}}{\pgfqpoint{10.427477in}{1.745958in}}%
\pgfpathcurveto{\pgfqpoint{10.423910in}{1.749524in}}{\pgfqpoint{10.419072in}{1.751528in}}{\pgfqpoint{10.414029in}{1.751528in}}%
\pgfpathcurveto{\pgfqpoint{10.408985in}{1.751528in}}{\pgfqpoint{10.404147in}{1.749524in}}{\pgfqpoint{10.400581in}{1.745958in}}%
\pgfpathcurveto{\pgfqpoint{10.397015in}{1.742391in}}{\pgfqpoint{10.395011in}{1.737554in}}{\pgfqpoint{10.395011in}{1.732510in}}%
\pgfpathcurveto{\pgfqpoint{10.395011in}{1.727466in}}{\pgfqpoint{10.397015in}{1.722629in}}{\pgfqpoint{10.400581in}{1.719062in}}%
\pgfpathcurveto{\pgfqpoint{10.404147in}{1.715496in}}{\pgfqpoint{10.408985in}{1.713492in}}{\pgfqpoint{10.414029in}{1.713492in}}%
\pgfpathclose%
\pgfusepath{fill}%
\end{pgfscope}%
\begin{pgfscope}%
\pgfpathrectangle{\pgfqpoint{6.572727in}{0.474100in}}{\pgfqpoint{4.227273in}{3.318700in}}%
\pgfusepath{clip}%
\pgfsetbuttcap%
\pgfsetroundjoin%
\definecolor{currentfill}{rgb}{0.267004,0.004874,0.329415}%
\pgfsetfillcolor{currentfill}%
\pgfsetfillopacity{0.700000}%
\pgfsetlinewidth{0.000000pt}%
\definecolor{currentstroke}{rgb}{0.000000,0.000000,0.000000}%
\pgfsetstrokecolor{currentstroke}%
\pgfsetstrokeopacity{0.700000}%
\pgfsetdash{}{0pt}%
\pgfpathmoveto{\pgfqpoint{8.703118in}{0.803521in}}%
\pgfpathcurveto{\pgfqpoint{8.708162in}{0.803521in}}{\pgfqpoint{8.713000in}{0.805525in}}{\pgfqpoint{8.716566in}{0.809092in}}%
\pgfpathcurveto{\pgfqpoint{8.720133in}{0.812658in}}{\pgfqpoint{8.722136in}{0.817496in}}{\pgfqpoint{8.722136in}{0.822540in}}%
\pgfpathcurveto{\pgfqpoint{8.722136in}{0.827583in}}{\pgfqpoint{8.720133in}{0.832421in}}{\pgfqpoint{8.716566in}{0.835987in}}%
\pgfpathcurveto{\pgfqpoint{8.713000in}{0.839554in}}{\pgfqpoint{8.708162in}{0.841558in}}{\pgfqpoint{8.703118in}{0.841558in}}%
\pgfpathcurveto{\pgfqpoint{8.698075in}{0.841558in}}{\pgfqpoint{8.693237in}{0.839554in}}{\pgfqpoint{8.689670in}{0.835987in}}%
\pgfpathcurveto{\pgfqpoint{8.686104in}{0.832421in}}{\pgfqpoint{8.684100in}{0.827583in}}{\pgfqpoint{8.684100in}{0.822540in}}%
\pgfpathcurveto{\pgfqpoint{8.684100in}{0.817496in}}{\pgfqpoint{8.686104in}{0.812658in}}{\pgfqpoint{8.689670in}{0.809092in}}%
\pgfpathcurveto{\pgfqpoint{8.693237in}{0.805525in}}{\pgfqpoint{8.698075in}{0.803521in}}{\pgfqpoint{8.703118in}{0.803521in}}%
\pgfpathclose%
\pgfusepath{fill}%
\end{pgfscope}%
\begin{pgfscope}%
\pgfpathrectangle{\pgfqpoint{6.572727in}{0.474100in}}{\pgfqpoint{4.227273in}{3.318700in}}%
\pgfusepath{clip}%
\pgfsetbuttcap%
\pgfsetroundjoin%
\definecolor{currentfill}{rgb}{0.993248,0.906157,0.143936}%
\pgfsetfillcolor{currentfill}%
\pgfsetfillopacity{0.700000}%
\pgfsetlinewidth{0.000000pt}%
\definecolor{currentstroke}{rgb}{0.000000,0.000000,0.000000}%
\pgfsetstrokecolor{currentstroke}%
\pgfsetstrokeopacity{0.700000}%
\pgfsetdash{}{0pt}%
\pgfpathmoveto{\pgfqpoint{9.323233in}{2.160461in}}%
\pgfpathcurveto{\pgfqpoint{9.328276in}{2.160461in}}{\pgfqpoint{9.333114in}{2.162465in}}{\pgfqpoint{9.336680in}{2.166032in}}%
\pgfpathcurveto{\pgfqpoint{9.340247in}{2.169598in}}{\pgfqpoint{9.342251in}{2.174436in}}{\pgfqpoint{9.342251in}{2.179479in}}%
\pgfpathcurveto{\pgfqpoint{9.342251in}{2.184523in}}{\pgfqpoint{9.340247in}{2.189361in}}{\pgfqpoint{9.336680in}{2.192927in}}%
\pgfpathcurveto{\pgfqpoint{9.333114in}{2.196494in}}{\pgfqpoint{9.328276in}{2.198498in}}{\pgfqpoint{9.323233in}{2.198498in}}%
\pgfpathcurveto{\pgfqpoint{9.318189in}{2.198498in}}{\pgfqpoint{9.313351in}{2.196494in}}{\pgfqpoint{9.309785in}{2.192927in}}%
\pgfpathcurveto{\pgfqpoint{9.306218in}{2.189361in}}{\pgfqpoint{9.304214in}{2.184523in}}{\pgfqpoint{9.304214in}{2.179479in}}%
\pgfpathcurveto{\pgfqpoint{9.304214in}{2.174436in}}{\pgfqpoint{9.306218in}{2.169598in}}{\pgfqpoint{9.309785in}{2.166032in}}%
\pgfpathcurveto{\pgfqpoint{9.313351in}{2.162465in}}{\pgfqpoint{9.318189in}{2.160461in}}{\pgfqpoint{9.323233in}{2.160461in}}%
\pgfpathclose%
\pgfusepath{fill}%
\end{pgfscope}%
\begin{pgfscope}%
\pgfpathrectangle{\pgfqpoint{6.572727in}{0.474100in}}{\pgfqpoint{4.227273in}{3.318700in}}%
\pgfusepath{clip}%
\pgfsetbuttcap%
\pgfsetroundjoin%
\definecolor{currentfill}{rgb}{0.127568,0.566949,0.550556}%
\pgfsetfillcolor{currentfill}%
\pgfsetfillopacity{0.700000}%
\pgfsetlinewidth{0.000000pt}%
\definecolor{currentstroke}{rgb}{0.000000,0.000000,0.000000}%
\pgfsetstrokecolor{currentstroke}%
\pgfsetstrokeopacity{0.700000}%
\pgfsetdash{}{0pt}%
\pgfpathmoveto{\pgfqpoint{7.856344in}{1.168467in}}%
\pgfpathcurveto{\pgfqpoint{7.861387in}{1.168467in}}{\pgfqpoint{7.866225in}{1.170470in}}{\pgfqpoint{7.869792in}{1.174037in}}%
\pgfpathcurveto{\pgfqpoint{7.873358in}{1.177603in}}{\pgfqpoint{7.875362in}{1.182441in}}{\pgfqpoint{7.875362in}{1.187485in}}%
\pgfpathcurveto{\pgfqpoint{7.875362in}{1.192528in}}{\pgfqpoint{7.873358in}{1.197366in}}{\pgfqpoint{7.869792in}{1.200933in}}%
\pgfpathcurveto{\pgfqpoint{7.866225in}{1.204499in}}{\pgfqpoint{7.861387in}{1.206503in}}{\pgfqpoint{7.856344in}{1.206503in}}%
\pgfpathcurveto{\pgfqpoint{7.851300in}{1.206503in}}{\pgfqpoint{7.846462in}{1.204499in}}{\pgfqpoint{7.842896in}{1.200933in}}%
\pgfpathcurveto{\pgfqpoint{7.839330in}{1.197366in}}{\pgfqpoint{7.837326in}{1.192528in}}{\pgfqpoint{7.837326in}{1.187485in}}%
\pgfpathcurveto{\pgfqpoint{7.837326in}{1.182441in}}{\pgfqpoint{7.839330in}{1.177603in}}{\pgfqpoint{7.842896in}{1.174037in}}%
\pgfpathcurveto{\pgfqpoint{7.846462in}{1.170470in}}{\pgfqpoint{7.851300in}{1.168467in}}{\pgfqpoint{7.856344in}{1.168467in}}%
\pgfpathclose%
\pgfusepath{fill}%
\end{pgfscope}%
\begin{pgfscope}%
\pgfpathrectangle{\pgfqpoint{6.572727in}{0.474100in}}{\pgfqpoint{4.227273in}{3.318700in}}%
\pgfusepath{clip}%
\pgfsetbuttcap%
\pgfsetroundjoin%
\definecolor{currentfill}{rgb}{0.127568,0.566949,0.550556}%
\pgfsetfillcolor{currentfill}%
\pgfsetfillopacity{0.700000}%
\pgfsetlinewidth{0.000000pt}%
\definecolor{currentstroke}{rgb}{0.000000,0.000000,0.000000}%
\pgfsetstrokecolor{currentstroke}%
\pgfsetstrokeopacity{0.700000}%
\pgfsetdash{}{0pt}%
\pgfpathmoveto{\pgfqpoint{7.767347in}{1.636610in}}%
\pgfpathcurveto{\pgfqpoint{7.772391in}{1.636610in}}{\pgfqpoint{7.777228in}{1.638613in}}{\pgfqpoint{7.780795in}{1.642180in}}%
\pgfpathcurveto{\pgfqpoint{7.784361in}{1.645746in}}{\pgfqpoint{7.786365in}{1.650584in}}{\pgfqpoint{7.786365in}{1.655628in}}%
\pgfpathcurveto{\pgfqpoint{7.786365in}{1.660671in}}{\pgfqpoint{7.784361in}{1.665509in}}{\pgfqpoint{7.780795in}{1.669076in}}%
\pgfpathcurveto{\pgfqpoint{7.777228in}{1.672642in}}{\pgfqpoint{7.772391in}{1.674646in}}{\pgfqpoint{7.767347in}{1.674646in}}%
\pgfpathcurveto{\pgfqpoint{7.762303in}{1.674646in}}{\pgfqpoint{7.757466in}{1.672642in}}{\pgfqpoint{7.753899in}{1.669076in}}%
\pgfpathcurveto{\pgfqpoint{7.750333in}{1.665509in}}{\pgfqpoint{7.748329in}{1.660671in}}{\pgfqpoint{7.748329in}{1.655628in}}%
\pgfpathcurveto{\pgfqpoint{7.748329in}{1.650584in}}{\pgfqpoint{7.750333in}{1.645746in}}{\pgfqpoint{7.753899in}{1.642180in}}%
\pgfpathcurveto{\pgfqpoint{7.757466in}{1.638613in}}{\pgfqpoint{7.762303in}{1.636610in}}{\pgfqpoint{7.767347in}{1.636610in}}%
\pgfpathclose%
\pgfusepath{fill}%
\end{pgfscope}%
\begin{pgfscope}%
\pgfpathrectangle{\pgfqpoint{6.572727in}{0.474100in}}{\pgfqpoint{4.227273in}{3.318700in}}%
\pgfusepath{clip}%
\pgfsetbuttcap%
\pgfsetroundjoin%
\definecolor{currentfill}{rgb}{0.993248,0.906157,0.143936}%
\pgfsetfillcolor{currentfill}%
\pgfsetfillopacity{0.700000}%
\pgfsetlinewidth{0.000000pt}%
\definecolor{currentstroke}{rgb}{0.000000,0.000000,0.000000}%
\pgfsetstrokecolor{currentstroke}%
\pgfsetstrokeopacity{0.700000}%
\pgfsetdash{}{0pt}%
\pgfpathmoveto{\pgfqpoint{9.706365in}{2.097766in}}%
\pgfpathcurveto{\pgfqpoint{9.711408in}{2.097766in}}{\pgfqpoint{9.716246in}{2.099770in}}{\pgfqpoint{9.719813in}{2.103336in}}%
\pgfpathcurveto{\pgfqpoint{9.723379in}{2.106902in}}{\pgfqpoint{9.725383in}{2.111740in}}{\pgfqpoint{9.725383in}{2.116784in}}%
\pgfpathcurveto{\pgfqpoint{9.725383in}{2.121828in}}{\pgfqpoint{9.723379in}{2.126665in}}{\pgfqpoint{9.719813in}{2.130232in}}%
\pgfpathcurveto{\pgfqpoint{9.716246in}{2.133798in}}{\pgfqpoint{9.711408in}{2.135802in}}{\pgfqpoint{9.706365in}{2.135802in}}%
\pgfpathcurveto{\pgfqpoint{9.701321in}{2.135802in}}{\pgfqpoint{9.696483in}{2.133798in}}{\pgfqpoint{9.692917in}{2.130232in}}%
\pgfpathcurveto{\pgfqpoint{9.689350in}{2.126665in}}{\pgfqpoint{9.687347in}{2.121828in}}{\pgfqpoint{9.687347in}{2.116784in}}%
\pgfpathcurveto{\pgfqpoint{9.687347in}{2.111740in}}{\pgfqpoint{9.689350in}{2.106902in}}{\pgfqpoint{9.692917in}{2.103336in}}%
\pgfpathcurveto{\pgfqpoint{9.696483in}{2.099770in}}{\pgfqpoint{9.701321in}{2.097766in}}{\pgfqpoint{9.706365in}{2.097766in}}%
\pgfpathclose%
\pgfusepath{fill}%
\end{pgfscope}%
\begin{pgfscope}%
\pgfpathrectangle{\pgfqpoint{6.572727in}{0.474100in}}{\pgfqpoint{4.227273in}{3.318700in}}%
\pgfusepath{clip}%
\pgfsetbuttcap%
\pgfsetroundjoin%
\definecolor{currentfill}{rgb}{0.127568,0.566949,0.550556}%
\pgfsetfillcolor{currentfill}%
\pgfsetfillopacity{0.700000}%
\pgfsetlinewidth{0.000000pt}%
\definecolor{currentstroke}{rgb}{0.000000,0.000000,0.000000}%
\pgfsetstrokecolor{currentstroke}%
\pgfsetstrokeopacity{0.700000}%
\pgfsetdash{}{0pt}%
\pgfpathmoveto{\pgfqpoint{7.866135in}{1.688261in}}%
\pgfpathcurveto{\pgfqpoint{7.871179in}{1.688261in}}{\pgfqpoint{7.876017in}{1.690265in}}{\pgfqpoint{7.879583in}{1.693831in}}%
\pgfpathcurveto{\pgfqpoint{7.883150in}{1.697397in}}{\pgfqpoint{7.885153in}{1.702235in}}{\pgfqpoint{7.885153in}{1.707279in}}%
\pgfpathcurveto{\pgfqpoint{7.885153in}{1.712323in}}{\pgfqpoint{7.883150in}{1.717160in}}{\pgfqpoint{7.879583in}{1.720727in}}%
\pgfpathcurveto{\pgfqpoint{7.876017in}{1.724293in}}{\pgfqpoint{7.871179in}{1.726297in}}{\pgfqpoint{7.866135in}{1.726297in}}%
\pgfpathcurveto{\pgfqpoint{7.861092in}{1.726297in}}{\pgfqpoint{7.856254in}{1.724293in}}{\pgfqpoint{7.852687in}{1.720727in}}%
\pgfpathcurveto{\pgfqpoint{7.849121in}{1.717160in}}{\pgfqpoint{7.847117in}{1.712323in}}{\pgfqpoint{7.847117in}{1.707279in}}%
\pgfpathcurveto{\pgfqpoint{7.847117in}{1.702235in}}{\pgfqpoint{7.849121in}{1.697397in}}{\pgfqpoint{7.852687in}{1.693831in}}%
\pgfpathcurveto{\pgfqpoint{7.856254in}{1.690265in}}{\pgfqpoint{7.861092in}{1.688261in}}{\pgfqpoint{7.866135in}{1.688261in}}%
\pgfpathclose%
\pgfusepath{fill}%
\end{pgfscope}%
\begin{pgfscope}%
\pgfpathrectangle{\pgfqpoint{6.572727in}{0.474100in}}{\pgfqpoint{4.227273in}{3.318700in}}%
\pgfusepath{clip}%
\pgfsetbuttcap%
\pgfsetroundjoin%
\definecolor{currentfill}{rgb}{0.127568,0.566949,0.550556}%
\pgfsetfillcolor{currentfill}%
\pgfsetfillopacity{0.700000}%
\pgfsetlinewidth{0.000000pt}%
\definecolor{currentstroke}{rgb}{0.000000,0.000000,0.000000}%
\pgfsetstrokecolor{currentstroke}%
\pgfsetstrokeopacity{0.700000}%
\pgfsetdash{}{0pt}%
\pgfpathmoveto{\pgfqpoint{8.416112in}{2.689800in}}%
\pgfpathcurveto{\pgfqpoint{8.421155in}{2.689800in}}{\pgfqpoint{8.425993in}{2.691804in}}{\pgfqpoint{8.429560in}{2.695370in}}%
\pgfpathcurveto{\pgfqpoint{8.433126in}{2.698937in}}{\pgfqpoint{8.435130in}{2.703775in}}{\pgfqpoint{8.435130in}{2.708818in}}%
\pgfpathcurveto{\pgfqpoint{8.435130in}{2.713862in}}{\pgfqpoint{8.433126in}{2.718700in}}{\pgfqpoint{8.429560in}{2.722266in}}%
\pgfpathcurveto{\pgfqpoint{8.425993in}{2.725833in}}{\pgfqpoint{8.421155in}{2.727836in}}{\pgfqpoint{8.416112in}{2.727836in}}%
\pgfpathcurveto{\pgfqpoint{8.411068in}{2.727836in}}{\pgfqpoint{8.406230in}{2.725833in}}{\pgfqpoint{8.402664in}{2.722266in}}%
\pgfpathcurveto{\pgfqpoint{8.399097in}{2.718700in}}{\pgfqpoint{8.397094in}{2.713862in}}{\pgfqpoint{8.397094in}{2.708818in}}%
\pgfpathcurveto{\pgfqpoint{8.397094in}{2.703775in}}{\pgfqpoint{8.399097in}{2.698937in}}{\pgfqpoint{8.402664in}{2.695370in}}%
\pgfpathcurveto{\pgfqpoint{8.406230in}{2.691804in}}{\pgfqpoint{8.411068in}{2.689800in}}{\pgfqpoint{8.416112in}{2.689800in}}%
\pgfpathclose%
\pgfusepath{fill}%
\end{pgfscope}%
\begin{pgfscope}%
\pgfpathrectangle{\pgfqpoint{6.572727in}{0.474100in}}{\pgfqpoint{4.227273in}{3.318700in}}%
\pgfusepath{clip}%
\pgfsetbuttcap%
\pgfsetroundjoin%
\definecolor{currentfill}{rgb}{0.993248,0.906157,0.143936}%
\pgfsetfillcolor{currentfill}%
\pgfsetfillopacity{0.700000}%
\pgfsetlinewidth{0.000000pt}%
\definecolor{currentstroke}{rgb}{0.000000,0.000000,0.000000}%
\pgfsetstrokecolor{currentstroke}%
\pgfsetstrokeopacity{0.700000}%
\pgfsetdash{}{0pt}%
\pgfpathmoveto{\pgfqpoint{9.629258in}{1.664976in}}%
\pgfpathcurveto{\pgfqpoint{9.634302in}{1.664976in}}{\pgfqpoint{9.639140in}{1.666980in}}{\pgfqpoint{9.642706in}{1.670546in}}%
\pgfpathcurveto{\pgfqpoint{9.646273in}{1.674113in}}{\pgfqpoint{9.648277in}{1.678951in}}{\pgfqpoint{9.648277in}{1.683994in}}%
\pgfpathcurveto{\pgfqpoint{9.648277in}{1.689038in}}{\pgfqpoint{9.646273in}{1.693876in}}{\pgfqpoint{9.642706in}{1.697442in}}%
\pgfpathcurveto{\pgfqpoint{9.639140in}{1.701009in}}{\pgfqpoint{9.634302in}{1.703012in}}{\pgfqpoint{9.629258in}{1.703012in}}%
\pgfpathcurveto{\pgfqpoint{9.624215in}{1.703012in}}{\pgfqpoint{9.619377in}{1.701009in}}{\pgfqpoint{9.615811in}{1.697442in}}%
\pgfpathcurveto{\pgfqpoint{9.612244in}{1.693876in}}{\pgfqpoint{9.610240in}{1.689038in}}{\pgfqpoint{9.610240in}{1.683994in}}%
\pgfpathcurveto{\pgfqpoint{9.610240in}{1.678951in}}{\pgfqpoint{9.612244in}{1.674113in}}{\pgfqpoint{9.615811in}{1.670546in}}%
\pgfpathcurveto{\pgfqpoint{9.619377in}{1.666980in}}{\pgfqpoint{9.624215in}{1.664976in}}{\pgfqpoint{9.629258in}{1.664976in}}%
\pgfpathclose%
\pgfusepath{fill}%
\end{pgfscope}%
\begin{pgfscope}%
\pgfpathrectangle{\pgfqpoint{6.572727in}{0.474100in}}{\pgfqpoint{4.227273in}{3.318700in}}%
\pgfusepath{clip}%
\pgfsetbuttcap%
\pgfsetroundjoin%
\definecolor{currentfill}{rgb}{0.993248,0.906157,0.143936}%
\pgfsetfillcolor{currentfill}%
\pgfsetfillopacity{0.700000}%
\pgfsetlinewidth{0.000000pt}%
\definecolor{currentstroke}{rgb}{0.000000,0.000000,0.000000}%
\pgfsetstrokecolor{currentstroke}%
\pgfsetstrokeopacity{0.700000}%
\pgfsetdash{}{0pt}%
\pgfpathmoveto{\pgfqpoint{8.962418in}{1.287431in}}%
\pgfpathcurveto{\pgfqpoint{8.967462in}{1.287431in}}{\pgfqpoint{8.972299in}{1.289435in}}{\pgfqpoint{8.975866in}{1.293001in}}%
\pgfpathcurveto{\pgfqpoint{8.979432in}{1.296568in}}{\pgfqpoint{8.981436in}{1.301405in}}{\pgfqpoint{8.981436in}{1.306449in}}%
\pgfpathcurveto{\pgfqpoint{8.981436in}{1.311493in}}{\pgfqpoint{8.979432in}{1.316331in}}{\pgfqpoint{8.975866in}{1.319897in}}%
\pgfpathcurveto{\pgfqpoint{8.972299in}{1.323463in}}{\pgfqpoint{8.967462in}{1.325467in}}{\pgfqpoint{8.962418in}{1.325467in}}%
\pgfpathcurveto{\pgfqpoint{8.957374in}{1.325467in}}{\pgfqpoint{8.952536in}{1.323463in}}{\pgfqpoint{8.948970in}{1.319897in}}%
\pgfpathcurveto{\pgfqpoint{8.945404in}{1.316331in}}{\pgfqpoint{8.943400in}{1.311493in}}{\pgfqpoint{8.943400in}{1.306449in}}%
\pgfpathcurveto{\pgfqpoint{8.943400in}{1.301405in}}{\pgfqpoint{8.945404in}{1.296568in}}{\pgfqpoint{8.948970in}{1.293001in}}%
\pgfpathcurveto{\pgfqpoint{8.952536in}{1.289435in}}{\pgfqpoint{8.957374in}{1.287431in}}{\pgfqpoint{8.962418in}{1.287431in}}%
\pgfpathclose%
\pgfusepath{fill}%
\end{pgfscope}%
\begin{pgfscope}%
\pgfpathrectangle{\pgfqpoint{6.572727in}{0.474100in}}{\pgfqpoint{4.227273in}{3.318700in}}%
\pgfusepath{clip}%
\pgfsetbuttcap%
\pgfsetroundjoin%
\definecolor{currentfill}{rgb}{0.993248,0.906157,0.143936}%
\pgfsetfillcolor{currentfill}%
\pgfsetfillopacity{0.700000}%
\pgfsetlinewidth{0.000000pt}%
\definecolor{currentstroke}{rgb}{0.000000,0.000000,0.000000}%
\pgfsetstrokecolor{currentstroke}%
\pgfsetstrokeopacity{0.700000}%
\pgfsetdash{}{0pt}%
\pgfpathmoveto{\pgfqpoint{9.790870in}{1.879159in}}%
\pgfpathcurveto{\pgfqpoint{9.795913in}{1.879159in}}{\pgfqpoint{9.800751in}{1.881163in}}{\pgfqpoint{9.804317in}{1.884729in}}%
\pgfpathcurveto{\pgfqpoint{9.807884in}{1.888296in}}{\pgfqpoint{9.809888in}{1.893134in}}{\pgfqpoint{9.809888in}{1.898177in}}%
\pgfpathcurveto{\pgfqpoint{9.809888in}{1.903221in}}{\pgfqpoint{9.807884in}{1.908059in}}{\pgfqpoint{9.804317in}{1.911625in}}%
\pgfpathcurveto{\pgfqpoint{9.800751in}{1.915192in}}{\pgfqpoint{9.795913in}{1.917195in}}{\pgfqpoint{9.790870in}{1.917195in}}%
\pgfpathcurveto{\pgfqpoint{9.785826in}{1.917195in}}{\pgfqpoint{9.780988in}{1.915192in}}{\pgfqpoint{9.777422in}{1.911625in}}%
\pgfpathcurveto{\pgfqpoint{9.773855in}{1.908059in}}{\pgfqpoint{9.771851in}{1.903221in}}{\pgfqpoint{9.771851in}{1.898177in}}%
\pgfpathcurveto{\pgfqpoint{9.771851in}{1.893134in}}{\pgfqpoint{9.773855in}{1.888296in}}{\pgfqpoint{9.777422in}{1.884729in}}%
\pgfpathcurveto{\pgfqpoint{9.780988in}{1.881163in}}{\pgfqpoint{9.785826in}{1.879159in}}{\pgfqpoint{9.790870in}{1.879159in}}%
\pgfpathclose%
\pgfusepath{fill}%
\end{pgfscope}%
\begin{pgfscope}%
\pgfpathrectangle{\pgfqpoint{6.572727in}{0.474100in}}{\pgfqpoint{4.227273in}{3.318700in}}%
\pgfusepath{clip}%
\pgfsetbuttcap%
\pgfsetroundjoin%
\definecolor{currentfill}{rgb}{0.127568,0.566949,0.550556}%
\pgfsetfillcolor{currentfill}%
\pgfsetfillopacity{0.700000}%
\pgfsetlinewidth{0.000000pt}%
\definecolor{currentstroke}{rgb}{0.000000,0.000000,0.000000}%
\pgfsetstrokecolor{currentstroke}%
\pgfsetstrokeopacity{0.700000}%
\pgfsetdash{}{0pt}%
\pgfpathmoveto{\pgfqpoint{7.402153in}{1.269046in}}%
\pgfpathcurveto{\pgfqpoint{7.407197in}{1.269046in}}{\pgfqpoint{7.412035in}{1.271050in}}{\pgfqpoint{7.415601in}{1.274617in}}%
\pgfpathcurveto{\pgfqpoint{7.419168in}{1.278183in}}{\pgfqpoint{7.421171in}{1.283021in}}{\pgfqpoint{7.421171in}{1.288064in}}%
\pgfpathcurveto{\pgfqpoint{7.421171in}{1.293108in}}{\pgfqpoint{7.419168in}{1.297946in}}{\pgfqpoint{7.415601in}{1.301512in}}%
\pgfpathcurveto{\pgfqpoint{7.412035in}{1.305079in}}{\pgfqpoint{7.407197in}{1.307083in}}{\pgfqpoint{7.402153in}{1.307083in}}%
\pgfpathcurveto{\pgfqpoint{7.397110in}{1.307083in}}{\pgfqpoint{7.392272in}{1.305079in}}{\pgfqpoint{7.388705in}{1.301512in}}%
\pgfpathcurveto{\pgfqpoint{7.385139in}{1.297946in}}{\pgfqpoint{7.383135in}{1.293108in}}{\pgfqpoint{7.383135in}{1.288064in}}%
\pgfpathcurveto{\pgfqpoint{7.383135in}{1.283021in}}{\pgfqpoint{7.385139in}{1.278183in}}{\pgfqpoint{7.388705in}{1.274617in}}%
\pgfpathcurveto{\pgfqpoint{7.392272in}{1.271050in}}{\pgfqpoint{7.397110in}{1.269046in}}{\pgfqpoint{7.402153in}{1.269046in}}%
\pgfpathclose%
\pgfusepath{fill}%
\end{pgfscope}%
\begin{pgfscope}%
\pgfpathrectangle{\pgfqpoint{6.572727in}{0.474100in}}{\pgfqpoint{4.227273in}{3.318700in}}%
\pgfusepath{clip}%
\pgfsetbuttcap%
\pgfsetroundjoin%
\definecolor{currentfill}{rgb}{0.993248,0.906157,0.143936}%
\pgfsetfillcolor{currentfill}%
\pgfsetfillopacity{0.700000}%
\pgfsetlinewidth{0.000000pt}%
\definecolor{currentstroke}{rgb}{0.000000,0.000000,0.000000}%
\pgfsetstrokecolor{currentstroke}%
\pgfsetstrokeopacity{0.700000}%
\pgfsetdash{}{0pt}%
\pgfpathmoveto{\pgfqpoint{9.794170in}{2.131494in}}%
\pgfpathcurveto{\pgfqpoint{9.799213in}{2.131494in}}{\pgfqpoint{9.804051in}{2.133498in}}{\pgfqpoint{9.807617in}{2.137065in}}%
\pgfpathcurveto{\pgfqpoint{9.811184in}{2.140631in}}{\pgfqpoint{9.813188in}{2.145469in}}{\pgfqpoint{9.813188in}{2.150512in}}%
\pgfpathcurveto{\pgfqpoint{9.813188in}{2.155556in}}{\pgfqpoint{9.811184in}{2.160394in}}{\pgfqpoint{9.807617in}{2.163960in}}%
\pgfpathcurveto{\pgfqpoint{9.804051in}{2.167527in}}{\pgfqpoint{9.799213in}{2.169531in}}{\pgfqpoint{9.794170in}{2.169531in}}%
\pgfpathcurveto{\pgfqpoint{9.789126in}{2.169531in}}{\pgfqpoint{9.784288in}{2.167527in}}{\pgfqpoint{9.780722in}{2.163960in}}%
\pgfpathcurveto{\pgfqpoint{9.777155in}{2.160394in}}{\pgfqpoint{9.775151in}{2.155556in}}{\pgfqpoint{9.775151in}{2.150512in}}%
\pgfpathcurveto{\pgfqpoint{9.775151in}{2.145469in}}{\pgfqpoint{9.777155in}{2.140631in}}{\pgfqpoint{9.780722in}{2.137065in}}%
\pgfpathcurveto{\pgfqpoint{9.784288in}{2.133498in}}{\pgfqpoint{9.789126in}{2.131494in}}{\pgfqpoint{9.794170in}{2.131494in}}%
\pgfpathclose%
\pgfusepath{fill}%
\end{pgfscope}%
\begin{pgfscope}%
\pgfpathrectangle{\pgfqpoint{6.572727in}{0.474100in}}{\pgfqpoint{4.227273in}{3.318700in}}%
\pgfusepath{clip}%
\pgfsetbuttcap%
\pgfsetroundjoin%
\definecolor{currentfill}{rgb}{0.127568,0.566949,0.550556}%
\pgfsetfillcolor{currentfill}%
\pgfsetfillopacity{0.700000}%
\pgfsetlinewidth{0.000000pt}%
\definecolor{currentstroke}{rgb}{0.000000,0.000000,0.000000}%
\pgfsetstrokecolor{currentstroke}%
\pgfsetstrokeopacity{0.700000}%
\pgfsetdash{}{0pt}%
\pgfpathmoveto{\pgfqpoint{7.970063in}{1.267306in}}%
\pgfpathcurveto{\pgfqpoint{7.975106in}{1.267306in}}{\pgfqpoint{7.979944in}{1.269310in}}{\pgfqpoint{7.983511in}{1.272877in}}%
\pgfpathcurveto{\pgfqpoint{7.987077in}{1.276443in}}{\pgfqpoint{7.989081in}{1.281281in}}{\pgfqpoint{7.989081in}{1.286324in}}%
\pgfpathcurveto{\pgfqpoint{7.989081in}{1.291368in}}{\pgfqpoint{7.987077in}{1.296206in}}{\pgfqpoint{7.983511in}{1.299772in}}%
\pgfpathcurveto{\pgfqpoint{7.979944in}{1.303339in}}{\pgfqpoint{7.975106in}{1.305343in}}{\pgfqpoint{7.970063in}{1.305343in}}%
\pgfpathcurveto{\pgfqpoint{7.965019in}{1.305343in}}{\pgfqpoint{7.960181in}{1.303339in}}{\pgfqpoint{7.956615in}{1.299772in}}%
\pgfpathcurveto{\pgfqpoint{7.953048in}{1.296206in}}{\pgfqpoint{7.951045in}{1.291368in}}{\pgfqpoint{7.951045in}{1.286324in}}%
\pgfpathcurveto{\pgfqpoint{7.951045in}{1.281281in}}{\pgfqpoint{7.953048in}{1.276443in}}{\pgfqpoint{7.956615in}{1.272877in}}%
\pgfpathcurveto{\pgfqpoint{7.960181in}{1.269310in}}{\pgfqpoint{7.965019in}{1.267306in}}{\pgfqpoint{7.970063in}{1.267306in}}%
\pgfpathclose%
\pgfusepath{fill}%
\end{pgfscope}%
\begin{pgfscope}%
\pgfpathrectangle{\pgfqpoint{6.572727in}{0.474100in}}{\pgfqpoint{4.227273in}{3.318700in}}%
\pgfusepath{clip}%
\pgfsetbuttcap%
\pgfsetroundjoin%
\definecolor{currentfill}{rgb}{0.127568,0.566949,0.550556}%
\pgfsetfillcolor{currentfill}%
\pgfsetfillopacity{0.700000}%
\pgfsetlinewidth{0.000000pt}%
\definecolor{currentstroke}{rgb}{0.000000,0.000000,0.000000}%
\pgfsetstrokecolor{currentstroke}%
\pgfsetstrokeopacity{0.700000}%
\pgfsetdash{}{0pt}%
\pgfpathmoveto{\pgfqpoint{7.979843in}{1.638995in}}%
\pgfpathcurveto{\pgfqpoint{7.984887in}{1.638995in}}{\pgfqpoint{7.989724in}{1.640999in}}{\pgfqpoint{7.993291in}{1.644565in}}%
\pgfpathcurveto{\pgfqpoint{7.996857in}{1.648132in}}{\pgfqpoint{7.998861in}{1.652970in}}{\pgfqpoint{7.998861in}{1.658013in}}%
\pgfpathcurveto{\pgfqpoint{7.998861in}{1.663057in}}{\pgfqpoint{7.996857in}{1.667895in}}{\pgfqpoint{7.993291in}{1.671461in}}%
\pgfpathcurveto{\pgfqpoint{7.989724in}{1.675028in}}{\pgfqpoint{7.984887in}{1.677031in}}{\pgfqpoint{7.979843in}{1.677031in}}%
\pgfpathcurveto{\pgfqpoint{7.974799in}{1.677031in}}{\pgfqpoint{7.969962in}{1.675028in}}{\pgfqpoint{7.966395in}{1.671461in}}%
\pgfpathcurveto{\pgfqpoint{7.962829in}{1.667895in}}{\pgfqpoint{7.960825in}{1.663057in}}{\pgfqpoint{7.960825in}{1.658013in}}%
\pgfpathcurveto{\pgfqpoint{7.960825in}{1.652970in}}{\pgfqpoint{7.962829in}{1.648132in}}{\pgfqpoint{7.966395in}{1.644565in}}%
\pgfpathcurveto{\pgfqpoint{7.969962in}{1.640999in}}{\pgfqpoint{7.974799in}{1.638995in}}{\pgfqpoint{7.979843in}{1.638995in}}%
\pgfpathclose%
\pgfusepath{fill}%
\end{pgfscope}%
\begin{pgfscope}%
\pgfpathrectangle{\pgfqpoint{6.572727in}{0.474100in}}{\pgfqpoint{4.227273in}{3.318700in}}%
\pgfusepath{clip}%
\pgfsetbuttcap%
\pgfsetroundjoin%
\definecolor{currentfill}{rgb}{0.127568,0.566949,0.550556}%
\pgfsetfillcolor{currentfill}%
\pgfsetfillopacity{0.700000}%
\pgfsetlinewidth{0.000000pt}%
\definecolor{currentstroke}{rgb}{0.000000,0.000000,0.000000}%
\pgfsetstrokecolor{currentstroke}%
\pgfsetstrokeopacity{0.700000}%
\pgfsetdash{}{0pt}%
\pgfpathmoveto{\pgfqpoint{7.864756in}{3.042516in}}%
\pgfpathcurveto{\pgfqpoint{7.869800in}{3.042516in}}{\pgfqpoint{7.874638in}{3.044520in}}{\pgfqpoint{7.878204in}{3.048087in}}%
\pgfpathcurveto{\pgfqpoint{7.881771in}{3.051653in}}{\pgfqpoint{7.883775in}{3.056491in}}{\pgfqpoint{7.883775in}{3.061535in}}%
\pgfpathcurveto{\pgfqpoint{7.883775in}{3.066578in}}{\pgfqpoint{7.881771in}{3.071416in}}{\pgfqpoint{7.878204in}{3.074982in}}%
\pgfpathcurveto{\pgfqpoint{7.874638in}{3.078549in}}{\pgfqpoint{7.869800in}{3.080553in}}{\pgfqpoint{7.864756in}{3.080553in}}%
\pgfpathcurveto{\pgfqpoint{7.859713in}{3.080553in}}{\pgfqpoint{7.854875in}{3.078549in}}{\pgfqpoint{7.851309in}{3.074982in}}%
\pgfpathcurveto{\pgfqpoint{7.847742in}{3.071416in}}{\pgfqpoint{7.845738in}{3.066578in}}{\pgfqpoint{7.845738in}{3.061535in}}%
\pgfpathcurveto{\pgfqpoint{7.845738in}{3.056491in}}{\pgfqpoint{7.847742in}{3.051653in}}{\pgfqpoint{7.851309in}{3.048087in}}%
\pgfpathcurveto{\pgfqpoint{7.854875in}{3.044520in}}{\pgfqpoint{7.859713in}{3.042516in}}{\pgfqpoint{7.864756in}{3.042516in}}%
\pgfpathclose%
\pgfusepath{fill}%
\end{pgfscope}%
\begin{pgfscope}%
\pgfpathrectangle{\pgfqpoint{6.572727in}{0.474100in}}{\pgfqpoint{4.227273in}{3.318700in}}%
\pgfusepath{clip}%
\pgfsetbuttcap%
\pgfsetroundjoin%
\definecolor{currentfill}{rgb}{0.993248,0.906157,0.143936}%
\pgfsetfillcolor{currentfill}%
\pgfsetfillopacity{0.700000}%
\pgfsetlinewidth{0.000000pt}%
\definecolor{currentstroke}{rgb}{0.000000,0.000000,0.000000}%
\pgfsetstrokecolor{currentstroke}%
\pgfsetstrokeopacity{0.700000}%
\pgfsetdash{}{0pt}%
\pgfpathmoveto{\pgfqpoint{9.403020in}{1.046733in}}%
\pgfpathcurveto{\pgfqpoint{9.408063in}{1.046733in}}{\pgfqpoint{9.412901in}{1.048737in}}{\pgfqpoint{9.416467in}{1.052303in}}%
\pgfpathcurveto{\pgfqpoint{9.420034in}{1.055870in}}{\pgfqpoint{9.422038in}{1.060708in}}{\pgfqpoint{9.422038in}{1.065751in}}%
\pgfpathcurveto{\pgfqpoint{9.422038in}{1.070795in}}{\pgfqpoint{9.420034in}{1.075633in}}{\pgfqpoint{9.416467in}{1.079199in}}%
\pgfpathcurveto{\pgfqpoint{9.412901in}{1.082766in}}{\pgfqpoint{9.408063in}{1.084769in}}{\pgfqpoint{9.403020in}{1.084769in}}%
\pgfpathcurveto{\pgfqpoint{9.397976in}{1.084769in}}{\pgfqpoint{9.393138in}{1.082766in}}{\pgfqpoint{9.389572in}{1.079199in}}%
\pgfpathcurveto{\pgfqpoint{9.386005in}{1.075633in}}{\pgfqpoint{9.384001in}{1.070795in}}{\pgfqpoint{9.384001in}{1.065751in}}%
\pgfpathcurveto{\pgfqpoint{9.384001in}{1.060708in}}{\pgfqpoint{9.386005in}{1.055870in}}{\pgfqpoint{9.389572in}{1.052303in}}%
\pgfpathcurveto{\pgfqpoint{9.393138in}{1.048737in}}{\pgfqpoint{9.397976in}{1.046733in}}{\pgfqpoint{9.403020in}{1.046733in}}%
\pgfpathclose%
\pgfusepath{fill}%
\end{pgfscope}%
\begin{pgfscope}%
\pgfpathrectangle{\pgfqpoint{6.572727in}{0.474100in}}{\pgfqpoint{4.227273in}{3.318700in}}%
\pgfusepath{clip}%
\pgfsetbuttcap%
\pgfsetroundjoin%
\definecolor{currentfill}{rgb}{0.127568,0.566949,0.550556}%
\pgfsetfillcolor{currentfill}%
\pgfsetfillopacity{0.700000}%
\pgfsetlinewidth{0.000000pt}%
\definecolor{currentstroke}{rgb}{0.000000,0.000000,0.000000}%
\pgfsetstrokecolor{currentstroke}%
\pgfsetstrokeopacity{0.700000}%
\pgfsetdash{}{0pt}%
\pgfpathmoveto{\pgfqpoint{7.918575in}{2.149727in}}%
\pgfpathcurveto{\pgfqpoint{7.923619in}{2.149727in}}{\pgfqpoint{7.928456in}{2.151731in}}{\pgfqpoint{7.932023in}{2.155297in}}%
\pgfpathcurveto{\pgfqpoint{7.935589in}{2.158864in}}{\pgfqpoint{7.937593in}{2.163702in}}{\pgfqpoint{7.937593in}{2.168745in}}%
\pgfpathcurveto{\pgfqpoint{7.937593in}{2.173789in}}{\pgfqpoint{7.935589in}{2.178627in}}{\pgfqpoint{7.932023in}{2.182193in}}%
\pgfpathcurveto{\pgfqpoint{7.928456in}{2.185759in}}{\pgfqpoint{7.923619in}{2.187763in}}{\pgfqpoint{7.918575in}{2.187763in}}%
\pgfpathcurveto{\pgfqpoint{7.913531in}{2.187763in}}{\pgfqpoint{7.908694in}{2.185759in}}{\pgfqpoint{7.905127in}{2.182193in}}%
\pgfpathcurveto{\pgfqpoint{7.901561in}{2.178627in}}{\pgfqpoint{7.899557in}{2.173789in}}{\pgfqpoint{7.899557in}{2.168745in}}%
\pgfpathcurveto{\pgfqpoint{7.899557in}{2.163702in}}{\pgfqpoint{7.901561in}{2.158864in}}{\pgfqpoint{7.905127in}{2.155297in}}%
\pgfpathcurveto{\pgfqpoint{7.908694in}{2.151731in}}{\pgfqpoint{7.913531in}{2.149727in}}{\pgfqpoint{7.918575in}{2.149727in}}%
\pgfpathclose%
\pgfusepath{fill}%
\end{pgfscope}%
\begin{pgfscope}%
\pgfpathrectangle{\pgfqpoint{6.572727in}{0.474100in}}{\pgfqpoint{4.227273in}{3.318700in}}%
\pgfusepath{clip}%
\pgfsetbuttcap%
\pgfsetroundjoin%
\definecolor{currentfill}{rgb}{0.127568,0.566949,0.550556}%
\pgfsetfillcolor{currentfill}%
\pgfsetfillopacity{0.700000}%
\pgfsetlinewidth{0.000000pt}%
\definecolor{currentstroke}{rgb}{0.000000,0.000000,0.000000}%
\pgfsetstrokecolor{currentstroke}%
\pgfsetstrokeopacity{0.700000}%
\pgfsetdash{}{0pt}%
\pgfpathmoveto{\pgfqpoint{8.004817in}{1.225383in}}%
\pgfpathcurveto{\pgfqpoint{8.009861in}{1.225383in}}{\pgfqpoint{8.014699in}{1.227387in}}{\pgfqpoint{8.018265in}{1.230953in}}%
\pgfpathcurveto{\pgfqpoint{8.021832in}{1.234520in}}{\pgfqpoint{8.023835in}{1.239357in}}{\pgfqpoint{8.023835in}{1.244401in}}%
\pgfpathcurveto{\pgfqpoint{8.023835in}{1.249445in}}{\pgfqpoint{8.021832in}{1.254283in}}{\pgfqpoint{8.018265in}{1.257849in}}%
\pgfpathcurveto{\pgfqpoint{8.014699in}{1.261415in}}{\pgfqpoint{8.009861in}{1.263419in}}{\pgfqpoint{8.004817in}{1.263419in}}%
\pgfpathcurveto{\pgfqpoint{7.999774in}{1.263419in}}{\pgfqpoint{7.994936in}{1.261415in}}{\pgfqpoint{7.991369in}{1.257849in}}%
\pgfpathcurveto{\pgfqpoint{7.987803in}{1.254283in}}{\pgfqpoint{7.985799in}{1.249445in}}{\pgfqpoint{7.985799in}{1.244401in}}%
\pgfpathcurveto{\pgfqpoint{7.985799in}{1.239357in}}{\pgfqpoint{7.987803in}{1.234520in}}{\pgfqpoint{7.991369in}{1.230953in}}%
\pgfpathcurveto{\pgfqpoint{7.994936in}{1.227387in}}{\pgfqpoint{7.999774in}{1.225383in}}{\pgfqpoint{8.004817in}{1.225383in}}%
\pgfpathclose%
\pgfusepath{fill}%
\end{pgfscope}%
\begin{pgfscope}%
\pgfpathrectangle{\pgfqpoint{6.572727in}{0.474100in}}{\pgfqpoint{4.227273in}{3.318700in}}%
\pgfusepath{clip}%
\pgfsetbuttcap%
\pgfsetroundjoin%
\definecolor{currentfill}{rgb}{0.127568,0.566949,0.550556}%
\pgfsetfillcolor{currentfill}%
\pgfsetfillopacity{0.700000}%
\pgfsetlinewidth{0.000000pt}%
\definecolor{currentstroke}{rgb}{0.000000,0.000000,0.000000}%
\pgfsetstrokecolor{currentstroke}%
\pgfsetstrokeopacity{0.700000}%
\pgfsetdash{}{0pt}%
\pgfpathmoveto{\pgfqpoint{8.196048in}{1.382449in}}%
\pgfpathcurveto{\pgfqpoint{8.201092in}{1.382449in}}{\pgfqpoint{8.205930in}{1.384453in}}{\pgfqpoint{8.209496in}{1.388019in}}%
\pgfpathcurveto{\pgfqpoint{8.213063in}{1.391585in}}{\pgfqpoint{8.215066in}{1.396423in}}{\pgfqpoint{8.215066in}{1.401467in}}%
\pgfpathcurveto{\pgfqpoint{8.215066in}{1.406511in}}{\pgfqpoint{8.213063in}{1.411348in}}{\pgfqpoint{8.209496in}{1.414915in}}%
\pgfpathcurveto{\pgfqpoint{8.205930in}{1.418481in}}{\pgfqpoint{8.201092in}{1.420485in}}{\pgfqpoint{8.196048in}{1.420485in}}%
\pgfpathcurveto{\pgfqpoint{8.191005in}{1.420485in}}{\pgfqpoint{8.186167in}{1.418481in}}{\pgfqpoint{8.182600in}{1.414915in}}%
\pgfpathcurveto{\pgfqpoint{8.179034in}{1.411348in}}{\pgfqpoint{8.177030in}{1.406511in}}{\pgfqpoint{8.177030in}{1.401467in}}%
\pgfpathcurveto{\pgfqpoint{8.177030in}{1.396423in}}{\pgfqpoint{8.179034in}{1.391585in}}{\pgfqpoint{8.182600in}{1.388019in}}%
\pgfpathcurveto{\pgfqpoint{8.186167in}{1.384453in}}{\pgfqpoint{8.191005in}{1.382449in}}{\pgfqpoint{8.196048in}{1.382449in}}%
\pgfpathclose%
\pgfusepath{fill}%
\end{pgfscope}%
\begin{pgfscope}%
\pgfpathrectangle{\pgfqpoint{6.572727in}{0.474100in}}{\pgfqpoint{4.227273in}{3.318700in}}%
\pgfusepath{clip}%
\pgfsetbuttcap%
\pgfsetroundjoin%
\definecolor{currentfill}{rgb}{0.127568,0.566949,0.550556}%
\pgfsetfillcolor{currentfill}%
\pgfsetfillopacity{0.700000}%
\pgfsetlinewidth{0.000000pt}%
\definecolor{currentstroke}{rgb}{0.000000,0.000000,0.000000}%
\pgfsetstrokecolor{currentstroke}%
\pgfsetstrokeopacity{0.700000}%
\pgfsetdash{}{0pt}%
\pgfpathmoveto{\pgfqpoint{8.345686in}{2.515201in}}%
\pgfpathcurveto{\pgfqpoint{8.350730in}{2.515201in}}{\pgfqpoint{8.355568in}{2.517205in}}{\pgfqpoint{8.359134in}{2.520771in}}%
\pgfpathcurveto{\pgfqpoint{8.362700in}{2.524338in}}{\pgfqpoint{8.364704in}{2.529175in}}{\pgfqpoint{8.364704in}{2.534219in}}%
\pgfpathcurveto{\pgfqpoint{8.364704in}{2.539263in}}{\pgfqpoint{8.362700in}{2.544101in}}{\pgfqpoint{8.359134in}{2.547667in}}%
\pgfpathcurveto{\pgfqpoint{8.355568in}{2.551233in}}{\pgfqpoint{8.350730in}{2.553237in}}{\pgfqpoint{8.345686in}{2.553237in}}%
\pgfpathcurveto{\pgfqpoint{8.340642in}{2.553237in}}{\pgfqpoint{8.335805in}{2.551233in}}{\pgfqpoint{8.332238in}{2.547667in}}%
\pgfpathcurveto{\pgfqpoint{8.328672in}{2.544101in}}{\pgfqpoint{8.326668in}{2.539263in}}{\pgfqpoint{8.326668in}{2.534219in}}%
\pgfpathcurveto{\pgfqpoint{8.326668in}{2.529175in}}{\pgfqpoint{8.328672in}{2.524338in}}{\pgfqpoint{8.332238in}{2.520771in}}%
\pgfpathcurveto{\pgfqpoint{8.335805in}{2.517205in}}{\pgfqpoint{8.340642in}{2.515201in}}{\pgfqpoint{8.345686in}{2.515201in}}%
\pgfpathclose%
\pgfusepath{fill}%
\end{pgfscope}%
\begin{pgfscope}%
\pgfpathrectangle{\pgfqpoint{6.572727in}{0.474100in}}{\pgfqpoint{4.227273in}{3.318700in}}%
\pgfusepath{clip}%
\pgfsetbuttcap%
\pgfsetroundjoin%
\definecolor{currentfill}{rgb}{0.993248,0.906157,0.143936}%
\pgfsetfillcolor{currentfill}%
\pgfsetfillopacity{0.700000}%
\pgfsetlinewidth{0.000000pt}%
\definecolor{currentstroke}{rgb}{0.000000,0.000000,0.000000}%
\pgfsetstrokecolor{currentstroke}%
\pgfsetstrokeopacity{0.700000}%
\pgfsetdash{}{0pt}%
\pgfpathmoveto{\pgfqpoint{9.600133in}{1.406614in}}%
\pgfpathcurveto{\pgfqpoint{9.605177in}{1.406614in}}{\pgfqpoint{9.610015in}{1.408618in}}{\pgfqpoint{9.613581in}{1.412185in}}%
\pgfpathcurveto{\pgfqpoint{9.617147in}{1.415751in}}{\pgfqpoint{9.619151in}{1.420589in}}{\pgfqpoint{9.619151in}{1.425632in}}%
\pgfpathcurveto{\pgfqpoint{9.619151in}{1.430676in}}{\pgfqpoint{9.617147in}{1.435514in}}{\pgfqpoint{9.613581in}{1.439080in}}%
\pgfpathcurveto{\pgfqpoint{9.610015in}{1.442647in}}{\pgfqpoint{9.605177in}{1.444651in}}{\pgfqpoint{9.600133in}{1.444651in}}%
\pgfpathcurveto{\pgfqpoint{9.595089in}{1.444651in}}{\pgfqpoint{9.590252in}{1.442647in}}{\pgfqpoint{9.586685in}{1.439080in}}%
\pgfpathcurveto{\pgfqpoint{9.583119in}{1.435514in}}{\pgfqpoint{9.581115in}{1.430676in}}{\pgfqpoint{9.581115in}{1.425632in}}%
\pgfpathcurveto{\pgfqpoint{9.581115in}{1.420589in}}{\pgfqpoint{9.583119in}{1.415751in}}{\pgfqpoint{9.586685in}{1.412185in}}%
\pgfpathcurveto{\pgfqpoint{9.590252in}{1.408618in}}{\pgfqpoint{9.595089in}{1.406614in}}{\pgfqpoint{9.600133in}{1.406614in}}%
\pgfpathclose%
\pgfusepath{fill}%
\end{pgfscope}%
\begin{pgfscope}%
\pgfpathrectangle{\pgfqpoint{6.572727in}{0.474100in}}{\pgfqpoint{4.227273in}{3.318700in}}%
\pgfusepath{clip}%
\pgfsetbuttcap%
\pgfsetroundjoin%
\definecolor{currentfill}{rgb}{0.127568,0.566949,0.550556}%
\pgfsetfillcolor{currentfill}%
\pgfsetfillopacity{0.700000}%
\pgfsetlinewidth{0.000000pt}%
\definecolor{currentstroke}{rgb}{0.000000,0.000000,0.000000}%
\pgfsetstrokecolor{currentstroke}%
\pgfsetstrokeopacity{0.700000}%
\pgfsetdash{}{0pt}%
\pgfpathmoveto{\pgfqpoint{8.391248in}{1.036682in}}%
\pgfpathcurveto{\pgfqpoint{8.396292in}{1.036682in}}{\pgfqpoint{8.401129in}{1.038685in}}{\pgfqpoint{8.404696in}{1.042252in}}%
\pgfpathcurveto{\pgfqpoint{8.408262in}{1.045818in}}{\pgfqpoint{8.410266in}{1.050656in}}{\pgfqpoint{8.410266in}{1.055700in}}%
\pgfpathcurveto{\pgfqpoint{8.410266in}{1.060743in}}{\pgfqpoint{8.408262in}{1.065581in}}{\pgfqpoint{8.404696in}{1.069148in}}%
\pgfpathcurveto{\pgfqpoint{8.401129in}{1.072714in}}{\pgfqpoint{8.396292in}{1.074718in}}{\pgfqpoint{8.391248in}{1.074718in}}%
\pgfpathcurveto{\pgfqpoint{8.386204in}{1.074718in}}{\pgfqpoint{8.381366in}{1.072714in}}{\pgfqpoint{8.377800in}{1.069148in}}%
\pgfpathcurveto{\pgfqpoint{8.374234in}{1.065581in}}{\pgfqpoint{8.372230in}{1.060743in}}{\pgfqpoint{8.372230in}{1.055700in}}%
\pgfpathcurveto{\pgfqpoint{8.372230in}{1.050656in}}{\pgfqpoint{8.374234in}{1.045818in}}{\pgfqpoint{8.377800in}{1.042252in}}%
\pgfpathcurveto{\pgfqpoint{8.381366in}{1.038685in}}{\pgfqpoint{8.386204in}{1.036682in}}{\pgfqpoint{8.391248in}{1.036682in}}%
\pgfpathclose%
\pgfusepath{fill}%
\end{pgfscope}%
\begin{pgfscope}%
\pgfpathrectangle{\pgfqpoint{6.572727in}{0.474100in}}{\pgfqpoint{4.227273in}{3.318700in}}%
\pgfusepath{clip}%
\pgfsetbuttcap%
\pgfsetroundjoin%
\definecolor{currentfill}{rgb}{0.993248,0.906157,0.143936}%
\pgfsetfillcolor{currentfill}%
\pgfsetfillopacity{0.700000}%
\pgfsetlinewidth{0.000000pt}%
\definecolor{currentstroke}{rgb}{0.000000,0.000000,0.000000}%
\pgfsetstrokecolor{currentstroke}%
\pgfsetstrokeopacity{0.700000}%
\pgfsetdash{}{0pt}%
\pgfpathmoveto{\pgfqpoint{9.446347in}{1.380990in}}%
\pgfpathcurveto{\pgfqpoint{9.451391in}{1.380990in}}{\pgfqpoint{9.456229in}{1.382994in}}{\pgfqpoint{9.459795in}{1.386560in}}%
\pgfpathcurveto{\pgfqpoint{9.463362in}{1.390127in}}{\pgfqpoint{9.465366in}{1.394964in}}{\pgfqpoint{9.465366in}{1.400008in}}%
\pgfpathcurveto{\pgfqpoint{9.465366in}{1.405052in}}{\pgfqpoint{9.463362in}{1.409889in}}{\pgfqpoint{9.459795in}{1.413456in}}%
\pgfpathcurveto{\pgfqpoint{9.456229in}{1.417022in}}{\pgfqpoint{9.451391in}{1.419026in}}{\pgfqpoint{9.446347in}{1.419026in}}%
\pgfpathcurveto{\pgfqpoint{9.441304in}{1.419026in}}{\pgfqpoint{9.436466in}{1.417022in}}{\pgfqpoint{9.432900in}{1.413456in}}%
\pgfpathcurveto{\pgfqpoint{9.429333in}{1.409889in}}{\pgfqpoint{9.427329in}{1.405052in}}{\pgfqpoint{9.427329in}{1.400008in}}%
\pgfpathcurveto{\pgfqpoint{9.427329in}{1.394964in}}{\pgfqpoint{9.429333in}{1.390127in}}{\pgfqpoint{9.432900in}{1.386560in}}%
\pgfpathcurveto{\pgfqpoint{9.436466in}{1.382994in}}{\pgfqpoint{9.441304in}{1.380990in}}{\pgfqpoint{9.446347in}{1.380990in}}%
\pgfpathclose%
\pgfusepath{fill}%
\end{pgfscope}%
\begin{pgfscope}%
\pgfpathrectangle{\pgfqpoint{6.572727in}{0.474100in}}{\pgfqpoint{4.227273in}{3.318700in}}%
\pgfusepath{clip}%
\pgfsetbuttcap%
\pgfsetroundjoin%
\definecolor{currentfill}{rgb}{0.127568,0.566949,0.550556}%
\pgfsetfillcolor{currentfill}%
\pgfsetfillopacity{0.700000}%
\pgfsetlinewidth{0.000000pt}%
\definecolor{currentstroke}{rgb}{0.000000,0.000000,0.000000}%
\pgfsetstrokecolor{currentstroke}%
\pgfsetstrokeopacity{0.700000}%
\pgfsetdash{}{0pt}%
\pgfpathmoveto{\pgfqpoint{8.137568in}{2.558601in}}%
\pgfpathcurveto{\pgfqpoint{8.142612in}{2.558601in}}{\pgfqpoint{8.147450in}{2.560605in}}{\pgfqpoint{8.151016in}{2.564171in}}%
\pgfpathcurveto{\pgfqpoint{8.154582in}{2.567738in}}{\pgfqpoint{8.156586in}{2.572575in}}{\pgfqpoint{8.156586in}{2.577619in}}%
\pgfpathcurveto{\pgfqpoint{8.156586in}{2.582663in}}{\pgfqpoint{8.154582in}{2.587500in}}{\pgfqpoint{8.151016in}{2.591067in}}%
\pgfpathcurveto{\pgfqpoint{8.147450in}{2.594633in}}{\pgfqpoint{8.142612in}{2.596637in}}{\pgfqpoint{8.137568in}{2.596637in}}%
\pgfpathcurveto{\pgfqpoint{8.132524in}{2.596637in}}{\pgfqpoint{8.127687in}{2.594633in}}{\pgfqpoint{8.124120in}{2.591067in}}%
\pgfpathcurveto{\pgfqpoint{8.120554in}{2.587500in}}{\pgfqpoint{8.118550in}{2.582663in}}{\pgfqpoint{8.118550in}{2.577619in}}%
\pgfpathcurveto{\pgfqpoint{8.118550in}{2.572575in}}{\pgfqpoint{8.120554in}{2.567738in}}{\pgfqpoint{8.124120in}{2.564171in}}%
\pgfpathcurveto{\pgfqpoint{8.127687in}{2.560605in}}{\pgfqpoint{8.132524in}{2.558601in}}{\pgfqpoint{8.137568in}{2.558601in}}%
\pgfpathclose%
\pgfusepath{fill}%
\end{pgfscope}%
\begin{pgfscope}%
\pgfpathrectangle{\pgfqpoint{6.572727in}{0.474100in}}{\pgfqpoint{4.227273in}{3.318700in}}%
\pgfusepath{clip}%
\pgfsetbuttcap%
\pgfsetroundjoin%
\definecolor{currentfill}{rgb}{0.127568,0.566949,0.550556}%
\pgfsetfillcolor{currentfill}%
\pgfsetfillopacity{0.700000}%
\pgfsetlinewidth{0.000000pt}%
\definecolor{currentstroke}{rgb}{0.000000,0.000000,0.000000}%
\pgfsetstrokecolor{currentstroke}%
\pgfsetstrokeopacity{0.700000}%
\pgfsetdash{}{0pt}%
\pgfpathmoveto{\pgfqpoint{8.005069in}{2.682855in}}%
\pgfpathcurveto{\pgfqpoint{8.010113in}{2.682855in}}{\pgfqpoint{8.014950in}{2.684859in}}{\pgfqpoint{8.018517in}{2.688425in}}%
\pgfpathcurveto{\pgfqpoint{8.022083in}{2.691992in}}{\pgfqpoint{8.024087in}{2.696829in}}{\pgfqpoint{8.024087in}{2.701873in}}%
\pgfpathcurveto{\pgfqpoint{8.024087in}{2.706917in}}{\pgfqpoint{8.022083in}{2.711755in}}{\pgfqpoint{8.018517in}{2.715321in}}%
\pgfpathcurveto{\pgfqpoint{8.014950in}{2.718887in}}{\pgfqpoint{8.010113in}{2.720891in}}{\pgfqpoint{8.005069in}{2.720891in}}%
\pgfpathcurveto{\pgfqpoint{8.000025in}{2.720891in}}{\pgfqpoint{7.995188in}{2.718887in}}{\pgfqpoint{7.991621in}{2.715321in}}%
\pgfpathcurveto{\pgfqpoint{7.988055in}{2.711755in}}{\pgfqpoint{7.986051in}{2.706917in}}{\pgfqpoint{7.986051in}{2.701873in}}%
\pgfpathcurveto{\pgfqpoint{7.986051in}{2.696829in}}{\pgfqpoint{7.988055in}{2.691992in}}{\pgfqpoint{7.991621in}{2.688425in}}%
\pgfpathcurveto{\pgfqpoint{7.995188in}{2.684859in}}{\pgfqpoint{8.000025in}{2.682855in}}{\pgfqpoint{8.005069in}{2.682855in}}%
\pgfpathclose%
\pgfusepath{fill}%
\end{pgfscope}%
\begin{pgfscope}%
\pgfpathrectangle{\pgfqpoint{6.572727in}{0.474100in}}{\pgfqpoint{4.227273in}{3.318700in}}%
\pgfusepath{clip}%
\pgfsetbuttcap%
\pgfsetroundjoin%
\definecolor{currentfill}{rgb}{0.127568,0.566949,0.550556}%
\pgfsetfillcolor{currentfill}%
\pgfsetfillopacity{0.700000}%
\pgfsetlinewidth{0.000000pt}%
\definecolor{currentstroke}{rgb}{0.000000,0.000000,0.000000}%
\pgfsetstrokecolor{currentstroke}%
\pgfsetstrokeopacity{0.700000}%
\pgfsetdash{}{0pt}%
\pgfpathmoveto{\pgfqpoint{8.138261in}{1.416787in}}%
\pgfpathcurveto{\pgfqpoint{8.143304in}{1.416787in}}{\pgfqpoint{8.148142in}{1.418791in}}{\pgfqpoint{8.151709in}{1.422358in}}%
\pgfpathcurveto{\pgfqpoint{8.155275in}{1.425924in}}{\pgfqpoint{8.157279in}{1.430762in}}{\pgfqpoint{8.157279in}{1.435805in}}%
\pgfpathcurveto{\pgfqpoint{8.157279in}{1.440849in}}{\pgfqpoint{8.155275in}{1.445687in}}{\pgfqpoint{8.151709in}{1.449253in}}%
\pgfpathcurveto{\pgfqpoint{8.148142in}{1.452820in}}{\pgfqpoint{8.143304in}{1.454824in}}{\pgfqpoint{8.138261in}{1.454824in}}%
\pgfpathcurveto{\pgfqpoint{8.133217in}{1.454824in}}{\pgfqpoint{8.128379in}{1.452820in}}{\pgfqpoint{8.124813in}{1.449253in}}%
\pgfpathcurveto{\pgfqpoint{8.121247in}{1.445687in}}{\pgfqpoint{8.119243in}{1.440849in}}{\pgfqpoint{8.119243in}{1.435805in}}%
\pgfpathcurveto{\pgfqpoint{8.119243in}{1.430762in}}{\pgfqpoint{8.121247in}{1.425924in}}{\pgfqpoint{8.124813in}{1.422358in}}%
\pgfpathcurveto{\pgfqpoint{8.128379in}{1.418791in}}{\pgfqpoint{8.133217in}{1.416787in}}{\pgfqpoint{8.138261in}{1.416787in}}%
\pgfpathclose%
\pgfusepath{fill}%
\end{pgfscope}%
\begin{pgfscope}%
\pgfpathrectangle{\pgfqpoint{6.572727in}{0.474100in}}{\pgfqpoint{4.227273in}{3.318700in}}%
\pgfusepath{clip}%
\pgfsetbuttcap%
\pgfsetroundjoin%
\definecolor{currentfill}{rgb}{0.127568,0.566949,0.550556}%
\pgfsetfillcolor{currentfill}%
\pgfsetfillopacity{0.700000}%
\pgfsetlinewidth{0.000000pt}%
\definecolor{currentstroke}{rgb}{0.000000,0.000000,0.000000}%
\pgfsetstrokecolor{currentstroke}%
\pgfsetstrokeopacity{0.700000}%
\pgfsetdash{}{0pt}%
\pgfpathmoveto{\pgfqpoint{8.041117in}{2.656134in}}%
\pgfpathcurveto{\pgfqpoint{8.046161in}{2.656134in}}{\pgfqpoint{8.050999in}{2.658138in}}{\pgfqpoint{8.054565in}{2.661704in}}%
\pgfpathcurveto{\pgfqpoint{8.058132in}{2.665271in}}{\pgfqpoint{8.060136in}{2.670109in}}{\pgfqpoint{8.060136in}{2.675152in}}%
\pgfpathcurveto{\pgfqpoint{8.060136in}{2.680196in}}{\pgfqpoint{8.058132in}{2.685034in}}{\pgfqpoint{8.054565in}{2.688600in}}%
\pgfpathcurveto{\pgfqpoint{8.050999in}{2.692167in}}{\pgfqpoint{8.046161in}{2.694170in}}{\pgfqpoint{8.041117in}{2.694170in}}%
\pgfpathcurveto{\pgfqpoint{8.036074in}{2.694170in}}{\pgfqpoint{8.031236in}{2.692167in}}{\pgfqpoint{8.027670in}{2.688600in}}%
\pgfpathcurveto{\pgfqpoint{8.024103in}{2.685034in}}{\pgfqpoint{8.022099in}{2.680196in}}{\pgfqpoint{8.022099in}{2.675152in}}%
\pgfpathcurveto{\pgfqpoint{8.022099in}{2.670109in}}{\pgfqpoint{8.024103in}{2.665271in}}{\pgfqpoint{8.027670in}{2.661704in}}%
\pgfpathcurveto{\pgfqpoint{8.031236in}{2.658138in}}{\pgfqpoint{8.036074in}{2.656134in}}{\pgfqpoint{8.041117in}{2.656134in}}%
\pgfpathclose%
\pgfusepath{fill}%
\end{pgfscope}%
\begin{pgfscope}%
\pgfpathrectangle{\pgfqpoint{6.572727in}{0.474100in}}{\pgfqpoint{4.227273in}{3.318700in}}%
\pgfusepath{clip}%
\pgfsetbuttcap%
\pgfsetroundjoin%
\definecolor{currentfill}{rgb}{0.127568,0.566949,0.550556}%
\pgfsetfillcolor{currentfill}%
\pgfsetfillopacity{0.700000}%
\pgfsetlinewidth{0.000000pt}%
\definecolor{currentstroke}{rgb}{0.000000,0.000000,0.000000}%
\pgfsetstrokecolor{currentstroke}%
\pgfsetstrokeopacity{0.700000}%
\pgfsetdash{}{0pt}%
\pgfpathmoveto{\pgfqpoint{8.234294in}{1.491787in}}%
\pgfpathcurveto{\pgfqpoint{8.239338in}{1.491787in}}{\pgfqpoint{8.244176in}{1.493791in}}{\pgfqpoint{8.247742in}{1.497357in}}%
\pgfpathcurveto{\pgfqpoint{8.251309in}{1.500924in}}{\pgfqpoint{8.253313in}{1.505762in}}{\pgfqpoint{8.253313in}{1.510805in}}%
\pgfpathcurveto{\pgfqpoint{8.253313in}{1.515849in}}{\pgfqpoint{8.251309in}{1.520687in}}{\pgfqpoint{8.247742in}{1.524253in}}%
\pgfpathcurveto{\pgfqpoint{8.244176in}{1.527820in}}{\pgfqpoint{8.239338in}{1.529823in}}{\pgfqpoint{8.234294in}{1.529823in}}%
\pgfpathcurveto{\pgfqpoint{8.229251in}{1.529823in}}{\pgfqpoint{8.224413in}{1.527820in}}{\pgfqpoint{8.220847in}{1.524253in}}%
\pgfpathcurveto{\pgfqpoint{8.217280in}{1.520687in}}{\pgfqpoint{8.215276in}{1.515849in}}{\pgfqpoint{8.215276in}{1.510805in}}%
\pgfpathcurveto{\pgfqpoint{8.215276in}{1.505762in}}{\pgfqpoint{8.217280in}{1.500924in}}{\pgfqpoint{8.220847in}{1.497357in}}%
\pgfpathcurveto{\pgfqpoint{8.224413in}{1.493791in}}{\pgfqpoint{8.229251in}{1.491787in}}{\pgfqpoint{8.234294in}{1.491787in}}%
\pgfpathclose%
\pgfusepath{fill}%
\end{pgfscope}%
\begin{pgfscope}%
\pgfpathrectangle{\pgfqpoint{6.572727in}{0.474100in}}{\pgfqpoint{4.227273in}{3.318700in}}%
\pgfusepath{clip}%
\pgfsetbuttcap%
\pgfsetroundjoin%
\definecolor{currentfill}{rgb}{0.127568,0.566949,0.550556}%
\pgfsetfillcolor{currentfill}%
\pgfsetfillopacity{0.700000}%
\pgfsetlinewidth{0.000000pt}%
\definecolor{currentstroke}{rgb}{0.000000,0.000000,0.000000}%
\pgfsetstrokecolor{currentstroke}%
\pgfsetstrokeopacity{0.700000}%
\pgfsetdash{}{0pt}%
\pgfpathmoveto{\pgfqpoint{7.417881in}{2.117797in}}%
\pgfpathcurveto{\pgfqpoint{7.422924in}{2.117797in}}{\pgfqpoint{7.427762in}{2.119801in}}{\pgfqpoint{7.431329in}{2.123368in}}%
\pgfpathcurveto{\pgfqpoint{7.434895in}{2.126934in}}{\pgfqpoint{7.436899in}{2.131772in}}{\pgfqpoint{7.436899in}{2.136815in}}%
\pgfpathcurveto{\pgfqpoint{7.436899in}{2.141859in}}{\pgfqpoint{7.434895in}{2.146697in}}{\pgfqpoint{7.431329in}{2.150263in}}%
\pgfpathcurveto{\pgfqpoint{7.427762in}{2.153830in}}{\pgfqpoint{7.422924in}{2.155834in}}{\pgfqpoint{7.417881in}{2.155834in}}%
\pgfpathcurveto{\pgfqpoint{7.412837in}{2.155834in}}{\pgfqpoint{7.407999in}{2.153830in}}{\pgfqpoint{7.404433in}{2.150263in}}%
\pgfpathcurveto{\pgfqpoint{7.400866in}{2.146697in}}{\pgfqpoint{7.398863in}{2.141859in}}{\pgfqpoint{7.398863in}{2.136815in}}%
\pgfpathcurveto{\pgfqpoint{7.398863in}{2.131772in}}{\pgfqpoint{7.400866in}{2.126934in}}{\pgfqpoint{7.404433in}{2.123368in}}%
\pgfpathcurveto{\pgfqpoint{7.407999in}{2.119801in}}{\pgfqpoint{7.412837in}{2.117797in}}{\pgfqpoint{7.417881in}{2.117797in}}%
\pgfpathclose%
\pgfusepath{fill}%
\end{pgfscope}%
\begin{pgfscope}%
\pgfpathrectangle{\pgfqpoint{6.572727in}{0.474100in}}{\pgfqpoint{4.227273in}{3.318700in}}%
\pgfusepath{clip}%
\pgfsetbuttcap%
\pgfsetroundjoin%
\definecolor{currentfill}{rgb}{0.127568,0.566949,0.550556}%
\pgfsetfillcolor{currentfill}%
\pgfsetfillopacity{0.700000}%
\pgfsetlinewidth{0.000000pt}%
\definecolor{currentstroke}{rgb}{0.000000,0.000000,0.000000}%
\pgfsetstrokecolor{currentstroke}%
\pgfsetstrokeopacity{0.700000}%
\pgfsetdash{}{0pt}%
\pgfpathmoveto{\pgfqpoint{7.862585in}{1.140430in}}%
\pgfpathcurveto{\pgfqpoint{7.867629in}{1.140430in}}{\pgfqpoint{7.872467in}{1.142434in}}{\pgfqpoint{7.876033in}{1.146001in}}%
\pgfpathcurveto{\pgfqpoint{7.879600in}{1.149567in}}{\pgfqpoint{7.881604in}{1.154405in}}{\pgfqpoint{7.881604in}{1.159448in}}%
\pgfpathcurveto{\pgfqpoint{7.881604in}{1.164492in}}{\pgfqpoint{7.879600in}{1.169330in}}{\pgfqpoint{7.876033in}{1.172896in}}%
\pgfpathcurveto{\pgfqpoint{7.872467in}{1.176463in}}{\pgfqpoint{7.867629in}{1.178467in}}{\pgfqpoint{7.862585in}{1.178467in}}%
\pgfpathcurveto{\pgfqpoint{7.857542in}{1.178467in}}{\pgfqpoint{7.852704in}{1.176463in}}{\pgfqpoint{7.849138in}{1.172896in}}%
\pgfpathcurveto{\pgfqpoint{7.845571in}{1.169330in}}{\pgfqpoint{7.843567in}{1.164492in}}{\pgfqpoint{7.843567in}{1.159448in}}%
\pgfpathcurveto{\pgfqpoint{7.843567in}{1.154405in}}{\pgfqpoint{7.845571in}{1.149567in}}{\pgfqpoint{7.849138in}{1.146001in}}%
\pgfpathcurveto{\pgfqpoint{7.852704in}{1.142434in}}{\pgfqpoint{7.857542in}{1.140430in}}{\pgfqpoint{7.862585in}{1.140430in}}%
\pgfpathclose%
\pgfusepath{fill}%
\end{pgfscope}%
\begin{pgfscope}%
\pgfpathrectangle{\pgfqpoint{6.572727in}{0.474100in}}{\pgfqpoint{4.227273in}{3.318700in}}%
\pgfusepath{clip}%
\pgfsetbuttcap%
\pgfsetroundjoin%
\definecolor{currentfill}{rgb}{0.127568,0.566949,0.550556}%
\pgfsetfillcolor{currentfill}%
\pgfsetfillopacity{0.700000}%
\pgfsetlinewidth{0.000000pt}%
\definecolor{currentstroke}{rgb}{0.000000,0.000000,0.000000}%
\pgfsetstrokecolor{currentstroke}%
\pgfsetstrokeopacity{0.700000}%
\pgfsetdash{}{0pt}%
\pgfpathmoveto{\pgfqpoint{8.169487in}{1.757332in}}%
\pgfpathcurveto{\pgfqpoint{8.174530in}{1.757332in}}{\pgfqpoint{8.179368in}{1.759336in}}{\pgfqpoint{8.182935in}{1.762902in}}%
\pgfpathcurveto{\pgfqpoint{8.186501in}{1.766468in}}{\pgfqpoint{8.188505in}{1.771306in}}{\pgfqpoint{8.188505in}{1.776350in}}%
\pgfpathcurveto{\pgfqpoint{8.188505in}{1.781393in}}{\pgfqpoint{8.186501in}{1.786231in}}{\pgfqpoint{8.182935in}{1.789798in}}%
\pgfpathcurveto{\pgfqpoint{8.179368in}{1.793364in}}{\pgfqpoint{8.174530in}{1.795368in}}{\pgfqpoint{8.169487in}{1.795368in}}%
\pgfpathcurveto{\pgfqpoint{8.164443in}{1.795368in}}{\pgfqpoint{8.159605in}{1.793364in}}{\pgfqpoint{8.156039in}{1.789798in}}%
\pgfpathcurveto{\pgfqpoint{8.152473in}{1.786231in}}{\pgfqpoint{8.150469in}{1.781393in}}{\pgfqpoint{8.150469in}{1.776350in}}%
\pgfpathcurveto{\pgfqpoint{8.150469in}{1.771306in}}{\pgfqpoint{8.152473in}{1.766468in}}{\pgfqpoint{8.156039in}{1.762902in}}%
\pgfpathcurveto{\pgfqpoint{8.159605in}{1.759336in}}{\pgfqpoint{8.164443in}{1.757332in}}{\pgfqpoint{8.169487in}{1.757332in}}%
\pgfpathclose%
\pgfusepath{fill}%
\end{pgfscope}%
\begin{pgfscope}%
\pgfpathrectangle{\pgfqpoint{6.572727in}{0.474100in}}{\pgfqpoint{4.227273in}{3.318700in}}%
\pgfusepath{clip}%
\pgfsetbuttcap%
\pgfsetroundjoin%
\definecolor{currentfill}{rgb}{0.127568,0.566949,0.550556}%
\pgfsetfillcolor{currentfill}%
\pgfsetfillopacity{0.700000}%
\pgfsetlinewidth{0.000000pt}%
\definecolor{currentstroke}{rgb}{0.000000,0.000000,0.000000}%
\pgfsetstrokecolor{currentstroke}%
\pgfsetstrokeopacity{0.700000}%
\pgfsetdash{}{0pt}%
\pgfpathmoveto{\pgfqpoint{7.683748in}{1.933337in}}%
\pgfpathcurveto{\pgfqpoint{7.688792in}{1.933337in}}{\pgfqpoint{7.693629in}{1.935341in}}{\pgfqpoint{7.697196in}{1.938907in}}%
\pgfpathcurveto{\pgfqpoint{7.700762in}{1.942474in}}{\pgfqpoint{7.702766in}{1.947311in}}{\pgfqpoint{7.702766in}{1.952355in}}%
\pgfpathcurveto{\pgfqpoint{7.702766in}{1.957399in}}{\pgfqpoint{7.700762in}{1.962236in}}{\pgfqpoint{7.697196in}{1.965803in}}%
\pgfpathcurveto{\pgfqpoint{7.693629in}{1.969369in}}{\pgfqpoint{7.688792in}{1.971373in}}{\pgfqpoint{7.683748in}{1.971373in}}%
\pgfpathcurveto{\pgfqpoint{7.678704in}{1.971373in}}{\pgfqpoint{7.673867in}{1.969369in}}{\pgfqpoint{7.670300in}{1.965803in}}%
\pgfpathcurveto{\pgfqpoint{7.666734in}{1.962236in}}{\pgfqpoint{7.664730in}{1.957399in}}{\pgfqpoint{7.664730in}{1.952355in}}%
\pgfpathcurveto{\pgfqpoint{7.664730in}{1.947311in}}{\pgfqpoint{7.666734in}{1.942474in}}{\pgfqpoint{7.670300in}{1.938907in}}%
\pgfpathcurveto{\pgfqpoint{7.673867in}{1.935341in}}{\pgfqpoint{7.678704in}{1.933337in}}{\pgfqpoint{7.683748in}{1.933337in}}%
\pgfpathclose%
\pgfusepath{fill}%
\end{pgfscope}%
\begin{pgfscope}%
\pgfpathrectangle{\pgfqpoint{6.572727in}{0.474100in}}{\pgfqpoint{4.227273in}{3.318700in}}%
\pgfusepath{clip}%
\pgfsetbuttcap%
\pgfsetroundjoin%
\definecolor{currentfill}{rgb}{0.993248,0.906157,0.143936}%
\pgfsetfillcolor{currentfill}%
\pgfsetfillopacity{0.700000}%
\pgfsetlinewidth{0.000000pt}%
\definecolor{currentstroke}{rgb}{0.000000,0.000000,0.000000}%
\pgfsetstrokecolor{currentstroke}%
\pgfsetstrokeopacity{0.700000}%
\pgfsetdash{}{0pt}%
\pgfpathmoveto{\pgfqpoint{9.038013in}{1.557534in}}%
\pgfpathcurveto{\pgfqpoint{9.043056in}{1.557534in}}{\pgfqpoint{9.047894in}{1.559538in}}{\pgfqpoint{9.051461in}{1.563104in}}%
\pgfpathcurveto{\pgfqpoint{9.055027in}{1.566671in}}{\pgfqpoint{9.057031in}{1.571508in}}{\pgfqpoint{9.057031in}{1.576552in}}%
\pgfpathcurveto{\pgfqpoint{9.057031in}{1.581596in}}{\pgfqpoint{9.055027in}{1.586434in}}{\pgfqpoint{9.051461in}{1.590000in}}%
\pgfpathcurveto{\pgfqpoint{9.047894in}{1.593566in}}{\pgfqpoint{9.043056in}{1.595570in}}{\pgfqpoint{9.038013in}{1.595570in}}%
\pgfpathcurveto{\pgfqpoint{9.032969in}{1.595570in}}{\pgfqpoint{9.028131in}{1.593566in}}{\pgfqpoint{9.024565in}{1.590000in}}%
\pgfpathcurveto{\pgfqpoint{9.020998in}{1.586434in}}{\pgfqpoint{9.018995in}{1.581596in}}{\pgfqpoint{9.018995in}{1.576552in}}%
\pgfpathcurveto{\pgfqpoint{9.018995in}{1.571508in}}{\pgfqpoint{9.020998in}{1.566671in}}{\pgfqpoint{9.024565in}{1.563104in}}%
\pgfpathcurveto{\pgfqpoint{9.028131in}{1.559538in}}{\pgfqpoint{9.032969in}{1.557534in}}{\pgfqpoint{9.038013in}{1.557534in}}%
\pgfpathclose%
\pgfusepath{fill}%
\end{pgfscope}%
\begin{pgfscope}%
\pgfpathrectangle{\pgfqpoint{6.572727in}{0.474100in}}{\pgfqpoint{4.227273in}{3.318700in}}%
\pgfusepath{clip}%
\pgfsetbuttcap%
\pgfsetroundjoin%
\definecolor{currentfill}{rgb}{0.127568,0.566949,0.550556}%
\pgfsetfillcolor{currentfill}%
\pgfsetfillopacity{0.700000}%
\pgfsetlinewidth{0.000000pt}%
\definecolor{currentstroke}{rgb}{0.000000,0.000000,0.000000}%
\pgfsetstrokecolor{currentstroke}%
\pgfsetstrokeopacity{0.700000}%
\pgfsetdash{}{0pt}%
\pgfpathmoveto{\pgfqpoint{8.481208in}{2.287272in}}%
\pgfpathcurveto{\pgfqpoint{8.486252in}{2.287272in}}{\pgfqpoint{8.491090in}{2.289276in}}{\pgfqpoint{8.494656in}{2.292842in}}%
\pgfpathcurveto{\pgfqpoint{8.498223in}{2.296409in}}{\pgfqpoint{8.500226in}{2.301246in}}{\pgfqpoint{8.500226in}{2.306290in}}%
\pgfpathcurveto{\pgfqpoint{8.500226in}{2.311334in}}{\pgfqpoint{8.498223in}{2.316171in}}{\pgfqpoint{8.494656in}{2.319738in}}%
\pgfpathcurveto{\pgfqpoint{8.491090in}{2.323304in}}{\pgfqpoint{8.486252in}{2.325308in}}{\pgfqpoint{8.481208in}{2.325308in}}%
\pgfpathcurveto{\pgfqpoint{8.476165in}{2.325308in}}{\pgfqpoint{8.471327in}{2.323304in}}{\pgfqpoint{8.467760in}{2.319738in}}%
\pgfpathcurveto{\pgfqpoint{8.464194in}{2.316171in}}{\pgfqpoint{8.462190in}{2.311334in}}{\pgfqpoint{8.462190in}{2.306290in}}%
\pgfpathcurveto{\pgfqpoint{8.462190in}{2.301246in}}{\pgfqpoint{8.464194in}{2.296409in}}{\pgfqpoint{8.467760in}{2.292842in}}%
\pgfpathcurveto{\pgfqpoint{8.471327in}{2.289276in}}{\pgfqpoint{8.476165in}{2.287272in}}{\pgfqpoint{8.481208in}{2.287272in}}%
\pgfpathclose%
\pgfusepath{fill}%
\end{pgfscope}%
\begin{pgfscope}%
\pgfpathrectangle{\pgfqpoint{6.572727in}{0.474100in}}{\pgfqpoint{4.227273in}{3.318700in}}%
\pgfusepath{clip}%
\pgfsetbuttcap%
\pgfsetroundjoin%
\definecolor{currentfill}{rgb}{0.993248,0.906157,0.143936}%
\pgfsetfillcolor{currentfill}%
\pgfsetfillopacity{0.700000}%
\pgfsetlinewidth{0.000000pt}%
\definecolor{currentstroke}{rgb}{0.000000,0.000000,0.000000}%
\pgfsetstrokecolor{currentstroke}%
\pgfsetstrokeopacity{0.700000}%
\pgfsetdash{}{0pt}%
\pgfpathmoveto{\pgfqpoint{9.314467in}{1.888475in}}%
\pgfpathcurveto{\pgfqpoint{9.319511in}{1.888475in}}{\pgfqpoint{9.324349in}{1.890478in}}{\pgfqpoint{9.327915in}{1.894045in}}%
\pgfpathcurveto{\pgfqpoint{9.331481in}{1.897611in}}{\pgfqpoint{9.333485in}{1.902449in}}{\pgfqpoint{9.333485in}{1.907493in}}%
\pgfpathcurveto{\pgfqpoint{9.333485in}{1.912536in}}{\pgfqpoint{9.331481in}{1.917374in}}{\pgfqpoint{9.327915in}{1.920941in}}%
\pgfpathcurveto{\pgfqpoint{9.324349in}{1.924507in}}{\pgfqpoint{9.319511in}{1.926511in}}{\pgfqpoint{9.314467in}{1.926511in}}%
\pgfpathcurveto{\pgfqpoint{9.309423in}{1.926511in}}{\pgfqpoint{9.304586in}{1.924507in}}{\pgfqpoint{9.301019in}{1.920941in}}%
\pgfpathcurveto{\pgfqpoint{9.297453in}{1.917374in}}{\pgfqpoint{9.295449in}{1.912536in}}{\pgfqpoint{9.295449in}{1.907493in}}%
\pgfpathcurveto{\pgfqpoint{9.295449in}{1.902449in}}{\pgfqpoint{9.297453in}{1.897611in}}{\pgfqpoint{9.301019in}{1.894045in}}%
\pgfpathcurveto{\pgfqpoint{9.304586in}{1.890478in}}{\pgfqpoint{9.309423in}{1.888475in}}{\pgfqpoint{9.314467in}{1.888475in}}%
\pgfpathclose%
\pgfusepath{fill}%
\end{pgfscope}%
\begin{pgfscope}%
\pgfpathrectangle{\pgfqpoint{6.572727in}{0.474100in}}{\pgfqpoint{4.227273in}{3.318700in}}%
\pgfusepath{clip}%
\pgfsetbuttcap%
\pgfsetroundjoin%
\definecolor{currentfill}{rgb}{0.127568,0.566949,0.550556}%
\pgfsetfillcolor{currentfill}%
\pgfsetfillopacity{0.700000}%
\pgfsetlinewidth{0.000000pt}%
\definecolor{currentstroke}{rgb}{0.000000,0.000000,0.000000}%
\pgfsetstrokecolor{currentstroke}%
\pgfsetstrokeopacity{0.700000}%
\pgfsetdash{}{0pt}%
\pgfpathmoveto{\pgfqpoint{7.843992in}{0.899686in}}%
\pgfpathcurveto{\pgfqpoint{7.849036in}{0.899686in}}{\pgfqpoint{7.853873in}{0.901690in}}{\pgfqpoint{7.857440in}{0.905256in}}%
\pgfpathcurveto{\pgfqpoint{7.861006in}{0.908823in}}{\pgfqpoint{7.863010in}{0.913661in}}{\pgfqpoint{7.863010in}{0.918704in}}%
\pgfpathcurveto{\pgfqpoint{7.863010in}{0.923748in}}{\pgfqpoint{7.861006in}{0.928586in}}{\pgfqpoint{7.857440in}{0.932152in}}%
\pgfpathcurveto{\pgfqpoint{7.853873in}{0.935718in}}{\pgfqpoint{7.849036in}{0.937722in}}{\pgfqpoint{7.843992in}{0.937722in}}%
\pgfpathcurveto{\pgfqpoint{7.838948in}{0.937722in}}{\pgfqpoint{7.834110in}{0.935718in}}{\pgfqpoint{7.830544in}{0.932152in}}%
\pgfpathcurveto{\pgfqpoint{7.826978in}{0.928586in}}{\pgfqpoint{7.824974in}{0.923748in}}{\pgfqpoint{7.824974in}{0.918704in}}%
\pgfpathcurveto{\pgfqpoint{7.824974in}{0.913661in}}{\pgfqpoint{7.826978in}{0.908823in}}{\pgfqpoint{7.830544in}{0.905256in}}%
\pgfpathcurveto{\pgfqpoint{7.834110in}{0.901690in}}{\pgfqpoint{7.838948in}{0.899686in}}{\pgfqpoint{7.843992in}{0.899686in}}%
\pgfpathclose%
\pgfusepath{fill}%
\end{pgfscope}%
\begin{pgfscope}%
\pgfpathrectangle{\pgfqpoint{6.572727in}{0.474100in}}{\pgfqpoint{4.227273in}{3.318700in}}%
\pgfusepath{clip}%
\pgfsetbuttcap%
\pgfsetroundjoin%
\definecolor{currentfill}{rgb}{0.127568,0.566949,0.550556}%
\pgfsetfillcolor{currentfill}%
\pgfsetfillopacity{0.700000}%
\pgfsetlinewidth{0.000000pt}%
\definecolor{currentstroke}{rgb}{0.000000,0.000000,0.000000}%
\pgfsetstrokecolor{currentstroke}%
\pgfsetstrokeopacity{0.700000}%
\pgfsetdash{}{0pt}%
\pgfpathmoveto{\pgfqpoint{8.501472in}{2.815219in}}%
\pgfpathcurveto{\pgfqpoint{8.506515in}{2.815219in}}{\pgfqpoint{8.511353in}{2.817223in}}{\pgfqpoint{8.514919in}{2.820790in}}%
\pgfpathcurveto{\pgfqpoint{8.518486in}{2.824356in}}{\pgfqpoint{8.520490in}{2.829194in}}{\pgfqpoint{8.520490in}{2.834237in}}%
\pgfpathcurveto{\pgfqpoint{8.520490in}{2.839281in}}{\pgfqpoint{8.518486in}{2.844119in}}{\pgfqpoint{8.514919in}{2.847685in}}%
\pgfpathcurveto{\pgfqpoint{8.511353in}{2.851252in}}{\pgfqpoint{8.506515in}{2.853256in}}{\pgfqpoint{8.501472in}{2.853256in}}%
\pgfpathcurveto{\pgfqpoint{8.496428in}{2.853256in}}{\pgfqpoint{8.491590in}{2.851252in}}{\pgfqpoint{8.488024in}{2.847685in}}%
\pgfpathcurveto{\pgfqpoint{8.484457in}{2.844119in}}{\pgfqpoint{8.482453in}{2.839281in}}{\pgfqpoint{8.482453in}{2.834237in}}%
\pgfpathcurveto{\pgfqpoint{8.482453in}{2.829194in}}{\pgfqpoint{8.484457in}{2.824356in}}{\pgfqpoint{8.488024in}{2.820790in}}%
\pgfpathcurveto{\pgfqpoint{8.491590in}{2.817223in}}{\pgfqpoint{8.496428in}{2.815219in}}{\pgfqpoint{8.501472in}{2.815219in}}%
\pgfpathclose%
\pgfusepath{fill}%
\end{pgfscope}%
\begin{pgfscope}%
\pgfpathrectangle{\pgfqpoint{6.572727in}{0.474100in}}{\pgfqpoint{4.227273in}{3.318700in}}%
\pgfusepath{clip}%
\pgfsetbuttcap%
\pgfsetroundjoin%
\definecolor{currentfill}{rgb}{0.127568,0.566949,0.550556}%
\pgfsetfillcolor{currentfill}%
\pgfsetfillopacity{0.700000}%
\pgfsetlinewidth{0.000000pt}%
\definecolor{currentstroke}{rgb}{0.000000,0.000000,0.000000}%
\pgfsetstrokecolor{currentstroke}%
\pgfsetstrokeopacity{0.700000}%
\pgfsetdash{}{0pt}%
\pgfpathmoveto{\pgfqpoint{8.110796in}{2.866503in}}%
\pgfpathcurveto{\pgfqpoint{8.115839in}{2.866503in}}{\pgfqpoint{8.120677in}{2.868507in}}{\pgfqpoint{8.124244in}{2.872073in}}%
\pgfpathcurveto{\pgfqpoint{8.127810in}{2.875640in}}{\pgfqpoint{8.129814in}{2.880477in}}{\pgfqpoint{8.129814in}{2.885521in}}%
\pgfpathcurveto{\pgfqpoint{8.129814in}{2.890565in}}{\pgfqpoint{8.127810in}{2.895403in}}{\pgfqpoint{8.124244in}{2.898969in}}%
\pgfpathcurveto{\pgfqpoint{8.120677in}{2.902535in}}{\pgfqpoint{8.115839in}{2.904539in}}{\pgfqpoint{8.110796in}{2.904539in}}%
\pgfpathcurveto{\pgfqpoint{8.105752in}{2.904539in}}{\pgfqpoint{8.100914in}{2.902535in}}{\pgfqpoint{8.097348in}{2.898969in}}%
\pgfpathcurveto{\pgfqpoint{8.093781in}{2.895403in}}{\pgfqpoint{8.091778in}{2.890565in}}{\pgfqpoint{8.091778in}{2.885521in}}%
\pgfpathcurveto{\pgfqpoint{8.091778in}{2.880477in}}{\pgfqpoint{8.093781in}{2.875640in}}{\pgfqpoint{8.097348in}{2.872073in}}%
\pgfpathcurveto{\pgfqpoint{8.100914in}{2.868507in}}{\pgfqpoint{8.105752in}{2.866503in}}{\pgfqpoint{8.110796in}{2.866503in}}%
\pgfpathclose%
\pgfusepath{fill}%
\end{pgfscope}%
\begin{pgfscope}%
\pgfpathrectangle{\pgfqpoint{6.572727in}{0.474100in}}{\pgfqpoint{4.227273in}{3.318700in}}%
\pgfusepath{clip}%
\pgfsetbuttcap%
\pgfsetroundjoin%
\definecolor{currentfill}{rgb}{0.127568,0.566949,0.550556}%
\pgfsetfillcolor{currentfill}%
\pgfsetfillopacity{0.700000}%
\pgfsetlinewidth{0.000000pt}%
\definecolor{currentstroke}{rgb}{0.000000,0.000000,0.000000}%
\pgfsetstrokecolor{currentstroke}%
\pgfsetstrokeopacity{0.700000}%
\pgfsetdash{}{0pt}%
\pgfpathmoveto{\pgfqpoint{8.297181in}{1.904391in}}%
\pgfpathcurveto{\pgfqpoint{8.302224in}{1.904391in}}{\pgfqpoint{8.307062in}{1.906395in}}{\pgfqpoint{8.310629in}{1.909961in}}%
\pgfpathcurveto{\pgfqpoint{8.314195in}{1.913528in}}{\pgfqpoint{8.316199in}{1.918365in}}{\pgfqpoint{8.316199in}{1.923409in}}%
\pgfpathcurveto{\pgfqpoint{8.316199in}{1.928453in}}{\pgfqpoint{8.314195in}{1.933291in}}{\pgfqpoint{8.310629in}{1.936857in}}%
\pgfpathcurveto{\pgfqpoint{8.307062in}{1.940423in}}{\pgfqpoint{8.302224in}{1.942427in}}{\pgfqpoint{8.297181in}{1.942427in}}%
\pgfpathcurveto{\pgfqpoint{8.292137in}{1.942427in}}{\pgfqpoint{8.287299in}{1.940423in}}{\pgfqpoint{8.283733in}{1.936857in}}%
\pgfpathcurveto{\pgfqpoint{8.280167in}{1.933291in}}{\pgfqpoint{8.278163in}{1.928453in}}{\pgfqpoint{8.278163in}{1.923409in}}%
\pgfpathcurveto{\pgfqpoint{8.278163in}{1.918365in}}{\pgfqpoint{8.280167in}{1.913528in}}{\pgfqpoint{8.283733in}{1.909961in}}%
\pgfpathcurveto{\pgfqpoint{8.287299in}{1.906395in}}{\pgfqpoint{8.292137in}{1.904391in}}{\pgfqpoint{8.297181in}{1.904391in}}%
\pgfpathclose%
\pgfusepath{fill}%
\end{pgfscope}%
\begin{pgfscope}%
\pgfpathrectangle{\pgfqpoint{6.572727in}{0.474100in}}{\pgfqpoint{4.227273in}{3.318700in}}%
\pgfusepath{clip}%
\pgfsetbuttcap%
\pgfsetroundjoin%
\definecolor{currentfill}{rgb}{0.127568,0.566949,0.550556}%
\pgfsetfillcolor{currentfill}%
\pgfsetfillopacity{0.700000}%
\pgfsetlinewidth{0.000000pt}%
\definecolor{currentstroke}{rgb}{0.000000,0.000000,0.000000}%
\pgfsetstrokecolor{currentstroke}%
\pgfsetstrokeopacity{0.700000}%
\pgfsetdash{}{0pt}%
\pgfpathmoveto{\pgfqpoint{8.738845in}{1.420968in}}%
\pgfpathcurveto{\pgfqpoint{8.743889in}{1.420968in}}{\pgfqpoint{8.748727in}{1.422972in}}{\pgfqpoint{8.752293in}{1.426538in}}%
\pgfpathcurveto{\pgfqpoint{8.755860in}{1.430105in}}{\pgfqpoint{8.757863in}{1.434942in}}{\pgfqpoint{8.757863in}{1.439986in}}%
\pgfpathcurveto{\pgfqpoint{8.757863in}{1.445030in}}{\pgfqpoint{8.755860in}{1.449867in}}{\pgfqpoint{8.752293in}{1.453434in}}%
\pgfpathcurveto{\pgfqpoint{8.748727in}{1.457000in}}{\pgfqpoint{8.743889in}{1.459004in}}{\pgfqpoint{8.738845in}{1.459004in}}%
\pgfpathcurveto{\pgfqpoint{8.733802in}{1.459004in}}{\pgfqpoint{8.728964in}{1.457000in}}{\pgfqpoint{8.725397in}{1.453434in}}%
\pgfpathcurveto{\pgfqpoint{8.721831in}{1.449867in}}{\pgfqpoint{8.719827in}{1.445030in}}{\pgfqpoint{8.719827in}{1.439986in}}%
\pgfpathcurveto{\pgfqpoint{8.719827in}{1.434942in}}{\pgfqpoint{8.721831in}{1.430105in}}{\pgfqpoint{8.725397in}{1.426538in}}%
\pgfpathcurveto{\pgfqpoint{8.728964in}{1.422972in}}{\pgfqpoint{8.733802in}{1.420968in}}{\pgfqpoint{8.738845in}{1.420968in}}%
\pgfpathclose%
\pgfusepath{fill}%
\end{pgfscope}%
\begin{pgfscope}%
\pgfpathrectangle{\pgfqpoint{6.572727in}{0.474100in}}{\pgfqpoint{4.227273in}{3.318700in}}%
\pgfusepath{clip}%
\pgfsetbuttcap%
\pgfsetroundjoin%
\definecolor{currentfill}{rgb}{0.993248,0.906157,0.143936}%
\pgfsetfillcolor{currentfill}%
\pgfsetfillopacity{0.700000}%
\pgfsetlinewidth{0.000000pt}%
\definecolor{currentstroke}{rgb}{0.000000,0.000000,0.000000}%
\pgfsetstrokecolor{currentstroke}%
\pgfsetstrokeopacity{0.700000}%
\pgfsetdash{}{0pt}%
\pgfpathmoveto{\pgfqpoint{9.144984in}{1.286895in}}%
\pgfpathcurveto{\pgfqpoint{9.150027in}{1.286895in}}{\pgfqpoint{9.154865in}{1.288899in}}{\pgfqpoint{9.158431in}{1.292466in}}%
\pgfpathcurveto{\pgfqpoint{9.161998in}{1.296032in}}{\pgfqpoint{9.164002in}{1.300870in}}{\pgfqpoint{9.164002in}{1.305914in}}%
\pgfpathcurveto{\pgfqpoint{9.164002in}{1.310957in}}{\pgfqpoint{9.161998in}{1.315795in}}{\pgfqpoint{9.158431in}{1.319361in}}%
\pgfpathcurveto{\pgfqpoint{9.154865in}{1.322928in}}{\pgfqpoint{9.150027in}{1.324932in}}{\pgfqpoint{9.144984in}{1.324932in}}%
\pgfpathcurveto{\pgfqpoint{9.139940in}{1.324932in}}{\pgfqpoint{9.135102in}{1.322928in}}{\pgfqpoint{9.131536in}{1.319361in}}%
\pgfpathcurveto{\pgfqpoint{9.127969in}{1.315795in}}{\pgfqpoint{9.125965in}{1.310957in}}{\pgfqpoint{9.125965in}{1.305914in}}%
\pgfpathcurveto{\pgfqpoint{9.125965in}{1.300870in}}{\pgfqpoint{9.127969in}{1.296032in}}{\pgfqpoint{9.131536in}{1.292466in}}%
\pgfpathcurveto{\pgfqpoint{9.135102in}{1.288899in}}{\pgfqpoint{9.139940in}{1.286895in}}{\pgfqpoint{9.144984in}{1.286895in}}%
\pgfpathclose%
\pgfusepath{fill}%
\end{pgfscope}%
\begin{pgfscope}%
\pgfpathrectangle{\pgfqpoint{6.572727in}{0.474100in}}{\pgfqpoint{4.227273in}{3.318700in}}%
\pgfusepath{clip}%
\pgfsetbuttcap%
\pgfsetroundjoin%
\definecolor{currentfill}{rgb}{0.993248,0.906157,0.143936}%
\pgfsetfillcolor{currentfill}%
\pgfsetfillopacity{0.700000}%
\pgfsetlinewidth{0.000000pt}%
\definecolor{currentstroke}{rgb}{0.000000,0.000000,0.000000}%
\pgfsetstrokecolor{currentstroke}%
\pgfsetstrokeopacity{0.700000}%
\pgfsetdash{}{0pt}%
\pgfpathmoveto{\pgfqpoint{9.338292in}{1.466118in}}%
\pgfpathcurveto{\pgfqpoint{9.343335in}{1.466118in}}{\pgfqpoint{9.348173in}{1.468122in}}{\pgfqpoint{9.351740in}{1.471688in}}%
\pgfpathcurveto{\pgfqpoint{9.355306in}{1.475255in}}{\pgfqpoint{9.357310in}{1.480092in}}{\pgfqpoint{9.357310in}{1.485136in}}%
\pgfpathcurveto{\pgfqpoint{9.357310in}{1.490180in}}{\pgfqpoint{9.355306in}{1.495018in}}{\pgfqpoint{9.351740in}{1.498584in}}%
\pgfpathcurveto{\pgfqpoint{9.348173in}{1.502150in}}{\pgfqpoint{9.343335in}{1.504154in}}{\pgfqpoint{9.338292in}{1.504154in}}%
\pgfpathcurveto{\pgfqpoint{9.333248in}{1.504154in}}{\pgfqpoint{9.328410in}{1.502150in}}{\pgfqpoint{9.324844in}{1.498584in}}%
\pgfpathcurveto{\pgfqpoint{9.321277in}{1.495018in}}{\pgfqpoint{9.319274in}{1.490180in}}{\pgfqpoint{9.319274in}{1.485136in}}%
\pgfpathcurveto{\pgfqpoint{9.319274in}{1.480092in}}{\pgfqpoint{9.321277in}{1.475255in}}{\pgfqpoint{9.324844in}{1.471688in}}%
\pgfpathcurveto{\pgfqpoint{9.328410in}{1.468122in}}{\pgfqpoint{9.333248in}{1.466118in}}{\pgfqpoint{9.338292in}{1.466118in}}%
\pgfpathclose%
\pgfusepath{fill}%
\end{pgfscope}%
\begin{pgfscope}%
\pgfpathrectangle{\pgfqpoint{6.572727in}{0.474100in}}{\pgfqpoint{4.227273in}{3.318700in}}%
\pgfusepath{clip}%
\pgfsetbuttcap%
\pgfsetroundjoin%
\definecolor{currentfill}{rgb}{0.127568,0.566949,0.550556}%
\pgfsetfillcolor{currentfill}%
\pgfsetfillopacity{0.700000}%
\pgfsetlinewidth{0.000000pt}%
\definecolor{currentstroke}{rgb}{0.000000,0.000000,0.000000}%
\pgfsetstrokecolor{currentstroke}%
\pgfsetstrokeopacity{0.700000}%
\pgfsetdash{}{0pt}%
\pgfpathmoveto{\pgfqpoint{8.082784in}{2.719942in}}%
\pgfpathcurveto{\pgfqpoint{8.087827in}{2.719942in}}{\pgfqpoint{8.092665in}{2.721946in}}{\pgfqpoint{8.096232in}{2.725512in}}%
\pgfpathcurveto{\pgfqpoint{8.099798in}{2.729079in}}{\pgfqpoint{8.101802in}{2.733917in}}{\pgfqpoint{8.101802in}{2.738960in}}%
\pgfpathcurveto{\pgfqpoint{8.101802in}{2.744004in}}{\pgfqpoint{8.099798in}{2.748842in}}{\pgfqpoint{8.096232in}{2.752408in}}%
\pgfpathcurveto{\pgfqpoint{8.092665in}{2.755975in}}{\pgfqpoint{8.087827in}{2.757978in}}{\pgfqpoint{8.082784in}{2.757978in}}%
\pgfpathcurveto{\pgfqpoint{8.077740in}{2.757978in}}{\pgfqpoint{8.072902in}{2.755975in}}{\pgfqpoint{8.069336in}{2.752408in}}%
\pgfpathcurveto{\pgfqpoint{8.065769in}{2.748842in}}{\pgfqpoint{8.063766in}{2.744004in}}{\pgfqpoint{8.063766in}{2.738960in}}%
\pgfpathcurveto{\pgfqpoint{8.063766in}{2.733917in}}{\pgfqpoint{8.065769in}{2.729079in}}{\pgfqpoint{8.069336in}{2.725512in}}%
\pgfpathcurveto{\pgfqpoint{8.072902in}{2.721946in}}{\pgfqpoint{8.077740in}{2.719942in}}{\pgfqpoint{8.082784in}{2.719942in}}%
\pgfpathclose%
\pgfusepath{fill}%
\end{pgfscope}%
\begin{pgfscope}%
\pgfpathrectangle{\pgfqpoint{6.572727in}{0.474100in}}{\pgfqpoint{4.227273in}{3.318700in}}%
\pgfusepath{clip}%
\pgfsetbuttcap%
\pgfsetroundjoin%
\definecolor{currentfill}{rgb}{0.993248,0.906157,0.143936}%
\pgfsetfillcolor{currentfill}%
\pgfsetfillopacity{0.700000}%
\pgfsetlinewidth{0.000000pt}%
\definecolor{currentstroke}{rgb}{0.000000,0.000000,0.000000}%
\pgfsetstrokecolor{currentstroke}%
\pgfsetstrokeopacity{0.700000}%
\pgfsetdash{}{0pt}%
\pgfpathmoveto{\pgfqpoint{10.523575in}{1.485629in}}%
\pgfpathcurveto{\pgfqpoint{10.528619in}{1.485629in}}{\pgfqpoint{10.533457in}{1.487633in}}{\pgfqpoint{10.537023in}{1.491199in}}%
\pgfpathcurveto{\pgfqpoint{10.540590in}{1.494766in}}{\pgfqpoint{10.542593in}{1.499603in}}{\pgfqpoint{10.542593in}{1.504647in}}%
\pgfpathcurveto{\pgfqpoint{10.542593in}{1.509691in}}{\pgfqpoint{10.540590in}{1.514529in}}{\pgfqpoint{10.537023in}{1.518095in}}%
\pgfpathcurveto{\pgfqpoint{10.533457in}{1.521661in}}{\pgfqpoint{10.528619in}{1.523665in}}{\pgfqpoint{10.523575in}{1.523665in}}%
\pgfpathcurveto{\pgfqpoint{10.518532in}{1.523665in}}{\pgfqpoint{10.513694in}{1.521661in}}{\pgfqpoint{10.510127in}{1.518095in}}%
\pgfpathcurveto{\pgfqpoint{10.506561in}{1.514529in}}{\pgfqpoint{10.504557in}{1.509691in}}{\pgfqpoint{10.504557in}{1.504647in}}%
\pgfpathcurveto{\pgfqpoint{10.504557in}{1.499603in}}{\pgfqpoint{10.506561in}{1.494766in}}{\pgfqpoint{10.510127in}{1.491199in}}%
\pgfpathcurveto{\pgfqpoint{10.513694in}{1.487633in}}{\pgfqpoint{10.518532in}{1.485629in}}{\pgfqpoint{10.523575in}{1.485629in}}%
\pgfpathclose%
\pgfusepath{fill}%
\end{pgfscope}%
\begin{pgfscope}%
\pgfpathrectangle{\pgfqpoint{6.572727in}{0.474100in}}{\pgfqpoint{4.227273in}{3.318700in}}%
\pgfusepath{clip}%
\pgfsetbuttcap%
\pgfsetroundjoin%
\definecolor{currentfill}{rgb}{0.127568,0.566949,0.550556}%
\pgfsetfillcolor{currentfill}%
\pgfsetfillopacity{0.700000}%
\pgfsetlinewidth{0.000000pt}%
\definecolor{currentstroke}{rgb}{0.000000,0.000000,0.000000}%
\pgfsetstrokecolor{currentstroke}%
\pgfsetstrokeopacity{0.700000}%
\pgfsetdash{}{0pt}%
\pgfpathmoveto{\pgfqpoint{7.985853in}{3.130262in}}%
\pgfpathcurveto{\pgfqpoint{7.990897in}{3.130262in}}{\pgfqpoint{7.995735in}{3.132266in}}{\pgfqpoint{7.999301in}{3.135832in}}%
\pgfpathcurveto{\pgfqpoint{8.002868in}{3.139399in}}{\pgfqpoint{8.004872in}{3.144236in}}{\pgfqpoint{8.004872in}{3.149280in}}%
\pgfpathcurveto{\pgfqpoint{8.004872in}{3.154324in}}{\pgfqpoint{8.002868in}{3.159162in}}{\pgfqpoint{7.999301in}{3.162728in}}%
\pgfpathcurveto{\pgfqpoint{7.995735in}{3.166294in}}{\pgfqpoint{7.990897in}{3.168298in}}{\pgfqpoint{7.985853in}{3.168298in}}%
\pgfpathcurveto{\pgfqpoint{7.980810in}{3.168298in}}{\pgfqpoint{7.975972in}{3.166294in}}{\pgfqpoint{7.972406in}{3.162728in}}%
\pgfpathcurveto{\pgfqpoint{7.968839in}{3.159162in}}{\pgfqpoint{7.966835in}{3.154324in}}{\pgfqpoint{7.966835in}{3.149280in}}%
\pgfpathcurveto{\pgfqpoint{7.966835in}{3.144236in}}{\pgfqpoint{7.968839in}{3.139399in}}{\pgfqpoint{7.972406in}{3.135832in}}%
\pgfpathcurveto{\pgfqpoint{7.975972in}{3.132266in}}{\pgfqpoint{7.980810in}{3.130262in}}{\pgfqpoint{7.985853in}{3.130262in}}%
\pgfpathclose%
\pgfusepath{fill}%
\end{pgfscope}%
\begin{pgfscope}%
\pgfpathrectangle{\pgfqpoint{6.572727in}{0.474100in}}{\pgfqpoint{4.227273in}{3.318700in}}%
\pgfusepath{clip}%
\pgfsetbuttcap%
\pgfsetroundjoin%
\definecolor{currentfill}{rgb}{0.993248,0.906157,0.143936}%
\pgfsetfillcolor{currentfill}%
\pgfsetfillopacity{0.700000}%
\pgfsetlinewidth{0.000000pt}%
\definecolor{currentstroke}{rgb}{0.000000,0.000000,0.000000}%
\pgfsetstrokecolor{currentstroke}%
\pgfsetstrokeopacity{0.700000}%
\pgfsetdash{}{0pt}%
\pgfpathmoveto{\pgfqpoint{9.121314in}{1.549120in}}%
\pgfpathcurveto{\pgfqpoint{9.126358in}{1.549120in}}{\pgfqpoint{9.131196in}{1.551124in}}{\pgfqpoint{9.134762in}{1.554690in}}%
\pgfpathcurveto{\pgfqpoint{9.138329in}{1.558256in}}{\pgfqpoint{9.140332in}{1.563094in}}{\pgfqpoint{9.140332in}{1.568138in}}%
\pgfpathcurveto{\pgfqpoint{9.140332in}{1.573182in}}{\pgfqpoint{9.138329in}{1.578019in}}{\pgfqpoint{9.134762in}{1.581586in}}%
\pgfpathcurveto{\pgfqpoint{9.131196in}{1.585152in}}{\pgfqpoint{9.126358in}{1.587156in}}{\pgfqpoint{9.121314in}{1.587156in}}%
\pgfpathcurveto{\pgfqpoint{9.116271in}{1.587156in}}{\pgfqpoint{9.111433in}{1.585152in}}{\pgfqpoint{9.107866in}{1.581586in}}%
\pgfpathcurveto{\pgfqpoint{9.104300in}{1.578019in}}{\pgfqpoint{9.102296in}{1.573182in}}{\pgfqpoint{9.102296in}{1.568138in}}%
\pgfpathcurveto{\pgfqpoint{9.102296in}{1.563094in}}{\pgfqpoint{9.104300in}{1.558256in}}{\pgfqpoint{9.107866in}{1.554690in}}%
\pgfpathcurveto{\pgfqpoint{9.111433in}{1.551124in}}{\pgfqpoint{9.116271in}{1.549120in}}{\pgfqpoint{9.121314in}{1.549120in}}%
\pgfpathclose%
\pgfusepath{fill}%
\end{pgfscope}%
\begin{pgfscope}%
\pgfpathrectangle{\pgfqpoint{6.572727in}{0.474100in}}{\pgfqpoint{4.227273in}{3.318700in}}%
\pgfusepath{clip}%
\pgfsetbuttcap%
\pgfsetroundjoin%
\definecolor{currentfill}{rgb}{0.993248,0.906157,0.143936}%
\pgfsetfillcolor{currentfill}%
\pgfsetfillopacity{0.700000}%
\pgfsetlinewidth{0.000000pt}%
\definecolor{currentstroke}{rgb}{0.000000,0.000000,0.000000}%
\pgfsetstrokecolor{currentstroke}%
\pgfsetstrokeopacity{0.700000}%
\pgfsetdash{}{0pt}%
\pgfpathmoveto{\pgfqpoint{9.732590in}{1.861824in}}%
\pgfpathcurveto{\pgfqpoint{9.737634in}{1.861824in}}{\pgfqpoint{9.742472in}{1.863828in}}{\pgfqpoint{9.746038in}{1.867394in}}%
\pgfpathcurveto{\pgfqpoint{9.749605in}{1.870960in}}{\pgfqpoint{9.751608in}{1.875798in}}{\pgfqpoint{9.751608in}{1.880842in}}%
\pgfpathcurveto{\pgfqpoint{9.751608in}{1.885885in}}{\pgfqpoint{9.749605in}{1.890723in}}{\pgfqpoint{9.746038in}{1.894290in}}%
\pgfpathcurveto{\pgfqpoint{9.742472in}{1.897856in}}{\pgfqpoint{9.737634in}{1.899860in}}{\pgfqpoint{9.732590in}{1.899860in}}%
\pgfpathcurveto{\pgfqpoint{9.727547in}{1.899860in}}{\pgfqpoint{9.722709in}{1.897856in}}{\pgfqpoint{9.719142in}{1.894290in}}%
\pgfpathcurveto{\pgfqpoint{9.715576in}{1.890723in}}{\pgfqpoint{9.713572in}{1.885885in}}{\pgfqpoint{9.713572in}{1.880842in}}%
\pgfpathcurveto{\pgfqpoint{9.713572in}{1.875798in}}{\pgfqpoint{9.715576in}{1.870960in}}{\pgfqpoint{9.719142in}{1.867394in}}%
\pgfpathcurveto{\pgfqpoint{9.722709in}{1.863828in}}{\pgfqpoint{9.727547in}{1.861824in}}{\pgfqpoint{9.732590in}{1.861824in}}%
\pgfpathclose%
\pgfusepath{fill}%
\end{pgfscope}%
\begin{pgfscope}%
\pgfpathrectangle{\pgfqpoint{6.572727in}{0.474100in}}{\pgfqpoint{4.227273in}{3.318700in}}%
\pgfusepath{clip}%
\pgfsetbuttcap%
\pgfsetroundjoin%
\definecolor{currentfill}{rgb}{0.127568,0.566949,0.550556}%
\pgfsetfillcolor{currentfill}%
\pgfsetfillopacity{0.700000}%
\pgfsetlinewidth{0.000000pt}%
\definecolor{currentstroke}{rgb}{0.000000,0.000000,0.000000}%
\pgfsetstrokecolor{currentstroke}%
\pgfsetstrokeopacity{0.700000}%
\pgfsetdash{}{0pt}%
\pgfpathmoveto{\pgfqpoint{7.764354in}{1.425265in}}%
\pgfpathcurveto{\pgfqpoint{7.769397in}{1.425265in}}{\pgfqpoint{7.774235in}{1.427269in}}{\pgfqpoint{7.777802in}{1.430836in}}%
\pgfpathcurveto{\pgfqpoint{7.781368in}{1.434402in}}{\pgfqpoint{7.783372in}{1.439240in}}{\pgfqpoint{7.783372in}{1.444283in}}%
\pgfpathcurveto{\pgfqpoint{7.783372in}{1.449327in}}{\pgfqpoint{7.781368in}{1.454165in}}{\pgfqpoint{7.777802in}{1.457731in}}%
\pgfpathcurveto{\pgfqpoint{7.774235in}{1.461298in}}{\pgfqpoint{7.769397in}{1.463302in}}{\pgfqpoint{7.764354in}{1.463302in}}%
\pgfpathcurveto{\pgfqpoint{7.759310in}{1.463302in}}{\pgfqpoint{7.754472in}{1.461298in}}{\pgfqpoint{7.750906in}{1.457731in}}%
\pgfpathcurveto{\pgfqpoint{7.747340in}{1.454165in}}{\pgfqpoint{7.745336in}{1.449327in}}{\pgfqpoint{7.745336in}{1.444283in}}%
\pgfpathcurveto{\pgfqpoint{7.745336in}{1.439240in}}{\pgfqpoint{7.747340in}{1.434402in}}{\pgfqpoint{7.750906in}{1.430836in}}%
\pgfpathcurveto{\pgfqpoint{7.754472in}{1.427269in}}{\pgfqpoint{7.759310in}{1.425265in}}{\pgfqpoint{7.764354in}{1.425265in}}%
\pgfpathclose%
\pgfusepath{fill}%
\end{pgfscope}%
\begin{pgfscope}%
\pgfpathrectangle{\pgfqpoint{6.572727in}{0.474100in}}{\pgfqpoint{4.227273in}{3.318700in}}%
\pgfusepath{clip}%
\pgfsetbuttcap%
\pgfsetroundjoin%
\definecolor{currentfill}{rgb}{0.993248,0.906157,0.143936}%
\pgfsetfillcolor{currentfill}%
\pgfsetfillopacity{0.700000}%
\pgfsetlinewidth{0.000000pt}%
\definecolor{currentstroke}{rgb}{0.000000,0.000000,0.000000}%
\pgfsetstrokecolor{currentstroke}%
\pgfsetstrokeopacity{0.700000}%
\pgfsetdash{}{0pt}%
\pgfpathmoveto{\pgfqpoint{10.185394in}{1.587454in}}%
\pgfpathcurveto{\pgfqpoint{10.190437in}{1.587454in}}{\pgfqpoint{10.195275in}{1.589458in}}{\pgfqpoint{10.198841in}{1.593025in}}%
\pgfpathcurveto{\pgfqpoint{10.202408in}{1.596591in}}{\pgfqpoint{10.204412in}{1.601429in}}{\pgfqpoint{10.204412in}{1.606472in}}%
\pgfpathcurveto{\pgfqpoint{10.204412in}{1.611516in}}{\pgfqpoint{10.202408in}{1.616354in}}{\pgfqpoint{10.198841in}{1.619920in}}%
\pgfpathcurveto{\pgfqpoint{10.195275in}{1.623487in}}{\pgfqpoint{10.190437in}{1.625491in}}{\pgfqpoint{10.185394in}{1.625491in}}%
\pgfpathcurveto{\pgfqpoint{10.180350in}{1.625491in}}{\pgfqpoint{10.175512in}{1.623487in}}{\pgfqpoint{10.171946in}{1.619920in}}%
\pgfpathcurveto{\pgfqpoint{10.168379in}{1.616354in}}{\pgfqpoint{10.166375in}{1.611516in}}{\pgfqpoint{10.166375in}{1.606472in}}%
\pgfpathcurveto{\pgfqpoint{10.166375in}{1.601429in}}{\pgfqpoint{10.168379in}{1.596591in}}{\pgfqpoint{10.171946in}{1.593025in}}%
\pgfpathcurveto{\pgfqpoint{10.175512in}{1.589458in}}{\pgfqpoint{10.180350in}{1.587454in}}{\pgfqpoint{10.185394in}{1.587454in}}%
\pgfpathclose%
\pgfusepath{fill}%
\end{pgfscope}%
\begin{pgfscope}%
\pgfpathrectangle{\pgfqpoint{6.572727in}{0.474100in}}{\pgfqpoint{4.227273in}{3.318700in}}%
\pgfusepath{clip}%
\pgfsetbuttcap%
\pgfsetroundjoin%
\definecolor{currentfill}{rgb}{0.993248,0.906157,0.143936}%
\pgfsetfillcolor{currentfill}%
\pgfsetfillopacity{0.700000}%
\pgfsetlinewidth{0.000000pt}%
\definecolor{currentstroke}{rgb}{0.000000,0.000000,0.000000}%
\pgfsetstrokecolor{currentstroke}%
\pgfsetstrokeopacity{0.700000}%
\pgfsetdash{}{0pt}%
\pgfpathmoveto{\pgfqpoint{9.875598in}{1.573172in}}%
\pgfpathcurveto{\pgfqpoint{9.880641in}{1.573172in}}{\pgfqpoint{9.885479in}{1.575176in}}{\pgfqpoint{9.889046in}{1.578743in}}%
\pgfpathcurveto{\pgfqpoint{9.892612in}{1.582309in}}{\pgfqpoint{9.894616in}{1.587147in}}{\pgfqpoint{9.894616in}{1.592190in}}%
\pgfpathcurveto{\pgfqpoint{9.894616in}{1.597234in}}{\pgfqpoint{9.892612in}{1.602072in}}{\pgfqpoint{9.889046in}{1.605638in}}%
\pgfpathcurveto{\pgfqpoint{9.885479in}{1.609205in}}{\pgfqpoint{9.880641in}{1.611209in}}{\pgfqpoint{9.875598in}{1.611209in}}%
\pgfpathcurveto{\pgfqpoint{9.870554in}{1.611209in}}{\pgfqpoint{9.865716in}{1.609205in}}{\pgfqpoint{9.862150in}{1.605638in}}%
\pgfpathcurveto{\pgfqpoint{9.858583in}{1.602072in}}{\pgfqpoint{9.856580in}{1.597234in}}{\pgfqpoint{9.856580in}{1.592190in}}%
\pgfpathcurveto{\pgfqpoint{9.856580in}{1.587147in}}{\pgfqpoint{9.858583in}{1.582309in}}{\pgfqpoint{9.862150in}{1.578743in}}%
\pgfpathcurveto{\pgfqpoint{9.865716in}{1.575176in}}{\pgfqpoint{9.870554in}{1.573172in}}{\pgfqpoint{9.875598in}{1.573172in}}%
\pgfpathclose%
\pgfusepath{fill}%
\end{pgfscope}%
\begin{pgfscope}%
\pgfpathrectangle{\pgfqpoint{6.572727in}{0.474100in}}{\pgfqpoint{4.227273in}{3.318700in}}%
\pgfusepath{clip}%
\pgfsetbuttcap%
\pgfsetroundjoin%
\definecolor{currentfill}{rgb}{0.993248,0.906157,0.143936}%
\pgfsetfillcolor{currentfill}%
\pgfsetfillopacity{0.700000}%
\pgfsetlinewidth{0.000000pt}%
\definecolor{currentstroke}{rgb}{0.000000,0.000000,0.000000}%
\pgfsetstrokecolor{currentstroke}%
\pgfsetstrokeopacity{0.700000}%
\pgfsetdash{}{0pt}%
\pgfpathmoveto{\pgfqpoint{9.118270in}{1.198404in}}%
\pgfpathcurveto{\pgfqpoint{9.123314in}{1.198404in}}{\pgfqpoint{9.128151in}{1.200407in}}{\pgfqpoint{9.131718in}{1.203974in}}%
\pgfpathcurveto{\pgfqpoint{9.135284in}{1.207540in}}{\pgfqpoint{9.137288in}{1.212378in}}{\pgfqpoint{9.137288in}{1.217422in}}%
\pgfpathcurveto{\pgfqpoint{9.137288in}{1.222465in}}{\pgfqpoint{9.135284in}{1.227303in}}{\pgfqpoint{9.131718in}{1.230870in}}%
\pgfpathcurveto{\pgfqpoint{9.128151in}{1.234436in}}{\pgfqpoint{9.123314in}{1.236440in}}{\pgfqpoint{9.118270in}{1.236440in}}%
\pgfpathcurveto{\pgfqpoint{9.113226in}{1.236440in}}{\pgfqpoint{9.108389in}{1.234436in}}{\pgfqpoint{9.104822in}{1.230870in}}%
\pgfpathcurveto{\pgfqpoint{9.101256in}{1.227303in}}{\pgfqpoint{9.099252in}{1.222465in}}{\pgfqpoint{9.099252in}{1.217422in}}%
\pgfpathcurveto{\pgfqpoint{9.099252in}{1.212378in}}{\pgfqpoint{9.101256in}{1.207540in}}{\pgfqpoint{9.104822in}{1.203974in}}%
\pgfpathcurveto{\pgfqpoint{9.108389in}{1.200407in}}{\pgfqpoint{9.113226in}{1.198404in}}{\pgfqpoint{9.118270in}{1.198404in}}%
\pgfpathclose%
\pgfusepath{fill}%
\end{pgfscope}%
\begin{pgfscope}%
\pgfpathrectangle{\pgfqpoint{6.572727in}{0.474100in}}{\pgfqpoint{4.227273in}{3.318700in}}%
\pgfusepath{clip}%
\pgfsetbuttcap%
\pgfsetroundjoin%
\definecolor{currentfill}{rgb}{0.127568,0.566949,0.550556}%
\pgfsetfillcolor{currentfill}%
\pgfsetfillopacity{0.700000}%
\pgfsetlinewidth{0.000000pt}%
\definecolor{currentstroke}{rgb}{0.000000,0.000000,0.000000}%
\pgfsetstrokecolor{currentstroke}%
\pgfsetstrokeopacity{0.700000}%
\pgfsetdash{}{0pt}%
\pgfpathmoveto{\pgfqpoint{8.034419in}{1.573831in}}%
\pgfpathcurveto{\pgfqpoint{8.039463in}{1.573831in}}{\pgfqpoint{8.044301in}{1.575834in}}{\pgfqpoint{8.047867in}{1.579401in}}%
\pgfpathcurveto{\pgfqpoint{8.051433in}{1.582967in}}{\pgfqpoint{8.053437in}{1.587805in}}{\pgfqpoint{8.053437in}{1.592849in}}%
\pgfpathcurveto{\pgfqpoint{8.053437in}{1.597892in}}{\pgfqpoint{8.051433in}{1.602730in}}{\pgfqpoint{8.047867in}{1.606297in}}%
\pgfpathcurveto{\pgfqpoint{8.044301in}{1.609863in}}{\pgfqpoint{8.039463in}{1.611867in}}{\pgfqpoint{8.034419in}{1.611867in}}%
\pgfpathcurveto{\pgfqpoint{8.029375in}{1.611867in}}{\pgfqpoint{8.024538in}{1.609863in}}{\pgfqpoint{8.020971in}{1.606297in}}%
\pgfpathcurveto{\pgfqpoint{8.017405in}{1.602730in}}{\pgfqpoint{8.015401in}{1.597892in}}{\pgfqpoint{8.015401in}{1.592849in}}%
\pgfpathcurveto{\pgfqpoint{8.015401in}{1.587805in}}{\pgfqpoint{8.017405in}{1.582967in}}{\pgfqpoint{8.020971in}{1.579401in}}%
\pgfpathcurveto{\pgfqpoint{8.024538in}{1.575834in}}{\pgfqpoint{8.029375in}{1.573831in}}{\pgfqpoint{8.034419in}{1.573831in}}%
\pgfpathclose%
\pgfusepath{fill}%
\end{pgfscope}%
\begin{pgfscope}%
\pgfpathrectangle{\pgfqpoint{6.572727in}{0.474100in}}{\pgfqpoint{4.227273in}{3.318700in}}%
\pgfusepath{clip}%
\pgfsetbuttcap%
\pgfsetroundjoin%
\definecolor{currentfill}{rgb}{0.127568,0.566949,0.550556}%
\pgfsetfillcolor{currentfill}%
\pgfsetfillopacity{0.700000}%
\pgfsetlinewidth{0.000000pt}%
\definecolor{currentstroke}{rgb}{0.000000,0.000000,0.000000}%
\pgfsetstrokecolor{currentstroke}%
\pgfsetstrokeopacity{0.700000}%
\pgfsetdash{}{0pt}%
\pgfpathmoveto{\pgfqpoint{7.957217in}{2.776938in}}%
\pgfpathcurveto{\pgfqpoint{7.962261in}{2.776938in}}{\pgfqpoint{7.967098in}{2.778941in}}{\pgfqpoint{7.970665in}{2.782508in}}%
\pgfpathcurveto{\pgfqpoint{7.974231in}{2.786074in}}{\pgfqpoint{7.976235in}{2.790912in}}{\pgfqpoint{7.976235in}{2.795956in}}%
\pgfpathcurveto{\pgfqpoint{7.976235in}{2.800999in}}{\pgfqpoint{7.974231in}{2.805837in}}{\pgfqpoint{7.970665in}{2.809404in}}%
\pgfpathcurveto{\pgfqpoint{7.967098in}{2.812970in}}{\pgfqpoint{7.962261in}{2.814974in}}{\pgfqpoint{7.957217in}{2.814974in}}%
\pgfpathcurveto{\pgfqpoint{7.952173in}{2.814974in}}{\pgfqpoint{7.947335in}{2.812970in}}{\pgfqpoint{7.943769in}{2.809404in}}%
\pgfpathcurveto{\pgfqpoint{7.940203in}{2.805837in}}{\pgfqpoint{7.938199in}{2.800999in}}{\pgfqpoint{7.938199in}{2.795956in}}%
\pgfpathcurveto{\pgfqpoint{7.938199in}{2.790912in}}{\pgfqpoint{7.940203in}{2.786074in}}{\pgfqpoint{7.943769in}{2.782508in}}%
\pgfpathcurveto{\pgfqpoint{7.947335in}{2.778941in}}{\pgfqpoint{7.952173in}{2.776938in}}{\pgfqpoint{7.957217in}{2.776938in}}%
\pgfpathclose%
\pgfusepath{fill}%
\end{pgfscope}%
\begin{pgfscope}%
\pgfpathrectangle{\pgfqpoint{6.572727in}{0.474100in}}{\pgfqpoint{4.227273in}{3.318700in}}%
\pgfusepath{clip}%
\pgfsetbuttcap%
\pgfsetroundjoin%
\definecolor{currentfill}{rgb}{0.127568,0.566949,0.550556}%
\pgfsetfillcolor{currentfill}%
\pgfsetfillopacity{0.700000}%
\pgfsetlinewidth{0.000000pt}%
\definecolor{currentstroke}{rgb}{0.000000,0.000000,0.000000}%
\pgfsetstrokecolor{currentstroke}%
\pgfsetstrokeopacity{0.700000}%
\pgfsetdash{}{0pt}%
\pgfpathmoveto{\pgfqpoint{7.447726in}{1.621090in}}%
\pgfpathcurveto{\pgfqpoint{7.452769in}{1.621090in}}{\pgfqpoint{7.457607in}{1.623094in}}{\pgfqpoint{7.461173in}{1.626661in}}%
\pgfpathcurveto{\pgfqpoint{7.464740in}{1.630227in}}{\pgfqpoint{7.466744in}{1.635065in}}{\pgfqpoint{7.466744in}{1.640108in}}%
\pgfpathcurveto{\pgfqpoint{7.466744in}{1.645152in}}{\pgfqpoint{7.464740in}{1.649990in}}{\pgfqpoint{7.461173in}{1.653556in}}%
\pgfpathcurveto{\pgfqpoint{7.457607in}{1.657123in}}{\pgfqpoint{7.452769in}{1.659127in}}{\pgfqpoint{7.447726in}{1.659127in}}%
\pgfpathcurveto{\pgfqpoint{7.442682in}{1.659127in}}{\pgfqpoint{7.437844in}{1.657123in}}{\pgfqpoint{7.434278in}{1.653556in}}%
\pgfpathcurveto{\pgfqpoint{7.430711in}{1.649990in}}{\pgfqpoint{7.428707in}{1.645152in}}{\pgfqpoint{7.428707in}{1.640108in}}%
\pgfpathcurveto{\pgfqpoint{7.428707in}{1.635065in}}{\pgfqpoint{7.430711in}{1.630227in}}{\pgfqpoint{7.434278in}{1.626661in}}%
\pgfpathcurveto{\pgfqpoint{7.437844in}{1.623094in}}{\pgfqpoint{7.442682in}{1.621090in}}{\pgfqpoint{7.447726in}{1.621090in}}%
\pgfpathclose%
\pgfusepath{fill}%
\end{pgfscope}%
\begin{pgfscope}%
\pgfpathrectangle{\pgfqpoint{6.572727in}{0.474100in}}{\pgfqpoint{4.227273in}{3.318700in}}%
\pgfusepath{clip}%
\pgfsetbuttcap%
\pgfsetroundjoin%
\definecolor{currentfill}{rgb}{0.127568,0.566949,0.550556}%
\pgfsetfillcolor{currentfill}%
\pgfsetfillopacity{0.700000}%
\pgfsetlinewidth{0.000000pt}%
\definecolor{currentstroke}{rgb}{0.000000,0.000000,0.000000}%
\pgfsetstrokecolor{currentstroke}%
\pgfsetstrokeopacity{0.700000}%
\pgfsetdash{}{0pt}%
\pgfpathmoveto{\pgfqpoint{8.375764in}{3.050893in}}%
\pgfpathcurveto{\pgfqpoint{8.380807in}{3.050893in}}{\pgfqpoint{8.385645in}{3.052897in}}{\pgfqpoint{8.389211in}{3.056463in}}%
\pgfpathcurveto{\pgfqpoint{8.392778in}{3.060029in}}{\pgfqpoint{8.394782in}{3.064867in}}{\pgfqpoint{8.394782in}{3.069911in}}%
\pgfpathcurveto{\pgfqpoint{8.394782in}{3.074954in}}{\pgfqpoint{8.392778in}{3.079792in}}{\pgfqpoint{8.389211in}{3.083359in}}%
\pgfpathcurveto{\pgfqpoint{8.385645in}{3.086925in}}{\pgfqpoint{8.380807in}{3.088929in}}{\pgfqpoint{8.375764in}{3.088929in}}%
\pgfpathcurveto{\pgfqpoint{8.370720in}{3.088929in}}{\pgfqpoint{8.365882in}{3.086925in}}{\pgfqpoint{8.362316in}{3.083359in}}%
\pgfpathcurveto{\pgfqpoint{8.358749in}{3.079792in}}{\pgfqpoint{8.356745in}{3.074954in}}{\pgfqpoint{8.356745in}{3.069911in}}%
\pgfpathcurveto{\pgfqpoint{8.356745in}{3.064867in}}{\pgfqpoint{8.358749in}{3.060029in}}{\pgfqpoint{8.362316in}{3.056463in}}%
\pgfpathcurveto{\pgfqpoint{8.365882in}{3.052897in}}{\pgfqpoint{8.370720in}{3.050893in}}{\pgfqpoint{8.375764in}{3.050893in}}%
\pgfpathclose%
\pgfusepath{fill}%
\end{pgfscope}%
\begin{pgfscope}%
\pgfpathrectangle{\pgfqpoint{6.572727in}{0.474100in}}{\pgfqpoint{4.227273in}{3.318700in}}%
\pgfusepath{clip}%
\pgfsetbuttcap%
\pgfsetroundjoin%
\definecolor{currentfill}{rgb}{0.127568,0.566949,0.550556}%
\pgfsetfillcolor{currentfill}%
\pgfsetfillopacity{0.700000}%
\pgfsetlinewidth{0.000000pt}%
\definecolor{currentstroke}{rgb}{0.000000,0.000000,0.000000}%
\pgfsetstrokecolor{currentstroke}%
\pgfsetstrokeopacity{0.700000}%
\pgfsetdash{}{0pt}%
\pgfpathmoveto{\pgfqpoint{7.971135in}{2.536822in}}%
\pgfpathcurveto{\pgfqpoint{7.976178in}{2.536822in}}{\pgfqpoint{7.981016in}{2.538826in}}{\pgfqpoint{7.984582in}{2.542392in}}%
\pgfpathcurveto{\pgfqpoint{7.988149in}{2.545959in}}{\pgfqpoint{7.990153in}{2.550797in}}{\pgfqpoint{7.990153in}{2.555840in}}%
\pgfpathcurveto{\pgfqpoint{7.990153in}{2.560884in}}{\pgfqpoint{7.988149in}{2.565722in}}{\pgfqpoint{7.984582in}{2.569288in}}%
\pgfpathcurveto{\pgfqpoint{7.981016in}{2.572855in}}{\pgfqpoint{7.976178in}{2.574858in}}{\pgfqpoint{7.971135in}{2.574858in}}%
\pgfpathcurveto{\pgfqpoint{7.966091in}{2.574858in}}{\pgfqpoint{7.961253in}{2.572855in}}{\pgfqpoint{7.957687in}{2.569288in}}%
\pgfpathcurveto{\pgfqpoint{7.954120in}{2.565722in}}{\pgfqpoint{7.952116in}{2.560884in}}{\pgfqpoint{7.952116in}{2.555840in}}%
\pgfpathcurveto{\pgfqpoint{7.952116in}{2.550797in}}{\pgfqpoint{7.954120in}{2.545959in}}{\pgfqpoint{7.957687in}{2.542392in}}%
\pgfpathcurveto{\pgfqpoint{7.961253in}{2.538826in}}{\pgfqpoint{7.966091in}{2.536822in}}{\pgfqpoint{7.971135in}{2.536822in}}%
\pgfpathclose%
\pgfusepath{fill}%
\end{pgfscope}%
\begin{pgfscope}%
\pgfpathrectangle{\pgfqpoint{6.572727in}{0.474100in}}{\pgfqpoint{4.227273in}{3.318700in}}%
\pgfusepath{clip}%
\pgfsetbuttcap%
\pgfsetroundjoin%
\definecolor{currentfill}{rgb}{0.127568,0.566949,0.550556}%
\pgfsetfillcolor{currentfill}%
\pgfsetfillopacity{0.700000}%
\pgfsetlinewidth{0.000000pt}%
\definecolor{currentstroke}{rgb}{0.000000,0.000000,0.000000}%
\pgfsetstrokecolor{currentstroke}%
\pgfsetstrokeopacity{0.700000}%
\pgfsetdash{}{0pt}%
\pgfpathmoveto{\pgfqpoint{7.306662in}{3.033135in}}%
\pgfpathcurveto{\pgfqpoint{7.311706in}{3.033135in}}{\pgfqpoint{7.316544in}{3.035139in}}{\pgfqpoint{7.320110in}{3.038705in}}%
\pgfpathcurveto{\pgfqpoint{7.323677in}{3.042272in}}{\pgfqpoint{7.325681in}{3.047110in}}{\pgfqpoint{7.325681in}{3.052153in}}%
\pgfpathcurveto{\pgfqpoint{7.325681in}{3.057197in}}{\pgfqpoint{7.323677in}{3.062035in}}{\pgfqpoint{7.320110in}{3.065601in}}%
\pgfpathcurveto{\pgfqpoint{7.316544in}{3.069168in}}{\pgfqpoint{7.311706in}{3.071171in}}{\pgfqpoint{7.306662in}{3.071171in}}%
\pgfpathcurveto{\pgfqpoint{7.301619in}{3.071171in}}{\pgfqpoint{7.296781in}{3.069168in}}{\pgfqpoint{7.293215in}{3.065601in}}%
\pgfpathcurveto{\pgfqpoint{7.289648in}{3.062035in}}{\pgfqpoint{7.287644in}{3.057197in}}{\pgfqpoint{7.287644in}{3.052153in}}%
\pgfpathcurveto{\pgfqpoint{7.287644in}{3.047110in}}{\pgfqpoint{7.289648in}{3.042272in}}{\pgfqpoint{7.293215in}{3.038705in}}%
\pgfpathcurveto{\pgfqpoint{7.296781in}{3.035139in}}{\pgfqpoint{7.301619in}{3.033135in}}{\pgfqpoint{7.306662in}{3.033135in}}%
\pgfpathclose%
\pgfusepath{fill}%
\end{pgfscope}%
\begin{pgfscope}%
\pgfpathrectangle{\pgfqpoint{6.572727in}{0.474100in}}{\pgfqpoint{4.227273in}{3.318700in}}%
\pgfusepath{clip}%
\pgfsetbuttcap%
\pgfsetroundjoin%
\definecolor{currentfill}{rgb}{0.127568,0.566949,0.550556}%
\pgfsetfillcolor{currentfill}%
\pgfsetfillopacity{0.700000}%
\pgfsetlinewidth{0.000000pt}%
\definecolor{currentstroke}{rgb}{0.000000,0.000000,0.000000}%
\pgfsetstrokecolor{currentstroke}%
\pgfsetstrokeopacity{0.700000}%
\pgfsetdash{}{0pt}%
\pgfpathmoveto{\pgfqpoint{7.989772in}{1.855730in}}%
\pgfpathcurveto{\pgfqpoint{7.994815in}{1.855730in}}{\pgfqpoint{7.999653in}{1.857734in}}{\pgfqpoint{8.003220in}{1.861300in}}%
\pgfpathcurveto{\pgfqpoint{8.006786in}{1.864867in}}{\pgfqpoint{8.008790in}{1.869705in}}{\pgfqpoint{8.008790in}{1.874748in}}%
\pgfpathcurveto{\pgfqpoint{8.008790in}{1.879792in}}{\pgfqpoint{8.006786in}{1.884630in}}{\pgfqpoint{8.003220in}{1.888196in}}%
\pgfpathcurveto{\pgfqpoint{7.999653in}{1.891762in}}{\pgfqpoint{7.994815in}{1.893766in}}{\pgfqpoint{7.989772in}{1.893766in}}%
\pgfpathcurveto{\pgfqpoint{7.984728in}{1.893766in}}{\pgfqpoint{7.979890in}{1.891762in}}{\pgfqpoint{7.976324in}{1.888196in}}%
\pgfpathcurveto{\pgfqpoint{7.972757in}{1.884630in}}{\pgfqpoint{7.970754in}{1.879792in}}{\pgfqpoint{7.970754in}{1.874748in}}%
\pgfpathcurveto{\pgfqpoint{7.970754in}{1.869705in}}{\pgfqpoint{7.972757in}{1.864867in}}{\pgfqpoint{7.976324in}{1.861300in}}%
\pgfpathcurveto{\pgfqpoint{7.979890in}{1.857734in}}{\pgfqpoint{7.984728in}{1.855730in}}{\pgfqpoint{7.989772in}{1.855730in}}%
\pgfpathclose%
\pgfusepath{fill}%
\end{pgfscope}%
\begin{pgfscope}%
\pgfpathrectangle{\pgfqpoint{6.572727in}{0.474100in}}{\pgfqpoint{4.227273in}{3.318700in}}%
\pgfusepath{clip}%
\pgfsetbuttcap%
\pgfsetroundjoin%
\definecolor{currentfill}{rgb}{0.127568,0.566949,0.550556}%
\pgfsetfillcolor{currentfill}%
\pgfsetfillopacity{0.700000}%
\pgfsetlinewidth{0.000000pt}%
\definecolor{currentstroke}{rgb}{0.000000,0.000000,0.000000}%
\pgfsetstrokecolor{currentstroke}%
\pgfsetstrokeopacity{0.700000}%
\pgfsetdash{}{0pt}%
\pgfpathmoveto{\pgfqpoint{8.164127in}{0.869917in}}%
\pgfpathcurveto{\pgfqpoint{8.169171in}{0.869917in}}{\pgfqpoint{8.174009in}{0.871921in}}{\pgfqpoint{8.177575in}{0.875487in}}%
\pgfpathcurveto{\pgfqpoint{8.181142in}{0.879054in}}{\pgfqpoint{8.183145in}{0.883891in}}{\pgfqpoint{8.183145in}{0.888935in}}%
\pgfpathcurveto{\pgfqpoint{8.183145in}{0.893979in}}{\pgfqpoint{8.181142in}{0.898816in}}{\pgfqpoint{8.177575in}{0.902383in}}%
\pgfpathcurveto{\pgfqpoint{8.174009in}{0.905949in}}{\pgfqpoint{8.169171in}{0.907953in}}{\pgfqpoint{8.164127in}{0.907953in}}%
\pgfpathcurveto{\pgfqpoint{8.159084in}{0.907953in}}{\pgfqpoint{8.154246in}{0.905949in}}{\pgfqpoint{8.150679in}{0.902383in}}%
\pgfpathcurveto{\pgfqpoint{8.147113in}{0.898816in}}{\pgfqpoint{8.145109in}{0.893979in}}{\pgfqpoint{8.145109in}{0.888935in}}%
\pgfpathcurveto{\pgfqpoint{8.145109in}{0.883891in}}{\pgfqpoint{8.147113in}{0.879054in}}{\pgfqpoint{8.150679in}{0.875487in}}%
\pgfpathcurveto{\pgfqpoint{8.154246in}{0.871921in}}{\pgfqpoint{8.159084in}{0.869917in}}{\pgfqpoint{8.164127in}{0.869917in}}%
\pgfpathclose%
\pgfusepath{fill}%
\end{pgfscope}%
\begin{pgfscope}%
\pgfpathrectangle{\pgfqpoint{6.572727in}{0.474100in}}{\pgfqpoint{4.227273in}{3.318700in}}%
\pgfusepath{clip}%
\pgfsetbuttcap%
\pgfsetroundjoin%
\definecolor{currentfill}{rgb}{0.127568,0.566949,0.550556}%
\pgfsetfillcolor{currentfill}%
\pgfsetfillopacity{0.700000}%
\pgfsetlinewidth{0.000000pt}%
\definecolor{currentstroke}{rgb}{0.000000,0.000000,0.000000}%
\pgfsetstrokecolor{currentstroke}%
\pgfsetstrokeopacity{0.700000}%
\pgfsetdash{}{0pt}%
\pgfpathmoveto{\pgfqpoint{8.302321in}{2.891813in}}%
\pgfpathcurveto{\pgfqpoint{8.307365in}{2.891813in}}{\pgfqpoint{8.312202in}{2.893817in}}{\pgfqpoint{8.315769in}{2.897383in}}%
\pgfpathcurveto{\pgfqpoint{8.319335in}{2.900950in}}{\pgfqpoint{8.321339in}{2.905787in}}{\pgfqpoint{8.321339in}{2.910831in}}%
\pgfpathcurveto{\pgfqpoint{8.321339in}{2.915875in}}{\pgfqpoint{8.319335in}{2.920713in}}{\pgfqpoint{8.315769in}{2.924279in}}%
\pgfpathcurveto{\pgfqpoint{8.312202in}{2.927845in}}{\pgfqpoint{8.307365in}{2.929849in}}{\pgfqpoint{8.302321in}{2.929849in}}%
\pgfpathcurveto{\pgfqpoint{8.297277in}{2.929849in}}{\pgfqpoint{8.292440in}{2.927845in}}{\pgfqpoint{8.288873in}{2.924279in}}%
\pgfpathcurveto{\pgfqpoint{8.285307in}{2.920713in}}{\pgfqpoint{8.283303in}{2.915875in}}{\pgfqpoint{8.283303in}{2.910831in}}%
\pgfpathcurveto{\pgfqpoint{8.283303in}{2.905787in}}{\pgfqpoint{8.285307in}{2.900950in}}{\pgfqpoint{8.288873in}{2.897383in}}%
\pgfpathcurveto{\pgfqpoint{8.292440in}{2.893817in}}{\pgfqpoint{8.297277in}{2.891813in}}{\pgfqpoint{8.302321in}{2.891813in}}%
\pgfpathclose%
\pgfusepath{fill}%
\end{pgfscope}%
\begin{pgfscope}%
\pgfpathrectangle{\pgfqpoint{6.572727in}{0.474100in}}{\pgfqpoint{4.227273in}{3.318700in}}%
\pgfusepath{clip}%
\pgfsetbuttcap%
\pgfsetroundjoin%
\definecolor{currentfill}{rgb}{0.127568,0.566949,0.550556}%
\pgfsetfillcolor{currentfill}%
\pgfsetfillopacity{0.700000}%
\pgfsetlinewidth{0.000000pt}%
\definecolor{currentstroke}{rgb}{0.000000,0.000000,0.000000}%
\pgfsetstrokecolor{currentstroke}%
\pgfsetstrokeopacity{0.700000}%
\pgfsetdash{}{0pt}%
\pgfpathmoveto{\pgfqpoint{7.920534in}{1.620271in}}%
\pgfpathcurveto{\pgfqpoint{7.925577in}{1.620271in}}{\pgfqpoint{7.930415in}{1.622275in}}{\pgfqpoint{7.933982in}{1.625842in}}%
\pgfpathcurveto{\pgfqpoint{7.937548in}{1.629408in}}{\pgfqpoint{7.939552in}{1.634246in}}{\pgfqpoint{7.939552in}{1.639290in}}%
\pgfpathcurveto{\pgfqpoint{7.939552in}{1.644333in}}{\pgfqpoint{7.937548in}{1.649171in}}{\pgfqpoint{7.933982in}{1.652737in}}%
\pgfpathcurveto{\pgfqpoint{7.930415in}{1.656304in}}{\pgfqpoint{7.925577in}{1.658308in}}{\pgfqpoint{7.920534in}{1.658308in}}%
\pgfpathcurveto{\pgfqpoint{7.915490in}{1.658308in}}{\pgfqpoint{7.910652in}{1.656304in}}{\pgfqpoint{7.907086in}{1.652737in}}%
\pgfpathcurveto{\pgfqpoint{7.903520in}{1.649171in}}{\pgfqpoint{7.901516in}{1.644333in}}{\pgfqpoint{7.901516in}{1.639290in}}%
\pgfpathcurveto{\pgfqpoint{7.901516in}{1.634246in}}{\pgfqpoint{7.903520in}{1.629408in}}{\pgfqpoint{7.907086in}{1.625842in}}%
\pgfpathcurveto{\pgfqpoint{7.910652in}{1.622275in}}{\pgfqpoint{7.915490in}{1.620271in}}{\pgfqpoint{7.920534in}{1.620271in}}%
\pgfpathclose%
\pgfusepath{fill}%
\end{pgfscope}%
\begin{pgfscope}%
\pgfpathrectangle{\pgfqpoint{6.572727in}{0.474100in}}{\pgfqpoint{4.227273in}{3.318700in}}%
\pgfusepath{clip}%
\pgfsetbuttcap%
\pgfsetroundjoin%
\definecolor{currentfill}{rgb}{0.127568,0.566949,0.550556}%
\pgfsetfillcolor{currentfill}%
\pgfsetfillopacity{0.700000}%
\pgfsetlinewidth{0.000000pt}%
\definecolor{currentstroke}{rgb}{0.000000,0.000000,0.000000}%
\pgfsetstrokecolor{currentstroke}%
\pgfsetstrokeopacity{0.700000}%
\pgfsetdash{}{0pt}%
\pgfpathmoveto{\pgfqpoint{7.973924in}{1.447028in}}%
\pgfpathcurveto{\pgfqpoint{7.978968in}{1.447028in}}{\pgfqpoint{7.983806in}{1.449032in}}{\pgfqpoint{7.987372in}{1.452598in}}%
\pgfpathcurveto{\pgfqpoint{7.990938in}{1.456165in}}{\pgfqpoint{7.992942in}{1.461002in}}{\pgfqpoint{7.992942in}{1.466046in}}%
\pgfpathcurveto{\pgfqpoint{7.992942in}{1.471090in}}{\pgfqpoint{7.990938in}{1.475928in}}{\pgfqpoint{7.987372in}{1.479494in}}%
\pgfpathcurveto{\pgfqpoint{7.983806in}{1.483060in}}{\pgfqpoint{7.978968in}{1.485064in}}{\pgfqpoint{7.973924in}{1.485064in}}%
\pgfpathcurveto{\pgfqpoint{7.968880in}{1.485064in}}{\pgfqpoint{7.964043in}{1.483060in}}{\pgfqpoint{7.960476in}{1.479494in}}%
\pgfpathcurveto{\pgfqpoint{7.956910in}{1.475928in}}{\pgfqpoint{7.954906in}{1.471090in}}{\pgfqpoint{7.954906in}{1.466046in}}%
\pgfpathcurveto{\pgfqpoint{7.954906in}{1.461002in}}{\pgfqpoint{7.956910in}{1.456165in}}{\pgfqpoint{7.960476in}{1.452598in}}%
\pgfpathcurveto{\pgfqpoint{7.964043in}{1.449032in}}{\pgfqpoint{7.968880in}{1.447028in}}{\pgfqpoint{7.973924in}{1.447028in}}%
\pgfpathclose%
\pgfusepath{fill}%
\end{pgfscope}%
\begin{pgfscope}%
\pgfpathrectangle{\pgfqpoint{6.572727in}{0.474100in}}{\pgfqpoint{4.227273in}{3.318700in}}%
\pgfusepath{clip}%
\pgfsetbuttcap%
\pgfsetroundjoin%
\definecolor{currentfill}{rgb}{0.993248,0.906157,0.143936}%
\pgfsetfillcolor{currentfill}%
\pgfsetfillopacity{0.700000}%
\pgfsetlinewidth{0.000000pt}%
\definecolor{currentstroke}{rgb}{0.000000,0.000000,0.000000}%
\pgfsetstrokecolor{currentstroke}%
\pgfsetstrokeopacity{0.700000}%
\pgfsetdash{}{0pt}%
\pgfpathmoveto{\pgfqpoint{9.695945in}{2.223342in}}%
\pgfpathcurveto{\pgfqpoint{9.700988in}{2.223342in}}{\pgfqpoint{9.705826in}{2.225346in}}{\pgfqpoint{9.709392in}{2.228913in}}%
\pgfpathcurveto{\pgfqpoint{9.712959in}{2.232479in}}{\pgfqpoint{9.714963in}{2.237317in}}{\pgfqpoint{9.714963in}{2.242360in}}%
\pgfpathcurveto{\pgfqpoint{9.714963in}{2.247404in}}{\pgfqpoint{9.712959in}{2.252242in}}{\pgfqpoint{9.709392in}{2.255808in}}%
\pgfpathcurveto{\pgfqpoint{9.705826in}{2.259375in}}{\pgfqpoint{9.700988in}{2.261379in}}{\pgfqpoint{9.695945in}{2.261379in}}%
\pgfpathcurveto{\pgfqpoint{9.690901in}{2.261379in}}{\pgfqpoint{9.686063in}{2.259375in}}{\pgfqpoint{9.682497in}{2.255808in}}%
\pgfpathcurveto{\pgfqpoint{9.678930in}{2.252242in}}{\pgfqpoint{9.676926in}{2.247404in}}{\pgfqpoint{9.676926in}{2.242360in}}%
\pgfpathcurveto{\pgfqpoint{9.676926in}{2.237317in}}{\pgfqpoint{9.678930in}{2.232479in}}{\pgfqpoint{9.682497in}{2.228913in}}%
\pgfpathcurveto{\pgfqpoint{9.686063in}{2.225346in}}{\pgfqpoint{9.690901in}{2.223342in}}{\pgfqpoint{9.695945in}{2.223342in}}%
\pgfpathclose%
\pgfusepath{fill}%
\end{pgfscope}%
\begin{pgfscope}%
\pgfpathrectangle{\pgfqpoint{6.572727in}{0.474100in}}{\pgfqpoint{4.227273in}{3.318700in}}%
\pgfusepath{clip}%
\pgfsetbuttcap%
\pgfsetroundjoin%
\definecolor{currentfill}{rgb}{0.127568,0.566949,0.550556}%
\pgfsetfillcolor{currentfill}%
\pgfsetfillopacity{0.700000}%
\pgfsetlinewidth{0.000000pt}%
\definecolor{currentstroke}{rgb}{0.000000,0.000000,0.000000}%
\pgfsetstrokecolor{currentstroke}%
\pgfsetstrokeopacity{0.700000}%
\pgfsetdash{}{0pt}%
\pgfpathmoveto{\pgfqpoint{8.414693in}{1.290831in}}%
\pgfpathcurveto{\pgfqpoint{8.419737in}{1.290831in}}{\pgfqpoint{8.424574in}{1.292835in}}{\pgfqpoint{8.428141in}{1.296401in}}%
\pgfpathcurveto{\pgfqpoint{8.431707in}{1.299967in}}{\pgfqpoint{8.433711in}{1.304805in}}{\pgfqpoint{8.433711in}{1.309849in}}%
\pgfpathcurveto{\pgfqpoint{8.433711in}{1.314892in}}{\pgfqpoint{8.431707in}{1.319730in}}{\pgfqpoint{8.428141in}{1.323297in}}%
\pgfpathcurveto{\pgfqpoint{8.424574in}{1.326863in}}{\pgfqpoint{8.419737in}{1.328867in}}{\pgfqpoint{8.414693in}{1.328867in}}%
\pgfpathcurveto{\pgfqpoint{8.409649in}{1.328867in}}{\pgfqpoint{8.404811in}{1.326863in}}{\pgfqpoint{8.401245in}{1.323297in}}%
\pgfpathcurveto{\pgfqpoint{8.397679in}{1.319730in}}{\pgfqpoint{8.395675in}{1.314892in}}{\pgfqpoint{8.395675in}{1.309849in}}%
\pgfpathcurveto{\pgfqpoint{8.395675in}{1.304805in}}{\pgfqpoint{8.397679in}{1.299967in}}{\pgfqpoint{8.401245in}{1.296401in}}%
\pgfpathcurveto{\pgfqpoint{8.404811in}{1.292835in}}{\pgfqpoint{8.409649in}{1.290831in}}{\pgfqpoint{8.414693in}{1.290831in}}%
\pgfpathclose%
\pgfusepath{fill}%
\end{pgfscope}%
\begin{pgfscope}%
\pgfpathrectangle{\pgfqpoint{6.572727in}{0.474100in}}{\pgfqpoint{4.227273in}{3.318700in}}%
\pgfusepath{clip}%
\pgfsetbuttcap%
\pgfsetroundjoin%
\definecolor{currentfill}{rgb}{0.127568,0.566949,0.550556}%
\pgfsetfillcolor{currentfill}%
\pgfsetfillopacity{0.700000}%
\pgfsetlinewidth{0.000000pt}%
\definecolor{currentstroke}{rgb}{0.000000,0.000000,0.000000}%
\pgfsetstrokecolor{currentstroke}%
\pgfsetstrokeopacity{0.700000}%
\pgfsetdash{}{0pt}%
\pgfpathmoveto{\pgfqpoint{8.411561in}{2.200381in}}%
\pgfpathcurveto{\pgfqpoint{8.416604in}{2.200381in}}{\pgfqpoint{8.421442in}{2.202385in}}{\pgfqpoint{8.425009in}{2.205952in}}%
\pgfpathcurveto{\pgfqpoint{8.428575in}{2.209518in}}{\pgfqpoint{8.430579in}{2.214356in}}{\pgfqpoint{8.430579in}{2.219399in}}%
\pgfpathcurveto{\pgfqpoint{8.430579in}{2.224443in}}{\pgfqpoint{8.428575in}{2.229281in}}{\pgfqpoint{8.425009in}{2.232847in}}%
\pgfpathcurveto{\pgfqpoint{8.421442in}{2.236414in}}{\pgfqpoint{8.416604in}{2.238418in}}{\pgfqpoint{8.411561in}{2.238418in}}%
\pgfpathcurveto{\pgfqpoint{8.406517in}{2.238418in}}{\pgfqpoint{8.401679in}{2.236414in}}{\pgfqpoint{8.398113in}{2.232847in}}%
\pgfpathcurveto{\pgfqpoint{8.394547in}{2.229281in}}{\pgfqpoint{8.392543in}{2.224443in}}{\pgfqpoint{8.392543in}{2.219399in}}%
\pgfpathcurveto{\pgfqpoint{8.392543in}{2.214356in}}{\pgfqpoint{8.394547in}{2.209518in}}{\pgfqpoint{8.398113in}{2.205952in}}%
\pgfpathcurveto{\pgfqpoint{8.401679in}{2.202385in}}{\pgfqpoint{8.406517in}{2.200381in}}{\pgfqpoint{8.411561in}{2.200381in}}%
\pgfpathclose%
\pgfusepath{fill}%
\end{pgfscope}%
\begin{pgfscope}%
\pgfpathrectangle{\pgfqpoint{6.572727in}{0.474100in}}{\pgfqpoint{4.227273in}{3.318700in}}%
\pgfusepath{clip}%
\pgfsetbuttcap%
\pgfsetroundjoin%
\definecolor{currentfill}{rgb}{0.127568,0.566949,0.550556}%
\pgfsetfillcolor{currentfill}%
\pgfsetfillopacity{0.700000}%
\pgfsetlinewidth{0.000000pt}%
\definecolor{currentstroke}{rgb}{0.000000,0.000000,0.000000}%
\pgfsetstrokecolor{currentstroke}%
\pgfsetstrokeopacity{0.700000}%
\pgfsetdash{}{0pt}%
\pgfpathmoveto{\pgfqpoint{7.618611in}{1.585276in}}%
\pgfpathcurveto{\pgfqpoint{7.623655in}{1.585276in}}{\pgfqpoint{7.628493in}{1.587280in}}{\pgfqpoint{7.632059in}{1.590846in}}%
\pgfpathcurveto{\pgfqpoint{7.635625in}{1.594413in}}{\pgfqpoint{7.637629in}{1.599251in}}{\pgfqpoint{7.637629in}{1.604294in}}%
\pgfpathcurveto{\pgfqpoint{7.637629in}{1.609338in}}{\pgfqpoint{7.635625in}{1.614176in}}{\pgfqpoint{7.632059in}{1.617742in}}%
\pgfpathcurveto{\pgfqpoint{7.628493in}{1.621309in}}{\pgfqpoint{7.623655in}{1.623312in}}{\pgfqpoint{7.618611in}{1.623312in}}%
\pgfpathcurveto{\pgfqpoint{7.613568in}{1.623312in}}{\pgfqpoint{7.608730in}{1.621309in}}{\pgfqpoint{7.605163in}{1.617742in}}%
\pgfpathcurveto{\pgfqpoint{7.601597in}{1.614176in}}{\pgfqpoint{7.599593in}{1.609338in}}{\pgfqpoint{7.599593in}{1.604294in}}%
\pgfpathcurveto{\pgfqpoint{7.599593in}{1.599251in}}{\pgfqpoint{7.601597in}{1.594413in}}{\pgfqpoint{7.605163in}{1.590846in}}%
\pgfpathcurveto{\pgfqpoint{7.608730in}{1.587280in}}{\pgfqpoint{7.613568in}{1.585276in}}{\pgfqpoint{7.618611in}{1.585276in}}%
\pgfpathclose%
\pgfusepath{fill}%
\end{pgfscope}%
\begin{pgfscope}%
\pgfpathrectangle{\pgfqpoint{6.572727in}{0.474100in}}{\pgfqpoint{4.227273in}{3.318700in}}%
\pgfusepath{clip}%
\pgfsetbuttcap%
\pgfsetroundjoin%
\definecolor{currentfill}{rgb}{0.127568,0.566949,0.550556}%
\pgfsetfillcolor{currentfill}%
\pgfsetfillopacity{0.700000}%
\pgfsetlinewidth{0.000000pt}%
\definecolor{currentstroke}{rgb}{0.000000,0.000000,0.000000}%
\pgfsetstrokecolor{currentstroke}%
\pgfsetstrokeopacity{0.700000}%
\pgfsetdash{}{0pt}%
\pgfpathmoveto{\pgfqpoint{8.589034in}{3.287556in}}%
\pgfpathcurveto{\pgfqpoint{8.594078in}{3.287556in}}{\pgfqpoint{8.598916in}{3.289560in}}{\pgfqpoint{8.602482in}{3.293126in}}%
\pgfpathcurveto{\pgfqpoint{8.606049in}{3.296693in}}{\pgfqpoint{8.608052in}{3.301530in}}{\pgfqpoint{8.608052in}{3.306574in}}%
\pgfpathcurveto{\pgfqpoint{8.608052in}{3.311618in}}{\pgfqpoint{8.606049in}{3.316456in}}{\pgfqpoint{8.602482in}{3.320022in}}%
\pgfpathcurveto{\pgfqpoint{8.598916in}{3.323588in}}{\pgfqpoint{8.594078in}{3.325592in}}{\pgfqpoint{8.589034in}{3.325592in}}%
\pgfpathcurveto{\pgfqpoint{8.583991in}{3.325592in}}{\pgfqpoint{8.579153in}{3.323588in}}{\pgfqpoint{8.575586in}{3.320022in}}%
\pgfpathcurveto{\pgfqpoint{8.572020in}{3.316456in}}{\pgfqpoint{8.570016in}{3.311618in}}{\pgfqpoint{8.570016in}{3.306574in}}%
\pgfpathcurveto{\pgfqpoint{8.570016in}{3.301530in}}{\pgfqpoint{8.572020in}{3.296693in}}{\pgfqpoint{8.575586in}{3.293126in}}%
\pgfpathcurveto{\pgfqpoint{8.579153in}{3.289560in}}{\pgfqpoint{8.583991in}{3.287556in}}{\pgfqpoint{8.589034in}{3.287556in}}%
\pgfpathclose%
\pgfusepath{fill}%
\end{pgfscope}%
\begin{pgfscope}%
\pgfpathrectangle{\pgfqpoint{6.572727in}{0.474100in}}{\pgfqpoint{4.227273in}{3.318700in}}%
\pgfusepath{clip}%
\pgfsetbuttcap%
\pgfsetroundjoin%
\definecolor{currentfill}{rgb}{0.127568,0.566949,0.550556}%
\pgfsetfillcolor{currentfill}%
\pgfsetfillopacity{0.700000}%
\pgfsetlinewidth{0.000000pt}%
\definecolor{currentstroke}{rgb}{0.000000,0.000000,0.000000}%
\pgfsetstrokecolor{currentstroke}%
\pgfsetstrokeopacity{0.700000}%
\pgfsetdash{}{0pt}%
\pgfpathmoveto{\pgfqpoint{7.887958in}{2.919524in}}%
\pgfpathcurveto{\pgfqpoint{7.893001in}{2.919524in}}{\pgfqpoint{7.897839in}{2.921528in}}{\pgfqpoint{7.901406in}{2.925094in}}%
\pgfpathcurveto{\pgfqpoint{7.904972in}{2.928660in}}{\pgfqpoint{7.906976in}{2.933498in}}{\pgfqpoint{7.906976in}{2.938542in}}%
\pgfpathcurveto{\pgfqpoint{7.906976in}{2.943585in}}{\pgfqpoint{7.904972in}{2.948423in}}{\pgfqpoint{7.901406in}{2.951990in}}%
\pgfpathcurveto{\pgfqpoint{7.897839in}{2.955556in}}{\pgfqpoint{7.893001in}{2.957560in}}{\pgfqpoint{7.887958in}{2.957560in}}%
\pgfpathcurveto{\pgfqpoint{7.882914in}{2.957560in}}{\pgfqpoint{7.878076in}{2.955556in}}{\pgfqpoint{7.874510in}{2.951990in}}%
\pgfpathcurveto{\pgfqpoint{7.870944in}{2.948423in}}{\pgfqpoint{7.868940in}{2.943585in}}{\pgfqpoint{7.868940in}{2.938542in}}%
\pgfpathcurveto{\pgfqpoint{7.868940in}{2.933498in}}{\pgfqpoint{7.870944in}{2.928660in}}{\pgfqpoint{7.874510in}{2.925094in}}%
\pgfpathcurveto{\pgfqpoint{7.878076in}{2.921528in}}{\pgfqpoint{7.882914in}{2.919524in}}{\pgfqpoint{7.887958in}{2.919524in}}%
\pgfpathclose%
\pgfusepath{fill}%
\end{pgfscope}%
\begin{pgfscope}%
\pgfpathrectangle{\pgfqpoint{6.572727in}{0.474100in}}{\pgfqpoint{4.227273in}{3.318700in}}%
\pgfusepath{clip}%
\pgfsetbuttcap%
\pgfsetroundjoin%
\definecolor{currentfill}{rgb}{0.127568,0.566949,0.550556}%
\pgfsetfillcolor{currentfill}%
\pgfsetfillopacity{0.700000}%
\pgfsetlinewidth{0.000000pt}%
\definecolor{currentstroke}{rgb}{0.000000,0.000000,0.000000}%
\pgfsetstrokecolor{currentstroke}%
\pgfsetstrokeopacity{0.700000}%
\pgfsetdash{}{0pt}%
\pgfpathmoveto{\pgfqpoint{8.083421in}{2.874764in}}%
\pgfpathcurveto{\pgfqpoint{8.088464in}{2.874764in}}{\pgfqpoint{8.093302in}{2.876768in}}{\pgfqpoint{8.096868in}{2.880334in}}%
\pgfpathcurveto{\pgfqpoint{8.100435in}{2.883900in}}{\pgfqpoint{8.102439in}{2.888738in}}{\pgfqpoint{8.102439in}{2.893782in}}%
\pgfpathcurveto{\pgfqpoint{8.102439in}{2.898826in}}{\pgfqpoint{8.100435in}{2.903663in}}{\pgfqpoint{8.096868in}{2.907230in}}%
\pgfpathcurveto{\pgfqpoint{8.093302in}{2.910796in}}{\pgfqpoint{8.088464in}{2.912800in}}{\pgfqpoint{8.083421in}{2.912800in}}%
\pgfpathcurveto{\pgfqpoint{8.078377in}{2.912800in}}{\pgfqpoint{8.073539in}{2.910796in}}{\pgfqpoint{8.069973in}{2.907230in}}%
\pgfpathcurveto{\pgfqpoint{8.066406in}{2.903663in}}{\pgfqpoint{8.064402in}{2.898826in}}{\pgfqpoint{8.064402in}{2.893782in}}%
\pgfpathcurveto{\pgfqpoint{8.064402in}{2.888738in}}{\pgfqpoint{8.066406in}{2.883900in}}{\pgfqpoint{8.069973in}{2.880334in}}%
\pgfpathcurveto{\pgfqpoint{8.073539in}{2.876768in}}{\pgfqpoint{8.078377in}{2.874764in}}{\pgfqpoint{8.083421in}{2.874764in}}%
\pgfpathclose%
\pgfusepath{fill}%
\end{pgfscope}%
\begin{pgfscope}%
\pgfpathrectangle{\pgfqpoint{6.572727in}{0.474100in}}{\pgfqpoint{4.227273in}{3.318700in}}%
\pgfusepath{clip}%
\pgfsetbuttcap%
\pgfsetroundjoin%
\definecolor{currentfill}{rgb}{0.993248,0.906157,0.143936}%
\pgfsetfillcolor{currentfill}%
\pgfsetfillopacity{0.700000}%
\pgfsetlinewidth{0.000000pt}%
\definecolor{currentstroke}{rgb}{0.000000,0.000000,0.000000}%
\pgfsetstrokecolor{currentstroke}%
\pgfsetstrokeopacity{0.700000}%
\pgfsetdash{}{0pt}%
\pgfpathmoveto{\pgfqpoint{9.490124in}{1.760659in}}%
\pgfpathcurveto{\pgfqpoint{9.495167in}{1.760659in}}{\pgfqpoint{9.500005in}{1.762663in}}{\pgfqpoint{9.503572in}{1.766229in}}%
\pgfpathcurveto{\pgfqpoint{9.507138in}{1.769796in}}{\pgfqpoint{9.509142in}{1.774634in}}{\pgfqpoint{9.509142in}{1.779677in}}%
\pgfpathcurveto{\pgfqpoint{9.509142in}{1.784721in}}{\pgfqpoint{9.507138in}{1.789559in}}{\pgfqpoint{9.503572in}{1.793125in}}%
\pgfpathcurveto{\pgfqpoint{9.500005in}{1.796692in}}{\pgfqpoint{9.495167in}{1.798695in}}{\pgfqpoint{9.490124in}{1.798695in}}%
\pgfpathcurveto{\pgfqpoint{9.485080in}{1.798695in}}{\pgfqpoint{9.480242in}{1.796692in}}{\pgfqpoint{9.476676in}{1.793125in}}%
\pgfpathcurveto{\pgfqpoint{9.473109in}{1.789559in}}{\pgfqpoint{9.471106in}{1.784721in}}{\pgfqpoint{9.471106in}{1.779677in}}%
\pgfpathcurveto{\pgfqpoint{9.471106in}{1.774634in}}{\pgfqpoint{9.473109in}{1.769796in}}{\pgfqpoint{9.476676in}{1.766229in}}%
\pgfpathcurveto{\pgfqpoint{9.480242in}{1.762663in}}{\pgfqpoint{9.485080in}{1.760659in}}{\pgfqpoint{9.490124in}{1.760659in}}%
\pgfpathclose%
\pgfusepath{fill}%
\end{pgfscope}%
\begin{pgfscope}%
\pgfpathrectangle{\pgfqpoint{6.572727in}{0.474100in}}{\pgfqpoint{4.227273in}{3.318700in}}%
\pgfusepath{clip}%
\pgfsetbuttcap%
\pgfsetroundjoin%
\definecolor{currentfill}{rgb}{0.127568,0.566949,0.550556}%
\pgfsetfillcolor{currentfill}%
\pgfsetfillopacity{0.700000}%
\pgfsetlinewidth{0.000000pt}%
\definecolor{currentstroke}{rgb}{0.000000,0.000000,0.000000}%
\pgfsetstrokecolor{currentstroke}%
\pgfsetstrokeopacity{0.700000}%
\pgfsetdash{}{0pt}%
\pgfpathmoveto{\pgfqpoint{7.472172in}{1.564153in}}%
\pgfpathcurveto{\pgfqpoint{7.477216in}{1.564153in}}{\pgfqpoint{7.482054in}{1.566157in}}{\pgfqpoint{7.485620in}{1.569723in}}%
\pgfpathcurveto{\pgfqpoint{7.489187in}{1.573289in}}{\pgfqpoint{7.491191in}{1.578127in}}{\pgfqpoint{7.491191in}{1.583171in}}%
\pgfpathcurveto{\pgfqpoint{7.491191in}{1.588214in}}{\pgfqpoint{7.489187in}{1.593052in}}{\pgfqpoint{7.485620in}{1.596619in}}%
\pgfpathcurveto{\pgfqpoint{7.482054in}{1.600185in}}{\pgfqpoint{7.477216in}{1.602189in}}{\pgfqpoint{7.472172in}{1.602189in}}%
\pgfpathcurveto{\pgfqpoint{7.467129in}{1.602189in}}{\pgfqpoint{7.462291in}{1.600185in}}{\pgfqpoint{7.458725in}{1.596619in}}%
\pgfpathcurveto{\pgfqpoint{7.455158in}{1.593052in}}{\pgfqpoint{7.453154in}{1.588214in}}{\pgfqpoint{7.453154in}{1.583171in}}%
\pgfpathcurveto{\pgfqpoint{7.453154in}{1.578127in}}{\pgfqpoint{7.455158in}{1.573289in}}{\pgfqpoint{7.458725in}{1.569723in}}%
\pgfpathcurveto{\pgfqpoint{7.462291in}{1.566157in}}{\pgfqpoint{7.467129in}{1.564153in}}{\pgfqpoint{7.472172in}{1.564153in}}%
\pgfpathclose%
\pgfusepath{fill}%
\end{pgfscope}%
\begin{pgfscope}%
\pgfpathrectangle{\pgfqpoint{6.572727in}{0.474100in}}{\pgfqpoint{4.227273in}{3.318700in}}%
\pgfusepath{clip}%
\pgfsetbuttcap%
\pgfsetroundjoin%
\definecolor{currentfill}{rgb}{0.993248,0.906157,0.143936}%
\pgfsetfillcolor{currentfill}%
\pgfsetfillopacity{0.700000}%
\pgfsetlinewidth{0.000000pt}%
\definecolor{currentstroke}{rgb}{0.000000,0.000000,0.000000}%
\pgfsetstrokecolor{currentstroke}%
\pgfsetstrokeopacity{0.700000}%
\pgfsetdash{}{0pt}%
\pgfpathmoveto{\pgfqpoint{8.958383in}{1.694388in}}%
\pgfpathcurveto{\pgfqpoint{8.963427in}{1.694388in}}{\pgfqpoint{8.968265in}{1.696392in}}{\pgfqpoint{8.971831in}{1.699958in}}%
\pgfpathcurveto{\pgfqpoint{8.975398in}{1.703524in}}{\pgfqpoint{8.977402in}{1.708362in}}{\pgfqpoint{8.977402in}{1.713406in}}%
\pgfpathcurveto{\pgfqpoint{8.977402in}{1.718450in}}{\pgfqpoint{8.975398in}{1.723287in}}{\pgfqpoint{8.971831in}{1.726854in}}%
\pgfpathcurveto{\pgfqpoint{8.968265in}{1.730420in}}{\pgfqpoint{8.963427in}{1.732424in}}{\pgfqpoint{8.958383in}{1.732424in}}%
\pgfpathcurveto{\pgfqpoint{8.953340in}{1.732424in}}{\pgfqpoint{8.948502in}{1.730420in}}{\pgfqpoint{8.944936in}{1.726854in}}%
\pgfpathcurveto{\pgfqpoint{8.941369in}{1.723287in}}{\pgfqpoint{8.939365in}{1.718450in}}{\pgfqpoint{8.939365in}{1.713406in}}%
\pgfpathcurveto{\pgfqpoint{8.939365in}{1.708362in}}{\pgfqpoint{8.941369in}{1.703524in}}{\pgfqpoint{8.944936in}{1.699958in}}%
\pgfpathcurveto{\pgfqpoint{8.948502in}{1.696392in}}{\pgfqpoint{8.953340in}{1.694388in}}{\pgfqpoint{8.958383in}{1.694388in}}%
\pgfpathclose%
\pgfusepath{fill}%
\end{pgfscope}%
\begin{pgfscope}%
\pgfpathrectangle{\pgfqpoint{6.572727in}{0.474100in}}{\pgfqpoint{4.227273in}{3.318700in}}%
\pgfusepath{clip}%
\pgfsetbuttcap%
\pgfsetroundjoin%
\definecolor{currentfill}{rgb}{0.993248,0.906157,0.143936}%
\pgfsetfillcolor{currentfill}%
\pgfsetfillopacity{0.700000}%
\pgfsetlinewidth{0.000000pt}%
\definecolor{currentstroke}{rgb}{0.000000,0.000000,0.000000}%
\pgfsetstrokecolor{currentstroke}%
\pgfsetstrokeopacity{0.700000}%
\pgfsetdash{}{0pt}%
\pgfpathmoveto{\pgfqpoint{9.053885in}{1.734843in}}%
\pgfpathcurveto{\pgfqpoint{9.058929in}{1.734843in}}{\pgfqpoint{9.063766in}{1.736846in}}{\pgfqpoint{9.067333in}{1.740413in}}%
\pgfpathcurveto{\pgfqpoint{9.070899in}{1.743979in}}{\pgfqpoint{9.072903in}{1.748817in}}{\pgfqpoint{9.072903in}{1.753861in}}%
\pgfpathcurveto{\pgfqpoint{9.072903in}{1.758904in}}{\pgfqpoint{9.070899in}{1.763742in}}{\pgfqpoint{9.067333in}{1.767309in}}%
\pgfpathcurveto{\pgfqpoint{9.063766in}{1.770875in}}{\pgfqpoint{9.058929in}{1.772879in}}{\pgfqpoint{9.053885in}{1.772879in}}%
\pgfpathcurveto{\pgfqpoint{9.048841in}{1.772879in}}{\pgfqpoint{9.044004in}{1.770875in}}{\pgfqpoint{9.040437in}{1.767309in}}%
\pgfpathcurveto{\pgfqpoint{9.036871in}{1.763742in}}{\pgfqpoint{9.034867in}{1.758904in}}{\pgfqpoint{9.034867in}{1.753861in}}%
\pgfpathcurveto{\pgfqpoint{9.034867in}{1.748817in}}{\pgfqpoint{9.036871in}{1.743979in}}{\pgfqpoint{9.040437in}{1.740413in}}%
\pgfpathcurveto{\pgfqpoint{9.044004in}{1.736846in}}{\pgfqpoint{9.048841in}{1.734843in}}{\pgfqpoint{9.053885in}{1.734843in}}%
\pgfpathclose%
\pgfusepath{fill}%
\end{pgfscope}%
\begin{pgfscope}%
\pgfpathrectangle{\pgfqpoint{6.572727in}{0.474100in}}{\pgfqpoint{4.227273in}{3.318700in}}%
\pgfusepath{clip}%
\pgfsetbuttcap%
\pgfsetroundjoin%
\definecolor{currentfill}{rgb}{0.993248,0.906157,0.143936}%
\pgfsetfillcolor{currentfill}%
\pgfsetfillopacity{0.700000}%
\pgfsetlinewidth{0.000000pt}%
\definecolor{currentstroke}{rgb}{0.000000,0.000000,0.000000}%
\pgfsetstrokecolor{currentstroke}%
\pgfsetstrokeopacity{0.700000}%
\pgfsetdash{}{0pt}%
\pgfpathmoveto{\pgfqpoint{9.016784in}{2.025438in}}%
\pgfpathcurveto{\pgfqpoint{9.021828in}{2.025438in}}{\pgfqpoint{9.026666in}{2.027442in}}{\pgfqpoint{9.030232in}{2.031009in}}%
\pgfpathcurveto{\pgfqpoint{9.033799in}{2.034575in}}{\pgfqpoint{9.035802in}{2.039413in}}{\pgfqpoint{9.035802in}{2.044457in}}%
\pgfpathcurveto{\pgfqpoint{9.035802in}{2.049500in}}{\pgfqpoint{9.033799in}{2.054338in}}{\pgfqpoint{9.030232in}{2.057904in}}%
\pgfpathcurveto{\pgfqpoint{9.026666in}{2.061471in}}{\pgfqpoint{9.021828in}{2.063475in}}{\pgfqpoint{9.016784in}{2.063475in}}%
\pgfpathcurveto{\pgfqpoint{9.011741in}{2.063475in}}{\pgfqpoint{9.006903in}{2.061471in}}{\pgfqpoint{9.003336in}{2.057904in}}%
\pgfpathcurveto{\pgfqpoint{8.999770in}{2.054338in}}{\pgfqpoint{8.997766in}{2.049500in}}{\pgfqpoint{8.997766in}{2.044457in}}%
\pgfpathcurveto{\pgfqpoint{8.997766in}{2.039413in}}{\pgfqpoint{8.999770in}{2.034575in}}{\pgfqpoint{9.003336in}{2.031009in}}%
\pgfpathcurveto{\pgfqpoint{9.006903in}{2.027442in}}{\pgfqpoint{9.011741in}{2.025438in}}{\pgfqpoint{9.016784in}{2.025438in}}%
\pgfpathclose%
\pgfusepath{fill}%
\end{pgfscope}%
\begin{pgfscope}%
\pgfpathrectangle{\pgfqpoint{6.572727in}{0.474100in}}{\pgfqpoint{4.227273in}{3.318700in}}%
\pgfusepath{clip}%
\pgfsetbuttcap%
\pgfsetroundjoin%
\definecolor{currentfill}{rgb}{0.127568,0.566949,0.550556}%
\pgfsetfillcolor{currentfill}%
\pgfsetfillopacity{0.700000}%
\pgfsetlinewidth{0.000000pt}%
\definecolor{currentstroke}{rgb}{0.000000,0.000000,0.000000}%
\pgfsetstrokecolor{currentstroke}%
\pgfsetstrokeopacity{0.700000}%
\pgfsetdash{}{0pt}%
\pgfpathmoveto{\pgfqpoint{7.885119in}{1.287685in}}%
\pgfpathcurveto{\pgfqpoint{7.890163in}{1.287685in}}{\pgfqpoint{7.895000in}{1.289688in}}{\pgfqpoint{7.898567in}{1.293255in}}%
\pgfpathcurveto{\pgfqpoint{7.902133in}{1.296821in}}{\pgfqpoint{7.904137in}{1.301659in}}{\pgfqpoint{7.904137in}{1.306703in}}%
\pgfpathcurveto{\pgfqpoint{7.904137in}{1.311746in}}{\pgfqpoint{7.902133in}{1.316584in}}{\pgfqpoint{7.898567in}{1.320151in}}%
\pgfpathcurveto{\pgfqpoint{7.895000in}{1.323717in}}{\pgfqpoint{7.890163in}{1.325721in}}{\pgfqpoint{7.885119in}{1.325721in}}%
\pgfpathcurveto{\pgfqpoint{7.880075in}{1.325721in}}{\pgfqpoint{7.875238in}{1.323717in}}{\pgfqpoint{7.871671in}{1.320151in}}%
\pgfpathcurveto{\pgfqpoint{7.868105in}{1.316584in}}{\pgfqpoint{7.866101in}{1.311746in}}{\pgfqpoint{7.866101in}{1.306703in}}%
\pgfpathcurveto{\pgfqpoint{7.866101in}{1.301659in}}{\pgfqpoint{7.868105in}{1.296821in}}{\pgfqpoint{7.871671in}{1.293255in}}%
\pgfpathcurveto{\pgfqpoint{7.875238in}{1.289688in}}{\pgfqpoint{7.880075in}{1.287685in}}{\pgfqpoint{7.885119in}{1.287685in}}%
\pgfpathclose%
\pgfusepath{fill}%
\end{pgfscope}%
\begin{pgfscope}%
\pgfpathrectangle{\pgfqpoint{6.572727in}{0.474100in}}{\pgfqpoint{4.227273in}{3.318700in}}%
\pgfusepath{clip}%
\pgfsetbuttcap%
\pgfsetroundjoin%
\definecolor{currentfill}{rgb}{0.127568,0.566949,0.550556}%
\pgfsetfillcolor{currentfill}%
\pgfsetfillopacity{0.700000}%
\pgfsetlinewidth{0.000000pt}%
\definecolor{currentstroke}{rgb}{0.000000,0.000000,0.000000}%
\pgfsetstrokecolor{currentstroke}%
\pgfsetstrokeopacity{0.700000}%
\pgfsetdash{}{0pt}%
\pgfpathmoveto{\pgfqpoint{7.830735in}{2.568934in}}%
\pgfpathcurveto{\pgfqpoint{7.835779in}{2.568934in}}{\pgfqpoint{7.840616in}{2.570938in}}{\pgfqpoint{7.844183in}{2.574504in}}%
\pgfpathcurveto{\pgfqpoint{7.847749in}{2.578071in}}{\pgfqpoint{7.849753in}{2.582908in}}{\pgfqpoint{7.849753in}{2.587952in}}%
\pgfpathcurveto{\pgfqpoint{7.849753in}{2.592996in}}{\pgfqpoint{7.847749in}{2.597834in}}{\pgfqpoint{7.844183in}{2.601400in}}%
\pgfpathcurveto{\pgfqpoint{7.840616in}{2.604966in}}{\pgfqpoint{7.835779in}{2.606970in}}{\pgfqpoint{7.830735in}{2.606970in}}%
\pgfpathcurveto{\pgfqpoint{7.825691in}{2.606970in}}{\pgfqpoint{7.820853in}{2.604966in}}{\pgfqpoint{7.817287in}{2.601400in}}%
\pgfpathcurveto{\pgfqpoint{7.813721in}{2.597834in}}{\pgfqpoint{7.811717in}{2.592996in}}{\pgfqpoint{7.811717in}{2.587952in}}%
\pgfpathcurveto{\pgfqpoint{7.811717in}{2.582908in}}{\pgfqpoint{7.813721in}{2.578071in}}{\pgfqpoint{7.817287in}{2.574504in}}%
\pgfpathcurveto{\pgfqpoint{7.820853in}{2.570938in}}{\pgfqpoint{7.825691in}{2.568934in}}{\pgfqpoint{7.830735in}{2.568934in}}%
\pgfpathclose%
\pgfusepath{fill}%
\end{pgfscope}%
\begin{pgfscope}%
\pgfpathrectangle{\pgfqpoint{6.572727in}{0.474100in}}{\pgfqpoint{4.227273in}{3.318700in}}%
\pgfusepath{clip}%
\pgfsetbuttcap%
\pgfsetroundjoin%
\definecolor{currentfill}{rgb}{0.127568,0.566949,0.550556}%
\pgfsetfillcolor{currentfill}%
\pgfsetfillopacity{0.700000}%
\pgfsetlinewidth{0.000000pt}%
\definecolor{currentstroke}{rgb}{0.000000,0.000000,0.000000}%
\pgfsetstrokecolor{currentstroke}%
\pgfsetstrokeopacity{0.700000}%
\pgfsetdash{}{0pt}%
\pgfpathmoveto{\pgfqpoint{7.642837in}{1.536358in}}%
\pgfpathcurveto{\pgfqpoint{7.647881in}{1.536358in}}{\pgfqpoint{7.652719in}{1.538361in}}{\pgfqpoint{7.656285in}{1.541928in}}%
\pgfpathcurveto{\pgfqpoint{7.659851in}{1.545494in}}{\pgfqpoint{7.661855in}{1.550332in}}{\pgfqpoint{7.661855in}{1.555376in}}%
\pgfpathcurveto{\pgfqpoint{7.661855in}{1.560419in}}{\pgfqpoint{7.659851in}{1.565257in}}{\pgfqpoint{7.656285in}{1.568824in}}%
\pgfpathcurveto{\pgfqpoint{7.652719in}{1.572390in}}{\pgfqpoint{7.647881in}{1.574394in}}{\pgfqpoint{7.642837in}{1.574394in}}%
\pgfpathcurveto{\pgfqpoint{7.637793in}{1.574394in}}{\pgfqpoint{7.632956in}{1.572390in}}{\pgfqpoint{7.629389in}{1.568824in}}%
\pgfpathcurveto{\pgfqpoint{7.625823in}{1.565257in}}{\pgfqpoint{7.623819in}{1.560419in}}{\pgfqpoint{7.623819in}{1.555376in}}%
\pgfpathcurveto{\pgfqpoint{7.623819in}{1.550332in}}{\pgfqpoint{7.625823in}{1.545494in}}{\pgfqpoint{7.629389in}{1.541928in}}%
\pgfpathcurveto{\pgfqpoint{7.632956in}{1.538361in}}{\pgfqpoint{7.637793in}{1.536358in}}{\pgfqpoint{7.642837in}{1.536358in}}%
\pgfpathclose%
\pgfusepath{fill}%
\end{pgfscope}%
\begin{pgfscope}%
\pgfpathrectangle{\pgfqpoint{6.572727in}{0.474100in}}{\pgfqpoint{4.227273in}{3.318700in}}%
\pgfusepath{clip}%
\pgfsetbuttcap%
\pgfsetroundjoin%
\definecolor{currentfill}{rgb}{0.127568,0.566949,0.550556}%
\pgfsetfillcolor{currentfill}%
\pgfsetfillopacity{0.700000}%
\pgfsetlinewidth{0.000000pt}%
\definecolor{currentstroke}{rgb}{0.000000,0.000000,0.000000}%
\pgfsetstrokecolor{currentstroke}%
\pgfsetstrokeopacity{0.700000}%
\pgfsetdash{}{0pt}%
\pgfpathmoveto{\pgfqpoint{8.592896in}{2.761975in}}%
\pgfpathcurveto{\pgfqpoint{8.597940in}{2.761975in}}{\pgfqpoint{8.602778in}{2.763979in}}{\pgfqpoint{8.606344in}{2.767545in}}%
\pgfpathcurveto{\pgfqpoint{8.609910in}{2.771111in}}{\pgfqpoint{8.611914in}{2.775949in}}{\pgfqpoint{8.611914in}{2.780993in}}%
\pgfpathcurveto{\pgfqpoint{8.611914in}{2.786037in}}{\pgfqpoint{8.609910in}{2.790874in}}{\pgfqpoint{8.606344in}{2.794441in}}%
\pgfpathcurveto{\pgfqpoint{8.602778in}{2.798007in}}{\pgfqpoint{8.597940in}{2.800011in}}{\pgfqpoint{8.592896in}{2.800011in}}%
\pgfpathcurveto{\pgfqpoint{8.587853in}{2.800011in}}{\pgfqpoint{8.583015in}{2.798007in}}{\pgfqpoint{8.579448in}{2.794441in}}%
\pgfpathcurveto{\pgfqpoint{8.575882in}{2.790874in}}{\pgfqpoint{8.573878in}{2.786037in}}{\pgfqpoint{8.573878in}{2.780993in}}%
\pgfpathcurveto{\pgfqpoint{8.573878in}{2.775949in}}{\pgfqpoint{8.575882in}{2.771111in}}{\pgfqpoint{8.579448in}{2.767545in}}%
\pgfpathcurveto{\pgfqpoint{8.583015in}{2.763979in}}{\pgfqpoint{8.587853in}{2.761975in}}{\pgfqpoint{8.592896in}{2.761975in}}%
\pgfpathclose%
\pgfusepath{fill}%
\end{pgfscope}%
\begin{pgfscope}%
\pgfpathrectangle{\pgfqpoint{6.572727in}{0.474100in}}{\pgfqpoint{4.227273in}{3.318700in}}%
\pgfusepath{clip}%
\pgfsetbuttcap%
\pgfsetroundjoin%
\definecolor{currentfill}{rgb}{0.127568,0.566949,0.550556}%
\pgfsetfillcolor{currentfill}%
\pgfsetfillopacity{0.700000}%
\pgfsetlinewidth{0.000000pt}%
\definecolor{currentstroke}{rgb}{0.000000,0.000000,0.000000}%
\pgfsetstrokecolor{currentstroke}%
\pgfsetstrokeopacity{0.700000}%
\pgfsetdash{}{0pt}%
\pgfpathmoveto{\pgfqpoint{8.318046in}{2.796986in}}%
\pgfpathcurveto{\pgfqpoint{8.323090in}{2.796986in}}{\pgfqpoint{8.327928in}{2.798990in}}{\pgfqpoint{8.331494in}{2.802556in}}%
\pgfpathcurveto{\pgfqpoint{8.335061in}{2.806122in}}{\pgfqpoint{8.337065in}{2.810960in}}{\pgfqpoint{8.337065in}{2.816004in}}%
\pgfpathcurveto{\pgfqpoint{8.337065in}{2.821048in}}{\pgfqpoint{8.335061in}{2.825885in}}{\pgfqpoint{8.331494in}{2.829452in}}%
\pgfpathcurveto{\pgfqpoint{8.327928in}{2.833018in}}{\pgfqpoint{8.323090in}{2.835022in}}{\pgfqpoint{8.318046in}{2.835022in}}%
\pgfpathcurveto{\pgfqpoint{8.313003in}{2.835022in}}{\pgfqpoint{8.308165in}{2.833018in}}{\pgfqpoint{8.304599in}{2.829452in}}%
\pgfpathcurveto{\pgfqpoint{8.301032in}{2.825885in}}{\pgfqpoint{8.299028in}{2.821048in}}{\pgfqpoint{8.299028in}{2.816004in}}%
\pgfpathcurveto{\pgfqpoint{8.299028in}{2.810960in}}{\pgfqpoint{8.301032in}{2.806122in}}{\pgfqpoint{8.304599in}{2.802556in}}%
\pgfpathcurveto{\pgfqpoint{8.308165in}{2.798990in}}{\pgfqpoint{8.313003in}{2.796986in}}{\pgfqpoint{8.318046in}{2.796986in}}%
\pgfpathclose%
\pgfusepath{fill}%
\end{pgfscope}%
\begin{pgfscope}%
\pgfpathrectangle{\pgfqpoint{6.572727in}{0.474100in}}{\pgfqpoint{4.227273in}{3.318700in}}%
\pgfusepath{clip}%
\pgfsetbuttcap%
\pgfsetroundjoin%
\definecolor{currentfill}{rgb}{0.127568,0.566949,0.550556}%
\pgfsetfillcolor{currentfill}%
\pgfsetfillopacity{0.700000}%
\pgfsetlinewidth{0.000000pt}%
\definecolor{currentstroke}{rgb}{0.000000,0.000000,0.000000}%
\pgfsetstrokecolor{currentstroke}%
\pgfsetstrokeopacity{0.700000}%
\pgfsetdash{}{0pt}%
\pgfpathmoveto{\pgfqpoint{8.800912in}{2.999439in}}%
\pgfpathcurveto{\pgfqpoint{8.805956in}{2.999439in}}{\pgfqpoint{8.810794in}{3.001443in}}{\pgfqpoint{8.814360in}{3.005009in}}%
\pgfpathcurveto{\pgfqpoint{8.817926in}{3.008575in}}{\pgfqpoint{8.819930in}{3.013413in}}{\pgfqpoint{8.819930in}{3.018457in}}%
\pgfpathcurveto{\pgfqpoint{8.819930in}{3.023500in}}{\pgfqpoint{8.817926in}{3.028338in}}{\pgfqpoint{8.814360in}{3.031905in}}%
\pgfpathcurveto{\pgfqpoint{8.810794in}{3.035471in}}{\pgfqpoint{8.805956in}{3.037475in}}{\pgfqpoint{8.800912in}{3.037475in}}%
\pgfpathcurveto{\pgfqpoint{8.795869in}{3.037475in}}{\pgfqpoint{8.791031in}{3.035471in}}{\pgfqpoint{8.787464in}{3.031905in}}%
\pgfpathcurveto{\pgfqpoint{8.783898in}{3.028338in}}{\pgfqpoint{8.781894in}{3.023500in}}{\pgfqpoint{8.781894in}{3.018457in}}%
\pgfpathcurveto{\pgfqpoint{8.781894in}{3.013413in}}{\pgfqpoint{8.783898in}{3.008575in}}{\pgfqpoint{8.787464in}{3.005009in}}%
\pgfpathcurveto{\pgfqpoint{8.791031in}{3.001443in}}{\pgfqpoint{8.795869in}{2.999439in}}{\pgfqpoint{8.800912in}{2.999439in}}%
\pgfpathclose%
\pgfusepath{fill}%
\end{pgfscope}%
\begin{pgfscope}%
\pgfpathrectangle{\pgfqpoint{6.572727in}{0.474100in}}{\pgfqpoint{4.227273in}{3.318700in}}%
\pgfusepath{clip}%
\pgfsetbuttcap%
\pgfsetroundjoin%
\definecolor{currentfill}{rgb}{0.127568,0.566949,0.550556}%
\pgfsetfillcolor{currentfill}%
\pgfsetfillopacity{0.700000}%
\pgfsetlinewidth{0.000000pt}%
\definecolor{currentstroke}{rgb}{0.000000,0.000000,0.000000}%
\pgfsetstrokecolor{currentstroke}%
\pgfsetstrokeopacity{0.700000}%
\pgfsetdash{}{0pt}%
\pgfpathmoveto{\pgfqpoint{8.195678in}{3.175359in}}%
\pgfpathcurveto{\pgfqpoint{8.200722in}{3.175359in}}{\pgfqpoint{8.205560in}{3.177363in}}{\pgfqpoint{8.209126in}{3.180929in}}%
\pgfpathcurveto{\pgfqpoint{8.212693in}{3.184495in}}{\pgfqpoint{8.214696in}{3.189333in}}{\pgfqpoint{8.214696in}{3.194377in}}%
\pgfpathcurveto{\pgfqpoint{8.214696in}{3.199420in}}{\pgfqpoint{8.212693in}{3.204258in}}{\pgfqpoint{8.209126in}{3.207825in}}%
\pgfpathcurveto{\pgfqpoint{8.205560in}{3.211391in}}{\pgfqpoint{8.200722in}{3.213395in}}{\pgfqpoint{8.195678in}{3.213395in}}%
\pgfpathcurveto{\pgfqpoint{8.190635in}{3.213395in}}{\pgfqpoint{8.185797in}{3.211391in}}{\pgfqpoint{8.182230in}{3.207825in}}%
\pgfpathcurveto{\pgfqpoint{8.178664in}{3.204258in}}{\pgfqpoint{8.176660in}{3.199420in}}{\pgfqpoint{8.176660in}{3.194377in}}%
\pgfpathcurveto{\pgfqpoint{8.176660in}{3.189333in}}{\pgfqpoint{8.178664in}{3.184495in}}{\pgfqpoint{8.182230in}{3.180929in}}%
\pgfpathcurveto{\pgfqpoint{8.185797in}{3.177363in}}{\pgfqpoint{8.190635in}{3.175359in}}{\pgfqpoint{8.195678in}{3.175359in}}%
\pgfpathclose%
\pgfusepath{fill}%
\end{pgfscope}%
\begin{pgfscope}%
\pgfpathrectangle{\pgfqpoint{6.572727in}{0.474100in}}{\pgfqpoint{4.227273in}{3.318700in}}%
\pgfusepath{clip}%
\pgfsetbuttcap%
\pgfsetroundjoin%
\definecolor{currentfill}{rgb}{0.127568,0.566949,0.550556}%
\pgfsetfillcolor{currentfill}%
\pgfsetfillopacity{0.700000}%
\pgfsetlinewidth{0.000000pt}%
\definecolor{currentstroke}{rgb}{0.000000,0.000000,0.000000}%
\pgfsetstrokecolor{currentstroke}%
\pgfsetstrokeopacity{0.700000}%
\pgfsetdash{}{0pt}%
\pgfpathmoveto{\pgfqpoint{8.146326in}{3.083046in}}%
\pgfpathcurveto{\pgfqpoint{8.151370in}{3.083046in}}{\pgfqpoint{8.156208in}{3.085049in}}{\pgfqpoint{8.159774in}{3.088616in}}%
\pgfpathcurveto{\pgfqpoint{8.163341in}{3.092182in}}{\pgfqpoint{8.165345in}{3.097020in}}{\pgfqpoint{8.165345in}{3.102064in}}%
\pgfpathcurveto{\pgfqpoint{8.165345in}{3.107107in}}{\pgfqpoint{8.163341in}{3.111945in}}{\pgfqpoint{8.159774in}{3.115512in}}%
\pgfpathcurveto{\pgfqpoint{8.156208in}{3.119078in}}{\pgfqpoint{8.151370in}{3.121082in}}{\pgfqpoint{8.146326in}{3.121082in}}%
\pgfpathcurveto{\pgfqpoint{8.141283in}{3.121082in}}{\pgfqpoint{8.136445in}{3.119078in}}{\pgfqpoint{8.132879in}{3.115512in}}%
\pgfpathcurveto{\pgfqpoint{8.129312in}{3.111945in}}{\pgfqpoint{8.127308in}{3.107107in}}{\pgfqpoint{8.127308in}{3.102064in}}%
\pgfpathcurveto{\pgfqpoint{8.127308in}{3.097020in}}{\pgfqpoint{8.129312in}{3.092182in}}{\pgfqpoint{8.132879in}{3.088616in}}%
\pgfpathcurveto{\pgfqpoint{8.136445in}{3.085049in}}{\pgfqpoint{8.141283in}{3.083046in}}{\pgfqpoint{8.146326in}{3.083046in}}%
\pgfpathclose%
\pgfusepath{fill}%
\end{pgfscope}%
\begin{pgfscope}%
\pgfpathrectangle{\pgfqpoint{6.572727in}{0.474100in}}{\pgfqpoint{4.227273in}{3.318700in}}%
\pgfusepath{clip}%
\pgfsetbuttcap%
\pgfsetroundjoin%
\definecolor{currentfill}{rgb}{0.127568,0.566949,0.550556}%
\pgfsetfillcolor{currentfill}%
\pgfsetfillopacity{0.700000}%
\pgfsetlinewidth{0.000000pt}%
\definecolor{currentstroke}{rgb}{0.000000,0.000000,0.000000}%
\pgfsetstrokecolor{currentstroke}%
\pgfsetstrokeopacity{0.700000}%
\pgfsetdash{}{0pt}%
\pgfpathmoveto{\pgfqpoint{7.797254in}{2.711690in}}%
\pgfpathcurveto{\pgfqpoint{7.802297in}{2.711690in}}{\pgfqpoint{7.807135in}{2.713694in}}{\pgfqpoint{7.810702in}{2.717261in}}%
\pgfpathcurveto{\pgfqpoint{7.814268in}{2.720827in}}{\pgfqpoint{7.816272in}{2.725665in}}{\pgfqpoint{7.816272in}{2.730708in}}%
\pgfpathcurveto{\pgfqpoint{7.816272in}{2.735752in}}{\pgfqpoint{7.814268in}{2.740590in}}{\pgfqpoint{7.810702in}{2.744156in}}%
\pgfpathcurveto{\pgfqpoint{7.807135in}{2.747723in}}{\pgfqpoint{7.802297in}{2.749727in}}{\pgfqpoint{7.797254in}{2.749727in}}%
\pgfpathcurveto{\pgfqpoint{7.792210in}{2.749727in}}{\pgfqpoint{7.787372in}{2.747723in}}{\pgfqpoint{7.783806in}{2.744156in}}%
\pgfpathcurveto{\pgfqpoint{7.780239in}{2.740590in}}{\pgfqpoint{7.778236in}{2.735752in}}{\pgfqpoint{7.778236in}{2.730708in}}%
\pgfpathcurveto{\pgfqpoint{7.778236in}{2.725665in}}{\pgfqpoint{7.780239in}{2.720827in}}{\pgfqpoint{7.783806in}{2.717261in}}%
\pgfpathcurveto{\pgfqpoint{7.787372in}{2.713694in}}{\pgfqpoint{7.792210in}{2.711690in}}{\pgfqpoint{7.797254in}{2.711690in}}%
\pgfpathclose%
\pgfusepath{fill}%
\end{pgfscope}%
\begin{pgfscope}%
\pgfpathrectangle{\pgfqpoint{6.572727in}{0.474100in}}{\pgfqpoint{4.227273in}{3.318700in}}%
\pgfusepath{clip}%
\pgfsetbuttcap%
\pgfsetroundjoin%
\definecolor{currentfill}{rgb}{0.127568,0.566949,0.550556}%
\pgfsetfillcolor{currentfill}%
\pgfsetfillopacity{0.700000}%
\pgfsetlinewidth{0.000000pt}%
\definecolor{currentstroke}{rgb}{0.000000,0.000000,0.000000}%
\pgfsetstrokecolor{currentstroke}%
\pgfsetstrokeopacity{0.700000}%
\pgfsetdash{}{0pt}%
\pgfpathmoveto{\pgfqpoint{7.873351in}{2.025215in}}%
\pgfpathcurveto{\pgfqpoint{7.878395in}{2.025215in}}{\pgfqpoint{7.883232in}{2.027219in}}{\pgfqpoint{7.886799in}{2.030785in}}%
\pgfpathcurveto{\pgfqpoint{7.890365in}{2.034352in}}{\pgfqpoint{7.892369in}{2.039190in}}{\pgfqpoint{7.892369in}{2.044233in}}%
\pgfpathcurveto{\pgfqpoint{7.892369in}{2.049277in}}{\pgfqpoint{7.890365in}{2.054115in}}{\pgfqpoint{7.886799in}{2.057681in}}%
\pgfpathcurveto{\pgfqpoint{7.883232in}{2.061248in}}{\pgfqpoint{7.878395in}{2.063251in}}{\pgfqpoint{7.873351in}{2.063251in}}%
\pgfpathcurveto{\pgfqpoint{7.868307in}{2.063251in}}{\pgfqpoint{7.863469in}{2.061248in}}{\pgfqpoint{7.859903in}{2.057681in}}%
\pgfpathcurveto{\pgfqpoint{7.856337in}{2.054115in}}{\pgfqpoint{7.854333in}{2.049277in}}{\pgfqpoint{7.854333in}{2.044233in}}%
\pgfpathcurveto{\pgfqpoint{7.854333in}{2.039190in}}{\pgfqpoint{7.856337in}{2.034352in}}{\pgfqpoint{7.859903in}{2.030785in}}%
\pgfpathcurveto{\pgfqpoint{7.863469in}{2.027219in}}{\pgfqpoint{7.868307in}{2.025215in}}{\pgfqpoint{7.873351in}{2.025215in}}%
\pgfpathclose%
\pgfusepath{fill}%
\end{pgfscope}%
\begin{pgfscope}%
\pgfpathrectangle{\pgfqpoint{6.572727in}{0.474100in}}{\pgfqpoint{4.227273in}{3.318700in}}%
\pgfusepath{clip}%
\pgfsetbuttcap%
\pgfsetroundjoin%
\definecolor{currentfill}{rgb}{0.127568,0.566949,0.550556}%
\pgfsetfillcolor{currentfill}%
\pgfsetfillopacity{0.700000}%
\pgfsetlinewidth{0.000000pt}%
\definecolor{currentstroke}{rgb}{0.000000,0.000000,0.000000}%
\pgfsetstrokecolor{currentstroke}%
\pgfsetstrokeopacity{0.700000}%
\pgfsetdash{}{0pt}%
\pgfpathmoveto{\pgfqpoint{8.143509in}{2.575959in}}%
\pgfpathcurveto{\pgfqpoint{8.148553in}{2.575959in}}{\pgfqpoint{8.153390in}{2.577963in}}{\pgfqpoint{8.156957in}{2.581529in}}%
\pgfpathcurveto{\pgfqpoint{8.160523in}{2.585096in}}{\pgfqpoint{8.162527in}{2.589934in}}{\pgfqpoint{8.162527in}{2.594977in}}%
\pgfpathcurveto{\pgfqpoint{8.162527in}{2.600021in}}{\pgfqpoint{8.160523in}{2.604859in}}{\pgfqpoint{8.156957in}{2.608425in}}%
\pgfpathcurveto{\pgfqpoint{8.153390in}{2.611992in}}{\pgfqpoint{8.148553in}{2.613995in}}{\pgfqpoint{8.143509in}{2.613995in}}%
\pgfpathcurveto{\pgfqpoint{8.138465in}{2.613995in}}{\pgfqpoint{8.133627in}{2.611992in}}{\pgfqpoint{8.130061in}{2.608425in}}%
\pgfpathcurveto{\pgfqpoint{8.126495in}{2.604859in}}{\pgfqpoint{8.124491in}{2.600021in}}{\pgfqpoint{8.124491in}{2.594977in}}%
\pgfpathcurveto{\pgfqpoint{8.124491in}{2.589934in}}{\pgfqpoint{8.126495in}{2.585096in}}{\pgfqpoint{8.130061in}{2.581529in}}%
\pgfpathcurveto{\pgfqpoint{8.133627in}{2.577963in}}{\pgfqpoint{8.138465in}{2.575959in}}{\pgfqpoint{8.143509in}{2.575959in}}%
\pgfpathclose%
\pgfusepath{fill}%
\end{pgfscope}%
\begin{pgfscope}%
\pgfpathrectangle{\pgfqpoint{6.572727in}{0.474100in}}{\pgfqpoint{4.227273in}{3.318700in}}%
\pgfusepath{clip}%
\pgfsetbuttcap%
\pgfsetroundjoin%
\definecolor{currentfill}{rgb}{0.993248,0.906157,0.143936}%
\pgfsetfillcolor{currentfill}%
\pgfsetfillopacity{0.700000}%
\pgfsetlinewidth{0.000000pt}%
\definecolor{currentstroke}{rgb}{0.000000,0.000000,0.000000}%
\pgfsetstrokecolor{currentstroke}%
\pgfsetstrokeopacity{0.700000}%
\pgfsetdash{}{0pt}%
\pgfpathmoveto{\pgfqpoint{10.118185in}{1.860835in}}%
\pgfpathcurveto{\pgfqpoint{10.123228in}{1.860835in}}{\pgfqpoint{10.128066in}{1.862839in}}{\pgfqpoint{10.131632in}{1.866406in}}%
\pgfpathcurveto{\pgfqpoint{10.135199in}{1.869972in}}{\pgfqpoint{10.137203in}{1.874810in}}{\pgfqpoint{10.137203in}{1.879854in}}%
\pgfpathcurveto{\pgfqpoint{10.137203in}{1.884897in}}{\pgfqpoint{10.135199in}{1.889735in}}{\pgfqpoint{10.131632in}{1.893301in}}%
\pgfpathcurveto{\pgfqpoint{10.128066in}{1.896868in}}{\pgfqpoint{10.123228in}{1.898872in}}{\pgfqpoint{10.118185in}{1.898872in}}%
\pgfpathcurveto{\pgfqpoint{10.113141in}{1.898872in}}{\pgfqpoint{10.108303in}{1.896868in}}{\pgfqpoint{10.104737in}{1.893301in}}%
\pgfpathcurveto{\pgfqpoint{10.101170in}{1.889735in}}{\pgfqpoint{10.099166in}{1.884897in}}{\pgfqpoint{10.099166in}{1.879854in}}%
\pgfpathcurveto{\pgfqpoint{10.099166in}{1.874810in}}{\pgfqpoint{10.101170in}{1.869972in}}{\pgfqpoint{10.104737in}{1.866406in}}%
\pgfpathcurveto{\pgfqpoint{10.108303in}{1.862839in}}{\pgfqpoint{10.113141in}{1.860835in}}{\pgfqpoint{10.118185in}{1.860835in}}%
\pgfpathclose%
\pgfusepath{fill}%
\end{pgfscope}%
\begin{pgfscope}%
\pgfpathrectangle{\pgfqpoint{6.572727in}{0.474100in}}{\pgfqpoint{4.227273in}{3.318700in}}%
\pgfusepath{clip}%
\pgfsetbuttcap%
\pgfsetroundjoin%
\definecolor{currentfill}{rgb}{0.127568,0.566949,0.550556}%
\pgfsetfillcolor{currentfill}%
\pgfsetfillopacity{0.700000}%
\pgfsetlinewidth{0.000000pt}%
\definecolor{currentstroke}{rgb}{0.000000,0.000000,0.000000}%
\pgfsetstrokecolor{currentstroke}%
\pgfsetstrokeopacity{0.700000}%
\pgfsetdash{}{0pt}%
\pgfpathmoveto{\pgfqpoint{8.077415in}{2.819816in}}%
\pgfpathcurveto{\pgfqpoint{8.082459in}{2.819816in}}{\pgfqpoint{8.087297in}{2.821820in}}{\pgfqpoint{8.090863in}{2.825386in}}%
\pgfpathcurveto{\pgfqpoint{8.094429in}{2.828953in}}{\pgfqpoint{8.096433in}{2.833791in}}{\pgfqpoint{8.096433in}{2.838834in}}%
\pgfpathcurveto{\pgfqpoint{8.096433in}{2.843878in}}{\pgfqpoint{8.094429in}{2.848716in}}{\pgfqpoint{8.090863in}{2.852282in}}%
\pgfpathcurveto{\pgfqpoint{8.087297in}{2.855849in}}{\pgfqpoint{8.082459in}{2.857852in}}{\pgfqpoint{8.077415in}{2.857852in}}%
\pgfpathcurveto{\pgfqpoint{8.072371in}{2.857852in}}{\pgfqpoint{8.067534in}{2.855849in}}{\pgfqpoint{8.063967in}{2.852282in}}%
\pgfpathcurveto{\pgfqpoint{8.060401in}{2.848716in}}{\pgfqpoint{8.058397in}{2.843878in}}{\pgfqpoint{8.058397in}{2.838834in}}%
\pgfpathcurveto{\pgfqpoint{8.058397in}{2.833791in}}{\pgfqpoint{8.060401in}{2.828953in}}{\pgfqpoint{8.063967in}{2.825386in}}%
\pgfpathcurveto{\pgfqpoint{8.067534in}{2.821820in}}{\pgfqpoint{8.072371in}{2.819816in}}{\pgfqpoint{8.077415in}{2.819816in}}%
\pgfpathclose%
\pgfusepath{fill}%
\end{pgfscope}%
\begin{pgfscope}%
\pgfpathrectangle{\pgfqpoint{6.572727in}{0.474100in}}{\pgfqpoint{4.227273in}{3.318700in}}%
\pgfusepath{clip}%
\pgfsetbuttcap%
\pgfsetroundjoin%
\definecolor{currentfill}{rgb}{0.127568,0.566949,0.550556}%
\pgfsetfillcolor{currentfill}%
\pgfsetfillopacity{0.700000}%
\pgfsetlinewidth{0.000000pt}%
\definecolor{currentstroke}{rgb}{0.000000,0.000000,0.000000}%
\pgfsetstrokecolor{currentstroke}%
\pgfsetstrokeopacity{0.700000}%
\pgfsetdash{}{0pt}%
\pgfpathmoveto{\pgfqpoint{8.196651in}{3.193915in}}%
\pgfpathcurveto{\pgfqpoint{8.201695in}{3.193915in}}{\pgfqpoint{8.206533in}{3.195919in}}{\pgfqpoint{8.210099in}{3.199485in}}%
\pgfpathcurveto{\pgfqpoint{8.213665in}{3.203051in}}{\pgfqpoint{8.215669in}{3.207889in}}{\pgfqpoint{8.215669in}{3.212933in}}%
\pgfpathcurveto{\pgfqpoint{8.215669in}{3.217977in}}{\pgfqpoint{8.213665in}{3.222814in}}{\pgfqpoint{8.210099in}{3.226381in}}%
\pgfpathcurveto{\pgfqpoint{8.206533in}{3.229947in}}{\pgfqpoint{8.201695in}{3.231951in}}{\pgfqpoint{8.196651in}{3.231951in}}%
\pgfpathcurveto{\pgfqpoint{8.191607in}{3.231951in}}{\pgfqpoint{8.186770in}{3.229947in}}{\pgfqpoint{8.183203in}{3.226381in}}%
\pgfpathcurveto{\pgfqpoint{8.179637in}{3.222814in}}{\pgfqpoint{8.177633in}{3.217977in}}{\pgfqpoint{8.177633in}{3.212933in}}%
\pgfpathcurveto{\pgfqpoint{8.177633in}{3.207889in}}{\pgfqpoint{8.179637in}{3.203051in}}{\pgfqpoint{8.183203in}{3.199485in}}%
\pgfpathcurveto{\pgfqpoint{8.186770in}{3.195919in}}{\pgfqpoint{8.191607in}{3.193915in}}{\pgfqpoint{8.196651in}{3.193915in}}%
\pgfpathclose%
\pgfusepath{fill}%
\end{pgfscope}%
\begin{pgfscope}%
\pgfpathrectangle{\pgfqpoint{6.572727in}{0.474100in}}{\pgfqpoint{4.227273in}{3.318700in}}%
\pgfusepath{clip}%
\pgfsetbuttcap%
\pgfsetroundjoin%
\definecolor{currentfill}{rgb}{0.127568,0.566949,0.550556}%
\pgfsetfillcolor{currentfill}%
\pgfsetfillopacity{0.700000}%
\pgfsetlinewidth{0.000000pt}%
\definecolor{currentstroke}{rgb}{0.000000,0.000000,0.000000}%
\pgfsetstrokecolor{currentstroke}%
\pgfsetstrokeopacity{0.700000}%
\pgfsetdash{}{0pt}%
\pgfpathmoveto{\pgfqpoint{8.566652in}{2.775355in}}%
\pgfpathcurveto{\pgfqpoint{8.571695in}{2.775355in}}{\pgfqpoint{8.576533in}{2.777359in}}{\pgfqpoint{8.580100in}{2.780926in}}%
\pgfpathcurveto{\pgfqpoint{8.583666in}{2.784492in}}{\pgfqpoint{8.585670in}{2.789330in}}{\pgfqpoint{8.585670in}{2.794374in}}%
\pgfpathcurveto{\pgfqpoint{8.585670in}{2.799417in}}{\pgfqpoint{8.583666in}{2.804255in}}{\pgfqpoint{8.580100in}{2.807821in}}%
\pgfpathcurveto{\pgfqpoint{8.576533in}{2.811388in}}{\pgfqpoint{8.571695in}{2.813392in}}{\pgfqpoint{8.566652in}{2.813392in}}%
\pgfpathcurveto{\pgfqpoint{8.561608in}{2.813392in}}{\pgfqpoint{8.556770in}{2.811388in}}{\pgfqpoint{8.553204in}{2.807821in}}%
\pgfpathcurveto{\pgfqpoint{8.549638in}{2.804255in}}{\pgfqpoint{8.547634in}{2.799417in}}{\pgfqpoint{8.547634in}{2.794374in}}%
\pgfpathcurveto{\pgfqpoint{8.547634in}{2.789330in}}{\pgfqpoint{8.549638in}{2.784492in}}{\pgfqpoint{8.553204in}{2.780926in}}%
\pgfpathcurveto{\pgfqpoint{8.556770in}{2.777359in}}{\pgfqpoint{8.561608in}{2.775355in}}{\pgfqpoint{8.566652in}{2.775355in}}%
\pgfpathclose%
\pgfusepath{fill}%
\end{pgfscope}%
\begin{pgfscope}%
\pgfpathrectangle{\pgfqpoint{6.572727in}{0.474100in}}{\pgfqpoint{4.227273in}{3.318700in}}%
\pgfusepath{clip}%
\pgfsetbuttcap%
\pgfsetroundjoin%
\definecolor{currentfill}{rgb}{0.993248,0.906157,0.143936}%
\pgfsetfillcolor{currentfill}%
\pgfsetfillopacity{0.700000}%
\pgfsetlinewidth{0.000000pt}%
\definecolor{currentstroke}{rgb}{0.000000,0.000000,0.000000}%
\pgfsetstrokecolor{currentstroke}%
\pgfsetstrokeopacity{0.700000}%
\pgfsetdash{}{0pt}%
\pgfpathmoveto{\pgfqpoint{9.713458in}{1.760334in}}%
\pgfpathcurveto{\pgfqpoint{9.718502in}{1.760334in}}{\pgfqpoint{9.723339in}{1.762338in}}{\pgfqpoint{9.726906in}{1.765904in}}%
\pgfpathcurveto{\pgfqpoint{9.730472in}{1.769471in}}{\pgfqpoint{9.732476in}{1.774308in}}{\pgfqpoint{9.732476in}{1.779352in}}%
\pgfpathcurveto{\pgfqpoint{9.732476in}{1.784396in}}{\pgfqpoint{9.730472in}{1.789233in}}{\pgfqpoint{9.726906in}{1.792800in}}%
\pgfpathcurveto{\pgfqpoint{9.723339in}{1.796366in}}{\pgfqpoint{9.718502in}{1.798370in}}{\pgfqpoint{9.713458in}{1.798370in}}%
\pgfpathcurveto{\pgfqpoint{9.708414in}{1.798370in}}{\pgfqpoint{9.703577in}{1.796366in}}{\pgfqpoint{9.700010in}{1.792800in}}%
\pgfpathcurveto{\pgfqpoint{9.696444in}{1.789233in}}{\pgfqpoint{9.694440in}{1.784396in}}{\pgfqpoint{9.694440in}{1.779352in}}%
\pgfpathcurveto{\pgfqpoint{9.694440in}{1.774308in}}{\pgfqpoint{9.696444in}{1.769471in}}{\pgfqpoint{9.700010in}{1.765904in}}%
\pgfpathcurveto{\pgfqpoint{9.703577in}{1.762338in}}{\pgfqpoint{9.708414in}{1.760334in}}{\pgfqpoint{9.713458in}{1.760334in}}%
\pgfpathclose%
\pgfusepath{fill}%
\end{pgfscope}%
\begin{pgfscope}%
\pgfpathrectangle{\pgfqpoint{6.572727in}{0.474100in}}{\pgfqpoint{4.227273in}{3.318700in}}%
\pgfusepath{clip}%
\pgfsetbuttcap%
\pgfsetroundjoin%
\definecolor{currentfill}{rgb}{0.993248,0.906157,0.143936}%
\pgfsetfillcolor{currentfill}%
\pgfsetfillopacity{0.700000}%
\pgfsetlinewidth{0.000000pt}%
\definecolor{currentstroke}{rgb}{0.000000,0.000000,0.000000}%
\pgfsetstrokecolor{currentstroke}%
\pgfsetstrokeopacity{0.700000}%
\pgfsetdash{}{0pt}%
\pgfpathmoveto{\pgfqpoint{9.549198in}{0.981676in}}%
\pgfpathcurveto{\pgfqpoint{9.554242in}{0.981676in}}{\pgfqpoint{9.559080in}{0.983680in}}{\pgfqpoint{9.562646in}{0.987246in}}%
\pgfpathcurveto{\pgfqpoint{9.566212in}{0.990813in}}{\pgfqpoint{9.568216in}{0.995651in}}{\pgfqpoint{9.568216in}{1.000694in}}%
\pgfpathcurveto{\pgfqpoint{9.568216in}{1.005738in}}{\pgfqpoint{9.566212in}{1.010576in}}{\pgfqpoint{9.562646in}{1.014142in}}%
\pgfpathcurveto{\pgfqpoint{9.559080in}{1.017708in}}{\pgfqpoint{9.554242in}{1.019712in}}{\pgfqpoint{9.549198in}{1.019712in}}%
\pgfpathcurveto{\pgfqpoint{9.544154in}{1.019712in}}{\pgfqpoint{9.539317in}{1.017708in}}{\pgfqpoint{9.535750in}{1.014142in}}%
\pgfpathcurveto{\pgfqpoint{9.532184in}{1.010576in}}{\pgfqpoint{9.530180in}{1.005738in}}{\pgfqpoint{9.530180in}{1.000694in}}%
\pgfpathcurveto{\pgfqpoint{9.530180in}{0.995651in}}{\pgfqpoint{9.532184in}{0.990813in}}{\pgfqpoint{9.535750in}{0.987246in}}%
\pgfpathcurveto{\pgfqpoint{9.539317in}{0.983680in}}{\pgfqpoint{9.544154in}{0.981676in}}{\pgfqpoint{9.549198in}{0.981676in}}%
\pgfpathclose%
\pgfusepath{fill}%
\end{pgfscope}%
\begin{pgfscope}%
\pgfpathrectangle{\pgfqpoint{6.572727in}{0.474100in}}{\pgfqpoint{4.227273in}{3.318700in}}%
\pgfusepath{clip}%
\pgfsetbuttcap%
\pgfsetroundjoin%
\definecolor{currentfill}{rgb}{0.993248,0.906157,0.143936}%
\pgfsetfillcolor{currentfill}%
\pgfsetfillopacity{0.700000}%
\pgfsetlinewidth{0.000000pt}%
\definecolor{currentstroke}{rgb}{0.000000,0.000000,0.000000}%
\pgfsetstrokecolor{currentstroke}%
\pgfsetstrokeopacity{0.700000}%
\pgfsetdash{}{0pt}%
\pgfpathmoveto{\pgfqpoint{9.938517in}{1.284958in}}%
\pgfpathcurveto{\pgfqpoint{9.943561in}{1.284958in}}{\pgfqpoint{9.948398in}{1.286962in}}{\pgfqpoint{9.951965in}{1.290529in}}%
\pgfpathcurveto{\pgfqpoint{9.955531in}{1.294095in}}{\pgfqpoint{9.957535in}{1.298933in}}{\pgfqpoint{9.957535in}{1.303977in}}%
\pgfpathcurveto{\pgfqpoint{9.957535in}{1.309020in}}{\pgfqpoint{9.955531in}{1.313858in}}{\pgfqpoint{9.951965in}{1.317424in}}%
\pgfpathcurveto{\pgfqpoint{9.948398in}{1.320991in}}{\pgfqpoint{9.943561in}{1.322995in}}{\pgfqpoint{9.938517in}{1.322995in}}%
\pgfpathcurveto{\pgfqpoint{9.933473in}{1.322995in}}{\pgfqpoint{9.928636in}{1.320991in}}{\pgfqpoint{9.925069in}{1.317424in}}%
\pgfpathcurveto{\pgfqpoint{9.921503in}{1.313858in}}{\pgfqpoint{9.919499in}{1.309020in}}{\pgfqpoint{9.919499in}{1.303977in}}%
\pgfpathcurveto{\pgfqpoint{9.919499in}{1.298933in}}{\pgfqpoint{9.921503in}{1.294095in}}{\pgfqpoint{9.925069in}{1.290529in}}%
\pgfpathcurveto{\pgfqpoint{9.928636in}{1.286962in}}{\pgfqpoint{9.933473in}{1.284958in}}{\pgfqpoint{9.938517in}{1.284958in}}%
\pgfpathclose%
\pgfusepath{fill}%
\end{pgfscope}%
\begin{pgfscope}%
\pgfpathrectangle{\pgfqpoint{6.572727in}{0.474100in}}{\pgfqpoint{4.227273in}{3.318700in}}%
\pgfusepath{clip}%
\pgfsetbuttcap%
\pgfsetroundjoin%
\definecolor{currentfill}{rgb}{0.993248,0.906157,0.143936}%
\pgfsetfillcolor{currentfill}%
\pgfsetfillopacity{0.700000}%
\pgfsetlinewidth{0.000000pt}%
\definecolor{currentstroke}{rgb}{0.000000,0.000000,0.000000}%
\pgfsetstrokecolor{currentstroke}%
\pgfsetstrokeopacity{0.700000}%
\pgfsetdash{}{0pt}%
\pgfpathmoveto{\pgfqpoint{9.637918in}{1.755463in}}%
\pgfpathcurveto{\pgfqpoint{9.642962in}{1.755463in}}{\pgfqpoint{9.647800in}{1.757466in}}{\pgfqpoint{9.651366in}{1.761033in}}%
\pgfpathcurveto{\pgfqpoint{9.654933in}{1.764599in}}{\pgfqpoint{9.656937in}{1.769437in}}{\pgfqpoint{9.656937in}{1.774481in}}%
\pgfpathcurveto{\pgfqpoint{9.656937in}{1.779524in}}{\pgfqpoint{9.654933in}{1.784362in}}{\pgfqpoint{9.651366in}{1.787929in}}%
\pgfpathcurveto{\pgfqpoint{9.647800in}{1.791495in}}{\pgfqpoint{9.642962in}{1.793499in}}{\pgfqpoint{9.637918in}{1.793499in}}%
\pgfpathcurveto{\pgfqpoint{9.632875in}{1.793499in}}{\pgfqpoint{9.628037in}{1.791495in}}{\pgfqpoint{9.624471in}{1.787929in}}%
\pgfpathcurveto{\pgfqpoint{9.620904in}{1.784362in}}{\pgfqpoint{9.618900in}{1.779524in}}{\pgfqpoint{9.618900in}{1.774481in}}%
\pgfpathcurveto{\pgfqpoint{9.618900in}{1.769437in}}{\pgfqpoint{9.620904in}{1.764599in}}{\pgfqpoint{9.624471in}{1.761033in}}%
\pgfpathcurveto{\pgfqpoint{9.628037in}{1.757466in}}{\pgfqpoint{9.632875in}{1.755463in}}{\pgfqpoint{9.637918in}{1.755463in}}%
\pgfpathclose%
\pgfusepath{fill}%
\end{pgfscope}%
\begin{pgfscope}%
\pgfpathrectangle{\pgfqpoint{6.572727in}{0.474100in}}{\pgfqpoint{4.227273in}{3.318700in}}%
\pgfusepath{clip}%
\pgfsetbuttcap%
\pgfsetroundjoin%
\definecolor{currentfill}{rgb}{0.993248,0.906157,0.143936}%
\pgfsetfillcolor{currentfill}%
\pgfsetfillopacity{0.700000}%
\pgfsetlinewidth{0.000000pt}%
\definecolor{currentstroke}{rgb}{0.000000,0.000000,0.000000}%
\pgfsetstrokecolor{currentstroke}%
\pgfsetstrokeopacity{0.700000}%
\pgfsetdash{}{0pt}%
\pgfpathmoveto{\pgfqpoint{9.935366in}{1.275067in}}%
\pgfpathcurveto{\pgfqpoint{9.940410in}{1.275067in}}{\pgfqpoint{9.945248in}{1.277071in}}{\pgfqpoint{9.948814in}{1.280637in}}%
\pgfpathcurveto{\pgfqpoint{9.952381in}{1.284204in}}{\pgfqpoint{9.954385in}{1.289041in}}{\pgfqpoint{9.954385in}{1.294085in}}%
\pgfpathcurveto{\pgfqpoint{9.954385in}{1.299129in}}{\pgfqpoint{9.952381in}{1.303966in}}{\pgfqpoint{9.948814in}{1.307533in}}%
\pgfpathcurveto{\pgfqpoint{9.945248in}{1.311099in}}{\pgfqpoint{9.940410in}{1.313103in}}{\pgfqpoint{9.935366in}{1.313103in}}%
\pgfpathcurveto{\pgfqpoint{9.930323in}{1.313103in}}{\pgfqpoint{9.925485in}{1.311099in}}{\pgfqpoint{9.921919in}{1.307533in}}%
\pgfpathcurveto{\pgfqpoint{9.918352in}{1.303966in}}{\pgfqpoint{9.916348in}{1.299129in}}{\pgfqpoint{9.916348in}{1.294085in}}%
\pgfpathcurveto{\pgfqpoint{9.916348in}{1.289041in}}{\pgfqpoint{9.918352in}{1.284204in}}{\pgfqpoint{9.921919in}{1.280637in}}%
\pgfpathcurveto{\pgfqpoint{9.925485in}{1.277071in}}{\pgfqpoint{9.930323in}{1.275067in}}{\pgfqpoint{9.935366in}{1.275067in}}%
\pgfpathclose%
\pgfusepath{fill}%
\end{pgfscope}%
\begin{pgfscope}%
\pgfpathrectangle{\pgfqpoint{6.572727in}{0.474100in}}{\pgfqpoint{4.227273in}{3.318700in}}%
\pgfusepath{clip}%
\pgfsetbuttcap%
\pgfsetroundjoin%
\definecolor{currentfill}{rgb}{0.993248,0.906157,0.143936}%
\pgfsetfillcolor{currentfill}%
\pgfsetfillopacity{0.700000}%
\pgfsetlinewidth{0.000000pt}%
\definecolor{currentstroke}{rgb}{0.000000,0.000000,0.000000}%
\pgfsetstrokecolor{currentstroke}%
\pgfsetstrokeopacity{0.700000}%
\pgfsetdash{}{0pt}%
\pgfpathmoveto{\pgfqpoint{9.632648in}{1.617971in}}%
\pgfpathcurveto{\pgfqpoint{9.637691in}{1.617971in}}{\pgfqpoint{9.642529in}{1.619974in}}{\pgfqpoint{9.646095in}{1.623541in}}%
\pgfpathcurveto{\pgfqpoint{9.649662in}{1.627107in}}{\pgfqpoint{9.651666in}{1.631945in}}{\pgfqpoint{9.651666in}{1.636989in}}%
\pgfpathcurveto{\pgfqpoint{9.651666in}{1.642032in}}{\pgfqpoint{9.649662in}{1.646870in}}{\pgfqpoint{9.646095in}{1.650437in}}%
\pgfpathcurveto{\pgfqpoint{9.642529in}{1.654003in}}{\pgfqpoint{9.637691in}{1.656007in}}{\pgfqpoint{9.632648in}{1.656007in}}%
\pgfpathcurveto{\pgfqpoint{9.627604in}{1.656007in}}{\pgfqpoint{9.622766in}{1.654003in}}{\pgfqpoint{9.619200in}{1.650437in}}%
\pgfpathcurveto{\pgfqpoint{9.615633in}{1.646870in}}{\pgfqpoint{9.613629in}{1.642032in}}{\pgfqpoint{9.613629in}{1.636989in}}%
\pgfpathcurveto{\pgfqpoint{9.613629in}{1.631945in}}{\pgfqpoint{9.615633in}{1.627107in}}{\pgfqpoint{9.619200in}{1.623541in}}%
\pgfpathcurveto{\pgfqpoint{9.622766in}{1.619974in}}{\pgfqpoint{9.627604in}{1.617971in}}{\pgfqpoint{9.632648in}{1.617971in}}%
\pgfpathclose%
\pgfusepath{fill}%
\end{pgfscope}%
\begin{pgfscope}%
\pgfpathrectangle{\pgfqpoint{6.572727in}{0.474100in}}{\pgfqpoint{4.227273in}{3.318700in}}%
\pgfusepath{clip}%
\pgfsetbuttcap%
\pgfsetroundjoin%
\definecolor{currentfill}{rgb}{0.127568,0.566949,0.550556}%
\pgfsetfillcolor{currentfill}%
\pgfsetfillopacity{0.700000}%
\pgfsetlinewidth{0.000000pt}%
\definecolor{currentstroke}{rgb}{0.000000,0.000000,0.000000}%
\pgfsetstrokecolor{currentstroke}%
\pgfsetstrokeopacity{0.700000}%
\pgfsetdash{}{0pt}%
\pgfpathmoveto{\pgfqpoint{8.044212in}{1.054902in}}%
\pgfpathcurveto{\pgfqpoint{8.049255in}{1.054902in}}{\pgfqpoint{8.054093in}{1.056906in}}{\pgfqpoint{8.057660in}{1.060472in}}%
\pgfpathcurveto{\pgfqpoint{8.061226in}{1.064038in}}{\pgfqpoint{8.063230in}{1.068876in}}{\pgfqpoint{8.063230in}{1.073920in}}%
\pgfpathcurveto{\pgfqpoint{8.063230in}{1.078964in}}{\pgfqpoint{8.061226in}{1.083801in}}{\pgfqpoint{8.057660in}{1.087368in}}%
\pgfpathcurveto{\pgfqpoint{8.054093in}{1.090934in}}{\pgfqpoint{8.049255in}{1.092938in}}{\pgfqpoint{8.044212in}{1.092938in}}%
\pgfpathcurveto{\pgfqpoint{8.039168in}{1.092938in}}{\pgfqpoint{8.034330in}{1.090934in}}{\pgfqpoint{8.030764in}{1.087368in}}%
\pgfpathcurveto{\pgfqpoint{8.027197in}{1.083801in}}{\pgfqpoint{8.025194in}{1.078964in}}{\pgfqpoint{8.025194in}{1.073920in}}%
\pgfpathcurveto{\pgfqpoint{8.025194in}{1.068876in}}{\pgfqpoint{8.027197in}{1.064038in}}{\pgfqpoint{8.030764in}{1.060472in}}%
\pgfpathcurveto{\pgfqpoint{8.034330in}{1.056906in}}{\pgfqpoint{8.039168in}{1.054902in}}{\pgfqpoint{8.044212in}{1.054902in}}%
\pgfpathclose%
\pgfusepath{fill}%
\end{pgfscope}%
\begin{pgfscope}%
\pgfpathrectangle{\pgfqpoint{6.572727in}{0.474100in}}{\pgfqpoint{4.227273in}{3.318700in}}%
\pgfusepath{clip}%
\pgfsetbuttcap%
\pgfsetroundjoin%
\definecolor{currentfill}{rgb}{0.127568,0.566949,0.550556}%
\pgfsetfillcolor{currentfill}%
\pgfsetfillopacity{0.700000}%
\pgfsetlinewidth{0.000000pt}%
\definecolor{currentstroke}{rgb}{0.000000,0.000000,0.000000}%
\pgfsetstrokecolor{currentstroke}%
\pgfsetstrokeopacity{0.700000}%
\pgfsetdash{}{0pt}%
\pgfpathmoveto{\pgfqpoint{7.670830in}{1.274676in}}%
\pgfpathcurveto{\pgfqpoint{7.675874in}{1.274676in}}{\pgfqpoint{7.680712in}{1.276680in}}{\pgfqpoint{7.684278in}{1.280246in}}%
\pgfpathcurveto{\pgfqpoint{7.687845in}{1.283813in}}{\pgfqpoint{7.689848in}{1.288650in}}{\pgfqpoint{7.689848in}{1.293694in}}%
\pgfpathcurveto{\pgfqpoint{7.689848in}{1.298738in}}{\pgfqpoint{7.687845in}{1.303575in}}{\pgfqpoint{7.684278in}{1.307142in}}%
\pgfpathcurveto{\pgfqpoint{7.680712in}{1.310708in}}{\pgfqpoint{7.675874in}{1.312712in}}{\pgfqpoint{7.670830in}{1.312712in}}%
\pgfpathcurveto{\pgfqpoint{7.665787in}{1.312712in}}{\pgfqpoint{7.660949in}{1.310708in}}{\pgfqpoint{7.657382in}{1.307142in}}%
\pgfpathcurveto{\pgfqpoint{7.653816in}{1.303575in}}{\pgfqpoint{7.651812in}{1.298738in}}{\pgfqpoint{7.651812in}{1.293694in}}%
\pgfpathcurveto{\pgfqpoint{7.651812in}{1.288650in}}{\pgfqpoint{7.653816in}{1.283813in}}{\pgfqpoint{7.657382in}{1.280246in}}%
\pgfpathcurveto{\pgfqpoint{7.660949in}{1.276680in}}{\pgfqpoint{7.665787in}{1.274676in}}{\pgfqpoint{7.670830in}{1.274676in}}%
\pgfpathclose%
\pgfusepath{fill}%
\end{pgfscope}%
\begin{pgfscope}%
\pgfpathrectangle{\pgfqpoint{6.572727in}{0.474100in}}{\pgfqpoint{4.227273in}{3.318700in}}%
\pgfusepath{clip}%
\pgfsetbuttcap%
\pgfsetroundjoin%
\definecolor{currentfill}{rgb}{0.993248,0.906157,0.143936}%
\pgfsetfillcolor{currentfill}%
\pgfsetfillopacity{0.700000}%
\pgfsetlinewidth{0.000000pt}%
\definecolor{currentstroke}{rgb}{0.000000,0.000000,0.000000}%
\pgfsetstrokecolor{currentstroke}%
\pgfsetstrokeopacity{0.700000}%
\pgfsetdash{}{0pt}%
\pgfpathmoveto{\pgfqpoint{9.727923in}{1.884600in}}%
\pgfpathcurveto{\pgfqpoint{9.732967in}{1.884600in}}{\pgfqpoint{9.737804in}{1.886604in}}{\pgfqpoint{9.741371in}{1.890170in}}%
\pgfpathcurveto{\pgfqpoint{9.744937in}{1.893737in}}{\pgfqpoint{9.746941in}{1.898575in}}{\pgfqpoint{9.746941in}{1.903618in}}%
\pgfpathcurveto{\pgfqpoint{9.746941in}{1.908662in}}{\pgfqpoint{9.744937in}{1.913500in}}{\pgfqpoint{9.741371in}{1.917066in}}%
\pgfpathcurveto{\pgfqpoint{9.737804in}{1.920632in}}{\pgfqpoint{9.732967in}{1.922636in}}{\pgfqpoint{9.727923in}{1.922636in}}%
\pgfpathcurveto{\pgfqpoint{9.722879in}{1.922636in}}{\pgfqpoint{9.718042in}{1.920632in}}{\pgfqpoint{9.714475in}{1.917066in}}%
\pgfpathcurveto{\pgfqpoint{9.710909in}{1.913500in}}{\pgfqpoint{9.708905in}{1.908662in}}{\pgfqpoint{9.708905in}{1.903618in}}%
\pgfpathcurveto{\pgfqpoint{9.708905in}{1.898575in}}{\pgfqpoint{9.710909in}{1.893737in}}{\pgfqpoint{9.714475in}{1.890170in}}%
\pgfpathcurveto{\pgfqpoint{9.718042in}{1.886604in}}{\pgfqpoint{9.722879in}{1.884600in}}{\pgfqpoint{9.727923in}{1.884600in}}%
\pgfpathclose%
\pgfusepath{fill}%
\end{pgfscope}%
\begin{pgfscope}%
\pgfpathrectangle{\pgfqpoint{6.572727in}{0.474100in}}{\pgfqpoint{4.227273in}{3.318700in}}%
\pgfusepath{clip}%
\pgfsetbuttcap%
\pgfsetroundjoin%
\definecolor{currentfill}{rgb}{0.127568,0.566949,0.550556}%
\pgfsetfillcolor{currentfill}%
\pgfsetfillopacity{0.700000}%
\pgfsetlinewidth{0.000000pt}%
\definecolor{currentstroke}{rgb}{0.000000,0.000000,0.000000}%
\pgfsetstrokecolor{currentstroke}%
\pgfsetstrokeopacity{0.700000}%
\pgfsetdash{}{0pt}%
\pgfpathmoveto{\pgfqpoint{7.486500in}{1.381530in}}%
\pgfpathcurveto{\pgfqpoint{7.491544in}{1.381530in}}{\pgfqpoint{7.496381in}{1.383533in}}{\pgfqpoint{7.499948in}{1.387100in}}%
\pgfpathcurveto{\pgfqpoint{7.503514in}{1.390666in}}{\pgfqpoint{7.505518in}{1.395504in}}{\pgfqpoint{7.505518in}{1.400548in}}%
\pgfpathcurveto{\pgfqpoint{7.505518in}{1.405591in}}{\pgfqpoint{7.503514in}{1.410429in}}{\pgfqpoint{7.499948in}{1.413996in}}%
\pgfpathcurveto{\pgfqpoint{7.496381in}{1.417562in}}{\pgfqpoint{7.491544in}{1.419566in}}{\pgfqpoint{7.486500in}{1.419566in}}%
\pgfpathcurveto{\pgfqpoint{7.481456in}{1.419566in}}{\pgfqpoint{7.476619in}{1.417562in}}{\pgfqpoint{7.473052in}{1.413996in}}%
\pgfpathcurveto{\pgfqpoint{7.469486in}{1.410429in}}{\pgfqpoint{7.467482in}{1.405591in}}{\pgfqpoint{7.467482in}{1.400548in}}%
\pgfpathcurveto{\pgfqpoint{7.467482in}{1.395504in}}{\pgfqpoint{7.469486in}{1.390666in}}{\pgfqpoint{7.473052in}{1.387100in}}%
\pgfpathcurveto{\pgfqpoint{7.476619in}{1.383533in}}{\pgfqpoint{7.481456in}{1.381530in}}{\pgfqpoint{7.486500in}{1.381530in}}%
\pgfpathclose%
\pgfusepath{fill}%
\end{pgfscope}%
\begin{pgfscope}%
\pgfpathrectangle{\pgfqpoint{6.572727in}{0.474100in}}{\pgfqpoint{4.227273in}{3.318700in}}%
\pgfusepath{clip}%
\pgfsetbuttcap%
\pgfsetroundjoin%
\definecolor{currentfill}{rgb}{0.127568,0.566949,0.550556}%
\pgfsetfillcolor{currentfill}%
\pgfsetfillopacity{0.700000}%
\pgfsetlinewidth{0.000000pt}%
\definecolor{currentstroke}{rgb}{0.000000,0.000000,0.000000}%
\pgfsetstrokecolor{currentstroke}%
\pgfsetstrokeopacity{0.700000}%
\pgfsetdash{}{0pt}%
\pgfpathmoveto{\pgfqpoint{7.183260in}{1.310549in}}%
\pgfpathcurveto{\pgfqpoint{7.188303in}{1.310549in}}{\pgfqpoint{7.193141in}{1.312553in}}{\pgfqpoint{7.196708in}{1.316119in}}%
\pgfpathcurveto{\pgfqpoint{7.200274in}{1.319686in}}{\pgfqpoint{7.202278in}{1.324523in}}{\pgfqpoint{7.202278in}{1.329567in}}%
\pgfpathcurveto{\pgfqpoint{7.202278in}{1.334611in}}{\pgfqpoint{7.200274in}{1.339449in}}{\pgfqpoint{7.196708in}{1.343015in}}%
\pgfpathcurveto{\pgfqpoint{7.193141in}{1.346581in}}{\pgfqpoint{7.188303in}{1.348585in}}{\pgfqpoint{7.183260in}{1.348585in}}%
\pgfpathcurveto{\pgfqpoint{7.178216in}{1.348585in}}{\pgfqpoint{7.173378in}{1.346581in}}{\pgfqpoint{7.169812in}{1.343015in}}%
\pgfpathcurveto{\pgfqpoint{7.166245in}{1.339449in}}{\pgfqpoint{7.164241in}{1.334611in}}{\pgfqpoint{7.164241in}{1.329567in}}%
\pgfpathcurveto{\pgfqpoint{7.164241in}{1.324523in}}{\pgfqpoint{7.166245in}{1.319686in}}{\pgfqpoint{7.169812in}{1.316119in}}%
\pgfpathcurveto{\pgfqpoint{7.173378in}{1.312553in}}{\pgfqpoint{7.178216in}{1.310549in}}{\pgfqpoint{7.183260in}{1.310549in}}%
\pgfpathclose%
\pgfusepath{fill}%
\end{pgfscope}%
\begin{pgfscope}%
\pgfpathrectangle{\pgfqpoint{6.572727in}{0.474100in}}{\pgfqpoint{4.227273in}{3.318700in}}%
\pgfusepath{clip}%
\pgfsetbuttcap%
\pgfsetroundjoin%
\definecolor{currentfill}{rgb}{0.127568,0.566949,0.550556}%
\pgfsetfillcolor{currentfill}%
\pgfsetfillopacity{0.700000}%
\pgfsetlinewidth{0.000000pt}%
\definecolor{currentstroke}{rgb}{0.000000,0.000000,0.000000}%
\pgfsetstrokecolor{currentstroke}%
\pgfsetstrokeopacity{0.700000}%
\pgfsetdash{}{0pt}%
\pgfpathmoveto{\pgfqpoint{8.591138in}{2.724221in}}%
\pgfpathcurveto{\pgfqpoint{8.596182in}{2.724221in}}{\pgfqpoint{8.601020in}{2.726225in}}{\pgfqpoint{8.604586in}{2.729792in}}%
\pgfpathcurveto{\pgfqpoint{8.608153in}{2.733358in}}{\pgfqpoint{8.610156in}{2.738196in}}{\pgfqpoint{8.610156in}{2.743239in}}%
\pgfpathcurveto{\pgfqpoint{8.610156in}{2.748283in}}{\pgfqpoint{8.608153in}{2.753121in}}{\pgfqpoint{8.604586in}{2.756687in}}%
\pgfpathcurveto{\pgfqpoint{8.601020in}{2.760254in}}{\pgfqpoint{8.596182in}{2.762258in}}{\pgfqpoint{8.591138in}{2.762258in}}%
\pgfpathcurveto{\pgfqpoint{8.586095in}{2.762258in}}{\pgfqpoint{8.581257in}{2.760254in}}{\pgfqpoint{8.577690in}{2.756687in}}%
\pgfpathcurveto{\pgfqpoint{8.574124in}{2.753121in}}{\pgfqpoint{8.572120in}{2.748283in}}{\pgfqpoint{8.572120in}{2.743239in}}%
\pgfpathcurveto{\pgfqpoint{8.572120in}{2.738196in}}{\pgfqpoint{8.574124in}{2.733358in}}{\pgfqpoint{8.577690in}{2.729792in}}%
\pgfpathcurveto{\pgfqpoint{8.581257in}{2.726225in}}{\pgfqpoint{8.586095in}{2.724221in}}{\pgfqpoint{8.591138in}{2.724221in}}%
\pgfpathclose%
\pgfusepath{fill}%
\end{pgfscope}%
\begin{pgfscope}%
\pgfpathrectangle{\pgfqpoint{6.572727in}{0.474100in}}{\pgfqpoint{4.227273in}{3.318700in}}%
\pgfusepath{clip}%
\pgfsetbuttcap%
\pgfsetroundjoin%
\definecolor{currentfill}{rgb}{0.267004,0.004874,0.329415}%
\pgfsetfillcolor{currentfill}%
\pgfsetfillopacity{0.700000}%
\pgfsetlinewidth{0.000000pt}%
\definecolor{currentstroke}{rgb}{0.000000,0.000000,0.000000}%
\pgfsetstrokecolor{currentstroke}%
\pgfsetstrokeopacity{0.700000}%
\pgfsetdash{}{0pt}%
\pgfpathmoveto{\pgfqpoint{7.203953in}{2.308809in}}%
\pgfpathcurveto{\pgfqpoint{7.208997in}{2.308809in}}{\pgfqpoint{7.213834in}{2.310813in}}{\pgfqpoint{7.217401in}{2.314379in}}%
\pgfpathcurveto{\pgfqpoint{7.220967in}{2.317946in}}{\pgfqpoint{7.222971in}{2.322783in}}{\pgfqpoint{7.222971in}{2.327827in}}%
\pgfpathcurveto{\pgfqpoint{7.222971in}{2.332871in}}{\pgfqpoint{7.220967in}{2.337708in}}{\pgfqpoint{7.217401in}{2.341275in}}%
\pgfpathcurveto{\pgfqpoint{7.213834in}{2.344841in}}{\pgfqpoint{7.208997in}{2.346845in}}{\pgfqpoint{7.203953in}{2.346845in}}%
\pgfpathcurveto{\pgfqpoint{7.198909in}{2.346845in}}{\pgfqpoint{7.194071in}{2.344841in}}{\pgfqpoint{7.190505in}{2.341275in}}%
\pgfpathcurveto{\pgfqpoint{7.186939in}{2.337708in}}{\pgfqpoint{7.184935in}{2.332871in}}{\pgfqpoint{7.184935in}{2.327827in}}%
\pgfpathcurveto{\pgfqpoint{7.184935in}{2.322783in}}{\pgfqpoint{7.186939in}{2.317946in}}{\pgfqpoint{7.190505in}{2.314379in}}%
\pgfpathcurveto{\pgfqpoint{7.194071in}{2.310813in}}{\pgfqpoint{7.198909in}{2.308809in}}{\pgfqpoint{7.203953in}{2.308809in}}%
\pgfpathclose%
\pgfusepath{fill}%
\end{pgfscope}%
\begin{pgfscope}%
\pgfpathrectangle{\pgfqpoint{6.572727in}{0.474100in}}{\pgfqpoint{4.227273in}{3.318700in}}%
\pgfusepath{clip}%
\pgfsetbuttcap%
\pgfsetroundjoin%
\definecolor{currentfill}{rgb}{0.127568,0.566949,0.550556}%
\pgfsetfillcolor{currentfill}%
\pgfsetfillopacity{0.700000}%
\pgfsetlinewidth{0.000000pt}%
\definecolor{currentstroke}{rgb}{0.000000,0.000000,0.000000}%
\pgfsetstrokecolor{currentstroke}%
\pgfsetstrokeopacity{0.700000}%
\pgfsetdash{}{0pt}%
\pgfpathmoveto{\pgfqpoint{8.146426in}{2.435695in}}%
\pgfpathcurveto{\pgfqpoint{8.151469in}{2.435695in}}{\pgfqpoint{8.156307in}{2.437699in}}{\pgfqpoint{8.159873in}{2.441265in}}%
\pgfpathcurveto{\pgfqpoint{8.163440in}{2.444831in}}{\pgfqpoint{8.165444in}{2.449669in}}{\pgfqpoint{8.165444in}{2.454713in}}%
\pgfpathcurveto{\pgfqpoint{8.165444in}{2.459756in}}{\pgfqpoint{8.163440in}{2.464594in}}{\pgfqpoint{8.159873in}{2.468161in}}%
\pgfpathcurveto{\pgfqpoint{8.156307in}{2.471727in}}{\pgfqpoint{8.151469in}{2.473731in}}{\pgfqpoint{8.146426in}{2.473731in}}%
\pgfpathcurveto{\pgfqpoint{8.141382in}{2.473731in}}{\pgfqpoint{8.136544in}{2.471727in}}{\pgfqpoint{8.132978in}{2.468161in}}%
\pgfpathcurveto{\pgfqpoint{8.129411in}{2.464594in}}{\pgfqpoint{8.127407in}{2.459756in}}{\pgfqpoint{8.127407in}{2.454713in}}%
\pgfpathcurveto{\pgfqpoint{8.127407in}{2.449669in}}{\pgfqpoint{8.129411in}{2.444831in}}{\pgfqpoint{8.132978in}{2.441265in}}%
\pgfpathcurveto{\pgfqpoint{8.136544in}{2.437699in}}{\pgfqpoint{8.141382in}{2.435695in}}{\pgfqpoint{8.146426in}{2.435695in}}%
\pgfpathclose%
\pgfusepath{fill}%
\end{pgfscope}%
\begin{pgfscope}%
\pgfpathrectangle{\pgfqpoint{6.572727in}{0.474100in}}{\pgfqpoint{4.227273in}{3.318700in}}%
\pgfusepath{clip}%
\pgfsetbuttcap%
\pgfsetroundjoin%
\definecolor{currentfill}{rgb}{0.993248,0.906157,0.143936}%
\pgfsetfillcolor{currentfill}%
\pgfsetfillopacity{0.700000}%
\pgfsetlinewidth{0.000000pt}%
\definecolor{currentstroke}{rgb}{0.000000,0.000000,0.000000}%
\pgfsetstrokecolor{currentstroke}%
\pgfsetstrokeopacity{0.700000}%
\pgfsetdash{}{0pt}%
\pgfpathmoveto{\pgfqpoint{9.958396in}{2.297978in}}%
\pgfpathcurveto{\pgfqpoint{9.963440in}{2.297978in}}{\pgfqpoint{9.968278in}{2.299982in}}{\pgfqpoint{9.971844in}{2.303549in}}%
\pgfpathcurveto{\pgfqpoint{9.975411in}{2.307115in}}{\pgfqpoint{9.977414in}{2.311953in}}{\pgfqpoint{9.977414in}{2.316996in}}%
\pgfpathcurveto{\pgfqpoint{9.977414in}{2.322040in}}{\pgfqpoint{9.975411in}{2.326878in}}{\pgfqpoint{9.971844in}{2.330444in}}%
\pgfpathcurveto{\pgfqpoint{9.968278in}{2.334011in}}{\pgfqpoint{9.963440in}{2.336015in}}{\pgfqpoint{9.958396in}{2.336015in}}%
\pgfpathcurveto{\pgfqpoint{9.953353in}{2.336015in}}{\pgfqpoint{9.948515in}{2.334011in}}{\pgfqpoint{9.944948in}{2.330444in}}%
\pgfpathcurveto{\pgfqpoint{9.941382in}{2.326878in}}{\pgfqpoint{9.939378in}{2.322040in}}{\pgfqpoint{9.939378in}{2.316996in}}%
\pgfpathcurveto{\pgfqpoint{9.939378in}{2.311953in}}{\pgfqpoint{9.941382in}{2.307115in}}{\pgfqpoint{9.944948in}{2.303549in}}%
\pgfpathcurveto{\pgfqpoint{9.948515in}{2.299982in}}{\pgfqpoint{9.953353in}{2.297978in}}{\pgfqpoint{9.958396in}{2.297978in}}%
\pgfpathclose%
\pgfusepath{fill}%
\end{pgfscope}%
\begin{pgfscope}%
\pgfpathrectangle{\pgfqpoint{6.572727in}{0.474100in}}{\pgfqpoint{4.227273in}{3.318700in}}%
\pgfusepath{clip}%
\pgfsetbuttcap%
\pgfsetroundjoin%
\definecolor{currentfill}{rgb}{0.127568,0.566949,0.550556}%
\pgfsetfillcolor{currentfill}%
\pgfsetfillopacity{0.700000}%
\pgfsetlinewidth{0.000000pt}%
\definecolor{currentstroke}{rgb}{0.000000,0.000000,0.000000}%
\pgfsetstrokecolor{currentstroke}%
\pgfsetstrokeopacity{0.700000}%
\pgfsetdash{}{0pt}%
\pgfpathmoveto{\pgfqpoint{8.131257in}{2.251934in}}%
\pgfpathcurveto{\pgfqpoint{8.136300in}{2.251934in}}{\pgfqpoint{8.141138in}{2.253938in}}{\pgfqpoint{8.144704in}{2.257505in}}%
\pgfpathcurveto{\pgfqpoint{8.148271in}{2.261071in}}{\pgfqpoint{8.150275in}{2.265909in}}{\pgfqpoint{8.150275in}{2.270952in}}%
\pgfpathcurveto{\pgfqpoint{8.150275in}{2.275996in}}{\pgfqpoint{8.148271in}{2.280834in}}{\pgfqpoint{8.144704in}{2.284400in}}%
\pgfpathcurveto{\pgfqpoint{8.141138in}{2.287967in}}{\pgfqpoint{8.136300in}{2.289971in}}{\pgfqpoint{8.131257in}{2.289971in}}%
\pgfpathcurveto{\pgfqpoint{8.126213in}{2.289971in}}{\pgfqpoint{8.121375in}{2.287967in}}{\pgfqpoint{8.117809in}{2.284400in}}%
\pgfpathcurveto{\pgfqpoint{8.114242in}{2.280834in}}{\pgfqpoint{8.112238in}{2.275996in}}{\pgfqpoint{8.112238in}{2.270952in}}%
\pgfpathcurveto{\pgfqpoint{8.112238in}{2.265909in}}{\pgfqpoint{8.114242in}{2.261071in}}{\pgfqpoint{8.117809in}{2.257505in}}%
\pgfpathcurveto{\pgfqpoint{8.121375in}{2.253938in}}{\pgfqpoint{8.126213in}{2.251934in}}{\pgfqpoint{8.131257in}{2.251934in}}%
\pgfpathclose%
\pgfusepath{fill}%
\end{pgfscope}%
\begin{pgfscope}%
\pgfpathrectangle{\pgfqpoint{6.572727in}{0.474100in}}{\pgfqpoint{4.227273in}{3.318700in}}%
\pgfusepath{clip}%
\pgfsetbuttcap%
\pgfsetroundjoin%
\definecolor{currentfill}{rgb}{0.127568,0.566949,0.550556}%
\pgfsetfillcolor{currentfill}%
\pgfsetfillopacity{0.700000}%
\pgfsetlinewidth{0.000000pt}%
\definecolor{currentstroke}{rgb}{0.000000,0.000000,0.000000}%
\pgfsetstrokecolor{currentstroke}%
\pgfsetstrokeopacity{0.700000}%
\pgfsetdash{}{0pt}%
\pgfpathmoveto{\pgfqpoint{8.039184in}{1.823604in}}%
\pgfpathcurveto{\pgfqpoint{8.044228in}{1.823604in}}{\pgfqpoint{8.049065in}{1.825608in}}{\pgfqpoint{8.052632in}{1.829175in}}%
\pgfpathcurveto{\pgfqpoint{8.056198in}{1.832741in}}{\pgfqpoint{8.058202in}{1.837579in}}{\pgfqpoint{8.058202in}{1.842623in}}%
\pgfpathcurveto{\pgfqpoint{8.058202in}{1.847666in}}{\pgfqpoint{8.056198in}{1.852504in}}{\pgfqpoint{8.052632in}{1.856070in}}%
\pgfpathcurveto{\pgfqpoint{8.049065in}{1.859637in}}{\pgfqpoint{8.044228in}{1.861641in}}{\pgfqpoint{8.039184in}{1.861641in}}%
\pgfpathcurveto{\pgfqpoint{8.034140in}{1.861641in}}{\pgfqpoint{8.029303in}{1.859637in}}{\pgfqpoint{8.025736in}{1.856070in}}%
\pgfpathcurveto{\pgfqpoint{8.022170in}{1.852504in}}{\pgfqpoint{8.020166in}{1.847666in}}{\pgfqpoint{8.020166in}{1.842623in}}%
\pgfpathcurveto{\pgfqpoint{8.020166in}{1.837579in}}{\pgfqpoint{8.022170in}{1.832741in}}{\pgfqpoint{8.025736in}{1.829175in}}%
\pgfpathcurveto{\pgfqpoint{8.029303in}{1.825608in}}{\pgfqpoint{8.034140in}{1.823604in}}{\pgfqpoint{8.039184in}{1.823604in}}%
\pgfpathclose%
\pgfusepath{fill}%
\end{pgfscope}%
\begin{pgfscope}%
\pgfpathrectangle{\pgfqpoint{6.572727in}{0.474100in}}{\pgfqpoint{4.227273in}{3.318700in}}%
\pgfusepath{clip}%
\pgfsetbuttcap%
\pgfsetroundjoin%
\definecolor{currentfill}{rgb}{0.127568,0.566949,0.550556}%
\pgfsetfillcolor{currentfill}%
\pgfsetfillopacity{0.700000}%
\pgfsetlinewidth{0.000000pt}%
\definecolor{currentstroke}{rgb}{0.000000,0.000000,0.000000}%
\pgfsetstrokecolor{currentstroke}%
\pgfsetstrokeopacity{0.700000}%
\pgfsetdash{}{0pt}%
\pgfpathmoveto{\pgfqpoint{8.147547in}{1.813627in}}%
\pgfpathcurveto{\pgfqpoint{8.152590in}{1.813627in}}{\pgfqpoint{8.157428in}{1.815631in}}{\pgfqpoint{8.160995in}{1.819198in}}%
\pgfpathcurveto{\pgfqpoint{8.164561in}{1.822764in}}{\pgfqpoint{8.166565in}{1.827602in}}{\pgfqpoint{8.166565in}{1.832646in}}%
\pgfpathcurveto{\pgfqpoint{8.166565in}{1.837689in}}{\pgfqpoint{8.164561in}{1.842527in}}{\pgfqpoint{8.160995in}{1.846094in}}%
\pgfpathcurveto{\pgfqpoint{8.157428in}{1.849660in}}{\pgfqpoint{8.152590in}{1.851664in}}{\pgfqpoint{8.147547in}{1.851664in}}%
\pgfpathcurveto{\pgfqpoint{8.142503in}{1.851664in}}{\pgfqpoint{8.137665in}{1.849660in}}{\pgfqpoint{8.134099in}{1.846094in}}%
\pgfpathcurveto{\pgfqpoint{8.130532in}{1.842527in}}{\pgfqpoint{8.128529in}{1.837689in}}{\pgfqpoint{8.128529in}{1.832646in}}%
\pgfpathcurveto{\pgfqpoint{8.128529in}{1.827602in}}{\pgfqpoint{8.130532in}{1.822764in}}{\pgfqpoint{8.134099in}{1.819198in}}%
\pgfpathcurveto{\pgfqpoint{8.137665in}{1.815631in}}{\pgfqpoint{8.142503in}{1.813627in}}{\pgfqpoint{8.147547in}{1.813627in}}%
\pgfpathclose%
\pgfusepath{fill}%
\end{pgfscope}%
\begin{pgfscope}%
\pgfpathrectangle{\pgfqpoint{6.572727in}{0.474100in}}{\pgfqpoint{4.227273in}{3.318700in}}%
\pgfusepath{clip}%
\pgfsetbuttcap%
\pgfsetroundjoin%
\definecolor{currentfill}{rgb}{0.127568,0.566949,0.550556}%
\pgfsetfillcolor{currentfill}%
\pgfsetfillopacity{0.700000}%
\pgfsetlinewidth{0.000000pt}%
\definecolor{currentstroke}{rgb}{0.000000,0.000000,0.000000}%
\pgfsetstrokecolor{currentstroke}%
\pgfsetstrokeopacity{0.700000}%
\pgfsetdash{}{0pt}%
\pgfpathmoveto{\pgfqpoint{7.346350in}{1.388649in}}%
\pgfpathcurveto{\pgfqpoint{7.351394in}{1.388649in}}{\pgfqpoint{7.356232in}{1.390653in}}{\pgfqpoint{7.359798in}{1.394220in}}%
\pgfpathcurveto{\pgfqpoint{7.363365in}{1.397786in}}{\pgfqpoint{7.365368in}{1.402624in}}{\pgfqpoint{7.365368in}{1.407667in}}%
\pgfpathcurveto{\pgfqpoint{7.365368in}{1.412711in}}{\pgfqpoint{7.363365in}{1.417549in}}{\pgfqpoint{7.359798in}{1.421115in}}%
\pgfpathcurveto{\pgfqpoint{7.356232in}{1.424682in}}{\pgfqpoint{7.351394in}{1.426686in}}{\pgfqpoint{7.346350in}{1.426686in}}%
\pgfpathcurveto{\pgfqpoint{7.341307in}{1.426686in}}{\pgfqpoint{7.336469in}{1.424682in}}{\pgfqpoint{7.332902in}{1.421115in}}%
\pgfpathcurveto{\pgfqpoint{7.329336in}{1.417549in}}{\pgfqpoint{7.327332in}{1.412711in}}{\pgfqpoint{7.327332in}{1.407667in}}%
\pgfpathcurveto{\pgfqpoint{7.327332in}{1.402624in}}{\pgfqpoint{7.329336in}{1.397786in}}{\pgfqpoint{7.332902in}{1.394220in}}%
\pgfpathcurveto{\pgfqpoint{7.336469in}{1.390653in}}{\pgfqpoint{7.341307in}{1.388649in}}{\pgfqpoint{7.346350in}{1.388649in}}%
\pgfpathclose%
\pgfusepath{fill}%
\end{pgfscope}%
\begin{pgfscope}%
\pgfpathrectangle{\pgfqpoint{6.572727in}{0.474100in}}{\pgfqpoint{4.227273in}{3.318700in}}%
\pgfusepath{clip}%
\pgfsetbuttcap%
\pgfsetroundjoin%
\definecolor{currentfill}{rgb}{0.993248,0.906157,0.143936}%
\pgfsetfillcolor{currentfill}%
\pgfsetfillopacity{0.700000}%
\pgfsetlinewidth{0.000000pt}%
\definecolor{currentstroke}{rgb}{0.000000,0.000000,0.000000}%
\pgfsetstrokecolor{currentstroke}%
\pgfsetstrokeopacity{0.700000}%
\pgfsetdash{}{0pt}%
\pgfpathmoveto{\pgfqpoint{9.787159in}{1.349847in}}%
\pgfpathcurveto{\pgfqpoint{9.792203in}{1.349847in}}{\pgfqpoint{9.797040in}{1.351851in}}{\pgfqpoint{9.800607in}{1.355418in}}%
\pgfpathcurveto{\pgfqpoint{9.804173in}{1.358984in}}{\pgfqpoint{9.806177in}{1.363822in}}{\pgfqpoint{9.806177in}{1.368865in}}%
\pgfpathcurveto{\pgfqpoint{9.806177in}{1.373909in}}{\pgfqpoint{9.804173in}{1.378747in}}{\pgfqpoint{9.800607in}{1.382313in}}%
\pgfpathcurveto{\pgfqpoint{9.797040in}{1.385880in}}{\pgfqpoint{9.792203in}{1.387884in}}{\pgfqpoint{9.787159in}{1.387884in}}%
\pgfpathcurveto{\pgfqpoint{9.782115in}{1.387884in}}{\pgfqpoint{9.777278in}{1.385880in}}{\pgfqpoint{9.773711in}{1.382313in}}%
\pgfpathcurveto{\pgfqpoint{9.770145in}{1.378747in}}{\pgfqpoint{9.768141in}{1.373909in}}{\pgfqpoint{9.768141in}{1.368865in}}%
\pgfpathcurveto{\pgfqpoint{9.768141in}{1.363822in}}{\pgfqpoint{9.770145in}{1.358984in}}{\pgfqpoint{9.773711in}{1.355418in}}%
\pgfpathcurveto{\pgfqpoint{9.777278in}{1.351851in}}{\pgfqpoint{9.782115in}{1.349847in}}{\pgfqpoint{9.787159in}{1.349847in}}%
\pgfpathclose%
\pgfusepath{fill}%
\end{pgfscope}%
\begin{pgfscope}%
\pgfpathrectangle{\pgfqpoint{6.572727in}{0.474100in}}{\pgfqpoint{4.227273in}{3.318700in}}%
\pgfusepath{clip}%
\pgfsetbuttcap%
\pgfsetroundjoin%
\definecolor{currentfill}{rgb}{0.127568,0.566949,0.550556}%
\pgfsetfillcolor{currentfill}%
\pgfsetfillopacity{0.700000}%
\pgfsetlinewidth{0.000000pt}%
\definecolor{currentstroke}{rgb}{0.000000,0.000000,0.000000}%
\pgfsetstrokecolor{currentstroke}%
\pgfsetstrokeopacity{0.700000}%
\pgfsetdash{}{0pt}%
\pgfpathmoveto{\pgfqpoint{8.359853in}{2.527548in}}%
\pgfpathcurveto{\pgfqpoint{8.364896in}{2.527548in}}{\pgfqpoint{8.369734in}{2.529552in}}{\pgfqpoint{8.373300in}{2.533118in}}%
\pgfpathcurveto{\pgfqpoint{8.376867in}{2.536684in}}{\pgfqpoint{8.378871in}{2.541522in}}{\pgfqpoint{8.378871in}{2.546566in}}%
\pgfpathcurveto{\pgfqpoint{8.378871in}{2.551610in}}{\pgfqpoint{8.376867in}{2.556447in}}{\pgfqpoint{8.373300in}{2.560014in}}%
\pgfpathcurveto{\pgfqpoint{8.369734in}{2.563580in}}{\pgfqpoint{8.364896in}{2.565584in}}{\pgfqpoint{8.359853in}{2.565584in}}%
\pgfpathcurveto{\pgfqpoint{8.354809in}{2.565584in}}{\pgfqpoint{8.349971in}{2.563580in}}{\pgfqpoint{8.346405in}{2.560014in}}%
\pgfpathcurveto{\pgfqpoint{8.342838in}{2.556447in}}{\pgfqpoint{8.340834in}{2.551610in}}{\pgfqpoint{8.340834in}{2.546566in}}%
\pgfpathcurveto{\pgfqpoint{8.340834in}{2.541522in}}{\pgfqpoint{8.342838in}{2.536684in}}{\pgfqpoint{8.346405in}{2.533118in}}%
\pgfpathcurveto{\pgfqpoint{8.349971in}{2.529552in}}{\pgfqpoint{8.354809in}{2.527548in}}{\pgfqpoint{8.359853in}{2.527548in}}%
\pgfpathclose%
\pgfusepath{fill}%
\end{pgfscope}%
\begin{pgfscope}%
\pgfpathrectangle{\pgfqpoint{6.572727in}{0.474100in}}{\pgfqpoint{4.227273in}{3.318700in}}%
\pgfusepath{clip}%
\pgfsetbuttcap%
\pgfsetroundjoin%
\definecolor{currentfill}{rgb}{0.993248,0.906157,0.143936}%
\pgfsetfillcolor{currentfill}%
\pgfsetfillopacity{0.700000}%
\pgfsetlinewidth{0.000000pt}%
\definecolor{currentstroke}{rgb}{0.000000,0.000000,0.000000}%
\pgfsetstrokecolor{currentstroke}%
\pgfsetstrokeopacity{0.700000}%
\pgfsetdash{}{0pt}%
\pgfpathmoveto{\pgfqpoint{9.451051in}{1.268015in}}%
\pgfpathcurveto{\pgfqpoint{9.456095in}{1.268015in}}{\pgfqpoint{9.460933in}{1.270019in}}{\pgfqpoint{9.464499in}{1.273585in}}%
\pgfpathcurveto{\pgfqpoint{9.468066in}{1.277152in}}{\pgfqpoint{9.470070in}{1.281990in}}{\pgfqpoint{9.470070in}{1.287033in}}%
\pgfpathcurveto{\pgfqpoint{9.470070in}{1.292077in}}{\pgfqpoint{9.468066in}{1.296915in}}{\pgfqpoint{9.464499in}{1.300481in}}%
\pgfpathcurveto{\pgfqpoint{9.460933in}{1.304047in}}{\pgfqpoint{9.456095in}{1.306051in}}{\pgfqpoint{9.451051in}{1.306051in}}%
\pgfpathcurveto{\pgfqpoint{9.446008in}{1.306051in}}{\pgfqpoint{9.441170in}{1.304047in}}{\pgfqpoint{9.437604in}{1.300481in}}%
\pgfpathcurveto{\pgfqpoint{9.434037in}{1.296915in}}{\pgfqpoint{9.432033in}{1.292077in}}{\pgfqpoint{9.432033in}{1.287033in}}%
\pgfpathcurveto{\pgfqpoint{9.432033in}{1.281990in}}{\pgfqpoint{9.434037in}{1.277152in}}{\pgfqpoint{9.437604in}{1.273585in}}%
\pgfpathcurveto{\pgfqpoint{9.441170in}{1.270019in}}{\pgfqpoint{9.446008in}{1.268015in}}{\pgfqpoint{9.451051in}{1.268015in}}%
\pgfpathclose%
\pgfusepath{fill}%
\end{pgfscope}%
\begin{pgfscope}%
\pgfpathrectangle{\pgfqpoint{6.572727in}{0.474100in}}{\pgfqpoint{4.227273in}{3.318700in}}%
\pgfusepath{clip}%
\pgfsetbuttcap%
\pgfsetroundjoin%
\definecolor{currentfill}{rgb}{0.127568,0.566949,0.550556}%
\pgfsetfillcolor{currentfill}%
\pgfsetfillopacity{0.700000}%
\pgfsetlinewidth{0.000000pt}%
\definecolor{currentstroke}{rgb}{0.000000,0.000000,0.000000}%
\pgfsetstrokecolor{currentstroke}%
\pgfsetstrokeopacity{0.700000}%
\pgfsetdash{}{0pt}%
\pgfpathmoveto{\pgfqpoint{8.622912in}{2.789715in}}%
\pgfpathcurveto{\pgfqpoint{8.627956in}{2.789715in}}{\pgfqpoint{8.632794in}{2.791718in}}{\pgfqpoint{8.636360in}{2.795285in}}%
\pgfpathcurveto{\pgfqpoint{8.639927in}{2.798851in}}{\pgfqpoint{8.641931in}{2.803689in}}{\pgfqpoint{8.641931in}{2.808733in}}%
\pgfpathcurveto{\pgfqpoint{8.641931in}{2.813776in}}{\pgfqpoint{8.639927in}{2.818614in}}{\pgfqpoint{8.636360in}{2.822181in}}%
\pgfpathcurveto{\pgfqpoint{8.632794in}{2.825747in}}{\pgfqpoint{8.627956in}{2.827751in}}{\pgfqpoint{8.622912in}{2.827751in}}%
\pgfpathcurveto{\pgfqpoint{8.617869in}{2.827751in}}{\pgfqpoint{8.613031in}{2.825747in}}{\pgfqpoint{8.609465in}{2.822181in}}%
\pgfpathcurveto{\pgfqpoint{8.605898in}{2.818614in}}{\pgfqpoint{8.603894in}{2.813776in}}{\pgfqpoint{8.603894in}{2.808733in}}%
\pgfpathcurveto{\pgfqpoint{8.603894in}{2.803689in}}{\pgfqpoint{8.605898in}{2.798851in}}{\pgfqpoint{8.609465in}{2.795285in}}%
\pgfpathcurveto{\pgfqpoint{8.613031in}{2.791718in}}{\pgfqpoint{8.617869in}{2.789715in}}{\pgfqpoint{8.622912in}{2.789715in}}%
\pgfpathclose%
\pgfusepath{fill}%
\end{pgfscope}%
\begin{pgfscope}%
\pgfpathrectangle{\pgfqpoint{6.572727in}{0.474100in}}{\pgfqpoint{4.227273in}{3.318700in}}%
\pgfusepath{clip}%
\pgfsetbuttcap%
\pgfsetroundjoin%
\definecolor{currentfill}{rgb}{0.993248,0.906157,0.143936}%
\pgfsetfillcolor{currentfill}%
\pgfsetfillopacity{0.700000}%
\pgfsetlinewidth{0.000000pt}%
\definecolor{currentstroke}{rgb}{0.000000,0.000000,0.000000}%
\pgfsetstrokecolor{currentstroke}%
\pgfsetstrokeopacity{0.700000}%
\pgfsetdash{}{0pt}%
\pgfpathmoveto{\pgfqpoint{9.626565in}{2.293616in}}%
\pgfpathcurveto{\pgfqpoint{9.631608in}{2.293616in}}{\pgfqpoint{9.636446in}{2.295620in}}{\pgfqpoint{9.640013in}{2.299187in}}%
\pgfpathcurveto{\pgfqpoint{9.643579in}{2.302753in}}{\pgfqpoint{9.645583in}{2.307591in}}{\pgfqpoint{9.645583in}{2.312634in}}%
\pgfpathcurveto{\pgfqpoint{9.645583in}{2.317678in}}{\pgfqpoint{9.643579in}{2.322516in}}{\pgfqpoint{9.640013in}{2.326082in}}%
\pgfpathcurveto{\pgfqpoint{9.636446in}{2.329649in}}{\pgfqpoint{9.631608in}{2.331653in}}{\pgfqpoint{9.626565in}{2.331653in}}%
\pgfpathcurveto{\pgfqpoint{9.621521in}{2.331653in}}{\pgfqpoint{9.616683in}{2.329649in}}{\pgfqpoint{9.613117in}{2.326082in}}%
\pgfpathcurveto{\pgfqpoint{9.609551in}{2.322516in}}{\pgfqpoint{9.607547in}{2.317678in}}{\pgfqpoint{9.607547in}{2.312634in}}%
\pgfpathcurveto{\pgfqpoint{9.607547in}{2.307591in}}{\pgfqpoint{9.609551in}{2.302753in}}{\pgfqpoint{9.613117in}{2.299187in}}%
\pgfpathcurveto{\pgfqpoint{9.616683in}{2.295620in}}{\pgfqpoint{9.621521in}{2.293616in}}{\pgfqpoint{9.626565in}{2.293616in}}%
\pgfpathclose%
\pgfusepath{fill}%
\end{pgfscope}%
\begin{pgfscope}%
\pgfpathrectangle{\pgfqpoint{6.572727in}{0.474100in}}{\pgfqpoint{4.227273in}{3.318700in}}%
\pgfusepath{clip}%
\pgfsetbuttcap%
\pgfsetroundjoin%
\definecolor{currentfill}{rgb}{0.993248,0.906157,0.143936}%
\pgfsetfillcolor{currentfill}%
\pgfsetfillopacity{0.700000}%
\pgfsetlinewidth{0.000000pt}%
\definecolor{currentstroke}{rgb}{0.000000,0.000000,0.000000}%
\pgfsetstrokecolor{currentstroke}%
\pgfsetstrokeopacity{0.700000}%
\pgfsetdash{}{0pt}%
\pgfpathmoveto{\pgfqpoint{10.067406in}{1.084939in}}%
\pgfpathcurveto{\pgfqpoint{10.072449in}{1.084939in}}{\pgfqpoint{10.077287in}{1.086943in}}{\pgfqpoint{10.080853in}{1.090509in}}%
\pgfpathcurveto{\pgfqpoint{10.084420in}{1.094076in}}{\pgfqpoint{10.086424in}{1.098914in}}{\pgfqpoint{10.086424in}{1.103957in}}%
\pgfpathcurveto{\pgfqpoint{10.086424in}{1.109001in}}{\pgfqpoint{10.084420in}{1.113839in}}{\pgfqpoint{10.080853in}{1.117405in}}%
\pgfpathcurveto{\pgfqpoint{10.077287in}{1.120972in}}{\pgfqpoint{10.072449in}{1.122975in}}{\pgfqpoint{10.067406in}{1.122975in}}%
\pgfpathcurveto{\pgfqpoint{10.062362in}{1.122975in}}{\pgfqpoint{10.057524in}{1.120972in}}{\pgfqpoint{10.053958in}{1.117405in}}%
\pgfpathcurveto{\pgfqpoint{10.050391in}{1.113839in}}{\pgfqpoint{10.048387in}{1.109001in}}{\pgfqpoint{10.048387in}{1.103957in}}%
\pgfpathcurveto{\pgfqpoint{10.048387in}{1.098914in}}{\pgfqpoint{10.050391in}{1.094076in}}{\pgfqpoint{10.053958in}{1.090509in}}%
\pgfpathcurveto{\pgfqpoint{10.057524in}{1.086943in}}{\pgfqpoint{10.062362in}{1.084939in}}{\pgfqpoint{10.067406in}{1.084939in}}%
\pgfpathclose%
\pgfusepath{fill}%
\end{pgfscope}%
\begin{pgfscope}%
\pgfpathrectangle{\pgfqpoint{6.572727in}{0.474100in}}{\pgfqpoint{4.227273in}{3.318700in}}%
\pgfusepath{clip}%
\pgfsetbuttcap%
\pgfsetroundjoin%
\definecolor{currentfill}{rgb}{0.127568,0.566949,0.550556}%
\pgfsetfillcolor{currentfill}%
\pgfsetfillopacity{0.700000}%
\pgfsetlinewidth{0.000000pt}%
\definecolor{currentstroke}{rgb}{0.000000,0.000000,0.000000}%
\pgfsetstrokecolor{currentstroke}%
\pgfsetstrokeopacity{0.700000}%
\pgfsetdash{}{0pt}%
\pgfpathmoveto{\pgfqpoint{7.406433in}{1.076929in}}%
\pgfpathcurveto{\pgfqpoint{7.411477in}{1.076929in}}{\pgfqpoint{7.416315in}{1.078933in}}{\pgfqpoint{7.419881in}{1.082500in}}%
\pgfpathcurveto{\pgfqpoint{7.423448in}{1.086066in}}{\pgfqpoint{7.425451in}{1.090904in}}{\pgfqpoint{7.425451in}{1.095947in}}%
\pgfpathcurveto{\pgfqpoint{7.425451in}{1.100991in}}{\pgfqpoint{7.423448in}{1.105829in}}{\pgfqpoint{7.419881in}{1.109395in}}%
\pgfpathcurveto{\pgfqpoint{7.416315in}{1.112962in}}{\pgfqpoint{7.411477in}{1.114966in}}{\pgfqpoint{7.406433in}{1.114966in}}%
\pgfpathcurveto{\pgfqpoint{7.401390in}{1.114966in}}{\pgfqpoint{7.396552in}{1.112962in}}{\pgfqpoint{7.392985in}{1.109395in}}%
\pgfpathcurveto{\pgfqpoint{7.389419in}{1.105829in}}{\pgfqpoint{7.387415in}{1.100991in}}{\pgfqpoint{7.387415in}{1.095947in}}%
\pgfpathcurveto{\pgfqpoint{7.387415in}{1.090904in}}{\pgfqpoint{7.389419in}{1.086066in}}{\pgfqpoint{7.392985in}{1.082500in}}%
\pgfpathcurveto{\pgfqpoint{7.396552in}{1.078933in}}{\pgfqpoint{7.401390in}{1.076929in}}{\pgfqpoint{7.406433in}{1.076929in}}%
\pgfpathclose%
\pgfusepath{fill}%
\end{pgfscope}%
\begin{pgfscope}%
\pgfpathrectangle{\pgfqpoint{6.572727in}{0.474100in}}{\pgfqpoint{4.227273in}{3.318700in}}%
\pgfusepath{clip}%
\pgfsetbuttcap%
\pgfsetroundjoin%
\definecolor{currentfill}{rgb}{0.127568,0.566949,0.550556}%
\pgfsetfillcolor{currentfill}%
\pgfsetfillopacity{0.700000}%
\pgfsetlinewidth{0.000000pt}%
\definecolor{currentstroke}{rgb}{0.000000,0.000000,0.000000}%
\pgfsetstrokecolor{currentstroke}%
\pgfsetstrokeopacity{0.700000}%
\pgfsetdash{}{0pt}%
\pgfpathmoveto{\pgfqpoint{7.584939in}{0.924488in}}%
\pgfpathcurveto{\pgfqpoint{7.589983in}{0.924488in}}{\pgfqpoint{7.594821in}{0.926492in}}{\pgfqpoint{7.598387in}{0.930059in}}%
\pgfpathcurveto{\pgfqpoint{7.601954in}{0.933625in}}{\pgfqpoint{7.603957in}{0.938463in}}{\pgfqpoint{7.603957in}{0.943507in}}%
\pgfpathcurveto{\pgfqpoint{7.603957in}{0.948550in}}{\pgfqpoint{7.601954in}{0.953388in}}{\pgfqpoint{7.598387in}{0.956954in}}%
\pgfpathcurveto{\pgfqpoint{7.594821in}{0.960521in}}{\pgfqpoint{7.589983in}{0.962525in}}{\pgfqpoint{7.584939in}{0.962525in}}%
\pgfpathcurveto{\pgfqpoint{7.579896in}{0.962525in}}{\pgfqpoint{7.575058in}{0.960521in}}{\pgfqpoint{7.571491in}{0.956954in}}%
\pgfpathcurveto{\pgfqpoint{7.567925in}{0.953388in}}{\pgfqpoint{7.565921in}{0.948550in}}{\pgfqpoint{7.565921in}{0.943507in}}%
\pgfpathcurveto{\pgfqpoint{7.565921in}{0.938463in}}{\pgfqpoint{7.567925in}{0.933625in}}{\pgfqpoint{7.571491in}{0.930059in}}%
\pgfpathcurveto{\pgfqpoint{7.575058in}{0.926492in}}{\pgfqpoint{7.579896in}{0.924488in}}{\pgfqpoint{7.584939in}{0.924488in}}%
\pgfpathclose%
\pgfusepath{fill}%
\end{pgfscope}%
\begin{pgfscope}%
\pgfpathrectangle{\pgfqpoint{6.572727in}{0.474100in}}{\pgfqpoint{4.227273in}{3.318700in}}%
\pgfusepath{clip}%
\pgfsetbuttcap%
\pgfsetroundjoin%
\definecolor{currentfill}{rgb}{0.127568,0.566949,0.550556}%
\pgfsetfillcolor{currentfill}%
\pgfsetfillopacity{0.700000}%
\pgfsetlinewidth{0.000000pt}%
\definecolor{currentstroke}{rgb}{0.000000,0.000000,0.000000}%
\pgfsetstrokecolor{currentstroke}%
\pgfsetstrokeopacity{0.700000}%
\pgfsetdash{}{0pt}%
\pgfpathmoveto{\pgfqpoint{7.637912in}{1.399358in}}%
\pgfpathcurveto{\pgfqpoint{7.642955in}{1.399358in}}{\pgfqpoint{7.647793in}{1.401362in}}{\pgfqpoint{7.651360in}{1.404928in}}%
\pgfpathcurveto{\pgfqpoint{7.654926in}{1.408494in}}{\pgfqpoint{7.656930in}{1.413332in}}{\pgfqpoint{7.656930in}{1.418376in}}%
\pgfpathcurveto{\pgfqpoint{7.656930in}{1.423420in}}{\pgfqpoint{7.654926in}{1.428257in}}{\pgfqpoint{7.651360in}{1.431824in}}%
\pgfpathcurveto{\pgfqpoint{7.647793in}{1.435390in}}{\pgfqpoint{7.642955in}{1.437394in}}{\pgfqpoint{7.637912in}{1.437394in}}%
\pgfpathcurveto{\pgfqpoint{7.632868in}{1.437394in}}{\pgfqpoint{7.628030in}{1.435390in}}{\pgfqpoint{7.624464in}{1.431824in}}%
\pgfpathcurveto{\pgfqpoint{7.620898in}{1.428257in}}{\pgfqpoint{7.618894in}{1.423420in}}{\pgfqpoint{7.618894in}{1.418376in}}%
\pgfpathcurveto{\pgfqpoint{7.618894in}{1.413332in}}{\pgfqpoint{7.620898in}{1.408494in}}{\pgfqpoint{7.624464in}{1.404928in}}%
\pgfpathcurveto{\pgfqpoint{7.628030in}{1.401362in}}{\pgfqpoint{7.632868in}{1.399358in}}{\pgfqpoint{7.637912in}{1.399358in}}%
\pgfpathclose%
\pgfusepath{fill}%
\end{pgfscope}%
\begin{pgfscope}%
\pgfpathrectangle{\pgfqpoint{6.572727in}{0.474100in}}{\pgfqpoint{4.227273in}{3.318700in}}%
\pgfusepath{clip}%
\pgfsetbuttcap%
\pgfsetroundjoin%
\definecolor{currentfill}{rgb}{0.127568,0.566949,0.550556}%
\pgfsetfillcolor{currentfill}%
\pgfsetfillopacity{0.700000}%
\pgfsetlinewidth{0.000000pt}%
\definecolor{currentstroke}{rgb}{0.000000,0.000000,0.000000}%
\pgfsetstrokecolor{currentstroke}%
\pgfsetstrokeopacity{0.700000}%
\pgfsetdash{}{0pt}%
\pgfpathmoveto{\pgfqpoint{7.696788in}{1.261967in}}%
\pgfpathcurveto{\pgfqpoint{7.701831in}{1.261967in}}{\pgfqpoint{7.706669in}{1.263971in}}{\pgfqpoint{7.710236in}{1.267537in}}%
\pgfpathcurveto{\pgfqpoint{7.713802in}{1.271103in}}{\pgfqpoint{7.715806in}{1.275941in}}{\pgfqpoint{7.715806in}{1.280985in}}%
\pgfpathcurveto{\pgfqpoint{7.715806in}{1.286028in}}{\pgfqpoint{7.713802in}{1.290866in}}{\pgfqpoint{7.710236in}{1.294433in}}%
\pgfpathcurveto{\pgfqpoint{7.706669in}{1.297999in}}{\pgfqpoint{7.701831in}{1.300003in}}{\pgfqpoint{7.696788in}{1.300003in}}%
\pgfpathcurveto{\pgfqpoint{7.691744in}{1.300003in}}{\pgfqpoint{7.686906in}{1.297999in}}{\pgfqpoint{7.683340in}{1.294433in}}%
\pgfpathcurveto{\pgfqpoint{7.679773in}{1.290866in}}{\pgfqpoint{7.677770in}{1.286028in}}{\pgfqpoint{7.677770in}{1.280985in}}%
\pgfpathcurveto{\pgfqpoint{7.677770in}{1.275941in}}{\pgfqpoint{7.679773in}{1.271103in}}{\pgfqpoint{7.683340in}{1.267537in}}%
\pgfpathcurveto{\pgfqpoint{7.686906in}{1.263971in}}{\pgfqpoint{7.691744in}{1.261967in}}{\pgfqpoint{7.696788in}{1.261967in}}%
\pgfpathclose%
\pgfusepath{fill}%
\end{pgfscope}%
\begin{pgfscope}%
\pgfpathrectangle{\pgfqpoint{6.572727in}{0.474100in}}{\pgfqpoint{4.227273in}{3.318700in}}%
\pgfusepath{clip}%
\pgfsetbuttcap%
\pgfsetroundjoin%
\definecolor{currentfill}{rgb}{0.127568,0.566949,0.550556}%
\pgfsetfillcolor{currentfill}%
\pgfsetfillopacity{0.700000}%
\pgfsetlinewidth{0.000000pt}%
\definecolor{currentstroke}{rgb}{0.000000,0.000000,0.000000}%
\pgfsetstrokecolor{currentstroke}%
\pgfsetstrokeopacity{0.700000}%
\pgfsetdash{}{0pt}%
\pgfpathmoveto{\pgfqpoint{7.882664in}{2.320399in}}%
\pgfpathcurveto{\pgfqpoint{7.887708in}{2.320399in}}{\pgfqpoint{7.892545in}{2.322403in}}{\pgfqpoint{7.896112in}{2.325969in}}%
\pgfpathcurveto{\pgfqpoint{7.899678in}{2.329536in}}{\pgfqpoint{7.901682in}{2.334373in}}{\pgfqpoint{7.901682in}{2.339417in}}%
\pgfpathcurveto{\pgfqpoint{7.901682in}{2.344461in}}{\pgfqpoint{7.899678in}{2.349299in}}{\pgfqpoint{7.896112in}{2.352865in}}%
\pgfpathcurveto{\pgfqpoint{7.892545in}{2.356431in}}{\pgfqpoint{7.887708in}{2.358435in}}{\pgfqpoint{7.882664in}{2.358435in}}%
\pgfpathcurveto{\pgfqpoint{7.877620in}{2.358435in}}{\pgfqpoint{7.872783in}{2.356431in}}{\pgfqpoint{7.869216in}{2.352865in}}%
\pgfpathcurveto{\pgfqpoint{7.865650in}{2.349299in}}{\pgfqpoint{7.863646in}{2.344461in}}{\pgfqpoint{7.863646in}{2.339417in}}%
\pgfpathcurveto{\pgfqpoint{7.863646in}{2.334373in}}{\pgfqpoint{7.865650in}{2.329536in}}{\pgfqpoint{7.869216in}{2.325969in}}%
\pgfpathcurveto{\pgfqpoint{7.872783in}{2.322403in}}{\pgfqpoint{7.877620in}{2.320399in}}{\pgfqpoint{7.882664in}{2.320399in}}%
\pgfpathclose%
\pgfusepath{fill}%
\end{pgfscope}%
\begin{pgfscope}%
\pgfpathrectangle{\pgfqpoint{6.572727in}{0.474100in}}{\pgfqpoint{4.227273in}{3.318700in}}%
\pgfusepath{clip}%
\pgfsetbuttcap%
\pgfsetroundjoin%
\definecolor{currentfill}{rgb}{0.993248,0.906157,0.143936}%
\pgfsetfillcolor{currentfill}%
\pgfsetfillopacity{0.700000}%
\pgfsetlinewidth{0.000000pt}%
\definecolor{currentstroke}{rgb}{0.000000,0.000000,0.000000}%
\pgfsetstrokecolor{currentstroke}%
\pgfsetstrokeopacity{0.700000}%
\pgfsetdash{}{0pt}%
\pgfpathmoveto{\pgfqpoint{9.492020in}{2.408279in}}%
\pgfpathcurveto{\pgfqpoint{9.497064in}{2.408279in}}{\pgfqpoint{9.501902in}{2.410283in}}{\pgfqpoint{9.505468in}{2.413849in}}%
\pgfpathcurveto{\pgfqpoint{9.509035in}{2.417415in}}{\pgfqpoint{9.511039in}{2.422253in}}{\pgfqpoint{9.511039in}{2.427297in}}%
\pgfpathcurveto{\pgfqpoint{9.511039in}{2.432341in}}{\pgfqpoint{9.509035in}{2.437178in}}{\pgfqpoint{9.505468in}{2.440745in}}%
\pgfpathcurveto{\pgfqpoint{9.501902in}{2.444311in}}{\pgfqpoint{9.497064in}{2.446315in}}{\pgfqpoint{9.492020in}{2.446315in}}%
\pgfpathcurveto{\pgfqpoint{9.486977in}{2.446315in}}{\pgfqpoint{9.482139in}{2.444311in}}{\pgfqpoint{9.478573in}{2.440745in}}%
\pgfpathcurveto{\pgfqpoint{9.475006in}{2.437178in}}{\pgfqpoint{9.473002in}{2.432341in}}{\pgfqpoint{9.473002in}{2.427297in}}%
\pgfpathcurveto{\pgfqpoint{9.473002in}{2.422253in}}{\pgfqpoint{9.475006in}{2.417415in}}{\pgfqpoint{9.478573in}{2.413849in}}%
\pgfpathcurveto{\pgfqpoint{9.482139in}{2.410283in}}{\pgfqpoint{9.486977in}{2.408279in}}{\pgfqpoint{9.492020in}{2.408279in}}%
\pgfpathclose%
\pgfusepath{fill}%
\end{pgfscope}%
\begin{pgfscope}%
\pgfpathrectangle{\pgfqpoint{6.572727in}{0.474100in}}{\pgfqpoint{4.227273in}{3.318700in}}%
\pgfusepath{clip}%
\pgfsetbuttcap%
\pgfsetroundjoin%
\definecolor{currentfill}{rgb}{0.127568,0.566949,0.550556}%
\pgfsetfillcolor{currentfill}%
\pgfsetfillopacity{0.700000}%
\pgfsetlinewidth{0.000000pt}%
\definecolor{currentstroke}{rgb}{0.000000,0.000000,0.000000}%
\pgfsetstrokecolor{currentstroke}%
\pgfsetstrokeopacity{0.700000}%
\pgfsetdash{}{0pt}%
\pgfpathmoveto{\pgfqpoint{8.117132in}{2.619036in}}%
\pgfpathcurveto{\pgfqpoint{8.122176in}{2.619036in}}{\pgfqpoint{8.127013in}{2.621040in}}{\pgfqpoint{8.130580in}{2.624606in}}%
\pgfpathcurveto{\pgfqpoint{8.134146in}{2.628173in}}{\pgfqpoint{8.136150in}{2.633010in}}{\pgfqpoint{8.136150in}{2.638054in}}%
\pgfpathcurveto{\pgfqpoint{8.136150in}{2.643098in}}{\pgfqpoint{8.134146in}{2.647936in}}{\pgfqpoint{8.130580in}{2.651502in}}%
\pgfpathcurveto{\pgfqpoint{8.127013in}{2.655068in}}{\pgfqpoint{8.122176in}{2.657072in}}{\pgfqpoint{8.117132in}{2.657072in}}%
\pgfpathcurveto{\pgfqpoint{8.112088in}{2.657072in}}{\pgfqpoint{8.107250in}{2.655068in}}{\pgfqpoint{8.103684in}{2.651502in}}%
\pgfpathcurveto{\pgfqpoint{8.100118in}{2.647936in}}{\pgfqpoint{8.098114in}{2.643098in}}{\pgfqpoint{8.098114in}{2.638054in}}%
\pgfpathcurveto{\pgfqpoint{8.098114in}{2.633010in}}{\pgfqpoint{8.100118in}{2.628173in}}{\pgfqpoint{8.103684in}{2.624606in}}%
\pgfpathcurveto{\pgfqpoint{8.107250in}{2.621040in}}{\pgfqpoint{8.112088in}{2.619036in}}{\pgfqpoint{8.117132in}{2.619036in}}%
\pgfpathclose%
\pgfusepath{fill}%
\end{pgfscope}%
\begin{pgfscope}%
\pgfpathrectangle{\pgfqpoint{6.572727in}{0.474100in}}{\pgfqpoint{4.227273in}{3.318700in}}%
\pgfusepath{clip}%
\pgfsetbuttcap%
\pgfsetroundjoin%
\definecolor{currentfill}{rgb}{0.127568,0.566949,0.550556}%
\pgfsetfillcolor{currentfill}%
\pgfsetfillopacity{0.700000}%
\pgfsetlinewidth{0.000000pt}%
\definecolor{currentstroke}{rgb}{0.000000,0.000000,0.000000}%
\pgfsetstrokecolor{currentstroke}%
\pgfsetstrokeopacity{0.700000}%
\pgfsetdash{}{0pt}%
\pgfpathmoveto{\pgfqpoint{7.589798in}{1.772762in}}%
\pgfpathcurveto{\pgfqpoint{7.594842in}{1.772762in}}{\pgfqpoint{7.599680in}{1.774766in}}{\pgfqpoint{7.603246in}{1.778333in}}%
\pgfpathcurveto{\pgfqpoint{7.606812in}{1.781899in}}{\pgfqpoint{7.608816in}{1.786737in}}{\pgfqpoint{7.608816in}{1.791780in}}%
\pgfpathcurveto{\pgfqpoint{7.608816in}{1.796824in}}{\pgfqpoint{7.606812in}{1.801662in}}{\pgfqpoint{7.603246in}{1.805228in}}%
\pgfpathcurveto{\pgfqpoint{7.599680in}{1.808795in}}{\pgfqpoint{7.594842in}{1.810799in}}{\pgfqpoint{7.589798in}{1.810799in}}%
\pgfpathcurveto{\pgfqpoint{7.584754in}{1.810799in}}{\pgfqpoint{7.579917in}{1.808795in}}{\pgfqpoint{7.576350in}{1.805228in}}%
\pgfpathcurveto{\pgfqpoint{7.572784in}{1.801662in}}{\pgfqpoint{7.570780in}{1.796824in}}{\pgfqpoint{7.570780in}{1.791780in}}%
\pgfpathcurveto{\pgfqpoint{7.570780in}{1.786737in}}{\pgfqpoint{7.572784in}{1.781899in}}{\pgfqpoint{7.576350in}{1.778333in}}%
\pgfpathcurveto{\pgfqpoint{7.579917in}{1.774766in}}{\pgfqpoint{7.584754in}{1.772762in}}{\pgfqpoint{7.589798in}{1.772762in}}%
\pgfpathclose%
\pgfusepath{fill}%
\end{pgfscope}%
\begin{pgfscope}%
\pgfpathrectangle{\pgfqpoint{6.572727in}{0.474100in}}{\pgfqpoint{4.227273in}{3.318700in}}%
\pgfusepath{clip}%
\pgfsetbuttcap%
\pgfsetroundjoin%
\definecolor{currentfill}{rgb}{0.993248,0.906157,0.143936}%
\pgfsetfillcolor{currentfill}%
\pgfsetfillopacity{0.700000}%
\pgfsetlinewidth{0.000000pt}%
\definecolor{currentstroke}{rgb}{0.000000,0.000000,0.000000}%
\pgfsetstrokecolor{currentstroke}%
\pgfsetstrokeopacity{0.700000}%
\pgfsetdash{}{0pt}%
\pgfpathmoveto{\pgfqpoint{9.617024in}{1.391560in}}%
\pgfpathcurveto{\pgfqpoint{9.622067in}{1.391560in}}{\pgfqpoint{9.626905in}{1.393564in}}{\pgfqpoint{9.630471in}{1.397130in}}%
\pgfpathcurveto{\pgfqpoint{9.634038in}{1.400697in}}{\pgfqpoint{9.636042in}{1.405534in}}{\pgfqpoint{9.636042in}{1.410578in}}%
\pgfpathcurveto{\pgfqpoint{9.636042in}{1.415622in}}{\pgfqpoint{9.634038in}{1.420459in}}{\pgfqpoint{9.630471in}{1.424026in}}%
\pgfpathcurveto{\pgfqpoint{9.626905in}{1.427592in}}{\pgfqpoint{9.622067in}{1.429596in}}{\pgfqpoint{9.617024in}{1.429596in}}%
\pgfpathcurveto{\pgfqpoint{9.611980in}{1.429596in}}{\pgfqpoint{9.607142in}{1.427592in}}{\pgfqpoint{9.603576in}{1.424026in}}%
\pgfpathcurveto{\pgfqpoint{9.600009in}{1.420459in}}{\pgfqpoint{9.598005in}{1.415622in}}{\pgfqpoint{9.598005in}{1.410578in}}%
\pgfpathcurveto{\pgfqpoint{9.598005in}{1.405534in}}{\pgfqpoint{9.600009in}{1.400697in}}{\pgfqpoint{9.603576in}{1.397130in}}%
\pgfpathcurveto{\pgfqpoint{9.607142in}{1.393564in}}{\pgfqpoint{9.611980in}{1.391560in}}{\pgfqpoint{9.617024in}{1.391560in}}%
\pgfpathclose%
\pgfusepath{fill}%
\end{pgfscope}%
\begin{pgfscope}%
\pgfpathrectangle{\pgfqpoint{6.572727in}{0.474100in}}{\pgfqpoint{4.227273in}{3.318700in}}%
\pgfusepath{clip}%
\pgfsetbuttcap%
\pgfsetroundjoin%
\definecolor{currentfill}{rgb}{0.127568,0.566949,0.550556}%
\pgfsetfillcolor{currentfill}%
\pgfsetfillopacity{0.700000}%
\pgfsetlinewidth{0.000000pt}%
\definecolor{currentstroke}{rgb}{0.000000,0.000000,0.000000}%
\pgfsetstrokecolor{currentstroke}%
\pgfsetstrokeopacity{0.700000}%
\pgfsetdash{}{0pt}%
\pgfpathmoveto{\pgfqpoint{7.555599in}{1.889150in}}%
\pgfpathcurveto{\pgfqpoint{7.560642in}{1.889150in}}{\pgfqpoint{7.565480in}{1.891154in}}{\pgfqpoint{7.569047in}{1.894721in}}%
\pgfpathcurveto{\pgfqpoint{7.572613in}{1.898287in}}{\pgfqpoint{7.574617in}{1.903125in}}{\pgfqpoint{7.574617in}{1.908169in}}%
\pgfpathcurveto{\pgfqpoint{7.574617in}{1.913212in}}{\pgfqpoint{7.572613in}{1.918050in}}{\pgfqpoint{7.569047in}{1.921616in}}%
\pgfpathcurveto{\pgfqpoint{7.565480in}{1.925183in}}{\pgfqpoint{7.560642in}{1.927187in}}{\pgfqpoint{7.555599in}{1.927187in}}%
\pgfpathcurveto{\pgfqpoint{7.550555in}{1.927187in}}{\pgfqpoint{7.545717in}{1.925183in}}{\pgfqpoint{7.542151in}{1.921616in}}%
\pgfpathcurveto{\pgfqpoint{7.538584in}{1.918050in}}{\pgfqpoint{7.536581in}{1.913212in}}{\pgfqpoint{7.536581in}{1.908169in}}%
\pgfpathcurveto{\pgfqpoint{7.536581in}{1.903125in}}{\pgfqpoint{7.538584in}{1.898287in}}{\pgfqpoint{7.542151in}{1.894721in}}%
\pgfpathcurveto{\pgfqpoint{7.545717in}{1.891154in}}{\pgfqpoint{7.550555in}{1.889150in}}{\pgfqpoint{7.555599in}{1.889150in}}%
\pgfpathclose%
\pgfusepath{fill}%
\end{pgfscope}%
\begin{pgfscope}%
\pgfpathrectangle{\pgfqpoint{6.572727in}{0.474100in}}{\pgfqpoint{4.227273in}{3.318700in}}%
\pgfusepath{clip}%
\pgfsetbuttcap%
\pgfsetroundjoin%
\definecolor{currentfill}{rgb}{0.127568,0.566949,0.550556}%
\pgfsetfillcolor{currentfill}%
\pgfsetfillopacity{0.700000}%
\pgfsetlinewidth{0.000000pt}%
\definecolor{currentstroke}{rgb}{0.000000,0.000000,0.000000}%
\pgfsetstrokecolor{currentstroke}%
\pgfsetstrokeopacity{0.700000}%
\pgfsetdash{}{0pt}%
\pgfpathmoveto{\pgfqpoint{7.765132in}{1.467946in}}%
\pgfpathcurveto{\pgfqpoint{7.770175in}{1.467946in}}{\pgfqpoint{7.775013in}{1.469950in}}{\pgfqpoint{7.778580in}{1.473517in}}%
\pgfpathcurveto{\pgfqpoint{7.782146in}{1.477083in}}{\pgfqpoint{7.784150in}{1.481921in}}{\pgfqpoint{7.784150in}{1.486965in}}%
\pgfpathcurveto{\pgfqpoint{7.784150in}{1.492008in}}{\pgfqpoint{7.782146in}{1.496846in}}{\pgfqpoint{7.778580in}{1.500412in}}%
\pgfpathcurveto{\pgfqpoint{7.775013in}{1.503979in}}{\pgfqpoint{7.770175in}{1.505983in}}{\pgfqpoint{7.765132in}{1.505983in}}%
\pgfpathcurveto{\pgfqpoint{7.760088in}{1.505983in}}{\pgfqpoint{7.755250in}{1.503979in}}{\pgfqpoint{7.751684in}{1.500412in}}%
\pgfpathcurveto{\pgfqpoint{7.748117in}{1.496846in}}{\pgfqpoint{7.746114in}{1.492008in}}{\pgfqpoint{7.746114in}{1.486965in}}%
\pgfpathcurveto{\pgfqpoint{7.746114in}{1.481921in}}{\pgfqpoint{7.748117in}{1.477083in}}{\pgfqpoint{7.751684in}{1.473517in}}%
\pgfpathcurveto{\pgfqpoint{7.755250in}{1.469950in}}{\pgfqpoint{7.760088in}{1.467946in}}{\pgfqpoint{7.765132in}{1.467946in}}%
\pgfpathclose%
\pgfusepath{fill}%
\end{pgfscope}%
\begin{pgfscope}%
\pgfpathrectangle{\pgfqpoint{6.572727in}{0.474100in}}{\pgfqpoint{4.227273in}{3.318700in}}%
\pgfusepath{clip}%
\pgfsetbuttcap%
\pgfsetroundjoin%
\definecolor{currentfill}{rgb}{0.127568,0.566949,0.550556}%
\pgfsetfillcolor{currentfill}%
\pgfsetfillopacity{0.700000}%
\pgfsetlinewidth{0.000000pt}%
\definecolor{currentstroke}{rgb}{0.000000,0.000000,0.000000}%
\pgfsetstrokecolor{currentstroke}%
\pgfsetstrokeopacity{0.700000}%
\pgfsetdash{}{0pt}%
\pgfpathmoveto{\pgfqpoint{8.390330in}{2.447414in}}%
\pgfpathcurveto{\pgfqpoint{8.395374in}{2.447414in}}{\pgfqpoint{8.400212in}{2.449418in}}{\pgfqpoint{8.403778in}{2.452985in}}%
\pgfpathcurveto{\pgfqpoint{8.407344in}{2.456551in}}{\pgfqpoint{8.409348in}{2.461389in}}{\pgfqpoint{8.409348in}{2.466433in}}%
\pgfpathcurveto{\pgfqpoint{8.409348in}{2.471476in}}{\pgfqpoint{8.407344in}{2.476314in}}{\pgfqpoint{8.403778in}{2.479880in}}%
\pgfpathcurveto{\pgfqpoint{8.400212in}{2.483447in}}{\pgfqpoint{8.395374in}{2.485451in}}{\pgfqpoint{8.390330in}{2.485451in}}%
\pgfpathcurveto{\pgfqpoint{8.385286in}{2.485451in}}{\pgfqpoint{8.380449in}{2.483447in}}{\pgfqpoint{8.376882in}{2.479880in}}%
\pgfpathcurveto{\pgfqpoint{8.373316in}{2.476314in}}{\pgfqpoint{8.371312in}{2.471476in}}{\pgfqpoint{8.371312in}{2.466433in}}%
\pgfpathcurveto{\pgfqpoint{8.371312in}{2.461389in}}{\pgfqpoint{8.373316in}{2.456551in}}{\pgfqpoint{8.376882in}{2.452985in}}%
\pgfpathcurveto{\pgfqpoint{8.380449in}{2.449418in}}{\pgfqpoint{8.385286in}{2.447414in}}{\pgfqpoint{8.390330in}{2.447414in}}%
\pgfpathclose%
\pgfusepath{fill}%
\end{pgfscope}%
\begin{pgfscope}%
\pgfpathrectangle{\pgfqpoint{6.572727in}{0.474100in}}{\pgfqpoint{4.227273in}{3.318700in}}%
\pgfusepath{clip}%
\pgfsetbuttcap%
\pgfsetroundjoin%
\definecolor{currentfill}{rgb}{0.127568,0.566949,0.550556}%
\pgfsetfillcolor{currentfill}%
\pgfsetfillopacity{0.700000}%
\pgfsetlinewidth{0.000000pt}%
\definecolor{currentstroke}{rgb}{0.000000,0.000000,0.000000}%
\pgfsetstrokecolor{currentstroke}%
\pgfsetstrokeopacity{0.700000}%
\pgfsetdash{}{0pt}%
\pgfpathmoveto{\pgfqpoint{7.505418in}{1.258751in}}%
\pgfpathcurveto{\pgfqpoint{7.510462in}{1.258751in}}{\pgfqpoint{7.515299in}{1.260755in}}{\pgfqpoint{7.518866in}{1.264321in}}%
\pgfpathcurveto{\pgfqpoint{7.522432in}{1.267888in}}{\pgfqpoint{7.524436in}{1.272725in}}{\pgfqpoint{7.524436in}{1.277769in}}%
\pgfpathcurveto{\pgfqpoint{7.524436in}{1.282813in}}{\pgfqpoint{7.522432in}{1.287650in}}{\pgfqpoint{7.518866in}{1.291217in}}%
\pgfpathcurveto{\pgfqpoint{7.515299in}{1.294783in}}{\pgfqpoint{7.510462in}{1.296787in}}{\pgfqpoint{7.505418in}{1.296787in}}%
\pgfpathcurveto{\pgfqpoint{7.500374in}{1.296787in}}{\pgfqpoint{7.495536in}{1.294783in}}{\pgfqpoint{7.491970in}{1.291217in}}%
\pgfpathcurveto{\pgfqpoint{7.488404in}{1.287650in}}{\pgfqpoint{7.486400in}{1.282813in}}{\pgfqpoint{7.486400in}{1.277769in}}%
\pgfpathcurveto{\pgfqpoint{7.486400in}{1.272725in}}{\pgfqpoint{7.488404in}{1.267888in}}{\pgfqpoint{7.491970in}{1.264321in}}%
\pgfpathcurveto{\pgfqpoint{7.495536in}{1.260755in}}{\pgfqpoint{7.500374in}{1.258751in}}{\pgfqpoint{7.505418in}{1.258751in}}%
\pgfpathclose%
\pgfusepath{fill}%
\end{pgfscope}%
\begin{pgfscope}%
\pgfpathrectangle{\pgfqpoint{6.572727in}{0.474100in}}{\pgfqpoint{4.227273in}{3.318700in}}%
\pgfusepath{clip}%
\pgfsetbuttcap%
\pgfsetroundjoin%
\definecolor{currentfill}{rgb}{0.993248,0.906157,0.143936}%
\pgfsetfillcolor{currentfill}%
\pgfsetfillopacity{0.700000}%
\pgfsetlinewidth{0.000000pt}%
\definecolor{currentstroke}{rgb}{0.000000,0.000000,0.000000}%
\pgfsetstrokecolor{currentstroke}%
\pgfsetstrokeopacity{0.700000}%
\pgfsetdash{}{0pt}%
\pgfpathmoveto{\pgfqpoint{9.308547in}{1.978042in}}%
\pgfpathcurveto{\pgfqpoint{9.313590in}{1.978042in}}{\pgfqpoint{9.318428in}{1.980045in}}{\pgfqpoint{9.321995in}{1.983612in}}%
\pgfpathcurveto{\pgfqpoint{9.325561in}{1.987178in}}{\pgfqpoint{9.327565in}{1.992016in}}{\pgfqpoint{9.327565in}{1.997060in}}%
\pgfpathcurveto{\pgfqpoint{9.327565in}{2.002103in}}{\pgfqpoint{9.325561in}{2.006941in}}{\pgfqpoint{9.321995in}{2.010508in}}%
\pgfpathcurveto{\pgfqpoint{9.318428in}{2.014074in}}{\pgfqpoint{9.313590in}{2.016078in}}{\pgfqpoint{9.308547in}{2.016078in}}%
\pgfpathcurveto{\pgfqpoint{9.303503in}{2.016078in}}{\pgfqpoint{9.298665in}{2.014074in}}{\pgfqpoint{9.295099in}{2.010508in}}%
\pgfpathcurveto{\pgfqpoint{9.291532in}{2.006941in}}{\pgfqpoint{9.289529in}{2.002103in}}{\pgfqpoint{9.289529in}{1.997060in}}%
\pgfpathcurveto{\pgfqpoint{9.289529in}{1.992016in}}{\pgfqpoint{9.291532in}{1.987178in}}{\pgfqpoint{9.295099in}{1.983612in}}%
\pgfpathcurveto{\pgfqpoint{9.298665in}{1.980045in}}{\pgfqpoint{9.303503in}{1.978042in}}{\pgfqpoint{9.308547in}{1.978042in}}%
\pgfpathclose%
\pgfusepath{fill}%
\end{pgfscope}%
\begin{pgfscope}%
\pgfpathrectangle{\pgfqpoint{6.572727in}{0.474100in}}{\pgfqpoint{4.227273in}{3.318700in}}%
\pgfusepath{clip}%
\pgfsetbuttcap%
\pgfsetroundjoin%
\definecolor{currentfill}{rgb}{0.127568,0.566949,0.550556}%
\pgfsetfillcolor{currentfill}%
\pgfsetfillopacity{0.700000}%
\pgfsetlinewidth{0.000000pt}%
\definecolor{currentstroke}{rgb}{0.000000,0.000000,0.000000}%
\pgfsetstrokecolor{currentstroke}%
\pgfsetstrokeopacity{0.700000}%
\pgfsetdash{}{0pt}%
\pgfpathmoveto{\pgfqpoint{7.873616in}{1.609758in}}%
\pgfpathcurveto{\pgfqpoint{7.878660in}{1.609758in}}{\pgfqpoint{7.883498in}{1.611762in}}{\pgfqpoint{7.887064in}{1.615328in}}%
\pgfpathcurveto{\pgfqpoint{7.890630in}{1.618894in}}{\pgfqpoint{7.892634in}{1.623732in}}{\pgfqpoint{7.892634in}{1.628776in}}%
\pgfpathcurveto{\pgfqpoint{7.892634in}{1.633820in}}{\pgfqpoint{7.890630in}{1.638657in}}{\pgfqpoint{7.887064in}{1.642224in}}%
\pgfpathcurveto{\pgfqpoint{7.883498in}{1.645790in}}{\pgfqpoint{7.878660in}{1.647794in}}{\pgfqpoint{7.873616in}{1.647794in}}%
\pgfpathcurveto{\pgfqpoint{7.868572in}{1.647794in}}{\pgfqpoint{7.863735in}{1.645790in}}{\pgfqpoint{7.860168in}{1.642224in}}%
\pgfpathcurveto{\pgfqpoint{7.856602in}{1.638657in}}{\pgfqpoint{7.854598in}{1.633820in}}{\pgfqpoint{7.854598in}{1.628776in}}%
\pgfpathcurveto{\pgfqpoint{7.854598in}{1.623732in}}{\pgfqpoint{7.856602in}{1.618894in}}{\pgfqpoint{7.860168in}{1.615328in}}%
\pgfpathcurveto{\pgfqpoint{7.863735in}{1.611762in}}{\pgfqpoint{7.868572in}{1.609758in}}{\pgfqpoint{7.873616in}{1.609758in}}%
\pgfpathclose%
\pgfusepath{fill}%
\end{pgfscope}%
\begin{pgfscope}%
\pgfpathrectangle{\pgfqpoint{6.572727in}{0.474100in}}{\pgfqpoint{4.227273in}{3.318700in}}%
\pgfusepath{clip}%
\pgfsetbuttcap%
\pgfsetroundjoin%
\definecolor{currentfill}{rgb}{0.993248,0.906157,0.143936}%
\pgfsetfillcolor{currentfill}%
\pgfsetfillopacity{0.700000}%
\pgfsetlinewidth{0.000000pt}%
\definecolor{currentstroke}{rgb}{0.000000,0.000000,0.000000}%
\pgfsetstrokecolor{currentstroke}%
\pgfsetstrokeopacity{0.700000}%
\pgfsetdash{}{0pt}%
\pgfpathmoveto{\pgfqpoint{9.908448in}{1.458857in}}%
\pgfpathcurveto{\pgfqpoint{9.913492in}{1.458857in}}{\pgfqpoint{9.918329in}{1.460861in}}{\pgfqpoint{9.921896in}{1.464428in}}%
\pgfpathcurveto{\pgfqpoint{9.925462in}{1.467994in}}{\pgfqpoint{9.927466in}{1.472832in}}{\pgfqpoint{9.927466in}{1.477876in}}%
\pgfpathcurveto{\pgfqpoint{9.927466in}{1.482919in}}{\pgfqpoint{9.925462in}{1.487757in}}{\pgfqpoint{9.921896in}{1.491323in}}%
\pgfpathcurveto{\pgfqpoint{9.918329in}{1.494890in}}{\pgfqpoint{9.913492in}{1.496894in}}{\pgfqpoint{9.908448in}{1.496894in}}%
\pgfpathcurveto{\pgfqpoint{9.903404in}{1.496894in}}{\pgfqpoint{9.898566in}{1.494890in}}{\pgfqpoint{9.895000in}{1.491323in}}%
\pgfpathcurveto{\pgfqpoint{9.891434in}{1.487757in}}{\pgfqpoint{9.889430in}{1.482919in}}{\pgfqpoint{9.889430in}{1.477876in}}%
\pgfpathcurveto{\pgfqpoint{9.889430in}{1.472832in}}{\pgfqpoint{9.891434in}{1.467994in}}{\pgfqpoint{9.895000in}{1.464428in}}%
\pgfpathcurveto{\pgfqpoint{9.898566in}{1.460861in}}{\pgfqpoint{9.903404in}{1.458857in}}{\pgfqpoint{9.908448in}{1.458857in}}%
\pgfpathclose%
\pgfusepath{fill}%
\end{pgfscope}%
\begin{pgfscope}%
\pgfpathrectangle{\pgfqpoint{6.572727in}{0.474100in}}{\pgfqpoint{4.227273in}{3.318700in}}%
\pgfusepath{clip}%
\pgfsetbuttcap%
\pgfsetroundjoin%
\definecolor{currentfill}{rgb}{0.127568,0.566949,0.550556}%
\pgfsetfillcolor{currentfill}%
\pgfsetfillopacity{0.700000}%
\pgfsetlinewidth{0.000000pt}%
\definecolor{currentstroke}{rgb}{0.000000,0.000000,0.000000}%
\pgfsetstrokecolor{currentstroke}%
\pgfsetstrokeopacity{0.700000}%
\pgfsetdash{}{0pt}%
\pgfpathmoveto{\pgfqpoint{7.982655in}{1.692162in}}%
\pgfpathcurveto{\pgfqpoint{7.987699in}{1.692162in}}{\pgfqpoint{7.992537in}{1.694165in}}{\pgfqpoint{7.996103in}{1.697732in}}%
\pgfpathcurveto{\pgfqpoint{7.999669in}{1.701298in}}{\pgfqpoint{8.001673in}{1.706136in}}{\pgfqpoint{8.001673in}{1.711180in}}%
\pgfpathcurveto{\pgfqpoint{8.001673in}{1.716223in}}{\pgfqpoint{7.999669in}{1.721061in}}{\pgfqpoint{7.996103in}{1.724628in}}%
\pgfpathcurveto{\pgfqpoint{7.992537in}{1.728194in}}{\pgfqpoint{7.987699in}{1.730198in}}{\pgfqpoint{7.982655in}{1.730198in}}%
\pgfpathcurveto{\pgfqpoint{7.977612in}{1.730198in}}{\pgfqpoint{7.972774in}{1.728194in}}{\pgfqpoint{7.969207in}{1.724628in}}%
\pgfpathcurveto{\pgfqpoint{7.965641in}{1.721061in}}{\pgfqpoint{7.963637in}{1.716223in}}{\pgfqpoint{7.963637in}{1.711180in}}%
\pgfpathcurveto{\pgfqpoint{7.963637in}{1.706136in}}{\pgfqpoint{7.965641in}{1.701298in}}{\pgfqpoint{7.969207in}{1.697732in}}%
\pgfpathcurveto{\pgfqpoint{7.972774in}{1.694165in}}{\pgfqpoint{7.977612in}{1.692162in}}{\pgfqpoint{7.982655in}{1.692162in}}%
\pgfpathclose%
\pgfusepath{fill}%
\end{pgfscope}%
\begin{pgfscope}%
\pgfpathrectangle{\pgfqpoint{6.572727in}{0.474100in}}{\pgfqpoint{4.227273in}{3.318700in}}%
\pgfusepath{clip}%
\pgfsetbuttcap%
\pgfsetroundjoin%
\definecolor{currentfill}{rgb}{0.127568,0.566949,0.550556}%
\pgfsetfillcolor{currentfill}%
\pgfsetfillopacity{0.700000}%
\pgfsetlinewidth{0.000000pt}%
\definecolor{currentstroke}{rgb}{0.000000,0.000000,0.000000}%
\pgfsetstrokecolor{currentstroke}%
\pgfsetstrokeopacity{0.700000}%
\pgfsetdash{}{0pt}%
\pgfpathmoveto{\pgfqpoint{7.948766in}{2.438947in}}%
\pgfpathcurveto{\pgfqpoint{7.953810in}{2.438947in}}{\pgfqpoint{7.958647in}{2.440951in}}{\pgfqpoint{7.962214in}{2.444517in}}%
\pgfpathcurveto{\pgfqpoint{7.965780in}{2.448084in}}{\pgfqpoint{7.967784in}{2.452922in}}{\pgfqpoint{7.967784in}{2.457965in}}%
\pgfpathcurveto{\pgfqpoint{7.967784in}{2.463009in}}{\pgfqpoint{7.965780in}{2.467847in}}{\pgfqpoint{7.962214in}{2.471413in}}%
\pgfpathcurveto{\pgfqpoint{7.958647in}{2.474980in}}{\pgfqpoint{7.953810in}{2.476983in}}{\pgfqpoint{7.948766in}{2.476983in}}%
\pgfpathcurveto{\pgfqpoint{7.943722in}{2.476983in}}{\pgfqpoint{7.938885in}{2.474980in}}{\pgfqpoint{7.935318in}{2.471413in}}%
\pgfpathcurveto{\pgfqpoint{7.931752in}{2.467847in}}{\pgfqpoint{7.929748in}{2.463009in}}{\pgfqpoint{7.929748in}{2.457965in}}%
\pgfpathcurveto{\pgfqpoint{7.929748in}{2.452922in}}{\pgfqpoint{7.931752in}{2.448084in}}{\pgfqpoint{7.935318in}{2.444517in}}%
\pgfpathcurveto{\pgfqpoint{7.938885in}{2.440951in}}{\pgfqpoint{7.943722in}{2.438947in}}{\pgfqpoint{7.948766in}{2.438947in}}%
\pgfpathclose%
\pgfusepath{fill}%
\end{pgfscope}%
\begin{pgfscope}%
\pgfpathrectangle{\pgfqpoint{6.572727in}{0.474100in}}{\pgfqpoint{4.227273in}{3.318700in}}%
\pgfusepath{clip}%
\pgfsetbuttcap%
\pgfsetroundjoin%
\definecolor{currentfill}{rgb}{0.127568,0.566949,0.550556}%
\pgfsetfillcolor{currentfill}%
\pgfsetfillopacity{0.700000}%
\pgfsetlinewidth{0.000000pt}%
\definecolor{currentstroke}{rgb}{0.000000,0.000000,0.000000}%
\pgfsetstrokecolor{currentstroke}%
\pgfsetstrokeopacity{0.700000}%
\pgfsetdash{}{0pt}%
\pgfpathmoveto{\pgfqpoint{7.267644in}{3.017274in}}%
\pgfpathcurveto{\pgfqpoint{7.272687in}{3.017274in}}{\pgfqpoint{7.277525in}{3.019278in}}{\pgfqpoint{7.281091in}{3.022845in}}%
\pgfpathcurveto{\pgfqpoint{7.284658in}{3.026411in}}{\pgfqpoint{7.286662in}{3.031249in}}{\pgfqpoint{7.286662in}{3.036292in}}%
\pgfpathcurveto{\pgfqpoint{7.286662in}{3.041336in}}{\pgfqpoint{7.284658in}{3.046174in}}{\pgfqpoint{7.281091in}{3.049740in}}%
\pgfpathcurveto{\pgfqpoint{7.277525in}{3.053307in}}{\pgfqpoint{7.272687in}{3.055311in}}{\pgfqpoint{7.267644in}{3.055311in}}%
\pgfpathcurveto{\pgfqpoint{7.262600in}{3.055311in}}{\pgfqpoint{7.257762in}{3.053307in}}{\pgfqpoint{7.254196in}{3.049740in}}%
\pgfpathcurveto{\pgfqpoint{7.250629in}{3.046174in}}{\pgfqpoint{7.248625in}{3.041336in}}{\pgfqpoint{7.248625in}{3.036292in}}%
\pgfpathcurveto{\pgfqpoint{7.248625in}{3.031249in}}{\pgfqpoint{7.250629in}{3.026411in}}{\pgfqpoint{7.254196in}{3.022845in}}%
\pgfpathcurveto{\pgfqpoint{7.257762in}{3.019278in}}{\pgfqpoint{7.262600in}{3.017274in}}{\pgfqpoint{7.267644in}{3.017274in}}%
\pgfpathclose%
\pgfusepath{fill}%
\end{pgfscope}%
\begin{pgfscope}%
\pgfpathrectangle{\pgfqpoint{6.572727in}{0.474100in}}{\pgfqpoint{4.227273in}{3.318700in}}%
\pgfusepath{clip}%
\pgfsetbuttcap%
\pgfsetroundjoin%
\definecolor{currentfill}{rgb}{0.993248,0.906157,0.143936}%
\pgfsetfillcolor{currentfill}%
\pgfsetfillopacity{0.700000}%
\pgfsetlinewidth{0.000000pt}%
\definecolor{currentstroke}{rgb}{0.000000,0.000000,0.000000}%
\pgfsetstrokecolor{currentstroke}%
\pgfsetstrokeopacity{0.700000}%
\pgfsetdash{}{0pt}%
\pgfpathmoveto{\pgfqpoint{9.764049in}{1.895044in}}%
\pgfpathcurveto{\pgfqpoint{9.769093in}{1.895044in}}{\pgfqpoint{9.773931in}{1.897048in}}{\pgfqpoint{9.777497in}{1.900614in}}%
\pgfpathcurveto{\pgfqpoint{9.781064in}{1.904181in}}{\pgfqpoint{9.783067in}{1.909018in}}{\pgfqpoint{9.783067in}{1.914062in}}%
\pgfpathcurveto{\pgfqpoint{9.783067in}{1.919106in}}{\pgfqpoint{9.781064in}{1.923943in}}{\pgfqpoint{9.777497in}{1.927510in}}%
\pgfpathcurveto{\pgfqpoint{9.773931in}{1.931076in}}{\pgfqpoint{9.769093in}{1.933080in}}{\pgfqpoint{9.764049in}{1.933080in}}%
\pgfpathcurveto{\pgfqpoint{9.759006in}{1.933080in}}{\pgfqpoint{9.754168in}{1.931076in}}{\pgfqpoint{9.750601in}{1.927510in}}%
\pgfpathcurveto{\pgfqpoint{9.747035in}{1.923943in}}{\pgfqpoint{9.745031in}{1.919106in}}{\pgfqpoint{9.745031in}{1.914062in}}%
\pgfpathcurveto{\pgfqpoint{9.745031in}{1.909018in}}{\pgfqpoint{9.747035in}{1.904181in}}{\pgfqpoint{9.750601in}{1.900614in}}%
\pgfpathcurveto{\pgfqpoint{9.754168in}{1.897048in}}{\pgfqpoint{9.759006in}{1.895044in}}{\pgfqpoint{9.764049in}{1.895044in}}%
\pgfpathclose%
\pgfusepath{fill}%
\end{pgfscope}%
\begin{pgfscope}%
\pgfpathrectangle{\pgfqpoint{6.572727in}{0.474100in}}{\pgfqpoint{4.227273in}{3.318700in}}%
\pgfusepath{clip}%
\pgfsetbuttcap%
\pgfsetroundjoin%
\definecolor{currentfill}{rgb}{0.127568,0.566949,0.550556}%
\pgfsetfillcolor{currentfill}%
\pgfsetfillopacity{0.700000}%
\pgfsetlinewidth{0.000000pt}%
\definecolor{currentstroke}{rgb}{0.000000,0.000000,0.000000}%
\pgfsetstrokecolor{currentstroke}%
\pgfsetstrokeopacity{0.700000}%
\pgfsetdash{}{0pt}%
\pgfpathmoveto{\pgfqpoint{7.877260in}{2.295941in}}%
\pgfpathcurveto{\pgfqpoint{7.882304in}{2.295941in}}{\pgfqpoint{7.887142in}{2.297945in}}{\pgfqpoint{7.890708in}{2.301511in}}%
\pgfpathcurveto{\pgfqpoint{7.894275in}{2.305078in}}{\pgfqpoint{7.896279in}{2.309916in}}{\pgfqpoint{7.896279in}{2.314959in}}%
\pgfpathcurveto{\pgfqpoint{7.896279in}{2.320003in}}{\pgfqpoint{7.894275in}{2.324841in}}{\pgfqpoint{7.890708in}{2.328407in}}%
\pgfpathcurveto{\pgfqpoint{7.887142in}{2.331973in}}{\pgfqpoint{7.882304in}{2.333977in}}{\pgfqpoint{7.877260in}{2.333977in}}%
\pgfpathcurveto{\pgfqpoint{7.872217in}{2.333977in}}{\pgfqpoint{7.867379in}{2.331973in}}{\pgfqpoint{7.863813in}{2.328407in}}%
\pgfpathcurveto{\pgfqpoint{7.860246in}{2.324841in}}{\pgfqpoint{7.858242in}{2.320003in}}{\pgfqpoint{7.858242in}{2.314959in}}%
\pgfpathcurveto{\pgfqpoint{7.858242in}{2.309916in}}{\pgfqpoint{7.860246in}{2.305078in}}{\pgfqpoint{7.863813in}{2.301511in}}%
\pgfpathcurveto{\pgfqpoint{7.867379in}{2.297945in}}{\pgfqpoint{7.872217in}{2.295941in}}{\pgfqpoint{7.877260in}{2.295941in}}%
\pgfpathclose%
\pgfusepath{fill}%
\end{pgfscope}%
\begin{pgfscope}%
\pgfpathrectangle{\pgfqpoint{6.572727in}{0.474100in}}{\pgfqpoint{4.227273in}{3.318700in}}%
\pgfusepath{clip}%
\pgfsetbuttcap%
\pgfsetroundjoin%
\definecolor{currentfill}{rgb}{0.127568,0.566949,0.550556}%
\pgfsetfillcolor{currentfill}%
\pgfsetfillopacity{0.700000}%
\pgfsetlinewidth{0.000000pt}%
\definecolor{currentstroke}{rgb}{0.000000,0.000000,0.000000}%
\pgfsetstrokecolor{currentstroke}%
\pgfsetstrokeopacity{0.700000}%
\pgfsetdash{}{0pt}%
\pgfpathmoveto{\pgfqpoint{8.057528in}{2.667054in}}%
\pgfpathcurveto{\pgfqpoint{8.062571in}{2.667054in}}{\pgfqpoint{8.067409in}{2.669058in}}{\pgfqpoint{8.070976in}{2.672624in}}%
\pgfpathcurveto{\pgfqpoint{8.074542in}{2.676190in}}{\pgfqpoint{8.076546in}{2.681028in}}{\pgfqpoint{8.076546in}{2.686072in}}%
\pgfpathcurveto{\pgfqpoint{8.076546in}{2.691116in}}{\pgfqpoint{8.074542in}{2.695953in}}{\pgfqpoint{8.070976in}{2.699520in}}%
\pgfpathcurveto{\pgfqpoint{8.067409in}{2.703086in}}{\pgfqpoint{8.062571in}{2.705090in}}{\pgfqpoint{8.057528in}{2.705090in}}%
\pgfpathcurveto{\pgfqpoint{8.052484in}{2.705090in}}{\pgfqpoint{8.047646in}{2.703086in}}{\pgfqpoint{8.044080in}{2.699520in}}%
\pgfpathcurveto{\pgfqpoint{8.040513in}{2.695953in}}{\pgfqpoint{8.038510in}{2.691116in}}{\pgfqpoint{8.038510in}{2.686072in}}%
\pgfpathcurveto{\pgfqpoint{8.038510in}{2.681028in}}{\pgfqpoint{8.040513in}{2.676190in}}{\pgfqpoint{8.044080in}{2.672624in}}%
\pgfpathcurveto{\pgfqpoint{8.047646in}{2.669058in}}{\pgfqpoint{8.052484in}{2.667054in}}{\pgfqpoint{8.057528in}{2.667054in}}%
\pgfpathclose%
\pgfusepath{fill}%
\end{pgfscope}%
\begin{pgfscope}%
\pgfpathrectangle{\pgfqpoint{6.572727in}{0.474100in}}{\pgfqpoint{4.227273in}{3.318700in}}%
\pgfusepath{clip}%
\pgfsetbuttcap%
\pgfsetroundjoin%
\definecolor{currentfill}{rgb}{0.993248,0.906157,0.143936}%
\pgfsetfillcolor{currentfill}%
\pgfsetfillopacity{0.700000}%
\pgfsetlinewidth{0.000000pt}%
\definecolor{currentstroke}{rgb}{0.000000,0.000000,0.000000}%
\pgfsetstrokecolor{currentstroke}%
\pgfsetstrokeopacity{0.700000}%
\pgfsetdash{}{0pt}%
\pgfpathmoveto{\pgfqpoint{9.430775in}{1.906199in}}%
\pgfpathcurveto{\pgfqpoint{9.435819in}{1.906199in}}{\pgfqpoint{9.440656in}{1.908203in}}{\pgfqpoint{9.444223in}{1.911769in}}%
\pgfpathcurveto{\pgfqpoint{9.447789in}{1.915336in}}{\pgfqpoint{9.449793in}{1.920173in}}{\pgfqpoint{9.449793in}{1.925217in}}%
\pgfpathcurveto{\pgfqpoint{9.449793in}{1.930261in}}{\pgfqpoint{9.447789in}{1.935099in}}{\pgfqpoint{9.444223in}{1.938665in}}%
\pgfpathcurveto{\pgfqpoint{9.440656in}{1.942231in}}{\pgfqpoint{9.435819in}{1.944235in}}{\pgfqpoint{9.430775in}{1.944235in}}%
\pgfpathcurveto{\pgfqpoint{9.425731in}{1.944235in}}{\pgfqpoint{9.420894in}{1.942231in}}{\pgfqpoint{9.417327in}{1.938665in}}%
\pgfpathcurveto{\pgfqpoint{9.413761in}{1.935099in}}{\pgfqpoint{9.411757in}{1.930261in}}{\pgfqpoint{9.411757in}{1.925217in}}%
\pgfpathcurveto{\pgfqpoint{9.411757in}{1.920173in}}{\pgfqpoint{9.413761in}{1.915336in}}{\pgfqpoint{9.417327in}{1.911769in}}%
\pgfpathcurveto{\pgfqpoint{9.420894in}{1.908203in}}{\pgfqpoint{9.425731in}{1.906199in}}{\pgfqpoint{9.430775in}{1.906199in}}%
\pgfpathclose%
\pgfusepath{fill}%
\end{pgfscope}%
\begin{pgfscope}%
\pgfpathrectangle{\pgfqpoint{6.572727in}{0.474100in}}{\pgfqpoint{4.227273in}{3.318700in}}%
\pgfusepath{clip}%
\pgfsetbuttcap%
\pgfsetroundjoin%
\definecolor{currentfill}{rgb}{0.127568,0.566949,0.550556}%
\pgfsetfillcolor{currentfill}%
\pgfsetfillopacity{0.700000}%
\pgfsetlinewidth{0.000000pt}%
\definecolor{currentstroke}{rgb}{0.000000,0.000000,0.000000}%
\pgfsetstrokecolor{currentstroke}%
\pgfsetstrokeopacity{0.700000}%
\pgfsetdash{}{0pt}%
\pgfpathmoveto{\pgfqpoint{7.415492in}{2.120343in}}%
\pgfpathcurveto{\pgfqpoint{7.420535in}{2.120343in}}{\pgfqpoint{7.425373in}{2.122347in}}{\pgfqpoint{7.428940in}{2.125913in}}%
\pgfpathcurveto{\pgfqpoint{7.432506in}{2.129480in}}{\pgfqpoint{7.434510in}{2.134317in}}{\pgfqpoint{7.434510in}{2.139361in}}%
\pgfpathcurveto{\pgfqpoint{7.434510in}{2.144405in}}{\pgfqpoint{7.432506in}{2.149242in}}{\pgfqpoint{7.428940in}{2.152809in}}%
\pgfpathcurveto{\pgfqpoint{7.425373in}{2.156375in}}{\pgfqpoint{7.420535in}{2.158379in}}{\pgfqpoint{7.415492in}{2.158379in}}%
\pgfpathcurveto{\pgfqpoint{7.410448in}{2.158379in}}{\pgfqpoint{7.405610in}{2.156375in}}{\pgfqpoint{7.402044in}{2.152809in}}%
\pgfpathcurveto{\pgfqpoint{7.398477in}{2.149242in}}{\pgfqpoint{7.396474in}{2.144405in}}{\pgfqpoint{7.396474in}{2.139361in}}%
\pgfpathcurveto{\pgfqpoint{7.396474in}{2.134317in}}{\pgfqpoint{7.398477in}{2.129480in}}{\pgfqpoint{7.402044in}{2.125913in}}%
\pgfpathcurveto{\pgfqpoint{7.405610in}{2.122347in}}{\pgfqpoint{7.410448in}{2.120343in}}{\pgfqpoint{7.415492in}{2.120343in}}%
\pgfpathclose%
\pgfusepath{fill}%
\end{pgfscope}%
\begin{pgfscope}%
\pgfpathrectangle{\pgfqpoint{6.572727in}{0.474100in}}{\pgfqpoint{4.227273in}{3.318700in}}%
\pgfusepath{clip}%
\pgfsetbuttcap%
\pgfsetroundjoin%
\definecolor{currentfill}{rgb}{0.127568,0.566949,0.550556}%
\pgfsetfillcolor{currentfill}%
\pgfsetfillopacity{0.700000}%
\pgfsetlinewidth{0.000000pt}%
\definecolor{currentstroke}{rgb}{0.000000,0.000000,0.000000}%
\pgfsetstrokecolor{currentstroke}%
\pgfsetstrokeopacity{0.700000}%
\pgfsetdash{}{0pt}%
\pgfpathmoveto{\pgfqpoint{8.174750in}{2.681549in}}%
\pgfpathcurveto{\pgfqpoint{8.179794in}{2.681549in}}{\pgfqpoint{8.184632in}{2.683552in}}{\pgfqpoint{8.188198in}{2.687119in}}%
\pgfpathcurveto{\pgfqpoint{8.191764in}{2.690685in}}{\pgfqpoint{8.193768in}{2.695523in}}{\pgfqpoint{8.193768in}{2.700567in}}%
\pgfpathcurveto{\pgfqpoint{8.193768in}{2.705610in}}{\pgfqpoint{8.191764in}{2.710448in}}{\pgfqpoint{8.188198in}{2.714015in}}%
\pgfpathcurveto{\pgfqpoint{8.184632in}{2.717581in}}{\pgfqpoint{8.179794in}{2.719585in}}{\pgfqpoint{8.174750in}{2.719585in}}%
\pgfpathcurveto{\pgfqpoint{8.169706in}{2.719585in}}{\pgfqpoint{8.164869in}{2.717581in}}{\pgfqpoint{8.161302in}{2.714015in}}%
\pgfpathcurveto{\pgfqpoint{8.157736in}{2.710448in}}{\pgfqpoint{8.155732in}{2.705610in}}{\pgfqpoint{8.155732in}{2.700567in}}%
\pgfpathcurveto{\pgfqpoint{8.155732in}{2.695523in}}{\pgfqpoint{8.157736in}{2.690685in}}{\pgfqpoint{8.161302in}{2.687119in}}%
\pgfpathcurveto{\pgfqpoint{8.164869in}{2.683552in}}{\pgfqpoint{8.169706in}{2.681549in}}{\pgfqpoint{8.174750in}{2.681549in}}%
\pgfpathclose%
\pgfusepath{fill}%
\end{pgfscope}%
\begin{pgfscope}%
\pgfpathrectangle{\pgfqpoint{6.572727in}{0.474100in}}{\pgfqpoint{4.227273in}{3.318700in}}%
\pgfusepath{clip}%
\pgfsetbuttcap%
\pgfsetroundjoin%
\definecolor{currentfill}{rgb}{0.127568,0.566949,0.550556}%
\pgfsetfillcolor{currentfill}%
\pgfsetfillopacity{0.700000}%
\pgfsetlinewidth{0.000000pt}%
\definecolor{currentstroke}{rgb}{0.000000,0.000000,0.000000}%
\pgfsetstrokecolor{currentstroke}%
\pgfsetstrokeopacity{0.700000}%
\pgfsetdash{}{0pt}%
\pgfpathmoveto{\pgfqpoint{7.873512in}{1.432447in}}%
\pgfpathcurveto{\pgfqpoint{7.878555in}{1.432447in}}{\pgfqpoint{7.883393in}{1.434451in}}{\pgfqpoint{7.886959in}{1.438017in}}%
\pgfpathcurveto{\pgfqpoint{7.890526in}{1.441583in}}{\pgfqpoint{7.892530in}{1.446421in}}{\pgfqpoint{7.892530in}{1.451465in}}%
\pgfpathcurveto{\pgfqpoint{7.892530in}{1.456509in}}{\pgfqpoint{7.890526in}{1.461346in}}{\pgfqpoint{7.886959in}{1.464913in}}%
\pgfpathcurveto{\pgfqpoint{7.883393in}{1.468479in}}{\pgfqpoint{7.878555in}{1.470483in}}{\pgfqpoint{7.873512in}{1.470483in}}%
\pgfpathcurveto{\pgfqpoint{7.868468in}{1.470483in}}{\pgfqpoint{7.863630in}{1.468479in}}{\pgfqpoint{7.860064in}{1.464913in}}%
\pgfpathcurveto{\pgfqpoint{7.856497in}{1.461346in}}{\pgfqpoint{7.854493in}{1.456509in}}{\pgfqpoint{7.854493in}{1.451465in}}%
\pgfpathcurveto{\pgfqpoint{7.854493in}{1.446421in}}{\pgfqpoint{7.856497in}{1.441583in}}{\pgfqpoint{7.860064in}{1.438017in}}%
\pgfpathcurveto{\pgfqpoint{7.863630in}{1.434451in}}{\pgfqpoint{7.868468in}{1.432447in}}{\pgfqpoint{7.873512in}{1.432447in}}%
\pgfpathclose%
\pgfusepath{fill}%
\end{pgfscope}%
\begin{pgfscope}%
\pgfpathrectangle{\pgfqpoint{6.572727in}{0.474100in}}{\pgfqpoint{4.227273in}{3.318700in}}%
\pgfusepath{clip}%
\pgfsetbuttcap%
\pgfsetroundjoin%
\definecolor{currentfill}{rgb}{0.127568,0.566949,0.550556}%
\pgfsetfillcolor{currentfill}%
\pgfsetfillopacity{0.700000}%
\pgfsetlinewidth{0.000000pt}%
\definecolor{currentstroke}{rgb}{0.000000,0.000000,0.000000}%
\pgfsetstrokecolor{currentstroke}%
\pgfsetstrokeopacity{0.700000}%
\pgfsetdash{}{0pt}%
\pgfpathmoveto{\pgfqpoint{8.491858in}{2.950591in}}%
\pgfpathcurveto{\pgfqpoint{8.496901in}{2.950591in}}{\pgfqpoint{8.501739in}{2.952594in}}{\pgfqpoint{8.505306in}{2.956161in}}%
\pgfpathcurveto{\pgfqpoint{8.508872in}{2.959727in}}{\pgfqpoint{8.510876in}{2.964565in}}{\pgfqpoint{8.510876in}{2.969609in}}%
\pgfpathcurveto{\pgfqpoint{8.510876in}{2.974652in}}{\pgfqpoint{8.508872in}{2.979490in}}{\pgfqpoint{8.505306in}{2.983057in}}%
\pgfpathcurveto{\pgfqpoint{8.501739in}{2.986623in}}{\pgfqpoint{8.496901in}{2.988627in}}{\pgfqpoint{8.491858in}{2.988627in}}%
\pgfpathcurveto{\pgfqpoint{8.486814in}{2.988627in}}{\pgfqpoint{8.481976in}{2.986623in}}{\pgfqpoint{8.478410in}{2.983057in}}%
\pgfpathcurveto{\pgfqpoint{8.474843in}{2.979490in}}{\pgfqpoint{8.472840in}{2.974652in}}{\pgfqpoint{8.472840in}{2.969609in}}%
\pgfpathcurveto{\pgfqpoint{8.472840in}{2.964565in}}{\pgfqpoint{8.474843in}{2.959727in}}{\pgfqpoint{8.478410in}{2.956161in}}%
\pgfpathcurveto{\pgfqpoint{8.481976in}{2.952594in}}{\pgfqpoint{8.486814in}{2.950591in}}{\pgfqpoint{8.491858in}{2.950591in}}%
\pgfpathclose%
\pgfusepath{fill}%
\end{pgfscope}%
\begin{pgfscope}%
\pgfpathrectangle{\pgfqpoint{6.572727in}{0.474100in}}{\pgfqpoint{4.227273in}{3.318700in}}%
\pgfusepath{clip}%
\pgfsetbuttcap%
\pgfsetroundjoin%
\definecolor{currentfill}{rgb}{0.127568,0.566949,0.550556}%
\pgfsetfillcolor{currentfill}%
\pgfsetfillopacity{0.700000}%
\pgfsetlinewidth{0.000000pt}%
\definecolor{currentstroke}{rgb}{0.000000,0.000000,0.000000}%
\pgfsetstrokecolor{currentstroke}%
\pgfsetstrokeopacity{0.700000}%
\pgfsetdash{}{0pt}%
\pgfpathmoveto{\pgfqpoint{7.709426in}{1.329065in}}%
\pgfpathcurveto{\pgfqpoint{7.714470in}{1.329065in}}{\pgfqpoint{7.719308in}{1.331069in}}{\pgfqpoint{7.722874in}{1.334636in}}%
\pgfpathcurveto{\pgfqpoint{7.726440in}{1.338202in}}{\pgfqpoint{7.728444in}{1.343040in}}{\pgfqpoint{7.728444in}{1.348083in}}%
\pgfpathcurveto{\pgfqpoint{7.728444in}{1.353127in}}{\pgfqpoint{7.726440in}{1.357965in}}{\pgfqpoint{7.722874in}{1.361531in}}%
\pgfpathcurveto{\pgfqpoint{7.719308in}{1.365098in}}{\pgfqpoint{7.714470in}{1.367102in}}{\pgfqpoint{7.709426in}{1.367102in}}%
\pgfpathcurveto{\pgfqpoint{7.704382in}{1.367102in}}{\pgfqpoint{7.699545in}{1.365098in}}{\pgfqpoint{7.695978in}{1.361531in}}%
\pgfpathcurveto{\pgfqpoint{7.692412in}{1.357965in}}{\pgfqpoint{7.690408in}{1.353127in}}{\pgfqpoint{7.690408in}{1.348083in}}%
\pgfpathcurveto{\pgfqpoint{7.690408in}{1.343040in}}{\pgfqpoint{7.692412in}{1.338202in}}{\pgfqpoint{7.695978in}{1.334636in}}%
\pgfpathcurveto{\pgfqpoint{7.699545in}{1.331069in}}{\pgfqpoint{7.704382in}{1.329065in}}{\pgfqpoint{7.709426in}{1.329065in}}%
\pgfpathclose%
\pgfusepath{fill}%
\end{pgfscope}%
\begin{pgfscope}%
\pgfpathrectangle{\pgfqpoint{6.572727in}{0.474100in}}{\pgfqpoint{4.227273in}{3.318700in}}%
\pgfusepath{clip}%
\pgfsetbuttcap%
\pgfsetroundjoin%
\definecolor{currentfill}{rgb}{0.127568,0.566949,0.550556}%
\pgfsetfillcolor{currentfill}%
\pgfsetfillopacity{0.700000}%
\pgfsetlinewidth{0.000000pt}%
\definecolor{currentstroke}{rgb}{0.000000,0.000000,0.000000}%
\pgfsetstrokecolor{currentstroke}%
\pgfsetstrokeopacity{0.700000}%
\pgfsetdash{}{0pt}%
\pgfpathmoveto{\pgfqpoint{8.343571in}{1.163724in}}%
\pgfpathcurveto{\pgfqpoint{8.348614in}{1.163724in}}{\pgfqpoint{8.353452in}{1.165728in}}{\pgfqpoint{8.357018in}{1.169294in}}%
\pgfpathcurveto{\pgfqpoint{8.360585in}{1.172861in}}{\pgfqpoint{8.362589in}{1.177698in}}{\pgfqpoint{8.362589in}{1.182742in}}%
\pgfpathcurveto{\pgfqpoint{8.362589in}{1.187786in}}{\pgfqpoint{8.360585in}{1.192624in}}{\pgfqpoint{8.357018in}{1.196190in}}%
\pgfpathcurveto{\pgfqpoint{8.353452in}{1.199756in}}{\pgfqpoint{8.348614in}{1.201760in}}{\pgfqpoint{8.343571in}{1.201760in}}%
\pgfpathcurveto{\pgfqpoint{8.338527in}{1.201760in}}{\pgfqpoint{8.333689in}{1.199756in}}{\pgfqpoint{8.330123in}{1.196190in}}%
\pgfpathcurveto{\pgfqpoint{8.326556in}{1.192624in}}{\pgfqpoint{8.324552in}{1.187786in}}{\pgfqpoint{8.324552in}{1.182742in}}%
\pgfpathcurveto{\pgfqpoint{8.324552in}{1.177698in}}{\pgfqpoint{8.326556in}{1.172861in}}{\pgfqpoint{8.330123in}{1.169294in}}%
\pgfpathcurveto{\pgfqpoint{8.333689in}{1.165728in}}{\pgfqpoint{8.338527in}{1.163724in}}{\pgfqpoint{8.343571in}{1.163724in}}%
\pgfpathclose%
\pgfusepath{fill}%
\end{pgfscope}%
\begin{pgfscope}%
\pgfpathrectangle{\pgfqpoint{6.572727in}{0.474100in}}{\pgfqpoint{4.227273in}{3.318700in}}%
\pgfusepath{clip}%
\pgfsetbuttcap%
\pgfsetroundjoin%
\definecolor{currentfill}{rgb}{0.127568,0.566949,0.550556}%
\pgfsetfillcolor{currentfill}%
\pgfsetfillopacity{0.700000}%
\pgfsetlinewidth{0.000000pt}%
\definecolor{currentstroke}{rgb}{0.000000,0.000000,0.000000}%
\pgfsetstrokecolor{currentstroke}%
\pgfsetstrokeopacity{0.700000}%
\pgfsetdash{}{0pt}%
\pgfpathmoveto{\pgfqpoint{7.792378in}{1.357340in}}%
\pgfpathcurveto{\pgfqpoint{7.797421in}{1.357340in}}{\pgfqpoint{7.802259in}{1.359344in}}{\pgfqpoint{7.805825in}{1.362910in}}%
\pgfpathcurveto{\pgfqpoint{7.809392in}{1.366477in}}{\pgfqpoint{7.811396in}{1.371314in}}{\pgfqpoint{7.811396in}{1.376358in}}%
\pgfpathcurveto{\pgfqpoint{7.811396in}{1.381402in}}{\pgfqpoint{7.809392in}{1.386239in}}{\pgfqpoint{7.805825in}{1.389806in}}%
\pgfpathcurveto{\pgfqpoint{7.802259in}{1.393372in}}{\pgfqpoint{7.797421in}{1.395376in}}{\pgfqpoint{7.792378in}{1.395376in}}%
\pgfpathcurveto{\pgfqpoint{7.787334in}{1.395376in}}{\pgfqpoint{7.782496in}{1.393372in}}{\pgfqpoint{7.778930in}{1.389806in}}%
\pgfpathcurveto{\pgfqpoint{7.775363in}{1.386239in}}{\pgfqpoint{7.773359in}{1.381402in}}{\pgfqpoint{7.773359in}{1.376358in}}%
\pgfpathcurveto{\pgfqpoint{7.773359in}{1.371314in}}{\pgfqpoint{7.775363in}{1.366477in}}{\pgfqpoint{7.778930in}{1.362910in}}%
\pgfpathcurveto{\pgfqpoint{7.782496in}{1.359344in}}{\pgfqpoint{7.787334in}{1.357340in}}{\pgfqpoint{7.792378in}{1.357340in}}%
\pgfpathclose%
\pgfusepath{fill}%
\end{pgfscope}%
\begin{pgfscope}%
\pgfpathrectangle{\pgfqpoint{6.572727in}{0.474100in}}{\pgfqpoint{4.227273in}{3.318700in}}%
\pgfusepath{clip}%
\pgfsetbuttcap%
\pgfsetroundjoin%
\definecolor{currentfill}{rgb}{0.127568,0.566949,0.550556}%
\pgfsetfillcolor{currentfill}%
\pgfsetfillopacity{0.700000}%
\pgfsetlinewidth{0.000000pt}%
\definecolor{currentstroke}{rgb}{0.000000,0.000000,0.000000}%
\pgfsetstrokecolor{currentstroke}%
\pgfsetstrokeopacity{0.700000}%
\pgfsetdash{}{0pt}%
\pgfpathmoveto{\pgfqpoint{7.940147in}{1.639148in}}%
\pgfpathcurveto{\pgfqpoint{7.945191in}{1.639148in}}{\pgfqpoint{7.950028in}{1.641151in}}{\pgfqpoint{7.953595in}{1.644718in}}%
\pgfpathcurveto{\pgfqpoint{7.957161in}{1.648284in}}{\pgfqpoint{7.959165in}{1.653122in}}{\pgfqpoint{7.959165in}{1.658166in}}%
\pgfpathcurveto{\pgfqpoint{7.959165in}{1.663209in}}{\pgfqpoint{7.957161in}{1.668047in}}{\pgfqpoint{7.953595in}{1.671614in}}%
\pgfpathcurveto{\pgfqpoint{7.950028in}{1.675180in}}{\pgfqpoint{7.945191in}{1.677184in}}{\pgfqpoint{7.940147in}{1.677184in}}%
\pgfpathcurveto{\pgfqpoint{7.935103in}{1.677184in}}{\pgfqpoint{7.930265in}{1.675180in}}{\pgfqpoint{7.926699in}{1.671614in}}%
\pgfpathcurveto{\pgfqpoint{7.923133in}{1.668047in}}{\pgfqpoint{7.921129in}{1.663209in}}{\pgfqpoint{7.921129in}{1.658166in}}%
\pgfpathcurveto{\pgfqpoint{7.921129in}{1.653122in}}{\pgfqpoint{7.923133in}{1.648284in}}{\pgfqpoint{7.926699in}{1.644718in}}%
\pgfpathcurveto{\pgfqpoint{7.930265in}{1.641151in}}{\pgfqpoint{7.935103in}{1.639148in}}{\pgfqpoint{7.940147in}{1.639148in}}%
\pgfpathclose%
\pgfusepath{fill}%
\end{pgfscope}%
\begin{pgfscope}%
\pgfpathrectangle{\pgfqpoint{6.572727in}{0.474100in}}{\pgfqpoint{4.227273in}{3.318700in}}%
\pgfusepath{clip}%
\pgfsetbuttcap%
\pgfsetroundjoin%
\definecolor{currentfill}{rgb}{0.127568,0.566949,0.550556}%
\pgfsetfillcolor{currentfill}%
\pgfsetfillopacity{0.700000}%
\pgfsetlinewidth{0.000000pt}%
\definecolor{currentstroke}{rgb}{0.000000,0.000000,0.000000}%
\pgfsetstrokecolor{currentstroke}%
\pgfsetstrokeopacity{0.700000}%
\pgfsetdash{}{0pt}%
\pgfpathmoveto{\pgfqpoint{8.380058in}{1.280001in}}%
\pgfpathcurveto{\pgfqpoint{8.385102in}{1.280001in}}{\pgfqpoint{8.389940in}{1.282005in}}{\pgfqpoint{8.393506in}{1.285572in}}%
\pgfpathcurveto{\pgfqpoint{8.397073in}{1.289138in}}{\pgfqpoint{8.399077in}{1.293976in}}{\pgfqpoint{8.399077in}{1.299019in}}%
\pgfpathcurveto{\pgfqpoint{8.399077in}{1.304063in}}{\pgfqpoint{8.397073in}{1.308901in}}{\pgfqpoint{8.393506in}{1.312467in}}%
\pgfpathcurveto{\pgfqpoint{8.389940in}{1.316034in}}{\pgfqpoint{8.385102in}{1.318038in}}{\pgfqpoint{8.380058in}{1.318038in}}%
\pgfpathcurveto{\pgfqpoint{8.375015in}{1.318038in}}{\pgfqpoint{8.370177in}{1.316034in}}{\pgfqpoint{8.366611in}{1.312467in}}%
\pgfpathcurveto{\pgfqpoint{8.363044in}{1.308901in}}{\pgfqpoint{8.361040in}{1.304063in}}{\pgfqpoint{8.361040in}{1.299019in}}%
\pgfpathcurveto{\pgfqpoint{8.361040in}{1.293976in}}{\pgfqpoint{8.363044in}{1.289138in}}{\pgfqpoint{8.366611in}{1.285572in}}%
\pgfpathcurveto{\pgfqpoint{8.370177in}{1.282005in}}{\pgfqpoint{8.375015in}{1.280001in}}{\pgfqpoint{8.380058in}{1.280001in}}%
\pgfpathclose%
\pgfusepath{fill}%
\end{pgfscope}%
\begin{pgfscope}%
\pgfpathrectangle{\pgfqpoint{6.572727in}{0.474100in}}{\pgfqpoint{4.227273in}{3.318700in}}%
\pgfusepath{clip}%
\pgfsetbuttcap%
\pgfsetroundjoin%
\definecolor{currentfill}{rgb}{0.993248,0.906157,0.143936}%
\pgfsetfillcolor{currentfill}%
\pgfsetfillopacity{0.700000}%
\pgfsetlinewidth{0.000000pt}%
\definecolor{currentstroke}{rgb}{0.000000,0.000000,0.000000}%
\pgfsetstrokecolor{currentstroke}%
\pgfsetstrokeopacity{0.700000}%
\pgfsetdash{}{0pt}%
\pgfpathmoveto{\pgfqpoint{9.698185in}{1.945955in}}%
\pgfpathcurveto{\pgfqpoint{9.703229in}{1.945955in}}{\pgfqpoint{9.708067in}{1.947959in}}{\pgfqpoint{9.711633in}{1.951525in}}%
\pgfpathcurveto{\pgfqpoint{9.715199in}{1.955092in}}{\pgfqpoint{9.717203in}{1.959929in}}{\pgfqpoint{9.717203in}{1.964973in}}%
\pgfpathcurveto{\pgfqpoint{9.717203in}{1.970017in}}{\pgfqpoint{9.715199in}{1.974855in}}{\pgfqpoint{9.711633in}{1.978421in}}%
\pgfpathcurveto{\pgfqpoint{9.708067in}{1.981987in}}{\pgfqpoint{9.703229in}{1.983991in}}{\pgfqpoint{9.698185in}{1.983991in}}%
\pgfpathcurveto{\pgfqpoint{9.693141in}{1.983991in}}{\pgfqpoint{9.688304in}{1.981987in}}{\pgfqpoint{9.684737in}{1.978421in}}%
\pgfpathcurveto{\pgfqpoint{9.681171in}{1.974855in}}{\pgfqpoint{9.679167in}{1.970017in}}{\pgfqpoint{9.679167in}{1.964973in}}%
\pgfpathcurveto{\pgfqpoint{9.679167in}{1.959929in}}{\pgfqpoint{9.681171in}{1.955092in}}{\pgfqpoint{9.684737in}{1.951525in}}%
\pgfpathcurveto{\pgfqpoint{9.688304in}{1.947959in}}{\pgfqpoint{9.693141in}{1.945955in}}{\pgfqpoint{9.698185in}{1.945955in}}%
\pgfpathclose%
\pgfusepath{fill}%
\end{pgfscope}%
\begin{pgfscope}%
\pgfpathrectangle{\pgfqpoint{6.572727in}{0.474100in}}{\pgfqpoint{4.227273in}{3.318700in}}%
\pgfusepath{clip}%
\pgfsetbuttcap%
\pgfsetroundjoin%
\definecolor{currentfill}{rgb}{0.993248,0.906157,0.143936}%
\pgfsetfillcolor{currentfill}%
\pgfsetfillopacity{0.700000}%
\pgfsetlinewidth{0.000000pt}%
\definecolor{currentstroke}{rgb}{0.000000,0.000000,0.000000}%
\pgfsetstrokecolor{currentstroke}%
\pgfsetstrokeopacity{0.700000}%
\pgfsetdash{}{0pt}%
\pgfpathmoveto{\pgfqpoint{9.397700in}{1.280531in}}%
\pgfpathcurveto{\pgfqpoint{9.402744in}{1.280531in}}{\pgfqpoint{9.407582in}{1.282535in}}{\pgfqpoint{9.411148in}{1.286102in}}%
\pgfpathcurveto{\pgfqpoint{9.414715in}{1.289668in}}{\pgfqpoint{9.416719in}{1.294506in}}{\pgfqpoint{9.416719in}{1.299550in}}%
\pgfpathcurveto{\pgfqpoint{9.416719in}{1.304593in}}{\pgfqpoint{9.414715in}{1.309431in}}{\pgfqpoint{9.411148in}{1.312997in}}%
\pgfpathcurveto{\pgfqpoint{9.407582in}{1.316564in}}{\pgfqpoint{9.402744in}{1.318568in}}{\pgfqpoint{9.397700in}{1.318568in}}%
\pgfpathcurveto{\pgfqpoint{9.392657in}{1.318568in}}{\pgfqpoint{9.387819in}{1.316564in}}{\pgfqpoint{9.384253in}{1.312997in}}%
\pgfpathcurveto{\pgfqpoint{9.380686in}{1.309431in}}{\pgfqpoint{9.378682in}{1.304593in}}{\pgfqpoint{9.378682in}{1.299550in}}%
\pgfpathcurveto{\pgfqpoint{9.378682in}{1.294506in}}{\pgfqpoint{9.380686in}{1.289668in}}{\pgfqpoint{9.384253in}{1.286102in}}%
\pgfpathcurveto{\pgfqpoint{9.387819in}{1.282535in}}{\pgfqpoint{9.392657in}{1.280531in}}{\pgfqpoint{9.397700in}{1.280531in}}%
\pgfpathclose%
\pgfusepath{fill}%
\end{pgfscope}%
\begin{pgfscope}%
\pgfpathrectangle{\pgfqpoint{6.572727in}{0.474100in}}{\pgfqpoint{4.227273in}{3.318700in}}%
\pgfusepath{clip}%
\pgfsetbuttcap%
\pgfsetroundjoin%
\definecolor{currentfill}{rgb}{0.127568,0.566949,0.550556}%
\pgfsetfillcolor{currentfill}%
\pgfsetfillopacity{0.700000}%
\pgfsetlinewidth{0.000000pt}%
\definecolor{currentstroke}{rgb}{0.000000,0.000000,0.000000}%
\pgfsetstrokecolor{currentstroke}%
\pgfsetstrokeopacity{0.700000}%
\pgfsetdash{}{0pt}%
\pgfpathmoveto{\pgfqpoint{8.575376in}{1.526192in}}%
\pgfpathcurveto{\pgfqpoint{8.580420in}{1.526192in}}{\pgfqpoint{8.585258in}{1.528196in}}{\pgfqpoint{8.588824in}{1.531762in}}%
\pgfpathcurveto{\pgfqpoint{8.592390in}{1.535328in}}{\pgfqpoint{8.594394in}{1.540166in}}{\pgfqpoint{8.594394in}{1.545210in}}%
\pgfpathcurveto{\pgfqpoint{8.594394in}{1.550254in}}{\pgfqpoint{8.592390in}{1.555091in}}{\pgfqpoint{8.588824in}{1.558658in}}%
\pgfpathcurveto{\pgfqpoint{8.585258in}{1.562224in}}{\pgfqpoint{8.580420in}{1.564228in}}{\pgfqpoint{8.575376in}{1.564228in}}%
\pgfpathcurveto{\pgfqpoint{8.570333in}{1.564228in}}{\pgfqpoint{8.565495in}{1.562224in}}{\pgfqpoint{8.561928in}{1.558658in}}%
\pgfpathcurveto{\pgfqpoint{8.558362in}{1.555091in}}{\pgfqpoint{8.556358in}{1.550254in}}{\pgfqpoint{8.556358in}{1.545210in}}%
\pgfpathcurveto{\pgfqpoint{8.556358in}{1.540166in}}{\pgfqpoint{8.558362in}{1.535328in}}{\pgfqpoint{8.561928in}{1.531762in}}%
\pgfpathcurveto{\pgfqpoint{8.565495in}{1.528196in}}{\pgfqpoint{8.570333in}{1.526192in}}{\pgfqpoint{8.575376in}{1.526192in}}%
\pgfpathclose%
\pgfusepath{fill}%
\end{pgfscope}%
\begin{pgfscope}%
\pgfpathrectangle{\pgfqpoint{6.572727in}{0.474100in}}{\pgfqpoint{4.227273in}{3.318700in}}%
\pgfusepath{clip}%
\pgfsetbuttcap%
\pgfsetroundjoin%
\definecolor{currentfill}{rgb}{0.127568,0.566949,0.550556}%
\pgfsetfillcolor{currentfill}%
\pgfsetfillopacity{0.700000}%
\pgfsetlinewidth{0.000000pt}%
\definecolor{currentstroke}{rgb}{0.000000,0.000000,0.000000}%
\pgfsetstrokecolor{currentstroke}%
\pgfsetstrokeopacity{0.700000}%
\pgfsetdash{}{0pt}%
\pgfpathmoveto{\pgfqpoint{7.961889in}{2.994783in}}%
\pgfpathcurveto{\pgfqpoint{7.966932in}{2.994783in}}{\pgfqpoint{7.971770in}{2.996787in}}{\pgfqpoint{7.975336in}{3.000354in}}%
\pgfpathcurveto{\pgfqpoint{7.978903in}{3.003920in}}{\pgfqpoint{7.980907in}{3.008758in}}{\pgfqpoint{7.980907in}{3.013801in}}%
\pgfpathcurveto{\pgfqpoint{7.980907in}{3.018845in}}{\pgfqpoint{7.978903in}{3.023683in}}{\pgfqpoint{7.975336in}{3.027249in}}%
\pgfpathcurveto{\pgfqpoint{7.971770in}{3.030816in}}{\pgfqpoint{7.966932in}{3.032820in}}{\pgfqpoint{7.961889in}{3.032820in}}%
\pgfpathcurveto{\pgfqpoint{7.956845in}{3.032820in}}{\pgfqpoint{7.952007in}{3.030816in}}{\pgfqpoint{7.948441in}{3.027249in}}%
\pgfpathcurveto{\pgfqpoint{7.944874in}{3.023683in}}{\pgfqpoint{7.942870in}{3.018845in}}{\pgfqpoint{7.942870in}{3.013801in}}%
\pgfpathcurveto{\pgfqpoint{7.942870in}{3.008758in}}{\pgfqpoint{7.944874in}{3.003920in}}{\pgfqpoint{7.948441in}{3.000354in}}%
\pgfpathcurveto{\pgfqpoint{7.952007in}{2.996787in}}{\pgfqpoint{7.956845in}{2.994783in}}{\pgfqpoint{7.961889in}{2.994783in}}%
\pgfpathclose%
\pgfusepath{fill}%
\end{pgfscope}%
\begin{pgfscope}%
\pgfpathrectangle{\pgfqpoint{6.572727in}{0.474100in}}{\pgfqpoint{4.227273in}{3.318700in}}%
\pgfusepath{clip}%
\pgfsetbuttcap%
\pgfsetroundjoin%
\definecolor{currentfill}{rgb}{0.993248,0.906157,0.143936}%
\pgfsetfillcolor{currentfill}%
\pgfsetfillopacity{0.700000}%
\pgfsetlinewidth{0.000000pt}%
\definecolor{currentstroke}{rgb}{0.000000,0.000000,0.000000}%
\pgfsetstrokecolor{currentstroke}%
\pgfsetstrokeopacity{0.700000}%
\pgfsetdash{}{0pt}%
\pgfpathmoveto{\pgfqpoint{9.692532in}{2.127383in}}%
\pgfpathcurveto{\pgfqpoint{9.697576in}{2.127383in}}{\pgfqpoint{9.702413in}{2.129387in}}{\pgfqpoint{9.705980in}{2.132953in}}%
\pgfpathcurveto{\pgfqpoint{9.709546in}{2.136519in}}{\pgfqpoint{9.711550in}{2.141357in}}{\pgfqpoint{9.711550in}{2.146401in}}%
\pgfpathcurveto{\pgfqpoint{9.711550in}{2.151444in}}{\pgfqpoint{9.709546in}{2.156282in}}{\pgfqpoint{9.705980in}{2.159849in}}%
\pgfpathcurveto{\pgfqpoint{9.702413in}{2.163415in}}{\pgfqpoint{9.697576in}{2.165419in}}{\pgfqpoint{9.692532in}{2.165419in}}%
\pgfpathcurveto{\pgfqpoint{9.687488in}{2.165419in}}{\pgfqpoint{9.682651in}{2.163415in}}{\pgfqpoint{9.679084in}{2.159849in}}%
\pgfpathcurveto{\pgfqpoint{9.675518in}{2.156282in}}{\pgfqpoint{9.673514in}{2.151444in}}{\pgfqpoint{9.673514in}{2.146401in}}%
\pgfpathcurveto{\pgfqpoint{9.673514in}{2.141357in}}{\pgfqpoint{9.675518in}{2.136519in}}{\pgfqpoint{9.679084in}{2.132953in}}%
\pgfpathcurveto{\pgfqpoint{9.682651in}{2.129387in}}{\pgfqpoint{9.687488in}{2.127383in}}{\pgfqpoint{9.692532in}{2.127383in}}%
\pgfpathclose%
\pgfusepath{fill}%
\end{pgfscope}%
\begin{pgfscope}%
\pgfpathrectangle{\pgfqpoint{6.572727in}{0.474100in}}{\pgfqpoint{4.227273in}{3.318700in}}%
\pgfusepath{clip}%
\pgfsetbuttcap%
\pgfsetroundjoin%
\definecolor{currentfill}{rgb}{0.127568,0.566949,0.550556}%
\pgfsetfillcolor{currentfill}%
\pgfsetfillopacity{0.700000}%
\pgfsetlinewidth{0.000000pt}%
\definecolor{currentstroke}{rgb}{0.000000,0.000000,0.000000}%
\pgfsetstrokecolor{currentstroke}%
\pgfsetstrokeopacity{0.700000}%
\pgfsetdash{}{0pt}%
\pgfpathmoveto{\pgfqpoint{8.102790in}{1.286592in}}%
\pgfpathcurveto{\pgfqpoint{8.107833in}{1.286592in}}{\pgfqpoint{8.112671in}{1.288596in}}{\pgfqpoint{8.116238in}{1.292163in}}%
\pgfpathcurveto{\pgfqpoint{8.119804in}{1.295729in}}{\pgfqpoint{8.121808in}{1.300567in}}{\pgfqpoint{8.121808in}{1.305611in}}%
\pgfpathcurveto{\pgfqpoint{8.121808in}{1.310654in}}{\pgfqpoint{8.119804in}{1.315492in}}{\pgfqpoint{8.116238in}{1.319058in}}%
\pgfpathcurveto{\pgfqpoint{8.112671in}{1.322625in}}{\pgfqpoint{8.107833in}{1.324629in}}{\pgfqpoint{8.102790in}{1.324629in}}%
\pgfpathcurveto{\pgfqpoint{8.097746in}{1.324629in}}{\pgfqpoint{8.092908in}{1.322625in}}{\pgfqpoint{8.089342in}{1.319058in}}%
\pgfpathcurveto{\pgfqpoint{8.085775in}{1.315492in}}{\pgfqpoint{8.083772in}{1.310654in}}{\pgfqpoint{8.083772in}{1.305611in}}%
\pgfpathcurveto{\pgfqpoint{8.083772in}{1.300567in}}{\pgfqpoint{8.085775in}{1.295729in}}{\pgfqpoint{8.089342in}{1.292163in}}%
\pgfpathcurveto{\pgfqpoint{8.092908in}{1.288596in}}{\pgfqpoint{8.097746in}{1.286592in}}{\pgfqpoint{8.102790in}{1.286592in}}%
\pgfpathclose%
\pgfusepath{fill}%
\end{pgfscope}%
\begin{pgfscope}%
\pgfpathrectangle{\pgfqpoint{6.572727in}{0.474100in}}{\pgfqpoint{4.227273in}{3.318700in}}%
\pgfusepath{clip}%
\pgfsetbuttcap%
\pgfsetroundjoin%
\definecolor{currentfill}{rgb}{0.127568,0.566949,0.550556}%
\pgfsetfillcolor{currentfill}%
\pgfsetfillopacity{0.700000}%
\pgfsetlinewidth{0.000000pt}%
\definecolor{currentstroke}{rgb}{0.000000,0.000000,0.000000}%
\pgfsetstrokecolor{currentstroke}%
\pgfsetstrokeopacity{0.700000}%
\pgfsetdash{}{0pt}%
\pgfpathmoveto{\pgfqpoint{7.747129in}{1.225143in}}%
\pgfpathcurveto{\pgfqpoint{7.752173in}{1.225143in}}{\pgfqpoint{7.757010in}{1.227147in}}{\pgfqpoint{7.760577in}{1.230713in}}%
\pgfpathcurveto{\pgfqpoint{7.764143in}{1.234280in}}{\pgfqpoint{7.766147in}{1.239118in}}{\pgfqpoint{7.766147in}{1.244161in}}%
\pgfpathcurveto{\pgfqpoint{7.766147in}{1.249205in}}{\pgfqpoint{7.764143in}{1.254043in}}{\pgfqpoint{7.760577in}{1.257609in}}%
\pgfpathcurveto{\pgfqpoint{7.757010in}{1.261176in}}{\pgfqpoint{7.752173in}{1.263179in}}{\pgfqpoint{7.747129in}{1.263179in}}%
\pgfpathcurveto{\pgfqpoint{7.742085in}{1.263179in}}{\pgfqpoint{7.737248in}{1.261176in}}{\pgfqpoint{7.733681in}{1.257609in}}%
\pgfpathcurveto{\pgfqpoint{7.730115in}{1.254043in}}{\pgfqpoint{7.728111in}{1.249205in}}{\pgfqpoint{7.728111in}{1.244161in}}%
\pgfpathcurveto{\pgfqpoint{7.728111in}{1.239118in}}{\pgfqpoint{7.730115in}{1.234280in}}{\pgfqpoint{7.733681in}{1.230713in}}%
\pgfpathcurveto{\pgfqpoint{7.737248in}{1.227147in}}{\pgfqpoint{7.742085in}{1.225143in}}{\pgfqpoint{7.747129in}{1.225143in}}%
\pgfpathclose%
\pgfusepath{fill}%
\end{pgfscope}%
\begin{pgfscope}%
\pgfpathrectangle{\pgfqpoint{6.572727in}{0.474100in}}{\pgfqpoint{4.227273in}{3.318700in}}%
\pgfusepath{clip}%
\pgfsetbuttcap%
\pgfsetroundjoin%
\definecolor{currentfill}{rgb}{0.127568,0.566949,0.550556}%
\pgfsetfillcolor{currentfill}%
\pgfsetfillopacity{0.700000}%
\pgfsetlinewidth{0.000000pt}%
\definecolor{currentstroke}{rgb}{0.000000,0.000000,0.000000}%
\pgfsetstrokecolor{currentstroke}%
\pgfsetstrokeopacity{0.700000}%
\pgfsetdash{}{0pt}%
\pgfpathmoveto{\pgfqpoint{8.573791in}{3.275027in}}%
\pgfpathcurveto{\pgfqpoint{8.578835in}{3.275027in}}{\pgfqpoint{8.583672in}{3.277031in}}{\pgfqpoint{8.587239in}{3.280597in}}%
\pgfpathcurveto{\pgfqpoint{8.590805in}{3.284163in}}{\pgfqpoint{8.592809in}{3.289001in}}{\pgfqpoint{8.592809in}{3.294045in}}%
\pgfpathcurveto{\pgfqpoint{8.592809in}{3.299088in}}{\pgfqpoint{8.590805in}{3.303926in}}{\pgfqpoint{8.587239in}{3.307493in}}%
\pgfpathcurveto{\pgfqpoint{8.583672in}{3.311059in}}{\pgfqpoint{8.578835in}{3.313063in}}{\pgfqpoint{8.573791in}{3.313063in}}%
\pgfpathcurveto{\pgfqpoint{8.568747in}{3.313063in}}{\pgfqpoint{8.563909in}{3.311059in}}{\pgfqpoint{8.560343in}{3.307493in}}%
\pgfpathcurveto{\pgfqpoint{8.556777in}{3.303926in}}{\pgfqpoint{8.554773in}{3.299088in}}{\pgfqpoint{8.554773in}{3.294045in}}%
\pgfpathcurveto{\pgfqpoint{8.554773in}{3.289001in}}{\pgfqpoint{8.556777in}{3.284163in}}{\pgfqpoint{8.560343in}{3.280597in}}%
\pgfpathcurveto{\pgfqpoint{8.563909in}{3.277031in}}{\pgfqpoint{8.568747in}{3.275027in}}{\pgfqpoint{8.573791in}{3.275027in}}%
\pgfpathclose%
\pgfusepath{fill}%
\end{pgfscope}%
\begin{pgfscope}%
\pgfpathrectangle{\pgfqpoint{6.572727in}{0.474100in}}{\pgfqpoint{4.227273in}{3.318700in}}%
\pgfusepath{clip}%
\pgfsetbuttcap%
\pgfsetroundjoin%
\definecolor{currentfill}{rgb}{0.127568,0.566949,0.550556}%
\pgfsetfillcolor{currentfill}%
\pgfsetfillopacity{0.700000}%
\pgfsetlinewidth{0.000000pt}%
\definecolor{currentstroke}{rgb}{0.000000,0.000000,0.000000}%
\pgfsetstrokecolor{currentstroke}%
\pgfsetstrokeopacity{0.700000}%
\pgfsetdash{}{0pt}%
\pgfpathmoveto{\pgfqpoint{7.415957in}{2.471074in}}%
\pgfpathcurveto{\pgfqpoint{7.421001in}{2.471074in}}{\pgfqpoint{7.425839in}{2.473078in}}{\pgfqpoint{7.429405in}{2.476644in}}%
\pgfpathcurveto{\pgfqpoint{7.432972in}{2.480211in}}{\pgfqpoint{7.434976in}{2.485048in}}{\pgfqpoint{7.434976in}{2.490092in}}%
\pgfpathcurveto{\pgfqpoint{7.434976in}{2.495136in}}{\pgfqpoint{7.432972in}{2.499973in}}{\pgfqpoint{7.429405in}{2.503540in}}%
\pgfpathcurveto{\pgfqpoint{7.425839in}{2.507106in}}{\pgfqpoint{7.421001in}{2.509110in}}{\pgfqpoint{7.415957in}{2.509110in}}%
\pgfpathcurveto{\pgfqpoint{7.410914in}{2.509110in}}{\pgfqpoint{7.406076in}{2.507106in}}{\pgfqpoint{7.402510in}{2.503540in}}%
\pgfpathcurveto{\pgfqpoint{7.398943in}{2.499973in}}{\pgfqpoint{7.396939in}{2.495136in}}{\pgfqpoint{7.396939in}{2.490092in}}%
\pgfpathcurveto{\pgfqpoint{7.396939in}{2.485048in}}{\pgfqpoint{7.398943in}{2.480211in}}{\pgfqpoint{7.402510in}{2.476644in}}%
\pgfpathcurveto{\pgfqpoint{7.406076in}{2.473078in}}{\pgfqpoint{7.410914in}{2.471074in}}{\pgfqpoint{7.415957in}{2.471074in}}%
\pgfpathclose%
\pgfusepath{fill}%
\end{pgfscope}%
\begin{pgfscope}%
\pgfpathrectangle{\pgfqpoint{6.572727in}{0.474100in}}{\pgfqpoint{4.227273in}{3.318700in}}%
\pgfusepath{clip}%
\pgfsetbuttcap%
\pgfsetroundjoin%
\definecolor{currentfill}{rgb}{0.127568,0.566949,0.550556}%
\pgfsetfillcolor{currentfill}%
\pgfsetfillopacity{0.700000}%
\pgfsetlinewidth{0.000000pt}%
\definecolor{currentstroke}{rgb}{0.000000,0.000000,0.000000}%
\pgfsetstrokecolor{currentstroke}%
\pgfsetstrokeopacity{0.700000}%
\pgfsetdash{}{0pt}%
\pgfpathmoveto{\pgfqpoint{8.074452in}{2.975412in}}%
\pgfpathcurveto{\pgfqpoint{8.079496in}{2.975412in}}{\pgfqpoint{8.084333in}{2.977416in}}{\pgfqpoint{8.087900in}{2.980982in}}%
\pgfpathcurveto{\pgfqpoint{8.091466in}{2.984549in}}{\pgfqpoint{8.093470in}{2.989386in}}{\pgfqpoint{8.093470in}{2.994430in}}%
\pgfpathcurveto{\pgfqpoint{8.093470in}{2.999474in}}{\pgfqpoint{8.091466in}{3.004311in}}{\pgfqpoint{8.087900in}{3.007878in}}%
\pgfpathcurveto{\pgfqpoint{8.084333in}{3.011444in}}{\pgfqpoint{8.079496in}{3.013448in}}{\pgfqpoint{8.074452in}{3.013448in}}%
\pgfpathcurveto{\pgfqpoint{8.069408in}{3.013448in}}{\pgfqpoint{8.064571in}{3.011444in}}{\pgfqpoint{8.061004in}{3.007878in}}%
\pgfpathcurveto{\pgfqpoint{8.057438in}{3.004311in}}{\pgfqpoint{8.055434in}{2.999474in}}{\pgfqpoint{8.055434in}{2.994430in}}%
\pgfpathcurveto{\pgfqpoint{8.055434in}{2.989386in}}{\pgfqpoint{8.057438in}{2.984549in}}{\pgfqpoint{8.061004in}{2.980982in}}%
\pgfpathcurveto{\pgfqpoint{8.064571in}{2.977416in}}{\pgfqpoint{8.069408in}{2.975412in}}{\pgfqpoint{8.074452in}{2.975412in}}%
\pgfpathclose%
\pgfusepath{fill}%
\end{pgfscope}%
\begin{pgfscope}%
\pgfpathrectangle{\pgfqpoint{6.572727in}{0.474100in}}{\pgfqpoint{4.227273in}{3.318700in}}%
\pgfusepath{clip}%
\pgfsetbuttcap%
\pgfsetroundjoin%
\definecolor{currentfill}{rgb}{0.127568,0.566949,0.550556}%
\pgfsetfillcolor{currentfill}%
\pgfsetfillopacity{0.700000}%
\pgfsetlinewidth{0.000000pt}%
\definecolor{currentstroke}{rgb}{0.000000,0.000000,0.000000}%
\pgfsetstrokecolor{currentstroke}%
\pgfsetstrokeopacity{0.700000}%
\pgfsetdash{}{0pt}%
\pgfpathmoveto{\pgfqpoint{8.042473in}{1.200590in}}%
\pgfpathcurveto{\pgfqpoint{8.047517in}{1.200590in}}{\pgfqpoint{8.052355in}{1.202594in}}{\pgfqpoint{8.055921in}{1.206160in}}%
\pgfpathcurveto{\pgfqpoint{8.059488in}{1.209727in}}{\pgfqpoint{8.061492in}{1.214564in}}{\pgfqpoint{8.061492in}{1.219608in}}%
\pgfpathcurveto{\pgfqpoint{8.061492in}{1.224652in}}{\pgfqpoint{8.059488in}{1.229489in}}{\pgfqpoint{8.055921in}{1.233056in}}%
\pgfpathcurveto{\pgfqpoint{8.052355in}{1.236622in}}{\pgfqpoint{8.047517in}{1.238626in}}{\pgfqpoint{8.042473in}{1.238626in}}%
\pgfpathcurveto{\pgfqpoint{8.037430in}{1.238626in}}{\pgfqpoint{8.032592in}{1.236622in}}{\pgfqpoint{8.029026in}{1.233056in}}%
\pgfpathcurveto{\pgfqpoint{8.025459in}{1.229489in}}{\pgfqpoint{8.023455in}{1.224652in}}{\pgfqpoint{8.023455in}{1.219608in}}%
\pgfpathcurveto{\pgfqpoint{8.023455in}{1.214564in}}{\pgfqpoint{8.025459in}{1.209727in}}{\pgfqpoint{8.029026in}{1.206160in}}%
\pgfpathcurveto{\pgfqpoint{8.032592in}{1.202594in}}{\pgfqpoint{8.037430in}{1.200590in}}{\pgfqpoint{8.042473in}{1.200590in}}%
\pgfpathclose%
\pgfusepath{fill}%
\end{pgfscope}%
\begin{pgfscope}%
\pgfpathrectangle{\pgfqpoint{6.572727in}{0.474100in}}{\pgfqpoint{4.227273in}{3.318700in}}%
\pgfusepath{clip}%
\pgfsetbuttcap%
\pgfsetroundjoin%
\definecolor{currentfill}{rgb}{0.127568,0.566949,0.550556}%
\pgfsetfillcolor{currentfill}%
\pgfsetfillopacity{0.700000}%
\pgfsetlinewidth{0.000000pt}%
\definecolor{currentstroke}{rgb}{0.000000,0.000000,0.000000}%
\pgfsetstrokecolor{currentstroke}%
\pgfsetstrokeopacity{0.700000}%
\pgfsetdash{}{0pt}%
\pgfpathmoveto{\pgfqpoint{8.057104in}{2.418049in}}%
\pgfpathcurveto{\pgfqpoint{8.062148in}{2.418049in}}{\pgfqpoint{8.066985in}{2.420053in}}{\pgfqpoint{8.070552in}{2.423619in}}%
\pgfpathcurveto{\pgfqpoint{8.074118in}{2.427186in}}{\pgfqpoint{8.076122in}{2.432024in}}{\pgfqpoint{8.076122in}{2.437067in}}%
\pgfpathcurveto{\pgfqpoint{8.076122in}{2.442111in}}{\pgfqpoint{8.074118in}{2.446949in}}{\pgfqpoint{8.070552in}{2.450515in}}%
\pgfpathcurveto{\pgfqpoint{8.066985in}{2.454082in}}{\pgfqpoint{8.062148in}{2.456085in}}{\pgfqpoint{8.057104in}{2.456085in}}%
\pgfpathcurveto{\pgfqpoint{8.052060in}{2.456085in}}{\pgfqpoint{8.047223in}{2.454082in}}{\pgfqpoint{8.043656in}{2.450515in}}%
\pgfpathcurveto{\pgfqpoint{8.040090in}{2.446949in}}{\pgfqpoint{8.038086in}{2.442111in}}{\pgfqpoint{8.038086in}{2.437067in}}%
\pgfpathcurveto{\pgfqpoint{8.038086in}{2.432024in}}{\pgfqpoint{8.040090in}{2.427186in}}{\pgfqpoint{8.043656in}{2.423619in}}%
\pgfpathcurveto{\pgfqpoint{8.047223in}{2.420053in}}{\pgfqpoint{8.052060in}{2.418049in}}{\pgfqpoint{8.057104in}{2.418049in}}%
\pgfpathclose%
\pgfusepath{fill}%
\end{pgfscope}%
\begin{pgfscope}%
\pgfpathrectangle{\pgfqpoint{6.572727in}{0.474100in}}{\pgfqpoint{4.227273in}{3.318700in}}%
\pgfusepath{clip}%
\pgfsetbuttcap%
\pgfsetroundjoin%
\definecolor{currentfill}{rgb}{0.993248,0.906157,0.143936}%
\pgfsetfillcolor{currentfill}%
\pgfsetfillopacity{0.700000}%
\pgfsetlinewidth{0.000000pt}%
\definecolor{currentstroke}{rgb}{0.000000,0.000000,0.000000}%
\pgfsetstrokecolor{currentstroke}%
\pgfsetstrokeopacity{0.700000}%
\pgfsetdash{}{0pt}%
\pgfpathmoveto{\pgfqpoint{9.446345in}{1.870362in}}%
\pgfpathcurveto{\pgfqpoint{9.451389in}{1.870362in}}{\pgfqpoint{9.456226in}{1.872366in}}{\pgfqpoint{9.459793in}{1.875932in}}%
\pgfpathcurveto{\pgfqpoint{9.463359in}{1.879499in}}{\pgfqpoint{9.465363in}{1.884336in}}{\pgfqpoint{9.465363in}{1.889380in}}%
\pgfpathcurveto{\pgfqpoint{9.465363in}{1.894424in}}{\pgfqpoint{9.463359in}{1.899261in}}{\pgfqpoint{9.459793in}{1.902828in}}%
\pgfpathcurveto{\pgfqpoint{9.456226in}{1.906394in}}{\pgfqpoint{9.451389in}{1.908398in}}{\pgfqpoint{9.446345in}{1.908398in}}%
\pgfpathcurveto{\pgfqpoint{9.441301in}{1.908398in}}{\pgfqpoint{9.436463in}{1.906394in}}{\pgfqpoint{9.432897in}{1.902828in}}%
\pgfpathcurveto{\pgfqpoint{9.429331in}{1.899261in}}{\pgfqpoint{9.427327in}{1.894424in}}{\pgfqpoint{9.427327in}{1.889380in}}%
\pgfpathcurveto{\pgfqpoint{9.427327in}{1.884336in}}{\pgfqpoint{9.429331in}{1.879499in}}{\pgfqpoint{9.432897in}{1.875932in}}%
\pgfpathcurveto{\pgfqpoint{9.436463in}{1.872366in}}{\pgfqpoint{9.441301in}{1.870362in}}{\pgfqpoint{9.446345in}{1.870362in}}%
\pgfpathclose%
\pgfusepath{fill}%
\end{pgfscope}%
\begin{pgfscope}%
\pgfpathrectangle{\pgfqpoint{6.572727in}{0.474100in}}{\pgfqpoint{4.227273in}{3.318700in}}%
\pgfusepath{clip}%
\pgfsetbuttcap%
\pgfsetroundjoin%
\definecolor{currentfill}{rgb}{0.993248,0.906157,0.143936}%
\pgfsetfillcolor{currentfill}%
\pgfsetfillopacity{0.700000}%
\pgfsetlinewidth{0.000000pt}%
\definecolor{currentstroke}{rgb}{0.000000,0.000000,0.000000}%
\pgfsetstrokecolor{currentstroke}%
\pgfsetstrokeopacity{0.700000}%
\pgfsetdash{}{0pt}%
\pgfpathmoveto{\pgfqpoint{9.648030in}{1.465150in}}%
\pgfpathcurveto{\pgfqpoint{9.653074in}{1.465150in}}{\pgfqpoint{9.657912in}{1.467154in}}{\pgfqpoint{9.661478in}{1.470720in}}%
\pgfpathcurveto{\pgfqpoint{9.665044in}{1.474287in}}{\pgfqpoint{9.667048in}{1.479125in}}{\pgfqpoint{9.667048in}{1.484168in}}%
\pgfpathcurveto{\pgfqpoint{9.667048in}{1.489212in}}{\pgfqpoint{9.665044in}{1.494050in}}{\pgfqpoint{9.661478in}{1.497616in}}%
\pgfpathcurveto{\pgfqpoint{9.657912in}{1.501183in}}{\pgfqpoint{9.653074in}{1.503186in}}{\pgfqpoint{9.648030in}{1.503186in}}%
\pgfpathcurveto{\pgfqpoint{9.642986in}{1.503186in}}{\pgfqpoint{9.638149in}{1.501183in}}{\pgfqpoint{9.634582in}{1.497616in}}%
\pgfpathcurveto{\pgfqpoint{9.631016in}{1.494050in}}{\pgfqpoint{9.629012in}{1.489212in}}{\pgfqpoint{9.629012in}{1.484168in}}%
\pgfpathcurveto{\pgfqpoint{9.629012in}{1.479125in}}{\pgfqpoint{9.631016in}{1.474287in}}{\pgfqpoint{9.634582in}{1.470720in}}%
\pgfpathcurveto{\pgfqpoint{9.638149in}{1.467154in}}{\pgfqpoint{9.642986in}{1.465150in}}{\pgfqpoint{9.648030in}{1.465150in}}%
\pgfpathclose%
\pgfusepath{fill}%
\end{pgfscope}%
\begin{pgfscope}%
\pgfpathrectangle{\pgfqpoint{6.572727in}{0.474100in}}{\pgfqpoint{4.227273in}{3.318700in}}%
\pgfusepath{clip}%
\pgfsetbuttcap%
\pgfsetroundjoin%
\definecolor{currentfill}{rgb}{0.127568,0.566949,0.550556}%
\pgfsetfillcolor{currentfill}%
\pgfsetfillopacity{0.700000}%
\pgfsetlinewidth{0.000000pt}%
\definecolor{currentstroke}{rgb}{0.000000,0.000000,0.000000}%
\pgfsetstrokecolor{currentstroke}%
\pgfsetstrokeopacity{0.700000}%
\pgfsetdash{}{0pt}%
\pgfpathmoveto{\pgfqpoint{7.824022in}{1.601359in}}%
\pgfpathcurveto{\pgfqpoint{7.829066in}{1.601359in}}{\pgfqpoint{7.833904in}{1.603363in}}{\pgfqpoint{7.837470in}{1.606929in}}%
\pgfpathcurveto{\pgfqpoint{7.841037in}{1.610496in}}{\pgfqpoint{7.843040in}{1.615333in}}{\pgfqpoint{7.843040in}{1.620377in}}%
\pgfpathcurveto{\pgfqpoint{7.843040in}{1.625421in}}{\pgfqpoint{7.841037in}{1.630259in}}{\pgfqpoint{7.837470in}{1.633825in}}%
\pgfpathcurveto{\pgfqpoint{7.833904in}{1.637391in}}{\pgfqpoint{7.829066in}{1.639395in}}{\pgfqpoint{7.824022in}{1.639395in}}%
\pgfpathcurveto{\pgfqpoint{7.818979in}{1.639395in}}{\pgfqpoint{7.814141in}{1.637391in}}{\pgfqpoint{7.810574in}{1.633825in}}%
\pgfpathcurveto{\pgfqpoint{7.807008in}{1.630259in}}{\pgfqpoint{7.805004in}{1.625421in}}{\pgfqpoint{7.805004in}{1.620377in}}%
\pgfpathcurveto{\pgfqpoint{7.805004in}{1.615333in}}{\pgfqpoint{7.807008in}{1.610496in}}{\pgfqpoint{7.810574in}{1.606929in}}%
\pgfpathcurveto{\pgfqpoint{7.814141in}{1.603363in}}{\pgfqpoint{7.818979in}{1.601359in}}{\pgfqpoint{7.824022in}{1.601359in}}%
\pgfpathclose%
\pgfusepath{fill}%
\end{pgfscope}%
\begin{pgfscope}%
\pgfpathrectangle{\pgfqpoint{6.572727in}{0.474100in}}{\pgfqpoint{4.227273in}{3.318700in}}%
\pgfusepath{clip}%
\pgfsetbuttcap%
\pgfsetroundjoin%
\definecolor{currentfill}{rgb}{0.127568,0.566949,0.550556}%
\pgfsetfillcolor{currentfill}%
\pgfsetfillopacity{0.700000}%
\pgfsetlinewidth{0.000000pt}%
\definecolor{currentstroke}{rgb}{0.000000,0.000000,0.000000}%
\pgfsetstrokecolor{currentstroke}%
\pgfsetstrokeopacity{0.700000}%
\pgfsetdash{}{0pt}%
\pgfpathmoveto{\pgfqpoint{8.428063in}{3.165375in}}%
\pgfpathcurveto{\pgfqpoint{8.433107in}{3.165375in}}{\pgfqpoint{8.437945in}{3.167379in}}{\pgfqpoint{8.441511in}{3.170946in}}%
\pgfpathcurveto{\pgfqpoint{8.445078in}{3.174512in}}{\pgfqpoint{8.447081in}{3.179350in}}{\pgfqpoint{8.447081in}{3.184393in}}%
\pgfpathcurveto{\pgfqpoint{8.447081in}{3.189437in}}{\pgfqpoint{8.445078in}{3.194275in}}{\pgfqpoint{8.441511in}{3.197841in}}%
\pgfpathcurveto{\pgfqpoint{8.437945in}{3.201408in}}{\pgfqpoint{8.433107in}{3.203412in}}{\pgfqpoint{8.428063in}{3.203412in}}%
\pgfpathcurveto{\pgfqpoint{8.423020in}{3.203412in}}{\pgfqpoint{8.418182in}{3.201408in}}{\pgfqpoint{8.414615in}{3.197841in}}%
\pgfpathcurveto{\pgfqpoint{8.411049in}{3.194275in}}{\pgfqpoint{8.409045in}{3.189437in}}{\pgfqpoint{8.409045in}{3.184393in}}%
\pgfpathcurveto{\pgfqpoint{8.409045in}{3.179350in}}{\pgfqpoint{8.411049in}{3.174512in}}{\pgfqpoint{8.414615in}{3.170946in}}%
\pgfpathcurveto{\pgfqpoint{8.418182in}{3.167379in}}{\pgfqpoint{8.423020in}{3.165375in}}{\pgfqpoint{8.428063in}{3.165375in}}%
\pgfpathclose%
\pgfusepath{fill}%
\end{pgfscope}%
\begin{pgfscope}%
\pgfpathrectangle{\pgfqpoint{6.572727in}{0.474100in}}{\pgfqpoint{4.227273in}{3.318700in}}%
\pgfusepath{clip}%
\pgfsetbuttcap%
\pgfsetroundjoin%
\definecolor{currentfill}{rgb}{0.993248,0.906157,0.143936}%
\pgfsetfillcolor{currentfill}%
\pgfsetfillopacity{0.700000}%
\pgfsetlinewidth{0.000000pt}%
\definecolor{currentstroke}{rgb}{0.000000,0.000000,0.000000}%
\pgfsetstrokecolor{currentstroke}%
\pgfsetstrokeopacity{0.700000}%
\pgfsetdash{}{0pt}%
\pgfpathmoveto{\pgfqpoint{9.258780in}{1.220225in}}%
\pgfpathcurveto{\pgfqpoint{9.263824in}{1.220225in}}{\pgfqpoint{9.268662in}{1.222229in}}{\pgfqpoint{9.272228in}{1.225795in}}%
\pgfpathcurveto{\pgfqpoint{9.275794in}{1.229362in}}{\pgfqpoint{9.277798in}{1.234199in}}{\pgfqpoint{9.277798in}{1.239243in}}%
\pgfpathcurveto{\pgfqpoint{9.277798in}{1.244287in}}{\pgfqpoint{9.275794in}{1.249125in}}{\pgfqpoint{9.272228in}{1.252691in}}%
\pgfpathcurveto{\pgfqpoint{9.268662in}{1.256257in}}{\pgfqpoint{9.263824in}{1.258261in}}{\pgfqpoint{9.258780in}{1.258261in}}%
\pgfpathcurveto{\pgfqpoint{9.253736in}{1.258261in}}{\pgfqpoint{9.248899in}{1.256257in}}{\pgfqpoint{9.245332in}{1.252691in}}%
\pgfpathcurveto{\pgfqpoint{9.241766in}{1.249125in}}{\pgfqpoint{9.239762in}{1.244287in}}{\pgfqpoint{9.239762in}{1.239243in}}%
\pgfpathcurveto{\pgfqpoint{9.239762in}{1.234199in}}{\pgfqpoint{9.241766in}{1.229362in}}{\pgfqpoint{9.245332in}{1.225795in}}%
\pgfpathcurveto{\pgfqpoint{9.248899in}{1.222229in}}{\pgfqpoint{9.253736in}{1.220225in}}{\pgfqpoint{9.258780in}{1.220225in}}%
\pgfpathclose%
\pgfusepath{fill}%
\end{pgfscope}%
\begin{pgfscope}%
\pgfpathrectangle{\pgfqpoint{6.572727in}{0.474100in}}{\pgfqpoint{4.227273in}{3.318700in}}%
\pgfusepath{clip}%
\pgfsetbuttcap%
\pgfsetroundjoin%
\definecolor{currentfill}{rgb}{0.127568,0.566949,0.550556}%
\pgfsetfillcolor{currentfill}%
\pgfsetfillopacity{0.700000}%
\pgfsetlinewidth{0.000000pt}%
\definecolor{currentstroke}{rgb}{0.000000,0.000000,0.000000}%
\pgfsetstrokecolor{currentstroke}%
\pgfsetstrokeopacity{0.700000}%
\pgfsetdash{}{0pt}%
\pgfpathmoveto{\pgfqpoint{7.915753in}{2.943696in}}%
\pgfpathcurveto{\pgfqpoint{7.920796in}{2.943696in}}{\pgfqpoint{7.925634in}{2.945700in}}{\pgfqpoint{7.929200in}{2.949267in}}%
\pgfpathcurveto{\pgfqpoint{7.932767in}{2.952833in}}{\pgfqpoint{7.934771in}{2.957671in}}{\pgfqpoint{7.934771in}{2.962714in}}%
\pgfpathcurveto{\pgfqpoint{7.934771in}{2.967758in}}{\pgfqpoint{7.932767in}{2.972596in}}{\pgfqpoint{7.929200in}{2.976162in}}%
\pgfpathcurveto{\pgfqpoint{7.925634in}{2.979729in}}{\pgfqpoint{7.920796in}{2.981733in}}{\pgfqpoint{7.915753in}{2.981733in}}%
\pgfpathcurveto{\pgfqpoint{7.910709in}{2.981733in}}{\pgfqpoint{7.905871in}{2.979729in}}{\pgfqpoint{7.902305in}{2.976162in}}%
\pgfpathcurveto{\pgfqpoint{7.898738in}{2.972596in}}{\pgfqpoint{7.896734in}{2.967758in}}{\pgfqpoint{7.896734in}{2.962714in}}%
\pgfpathcurveto{\pgfqpoint{7.896734in}{2.957671in}}{\pgfqpoint{7.898738in}{2.952833in}}{\pgfqpoint{7.902305in}{2.949267in}}%
\pgfpathcurveto{\pgfqpoint{7.905871in}{2.945700in}}{\pgfqpoint{7.910709in}{2.943696in}}{\pgfqpoint{7.915753in}{2.943696in}}%
\pgfpathclose%
\pgfusepath{fill}%
\end{pgfscope}%
\begin{pgfscope}%
\pgfpathrectangle{\pgfqpoint{6.572727in}{0.474100in}}{\pgfqpoint{4.227273in}{3.318700in}}%
\pgfusepath{clip}%
\pgfsetbuttcap%
\pgfsetroundjoin%
\definecolor{currentfill}{rgb}{0.993248,0.906157,0.143936}%
\pgfsetfillcolor{currentfill}%
\pgfsetfillopacity{0.700000}%
\pgfsetlinewidth{0.000000pt}%
\definecolor{currentstroke}{rgb}{0.000000,0.000000,0.000000}%
\pgfsetstrokecolor{currentstroke}%
\pgfsetstrokeopacity{0.700000}%
\pgfsetdash{}{0pt}%
\pgfpathmoveto{\pgfqpoint{9.496340in}{1.471711in}}%
\pgfpathcurveto{\pgfqpoint{9.501384in}{1.471711in}}{\pgfqpoint{9.506222in}{1.473715in}}{\pgfqpoint{9.509788in}{1.477281in}}%
\pgfpathcurveto{\pgfqpoint{9.513354in}{1.480848in}}{\pgfqpoint{9.515358in}{1.485686in}}{\pgfqpoint{9.515358in}{1.490729in}}%
\pgfpathcurveto{\pgfqpoint{9.515358in}{1.495773in}}{\pgfqpoint{9.513354in}{1.500611in}}{\pgfqpoint{9.509788in}{1.504177in}}%
\pgfpathcurveto{\pgfqpoint{9.506222in}{1.507744in}}{\pgfqpoint{9.501384in}{1.509747in}}{\pgfqpoint{9.496340in}{1.509747in}}%
\pgfpathcurveto{\pgfqpoint{9.491296in}{1.509747in}}{\pgfqpoint{9.486459in}{1.507744in}}{\pgfqpoint{9.482892in}{1.504177in}}%
\pgfpathcurveto{\pgfqpoint{9.479326in}{1.500611in}}{\pgfqpoint{9.477322in}{1.495773in}}{\pgfqpoint{9.477322in}{1.490729in}}%
\pgfpathcurveto{\pgfqpoint{9.477322in}{1.485686in}}{\pgfqpoint{9.479326in}{1.480848in}}{\pgfqpoint{9.482892in}{1.477281in}}%
\pgfpathcurveto{\pgfqpoint{9.486459in}{1.473715in}}{\pgfqpoint{9.491296in}{1.471711in}}{\pgfqpoint{9.496340in}{1.471711in}}%
\pgfpathclose%
\pgfusepath{fill}%
\end{pgfscope}%
\begin{pgfscope}%
\pgfpathrectangle{\pgfqpoint{6.572727in}{0.474100in}}{\pgfqpoint{4.227273in}{3.318700in}}%
\pgfusepath{clip}%
\pgfsetbuttcap%
\pgfsetroundjoin%
\definecolor{currentfill}{rgb}{0.993248,0.906157,0.143936}%
\pgfsetfillcolor{currentfill}%
\pgfsetfillopacity{0.700000}%
\pgfsetlinewidth{0.000000pt}%
\definecolor{currentstroke}{rgb}{0.000000,0.000000,0.000000}%
\pgfsetstrokecolor{currentstroke}%
\pgfsetstrokeopacity{0.700000}%
\pgfsetdash{}{0pt}%
\pgfpathmoveto{\pgfqpoint{10.133801in}{1.006054in}}%
\pgfpathcurveto{\pgfqpoint{10.138845in}{1.006054in}}{\pgfqpoint{10.143683in}{1.008058in}}{\pgfqpoint{10.147249in}{1.011624in}}%
\pgfpathcurveto{\pgfqpoint{10.150816in}{1.015191in}}{\pgfqpoint{10.152820in}{1.020028in}}{\pgfqpoint{10.152820in}{1.025072in}}%
\pgfpathcurveto{\pgfqpoint{10.152820in}{1.030116in}}{\pgfqpoint{10.150816in}{1.034954in}}{\pgfqpoint{10.147249in}{1.038520in}}%
\pgfpathcurveto{\pgfqpoint{10.143683in}{1.042086in}}{\pgfqpoint{10.138845in}{1.044090in}}{\pgfqpoint{10.133801in}{1.044090in}}%
\pgfpathcurveto{\pgfqpoint{10.128758in}{1.044090in}}{\pgfqpoint{10.123920in}{1.042086in}}{\pgfqpoint{10.120354in}{1.038520in}}%
\pgfpathcurveto{\pgfqpoint{10.116787in}{1.034954in}}{\pgfqpoint{10.114783in}{1.030116in}}{\pgfqpoint{10.114783in}{1.025072in}}%
\pgfpathcurveto{\pgfqpoint{10.114783in}{1.020028in}}{\pgfqpoint{10.116787in}{1.015191in}}{\pgfqpoint{10.120354in}{1.011624in}}%
\pgfpathcurveto{\pgfqpoint{10.123920in}{1.008058in}}{\pgfqpoint{10.128758in}{1.006054in}}{\pgfqpoint{10.133801in}{1.006054in}}%
\pgfpathclose%
\pgfusepath{fill}%
\end{pgfscope}%
\begin{pgfscope}%
\pgfpathrectangle{\pgfqpoint{6.572727in}{0.474100in}}{\pgfqpoint{4.227273in}{3.318700in}}%
\pgfusepath{clip}%
\pgfsetbuttcap%
\pgfsetroundjoin%
\definecolor{currentfill}{rgb}{0.993248,0.906157,0.143936}%
\pgfsetfillcolor{currentfill}%
\pgfsetfillopacity{0.700000}%
\pgfsetlinewidth{0.000000pt}%
\definecolor{currentstroke}{rgb}{0.000000,0.000000,0.000000}%
\pgfsetstrokecolor{currentstroke}%
\pgfsetstrokeopacity{0.700000}%
\pgfsetdash{}{0pt}%
\pgfpathmoveto{\pgfqpoint{9.791706in}{1.650366in}}%
\pgfpathcurveto{\pgfqpoint{9.796749in}{1.650366in}}{\pgfqpoint{9.801587in}{1.652370in}}{\pgfqpoint{9.805153in}{1.655937in}}%
\pgfpathcurveto{\pgfqpoint{9.808720in}{1.659503in}}{\pgfqpoint{9.810724in}{1.664341in}}{\pgfqpoint{9.810724in}{1.669384in}}%
\pgfpathcurveto{\pgfqpoint{9.810724in}{1.674428in}}{\pgfqpoint{9.808720in}{1.679266in}}{\pgfqpoint{9.805153in}{1.682832in}}%
\pgfpathcurveto{\pgfqpoint{9.801587in}{1.686399in}}{\pgfqpoint{9.796749in}{1.688403in}}{\pgfqpoint{9.791706in}{1.688403in}}%
\pgfpathcurveto{\pgfqpoint{9.786662in}{1.688403in}}{\pgfqpoint{9.781824in}{1.686399in}}{\pgfqpoint{9.778258in}{1.682832in}}%
\pgfpathcurveto{\pgfqpoint{9.774691in}{1.679266in}}{\pgfqpoint{9.772687in}{1.674428in}}{\pgfqpoint{9.772687in}{1.669384in}}%
\pgfpathcurveto{\pgfqpoint{9.772687in}{1.664341in}}{\pgfqpoint{9.774691in}{1.659503in}}{\pgfqpoint{9.778258in}{1.655937in}}%
\pgfpathcurveto{\pgfqpoint{9.781824in}{1.652370in}}{\pgfqpoint{9.786662in}{1.650366in}}{\pgfqpoint{9.791706in}{1.650366in}}%
\pgfpathclose%
\pgfusepath{fill}%
\end{pgfscope}%
\begin{pgfscope}%
\pgfpathrectangle{\pgfqpoint{6.572727in}{0.474100in}}{\pgfqpoint{4.227273in}{3.318700in}}%
\pgfusepath{clip}%
\pgfsetbuttcap%
\pgfsetroundjoin%
\definecolor{currentfill}{rgb}{0.127568,0.566949,0.550556}%
\pgfsetfillcolor{currentfill}%
\pgfsetfillopacity{0.700000}%
\pgfsetlinewidth{0.000000pt}%
\definecolor{currentstroke}{rgb}{0.000000,0.000000,0.000000}%
\pgfsetstrokecolor{currentstroke}%
\pgfsetstrokeopacity{0.700000}%
\pgfsetdash{}{0pt}%
\pgfpathmoveto{\pgfqpoint{8.327714in}{2.964758in}}%
\pgfpathcurveto{\pgfqpoint{8.332758in}{2.964758in}}{\pgfqpoint{8.337595in}{2.966762in}}{\pgfqpoint{8.341162in}{2.970329in}}%
\pgfpathcurveto{\pgfqpoint{8.344728in}{2.973895in}}{\pgfqpoint{8.346732in}{2.978733in}}{\pgfqpoint{8.346732in}{2.983777in}}%
\pgfpathcurveto{\pgfqpoint{8.346732in}{2.988820in}}{\pgfqpoint{8.344728in}{2.993658in}}{\pgfqpoint{8.341162in}{2.997224in}}%
\pgfpathcurveto{\pgfqpoint{8.337595in}{3.000791in}}{\pgfqpoint{8.332758in}{3.002795in}}{\pgfqpoint{8.327714in}{3.002795in}}%
\pgfpathcurveto{\pgfqpoint{8.322670in}{3.002795in}}{\pgfqpoint{8.317833in}{3.000791in}}{\pgfqpoint{8.314266in}{2.997224in}}%
\pgfpathcurveto{\pgfqpoint{8.310700in}{2.993658in}}{\pgfqpoint{8.308696in}{2.988820in}}{\pgfqpoint{8.308696in}{2.983777in}}%
\pgfpathcurveto{\pgfqpoint{8.308696in}{2.978733in}}{\pgfqpoint{8.310700in}{2.973895in}}{\pgfqpoint{8.314266in}{2.970329in}}%
\pgfpathcurveto{\pgfqpoint{8.317833in}{2.966762in}}{\pgfqpoint{8.322670in}{2.964758in}}{\pgfqpoint{8.327714in}{2.964758in}}%
\pgfpathclose%
\pgfusepath{fill}%
\end{pgfscope}%
\begin{pgfscope}%
\pgfpathrectangle{\pgfqpoint{6.572727in}{0.474100in}}{\pgfqpoint{4.227273in}{3.318700in}}%
\pgfusepath{clip}%
\pgfsetbuttcap%
\pgfsetroundjoin%
\definecolor{currentfill}{rgb}{0.993248,0.906157,0.143936}%
\pgfsetfillcolor{currentfill}%
\pgfsetfillopacity{0.700000}%
\pgfsetlinewidth{0.000000pt}%
\definecolor{currentstroke}{rgb}{0.000000,0.000000,0.000000}%
\pgfsetstrokecolor{currentstroke}%
\pgfsetstrokeopacity{0.700000}%
\pgfsetdash{}{0pt}%
\pgfpathmoveto{\pgfqpoint{9.386556in}{1.606792in}}%
\pgfpathcurveto{\pgfqpoint{9.391600in}{1.606792in}}{\pgfqpoint{9.396438in}{1.608796in}}{\pgfqpoint{9.400004in}{1.612362in}}%
\pgfpathcurveto{\pgfqpoint{9.403570in}{1.615929in}}{\pgfqpoint{9.405574in}{1.620767in}}{\pgfqpoint{9.405574in}{1.625810in}}%
\pgfpathcurveto{\pgfqpoint{9.405574in}{1.630854in}}{\pgfqpoint{9.403570in}{1.635692in}}{\pgfqpoint{9.400004in}{1.639258in}}%
\pgfpathcurveto{\pgfqpoint{9.396438in}{1.642825in}}{\pgfqpoint{9.391600in}{1.644828in}}{\pgfqpoint{9.386556in}{1.644828in}}%
\pgfpathcurveto{\pgfqpoint{9.381513in}{1.644828in}}{\pgfqpoint{9.376675in}{1.642825in}}{\pgfqpoint{9.373108in}{1.639258in}}%
\pgfpathcurveto{\pgfqpoint{9.369542in}{1.635692in}}{\pgfqpoint{9.367538in}{1.630854in}}{\pgfqpoint{9.367538in}{1.625810in}}%
\pgfpathcurveto{\pgfqpoint{9.367538in}{1.620767in}}{\pgfqpoint{9.369542in}{1.615929in}}{\pgfqpoint{9.373108in}{1.612362in}}%
\pgfpathcurveto{\pgfqpoint{9.376675in}{1.608796in}}{\pgfqpoint{9.381513in}{1.606792in}}{\pgfqpoint{9.386556in}{1.606792in}}%
\pgfpathclose%
\pgfusepath{fill}%
\end{pgfscope}%
\begin{pgfscope}%
\pgfpathrectangle{\pgfqpoint{6.572727in}{0.474100in}}{\pgfqpoint{4.227273in}{3.318700in}}%
\pgfusepath{clip}%
\pgfsetbuttcap%
\pgfsetroundjoin%
\definecolor{currentfill}{rgb}{0.127568,0.566949,0.550556}%
\pgfsetfillcolor{currentfill}%
\pgfsetfillopacity{0.700000}%
\pgfsetlinewidth{0.000000pt}%
\definecolor{currentstroke}{rgb}{0.000000,0.000000,0.000000}%
\pgfsetstrokecolor{currentstroke}%
\pgfsetstrokeopacity{0.700000}%
\pgfsetdash{}{0pt}%
\pgfpathmoveto{\pgfqpoint{7.638030in}{1.288159in}}%
\pgfpathcurveto{\pgfqpoint{7.643074in}{1.288159in}}{\pgfqpoint{7.647912in}{1.290163in}}{\pgfqpoint{7.651478in}{1.293729in}}%
\pgfpathcurveto{\pgfqpoint{7.655045in}{1.297296in}}{\pgfqpoint{7.657049in}{1.302133in}}{\pgfqpoint{7.657049in}{1.307177in}}%
\pgfpathcurveto{\pgfqpoint{7.657049in}{1.312221in}}{\pgfqpoint{7.655045in}{1.317058in}}{\pgfqpoint{7.651478in}{1.320625in}}%
\pgfpathcurveto{\pgfqpoint{7.647912in}{1.324191in}}{\pgfqpoint{7.643074in}{1.326195in}}{\pgfqpoint{7.638030in}{1.326195in}}%
\pgfpathcurveto{\pgfqpoint{7.632987in}{1.326195in}}{\pgfqpoint{7.628149in}{1.324191in}}{\pgfqpoint{7.624583in}{1.320625in}}%
\pgfpathcurveto{\pgfqpoint{7.621016in}{1.317058in}}{\pgfqpoint{7.619012in}{1.312221in}}{\pgfqpoint{7.619012in}{1.307177in}}%
\pgfpathcurveto{\pgfqpoint{7.619012in}{1.302133in}}{\pgfqpoint{7.621016in}{1.297296in}}{\pgfqpoint{7.624583in}{1.293729in}}%
\pgfpathcurveto{\pgfqpoint{7.628149in}{1.290163in}}{\pgfqpoint{7.632987in}{1.288159in}}{\pgfqpoint{7.638030in}{1.288159in}}%
\pgfpathclose%
\pgfusepath{fill}%
\end{pgfscope}%
\begin{pgfscope}%
\pgfpathrectangle{\pgfqpoint{6.572727in}{0.474100in}}{\pgfqpoint{4.227273in}{3.318700in}}%
\pgfusepath{clip}%
\pgfsetbuttcap%
\pgfsetroundjoin%
\definecolor{currentfill}{rgb}{0.127568,0.566949,0.550556}%
\pgfsetfillcolor{currentfill}%
\pgfsetfillopacity{0.700000}%
\pgfsetlinewidth{0.000000pt}%
\definecolor{currentstroke}{rgb}{0.000000,0.000000,0.000000}%
\pgfsetstrokecolor{currentstroke}%
\pgfsetstrokeopacity{0.700000}%
\pgfsetdash{}{0pt}%
\pgfpathmoveto{\pgfqpoint{7.604546in}{2.376934in}}%
\pgfpathcurveto{\pgfqpoint{7.609590in}{2.376934in}}{\pgfqpoint{7.614427in}{2.378938in}}{\pgfqpoint{7.617994in}{2.382504in}}%
\pgfpathcurveto{\pgfqpoint{7.621560in}{2.386070in}}{\pgfqpoint{7.623564in}{2.390908in}}{\pgfqpoint{7.623564in}{2.395952in}}%
\pgfpathcurveto{\pgfqpoint{7.623564in}{2.400995in}}{\pgfqpoint{7.621560in}{2.405833in}}{\pgfqpoint{7.617994in}{2.409400in}}%
\pgfpathcurveto{\pgfqpoint{7.614427in}{2.412966in}}{\pgfqpoint{7.609590in}{2.414970in}}{\pgfqpoint{7.604546in}{2.414970in}}%
\pgfpathcurveto{\pgfqpoint{7.599502in}{2.414970in}}{\pgfqpoint{7.594665in}{2.412966in}}{\pgfqpoint{7.591098in}{2.409400in}}%
\pgfpathcurveto{\pgfqpoint{7.587532in}{2.405833in}}{\pgfqpoint{7.585528in}{2.400995in}}{\pgfqpoint{7.585528in}{2.395952in}}%
\pgfpathcurveto{\pgfqpoint{7.585528in}{2.390908in}}{\pgfqpoint{7.587532in}{2.386070in}}{\pgfqpoint{7.591098in}{2.382504in}}%
\pgfpathcurveto{\pgfqpoint{7.594665in}{2.378938in}}{\pgfqpoint{7.599502in}{2.376934in}}{\pgfqpoint{7.604546in}{2.376934in}}%
\pgfpathclose%
\pgfusepath{fill}%
\end{pgfscope}%
\begin{pgfscope}%
\pgfpathrectangle{\pgfqpoint{6.572727in}{0.474100in}}{\pgfqpoint{4.227273in}{3.318700in}}%
\pgfusepath{clip}%
\pgfsetbuttcap%
\pgfsetroundjoin%
\definecolor{currentfill}{rgb}{0.127568,0.566949,0.550556}%
\pgfsetfillcolor{currentfill}%
\pgfsetfillopacity{0.700000}%
\pgfsetlinewidth{0.000000pt}%
\definecolor{currentstroke}{rgb}{0.000000,0.000000,0.000000}%
\pgfsetstrokecolor{currentstroke}%
\pgfsetstrokeopacity{0.700000}%
\pgfsetdash{}{0pt}%
\pgfpathmoveto{\pgfqpoint{8.385452in}{2.802472in}}%
\pgfpathcurveto{\pgfqpoint{8.390496in}{2.802472in}}{\pgfqpoint{8.395334in}{2.804476in}}{\pgfqpoint{8.398900in}{2.808043in}}%
\pgfpathcurveto{\pgfqpoint{8.402467in}{2.811609in}}{\pgfqpoint{8.404470in}{2.816447in}}{\pgfqpoint{8.404470in}{2.821490in}}%
\pgfpathcurveto{\pgfqpoint{8.404470in}{2.826534in}}{\pgfqpoint{8.402467in}{2.831372in}}{\pgfqpoint{8.398900in}{2.834938in}}%
\pgfpathcurveto{\pgfqpoint{8.395334in}{2.838505in}}{\pgfqpoint{8.390496in}{2.840509in}}{\pgfqpoint{8.385452in}{2.840509in}}%
\pgfpathcurveto{\pgfqpoint{8.380409in}{2.840509in}}{\pgfqpoint{8.375571in}{2.838505in}}{\pgfqpoint{8.372004in}{2.834938in}}%
\pgfpathcurveto{\pgfqpoint{8.368438in}{2.831372in}}{\pgfqpoint{8.366434in}{2.826534in}}{\pgfqpoint{8.366434in}{2.821490in}}%
\pgfpathcurveto{\pgfqpoint{8.366434in}{2.816447in}}{\pgfqpoint{8.368438in}{2.811609in}}{\pgfqpoint{8.372004in}{2.808043in}}%
\pgfpathcurveto{\pgfqpoint{8.375571in}{2.804476in}}{\pgfqpoint{8.380409in}{2.802472in}}{\pgfqpoint{8.385452in}{2.802472in}}%
\pgfpathclose%
\pgfusepath{fill}%
\end{pgfscope}%
\begin{pgfscope}%
\pgfpathrectangle{\pgfqpoint{6.572727in}{0.474100in}}{\pgfqpoint{4.227273in}{3.318700in}}%
\pgfusepath{clip}%
\pgfsetbuttcap%
\pgfsetroundjoin%
\definecolor{currentfill}{rgb}{0.127568,0.566949,0.550556}%
\pgfsetfillcolor{currentfill}%
\pgfsetfillopacity{0.700000}%
\pgfsetlinewidth{0.000000pt}%
\definecolor{currentstroke}{rgb}{0.000000,0.000000,0.000000}%
\pgfsetstrokecolor{currentstroke}%
\pgfsetstrokeopacity{0.700000}%
\pgfsetdash{}{0pt}%
\pgfpathmoveto{\pgfqpoint{8.414996in}{2.907545in}}%
\pgfpathcurveto{\pgfqpoint{8.420040in}{2.907545in}}{\pgfqpoint{8.424878in}{2.909549in}}{\pgfqpoint{8.428444in}{2.913115in}}%
\pgfpathcurveto{\pgfqpoint{8.432010in}{2.916682in}}{\pgfqpoint{8.434014in}{2.921520in}}{\pgfqpoint{8.434014in}{2.926563in}}%
\pgfpathcurveto{\pgfqpoint{8.434014in}{2.931607in}}{\pgfqpoint{8.432010in}{2.936445in}}{\pgfqpoint{8.428444in}{2.940011in}}%
\pgfpathcurveto{\pgfqpoint{8.424878in}{2.943577in}}{\pgfqpoint{8.420040in}{2.945581in}}{\pgfqpoint{8.414996in}{2.945581in}}%
\pgfpathcurveto{\pgfqpoint{8.409952in}{2.945581in}}{\pgfqpoint{8.405115in}{2.943577in}}{\pgfqpoint{8.401548in}{2.940011in}}%
\pgfpathcurveto{\pgfqpoint{8.397982in}{2.936445in}}{\pgfqpoint{8.395978in}{2.931607in}}{\pgfqpoint{8.395978in}{2.926563in}}%
\pgfpathcurveto{\pgfqpoint{8.395978in}{2.921520in}}{\pgfqpoint{8.397982in}{2.916682in}}{\pgfqpoint{8.401548in}{2.913115in}}%
\pgfpathcurveto{\pgfqpoint{8.405115in}{2.909549in}}{\pgfqpoint{8.409952in}{2.907545in}}{\pgfqpoint{8.414996in}{2.907545in}}%
\pgfpathclose%
\pgfusepath{fill}%
\end{pgfscope}%
\begin{pgfscope}%
\pgfpathrectangle{\pgfqpoint{6.572727in}{0.474100in}}{\pgfqpoint{4.227273in}{3.318700in}}%
\pgfusepath{clip}%
\pgfsetbuttcap%
\pgfsetroundjoin%
\definecolor{currentfill}{rgb}{0.993248,0.906157,0.143936}%
\pgfsetfillcolor{currentfill}%
\pgfsetfillopacity{0.700000}%
\pgfsetlinewidth{0.000000pt}%
\definecolor{currentstroke}{rgb}{0.000000,0.000000,0.000000}%
\pgfsetstrokecolor{currentstroke}%
\pgfsetstrokeopacity{0.700000}%
\pgfsetdash{}{0pt}%
\pgfpathmoveto{\pgfqpoint{9.486798in}{1.234783in}}%
\pgfpathcurveto{\pgfqpoint{9.491841in}{1.234783in}}{\pgfqpoint{9.496679in}{1.236787in}}{\pgfqpoint{9.500246in}{1.240353in}}%
\pgfpathcurveto{\pgfqpoint{9.503812in}{1.243920in}}{\pgfqpoint{9.505816in}{1.248757in}}{\pgfqpoint{9.505816in}{1.253801in}}%
\pgfpathcurveto{\pgfqpoint{9.505816in}{1.258845in}}{\pgfqpoint{9.503812in}{1.263682in}}{\pgfqpoint{9.500246in}{1.267249in}}%
\pgfpathcurveto{\pgfqpoint{9.496679in}{1.270815in}}{\pgfqpoint{9.491841in}{1.272819in}}{\pgfqpoint{9.486798in}{1.272819in}}%
\pgfpathcurveto{\pgfqpoint{9.481754in}{1.272819in}}{\pgfqpoint{9.476916in}{1.270815in}}{\pgfqpoint{9.473350in}{1.267249in}}%
\pgfpathcurveto{\pgfqpoint{9.469783in}{1.263682in}}{\pgfqpoint{9.467780in}{1.258845in}}{\pgfqpoint{9.467780in}{1.253801in}}%
\pgfpathcurveto{\pgfqpoint{9.467780in}{1.248757in}}{\pgfqpoint{9.469783in}{1.243920in}}{\pgfqpoint{9.473350in}{1.240353in}}%
\pgfpathcurveto{\pgfqpoint{9.476916in}{1.236787in}}{\pgfqpoint{9.481754in}{1.234783in}}{\pgfqpoint{9.486798in}{1.234783in}}%
\pgfpathclose%
\pgfusepath{fill}%
\end{pgfscope}%
\begin{pgfscope}%
\pgfpathrectangle{\pgfqpoint{6.572727in}{0.474100in}}{\pgfqpoint{4.227273in}{3.318700in}}%
\pgfusepath{clip}%
\pgfsetbuttcap%
\pgfsetroundjoin%
\definecolor{currentfill}{rgb}{0.127568,0.566949,0.550556}%
\pgfsetfillcolor{currentfill}%
\pgfsetfillopacity{0.700000}%
\pgfsetlinewidth{0.000000pt}%
\definecolor{currentstroke}{rgb}{0.000000,0.000000,0.000000}%
\pgfsetstrokecolor{currentstroke}%
\pgfsetstrokeopacity{0.700000}%
\pgfsetdash{}{0pt}%
\pgfpathmoveto{\pgfqpoint{7.913414in}{1.778866in}}%
\pgfpathcurveto{\pgfqpoint{7.918458in}{1.778866in}}{\pgfqpoint{7.923295in}{1.780870in}}{\pgfqpoint{7.926862in}{1.784437in}}%
\pgfpathcurveto{\pgfqpoint{7.930428in}{1.788003in}}{\pgfqpoint{7.932432in}{1.792841in}}{\pgfqpoint{7.932432in}{1.797884in}}%
\pgfpathcurveto{\pgfqpoint{7.932432in}{1.802928in}}{\pgfqpoint{7.930428in}{1.807766in}}{\pgfqpoint{7.926862in}{1.811332in}}%
\pgfpathcurveto{\pgfqpoint{7.923295in}{1.814899in}}{\pgfqpoint{7.918458in}{1.816903in}}{\pgfqpoint{7.913414in}{1.816903in}}%
\pgfpathcurveto{\pgfqpoint{7.908370in}{1.816903in}}{\pgfqpoint{7.903532in}{1.814899in}}{\pgfqpoint{7.899966in}{1.811332in}}%
\pgfpathcurveto{\pgfqpoint{7.896400in}{1.807766in}}{\pgfqpoint{7.894396in}{1.802928in}}{\pgfqpoint{7.894396in}{1.797884in}}%
\pgfpathcurveto{\pgfqpoint{7.894396in}{1.792841in}}{\pgfqpoint{7.896400in}{1.788003in}}{\pgfqpoint{7.899966in}{1.784437in}}%
\pgfpathcurveto{\pgfqpoint{7.903532in}{1.780870in}}{\pgfqpoint{7.908370in}{1.778866in}}{\pgfqpoint{7.913414in}{1.778866in}}%
\pgfpathclose%
\pgfusepath{fill}%
\end{pgfscope}%
\begin{pgfscope}%
\pgfpathrectangle{\pgfqpoint{6.572727in}{0.474100in}}{\pgfqpoint{4.227273in}{3.318700in}}%
\pgfusepath{clip}%
\pgfsetbuttcap%
\pgfsetroundjoin%
\definecolor{currentfill}{rgb}{0.127568,0.566949,0.550556}%
\pgfsetfillcolor{currentfill}%
\pgfsetfillopacity{0.700000}%
\pgfsetlinewidth{0.000000pt}%
\definecolor{currentstroke}{rgb}{0.000000,0.000000,0.000000}%
\pgfsetstrokecolor{currentstroke}%
\pgfsetstrokeopacity{0.700000}%
\pgfsetdash{}{0pt}%
\pgfpathmoveto{\pgfqpoint{7.821927in}{2.085040in}}%
\pgfpathcurveto{\pgfqpoint{7.826971in}{2.085040in}}{\pgfqpoint{7.831808in}{2.087044in}}{\pgfqpoint{7.835375in}{2.090610in}}%
\pgfpathcurveto{\pgfqpoint{7.838941in}{2.094176in}}{\pgfqpoint{7.840945in}{2.099014in}}{\pgfqpoint{7.840945in}{2.104058in}}%
\pgfpathcurveto{\pgfqpoint{7.840945in}{2.109101in}}{\pgfqpoint{7.838941in}{2.113939in}}{\pgfqpoint{7.835375in}{2.117506in}}%
\pgfpathcurveto{\pgfqpoint{7.831808in}{2.121072in}}{\pgfqpoint{7.826971in}{2.123076in}}{\pgfqpoint{7.821927in}{2.123076in}}%
\pgfpathcurveto{\pgfqpoint{7.816883in}{2.123076in}}{\pgfqpoint{7.812045in}{2.121072in}}{\pgfqpoint{7.808479in}{2.117506in}}%
\pgfpathcurveto{\pgfqpoint{7.804913in}{2.113939in}}{\pgfqpoint{7.802909in}{2.109101in}}{\pgfqpoint{7.802909in}{2.104058in}}%
\pgfpathcurveto{\pgfqpoint{7.802909in}{2.099014in}}{\pgfqpoint{7.804913in}{2.094176in}}{\pgfqpoint{7.808479in}{2.090610in}}%
\pgfpathcurveto{\pgfqpoint{7.812045in}{2.087044in}}{\pgfqpoint{7.816883in}{2.085040in}}{\pgfqpoint{7.821927in}{2.085040in}}%
\pgfpathclose%
\pgfusepath{fill}%
\end{pgfscope}%
\begin{pgfscope}%
\pgfpathrectangle{\pgfqpoint{6.572727in}{0.474100in}}{\pgfqpoint{4.227273in}{3.318700in}}%
\pgfusepath{clip}%
\pgfsetbuttcap%
\pgfsetroundjoin%
\definecolor{currentfill}{rgb}{0.127568,0.566949,0.550556}%
\pgfsetfillcolor{currentfill}%
\pgfsetfillopacity{0.700000}%
\pgfsetlinewidth{0.000000pt}%
\definecolor{currentstroke}{rgb}{0.000000,0.000000,0.000000}%
\pgfsetstrokecolor{currentstroke}%
\pgfsetstrokeopacity{0.700000}%
\pgfsetdash{}{0pt}%
\pgfpathmoveto{\pgfqpoint{8.164135in}{2.450441in}}%
\pgfpathcurveto{\pgfqpoint{8.169179in}{2.450441in}}{\pgfqpoint{8.174016in}{2.452445in}}{\pgfqpoint{8.177583in}{2.456011in}}%
\pgfpathcurveto{\pgfqpoint{8.181149in}{2.459578in}}{\pgfqpoint{8.183153in}{2.464415in}}{\pgfqpoint{8.183153in}{2.469459in}}%
\pgfpathcurveto{\pgfqpoint{8.183153in}{2.474503in}}{\pgfqpoint{8.181149in}{2.479340in}}{\pgfqpoint{8.177583in}{2.482907in}}%
\pgfpathcurveto{\pgfqpoint{8.174016in}{2.486473in}}{\pgfqpoint{8.169179in}{2.488477in}}{\pgfqpoint{8.164135in}{2.488477in}}%
\pgfpathcurveto{\pgfqpoint{8.159091in}{2.488477in}}{\pgfqpoint{8.154253in}{2.486473in}}{\pgfqpoint{8.150687in}{2.482907in}}%
\pgfpathcurveto{\pgfqpoint{8.147121in}{2.479340in}}{\pgfqpoint{8.145117in}{2.474503in}}{\pgfqpoint{8.145117in}{2.469459in}}%
\pgfpathcurveto{\pgfqpoint{8.145117in}{2.464415in}}{\pgfqpoint{8.147121in}{2.459578in}}{\pgfqpoint{8.150687in}{2.456011in}}%
\pgfpathcurveto{\pgfqpoint{8.154253in}{2.452445in}}{\pgfqpoint{8.159091in}{2.450441in}}{\pgfqpoint{8.164135in}{2.450441in}}%
\pgfpathclose%
\pgfusepath{fill}%
\end{pgfscope}%
\begin{pgfscope}%
\pgfpathrectangle{\pgfqpoint{6.572727in}{0.474100in}}{\pgfqpoint{4.227273in}{3.318700in}}%
\pgfusepath{clip}%
\pgfsetbuttcap%
\pgfsetroundjoin%
\definecolor{currentfill}{rgb}{0.127568,0.566949,0.550556}%
\pgfsetfillcolor{currentfill}%
\pgfsetfillopacity{0.700000}%
\pgfsetlinewidth{0.000000pt}%
\definecolor{currentstroke}{rgb}{0.000000,0.000000,0.000000}%
\pgfsetstrokecolor{currentstroke}%
\pgfsetstrokeopacity{0.700000}%
\pgfsetdash{}{0pt}%
\pgfpathmoveto{\pgfqpoint{7.910866in}{2.671667in}}%
\pgfpathcurveto{\pgfqpoint{7.915910in}{2.671667in}}{\pgfqpoint{7.920748in}{2.673671in}}{\pgfqpoint{7.924314in}{2.677237in}}%
\pgfpathcurveto{\pgfqpoint{7.927881in}{2.680804in}}{\pgfqpoint{7.929885in}{2.685641in}}{\pgfqpoint{7.929885in}{2.690685in}}%
\pgfpathcurveto{\pgfqpoint{7.929885in}{2.695729in}}{\pgfqpoint{7.927881in}{2.700566in}}{\pgfqpoint{7.924314in}{2.704133in}}%
\pgfpathcurveto{\pgfqpoint{7.920748in}{2.707699in}}{\pgfqpoint{7.915910in}{2.709703in}}{\pgfqpoint{7.910866in}{2.709703in}}%
\pgfpathcurveto{\pgfqpoint{7.905823in}{2.709703in}}{\pgfqpoint{7.900985in}{2.707699in}}{\pgfqpoint{7.897419in}{2.704133in}}%
\pgfpathcurveto{\pgfqpoint{7.893852in}{2.700566in}}{\pgfqpoint{7.891848in}{2.695729in}}{\pgfqpoint{7.891848in}{2.690685in}}%
\pgfpathcurveto{\pgfqpoint{7.891848in}{2.685641in}}{\pgfqpoint{7.893852in}{2.680804in}}{\pgfqpoint{7.897419in}{2.677237in}}%
\pgfpathcurveto{\pgfqpoint{7.900985in}{2.673671in}}{\pgfqpoint{7.905823in}{2.671667in}}{\pgfqpoint{7.910866in}{2.671667in}}%
\pgfpathclose%
\pgfusepath{fill}%
\end{pgfscope}%
\begin{pgfscope}%
\pgfpathrectangle{\pgfqpoint{6.572727in}{0.474100in}}{\pgfqpoint{4.227273in}{3.318700in}}%
\pgfusepath{clip}%
\pgfsetbuttcap%
\pgfsetroundjoin%
\definecolor{currentfill}{rgb}{0.127568,0.566949,0.550556}%
\pgfsetfillcolor{currentfill}%
\pgfsetfillopacity{0.700000}%
\pgfsetlinewidth{0.000000pt}%
\definecolor{currentstroke}{rgb}{0.000000,0.000000,0.000000}%
\pgfsetstrokecolor{currentstroke}%
\pgfsetstrokeopacity{0.700000}%
\pgfsetdash{}{0pt}%
\pgfpathmoveto{\pgfqpoint{7.957590in}{2.138552in}}%
\pgfpathcurveto{\pgfqpoint{7.962633in}{2.138552in}}{\pgfqpoint{7.967471in}{2.140556in}}{\pgfqpoint{7.971037in}{2.144122in}}%
\pgfpathcurveto{\pgfqpoint{7.974604in}{2.147689in}}{\pgfqpoint{7.976608in}{2.152527in}}{\pgfqpoint{7.976608in}{2.157570in}}%
\pgfpathcurveto{\pgfqpoint{7.976608in}{2.162614in}}{\pgfqpoint{7.974604in}{2.167452in}}{\pgfqpoint{7.971037in}{2.171018in}}%
\pgfpathcurveto{\pgfqpoint{7.967471in}{2.174585in}}{\pgfqpoint{7.962633in}{2.176588in}}{\pgfqpoint{7.957590in}{2.176588in}}%
\pgfpathcurveto{\pgfqpoint{7.952546in}{2.176588in}}{\pgfqpoint{7.947708in}{2.174585in}}{\pgfqpoint{7.944142in}{2.171018in}}%
\pgfpathcurveto{\pgfqpoint{7.940575in}{2.167452in}}{\pgfqpoint{7.938571in}{2.162614in}}{\pgfqpoint{7.938571in}{2.157570in}}%
\pgfpathcurveto{\pgfqpoint{7.938571in}{2.152527in}}{\pgfqpoint{7.940575in}{2.147689in}}{\pgfqpoint{7.944142in}{2.144122in}}%
\pgfpathcurveto{\pgfqpoint{7.947708in}{2.140556in}}{\pgfqpoint{7.952546in}{2.138552in}}{\pgfqpoint{7.957590in}{2.138552in}}%
\pgfpathclose%
\pgfusepath{fill}%
\end{pgfscope}%
\begin{pgfscope}%
\pgfpathrectangle{\pgfqpoint{6.572727in}{0.474100in}}{\pgfqpoint{4.227273in}{3.318700in}}%
\pgfusepath{clip}%
\pgfsetbuttcap%
\pgfsetroundjoin%
\definecolor{currentfill}{rgb}{0.993248,0.906157,0.143936}%
\pgfsetfillcolor{currentfill}%
\pgfsetfillopacity{0.700000}%
\pgfsetlinewidth{0.000000pt}%
\definecolor{currentstroke}{rgb}{0.000000,0.000000,0.000000}%
\pgfsetstrokecolor{currentstroke}%
\pgfsetstrokeopacity{0.700000}%
\pgfsetdash{}{0pt}%
\pgfpathmoveto{\pgfqpoint{10.137840in}{1.609582in}}%
\pgfpathcurveto{\pgfqpoint{10.142883in}{1.609582in}}{\pgfqpoint{10.147721in}{1.611586in}}{\pgfqpoint{10.151288in}{1.615152in}}%
\pgfpathcurveto{\pgfqpoint{10.154854in}{1.618718in}}{\pgfqpoint{10.156858in}{1.623556in}}{\pgfqpoint{10.156858in}{1.628600in}}%
\pgfpathcurveto{\pgfqpoint{10.156858in}{1.633644in}}{\pgfqpoint{10.154854in}{1.638481in}}{\pgfqpoint{10.151288in}{1.642048in}}%
\pgfpathcurveto{\pgfqpoint{10.147721in}{1.645614in}}{\pgfqpoint{10.142883in}{1.647618in}}{\pgfqpoint{10.137840in}{1.647618in}}%
\pgfpathcurveto{\pgfqpoint{10.132796in}{1.647618in}}{\pgfqpoint{10.127958in}{1.645614in}}{\pgfqpoint{10.124392in}{1.642048in}}%
\pgfpathcurveto{\pgfqpoint{10.120825in}{1.638481in}}{\pgfqpoint{10.118822in}{1.633644in}}{\pgfqpoint{10.118822in}{1.628600in}}%
\pgfpathcurveto{\pgfqpoint{10.118822in}{1.623556in}}{\pgfqpoint{10.120825in}{1.618718in}}{\pgfqpoint{10.124392in}{1.615152in}}%
\pgfpathcurveto{\pgfqpoint{10.127958in}{1.611586in}}{\pgfqpoint{10.132796in}{1.609582in}}{\pgfqpoint{10.137840in}{1.609582in}}%
\pgfpathclose%
\pgfusepath{fill}%
\end{pgfscope}%
\begin{pgfscope}%
\pgfpathrectangle{\pgfqpoint{6.572727in}{0.474100in}}{\pgfqpoint{4.227273in}{3.318700in}}%
\pgfusepath{clip}%
\pgfsetbuttcap%
\pgfsetroundjoin%
\definecolor{currentfill}{rgb}{0.127568,0.566949,0.550556}%
\pgfsetfillcolor{currentfill}%
\pgfsetfillopacity{0.700000}%
\pgfsetlinewidth{0.000000pt}%
\definecolor{currentstroke}{rgb}{0.000000,0.000000,0.000000}%
\pgfsetstrokecolor{currentstroke}%
\pgfsetstrokeopacity{0.700000}%
\pgfsetdash{}{0pt}%
\pgfpathmoveto{\pgfqpoint{7.864466in}{1.589054in}}%
\pgfpathcurveto{\pgfqpoint{7.869509in}{1.589054in}}{\pgfqpoint{7.874347in}{1.591058in}}{\pgfqpoint{7.877913in}{1.594625in}}%
\pgfpathcurveto{\pgfqpoint{7.881480in}{1.598191in}}{\pgfqpoint{7.883484in}{1.603029in}}{\pgfqpoint{7.883484in}{1.608073in}}%
\pgfpathcurveto{\pgfqpoint{7.883484in}{1.613116in}}{\pgfqpoint{7.881480in}{1.617954in}}{\pgfqpoint{7.877913in}{1.621520in}}%
\pgfpathcurveto{\pgfqpoint{7.874347in}{1.625087in}}{\pgfqpoint{7.869509in}{1.627091in}}{\pgfqpoint{7.864466in}{1.627091in}}%
\pgfpathcurveto{\pgfqpoint{7.859422in}{1.627091in}}{\pgfqpoint{7.854584in}{1.625087in}}{\pgfqpoint{7.851018in}{1.621520in}}%
\pgfpathcurveto{\pgfqpoint{7.847451in}{1.617954in}}{\pgfqpoint{7.845447in}{1.613116in}}{\pgfqpoint{7.845447in}{1.608073in}}%
\pgfpathcurveto{\pgfqpoint{7.845447in}{1.603029in}}{\pgfqpoint{7.847451in}{1.598191in}}{\pgfqpoint{7.851018in}{1.594625in}}%
\pgfpathcurveto{\pgfqpoint{7.854584in}{1.591058in}}{\pgfqpoint{7.859422in}{1.589054in}}{\pgfqpoint{7.864466in}{1.589054in}}%
\pgfpathclose%
\pgfusepath{fill}%
\end{pgfscope}%
\begin{pgfscope}%
\pgfpathrectangle{\pgfqpoint{6.572727in}{0.474100in}}{\pgfqpoint{4.227273in}{3.318700in}}%
\pgfusepath{clip}%
\pgfsetbuttcap%
\pgfsetroundjoin%
\definecolor{currentfill}{rgb}{0.993248,0.906157,0.143936}%
\pgfsetfillcolor{currentfill}%
\pgfsetfillopacity{0.700000}%
\pgfsetlinewidth{0.000000pt}%
\definecolor{currentstroke}{rgb}{0.000000,0.000000,0.000000}%
\pgfsetstrokecolor{currentstroke}%
\pgfsetstrokeopacity{0.700000}%
\pgfsetdash{}{0pt}%
\pgfpathmoveto{\pgfqpoint{9.629775in}{1.528945in}}%
\pgfpathcurveto{\pgfqpoint{9.634819in}{1.528945in}}{\pgfqpoint{9.639657in}{1.530949in}}{\pgfqpoint{9.643223in}{1.534515in}}%
\pgfpathcurveto{\pgfqpoint{9.646790in}{1.538081in}}{\pgfqpoint{9.648793in}{1.542919in}}{\pgfqpoint{9.648793in}{1.547963in}}%
\pgfpathcurveto{\pgfqpoint{9.648793in}{1.553006in}}{\pgfqpoint{9.646790in}{1.557844in}}{\pgfqpoint{9.643223in}{1.561411in}}%
\pgfpathcurveto{\pgfqpoint{9.639657in}{1.564977in}}{\pgfqpoint{9.634819in}{1.566981in}}{\pgfqpoint{9.629775in}{1.566981in}}%
\pgfpathcurveto{\pgfqpoint{9.624732in}{1.566981in}}{\pgfqpoint{9.619894in}{1.564977in}}{\pgfqpoint{9.616327in}{1.561411in}}%
\pgfpathcurveto{\pgfqpoint{9.612761in}{1.557844in}}{\pgfqpoint{9.610757in}{1.553006in}}{\pgfqpoint{9.610757in}{1.547963in}}%
\pgfpathcurveto{\pgfqpoint{9.610757in}{1.542919in}}{\pgfqpoint{9.612761in}{1.538081in}}{\pgfqpoint{9.616327in}{1.534515in}}%
\pgfpathcurveto{\pgfqpoint{9.619894in}{1.530949in}}{\pgfqpoint{9.624732in}{1.528945in}}{\pgfqpoint{9.629775in}{1.528945in}}%
\pgfpathclose%
\pgfusepath{fill}%
\end{pgfscope}%
\begin{pgfscope}%
\pgfpathrectangle{\pgfqpoint{6.572727in}{0.474100in}}{\pgfqpoint{4.227273in}{3.318700in}}%
\pgfusepath{clip}%
\pgfsetbuttcap%
\pgfsetroundjoin%
\definecolor{currentfill}{rgb}{0.993248,0.906157,0.143936}%
\pgfsetfillcolor{currentfill}%
\pgfsetfillopacity{0.700000}%
\pgfsetlinewidth{0.000000pt}%
\definecolor{currentstroke}{rgb}{0.000000,0.000000,0.000000}%
\pgfsetstrokecolor{currentstroke}%
\pgfsetstrokeopacity{0.700000}%
\pgfsetdash{}{0pt}%
\pgfpathmoveto{\pgfqpoint{9.785555in}{1.629689in}}%
\pgfpathcurveto{\pgfqpoint{9.790598in}{1.629689in}}{\pgfqpoint{9.795436in}{1.631693in}}{\pgfqpoint{9.799002in}{1.635260in}}%
\pgfpathcurveto{\pgfqpoint{9.802569in}{1.638826in}}{\pgfqpoint{9.804573in}{1.643664in}}{\pgfqpoint{9.804573in}{1.648708in}}%
\pgfpathcurveto{\pgfqpoint{9.804573in}{1.653751in}}{\pgfqpoint{9.802569in}{1.658589in}}{\pgfqpoint{9.799002in}{1.662155in}}%
\pgfpathcurveto{\pgfqpoint{9.795436in}{1.665722in}}{\pgfqpoint{9.790598in}{1.667726in}}{\pgfqpoint{9.785555in}{1.667726in}}%
\pgfpathcurveto{\pgfqpoint{9.780511in}{1.667726in}}{\pgfqpoint{9.775673in}{1.665722in}}{\pgfqpoint{9.772107in}{1.662155in}}%
\pgfpathcurveto{\pgfqpoint{9.768540in}{1.658589in}}{\pgfqpoint{9.766536in}{1.653751in}}{\pgfqpoint{9.766536in}{1.648708in}}%
\pgfpathcurveto{\pgfqpoint{9.766536in}{1.643664in}}{\pgfqpoint{9.768540in}{1.638826in}}{\pgfqpoint{9.772107in}{1.635260in}}%
\pgfpathcurveto{\pgfqpoint{9.775673in}{1.631693in}}{\pgfqpoint{9.780511in}{1.629689in}}{\pgfqpoint{9.785555in}{1.629689in}}%
\pgfpathclose%
\pgfusepath{fill}%
\end{pgfscope}%
\begin{pgfscope}%
\pgfpathrectangle{\pgfqpoint{6.572727in}{0.474100in}}{\pgfqpoint{4.227273in}{3.318700in}}%
\pgfusepath{clip}%
\pgfsetbuttcap%
\pgfsetroundjoin%
\definecolor{currentfill}{rgb}{0.993248,0.906157,0.143936}%
\pgfsetfillcolor{currentfill}%
\pgfsetfillopacity{0.700000}%
\pgfsetlinewidth{0.000000pt}%
\definecolor{currentstroke}{rgb}{0.000000,0.000000,0.000000}%
\pgfsetstrokecolor{currentstroke}%
\pgfsetstrokeopacity{0.700000}%
\pgfsetdash{}{0pt}%
\pgfpathmoveto{\pgfqpoint{9.315914in}{1.238980in}}%
\pgfpathcurveto{\pgfqpoint{9.320957in}{1.238980in}}{\pgfqpoint{9.325795in}{1.240983in}}{\pgfqpoint{9.329362in}{1.244550in}}%
\pgfpathcurveto{\pgfqpoint{9.332928in}{1.248116in}}{\pgfqpoint{9.334932in}{1.252954in}}{\pgfqpoint{9.334932in}{1.257998in}}%
\pgfpathcurveto{\pgfqpoint{9.334932in}{1.263041in}}{\pgfqpoint{9.332928in}{1.267879in}}{\pgfqpoint{9.329362in}{1.271446in}}%
\pgfpathcurveto{\pgfqpoint{9.325795in}{1.275012in}}{\pgfqpoint{9.320957in}{1.277016in}}{\pgfqpoint{9.315914in}{1.277016in}}%
\pgfpathcurveto{\pgfqpoint{9.310870in}{1.277016in}}{\pgfqpoint{9.306032in}{1.275012in}}{\pgfqpoint{9.302466in}{1.271446in}}%
\pgfpathcurveto{\pgfqpoint{9.298899in}{1.267879in}}{\pgfqpoint{9.296896in}{1.263041in}}{\pgfqpoint{9.296896in}{1.257998in}}%
\pgfpathcurveto{\pgfqpoint{9.296896in}{1.252954in}}{\pgfqpoint{9.298899in}{1.248116in}}{\pgfqpoint{9.302466in}{1.244550in}}%
\pgfpathcurveto{\pgfqpoint{9.306032in}{1.240983in}}{\pgfqpoint{9.310870in}{1.238980in}}{\pgfqpoint{9.315914in}{1.238980in}}%
\pgfpathclose%
\pgfusepath{fill}%
\end{pgfscope}%
\begin{pgfscope}%
\pgfpathrectangle{\pgfqpoint{6.572727in}{0.474100in}}{\pgfqpoint{4.227273in}{3.318700in}}%
\pgfusepath{clip}%
\pgfsetbuttcap%
\pgfsetroundjoin%
\definecolor{currentfill}{rgb}{0.127568,0.566949,0.550556}%
\pgfsetfillcolor{currentfill}%
\pgfsetfillopacity{0.700000}%
\pgfsetlinewidth{0.000000pt}%
\definecolor{currentstroke}{rgb}{0.000000,0.000000,0.000000}%
\pgfsetstrokecolor{currentstroke}%
\pgfsetstrokeopacity{0.700000}%
\pgfsetdash{}{0pt}%
\pgfpathmoveto{\pgfqpoint{8.526535in}{2.933432in}}%
\pgfpathcurveto{\pgfqpoint{8.531579in}{2.933432in}}{\pgfqpoint{8.536417in}{2.935436in}}{\pgfqpoint{8.539983in}{2.939002in}}%
\pgfpathcurveto{\pgfqpoint{8.543550in}{2.942569in}}{\pgfqpoint{8.545553in}{2.947406in}}{\pgfqpoint{8.545553in}{2.952450in}}%
\pgfpathcurveto{\pgfqpoint{8.545553in}{2.957494in}}{\pgfqpoint{8.543550in}{2.962331in}}{\pgfqpoint{8.539983in}{2.965898in}}%
\pgfpathcurveto{\pgfqpoint{8.536417in}{2.969464in}}{\pgfqpoint{8.531579in}{2.971468in}}{\pgfqpoint{8.526535in}{2.971468in}}%
\pgfpathcurveto{\pgfqpoint{8.521492in}{2.971468in}}{\pgfqpoint{8.516654in}{2.969464in}}{\pgfqpoint{8.513087in}{2.965898in}}%
\pgfpathcurveto{\pgfqpoint{8.509521in}{2.962331in}}{\pgfqpoint{8.507517in}{2.957494in}}{\pgfqpoint{8.507517in}{2.952450in}}%
\pgfpathcurveto{\pgfqpoint{8.507517in}{2.947406in}}{\pgfqpoint{8.509521in}{2.942569in}}{\pgfqpoint{8.513087in}{2.939002in}}%
\pgfpathcurveto{\pgfqpoint{8.516654in}{2.935436in}}{\pgfqpoint{8.521492in}{2.933432in}}{\pgfqpoint{8.526535in}{2.933432in}}%
\pgfpathclose%
\pgfusepath{fill}%
\end{pgfscope}%
\begin{pgfscope}%
\pgfpathrectangle{\pgfqpoint{6.572727in}{0.474100in}}{\pgfqpoint{4.227273in}{3.318700in}}%
\pgfusepath{clip}%
\pgfsetbuttcap%
\pgfsetroundjoin%
\definecolor{currentfill}{rgb}{0.127568,0.566949,0.550556}%
\pgfsetfillcolor{currentfill}%
\pgfsetfillopacity{0.700000}%
\pgfsetlinewidth{0.000000pt}%
\definecolor{currentstroke}{rgb}{0.000000,0.000000,0.000000}%
\pgfsetstrokecolor{currentstroke}%
\pgfsetstrokeopacity{0.700000}%
\pgfsetdash{}{0pt}%
\pgfpathmoveto{\pgfqpoint{7.632062in}{1.188426in}}%
\pgfpathcurveto{\pgfqpoint{7.637105in}{1.188426in}}{\pgfqpoint{7.641943in}{1.190430in}}{\pgfqpoint{7.645510in}{1.193996in}}%
\pgfpathcurveto{\pgfqpoint{7.649076in}{1.197563in}}{\pgfqpoint{7.651080in}{1.202400in}}{\pgfqpoint{7.651080in}{1.207444in}}%
\pgfpathcurveto{\pgfqpoint{7.651080in}{1.212488in}}{\pgfqpoint{7.649076in}{1.217325in}}{\pgfqpoint{7.645510in}{1.220892in}}%
\pgfpathcurveto{\pgfqpoint{7.641943in}{1.224458in}}{\pgfqpoint{7.637105in}{1.226462in}}{\pgfqpoint{7.632062in}{1.226462in}}%
\pgfpathcurveto{\pgfqpoint{7.627018in}{1.226462in}}{\pgfqpoint{7.622180in}{1.224458in}}{\pgfqpoint{7.618614in}{1.220892in}}%
\pgfpathcurveto{\pgfqpoint{7.615047in}{1.217325in}}{\pgfqpoint{7.613044in}{1.212488in}}{\pgfqpoint{7.613044in}{1.207444in}}%
\pgfpathcurveto{\pgfqpoint{7.613044in}{1.202400in}}{\pgfqpoint{7.615047in}{1.197563in}}{\pgfqpoint{7.618614in}{1.193996in}}%
\pgfpathcurveto{\pgfqpoint{7.622180in}{1.190430in}}{\pgfqpoint{7.627018in}{1.188426in}}{\pgfqpoint{7.632062in}{1.188426in}}%
\pgfpathclose%
\pgfusepath{fill}%
\end{pgfscope}%
\begin{pgfscope}%
\pgfpathrectangle{\pgfqpoint{6.572727in}{0.474100in}}{\pgfqpoint{4.227273in}{3.318700in}}%
\pgfusepath{clip}%
\pgfsetbuttcap%
\pgfsetroundjoin%
\definecolor{currentfill}{rgb}{0.993248,0.906157,0.143936}%
\pgfsetfillcolor{currentfill}%
\pgfsetfillopacity{0.700000}%
\pgfsetlinewidth{0.000000pt}%
\definecolor{currentstroke}{rgb}{0.000000,0.000000,0.000000}%
\pgfsetstrokecolor{currentstroke}%
\pgfsetstrokeopacity{0.700000}%
\pgfsetdash{}{0pt}%
\pgfpathmoveto{\pgfqpoint{9.806089in}{1.309000in}}%
\pgfpathcurveto{\pgfqpoint{9.811133in}{1.309000in}}{\pgfqpoint{9.815971in}{1.311004in}}{\pgfqpoint{9.819537in}{1.314570in}}%
\pgfpathcurveto{\pgfqpoint{9.823103in}{1.318137in}}{\pgfqpoint{9.825107in}{1.322975in}}{\pgfqpoint{9.825107in}{1.328018in}}%
\pgfpathcurveto{\pgfqpoint{9.825107in}{1.333062in}}{\pgfqpoint{9.823103in}{1.337900in}}{\pgfqpoint{9.819537in}{1.341466in}}%
\pgfpathcurveto{\pgfqpoint{9.815971in}{1.345033in}}{\pgfqpoint{9.811133in}{1.347036in}}{\pgfqpoint{9.806089in}{1.347036in}}%
\pgfpathcurveto{\pgfqpoint{9.801045in}{1.347036in}}{\pgfqpoint{9.796208in}{1.345033in}}{\pgfqpoint{9.792641in}{1.341466in}}%
\pgfpathcurveto{\pgfqpoint{9.789075in}{1.337900in}}{\pgfqpoint{9.787071in}{1.333062in}}{\pgfqpoint{9.787071in}{1.328018in}}%
\pgfpathcurveto{\pgfqpoint{9.787071in}{1.322975in}}{\pgfqpoint{9.789075in}{1.318137in}}{\pgfqpoint{9.792641in}{1.314570in}}%
\pgfpathcurveto{\pgfqpoint{9.796208in}{1.311004in}}{\pgfqpoint{9.801045in}{1.309000in}}{\pgfqpoint{9.806089in}{1.309000in}}%
\pgfpathclose%
\pgfusepath{fill}%
\end{pgfscope}%
\begin{pgfscope}%
\pgfpathrectangle{\pgfqpoint{6.572727in}{0.474100in}}{\pgfqpoint{4.227273in}{3.318700in}}%
\pgfusepath{clip}%
\pgfsetbuttcap%
\pgfsetroundjoin%
\definecolor{currentfill}{rgb}{0.127568,0.566949,0.550556}%
\pgfsetfillcolor{currentfill}%
\pgfsetfillopacity{0.700000}%
\pgfsetlinewidth{0.000000pt}%
\definecolor{currentstroke}{rgb}{0.000000,0.000000,0.000000}%
\pgfsetstrokecolor{currentstroke}%
\pgfsetstrokeopacity{0.700000}%
\pgfsetdash{}{0pt}%
\pgfpathmoveto{\pgfqpoint{7.583548in}{1.188066in}}%
\pgfpathcurveto{\pgfqpoint{7.588592in}{1.188066in}}{\pgfqpoint{7.593429in}{1.190070in}}{\pgfqpoint{7.596996in}{1.193636in}}%
\pgfpathcurveto{\pgfqpoint{7.600562in}{1.197202in}}{\pgfqpoint{7.602566in}{1.202040in}}{\pgfqpoint{7.602566in}{1.207084in}}%
\pgfpathcurveto{\pgfqpoint{7.602566in}{1.212128in}}{\pgfqpoint{7.600562in}{1.216965in}}{\pgfqpoint{7.596996in}{1.220532in}}%
\pgfpathcurveto{\pgfqpoint{7.593429in}{1.224098in}}{\pgfqpoint{7.588592in}{1.226102in}}{\pgfqpoint{7.583548in}{1.226102in}}%
\pgfpathcurveto{\pgfqpoint{7.578504in}{1.226102in}}{\pgfqpoint{7.573666in}{1.224098in}}{\pgfqpoint{7.570100in}{1.220532in}}%
\pgfpathcurveto{\pgfqpoint{7.566534in}{1.216965in}}{\pgfqpoint{7.564530in}{1.212128in}}{\pgfqpoint{7.564530in}{1.207084in}}%
\pgfpathcurveto{\pgfqpoint{7.564530in}{1.202040in}}{\pgfqpoint{7.566534in}{1.197202in}}{\pgfqpoint{7.570100in}{1.193636in}}%
\pgfpathcurveto{\pgfqpoint{7.573666in}{1.190070in}}{\pgfqpoint{7.578504in}{1.188066in}}{\pgfqpoint{7.583548in}{1.188066in}}%
\pgfpathclose%
\pgfusepath{fill}%
\end{pgfscope}%
\begin{pgfscope}%
\pgfpathrectangle{\pgfqpoint{6.572727in}{0.474100in}}{\pgfqpoint{4.227273in}{3.318700in}}%
\pgfusepath{clip}%
\pgfsetbuttcap%
\pgfsetroundjoin%
\definecolor{currentfill}{rgb}{0.127568,0.566949,0.550556}%
\pgfsetfillcolor{currentfill}%
\pgfsetfillopacity{0.700000}%
\pgfsetlinewidth{0.000000pt}%
\definecolor{currentstroke}{rgb}{0.000000,0.000000,0.000000}%
\pgfsetstrokecolor{currentstroke}%
\pgfsetstrokeopacity{0.700000}%
\pgfsetdash{}{0pt}%
\pgfpathmoveto{\pgfqpoint{7.070485in}{1.250366in}}%
\pgfpathcurveto{\pgfqpoint{7.075528in}{1.250366in}}{\pgfqpoint{7.080366in}{1.252370in}}{\pgfqpoint{7.083933in}{1.255937in}}%
\pgfpathcurveto{\pgfqpoint{7.087499in}{1.259503in}}{\pgfqpoint{7.089503in}{1.264341in}}{\pgfqpoint{7.089503in}{1.269385in}}%
\pgfpathcurveto{\pgfqpoint{7.089503in}{1.274428in}}{\pgfqpoint{7.087499in}{1.279266in}}{\pgfqpoint{7.083933in}{1.282832in}}%
\pgfpathcurveto{\pgfqpoint{7.080366in}{1.286399in}}{\pgfqpoint{7.075528in}{1.288403in}}{\pgfqpoint{7.070485in}{1.288403in}}%
\pgfpathcurveto{\pgfqpoint{7.065441in}{1.288403in}}{\pgfqpoint{7.060603in}{1.286399in}}{\pgfqpoint{7.057037in}{1.282832in}}%
\pgfpathcurveto{\pgfqpoint{7.053470in}{1.279266in}}{\pgfqpoint{7.051467in}{1.274428in}}{\pgfqpoint{7.051467in}{1.269385in}}%
\pgfpathcurveto{\pgfqpoint{7.051467in}{1.264341in}}{\pgfqpoint{7.053470in}{1.259503in}}{\pgfqpoint{7.057037in}{1.255937in}}%
\pgfpathcurveto{\pgfqpoint{7.060603in}{1.252370in}}{\pgfqpoint{7.065441in}{1.250366in}}{\pgfqpoint{7.070485in}{1.250366in}}%
\pgfpathclose%
\pgfusepath{fill}%
\end{pgfscope}%
\begin{pgfscope}%
\pgfpathrectangle{\pgfqpoint{6.572727in}{0.474100in}}{\pgfqpoint{4.227273in}{3.318700in}}%
\pgfusepath{clip}%
\pgfsetbuttcap%
\pgfsetroundjoin%
\definecolor{currentfill}{rgb}{0.127568,0.566949,0.550556}%
\pgfsetfillcolor{currentfill}%
\pgfsetfillopacity{0.700000}%
\pgfsetlinewidth{0.000000pt}%
\definecolor{currentstroke}{rgb}{0.000000,0.000000,0.000000}%
\pgfsetstrokecolor{currentstroke}%
\pgfsetstrokeopacity{0.700000}%
\pgfsetdash{}{0pt}%
\pgfpathmoveto{\pgfqpoint{7.793212in}{1.589804in}}%
\pgfpathcurveto{\pgfqpoint{7.798255in}{1.589804in}}{\pgfqpoint{7.803093in}{1.591808in}}{\pgfqpoint{7.806660in}{1.595374in}}%
\pgfpathcurveto{\pgfqpoint{7.810226in}{1.598941in}}{\pgfqpoint{7.812230in}{1.603778in}}{\pgfqpoint{7.812230in}{1.608822in}}%
\pgfpathcurveto{\pgfqpoint{7.812230in}{1.613866in}}{\pgfqpoint{7.810226in}{1.618704in}}{\pgfqpoint{7.806660in}{1.622270in}}%
\pgfpathcurveto{\pgfqpoint{7.803093in}{1.625836in}}{\pgfqpoint{7.798255in}{1.627840in}}{\pgfqpoint{7.793212in}{1.627840in}}%
\pgfpathcurveto{\pgfqpoint{7.788168in}{1.627840in}}{\pgfqpoint{7.783330in}{1.625836in}}{\pgfqpoint{7.779764in}{1.622270in}}%
\pgfpathcurveto{\pgfqpoint{7.776197in}{1.618704in}}{\pgfqpoint{7.774194in}{1.613866in}}{\pgfqpoint{7.774194in}{1.608822in}}%
\pgfpathcurveto{\pgfqpoint{7.774194in}{1.603778in}}{\pgfqpoint{7.776197in}{1.598941in}}{\pgfqpoint{7.779764in}{1.595374in}}%
\pgfpathcurveto{\pgfqpoint{7.783330in}{1.591808in}}{\pgfqpoint{7.788168in}{1.589804in}}{\pgfqpoint{7.793212in}{1.589804in}}%
\pgfpathclose%
\pgfusepath{fill}%
\end{pgfscope}%
\begin{pgfscope}%
\pgfpathrectangle{\pgfqpoint{6.572727in}{0.474100in}}{\pgfqpoint{4.227273in}{3.318700in}}%
\pgfusepath{clip}%
\pgfsetbuttcap%
\pgfsetroundjoin%
\definecolor{currentfill}{rgb}{0.127568,0.566949,0.550556}%
\pgfsetfillcolor{currentfill}%
\pgfsetfillopacity{0.700000}%
\pgfsetlinewidth{0.000000pt}%
\definecolor{currentstroke}{rgb}{0.000000,0.000000,0.000000}%
\pgfsetstrokecolor{currentstroke}%
\pgfsetstrokeopacity{0.700000}%
\pgfsetdash{}{0pt}%
\pgfpathmoveto{\pgfqpoint{8.054637in}{2.485720in}}%
\pgfpathcurveto{\pgfqpoint{8.059681in}{2.485720in}}{\pgfqpoint{8.064518in}{2.487724in}}{\pgfqpoint{8.068085in}{2.491291in}}%
\pgfpathcurveto{\pgfqpoint{8.071651in}{2.494857in}}{\pgfqpoint{8.073655in}{2.499695in}}{\pgfqpoint{8.073655in}{2.504738in}}%
\pgfpathcurveto{\pgfqpoint{8.073655in}{2.509782in}}{\pgfqpoint{8.071651in}{2.514620in}}{\pgfqpoint{8.068085in}{2.518186in}}%
\pgfpathcurveto{\pgfqpoint{8.064518in}{2.521753in}}{\pgfqpoint{8.059681in}{2.523757in}}{\pgfqpoint{8.054637in}{2.523757in}}%
\pgfpathcurveto{\pgfqpoint{8.049593in}{2.523757in}}{\pgfqpoint{8.044755in}{2.521753in}}{\pgfqpoint{8.041189in}{2.518186in}}%
\pgfpathcurveto{\pgfqpoint{8.037623in}{2.514620in}}{\pgfqpoint{8.035619in}{2.509782in}}{\pgfqpoint{8.035619in}{2.504738in}}%
\pgfpathcurveto{\pgfqpoint{8.035619in}{2.499695in}}{\pgfqpoint{8.037623in}{2.494857in}}{\pgfqpoint{8.041189in}{2.491291in}}%
\pgfpathcurveto{\pgfqpoint{8.044755in}{2.487724in}}{\pgfqpoint{8.049593in}{2.485720in}}{\pgfqpoint{8.054637in}{2.485720in}}%
\pgfpathclose%
\pgfusepath{fill}%
\end{pgfscope}%
\begin{pgfscope}%
\pgfpathrectangle{\pgfqpoint{6.572727in}{0.474100in}}{\pgfqpoint{4.227273in}{3.318700in}}%
\pgfusepath{clip}%
\pgfsetbuttcap%
\pgfsetroundjoin%
\definecolor{currentfill}{rgb}{0.127568,0.566949,0.550556}%
\pgfsetfillcolor{currentfill}%
\pgfsetfillopacity{0.700000}%
\pgfsetlinewidth{0.000000pt}%
\definecolor{currentstroke}{rgb}{0.000000,0.000000,0.000000}%
\pgfsetstrokecolor{currentstroke}%
\pgfsetstrokeopacity{0.700000}%
\pgfsetdash{}{0pt}%
\pgfpathmoveto{\pgfqpoint{7.599768in}{1.282891in}}%
\pgfpathcurveto{\pgfqpoint{7.604812in}{1.282891in}}{\pgfqpoint{7.609650in}{1.284895in}}{\pgfqpoint{7.613216in}{1.288461in}}%
\pgfpathcurveto{\pgfqpoint{7.616782in}{1.292028in}}{\pgfqpoint{7.618786in}{1.296866in}}{\pgfqpoint{7.618786in}{1.301909in}}%
\pgfpathcurveto{\pgfqpoint{7.618786in}{1.306953in}}{\pgfqpoint{7.616782in}{1.311791in}}{\pgfqpoint{7.613216in}{1.315357in}}%
\pgfpathcurveto{\pgfqpoint{7.609650in}{1.318924in}}{\pgfqpoint{7.604812in}{1.320927in}}{\pgfqpoint{7.599768in}{1.320927in}}%
\pgfpathcurveto{\pgfqpoint{7.594724in}{1.320927in}}{\pgfqpoint{7.589887in}{1.318924in}}{\pgfqpoint{7.586320in}{1.315357in}}%
\pgfpathcurveto{\pgfqpoint{7.582754in}{1.311791in}}{\pgfqpoint{7.580750in}{1.306953in}}{\pgfqpoint{7.580750in}{1.301909in}}%
\pgfpathcurveto{\pgfqpoint{7.580750in}{1.296866in}}{\pgfqpoint{7.582754in}{1.292028in}}{\pgfqpoint{7.586320in}{1.288461in}}%
\pgfpathcurveto{\pgfqpoint{7.589887in}{1.284895in}}{\pgfqpoint{7.594724in}{1.282891in}}{\pgfqpoint{7.599768in}{1.282891in}}%
\pgfpathclose%
\pgfusepath{fill}%
\end{pgfscope}%
\begin{pgfscope}%
\pgfpathrectangle{\pgfqpoint{6.572727in}{0.474100in}}{\pgfqpoint{4.227273in}{3.318700in}}%
\pgfusepath{clip}%
\pgfsetbuttcap%
\pgfsetroundjoin%
\definecolor{currentfill}{rgb}{0.127568,0.566949,0.550556}%
\pgfsetfillcolor{currentfill}%
\pgfsetfillopacity{0.700000}%
\pgfsetlinewidth{0.000000pt}%
\definecolor{currentstroke}{rgb}{0.000000,0.000000,0.000000}%
\pgfsetstrokecolor{currentstroke}%
\pgfsetstrokeopacity{0.700000}%
\pgfsetdash{}{0pt}%
\pgfpathmoveto{\pgfqpoint{8.375172in}{3.020818in}}%
\pgfpathcurveto{\pgfqpoint{8.380216in}{3.020818in}}{\pgfqpoint{8.385054in}{3.022822in}}{\pgfqpoint{8.388620in}{3.026388in}}%
\pgfpathcurveto{\pgfqpoint{8.392187in}{3.029954in}}{\pgfqpoint{8.394191in}{3.034792in}}{\pgfqpoint{8.394191in}{3.039836in}}%
\pgfpathcurveto{\pgfqpoint{8.394191in}{3.044880in}}{\pgfqpoint{8.392187in}{3.049717in}}{\pgfqpoint{8.388620in}{3.053284in}}%
\pgfpathcurveto{\pgfqpoint{8.385054in}{3.056850in}}{\pgfqpoint{8.380216in}{3.058854in}}{\pgfqpoint{8.375172in}{3.058854in}}%
\pgfpathcurveto{\pgfqpoint{8.370129in}{3.058854in}}{\pgfqpoint{8.365291in}{3.056850in}}{\pgfqpoint{8.361725in}{3.053284in}}%
\pgfpathcurveto{\pgfqpoint{8.358158in}{3.049717in}}{\pgfqpoint{8.356154in}{3.044880in}}{\pgfqpoint{8.356154in}{3.039836in}}%
\pgfpathcurveto{\pgfqpoint{8.356154in}{3.034792in}}{\pgfqpoint{8.358158in}{3.029954in}}{\pgfqpoint{8.361725in}{3.026388in}}%
\pgfpathcurveto{\pgfqpoint{8.365291in}{3.022822in}}{\pgfqpoint{8.370129in}{3.020818in}}{\pgfqpoint{8.375172in}{3.020818in}}%
\pgfpathclose%
\pgfusepath{fill}%
\end{pgfscope}%
\begin{pgfscope}%
\pgfpathrectangle{\pgfqpoint{6.572727in}{0.474100in}}{\pgfqpoint{4.227273in}{3.318700in}}%
\pgfusepath{clip}%
\pgfsetbuttcap%
\pgfsetroundjoin%
\definecolor{currentfill}{rgb}{0.127568,0.566949,0.550556}%
\pgfsetfillcolor{currentfill}%
\pgfsetfillopacity{0.700000}%
\pgfsetlinewidth{0.000000pt}%
\definecolor{currentstroke}{rgb}{0.000000,0.000000,0.000000}%
\pgfsetstrokecolor{currentstroke}%
\pgfsetstrokeopacity{0.700000}%
\pgfsetdash{}{0pt}%
\pgfpathmoveto{\pgfqpoint{7.989577in}{2.763777in}}%
\pgfpathcurveto{\pgfqpoint{7.994620in}{2.763777in}}{\pgfqpoint{7.999458in}{2.765781in}}{\pgfqpoint{8.003025in}{2.769348in}}%
\pgfpathcurveto{\pgfqpoint{8.006591in}{2.772914in}}{\pgfqpoint{8.008595in}{2.777752in}}{\pgfqpoint{8.008595in}{2.782795in}}%
\pgfpathcurveto{\pgfqpoint{8.008595in}{2.787839in}}{\pgfqpoint{8.006591in}{2.792677in}}{\pgfqpoint{8.003025in}{2.796243in}}%
\pgfpathcurveto{\pgfqpoint{7.999458in}{2.799810in}}{\pgfqpoint{7.994620in}{2.801814in}}{\pgfqpoint{7.989577in}{2.801814in}}%
\pgfpathcurveto{\pgfqpoint{7.984533in}{2.801814in}}{\pgfqpoint{7.979695in}{2.799810in}}{\pgfqpoint{7.976129in}{2.796243in}}%
\pgfpathcurveto{\pgfqpoint{7.972562in}{2.792677in}}{\pgfqpoint{7.970559in}{2.787839in}}{\pgfqpoint{7.970559in}{2.782795in}}%
\pgfpathcurveto{\pgfqpoint{7.970559in}{2.777752in}}{\pgfqpoint{7.972562in}{2.772914in}}{\pgfqpoint{7.976129in}{2.769348in}}%
\pgfpathcurveto{\pgfqpoint{7.979695in}{2.765781in}}{\pgfqpoint{7.984533in}{2.763777in}}{\pgfqpoint{7.989577in}{2.763777in}}%
\pgfpathclose%
\pgfusepath{fill}%
\end{pgfscope}%
\begin{pgfscope}%
\pgfpathrectangle{\pgfqpoint{6.572727in}{0.474100in}}{\pgfqpoint{4.227273in}{3.318700in}}%
\pgfusepath{clip}%
\pgfsetbuttcap%
\pgfsetroundjoin%
\definecolor{currentfill}{rgb}{0.127568,0.566949,0.550556}%
\pgfsetfillcolor{currentfill}%
\pgfsetfillopacity{0.700000}%
\pgfsetlinewidth{0.000000pt}%
\definecolor{currentstroke}{rgb}{0.000000,0.000000,0.000000}%
\pgfsetstrokecolor{currentstroke}%
\pgfsetstrokeopacity{0.700000}%
\pgfsetdash{}{0pt}%
\pgfpathmoveto{\pgfqpoint{8.127369in}{1.497664in}}%
\pgfpathcurveto{\pgfqpoint{8.132413in}{1.497664in}}{\pgfqpoint{8.137251in}{1.499668in}}{\pgfqpoint{8.140817in}{1.503234in}}%
\pgfpathcurveto{\pgfqpoint{8.144384in}{1.506801in}}{\pgfqpoint{8.146387in}{1.511638in}}{\pgfqpoint{8.146387in}{1.516682in}}%
\pgfpathcurveto{\pgfqpoint{8.146387in}{1.521726in}}{\pgfqpoint{8.144384in}{1.526564in}}{\pgfqpoint{8.140817in}{1.530130in}}%
\pgfpathcurveto{\pgfqpoint{8.137251in}{1.533696in}}{\pgfqpoint{8.132413in}{1.535700in}}{\pgfqpoint{8.127369in}{1.535700in}}%
\pgfpathcurveto{\pgfqpoint{8.122326in}{1.535700in}}{\pgfqpoint{8.117488in}{1.533696in}}{\pgfqpoint{8.113921in}{1.530130in}}%
\pgfpathcurveto{\pgfqpoint{8.110355in}{1.526564in}}{\pgfqpoint{8.108351in}{1.521726in}}{\pgfqpoint{8.108351in}{1.516682in}}%
\pgfpathcurveto{\pgfqpoint{8.108351in}{1.511638in}}{\pgfqpoint{8.110355in}{1.506801in}}{\pgfqpoint{8.113921in}{1.503234in}}%
\pgfpathcurveto{\pgfqpoint{8.117488in}{1.499668in}}{\pgfqpoint{8.122326in}{1.497664in}}{\pgfqpoint{8.127369in}{1.497664in}}%
\pgfpathclose%
\pgfusepath{fill}%
\end{pgfscope}%
\begin{pgfscope}%
\pgfpathrectangle{\pgfqpoint{6.572727in}{0.474100in}}{\pgfqpoint{4.227273in}{3.318700in}}%
\pgfusepath{clip}%
\pgfsetbuttcap%
\pgfsetroundjoin%
\definecolor{currentfill}{rgb}{0.993248,0.906157,0.143936}%
\pgfsetfillcolor{currentfill}%
\pgfsetfillopacity{0.700000}%
\pgfsetlinewidth{0.000000pt}%
\definecolor{currentstroke}{rgb}{0.000000,0.000000,0.000000}%
\pgfsetstrokecolor{currentstroke}%
\pgfsetstrokeopacity{0.700000}%
\pgfsetdash{}{0pt}%
\pgfpathmoveto{\pgfqpoint{9.781117in}{1.506190in}}%
\pgfpathcurveto{\pgfqpoint{9.786161in}{1.506190in}}{\pgfqpoint{9.790999in}{1.508194in}}{\pgfqpoint{9.794565in}{1.511760in}}%
\pgfpathcurveto{\pgfqpoint{9.798132in}{1.515326in}}{\pgfqpoint{9.800136in}{1.520164in}}{\pgfqpoint{9.800136in}{1.525208in}}%
\pgfpathcurveto{\pgfqpoint{9.800136in}{1.530252in}}{\pgfqpoint{9.798132in}{1.535089in}}{\pgfqpoint{9.794565in}{1.538656in}}%
\pgfpathcurveto{\pgfqpoint{9.790999in}{1.542222in}}{\pgfqpoint{9.786161in}{1.544226in}}{\pgfqpoint{9.781117in}{1.544226in}}%
\pgfpathcurveto{\pgfqpoint{9.776074in}{1.544226in}}{\pgfqpoint{9.771236in}{1.542222in}}{\pgfqpoint{9.767670in}{1.538656in}}%
\pgfpathcurveto{\pgfqpoint{9.764103in}{1.535089in}}{\pgfqpoint{9.762099in}{1.530252in}}{\pgfqpoint{9.762099in}{1.525208in}}%
\pgfpathcurveto{\pgfqpoint{9.762099in}{1.520164in}}{\pgfqpoint{9.764103in}{1.515326in}}{\pgfqpoint{9.767670in}{1.511760in}}%
\pgfpathcurveto{\pgfqpoint{9.771236in}{1.508194in}}{\pgfqpoint{9.776074in}{1.506190in}}{\pgfqpoint{9.781117in}{1.506190in}}%
\pgfpathclose%
\pgfusepath{fill}%
\end{pgfscope}%
\begin{pgfscope}%
\pgfpathrectangle{\pgfqpoint{6.572727in}{0.474100in}}{\pgfqpoint{4.227273in}{3.318700in}}%
\pgfusepath{clip}%
\pgfsetbuttcap%
\pgfsetroundjoin%
\definecolor{currentfill}{rgb}{0.993248,0.906157,0.143936}%
\pgfsetfillcolor{currentfill}%
\pgfsetfillopacity{0.700000}%
\pgfsetlinewidth{0.000000pt}%
\definecolor{currentstroke}{rgb}{0.000000,0.000000,0.000000}%
\pgfsetstrokecolor{currentstroke}%
\pgfsetstrokeopacity{0.700000}%
\pgfsetdash{}{0pt}%
\pgfpathmoveto{\pgfqpoint{9.497266in}{1.564599in}}%
\pgfpathcurveto{\pgfqpoint{9.502309in}{1.564599in}}{\pgfqpoint{9.507147in}{1.566603in}}{\pgfqpoint{9.510713in}{1.570169in}}%
\pgfpathcurveto{\pgfqpoint{9.514280in}{1.573736in}}{\pgfqpoint{9.516284in}{1.578574in}}{\pgfqpoint{9.516284in}{1.583617in}}%
\pgfpathcurveto{\pgfqpoint{9.516284in}{1.588661in}}{\pgfqpoint{9.514280in}{1.593499in}}{\pgfqpoint{9.510713in}{1.597065in}}%
\pgfpathcurveto{\pgfqpoint{9.507147in}{1.600631in}}{\pgfqpoint{9.502309in}{1.602635in}}{\pgfqpoint{9.497266in}{1.602635in}}%
\pgfpathcurveto{\pgfqpoint{9.492222in}{1.602635in}}{\pgfqpoint{9.487384in}{1.600631in}}{\pgfqpoint{9.483818in}{1.597065in}}%
\pgfpathcurveto{\pgfqpoint{9.480251in}{1.593499in}}{\pgfqpoint{9.478247in}{1.588661in}}{\pgfqpoint{9.478247in}{1.583617in}}%
\pgfpathcurveto{\pgfqpoint{9.478247in}{1.578574in}}{\pgfqpoint{9.480251in}{1.573736in}}{\pgfqpoint{9.483818in}{1.570169in}}%
\pgfpathcurveto{\pgfqpoint{9.487384in}{1.566603in}}{\pgfqpoint{9.492222in}{1.564599in}}{\pgfqpoint{9.497266in}{1.564599in}}%
\pgfpathclose%
\pgfusepath{fill}%
\end{pgfscope}%
\begin{pgfscope}%
\pgfpathrectangle{\pgfqpoint{6.572727in}{0.474100in}}{\pgfqpoint{4.227273in}{3.318700in}}%
\pgfusepath{clip}%
\pgfsetbuttcap%
\pgfsetroundjoin%
\definecolor{currentfill}{rgb}{0.127568,0.566949,0.550556}%
\pgfsetfillcolor{currentfill}%
\pgfsetfillopacity{0.700000}%
\pgfsetlinewidth{0.000000pt}%
\definecolor{currentstroke}{rgb}{0.000000,0.000000,0.000000}%
\pgfsetstrokecolor{currentstroke}%
\pgfsetstrokeopacity{0.700000}%
\pgfsetdash{}{0pt}%
\pgfpathmoveto{\pgfqpoint{7.090086in}{1.496827in}}%
\pgfpathcurveto{\pgfqpoint{7.095130in}{1.496827in}}{\pgfqpoint{7.099967in}{1.498831in}}{\pgfqpoint{7.103534in}{1.502397in}}%
\pgfpathcurveto{\pgfqpoint{7.107100in}{1.505963in}}{\pgfqpoint{7.109104in}{1.510801in}}{\pgfqpoint{7.109104in}{1.515845in}}%
\pgfpathcurveto{\pgfqpoint{7.109104in}{1.520889in}}{\pgfqpoint{7.107100in}{1.525726in}}{\pgfqpoint{7.103534in}{1.529293in}}%
\pgfpathcurveto{\pgfqpoint{7.099967in}{1.532859in}}{\pgfqpoint{7.095130in}{1.534863in}}{\pgfqpoint{7.090086in}{1.534863in}}%
\pgfpathcurveto{\pgfqpoint{7.085042in}{1.534863in}}{\pgfqpoint{7.080205in}{1.532859in}}{\pgfqpoint{7.076638in}{1.529293in}}%
\pgfpathcurveto{\pgfqpoint{7.073072in}{1.525726in}}{\pgfqpoint{7.071068in}{1.520889in}}{\pgfqpoint{7.071068in}{1.515845in}}%
\pgfpathcurveto{\pgfqpoint{7.071068in}{1.510801in}}{\pgfqpoint{7.073072in}{1.505963in}}{\pgfqpoint{7.076638in}{1.502397in}}%
\pgfpathcurveto{\pgfqpoint{7.080205in}{1.498831in}}{\pgfqpoint{7.085042in}{1.496827in}}{\pgfqpoint{7.090086in}{1.496827in}}%
\pgfpathclose%
\pgfusepath{fill}%
\end{pgfscope}%
\begin{pgfscope}%
\pgfpathrectangle{\pgfqpoint{6.572727in}{0.474100in}}{\pgfqpoint{4.227273in}{3.318700in}}%
\pgfusepath{clip}%
\pgfsetbuttcap%
\pgfsetroundjoin%
\definecolor{currentfill}{rgb}{0.127568,0.566949,0.550556}%
\pgfsetfillcolor{currentfill}%
\pgfsetfillopacity{0.700000}%
\pgfsetlinewidth{0.000000pt}%
\definecolor{currentstroke}{rgb}{0.000000,0.000000,0.000000}%
\pgfsetstrokecolor{currentstroke}%
\pgfsetstrokeopacity{0.700000}%
\pgfsetdash{}{0pt}%
\pgfpathmoveto{\pgfqpoint{8.236202in}{3.175845in}}%
\pgfpathcurveto{\pgfqpoint{8.241246in}{3.175845in}}{\pgfqpoint{8.246084in}{3.177849in}}{\pgfqpoint{8.249650in}{3.181415in}}%
\pgfpathcurveto{\pgfqpoint{8.253217in}{3.184982in}}{\pgfqpoint{8.255220in}{3.189820in}}{\pgfqpoint{8.255220in}{3.194863in}}%
\pgfpathcurveto{\pgfqpoint{8.255220in}{3.199907in}}{\pgfqpoint{8.253217in}{3.204745in}}{\pgfqpoint{8.249650in}{3.208311in}}%
\pgfpathcurveto{\pgfqpoint{8.246084in}{3.211878in}}{\pgfqpoint{8.241246in}{3.213881in}}{\pgfqpoint{8.236202in}{3.213881in}}%
\pgfpathcurveto{\pgfqpoint{8.231159in}{3.213881in}}{\pgfqpoint{8.226321in}{3.211878in}}{\pgfqpoint{8.222754in}{3.208311in}}%
\pgfpathcurveto{\pgfqpoint{8.219188in}{3.204745in}}{\pgfqpoint{8.217184in}{3.199907in}}{\pgfqpoint{8.217184in}{3.194863in}}%
\pgfpathcurveto{\pgfqpoint{8.217184in}{3.189820in}}{\pgfqpoint{8.219188in}{3.184982in}}{\pgfqpoint{8.222754in}{3.181415in}}%
\pgfpathcurveto{\pgfqpoint{8.226321in}{3.177849in}}{\pgfqpoint{8.231159in}{3.175845in}}{\pgfqpoint{8.236202in}{3.175845in}}%
\pgfpathclose%
\pgfusepath{fill}%
\end{pgfscope}%
\begin{pgfscope}%
\pgfpathrectangle{\pgfqpoint{6.572727in}{0.474100in}}{\pgfqpoint{4.227273in}{3.318700in}}%
\pgfusepath{clip}%
\pgfsetbuttcap%
\pgfsetroundjoin%
\definecolor{currentfill}{rgb}{0.127568,0.566949,0.550556}%
\pgfsetfillcolor{currentfill}%
\pgfsetfillopacity{0.700000}%
\pgfsetlinewidth{0.000000pt}%
\definecolor{currentstroke}{rgb}{0.000000,0.000000,0.000000}%
\pgfsetstrokecolor{currentstroke}%
\pgfsetstrokeopacity{0.700000}%
\pgfsetdash{}{0pt}%
\pgfpathmoveto{\pgfqpoint{7.645761in}{2.969141in}}%
\pgfpathcurveto{\pgfqpoint{7.650804in}{2.969141in}}{\pgfqpoint{7.655642in}{2.971145in}}{\pgfqpoint{7.659208in}{2.974712in}}%
\pgfpathcurveto{\pgfqpoint{7.662775in}{2.978278in}}{\pgfqpoint{7.664779in}{2.983116in}}{\pgfqpoint{7.664779in}{2.988159in}}%
\pgfpathcurveto{\pgfqpoint{7.664779in}{2.993203in}}{\pgfqpoint{7.662775in}{2.998041in}}{\pgfqpoint{7.659208in}{3.001607in}}%
\pgfpathcurveto{\pgfqpoint{7.655642in}{3.005174in}}{\pgfqpoint{7.650804in}{3.007178in}}{\pgfqpoint{7.645761in}{3.007178in}}%
\pgfpathcurveto{\pgfqpoint{7.640717in}{3.007178in}}{\pgfqpoint{7.635879in}{3.005174in}}{\pgfqpoint{7.632313in}{3.001607in}}%
\pgfpathcurveto{\pgfqpoint{7.628746in}{2.998041in}}{\pgfqpoint{7.626742in}{2.993203in}}{\pgfqpoint{7.626742in}{2.988159in}}%
\pgfpathcurveto{\pgfqpoint{7.626742in}{2.983116in}}{\pgfqpoint{7.628746in}{2.978278in}}{\pgfqpoint{7.632313in}{2.974712in}}%
\pgfpathcurveto{\pgfqpoint{7.635879in}{2.971145in}}{\pgfqpoint{7.640717in}{2.969141in}}{\pgfqpoint{7.645761in}{2.969141in}}%
\pgfpathclose%
\pgfusepath{fill}%
\end{pgfscope}%
\begin{pgfscope}%
\pgfpathrectangle{\pgfqpoint{6.572727in}{0.474100in}}{\pgfqpoint{4.227273in}{3.318700in}}%
\pgfusepath{clip}%
\pgfsetbuttcap%
\pgfsetroundjoin%
\definecolor{currentfill}{rgb}{0.127568,0.566949,0.550556}%
\pgfsetfillcolor{currentfill}%
\pgfsetfillopacity{0.700000}%
\pgfsetlinewidth{0.000000pt}%
\definecolor{currentstroke}{rgb}{0.000000,0.000000,0.000000}%
\pgfsetstrokecolor{currentstroke}%
\pgfsetstrokeopacity{0.700000}%
\pgfsetdash{}{0pt}%
\pgfpathmoveto{\pgfqpoint{8.310008in}{2.679224in}}%
\pgfpathcurveto{\pgfqpoint{8.315052in}{2.679224in}}{\pgfqpoint{8.319890in}{2.681228in}}{\pgfqpoint{8.323456in}{2.684794in}}%
\pgfpathcurveto{\pgfqpoint{8.327023in}{2.688361in}}{\pgfqpoint{8.329026in}{2.693199in}}{\pgfqpoint{8.329026in}{2.698242in}}%
\pgfpathcurveto{\pgfqpoint{8.329026in}{2.703286in}}{\pgfqpoint{8.327023in}{2.708124in}}{\pgfqpoint{8.323456in}{2.711690in}}%
\pgfpathcurveto{\pgfqpoint{8.319890in}{2.715257in}}{\pgfqpoint{8.315052in}{2.717260in}}{\pgfqpoint{8.310008in}{2.717260in}}%
\pgfpathcurveto{\pgfqpoint{8.304965in}{2.717260in}}{\pgfqpoint{8.300127in}{2.715257in}}{\pgfqpoint{8.296560in}{2.711690in}}%
\pgfpathcurveto{\pgfqpoint{8.292994in}{2.708124in}}{\pgfqpoint{8.290990in}{2.703286in}}{\pgfqpoint{8.290990in}{2.698242in}}%
\pgfpathcurveto{\pgfqpoint{8.290990in}{2.693199in}}{\pgfqpoint{8.292994in}{2.688361in}}{\pgfqpoint{8.296560in}{2.684794in}}%
\pgfpathcurveto{\pgfqpoint{8.300127in}{2.681228in}}{\pgfqpoint{8.304965in}{2.679224in}}{\pgfqpoint{8.310008in}{2.679224in}}%
\pgfpathclose%
\pgfusepath{fill}%
\end{pgfscope}%
\begin{pgfscope}%
\pgfpathrectangle{\pgfqpoint{6.572727in}{0.474100in}}{\pgfqpoint{4.227273in}{3.318700in}}%
\pgfusepath{clip}%
\pgfsetbuttcap%
\pgfsetroundjoin%
\definecolor{currentfill}{rgb}{0.993248,0.906157,0.143936}%
\pgfsetfillcolor{currentfill}%
\pgfsetfillopacity{0.700000}%
\pgfsetlinewidth{0.000000pt}%
\definecolor{currentstroke}{rgb}{0.000000,0.000000,0.000000}%
\pgfsetstrokecolor{currentstroke}%
\pgfsetstrokeopacity{0.700000}%
\pgfsetdash{}{0pt}%
\pgfpathmoveto{\pgfqpoint{9.612338in}{1.447560in}}%
\pgfpathcurveto{\pgfqpoint{9.617382in}{1.447560in}}{\pgfqpoint{9.622220in}{1.449564in}}{\pgfqpoint{9.625786in}{1.453131in}}%
\pgfpathcurveto{\pgfqpoint{9.629353in}{1.456697in}}{\pgfqpoint{9.631357in}{1.461535in}}{\pgfqpoint{9.631357in}{1.466579in}}%
\pgfpathcurveto{\pgfqpoint{9.631357in}{1.471622in}}{\pgfqpoint{9.629353in}{1.476460in}}{\pgfqpoint{9.625786in}{1.480026in}}%
\pgfpathcurveto{\pgfqpoint{9.622220in}{1.483593in}}{\pgfqpoint{9.617382in}{1.485597in}}{\pgfqpoint{9.612338in}{1.485597in}}%
\pgfpathcurveto{\pgfqpoint{9.607295in}{1.485597in}}{\pgfqpoint{9.602457in}{1.483593in}}{\pgfqpoint{9.598891in}{1.480026in}}%
\pgfpathcurveto{\pgfqpoint{9.595324in}{1.476460in}}{\pgfqpoint{9.593320in}{1.471622in}}{\pgfqpoint{9.593320in}{1.466579in}}%
\pgfpathcurveto{\pgfqpoint{9.593320in}{1.461535in}}{\pgfqpoint{9.595324in}{1.456697in}}{\pgfqpoint{9.598891in}{1.453131in}}%
\pgfpathcurveto{\pgfqpoint{9.602457in}{1.449564in}}{\pgfqpoint{9.607295in}{1.447560in}}{\pgfqpoint{9.612338in}{1.447560in}}%
\pgfpathclose%
\pgfusepath{fill}%
\end{pgfscope}%
\begin{pgfscope}%
\pgfpathrectangle{\pgfqpoint{6.572727in}{0.474100in}}{\pgfqpoint{4.227273in}{3.318700in}}%
\pgfusepath{clip}%
\pgfsetbuttcap%
\pgfsetroundjoin%
\definecolor{currentfill}{rgb}{0.993248,0.906157,0.143936}%
\pgfsetfillcolor{currentfill}%
\pgfsetfillopacity{0.700000}%
\pgfsetlinewidth{0.000000pt}%
\definecolor{currentstroke}{rgb}{0.000000,0.000000,0.000000}%
\pgfsetstrokecolor{currentstroke}%
\pgfsetstrokeopacity{0.700000}%
\pgfsetdash{}{0pt}%
\pgfpathmoveto{\pgfqpoint{9.350576in}{1.352473in}}%
\pgfpathcurveto{\pgfqpoint{9.355619in}{1.352473in}}{\pgfqpoint{9.360457in}{1.354477in}}{\pgfqpoint{9.364024in}{1.358043in}}%
\pgfpathcurveto{\pgfqpoint{9.367590in}{1.361610in}}{\pgfqpoint{9.369594in}{1.366447in}}{\pgfqpoint{9.369594in}{1.371491in}}%
\pgfpathcurveto{\pgfqpoint{9.369594in}{1.376535in}}{\pgfqpoint{9.367590in}{1.381373in}}{\pgfqpoint{9.364024in}{1.384939in}}%
\pgfpathcurveto{\pgfqpoint{9.360457in}{1.388505in}}{\pgfqpoint{9.355619in}{1.390509in}}{\pgfqpoint{9.350576in}{1.390509in}}%
\pgfpathcurveto{\pgfqpoint{9.345532in}{1.390509in}}{\pgfqpoint{9.340694in}{1.388505in}}{\pgfqpoint{9.337128in}{1.384939in}}%
\pgfpathcurveto{\pgfqpoint{9.333561in}{1.381373in}}{\pgfqpoint{9.331558in}{1.376535in}}{\pgfqpoint{9.331558in}{1.371491in}}%
\pgfpathcurveto{\pgfqpoint{9.331558in}{1.366447in}}{\pgfqpoint{9.333561in}{1.361610in}}{\pgfqpoint{9.337128in}{1.358043in}}%
\pgfpathcurveto{\pgfqpoint{9.340694in}{1.354477in}}{\pgfqpoint{9.345532in}{1.352473in}}{\pgfqpoint{9.350576in}{1.352473in}}%
\pgfpathclose%
\pgfusepath{fill}%
\end{pgfscope}%
\begin{pgfscope}%
\pgfpathrectangle{\pgfqpoint{6.572727in}{0.474100in}}{\pgfqpoint{4.227273in}{3.318700in}}%
\pgfusepath{clip}%
\pgfsetbuttcap%
\pgfsetroundjoin%
\definecolor{currentfill}{rgb}{0.127568,0.566949,0.550556}%
\pgfsetfillcolor{currentfill}%
\pgfsetfillopacity{0.700000}%
\pgfsetlinewidth{0.000000pt}%
\definecolor{currentstroke}{rgb}{0.000000,0.000000,0.000000}%
\pgfsetstrokecolor{currentstroke}%
\pgfsetstrokeopacity{0.700000}%
\pgfsetdash{}{0pt}%
\pgfpathmoveto{\pgfqpoint{8.407916in}{2.925847in}}%
\pgfpathcurveto{\pgfqpoint{8.412960in}{2.925847in}}{\pgfqpoint{8.417797in}{2.927851in}}{\pgfqpoint{8.421364in}{2.931417in}}%
\pgfpathcurveto{\pgfqpoint{8.424930in}{2.934984in}}{\pgfqpoint{8.426934in}{2.939822in}}{\pgfqpoint{8.426934in}{2.944865in}}%
\pgfpathcurveto{\pgfqpoint{8.426934in}{2.949909in}}{\pgfqpoint{8.424930in}{2.954747in}}{\pgfqpoint{8.421364in}{2.958313in}}%
\pgfpathcurveto{\pgfqpoint{8.417797in}{2.961880in}}{\pgfqpoint{8.412960in}{2.963883in}}{\pgfqpoint{8.407916in}{2.963883in}}%
\pgfpathcurveto{\pgfqpoint{8.402872in}{2.963883in}}{\pgfqpoint{8.398035in}{2.961880in}}{\pgfqpoint{8.394468in}{2.958313in}}%
\pgfpathcurveto{\pgfqpoint{8.390902in}{2.954747in}}{\pgfqpoint{8.388898in}{2.949909in}}{\pgfqpoint{8.388898in}{2.944865in}}%
\pgfpathcurveto{\pgfqpoint{8.388898in}{2.939822in}}{\pgfqpoint{8.390902in}{2.934984in}}{\pgfqpoint{8.394468in}{2.931417in}}%
\pgfpathcurveto{\pgfqpoint{8.398035in}{2.927851in}}{\pgfqpoint{8.402872in}{2.925847in}}{\pgfqpoint{8.407916in}{2.925847in}}%
\pgfpathclose%
\pgfusepath{fill}%
\end{pgfscope}%
\begin{pgfscope}%
\pgfpathrectangle{\pgfqpoint{6.572727in}{0.474100in}}{\pgfqpoint{4.227273in}{3.318700in}}%
\pgfusepath{clip}%
\pgfsetbuttcap%
\pgfsetroundjoin%
\definecolor{currentfill}{rgb}{0.993248,0.906157,0.143936}%
\pgfsetfillcolor{currentfill}%
\pgfsetfillopacity{0.700000}%
\pgfsetlinewidth{0.000000pt}%
\definecolor{currentstroke}{rgb}{0.000000,0.000000,0.000000}%
\pgfsetstrokecolor{currentstroke}%
\pgfsetstrokeopacity{0.700000}%
\pgfsetdash{}{0pt}%
\pgfpathmoveto{\pgfqpoint{9.464407in}{1.748516in}}%
\pgfpathcurveto{\pgfqpoint{9.469451in}{1.748516in}}{\pgfqpoint{9.474288in}{1.750520in}}{\pgfqpoint{9.477855in}{1.754087in}}%
\pgfpathcurveto{\pgfqpoint{9.481421in}{1.757653in}}{\pgfqpoint{9.483425in}{1.762491in}}{\pgfqpoint{9.483425in}{1.767534in}}%
\pgfpathcurveto{\pgfqpoint{9.483425in}{1.772578in}}{\pgfqpoint{9.481421in}{1.777416in}}{\pgfqpoint{9.477855in}{1.780982in}}%
\pgfpathcurveto{\pgfqpoint{9.474288in}{1.784549in}}{\pgfqpoint{9.469451in}{1.786553in}}{\pgfqpoint{9.464407in}{1.786553in}}%
\pgfpathcurveto{\pgfqpoint{9.459363in}{1.786553in}}{\pgfqpoint{9.454526in}{1.784549in}}{\pgfqpoint{9.450959in}{1.780982in}}%
\pgfpathcurveto{\pgfqpoint{9.447393in}{1.777416in}}{\pgfqpoint{9.445389in}{1.772578in}}{\pgfqpoint{9.445389in}{1.767534in}}%
\pgfpathcurveto{\pgfqpoint{9.445389in}{1.762491in}}{\pgfqpoint{9.447393in}{1.757653in}}{\pgfqpoint{9.450959in}{1.754087in}}%
\pgfpathcurveto{\pgfqpoint{9.454526in}{1.750520in}}{\pgfqpoint{9.459363in}{1.748516in}}{\pgfqpoint{9.464407in}{1.748516in}}%
\pgfpathclose%
\pgfusepath{fill}%
\end{pgfscope}%
\begin{pgfscope}%
\pgfpathrectangle{\pgfqpoint{6.572727in}{0.474100in}}{\pgfqpoint{4.227273in}{3.318700in}}%
\pgfusepath{clip}%
\pgfsetbuttcap%
\pgfsetroundjoin%
\definecolor{currentfill}{rgb}{0.127568,0.566949,0.550556}%
\pgfsetfillcolor{currentfill}%
\pgfsetfillopacity{0.700000}%
\pgfsetlinewidth{0.000000pt}%
\definecolor{currentstroke}{rgb}{0.000000,0.000000,0.000000}%
\pgfsetstrokecolor{currentstroke}%
\pgfsetstrokeopacity{0.700000}%
\pgfsetdash{}{0pt}%
\pgfpathmoveto{\pgfqpoint{8.176808in}{2.543123in}}%
\pgfpathcurveto{\pgfqpoint{8.181852in}{2.543123in}}{\pgfqpoint{8.186689in}{2.545127in}}{\pgfqpoint{8.190256in}{2.548693in}}%
\pgfpathcurveto{\pgfqpoint{8.193822in}{2.552260in}}{\pgfqpoint{8.195826in}{2.557098in}}{\pgfqpoint{8.195826in}{2.562141in}}%
\pgfpathcurveto{\pgfqpoint{8.195826in}{2.567185in}}{\pgfqpoint{8.193822in}{2.572023in}}{\pgfqpoint{8.190256in}{2.575589in}}%
\pgfpathcurveto{\pgfqpoint{8.186689in}{2.579156in}}{\pgfqpoint{8.181852in}{2.581159in}}{\pgfqpoint{8.176808in}{2.581159in}}%
\pgfpathcurveto{\pgfqpoint{8.171764in}{2.581159in}}{\pgfqpoint{8.166926in}{2.579156in}}{\pgfqpoint{8.163360in}{2.575589in}}%
\pgfpathcurveto{\pgfqpoint{8.159794in}{2.572023in}}{\pgfqpoint{8.157790in}{2.567185in}}{\pgfqpoint{8.157790in}{2.562141in}}%
\pgfpathcurveto{\pgfqpoint{8.157790in}{2.557098in}}{\pgfqpoint{8.159794in}{2.552260in}}{\pgfqpoint{8.163360in}{2.548693in}}%
\pgfpathcurveto{\pgfqpoint{8.166926in}{2.545127in}}{\pgfqpoint{8.171764in}{2.543123in}}{\pgfqpoint{8.176808in}{2.543123in}}%
\pgfpathclose%
\pgfusepath{fill}%
\end{pgfscope}%
\begin{pgfscope}%
\pgfpathrectangle{\pgfqpoint{6.572727in}{0.474100in}}{\pgfqpoint{4.227273in}{3.318700in}}%
\pgfusepath{clip}%
\pgfsetbuttcap%
\pgfsetroundjoin%
\definecolor{currentfill}{rgb}{0.993248,0.906157,0.143936}%
\pgfsetfillcolor{currentfill}%
\pgfsetfillopacity{0.700000}%
\pgfsetlinewidth{0.000000pt}%
\definecolor{currentstroke}{rgb}{0.000000,0.000000,0.000000}%
\pgfsetstrokecolor{currentstroke}%
\pgfsetstrokeopacity{0.700000}%
\pgfsetdash{}{0pt}%
\pgfpathmoveto{\pgfqpoint{9.450364in}{1.850104in}}%
\pgfpathcurveto{\pgfqpoint{9.455408in}{1.850104in}}{\pgfqpoint{9.460246in}{1.852108in}}{\pgfqpoint{9.463812in}{1.855674in}}%
\pgfpathcurveto{\pgfqpoint{9.467379in}{1.859240in}}{\pgfqpoint{9.469382in}{1.864078in}}{\pgfqpoint{9.469382in}{1.869122in}}%
\pgfpathcurveto{\pgfqpoint{9.469382in}{1.874166in}}{\pgfqpoint{9.467379in}{1.879003in}}{\pgfqpoint{9.463812in}{1.882570in}}%
\pgfpathcurveto{\pgfqpoint{9.460246in}{1.886136in}}{\pgfqpoint{9.455408in}{1.888140in}}{\pgfqpoint{9.450364in}{1.888140in}}%
\pgfpathcurveto{\pgfqpoint{9.445321in}{1.888140in}}{\pgfqpoint{9.440483in}{1.886136in}}{\pgfqpoint{9.436916in}{1.882570in}}%
\pgfpathcurveto{\pgfqpoint{9.433350in}{1.879003in}}{\pgfqpoint{9.431346in}{1.874166in}}{\pgfqpoint{9.431346in}{1.869122in}}%
\pgfpathcurveto{\pgfqpoint{9.431346in}{1.864078in}}{\pgfqpoint{9.433350in}{1.859240in}}{\pgfqpoint{9.436916in}{1.855674in}}%
\pgfpathcurveto{\pgfqpoint{9.440483in}{1.852108in}}{\pgfqpoint{9.445321in}{1.850104in}}{\pgfqpoint{9.450364in}{1.850104in}}%
\pgfpathclose%
\pgfusepath{fill}%
\end{pgfscope}%
\begin{pgfscope}%
\pgfpathrectangle{\pgfqpoint{6.572727in}{0.474100in}}{\pgfqpoint{4.227273in}{3.318700in}}%
\pgfusepath{clip}%
\pgfsetbuttcap%
\pgfsetroundjoin%
\definecolor{currentfill}{rgb}{0.127568,0.566949,0.550556}%
\pgfsetfillcolor{currentfill}%
\pgfsetfillopacity{0.700000}%
\pgfsetlinewidth{0.000000pt}%
\definecolor{currentstroke}{rgb}{0.000000,0.000000,0.000000}%
\pgfsetstrokecolor{currentstroke}%
\pgfsetstrokeopacity{0.700000}%
\pgfsetdash{}{0pt}%
\pgfpathmoveto{\pgfqpoint{9.039734in}{3.262138in}}%
\pgfpathcurveto{\pgfqpoint{9.044777in}{3.262138in}}{\pgfqpoint{9.049615in}{3.264142in}}{\pgfqpoint{9.053182in}{3.267708in}}%
\pgfpathcurveto{\pgfqpoint{9.056748in}{3.271274in}}{\pgfqpoint{9.058752in}{3.276112in}}{\pgfqpoint{9.058752in}{3.281156in}}%
\pgfpathcurveto{\pgfqpoint{9.058752in}{3.286200in}}{\pgfqpoint{9.056748in}{3.291037in}}{\pgfqpoint{9.053182in}{3.294604in}}%
\pgfpathcurveto{\pgfqpoint{9.049615in}{3.298170in}}{\pgfqpoint{9.044777in}{3.300174in}}{\pgfqpoint{9.039734in}{3.300174in}}%
\pgfpathcurveto{\pgfqpoint{9.034690in}{3.300174in}}{\pgfqpoint{9.029852in}{3.298170in}}{\pgfqpoint{9.026286in}{3.294604in}}%
\pgfpathcurveto{\pgfqpoint{9.022720in}{3.291037in}}{\pgfqpoint{9.020716in}{3.286200in}}{\pgfqpoint{9.020716in}{3.281156in}}%
\pgfpathcurveto{\pgfqpoint{9.020716in}{3.276112in}}{\pgfqpoint{9.022720in}{3.271274in}}{\pgfqpoint{9.026286in}{3.267708in}}%
\pgfpathcurveto{\pgfqpoint{9.029852in}{3.264142in}}{\pgfqpoint{9.034690in}{3.262138in}}{\pgfqpoint{9.039734in}{3.262138in}}%
\pgfpathclose%
\pgfusepath{fill}%
\end{pgfscope}%
\begin{pgfscope}%
\pgfpathrectangle{\pgfqpoint{6.572727in}{0.474100in}}{\pgfqpoint{4.227273in}{3.318700in}}%
\pgfusepath{clip}%
\pgfsetbuttcap%
\pgfsetroundjoin%
\definecolor{currentfill}{rgb}{0.127568,0.566949,0.550556}%
\pgfsetfillcolor{currentfill}%
\pgfsetfillopacity{0.700000}%
\pgfsetlinewidth{0.000000pt}%
\definecolor{currentstroke}{rgb}{0.000000,0.000000,0.000000}%
\pgfsetstrokecolor{currentstroke}%
\pgfsetstrokeopacity{0.700000}%
\pgfsetdash{}{0pt}%
\pgfpathmoveto{\pgfqpoint{8.006949in}{1.357047in}}%
\pgfpathcurveto{\pgfqpoint{8.011993in}{1.357047in}}{\pgfqpoint{8.016831in}{1.359051in}}{\pgfqpoint{8.020397in}{1.362618in}}%
\pgfpathcurveto{\pgfqpoint{8.023963in}{1.366184in}}{\pgfqpoint{8.025967in}{1.371022in}}{\pgfqpoint{8.025967in}{1.376066in}}%
\pgfpathcurveto{\pgfqpoint{8.025967in}{1.381109in}}{\pgfqpoint{8.023963in}{1.385947in}}{\pgfqpoint{8.020397in}{1.389513in}}%
\pgfpathcurveto{\pgfqpoint{8.016831in}{1.393080in}}{\pgfqpoint{8.011993in}{1.395084in}}{\pgfqpoint{8.006949in}{1.395084in}}%
\pgfpathcurveto{\pgfqpoint{8.001905in}{1.395084in}}{\pgfqpoint{7.997068in}{1.393080in}}{\pgfqpoint{7.993501in}{1.389513in}}%
\pgfpathcurveto{\pgfqpoint{7.989935in}{1.385947in}}{\pgfqpoint{7.987931in}{1.381109in}}{\pgfqpoint{7.987931in}{1.376066in}}%
\pgfpathcurveto{\pgfqpoint{7.987931in}{1.371022in}}{\pgfqpoint{7.989935in}{1.366184in}}{\pgfqpoint{7.993501in}{1.362618in}}%
\pgfpathcurveto{\pgfqpoint{7.997068in}{1.359051in}}{\pgfqpoint{8.001905in}{1.357047in}}{\pgfqpoint{8.006949in}{1.357047in}}%
\pgfpathclose%
\pgfusepath{fill}%
\end{pgfscope}%
\begin{pgfscope}%
\pgfpathrectangle{\pgfqpoint{6.572727in}{0.474100in}}{\pgfqpoint{4.227273in}{3.318700in}}%
\pgfusepath{clip}%
\pgfsetbuttcap%
\pgfsetroundjoin%
\definecolor{currentfill}{rgb}{0.127568,0.566949,0.550556}%
\pgfsetfillcolor{currentfill}%
\pgfsetfillopacity{0.700000}%
\pgfsetlinewidth{0.000000pt}%
\definecolor{currentstroke}{rgb}{0.000000,0.000000,0.000000}%
\pgfsetstrokecolor{currentstroke}%
\pgfsetstrokeopacity{0.700000}%
\pgfsetdash{}{0pt}%
\pgfpathmoveto{\pgfqpoint{7.678302in}{0.972630in}}%
\pgfpathcurveto{\pgfqpoint{7.683346in}{0.972630in}}{\pgfqpoint{7.688184in}{0.974634in}}{\pgfqpoint{7.691750in}{0.978200in}}%
\pgfpathcurveto{\pgfqpoint{7.695317in}{0.981767in}}{\pgfqpoint{7.697321in}{0.986604in}}{\pgfqpoint{7.697321in}{0.991648in}}%
\pgfpathcurveto{\pgfqpoint{7.697321in}{0.996692in}}{\pgfqpoint{7.695317in}{1.001529in}}{\pgfqpoint{7.691750in}{1.005096in}}%
\pgfpathcurveto{\pgfqpoint{7.688184in}{1.008662in}}{\pgfqpoint{7.683346in}{1.010666in}}{\pgfqpoint{7.678302in}{1.010666in}}%
\pgfpathcurveto{\pgfqpoint{7.673259in}{1.010666in}}{\pgfqpoint{7.668421in}{1.008662in}}{\pgfqpoint{7.664855in}{1.005096in}}%
\pgfpathcurveto{\pgfqpoint{7.661288in}{1.001529in}}{\pgfqpoint{7.659284in}{0.996692in}}{\pgfqpoint{7.659284in}{0.991648in}}%
\pgfpathcurveto{\pgfqpoint{7.659284in}{0.986604in}}{\pgfqpoint{7.661288in}{0.981767in}}{\pgfqpoint{7.664855in}{0.978200in}}%
\pgfpathcurveto{\pgfqpoint{7.668421in}{0.974634in}}{\pgfqpoint{7.673259in}{0.972630in}}{\pgfqpoint{7.678302in}{0.972630in}}%
\pgfpathclose%
\pgfusepath{fill}%
\end{pgfscope}%
\begin{pgfscope}%
\pgfpathrectangle{\pgfqpoint{6.572727in}{0.474100in}}{\pgfqpoint{4.227273in}{3.318700in}}%
\pgfusepath{clip}%
\pgfsetbuttcap%
\pgfsetroundjoin%
\definecolor{currentfill}{rgb}{0.993248,0.906157,0.143936}%
\pgfsetfillcolor{currentfill}%
\pgfsetfillopacity{0.700000}%
\pgfsetlinewidth{0.000000pt}%
\definecolor{currentstroke}{rgb}{0.000000,0.000000,0.000000}%
\pgfsetstrokecolor{currentstroke}%
\pgfsetstrokeopacity{0.700000}%
\pgfsetdash{}{0pt}%
\pgfpathmoveto{\pgfqpoint{9.677000in}{1.228677in}}%
\pgfpathcurveto{\pgfqpoint{9.682043in}{1.228677in}}{\pgfqpoint{9.686881in}{1.230680in}}{\pgfqpoint{9.690448in}{1.234247in}}%
\pgfpathcurveto{\pgfqpoint{9.694014in}{1.237813in}}{\pgfqpoint{9.696018in}{1.242651in}}{\pgfqpoint{9.696018in}{1.247695in}}%
\pgfpathcurveto{\pgfqpoint{9.696018in}{1.252738in}}{\pgfqpoint{9.694014in}{1.257576in}}{\pgfqpoint{9.690448in}{1.261143in}}%
\pgfpathcurveto{\pgfqpoint{9.686881in}{1.264709in}}{\pgfqpoint{9.682043in}{1.266713in}}{\pgfqpoint{9.677000in}{1.266713in}}%
\pgfpathcurveto{\pgfqpoint{9.671956in}{1.266713in}}{\pgfqpoint{9.667118in}{1.264709in}}{\pgfqpoint{9.663552in}{1.261143in}}%
\pgfpathcurveto{\pgfqpoint{9.659985in}{1.257576in}}{\pgfqpoint{9.657982in}{1.252738in}}{\pgfqpoint{9.657982in}{1.247695in}}%
\pgfpathcurveto{\pgfqpoint{9.657982in}{1.242651in}}{\pgfqpoint{9.659985in}{1.237813in}}{\pgfqpoint{9.663552in}{1.234247in}}%
\pgfpathcurveto{\pgfqpoint{9.667118in}{1.230680in}}{\pgfqpoint{9.671956in}{1.228677in}}{\pgfqpoint{9.677000in}{1.228677in}}%
\pgfpathclose%
\pgfusepath{fill}%
\end{pgfscope}%
\begin{pgfscope}%
\pgfpathrectangle{\pgfqpoint{6.572727in}{0.474100in}}{\pgfqpoint{4.227273in}{3.318700in}}%
\pgfusepath{clip}%
\pgfsetbuttcap%
\pgfsetroundjoin%
\definecolor{currentfill}{rgb}{0.127568,0.566949,0.550556}%
\pgfsetfillcolor{currentfill}%
\pgfsetfillopacity{0.700000}%
\pgfsetlinewidth{0.000000pt}%
\definecolor{currentstroke}{rgb}{0.000000,0.000000,0.000000}%
\pgfsetstrokecolor{currentstroke}%
\pgfsetstrokeopacity{0.700000}%
\pgfsetdash{}{0pt}%
\pgfpathmoveto{\pgfqpoint{7.666068in}{1.317929in}}%
\pgfpathcurveto{\pgfqpoint{7.671112in}{1.317929in}}{\pgfqpoint{7.675950in}{1.319932in}}{\pgfqpoint{7.679516in}{1.323499in}}%
\pgfpathcurveto{\pgfqpoint{7.683083in}{1.327065in}}{\pgfqpoint{7.685086in}{1.331903in}}{\pgfqpoint{7.685086in}{1.336947in}}%
\pgfpathcurveto{\pgfqpoint{7.685086in}{1.341990in}}{\pgfqpoint{7.683083in}{1.346828in}}{\pgfqpoint{7.679516in}{1.350395in}}%
\pgfpathcurveto{\pgfqpoint{7.675950in}{1.353961in}}{\pgfqpoint{7.671112in}{1.355965in}}{\pgfqpoint{7.666068in}{1.355965in}}%
\pgfpathcurveto{\pgfqpoint{7.661025in}{1.355965in}}{\pgfqpoint{7.656187in}{1.353961in}}{\pgfqpoint{7.652620in}{1.350395in}}%
\pgfpathcurveto{\pgfqpoint{7.649054in}{1.346828in}}{\pgfqpoint{7.647050in}{1.341990in}}{\pgfqpoint{7.647050in}{1.336947in}}%
\pgfpathcurveto{\pgfqpoint{7.647050in}{1.331903in}}{\pgfqpoint{7.649054in}{1.327065in}}{\pgfqpoint{7.652620in}{1.323499in}}%
\pgfpathcurveto{\pgfqpoint{7.656187in}{1.319932in}}{\pgfqpoint{7.661025in}{1.317929in}}{\pgfqpoint{7.666068in}{1.317929in}}%
\pgfpathclose%
\pgfusepath{fill}%
\end{pgfscope}%
\begin{pgfscope}%
\pgfpathrectangle{\pgfqpoint{6.572727in}{0.474100in}}{\pgfqpoint{4.227273in}{3.318700in}}%
\pgfusepath{clip}%
\pgfsetbuttcap%
\pgfsetroundjoin%
\definecolor{currentfill}{rgb}{0.993248,0.906157,0.143936}%
\pgfsetfillcolor{currentfill}%
\pgfsetfillopacity{0.700000}%
\pgfsetlinewidth{0.000000pt}%
\definecolor{currentstroke}{rgb}{0.000000,0.000000,0.000000}%
\pgfsetstrokecolor{currentstroke}%
\pgfsetstrokeopacity{0.700000}%
\pgfsetdash{}{0pt}%
\pgfpathmoveto{\pgfqpoint{9.636833in}{1.728385in}}%
\pgfpathcurveto{\pgfqpoint{9.641876in}{1.728385in}}{\pgfqpoint{9.646714in}{1.730389in}}{\pgfqpoint{9.650280in}{1.733955in}}%
\pgfpathcurveto{\pgfqpoint{9.653847in}{1.737522in}}{\pgfqpoint{9.655851in}{1.742359in}}{\pgfqpoint{9.655851in}{1.747403in}}%
\pgfpathcurveto{\pgfqpoint{9.655851in}{1.752447in}}{\pgfqpoint{9.653847in}{1.757285in}}{\pgfqpoint{9.650280in}{1.760851in}}%
\pgfpathcurveto{\pgfqpoint{9.646714in}{1.764417in}}{\pgfqpoint{9.641876in}{1.766421in}}{\pgfqpoint{9.636833in}{1.766421in}}%
\pgfpathcurveto{\pgfqpoint{9.631789in}{1.766421in}}{\pgfqpoint{9.626951in}{1.764417in}}{\pgfqpoint{9.623385in}{1.760851in}}%
\pgfpathcurveto{\pgfqpoint{9.619818in}{1.757285in}}{\pgfqpoint{9.617814in}{1.752447in}}{\pgfqpoint{9.617814in}{1.747403in}}%
\pgfpathcurveto{\pgfqpoint{9.617814in}{1.742359in}}{\pgfqpoint{9.619818in}{1.737522in}}{\pgfqpoint{9.623385in}{1.733955in}}%
\pgfpathcurveto{\pgfqpoint{9.626951in}{1.730389in}}{\pgfqpoint{9.631789in}{1.728385in}}{\pgfqpoint{9.636833in}{1.728385in}}%
\pgfpathclose%
\pgfusepath{fill}%
\end{pgfscope}%
\begin{pgfscope}%
\pgfpathrectangle{\pgfqpoint{6.572727in}{0.474100in}}{\pgfqpoint{4.227273in}{3.318700in}}%
\pgfusepath{clip}%
\pgfsetbuttcap%
\pgfsetroundjoin%
\definecolor{currentfill}{rgb}{0.993248,0.906157,0.143936}%
\pgfsetfillcolor{currentfill}%
\pgfsetfillopacity{0.700000}%
\pgfsetlinewidth{0.000000pt}%
\definecolor{currentstroke}{rgb}{0.000000,0.000000,0.000000}%
\pgfsetstrokecolor{currentstroke}%
\pgfsetstrokeopacity{0.700000}%
\pgfsetdash{}{0pt}%
\pgfpathmoveto{\pgfqpoint{9.946611in}{1.902850in}}%
\pgfpathcurveto{\pgfqpoint{9.951655in}{1.902850in}}{\pgfqpoint{9.956493in}{1.904853in}}{\pgfqpoint{9.960059in}{1.908420in}}%
\pgfpathcurveto{\pgfqpoint{9.963625in}{1.911986in}}{\pgfqpoint{9.965629in}{1.916824in}}{\pgfqpoint{9.965629in}{1.921868in}}%
\pgfpathcurveto{\pgfqpoint{9.965629in}{1.926911in}}{\pgfqpoint{9.963625in}{1.931749in}}{\pgfqpoint{9.960059in}{1.935316in}}%
\pgfpathcurveto{\pgfqpoint{9.956493in}{1.938882in}}{\pgfqpoint{9.951655in}{1.940886in}}{\pgfqpoint{9.946611in}{1.940886in}}%
\pgfpathcurveto{\pgfqpoint{9.941568in}{1.940886in}}{\pgfqpoint{9.936730in}{1.938882in}}{\pgfqpoint{9.933163in}{1.935316in}}%
\pgfpathcurveto{\pgfqpoint{9.929597in}{1.931749in}}{\pgfqpoint{9.927593in}{1.926911in}}{\pgfqpoint{9.927593in}{1.921868in}}%
\pgfpathcurveto{\pgfqpoint{9.927593in}{1.916824in}}{\pgfqpoint{9.929597in}{1.911986in}}{\pgfqpoint{9.933163in}{1.908420in}}%
\pgfpathcurveto{\pgfqpoint{9.936730in}{1.904853in}}{\pgfqpoint{9.941568in}{1.902850in}}{\pgfqpoint{9.946611in}{1.902850in}}%
\pgfpathclose%
\pgfusepath{fill}%
\end{pgfscope}%
\begin{pgfscope}%
\pgfpathrectangle{\pgfqpoint{6.572727in}{0.474100in}}{\pgfqpoint{4.227273in}{3.318700in}}%
\pgfusepath{clip}%
\pgfsetbuttcap%
\pgfsetroundjoin%
\definecolor{currentfill}{rgb}{0.127568,0.566949,0.550556}%
\pgfsetfillcolor{currentfill}%
\pgfsetfillopacity{0.700000}%
\pgfsetlinewidth{0.000000pt}%
\definecolor{currentstroke}{rgb}{0.000000,0.000000,0.000000}%
\pgfsetstrokecolor{currentstroke}%
\pgfsetstrokeopacity{0.700000}%
\pgfsetdash{}{0pt}%
\pgfpathmoveto{\pgfqpoint{8.294514in}{1.454814in}}%
\pgfpathcurveto{\pgfqpoint{8.299558in}{1.454814in}}{\pgfqpoint{8.304396in}{1.456818in}}{\pgfqpoint{8.307962in}{1.460385in}}%
\pgfpathcurveto{\pgfqpoint{8.311528in}{1.463951in}}{\pgfqpoint{8.313532in}{1.468789in}}{\pgfqpoint{8.313532in}{1.473833in}}%
\pgfpathcurveto{\pgfqpoint{8.313532in}{1.478876in}}{\pgfqpoint{8.311528in}{1.483714in}}{\pgfqpoint{8.307962in}{1.487280in}}%
\pgfpathcurveto{\pgfqpoint{8.304396in}{1.490847in}}{\pgfqpoint{8.299558in}{1.492851in}}{\pgfqpoint{8.294514in}{1.492851in}}%
\pgfpathcurveto{\pgfqpoint{8.289470in}{1.492851in}}{\pgfqpoint{8.284633in}{1.490847in}}{\pgfqpoint{8.281066in}{1.487280in}}%
\pgfpathcurveto{\pgfqpoint{8.277500in}{1.483714in}}{\pgfqpoint{8.275496in}{1.478876in}}{\pgfqpoint{8.275496in}{1.473833in}}%
\pgfpathcurveto{\pgfqpoint{8.275496in}{1.468789in}}{\pgfqpoint{8.277500in}{1.463951in}}{\pgfqpoint{8.281066in}{1.460385in}}%
\pgfpathcurveto{\pgfqpoint{8.284633in}{1.456818in}}{\pgfqpoint{8.289470in}{1.454814in}}{\pgfqpoint{8.294514in}{1.454814in}}%
\pgfpathclose%
\pgfusepath{fill}%
\end{pgfscope}%
\begin{pgfscope}%
\pgfpathrectangle{\pgfqpoint{6.572727in}{0.474100in}}{\pgfqpoint{4.227273in}{3.318700in}}%
\pgfusepath{clip}%
\pgfsetbuttcap%
\pgfsetroundjoin%
\definecolor{currentfill}{rgb}{0.127568,0.566949,0.550556}%
\pgfsetfillcolor{currentfill}%
\pgfsetfillopacity{0.700000}%
\pgfsetlinewidth{0.000000pt}%
\definecolor{currentstroke}{rgb}{0.000000,0.000000,0.000000}%
\pgfsetstrokecolor{currentstroke}%
\pgfsetstrokeopacity{0.700000}%
\pgfsetdash{}{0pt}%
\pgfpathmoveto{\pgfqpoint{7.598975in}{1.203723in}}%
\pgfpathcurveto{\pgfqpoint{7.604018in}{1.203723in}}{\pgfqpoint{7.608856in}{1.205727in}}{\pgfqpoint{7.612422in}{1.209293in}}%
\pgfpathcurveto{\pgfqpoint{7.615989in}{1.212859in}}{\pgfqpoint{7.617993in}{1.217697in}}{\pgfqpoint{7.617993in}{1.222741in}}%
\pgfpathcurveto{\pgfqpoint{7.617993in}{1.227784in}}{\pgfqpoint{7.615989in}{1.232622in}}{\pgfqpoint{7.612422in}{1.236189in}}%
\pgfpathcurveto{\pgfqpoint{7.608856in}{1.239755in}}{\pgfqpoint{7.604018in}{1.241759in}}{\pgfqpoint{7.598975in}{1.241759in}}%
\pgfpathcurveto{\pgfqpoint{7.593931in}{1.241759in}}{\pgfqpoint{7.589093in}{1.239755in}}{\pgfqpoint{7.585527in}{1.236189in}}%
\pgfpathcurveto{\pgfqpoint{7.581960in}{1.232622in}}{\pgfqpoint{7.579956in}{1.227784in}}{\pgfqpoint{7.579956in}{1.222741in}}%
\pgfpathcurveto{\pgfqpoint{7.579956in}{1.217697in}}{\pgfqpoint{7.581960in}{1.212859in}}{\pgfqpoint{7.585527in}{1.209293in}}%
\pgfpathcurveto{\pgfqpoint{7.589093in}{1.205727in}}{\pgfqpoint{7.593931in}{1.203723in}}{\pgfqpoint{7.598975in}{1.203723in}}%
\pgfpathclose%
\pgfusepath{fill}%
\end{pgfscope}%
\begin{pgfscope}%
\pgfpathrectangle{\pgfqpoint{6.572727in}{0.474100in}}{\pgfqpoint{4.227273in}{3.318700in}}%
\pgfusepath{clip}%
\pgfsetbuttcap%
\pgfsetroundjoin%
\definecolor{currentfill}{rgb}{0.127568,0.566949,0.550556}%
\pgfsetfillcolor{currentfill}%
\pgfsetfillopacity{0.700000}%
\pgfsetlinewidth{0.000000pt}%
\definecolor{currentstroke}{rgb}{0.000000,0.000000,0.000000}%
\pgfsetstrokecolor{currentstroke}%
\pgfsetstrokeopacity{0.700000}%
\pgfsetdash{}{0pt}%
\pgfpathmoveto{\pgfqpoint{7.973042in}{3.331380in}}%
\pgfpathcurveto{\pgfqpoint{7.978086in}{3.331380in}}{\pgfqpoint{7.982924in}{3.333384in}}{\pgfqpoint{7.986490in}{3.336950in}}%
\pgfpathcurveto{\pgfqpoint{7.990057in}{3.340516in}}{\pgfqpoint{7.992061in}{3.345354in}}{\pgfqpoint{7.992061in}{3.350398in}}%
\pgfpathcurveto{\pgfqpoint{7.992061in}{3.355441in}}{\pgfqpoint{7.990057in}{3.360279in}}{\pgfqpoint{7.986490in}{3.363846in}}%
\pgfpathcurveto{\pgfqpoint{7.982924in}{3.367412in}}{\pgfqpoint{7.978086in}{3.369416in}}{\pgfqpoint{7.973042in}{3.369416in}}%
\pgfpathcurveto{\pgfqpoint{7.967999in}{3.369416in}}{\pgfqpoint{7.963161in}{3.367412in}}{\pgfqpoint{7.959595in}{3.363846in}}%
\pgfpathcurveto{\pgfqpoint{7.956028in}{3.360279in}}{\pgfqpoint{7.954024in}{3.355441in}}{\pgfqpoint{7.954024in}{3.350398in}}%
\pgfpathcurveto{\pgfqpoint{7.954024in}{3.345354in}}{\pgfqpoint{7.956028in}{3.340516in}}{\pgfqpoint{7.959595in}{3.336950in}}%
\pgfpathcurveto{\pgfqpoint{7.963161in}{3.333384in}}{\pgfqpoint{7.967999in}{3.331380in}}{\pgfqpoint{7.973042in}{3.331380in}}%
\pgfpathclose%
\pgfusepath{fill}%
\end{pgfscope}%
\begin{pgfscope}%
\pgfpathrectangle{\pgfqpoint{6.572727in}{0.474100in}}{\pgfqpoint{4.227273in}{3.318700in}}%
\pgfusepath{clip}%
\pgfsetbuttcap%
\pgfsetroundjoin%
\definecolor{currentfill}{rgb}{0.127568,0.566949,0.550556}%
\pgfsetfillcolor{currentfill}%
\pgfsetfillopacity{0.700000}%
\pgfsetlinewidth{0.000000pt}%
\definecolor{currentstroke}{rgb}{0.000000,0.000000,0.000000}%
\pgfsetstrokecolor{currentstroke}%
\pgfsetstrokeopacity{0.700000}%
\pgfsetdash{}{0pt}%
\pgfpathmoveto{\pgfqpoint{7.046167in}{1.754201in}}%
\pgfpathcurveto{\pgfqpoint{7.051210in}{1.754201in}}{\pgfqpoint{7.056048in}{1.756205in}}{\pgfqpoint{7.059614in}{1.759771in}}%
\pgfpathcurveto{\pgfqpoint{7.063181in}{1.763338in}}{\pgfqpoint{7.065185in}{1.768175in}}{\pgfqpoint{7.065185in}{1.773219in}}%
\pgfpathcurveto{\pgfqpoint{7.065185in}{1.778263in}}{\pgfqpoint{7.063181in}{1.783101in}}{\pgfqpoint{7.059614in}{1.786667in}}%
\pgfpathcurveto{\pgfqpoint{7.056048in}{1.790233in}}{\pgfqpoint{7.051210in}{1.792237in}}{\pgfqpoint{7.046167in}{1.792237in}}%
\pgfpathcurveto{\pgfqpoint{7.041123in}{1.792237in}}{\pgfqpoint{7.036285in}{1.790233in}}{\pgfqpoint{7.032719in}{1.786667in}}%
\pgfpathcurveto{\pgfqpoint{7.029152in}{1.783101in}}{\pgfqpoint{7.027148in}{1.778263in}}{\pgfqpoint{7.027148in}{1.773219in}}%
\pgfpathcurveto{\pgfqpoint{7.027148in}{1.768175in}}{\pgfqpoint{7.029152in}{1.763338in}}{\pgfqpoint{7.032719in}{1.759771in}}%
\pgfpathcurveto{\pgfqpoint{7.036285in}{1.756205in}}{\pgfqpoint{7.041123in}{1.754201in}}{\pgfqpoint{7.046167in}{1.754201in}}%
\pgfpathclose%
\pgfusepath{fill}%
\end{pgfscope}%
\begin{pgfscope}%
\pgfpathrectangle{\pgfqpoint{6.572727in}{0.474100in}}{\pgfqpoint{4.227273in}{3.318700in}}%
\pgfusepath{clip}%
\pgfsetbuttcap%
\pgfsetroundjoin%
\definecolor{currentfill}{rgb}{0.127568,0.566949,0.550556}%
\pgfsetfillcolor{currentfill}%
\pgfsetfillopacity{0.700000}%
\pgfsetlinewidth{0.000000pt}%
\definecolor{currentstroke}{rgb}{0.000000,0.000000,0.000000}%
\pgfsetstrokecolor{currentstroke}%
\pgfsetstrokeopacity{0.700000}%
\pgfsetdash{}{0pt}%
\pgfpathmoveto{\pgfqpoint{8.564128in}{2.906799in}}%
\pgfpathcurveto{\pgfqpoint{8.569171in}{2.906799in}}{\pgfqpoint{8.574009in}{2.908803in}}{\pgfqpoint{8.577575in}{2.912370in}}%
\pgfpathcurveto{\pgfqpoint{8.581142in}{2.915936in}}{\pgfqpoint{8.583146in}{2.920774in}}{\pgfqpoint{8.583146in}{2.925817in}}%
\pgfpathcurveto{\pgfqpoint{8.583146in}{2.930861in}}{\pgfqpoint{8.581142in}{2.935699in}}{\pgfqpoint{8.577575in}{2.939265in}}%
\pgfpathcurveto{\pgfqpoint{8.574009in}{2.942832in}}{\pgfqpoint{8.569171in}{2.944836in}}{\pgfqpoint{8.564128in}{2.944836in}}%
\pgfpathcurveto{\pgfqpoint{8.559084in}{2.944836in}}{\pgfqpoint{8.554246in}{2.942832in}}{\pgfqpoint{8.550680in}{2.939265in}}%
\pgfpathcurveto{\pgfqpoint{8.547113in}{2.935699in}}{\pgfqpoint{8.545109in}{2.930861in}}{\pgfqpoint{8.545109in}{2.925817in}}%
\pgfpathcurveto{\pgfqpoint{8.545109in}{2.920774in}}{\pgfqpoint{8.547113in}{2.915936in}}{\pgfqpoint{8.550680in}{2.912370in}}%
\pgfpathcurveto{\pgfqpoint{8.554246in}{2.908803in}}{\pgfqpoint{8.559084in}{2.906799in}}{\pgfqpoint{8.564128in}{2.906799in}}%
\pgfpathclose%
\pgfusepath{fill}%
\end{pgfscope}%
\begin{pgfscope}%
\pgfpathrectangle{\pgfqpoint{6.572727in}{0.474100in}}{\pgfqpoint{4.227273in}{3.318700in}}%
\pgfusepath{clip}%
\pgfsetbuttcap%
\pgfsetroundjoin%
\definecolor{currentfill}{rgb}{0.993248,0.906157,0.143936}%
\pgfsetfillcolor{currentfill}%
\pgfsetfillopacity{0.700000}%
\pgfsetlinewidth{0.000000pt}%
\definecolor{currentstroke}{rgb}{0.000000,0.000000,0.000000}%
\pgfsetstrokecolor{currentstroke}%
\pgfsetstrokeopacity{0.700000}%
\pgfsetdash{}{0pt}%
\pgfpathmoveto{\pgfqpoint{10.011983in}{1.213553in}}%
\pgfpathcurveto{\pgfqpoint{10.017026in}{1.213553in}}{\pgfqpoint{10.021864in}{1.215557in}}{\pgfqpoint{10.025431in}{1.219123in}}%
\pgfpathcurveto{\pgfqpoint{10.028997in}{1.222690in}}{\pgfqpoint{10.031001in}{1.227528in}}{\pgfqpoint{10.031001in}{1.232571in}}%
\pgfpathcurveto{\pgfqpoint{10.031001in}{1.237615in}}{\pgfqpoint{10.028997in}{1.242453in}}{\pgfqpoint{10.025431in}{1.246019in}}%
\pgfpathcurveto{\pgfqpoint{10.021864in}{1.249586in}}{\pgfqpoint{10.017026in}{1.251589in}}{\pgfqpoint{10.011983in}{1.251589in}}%
\pgfpathcurveto{\pgfqpoint{10.006939in}{1.251589in}}{\pgfqpoint{10.002101in}{1.249586in}}{\pgfqpoint{9.998535in}{1.246019in}}%
\pgfpathcurveto{\pgfqpoint{9.994968in}{1.242453in}}{\pgfqpoint{9.992965in}{1.237615in}}{\pgfqpoint{9.992965in}{1.232571in}}%
\pgfpathcurveto{\pgfqpoint{9.992965in}{1.227528in}}{\pgfqpoint{9.994968in}{1.222690in}}{\pgfqpoint{9.998535in}{1.219123in}}%
\pgfpathcurveto{\pgfqpoint{10.002101in}{1.215557in}}{\pgfqpoint{10.006939in}{1.213553in}}{\pgfqpoint{10.011983in}{1.213553in}}%
\pgfpathclose%
\pgfusepath{fill}%
\end{pgfscope}%
\begin{pgfscope}%
\pgfpathrectangle{\pgfqpoint{6.572727in}{0.474100in}}{\pgfqpoint{4.227273in}{3.318700in}}%
\pgfusepath{clip}%
\pgfsetbuttcap%
\pgfsetroundjoin%
\definecolor{currentfill}{rgb}{0.993248,0.906157,0.143936}%
\pgfsetfillcolor{currentfill}%
\pgfsetfillopacity{0.700000}%
\pgfsetlinewidth{0.000000pt}%
\definecolor{currentstroke}{rgb}{0.000000,0.000000,0.000000}%
\pgfsetstrokecolor{currentstroke}%
\pgfsetstrokeopacity{0.700000}%
\pgfsetdash{}{0pt}%
\pgfpathmoveto{\pgfqpoint{9.948581in}{1.465199in}}%
\pgfpathcurveto{\pgfqpoint{9.953625in}{1.465199in}}{\pgfqpoint{9.958462in}{1.467202in}}{\pgfqpoint{9.962029in}{1.470769in}}%
\pgfpathcurveto{\pgfqpoint{9.965595in}{1.474335in}}{\pgfqpoint{9.967599in}{1.479173in}}{\pgfqpoint{9.967599in}{1.484217in}}%
\pgfpathcurveto{\pgfqpoint{9.967599in}{1.489260in}}{\pgfqpoint{9.965595in}{1.494098in}}{\pgfqpoint{9.962029in}{1.497665in}}%
\pgfpathcurveto{\pgfqpoint{9.958462in}{1.501231in}}{\pgfqpoint{9.953625in}{1.503235in}}{\pgfqpoint{9.948581in}{1.503235in}}%
\pgfpathcurveto{\pgfqpoint{9.943537in}{1.503235in}}{\pgfqpoint{9.938700in}{1.501231in}}{\pgfqpoint{9.935133in}{1.497665in}}%
\pgfpathcurveto{\pgfqpoint{9.931567in}{1.494098in}}{\pgfqpoint{9.929563in}{1.489260in}}{\pgfqpoint{9.929563in}{1.484217in}}%
\pgfpathcurveto{\pgfqpoint{9.929563in}{1.479173in}}{\pgfqpoint{9.931567in}{1.474335in}}{\pgfqpoint{9.935133in}{1.470769in}}%
\pgfpathcurveto{\pgfqpoint{9.938700in}{1.467202in}}{\pgfqpoint{9.943537in}{1.465199in}}{\pgfqpoint{9.948581in}{1.465199in}}%
\pgfpathclose%
\pgfusepath{fill}%
\end{pgfscope}%
\begin{pgfscope}%
\pgfpathrectangle{\pgfqpoint{6.572727in}{0.474100in}}{\pgfqpoint{4.227273in}{3.318700in}}%
\pgfusepath{clip}%
\pgfsetbuttcap%
\pgfsetroundjoin%
\definecolor{currentfill}{rgb}{0.127568,0.566949,0.550556}%
\pgfsetfillcolor{currentfill}%
\pgfsetfillopacity{0.700000}%
\pgfsetlinewidth{0.000000pt}%
\definecolor{currentstroke}{rgb}{0.000000,0.000000,0.000000}%
\pgfsetstrokecolor{currentstroke}%
\pgfsetstrokeopacity{0.700000}%
\pgfsetdash{}{0pt}%
\pgfpathmoveto{\pgfqpoint{8.444460in}{2.961181in}}%
\pgfpathcurveto{\pgfqpoint{8.449504in}{2.961181in}}{\pgfqpoint{8.454342in}{2.963185in}}{\pgfqpoint{8.457908in}{2.966751in}}%
\pgfpathcurveto{\pgfqpoint{8.461475in}{2.970317in}}{\pgfqpoint{8.463479in}{2.975155in}}{\pgfqpoint{8.463479in}{2.980199in}}%
\pgfpathcurveto{\pgfqpoint{8.463479in}{2.985243in}}{\pgfqpoint{8.461475in}{2.990080in}}{\pgfqpoint{8.457908in}{2.993647in}}%
\pgfpathcurveto{\pgfqpoint{8.454342in}{2.997213in}}{\pgfqpoint{8.449504in}{2.999217in}}{\pgfqpoint{8.444460in}{2.999217in}}%
\pgfpathcurveto{\pgfqpoint{8.439417in}{2.999217in}}{\pgfqpoint{8.434579in}{2.997213in}}{\pgfqpoint{8.431013in}{2.993647in}}%
\pgfpathcurveto{\pgfqpoint{8.427446in}{2.990080in}}{\pgfqpoint{8.425442in}{2.985243in}}{\pgfqpoint{8.425442in}{2.980199in}}%
\pgfpathcurveto{\pgfqpoint{8.425442in}{2.975155in}}{\pgfqpoint{8.427446in}{2.970317in}}{\pgfqpoint{8.431013in}{2.966751in}}%
\pgfpathcurveto{\pgfqpoint{8.434579in}{2.963185in}}{\pgfqpoint{8.439417in}{2.961181in}}{\pgfqpoint{8.444460in}{2.961181in}}%
\pgfpathclose%
\pgfusepath{fill}%
\end{pgfscope}%
\begin{pgfscope}%
\pgfpathrectangle{\pgfqpoint{6.572727in}{0.474100in}}{\pgfqpoint{4.227273in}{3.318700in}}%
\pgfusepath{clip}%
\pgfsetbuttcap%
\pgfsetroundjoin%
\definecolor{currentfill}{rgb}{0.127568,0.566949,0.550556}%
\pgfsetfillcolor{currentfill}%
\pgfsetfillopacity{0.700000}%
\pgfsetlinewidth{0.000000pt}%
\definecolor{currentstroke}{rgb}{0.000000,0.000000,0.000000}%
\pgfsetstrokecolor{currentstroke}%
\pgfsetstrokeopacity{0.700000}%
\pgfsetdash{}{0pt}%
\pgfpathmoveto{\pgfqpoint{7.196845in}{1.258952in}}%
\pgfpathcurveto{\pgfqpoint{7.201889in}{1.258952in}}{\pgfqpoint{7.206727in}{1.260956in}}{\pgfqpoint{7.210293in}{1.264522in}}%
\pgfpathcurveto{\pgfqpoint{7.213860in}{1.268089in}}{\pgfqpoint{7.215864in}{1.272926in}}{\pgfqpoint{7.215864in}{1.277970in}}%
\pgfpathcurveto{\pgfqpoint{7.215864in}{1.283014in}}{\pgfqpoint{7.213860in}{1.287851in}}{\pgfqpoint{7.210293in}{1.291418in}}%
\pgfpathcurveto{\pgfqpoint{7.206727in}{1.294984in}}{\pgfqpoint{7.201889in}{1.296988in}}{\pgfqpoint{7.196845in}{1.296988in}}%
\pgfpathcurveto{\pgfqpoint{7.191802in}{1.296988in}}{\pgfqpoint{7.186964in}{1.294984in}}{\pgfqpoint{7.183398in}{1.291418in}}%
\pgfpathcurveto{\pgfqpoint{7.179831in}{1.287851in}}{\pgfqpoint{7.177827in}{1.283014in}}{\pgfqpoint{7.177827in}{1.277970in}}%
\pgfpathcurveto{\pgfqpoint{7.177827in}{1.272926in}}{\pgfqpoint{7.179831in}{1.268089in}}{\pgfqpoint{7.183398in}{1.264522in}}%
\pgfpathcurveto{\pgfqpoint{7.186964in}{1.260956in}}{\pgfqpoint{7.191802in}{1.258952in}}{\pgfqpoint{7.196845in}{1.258952in}}%
\pgfpathclose%
\pgfusepath{fill}%
\end{pgfscope}%
\begin{pgfscope}%
\pgfpathrectangle{\pgfqpoint{6.572727in}{0.474100in}}{\pgfqpoint{4.227273in}{3.318700in}}%
\pgfusepath{clip}%
\pgfsetbuttcap%
\pgfsetroundjoin%
\definecolor{currentfill}{rgb}{0.127568,0.566949,0.550556}%
\pgfsetfillcolor{currentfill}%
\pgfsetfillopacity{0.700000}%
\pgfsetlinewidth{0.000000pt}%
\definecolor{currentstroke}{rgb}{0.000000,0.000000,0.000000}%
\pgfsetstrokecolor{currentstroke}%
\pgfsetstrokeopacity{0.700000}%
\pgfsetdash{}{0pt}%
\pgfpathmoveto{\pgfqpoint{8.261624in}{2.751186in}}%
\pgfpathcurveto{\pgfqpoint{8.266668in}{2.751186in}}{\pgfqpoint{8.271506in}{2.753190in}}{\pgfqpoint{8.275072in}{2.756756in}}%
\pgfpathcurveto{\pgfqpoint{8.278639in}{2.760322in}}{\pgfqpoint{8.280643in}{2.765160in}}{\pgfqpoint{8.280643in}{2.770204in}}%
\pgfpathcurveto{\pgfqpoint{8.280643in}{2.775248in}}{\pgfqpoint{8.278639in}{2.780085in}}{\pgfqpoint{8.275072in}{2.783652in}}%
\pgfpathcurveto{\pgfqpoint{8.271506in}{2.787218in}}{\pgfqpoint{8.266668in}{2.789222in}}{\pgfqpoint{8.261624in}{2.789222in}}%
\pgfpathcurveto{\pgfqpoint{8.256581in}{2.789222in}}{\pgfqpoint{8.251743in}{2.787218in}}{\pgfqpoint{8.248177in}{2.783652in}}%
\pgfpathcurveto{\pgfqpoint{8.244610in}{2.780085in}}{\pgfqpoint{8.242606in}{2.775248in}}{\pgfqpoint{8.242606in}{2.770204in}}%
\pgfpathcurveto{\pgfqpoint{8.242606in}{2.765160in}}{\pgfqpoint{8.244610in}{2.760322in}}{\pgfqpoint{8.248177in}{2.756756in}}%
\pgfpathcurveto{\pgfqpoint{8.251743in}{2.753190in}}{\pgfqpoint{8.256581in}{2.751186in}}{\pgfqpoint{8.261624in}{2.751186in}}%
\pgfpathclose%
\pgfusepath{fill}%
\end{pgfscope}%
\begin{pgfscope}%
\pgfpathrectangle{\pgfqpoint{6.572727in}{0.474100in}}{\pgfqpoint{4.227273in}{3.318700in}}%
\pgfusepath{clip}%
\pgfsetbuttcap%
\pgfsetroundjoin%
\definecolor{currentfill}{rgb}{0.993248,0.906157,0.143936}%
\pgfsetfillcolor{currentfill}%
\pgfsetfillopacity{0.700000}%
\pgfsetlinewidth{0.000000pt}%
\definecolor{currentstroke}{rgb}{0.000000,0.000000,0.000000}%
\pgfsetstrokecolor{currentstroke}%
\pgfsetstrokeopacity{0.700000}%
\pgfsetdash{}{0pt}%
\pgfpathmoveto{\pgfqpoint{9.804221in}{1.719766in}}%
\pgfpathcurveto{\pgfqpoint{9.809265in}{1.719766in}}{\pgfqpoint{9.814103in}{1.721770in}}{\pgfqpoint{9.817669in}{1.725336in}}%
\pgfpathcurveto{\pgfqpoint{9.821236in}{1.728903in}}{\pgfqpoint{9.823239in}{1.733741in}}{\pgfqpoint{9.823239in}{1.738784in}}%
\pgfpathcurveto{\pgfqpoint{9.823239in}{1.743828in}}{\pgfqpoint{9.821236in}{1.748666in}}{\pgfqpoint{9.817669in}{1.752232in}}%
\pgfpathcurveto{\pgfqpoint{9.814103in}{1.755799in}}{\pgfqpoint{9.809265in}{1.757802in}}{\pgfqpoint{9.804221in}{1.757802in}}%
\pgfpathcurveto{\pgfqpoint{9.799178in}{1.757802in}}{\pgfqpoint{9.794340in}{1.755799in}}{\pgfqpoint{9.790773in}{1.752232in}}%
\pgfpathcurveto{\pgfqpoint{9.787207in}{1.748666in}}{\pgfqpoint{9.785203in}{1.743828in}}{\pgfqpoint{9.785203in}{1.738784in}}%
\pgfpathcurveto{\pgfqpoint{9.785203in}{1.733741in}}{\pgfqpoint{9.787207in}{1.728903in}}{\pgfqpoint{9.790773in}{1.725336in}}%
\pgfpathcurveto{\pgfqpoint{9.794340in}{1.721770in}}{\pgfqpoint{9.799178in}{1.719766in}}{\pgfqpoint{9.804221in}{1.719766in}}%
\pgfpathclose%
\pgfusepath{fill}%
\end{pgfscope}%
\begin{pgfscope}%
\pgfpathrectangle{\pgfqpoint{6.572727in}{0.474100in}}{\pgfqpoint{4.227273in}{3.318700in}}%
\pgfusepath{clip}%
\pgfsetbuttcap%
\pgfsetroundjoin%
\definecolor{currentfill}{rgb}{0.993248,0.906157,0.143936}%
\pgfsetfillcolor{currentfill}%
\pgfsetfillopacity{0.700000}%
\pgfsetlinewidth{0.000000pt}%
\definecolor{currentstroke}{rgb}{0.000000,0.000000,0.000000}%
\pgfsetstrokecolor{currentstroke}%
\pgfsetstrokeopacity{0.700000}%
\pgfsetdash{}{0pt}%
\pgfpathmoveto{\pgfqpoint{9.471324in}{1.421160in}}%
\pgfpathcurveto{\pgfqpoint{9.476368in}{1.421160in}}{\pgfqpoint{9.481206in}{1.423164in}}{\pgfqpoint{9.484772in}{1.426730in}}%
\pgfpathcurveto{\pgfqpoint{9.488339in}{1.430297in}}{\pgfqpoint{9.490343in}{1.435135in}}{\pgfqpoint{9.490343in}{1.440178in}}%
\pgfpathcurveto{\pgfqpoint{9.490343in}{1.445222in}}{\pgfqpoint{9.488339in}{1.450060in}}{\pgfqpoint{9.484772in}{1.453626in}}%
\pgfpathcurveto{\pgfqpoint{9.481206in}{1.457192in}}{\pgfqpoint{9.476368in}{1.459196in}}{\pgfqpoint{9.471324in}{1.459196in}}%
\pgfpathcurveto{\pgfqpoint{9.466281in}{1.459196in}}{\pgfqpoint{9.461443in}{1.457192in}}{\pgfqpoint{9.457877in}{1.453626in}}%
\pgfpathcurveto{\pgfqpoint{9.454310in}{1.450060in}}{\pgfqpoint{9.452306in}{1.445222in}}{\pgfqpoint{9.452306in}{1.440178in}}%
\pgfpathcurveto{\pgfqpoint{9.452306in}{1.435135in}}{\pgfqpoint{9.454310in}{1.430297in}}{\pgfqpoint{9.457877in}{1.426730in}}%
\pgfpathcurveto{\pgfqpoint{9.461443in}{1.423164in}}{\pgfqpoint{9.466281in}{1.421160in}}{\pgfqpoint{9.471324in}{1.421160in}}%
\pgfpathclose%
\pgfusepath{fill}%
\end{pgfscope}%
\begin{pgfscope}%
\pgfpathrectangle{\pgfqpoint{6.572727in}{0.474100in}}{\pgfqpoint{4.227273in}{3.318700in}}%
\pgfusepath{clip}%
\pgfsetbuttcap%
\pgfsetroundjoin%
\definecolor{currentfill}{rgb}{0.993248,0.906157,0.143936}%
\pgfsetfillcolor{currentfill}%
\pgfsetfillopacity{0.700000}%
\pgfsetlinewidth{0.000000pt}%
\definecolor{currentstroke}{rgb}{0.000000,0.000000,0.000000}%
\pgfsetstrokecolor{currentstroke}%
\pgfsetstrokeopacity{0.700000}%
\pgfsetdash{}{0pt}%
\pgfpathmoveto{\pgfqpoint{9.078876in}{2.326257in}}%
\pgfpathcurveto{\pgfqpoint{9.083920in}{2.326257in}}{\pgfqpoint{9.088758in}{2.328261in}}{\pgfqpoint{9.092324in}{2.331827in}}%
\pgfpathcurveto{\pgfqpoint{9.095891in}{2.335394in}}{\pgfqpoint{9.097894in}{2.340232in}}{\pgfqpoint{9.097894in}{2.345275in}}%
\pgfpathcurveto{\pgfqpoint{9.097894in}{2.350319in}}{\pgfqpoint{9.095891in}{2.355157in}}{\pgfqpoint{9.092324in}{2.358723in}}%
\pgfpathcurveto{\pgfqpoint{9.088758in}{2.362289in}}{\pgfqpoint{9.083920in}{2.364293in}}{\pgfqpoint{9.078876in}{2.364293in}}%
\pgfpathcurveto{\pgfqpoint{9.073833in}{2.364293in}}{\pgfqpoint{9.068995in}{2.362289in}}{\pgfqpoint{9.065428in}{2.358723in}}%
\pgfpathcurveto{\pgfqpoint{9.061862in}{2.355157in}}{\pgfqpoint{9.059858in}{2.350319in}}{\pgfqpoint{9.059858in}{2.345275in}}%
\pgfpathcurveto{\pgfqpoint{9.059858in}{2.340232in}}{\pgfqpoint{9.061862in}{2.335394in}}{\pgfqpoint{9.065428in}{2.331827in}}%
\pgfpathcurveto{\pgfqpoint{9.068995in}{2.328261in}}{\pgfqpoint{9.073833in}{2.326257in}}{\pgfqpoint{9.078876in}{2.326257in}}%
\pgfpathclose%
\pgfusepath{fill}%
\end{pgfscope}%
\begin{pgfscope}%
\pgfpathrectangle{\pgfqpoint{6.572727in}{0.474100in}}{\pgfqpoint{4.227273in}{3.318700in}}%
\pgfusepath{clip}%
\pgfsetbuttcap%
\pgfsetroundjoin%
\definecolor{currentfill}{rgb}{0.127568,0.566949,0.550556}%
\pgfsetfillcolor{currentfill}%
\pgfsetfillopacity{0.700000}%
\pgfsetlinewidth{0.000000pt}%
\definecolor{currentstroke}{rgb}{0.000000,0.000000,0.000000}%
\pgfsetstrokecolor{currentstroke}%
\pgfsetstrokeopacity{0.700000}%
\pgfsetdash{}{0pt}%
\pgfpathmoveto{\pgfqpoint{8.182845in}{1.344732in}}%
\pgfpathcurveto{\pgfqpoint{8.187889in}{1.344732in}}{\pgfqpoint{8.192727in}{1.346736in}}{\pgfqpoint{8.196293in}{1.350302in}}%
\pgfpathcurveto{\pgfqpoint{8.199860in}{1.353869in}}{\pgfqpoint{8.201863in}{1.358706in}}{\pgfqpoint{8.201863in}{1.363750in}}%
\pgfpathcurveto{\pgfqpoint{8.201863in}{1.368794in}}{\pgfqpoint{8.199860in}{1.373631in}}{\pgfqpoint{8.196293in}{1.377198in}}%
\pgfpathcurveto{\pgfqpoint{8.192727in}{1.380764in}}{\pgfqpoint{8.187889in}{1.382768in}}{\pgfqpoint{8.182845in}{1.382768in}}%
\pgfpathcurveto{\pgfqpoint{8.177802in}{1.382768in}}{\pgfqpoint{8.172964in}{1.380764in}}{\pgfqpoint{8.169397in}{1.377198in}}%
\pgfpathcurveto{\pgfqpoint{8.165831in}{1.373631in}}{\pgfqpoint{8.163827in}{1.368794in}}{\pgfqpoint{8.163827in}{1.363750in}}%
\pgfpathcurveto{\pgfqpoint{8.163827in}{1.358706in}}{\pgfqpoint{8.165831in}{1.353869in}}{\pgfqpoint{8.169397in}{1.350302in}}%
\pgfpathcurveto{\pgfqpoint{8.172964in}{1.346736in}}{\pgfqpoint{8.177802in}{1.344732in}}{\pgfqpoint{8.182845in}{1.344732in}}%
\pgfpathclose%
\pgfusepath{fill}%
\end{pgfscope}%
\begin{pgfscope}%
\pgfpathrectangle{\pgfqpoint{6.572727in}{0.474100in}}{\pgfqpoint{4.227273in}{3.318700in}}%
\pgfusepath{clip}%
\pgfsetbuttcap%
\pgfsetroundjoin%
\definecolor{currentfill}{rgb}{0.127568,0.566949,0.550556}%
\pgfsetfillcolor{currentfill}%
\pgfsetfillopacity{0.700000}%
\pgfsetlinewidth{0.000000pt}%
\definecolor{currentstroke}{rgb}{0.000000,0.000000,0.000000}%
\pgfsetstrokecolor{currentstroke}%
\pgfsetstrokeopacity{0.700000}%
\pgfsetdash{}{0pt}%
\pgfpathmoveto{\pgfqpoint{7.817637in}{1.871247in}}%
\pgfpathcurveto{\pgfqpoint{7.822681in}{1.871247in}}{\pgfqpoint{7.827519in}{1.873251in}}{\pgfqpoint{7.831085in}{1.876817in}}%
\pgfpathcurveto{\pgfqpoint{7.834652in}{1.880384in}}{\pgfqpoint{7.836655in}{1.885221in}}{\pgfqpoint{7.836655in}{1.890265in}}%
\pgfpathcurveto{\pgfqpoint{7.836655in}{1.895309in}}{\pgfqpoint{7.834652in}{1.900147in}}{\pgfqpoint{7.831085in}{1.903713in}}%
\pgfpathcurveto{\pgfqpoint{7.827519in}{1.907279in}}{\pgfqpoint{7.822681in}{1.909283in}}{\pgfqpoint{7.817637in}{1.909283in}}%
\pgfpathcurveto{\pgfqpoint{7.812594in}{1.909283in}}{\pgfqpoint{7.807756in}{1.907279in}}{\pgfqpoint{7.804189in}{1.903713in}}%
\pgfpathcurveto{\pgfqpoint{7.800623in}{1.900147in}}{\pgfqpoint{7.798619in}{1.895309in}}{\pgfqpoint{7.798619in}{1.890265in}}%
\pgfpathcurveto{\pgfqpoint{7.798619in}{1.885221in}}{\pgfqpoint{7.800623in}{1.880384in}}{\pgfqpoint{7.804189in}{1.876817in}}%
\pgfpathcurveto{\pgfqpoint{7.807756in}{1.873251in}}{\pgfqpoint{7.812594in}{1.871247in}}{\pgfqpoint{7.817637in}{1.871247in}}%
\pgfpathclose%
\pgfusepath{fill}%
\end{pgfscope}%
\begin{pgfscope}%
\pgfpathrectangle{\pgfqpoint{6.572727in}{0.474100in}}{\pgfqpoint{4.227273in}{3.318700in}}%
\pgfusepath{clip}%
\pgfsetbuttcap%
\pgfsetroundjoin%
\definecolor{currentfill}{rgb}{0.127568,0.566949,0.550556}%
\pgfsetfillcolor{currentfill}%
\pgfsetfillopacity{0.700000}%
\pgfsetlinewidth{0.000000pt}%
\definecolor{currentstroke}{rgb}{0.000000,0.000000,0.000000}%
\pgfsetstrokecolor{currentstroke}%
\pgfsetstrokeopacity{0.700000}%
\pgfsetdash{}{0pt}%
\pgfpathmoveto{\pgfqpoint{7.984190in}{0.946791in}}%
\pgfpathcurveto{\pgfqpoint{7.989234in}{0.946791in}}{\pgfqpoint{7.994072in}{0.948795in}}{\pgfqpoint{7.997638in}{0.952361in}}%
\pgfpathcurveto{\pgfqpoint{8.001204in}{0.955928in}}{\pgfqpoint{8.003208in}{0.960765in}}{\pgfqpoint{8.003208in}{0.965809in}}%
\pgfpathcurveto{\pgfqpoint{8.003208in}{0.970853in}}{\pgfqpoint{8.001204in}{0.975690in}}{\pgfqpoint{7.997638in}{0.979257in}}%
\pgfpathcurveto{\pgfqpoint{7.994072in}{0.982823in}}{\pgfqpoint{7.989234in}{0.984827in}}{\pgfqpoint{7.984190in}{0.984827in}}%
\pgfpathcurveto{\pgfqpoint{7.979146in}{0.984827in}}{\pgfqpoint{7.974309in}{0.982823in}}{\pgfqpoint{7.970742in}{0.979257in}}%
\pgfpathcurveto{\pgfqpoint{7.967176in}{0.975690in}}{\pgfqpoint{7.965172in}{0.970853in}}{\pgfqpoint{7.965172in}{0.965809in}}%
\pgfpathcurveto{\pgfqpoint{7.965172in}{0.960765in}}{\pgfqpoint{7.967176in}{0.955928in}}{\pgfqpoint{7.970742in}{0.952361in}}%
\pgfpathcurveto{\pgfqpoint{7.974309in}{0.948795in}}{\pgfqpoint{7.979146in}{0.946791in}}{\pgfqpoint{7.984190in}{0.946791in}}%
\pgfpathclose%
\pgfusepath{fill}%
\end{pgfscope}%
\begin{pgfscope}%
\pgfpathrectangle{\pgfqpoint{6.572727in}{0.474100in}}{\pgfqpoint{4.227273in}{3.318700in}}%
\pgfusepath{clip}%
\pgfsetbuttcap%
\pgfsetroundjoin%
\definecolor{currentfill}{rgb}{0.127568,0.566949,0.550556}%
\pgfsetfillcolor{currentfill}%
\pgfsetfillopacity{0.700000}%
\pgfsetlinewidth{0.000000pt}%
\definecolor{currentstroke}{rgb}{0.000000,0.000000,0.000000}%
\pgfsetstrokecolor{currentstroke}%
\pgfsetstrokeopacity{0.700000}%
\pgfsetdash{}{0pt}%
\pgfpathmoveto{\pgfqpoint{8.648901in}{3.033687in}}%
\pgfpathcurveto{\pgfqpoint{8.653945in}{3.033687in}}{\pgfqpoint{8.658783in}{3.035691in}}{\pgfqpoint{8.662349in}{3.039257in}}%
\pgfpathcurveto{\pgfqpoint{8.665916in}{3.042824in}}{\pgfqpoint{8.667920in}{3.047661in}}{\pgfqpoint{8.667920in}{3.052705in}}%
\pgfpathcurveto{\pgfqpoint{8.667920in}{3.057749in}}{\pgfqpoint{8.665916in}{3.062586in}}{\pgfqpoint{8.662349in}{3.066153in}}%
\pgfpathcurveto{\pgfqpoint{8.658783in}{3.069719in}}{\pgfqpoint{8.653945in}{3.071723in}}{\pgfqpoint{8.648901in}{3.071723in}}%
\pgfpathcurveto{\pgfqpoint{8.643858in}{3.071723in}}{\pgfqpoint{8.639020in}{3.069719in}}{\pgfqpoint{8.635454in}{3.066153in}}%
\pgfpathcurveto{\pgfqpoint{8.631887in}{3.062586in}}{\pgfqpoint{8.629883in}{3.057749in}}{\pgfqpoint{8.629883in}{3.052705in}}%
\pgfpathcurveto{\pgfqpoint{8.629883in}{3.047661in}}{\pgfqpoint{8.631887in}{3.042824in}}{\pgfqpoint{8.635454in}{3.039257in}}%
\pgfpathcurveto{\pgfqpoint{8.639020in}{3.035691in}}{\pgfqpoint{8.643858in}{3.033687in}}{\pgfqpoint{8.648901in}{3.033687in}}%
\pgfpathclose%
\pgfusepath{fill}%
\end{pgfscope}%
\begin{pgfscope}%
\pgfpathrectangle{\pgfqpoint{6.572727in}{0.474100in}}{\pgfqpoint{4.227273in}{3.318700in}}%
\pgfusepath{clip}%
\pgfsetbuttcap%
\pgfsetroundjoin%
\definecolor{currentfill}{rgb}{0.127568,0.566949,0.550556}%
\pgfsetfillcolor{currentfill}%
\pgfsetfillopacity{0.700000}%
\pgfsetlinewidth{0.000000pt}%
\definecolor{currentstroke}{rgb}{0.000000,0.000000,0.000000}%
\pgfsetstrokecolor{currentstroke}%
\pgfsetstrokeopacity{0.700000}%
\pgfsetdash{}{0pt}%
\pgfpathmoveto{\pgfqpoint{7.866339in}{2.696607in}}%
\pgfpathcurveto{\pgfqpoint{7.871383in}{2.696607in}}{\pgfqpoint{7.876221in}{2.698611in}}{\pgfqpoint{7.879787in}{2.702177in}}%
\pgfpathcurveto{\pgfqpoint{7.883354in}{2.705744in}}{\pgfqpoint{7.885357in}{2.710582in}}{\pgfqpoint{7.885357in}{2.715625in}}%
\pgfpathcurveto{\pgfqpoint{7.885357in}{2.720669in}}{\pgfqpoint{7.883354in}{2.725507in}}{\pgfqpoint{7.879787in}{2.729073in}}%
\pgfpathcurveto{\pgfqpoint{7.876221in}{2.732639in}}{\pgfqpoint{7.871383in}{2.734643in}}{\pgfqpoint{7.866339in}{2.734643in}}%
\pgfpathcurveto{\pgfqpoint{7.861296in}{2.734643in}}{\pgfqpoint{7.856458in}{2.732639in}}{\pgfqpoint{7.852891in}{2.729073in}}%
\pgfpathcurveto{\pgfqpoint{7.849325in}{2.725507in}}{\pgfqpoint{7.847321in}{2.720669in}}{\pgfqpoint{7.847321in}{2.715625in}}%
\pgfpathcurveto{\pgfqpoint{7.847321in}{2.710582in}}{\pgfqpoint{7.849325in}{2.705744in}}{\pgfqpoint{7.852891in}{2.702177in}}%
\pgfpathcurveto{\pgfqpoint{7.856458in}{2.698611in}}{\pgfqpoint{7.861296in}{2.696607in}}{\pgfqpoint{7.866339in}{2.696607in}}%
\pgfpathclose%
\pgfusepath{fill}%
\end{pgfscope}%
\begin{pgfscope}%
\pgfpathrectangle{\pgfqpoint{6.572727in}{0.474100in}}{\pgfqpoint{4.227273in}{3.318700in}}%
\pgfusepath{clip}%
\pgfsetbuttcap%
\pgfsetroundjoin%
\definecolor{currentfill}{rgb}{0.127568,0.566949,0.550556}%
\pgfsetfillcolor{currentfill}%
\pgfsetfillopacity{0.700000}%
\pgfsetlinewidth{0.000000pt}%
\definecolor{currentstroke}{rgb}{0.000000,0.000000,0.000000}%
\pgfsetstrokecolor{currentstroke}%
\pgfsetstrokeopacity{0.700000}%
\pgfsetdash{}{0pt}%
\pgfpathmoveto{\pgfqpoint{8.411086in}{2.846783in}}%
\pgfpathcurveto{\pgfqpoint{8.416129in}{2.846783in}}{\pgfqpoint{8.420967in}{2.848787in}}{\pgfqpoint{8.424534in}{2.852353in}}%
\pgfpathcurveto{\pgfqpoint{8.428100in}{2.855919in}}{\pgfqpoint{8.430104in}{2.860757in}}{\pgfqpoint{8.430104in}{2.865801in}}%
\pgfpathcurveto{\pgfqpoint{8.430104in}{2.870845in}}{\pgfqpoint{8.428100in}{2.875682in}}{\pgfqpoint{8.424534in}{2.879249in}}%
\pgfpathcurveto{\pgfqpoint{8.420967in}{2.882815in}}{\pgfqpoint{8.416129in}{2.884819in}}{\pgfqpoint{8.411086in}{2.884819in}}%
\pgfpathcurveto{\pgfqpoint{8.406042in}{2.884819in}}{\pgfqpoint{8.401204in}{2.882815in}}{\pgfqpoint{8.397638in}{2.879249in}}%
\pgfpathcurveto{\pgfqpoint{8.394071in}{2.875682in}}{\pgfqpoint{8.392068in}{2.870845in}}{\pgfqpoint{8.392068in}{2.865801in}}%
\pgfpathcurveto{\pgfqpoint{8.392068in}{2.860757in}}{\pgfqpoint{8.394071in}{2.855919in}}{\pgfqpoint{8.397638in}{2.852353in}}%
\pgfpathcurveto{\pgfqpoint{8.401204in}{2.848787in}}{\pgfqpoint{8.406042in}{2.846783in}}{\pgfqpoint{8.411086in}{2.846783in}}%
\pgfpathclose%
\pgfusepath{fill}%
\end{pgfscope}%
\begin{pgfscope}%
\pgfpathrectangle{\pgfqpoint{6.572727in}{0.474100in}}{\pgfqpoint{4.227273in}{3.318700in}}%
\pgfusepath{clip}%
\pgfsetbuttcap%
\pgfsetroundjoin%
\definecolor{currentfill}{rgb}{0.127568,0.566949,0.550556}%
\pgfsetfillcolor{currentfill}%
\pgfsetfillopacity{0.700000}%
\pgfsetlinewidth{0.000000pt}%
\definecolor{currentstroke}{rgb}{0.000000,0.000000,0.000000}%
\pgfsetstrokecolor{currentstroke}%
\pgfsetstrokeopacity{0.700000}%
\pgfsetdash{}{0pt}%
\pgfpathmoveto{\pgfqpoint{7.301251in}{1.737872in}}%
\pgfpathcurveto{\pgfqpoint{7.306294in}{1.737872in}}{\pgfqpoint{7.311132in}{1.739876in}}{\pgfqpoint{7.314698in}{1.743442in}}%
\pgfpathcurveto{\pgfqpoint{7.318265in}{1.747008in}}{\pgfqpoint{7.320269in}{1.751846in}}{\pgfqpoint{7.320269in}{1.756890in}}%
\pgfpathcurveto{\pgfqpoint{7.320269in}{1.761934in}}{\pgfqpoint{7.318265in}{1.766771in}}{\pgfqpoint{7.314698in}{1.770338in}}%
\pgfpathcurveto{\pgfqpoint{7.311132in}{1.773904in}}{\pgfqpoint{7.306294in}{1.775908in}}{\pgfqpoint{7.301251in}{1.775908in}}%
\pgfpathcurveto{\pgfqpoint{7.296207in}{1.775908in}}{\pgfqpoint{7.291369in}{1.773904in}}{\pgfqpoint{7.287803in}{1.770338in}}%
\pgfpathcurveto{\pgfqpoint{7.284236in}{1.766771in}}{\pgfqpoint{7.282232in}{1.761934in}}{\pgfqpoint{7.282232in}{1.756890in}}%
\pgfpathcurveto{\pgfqpoint{7.282232in}{1.751846in}}{\pgfqpoint{7.284236in}{1.747008in}}{\pgfqpoint{7.287803in}{1.743442in}}%
\pgfpathcurveto{\pgfqpoint{7.291369in}{1.739876in}}{\pgfqpoint{7.296207in}{1.737872in}}{\pgfqpoint{7.301251in}{1.737872in}}%
\pgfpathclose%
\pgfusepath{fill}%
\end{pgfscope}%
\begin{pgfscope}%
\pgfpathrectangle{\pgfqpoint{6.572727in}{0.474100in}}{\pgfqpoint{4.227273in}{3.318700in}}%
\pgfusepath{clip}%
\pgfsetbuttcap%
\pgfsetroundjoin%
\definecolor{currentfill}{rgb}{0.993248,0.906157,0.143936}%
\pgfsetfillcolor{currentfill}%
\pgfsetfillopacity{0.700000}%
\pgfsetlinewidth{0.000000pt}%
\definecolor{currentstroke}{rgb}{0.000000,0.000000,0.000000}%
\pgfsetstrokecolor{currentstroke}%
\pgfsetstrokeopacity{0.700000}%
\pgfsetdash{}{0pt}%
\pgfpathmoveto{\pgfqpoint{9.701916in}{1.218985in}}%
\pgfpathcurveto{\pgfqpoint{9.706960in}{1.218985in}}{\pgfqpoint{9.711798in}{1.220989in}}{\pgfqpoint{9.715364in}{1.224556in}}%
\pgfpathcurveto{\pgfqpoint{9.718931in}{1.228122in}}{\pgfqpoint{9.720935in}{1.232960in}}{\pgfqpoint{9.720935in}{1.238003in}}%
\pgfpathcurveto{\pgfqpoint{9.720935in}{1.243047in}}{\pgfqpoint{9.718931in}{1.247885in}}{\pgfqpoint{9.715364in}{1.251451in}}%
\pgfpathcurveto{\pgfqpoint{9.711798in}{1.255018in}}{\pgfqpoint{9.706960in}{1.257022in}}{\pgfqpoint{9.701916in}{1.257022in}}%
\pgfpathcurveto{\pgfqpoint{9.696873in}{1.257022in}}{\pgfqpoint{9.692035in}{1.255018in}}{\pgfqpoint{9.688469in}{1.251451in}}%
\pgfpathcurveto{\pgfqpoint{9.684902in}{1.247885in}}{\pgfqpoint{9.682898in}{1.243047in}}{\pgfqpoint{9.682898in}{1.238003in}}%
\pgfpathcurveto{\pgfqpoint{9.682898in}{1.232960in}}{\pgfqpoint{9.684902in}{1.228122in}}{\pgfqpoint{9.688469in}{1.224556in}}%
\pgfpathcurveto{\pgfqpoint{9.692035in}{1.220989in}}{\pgfqpoint{9.696873in}{1.218985in}}{\pgfqpoint{9.701916in}{1.218985in}}%
\pgfpathclose%
\pgfusepath{fill}%
\end{pgfscope}%
\begin{pgfscope}%
\pgfpathrectangle{\pgfqpoint{6.572727in}{0.474100in}}{\pgfqpoint{4.227273in}{3.318700in}}%
\pgfusepath{clip}%
\pgfsetbuttcap%
\pgfsetroundjoin%
\definecolor{currentfill}{rgb}{0.993248,0.906157,0.143936}%
\pgfsetfillcolor{currentfill}%
\pgfsetfillopacity{0.700000}%
\pgfsetlinewidth{0.000000pt}%
\definecolor{currentstroke}{rgb}{0.000000,0.000000,0.000000}%
\pgfsetstrokecolor{currentstroke}%
\pgfsetstrokeopacity{0.700000}%
\pgfsetdash{}{0pt}%
\pgfpathmoveto{\pgfqpoint{9.674932in}{2.171266in}}%
\pgfpathcurveto{\pgfqpoint{9.679975in}{2.171266in}}{\pgfqpoint{9.684813in}{2.173270in}}{\pgfqpoint{9.688379in}{2.176836in}}%
\pgfpathcurveto{\pgfqpoint{9.691946in}{2.180403in}}{\pgfqpoint{9.693950in}{2.185241in}}{\pgfqpoint{9.693950in}{2.190284in}}%
\pgfpathcurveto{\pgfqpoint{9.693950in}{2.195328in}}{\pgfqpoint{9.691946in}{2.200166in}}{\pgfqpoint{9.688379in}{2.203732in}}%
\pgfpathcurveto{\pgfqpoint{9.684813in}{2.207299in}}{\pgfqpoint{9.679975in}{2.209302in}}{\pgfqpoint{9.674932in}{2.209302in}}%
\pgfpathcurveto{\pgfqpoint{9.669888in}{2.209302in}}{\pgfqpoint{9.665050in}{2.207299in}}{\pgfqpoint{9.661484in}{2.203732in}}%
\pgfpathcurveto{\pgfqpoint{9.657917in}{2.200166in}}{\pgfqpoint{9.655913in}{2.195328in}}{\pgfqpoint{9.655913in}{2.190284in}}%
\pgfpathcurveto{\pgfqpoint{9.655913in}{2.185241in}}{\pgfqpoint{9.657917in}{2.180403in}}{\pgfqpoint{9.661484in}{2.176836in}}%
\pgfpathcurveto{\pgfqpoint{9.665050in}{2.173270in}}{\pgfqpoint{9.669888in}{2.171266in}}{\pgfqpoint{9.674932in}{2.171266in}}%
\pgfpathclose%
\pgfusepath{fill}%
\end{pgfscope}%
\begin{pgfscope}%
\pgfpathrectangle{\pgfqpoint{6.572727in}{0.474100in}}{\pgfqpoint{4.227273in}{3.318700in}}%
\pgfusepath{clip}%
\pgfsetbuttcap%
\pgfsetroundjoin%
\definecolor{currentfill}{rgb}{0.127568,0.566949,0.550556}%
\pgfsetfillcolor{currentfill}%
\pgfsetfillopacity{0.700000}%
\pgfsetlinewidth{0.000000pt}%
\definecolor{currentstroke}{rgb}{0.000000,0.000000,0.000000}%
\pgfsetstrokecolor{currentstroke}%
\pgfsetstrokeopacity{0.700000}%
\pgfsetdash{}{0pt}%
\pgfpathmoveto{\pgfqpoint{8.138498in}{2.002133in}}%
\pgfpathcurveto{\pgfqpoint{8.143542in}{2.002133in}}{\pgfqpoint{8.148380in}{2.004137in}}{\pgfqpoint{8.151946in}{2.007703in}}%
\pgfpathcurveto{\pgfqpoint{8.155512in}{2.011270in}}{\pgfqpoint{8.157516in}{2.016108in}}{\pgfqpoint{8.157516in}{2.021151in}}%
\pgfpathcurveto{\pgfqpoint{8.157516in}{2.026195in}}{\pgfqpoint{8.155512in}{2.031033in}}{\pgfqpoint{8.151946in}{2.034599in}}%
\pgfpathcurveto{\pgfqpoint{8.148380in}{2.038165in}}{\pgfqpoint{8.143542in}{2.040169in}}{\pgfqpoint{8.138498in}{2.040169in}}%
\pgfpathcurveto{\pgfqpoint{8.133454in}{2.040169in}}{\pgfqpoint{8.128617in}{2.038165in}}{\pgfqpoint{8.125050in}{2.034599in}}%
\pgfpathcurveto{\pgfqpoint{8.121484in}{2.031033in}}{\pgfqpoint{8.119480in}{2.026195in}}{\pgfqpoint{8.119480in}{2.021151in}}%
\pgfpathcurveto{\pgfqpoint{8.119480in}{2.016108in}}{\pgfqpoint{8.121484in}{2.011270in}}{\pgfqpoint{8.125050in}{2.007703in}}%
\pgfpathcurveto{\pgfqpoint{8.128617in}{2.004137in}}{\pgfqpoint{8.133454in}{2.002133in}}{\pgfqpoint{8.138498in}{2.002133in}}%
\pgfpathclose%
\pgfusepath{fill}%
\end{pgfscope}%
\begin{pgfscope}%
\pgfpathrectangle{\pgfqpoint{6.572727in}{0.474100in}}{\pgfqpoint{4.227273in}{3.318700in}}%
\pgfusepath{clip}%
\pgfsetbuttcap%
\pgfsetroundjoin%
\definecolor{currentfill}{rgb}{0.993248,0.906157,0.143936}%
\pgfsetfillcolor{currentfill}%
\pgfsetfillopacity{0.700000}%
\pgfsetlinewidth{0.000000pt}%
\definecolor{currentstroke}{rgb}{0.000000,0.000000,0.000000}%
\pgfsetstrokecolor{currentstroke}%
\pgfsetstrokeopacity{0.700000}%
\pgfsetdash{}{0pt}%
\pgfpathmoveto{\pgfqpoint{9.293637in}{1.740326in}}%
\pgfpathcurveto{\pgfqpoint{9.298680in}{1.740326in}}{\pgfqpoint{9.303518in}{1.742330in}}{\pgfqpoint{9.307084in}{1.745897in}}%
\pgfpathcurveto{\pgfqpoint{9.310651in}{1.749463in}}{\pgfqpoint{9.312655in}{1.754301in}}{\pgfqpoint{9.312655in}{1.759345in}}%
\pgfpathcurveto{\pgfqpoint{9.312655in}{1.764388in}}{\pgfqpoint{9.310651in}{1.769226in}}{\pgfqpoint{9.307084in}{1.772792in}}%
\pgfpathcurveto{\pgfqpoint{9.303518in}{1.776359in}}{\pgfqpoint{9.298680in}{1.778363in}}{\pgfqpoint{9.293637in}{1.778363in}}%
\pgfpathcurveto{\pgfqpoint{9.288593in}{1.778363in}}{\pgfqpoint{9.283755in}{1.776359in}}{\pgfqpoint{9.280189in}{1.772792in}}%
\pgfpathcurveto{\pgfqpoint{9.276622in}{1.769226in}}{\pgfqpoint{9.274618in}{1.764388in}}{\pgfqpoint{9.274618in}{1.759345in}}%
\pgfpathcurveto{\pgfqpoint{9.274618in}{1.754301in}}{\pgfqpoint{9.276622in}{1.749463in}}{\pgfqpoint{9.280189in}{1.745897in}}%
\pgfpathcurveto{\pgfqpoint{9.283755in}{1.742330in}}{\pgfqpoint{9.288593in}{1.740326in}}{\pgfqpoint{9.293637in}{1.740326in}}%
\pgfpathclose%
\pgfusepath{fill}%
\end{pgfscope}%
\begin{pgfscope}%
\pgfpathrectangle{\pgfqpoint{6.572727in}{0.474100in}}{\pgfqpoint{4.227273in}{3.318700in}}%
\pgfusepath{clip}%
\pgfsetbuttcap%
\pgfsetroundjoin%
\definecolor{currentfill}{rgb}{0.993248,0.906157,0.143936}%
\pgfsetfillcolor{currentfill}%
\pgfsetfillopacity{0.700000}%
\pgfsetlinewidth{0.000000pt}%
\definecolor{currentstroke}{rgb}{0.000000,0.000000,0.000000}%
\pgfsetstrokecolor{currentstroke}%
\pgfsetstrokeopacity{0.700000}%
\pgfsetdash{}{0pt}%
\pgfpathmoveto{\pgfqpoint{9.159888in}{1.566051in}}%
\pgfpathcurveto{\pgfqpoint{9.164932in}{1.566051in}}{\pgfqpoint{9.169769in}{1.568055in}}{\pgfqpoint{9.173336in}{1.571621in}}%
\pgfpathcurveto{\pgfqpoint{9.176902in}{1.575188in}}{\pgfqpoint{9.178906in}{1.580025in}}{\pgfqpoint{9.178906in}{1.585069in}}%
\pgfpathcurveto{\pgfqpoint{9.178906in}{1.590113in}}{\pgfqpoint{9.176902in}{1.594951in}}{\pgfqpoint{9.173336in}{1.598517in}}%
\pgfpathcurveto{\pgfqpoint{9.169769in}{1.602083in}}{\pgfqpoint{9.164932in}{1.604087in}}{\pgfqpoint{9.159888in}{1.604087in}}%
\pgfpathcurveto{\pgfqpoint{9.154844in}{1.604087in}}{\pgfqpoint{9.150006in}{1.602083in}}{\pgfqpoint{9.146440in}{1.598517in}}%
\pgfpathcurveto{\pgfqpoint{9.142874in}{1.594951in}}{\pgfqpoint{9.140870in}{1.590113in}}{\pgfqpoint{9.140870in}{1.585069in}}%
\pgfpathcurveto{\pgfqpoint{9.140870in}{1.580025in}}{\pgfqpoint{9.142874in}{1.575188in}}{\pgfqpoint{9.146440in}{1.571621in}}%
\pgfpathcurveto{\pgfqpoint{9.150006in}{1.568055in}}{\pgfqpoint{9.154844in}{1.566051in}}{\pgfqpoint{9.159888in}{1.566051in}}%
\pgfpathclose%
\pgfusepath{fill}%
\end{pgfscope}%
\begin{pgfscope}%
\pgfpathrectangle{\pgfqpoint{6.572727in}{0.474100in}}{\pgfqpoint{4.227273in}{3.318700in}}%
\pgfusepath{clip}%
\pgfsetbuttcap%
\pgfsetroundjoin%
\definecolor{currentfill}{rgb}{0.993248,0.906157,0.143936}%
\pgfsetfillcolor{currentfill}%
\pgfsetfillopacity{0.700000}%
\pgfsetlinewidth{0.000000pt}%
\definecolor{currentstroke}{rgb}{0.000000,0.000000,0.000000}%
\pgfsetstrokecolor{currentstroke}%
\pgfsetstrokeopacity{0.700000}%
\pgfsetdash{}{0pt}%
\pgfpathmoveto{\pgfqpoint{9.820642in}{1.508687in}}%
\pgfpathcurveto{\pgfqpoint{9.825686in}{1.508687in}}{\pgfqpoint{9.830523in}{1.510691in}}{\pgfqpoint{9.834090in}{1.514257in}}%
\pgfpathcurveto{\pgfqpoint{9.837656in}{1.517823in}}{\pgfqpoint{9.839660in}{1.522661in}}{\pgfqpoint{9.839660in}{1.527705in}}%
\pgfpathcurveto{\pgfqpoint{9.839660in}{1.532749in}}{\pgfqpoint{9.837656in}{1.537586in}}{\pgfqpoint{9.834090in}{1.541153in}}%
\pgfpathcurveto{\pgfqpoint{9.830523in}{1.544719in}}{\pgfqpoint{9.825686in}{1.546723in}}{\pgfqpoint{9.820642in}{1.546723in}}%
\pgfpathcurveto{\pgfqpoint{9.815598in}{1.546723in}}{\pgfqpoint{9.810760in}{1.544719in}}{\pgfqpoint{9.807194in}{1.541153in}}%
\pgfpathcurveto{\pgfqpoint{9.803628in}{1.537586in}}{\pgfqpoint{9.801624in}{1.532749in}}{\pgfqpoint{9.801624in}{1.527705in}}%
\pgfpathcurveto{\pgfqpoint{9.801624in}{1.522661in}}{\pgfqpoint{9.803628in}{1.517823in}}{\pgfqpoint{9.807194in}{1.514257in}}%
\pgfpathcurveto{\pgfqpoint{9.810760in}{1.510691in}}{\pgfqpoint{9.815598in}{1.508687in}}{\pgfqpoint{9.820642in}{1.508687in}}%
\pgfpathclose%
\pgfusepath{fill}%
\end{pgfscope}%
\begin{pgfscope}%
\pgfpathrectangle{\pgfqpoint{6.572727in}{0.474100in}}{\pgfqpoint{4.227273in}{3.318700in}}%
\pgfusepath{clip}%
\pgfsetbuttcap%
\pgfsetroundjoin%
\definecolor{currentfill}{rgb}{0.127568,0.566949,0.550556}%
\pgfsetfillcolor{currentfill}%
\pgfsetfillopacity{0.700000}%
\pgfsetlinewidth{0.000000pt}%
\definecolor{currentstroke}{rgb}{0.000000,0.000000,0.000000}%
\pgfsetstrokecolor{currentstroke}%
\pgfsetstrokeopacity{0.700000}%
\pgfsetdash{}{0pt}%
\pgfpathmoveto{\pgfqpoint{7.495660in}{1.923523in}}%
\pgfpathcurveto{\pgfqpoint{7.500704in}{1.923523in}}{\pgfqpoint{7.505541in}{1.925526in}}{\pgfqpoint{7.509108in}{1.929093in}}%
\pgfpathcurveto{\pgfqpoint{7.512674in}{1.932659in}}{\pgfqpoint{7.514678in}{1.937497in}}{\pgfqpoint{7.514678in}{1.942541in}}%
\pgfpathcurveto{\pgfqpoint{7.514678in}{1.947584in}}{\pgfqpoint{7.512674in}{1.952422in}}{\pgfqpoint{7.509108in}{1.955989in}}%
\pgfpathcurveto{\pgfqpoint{7.505541in}{1.959555in}}{\pgfqpoint{7.500704in}{1.961559in}}{\pgfqpoint{7.495660in}{1.961559in}}%
\pgfpathcurveto{\pgfqpoint{7.490616in}{1.961559in}}{\pgfqpoint{7.485778in}{1.959555in}}{\pgfqpoint{7.482212in}{1.955989in}}%
\pgfpathcurveto{\pgfqpoint{7.478646in}{1.952422in}}{\pgfqpoint{7.476642in}{1.947584in}}{\pgfqpoint{7.476642in}{1.942541in}}%
\pgfpathcurveto{\pgfqpoint{7.476642in}{1.937497in}}{\pgfqpoint{7.478646in}{1.932659in}}{\pgfqpoint{7.482212in}{1.929093in}}%
\pgfpathcurveto{\pgfqpoint{7.485778in}{1.925526in}}{\pgfqpoint{7.490616in}{1.923523in}}{\pgfqpoint{7.495660in}{1.923523in}}%
\pgfpathclose%
\pgfusepath{fill}%
\end{pgfscope}%
\begin{pgfscope}%
\pgfpathrectangle{\pgfqpoint{6.572727in}{0.474100in}}{\pgfqpoint{4.227273in}{3.318700in}}%
\pgfusepath{clip}%
\pgfsetbuttcap%
\pgfsetroundjoin%
\definecolor{currentfill}{rgb}{0.127568,0.566949,0.550556}%
\pgfsetfillcolor{currentfill}%
\pgfsetfillopacity{0.700000}%
\pgfsetlinewidth{0.000000pt}%
\definecolor{currentstroke}{rgb}{0.000000,0.000000,0.000000}%
\pgfsetstrokecolor{currentstroke}%
\pgfsetstrokeopacity{0.700000}%
\pgfsetdash{}{0pt}%
\pgfpathmoveto{\pgfqpoint{8.061798in}{3.460992in}}%
\pgfpathcurveto{\pgfqpoint{8.066841in}{3.460992in}}{\pgfqpoint{8.071679in}{3.462996in}}{\pgfqpoint{8.075245in}{3.466562in}}%
\pgfpathcurveto{\pgfqpoint{8.078812in}{3.470129in}}{\pgfqpoint{8.080816in}{3.474966in}}{\pgfqpoint{8.080816in}{3.480010in}}%
\pgfpathcurveto{\pgfqpoint{8.080816in}{3.485054in}}{\pgfqpoint{8.078812in}{3.489892in}}{\pgfqpoint{8.075245in}{3.493458in}}%
\pgfpathcurveto{\pgfqpoint{8.071679in}{3.497024in}}{\pgfqpoint{8.066841in}{3.499028in}}{\pgfqpoint{8.061798in}{3.499028in}}%
\pgfpathcurveto{\pgfqpoint{8.056754in}{3.499028in}}{\pgfqpoint{8.051916in}{3.497024in}}{\pgfqpoint{8.048350in}{3.493458in}}%
\pgfpathcurveto{\pgfqpoint{8.044783in}{3.489892in}}{\pgfqpoint{8.042779in}{3.485054in}}{\pgfqpoint{8.042779in}{3.480010in}}%
\pgfpathcurveto{\pgfqpoint{8.042779in}{3.474966in}}{\pgfqpoint{8.044783in}{3.470129in}}{\pgfqpoint{8.048350in}{3.466562in}}%
\pgfpathcurveto{\pgfqpoint{8.051916in}{3.462996in}}{\pgfqpoint{8.056754in}{3.460992in}}{\pgfqpoint{8.061798in}{3.460992in}}%
\pgfpathclose%
\pgfusepath{fill}%
\end{pgfscope}%
\begin{pgfscope}%
\pgfpathrectangle{\pgfqpoint{6.572727in}{0.474100in}}{\pgfqpoint{4.227273in}{3.318700in}}%
\pgfusepath{clip}%
\pgfsetbuttcap%
\pgfsetroundjoin%
\definecolor{currentfill}{rgb}{0.993248,0.906157,0.143936}%
\pgfsetfillcolor{currentfill}%
\pgfsetfillopacity{0.700000}%
\pgfsetlinewidth{0.000000pt}%
\definecolor{currentstroke}{rgb}{0.000000,0.000000,0.000000}%
\pgfsetstrokecolor{currentstroke}%
\pgfsetstrokeopacity{0.700000}%
\pgfsetdash{}{0pt}%
\pgfpathmoveto{\pgfqpoint{9.826447in}{1.331738in}}%
\pgfpathcurveto{\pgfqpoint{9.831491in}{1.331738in}}{\pgfqpoint{9.836329in}{1.333742in}}{\pgfqpoint{9.839895in}{1.337308in}}%
\pgfpathcurveto{\pgfqpoint{9.843461in}{1.340874in}}{\pgfqpoint{9.845465in}{1.345712in}}{\pgfqpoint{9.845465in}{1.350756in}}%
\pgfpathcurveto{\pgfqpoint{9.845465in}{1.355800in}}{\pgfqpoint{9.843461in}{1.360637in}}{\pgfqpoint{9.839895in}{1.364204in}}%
\pgfpathcurveto{\pgfqpoint{9.836329in}{1.367770in}}{\pgfqpoint{9.831491in}{1.369774in}}{\pgfqpoint{9.826447in}{1.369774in}}%
\pgfpathcurveto{\pgfqpoint{9.821404in}{1.369774in}}{\pgfqpoint{9.816566in}{1.367770in}}{\pgfqpoint{9.812999in}{1.364204in}}%
\pgfpathcurveto{\pgfqpoint{9.809433in}{1.360637in}}{\pgfqpoint{9.807429in}{1.355800in}}{\pgfqpoint{9.807429in}{1.350756in}}%
\pgfpathcurveto{\pgfqpoint{9.807429in}{1.345712in}}{\pgfqpoint{9.809433in}{1.340874in}}{\pgfqpoint{9.812999in}{1.337308in}}%
\pgfpathcurveto{\pgfqpoint{9.816566in}{1.333742in}}{\pgfqpoint{9.821404in}{1.331738in}}{\pgfqpoint{9.826447in}{1.331738in}}%
\pgfpathclose%
\pgfusepath{fill}%
\end{pgfscope}%
\begin{pgfscope}%
\pgfpathrectangle{\pgfqpoint{6.572727in}{0.474100in}}{\pgfqpoint{4.227273in}{3.318700in}}%
\pgfusepath{clip}%
\pgfsetbuttcap%
\pgfsetroundjoin%
\definecolor{currentfill}{rgb}{0.993248,0.906157,0.143936}%
\pgfsetfillcolor{currentfill}%
\pgfsetfillopacity{0.700000}%
\pgfsetlinewidth{0.000000pt}%
\definecolor{currentstroke}{rgb}{0.000000,0.000000,0.000000}%
\pgfsetstrokecolor{currentstroke}%
\pgfsetstrokeopacity{0.700000}%
\pgfsetdash{}{0pt}%
\pgfpathmoveto{\pgfqpoint{10.093114in}{1.259359in}}%
\pgfpathcurveto{\pgfqpoint{10.098158in}{1.259359in}}{\pgfqpoint{10.102996in}{1.261363in}}{\pgfqpoint{10.106562in}{1.264929in}}%
\pgfpathcurveto{\pgfqpoint{10.110129in}{1.268496in}}{\pgfqpoint{10.112133in}{1.273334in}}{\pgfqpoint{10.112133in}{1.278377in}}%
\pgfpathcurveto{\pgfqpoint{10.112133in}{1.283421in}}{\pgfqpoint{10.110129in}{1.288259in}}{\pgfqpoint{10.106562in}{1.291825in}}%
\pgfpathcurveto{\pgfqpoint{10.102996in}{1.295391in}}{\pgfqpoint{10.098158in}{1.297395in}}{\pgfqpoint{10.093114in}{1.297395in}}%
\pgfpathcurveto{\pgfqpoint{10.088071in}{1.297395in}}{\pgfqpoint{10.083233in}{1.295391in}}{\pgfqpoint{10.079667in}{1.291825in}}%
\pgfpathcurveto{\pgfqpoint{10.076100in}{1.288259in}}{\pgfqpoint{10.074096in}{1.283421in}}{\pgfqpoint{10.074096in}{1.278377in}}%
\pgfpathcurveto{\pgfqpoint{10.074096in}{1.273334in}}{\pgfqpoint{10.076100in}{1.268496in}}{\pgfqpoint{10.079667in}{1.264929in}}%
\pgfpathcurveto{\pgfqpoint{10.083233in}{1.261363in}}{\pgfqpoint{10.088071in}{1.259359in}}{\pgfqpoint{10.093114in}{1.259359in}}%
\pgfpathclose%
\pgfusepath{fill}%
\end{pgfscope}%
\begin{pgfscope}%
\pgfpathrectangle{\pgfqpoint{6.572727in}{0.474100in}}{\pgfqpoint{4.227273in}{3.318700in}}%
\pgfusepath{clip}%
\pgfsetbuttcap%
\pgfsetroundjoin%
\definecolor{currentfill}{rgb}{0.993248,0.906157,0.143936}%
\pgfsetfillcolor{currentfill}%
\pgfsetfillopacity{0.700000}%
\pgfsetlinewidth{0.000000pt}%
\definecolor{currentstroke}{rgb}{0.000000,0.000000,0.000000}%
\pgfsetstrokecolor{currentstroke}%
\pgfsetstrokeopacity{0.700000}%
\pgfsetdash{}{0pt}%
\pgfpathmoveto{\pgfqpoint{9.257791in}{2.241229in}}%
\pgfpathcurveto{\pgfqpoint{9.262835in}{2.241229in}}{\pgfqpoint{9.267673in}{2.243233in}}{\pgfqpoint{9.271239in}{2.246799in}}%
\pgfpathcurveto{\pgfqpoint{9.274806in}{2.250366in}}{\pgfqpoint{9.276810in}{2.255204in}}{\pgfqpoint{9.276810in}{2.260247in}}%
\pgfpathcurveto{\pgfqpoint{9.276810in}{2.265291in}}{\pgfqpoint{9.274806in}{2.270129in}}{\pgfqpoint{9.271239in}{2.273695in}}%
\pgfpathcurveto{\pgfqpoint{9.267673in}{2.277262in}}{\pgfqpoint{9.262835in}{2.279265in}}{\pgfqpoint{9.257791in}{2.279265in}}%
\pgfpathcurveto{\pgfqpoint{9.252748in}{2.279265in}}{\pgfqpoint{9.247910in}{2.277262in}}{\pgfqpoint{9.244344in}{2.273695in}}%
\pgfpathcurveto{\pgfqpoint{9.240777in}{2.270129in}}{\pgfqpoint{9.238773in}{2.265291in}}{\pgfqpoint{9.238773in}{2.260247in}}%
\pgfpathcurveto{\pgfqpoint{9.238773in}{2.255204in}}{\pgfqpoint{9.240777in}{2.250366in}}{\pgfqpoint{9.244344in}{2.246799in}}%
\pgfpathcurveto{\pgfqpoint{9.247910in}{2.243233in}}{\pgfqpoint{9.252748in}{2.241229in}}{\pgfqpoint{9.257791in}{2.241229in}}%
\pgfpathclose%
\pgfusepath{fill}%
\end{pgfscope}%
\begin{pgfscope}%
\pgfpathrectangle{\pgfqpoint{6.572727in}{0.474100in}}{\pgfqpoint{4.227273in}{3.318700in}}%
\pgfusepath{clip}%
\pgfsetbuttcap%
\pgfsetroundjoin%
\definecolor{currentfill}{rgb}{0.127568,0.566949,0.550556}%
\pgfsetfillcolor{currentfill}%
\pgfsetfillopacity{0.700000}%
\pgfsetlinewidth{0.000000pt}%
\definecolor{currentstroke}{rgb}{0.000000,0.000000,0.000000}%
\pgfsetstrokecolor{currentstroke}%
\pgfsetstrokeopacity{0.700000}%
\pgfsetdash{}{0pt}%
\pgfpathmoveto{\pgfqpoint{8.069953in}{1.296751in}}%
\pgfpathcurveto{\pgfqpoint{8.074997in}{1.296751in}}{\pgfqpoint{8.079834in}{1.298755in}}{\pgfqpoint{8.083401in}{1.302322in}}%
\pgfpathcurveto{\pgfqpoint{8.086967in}{1.305888in}}{\pgfqpoint{8.088971in}{1.310726in}}{\pgfqpoint{8.088971in}{1.315769in}}%
\pgfpathcurveto{\pgfqpoint{8.088971in}{1.320813in}}{\pgfqpoint{8.086967in}{1.325651in}}{\pgfqpoint{8.083401in}{1.329217in}}%
\pgfpathcurveto{\pgfqpoint{8.079834in}{1.332784in}}{\pgfqpoint{8.074997in}{1.334788in}}{\pgfqpoint{8.069953in}{1.334788in}}%
\pgfpathcurveto{\pgfqpoint{8.064909in}{1.334788in}}{\pgfqpoint{8.060071in}{1.332784in}}{\pgfqpoint{8.056505in}{1.329217in}}%
\pgfpathcurveto{\pgfqpoint{8.052939in}{1.325651in}}{\pgfqpoint{8.050935in}{1.320813in}}{\pgfqpoint{8.050935in}{1.315769in}}%
\pgfpathcurveto{\pgfqpoint{8.050935in}{1.310726in}}{\pgfqpoint{8.052939in}{1.305888in}}{\pgfqpoint{8.056505in}{1.302322in}}%
\pgfpathcurveto{\pgfqpoint{8.060071in}{1.298755in}}{\pgfqpoint{8.064909in}{1.296751in}}{\pgfqpoint{8.069953in}{1.296751in}}%
\pgfpathclose%
\pgfusepath{fill}%
\end{pgfscope}%
\begin{pgfscope}%
\pgfpathrectangle{\pgfqpoint{6.572727in}{0.474100in}}{\pgfqpoint{4.227273in}{3.318700in}}%
\pgfusepath{clip}%
\pgfsetbuttcap%
\pgfsetroundjoin%
\definecolor{currentfill}{rgb}{0.993248,0.906157,0.143936}%
\pgfsetfillcolor{currentfill}%
\pgfsetfillopacity{0.700000}%
\pgfsetlinewidth{0.000000pt}%
\definecolor{currentstroke}{rgb}{0.000000,0.000000,0.000000}%
\pgfsetstrokecolor{currentstroke}%
\pgfsetstrokeopacity{0.700000}%
\pgfsetdash{}{0pt}%
\pgfpathmoveto{\pgfqpoint{10.028396in}{1.690765in}}%
\pgfpathcurveto{\pgfqpoint{10.033440in}{1.690765in}}{\pgfqpoint{10.038278in}{1.692769in}}{\pgfqpoint{10.041844in}{1.696335in}}%
\pgfpathcurveto{\pgfqpoint{10.045411in}{1.699902in}}{\pgfqpoint{10.047414in}{1.704739in}}{\pgfqpoint{10.047414in}{1.709783in}}%
\pgfpathcurveto{\pgfqpoint{10.047414in}{1.714827in}}{\pgfqpoint{10.045411in}{1.719665in}}{\pgfqpoint{10.041844in}{1.723231in}}%
\pgfpathcurveto{\pgfqpoint{10.038278in}{1.726797in}}{\pgfqpoint{10.033440in}{1.728801in}}{\pgfqpoint{10.028396in}{1.728801in}}%
\pgfpathcurveto{\pgfqpoint{10.023353in}{1.728801in}}{\pgfqpoint{10.018515in}{1.726797in}}{\pgfqpoint{10.014948in}{1.723231in}}%
\pgfpathcurveto{\pgfqpoint{10.011382in}{1.719665in}}{\pgfqpoint{10.009378in}{1.714827in}}{\pgfqpoint{10.009378in}{1.709783in}}%
\pgfpathcurveto{\pgfqpoint{10.009378in}{1.704739in}}{\pgfqpoint{10.011382in}{1.699902in}}{\pgfqpoint{10.014948in}{1.696335in}}%
\pgfpathcurveto{\pgfqpoint{10.018515in}{1.692769in}}{\pgfqpoint{10.023353in}{1.690765in}}{\pgfqpoint{10.028396in}{1.690765in}}%
\pgfpathclose%
\pgfusepath{fill}%
\end{pgfscope}%
\begin{pgfscope}%
\pgfpathrectangle{\pgfqpoint{6.572727in}{0.474100in}}{\pgfqpoint{4.227273in}{3.318700in}}%
\pgfusepath{clip}%
\pgfsetbuttcap%
\pgfsetroundjoin%
\definecolor{currentfill}{rgb}{0.267004,0.004874,0.329415}%
\pgfsetfillcolor{currentfill}%
\pgfsetfillopacity{0.700000}%
\pgfsetlinewidth{0.000000pt}%
\definecolor{currentstroke}{rgb}{0.000000,0.000000,0.000000}%
\pgfsetstrokecolor{currentstroke}%
\pgfsetstrokeopacity{0.700000}%
\pgfsetdash{}{0pt}%
\pgfpathmoveto{\pgfqpoint{8.604466in}{3.565402in}}%
\pgfpathcurveto{\pgfqpoint{8.609510in}{3.565402in}}{\pgfqpoint{8.614348in}{3.567406in}}{\pgfqpoint{8.617914in}{3.570972in}}%
\pgfpathcurveto{\pgfqpoint{8.621481in}{3.574539in}}{\pgfqpoint{8.623485in}{3.579376in}}{\pgfqpoint{8.623485in}{3.584420in}}%
\pgfpathcurveto{\pgfqpoint{8.623485in}{3.589464in}}{\pgfqpoint{8.621481in}{3.594301in}}{\pgfqpoint{8.617914in}{3.597868in}}%
\pgfpathcurveto{\pgfqpoint{8.614348in}{3.601434in}}{\pgfqpoint{8.609510in}{3.603438in}}{\pgfqpoint{8.604466in}{3.603438in}}%
\pgfpathcurveto{\pgfqpoint{8.599423in}{3.603438in}}{\pgfqpoint{8.594585in}{3.601434in}}{\pgfqpoint{8.591019in}{3.597868in}}%
\pgfpathcurveto{\pgfqpoint{8.587452in}{3.594301in}}{\pgfqpoint{8.585448in}{3.589464in}}{\pgfqpoint{8.585448in}{3.584420in}}%
\pgfpathcurveto{\pgfqpoint{8.585448in}{3.579376in}}{\pgfqpoint{8.587452in}{3.574539in}}{\pgfqpoint{8.591019in}{3.570972in}}%
\pgfpathcurveto{\pgfqpoint{8.594585in}{3.567406in}}{\pgfqpoint{8.599423in}{3.565402in}}{\pgfqpoint{8.604466in}{3.565402in}}%
\pgfpathclose%
\pgfusepath{fill}%
\end{pgfscope}%
\begin{pgfscope}%
\pgfpathrectangle{\pgfqpoint{6.572727in}{0.474100in}}{\pgfqpoint{4.227273in}{3.318700in}}%
\pgfusepath{clip}%
\pgfsetbuttcap%
\pgfsetroundjoin%
\definecolor{currentfill}{rgb}{0.127568,0.566949,0.550556}%
\pgfsetfillcolor{currentfill}%
\pgfsetfillopacity{0.700000}%
\pgfsetlinewidth{0.000000pt}%
\definecolor{currentstroke}{rgb}{0.000000,0.000000,0.000000}%
\pgfsetstrokecolor{currentstroke}%
\pgfsetstrokeopacity{0.700000}%
\pgfsetdash{}{0pt}%
\pgfpathmoveto{\pgfqpoint{7.838982in}{2.831448in}}%
\pgfpathcurveto{\pgfqpoint{7.844026in}{2.831448in}}{\pgfqpoint{7.848863in}{2.833451in}}{\pgfqpoint{7.852430in}{2.837018in}}%
\pgfpathcurveto{\pgfqpoint{7.855996in}{2.840584in}}{\pgfqpoint{7.858000in}{2.845422in}}{\pgfqpoint{7.858000in}{2.850466in}}%
\pgfpathcurveto{\pgfqpoint{7.858000in}{2.855509in}}{\pgfqpoint{7.855996in}{2.860347in}}{\pgfqpoint{7.852430in}{2.863914in}}%
\pgfpathcurveto{\pgfqpoint{7.848863in}{2.867480in}}{\pgfqpoint{7.844026in}{2.869484in}}{\pgfqpoint{7.838982in}{2.869484in}}%
\pgfpathcurveto{\pgfqpoint{7.833938in}{2.869484in}}{\pgfqpoint{7.829100in}{2.867480in}}{\pgfqpoint{7.825534in}{2.863914in}}%
\pgfpathcurveto{\pgfqpoint{7.821968in}{2.860347in}}{\pgfqpoint{7.819964in}{2.855509in}}{\pgfqpoint{7.819964in}{2.850466in}}%
\pgfpathcurveto{\pgfqpoint{7.819964in}{2.845422in}}{\pgfqpoint{7.821968in}{2.840584in}}{\pgfqpoint{7.825534in}{2.837018in}}%
\pgfpathcurveto{\pgfqpoint{7.829100in}{2.833451in}}{\pgfqpoint{7.833938in}{2.831448in}}{\pgfqpoint{7.838982in}{2.831448in}}%
\pgfpathclose%
\pgfusepath{fill}%
\end{pgfscope}%
\begin{pgfscope}%
\pgfpathrectangle{\pgfqpoint{6.572727in}{0.474100in}}{\pgfqpoint{4.227273in}{3.318700in}}%
\pgfusepath{clip}%
\pgfsetbuttcap%
\pgfsetroundjoin%
\definecolor{currentfill}{rgb}{0.127568,0.566949,0.550556}%
\pgfsetfillcolor{currentfill}%
\pgfsetfillopacity{0.700000}%
\pgfsetlinewidth{0.000000pt}%
\definecolor{currentstroke}{rgb}{0.000000,0.000000,0.000000}%
\pgfsetstrokecolor{currentstroke}%
\pgfsetstrokeopacity{0.700000}%
\pgfsetdash{}{0pt}%
\pgfpathmoveto{\pgfqpoint{7.485068in}{1.086535in}}%
\pgfpathcurveto{\pgfqpoint{7.490112in}{1.086535in}}{\pgfqpoint{7.494949in}{1.088538in}}{\pgfqpoint{7.498516in}{1.092105in}}%
\pgfpathcurveto{\pgfqpoint{7.502082in}{1.095671in}}{\pgfqpoint{7.504086in}{1.100509in}}{\pgfqpoint{7.504086in}{1.105553in}}%
\pgfpathcurveto{\pgfqpoint{7.504086in}{1.110596in}}{\pgfqpoint{7.502082in}{1.115434in}}{\pgfqpoint{7.498516in}{1.119001in}}%
\pgfpathcurveto{\pgfqpoint{7.494949in}{1.122567in}}{\pgfqpoint{7.490112in}{1.124571in}}{\pgfqpoint{7.485068in}{1.124571in}}%
\pgfpathcurveto{\pgfqpoint{7.480024in}{1.124571in}}{\pgfqpoint{7.475187in}{1.122567in}}{\pgfqpoint{7.471620in}{1.119001in}}%
\pgfpathcurveto{\pgfqpoint{7.468054in}{1.115434in}}{\pgfqpoint{7.466050in}{1.110596in}}{\pgfqpoint{7.466050in}{1.105553in}}%
\pgfpathcurveto{\pgfqpoint{7.466050in}{1.100509in}}{\pgfqpoint{7.468054in}{1.095671in}}{\pgfqpoint{7.471620in}{1.092105in}}%
\pgfpathcurveto{\pgfqpoint{7.475187in}{1.088538in}}{\pgfqpoint{7.480024in}{1.086535in}}{\pgfqpoint{7.485068in}{1.086535in}}%
\pgfpathclose%
\pgfusepath{fill}%
\end{pgfscope}%
\begin{pgfscope}%
\pgfpathrectangle{\pgfqpoint{6.572727in}{0.474100in}}{\pgfqpoint{4.227273in}{3.318700in}}%
\pgfusepath{clip}%
\pgfsetbuttcap%
\pgfsetroundjoin%
\definecolor{currentfill}{rgb}{0.127568,0.566949,0.550556}%
\pgfsetfillcolor{currentfill}%
\pgfsetfillopacity{0.700000}%
\pgfsetlinewidth{0.000000pt}%
\definecolor{currentstroke}{rgb}{0.000000,0.000000,0.000000}%
\pgfsetstrokecolor{currentstroke}%
\pgfsetstrokeopacity{0.700000}%
\pgfsetdash{}{0pt}%
\pgfpathmoveto{\pgfqpoint{7.363760in}{1.436366in}}%
\pgfpathcurveto{\pgfqpoint{7.368803in}{1.436366in}}{\pgfqpoint{7.373641in}{1.438370in}}{\pgfqpoint{7.377207in}{1.441936in}}%
\pgfpathcurveto{\pgfqpoint{7.380774in}{1.445502in}}{\pgfqpoint{7.382778in}{1.450340in}}{\pgfqpoint{7.382778in}{1.455384in}}%
\pgfpathcurveto{\pgfqpoint{7.382778in}{1.460428in}}{\pgfqpoint{7.380774in}{1.465265in}}{\pgfqpoint{7.377207in}{1.468832in}}%
\pgfpathcurveto{\pgfqpoint{7.373641in}{1.472398in}}{\pgfqpoint{7.368803in}{1.474402in}}{\pgfqpoint{7.363760in}{1.474402in}}%
\pgfpathcurveto{\pgfqpoint{7.358716in}{1.474402in}}{\pgfqpoint{7.353878in}{1.472398in}}{\pgfqpoint{7.350312in}{1.468832in}}%
\pgfpathcurveto{\pgfqpoint{7.346745in}{1.465265in}}{\pgfqpoint{7.344741in}{1.460428in}}{\pgfqpoint{7.344741in}{1.455384in}}%
\pgfpathcurveto{\pgfqpoint{7.344741in}{1.450340in}}{\pgfqpoint{7.346745in}{1.445502in}}{\pgfqpoint{7.350312in}{1.441936in}}%
\pgfpathcurveto{\pgfqpoint{7.353878in}{1.438370in}}{\pgfqpoint{7.358716in}{1.436366in}}{\pgfqpoint{7.363760in}{1.436366in}}%
\pgfpathclose%
\pgfusepath{fill}%
\end{pgfscope}%
\begin{pgfscope}%
\pgfpathrectangle{\pgfqpoint{6.572727in}{0.474100in}}{\pgfqpoint{4.227273in}{3.318700in}}%
\pgfusepath{clip}%
\pgfsetbuttcap%
\pgfsetroundjoin%
\definecolor{currentfill}{rgb}{0.127568,0.566949,0.550556}%
\pgfsetfillcolor{currentfill}%
\pgfsetfillopacity{0.700000}%
\pgfsetlinewidth{0.000000pt}%
\definecolor{currentstroke}{rgb}{0.000000,0.000000,0.000000}%
\pgfsetstrokecolor{currentstroke}%
\pgfsetstrokeopacity{0.700000}%
\pgfsetdash{}{0pt}%
\pgfpathmoveto{\pgfqpoint{7.886747in}{2.037674in}}%
\pgfpathcurveto{\pgfqpoint{7.891791in}{2.037674in}}{\pgfqpoint{7.896629in}{2.039678in}}{\pgfqpoint{7.900195in}{2.043244in}}%
\pgfpathcurveto{\pgfqpoint{7.903761in}{2.046811in}}{\pgfqpoint{7.905765in}{2.051648in}}{\pgfqpoint{7.905765in}{2.056692in}}%
\pgfpathcurveto{\pgfqpoint{7.905765in}{2.061736in}}{\pgfqpoint{7.903761in}{2.066573in}}{\pgfqpoint{7.900195in}{2.070140in}}%
\pgfpathcurveto{\pgfqpoint{7.896629in}{2.073706in}}{\pgfqpoint{7.891791in}{2.075710in}}{\pgfqpoint{7.886747in}{2.075710in}}%
\pgfpathcurveto{\pgfqpoint{7.881703in}{2.075710in}}{\pgfqpoint{7.876866in}{2.073706in}}{\pgfqpoint{7.873299in}{2.070140in}}%
\pgfpathcurveto{\pgfqpoint{7.869733in}{2.066573in}}{\pgfqpoint{7.867729in}{2.061736in}}{\pgfqpoint{7.867729in}{2.056692in}}%
\pgfpathcurveto{\pgfqpoint{7.867729in}{2.051648in}}{\pgfqpoint{7.869733in}{2.046811in}}{\pgfqpoint{7.873299in}{2.043244in}}%
\pgfpathcurveto{\pgfqpoint{7.876866in}{2.039678in}}{\pgfqpoint{7.881703in}{2.037674in}}{\pgfqpoint{7.886747in}{2.037674in}}%
\pgfpathclose%
\pgfusepath{fill}%
\end{pgfscope}%
\begin{pgfscope}%
\pgfpathrectangle{\pgfqpoint{6.572727in}{0.474100in}}{\pgfqpoint{4.227273in}{3.318700in}}%
\pgfusepath{clip}%
\pgfsetbuttcap%
\pgfsetroundjoin%
\definecolor{currentfill}{rgb}{0.267004,0.004874,0.329415}%
\pgfsetfillcolor{currentfill}%
\pgfsetfillopacity{0.700000}%
\pgfsetlinewidth{0.000000pt}%
\definecolor{currentstroke}{rgb}{0.000000,0.000000,0.000000}%
\pgfsetstrokecolor{currentstroke}%
\pgfsetstrokeopacity{0.700000}%
\pgfsetdash{}{0pt}%
\pgfpathmoveto{\pgfqpoint{7.008253in}{2.027720in}}%
\pgfpathcurveto{\pgfqpoint{7.013296in}{2.027720in}}{\pgfqpoint{7.018134in}{2.029724in}}{\pgfqpoint{7.021700in}{2.033290in}}%
\pgfpathcurveto{\pgfqpoint{7.025267in}{2.036857in}}{\pgfqpoint{7.027271in}{2.041694in}}{\pgfqpoint{7.027271in}{2.046738in}}%
\pgfpathcurveto{\pgfqpoint{7.027271in}{2.051782in}}{\pgfqpoint{7.025267in}{2.056619in}}{\pgfqpoint{7.021700in}{2.060186in}}%
\pgfpathcurveto{\pgfqpoint{7.018134in}{2.063752in}}{\pgfqpoint{7.013296in}{2.065756in}}{\pgfqpoint{7.008253in}{2.065756in}}%
\pgfpathcurveto{\pgfqpoint{7.003209in}{2.065756in}}{\pgfqpoint{6.998371in}{2.063752in}}{\pgfqpoint{6.994805in}{2.060186in}}%
\pgfpathcurveto{\pgfqpoint{6.991238in}{2.056619in}}{\pgfqpoint{6.989234in}{2.051782in}}{\pgfqpoint{6.989234in}{2.046738in}}%
\pgfpathcurveto{\pgfqpoint{6.989234in}{2.041694in}}{\pgfqpoint{6.991238in}{2.036857in}}{\pgfqpoint{6.994805in}{2.033290in}}%
\pgfpathcurveto{\pgfqpoint{6.998371in}{2.029724in}}{\pgfqpoint{7.003209in}{2.027720in}}{\pgfqpoint{7.008253in}{2.027720in}}%
\pgfpathclose%
\pgfusepath{fill}%
\end{pgfscope}%
\begin{pgfscope}%
\pgfpathrectangle{\pgfqpoint{6.572727in}{0.474100in}}{\pgfqpoint{4.227273in}{3.318700in}}%
\pgfusepath{clip}%
\pgfsetbuttcap%
\pgfsetroundjoin%
\definecolor{currentfill}{rgb}{0.127568,0.566949,0.550556}%
\pgfsetfillcolor{currentfill}%
\pgfsetfillopacity{0.700000}%
\pgfsetlinewidth{0.000000pt}%
\definecolor{currentstroke}{rgb}{0.000000,0.000000,0.000000}%
\pgfsetstrokecolor{currentstroke}%
\pgfsetstrokeopacity{0.700000}%
\pgfsetdash{}{0pt}%
\pgfpathmoveto{\pgfqpoint{7.762669in}{1.978695in}}%
\pgfpathcurveto{\pgfqpoint{7.767713in}{1.978695in}}{\pgfqpoint{7.772551in}{1.980699in}}{\pgfqpoint{7.776117in}{1.984265in}}%
\pgfpathcurveto{\pgfqpoint{7.779684in}{1.987831in}}{\pgfqpoint{7.781688in}{1.992669in}}{\pgfqpoint{7.781688in}{1.997713in}}%
\pgfpathcurveto{\pgfqpoint{7.781688in}{2.002756in}}{\pgfqpoint{7.779684in}{2.007594in}}{\pgfqpoint{7.776117in}{2.011161in}}%
\pgfpathcurveto{\pgfqpoint{7.772551in}{2.014727in}}{\pgfqpoint{7.767713in}{2.016731in}}{\pgfqpoint{7.762669in}{2.016731in}}%
\pgfpathcurveto{\pgfqpoint{7.757626in}{2.016731in}}{\pgfqpoint{7.752788in}{2.014727in}}{\pgfqpoint{7.749222in}{2.011161in}}%
\pgfpathcurveto{\pgfqpoint{7.745655in}{2.007594in}}{\pgfqpoint{7.743651in}{2.002756in}}{\pgfqpoint{7.743651in}{1.997713in}}%
\pgfpathcurveto{\pgfqpoint{7.743651in}{1.992669in}}{\pgfqpoint{7.745655in}{1.987831in}}{\pgfqpoint{7.749222in}{1.984265in}}%
\pgfpathcurveto{\pgfqpoint{7.752788in}{1.980699in}}{\pgfqpoint{7.757626in}{1.978695in}}{\pgfqpoint{7.762669in}{1.978695in}}%
\pgfpathclose%
\pgfusepath{fill}%
\end{pgfscope}%
\begin{pgfscope}%
\pgfpathrectangle{\pgfqpoint{6.572727in}{0.474100in}}{\pgfqpoint{4.227273in}{3.318700in}}%
\pgfusepath{clip}%
\pgfsetbuttcap%
\pgfsetroundjoin%
\definecolor{currentfill}{rgb}{0.127568,0.566949,0.550556}%
\pgfsetfillcolor{currentfill}%
\pgfsetfillopacity{0.700000}%
\pgfsetlinewidth{0.000000pt}%
\definecolor{currentstroke}{rgb}{0.000000,0.000000,0.000000}%
\pgfsetstrokecolor{currentstroke}%
\pgfsetstrokeopacity{0.700000}%
\pgfsetdash{}{0pt}%
\pgfpathmoveto{\pgfqpoint{8.051190in}{1.564707in}}%
\pgfpathcurveto{\pgfqpoint{8.056233in}{1.564707in}}{\pgfqpoint{8.061071in}{1.566711in}}{\pgfqpoint{8.064638in}{1.570277in}}%
\pgfpathcurveto{\pgfqpoint{8.068204in}{1.573843in}}{\pgfqpoint{8.070208in}{1.578681in}}{\pgfqpoint{8.070208in}{1.583725in}}%
\pgfpathcurveto{\pgfqpoint{8.070208in}{1.588769in}}{\pgfqpoint{8.068204in}{1.593606in}}{\pgfqpoint{8.064638in}{1.597173in}}%
\pgfpathcurveto{\pgfqpoint{8.061071in}{1.600739in}}{\pgfqpoint{8.056233in}{1.602743in}}{\pgfqpoint{8.051190in}{1.602743in}}%
\pgfpathcurveto{\pgfqpoint{8.046146in}{1.602743in}}{\pgfqpoint{8.041308in}{1.600739in}}{\pgfqpoint{8.037742in}{1.597173in}}%
\pgfpathcurveto{\pgfqpoint{8.034176in}{1.593606in}}{\pgfqpoint{8.032172in}{1.588769in}}{\pgfqpoint{8.032172in}{1.583725in}}%
\pgfpathcurveto{\pgfqpoint{8.032172in}{1.578681in}}{\pgfqpoint{8.034176in}{1.573843in}}{\pgfqpoint{8.037742in}{1.570277in}}%
\pgfpathcurveto{\pgfqpoint{8.041308in}{1.566711in}}{\pgfqpoint{8.046146in}{1.564707in}}{\pgfqpoint{8.051190in}{1.564707in}}%
\pgfpathclose%
\pgfusepath{fill}%
\end{pgfscope}%
\begin{pgfscope}%
\pgfpathrectangle{\pgfqpoint{6.572727in}{0.474100in}}{\pgfqpoint{4.227273in}{3.318700in}}%
\pgfusepath{clip}%
\pgfsetbuttcap%
\pgfsetroundjoin%
\definecolor{currentfill}{rgb}{0.127568,0.566949,0.550556}%
\pgfsetfillcolor{currentfill}%
\pgfsetfillopacity{0.700000}%
\pgfsetlinewidth{0.000000pt}%
\definecolor{currentstroke}{rgb}{0.000000,0.000000,0.000000}%
\pgfsetstrokecolor{currentstroke}%
\pgfsetstrokeopacity{0.700000}%
\pgfsetdash{}{0pt}%
\pgfpathmoveto{\pgfqpoint{7.897754in}{2.676451in}}%
\pgfpathcurveto{\pgfqpoint{7.902798in}{2.676451in}}{\pgfqpoint{7.907635in}{2.678455in}}{\pgfqpoint{7.911202in}{2.682022in}}%
\pgfpathcurveto{\pgfqpoint{7.914768in}{2.685588in}}{\pgfqpoint{7.916772in}{2.690426in}}{\pgfqpoint{7.916772in}{2.695470in}}%
\pgfpathcurveto{\pgfqpoint{7.916772in}{2.700513in}}{\pgfqpoint{7.914768in}{2.705351in}}{\pgfqpoint{7.911202in}{2.708917in}}%
\pgfpathcurveto{\pgfqpoint{7.907635in}{2.712484in}}{\pgfqpoint{7.902798in}{2.714488in}}{\pgfqpoint{7.897754in}{2.714488in}}%
\pgfpathcurveto{\pgfqpoint{7.892710in}{2.714488in}}{\pgfqpoint{7.887872in}{2.712484in}}{\pgfqpoint{7.884306in}{2.708917in}}%
\pgfpathcurveto{\pgfqpoint{7.880740in}{2.705351in}}{\pgfqpoint{7.878736in}{2.700513in}}{\pgfqpoint{7.878736in}{2.695470in}}%
\pgfpathcurveto{\pgfqpoint{7.878736in}{2.690426in}}{\pgfqpoint{7.880740in}{2.685588in}}{\pgfqpoint{7.884306in}{2.682022in}}%
\pgfpathcurveto{\pgfqpoint{7.887872in}{2.678455in}}{\pgfqpoint{7.892710in}{2.676451in}}{\pgfqpoint{7.897754in}{2.676451in}}%
\pgfpathclose%
\pgfusepath{fill}%
\end{pgfscope}%
\begin{pgfscope}%
\pgfpathrectangle{\pgfqpoint{6.572727in}{0.474100in}}{\pgfqpoint{4.227273in}{3.318700in}}%
\pgfusepath{clip}%
\pgfsetbuttcap%
\pgfsetroundjoin%
\definecolor{currentfill}{rgb}{0.127568,0.566949,0.550556}%
\pgfsetfillcolor{currentfill}%
\pgfsetfillopacity{0.700000}%
\pgfsetlinewidth{0.000000pt}%
\definecolor{currentstroke}{rgb}{0.000000,0.000000,0.000000}%
\pgfsetstrokecolor{currentstroke}%
\pgfsetstrokeopacity{0.700000}%
\pgfsetdash{}{0pt}%
\pgfpathmoveto{\pgfqpoint{8.591351in}{1.895249in}}%
\pgfpathcurveto{\pgfqpoint{8.596394in}{1.895249in}}{\pgfqpoint{8.601232in}{1.897253in}}{\pgfqpoint{8.604799in}{1.900819in}}%
\pgfpathcurveto{\pgfqpoint{8.608365in}{1.904385in}}{\pgfqpoint{8.610369in}{1.909223in}}{\pgfqpoint{8.610369in}{1.914267in}}%
\pgfpathcurveto{\pgfqpoint{8.610369in}{1.919311in}}{\pgfqpoint{8.608365in}{1.924148in}}{\pgfqpoint{8.604799in}{1.927715in}}%
\pgfpathcurveto{\pgfqpoint{8.601232in}{1.931281in}}{\pgfqpoint{8.596394in}{1.933285in}}{\pgfqpoint{8.591351in}{1.933285in}}%
\pgfpathcurveto{\pgfqpoint{8.586307in}{1.933285in}}{\pgfqpoint{8.581469in}{1.931281in}}{\pgfqpoint{8.577903in}{1.927715in}}%
\pgfpathcurveto{\pgfqpoint{8.574336in}{1.924148in}}{\pgfqpoint{8.572333in}{1.919311in}}{\pgfqpoint{8.572333in}{1.914267in}}%
\pgfpathcurveto{\pgfqpoint{8.572333in}{1.909223in}}{\pgfqpoint{8.574336in}{1.904385in}}{\pgfqpoint{8.577903in}{1.900819in}}%
\pgfpathcurveto{\pgfqpoint{8.581469in}{1.897253in}}{\pgfqpoint{8.586307in}{1.895249in}}{\pgfqpoint{8.591351in}{1.895249in}}%
\pgfpathclose%
\pgfusepath{fill}%
\end{pgfscope}%
\begin{pgfscope}%
\pgfpathrectangle{\pgfqpoint{6.572727in}{0.474100in}}{\pgfqpoint{4.227273in}{3.318700in}}%
\pgfusepath{clip}%
\pgfsetbuttcap%
\pgfsetroundjoin%
\definecolor{currentfill}{rgb}{0.127568,0.566949,0.550556}%
\pgfsetfillcolor{currentfill}%
\pgfsetfillopacity{0.700000}%
\pgfsetlinewidth{0.000000pt}%
\definecolor{currentstroke}{rgb}{0.000000,0.000000,0.000000}%
\pgfsetstrokecolor{currentstroke}%
\pgfsetstrokeopacity{0.700000}%
\pgfsetdash{}{0pt}%
\pgfpathmoveto{\pgfqpoint{8.008299in}{2.068373in}}%
\pgfpathcurveto{\pgfqpoint{8.013343in}{2.068373in}}{\pgfqpoint{8.018181in}{2.070377in}}{\pgfqpoint{8.021747in}{2.073943in}}%
\pgfpathcurveto{\pgfqpoint{8.025313in}{2.077510in}}{\pgfqpoint{8.027317in}{2.082347in}}{\pgfqpoint{8.027317in}{2.087391in}}%
\pgfpathcurveto{\pgfqpoint{8.027317in}{2.092435in}}{\pgfqpoint{8.025313in}{2.097272in}}{\pgfqpoint{8.021747in}{2.100839in}}%
\pgfpathcurveto{\pgfqpoint{8.018181in}{2.104405in}}{\pgfqpoint{8.013343in}{2.106409in}}{\pgfqpoint{8.008299in}{2.106409in}}%
\pgfpathcurveto{\pgfqpoint{8.003255in}{2.106409in}}{\pgfqpoint{7.998418in}{2.104405in}}{\pgfqpoint{7.994851in}{2.100839in}}%
\pgfpathcurveto{\pgfqpoint{7.991285in}{2.097272in}}{\pgfqpoint{7.989281in}{2.092435in}}{\pgfqpoint{7.989281in}{2.087391in}}%
\pgfpathcurveto{\pgfqpoint{7.989281in}{2.082347in}}{\pgfqpoint{7.991285in}{2.077510in}}{\pgfqpoint{7.994851in}{2.073943in}}%
\pgfpathcurveto{\pgfqpoint{7.998418in}{2.070377in}}{\pgfqpoint{8.003255in}{2.068373in}}{\pgfqpoint{8.008299in}{2.068373in}}%
\pgfpathclose%
\pgfusepath{fill}%
\end{pgfscope}%
\begin{pgfscope}%
\pgfpathrectangle{\pgfqpoint{6.572727in}{0.474100in}}{\pgfqpoint{4.227273in}{3.318700in}}%
\pgfusepath{clip}%
\pgfsetbuttcap%
\pgfsetroundjoin%
\definecolor{currentfill}{rgb}{0.127568,0.566949,0.550556}%
\pgfsetfillcolor{currentfill}%
\pgfsetfillopacity{0.700000}%
\pgfsetlinewidth{0.000000pt}%
\definecolor{currentstroke}{rgb}{0.000000,0.000000,0.000000}%
\pgfsetstrokecolor{currentstroke}%
\pgfsetstrokeopacity{0.700000}%
\pgfsetdash{}{0pt}%
\pgfpathmoveto{\pgfqpoint{8.025682in}{2.796327in}}%
\pgfpathcurveto{\pgfqpoint{8.030726in}{2.796327in}}{\pgfqpoint{8.035564in}{2.798331in}}{\pgfqpoint{8.039130in}{2.801898in}}%
\pgfpathcurveto{\pgfqpoint{8.042696in}{2.805464in}}{\pgfqpoint{8.044700in}{2.810302in}}{\pgfqpoint{8.044700in}{2.815345in}}%
\pgfpathcurveto{\pgfqpoint{8.044700in}{2.820389in}}{\pgfqpoint{8.042696in}{2.825227in}}{\pgfqpoint{8.039130in}{2.828793in}}%
\pgfpathcurveto{\pgfqpoint{8.035564in}{2.832360in}}{\pgfqpoint{8.030726in}{2.834364in}}{\pgfqpoint{8.025682in}{2.834364in}}%
\pgfpathcurveto{\pgfqpoint{8.020638in}{2.834364in}}{\pgfqpoint{8.015801in}{2.832360in}}{\pgfqpoint{8.012234in}{2.828793in}}%
\pgfpathcurveto{\pgfqpoint{8.008668in}{2.825227in}}{\pgfqpoint{8.006664in}{2.820389in}}{\pgfqpoint{8.006664in}{2.815345in}}%
\pgfpathcurveto{\pgfqpoint{8.006664in}{2.810302in}}{\pgfqpoint{8.008668in}{2.805464in}}{\pgfqpoint{8.012234in}{2.801898in}}%
\pgfpathcurveto{\pgfqpoint{8.015801in}{2.798331in}}{\pgfqpoint{8.020638in}{2.796327in}}{\pgfqpoint{8.025682in}{2.796327in}}%
\pgfpathclose%
\pgfusepath{fill}%
\end{pgfscope}%
\begin{pgfscope}%
\pgfpathrectangle{\pgfqpoint{6.572727in}{0.474100in}}{\pgfqpoint{4.227273in}{3.318700in}}%
\pgfusepath{clip}%
\pgfsetbuttcap%
\pgfsetroundjoin%
\definecolor{currentfill}{rgb}{0.127568,0.566949,0.550556}%
\pgfsetfillcolor{currentfill}%
\pgfsetfillopacity{0.700000}%
\pgfsetlinewidth{0.000000pt}%
\definecolor{currentstroke}{rgb}{0.000000,0.000000,0.000000}%
\pgfsetstrokecolor{currentstroke}%
\pgfsetstrokeopacity{0.700000}%
\pgfsetdash{}{0pt}%
\pgfpathmoveto{\pgfqpoint{7.696817in}{2.961298in}}%
\pgfpathcurveto{\pgfqpoint{7.701861in}{2.961298in}}{\pgfqpoint{7.706699in}{2.963302in}}{\pgfqpoint{7.710265in}{2.966868in}}%
\pgfpathcurveto{\pgfqpoint{7.713831in}{2.970434in}}{\pgfqpoint{7.715835in}{2.975272in}}{\pgfqpoint{7.715835in}{2.980316in}}%
\pgfpathcurveto{\pgfqpoint{7.715835in}{2.985360in}}{\pgfqpoint{7.713831in}{2.990197in}}{\pgfqpoint{7.710265in}{2.993764in}}%
\pgfpathcurveto{\pgfqpoint{7.706699in}{2.997330in}}{\pgfqpoint{7.701861in}{2.999334in}}{\pgfqpoint{7.696817in}{2.999334in}}%
\pgfpathcurveto{\pgfqpoint{7.691774in}{2.999334in}}{\pgfqpoint{7.686936in}{2.997330in}}{\pgfqpoint{7.683369in}{2.993764in}}%
\pgfpathcurveto{\pgfqpoint{7.679803in}{2.990197in}}{\pgfqpoint{7.677799in}{2.985360in}}{\pgfqpoint{7.677799in}{2.980316in}}%
\pgfpathcurveto{\pgfqpoint{7.677799in}{2.975272in}}{\pgfqpoint{7.679803in}{2.970434in}}{\pgfqpoint{7.683369in}{2.966868in}}%
\pgfpathcurveto{\pgfqpoint{7.686936in}{2.963302in}}{\pgfqpoint{7.691774in}{2.961298in}}{\pgfqpoint{7.696817in}{2.961298in}}%
\pgfpathclose%
\pgfusepath{fill}%
\end{pgfscope}%
\begin{pgfscope}%
\pgfpathrectangle{\pgfqpoint{6.572727in}{0.474100in}}{\pgfqpoint{4.227273in}{3.318700in}}%
\pgfusepath{clip}%
\pgfsetbuttcap%
\pgfsetroundjoin%
\definecolor{currentfill}{rgb}{0.993248,0.906157,0.143936}%
\pgfsetfillcolor{currentfill}%
\pgfsetfillopacity{0.700000}%
\pgfsetlinewidth{0.000000pt}%
\definecolor{currentstroke}{rgb}{0.000000,0.000000,0.000000}%
\pgfsetstrokecolor{currentstroke}%
\pgfsetstrokeopacity{0.700000}%
\pgfsetdash{}{0pt}%
\pgfpathmoveto{\pgfqpoint{9.263842in}{1.358747in}}%
\pgfpathcurveto{\pgfqpoint{9.268886in}{1.358747in}}{\pgfqpoint{9.273724in}{1.360751in}}{\pgfqpoint{9.277290in}{1.364318in}}%
\pgfpathcurveto{\pgfqpoint{9.280857in}{1.367884in}}{\pgfqpoint{9.282860in}{1.372722in}}{\pgfqpoint{9.282860in}{1.377766in}}%
\pgfpathcurveto{\pgfqpoint{9.282860in}{1.382809in}}{\pgfqpoint{9.280857in}{1.387647in}}{\pgfqpoint{9.277290in}{1.391213in}}%
\pgfpathcurveto{\pgfqpoint{9.273724in}{1.394780in}}{\pgfqpoint{9.268886in}{1.396784in}}{\pgfqpoint{9.263842in}{1.396784in}}%
\pgfpathcurveto{\pgfqpoint{9.258799in}{1.396784in}}{\pgfqpoint{9.253961in}{1.394780in}}{\pgfqpoint{9.250394in}{1.391213in}}%
\pgfpathcurveto{\pgfqpoint{9.246828in}{1.387647in}}{\pgfqpoint{9.244824in}{1.382809in}}{\pgfqpoint{9.244824in}{1.377766in}}%
\pgfpathcurveto{\pgfqpoint{9.244824in}{1.372722in}}{\pgfqpoint{9.246828in}{1.367884in}}{\pgfqpoint{9.250394in}{1.364318in}}%
\pgfpathcurveto{\pgfqpoint{9.253961in}{1.360751in}}{\pgfqpoint{9.258799in}{1.358747in}}{\pgfqpoint{9.263842in}{1.358747in}}%
\pgfpathclose%
\pgfusepath{fill}%
\end{pgfscope}%
\begin{pgfscope}%
\pgfpathrectangle{\pgfqpoint{6.572727in}{0.474100in}}{\pgfqpoint{4.227273in}{3.318700in}}%
\pgfusepath{clip}%
\pgfsetbuttcap%
\pgfsetroundjoin%
\definecolor{currentfill}{rgb}{0.993248,0.906157,0.143936}%
\pgfsetfillcolor{currentfill}%
\pgfsetfillopacity{0.700000}%
\pgfsetlinewidth{0.000000pt}%
\definecolor{currentstroke}{rgb}{0.000000,0.000000,0.000000}%
\pgfsetstrokecolor{currentstroke}%
\pgfsetstrokeopacity{0.700000}%
\pgfsetdash{}{0pt}%
\pgfpathmoveto{\pgfqpoint{9.568155in}{1.588647in}}%
\pgfpathcurveto{\pgfqpoint{9.573199in}{1.588647in}}{\pgfqpoint{9.578036in}{1.590651in}}{\pgfqpoint{9.581603in}{1.594217in}}%
\pgfpathcurveto{\pgfqpoint{9.585169in}{1.597784in}}{\pgfqpoint{9.587173in}{1.602621in}}{\pgfqpoint{9.587173in}{1.607665in}}%
\pgfpathcurveto{\pgfqpoint{9.587173in}{1.612709in}}{\pgfqpoint{9.585169in}{1.617547in}}{\pgfqpoint{9.581603in}{1.621113in}}%
\pgfpathcurveto{\pgfqpoint{9.578036in}{1.624679in}}{\pgfqpoint{9.573199in}{1.626683in}}{\pgfqpoint{9.568155in}{1.626683in}}%
\pgfpathcurveto{\pgfqpoint{9.563111in}{1.626683in}}{\pgfqpoint{9.558273in}{1.624679in}}{\pgfqpoint{9.554707in}{1.621113in}}%
\pgfpathcurveto{\pgfqpoint{9.551141in}{1.617547in}}{\pgfqpoint{9.549137in}{1.612709in}}{\pgfqpoint{9.549137in}{1.607665in}}%
\pgfpathcurveto{\pgfqpoint{9.549137in}{1.602621in}}{\pgfqpoint{9.551141in}{1.597784in}}{\pgfqpoint{9.554707in}{1.594217in}}%
\pgfpathcurveto{\pgfqpoint{9.558273in}{1.590651in}}{\pgfqpoint{9.563111in}{1.588647in}}{\pgfqpoint{9.568155in}{1.588647in}}%
\pgfpathclose%
\pgfusepath{fill}%
\end{pgfscope}%
\begin{pgfscope}%
\pgfpathrectangle{\pgfqpoint{6.572727in}{0.474100in}}{\pgfqpoint{4.227273in}{3.318700in}}%
\pgfusepath{clip}%
\pgfsetbuttcap%
\pgfsetroundjoin%
\definecolor{currentfill}{rgb}{0.993248,0.906157,0.143936}%
\pgfsetfillcolor{currentfill}%
\pgfsetfillopacity{0.700000}%
\pgfsetlinewidth{0.000000pt}%
\definecolor{currentstroke}{rgb}{0.000000,0.000000,0.000000}%
\pgfsetstrokecolor{currentstroke}%
\pgfsetstrokeopacity{0.700000}%
\pgfsetdash{}{0pt}%
\pgfpathmoveto{\pgfqpoint{9.920171in}{1.154693in}}%
\pgfpathcurveto{\pgfqpoint{9.925215in}{1.154693in}}{\pgfqpoint{9.930053in}{1.156696in}}{\pgfqpoint{9.933619in}{1.160263in}}%
\pgfpathcurveto{\pgfqpoint{9.937186in}{1.163829in}}{\pgfqpoint{9.939190in}{1.168667in}}{\pgfqpoint{9.939190in}{1.173711in}}%
\pgfpathcurveto{\pgfqpoint{9.939190in}{1.178754in}}{\pgfqpoint{9.937186in}{1.183592in}}{\pgfqpoint{9.933619in}{1.187159in}}%
\pgfpathcurveto{\pgfqpoint{9.930053in}{1.190725in}}{\pgfqpoint{9.925215in}{1.192729in}}{\pgfqpoint{9.920171in}{1.192729in}}%
\pgfpathcurveto{\pgfqpoint{9.915128in}{1.192729in}}{\pgfqpoint{9.910290in}{1.190725in}}{\pgfqpoint{9.906724in}{1.187159in}}%
\pgfpathcurveto{\pgfqpoint{9.903157in}{1.183592in}}{\pgfqpoint{9.901153in}{1.178754in}}{\pgfqpoint{9.901153in}{1.173711in}}%
\pgfpathcurveto{\pgfqpoint{9.901153in}{1.168667in}}{\pgfqpoint{9.903157in}{1.163829in}}{\pgfqpoint{9.906724in}{1.160263in}}%
\pgfpathcurveto{\pgfqpoint{9.910290in}{1.156696in}}{\pgfqpoint{9.915128in}{1.154693in}}{\pgfqpoint{9.920171in}{1.154693in}}%
\pgfpathclose%
\pgfusepath{fill}%
\end{pgfscope}%
\begin{pgfscope}%
\pgfpathrectangle{\pgfqpoint{6.572727in}{0.474100in}}{\pgfqpoint{4.227273in}{3.318700in}}%
\pgfusepath{clip}%
\pgfsetbuttcap%
\pgfsetroundjoin%
\definecolor{currentfill}{rgb}{0.127568,0.566949,0.550556}%
\pgfsetfillcolor{currentfill}%
\pgfsetfillopacity{0.700000}%
\pgfsetlinewidth{0.000000pt}%
\definecolor{currentstroke}{rgb}{0.000000,0.000000,0.000000}%
\pgfsetstrokecolor{currentstroke}%
\pgfsetstrokeopacity{0.700000}%
\pgfsetdash{}{0pt}%
\pgfpathmoveto{\pgfqpoint{7.993456in}{3.370458in}}%
\pgfpathcurveto{\pgfqpoint{7.998500in}{3.370458in}}{\pgfqpoint{8.003338in}{3.372462in}}{\pgfqpoint{8.006904in}{3.376028in}}%
\pgfpathcurveto{\pgfqpoint{8.010470in}{3.379595in}}{\pgfqpoint{8.012474in}{3.384433in}}{\pgfqpoint{8.012474in}{3.389476in}}%
\pgfpathcurveto{\pgfqpoint{8.012474in}{3.394520in}}{\pgfqpoint{8.010470in}{3.399358in}}{\pgfqpoint{8.006904in}{3.402924in}}%
\pgfpathcurveto{\pgfqpoint{8.003338in}{3.406491in}}{\pgfqpoint{7.998500in}{3.408494in}}{\pgfqpoint{7.993456in}{3.408494in}}%
\pgfpathcurveto{\pgfqpoint{7.988412in}{3.408494in}}{\pgfqpoint{7.983575in}{3.406491in}}{\pgfqpoint{7.980008in}{3.402924in}}%
\pgfpathcurveto{\pgfqpoint{7.976442in}{3.399358in}}{\pgfqpoint{7.974438in}{3.394520in}}{\pgfqpoint{7.974438in}{3.389476in}}%
\pgfpathcurveto{\pgfqpoint{7.974438in}{3.384433in}}{\pgfqpoint{7.976442in}{3.379595in}}{\pgfqpoint{7.980008in}{3.376028in}}%
\pgfpathcurveto{\pgfqpoint{7.983575in}{3.372462in}}{\pgfqpoint{7.988412in}{3.370458in}}{\pgfqpoint{7.993456in}{3.370458in}}%
\pgfpathclose%
\pgfusepath{fill}%
\end{pgfscope}%
\begin{pgfscope}%
\pgfpathrectangle{\pgfqpoint{6.572727in}{0.474100in}}{\pgfqpoint{4.227273in}{3.318700in}}%
\pgfusepath{clip}%
\pgfsetbuttcap%
\pgfsetroundjoin%
\definecolor{currentfill}{rgb}{0.127568,0.566949,0.550556}%
\pgfsetfillcolor{currentfill}%
\pgfsetfillopacity{0.700000}%
\pgfsetlinewidth{0.000000pt}%
\definecolor{currentstroke}{rgb}{0.000000,0.000000,0.000000}%
\pgfsetstrokecolor{currentstroke}%
\pgfsetstrokeopacity{0.700000}%
\pgfsetdash{}{0pt}%
\pgfpathmoveto{\pgfqpoint{7.379559in}{1.546080in}}%
\pgfpathcurveto{\pgfqpoint{7.384602in}{1.546080in}}{\pgfqpoint{7.389440in}{1.548083in}}{\pgfqpoint{7.393007in}{1.551650in}}%
\pgfpathcurveto{\pgfqpoint{7.396573in}{1.555216in}}{\pgfqpoint{7.398577in}{1.560054in}}{\pgfqpoint{7.398577in}{1.565098in}}%
\pgfpathcurveto{\pgfqpoint{7.398577in}{1.570141in}}{\pgfqpoint{7.396573in}{1.574979in}}{\pgfqpoint{7.393007in}{1.578546in}}%
\pgfpathcurveto{\pgfqpoint{7.389440in}{1.582112in}}{\pgfqpoint{7.384602in}{1.584116in}}{\pgfqpoint{7.379559in}{1.584116in}}%
\pgfpathcurveto{\pgfqpoint{7.374515in}{1.584116in}}{\pgfqpoint{7.369677in}{1.582112in}}{\pgfqpoint{7.366111in}{1.578546in}}%
\pgfpathcurveto{\pgfqpoint{7.362544in}{1.574979in}}{\pgfqpoint{7.360541in}{1.570141in}}{\pgfqpoint{7.360541in}{1.565098in}}%
\pgfpathcurveto{\pgfqpoint{7.360541in}{1.560054in}}{\pgfqpoint{7.362544in}{1.555216in}}{\pgfqpoint{7.366111in}{1.551650in}}%
\pgfpathcurveto{\pgfqpoint{7.369677in}{1.548083in}}{\pgfqpoint{7.374515in}{1.546080in}}{\pgfqpoint{7.379559in}{1.546080in}}%
\pgfpathclose%
\pgfusepath{fill}%
\end{pgfscope}%
\begin{pgfscope}%
\pgfpathrectangle{\pgfqpoint{6.572727in}{0.474100in}}{\pgfqpoint{4.227273in}{3.318700in}}%
\pgfusepath{clip}%
\pgfsetbuttcap%
\pgfsetroundjoin%
\definecolor{currentfill}{rgb}{0.127568,0.566949,0.550556}%
\pgfsetfillcolor{currentfill}%
\pgfsetfillopacity{0.700000}%
\pgfsetlinewidth{0.000000pt}%
\definecolor{currentstroke}{rgb}{0.000000,0.000000,0.000000}%
\pgfsetstrokecolor{currentstroke}%
\pgfsetstrokeopacity{0.700000}%
\pgfsetdash{}{0pt}%
\pgfpathmoveto{\pgfqpoint{8.224182in}{2.813490in}}%
\pgfpathcurveto{\pgfqpoint{8.229226in}{2.813490in}}{\pgfqpoint{8.234064in}{2.815494in}}{\pgfqpoint{8.237630in}{2.819061in}}%
\pgfpathcurveto{\pgfqpoint{8.241197in}{2.822627in}}{\pgfqpoint{8.243201in}{2.827465in}}{\pgfqpoint{8.243201in}{2.832509in}}%
\pgfpathcurveto{\pgfqpoint{8.243201in}{2.837552in}}{\pgfqpoint{8.241197in}{2.842390in}}{\pgfqpoint{8.237630in}{2.845956in}}%
\pgfpathcurveto{\pgfqpoint{8.234064in}{2.849523in}}{\pgfqpoint{8.229226in}{2.851527in}}{\pgfqpoint{8.224182in}{2.851527in}}%
\pgfpathcurveto{\pgfqpoint{8.219139in}{2.851527in}}{\pgfqpoint{8.214301in}{2.849523in}}{\pgfqpoint{8.210735in}{2.845956in}}%
\pgfpathcurveto{\pgfqpoint{8.207168in}{2.842390in}}{\pgfqpoint{8.205164in}{2.837552in}}{\pgfqpoint{8.205164in}{2.832509in}}%
\pgfpathcurveto{\pgfqpoint{8.205164in}{2.827465in}}{\pgfqpoint{8.207168in}{2.822627in}}{\pgfqpoint{8.210735in}{2.819061in}}%
\pgfpathcurveto{\pgfqpoint{8.214301in}{2.815494in}}{\pgfqpoint{8.219139in}{2.813490in}}{\pgfqpoint{8.224182in}{2.813490in}}%
\pgfpathclose%
\pgfusepath{fill}%
\end{pgfscope}%
\begin{pgfscope}%
\pgfpathrectangle{\pgfqpoint{6.572727in}{0.474100in}}{\pgfqpoint{4.227273in}{3.318700in}}%
\pgfusepath{clip}%
\pgfsetbuttcap%
\pgfsetroundjoin%
\definecolor{currentfill}{rgb}{0.993248,0.906157,0.143936}%
\pgfsetfillcolor{currentfill}%
\pgfsetfillopacity{0.700000}%
\pgfsetlinewidth{0.000000pt}%
\definecolor{currentstroke}{rgb}{0.000000,0.000000,0.000000}%
\pgfsetstrokecolor{currentstroke}%
\pgfsetstrokeopacity{0.700000}%
\pgfsetdash{}{0pt}%
\pgfpathmoveto{\pgfqpoint{9.570389in}{1.144786in}}%
\pgfpathcurveto{\pgfqpoint{9.575432in}{1.144786in}}{\pgfqpoint{9.580270in}{1.146790in}}{\pgfqpoint{9.583836in}{1.150356in}}%
\pgfpathcurveto{\pgfqpoint{9.587403in}{1.153922in}}{\pgfqpoint{9.589407in}{1.158760in}}{\pgfqpoint{9.589407in}{1.163804in}}%
\pgfpathcurveto{\pgfqpoint{9.589407in}{1.168847in}}{\pgfqpoint{9.587403in}{1.173685in}}{\pgfqpoint{9.583836in}{1.177252in}}%
\pgfpathcurveto{\pgfqpoint{9.580270in}{1.180818in}}{\pgfqpoint{9.575432in}{1.182822in}}{\pgfqpoint{9.570389in}{1.182822in}}%
\pgfpathcurveto{\pgfqpoint{9.565345in}{1.182822in}}{\pgfqpoint{9.560507in}{1.180818in}}{\pgfqpoint{9.556941in}{1.177252in}}%
\pgfpathcurveto{\pgfqpoint{9.553374in}{1.173685in}}{\pgfqpoint{9.551370in}{1.168847in}}{\pgfqpoint{9.551370in}{1.163804in}}%
\pgfpathcurveto{\pgfqpoint{9.551370in}{1.158760in}}{\pgfqpoint{9.553374in}{1.153922in}}{\pgfqpoint{9.556941in}{1.150356in}}%
\pgfpathcurveto{\pgfqpoint{9.560507in}{1.146790in}}{\pgfqpoint{9.565345in}{1.144786in}}{\pgfqpoint{9.570389in}{1.144786in}}%
\pgfpathclose%
\pgfusepath{fill}%
\end{pgfscope}%
\begin{pgfscope}%
\pgfpathrectangle{\pgfqpoint{6.572727in}{0.474100in}}{\pgfqpoint{4.227273in}{3.318700in}}%
\pgfusepath{clip}%
\pgfsetbuttcap%
\pgfsetroundjoin%
\definecolor{currentfill}{rgb}{0.993248,0.906157,0.143936}%
\pgfsetfillcolor{currentfill}%
\pgfsetfillopacity{0.700000}%
\pgfsetlinewidth{0.000000pt}%
\definecolor{currentstroke}{rgb}{0.000000,0.000000,0.000000}%
\pgfsetstrokecolor{currentstroke}%
\pgfsetstrokeopacity{0.700000}%
\pgfsetdash{}{0pt}%
\pgfpathmoveto{\pgfqpoint{9.253134in}{1.280960in}}%
\pgfpathcurveto{\pgfqpoint{9.258177in}{1.280960in}}{\pgfqpoint{9.263015in}{1.282964in}}{\pgfqpoint{9.266581in}{1.286530in}}%
\pgfpathcurveto{\pgfqpoint{9.270148in}{1.290097in}}{\pgfqpoint{9.272152in}{1.294935in}}{\pgfqpoint{9.272152in}{1.299978in}}%
\pgfpathcurveto{\pgfqpoint{9.272152in}{1.305022in}}{\pgfqpoint{9.270148in}{1.309860in}}{\pgfqpoint{9.266581in}{1.313426in}}%
\pgfpathcurveto{\pgfqpoint{9.263015in}{1.316993in}}{\pgfqpoint{9.258177in}{1.318996in}}{\pgfqpoint{9.253134in}{1.318996in}}%
\pgfpathcurveto{\pgfqpoint{9.248090in}{1.318996in}}{\pgfqpoint{9.243252in}{1.316993in}}{\pgfqpoint{9.239686in}{1.313426in}}%
\pgfpathcurveto{\pgfqpoint{9.236119in}{1.309860in}}{\pgfqpoint{9.234115in}{1.305022in}}{\pgfqpoint{9.234115in}{1.299978in}}%
\pgfpathcurveto{\pgfqpoint{9.234115in}{1.294935in}}{\pgfqpoint{9.236119in}{1.290097in}}{\pgfqpoint{9.239686in}{1.286530in}}%
\pgfpathcurveto{\pgfqpoint{9.243252in}{1.282964in}}{\pgfqpoint{9.248090in}{1.280960in}}{\pgfqpoint{9.253134in}{1.280960in}}%
\pgfpathclose%
\pgfusepath{fill}%
\end{pgfscope}%
\begin{pgfscope}%
\pgfpathrectangle{\pgfqpoint{6.572727in}{0.474100in}}{\pgfqpoint{4.227273in}{3.318700in}}%
\pgfusepath{clip}%
\pgfsetbuttcap%
\pgfsetroundjoin%
\definecolor{currentfill}{rgb}{0.127568,0.566949,0.550556}%
\pgfsetfillcolor{currentfill}%
\pgfsetfillopacity{0.700000}%
\pgfsetlinewidth{0.000000pt}%
\definecolor{currentstroke}{rgb}{0.000000,0.000000,0.000000}%
\pgfsetstrokecolor{currentstroke}%
\pgfsetstrokeopacity{0.700000}%
\pgfsetdash{}{0pt}%
\pgfpathmoveto{\pgfqpoint{8.658814in}{2.451817in}}%
\pgfpathcurveto{\pgfqpoint{8.663858in}{2.451817in}}{\pgfqpoint{8.668696in}{2.453821in}}{\pgfqpoint{8.672262in}{2.457388in}}%
\pgfpathcurveto{\pgfqpoint{8.675829in}{2.460954in}}{\pgfqpoint{8.677833in}{2.465792in}}{\pgfqpoint{8.677833in}{2.470836in}}%
\pgfpathcurveto{\pgfqpoint{8.677833in}{2.475879in}}{\pgfqpoint{8.675829in}{2.480717in}}{\pgfqpoint{8.672262in}{2.484283in}}%
\pgfpathcurveto{\pgfqpoint{8.668696in}{2.487850in}}{\pgfqpoint{8.663858in}{2.489854in}}{\pgfqpoint{8.658814in}{2.489854in}}%
\pgfpathcurveto{\pgfqpoint{8.653771in}{2.489854in}}{\pgfqpoint{8.648933in}{2.487850in}}{\pgfqpoint{8.645367in}{2.484283in}}%
\pgfpathcurveto{\pgfqpoint{8.641800in}{2.480717in}}{\pgfqpoint{8.639796in}{2.475879in}}{\pgfqpoint{8.639796in}{2.470836in}}%
\pgfpathcurveto{\pgfqpoint{8.639796in}{2.465792in}}{\pgfqpoint{8.641800in}{2.460954in}}{\pgfqpoint{8.645367in}{2.457388in}}%
\pgfpathcurveto{\pgfqpoint{8.648933in}{2.453821in}}{\pgfqpoint{8.653771in}{2.451817in}}{\pgfqpoint{8.658814in}{2.451817in}}%
\pgfpathclose%
\pgfusepath{fill}%
\end{pgfscope}%
\begin{pgfscope}%
\pgfpathrectangle{\pgfqpoint{6.572727in}{0.474100in}}{\pgfqpoint{4.227273in}{3.318700in}}%
\pgfusepath{clip}%
\pgfsetbuttcap%
\pgfsetroundjoin%
\definecolor{currentfill}{rgb}{0.127568,0.566949,0.550556}%
\pgfsetfillcolor{currentfill}%
\pgfsetfillopacity{0.700000}%
\pgfsetlinewidth{0.000000pt}%
\definecolor{currentstroke}{rgb}{0.000000,0.000000,0.000000}%
\pgfsetstrokecolor{currentstroke}%
\pgfsetstrokeopacity{0.700000}%
\pgfsetdash{}{0pt}%
\pgfpathmoveto{\pgfqpoint{7.867878in}{0.993533in}}%
\pgfpathcurveto{\pgfqpoint{7.872922in}{0.993533in}}{\pgfqpoint{7.877760in}{0.995537in}}{\pgfqpoint{7.881326in}{0.999103in}}%
\pgfpathcurveto{\pgfqpoint{7.884892in}{1.002669in}}{\pgfqpoint{7.886896in}{1.007507in}}{\pgfqpoint{7.886896in}{1.012551in}}%
\pgfpathcurveto{\pgfqpoint{7.886896in}{1.017595in}}{\pgfqpoint{7.884892in}{1.022432in}}{\pgfqpoint{7.881326in}{1.025999in}}%
\pgfpathcurveto{\pgfqpoint{7.877760in}{1.029565in}}{\pgfqpoint{7.872922in}{1.031569in}}{\pgfqpoint{7.867878in}{1.031569in}}%
\pgfpathcurveto{\pgfqpoint{7.862834in}{1.031569in}}{\pgfqpoint{7.857997in}{1.029565in}}{\pgfqpoint{7.854430in}{1.025999in}}%
\pgfpathcurveto{\pgfqpoint{7.850864in}{1.022432in}}{\pgfqpoint{7.848860in}{1.017595in}}{\pgfqpoint{7.848860in}{1.012551in}}%
\pgfpathcurveto{\pgfqpoint{7.848860in}{1.007507in}}{\pgfqpoint{7.850864in}{1.002669in}}{\pgfqpoint{7.854430in}{0.999103in}}%
\pgfpathcurveto{\pgfqpoint{7.857997in}{0.995537in}}{\pgfqpoint{7.862834in}{0.993533in}}{\pgfqpoint{7.867878in}{0.993533in}}%
\pgfpathclose%
\pgfusepath{fill}%
\end{pgfscope}%
\begin{pgfscope}%
\pgfpathrectangle{\pgfqpoint{6.572727in}{0.474100in}}{\pgfqpoint{4.227273in}{3.318700in}}%
\pgfusepath{clip}%
\pgfsetbuttcap%
\pgfsetroundjoin%
\definecolor{currentfill}{rgb}{0.993248,0.906157,0.143936}%
\pgfsetfillcolor{currentfill}%
\pgfsetfillopacity{0.700000}%
\pgfsetlinewidth{0.000000pt}%
\definecolor{currentstroke}{rgb}{0.000000,0.000000,0.000000}%
\pgfsetstrokecolor{currentstroke}%
\pgfsetstrokeopacity{0.700000}%
\pgfsetdash{}{0pt}%
\pgfpathmoveto{\pgfqpoint{9.600641in}{1.318018in}}%
\pgfpathcurveto{\pgfqpoint{9.605684in}{1.318018in}}{\pgfqpoint{9.610522in}{1.320021in}}{\pgfqpoint{9.614088in}{1.323588in}}%
\pgfpathcurveto{\pgfqpoint{9.617655in}{1.327154in}}{\pgfqpoint{9.619659in}{1.331992in}}{\pgfqpoint{9.619659in}{1.337036in}}%
\pgfpathcurveto{\pgfqpoint{9.619659in}{1.342079in}}{\pgfqpoint{9.617655in}{1.346917in}}{\pgfqpoint{9.614088in}{1.350484in}}%
\pgfpathcurveto{\pgfqpoint{9.610522in}{1.354050in}}{\pgfqpoint{9.605684in}{1.356054in}}{\pgfqpoint{9.600641in}{1.356054in}}%
\pgfpathcurveto{\pgfqpoint{9.595597in}{1.356054in}}{\pgfqpoint{9.590759in}{1.354050in}}{\pgfqpoint{9.587193in}{1.350484in}}%
\pgfpathcurveto{\pgfqpoint{9.583626in}{1.346917in}}{\pgfqpoint{9.581622in}{1.342079in}}{\pgfqpoint{9.581622in}{1.337036in}}%
\pgfpathcurveto{\pgfqpoint{9.581622in}{1.331992in}}{\pgfqpoint{9.583626in}{1.327154in}}{\pgfqpoint{9.587193in}{1.323588in}}%
\pgfpathcurveto{\pgfqpoint{9.590759in}{1.320021in}}{\pgfqpoint{9.595597in}{1.318018in}}{\pgfqpoint{9.600641in}{1.318018in}}%
\pgfpathclose%
\pgfusepath{fill}%
\end{pgfscope}%
\begin{pgfscope}%
\pgfpathrectangle{\pgfqpoint{6.572727in}{0.474100in}}{\pgfqpoint{4.227273in}{3.318700in}}%
\pgfusepath{clip}%
\pgfsetbuttcap%
\pgfsetroundjoin%
\definecolor{currentfill}{rgb}{0.127568,0.566949,0.550556}%
\pgfsetfillcolor{currentfill}%
\pgfsetfillopacity{0.700000}%
\pgfsetlinewidth{0.000000pt}%
\definecolor{currentstroke}{rgb}{0.000000,0.000000,0.000000}%
\pgfsetstrokecolor{currentstroke}%
\pgfsetstrokeopacity{0.700000}%
\pgfsetdash{}{0pt}%
\pgfpathmoveto{\pgfqpoint{7.764700in}{3.369982in}}%
\pgfpathcurveto{\pgfqpoint{7.769744in}{3.369982in}}{\pgfqpoint{7.774582in}{3.371986in}}{\pgfqpoint{7.778148in}{3.375552in}}%
\pgfpathcurveto{\pgfqpoint{7.781714in}{3.379119in}}{\pgfqpoint{7.783718in}{3.383956in}}{\pgfqpoint{7.783718in}{3.389000in}}%
\pgfpathcurveto{\pgfqpoint{7.783718in}{3.394044in}}{\pgfqpoint{7.781714in}{3.398882in}}{\pgfqpoint{7.778148in}{3.402448in}}%
\pgfpathcurveto{\pgfqpoint{7.774582in}{3.406014in}}{\pgfqpoint{7.769744in}{3.408018in}}{\pgfqpoint{7.764700in}{3.408018in}}%
\pgfpathcurveto{\pgfqpoint{7.759657in}{3.408018in}}{\pgfqpoint{7.754819in}{3.406014in}}{\pgfqpoint{7.751252in}{3.402448in}}%
\pgfpathcurveto{\pgfqpoint{7.747686in}{3.398882in}}{\pgfqpoint{7.745682in}{3.394044in}}{\pgfqpoint{7.745682in}{3.389000in}}%
\pgfpathcurveto{\pgfqpoint{7.745682in}{3.383956in}}{\pgfqpoint{7.747686in}{3.379119in}}{\pgfqpoint{7.751252in}{3.375552in}}%
\pgfpathcurveto{\pgfqpoint{7.754819in}{3.371986in}}{\pgfqpoint{7.759657in}{3.369982in}}{\pgfqpoint{7.764700in}{3.369982in}}%
\pgfpathclose%
\pgfusepath{fill}%
\end{pgfscope}%
\begin{pgfscope}%
\pgfpathrectangle{\pgfqpoint{6.572727in}{0.474100in}}{\pgfqpoint{4.227273in}{3.318700in}}%
\pgfusepath{clip}%
\pgfsetbuttcap%
\pgfsetroundjoin%
\definecolor{currentfill}{rgb}{0.993248,0.906157,0.143936}%
\pgfsetfillcolor{currentfill}%
\pgfsetfillopacity{0.700000}%
\pgfsetlinewidth{0.000000pt}%
\definecolor{currentstroke}{rgb}{0.000000,0.000000,0.000000}%
\pgfsetstrokecolor{currentstroke}%
\pgfsetstrokeopacity{0.700000}%
\pgfsetdash{}{0pt}%
\pgfpathmoveto{\pgfqpoint{10.262451in}{1.458038in}}%
\pgfpathcurveto{\pgfqpoint{10.267495in}{1.458038in}}{\pgfqpoint{10.272332in}{1.460042in}}{\pgfqpoint{10.275899in}{1.463608in}}%
\pgfpathcurveto{\pgfqpoint{10.279465in}{1.467175in}}{\pgfqpoint{10.281469in}{1.472013in}}{\pgfqpoint{10.281469in}{1.477056in}}%
\pgfpathcurveto{\pgfqpoint{10.281469in}{1.482100in}}{\pgfqpoint{10.279465in}{1.486938in}}{\pgfqpoint{10.275899in}{1.490504in}}%
\pgfpathcurveto{\pgfqpoint{10.272332in}{1.494070in}}{\pgfqpoint{10.267495in}{1.496074in}}{\pgfqpoint{10.262451in}{1.496074in}}%
\pgfpathcurveto{\pgfqpoint{10.257407in}{1.496074in}}{\pgfqpoint{10.252570in}{1.494070in}}{\pgfqpoint{10.249003in}{1.490504in}}%
\pgfpathcurveto{\pgfqpoint{10.245437in}{1.486938in}}{\pgfqpoint{10.243433in}{1.482100in}}{\pgfqpoint{10.243433in}{1.477056in}}%
\pgfpathcurveto{\pgfqpoint{10.243433in}{1.472013in}}{\pgfqpoint{10.245437in}{1.467175in}}{\pgfqpoint{10.249003in}{1.463608in}}%
\pgfpathcurveto{\pgfqpoint{10.252570in}{1.460042in}}{\pgfqpoint{10.257407in}{1.458038in}}{\pgfqpoint{10.262451in}{1.458038in}}%
\pgfpathclose%
\pgfusepath{fill}%
\end{pgfscope}%
\begin{pgfscope}%
\pgfpathrectangle{\pgfqpoint{6.572727in}{0.474100in}}{\pgfqpoint{4.227273in}{3.318700in}}%
\pgfusepath{clip}%
\pgfsetbuttcap%
\pgfsetroundjoin%
\definecolor{currentfill}{rgb}{0.993248,0.906157,0.143936}%
\pgfsetfillcolor{currentfill}%
\pgfsetfillopacity{0.700000}%
\pgfsetlinewidth{0.000000pt}%
\definecolor{currentstroke}{rgb}{0.000000,0.000000,0.000000}%
\pgfsetstrokecolor{currentstroke}%
\pgfsetstrokeopacity{0.700000}%
\pgfsetdash{}{0pt}%
\pgfpathmoveto{\pgfqpoint{9.255775in}{1.308261in}}%
\pgfpathcurveto{\pgfqpoint{9.260819in}{1.308261in}}{\pgfqpoint{9.265656in}{1.310265in}}{\pgfqpoint{9.269223in}{1.313831in}}%
\pgfpathcurveto{\pgfqpoint{9.272789in}{1.317397in}}{\pgfqpoint{9.274793in}{1.322235in}}{\pgfqpoint{9.274793in}{1.327279in}}%
\pgfpathcurveto{\pgfqpoint{9.274793in}{1.332323in}}{\pgfqpoint{9.272789in}{1.337160in}}{\pgfqpoint{9.269223in}{1.340727in}}%
\pgfpathcurveto{\pgfqpoint{9.265656in}{1.344293in}}{\pgfqpoint{9.260819in}{1.346297in}}{\pgfqpoint{9.255775in}{1.346297in}}%
\pgfpathcurveto{\pgfqpoint{9.250731in}{1.346297in}}{\pgfqpoint{9.245894in}{1.344293in}}{\pgfqpoint{9.242327in}{1.340727in}}%
\pgfpathcurveto{\pgfqpoint{9.238761in}{1.337160in}}{\pgfqpoint{9.236757in}{1.332323in}}{\pgfqpoint{9.236757in}{1.327279in}}%
\pgfpathcurveto{\pgfqpoint{9.236757in}{1.322235in}}{\pgfqpoint{9.238761in}{1.317397in}}{\pgfqpoint{9.242327in}{1.313831in}}%
\pgfpathcurveto{\pgfqpoint{9.245894in}{1.310265in}}{\pgfqpoint{9.250731in}{1.308261in}}{\pgfqpoint{9.255775in}{1.308261in}}%
\pgfpathclose%
\pgfusepath{fill}%
\end{pgfscope}%
\begin{pgfscope}%
\pgfpathrectangle{\pgfqpoint{6.572727in}{0.474100in}}{\pgfqpoint{4.227273in}{3.318700in}}%
\pgfusepath{clip}%
\pgfsetbuttcap%
\pgfsetroundjoin%
\definecolor{currentfill}{rgb}{0.127568,0.566949,0.550556}%
\pgfsetfillcolor{currentfill}%
\pgfsetfillopacity{0.700000}%
\pgfsetlinewidth{0.000000pt}%
\definecolor{currentstroke}{rgb}{0.000000,0.000000,0.000000}%
\pgfsetstrokecolor{currentstroke}%
\pgfsetstrokeopacity{0.700000}%
\pgfsetdash{}{0pt}%
\pgfpathmoveto{\pgfqpoint{7.691415in}{1.439742in}}%
\pgfpathcurveto{\pgfqpoint{7.696459in}{1.439742in}}{\pgfqpoint{7.701297in}{1.441746in}}{\pgfqpoint{7.704863in}{1.445312in}}%
\pgfpathcurveto{\pgfqpoint{7.708430in}{1.448879in}}{\pgfqpoint{7.710433in}{1.453716in}}{\pgfqpoint{7.710433in}{1.458760in}}%
\pgfpathcurveto{\pgfqpoint{7.710433in}{1.463804in}}{\pgfqpoint{7.708430in}{1.468641in}}{\pgfqpoint{7.704863in}{1.472208in}}%
\pgfpathcurveto{\pgfqpoint{7.701297in}{1.475774in}}{\pgfqpoint{7.696459in}{1.477778in}}{\pgfqpoint{7.691415in}{1.477778in}}%
\pgfpathcurveto{\pgfqpoint{7.686372in}{1.477778in}}{\pgfqpoint{7.681534in}{1.475774in}}{\pgfqpoint{7.677967in}{1.472208in}}%
\pgfpathcurveto{\pgfqpoint{7.674401in}{1.468641in}}{\pgfqpoint{7.672397in}{1.463804in}}{\pgfqpoint{7.672397in}{1.458760in}}%
\pgfpathcurveto{\pgfqpoint{7.672397in}{1.453716in}}{\pgfqpoint{7.674401in}{1.448879in}}{\pgfqpoint{7.677967in}{1.445312in}}%
\pgfpathcurveto{\pgfqpoint{7.681534in}{1.441746in}}{\pgfqpoint{7.686372in}{1.439742in}}{\pgfqpoint{7.691415in}{1.439742in}}%
\pgfpathclose%
\pgfusepath{fill}%
\end{pgfscope}%
\begin{pgfscope}%
\pgfpathrectangle{\pgfqpoint{6.572727in}{0.474100in}}{\pgfqpoint{4.227273in}{3.318700in}}%
\pgfusepath{clip}%
\pgfsetbuttcap%
\pgfsetroundjoin%
\definecolor{currentfill}{rgb}{0.127568,0.566949,0.550556}%
\pgfsetfillcolor{currentfill}%
\pgfsetfillopacity{0.700000}%
\pgfsetlinewidth{0.000000pt}%
\definecolor{currentstroke}{rgb}{0.000000,0.000000,0.000000}%
\pgfsetstrokecolor{currentstroke}%
\pgfsetstrokeopacity{0.700000}%
\pgfsetdash{}{0pt}%
\pgfpathmoveto{\pgfqpoint{7.949581in}{2.582726in}}%
\pgfpathcurveto{\pgfqpoint{7.954624in}{2.582726in}}{\pgfqpoint{7.959462in}{2.584729in}}{\pgfqpoint{7.963029in}{2.588296in}}%
\pgfpathcurveto{\pgfqpoint{7.966595in}{2.591862in}}{\pgfqpoint{7.968599in}{2.596700in}}{\pgfqpoint{7.968599in}{2.601744in}}%
\pgfpathcurveto{\pgfqpoint{7.968599in}{2.606787in}}{\pgfqpoint{7.966595in}{2.611625in}}{\pgfqpoint{7.963029in}{2.615192in}}%
\pgfpathcurveto{\pgfqpoint{7.959462in}{2.618758in}}{\pgfqpoint{7.954624in}{2.620762in}}{\pgfqpoint{7.949581in}{2.620762in}}%
\pgfpathcurveto{\pgfqpoint{7.944537in}{2.620762in}}{\pgfqpoint{7.939699in}{2.618758in}}{\pgfqpoint{7.936133in}{2.615192in}}%
\pgfpathcurveto{\pgfqpoint{7.932566in}{2.611625in}}{\pgfqpoint{7.930563in}{2.606787in}}{\pgfqpoint{7.930563in}{2.601744in}}%
\pgfpathcurveto{\pgfqpoint{7.930563in}{2.596700in}}{\pgfqpoint{7.932566in}{2.591862in}}{\pgfqpoint{7.936133in}{2.588296in}}%
\pgfpathcurveto{\pgfqpoint{7.939699in}{2.584729in}}{\pgfqpoint{7.944537in}{2.582726in}}{\pgfqpoint{7.949581in}{2.582726in}}%
\pgfpathclose%
\pgfusepath{fill}%
\end{pgfscope}%
\begin{pgfscope}%
\pgfpathrectangle{\pgfqpoint{6.572727in}{0.474100in}}{\pgfqpoint{4.227273in}{3.318700in}}%
\pgfusepath{clip}%
\pgfsetbuttcap%
\pgfsetroundjoin%
\definecolor{currentfill}{rgb}{0.127568,0.566949,0.550556}%
\pgfsetfillcolor{currentfill}%
\pgfsetfillopacity{0.700000}%
\pgfsetlinewidth{0.000000pt}%
\definecolor{currentstroke}{rgb}{0.000000,0.000000,0.000000}%
\pgfsetstrokecolor{currentstroke}%
\pgfsetstrokeopacity{0.700000}%
\pgfsetdash{}{0pt}%
\pgfpathmoveto{\pgfqpoint{7.651143in}{1.871028in}}%
\pgfpathcurveto{\pgfqpoint{7.656187in}{1.871028in}}{\pgfqpoint{7.661025in}{1.873031in}}{\pgfqpoint{7.664591in}{1.876598in}}%
\pgfpathcurveto{\pgfqpoint{7.668157in}{1.880164in}}{\pgfqpoint{7.670161in}{1.885002in}}{\pgfqpoint{7.670161in}{1.890046in}}%
\pgfpathcurveto{\pgfqpoint{7.670161in}{1.895089in}}{\pgfqpoint{7.668157in}{1.899927in}}{\pgfqpoint{7.664591in}{1.903494in}}%
\pgfpathcurveto{\pgfqpoint{7.661025in}{1.907060in}}{\pgfqpoint{7.656187in}{1.909064in}}{\pgfqpoint{7.651143in}{1.909064in}}%
\pgfpathcurveto{\pgfqpoint{7.646099in}{1.909064in}}{\pgfqpoint{7.641262in}{1.907060in}}{\pgfqpoint{7.637695in}{1.903494in}}%
\pgfpathcurveto{\pgfqpoint{7.634129in}{1.899927in}}{\pgfqpoint{7.632125in}{1.895089in}}{\pgfqpoint{7.632125in}{1.890046in}}%
\pgfpathcurveto{\pgfqpoint{7.632125in}{1.885002in}}{\pgfqpoint{7.634129in}{1.880164in}}{\pgfqpoint{7.637695in}{1.876598in}}%
\pgfpathcurveto{\pgfqpoint{7.641262in}{1.873031in}}{\pgfqpoint{7.646099in}{1.871028in}}{\pgfqpoint{7.651143in}{1.871028in}}%
\pgfpathclose%
\pgfusepath{fill}%
\end{pgfscope}%
\begin{pgfscope}%
\pgfpathrectangle{\pgfqpoint{6.572727in}{0.474100in}}{\pgfqpoint{4.227273in}{3.318700in}}%
\pgfusepath{clip}%
\pgfsetbuttcap%
\pgfsetroundjoin%
\definecolor{currentfill}{rgb}{0.127568,0.566949,0.550556}%
\pgfsetfillcolor{currentfill}%
\pgfsetfillopacity{0.700000}%
\pgfsetlinewidth{0.000000pt}%
\definecolor{currentstroke}{rgb}{0.000000,0.000000,0.000000}%
\pgfsetstrokecolor{currentstroke}%
\pgfsetstrokeopacity{0.700000}%
\pgfsetdash{}{0pt}%
\pgfpathmoveto{\pgfqpoint{7.946003in}{2.235074in}}%
\pgfpathcurveto{\pgfqpoint{7.951046in}{2.235074in}}{\pgfqpoint{7.955884in}{2.237078in}}{\pgfqpoint{7.959451in}{2.240644in}}%
\pgfpathcurveto{\pgfqpoint{7.963017in}{2.244211in}}{\pgfqpoint{7.965021in}{2.249049in}}{\pgfqpoint{7.965021in}{2.254092in}}%
\pgfpathcurveto{\pgfqpoint{7.965021in}{2.259136in}}{\pgfqpoint{7.963017in}{2.263974in}}{\pgfqpoint{7.959451in}{2.267540in}}%
\pgfpathcurveto{\pgfqpoint{7.955884in}{2.271107in}}{\pgfqpoint{7.951046in}{2.273110in}}{\pgfqpoint{7.946003in}{2.273110in}}%
\pgfpathcurveto{\pgfqpoint{7.940959in}{2.273110in}}{\pgfqpoint{7.936121in}{2.271107in}}{\pgfqpoint{7.932555in}{2.267540in}}%
\pgfpathcurveto{\pgfqpoint{7.928988in}{2.263974in}}{\pgfqpoint{7.926985in}{2.259136in}}{\pgfqpoint{7.926985in}{2.254092in}}%
\pgfpathcurveto{\pgfqpoint{7.926985in}{2.249049in}}{\pgfqpoint{7.928988in}{2.244211in}}{\pgfqpoint{7.932555in}{2.240644in}}%
\pgfpathcurveto{\pgfqpoint{7.936121in}{2.237078in}}{\pgfqpoint{7.940959in}{2.235074in}}{\pgfqpoint{7.946003in}{2.235074in}}%
\pgfpathclose%
\pgfusepath{fill}%
\end{pgfscope}%
\begin{pgfscope}%
\pgfpathrectangle{\pgfqpoint{6.572727in}{0.474100in}}{\pgfqpoint{4.227273in}{3.318700in}}%
\pgfusepath{clip}%
\pgfsetbuttcap%
\pgfsetroundjoin%
\definecolor{currentfill}{rgb}{0.127568,0.566949,0.550556}%
\pgfsetfillcolor{currentfill}%
\pgfsetfillopacity{0.700000}%
\pgfsetlinewidth{0.000000pt}%
\definecolor{currentstroke}{rgb}{0.000000,0.000000,0.000000}%
\pgfsetstrokecolor{currentstroke}%
\pgfsetstrokeopacity{0.700000}%
\pgfsetdash{}{0pt}%
\pgfpathmoveto{\pgfqpoint{7.762822in}{1.076009in}}%
\pgfpathcurveto{\pgfqpoint{7.767866in}{1.076009in}}{\pgfqpoint{7.772703in}{1.078013in}}{\pgfqpoint{7.776270in}{1.081579in}}%
\pgfpathcurveto{\pgfqpoint{7.779836in}{1.085146in}}{\pgfqpoint{7.781840in}{1.089983in}}{\pgfqpoint{7.781840in}{1.095027in}}%
\pgfpathcurveto{\pgfqpoint{7.781840in}{1.100071in}}{\pgfqpoint{7.779836in}{1.104909in}}{\pgfqpoint{7.776270in}{1.108475in}}%
\pgfpathcurveto{\pgfqpoint{7.772703in}{1.112041in}}{\pgfqpoint{7.767866in}{1.114045in}}{\pgfqpoint{7.762822in}{1.114045in}}%
\pgfpathcurveto{\pgfqpoint{7.757778in}{1.114045in}}{\pgfqpoint{7.752941in}{1.112041in}}{\pgfqpoint{7.749374in}{1.108475in}}%
\pgfpathcurveto{\pgfqpoint{7.745808in}{1.104909in}}{\pgfqpoint{7.743804in}{1.100071in}}{\pgfqpoint{7.743804in}{1.095027in}}%
\pgfpathcurveto{\pgfqpoint{7.743804in}{1.089983in}}{\pgfqpoint{7.745808in}{1.085146in}}{\pgfqpoint{7.749374in}{1.081579in}}%
\pgfpathcurveto{\pgfqpoint{7.752941in}{1.078013in}}{\pgfqpoint{7.757778in}{1.076009in}}{\pgfqpoint{7.762822in}{1.076009in}}%
\pgfpathclose%
\pgfusepath{fill}%
\end{pgfscope}%
\begin{pgfscope}%
\pgfpathrectangle{\pgfqpoint{6.572727in}{0.474100in}}{\pgfqpoint{4.227273in}{3.318700in}}%
\pgfusepath{clip}%
\pgfsetbuttcap%
\pgfsetroundjoin%
\definecolor{currentfill}{rgb}{0.127568,0.566949,0.550556}%
\pgfsetfillcolor{currentfill}%
\pgfsetfillopacity{0.700000}%
\pgfsetlinewidth{0.000000pt}%
\definecolor{currentstroke}{rgb}{0.000000,0.000000,0.000000}%
\pgfsetstrokecolor{currentstroke}%
\pgfsetstrokeopacity{0.700000}%
\pgfsetdash{}{0pt}%
\pgfpathmoveto{\pgfqpoint{8.124940in}{1.511999in}}%
\pgfpathcurveto{\pgfqpoint{8.129983in}{1.511999in}}{\pgfqpoint{8.134821in}{1.514003in}}{\pgfqpoint{8.138387in}{1.517569in}}%
\pgfpathcurveto{\pgfqpoint{8.141954in}{1.521136in}}{\pgfqpoint{8.143958in}{1.525973in}}{\pgfqpoint{8.143958in}{1.531017in}}%
\pgfpathcurveto{\pgfqpoint{8.143958in}{1.536061in}}{\pgfqpoint{8.141954in}{1.540899in}}{\pgfqpoint{8.138387in}{1.544465in}}%
\pgfpathcurveto{\pgfqpoint{8.134821in}{1.548031in}}{\pgfqpoint{8.129983in}{1.550035in}}{\pgfqpoint{8.124940in}{1.550035in}}%
\pgfpathcurveto{\pgfqpoint{8.119896in}{1.550035in}}{\pgfqpoint{8.115058in}{1.548031in}}{\pgfqpoint{8.111492in}{1.544465in}}%
\pgfpathcurveto{\pgfqpoint{8.107925in}{1.540899in}}{\pgfqpoint{8.105921in}{1.536061in}}{\pgfqpoint{8.105921in}{1.531017in}}%
\pgfpathcurveto{\pgfqpoint{8.105921in}{1.525973in}}{\pgfqpoint{8.107925in}{1.521136in}}{\pgfqpoint{8.111492in}{1.517569in}}%
\pgfpathcurveto{\pgfqpoint{8.115058in}{1.514003in}}{\pgfqpoint{8.119896in}{1.511999in}}{\pgfqpoint{8.124940in}{1.511999in}}%
\pgfpathclose%
\pgfusepath{fill}%
\end{pgfscope}%
\begin{pgfscope}%
\pgfpathrectangle{\pgfqpoint{6.572727in}{0.474100in}}{\pgfqpoint{4.227273in}{3.318700in}}%
\pgfusepath{clip}%
\pgfsetbuttcap%
\pgfsetroundjoin%
\definecolor{currentfill}{rgb}{0.993248,0.906157,0.143936}%
\pgfsetfillcolor{currentfill}%
\pgfsetfillopacity{0.700000}%
\pgfsetlinewidth{0.000000pt}%
\definecolor{currentstroke}{rgb}{0.000000,0.000000,0.000000}%
\pgfsetstrokecolor{currentstroke}%
\pgfsetstrokeopacity{0.700000}%
\pgfsetdash{}{0pt}%
\pgfpathmoveto{\pgfqpoint{10.010201in}{1.875300in}}%
\pgfpathcurveto{\pgfqpoint{10.015244in}{1.875300in}}{\pgfqpoint{10.020082in}{1.877304in}}{\pgfqpoint{10.023649in}{1.880870in}}%
\pgfpathcurveto{\pgfqpoint{10.027215in}{1.884437in}}{\pgfqpoint{10.029219in}{1.889275in}}{\pgfqpoint{10.029219in}{1.894318in}}%
\pgfpathcurveto{\pgfqpoint{10.029219in}{1.899362in}}{\pgfqpoint{10.027215in}{1.904200in}}{\pgfqpoint{10.023649in}{1.907766in}}%
\pgfpathcurveto{\pgfqpoint{10.020082in}{1.911332in}}{\pgfqpoint{10.015244in}{1.913336in}}{\pgfqpoint{10.010201in}{1.913336in}}%
\pgfpathcurveto{\pgfqpoint{10.005157in}{1.913336in}}{\pgfqpoint{10.000319in}{1.911332in}}{\pgfqpoint{9.996753in}{1.907766in}}%
\pgfpathcurveto{\pgfqpoint{9.993186in}{1.904200in}}{\pgfqpoint{9.991183in}{1.899362in}}{\pgfqpoint{9.991183in}{1.894318in}}%
\pgfpathcurveto{\pgfqpoint{9.991183in}{1.889275in}}{\pgfqpoint{9.993186in}{1.884437in}}{\pgfqpoint{9.996753in}{1.880870in}}%
\pgfpathcurveto{\pgfqpoint{10.000319in}{1.877304in}}{\pgfqpoint{10.005157in}{1.875300in}}{\pgfqpoint{10.010201in}{1.875300in}}%
\pgfpathclose%
\pgfusepath{fill}%
\end{pgfscope}%
\begin{pgfscope}%
\pgfpathrectangle{\pgfqpoint{6.572727in}{0.474100in}}{\pgfqpoint{4.227273in}{3.318700in}}%
\pgfusepath{clip}%
\pgfsetbuttcap%
\pgfsetroundjoin%
\definecolor{currentfill}{rgb}{0.127568,0.566949,0.550556}%
\pgfsetfillcolor{currentfill}%
\pgfsetfillopacity{0.700000}%
\pgfsetlinewidth{0.000000pt}%
\definecolor{currentstroke}{rgb}{0.000000,0.000000,0.000000}%
\pgfsetstrokecolor{currentstroke}%
\pgfsetstrokeopacity{0.700000}%
\pgfsetdash{}{0pt}%
\pgfpathmoveto{\pgfqpoint{8.010571in}{2.680736in}}%
\pgfpathcurveto{\pgfqpoint{8.015614in}{2.680736in}}{\pgfqpoint{8.020452in}{2.682740in}}{\pgfqpoint{8.024018in}{2.686307in}}%
\pgfpathcurveto{\pgfqpoint{8.027585in}{2.689873in}}{\pgfqpoint{8.029589in}{2.694711in}}{\pgfqpoint{8.029589in}{2.699754in}}%
\pgfpathcurveto{\pgfqpoint{8.029589in}{2.704798in}}{\pgfqpoint{8.027585in}{2.709636in}}{\pgfqpoint{8.024018in}{2.713202in}}%
\pgfpathcurveto{\pgfqpoint{8.020452in}{2.716769in}}{\pgfqpoint{8.015614in}{2.718773in}}{\pgfqpoint{8.010571in}{2.718773in}}%
\pgfpathcurveto{\pgfqpoint{8.005527in}{2.718773in}}{\pgfqpoint{8.000689in}{2.716769in}}{\pgfqpoint{7.997123in}{2.713202in}}%
\pgfpathcurveto{\pgfqpoint{7.993556in}{2.709636in}}{\pgfqpoint{7.991552in}{2.704798in}}{\pgfqpoint{7.991552in}{2.699754in}}%
\pgfpathcurveto{\pgfqpoint{7.991552in}{2.694711in}}{\pgfqpoint{7.993556in}{2.689873in}}{\pgfqpoint{7.997123in}{2.686307in}}%
\pgfpathcurveto{\pgfqpoint{8.000689in}{2.682740in}}{\pgfqpoint{8.005527in}{2.680736in}}{\pgfqpoint{8.010571in}{2.680736in}}%
\pgfpathclose%
\pgfusepath{fill}%
\end{pgfscope}%
\begin{pgfscope}%
\pgfpathrectangle{\pgfqpoint{6.572727in}{0.474100in}}{\pgfqpoint{4.227273in}{3.318700in}}%
\pgfusepath{clip}%
\pgfsetbuttcap%
\pgfsetroundjoin%
\definecolor{currentfill}{rgb}{0.993248,0.906157,0.143936}%
\pgfsetfillcolor{currentfill}%
\pgfsetfillopacity{0.700000}%
\pgfsetlinewidth{0.000000pt}%
\definecolor{currentstroke}{rgb}{0.000000,0.000000,0.000000}%
\pgfsetstrokecolor{currentstroke}%
\pgfsetstrokeopacity{0.700000}%
\pgfsetdash{}{0pt}%
\pgfpathmoveto{\pgfqpoint{10.233045in}{1.081419in}}%
\pgfpathcurveto{\pgfqpoint{10.238089in}{1.081419in}}{\pgfqpoint{10.242927in}{1.083422in}}{\pgfqpoint{10.246493in}{1.086989in}}%
\pgfpathcurveto{\pgfqpoint{10.250060in}{1.090555in}}{\pgfqpoint{10.252064in}{1.095393in}}{\pgfqpoint{10.252064in}{1.100437in}}%
\pgfpathcurveto{\pgfqpoint{10.252064in}{1.105480in}}{\pgfqpoint{10.250060in}{1.110318in}}{\pgfqpoint{10.246493in}{1.113885in}}%
\pgfpathcurveto{\pgfqpoint{10.242927in}{1.117451in}}{\pgfqpoint{10.238089in}{1.119455in}}{\pgfqpoint{10.233045in}{1.119455in}}%
\pgfpathcurveto{\pgfqpoint{10.228002in}{1.119455in}}{\pgfqpoint{10.223164in}{1.117451in}}{\pgfqpoint{10.219598in}{1.113885in}}%
\pgfpathcurveto{\pgfqpoint{10.216031in}{1.110318in}}{\pgfqpoint{10.214027in}{1.105480in}}{\pgfqpoint{10.214027in}{1.100437in}}%
\pgfpathcurveto{\pgfqpoint{10.214027in}{1.095393in}}{\pgfqpoint{10.216031in}{1.090555in}}{\pgfqpoint{10.219598in}{1.086989in}}%
\pgfpathcurveto{\pgfqpoint{10.223164in}{1.083422in}}{\pgfqpoint{10.228002in}{1.081419in}}{\pgfqpoint{10.233045in}{1.081419in}}%
\pgfpathclose%
\pgfusepath{fill}%
\end{pgfscope}%
\begin{pgfscope}%
\pgfpathrectangle{\pgfqpoint{6.572727in}{0.474100in}}{\pgfqpoint{4.227273in}{3.318700in}}%
\pgfusepath{clip}%
\pgfsetbuttcap%
\pgfsetroundjoin%
\definecolor{currentfill}{rgb}{0.127568,0.566949,0.550556}%
\pgfsetfillcolor{currentfill}%
\pgfsetfillopacity{0.700000}%
\pgfsetlinewidth{0.000000pt}%
\definecolor{currentstroke}{rgb}{0.000000,0.000000,0.000000}%
\pgfsetstrokecolor{currentstroke}%
\pgfsetstrokeopacity{0.700000}%
\pgfsetdash{}{0pt}%
\pgfpathmoveto{\pgfqpoint{7.491528in}{1.482655in}}%
\pgfpathcurveto{\pgfqpoint{7.496572in}{1.482655in}}{\pgfqpoint{7.501409in}{1.484659in}}{\pgfqpoint{7.504976in}{1.488225in}}%
\pgfpathcurveto{\pgfqpoint{7.508542in}{1.491791in}}{\pgfqpoint{7.510546in}{1.496629in}}{\pgfqpoint{7.510546in}{1.501673in}}%
\pgfpathcurveto{\pgfqpoint{7.510546in}{1.506716in}}{\pgfqpoint{7.508542in}{1.511554in}}{\pgfqpoint{7.504976in}{1.515121in}}%
\pgfpathcurveto{\pgfqpoint{7.501409in}{1.518687in}}{\pgfqpoint{7.496572in}{1.520691in}}{\pgfqpoint{7.491528in}{1.520691in}}%
\pgfpathcurveto{\pgfqpoint{7.486484in}{1.520691in}}{\pgfqpoint{7.481646in}{1.518687in}}{\pgfqpoint{7.478080in}{1.515121in}}%
\pgfpathcurveto{\pgfqpoint{7.474514in}{1.511554in}}{\pgfqpoint{7.472510in}{1.506716in}}{\pgfqpoint{7.472510in}{1.501673in}}%
\pgfpathcurveto{\pgfqpoint{7.472510in}{1.496629in}}{\pgfqpoint{7.474514in}{1.491791in}}{\pgfqpoint{7.478080in}{1.488225in}}%
\pgfpathcurveto{\pgfqpoint{7.481646in}{1.484659in}}{\pgfqpoint{7.486484in}{1.482655in}}{\pgfqpoint{7.491528in}{1.482655in}}%
\pgfpathclose%
\pgfusepath{fill}%
\end{pgfscope}%
\begin{pgfscope}%
\pgfpathrectangle{\pgfqpoint{6.572727in}{0.474100in}}{\pgfqpoint{4.227273in}{3.318700in}}%
\pgfusepath{clip}%
\pgfsetbuttcap%
\pgfsetroundjoin%
\definecolor{currentfill}{rgb}{0.127568,0.566949,0.550556}%
\pgfsetfillcolor{currentfill}%
\pgfsetfillopacity{0.700000}%
\pgfsetlinewidth{0.000000pt}%
\definecolor{currentstroke}{rgb}{0.000000,0.000000,0.000000}%
\pgfsetstrokecolor{currentstroke}%
\pgfsetstrokeopacity{0.700000}%
\pgfsetdash{}{0pt}%
\pgfpathmoveto{\pgfqpoint{8.031887in}{1.319203in}}%
\pgfpathcurveto{\pgfqpoint{8.036930in}{1.319203in}}{\pgfqpoint{8.041768in}{1.321207in}}{\pgfqpoint{8.045334in}{1.324773in}}%
\pgfpathcurveto{\pgfqpoint{8.048901in}{1.328340in}}{\pgfqpoint{8.050905in}{1.333177in}}{\pgfqpoint{8.050905in}{1.338221in}}%
\pgfpathcurveto{\pgfqpoint{8.050905in}{1.343265in}}{\pgfqpoint{8.048901in}{1.348103in}}{\pgfqpoint{8.045334in}{1.351669in}}%
\pgfpathcurveto{\pgfqpoint{8.041768in}{1.355235in}}{\pgfqpoint{8.036930in}{1.357239in}}{\pgfqpoint{8.031887in}{1.357239in}}%
\pgfpathcurveto{\pgfqpoint{8.026843in}{1.357239in}}{\pgfqpoint{8.022005in}{1.355235in}}{\pgfqpoint{8.018439in}{1.351669in}}%
\pgfpathcurveto{\pgfqpoint{8.014872in}{1.348103in}}{\pgfqpoint{8.012868in}{1.343265in}}{\pgfqpoint{8.012868in}{1.338221in}}%
\pgfpathcurveto{\pgfqpoint{8.012868in}{1.333177in}}{\pgfqpoint{8.014872in}{1.328340in}}{\pgfqpoint{8.018439in}{1.324773in}}%
\pgfpathcurveto{\pgfqpoint{8.022005in}{1.321207in}}{\pgfqpoint{8.026843in}{1.319203in}}{\pgfqpoint{8.031887in}{1.319203in}}%
\pgfpathclose%
\pgfusepath{fill}%
\end{pgfscope}%
\begin{pgfscope}%
\pgfpathrectangle{\pgfqpoint{6.572727in}{0.474100in}}{\pgfqpoint{4.227273in}{3.318700in}}%
\pgfusepath{clip}%
\pgfsetbuttcap%
\pgfsetroundjoin%
\definecolor{currentfill}{rgb}{0.127568,0.566949,0.550556}%
\pgfsetfillcolor{currentfill}%
\pgfsetfillopacity{0.700000}%
\pgfsetlinewidth{0.000000pt}%
\definecolor{currentstroke}{rgb}{0.000000,0.000000,0.000000}%
\pgfsetstrokecolor{currentstroke}%
\pgfsetstrokeopacity{0.700000}%
\pgfsetdash{}{0pt}%
\pgfpathmoveto{\pgfqpoint{8.073258in}{2.338818in}}%
\pgfpathcurveto{\pgfqpoint{8.078302in}{2.338818in}}{\pgfqpoint{8.083139in}{2.340822in}}{\pgfqpoint{8.086706in}{2.344389in}}%
\pgfpathcurveto{\pgfqpoint{8.090272in}{2.347955in}}{\pgfqpoint{8.092276in}{2.352793in}}{\pgfqpoint{8.092276in}{2.357837in}}%
\pgfpathcurveto{\pgfqpoint{8.092276in}{2.362880in}}{\pgfqpoint{8.090272in}{2.367718in}}{\pgfqpoint{8.086706in}{2.371284in}}%
\pgfpathcurveto{\pgfqpoint{8.083139in}{2.374851in}}{\pgfqpoint{8.078302in}{2.376855in}}{\pgfqpoint{8.073258in}{2.376855in}}%
\pgfpathcurveto{\pgfqpoint{8.068214in}{2.376855in}}{\pgfqpoint{8.063376in}{2.374851in}}{\pgfqpoint{8.059810in}{2.371284in}}%
\pgfpathcurveto{\pgfqpoint{8.056244in}{2.367718in}}{\pgfqpoint{8.054240in}{2.362880in}}{\pgfqpoint{8.054240in}{2.357837in}}%
\pgfpathcurveto{\pgfqpoint{8.054240in}{2.352793in}}{\pgfqpoint{8.056244in}{2.347955in}}{\pgfqpoint{8.059810in}{2.344389in}}%
\pgfpathcurveto{\pgfqpoint{8.063376in}{2.340822in}}{\pgfqpoint{8.068214in}{2.338818in}}{\pgfqpoint{8.073258in}{2.338818in}}%
\pgfpathclose%
\pgfusepath{fill}%
\end{pgfscope}%
\begin{pgfscope}%
\pgfpathrectangle{\pgfqpoint{6.572727in}{0.474100in}}{\pgfqpoint{4.227273in}{3.318700in}}%
\pgfusepath{clip}%
\pgfsetbuttcap%
\pgfsetroundjoin%
\definecolor{currentfill}{rgb}{0.127568,0.566949,0.550556}%
\pgfsetfillcolor{currentfill}%
\pgfsetfillopacity{0.700000}%
\pgfsetlinewidth{0.000000pt}%
\definecolor{currentstroke}{rgb}{0.000000,0.000000,0.000000}%
\pgfsetstrokecolor{currentstroke}%
\pgfsetstrokeopacity{0.700000}%
\pgfsetdash{}{0pt}%
\pgfpathmoveto{\pgfqpoint{7.636168in}{1.266710in}}%
\pgfpathcurveto{\pgfqpoint{7.641212in}{1.266710in}}{\pgfqpoint{7.646049in}{1.268714in}}{\pgfqpoint{7.649616in}{1.272280in}}%
\pgfpathcurveto{\pgfqpoint{7.653182in}{1.275847in}}{\pgfqpoint{7.655186in}{1.280685in}}{\pgfqpoint{7.655186in}{1.285728in}}%
\pgfpathcurveto{\pgfqpoint{7.655186in}{1.290772in}}{\pgfqpoint{7.653182in}{1.295610in}}{\pgfqpoint{7.649616in}{1.299176in}}%
\pgfpathcurveto{\pgfqpoint{7.646049in}{1.302743in}}{\pgfqpoint{7.641212in}{1.304746in}}{\pgfqpoint{7.636168in}{1.304746in}}%
\pgfpathcurveto{\pgfqpoint{7.631124in}{1.304746in}}{\pgfqpoint{7.626286in}{1.302743in}}{\pgfqpoint{7.622720in}{1.299176in}}%
\pgfpathcurveto{\pgfqpoint{7.619154in}{1.295610in}}{\pgfqpoint{7.617150in}{1.290772in}}{\pgfqpoint{7.617150in}{1.285728in}}%
\pgfpathcurveto{\pgfqpoint{7.617150in}{1.280685in}}{\pgfqpoint{7.619154in}{1.275847in}}{\pgfqpoint{7.622720in}{1.272280in}}%
\pgfpathcurveto{\pgfqpoint{7.626286in}{1.268714in}}{\pgfqpoint{7.631124in}{1.266710in}}{\pgfqpoint{7.636168in}{1.266710in}}%
\pgfpathclose%
\pgfusepath{fill}%
\end{pgfscope}%
\begin{pgfscope}%
\pgfpathrectangle{\pgfqpoint{6.572727in}{0.474100in}}{\pgfqpoint{4.227273in}{3.318700in}}%
\pgfusepath{clip}%
\pgfsetbuttcap%
\pgfsetroundjoin%
\definecolor{currentfill}{rgb}{0.993248,0.906157,0.143936}%
\pgfsetfillcolor{currentfill}%
\pgfsetfillopacity{0.700000}%
\pgfsetlinewidth{0.000000pt}%
\definecolor{currentstroke}{rgb}{0.000000,0.000000,0.000000}%
\pgfsetstrokecolor{currentstroke}%
\pgfsetstrokeopacity{0.700000}%
\pgfsetdash{}{0pt}%
\pgfpathmoveto{\pgfqpoint{9.859125in}{1.753705in}}%
\pgfpathcurveto{\pgfqpoint{9.864169in}{1.753705in}}{\pgfqpoint{9.869007in}{1.755709in}}{\pgfqpoint{9.872573in}{1.759275in}}%
\pgfpathcurveto{\pgfqpoint{9.876139in}{1.762842in}}{\pgfqpoint{9.878143in}{1.767680in}}{\pgfqpoint{9.878143in}{1.772723in}}%
\pgfpathcurveto{\pgfqpoint{9.878143in}{1.777767in}}{\pgfqpoint{9.876139in}{1.782605in}}{\pgfqpoint{9.872573in}{1.786171in}}%
\pgfpathcurveto{\pgfqpoint{9.869007in}{1.789738in}}{\pgfqpoint{9.864169in}{1.791741in}}{\pgfqpoint{9.859125in}{1.791741in}}%
\pgfpathcurveto{\pgfqpoint{9.854082in}{1.791741in}}{\pgfqpoint{9.849244in}{1.789738in}}{\pgfqpoint{9.845677in}{1.786171in}}%
\pgfpathcurveto{\pgfqpoint{9.842111in}{1.782605in}}{\pgfqpoint{9.840107in}{1.777767in}}{\pgfqpoint{9.840107in}{1.772723in}}%
\pgfpathcurveto{\pgfqpoint{9.840107in}{1.767680in}}{\pgfqpoint{9.842111in}{1.762842in}}{\pgfqpoint{9.845677in}{1.759275in}}%
\pgfpathcurveto{\pgfqpoint{9.849244in}{1.755709in}}{\pgfqpoint{9.854082in}{1.753705in}}{\pgfqpoint{9.859125in}{1.753705in}}%
\pgfpathclose%
\pgfusepath{fill}%
\end{pgfscope}%
\begin{pgfscope}%
\pgfpathrectangle{\pgfqpoint{6.572727in}{0.474100in}}{\pgfqpoint{4.227273in}{3.318700in}}%
\pgfusepath{clip}%
\pgfsetbuttcap%
\pgfsetroundjoin%
\definecolor{currentfill}{rgb}{0.127568,0.566949,0.550556}%
\pgfsetfillcolor{currentfill}%
\pgfsetfillopacity{0.700000}%
\pgfsetlinewidth{0.000000pt}%
\definecolor{currentstroke}{rgb}{0.000000,0.000000,0.000000}%
\pgfsetstrokecolor{currentstroke}%
\pgfsetstrokeopacity{0.700000}%
\pgfsetdash{}{0pt}%
\pgfpathmoveto{\pgfqpoint{7.248329in}{1.327273in}}%
\pgfpathcurveto{\pgfqpoint{7.253373in}{1.327273in}}{\pgfqpoint{7.258211in}{1.329277in}}{\pgfqpoint{7.261777in}{1.332843in}}%
\pgfpathcurveto{\pgfqpoint{7.265344in}{1.336410in}}{\pgfqpoint{7.267347in}{1.341248in}}{\pgfqpoint{7.267347in}{1.346291in}}%
\pgfpathcurveto{\pgfqpoint{7.267347in}{1.351335in}}{\pgfqpoint{7.265344in}{1.356173in}}{\pgfqpoint{7.261777in}{1.359739in}}%
\pgfpathcurveto{\pgfqpoint{7.258211in}{1.363306in}}{\pgfqpoint{7.253373in}{1.365309in}}{\pgfqpoint{7.248329in}{1.365309in}}%
\pgfpathcurveto{\pgfqpoint{7.243286in}{1.365309in}}{\pgfqpoint{7.238448in}{1.363306in}}{\pgfqpoint{7.234881in}{1.359739in}}%
\pgfpathcurveto{\pgfqpoint{7.231315in}{1.356173in}}{\pgfqpoint{7.229311in}{1.351335in}}{\pgfqpoint{7.229311in}{1.346291in}}%
\pgfpathcurveto{\pgfqpoint{7.229311in}{1.341248in}}{\pgfqpoint{7.231315in}{1.336410in}}{\pgfqpoint{7.234881in}{1.332843in}}%
\pgfpathcurveto{\pgfqpoint{7.238448in}{1.329277in}}{\pgfqpoint{7.243286in}{1.327273in}}{\pgfqpoint{7.248329in}{1.327273in}}%
\pgfpathclose%
\pgfusepath{fill}%
\end{pgfscope}%
\begin{pgfscope}%
\pgfpathrectangle{\pgfqpoint{6.572727in}{0.474100in}}{\pgfqpoint{4.227273in}{3.318700in}}%
\pgfusepath{clip}%
\pgfsetbuttcap%
\pgfsetroundjoin%
\definecolor{currentfill}{rgb}{0.993248,0.906157,0.143936}%
\pgfsetfillcolor{currentfill}%
\pgfsetfillopacity{0.700000}%
\pgfsetlinewidth{0.000000pt}%
\definecolor{currentstroke}{rgb}{0.000000,0.000000,0.000000}%
\pgfsetstrokecolor{currentstroke}%
\pgfsetstrokeopacity{0.700000}%
\pgfsetdash{}{0pt}%
\pgfpathmoveto{\pgfqpoint{9.765802in}{1.949143in}}%
\pgfpathcurveto{\pgfqpoint{9.770846in}{1.949143in}}{\pgfqpoint{9.775684in}{1.951147in}}{\pgfqpoint{9.779250in}{1.954713in}}%
\pgfpathcurveto{\pgfqpoint{9.782817in}{1.958280in}}{\pgfqpoint{9.784820in}{1.963117in}}{\pgfqpoint{9.784820in}{1.968161in}}%
\pgfpathcurveto{\pgfqpoint{9.784820in}{1.973205in}}{\pgfqpoint{9.782817in}{1.978043in}}{\pgfqpoint{9.779250in}{1.981609in}}%
\pgfpathcurveto{\pgfqpoint{9.775684in}{1.985175in}}{\pgfqpoint{9.770846in}{1.987179in}}{\pgfqpoint{9.765802in}{1.987179in}}%
\pgfpathcurveto{\pgfqpoint{9.760759in}{1.987179in}}{\pgfqpoint{9.755921in}{1.985175in}}{\pgfqpoint{9.752354in}{1.981609in}}%
\pgfpathcurveto{\pgfqpoint{9.748788in}{1.978043in}}{\pgfqpoint{9.746784in}{1.973205in}}{\pgfqpoint{9.746784in}{1.968161in}}%
\pgfpathcurveto{\pgfqpoint{9.746784in}{1.963117in}}{\pgfqpoint{9.748788in}{1.958280in}}{\pgfqpoint{9.752354in}{1.954713in}}%
\pgfpathcurveto{\pgfqpoint{9.755921in}{1.951147in}}{\pgfqpoint{9.760759in}{1.949143in}}{\pgfqpoint{9.765802in}{1.949143in}}%
\pgfpathclose%
\pgfusepath{fill}%
\end{pgfscope}%
\begin{pgfscope}%
\pgfpathrectangle{\pgfqpoint{6.572727in}{0.474100in}}{\pgfqpoint{4.227273in}{3.318700in}}%
\pgfusepath{clip}%
\pgfsetbuttcap%
\pgfsetroundjoin%
\definecolor{currentfill}{rgb}{0.127568,0.566949,0.550556}%
\pgfsetfillcolor{currentfill}%
\pgfsetfillopacity{0.700000}%
\pgfsetlinewidth{0.000000pt}%
\definecolor{currentstroke}{rgb}{0.000000,0.000000,0.000000}%
\pgfsetstrokecolor{currentstroke}%
\pgfsetstrokeopacity{0.700000}%
\pgfsetdash{}{0pt}%
\pgfpathmoveto{\pgfqpoint{8.297741in}{3.301622in}}%
\pgfpathcurveto{\pgfqpoint{8.302785in}{3.301622in}}{\pgfqpoint{8.307623in}{3.303626in}}{\pgfqpoint{8.311189in}{3.307192in}}%
\pgfpathcurveto{\pgfqpoint{8.314756in}{3.310758in}}{\pgfqpoint{8.316759in}{3.315596in}}{\pgfqpoint{8.316759in}{3.320640in}}%
\pgfpathcurveto{\pgfqpoint{8.316759in}{3.325684in}}{\pgfqpoint{8.314756in}{3.330521in}}{\pgfqpoint{8.311189in}{3.334088in}}%
\pgfpathcurveto{\pgfqpoint{8.307623in}{3.337654in}}{\pgfqpoint{8.302785in}{3.339658in}}{\pgfqpoint{8.297741in}{3.339658in}}%
\pgfpathcurveto{\pgfqpoint{8.292698in}{3.339658in}}{\pgfqpoint{8.287860in}{3.337654in}}{\pgfqpoint{8.284293in}{3.334088in}}%
\pgfpathcurveto{\pgfqpoint{8.280727in}{3.330521in}}{\pgfqpoint{8.278723in}{3.325684in}}{\pgfqpoint{8.278723in}{3.320640in}}%
\pgfpathcurveto{\pgfqpoint{8.278723in}{3.315596in}}{\pgfqpoint{8.280727in}{3.310758in}}{\pgfqpoint{8.284293in}{3.307192in}}%
\pgfpathcurveto{\pgfqpoint{8.287860in}{3.303626in}}{\pgfqpoint{8.292698in}{3.301622in}}{\pgfqpoint{8.297741in}{3.301622in}}%
\pgfpathclose%
\pgfusepath{fill}%
\end{pgfscope}%
\begin{pgfscope}%
\pgfpathrectangle{\pgfqpoint{6.572727in}{0.474100in}}{\pgfqpoint{4.227273in}{3.318700in}}%
\pgfusepath{clip}%
\pgfsetbuttcap%
\pgfsetroundjoin%
\definecolor{currentfill}{rgb}{0.127568,0.566949,0.550556}%
\pgfsetfillcolor{currentfill}%
\pgfsetfillopacity{0.700000}%
\pgfsetlinewidth{0.000000pt}%
\definecolor{currentstroke}{rgb}{0.000000,0.000000,0.000000}%
\pgfsetstrokecolor{currentstroke}%
\pgfsetstrokeopacity{0.700000}%
\pgfsetdash{}{0pt}%
\pgfpathmoveto{\pgfqpoint{8.332766in}{1.731264in}}%
\pgfpathcurveto{\pgfqpoint{8.337809in}{1.731264in}}{\pgfqpoint{8.342647in}{1.733268in}}{\pgfqpoint{8.346214in}{1.736834in}}%
\pgfpathcurveto{\pgfqpoint{8.349780in}{1.740401in}}{\pgfqpoint{8.351784in}{1.745239in}}{\pgfqpoint{8.351784in}{1.750282in}}%
\pgfpathcurveto{\pgfqpoint{8.351784in}{1.755326in}}{\pgfqpoint{8.349780in}{1.760164in}}{\pgfqpoint{8.346214in}{1.763730in}}%
\pgfpathcurveto{\pgfqpoint{8.342647in}{1.767297in}}{\pgfqpoint{8.337809in}{1.769300in}}{\pgfqpoint{8.332766in}{1.769300in}}%
\pgfpathcurveto{\pgfqpoint{8.327722in}{1.769300in}}{\pgfqpoint{8.322884in}{1.767297in}}{\pgfqpoint{8.319318in}{1.763730in}}%
\pgfpathcurveto{\pgfqpoint{8.315751in}{1.760164in}}{\pgfqpoint{8.313748in}{1.755326in}}{\pgfqpoint{8.313748in}{1.750282in}}%
\pgfpathcurveto{\pgfqpoint{8.313748in}{1.745239in}}{\pgfqpoint{8.315751in}{1.740401in}}{\pgfqpoint{8.319318in}{1.736834in}}%
\pgfpathcurveto{\pgfqpoint{8.322884in}{1.733268in}}{\pgfqpoint{8.327722in}{1.731264in}}{\pgfqpoint{8.332766in}{1.731264in}}%
\pgfpathclose%
\pgfusepath{fill}%
\end{pgfscope}%
\begin{pgfscope}%
\pgfpathrectangle{\pgfqpoint{6.572727in}{0.474100in}}{\pgfqpoint{4.227273in}{3.318700in}}%
\pgfusepath{clip}%
\pgfsetbuttcap%
\pgfsetroundjoin%
\definecolor{currentfill}{rgb}{0.993248,0.906157,0.143936}%
\pgfsetfillcolor{currentfill}%
\pgfsetfillopacity{0.700000}%
\pgfsetlinewidth{0.000000pt}%
\definecolor{currentstroke}{rgb}{0.000000,0.000000,0.000000}%
\pgfsetstrokecolor{currentstroke}%
\pgfsetstrokeopacity{0.700000}%
\pgfsetdash{}{0pt}%
\pgfpathmoveto{\pgfqpoint{9.636829in}{1.362323in}}%
\pgfpathcurveto{\pgfqpoint{9.641873in}{1.362323in}}{\pgfqpoint{9.646711in}{1.364327in}}{\pgfqpoint{9.650277in}{1.367893in}}%
\pgfpathcurveto{\pgfqpoint{9.653843in}{1.371459in}}{\pgfqpoint{9.655847in}{1.376297in}}{\pgfqpoint{9.655847in}{1.381341in}}%
\pgfpathcurveto{\pgfqpoint{9.655847in}{1.386385in}}{\pgfqpoint{9.653843in}{1.391222in}}{\pgfqpoint{9.650277in}{1.394789in}}%
\pgfpathcurveto{\pgfqpoint{9.646711in}{1.398355in}}{\pgfqpoint{9.641873in}{1.400359in}}{\pgfqpoint{9.636829in}{1.400359in}}%
\pgfpathcurveto{\pgfqpoint{9.631786in}{1.400359in}}{\pgfqpoint{9.626948in}{1.398355in}}{\pgfqpoint{9.623381in}{1.394789in}}%
\pgfpathcurveto{\pgfqpoint{9.619815in}{1.391222in}}{\pgfqpoint{9.617811in}{1.386385in}}{\pgfqpoint{9.617811in}{1.381341in}}%
\pgfpathcurveto{\pgfqpoint{9.617811in}{1.376297in}}{\pgfqpoint{9.619815in}{1.371459in}}{\pgfqpoint{9.623381in}{1.367893in}}%
\pgfpathcurveto{\pgfqpoint{9.626948in}{1.364327in}}{\pgfqpoint{9.631786in}{1.362323in}}{\pgfqpoint{9.636829in}{1.362323in}}%
\pgfpathclose%
\pgfusepath{fill}%
\end{pgfscope}%
\begin{pgfscope}%
\pgfpathrectangle{\pgfqpoint{6.572727in}{0.474100in}}{\pgfqpoint{4.227273in}{3.318700in}}%
\pgfusepath{clip}%
\pgfsetbuttcap%
\pgfsetroundjoin%
\definecolor{currentfill}{rgb}{0.127568,0.566949,0.550556}%
\pgfsetfillcolor{currentfill}%
\pgfsetfillopacity{0.700000}%
\pgfsetlinewidth{0.000000pt}%
\definecolor{currentstroke}{rgb}{0.000000,0.000000,0.000000}%
\pgfsetstrokecolor{currentstroke}%
\pgfsetstrokeopacity{0.700000}%
\pgfsetdash{}{0pt}%
\pgfpathmoveto{\pgfqpoint{7.857265in}{3.089012in}}%
\pgfpathcurveto{\pgfqpoint{7.862309in}{3.089012in}}{\pgfqpoint{7.867147in}{3.091015in}}{\pgfqpoint{7.870713in}{3.094582in}}%
\pgfpathcurveto{\pgfqpoint{7.874280in}{3.098148in}}{\pgfqpoint{7.876283in}{3.102986in}}{\pgfqpoint{7.876283in}{3.108030in}}%
\pgfpathcurveto{\pgfqpoint{7.876283in}{3.113073in}}{\pgfqpoint{7.874280in}{3.117911in}}{\pgfqpoint{7.870713in}{3.121478in}}%
\pgfpathcurveto{\pgfqpoint{7.867147in}{3.125044in}}{\pgfqpoint{7.862309in}{3.127048in}}{\pgfqpoint{7.857265in}{3.127048in}}%
\pgfpathcurveto{\pgfqpoint{7.852222in}{3.127048in}}{\pgfqpoint{7.847384in}{3.125044in}}{\pgfqpoint{7.843817in}{3.121478in}}%
\pgfpathcurveto{\pgfqpoint{7.840251in}{3.117911in}}{\pgfqpoint{7.838247in}{3.113073in}}{\pgfqpoint{7.838247in}{3.108030in}}%
\pgfpathcurveto{\pgfqpoint{7.838247in}{3.102986in}}{\pgfqpoint{7.840251in}{3.098148in}}{\pgfqpoint{7.843817in}{3.094582in}}%
\pgfpathcurveto{\pgfqpoint{7.847384in}{3.091015in}}{\pgfqpoint{7.852222in}{3.089012in}}{\pgfqpoint{7.857265in}{3.089012in}}%
\pgfpathclose%
\pgfusepath{fill}%
\end{pgfscope}%
\begin{pgfscope}%
\pgfpathrectangle{\pgfqpoint{6.572727in}{0.474100in}}{\pgfqpoint{4.227273in}{3.318700in}}%
\pgfusepath{clip}%
\pgfsetbuttcap%
\pgfsetroundjoin%
\definecolor{currentfill}{rgb}{0.127568,0.566949,0.550556}%
\pgfsetfillcolor{currentfill}%
\pgfsetfillopacity{0.700000}%
\pgfsetlinewidth{0.000000pt}%
\definecolor{currentstroke}{rgb}{0.000000,0.000000,0.000000}%
\pgfsetstrokecolor{currentstroke}%
\pgfsetstrokeopacity{0.700000}%
\pgfsetdash{}{0pt}%
\pgfpathmoveto{\pgfqpoint{7.903005in}{2.754068in}}%
\pgfpathcurveto{\pgfqpoint{7.908049in}{2.754068in}}{\pgfqpoint{7.912886in}{2.756072in}}{\pgfqpoint{7.916453in}{2.759638in}}%
\pgfpathcurveto{\pgfqpoint{7.920019in}{2.763205in}}{\pgfqpoint{7.922023in}{2.768042in}}{\pgfqpoint{7.922023in}{2.773086in}}%
\pgfpathcurveto{\pgfqpoint{7.922023in}{2.778130in}}{\pgfqpoint{7.920019in}{2.782967in}}{\pgfqpoint{7.916453in}{2.786534in}}%
\pgfpathcurveto{\pgfqpoint{7.912886in}{2.790100in}}{\pgfqpoint{7.908049in}{2.792104in}}{\pgfqpoint{7.903005in}{2.792104in}}%
\pgfpathcurveto{\pgfqpoint{7.897961in}{2.792104in}}{\pgfqpoint{7.893124in}{2.790100in}}{\pgfqpoint{7.889557in}{2.786534in}}%
\pgfpathcurveto{\pgfqpoint{7.885991in}{2.782967in}}{\pgfqpoint{7.883987in}{2.778130in}}{\pgfqpoint{7.883987in}{2.773086in}}%
\pgfpathcurveto{\pgfqpoint{7.883987in}{2.768042in}}{\pgfqpoint{7.885991in}{2.763205in}}{\pgfqpoint{7.889557in}{2.759638in}}%
\pgfpathcurveto{\pgfqpoint{7.893124in}{2.756072in}}{\pgfqpoint{7.897961in}{2.754068in}}{\pgfqpoint{7.903005in}{2.754068in}}%
\pgfpathclose%
\pgfusepath{fill}%
\end{pgfscope}%
\begin{pgfscope}%
\pgfpathrectangle{\pgfqpoint{6.572727in}{0.474100in}}{\pgfqpoint{4.227273in}{3.318700in}}%
\pgfusepath{clip}%
\pgfsetbuttcap%
\pgfsetroundjoin%
\definecolor{currentfill}{rgb}{0.127568,0.566949,0.550556}%
\pgfsetfillcolor{currentfill}%
\pgfsetfillopacity{0.700000}%
\pgfsetlinewidth{0.000000pt}%
\definecolor{currentstroke}{rgb}{0.000000,0.000000,0.000000}%
\pgfsetstrokecolor{currentstroke}%
\pgfsetstrokeopacity{0.700000}%
\pgfsetdash{}{0pt}%
\pgfpathmoveto{\pgfqpoint{8.694913in}{2.835703in}}%
\pgfpathcurveto{\pgfqpoint{8.699956in}{2.835703in}}{\pgfqpoint{8.704794in}{2.837707in}}{\pgfqpoint{8.708361in}{2.841274in}}%
\pgfpathcurveto{\pgfqpoint{8.711927in}{2.844840in}}{\pgfqpoint{8.713931in}{2.849678in}}{\pgfqpoint{8.713931in}{2.854721in}}%
\pgfpathcurveto{\pgfqpoint{8.713931in}{2.859765in}}{\pgfqpoint{8.711927in}{2.864603in}}{\pgfqpoint{8.708361in}{2.868169in}}%
\pgfpathcurveto{\pgfqpoint{8.704794in}{2.871736in}}{\pgfqpoint{8.699956in}{2.873740in}}{\pgfqpoint{8.694913in}{2.873740in}}%
\pgfpathcurveto{\pgfqpoint{8.689869in}{2.873740in}}{\pgfqpoint{8.685031in}{2.871736in}}{\pgfqpoint{8.681465in}{2.868169in}}%
\pgfpathcurveto{\pgfqpoint{8.677898in}{2.864603in}}{\pgfqpoint{8.675895in}{2.859765in}}{\pgfqpoint{8.675895in}{2.854721in}}%
\pgfpathcurveto{\pgfqpoint{8.675895in}{2.849678in}}{\pgfqpoint{8.677898in}{2.844840in}}{\pgfqpoint{8.681465in}{2.841274in}}%
\pgfpathcurveto{\pgfqpoint{8.685031in}{2.837707in}}{\pgfqpoint{8.689869in}{2.835703in}}{\pgfqpoint{8.694913in}{2.835703in}}%
\pgfpathclose%
\pgfusepath{fill}%
\end{pgfscope}%
\begin{pgfscope}%
\pgfpathrectangle{\pgfqpoint{6.572727in}{0.474100in}}{\pgfqpoint{4.227273in}{3.318700in}}%
\pgfusepath{clip}%
\pgfsetbuttcap%
\pgfsetroundjoin%
\definecolor{currentfill}{rgb}{0.127568,0.566949,0.550556}%
\pgfsetfillcolor{currentfill}%
\pgfsetfillopacity{0.700000}%
\pgfsetlinewidth{0.000000pt}%
\definecolor{currentstroke}{rgb}{0.000000,0.000000,0.000000}%
\pgfsetstrokecolor{currentstroke}%
\pgfsetstrokeopacity{0.700000}%
\pgfsetdash{}{0pt}%
\pgfpathmoveto{\pgfqpoint{7.486427in}{1.070516in}}%
\pgfpathcurveto{\pgfqpoint{7.491471in}{1.070516in}}{\pgfqpoint{7.496309in}{1.072519in}}{\pgfqpoint{7.499875in}{1.076086in}}%
\pgfpathcurveto{\pgfqpoint{7.503441in}{1.079652in}}{\pgfqpoint{7.505445in}{1.084490in}}{\pgfqpoint{7.505445in}{1.089534in}}%
\pgfpathcurveto{\pgfqpoint{7.505445in}{1.094577in}}{\pgfqpoint{7.503441in}{1.099415in}}{\pgfqpoint{7.499875in}{1.102982in}}%
\pgfpathcurveto{\pgfqpoint{7.496309in}{1.106548in}}{\pgfqpoint{7.491471in}{1.108552in}}{\pgfqpoint{7.486427in}{1.108552in}}%
\pgfpathcurveto{\pgfqpoint{7.481383in}{1.108552in}}{\pgfqpoint{7.476546in}{1.106548in}}{\pgfqpoint{7.472979in}{1.102982in}}%
\pgfpathcurveto{\pgfqpoint{7.469413in}{1.099415in}}{\pgfqpoint{7.467409in}{1.094577in}}{\pgfqpoint{7.467409in}{1.089534in}}%
\pgfpathcurveto{\pgfqpoint{7.467409in}{1.084490in}}{\pgfqpoint{7.469413in}{1.079652in}}{\pgfqpoint{7.472979in}{1.076086in}}%
\pgfpathcurveto{\pgfqpoint{7.476546in}{1.072519in}}{\pgfqpoint{7.481383in}{1.070516in}}{\pgfqpoint{7.486427in}{1.070516in}}%
\pgfpathclose%
\pgfusepath{fill}%
\end{pgfscope}%
\begin{pgfscope}%
\pgfpathrectangle{\pgfqpoint{6.572727in}{0.474100in}}{\pgfqpoint{4.227273in}{3.318700in}}%
\pgfusepath{clip}%
\pgfsetbuttcap%
\pgfsetroundjoin%
\definecolor{currentfill}{rgb}{0.127568,0.566949,0.550556}%
\pgfsetfillcolor{currentfill}%
\pgfsetfillopacity{0.700000}%
\pgfsetlinewidth{0.000000pt}%
\definecolor{currentstroke}{rgb}{0.000000,0.000000,0.000000}%
\pgfsetstrokecolor{currentstroke}%
\pgfsetstrokeopacity{0.700000}%
\pgfsetdash{}{0pt}%
\pgfpathmoveto{\pgfqpoint{7.674943in}{1.334929in}}%
\pgfpathcurveto{\pgfqpoint{7.679987in}{1.334929in}}{\pgfqpoint{7.684825in}{1.336933in}}{\pgfqpoint{7.688391in}{1.340499in}}%
\pgfpathcurveto{\pgfqpoint{7.691957in}{1.344066in}}{\pgfqpoint{7.693961in}{1.348904in}}{\pgfqpoint{7.693961in}{1.353947in}}%
\pgfpathcurveto{\pgfqpoint{7.693961in}{1.358991in}}{\pgfqpoint{7.691957in}{1.363829in}}{\pgfqpoint{7.688391in}{1.367395in}}%
\pgfpathcurveto{\pgfqpoint{7.684825in}{1.370962in}}{\pgfqpoint{7.679987in}{1.372965in}}{\pgfqpoint{7.674943in}{1.372965in}}%
\pgfpathcurveto{\pgfqpoint{7.669899in}{1.372965in}}{\pgfqpoint{7.665062in}{1.370962in}}{\pgfqpoint{7.661495in}{1.367395in}}%
\pgfpathcurveto{\pgfqpoint{7.657929in}{1.363829in}}{\pgfqpoint{7.655925in}{1.358991in}}{\pgfqpoint{7.655925in}{1.353947in}}%
\pgfpathcurveto{\pgfqpoint{7.655925in}{1.348904in}}{\pgfqpoint{7.657929in}{1.344066in}}{\pgfqpoint{7.661495in}{1.340499in}}%
\pgfpathcurveto{\pgfqpoint{7.665062in}{1.336933in}}{\pgfqpoint{7.669899in}{1.334929in}}{\pgfqpoint{7.674943in}{1.334929in}}%
\pgfpathclose%
\pgfusepath{fill}%
\end{pgfscope}%
\begin{pgfscope}%
\pgfpathrectangle{\pgfqpoint{6.572727in}{0.474100in}}{\pgfqpoint{4.227273in}{3.318700in}}%
\pgfusepath{clip}%
\pgfsetbuttcap%
\pgfsetroundjoin%
\definecolor{currentfill}{rgb}{0.127568,0.566949,0.550556}%
\pgfsetfillcolor{currentfill}%
\pgfsetfillopacity{0.700000}%
\pgfsetlinewidth{0.000000pt}%
\definecolor{currentstroke}{rgb}{0.000000,0.000000,0.000000}%
\pgfsetstrokecolor{currentstroke}%
\pgfsetstrokeopacity{0.700000}%
\pgfsetdash{}{0pt}%
\pgfpathmoveto{\pgfqpoint{8.347863in}{2.480927in}}%
\pgfpathcurveto{\pgfqpoint{8.352907in}{2.480927in}}{\pgfqpoint{8.357745in}{2.482931in}}{\pgfqpoint{8.361311in}{2.486497in}}%
\pgfpathcurveto{\pgfqpoint{8.364878in}{2.490064in}}{\pgfqpoint{8.366882in}{2.494902in}}{\pgfqpoint{8.366882in}{2.499945in}}%
\pgfpathcurveto{\pgfqpoint{8.366882in}{2.504989in}}{\pgfqpoint{8.364878in}{2.509827in}}{\pgfqpoint{8.361311in}{2.513393in}}%
\pgfpathcurveto{\pgfqpoint{8.357745in}{2.516959in}}{\pgfqpoint{8.352907in}{2.518963in}}{\pgfqpoint{8.347863in}{2.518963in}}%
\pgfpathcurveto{\pgfqpoint{8.342820in}{2.518963in}}{\pgfqpoint{8.337982in}{2.516959in}}{\pgfqpoint{8.334416in}{2.513393in}}%
\pgfpathcurveto{\pgfqpoint{8.330849in}{2.509827in}}{\pgfqpoint{8.328845in}{2.504989in}}{\pgfqpoint{8.328845in}{2.499945in}}%
\pgfpathcurveto{\pgfqpoint{8.328845in}{2.494902in}}{\pgfqpoint{8.330849in}{2.490064in}}{\pgfqpoint{8.334416in}{2.486497in}}%
\pgfpathcurveto{\pgfqpoint{8.337982in}{2.482931in}}{\pgfqpoint{8.342820in}{2.480927in}}{\pgfqpoint{8.347863in}{2.480927in}}%
\pgfpathclose%
\pgfusepath{fill}%
\end{pgfscope}%
\begin{pgfscope}%
\pgfpathrectangle{\pgfqpoint{6.572727in}{0.474100in}}{\pgfqpoint{4.227273in}{3.318700in}}%
\pgfusepath{clip}%
\pgfsetbuttcap%
\pgfsetroundjoin%
\definecolor{currentfill}{rgb}{0.993248,0.906157,0.143936}%
\pgfsetfillcolor{currentfill}%
\pgfsetfillopacity{0.700000}%
\pgfsetlinewidth{0.000000pt}%
\definecolor{currentstroke}{rgb}{0.000000,0.000000,0.000000}%
\pgfsetstrokecolor{currentstroke}%
\pgfsetstrokeopacity{0.700000}%
\pgfsetdash{}{0pt}%
\pgfpathmoveto{\pgfqpoint{9.203207in}{1.116841in}}%
\pgfpathcurveto{\pgfqpoint{9.208251in}{1.116841in}}{\pgfqpoint{9.213089in}{1.118845in}}{\pgfqpoint{9.216655in}{1.122411in}}%
\pgfpathcurveto{\pgfqpoint{9.220222in}{1.125978in}}{\pgfqpoint{9.222226in}{1.130815in}}{\pgfqpoint{9.222226in}{1.135859in}}%
\pgfpathcurveto{\pgfqpoint{9.222226in}{1.140903in}}{\pgfqpoint{9.220222in}{1.145741in}}{\pgfqpoint{9.216655in}{1.149307in}}%
\pgfpathcurveto{\pgfqpoint{9.213089in}{1.152873in}}{\pgfqpoint{9.208251in}{1.154877in}}{\pgfqpoint{9.203207in}{1.154877in}}%
\pgfpathcurveto{\pgfqpoint{9.198164in}{1.154877in}}{\pgfqpoint{9.193326in}{1.152873in}}{\pgfqpoint{9.189760in}{1.149307in}}%
\pgfpathcurveto{\pgfqpoint{9.186193in}{1.145741in}}{\pgfqpoint{9.184189in}{1.140903in}}{\pgfqpoint{9.184189in}{1.135859in}}%
\pgfpathcurveto{\pgfqpoint{9.184189in}{1.130815in}}{\pgfqpoint{9.186193in}{1.125978in}}{\pgfqpoint{9.189760in}{1.122411in}}%
\pgfpathcurveto{\pgfqpoint{9.193326in}{1.118845in}}{\pgfqpoint{9.198164in}{1.116841in}}{\pgfqpoint{9.203207in}{1.116841in}}%
\pgfpathclose%
\pgfusepath{fill}%
\end{pgfscope}%
\begin{pgfscope}%
\pgfpathrectangle{\pgfqpoint{6.572727in}{0.474100in}}{\pgfqpoint{4.227273in}{3.318700in}}%
\pgfusepath{clip}%
\pgfsetbuttcap%
\pgfsetroundjoin%
\definecolor{currentfill}{rgb}{0.993248,0.906157,0.143936}%
\pgfsetfillcolor{currentfill}%
\pgfsetfillopacity{0.700000}%
\pgfsetlinewidth{0.000000pt}%
\definecolor{currentstroke}{rgb}{0.000000,0.000000,0.000000}%
\pgfsetstrokecolor{currentstroke}%
\pgfsetstrokeopacity{0.700000}%
\pgfsetdash{}{0pt}%
\pgfpathmoveto{\pgfqpoint{9.020750in}{0.960771in}}%
\pgfpathcurveto{\pgfqpoint{9.025794in}{0.960771in}}{\pgfqpoint{9.030632in}{0.962774in}}{\pgfqpoint{9.034198in}{0.966341in}}%
\pgfpathcurveto{\pgfqpoint{9.037765in}{0.969907in}}{\pgfqpoint{9.039769in}{0.974745in}}{\pgfqpoint{9.039769in}{0.979789in}}%
\pgfpathcurveto{\pgfqpoint{9.039769in}{0.984832in}}{\pgfqpoint{9.037765in}{0.989670in}}{\pgfqpoint{9.034198in}{0.993237in}}%
\pgfpathcurveto{\pgfqpoint{9.030632in}{0.996803in}}{\pgfqpoint{9.025794in}{0.998807in}}{\pgfqpoint{9.020750in}{0.998807in}}%
\pgfpathcurveto{\pgfqpoint{9.015707in}{0.998807in}}{\pgfqpoint{9.010869in}{0.996803in}}{\pgfqpoint{9.007303in}{0.993237in}}%
\pgfpathcurveto{\pgfqpoint{9.003736in}{0.989670in}}{\pgfqpoint{9.001732in}{0.984832in}}{\pgfqpoint{9.001732in}{0.979789in}}%
\pgfpathcurveto{\pgfqpoint{9.001732in}{0.974745in}}{\pgfqpoint{9.003736in}{0.969907in}}{\pgfqpoint{9.007303in}{0.966341in}}%
\pgfpathcurveto{\pgfqpoint{9.010869in}{0.962774in}}{\pgfqpoint{9.015707in}{0.960771in}}{\pgfqpoint{9.020750in}{0.960771in}}%
\pgfpathclose%
\pgfusepath{fill}%
\end{pgfscope}%
\begin{pgfscope}%
\pgfpathrectangle{\pgfqpoint{6.572727in}{0.474100in}}{\pgfqpoint{4.227273in}{3.318700in}}%
\pgfusepath{clip}%
\pgfsetbuttcap%
\pgfsetroundjoin%
\definecolor{currentfill}{rgb}{0.127568,0.566949,0.550556}%
\pgfsetfillcolor{currentfill}%
\pgfsetfillopacity{0.700000}%
\pgfsetlinewidth{0.000000pt}%
\definecolor{currentstroke}{rgb}{0.000000,0.000000,0.000000}%
\pgfsetstrokecolor{currentstroke}%
\pgfsetstrokeopacity{0.700000}%
\pgfsetdash{}{0pt}%
\pgfpathmoveto{\pgfqpoint{8.228015in}{2.887898in}}%
\pgfpathcurveto{\pgfqpoint{8.233058in}{2.887898in}}{\pgfqpoint{8.237896in}{2.889902in}}{\pgfqpoint{8.241463in}{2.893469in}}%
\pgfpathcurveto{\pgfqpoint{8.245029in}{2.897035in}}{\pgfqpoint{8.247033in}{2.901873in}}{\pgfqpoint{8.247033in}{2.906917in}}%
\pgfpathcurveto{\pgfqpoint{8.247033in}{2.911960in}}{\pgfqpoint{8.245029in}{2.916798in}}{\pgfqpoint{8.241463in}{2.920364in}}%
\pgfpathcurveto{\pgfqpoint{8.237896in}{2.923931in}}{\pgfqpoint{8.233058in}{2.925935in}}{\pgfqpoint{8.228015in}{2.925935in}}%
\pgfpathcurveto{\pgfqpoint{8.222971in}{2.925935in}}{\pgfqpoint{8.218133in}{2.923931in}}{\pgfqpoint{8.214567in}{2.920364in}}%
\pgfpathcurveto{\pgfqpoint{8.211000in}{2.916798in}}{\pgfqpoint{8.208997in}{2.911960in}}{\pgfqpoint{8.208997in}{2.906917in}}%
\pgfpathcurveto{\pgfqpoint{8.208997in}{2.901873in}}{\pgfqpoint{8.211000in}{2.897035in}}{\pgfqpoint{8.214567in}{2.893469in}}%
\pgfpathcurveto{\pgfqpoint{8.218133in}{2.889902in}}{\pgfqpoint{8.222971in}{2.887898in}}{\pgfqpoint{8.228015in}{2.887898in}}%
\pgfpathclose%
\pgfusepath{fill}%
\end{pgfscope}%
\begin{pgfscope}%
\pgfpathrectangle{\pgfqpoint{6.572727in}{0.474100in}}{\pgfqpoint{4.227273in}{3.318700in}}%
\pgfusepath{clip}%
\pgfsetbuttcap%
\pgfsetroundjoin%
\definecolor{currentfill}{rgb}{0.993248,0.906157,0.143936}%
\pgfsetfillcolor{currentfill}%
\pgfsetfillopacity{0.700000}%
\pgfsetlinewidth{0.000000pt}%
\definecolor{currentstroke}{rgb}{0.000000,0.000000,0.000000}%
\pgfsetstrokecolor{currentstroke}%
\pgfsetstrokeopacity{0.700000}%
\pgfsetdash{}{0pt}%
\pgfpathmoveto{\pgfqpoint{9.623363in}{1.299352in}}%
\pgfpathcurveto{\pgfqpoint{9.628406in}{1.299352in}}{\pgfqpoint{9.633244in}{1.301355in}}{\pgfqpoint{9.636810in}{1.304922in}}%
\pgfpathcurveto{\pgfqpoint{9.640377in}{1.308488in}}{\pgfqpoint{9.642381in}{1.313326in}}{\pgfqpoint{9.642381in}{1.318370in}}%
\pgfpathcurveto{\pgfqpoint{9.642381in}{1.323413in}}{\pgfqpoint{9.640377in}{1.328251in}}{\pgfqpoint{9.636810in}{1.331818in}}%
\pgfpathcurveto{\pgfqpoint{9.633244in}{1.335384in}}{\pgfqpoint{9.628406in}{1.337388in}}{\pgfqpoint{9.623363in}{1.337388in}}%
\pgfpathcurveto{\pgfqpoint{9.618319in}{1.337388in}}{\pgfqpoint{9.613481in}{1.335384in}}{\pgfqpoint{9.609915in}{1.331818in}}%
\pgfpathcurveto{\pgfqpoint{9.606348in}{1.328251in}}{\pgfqpoint{9.604344in}{1.323413in}}{\pgfqpoint{9.604344in}{1.318370in}}%
\pgfpathcurveto{\pgfqpoint{9.604344in}{1.313326in}}{\pgfqpoint{9.606348in}{1.308488in}}{\pgfqpoint{9.609915in}{1.304922in}}%
\pgfpathcurveto{\pgfqpoint{9.613481in}{1.301355in}}{\pgfqpoint{9.618319in}{1.299352in}}{\pgfqpoint{9.623363in}{1.299352in}}%
\pgfpathclose%
\pgfusepath{fill}%
\end{pgfscope}%
\begin{pgfscope}%
\pgfpathrectangle{\pgfqpoint{6.572727in}{0.474100in}}{\pgfqpoint{4.227273in}{3.318700in}}%
\pgfusepath{clip}%
\pgfsetbuttcap%
\pgfsetroundjoin%
\definecolor{currentfill}{rgb}{0.993248,0.906157,0.143936}%
\pgfsetfillcolor{currentfill}%
\pgfsetfillopacity{0.700000}%
\pgfsetlinewidth{0.000000pt}%
\definecolor{currentstroke}{rgb}{0.000000,0.000000,0.000000}%
\pgfsetstrokecolor{currentstroke}%
\pgfsetstrokeopacity{0.700000}%
\pgfsetdash{}{0pt}%
\pgfpathmoveto{\pgfqpoint{9.234232in}{1.370013in}}%
\pgfpathcurveto{\pgfqpoint{9.239276in}{1.370013in}}{\pgfqpoint{9.244114in}{1.372017in}}{\pgfqpoint{9.247680in}{1.375583in}}%
\pgfpathcurveto{\pgfqpoint{9.251246in}{1.379150in}}{\pgfqpoint{9.253250in}{1.383987in}}{\pgfqpoint{9.253250in}{1.389031in}}%
\pgfpathcurveto{\pgfqpoint{9.253250in}{1.394075in}}{\pgfqpoint{9.251246in}{1.398913in}}{\pgfqpoint{9.247680in}{1.402479in}}%
\pgfpathcurveto{\pgfqpoint{9.244114in}{1.406045in}}{\pgfqpoint{9.239276in}{1.408049in}}{\pgfqpoint{9.234232in}{1.408049in}}%
\pgfpathcurveto{\pgfqpoint{9.229188in}{1.408049in}}{\pgfqpoint{9.224351in}{1.406045in}}{\pgfqpoint{9.220784in}{1.402479in}}%
\pgfpathcurveto{\pgfqpoint{9.217218in}{1.398913in}}{\pgfqpoint{9.215214in}{1.394075in}}{\pgfqpoint{9.215214in}{1.389031in}}%
\pgfpathcurveto{\pgfqpoint{9.215214in}{1.383987in}}{\pgfqpoint{9.217218in}{1.379150in}}{\pgfqpoint{9.220784in}{1.375583in}}%
\pgfpathcurveto{\pgfqpoint{9.224351in}{1.372017in}}{\pgfqpoint{9.229188in}{1.370013in}}{\pgfqpoint{9.234232in}{1.370013in}}%
\pgfpathclose%
\pgfusepath{fill}%
\end{pgfscope}%
\begin{pgfscope}%
\pgfpathrectangle{\pgfqpoint{6.572727in}{0.474100in}}{\pgfqpoint{4.227273in}{3.318700in}}%
\pgfusepath{clip}%
\pgfsetbuttcap%
\pgfsetroundjoin%
\definecolor{currentfill}{rgb}{0.127568,0.566949,0.550556}%
\pgfsetfillcolor{currentfill}%
\pgfsetfillopacity{0.700000}%
\pgfsetlinewidth{0.000000pt}%
\definecolor{currentstroke}{rgb}{0.000000,0.000000,0.000000}%
\pgfsetstrokecolor{currentstroke}%
\pgfsetstrokeopacity{0.700000}%
\pgfsetdash{}{0pt}%
\pgfpathmoveto{\pgfqpoint{7.543704in}{1.881352in}}%
\pgfpathcurveto{\pgfqpoint{7.548748in}{1.881352in}}{\pgfqpoint{7.553585in}{1.883356in}}{\pgfqpoint{7.557152in}{1.886923in}}%
\pgfpathcurveto{\pgfqpoint{7.560718in}{1.890489in}}{\pgfqpoint{7.562722in}{1.895327in}}{\pgfqpoint{7.562722in}{1.900371in}}%
\pgfpathcurveto{\pgfqpoint{7.562722in}{1.905414in}}{\pgfqpoint{7.560718in}{1.910252in}}{\pgfqpoint{7.557152in}{1.913818in}}%
\pgfpathcurveto{\pgfqpoint{7.553585in}{1.917385in}}{\pgfqpoint{7.548748in}{1.919389in}}{\pgfqpoint{7.543704in}{1.919389in}}%
\pgfpathcurveto{\pgfqpoint{7.538660in}{1.919389in}}{\pgfqpoint{7.533823in}{1.917385in}}{\pgfqpoint{7.530256in}{1.913818in}}%
\pgfpathcurveto{\pgfqpoint{7.526690in}{1.910252in}}{\pgfqpoint{7.524686in}{1.905414in}}{\pgfqpoint{7.524686in}{1.900371in}}%
\pgfpathcurveto{\pgfqpoint{7.524686in}{1.895327in}}{\pgfqpoint{7.526690in}{1.890489in}}{\pgfqpoint{7.530256in}{1.886923in}}%
\pgfpathcurveto{\pgfqpoint{7.533823in}{1.883356in}}{\pgfqpoint{7.538660in}{1.881352in}}{\pgfqpoint{7.543704in}{1.881352in}}%
\pgfpathclose%
\pgfusepath{fill}%
\end{pgfscope}%
\begin{pgfscope}%
\pgfpathrectangle{\pgfqpoint{6.572727in}{0.474100in}}{\pgfqpoint{4.227273in}{3.318700in}}%
\pgfusepath{clip}%
\pgfsetbuttcap%
\pgfsetroundjoin%
\definecolor{currentfill}{rgb}{0.127568,0.566949,0.550556}%
\pgfsetfillcolor{currentfill}%
\pgfsetfillopacity{0.700000}%
\pgfsetlinewidth{0.000000pt}%
\definecolor{currentstroke}{rgb}{0.000000,0.000000,0.000000}%
\pgfsetstrokecolor{currentstroke}%
\pgfsetstrokeopacity{0.700000}%
\pgfsetdash{}{0pt}%
\pgfpathmoveto{\pgfqpoint{7.281759in}{1.846736in}}%
\pgfpathcurveto{\pgfqpoint{7.286803in}{1.846736in}}{\pgfqpoint{7.291640in}{1.848740in}}{\pgfqpoint{7.295207in}{1.852306in}}%
\pgfpathcurveto{\pgfqpoint{7.298773in}{1.855873in}}{\pgfqpoint{7.300777in}{1.860711in}}{\pgfqpoint{7.300777in}{1.865754in}}%
\pgfpathcurveto{\pgfqpoint{7.300777in}{1.870798in}}{\pgfqpoint{7.298773in}{1.875636in}}{\pgfqpoint{7.295207in}{1.879202in}}%
\pgfpathcurveto{\pgfqpoint{7.291640in}{1.882769in}}{\pgfqpoint{7.286803in}{1.884772in}}{\pgfqpoint{7.281759in}{1.884772in}}%
\pgfpathcurveto{\pgfqpoint{7.276715in}{1.884772in}}{\pgfqpoint{7.271878in}{1.882769in}}{\pgfqpoint{7.268311in}{1.879202in}}%
\pgfpathcurveto{\pgfqpoint{7.264745in}{1.875636in}}{\pgfqpoint{7.262741in}{1.870798in}}{\pgfqpoint{7.262741in}{1.865754in}}%
\pgfpathcurveto{\pgfqpoint{7.262741in}{1.860711in}}{\pgfqpoint{7.264745in}{1.855873in}}{\pgfqpoint{7.268311in}{1.852306in}}%
\pgfpathcurveto{\pgfqpoint{7.271878in}{1.848740in}}{\pgfqpoint{7.276715in}{1.846736in}}{\pgfqpoint{7.281759in}{1.846736in}}%
\pgfpathclose%
\pgfusepath{fill}%
\end{pgfscope}%
\begin{pgfscope}%
\pgfpathrectangle{\pgfqpoint{6.572727in}{0.474100in}}{\pgfqpoint{4.227273in}{3.318700in}}%
\pgfusepath{clip}%
\pgfsetbuttcap%
\pgfsetroundjoin%
\definecolor{currentfill}{rgb}{0.127568,0.566949,0.550556}%
\pgfsetfillcolor{currentfill}%
\pgfsetfillopacity{0.700000}%
\pgfsetlinewidth{0.000000pt}%
\definecolor{currentstroke}{rgb}{0.000000,0.000000,0.000000}%
\pgfsetstrokecolor{currentstroke}%
\pgfsetstrokeopacity{0.700000}%
\pgfsetdash{}{0pt}%
\pgfpathmoveto{\pgfqpoint{7.755262in}{2.153809in}}%
\pgfpathcurveto{\pgfqpoint{7.760305in}{2.153809in}}{\pgfqpoint{7.765143in}{2.155813in}}{\pgfqpoint{7.768710in}{2.159379in}}%
\pgfpathcurveto{\pgfqpoint{7.772276in}{2.162945in}}{\pgfqpoint{7.774280in}{2.167783in}}{\pgfqpoint{7.774280in}{2.172827in}}%
\pgfpathcurveto{\pgfqpoint{7.774280in}{2.177870in}}{\pgfqpoint{7.772276in}{2.182708in}}{\pgfqpoint{7.768710in}{2.186275in}}%
\pgfpathcurveto{\pgfqpoint{7.765143in}{2.189841in}}{\pgfqpoint{7.760305in}{2.191845in}}{\pgfqpoint{7.755262in}{2.191845in}}%
\pgfpathcurveto{\pgfqpoint{7.750218in}{2.191845in}}{\pgfqpoint{7.745380in}{2.189841in}}{\pgfqpoint{7.741814in}{2.186275in}}%
\pgfpathcurveto{\pgfqpoint{7.738248in}{2.182708in}}{\pgfqpoint{7.736244in}{2.177870in}}{\pgfqpoint{7.736244in}{2.172827in}}%
\pgfpathcurveto{\pgfqpoint{7.736244in}{2.167783in}}{\pgfqpoint{7.738248in}{2.162945in}}{\pgfqpoint{7.741814in}{2.159379in}}%
\pgfpathcurveto{\pgfqpoint{7.745380in}{2.155813in}}{\pgfqpoint{7.750218in}{2.153809in}}{\pgfqpoint{7.755262in}{2.153809in}}%
\pgfpathclose%
\pgfusepath{fill}%
\end{pgfscope}%
\begin{pgfscope}%
\pgfpathrectangle{\pgfqpoint{6.572727in}{0.474100in}}{\pgfqpoint{4.227273in}{3.318700in}}%
\pgfusepath{clip}%
\pgfsetbuttcap%
\pgfsetroundjoin%
\definecolor{currentfill}{rgb}{0.127568,0.566949,0.550556}%
\pgfsetfillcolor{currentfill}%
\pgfsetfillopacity{0.700000}%
\pgfsetlinewidth{0.000000pt}%
\definecolor{currentstroke}{rgb}{0.000000,0.000000,0.000000}%
\pgfsetstrokecolor{currentstroke}%
\pgfsetstrokeopacity{0.700000}%
\pgfsetdash{}{0pt}%
\pgfpathmoveto{\pgfqpoint{8.323257in}{2.408679in}}%
\pgfpathcurveto{\pgfqpoint{8.328301in}{2.408679in}}{\pgfqpoint{8.333139in}{2.410683in}}{\pgfqpoint{8.336705in}{2.414250in}}%
\pgfpathcurveto{\pgfqpoint{8.340271in}{2.417816in}}{\pgfqpoint{8.342275in}{2.422654in}}{\pgfqpoint{8.342275in}{2.427698in}}%
\pgfpathcurveto{\pgfqpoint{8.342275in}{2.432741in}}{\pgfqpoint{8.340271in}{2.437579in}}{\pgfqpoint{8.336705in}{2.441145in}}%
\pgfpathcurveto{\pgfqpoint{8.333139in}{2.444712in}}{\pgfqpoint{8.328301in}{2.446716in}}{\pgfqpoint{8.323257in}{2.446716in}}%
\pgfpathcurveto{\pgfqpoint{8.318213in}{2.446716in}}{\pgfqpoint{8.313376in}{2.444712in}}{\pgfqpoint{8.309809in}{2.441145in}}%
\pgfpathcurveto{\pgfqpoint{8.306243in}{2.437579in}}{\pgfqpoint{8.304239in}{2.432741in}}{\pgfqpoint{8.304239in}{2.427698in}}%
\pgfpathcurveto{\pgfqpoint{8.304239in}{2.422654in}}{\pgfqpoint{8.306243in}{2.417816in}}{\pgfqpoint{8.309809in}{2.414250in}}%
\pgfpathcurveto{\pgfqpoint{8.313376in}{2.410683in}}{\pgfqpoint{8.318213in}{2.408679in}}{\pgfqpoint{8.323257in}{2.408679in}}%
\pgfpathclose%
\pgfusepath{fill}%
\end{pgfscope}%
\begin{pgfscope}%
\pgfpathrectangle{\pgfqpoint{6.572727in}{0.474100in}}{\pgfqpoint{4.227273in}{3.318700in}}%
\pgfusepath{clip}%
\pgfsetbuttcap%
\pgfsetroundjoin%
\definecolor{currentfill}{rgb}{0.127568,0.566949,0.550556}%
\pgfsetfillcolor{currentfill}%
\pgfsetfillopacity{0.700000}%
\pgfsetlinewidth{0.000000pt}%
\definecolor{currentstroke}{rgb}{0.000000,0.000000,0.000000}%
\pgfsetstrokecolor{currentstroke}%
\pgfsetstrokeopacity{0.700000}%
\pgfsetdash{}{0pt}%
\pgfpathmoveto{\pgfqpoint{8.300424in}{1.418839in}}%
\pgfpathcurveto{\pgfqpoint{8.305468in}{1.418839in}}{\pgfqpoint{8.310306in}{1.420843in}}{\pgfqpoint{8.313872in}{1.424409in}}%
\pgfpathcurveto{\pgfqpoint{8.317439in}{1.427975in}}{\pgfqpoint{8.319442in}{1.432813in}}{\pgfqpoint{8.319442in}{1.437857in}}%
\pgfpathcurveto{\pgfqpoint{8.319442in}{1.442900in}}{\pgfqpoint{8.317439in}{1.447738in}}{\pgfqpoint{8.313872in}{1.451305in}}%
\pgfpathcurveto{\pgfqpoint{8.310306in}{1.454871in}}{\pgfqpoint{8.305468in}{1.456875in}}{\pgfqpoint{8.300424in}{1.456875in}}%
\pgfpathcurveto{\pgfqpoint{8.295381in}{1.456875in}}{\pgfqpoint{8.290543in}{1.454871in}}{\pgfqpoint{8.286976in}{1.451305in}}%
\pgfpathcurveto{\pgfqpoint{8.283410in}{1.447738in}}{\pgfqpoint{8.281406in}{1.442900in}}{\pgfqpoint{8.281406in}{1.437857in}}%
\pgfpathcurveto{\pgfqpoint{8.281406in}{1.432813in}}{\pgfqpoint{8.283410in}{1.427975in}}{\pgfqpoint{8.286976in}{1.424409in}}%
\pgfpathcurveto{\pgfqpoint{8.290543in}{1.420843in}}{\pgfqpoint{8.295381in}{1.418839in}}{\pgfqpoint{8.300424in}{1.418839in}}%
\pgfpathclose%
\pgfusepath{fill}%
\end{pgfscope}%
\begin{pgfscope}%
\pgfpathrectangle{\pgfqpoint{6.572727in}{0.474100in}}{\pgfqpoint{4.227273in}{3.318700in}}%
\pgfusepath{clip}%
\pgfsetbuttcap%
\pgfsetroundjoin%
\definecolor{currentfill}{rgb}{0.993248,0.906157,0.143936}%
\pgfsetfillcolor{currentfill}%
\pgfsetfillopacity{0.700000}%
\pgfsetlinewidth{0.000000pt}%
\definecolor{currentstroke}{rgb}{0.000000,0.000000,0.000000}%
\pgfsetstrokecolor{currentstroke}%
\pgfsetstrokeopacity{0.700000}%
\pgfsetdash{}{0pt}%
\pgfpathmoveto{\pgfqpoint{9.664183in}{1.336627in}}%
\pgfpathcurveto{\pgfqpoint{9.669227in}{1.336627in}}{\pgfqpoint{9.674065in}{1.338631in}}{\pgfqpoint{9.677631in}{1.342197in}}%
\pgfpathcurveto{\pgfqpoint{9.681198in}{1.345764in}}{\pgfqpoint{9.683202in}{1.350602in}}{\pgfqpoint{9.683202in}{1.355645in}}%
\pgfpathcurveto{\pgfqpoint{9.683202in}{1.360689in}}{\pgfqpoint{9.681198in}{1.365527in}}{\pgfqpoint{9.677631in}{1.369093in}}%
\pgfpathcurveto{\pgfqpoint{9.674065in}{1.372660in}}{\pgfqpoint{9.669227in}{1.374663in}}{\pgfqpoint{9.664183in}{1.374663in}}%
\pgfpathcurveto{\pgfqpoint{9.659140in}{1.374663in}}{\pgfqpoint{9.654302in}{1.372660in}}{\pgfqpoint{9.650736in}{1.369093in}}%
\pgfpathcurveto{\pgfqpoint{9.647169in}{1.365527in}}{\pgfqpoint{9.645165in}{1.360689in}}{\pgfqpoint{9.645165in}{1.355645in}}%
\pgfpathcurveto{\pgfqpoint{9.645165in}{1.350602in}}{\pgfqpoint{9.647169in}{1.345764in}}{\pgfqpoint{9.650736in}{1.342197in}}%
\pgfpathcurveto{\pgfqpoint{9.654302in}{1.338631in}}{\pgfqpoint{9.659140in}{1.336627in}}{\pgfqpoint{9.664183in}{1.336627in}}%
\pgfpathclose%
\pgfusepath{fill}%
\end{pgfscope}%
\begin{pgfscope}%
\pgfpathrectangle{\pgfqpoint{6.572727in}{0.474100in}}{\pgfqpoint{4.227273in}{3.318700in}}%
\pgfusepath{clip}%
\pgfsetbuttcap%
\pgfsetroundjoin%
\definecolor{currentfill}{rgb}{0.127568,0.566949,0.550556}%
\pgfsetfillcolor{currentfill}%
\pgfsetfillopacity{0.700000}%
\pgfsetlinewidth{0.000000pt}%
\definecolor{currentstroke}{rgb}{0.000000,0.000000,0.000000}%
\pgfsetstrokecolor{currentstroke}%
\pgfsetstrokeopacity{0.700000}%
\pgfsetdash{}{0pt}%
\pgfpathmoveto{\pgfqpoint{7.833822in}{1.085122in}}%
\pgfpathcurveto{\pgfqpoint{7.838865in}{1.085122in}}{\pgfqpoint{7.843703in}{1.087126in}}{\pgfqpoint{7.847269in}{1.090692in}}%
\pgfpathcurveto{\pgfqpoint{7.850836in}{1.094259in}}{\pgfqpoint{7.852840in}{1.099096in}}{\pgfqpoint{7.852840in}{1.104140in}}%
\pgfpathcurveto{\pgfqpoint{7.852840in}{1.109184in}}{\pgfqpoint{7.850836in}{1.114021in}}{\pgfqpoint{7.847269in}{1.117588in}}%
\pgfpathcurveto{\pgfqpoint{7.843703in}{1.121154in}}{\pgfqpoint{7.838865in}{1.123158in}}{\pgfqpoint{7.833822in}{1.123158in}}%
\pgfpathcurveto{\pgfqpoint{7.828778in}{1.123158in}}{\pgfqpoint{7.823940in}{1.121154in}}{\pgfqpoint{7.820374in}{1.117588in}}%
\pgfpathcurveto{\pgfqpoint{7.816807in}{1.114021in}}{\pgfqpoint{7.814803in}{1.109184in}}{\pgfqpoint{7.814803in}{1.104140in}}%
\pgfpathcurveto{\pgfqpoint{7.814803in}{1.099096in}}{\pgfqpoint{7.816807in}{1.094259in}}{\pgfqpoint{7.820374in}{1.090692in}}%
\pgfpathcurveto{\pgfqpoint{7.823940in}{1.087126in}}{\pgfqpoint{7.828778in}{1.085122in}}{\pgfqpoint{7.833822in}{1.085122in}}%
\pgfpathclose%
\pgfusepath{fill}%
\end{pgfscope}%
\begin{pgfscope}%
\pgfpathrectangle{\pgfqpoint{6.572727in}{0.474100in}}{\pgfqpoint{4.227273in}{3.318700in}}%
\pgfusepath{clip}%
\pgfsetbuttcap%
\pgfsetroundjoin%
\definecolor{currentfill}{rgb}{0.127568,0.566949,0.550556}%
\pgfsetfillcolor{currentfill}%
\pgfsetfillopacity{0.700000}%
\pgfsetlinewidth{0.000000pt}%
\definecolor{currentstroke}{rgb}{0.000000,0.000000,0.000000}%
\pgfsetstrokecolor{currentstroke}%
\pgfsetstrokeopacity{0.700000}%
\pgfsetdash{}{0pt}%
\pgfpathmoveto{\pgfqpoint{8.321344in}{3.141455in}}%
\pgfpathcurveto{\pgfqpoint{8.326388in}{3.141455in}}{\pgfqpoint{8.331226in}{3.143459in}}{\pgfqpoint{8.334792in}{3.147025in}}%
\pgfpathcurveto{\pgfqpoint{8.338359in}{3.150592in}}{\pgfqpoint{8.340363in}{3.155429in}}{\pgfqpoint{8.340363in}{3.160473in}}%
\pgfpathcurveto{\pgfqpoint{8.340363in}{3.165517in}}{\pgfqpoint{8.338359in}{3.170354in}}{\pgfqpoint{8.334792in}{3.173921in}}%
\pgfpathcurveto{\pgfqpoint{8.331226in}{3.177487in}}{\pgfqpoint{8.326388in}{3.179491in}}{\pgfqpoint{8.321344in}{3.179491in}}%
\pgfpathcurveto{\pgfqpoint{8.316301in}{3.179491in}}{\pgfqpoint{8.311463in}{3.177487in}}{\pgfqpoint{8.307897in}{3.173921in}}%
\pgfpathcurveto{\pgfqpoint{8.304330in}{3.170354in}}{\pgfqpoint{8.302326in}{3.165517in}}{\pgfqpoint{8.302326in}{3.160473in}}%
\pgfpathcurveto{\pgfqpoint{8.302326in}{3.155429in}}{\pgfqpoint{8.304330in}{3.150592in}}{\pgfqpoint{8.307897in}{3.147025in}}%
\pgfpathcurveto{\pgfqpoint{8.311463in}{3.143459in}}{\pgfqpoint{8.316301in}{3.141455in}}{\pgfqpoint{8.321344in}{3.141455in}}%
\pgfpathclose%
\pgfusepath{fill}%
\end{pgfscope}%
\begin{pgfscope}%
\pgfpathrectangle{\pgfqpoint{6.572727in}{0.474100in}}{\pgfqpoint{4.227273in}{3.318700in}}%
\pgfusepath{clip}%
\pgfsetbuttcap%
\pgfsetroundjoin%
\definecolor{currentfill}{rgb}{0.127568,0.566949,0.550556}%
\pgfsetfillcolor{currentfill}%
\pgfsetfillopacity{0.700000}%
\pgfsetlinewidth{0.000000pt}%
\definecolor{currentstroke}{rgb}{0.000000,0.000000,0.000000}%
\pgfsetstrokecolor{currentstroke}%
\pgfsetstrokeopacity{0.700000}%
\pgfsetdash{}{0pt}%
\pgfpathmoveto{\pgfqpoint{8.114934in}{2.271897in}}%
\pgfpathcurveto{\pgfqpoint{8.119978in}{2.271897in}}{\pgfqpoint{8.124815in}{2.273900in}}{\pgfqpoint{8.128382in}{2.277467in}}%
\pgfpathcurveto{\pgfqpoint{8.131948in}{2.281033in}}{\pgfqpoint{8.133952in}{2.285871in}}{\pgfqpoint{8.133952in}{2.290915in}}%
\pgfpathcurveto{\pgfqpoint{8.133952in}{2.295958in}}{\pgfqpoint{8.131948in}{2.300796in}}{\pgfqpoint{8.128382in}{2.304363in}}%
\pgfpathcurveto{\pgfqpoint{8.124815in}{2.307929in}}{\pgfqpoint{8.119978in}{2.309933in}}{\pgfqpoint{8.114934in}{2.309933in}}%
\pgfpathcurveto{\pgfqpoint{8.109890in}{2.309933in}}{\pgfqpoint{8.105053in}{2.307929in}}{\pgfqpoint{8.101486in}{2.304363in}}%
\pgfpathcurveto{\pgfqpoint{8.097920in}{2.300796in}}{\pgfqpoint{8.095916in}{2.295958in}}{\pgfqpoint{8.095916in}{2.290915in}}%
\pgfpathcurveto{\pgfqpoint{8.095916in}{2.285871in}}{\pgfqpoint{8.097920in}{2.281033in}}{\pgfqpoint{8.101486in}{2.277467in}}%
\pgfpathcurveto{\pgfqpoint{8.105053in}{2.273900in}}{\pgfqpoint{8.109890in}{2.271897in}}{\pgfqpoint{8.114934in}{2.271897in}}%
\pgfpathclose%
\pgfusepath{fill}%
\end{pgfscope}%
\begin{pgfscope}%
\pgfpathrectangle{\pgfqpoint{6.572727in}{0.474100in}}{\pgfqpoint{4.227273in}{3.318700in}}%
\pgfusepath{clip}%
\pgfsetbuttcap%
\pgfsetroundjoin%
\definecolor{currentfill}{rgb}{0.993248,0.906157,0.143936}%
\pgfsetfillcolor{currentfill}%
\pgfsetfillopacity{0.700000}%
\pgfsetlinewidth{0.000000pt}%
\definecolor{currentstroke}{rgb}{0.000000,0.000000,0.000000}%
\pgfsetstrokecolor{currentstroke}%
\pgfsetstrokeopacity{0.700000}%
\pgfsetdash{}{0pt}%
\pgfpathmoveto{\pgfqpoint{9.606244in}{1.582986in}}%
\pgfpathcurveto{\pgfqpoint{9.611288in}{1.582986in}}{\pgfqpoint{9.616126in}{1.584990in}}{\pgfqpoint{9.619692in}{1.588556in}}%
\pgfpathcurveto{\pgfqpoint{9.623258in}{1.592122in}}{\pgfqpoint{9.625262in}{1.596960in}}{\pgfqpoint{9.625262in}{1.602004in}}%
\pgfpathcurveto{\pgfqpoint{9.625262in}{1.607047in}}{\pgfqpoint{9.623258in}{1.611885in}}{\pgfqpoint{9.619692in}{1.615452in}}%
\pgfpathcurveto{\pgfqpoint{9.616126in}{1.619018in}}{\pgfqpoint{9.611288in}{1.621022in}}{\pgfqpoint{9.606244in}{1.621022in}}%
\pgfpathcurveto{\pgfqpoint{9.601201in}{1.621022in}}{\pgfqpoint{9.596363in}{1.619018in}}{\pgfqpoint{9.592796in}{1.615452in}}%
\pgfpathcurveto{\pgfqpoint{9.589230in}{1.611885in}}{\pgfqpoint{9.587226in}{1.607047in}}{\pgfqpoint{9.587226in}{1.602004in}}%
\pgfpathcurveto{\pgfqpoint{9.587226in}{1.596960in}}{\pgfqpoint{9.589230in}{1.592122in}}{\pgfqpoint{9.592796in}{1.588556in}}%
\pgfpathcurveto{\pgfqpoint{9.596363in}{1.584990in}}{\pgfqpoint{9.601201in}{1.582986in}}{\pgfqpoint{9.606244in}{1.582986in}}%
\pgfpathclose%
\pgfusepath{fill}%
\end{pgfscope}%
\begin{pgfscope}%
\pgfpathrectangle{\pgfqpoint{6.572727in}{0.474100in}}{\pgfqpoint{4.227273in}{3.318700in}}%
\pgfusepath{clip}%
\pgfsetbuttcap%
\pgfsetroundjoin%
\definecolor{currentfill}{rgb}{0.127568,0.566949,0.550556}%
\pgfsetfillcolor{currentfill}%
\pgfsetfillopacity{0.700000}%
\pgfsetlinewidth{0.000000pt}%
\definecolor{currentstroke}{rgb}{0.000000,0.000000,0.000000}%
\pgfsetstrokecolor{currentstroke}%
\pgfsetstrokeopacity{0.700000}%
\pgfsetdash{}{0pt}%
\pgfpathmoveto{\pgfqpoint{7.802058in}{2.341009in}}%
\pgfpathcurveto{\pgfqpoint{7.807101in}{2.341009in}}{\pgfqpoint{7.811939in}{2.343012in}}{\pgfqpoint{7.815505in}{2.346579in}}%
\pgfpathcurveto{\pgfqpoint{7.819072in}{2.350145in}}{\pgfqpoint{7.821076in}{2.354983in}}{\pgfqpoint{7.821076in}{2.360027in}}%
\pgfpathcurveto{\pgfqpoint{7.821076in}{2.365070in}}{\pgfqpoint{7.819072in}{2.369908in}}{\pgfqpoint{7.815505in}{2.373475in}}%
\pgfpathcurveto{\pgfqpoint{7.811939in}{2.377041in}}{\pgfqpoint{7.807101in}{2.379045in}}{\pgfqpoint{7.802058in}{2.379045in}}%
\pgfpathcurveto{\pgfqpoint{7.797014in}{2.379045in}}{\pgfqpoint{7.792176in}{2.377041in}}{\pgfqpoint{7.788610in}{2.373475in}}%
\pgfpathcurveto{\pgfqpoint{7.785043in}{2.369908in}}{\pgfqpoint{7.783039in}{2.365070in}}{\pgfqpoint{7.783039in}{2.360027in}}%
\pgfpathcurveto{\pgfqpoint{7.783039in}{2.354983in}}{\pgfqpoint{7.785043in}{2.350145in}}{\pgfqpoint{7.788610in}{2.346579in}}%
\pgfpathcurveto{\pgfqpoint{7.792176in}{2.343012in}}{\pgfqpoint{7.797014in}{2.341009in}}{\pgfqpoint{7.802058in}{2.341009in}}%
\pgfpathclose%
\pgfusepath{fill}%
\end{pgfscope}%
\begin{pgfscope}%
\pgfpathrectangle{\pgfqpoint{6.572727in}{0.474100in}}{\pgfqpoint{4.227273in}{3.318700in}}%
\pgfusepath{clip}%
\pgfsetbuttcap%
\pgfsetroundjoin%
\definecolor{currentfill}{rgb}{0.993248,0.906157,0.143936}%
\pgfsetfillcolor{currentfill}%
\pgfsetfillopacity{0.700000}%
\pgfsetlinewidth{0.000000pt}%
\definecolor{currentstroke}{rgb}{0.000000,0.000000,0.000000}%
\pgfsetstrokecolor{currentstroke}%
\pgfsetstrokeopacity{0.700000}%
\pgfsetdash{}{0pt}%
\pgfpathmoveto{\pgfqpoint{9.261556in}{1.255216in}}%
\pgfpathcurveto{\pgfqpoint{9.266599in}{1.255216in}}{\pgfqpoint{9.271437in}{1.257220in}}{\pgfqpoint{9.275003in}{1.260786in}}%
\pgfpathcurveto{\pgfqpoint{9.278570in}{1.264353in}}{\pgfqpoint{9.280574in}{1.269190in}}{\pgfqpoint{9.280574in}{1.274234in}}%
\pgfpathcurveto{\pgfqpoint{9.280574in}{1.279278in}}{\pgfqpoint{9.278570in}{1.284115in}}{\pgfqpoint{9.275003in}{1.287682in}}%
\pgfpathcurveto{\pgfqpoint{9.271437in}{1.291248in}}{\pgfqpoint{9.266599in}{1.293252in}}{\pgfqpoint{9.261556in}{1.293252in}}%
\pgfpathcurveto{\pgfqpoint{9.256512in}{1.293252in}}{\pgfqpoint{9.251674in}{1.291248in}}{\pgfqpoint{9.248108in}{1.287682in}}%
\pgfpathcurveto{\pgfqpoint{9.244541in}{1.284115in}}{\pgfqpoint{9.242537in}{1.279278in}}{\pgfqpoint{9.242537in}{1.274234in}}%
\pgfpathcurveto{\pgfqpoint{9.242537in}{1.269190in}}{\pgfqpoint{9.244541in}{1.264353in}}{\pgfqpoint{9.248108in}{1.260786in}}%
\pgfpathcurveto{\pgfqpoint{9.251674in}{1.257220in}}{\pgfqpoint{9.256512in}{1.255216in}}{\pgfqpoint{9.261556in}{1.255216in}}%
\pgfpathclose%
\pgfusepath{fill}%
\end{pgfscope}%
\begin{pgfscope}%
\pgfpathrectangle{\pgfqpoint{6.572727in}{0.474100in}}{\pgfqpoint{4.227273in}{3.318700in}}%
\pgfusepath{clip}%
\pgfsetbuttcap%
\pgfsetroundjoin%
\definecolor{currentfill}{rgb}{0.127568,0.566949,0.550556}%
\pgfsetfillcolor{currentfill}%
\pgfsetfillopacity{0.700000}%
\pgfsetlinewidth{0.000000pt}%
\definecolor{currentstroke}{rgb}{0.000000,0.000000,0.000000}%
\pgfsetstrokecolor{currentstroke}%
\pgfsetstrokeopacity{0.700000}%
\pgfsetdash{}{0pt}%
\pgfpathmoveto{\pgfqpoint{8.208631in}{2.591468in}}%
\pgfpathcurveto{\pgfqpoint{8.213675in}{2.591468in}}{\pgfqpoint{8.218513in}{2.593472in}}{\pgfqpoint{8.222079in}{2.597039in}}%
\pgfpathcurveto{\pgfqpoint{8.225645in}{2.600605in}}{\pgfqpoint{8.227649in}{2.605443in}}{\pgfqpoint{8.227649in}{2.610486in}}%
\pgfpathcurveto{\pgfqpoint{8.227649in}{2.615530in}}{\pgfqpoint{8.225645in}{2.620368in}}{\pgfqpoint{8.222079in}{2.623934in}}%
\pgfpathcurveto{\pgfqpoint{8.218513in}{2.627501in}}{\pgfqpoint{8.213675in}{2.629505in}}{\pgfqpoint{8.208631in}{2.629505in}}%
\pgfpathcurveto{\pgfqpoint{8.203588in}{2.629505in}}{\pgfqpoint{8.198750in}{2.627501in}}{\pgfqpoint{8.195183in}{2.623934in}}%
\pgfpathcurveto{\pgfqpoint{8.191617in}{2.620368in}}{\pgfqpoint{8.189613in}{2.615530in}}{\pgfqpoint{8.189613in}{2.610486in}}%
\pgfpathcurveto{\pgfqpoint{8.189613in}{2.605443in}}{\pgfqpoint{8.191617in}{2.600605in}}{\pgfqpoint{8.195183in}{2.597039in}}%
\pgfpathcurveto{\pgfqpoint{8.198750in}{2.593472in}}{\pgfqpoint{8.203588in}{2.591468in}}{\pgfqpoint{8.208631in}{2.591468in}}%
\pgfpathclose%
\pgfusepath{fill}%
\end{pgfscope}%
\begin{pgfscope}%
\pgfpathrectangle{\pgfqpoint{6.572727in}{0.474100in}}{\pgfqpoint{4.227273in}{3.318700in}}%
\pgfusepath{clip}%
\pgfsetbuttcap%
\pgfsetroundjoin%
\definecolor{currentfill}{rgb}{0.127568,0.566949,0.550556}%
\pgfsetfillcolor{currentfill}%
\pgfsetfillopacity{0.700000}%
\pgfsetlinewidth{0.000000pt}%
\definecolor{currentstroke}{rgb}{0.000000,0.000000,0.000000}%
\pgfsetstrokecolor{currentstroke}%
\pgfsetstrokeopacity{0.700000}%
\pgfsetdash{}{0pt}%
\pgfpathmoveto{\pgfqpoint{7.589720in}{1.922219in}}%
\pgfpathcurveto{\pgfqpoint{7.594764in}{1.922219in}}{\pgfqpoint{7.599601in}{1.924223in}}{\pgfqpoint{7.603168in}{1.927789in}}%
\pgfpathcurveto{\pgfqpoint{7.606734in}{1.931355in}}{\pgfqpoint{7.608738in}{1.936193in}}{\pgfqpoint{7.608738in}{1.941237in}}%
\pgfpathcurveto{\pgfqpoint{7.608738in}{1.946280in}}{\pgfqpoint{7.606734in}{1.951118in}}{\pgfqpoint{7.603168in}{1.954685in}}%
\pgfpathcurveto{\pgfqpoint{7.599601in}{1.958251in}}{\pgfqpoint{7.594764in}{1.960255in}}{\pgfqpoint{7.589720in}{1.960255in}}%
\pgfpathcurveto{\pgfqpoint{7.584676in}{1.960255in}}{\pgfqpoint{7.579838in}{1.958251in}}{\pgfqpoint{7.576272in}{1.954685in}}%
\pgfpathcurveto{\pgfqpoint{7.572706in}{1.951118in}}{\pgfqpoint{7.570702in}{1.946280in}}{\pgfqpoint{7.570702in}{1.941237in}}%
\pgfpathcurveto{\pgfqpoint{7.570702in}{1.936193in}}{\pgfqpoint{7.572706in}{1.931355in}}{\pgfqpoint{7.576272in}{1.927789in}}%
\pgfpathcurveto{\pgfqpoint{7.579838in}{1.924223in}}{\pgfqpoint{7.584676in}{1.922219in}}{\pgfqpoint{7.589720in}{1.922219in}}%
\pgfpathclose%
\pgfusepath{fill}%
\end{pgfscope}%
\begin{pgfscope}%
\pgfpathrectangle{\pgfqpoint{6.572727in}{0.474100in}}{\pgfqpoint{4.227273in}{3.318700in}}%
\pgfusepath{clip}%
\pgfsetbuttcap%
\pgfsetroundjoin%
\definecolor{currentfill}{rgb}{0.127568,0.566949,0.550556}%
\pgfsetfillcolor{currentfill}%
\pgfsetfillopacity{0.700000}%
\pgfsetlinewidth{0.000000pt}%
\definecolor{currentstroke}{rgb}{0.000000,0.000000,0.000000}%
\pgfsetstrokecolor{currentstroke}%
\pgfsetstrokeopacity{0.700000}%
\pgfsetdash{}{0pt}%
\pgfpathmoveto{\pgfqpoint{7.919911in}{1.991658in}}%
\pgfpathcurveto{\pgfqpoint{7.924954in}{1.991658in}}{\pgfqpoint{7.929792in}{1.993662in}}{\pgfqpoint{7.933358in}{1.997228in}}%
\pgfpathcurveto{\pgfqpoint{7.936925in}{2.000795in}}{\pgfqpoint{7.938929in}{2.005633in}}{\pgfqpoint{7.938929in}{2.010676in}}%
\pgfpathcurveto{\pgfqpoint{7.938929in}{2.015720in}}{\pgfqpoint{7.936925in}{2.020558in}}{\pgfqpoint{7.933358in}{2.024124in}}%
\pgfpathcurveto{\pgfqpoint{7.929792in}{2.027690in}}{\pgfqpoint{7.924954in}{2.029694in}}{\pgfqpoint{7.919911in}{2.029694in}}%
\pgfpathcurveto{\pgfqpoint{7.914867in}{2.029694in}}{\pgfqpoint{7.910029in}{2.027690in}}{\pgfqpoint{7.906463in}{2.024124in}}%
\pgfpathcurveto{\pgfqpoint{7.902896in}{2.020558in}}{\pgfqpoint{7.900892in}{2.015720in}}{\pgfqpoint{7.900892in}{2.010676in}}%
\pgfpathcurveto{\pgfqpoint{7.900892in}{2.005633in}}{\pgfqpoint{7.902896in}{2.000795in}}{\pgfqpoint{7.906463in}{1.997228in}}%
\pgfpathcurveto{\pgfqpoint{7.910029in}{1.993662in}}{\pgfqpoint{7.914867in}{1.991658in}}{\pgfqpoint{7.919911in}{1.991658in}}%
\pgfpathclose%
\pgfusepath{fill}%
\end{pgfscope}%
\begin{pgfscope}%
\pgfpathrectangle{\pgfqpoint{6.572727in}{0.474100in}}{\pgfqpoint{4.227273in}{3.318700in}}%
\pgfusepath{clip}%
\pgfsetbuttcap%
\pgfsetroundjoin%
\definecolor{currentfill}{rgb}{0.993248,0.906157,0.143936}%
\pgfsetfillcolor{currentfill}%
\pgfsetfillopacity{0.700000}%
\pgfsetlinewidth{0.000000pt}%
\definecolor{currentstroke}{rgb}{0.000000,0.000000,0.000000}%
\pgfsetstrokecolor{currentstroke}%
\pgfsetstrokeopacity{0.700000}%
\pgfsetdash{}{0pt}%
\pgfpathmoveto{\pgfqpoint{9.923362in}{1.636330in}}%
\pgfpathcurveto{\pgfqpoint{9.928405in}{1.636330in}}{\pgfqpoint{9.933243in}{1.638334in}}{\pgfqpoint{9.936809in}{1.641900in}}%
\pgfpathcurveto{\pgfqpoint{9.940376in}{1.645467in}}{\pgfqpoint{9.942380in}{1.650305in}}{\pgfqpoint{9.942380in}{1.655348in}}%
\pgfpathcurveto{\pgfqpoint{9.942380in}{1.660392in}}{\pgfqpoint{9.940376in}{1.665230in}}{\pgfqpoint{9.936809in}{1.668796in}}%
\pgfpathcurveto{\pgfqpoint{9.933243in}{1.672362in}}{\pgfqpoint{9.928405in}{1.674366in}}{\pgfqpoint{9.923362in}{1.674366in}}%
\pgfpathcurveto{\pgfqpoint{9.918318in}{1.674366in}}{\pgfqpoint{9.913480in}{1.672362in}}{\pgfqpoint{9.909914in}{1.668796in}}%
\pgfpathcurveto{\pgfqpoint{9.906347in}{1.665230in}}{\pgfqpoint{9.904343in}{1.660392in}}{\pgfqpoint{9.904343in}{1.655348in}}%
\pgfpathcurveto{\pgfqpoint{9.904343in}{1.650305in}}{\pgfqpoint{9.906347in}{1.645467in}}{\pgfqpoint{9.909914in}{1.641900in}}%
\pgfpathcurveto{\pgfqpoint{9.913480in}{1.638334in}}{\pgfqpoint{9.918318in}{1.636330in}}{\pgfqpoint{9.923362in}{1.636330in}}%
\pgfpathclose%
\pgfusepath{fill}%
\end{pgfscope}%
\begin{pgfscope}%
\pgfpathrectangle{\pgfqpoint{6.572727in}{0.474100in}}{\pgfqpoint{4.227273in}{3.318700in}}%
\pgfusepath{clip}%
\pgfsetbuttcap%
\pgfsetroundjoin%
\definecolor{currentfill}{rgb}{0.127568,0.566949,0.550556}%
\pgfsetfillcolor{currentfill}%
\pgfsetfillopacity{0.700000}%
\pgfsetlinewidth{0.000000pt}%
\definecolor{currentstroke}{rgb}{0.000000,0.000000,0.000000}%
\pgfsetstrokecolor{currentstroke}%
\pgfsetstrokeopacity{0.700000}%
\pgfsetdash{}{0pt}%
\pgfpathmoveto{\pgfqpoint{8.561788in}{1.923436in}}%
\pgfpathcurveto{\pgfqpoint{8.566831in}{1.923436in}}{\pgfqpoint{8.571669in}{1.925440in}}{\pgfqpoint{8.575236in}{1.929007in}}%
\pgfpathcurveto{\pgfqpoint{8.578802in}{1.932573in}}{\pgfqpoint{8.580806in}{1.937411in}}{\pgfqpoint{8.580806in}{1.942454in}}%
\pgfpathcurveto{\pgfqpoint{8.580806in}{1.947498in}}{\pgfqpoint{8.578802in}{1.952336in}}{\pgfqpoint{8.575236in}{1.955902in}}%
\pgfpathcurveto{\pgfqpoint{8.571669in}{1.959469in}}{\pgfqpoint{8.566831in}{1.961473in}}{\pgfqpoint{8.561788in}{1.961473in}}%
\pgfpathcurveto{\pgfqpoint{8.556744in}{1.961473in}}{\pgfqpoint{8.551906in}{1.959469in}}{\pgfqpoint{8.548340in}{1.955902in}}%
\pgfpathcurveto{\pgfqpoint{8.544773in}{1.952336in}}{\pgfqpoint{8.542770in}{1.947498in}}{\pgfqpoint{8.542770in}{1.942454in}}%
\pgfpathcurveto{\pgfqpoint{8.542770in}{1.937411in}}{\pgfqpoint{8.544773in}{1.932573in}}{\pgfqpoint{8.548340in}{1.929007in}}%
\pgfpathcurveto{\pgfqpoint{8.551906in}{1.925440in}}{\pgfqpoint{8.556744in}{1.923436in}}{\pgfqpoint{8.561788in}{1.923436in}}%
\pgfpathclose%
\pgfusepath{fill}%
\end{pgfscope}%
\begin{pgfscope}%
\pgfpathrectangle{\pgfqpoint{6.572727in}{0.474100in}}{\pgfqpoint{4.227273in}{3.318700in}}%
\pgfusepath{clip}%
\pgfsetbuttcap%
\pgfsetroundjoin%
\definecolor{currentfill}{rgb}{0.127568,0.566949,0.550556}%
\pgfsetfillcolor{currentfill}%
\pgfsetfillopacity{0.700000}%
\pgfsetlinewidth{0.000000pt}%
\definecolor{currentstroke}{rgb}{0.000000,0.000000,0.000000}%
\pgfsetstrokecolor{currentstroke}%
\pgfsetstrokeopacity{0.700000}%
\pgfsetdash{}{0pt}%
\pgfpathmoveto{\pgfqpoint{8.647684in}{2.735414in}}%
\pgfpathcurveto{\pgfqpoint{8.652728in}{2.735414in}}{\pgfqpoint{8.657566in}{2.737418in}}{\pgfqpoint{8.661132in}{2.740985in}}%
\pgfpathcurveto{\pgfqpoint{8.664698in}{2.744551in}}{\pgfqpoint{8.666702in}{2.749389in}}{\pgfqpoint{8.666702in}{2.754433in}}%
\pgfpathcurveto{\pgfqpoint{8.666702in}{2.759476in}}{\pgfqpoint{8.664698in}{2.764314in}}{\pgfqpoint{8.661132in}{2.767880in}}%
\pgfpathcurveto{\pgfqpoint{8.657566in}{2.771447in}}{\pgfqpoint{8.652728in}{2.773451in}}{\pgfqpoint{8.647684in}{2.773451in}}%
\pgfpathcurveto{\pgfqpoint{8.642640in}{2.773451in}}{\pgfqpoint{8.637803in}{2.771447in}}{\pgfqpoint{8.634236in}{2.767880in}}%
\pgfpathcurveto{\pgfqpoint{8.630670in}{2.764314in}}{\pgfqpoint{8.628666in}{2.759476in}}{\pgfqpoint{8.628666in}{2.754433in}}%
\pgfpathcurveto{\pgfqpoint{8.628666in}{2.749389in}}{\pgfqpoint{8.630670in}{2.744551in}}{\pgfqpoint{8.634236in}{2.740985in}}%
\pgfpathcurveto{\pgfqpoint{8.637803in}{2.737418in}}{\pgfqpoint{8.642640in}{2.735414in}}{\pgfqpoint{8.647684in}{2.735414in}}%
\pgfpathclose%
\pgfusepath{fill}%
\end{pgfscope}%
\begin{pgfscope}%
\pgfpathrectangle{\pgfqpoint{6.572727in}{0.474100in}}{\pgfqpoint{4.227273in}{3.318700in}}%
\pgfusepath{clip}%
\pgfsetbuttcap%
\pgfsetroundjoin%
\definecolor{currentfill}{rgb}{0.127568,0.566949,0.550556}%
\pgfsetfillcolor{currentfill}%
\pgfsetfillopacity{0.700000}%
\pgfsetlinewidth{0.000000pt}%
\definecolor{currentstroke}{rgb}{0.000000,0.000000,0.000000}%
\pgfsetstrokecolor{currentstroke}%
\pgfsetstrokeopacity{0.700000}%
\pgfsetdash{}{0pt}%
\pgfpathmoveto{\pgfqpoint{8.316492in}{2.216537in}}%
\pgfpathcurveto{\pgfqpoint{8.321535in}{2.216537in}}{\pgfqpoint{8.326373in}{2.218541in}}{\pgfqpoint{8.329940in}{2.222107in}}%
\pgfpathcurveto{\pgfqpoint{8.333506in}{2.225673in}}{\pgfqpoint{8.335510in}{2.230511in}}{\pgfqpoint{8.335510in}{2.235555in}}%
\pgfpathcurveto{\pgfqpoint{8.335510in}{2.240599in}}{\pgfqpoint{8.333506in}{2.245436in}}{\pgfqpoint{8.329940in}{2.249003in}}%
\pgfpathcurveto{\pgfqpoint{8.326373in}{2.252569in}}{\pgfqpoint{8.321535in}{2.254573in}}{\pgfqpoint{8.316492in}{2.254573in}}%
\pgfpathcurveto{\pgfqpoint{8.311448in}{2.254573in}}{\pgfqpoint{8.306610in}{2.252569in}}{\pgfqpoint{8.303044in}{2.249003in}}%
\pgfpathcurveto{\pgfqpoint{8.299478in}{2.245436in}}{\pgfqpoint{8.297474in}{2.240599in}}{\pgfqpoint{8.297474in}{2.235555in}}%
\pgfpathcurveto{\pgfqpoint{8.297474in}{2.230511in}}{\pgfqpoint{8.299478in}{2.225673in}}{\pgfqpoint{8.303044in}{2.222107in}}%
\pgfpathcurveto{\pgfqpoint{8.306610in}{2.218541in}}{\pgfqpoint{8.311448in}{2.216537in}}{\pgfqpoint{8.316492in}{2.216537in}}%
\pgfpathclose%
\pgfusepath{fill}%
\end{pgfscope}%
\begin{pgfscope}%
\pgfpathrectangle{\pgfqpoint{6.572727in}{0.474100in}}{\pgfqpoint{4.227273in}{3.318700in}}%
\pgfusepath{clip}%
\pgfsetbuttcap%
\pgfsetroundjoin%
\definecolor{currentfill}{rgb}{0.127568,0.566949,0.550556}%
\pgfsetfillcolor{currentfill}%
\pgfsetfillopacity{0.700000}%
\pgfsetlinewidth{0.000000pt}%
\definecolor{currentstroke}{rgb}{0.000000,0.000000,0.000000}%
\pgfsetstrokecolor{currentstroke}%
\pgfsetstrokeopacity{0.700000}%
\pgfsetdash{}{0pt}%
\pgfpathmoveto{\pgfqpoint{8.881533in}{3.043491in}}%
\pgfpathcurveto{\pgfqpoint{8.886576in}{3.043491in}}{\pgfqpoint{8.891414in}{3.045495in}}{\pgfqpoint{8.894980in}{3.049061in}}%
\pgfpathcurveto{\pgfqpoint{8.898547in}{3.052628in}}{\pgfqpoint{8.900551in}{3.057466in}}{\pgfqpoint{8.900551in}{3.062509in}}%
\pgfpathcurveto{\pgfqpoint{8.900551in}{3.067553in}}{\pgfqpoint{8.898547in}{3.072391in}}{\pgfqpoint{8.894980in}{3.075957in}}%
\pgfpathcurveto{\pgfqpoint{8.891414in}{3.079524in}}{\pgfqpoint{8.886576in}{3.081527in}}{\pgfqpoint{8.881533in}{3.081527in}}%
\pgfpathcurveto{\pgfqpoint{8.876489in}{3.081527in}}{\pgfqpoint{8.871651in}{3.079524in}}{\pgfqpoint{8.868085in}{3.075957in}}%
\pgfpathcurveto{\pgfqpoint{8.864518in}{3.072391in}}{\pgfqpoint{8.862514in}{3.067553in}}{\pgfqpoint{8.862514in}{3.062509in}}%
\pgfpathcurveto{\pgfqpoint{8.862514in}{3.057466in}}{\pgfqpoint{8.864518in}{3.052628in}}{\pgfqpoint{8.868085in}{3.049061in}}%
\pgfpathcurveto{\pgfqpoint{8.871651in}{3.045495in}}{\pgfqpoint{8.876489in}{3.043491in}}{\pgfqpoint{8.881533in}{3.043491in}}%
\pgfpathclose%
\pgfusepath{fill}%
\end{pgfscope}%
\begin{pgfscope}%
\pgfpathrectangle{\pgfqpoint{6.572727in}{0.474100in}}{\pgfqpoint{4.227273in}{3.318700in}}%
\pgfusepath{clip}%
\pgfsetbuttcap%
\pgfsetroundjoin%
\definecolor{currentfill}{rgb}{0.127568,0.566949,0.550556}%
\pgfsetfillcolor{currentfill}%
\pgfsetfillopacity{0.700000}%
\pgfsetlinewidth{0.000000pt}%
\definecolor{currentstroke}{rgb}{0.000000,0.000000,0.000000}%
\pgfsetstrokecolor{currentstroke}%
\pgfsetstrokeopacity{0.700000}%
\pgfsetdash{}{0pt}%
\pgfpathmoveto{\pgfqpoint{7.455704in}{1.996569in}}%
\pgfpathcurveto{\pgfqpoint{7.460748in}{1.996569in}}{\pgfqpoint{7.465586in}{1.998573in}}{\pgfqpoint{7.469152in}{2.002139in}}%
\pgfpathcurveto{\pgfqpoint{7.472718in}{2.005706in}}{\pgfqpoint{7.474722in}{2.010544in}}{\pgfqpoint{7.474722in}{2.015587in}}%
\pgfpathcurveto{\pgfqpoint{7.474722in}{2.020631in}}{\pgfqpoint{7.472718in}{2.025469in}}{\pgfqpoint{7.469152in}{2.029035in}}%
\pgfpathcurveto{\pgfqpoint{7.465586in}{2.032601in}}{\pgfqpoint{7.460748in}{2.034605in}}{\pgfqpoint{7.455704in}{2.034605in}}%
\pgfpathcurveto{\pgfqpoint{7.450661in}{2.034605in}}{\pgfqpoint{7.445823in}{2.032601in}}{\pgfqpoint{7.442256in}{2.029035in}}%
\pgfpathcurveto{\pgfqpoint{7.438690in}{2.025469in}}{\pgfqpoint{7.436686in}{2.020631in}}{\pgfqpoint{7.436686in}{2.015587in}}%
\pgfpathcurveto{\pgfqpoint{7.436686in}{2.010544in}}{\pgfqpoint{7.438690in}{2.005706in}}{\pgfqpoint{7.442256in}{2.002139in}}%
\pgfpathcurveto{\pgfqpoint{7.445823in}{1.998573in}}{\pgfqpoint{7.450661in}{1.996569in}}{\pgfqpoint{7.455704in}{1.996569in}}%
\pgfpathclose%
\pgfusepath{fill}%
\end{pgfscope}%
\begin{pgfscope}%
\pgfpathrectangle{\pgfqpoint{6.572727in}{0.474100in}}{\pgfqpoint{4.227273in}{3.318700in}}%
\pgfusepath{clip}%
\pgfsetbuttcap%
\pgfsetroundjoin%
\definecolor{currentfill}{rgb}{0.127568,0.566949,0.550556}%
\pgfsetfillcolor{currentfill}%
\pgfsetfillopacity{0.700000}%
\pgfsetlinewidth{0.000000pt}%
\definecolor{currentstroke}{rgb}{0.000000,0.000000,0.000000}%
\pgfsetstrokecolor{currentstroke}%
\pgfsetstrokeopacity{0.700000}%
\pgfsetdash{}{0pt}%
\pgfpathmoveto{\pgfqpoint{8.261768in}{1.763173in}}%
\pgfpathcurveto{\pgfqpoint{8.266812in}{1.763173in}}{\pgfqpoint{8.271650in}{1.765177in}}{\pgfqpoint{8.275216in}{1.768743in}}%
\pgfpathcurveto{\pgfqpoint{8.278783in}{1.772310in}}{\pgfqpoint{8.280786in}{1.777148in}}{\pgfqpoint{8.280786in}{1.782191in}}%
\pgfpathcurveto{\pgfqpoint{8.280786in}{1.787235in}}{\pgfqpoint{8.278783in}{1.792073in}}{\pgfqpoint{8.275216in}{1.795639in}}%
\pgfpathcurveto{\pgfqpoint{8.271650in}{1.799205in}}{\pgfqpoint{8.266812in}{1.801209in}}{\pgfqpoint{8.261768in}{1.801209in}}%
\pgfpathcurveto{\pgfqpoint{8.256725in}{1.801209in}}{\pgfqpoint{8.251887in}{1.799205in}}{\pgfqpoint{8.248320in}{1.795639in}}%
\pgfpathcurveto{\pgfqpoint{8.244754in}{1.792073in}}{\pgfqpoint{8.242750in}{1.787235in}}{\pgfqpoint{8.242750in}{1.782191in}}%
\pgfpathcurveto{\pgfqpoint{8.242750in}{1.777148in}}{\pgfqpoint{8.244754in}{1.772310in}}{\pgfqpoint{8.248320in}{1.768743in}}%
\pgfpathcurveto{\pgfqpoint{8.251887in}{1.765177in}}{\pgfqpoint{8.256725in}{1.763173in}}{\pgfqpoint{8.261768in}{1.763173in}}%
\pgfpathclose%
\pgfusepath{fill}%
\end{pgfscope}%
\begin{pgfscope}%
\pgfpathrectangle{\pgfqpoint{6.572727in}{0.474100in}}{\pgfqpoint{4.227273in}{3.318700in}}%
\pgfusepath{clip}%
\pgfsetbuttcap%
\pgfsetroundjoin%
\definecolor{currentfill}{rgb}{0.993248,0.906157,0.143936}%
\pgfsetfillcolor{currentfill}%
\pgfsetfillopacity{0.700000}%
\pgfsetlinewidth{0.000000pt}%
\definecolor{currentstroke}{rgb}{0.000000,0.000000,0.000000}%
\pgfsetstrokecolor{currentstroke}%
\pgfsetstrokeopacity{0.700000}%
\pgfsetdash{}{0pt}%
\pgfpathmoveto{\pgfqpoint{9.843072in}{1.470502in}}%
\pgfpathcurveto{\pgfqpoint{9.848116in}{1.470502in}}{\pgfqpoint{9.852954in}{1.472506in}}{\pgfqpoint{9.856520in}{1.476072in}}%
\pgfpathcurveto{\pgfqpoint{9.860086in}{1.479639in}}{\pgfqpoint{9.862090in}{1.484476in}}{\pgfqpoint{9.862090in}{1.489520in}}%
\pgfpathcurveto{\pgfqpoint{9.862090in}{1.494564in}}{\pgfqpoint{9.860086in}{1.499401in}}{\pgfqpoint{9.856520in}{1.502968in}}%
\pgfpathcurveto{\pgfqpoint{9.852954in}{1.506534in}}{\pgfqpoint{9.848116in}{1.508538in}}{\pgfqpoint{9.843072in}{1.508538in}}%
\pgfpathcurveto{\pgfqpoint{9.838029in}{1.508538in}}{\pgfqpoint{9.833191in}{1.506534in}}{\pgfqpoint{9.829624in}{1.502968in}}%
\pgfpathcurveto{\pgfqpoint{9.826058in}{1.499401in}}{\pgfqpoint{9.824054in}{1.494564in}}{\pgfqpoint{9.824054in}{1.489520in}}%
\pgfpathcurveto{\pgfqpoint{9.824054in}{1.484476in}}{\pgfqpoint{9.826058in}{1.479639in}}{\pgfqpoint{9.829624in}{1.476072in}}%
\pgfpathcurveto{\pgfqpoint{9.833191in}{1.472506in}}{\pgfqpoint{9.838029in}{1.470502in}}{\pgfqpoint{9.843072in}{1.470502in}}%
\pgfpathclose%
\pgfusepath{fill}%
\end{pgfscope}%
\begin{pgfscope}%
\pgfpathrectangle{\pgfqpoint{6.572727in}{0.474100in}}{\pgfqpoint{4.227273in}{3.318700in}}%
\pgfusepath{clip}%
\pgfsetbuttcap%
\pgfsetroundjoin%
\definecolor{currentfill}{rgb}{0.127568,0.566949,0.550556}%
\pgfsetfillcolor{currentfill}%
\pgfsetfillopacity{0.700000}%
\pgfsetlinewidth{0.000000pt}%
\definecolor{currentstroke}{rgb}{0.000000,0.000000,0.000000}%
\pgfsetstrokecolor{currentstroke}%
\pgfsetstrokeopacity{0.700000}%
\pgfsetdash{}{0pt}%
\pgfpathmoveto{\pgfqpoint{8.493389in}{2.685854in}}%
\pgfpathcurveto{\pgfqpoint{8.498432in}{2.685854in}}{\pgfqpoint{8.503270in}{2.687858in}}{\pgfqpoint{8.506836in}{2.691425in}}%
\pgfpathcurveto{\pgfqpoint{8.510403in}{2.694991in}}{\pgfqpoint{8.512407in}{2.699829in}}{\pgfqpoint{8.512407in}{2.704873in}}%
\pgfpathcurveto{\pgfqpoint{8.512407in}{2.709916in}}{\pgfqpoint{8.510403in}{2.714754in}}{\pgfqpoint{8.506836in}{2.718320in}}%
\pgfpathcurveto{\pgfqpoint{8.503270in}{2.721887in}}{\pgfqpoint{8.498432in}{2.723891in}}{\pgfqpoint{8.493389in}{2.723891in}}%
\pgfpathcurveto{\pgfqpoint{8.488345in}{2.723891in}}{\pgfqpoint{8.483507in}{2.721887in}}{\pgfqpoint{8.479941in}{2.718320in}}%
\pgfpathcurveto{\pgfqpoint{8.476374in}{2.714754in}}{\pgfqpoint{8.474370in}{2.709916in}}{\pgfqpoint{8.474370in}{2.704873in}}%
\pgfpathcurveto{\pgfqpoint{8.474370in}{2.699829in}}{\pgfqpoint{8.476374in}{2.694991in}}{\pgfqpoint{8.479941in}{2.691425in}}%
\pgfpathcurveto{\pgfqpoint{8.483507in}{2.687858in}}{\pgfqpoint{8.488345in}{2.685854in}}{\pgfqpoint{8.493389in}{2.685854in}}%
\pgfpathclose%
\pgfusepath{fill}%
\end{pgfscope}%
\begin{pgfscope}%
\pgfpathrectangle{\pgfqpoint{6.572727in}{0.474100in}}{\pgfqpoint{4.227273in}{3.318700in}}%
\pgfusepath{clip}%
\pgfsetbuttcap%
\pgfsetroundjoin%
\definecolor{currentfill}{rgb}{0.127568,0.566949,0.550556}%
\pgfsetfillcolor{currentfill}%
\pgfsetfillopacity{0.700000}%
\pgfsetlinewidth{0.000000pt}%
\definecolor{currentstroke}{rgb}{0.000000,0.000000,0.000000}%
\pgfsetstrokecolor{currentstroke}%
\pgfsetstrokeopacity{0.700000}%
\pgfsetdash{}{0pt}%
\pgfpathmoveto{\pgfqpoint{7.884330in}{1.732476in}}%
\pgfpathcurveto{\pgfqpoint{7.889374in}{1.732476in}}{\pgfqpoint{7.894212in}{1.734480in}}{\pgfqpoint{7.897778in}{1.738046in}}%
\pgfpathcurveto{\pgfqpoint{7.901345in}{1.741613in}}{\pgfqpoint{7.903349in}{1.746451in}}{\pgfqpoint{7.903349in}{1.751494in}}%
\pgfpathcurveto{\pgfqpoint{7.903349in}{1.756538in}}{\pgfqpoint{7.901345in}{1.761376in}}{\pgfqpoint{7.897778in}{1.764942in}}%
\pgfpathcurveto{\pgfqpoint{7.894212in}{1.768509in}}{\pgfqpoint{7.889374in}{1.770512in}}{\pgfqpoint{7.884330in}{1.770512in}}%
\pgfpathcurveto{\pgfqpoint{7.879287in}{1.770512in}}{\pgfqpoint{7.874449in}{1.768509in}}{\pgfqpoint{7.870883in}{1.764942in}}%
\pgfpathcurveto{\pgfqpoint{7.867316in}{1.761376in}}{\pgfqpoint{7.865312in}{1.756538in}}{\pgfqpoint{7.865312in}{1.751494in}}%
\pgfpathcurveto{\pgfqpoint{7.865312in}{1.746451in}}{\pgfqpoint{7.867316in}{1.741613in}}{\pgfqpoint{7.870883in}{1.738046in}}%
\pgfpathcurveto{\pgfqpoint{7.874449in}{1.734480in}}{\pgfqpoint{7.879287in}{1.732476in}}{\pgfqpoint{7.884330in}{1.732476in}}%
\pgfpathclose%
\pgfusepath{fill}%
\end{pgfscope}%
\begin{pgfscope}%
\pgfpathrectangle{\pgfqpoint{6.572727in}{0.474100in}}{\pgfqpoint{4.227273in}{3.318700in}}%
\pgfusepath{clip}%
\pgfsetbuttcap%
\pgfsetroundjoin%
\definecolor{currentfill}{rgb}{0.127568,0.566949,0.550556}%
\pgfsetfillcolor{currentfill}%
\pgfsetfillopacity{0.700000}%
\pgfsetlinewidth{0.000000pt}%
\definecolor{currentstroke}{rgb}{0.000000,0.000000,0.000000}%
\pgfsetstrokecolor{currentstroke}%
\pgfsetstrokeopacity{0.700000}%
\pgfsetdash{}{0pt}%
\pgfpathmoveto{\pgfqpoint{7.794499in}{3.097098in}}%
\pgfpathcurveto{\pgfqpoint{7.799542in}{3.097098in}}{\pgfqpoint{7.804380in}{3.099102in}}{\pgfqpoint{7.807947in}{3.102668in}}%
\pgfpathcurveto{\pgfqpoint{7.811513in}{3.106235in}}{\pgfqpoint{7.813517in}{3.111073in}}{\pgfqpoint{7.813517in}{3.116116in}}%
\pgfpathcurveto{\pgfqpoint{7.813517in}{3.121160in}}{\pgfqpoint{7.811513in}{3.125998in}}{\pgfqpoint{7.807947in}{3.129564in}}%
\pgfpathcurveto{\pgfqpoint{7.804380in}{3.133131in}}{\pgfqpoint{7.799542in}{3.135134in}}{\pgfqpoint{7.794499in}{3.135134in}}%
\pgfpathcurveto{\pgfqpoint{7.789455in}{3.135134in}}{\pgfqpoint{7.784617in}{3.133131in}}{\pgfqpoint{7.781051in}{3.129564in}}%
\pgfpathcurveto{\pgfqpoint{7.777485in}{3.125998in}}{\pgfqpoint{7.775481in}{3.121160in}}{\pgfqpoint{7.775481in}{3.116116in}}%
\pgfpathcurveto{\pgfqpoint{7.775481in}{3.111073in}}{\pgfqpoint{7.777485in}{3.106235in}}{\pgfqpoint{7.781051in}{3.102668in}}%
\pgfpathcurveto{\pgfqpoint{7.784617in}{3.099102in}}{\pgfqpoint{7.789455in}{3.097098in}}{\pgfqpoint{7.794499in}{3.097098in}}%
\pgfpathclose%
\pgfusepath{fill}%
\end{pgfscope}%
\begin{pgfscope}%
\pgfpathrectangle{\pgfqpoint{6.572727in}{0.474100in}}{\pgfqpoint{4.227273in}{3.318700in}}%
\pgfusepath{clip}%
\pgfsetbuttcap%
\pgfsetroundjoin%
\definecolor{currentfill}{rgb}{0.993248,0.906157,0.143936}%
\pgfsetfillcolor{currentfill}%
\pgfsetfillopacity{0.700000}%
\pgfsetlinewidth{0.000000pt}%
\definecolor{currentstroke}{rgb}{0.000000,0.000000,0.000000}%
\pgfsetstrokecolor{currentstroke}%
\pgfsetstrokeopacity{0.700000}%
\pgfsetdash{}{0pt}%
\pgfpathmoveto{\pgfqpoint{9.464640in}{1.941879in}}%
\pgfpathcurveto{\pgfqpoint{9.469684in}{1.941879in}}{\pgfqpoint{9.474522in}{1.943883in}}{\pgfqpoint{9.478088in}{1.947449in}}%
\pgfpathcurveto{\pgfqpoint{9.481654in}{1.951016in}}{\pgfqpoint{9.483658in}{1.955854in}}{\pgfqpoint{9.483658in}{1.960897in}}%
\pgfpathcurveto{\pgfqpoint{9.483658in}{1.965941in}}{\pgfqpoint{9.481654in}{1.970779in}}{\pgfqpoint{9.478088in}{1.974345in}}%
\pgfpathcurveto{\pgfqpoint{9.474522in}{1.977912in}}{\pgfqpoint{9.469684in}{1.979915in}}{\pgfqpoint{9.464640in}{1.979915in}}%
\pgfpathcurveto{\pgfqpoint{9.459596in}{1.979915in}}{\pgfqpoint{9.454759in}{1.977912in}}{\pgfqpoint{9.451192in}{1.974345in}}%
\pgfpathcurveto{\pgfqpoint{9.447626in}{1.970779in}}{\pgfqpoint{9.445622in}{1.965941in}}{\pgfqpoint{9.445622in}{1.960897in}}%
\pgfpathcurveto{\pgfqpoint{9.445622in}{1.955854in}}{\pgfqpoint{9.447626in}{1.951016in}}{\pgfqpoint{9.451192in}{1.947449in}}%
\pgfpathcurveto{\pgfqpoint{9.454759in}{1.943883in}}{\pgfqpoint{9.459596in}{1.941879in}}{\pgfqpoint{9.464640in}{1.941879in}}%
\pgfpathclose%
\pgfusepath{fill}%
\end{pgfscope}%
\begin{pgfscope}%
\pgfpathrectangle{\pgfqpoint{6.572727in}{0.474100in}}{\pgfqpoint{4.227273in}{3.318700in}}%
\pgfusepath{clip}%
\pgfsetbuttcap%
\pgfsetroundjoin%
\definecolor{currentfill}{rgb}{0.127568,0.566949,0.550556}%
\pgfsetfillcolor{currentfill}%
\pgfsetfillopacity{0.700000}%
\pgfsetlinewidth{0.000000pt}%
\definecolor{currentstroke}{rgb}{0.000000,0.000000,0.000000}%
\pgfsetstrokecolor{currentstroke}%
\pgfsetstrokeopacity{0.700000}%
\pgfsetdash{}{0pt}%
\pgfpathmoveto{\pgfqpoint{8.272907in}{1.232988in}}%
\pgfpathcurveto{\pgfqpoint{8.277951in}{1.232988in}}{\pgfqpoint{8.282788in}{1.234992in}}{\pgfqpoint{8.286355in}{1.238559in}}%
\pgfpathcurveto{\pgfqpoint{8.289921in}{1.242125in}}{\pgfqpoint{8.291925in}{1.246963in}}{\pgfqpoint{8.291925in}{1.252007in}}%
\pgfpathcurveto{\pgfqpoint{8.291925in}{1.257050in}}{\pgfqpoint{8.289921in}{1.261888in}}{\pgfqpoint{8.286355in}{1.265454in}}%
\pgfpathcurveto{\pgfqpoint{8.282788in}{1.269021in}}{\pgfqpoint{8.277951in}{1.271025in}}{\pgfqpoint{8.272907in}{1.271025in}}%
\pgfpathcurveto{\pgfqpoint{8.267863in}{1.271025in}}{\pgfqpoint{8.263025in}{1.269021in}}{\pgfqpoint{8.259459in}{1.265454in}}%
\pgfpathcurveto{\pgfqpoint{8.255893in}{1.261888in}}{\pgfqpoint{8.253889in}{1.257050in}}{\pgfqpoint{8.253889in}{1.252007in}}%
\pgfpathcurveto{\pgfqpoint{8.253889in}{1.246963in}}{\pgfqpoint{8.255893in}{1.242125in}}{\pgfqpoint{8.259459in}{1.238559in}}%
\pgfpathcurveto{\pgfqpoint{8.263025in}{1.234992in}}{\pgfqpoint{8.267863in}{1.232988in}}{\pgfqpoint{8.272907in}{1.232988in}}%
\pgfpathclose%
\pgfusepath{fill}%
\end{pgfscope}%
\begin{pgfscope}%
\pgfpathrectangle{\pgfqpoint{6.572727in}{0.474100in}}{\pgfqpoint{4.227273in}{3.318700in}}%
\pgfusepath{clip}%
\pgfsetbuttcap%
\pgfsetroundjoin%
\definecolor{currentfill}{rgb}{0.127568,0.566949,0.550556}%
\pgfsetfillcolor{currentfill}%
\pgfsetfillopacity{0.700000}%
\pgfsetlinewidth{0.000000pt}%
\definecolor{currentstroke}{rgb}{0.000000,0.000000,0.000000}%
\pgfsetstrokecolor{currentstroke}%
\pgfsetstrokeopacity{0.700000}%
\pgfsetdash{}{0pt}%
\pgfpathmoveto{\pgfqpoint{8.250350in}{1.627376in}}%
\pgfpathcurveto{\pgfqpoint{8.255394in}{1.627376in}}{\pgfqpoint{8.260232in}{1.629379in}}{\pgfqpoint{8.263798in}{1.632946in}}%
\pgfpathcurveto{\pgfqpoint{8.267364in}{1.636512in}}{\pgfqpoint{8.269368in}{1.641350in}}{\pgfqpoint{8.269368in}{1.646394in}}%
\pgfpathcurveto{\pgfqpoint{8.269368in}{1.651437in}}{\pgfqpoint{8.267364in}{1.656275in}}{\pgfqpoint{8.263798in}{1.659842in}}%
\pgfpathcurveto{\pgfqpoint{8.260232in}{1.663408in}}{\pgfqpoint{8.255394in}{1.665412in}}{\pgfqpoint{8.250350in}{1.665412in}}%
\pgfpathcurveto{\pgfqpoint{8.245307in}{1.665412in}}{\pgfqpoint{8.240469in}{1.663408in}}{\pgfqpoint{8.236902in}{1.659842in}}%
\pgfpathcurveto{\pgfqpoint{8.233336in}{1.656275in}}{\pgfqpoint{8.231332in}{1.651437in}}{\pgfqpoint{8.231332in}{1.646394in}}%
\pgfpathcurveto{\pgfqpoint{8.231332in}{1.641350in}}{\pgfqpoint{8.233336in}{1.636512in}}{\pgfqpoint{8.236902in}{1.632946in}}%
\pgfpathcurveto{\pgfqpoint{8.240469in}{1.629379in}}{\pgfqpoint{8.245307in}{1.627376in}}{\pgfqpoint{8.250350in}{1.627376in}}%
\pgfpathclose%
\pgfusepath{fill}%
\end{pgfscope}%
\begin{pgfscope}%
\pgfpathrectangle{\pgfqpoint{6.572727in}{0.474100in}}{\pgfqpoint{4.227273in}{3.318700in}}%
\pgfusepath{clip}%
\pgfsetbuttcap%
\pgfsetroundjoin%
\definecolor{currentfill}{rgb}{0.993248,0.906157,0.143936}%
\pgfsetfillcolor{currentfill}%
\pgfsetfillopacity{0.700000}%
\pgfsetlinewidth{0.000000pt}%
\definecolor{currentstroke}{rgb}{0.000000,0.000000,0.000000}%
\pgfsetstrokecolor{currentstroke}%
\pgfsetstrokeopacity{0.700000}%
\pgfsetdash{}{0pt}%
\pgfpathmoveto{\pgfqpoint{9.945971in}{1.769533in}}%
\pgfpathcurveto{\pgfqpoint{9.951015in}{1.769533in}}{\pgfqpoint{9.955853in}{1.771537in}}{\pgfqpoint{9.959419in}{1.775103in}}%
\pgfpathcurveto{\pgfqpoint{9.962986in}{1.778669in}}{\pgfqpoint{9.964989in}{1.783507in}}{\pgfqpoint{9.964989in}{1.788551in}}%
\pgfpathcurveto{\pgfqpoint{9.964989in}{1.793595in}}{\pgfqpoint{9.962986in}{1.798432in}}{\pgfqpoint{9.959419in}{1.801999in}}%
\pgfpathcurveto{\pgfqpoint{9.955853in}{1.805565in}}{\pgfqpoint{9.951015in}{1.807569in}}{\pgfqpoint{9.945971in}{1.807569in}}%
\pgfpathcurveto{\pgfqpoint{9.940928in}{1.807569in}}{\pgfqpoint{9.936090in}{1.805565in}}{\pgfqpoint{9.932523in}{1.801999in}}%
\pgfpathcurveto{\pgfqpoint{9.928957in}{1.798432in}}{\pgfqpoint{9.926953in}{1.793595in}}{\pgfqpoint{9.926953in}{1.788551in}}%
\pgfpathcurveto{\pgfqpoint{9.926953in}{1.783507in}}{\pgfqpoint{9.928957in}{1.778669in}}{\pgfqpoint{9.932523in}{1.775103in}}%
\pgfpathcurveto{\pgfqpoint{9.936090in}{1.771537in}}{\pgfqpoint{9.940928in}{1.769533in}}{\pgfqpoint{9.945971in}{1.769533in}}%
\pgfpathclose%
\pgfusepath{fill}%
\end{pgfscope}%
\begin{pgfscope}%
\pgfpathrectangle{\pgfqpoint{6.572727in}{0.474100in}}{\pgfqpoint{4.227273in}{3.318700in}}%
\pgfusepath{clip}%
\pgfsetbuttcap%
\pgfsetroundjoin%
\definecolor{currentfill}{rgb}{0.127568,0.566949,0.550556}%
\pgfsetfillcolor{currentfill}%
\pgfsetfillopacity{0.700000}%
\pgfsetlinewidth{0.000000pt}%
\definecolor{currentstroke}{rgb}{0.000000,0.000000,0.000000}%
\pgfsetstrokecolor{currentstroke}%
\pgfsetstrokeopacity{0.700000}%
\pgfsetdash{}{0pt}%
\pgfpathmoveto{\pgfqpoint{7.906949in}{1.874768in}}%
\pgfpathcurveto{\pgfqpoint{7.911993in}{1.874768in}}{\pgfqpoint{7.916831in}{1.876772in}}{\pgfqpoint{7.920397in}{1.880338in}}%
\pgfpathcurveto{\pgfqpoint{7.923963in}{1.883905in}}{\pgfqpoint{7.925967in}{1.888742in}}{\pgfqpoint{7.925967in}{1.893786in}}%
\pgfpathcurveto{\pgfqpoint{7.925967in}{1.898830in}}{\pgfqpoint{7.923963in}{1.903667in}}{\pgfqpoint{7.920397in}{1.907234in}}%
\pgfpathcurveto{\pgfqpoint{7.916831in}{1.910800in}}{\pgfqpoint{7.911993in}{1.912804in}}{\pgfqpoint{7.906949in}{1.912804in}}%
\pgfpathcurveto{\pgfqpoint{7.901905in}{1.912804in}}{\pgfqpoint{7.897068in}{1.910800in}}{\pgfqpoint{7.893501in}{1.907234in}}%
\pgfpathcurveto{\pgfqpoint{7.889935in}{1.903667in}}{\pgfqpoint{7.887931in}{1.898830in}}{\pgfqpoint{7.887931in}{1.893786in}}%
\pgfpathcurveto{\pgfqpoint{7.887931in}{1.888742in}}{\pgfqpoint{7.889935in}{1.883905in}}{\pgfqpoint{7.893501in}{1.880338in}}%
\pgfpathcurveto{\pgfqpoint{7.897068in}{1.876772in}}{\pgfqpoint{7.901905in}{1.874768in}}{\pgfqpoint{7.906949in}{1.874768in}}%
\pgfpathclose%
\pgfusepath{fill}%
\end{pgfscope}%
\begin{pgfscope}%
\pgfpathrectangle{\pgfqpoint{6.572727in}{0.474100in}}{\pgfqpoint{4.227273in}{3.318700in}}%
\pgfusepath{clip}%
\pgfsetbuttcap%
\pgfsetroundjoin%
\definecolor{currentfill}{rgb}{0.993248,0.906157,0.143936}%
\pgfsetfillcolor{currentfill}%
\pgfsetfillopacity{0.700000}%
\pgfsetlinewidth{0.000000pt}%
\definecolor{currentstroke}{rgb}{0.000000,0.000000,0.000000}%
\pgfsetstrokecolor{currentstroke}%
\pgfsetstrokeopacity{0.700000}%
\pgfsetdash{}{0pt}%
\pgfpathmoveto{\pgfqpoint{9.251025in}{1.584946in}}%
\pgfpathcurveto{\pgfqpoint{9.256069in}{1.584946in}}{\pgfqpoint{9.260907in}{1.586950in}}{\pgfqpoint{9.264473in}{1.590516in}}%
\pgfpathcurveto{\pgfqpoint{9.268040in}{1.594083in}}{\pgfqpoint{9.270044in}{1.598921in}}{\pgfqpoint{9.270044in}{1.603964in}}%
\pgfpathcurveto{\pgfqpoint{9.270044in}{1.609008in}}{\pgfqpoint{9.268040in}{1.613846in}}{\pgfqpoint{9.264473in}{1.617412in}}%
\pgfpathcurveto{\pgfqpoint{9.260907in}{1.620979in}}{\pgfqpoint{9.256069in}{1.622982in}}{\pgfqpoint{9.251025in}{1.622982in}}%
\pgfpathcurveto{\pgfqpoint{9.245982in}{1.622982in}}{\pgfqpoint{9.241144in}{1.620979in}}{\pgfqpoint{9.237578in}{1.617412in}}%
\pgfpathcurveto{\pgfqpoint{9.234011in}{1.613846in}}{\pgfqpoint{9.232007in}{1.609008in}}{\pgfqpoint{9.232007in}{1.603964in}}%
\pgfpathcurveto{\pgfqpoint{9.232007in}{1.598921in}}{\pgfqpoint{9.234011in}{1.594083in}}{\pgfqpoint{9.237578in}{1.590516in}}%
\pgfpathcurveto{\pgfqpoint{9.241144in}{1.586950in}}{\pgfqpoint{9.245982in}{1.584946in}}{\pgfqpoint{9.251025in}{1.584946in}}%
\pgfpathclose%
\pgfusepath{fill}%
\end{pgfscope}%
\begin{pgfscope}%
\pgfpathrectangle{\pgfqpoint{6.572727in}{0.474100in}}{\pgfqpoint{4.227273in}{3.318700in}}%
\pgfusepath{clip}%
\pgfsetbuttcap%
\pgfsetroundjoin%
\definecolor{currentfill}{rgb}{0.127568,0.566949,0.550556}%
\pgfsetfillcolor{currentfill}%
\pgfsetfillopacity{0.700000}%
\pgfsetlinewidth{0.000000pt}%
\definecolor{currentstroke}{rgb}{0.000000,0.000000,0.000000}%
\pgfsetstrokecolor{currentstroke}%
\pgfsetstrokeopacity{0.700000}%
\pgfsetdash{}{0pt}%
\pgfpathmoveto{\pgfqpoint{7.804424in}{1.847281in}}%
\pgfpathcurveto{\pgfqpoint{7.809468in}{1.847281in}}{\pgfqpoint{7.814306in}{1.849284in}}{\pgfqpoint{7.817872in}{1.852851in}}%
\pgfpathcurveto{\pgfqpoint{7.821439in}{1.856417in}}{\pgfqpoint{7.823442in}{1.861255in}}{\pgfqpoint{7.823442in}{1.866299in}}%
\pgfpathcurveto{\pgfqpoint{7.823442in}{1.871342in}}{\pgfqpoint{7.821439in}{1.876180in}}{\pgfqpoint{7.817872in}{1.879747in}}%
\pgfpathcurveto{\pgfqpoint{7.814306in}{1.883313in}}{\pgfqpoint{7.809468in}{1.885317in}}{\pgfqpoint{7.804424in}{1.885317in}}%
\pgfpathcurveto{\pgfqpoint{7.799381in}{1.885317in}}{\pgfqpoint{7.794543in}{1.883313in}}{\pgfqpoint{7.790976in}{1.879747in}}%
\pgfpathcurveto{\pgfqpoint{7.787410in}{1.876180in}}{\pgfqpoint{7.785406in}{1.871342in}}{\pgfqpoint{7.785406in}{1.866299in}}%
\pgfpathcurveto{\pgfqpoint{7.785406in}{1.861255in}}{\pgfqpoint{7.787410in}{1.856417in}}{\pgfqpoint{7.790976in}{1.852851in}}%
\pgfpathcurveto{\pgfqpoint{7.794543in}{1.849284in}}{\pgfqpoint{7.799381in}{1.847281in}}{\pgfqpoint{7.804424in}{1.847281in}}%
\pgfpathclose%
\pgfusepath{fill}%
\end{pgfscope}%
\begin{pgfscope}%
\pgfpathrectangle{\pgfqpoint{6.572727in}{0.474100in}}{\pgfqpoint{4.227273in}{3.318700in}}%
\pgfusepath{clip}%
\pgfsetbuttcap%
\pgfsetroundjoin%
\definecolor{currentfill}{rgb}{0.127568,0.566949,0.550556}%
\pgfsetfillcolor{currentfill}%
\pgfsetfillopacity{0.700000}%
\pgfsetlinewidth{0.000000pt}%
\definecolor{currentstroke}{rgb}{0.000000,0.000000,0.000000}%
\pgfsetstrokecolor{currentstroke}%
\pgfsetstrokeopacity{0.700000}%
\pgfsetdash{}{0pt}%
\pgfpathmoveto{\pgfqpoint{8.593939in}{2.763200in}}%
\pgfpathcurveto{\pgfqpoint{8.598982in}{2.763200in}}{\pgfqpoint{8.603820in}{2.765204in}}{\pgfqpoint{8.607386in}{2.768770in}}%
\pgfpathcurveto{\pgfqpoint{8.610953in}{2.772336in}}{\pgfqpoint{8.612957in}{2.777174in}}{\pgfqpoint{8.612957in}{2.782218in}}%
\pgfpathcurveto{\pgfqpoint{8.612957in}{2.787261in}}{\pgfqpoint{8.610953in}{2.792099in}}{\pgfqpoint{8.607386in}{2.795666in}}%
\pgfpathcurveto{\pgfqpoint{8.603820in}{2.799232in}}{\pgfqpoint{8.598982in}{2.801236in}}{\pgfqpoint{8.593939in}{2.801236in}}%
\pgfpathcurveto{\pgfqpoint{8.588895in}{2.801236in}}{\pgfqpoint{8.584057in}{2.799232in}}{\pgfqpoint{8.580491in}{2.795666in}}%
\pgfpathcurveto{\pgfqpoint{8.576924in}{2.792099in}}{\pgfqpoint{8.574920in}{2.787261in}}{\pgfqpoint{8.574920in}{2.782218in}}%
\pgfpathcurveto{\pgfqpoint{8.574920in}{2.777174in}}{\pgfqpoint{8.576924in}{2.772336in}}{\pgfqpoint{8.580491in}{2.768770in}}%
\pgfpathcurveto{\pgfqpoint{8.584057in}{2.765204in}}{\pgfqpoint{8.588895in}{2.763200in}}{\pgfqpoint{8.593939in}{2.763200in}}%
\pgfpathclose%
\pgfusepath{fill}%
\end{pgfscope}%
\begin{pgfscope}%
\pgfpathrectangle{\pgfqpoint{6.572727in}{0.474100in}}{\pgfqpoint{4.227273in}{3.318700in}}%
\pgfusepath{clip}%
\pgfsetbuttcap%
\pgfsetroundjoin%
\definecolor{currentfill}{rgb}{0.993248,0.906157,0.143936}%
\pgfsetfillcolor{currentfill}%
\pgfsetfillopacity{0.700000}%
\pgfsetlinewidth{0.000000pt}%
\definecolor{currentstroke}{rgb}{0.000000,0.000000,0.000000}%
\pgfsetstrokecolor{currentstroke}%
\pgfsetstrokeopacity{0.700000}%
\pgfsetdash{}{0pt}%
\pgfpathmoveto{\pgfqpoint{9.787090in}{1.591905in}}%
\pgfpathcurveto{\pgfqpoint{9.792134in}{1.591905in}}{\pgfqpoint{9.796971in}{1.593909in}}{\pgfqpoint{9.800538in}{1.597475in}}%
\pgfpathcurveto{\pgfqpoint{9.804104in}{1.601042in}}{\pgfqpoint{9.806108in}{1.605879in}}{\pgfqpoint{9.806108in}{1.610923in}}%
\pgfpathcurveto{\pgfqpoint{9.806108in}{1.615967in}}{\pgfqpoint{9.804104in}{1.620805in}}{\pgfqpoint{9.800538in}{1.624371in}}%
\pgfpathcurveto{\pgfqpoint{9.796971in}{1.627937in}}{\pgfqpoint{9.792134in}{1.629941in}}{\pgfqpoint{9.787090in}{1.629941in}}%
\pgfpathcurveto{\pgfqpoint{9.782046in}{1.629941in}}{\pgfqpoint{9.777208in}{1.627937in}}{\pgfqpoint{9.773642in}{1.624371in}}%
\pgfpathcurveto{\pgfqpoint{9.770076in}{1.620805in}}{\pgfqpoint{9.768072in}{1.615967in}}{\pgfqpoint{9.768072in}{1.610923in}}%
\pgfpathcurveto{\pgfqpoint{9.768072in}{1.605879in}}{\pgfqpoint{9.770076in}{1.601042in}}{\pgfqpoint{9.773642in}{1.597475in}}%
\pgfpathcurveto{\pgfqpoint{9.777208in}{1.593909in}}{\pgfqpoint{9.782046in}{1.591905in}}{\pgfqpoint{9.787090in}{1.591905in}}%
\pgfpathclose%
\pgfusepath{fill}%
\end{pgfscope}%
\begin{pgfscope}%
\pgfpathrectangle{\pgfqpoint{6.572727in}{0.474100in}}{\pgfqpoint{4.227273in}{3.318700in}}%
\pgfusepath{clip}%
\pgfsetbuttcap%
\pgfsetroundjoin%
\definecolor{currentfill}{rgb}{0.127568,0.566949,0.550556}%
\pgfsetfillcolor{currentfill}%
\pgfsetfillopacity{0.700000}%
\pgfsetlinewidth{0.000000pt}%
\definecolor{currentstroke}{rgb}{0.000000,0.000000,0.000000}%
\pgfsetstrokecolor{currentstroke}%
\pgfsetstrokeopacity{0.700000}%
\pgfsetdash{}{0pt}%
\pgfpathmoveto{\pgfqpoint{8.258408in}{3.071458in}}%
\pgfpathcurveto{\pgfqpoint{8.263452in}{3.071458in}}{\pgfqpoint{8.268290in}{3.073462in}}{\pgfqpoint{8.271856in}{3.077029in}}%
\pgfpathcurveto{\pgfqpoint{8.275422in}{3.080595in}}{\pgfqpoint{8.277426in}{3.085433in}}{\pgfqpoint{8.277426in}{3.090477in}}%
\pgfpathcurveto{\pgfqpoint{8.277426in}{3.095520in}}{\pgfqpoint{8.275422in}{3.100358in}}{\pgfqpoint{8.271856in}{3.103924in}}%
\pgfpathcurveto{\pgfqpoint{8.268290in}{3.107491in}}{\pgfqpoint{8.263452in}{3.109495in}}{\pgfqpoint{8.258408in}{3.109495in}}%
\pgfpathcurveto{\pgfqpoint{8.253364in}{3.109495in}}{\pgfqpoint{8.248527in}{3.107491in}}{\pgfqpoint{8.244960in}{3.103924in}}%
\pgfpathcurveto{\pgfqpoint{8.241394in}{3.100358in}}{\pgfqpoint{8.239390in}{3.095520in}}{\pgfqpoint{8.239390in}{3.090477in}}%
\pgfpathcurveto{\pgfqpoint{8.239390in}{3.085433in}}{\pgfqpoint{8.241394in}{3.080595in}}{\pgfqpoint{8.244960in}{3.077029in}}%
\pgfpathcurveto{\pgfqpoint{8.248527in}{3.073462in}}{\pgfqpoint{8.253364in}{3.071458in}}{\pgfqpoint{8.258408in}{3.071458in}}%
\pgfpathclose%
\pgfusepath{fill}%
\end{pgfscope}%
\begin{pgfscope}%
\pgfpathrectangle{\pgfqpoint{6.572727in}{0.474100in}}{\pgfqpoint{4.227273in}{3.318700in}}%
\pgfusepath{clip}%
\pgfsetbuttcap%
\pgfsetroundjoin%
\definecolor{currentfill}{rgb}{0.127568,0.566949,0.550556}%
\pgfsetfillcolor{currentfill}%
\pgfsetfillopacity{0.700000}%
\pgfsetlinewidth{0.000000pt}%
\definecolor{currentstroke}{rgb}{0.000000,0.000000,0.000000}%
\pgfsetstrokecolor{currentstroke}%
\pgfsetstrokeopacity{0.700000}%
\pgfsetdash{}{0pt}%
\pgfpathmoveto{\pgfqpoint{8.370768in}{1.469745in}}%
\pgfpathcurveto{\pgfqpoint{8.375811in}{1.469745in}}{\pgfqpoint{8.380649in}{1.471749in}}{\pgfqpoint{8.384215in}{1.475316in}}%
\pgfpathcurveto{\pgfqpoint{8.387782in}{1.478882in}}{\pgfqpoint{8.389786in}{1.483720in}}{\pgfqpoint{8.389786in}{1.488764in}}%
\pgfpathcurveto{\pgfqpoint{8.389786in}{1.493807in}}{\pgfqpoint{8.387782in}{1.498645in}}{\pgfqpoint{8.384215in}{1.502211in}}%
\pgfpathcurveto{\pgfqpoint{8.380649in}{1.505778in}}{\pgfqpoint{8.375811in}{1.507782in}}{\pgfqpoint{8.370768in}{1.507782in}}%
\pgfpathcurveto{\pgfqpoint{8.365724in}{1.507782in}}{\pgfqpoint{8.360886in}{1.505778in}}{\pgfqpoint{8.357320in}{1.502211in}}%
\pgfpathcurveto{\pgfqpoint{8.353753in}{1.498645in}}{\pgfqpoint{8.351749in}{1.493807in}}{\pgfqpoint{8.351749in}{1.488764in}}%
\pgfpathcurveto{\pgfqpoint{8.351749in}{1.483720in}}{\pgfqpoint{8.353753in}{1.478882in}}{\pgfqpoint{8.357320in}{1.475316in}}%
\pgfpathcurveto{\pgfqpoint{8.360886in}{1.471749in}}{\pgfqpoint{8.365724in}{1.469745in}}{\pgfqpoint{8.370768in}{1.469745in}}%
\pgfpathclose%
\pgfusepath{fill}%
\end{pgfscope}%
\begin{pgfscope}%
\pgfpathrectangle{\pgfqpoint{6.572727in}{0.474100in}}{\pgfqpoint{4.227273in}{3.318700in}}%
\pgfusepath{clip}%
\pgfsetbuttcap%
\pgfsetroundjoin%
\definecolor{currentfill}{rgb}{0.127568,0.566949,0.550556}%
\pgfsetfillcolor{currentfill}%
\pgfsetfillopacity{0.700000}%
\pgfsetlinewidth{0.000000pt}%
\definecolor{currentstroke}{rgb}{0.000000,0.000000,0.000000}%
\pgfsetstrokecolor{currentstroke}%
\pgfsetstrokeopacity{0.700000}%
\pgfsetdash{}{0pt}%
\pgfpathmoveto{\pgfqpoint{7.401273in}{1.192665in}}%
\pgfpathcurveto{\pgfqpoint{7.406317in}{1.192665in}}{\pgfqpoint{7.411155in}{1.194669in}}{\pgfqpoint{7.414721in}{1.198235in}}%
\pgfpathcurveto{\pgfqpoint{7.418288in}{1.201801in}}{\pgfqpoint{7.420291in}{1.206639in}}{\pgfqpoint{7.420291in}{1.211683in}}%
\pgfpathcurveto{\pgfqpoint{7.420291in}{1.216727in}}{\pgfqpoint{7.418288in}{1.221564in}}{\pgfqpoint{7.414721in}{1.225131in}}%
\pgfpathcurveto{\pgfqpoint{7.411155in}{1.228697in}}{\pgfqpoint{7.406317in}{1.230701in}}{\pgfqpoint{7.401273in}{1.230701in}}%
\pgfpathcurveto{\pgfqpoint{7.396230in}{1.230701in}}{\pgfqpoint{7.391392in}{1.228697in}}{\pgfqpoint{7.387825in}{1.225131in}}%
\pgfpathcurveto{\pgfqpoint{7.384259in}{1.221564in}}{\pgfqpoint{7.382255in}{1.216727in}}{\pgfqpoint{7.382255in}{1.211683in}}%
\pgfpathcurveto{\pgfqpoint{7.382255in}{1.206639in}}{\pgfqpoint{7.384259in}{1.201801in}}{\pgfqpoint{7.387825in}{1.198235in}}%
\pgfpathcurveto{\pgfqpoint{7.391392in}{1.194669in}}{\pgfqpoint{7.396230in}{1.192665in}}{\pgfqpoint{7.401273in}{1.192665in}}%
\pgfpathclose%
\pgfusepath{fill}%
\end{pgfscope}%
\begin{pgfscope}%
\pgfpathrectangle{\pgfqpoint{6.572727in}{0.474100in}}{\pgfqpoint{4.227273in}{3.318700in}}%
\pgfusepath{clip}%
\pgfsetbuttcap%
\pgfsetroundjoin%
\definecolor{currentfill}{rgb}{0.127568,0.566949,0.550556}%
\pgfsetfillcolor{currentfill}%
\pgfsetfillopacity{0.700000}%
\pgfsetlinewidth{0.000000pt}%
\definecolor{currentstroke}{rgb}{0.000000,0.000000,0.000000}%
\pgfsetstrokecolor{currentstroke}%
\pgfsetstrokeopacity{0.700000}%
\pgfsetdash{}{0pt}%
\pgfpathmoveto{\pgfqpoint{8.392658in}{2.584315in}}%
\pgfpathcurveto{\pgfqpoint{8.397702in}{2.584315in}}{\pgfqpoint{8.402539in}{2.586319in}}{\pgfqpoint{8.406106in}{2.589885in}}%
\pgfpathcurveto{\pgfqpoint{8.409672in}{2.593452in}}{\pgfqpoint{8.411676in}{2.598289in}}{\pgfqpoint{8.411676in}{2.603333in}}%
\pgfpathcurveto{\pgfqpoint{8.411676in}{2.608377in}}{\pgfqpoint{8.409672in}{2.613215in}}{\pgfqpoint{8.406106in}{2.616781in}}%
\pgfpathcurveto{\pgfqpoint{8.402539in}{2.620347in}}{\pgfqpoint{8.397702in}{2.622351in}}{\pgfqpoint{8.392658in}{2.622351in}}%
\pgfpathcurveto{\pgfqpoint{8.387614in}{2.622351in}}{\pgfqpoint{8.382777in}{2.620347in}}{\pgfqpoint{8.379210in}{2.616781in}}%
\pgfpathcurveto{\pgfqpoint{8.375644in}{2.613215in}}{\pgfqpoint{8.373640in}{2.608377in}}{\pgfqpoint{8.373640in}{2.603333in}}%
\pgfpathcurveto{\pgfqpoint{8.373640in}{2.598289in}}{\pgfqpoint{8.375644in}{2.593452in}}{\pgfqpoint{8.379210in}{2.589885in}}%
\pgfpathcurveto{\pgfqpoint{8.382777in}{2.586319in}}{\pgfqpoint{8.387614in}{2.584315in}}{\pgfqpoint{8.392658in}{2.584315in}}%
\pgfpathclose%
\pgfusepath{fill}%
\end{pgfscope}%
\begin{pgfscope}%
\pgfpathrectangle{\pgfqpoint{6.572727in}{0.474100in}}{\pgfqpoint{4.227273in}{3.318700in}}%
\pgfusepath{clip}%
\pgfsetbuttcap%
\pgfsetroundjoin%
\definecolor{currentfill}{rgb}{0.127568,0.566949,0.550556}%
\pgfsetfillcolor{currentfill}%
\pgfsetfillopacity{0.700000}%
\pgfsetlinewidth{0.000000pt}%
\definecolor{currentstroke}{rgb}{0.000000,0.000000,0.000000}%
\pgfsetstrokecolor{currentstroke}%
\pgfsetstrokeopacity{0.700000}%
\pgfsetdash{}{0pt}%
\pgfpathmoveto{\pgfqpoint{8.593693in}{2.843314in}}%
\pgfpathcurveto{\pgfqpoint{8.598737in}{2.843314in}}{\pgfqpoint{8.603575in}{2.845318in}}{\pgfqpoint{8.607141in}{2.848884in}}%
\pgfpathcurveto{\pgfqpoint{8.610707in}{2.852450in}}{\pgfqpoint{8.612711in}{2.857288in}}{\pgfqpoint{8.612711in}{2.862332in}}%
\pgfpathcurveto{\pgfqpoint{8.612711in}{2.867376in}}{\pgfqpoint{8.610707in}{2.872213in}}{\pgfqpoint{8.607141in}{2.875780in}}%
\pgfpathcurveto{\pgfqpoint{8.603575in}{2.879346in}}{\pgfqpoint{8.598737in}{2.881350in}}{\pgfqpoint{8.593693in}{2.881350in}}%
\pgfpathcurveto{\pgfqpoint{8.588650in}{2.881350in}}{\pgfqpoint{8.583812in}{2.879346in}}{\pgfqpoint{8.580245in}{2.875780in}}%
\pgfpathcurveto{\pgfqpoint{8.576679in}{2.872213in}}{\pgfqpoint{8.574675in}{2.867376in}}{\pgfqpoint{8.574675in}{2.862332in}}%
\pgfpathcurveto{\pgfqpoint{8.574675in}{2.857288in}}{\pgfqpoint{8.576679in}{2.852450in}}{\pgfqpoint{8.580245in}{2.848884in}}%
\pgfpathcurveto{\pgfqpoint{8.583812in}{2.845318in}}{\pgfqpoint{8.588650in}{2.843314in}}{\pgfqpoint{8.593693in}{2.843314in}}%
\pgfpathclose%
\pgfusepath{fill}%
\end{pgfscope}%
\begin{pgfscope}%
\pgfpathrectangle{\pgfqpoint{6.572727in}{0.474100in}}{\pgfqpoint{4.227273in}{3.318700in}}%
\pgfusepath{clip}%
\pgfsetbuttcap%
\pgfsetroundjoin%
\definecolor{currentfill}{rgb}{0.993248,0.906157,0.143936}%
\pgfsetfillcolor{currentfill}%
\pgfsetfillopacity{0.700000}%
\pgfsetlinewidth{0.000000pt}%
\definecolor{currentstroke}{rgb}{0.000000,0.000000,0.000000}%
\pgfsetstrokecolor{currentstroke}%
\pgfsetstrokeopacity{0.700000}%
\pgfsetdash{}{0pt}%
\pgfpathmoveto{\pgfqpoint{8.962311in}{1.199849in}}%
\pgfpathcurveto{\pgfqpoint{8.967355in}{1.199849in}}{\pgfqpoint{8.972192in}{1.201852in}}{\pgfqpoint{8.975759in}{1.205419in}}%
\pgfpathcurveto{\pgfqpoint{8.979325in}{1.208985in}}{\pgfqpoint{8.981329in}{1.213823in}}{\pgfqpoint{8.981329in}{1.218867in}}%
\pgfpathcurveto{\pgfqpoint{8.981329in}{1.223910in}}{\pgfqpoint{8.979325in}{1.228748in}}{\pgfqpoint{8.975759in}{1.232315in}}%
\pgfpathcurveto{\pgfqpoint{8.972192in}{1.235881in}}{\pgfqpoint{8.967355in}{1.237885in}}{\pgfqpoint{8.962311in}{1.237885in}}%
\pgfpathcurveto{\pgfqpoint{8.957267in}{1.237885in}}{\pgfqpoint{8.952429in}{1.235881in}}{\pgfqpoint{8.948863in}{1.232315in}}%
\pgfpathcurveto{\pgfqpoint{8.945297in}{1.228748in}}{\pgfqpoint{8.943293in}{1.223910in}}{\pgfqpoint{8.943293in}{1.218867in}}%
\pgfpathcurveto{\pgfqpoint{8.943293in}{1.213823in}}{\pgfqpoint{8.945297in}{1.208985in}}{\pgfqpoint{8.948863in}{1.205419in}}%
\pgfpathcurveto{\pgfqpoint{8.952429in}{1.201852in}}{\pgfqpoint{8.957267in}{1.199849in}}{\pgfqpoint{8.962311in}{1.199849in}}%
\pgfpathclose%
\pgfusepath{fill}%
\end{pgfscope}%
\begin{pgfscope}%
\pgfpathrectangle{\pgfqpoint{6.572727in}{0.474100in}}{\pgfqpoint{4.227273in}{3.318700in}}%
\pgfusepath{clip}%
\pgfsetbuttcap%
\pgfsetroundjoin%
\definecolor{currentfill}{rgb}{0.127568,0.566949,0.550556}%
\pgfsetfillcolor{currentfill}%
\pgfsetfillopacity{0.700000}%
\pgfsetlinewidth{0.000000pt}%
\definecolor{currentstroke}{rgb}{0.000000,0.000000,0.000000}%
\pgfsetstrokecolor{currentstroke}%
\pgfsetstrokeopacity{0.700000}%
\pgfsetdash{}{0pt}%
\pgfpathmoveto{\pgfqpoint{8.247980in}{2.990677in}}%
\pgfpathcurveto{\pgfqpoint{8.253024in}{2.990677in}}{\pgfqpoint{8.257862in}{2.992681in}}{\pgfqpoint{8.261428in}{2.996247in}}%
\pgfpathcurveto{\pgfqpoint{8.264994in}{2.999814in}}{\pgfqpoint{8.266998in}{3.004652in}}{\pgfqpoint{8.266998in}{3.009695in}}%
\pgfpathcurveto{\pgfqpoint{8.266998in}{3.014739in}}{\pgfqpoint{8.264994in}{3.019577in}}{\pgfqpoint{8.261428in}{3.023143in}}%
\pgfpathcurveto{\pgfqpoint{8.257862in}{3.026710in}}{\pgfqpoint{8.253024in}{3.028713in}}{\pgfqpoint{8.247980in}{3.028713in}}%
\pgfpathcurveto{\pgfqpoint{8.242936in}{3.028713in}}{\pgfqpoint{8.238099in}{3.026710in}}{\pgfqpoint{8.234532in}{3.023143in}}%
\pgfpathcurveto{\pgfqpoint{8.230966in}{3.019577in}}{\pgfqpoint{8.228962in}{3.014739in}}{\pgfqpoint{8.228962in}{3.009695in}}%
\pgfpathcurveto{\pgfqpoint{8.228962in}{3.004652in}}{\pgfqpoint{8.230966in}{2.999814in}}{\pgfqpoint{8.234532in}{2.996247in}}%
\pgfpathcurveto{\pgfqpoint{8.238099in}{2.992681in}}{\pgfqpoint{8.242936in}{2.990677in}}{\pgfqpoint{8.247980in}{2.990677in}}%
\pgfpathclose%
\pgfusepath{fill}%
\end{pgfscope}%
\begin{pgfscope}%
\pgfpathrectangle{\pgfqpoint{6.572727in}{0.474100in}}{\pgfqpoint{4.227273in}{3.318700in}}%
\pgfusepath{clip}%
\pgfsetbuttcap%
\pgfsetroundjoin%
\definecolor{currentfill}{rgb}{0.127568,0.566949,0.550556}%
\pgfsetfillcolor{currentfill}%
\pgfsetfillopacity{0.700000}%
\pgfsetlinewidth{0.000000pt}%
\definecolor{currentstroke}{rgb}{0.000000,0.000000,0.000000}%
\pgfsetstrokecolor{currentstroke}%
\pgfsetstrokeopacity{0.700000}%
\pgfsetdash{}{0pt}%
\pgfpathmoveto{\pgfqpoint{8.512728in}{2.857207in}}%
\pgfpathcurveto{\pgfqpoint{8.517772in}{2.857207in}}{\pgfqpoint{8.522610in}{2.859211in}}{\pgfqpoint{8.526176in}{2.862777in}}%
\pgfpathcurveto{\pgfqpoint{8.529743in}{2.866344in}}{\pgfqpoint{8.531747in}{2.871182in}}{\pgfqpoint{8.531747in}{2.876225in}}%
\pgfpathcurveto{\pgfqpoint{8.531747in}{2.881269in}}{\pgfqpoint{8.529743in}{2.886107in}}{\pgfqpoint{8.526176in}{2.889673in}}%
\pgfpathcurveto{\pgfqpoint{8.522610in}{2.893240in}}{\pgfqpoint{8.517772in}{2.895243in}}{\pgfqpoint{8.512728in}{2.895243in}}%
\pgfpathcurveto{\pgfqpoint{8.507685in}{2.895243in}}{\pgfqpoint{8.502847in}{2.893240in}}{\pgfqpoint{8.499281in}{2.889673in}}%
\pgfpathcurveto{\pgfqpoint{8.495714in}{2.886107in}}{\pgfqpoint{8.493710in}{2.881269in}}{\pgfqpoint{8.493710in}{2.876225in}}%
\pgfpathcurveto{\pgfqpoint{8.493710in}{2.871182in}}{\pgfqpoint{8.495714in}{2.866344in}}{\pgfqpoint{8.499281in}{2.862777in}}%
\pgfpathcurveto{\pgfqpoint{8.502847in}{2.859211in}}{\pgfqpoint{8.507685in}{2.857207in}}{\pgfqpoint{8.512728in}{2.857207in}}%
\pgfpathclose%
\pgfusepath{fill}%
\end{pgfscope}%
\begin{pgfscope}%
\pgfpathrectangle{\pgfqpoint{6.572727in}{0.474100in}}{\pgfqpoint{4.227273in}{3.318700in}}%
\pgfusepath{clip}%
\pgfsetbuttcap%
\pgfsetroundjoin%
\definecolor{currentfill}{rgb}{0.127568,0.566949,0.550556}%
\pgfsetfillcolor{currentfill}%
\pgfsetfillopacity{0.700000}%
\pgfsetlinewidth{0.000000pt}%
\definecolor{currentstroke}{rgb}{0.000000,0.000000,0.000000}%
\pgfsetstrokecolor{currentstroke}%
\pgfsetstrokeopacity{0.700000}%
\pgfsetdash{}{0pt}%
\pgfpathmoveto{\pgfqpoint{7.567537in}{2.653674in}}%
\pgfpathcurveto{\pgfqpoint{7.572581in}{2.653674in}}{\pgfqpoint{7.577418in}{2.655678in}}{\pgfqpoint{7.580985in}{2.659244in}}%
\pgfpathcurveto{\pgfqpoint{7.584551in}{2.662811in}}{\pgfqpoint{7.586555in}{2.667649in}}{\pgfqpoint{7.586555in}{2.672692in}}%
\pgfpathcurveto{\pgfqpoint{7.586555in}{2.677736in}}{\pgfqpoint{7.584551in}{2.682574in}}{\pgfqpoint{7.580985in}{2.686140in}}%
\pgfpathcurveto{\pgfqpoint{7.577418in}{2.689706in}}{\pgfqpoint{7.572581in}{2.691710in}}{\pgfqpoint{7.567537in}{2.691710in}}%
\pgfpathcurveto{\pgfqpoint{7.562493in}{2.691710in}}{\pgfqpoint{7.557655in}{2.689706in}}{\pgfqpoint{7.554089in}{2.686140in}}%
\pgfpathcurveto{\pgfqpoint{7.550523in}{2.682574in}}{\pgfqpoint{7.548519in}{2.677736in}}{\pgfqpoint{7.548519in}{2.672692in}}%
\pgfpathcurveto{\pgfqpoint{7.548519in}{2.667649in}}{\pgfqpoint{7.550523in}{2.662811in}}{\pgfqpoint{7.554089in}{2.659244in}}%
\pgfpathcurveto{\pgfqpoint{7.557655in}{2.655678in}}{\pgfqpoint{7.562493in}{2.653674in}}{\pgfqpoint{7.567537in}{2.653674in}}%
\pgfpathclose%
\pgfusepath{fill}%
\end{pgfscope}%
\begin{pgfscope}%
\pgfpathrectangle{\pgfqpoint{6.572727in}{0.474100in}}{\pgfqpoint{4.227273in}{3.318700in}}%
\pgfusepath{clip}%
\pgfsetbuttcap%
\pgfsetroundjoin%
\definecolor{currentfill}{rgb}{0.127568,0.566949,0.550556}%
\pgfsetfillcolor{currentfill}%
\pgfsetfillopacity{0.700000}%
\pgfsetlinewidth{0.000000pt}%
\definecolor{currentstroke}{rgb}{0.000000,0.000000,0.000000}%
\pgfsetstrokecolor{currentstroke}%
\pgfsetstrokeopacity{0.700000}%
\pgfsetdash{}{0pt}%
\pgfpathmoveto{\pgfqpoint{8.304035in}{1.819960in}}%
\pgfpathcurveto{\pgfqpoint{8.309079in}{1.819960in}}{\pgfqpoint{8.313917in}{1.821964in}}{\pgfqpoint{8.317483in}{1.825530in}}%
\pgfpathcurveto{\pgfqpoint{8.321050in}{1.829097in}}{\pgfqpoint{8.323054in}{1.833934in}}{\pgfqpoint{8.323054in}{1.838978in}}%
\pgfpathcurveto{\pgfqpoint{8.323054in}{1.844022in}}{\pgfqpoint{8.321050in}{1.848860in}}{\pgfqpoint{8.317483in}{1.852426in}}%
\pgfpathcurveto{\pgfqpoint{8.313917in}{1.855992in}}{\pgfqpoint{8.309079in}{1.857996in}}{\pgfqpoint{8.304035in}{1.857996in}}%
\pgfpathcurveto{\pgfqpoint{8.298992in}{1.857996in}}{\pgfqpoint{8.294154in}{1.855992in}}{\pgfqpoint{8.290588in}{1.852426in}}%
\pgfpathcurveto{\pgfqpoint{8.287021in}{1.848860in}}{\pgfqpoint{8.285017in}{1.844022in}}{\pgfqpoint{8.285017in}{1.838978in}}%
\pgfpathcurveto{\pgfqpoint{8.285017in}{1.833934in}}{\pgfqpoint{8.287021in}{1.829097in}}{\pgfqpoint{8.290588in}{1.825530in}}%
\pgfpathcurveto{\pgfqpoint{8.294154in}{1.821964in}}{\pgfqpoint{8.298992in}{1.819960in}}{\pgfqpoint{8.304035in}{1.819960in}}%
\pgfpathclose%
\pgfusepath{fill}%
\end{pgfscope}%
\begin{pgfscope}%
\pgfpathrectangle{\pgfqpoint{6.572727in}{0.474100in}}{\pgfqpoint{4.227273in}{3.318700in}}%
\pgfusepath{clip}%
\pgfsetbuttcap%
\pgfsetroundjoin%
\definecolor{currentfill}{rgb}{0.127568,0.566949,0.550556}%
\pgfsetfillcolor{currentfill}%
\pgfsetfillopacity{0.700000}%
\pgfsetlinewidth{0.000000pt}%
\definecolor{currentstroke}{rgb}{0.000000,0.000000,0.000000}%
\pgfsetstrokecolor{currentstroke}%
\pgfsetstrokeopacity{0.700000}%
\pgfsetdash{}{0pt}%
\pgfpathmoveto{\pgfqpoint{7.945251in}{1.737565in}}%
\pgfpathcurveto{\pgfqpoint{7.950295in}{1.737565in}}{\pgfqpoint{7.955132in}{1.739569in}}{\pgfqpoint{7.958699in}{1.743136in}}%
\pgfpathcurveto{\pgfqpoint{7.962265in}{1.746702in}}{\pgfqpoint{7.964269in}{1.751540in}}{\pgfqpoint{7.964269in}{1.756584in}}%
\pgfpathcurveto{\pgfqpoint{7.964269in}{1.761627in}}{\pgfqpoint{7.962265in}{1.766465in}}{\pgfqpoint{7.958699in}{1.770031in}}%
\pgfpathcurveto{\pgfqpoint{7.955132in}{1.773598in}}{\pgfqpoint{7.950295in}{1.775602in}}{\pgfqpoint{7.945251in}{1.775602in}}%
\pgfpathcurveto{\pgfqpoint{7.940207in}{1.775602in}}{\pgfqpoint{7.935370in}{1.773598in}}{\pgfqpoint{7.931803in}{1.770031in}}%
\pgfpathcurveto{\pgfqpoint{7.928237in}{1.766465in}}{\pgfqpoint{7.926233in}{1.761627in}}{\pgfqpoint{7.926233in}{1.756584in}}%
\pgfpathcurveto{\pgfqpoint{7.926233in}{1.751540in}}{\pgfqpoint{7.928237in}{1.746702in}}{\pgfqpoint{7.931803in}{1.743136in}}%
\pgfpathcurveto{\pgfqpoint{7.935370in}{1.739569in}}{\pgfqpoint{7.940207in}{1.737565in}}{\pgfqpoint{7.945251in}{1.737565in}}%
\pgfpathclose%
\pgfusepath{fill}%
\end{pgfscope}%
\begin{pgfscope}%
\pgfpathrectangle{\pgfqpoint{6.572727in}{0.474100in}}{\pgfqpoint{4.227273in}{3.318700in}}%
\pgfusepath{clip}%
\pgfsetbuttcap%
\pgfsetroundjoin%
\definecolor{currentfill}{rgb}{0.127568,0.566949,0.550556}%
\pgfsetfillcolor{currentfill}%
\pgfsetfillopacity{0.700000}%
\pgfsetlinewidth{0.000000pt}%
\definecolor{currentstroke}{rgb}{0.000000,0.000000,0.000000}%
\pgfsetstrokecolor{currentstroke}%
\pgfsetstrokeopacity{0.700000}%
\pgfsetdash{}{0pt}%
\pgfpathmoveto{\pgfqpoint{8.374078in}{2.941125in}}%
\pgfpathcurveto{\pgfqpoint{8.379122in}{2.941125in}}{\pgfqpoint{8.383960in}{2.943129in}}{\pgfqpoint{8.387526in}{2.946696in}}%
\pgfpathcurveto{\pgfqpoint{8.391092in}{2.950262in}}{\pgfqpoint{8.393096in}{2.955100in}}{\pgfqpoint{8.393096in}{2.960143in}}%
\pgfpathcurveto{\pgfqpoint{8.393096in}{2.965187in}}{\pgfqpoint{8.391092in}{2.970025in}}{\pgfqpoint{8.387526in}{2.973591in}}%
\pgfpathcurveto{\pgfqpoint{8.383960in}{2.977158in}}{\pgfqpoint{8.379122in}{2.979162in}}{\pgfqpoint{8.374078in}{2.979162in}}%
\pgfpathcurveto{\pgfqpoint{8.369035in}{2.979162in}}{\pgfqpoint{8.364197in}{2.977158in}}{\pgfqpoint{8.360630in}{2.973591in}}%
\pgfpathcurveto{\pgfqpoint{8.357064in}{2.970025in}}{\pgfqpoint{8.355060in}{2.965187in}}{\pgfqpoint{8.355060in}{2.960143in}}%
\pgfpathcurveto{\pgfqpoint{8.355060in}{2.955100in}}{\pgfqpoint{8.357064in}{2.950262in}}{\pgfqpoint{8.360630in}{2.946696in}}%
\pgfpathcurveto{\pgfqpoint{8.364197in}{2.943129in}}{\pgfqpoint{8.369035in}{2.941125in}}{\pgfqpoint{8.374078in}{2.941125in}}%
\pgfpathclose%
\pgfusepath{fill}%
\end{pgfscope}%
\begin{pgfscope}%
\pgfpathrectangle{\pgfqpoint{6.572727in}{0.474100in}}{\pgfqpoint{4.227273in}{3.318700in}}%
\pgfusepath{clip}%
\pgfsetbuttcap%
\pgfsetroundjoin%
\definecolor{currentfill}{rgb}{0.127568,0.566949,0.550556}%
\pgfsetfillcolor{currentfill}%
\pgfsetfillopacity{0.700000}%
\pgfsetlinewidth{0.000000pt}%
\definecolor{currentstroke}{rgb}{0.000000,0.000000,0.000000}%
\pgfsetstrokecolor{currentstroke}%
\pgfsetstrokeopacity{0.700000}%
\pgfsetdash{}{0pt}%
\pgfpathmoveto{\pgfqpoint{8.391447in}{1.884211in}}%
\pgfpathcurveto{\pgfqpoint{8.396491in}{1.884211in}}{\pgfqpoint{8.401329in}{1.886215in}}{\pgfqpoint{8.404895in}{1.889781in}}%
\pgfpathcurveto{\pgfqpoint{8.408461in}{1.893348in}}{\pgfqpoint{8.410465in}{1.898186in}}{\pgfqpoint{8.410465in}{1.903229in}}%
\pgfpathcurveto{\pgfqpoint{8.410465in}{1.908273in}}{\pgfqpoint{8.408461in}{1.913111in}}{\pgfqpoint{8.404895in}{1.916677in}}%
\pgfpathcurveto{\pgfqpoint{8.401329in}{1.920243in}}{\pgfqpoint{8.396491in}{1.922247in}}{\pgfqpoint{8.391447in}{1.922247in}}%
\pgfpathcurveto{\pgfqpoint{8.386403in}{1.922247in}}{\pgfqpoint{8.381566in}{1.920243in}}{\pgfqpoint{8.377999in}{1.916677in}}%
\pgfpathcurveto{\pgfqpoint{8.374433in}{1.913111in}}{\pgfqpoint{8.372429in}{1.908273in}}{\pgfqpoint{8.372429in}{1.903229in}}%
\pgfpathcurveto{\pgfqpoint{8.372429in}{1.898186in}}{\pgfqpoint{8.374433in}{1.893348in}}{\pgfqpoint{8.377999in}{1.889781in}}%
\pgfpathcurveto{\pgfqpoint{8.381566in}{1.886215in}}{\pgfqpoint{8.386403in}{1.884211in}}{\pgfqpoint{8.391447in}{1.884211in}}%
\pgfpathclose%
\pgfusepath{fill}%
\end{pgfscope}%
\begin{pgfscope}%
\pgfpathrectangle{\pgfqpoint{6.572727in}{0.474100in}}{\pgfqpoint{4.227273in}{3.318700in}}%
\pgfusepath{clip}%
\pgfsetbuttcap%
\pgfsetroundjoin%
\definecolor{currentfill}{rgb}{0.993248,0.906157,0.143936}%
\pgfsetfillcolor{currentfill}%
\pgfsetfillopacity{0.700000}%
\pgfsetlinewidth{0.000000pt}%
\definecolor{currentstroke}{rgb}{0.000000,0.000000,0.000000}%
\pgfsetstrokecolor{currentstroke}%
\pgfsetstrokeopacity{0.700000}%
\pgfsetdash{}{0pt}%
\pgfpathmoveto{\pgfqpoint{9.050304in}{1.386428in}}%
\pgfpathcurveto{\pgfqpoint{9.055348in}{1.386428in}}{\pgfqpoint{9.060186in}{1.388432in}}{\pgfqpoint{9.063752in}{1.391998in}}%
\pgfpathcurveto{\pgfqpoint{9.067319in}{1.395564in}}{\pgfqpoint{9.069322in}{1.400402in}}{\pgfqpoint{9.069322in}{1.405446in}}%
\pgfpathcurveto{\pgfqpoint{9.069322in}{1.410489in}}{\pgfqpoint{9.067319in}{1.415327in}}{\pgfqpoint{9.063752in}{1.418894in}}%
\pgfpathcurveto{\pgfqpoint{9.060186in}{1.422460in}}{\pgfqpoint{9.055348in}{1.424464in}}{\pgfqpoint{9.050304in}{1.424464in}}%
\pgfpathcurveto{\pgfqpoint{9.045261in}{1.424464in}}{\pgfqpoint{9.040423in}{1.422460in}}{\pgfqpoint{9.036856in}{1.418894in}}%
\pgfpathcurveto{\pgfqpoint{9.033290in}{1.415327in}}{\pgfqpoint{9.031286in}{1.410489in}}{\pgfqpoint{9.031286in}{1.405446in}}%
\pgfpathcurveto{\pgfqpoint{9.031286in}{1.400402in}}{\pgfqpoint{9.033290in}{1.395564in}}{\pgfqpoint{9.036856in}{1.391998in}}%
\pgfpathcurveto{\pgfqpoint{9.040423in}{1.388432in}}{\pgfqpoint{9.045261in}{1.386428in}}{\pgfqpoint{9.050304in}{1.386428in}}%
\pgfpathclose%
\pgfusepath{fill}%
\end{pgfscope}%
\begin{pgfscope}%
\pgfpathrectangle{\pgfqpoint{6.572727in}{0.474100in}}{\pgfqpoint{4.227273in}{3.318700in}}%
\pgfusepath{clip}%
\pgfsetbuttcap%
\pgfsetroundjoin%
\definecolor{currentfill}{rgb}{0.127568,0.566949,0.550556}%
\pgfsetfillcolor{currentfill}%
\pgfsetfillopacity{0.700000}%
\pgfsetlinewidth{0.000000pt}%
\definecolor{currentstroke}{rgb}{0.000000,0.000000,0.000000}%
\pgfsetstrokecolor{currentstroke}%
\pgfsetstrokeopacity{0.700000}%
\pgfsetdash{}{0pt}%
\pgfpathmoveto{\pgfqpoint{8.129471in}{1.783937in}}%
\pgfpathcurveto{\pgfqpoint{8.134515in}{1.783937in}}{\pgfqpoint{8.139353in}{1.785941in}}{\pgfqpoint{8.142919in}{1.789507in}}%
\pgfpathcurveto{\pgfqpoint{8.146486in}{1.793074in}}{\pgfqpoint{8.148490in}{1.797912in}}{\pgfqpoint{8.148490in}{1.802955in}}%
\pgfpathcurveto{\pgfqpoint{8.148490in}{1.807999in}}{\pgfqpoint{8.146486in}{1.812837in}}{\pgfqpoint{8.142919in}{1.816403in}}%
\pgfpathcurveto{\pgfqpoint{8.139353in}{1.819970in}}{\pgfqpoint{8.134515in}{1.821973in}}{\pgfqpoint{8.129471in}{1.821973in}}%
\pgfpathcurveto{\pgfqpoint{8.124428in}{1.821973in}}{\pgfqpoint{8.119590in}{1.819970in}}{\pgfqpoint{8.116024in}{1.816403in}}%
\pgfpathcurveto{\pgfqpoint{8.112457in}{1.812837in}}{\pgfqpoint{8.110453in}{1.807999in}}{\pgfqpoint{8.110453in}{1.802955in}}%
\pgfpathcurveto{\pgfqpoint{8.110453in}{1.797912in}}{\pgfqpoint{8.112457in}{1.793074in}}{\pgfqpoint{8.116024in}{1.789507in}}%
\pgfpathcurveto{\pgfqpoint{8.119590in}{1.785941in}}{\pgfqpoint{8.124428in}{1.783937in}}{\pgfqpoint{8.129471in}{1.783937in}}%
\pgfpathclose%
\pgfusepath{fill}%
\end{pgfscope}%
\begin{pgfscope}%
\pgfpathrectangle{\pgfqpoint{6.572727in}{0.474100in}}{\pgfqpoint{4.227273in}{3.318700in}}%
\pgfusepath{clip}%
\pgfsetbuttcap%
\pgfsetroundjoin%
\definecolor{currentfill}{rgb}{0.127568,0.566949,0.550556}%
\pgfsetfillcolor{currentfill}%
\pgfsetfillopacity{0.700000}%
\pgfsetlinewidth{0.000000pt}%
\definecolor{currentstroke}{rgb}{0.000000,0.000000,0.000000}%
\pgfsetstrokecolor{currentstroke}%
\pgfsetstrokeopacity{0.700000}%
\pgfsetdash{}{0pt}%
\pgfpathmoveto{\pgfqpoint{7.962267in}{2.481758in}}%
\pgfpathcurveto{\pgfqpoint{7.967311in}{2.481758in}}{\pgfqpoint{7.972149in}{2.483762in}}{\pgfqpoint{7.975715in}{2.487329in}}%
\pgfpathcurveto{\pgfqpoint{7.979282in}{2.490895in}}{\pgfqpoint{7.981286in}{2.495733in}}{\pgfqpoint{7.981286in}{2.500776in}}%
\pgfpathcurveto{\pgfqpoint{7.981286in}{2.505820in}}{\pgfqpoint{7.979282in}{2.510658in}}{\pgfqpoint{7.975715in}{2.514224in}}%
\pgfpathcurveto{\pgfqpoint{7.972149in}{2.517791in}}{\pgfqpoint{7.967311in}{2.519795in}}{\pgfqpoint{7.962267in}{2.519795in}}%
\pgfpathcurveto{\pgfqpoint{7.957224in}{2.519795in}}{\pgfqpoint{7.952386in}{2.517791in}}{\pgfqpoint{7.948820in}{2.514224in}}%
\pgfpathcurveto{\pgfqpoint{7.945253in}{2.510658in}}{\pgfqpoint{7.943249in}{2.505820in}}{\pgfqpoint{7.943249in}{2.500776in}}%
\pgfpathcurveto{\pgfqpoint{7.943249in}{2.495733in}}{\pgfqpoint{7.945253in}{2.490895in}}{\pgfqpoint{7.948820in}{2.487329in}}%
\pgfpathcurveto{\pgfqpoint{7.952386in}{2.483762in}}{\pgfqpoint{7.957224in}{2.481758in}}{\pgfqpoint{7.962267in}{2.481758in}}%
\pgfpathclose%
\pgfusepath{fill}%
\end{pgfscope}%
\begin{pgfscope}%
\pgfpathrectangle{\pgfqpoint{6.572727in}{0.474100in}}{\pgfqpoint{4.227273in}{3.318700in}}%
\pgfusepath{clip}%
\pgfsetbuttcap%
\pgfsetroundjoin%
\definecolor{currentfill}{rgb}{0.993248,0.906157,0.143936}%
\pgfsetfillcolor{currentfill}%
\pgfsetfillopacity{0.700000}%
\pgfsetlinewidth{0.000000pt}%
\definecolor{currentstroke}{rgb}{0.000000,0.000000,0.000000}%
\pgfsetstrokecolor{currentstroke}%
\pgfsetstrokeopacity{0.700000}%
\pgfsetdash{}{0pt}%
\pgfpathmoveto{\pgfqpoint{9.766269in}{2.089141in}}%
\pgfpathcurveto{\pgfqpoint{9.771313in}{2.089141in}}{\pgfqpoint{9.776151in}{2.091145in}}{\pgfqpoint{9.779717in}{2.094712in}}%
\pgfpathcurveto{\pgfqpoint{9.783284in}{2.098278in}}{\pgfqpoint{9.785287in}{2.103116in}}{\pgfqpoint{9.785287in}{2.108160in}}%
\pgfpathcurveto{\pgfqpoint{9.785287in}{2.113203in}}{\pgfqpoint{9.783284in}{2.118041in}}{\pgfqpoint{9.779717in}{2.121607in}}%
\pgfpathcurveto{\pgfqpoint{9.776151in}{2.125174in}}{\pgfqpoint{9.771313in}{2.127178in}}{\pgfqpoint{9.766269in}{2.127178in}}%
\pgfpathcurveto{\pgfqpoint{9.761226in}{2.127178in}}{\pgfqpoint{9.756388in}{2.125174in}}{\pgfqpoint{9.752821in}{2.121607in}}%
\pgfpathcurveto{\pgfqpoint{9.749255in}{2.118041in}}{\pgfqpoint{9.747251in}{2.113203in}}{\pgfqpoint{9.747251in}{2.108160in}}%
\pgfpathcurveto{\pgfqpoint{9.747251in}{2.103116in}}{\pgfqpoint{9.749255in}{2.098278in}}{\pgfqpoint{9.752821in}{2.094712in}}%
\pgfpathcurveto{\pgfqpoint{9.756388in}{2.091145in}}{\pgfqpoint{9.761226in}{2.089141in}}{\pgfqpoint{9.766269in}{2.089141in}}%
\pgfpathclose%
\pgfusepath{fill}%
\end{pgfscope}%
\begin{pgfscope}%
\pgfpathrectangle{\pgfqpoint{6.572727in}{0.474100in}}{\pgfqpoint{4.227273in}{3.318700in}}%
\pgfusepath{clip}%
\pgfsetbuttcap%
\pgfsetroundjoin%
\definecolor{currentfill}{rgb}{0.127568,0.566949,0.550556}%
\pgfsetfillcolor{currentfill}%
\pgfsetfillopacity{0.700000}%
\pgfsetlinewidth{0.000000pt}%
\definecolor{currentstroke}{rgb}{0.000000,0.000000,0.000000}%
\pgfsetstrokecolor{currentstroke}%
\pgfsetstrokeopacity{0.700000}%
\pgfsetdash{}{0pt}%
\pgfpathmoveto{\pgfqpoint{7.789275in}{1.797905in}}%
\pgfpathcurveto{\pgfqpoint{7.794319in}{1.797905in}}{\pgfqpoint{7.799157in}{1.799909in}}{\pgfqpoint{7.802723in}{1.803475in}}%
\pgfpathcurveto{\pgfqpoint{7.806289in}{1.807042in}}{\pgfqpoint{7.808293in}{1.811880in}}{\pgfqpoint{7.808293in}{1.816923in}}%
\pgfpathcurveto{\pgfqpoint{7.808293in}{1.821967in}}{\pgfqpoint{7.806289in}{1.826805in}}{\pgfqpoint{7.802723in}{1.830371in}}%
\pgfpathcurveto{\pgfqpoint{7.799157in}{1.833937in}}{\pgfqpoint{7.794319in}{1.835941in}}{\pgfqpoint{7.789275in}{1.835941in}}%
\pgfpathcurveto{\pgfqpoint{7.784232in}{1.835941in}}{\pgfqpoint{7.779394in}{1.833937in}}{\pgfqpoint{7.775827in}{1.830371in}}%
\pgfpathcurveto{\pgfqpoint{7.772261in}{1.826805in}}{\pgfqpoint{7.770257in}{1.821967in}}{\pgfqpoint{7.770257in}{1.816923in}}%
\pgfpathcurveto{\pgfqpoint{7.770257in}{1.811880in}}{\pgfqpoint{7.772261in}{1.807042in}}{\pgfqpoint{7.775827in}{1.803475in}}%
\pgfpathcurveto{\pgfqpoint{7.779394in}{1.799909in}}{\pgfqpoint{7.784232in}{1.797905in}}{\pgfqpoint{7.789275in}{1.797905in}}%
\pgfpathclose%
\pgfusepath{fill}%
\end{pgfscope}%
\begin{pgfscope}%
\pgfpathrectangle{\pgfqpoint{6.572727in}{0.474100in}}{\pgfqpoint{4.227273in}{3.318700in}}%
\pgfusepath{clip}%
\pgfsetbuttcap%
\pgfsetroundjoin%
\definecolor{currentfill}{rgb}{0.993248,0.906157,0.143936}%
\pgfsetfillcolor{currentfill}%
\pgfsetfillopacity{0.700000}%
\pgfsetlinewidth{0.000000pt}%
\definecolor{currentstroke}{rgb}{0.000000,0.000000,0.000000}%
\pgfsetstrokecolor{currentstroke}%
\pgfsetstrokeopacity{0.700000}%
\pgfsetdash{}{0pt}%
\pgfpathmoveto{\pgfqpoint{10.206277in}{1.340748in}}%
\pgfpathcurveto{\pgfqpoint{10.211320in}{1.340748in}}{\pgfqpoint{10.216158in}{1.342751in}}{\pgfqpoint{10.219724in}{1.346318in}}%
\pgfpathcurveto{\pgfqpoint{10.223291in}{1.349884in}}{\pgfqpoint{10.225295in}{1.354722in}}{\pgfqpoint{10.225295in}{1.359766in}}%
\pgfpathcurveto{\pgfqpoint{10.225295in}{1.364809in}}{\pgfqpoint{10.223291in}{1.369647in}}{\pgfqpoint{10.219724in}{1.373214in}}%
\pgfpathcurveto{\pgfqpoint{10.216158in}{1.376780in}}{\pgfqpoint{10.211320in}{1.378784in}}{\pgfqpoint{10.206277in}{1.378784in}}%
\pgfpathcurveto{\pgfqpoint{10.201233in}{1.378784in}}{\pgfqpoint{10.196395in}{1.376780in}}{\pgfqpoint{10.192829in}{1.373214in}}%
\pgfpathcurveto{\pgfqpoint{10.189262in}{1.369647in}}{\pgfqpoint{10.187258in}{1.364809in}}{\pgfqpoint{10.187258in}{1.359766in}}%
\pgfpathcurveto{\pgfqpoint{10.187258in}{1.354722in}}{\pgfqpoint{10.189262in}{1.349884in}}{\pgfqpoint{10.192829in}{1.346318in}}%
\pgfpathcurveto{\pgfqpoint{10.196395in}{1.342751in}}{\pgfqpoint{10.201233in}{1.340748in}}{\pgfqpoint{10.206277in}{1.340748in}}%
\pgfpathclose%
\pgfusepath{fill}%
\end{pgfscope}%
\begin{pgfscope}%
\pgfpathrectangle{\pgfqpoint{6.572727in}{0.474100in}}{\pgfqpoint{4.227273in}{3.318700in}}%
\pgfusepath{clip}%
\pgfsetbuttcap%
\pgfsetroundjoin%
\definecolor{currentfill}{rgb}{0.127568,0.566949,0.550556}%
\pgfsetfillcolor{currentfill}%
\pgfsetfillopacity{0.700000}%
\pgfsetlinewidth{0.000000pt}%
\definecolor{currentstroke}{rgb}{0.000000,0.000000,0.000000}%
\pgfsetstrokecolor{currentstroke}%
\pgfsetstrokeopacity{0.700000}%
\pgfsetdash{}{0pt}%
\pgfpathmoveto{\pgfqpoint{7.979880in}{1.698175in}}%
\pgfpathcurveto{\pgfqpoint{7.984924in}{1.698175in}}{\pgfqpoint{7.989762in}{1.700179in}}{\pgfqpoint{7.993328in}{1.703745in}}%
\pgfpathcurveto{\pgfqpoint{7.996895in}{1.707312in}}{\pgfqpoint{7.998898in}{1.712150in}}{\pgfqpoint{7.998898in}{1.717193in}}%
\pgfpathcurveto{\pgfqpoint{7.998898in}{1.722237in}}{\pgfqpoint{7.996895in}{1.727075in}}{\pgfqpoint{7.993328in}{1.730641in}}%
\pgfpathcurveto{\pgfqpoint{7.989762in}{1.734208in}}{\pgfqpoint{7.984924in}{1.736211in}}{\pgfqpoint{7.979880in}{1.736211in}}%
\pgfpathcurveto{\pgfqpoint{7.974837in}{1.736211in}}{\pgfqpoint{7.969999in}{1.734208in}}{\pgfqpoint{7.966432in}{1.730641in}}%
\pgfpathcurveto{\pgfqpoint{7.962866in}{1.727075in}}{\pgfqpoint{7.960862in}{1.722237in}}{\pgfqpoint{7.960862in}{1.717193in}}%
\pgfpathcurveto{\pgfqpoint{7.960862in}{1.712150in}}{\pgfqpoint{7.962866in}{1.707312in}}{\pgfqpoint{7.966432in}{1.703745in}}%
\pgfpathcurveto{\pgfqpoint{7.969999in}{1.700179in}}{\pgfqpoint{7.974837in}{1.698175in}}{\pgfqpoint{7.979880in}{1.698175in}}%
\pgfpathclose%
\pgfusepath{fill}%
\end{pgfscope}%
\begin{pgfscope}%
\pgfpathrectangle{\pgfqpoint{6.572727in}{0.474100in}}{\pgfqpoint{4.227273in}{3.318700in}}%
\pgfusepath{clip}%
\pgfsetbuttcap%
\pgfsetroundjoin%
\definecolor{currentfill}{rgb}{0.127568,0.566949,0.550556}%
\pgfsetfillcolor{currentfill}%
\pgfsetfillopacity{0.700000}%
\pgfsetlinewidth{0.000000pt}%
\definecolor{currentstroke}{rgb}{0.000000,0.000000,0.000000}%
\pgfsetstrokecolor{currentstroke}%
\pgfsetstrokeopacity{0.700000}%
\pgfsetdash{}{0pt}%
\pgfpathmoveto{\pgfqpoint{7.675073in}{2.953896in}}%
\pgfpathcurveto{\pgfqpoint{7.680117in}{2.953896in}}{\pgfqpoint{7.684955in}{2.955900in}}{\pgfqpoint{7.688521in}{2.959466in}}%
\pgfpathcurveto{\pgfqpoint{7.692088in}{2.963033in}}{\pgfqpoint{7.694091in}{2.967870in}}{\pgfqpoint{7.694091in}{2.972914in}}%
\pgfpathcurveto{\pgfqpoint{7.694091in}{2.977958in}}{\pgfqpoint{7.692088in}{2.982796in}}{\pgfqpoint{7.688521in}{2.986362in}}%
\pgfpathcurveto{\pgfqpoint{7.684955in}{2.989928in}}{\pgfqpoint{7.680117in}{2.991932in}}{\pgfqpoint{7.675073in}{2.991932in}}%
\pgfpathcurveto{\pgfqpoint{7.670030in}{2.991932in}}{\pgfqpoint{7.665192in}{2.989928in}}{\pgfqpoint{7.661625in}{2.986362in}}%
\pgfpathcurveto{\pgfqpoint{7.658059in}{2.982796in}}{\pgfqpoint{7.656055in}{2.977958in}}{\pgfqpoint{7.656055in}{2.972914in}}%
\pgfpathcurveto{\pgfqpoint{7.656055in}{2.967870in}}{\pgfqpoint{7.658059in}{2.963033in}}{\pgfqpoint{7.661625in}{2.959466in}}%
\pgfpathcurveto{\pgfqpoint{7.665192in}{2.955900in}}{\pgfqpoint{7.670030in}{2.953896in}}{\pgfqpoint{7.675073in}{2.953896in}}%
\pgfpathclose%
\pgfusepath{fill}%
\end{pgfscope}%
\begin{pgfscope}%
\pgfpathrectangle{\pgfqpoint{6.572727in}{0.474100in}}{\pgfqpoint{4.227273in}{3.318700in}}%
\pgfusepath{clip}%
\pgfsetbuttcap%
\pgfsetroundjoin%
\definecolor{currentfill}{rgb}{0.127568,0.566949,0.550556}%
\pgfsetfillcolor{currentfill}%
\pgfsetfillopacity{0.700000}%
\pgfsetlinewidth{0.000000pt}%
\definecolor{currentstroke}{rgb}{0.000000,0.000000,0.000000}%
\pgfsetstrokecolor{currentstroke}%
\pgfsetstrokeopacity{0.700000}%
\pgfsetdash{}{0pt}%
\pgfpathmoveto{\pgfqpoint{8.025314in}{2.734172in}}%
\pgfpathcurveto{\pgfqpoint{8.030358in}{2.734172in}}{\pgfqpoint{8.035195in}{2.736176in}}{\pgfqpoint{8.038762in}{2.739742in}}%
\pgfpathcurveto{\pgfqpoint{8.042328in}{2.743308in}}{\pgfqpoint{8.044332in}{2.748146in}}{\pgfqpoint{8.044332in}{2.753190in}}%
\pgfpathcurveto{\pgfqpoint{8.044332in}{2.758233in}}{\pgfqpoint{8.042328in}{2.763071in}}{\pgfqpoint{8.038762in}{2.766638in}}%
\pgfpathcurveto{\pgfqpoint{8.035195in}{2.770204in}}{\pgfqpoint{8.030358in}{2.772208in}}{\pgfqpoint{8.025314in}{2.772208in}}%
\pgfpathcurveto{\pgfqpoint{8.020270in}{2.772208in}}{\pgfqpoint{8.015432in}{2.770204in}}{\pgfqpoint{8.011866in}{2.766638in}}%
\pgfpathcurveto{\pgfqpoint{8.008300in}{2.763071in}}{\pgfqpoint{8.006296in}{2.758233in}}{\pgfqpoint{8.006296in}{2.753190in}}%
\pgfpathcurveto{\pgfqpoint{8.006296in}{2.748146in}}{\pgfqpoint{8.008300in}{2.743308in}}{\pgfqpoint{8.011866in}{2.739742in}}%
\pgfpathcurveto{\pgfqpoint{8.015432in}{2.736176in}}{\pgfqpoint{8.020270in}{2.734172in}}{\pgfqpoint{8.025314in}{2.734172in}}%
\pgfpathclose%
\pgfusepath{fill}%
\end{pgfscope}%
\begin{pgfscope}%
\pgfpathrectangle{\pgfqpoint{6.572727in}{0.474100in}}{\pgfqpoint{4.227273in}{3.318700in}}%
\pgfusepath{clip}%
\pgfsetbuttcap%
\pgfsetroundjoin%
\definecolor{currentfill}{rgb}{0.127568,0.566949,0.550556}%
\pgfsetfillcolor{currentfill}%
\pgfsetfillopacity{0.700000}%
\pgfsetlinewidth{0.000000pt}%
\definecolor{currentstroke}{rgb}{0.000000,0.000000,0.000000}%
\pgfsetstrokecolor{currentstroke}%
\pgfsetstrokeopacity{0.700000}%
\pgfsetdash{}{0pt}%
\pgfpathmoveto{\pgfqpoint{7.433749in}{1.104281in}}%
\pgfpathcurveto{\pgfqpoint{7.438793in}{1.104281in}}{\pgfqpoint{7.443631in}{1.106285in}}{\pgfqpoint{7.447197in}{1.109852in}}%
\pgfpathcurveto{\pgfqpoint{7.450763in}{1.113418in}}{\pgfqpoint{7.452767in}{1.118256in}}{\pgfqpoint{7.452767in}{1.123300in}}%
\pgfpathcurveto{\pgfqpoint{7.452767in}{1.128343in}}{\pgfqpoint{7.450763in}{1.133181in}}{\pgfqpoint{7.447197in}{1.136747in}}%
\pgfpathcurveto{\pgfqpoint{7.443631in}{1.140314in}}{\pgfqpoint{7.438793in}{1.142318in}}{\pgfqpoint{7.433749in}{1.142318in}}%
\pgfpathcurveto{\pgfqpoint{7.428705in}{1.142318in}}{\pgfqpoint{7.423868in}{1.140314in}}{\pgfqpoint{7.420301in}{1.136747in}}%
\pgfpathcurveto{\pgfqpoint{7.416735in}{1.133181in}}{\pgfqpoint{7.414731in}{1.128343in}}{\pgfqpoint{7.414731in}{1.123300in}}%
\pgfpathcurveto{\pgfqpoint{7.414731in}{1.118256in}}{\pgfqpoint{7.416735in}{1.113418in}}{\pgfqpoint{7.420301in}{1.109852in}}%
\pgfpathcurveto{\pgfqpoint{7.423868in}{1.106285in}}{\pgfqpoint{7.428705in}{1.104281in}}{\pgfqpoint{7.433749in}{1.104281in}}%
\pgfpathclose%
\pgfusepath{fill}%
\end{pgfscope}%
\begin{pgfscope}%
\pgfpathrectangle{\pgfqpoint{6.572727in}{0.474100in}}{\pgfqpoint{4.227273in}{3.318700in}}%
\pgfusepath{clip}%
\pgfsetbuttcap%
\pgfsetroundjoin%
\definecolor{currentfill}{rgb}{0.127568,0.566949,0.550556}%
\pgfsetfillcolor{currentfill}%
\pgfsetfillopacity{0.700000}%
\pgfsetlinewidth{0.000000pt}%
\definecolor{currentstroke}{rgb}{0.000000,0.000000,0.000000}%
\pgfsetstrokecolor{currentstroke}%
\pgfsetstrokeopacity{0.700000}%
\pgfsetdash{}{0pt}%
\pgfpathmoveto{\pgfqpoint{7.968491in}{1.547670in}}%
\pgfpathcurveto{\pgfqpoint{7.973535in}{1.547670in}}{\pgfqpoint{7.978373in}{1.549674in}}{\pgfqpoint{7.981939in}{1.553241in}}%
\pgfpathcurveto{\pgfqpoint{7.985506in}{1.556807in}}{\pgfqpoint{7.987510in}{1.561645in}}{\pgfqpoint{7.987510in}{1.566688in}}%
\pgfpathcurveto{\pgfqpoint{7.987510in}{1.571732in}}{\pgfqpoint{7.985506in}{1.576570in}}{\pgfqpoint{7.981939in}{1.580136in}}%
\pgfpathcurveto{\pgfqpoint{7.978373in}{1.583703in}}{\pgfqpoint{7.973535in}{1.585707in}}{\pgfqpoint{7.968491in}{1.585707in}}%
\pgfpathcurveto{\pgfqpoint{7.963448in}{1.585707in}}{\pgfqpoint{7.958610in}{1.583703in}}{\pgfqpoint{7.955044in}{1.580136in}}%
\pgfpathcurveto{\pgfqpoint{7.951477in}{1.576570in}}{\pgfqpoint{7.949473in}{1.571732in}}{\pgfqpoint{7.949473in}{1.566688in}}%
\pgfpathcurveto{\pgfqpoint{7.949473in}{1.561645in}}{\pgfqpoint{7.951477in}{1.556807in}}{\pgfqpoint{7.955044in}{1.553241in}}%
\pgfpathcurveto{\pgfqpoint{7.958610in}{1.549674in}}{\pgfqpoint{7.963448in}{1.547670in}}{\pgfqpoint{7.968491in}{1.547670in}}%
\pgfpathclose%
\pgfusepath{fill}%
\end{pgfscope}%
\begin{pgfscope}%
\pgfpathrectangle{\pgfqpoint{6.572727in}{0.474100in}}{\pgfqpoint{4.227273in}{3.318700in}}%
\pgfusepath{clip}%
\pgfsetbuttcap%
\pgfsetroundjoin%
\definecolor{currentfill}{rgb}{0.127568,0.566949,0.550556}%
\pgfsetfillcolor{currentfill}%
\pgfsetfillopacity{0.700000}%
\pgfsetlinewidth{0.000000pt}%
\definecolor{currentstroke}{rgb}{0.000000,0.000000,0.000000}%
\pgfsetstrokecolor{currentstroke}%
\pgfsetstrokeopacity{0.700000}%
\pgfsetdash{}{0pt}%
\pgfpathmoveto{\pgfqpoint{7.977755in}{1.376184in}}%
\pgfpathcurveto{\pgfqpoint{7.982799in}{1.376184in}}{\pgfqpoint{7.987637in}{1.378188in}}{\pgfqpoint{7.991203in}{1.381754in}}%
\pgfpathcurveto{\pgfqpoint{7.994769in}{1.385320in}}{\pgfqpoint{7.996773in}{1.390158in}}{\pgfqpoint{7.996773in}{1.395202in}}%
\pgfpathcurveto{\pgfqpoint{7.996773in}{1.400245in}}{\pgfqpoint{7.994769in}{1.405083in}}{\pgfqpoint{7.991203in}{1.408650in}}%
\pgfpathcurveto{\pgfqpoint{7.987637in}{1.412216in}}{\pgfqpoint{7.982799in}{1.414220in}}{\pgfqpoint{7.977755in}{1.414220in}}%
\pgfpathcurveto{\pgfqpoint{7.972711in}{1.414220in}}{\pgfqpoint{7.967874in}{1.412216in}}{\pgfqpoint{7.964307in}{1.408650in}}%
\pgfpathcurveto{\pgfqpoint{7.960741in}{1.405083in}}{\pgfqpoint{7.958737in}{1.400245in}}{\pgfqpoint{7.958737in}{1.395202in}}%
\pgfpathcurveto{\pgfqpoint{7.958737in}{1.390158in}}{\pgfqpoint{7.960741in}{1.385320in}}{\pgfqpoint{7.964307in}{1.381754in}}%
\pgfpathcurveto{\pgfqpoint{7.967874in}{1.378188in}}{\pgfqpoint{7.972711in}{1.376184in}}{\pgfqpoint{7.977755in}{1.376184in}}%
\pgfpathclose%
\pgfusepath{fill}%
\end{pgfscope}%
\begin{pgfscope}%
\pgfpathrectangle{\pgfqpoint{6.572727in}{0.474100in}}{\pgfqpoint{4.227273in}{3.318700in}}%
\pgfusepath{clip}%
\pgfsetbuttcap%
\pgfsetroundjoin%
\definecolor{currentfill}{rgb}{0.127568,0.566949,0.550556}%
\pgfsetfillcolor{currentfill}%
\pgfsetfillopacity{0.700000}%
\pgfsetlinewidth{0.000000pt}%
\definecolor{currentstroke}{rgb}{0.000000,0.000000,0.000000}%
\pgfsetstrokecolor{currentstroke}%
\pgfsetstrokeopacity{0.700000}%
\pgfsetdash{}{0pt}%
\pgfpathmoveto{\pgfqpoint{8.078050in}{2.949183in}}%
\pgfpathcurveto{\pgfqpoint{8.083094in}{2.949183in}}{\pgfqpoint{8.087932in}{2.951187in}}{\pgfqpoint{8.091498in}{2.954753in}}%
\pgfpathcurveto{\pgfqpoint{8.095064in}{2.958320in}}{\pgfqpoint{8.097068in}{2.963158in}}{\pgfqpoint{8.097068in}{2.968201in}}%
\pgfpathcurveto{\pgfqpoint{8.097068in}{2.973245in}}{\pgfqpoint{8.095064in}{2.978083in}}{\pgfqpoint{8.091498in}{2.981649in}}%
\pgfpathcurveto{\pgfqpoint{8.087932in}{2.985215in}}{\pgfqpoint{8.083094in}{2.987219in}}{\pgfqpoint{8.078050in}{2.987219in}}%
\pgfpathcurveto{\pgfqpoint{8.073007in}{2.987219in}}{\pgfqpoint{8.068169in}{2.985215in}}{\pgfqpoint{8.064602in}{2.981649in}}%
\pgfpathcurveto{\pgfqpoint{8.061036in}{2.978083in}}{\pgfqpoint{8.059032in}{2.973245in}}{\pgfqpoint{8.059032in}{2.968201in}}%
\pgfpathcurveto{\pgfqpoint{8.059032in}{2.963158in}}{\pgfqpoint{8.061036in}{2.958320in}}{\pgfqpoint{8.064602in}{2.954753in}}%
\pgfpathcurveto{\pgfqpoint{8.068169in}{2.951187in}}{\pgfqpoint{8.073007in}{2.949183in}}{\pgfqpoint{8.078050in}{2.949183in}}%
\pgfpathclose%
\pgfusepath{fill}%
\end{pgfscope}%
\begin{pgfscope}%
\pgfpathrectangle{\pgfqpoint{6.572727in}{0.474100in}}{\pgfqpoint{4.227273in}{3.318700in}}%
\pgfusepath{clip}%
\pgfsetbuttcap%
\pgfsetroundjoin%
\definecolor{currentfill}{rgb}{0.127568,0.566949,0.550556}%
\pgfsetfillcolor{currentfill}%
\pgfsetfillopacity{0.700000}%
\pgfsetlinewidth{0.000000pt}%
\definecolor{currentstroke}{rgb}{0.000000,0.000000,0.000000}%
\pgfsetstrokecolor{currentstroke}%
\pgfsetstrokeopacity{0.700000}%
\pgfsetdash{}{0pt}%
\pgfpathmoveto{\pgfqpoint{8.105524in}{1.262040in}}%
\pgfpathcurveto{\pgfqpoint{8.110568in}{1.262040in}}{\pgfqpoint{8.115406in}{1.264044in}}{\pgfqpoint{8.118972in}{1.267611in}}%
\pgfpathcurveto{\pgfqpoint{8.122539in}{1.271177in}}{\pgfqpoint{8.124543in}{1.276015in}}{\pgfqpoint{8.124543in}{1.281059in}}%
\pgfpathcurveto{\pgfqpoint{8.124543in}{1.286102in}}{\pgfqpoint{8.122539in}{1.290940in}}{\pgfqpoint{8.118972in}{1.294506in}}%
\pgfpathcurveto{\pgfqpoint{8.115406in}{1.298073in}}{\pgfqpoint{8.110568in}{1.300077in}}{\pgfqpoint{8.105524in}{1.300077in}}%
\pgfpathcurveto{\pgfqpoint{8.100481in}{1.300077in}}{\pgfqpoint{8.095643in}{1.298073in}}{\pgfqpoint{8.092077in}{1.294506in}}%
\pgfpathcurveto{\pgfqpoint{8.088510in}{1.290940in}}{\pgfqpoint{8.086506in}{1.286102in}}{\pgfqpoint{8.086506in}{1.281059in}}%
\pgfpathcurveto{\pgfqpoint{8.086506in}{1.276015in}}{\pgfqpoint{8.088510in}{1.271177in}}{\pgfqpoint{8.092077in}{1.267611in}}%
\pgfpathcurveto{\pgfqpoint{8.095643in}{1.264044in}}{\pgfqpoint{8.100481in}{1.262040in}}{\pgfqpoint{8.105524in}{1.262040in}}%
\pgfpathclose%
\pgfusepath{fill}%
\end{pgfscope}%
\begin{pgfscope}%
\pgfpathrectangle{\pgfqpoint{6.572727in}{0.474100in}}{\pgfqpoint{4.227273in}{3.318700in}}%
\pgfusepath{clip}%
\pgfsetbuttcap%
\pgfsetroundjoin%
\definecolor{currentfill}{rgb}{0.127568,0.566949,0.550556}%
\pgfsetfillcolor{currentfill}%
\pgfsetfillopacity{0.700000}%
\pgfsetlinewidth{0.000000pt}%
\definecolor{currentstroke}{rgb}{0.000000,0.000000,0.000000}%
\pgfsetstrokecolor{currentstroke}%
\pgfsetstrokeopacity{0.700000}%
\pgfsetdash{}{0pt}%
\pgfpathmoveto{\pgfqpoint{7.643309in}{1.385335in}}%
\pgfpathcurveto{\pgfqpoint{7.648352in}{1.385335in}}{\pgfqpoint{7.653190in}{1.387339in}}{\pgfqpoint{7.656757in}{1.390905in}}%
\pgfpathcurveto{\pgfqpoint{7.660323in}{1.394472in}}{\pgfqpoint{7.662327in}{1.399310in}}{\pgfqpoint{7.662327in}{1.404353in}}%
\pgfpathcurveto{\pgfqpoint{7.662327in}{1.409397in}}{\pgfqpoint{7.660323in}{1.414235in}}{\pgfqpoint{7.656757in}{1.417801in}}%
\pgfpathcurveto{\pgfqpoint{7.653190in}{1.421368in}}{\pgfqpoint{7.648352in}{1.423371in}}{\pgfqpoint{7.643309in}{1.423371in}}%
\pgfpathcurveto{\pgfqpoint{7.638265in}{1.423371in}}{\pgfqpoint{7.633427in}{1.421368in}}{\pgfqpoint{7.629861in}{1.417801in}}%
\pgfpathcurveto{\pgfqpoint{7.626294in}{1.414235in}}{\pgfqpoint{7.624291in}{1.409397in}}{\pgfqpoint{7.624291in}{1.404353in}}%
\pgfpathcurveto{\pgfqpoint{7.624291in}{1.399310in}}{\pgfqpoint{7.626294in}{1.394472in}}{\pgfqpoint{7.629861in}{1.390905in}}%
\pgfpathcurveto{\pgfqpoint{7.633427in}{1.387339in}}{\pgfqpoint{7.638265in}{1.385335in}}{\pgfqpoint{7.643309in}{1.385335in}}%
\pgfpathclose%
\pgfusepath{fill}%
\end{pgfscope}%
\begin{pgfscope}%
\pgfpathrectangle{\pgfqpoint{6.572727in}{0.474100in}}{\pgfqpoint{4.227273in}{3.318700in}}%
\pgfusepath{clip}%
\pgfsetbuttcap%
\pgfsetroundjoin%
\definecolor{currentfill}{rgb}{0.127568,0.566949,0.550556}%
\pgfsetfillcolor{currentfill}%
\pgfsetfillopacity{0.700000}%
\pgfsetlinewidth{0.000000pt}%
\definecolor{currentstroke}{rgb}{0.000000,0.000000,0.000000}%
\pgfsetstrokecolor{currentstroke}%
\pgfsetstrokeopacity{0.700000}%
\pgfsetdash{}{0pt}%
\pgfpathmoveto{\pgfqpoint{7.772328in}{2.354798in}}%
\pgfpathcurveto{\pgfqpoint{7.777372in}{2.354798in}}{\pgfqpoint{7.782210in}{2.356801in}}{\pgfqpoint{7.785776in}{2.360368in}}%
\pgfpathcurveto{\pgfqpoint{7.789343in}{2.363934in}}{\pgfqpoint{7.791347in}{2.368772in}}{\pgfqpoint{7.791347in}{2.373816in}}%
\pgfpathcurveto{\pgfqpoint{7.791347in}{2.378859in}}{\pgfqpoint{7.789343in}{2.383697in}}{\pgfqpoint{7.785776in}{2.387264in}}%
\pgfpathcurveto{\pgfqpoint{7.782210in}{2.390830in}}{\pgfqpoint{7.777372in}{2.392834in}}{\pgfqpoint{7.772328in}{2.392834in}}%
\pgfpathcurveto{\pgfqpoint{7.767285in}{2.392834in}}{\pgfqpoint{7.762447in}{2.390830in}}{\pgfqpoint{7.758881in}{2.387264in}}%
\pgfpathcurveto{\pgfqpoint{7.755314in}{2.383697in}}{\pgfqpoint{7.753310in}{2.378859in}}{\pgfqpoint{7.753310in}{2.373816in}}%
\pgfpathcurveto{\pgfqpoint{7.753310in}{2.368772in}}{\pgfqpoint{7.755314in}{2.363934in}}{\pgfqpoint{7.758881in}{2.360368in}}%
\pgfpathcurveto{\pgfqpoint{7.762447in}{2.356801in}}{\pgfqpoint{7.767285in}{2.354798in}}{\pgfqpoint{7.772328in}{2.354798in}}%
\pgfpathclose%
\pgfusepath{fill}%
\end{pgfscope}%
\begin{pgfscope}%
\pgfpathrectangle{\pgfqpoint{6.572727in}{0.474100in}}{\pgfqpoint{4.227273in}{3.318700in}}%
\pgfusepath{clip}%
\pgfsetbuttcap%
\pgfsetroundjoin%
\definecolor{currentfill}{rgb}{0.127568,0.566949,0.550556}%
\pgfsetfillcolor{currentfill}%
\pgfsetfillopacity{0.700000}%
\pgfsetlinewidth{0.000000pt}%
\definecolor{currentstroke}{rgb}{0.000000,0.000000,0.000000}%
\pgfsetstrokecolor{currentstroke}%
\pgfsetstrokeopacity{0.700000}%
\pgfsetdash{}{0pt}%
\pgfpathmoveto{\pgfqpoint{8.983886in}{2.799000in}}%
\pgfpathcurveto{\pgfqpoint{8.988930in}{2.799000in}}{\pgfqpoint{8.993767in}{2.801004in}}{\pgfqpoint{8.997334in}{2.804571in}}%
\pgfpathcurveto{\pgfqpoint{9.000900in}{2.808137in}}{\pgfqpoint{9.002904in}{2.812975in}}{\pgfqpoint{9.002904in}{2.818019in}}%
\pgfpathcurveto{\pgfqpoint{9.002904in}{2.823062in}}{\pgfqpoint{9.000900in}{2.827900in}}{\pgfqpoint{8.997334in}{2.831467in}}%
\pgfpathcurveto{\pgfqpoint{8.993767in}{2.835033in}}{\pgfqpoint{8.988930in}{2.837037in}}{\pgfqpoint{8.983886in}{2.837037in}}%
\pgfpathcurveto{\pgfqpoint{8.978842in}{2.837037in}}{\pgfqpoint{8.974005in}{2.835033in}}{\pgfqpoint{8.970438in}{2.831467in}}%
\pgfpathcurveto{\pgfqpoint{8.966872in}{2.827900in}}{\pgfqpoint{8.964868in}{2.823062in}}{\pgfqpoint{8.964868in}{2.818019in}}%
\pgfpathcurveto{\pgfqpoint{8.964868in}{2.812975in}}{\pgfqpoint{8.966872in}{2.808137in}}{\pgfqpoint{8.970438in}{2.804571in}}%
\pgfpathcurveto{\pgfqpoint{8.974005in}{2.801004in}}{\pgfqpoint{8.978842in}{2.799000in}}{\pgfqpoint{8.983886in}{2.799000in}}%
\pgfpathclose%
\pgfusepath{fill}%
\end{pgfscope}%
\begin{pgfscope}%
\pgfpathrectangle{\pgfqpoint{6.572727in}{0.474100in}}{\pgfqpoint{4.227273in}{3.318700in}}%
\pgfusepath{clip}%
\pgfsetbuttcap%
\pgfsetroundjoin%
\definecolor{currentfill}{rgb}{0.127568,0.566949,0.550556}%
\pgfsetfillcolor{currentfill}%
\pgfsetfillopacity{0.700000}%
\pgfsetlinewidth{0.000000pt}%
\definecolor{currentstroke}{rgb}{0.000000,0.000000,0.000000}%
\pgfsetstrokecolor{currentstroke}%
\pgfsetstrokeopacity{0.700000}%
\pgfsetdash{}{0pt}%
\pgfpathmoveto{\pgfqpoint{8.129588in}{1.544685in}}%
\pgfpathcurveto{\pgfqpoint{8.134631in}{1.544685in}}{\pgfqpoint{8.139469in}{1.546689in}}{\pgfqpoint{8.143036in}{1.550255in}}%
\pgfpathcurveto{\pgfqpoint{8.146602in}{1.553822in}}{\pgfqpoint{8.148606in}{1.558659in}}{\pgfqpoint{8.148606in}{1.563703in}}%
\pgfpathcurveto{\pgfqpoint{8.148606in}{1.568747in}}{\pgfqpoint{8.146602in}{1.573584in}}{\pgfqpoint{8.143036in}{1.577151in}}%
\pgfpathcurveto{\pgfqpoint{8.139469in}{1.580717in}}{\pgfqpoint{8.134631in}{1.582721in}}{\pgfqpoint{8.129588in}{1.582721in}}%
\pgfpathcurveto{\pgfqpoint{8.124544in}{1.582721in}}{\pgfqpoint{8.119706in}{1.580717in}}{\pgfqpoint{8.116140in}{1.577151in}}%
\pgfpathcurveto{\pgfqpoint{8.112574in}{1.573584in}}{\pgfqpoint{8.110570in}{1.568747in}}{\pgfqpoint{8.110570in}{1.563703in}}%
\pgfpathcurveto{\pgfqpoint{8.110570in}{1.558659in}}{\pgfqpoint{8.112574in}{1.553822in}}{\pgfqpoint{8.116140in}{1.550255in}}%
\pgfpathcurveto{\pgfqpoint{8.119706in}{1.546689in}}{\pgfqpoint{8.124544in}{1.544685in}}{\pgfqpoint{8.129588in}{1.544685in}}%
\pgfpathclose%
\pgfusepath{fill}%
\end{pgfscope}%
\begin{pgfscope}%
\pgfpathrectangle{\pgfqpoint{6.572727in}{0.474100in}}{\pgfqpoint{4.227273in}{3.318700in}}%
\pgfusepath{clip}%
\pgfsetbuttcap%
\pgfsetroundjoin%
\definecolor{currentfill}{rgb}{0.127568,0.566949,0.550556}%
\pgfsetfillcolor{currentfill}%
\pgfsetfillopacity{0.700000}%
\pgfsetlinewidth{0.000000pt}%
\definecolor{currentstroke}{rgb}{0.000000,0.000000,0.000000}%
\pgfsetstrokecolor{currentstroke}%
\pgfsetstrokeopacity{0.700000}%
\pgfsetdash{}{0pt}%
\pgfpathmoveto{\pgfqpoint{8.301748in}{2.940765in}}%
\pgfpathcurveto{\pgfqpoint{8.306792in}{2.940765in}}{\pgfqpoint{8.311630in}{2.942769in}}{\pgfqpoint{8.315196in}{2.946336in}}%
\pgfpathcurveto{\pgfqpoint{8.318762in}{2.949902in}}{\pgfqpoint{8.320766in}{2.954740in}}{\pgfqpoint{8.320766in}{2.959784in}}%
\pgfpathcurveto{\pgfqpoint{8.320766in}{2.964827in}}{\pgfqpoint{8.318762in}{2.969665in}}{\pgfqpoint{8.315196in}{2.973231in}}%
\pgfpathcurveto{\pgfqpoint{8.311630in}{2.976798in}}{\pgfqpoint{8.306792in}{2.978802in}}{\pgfqpoint{8.301748in}{2.978802in}}%
\pgfpathcurveto{\pgfqpoint{8.296704in}{2.978802in}}{\pgfqpoint{8.291867in}{2.976798in}}{\pgfqpoint{8.288300in}{2.973231in}}%
\pgfpathcurveto{\pgfqpoint{8.284734in}{2.969665in}}{\pgfqpoint{8.282730in}{2.964827in}}{\pgfqpoint{8.282730in}{2.959784in}}%
\pgfpathcurveto{\pgfqpoint{8.282730in}{2.954740in}}{\pgfqpoint{8.284734in}{2.949902in}}{\pgfqpoint{8.288300in}{2.946336in}}%
\pgfpathcurveto{\pgfqpoint{8.291867in}{2.942769in}}{\pgfqpoint{8.296704in}{2.940765in}}{\pgfqpoint{8.301748in}{2.940765in}}%
\pgfpathclose%
\pgfusepath{fill}%
\end{pgfscope}%
\begin{pgfscope}%
\pgfpathrectangle{\pgfqpoint{6.572727in}{0.474100in}}{\pgfqpoint{4.227273in}{3.318700in}}%
\pgfusepath{clip}%
\pgfsetbuttcap%
\pgfsetroundjoin%
\definecolor{currentfill}{rgb}{0.993248,0.906157,0.143936}%
\pgfsetfillcolor{currentfill}%
\pgfsetfillopacity{0.700000}%
\pgfsetlinewidth{0.000000pt}%
\definecolor{currentstroke}{rgb}{0.000000,0.000000,0.000000}%
\pgfsetstrokecolor{currentstroke}%
\pgfsetstrokeopacity{0.700000}%
\pgfsetdash{}{0pt}%
\pgfpathmoveto{\pgfqpoint{9.998091in}{1.794103in}}%
\pgfpathcurveto{\pgfqpoint{10.003135in}{1.794103in}}{\pgfqpoint{10.007973in}{1.796107in}}{\pgfqpoint{10.011539in}{1.799673in}}%
\pgfpathcurveto{\pgfqpoint{10.015106in}{1.803240in}}{\pgfqpoint{10.017110in}{1.808077in}}{\pgfqpoint{10.017110in}{1.813121in}}%
\pgfpathcurveto{\pgfqpoint{10.017110in}{1.818165in}}{\pgfqpoint{10.015106in}{1.823002in}}{\pgfqpoint{10.011539in}{1.826569in}}%
\pgfpathcurveto{\pgfqpoint{10.007973in}{1.830135in}}{\pgfqpoint{10.003135in}{1.832139in}}{\pgfqpoint{9.998091in}{1.832139in}}%
\pgfpathcurveto{\pgfqpoint{9.993048in}{1.832139in}}{\pgfqpoint{9.988210in}{1.830135in}}{\pgfqpoint{9.984644in}{1.826569in}}%
\pgfpathcurveto{\pgfqpoint{9.981077in}{1.823002in}}{\pgfqpoint{9.979073in}{1.818165in}}{\pgfqpoint{9.979073in}{1.813121in}}%
\pgfpathcurveto{\pgfqpoint{9.979073in}{1.808077in}}{\pgfqpoint{9.981077in}{1.803240in}}{\pgfqpoint{9.984644in}{1.799673in}}%
\pgfpathcurveto{\pgfqpoint{9.988210in}{1.796107in}}{\pgfqpoint{9.993048in}{1.794103in}}{\pgfqpoint{9.998091in}{1.794103in}}%
\pgfpathclose%
\pgfusepath{fill}%
\end{pgfscope}%
\begin{pgfscope}%
\pgfpathrectangle{\pgfqpoint{6.572727in}{0.474100in}}{\pgfqpoint{4.227273in}{3.318700in}}%
\pgfusepath{clip}%
\pgfsetbuttcap%
\pgfsetroundjoin%
\definecolor{currentfill}{rgb}{0.993248,0.906157,0.143936}%
\pgfsetfillcolor{currentfill}%
\pgfsetfillopacity{0.700000}%
\pgfsetlinewidth{0.000000pt}%
\definecolor{currentstroke}{rgb}{0.000000,0.000000,0.000000}%
\pgfsetstrokecolor{currentstroke}%
\pgfsetstrokeopacity{0.700000}%
\pgfsetdash{}{0pt}%
\pgfpathmoveto{\pgfqpoint{9.226196in}{1.518150in}}%
\pgfpathcurveto{\pgfqpoint{9.231239in}{1.518150in}}{\pgfqpoint{9.236077in}{1.520154in}}{\pgfqpoint{9.239644in}{1.523720in}}%
\pgfpathcurveto{\pgfqpoint{9.243210in}{1.527287in}}{\pgfqpoint{9.245214in}{1.532124in}}{\pgfqpoint{9.245214in}{1.537168in}}%
\pgfpathcurveto{\pgfqpoint{9.245214in}{1.542212in}}{\pgfqpoint{9.243210in}{1.547049in}}{\pgfqpoint{9.239644in}{1.550616in}}%
\pgfpathcurveto{\pgfqpoint{9.236077in}{1.554182in}}{\pgfqpoint{9.231239in}{1.556186in}}{\pgfqpoint{9.226196in}{1.556186in}}%
\pgfpathcurveto{\pgfqpoint{9.221152in}{1.556186in}}{\pgfqpoint{9.216314in}{1.554182in}}{\pgfqpoint{9.212748in}{1.550616in}}%
\pgfpathcurveto{\pgfqpoint{9.209181in}{1.547049in}}{\pgfqpoint{9.207178in}{1.542212in}}{\pgfqpoint{9.207178in}{1.537168in}}%
\pgfpathcurveto{\pgfqpoint{9.207178in}{1.532124in}}{\pgfqpoint{9.209181in}{1.527287in}}{\pgfqpoint{9.212748in}{1.523720in}}%
\pgfpathcurveto{\pgfqpoint{9.216314in}{1.520154in}}{\pgfqpoint{9.221152in}{1.518150in}}{\pgfqpoint{9.226196in}{1.518150in}}%
\pgfpathclose%
\pgfusepath{fill}%
\end{pgfscope}%
\begin{pgfscope}%
\pgfpathrectangle{\pgfqpoint{6.572727in}{0.474100in}}{\pgfqpoint{4.227273in}{3.318700in}}%
\pgfusepath{clip}%
\pgfsetbuttcap%
\pgfsetroundjoin%
\definecolor{currentfill}{rgb}{0.127568,0.566949,0.550556}%
\pgfsetfillcolor{currentfill}%
\pgfsetfillopacity{0.700000}%
\pgfsetlinewidth{0.000000pt}%
\definecolor{currentstroke}{rgb}{0.000000,0.000000,0.000000}%
\pgfsetstrokecolor{currentstroke}%
\pgfsetstrokeopacity{0.700000}%
\pgfsetdash{}{0pt}%
\pgfpathmoveto{\pgfqpoint{7.568149in}{1.933509in}}%
\pgfpathcurveto{\pgfqpoint{7.573193in}{1.933509in}}{\pgfqpoint{7.578031in}{1.935513in}}{\pgfqpoint{7.581597in}{1.939079in}}%
\pgfpathcurveto{\pgfqpoint{7.585164in}{1.942646in}}{\pgfqpoint{7.587168in}{1.947484in}}{\pgfqpoint{7.587168in}{1.952527in}}%
\pgfpathcurveto{\pgfqpoint{7.587168in}{1.957571in}}{\pgfqpoint{7.585164in}{1.962409in}}{\pgfqpoint{7.581597in}{1.965975in}}%
\pgfpathcurveto{\pgfqpoint{7.578031in}{1.969542in}}{\pgfqpoint{7.573193in}{1.971545in}}{\pgfqpoint{7.568149in}{1.971545in}}%
\pgfpathcurveto{\pgfqpoint{7.563106in}{1.971545in}}{\pgfqpoint{7.558268in}{1.969542in}}{\pgfqpoint{7.554702in}{1.965975in}}%
\pgfpathcurveto{\pgfqpoint{7.551135in}{1.962409in}}{\pgfqpoint{7.549131in}{1.957571in}}{\pgfqpoint{7.549131in}{1.952527in}}%
\pgfpathcurveto{\pgfqpoint{7.549131in}{1.947484in}}{\pgfqpoint{7.551135in}{1.942646in}}{\pgfqpoint{7.554702in}{1.939079in}}%
\pgfpathcurveto{\pgfqpoint{7.558268in}{1.935513in}}{\pgfqpoint{7.563106in}{1.933509in}}{\pgfqpoint{7.568149in}{1.933509in}}%
\pgfpathclose%
\pgfusepath{fill}%
\end{pgfscope}%
\begin{pgfscope}%
\pgfpathrectangle{\pgfqpoint{6.572727in}{0.474100in}}{\pgfqpoint{4.227273in}{3.318700in}}%
\pgfusepath{clip}%
\pgfsetbuttcap%
\pgfsetroundjoin%
\definecolor{currentfill}{rgb}{0.127568,0.566949,0.550556}%
\pgfsetfillcolor{currentfill}%
\pgfsetfillopacity{0.700000}%
\pgfsetlinewidth{0.000000pt}%
\definecolor{currentstroke}{rgb}{0.000000,0.000000,0.000000}%
\pgfsetstrokecolor{currentstroke}%
\pgfsetstrokeopacity{0.700000}%
\pgfsetdash{}{0pt}%
\pgfpathmoveto{\pgfqpoint{7.540108in}{2.120459in}}%
\pgfpathcurveto{\pgfqpoint{7.545151in}{2.120459in}}{\pgfqpoint{7.549989in}{2.122462in}}{\pgfqpoint{7.553556in}{2.126029in}}%
\pgfpathcurveto{\pgfqpoint{7.557122in}{2.129595in}}{\pgfqpoint{7.559126in}{2.134433in}}{\pgfqpoint{7.559126in}{2.139477in}}%
\pgfpathcurveto{\pgfqpoint{7.559126in}{2.144520in}}{\pgfqpoint{7.557122in}{2.149358in}}{\pgfqpoint{7.553556in}{2.152925in}}%
\pgfpathcurveto{\pgfqpoint{7.549989in}{2.156491in}}{\pgfqpoint{7.545151in}{2.158495in}}{\pgfqpoint{7.540108in}{2.158495in}}%
\pgfpathcurveto{\pgfqpoint{7.535064in}{2.158495in}}{\pgfqpoint{7.530226in}{2.156491in}}{\pgfqpoint{7.526660in}{2.152925in}}%
\pgfpathcurveto{\pgfqpoint{7.523093in}{2.149358in}}{\pgfqpoint{7.521090in}{2.144520in}}{\pgfqpoint{7.521090in}{2.139477in}}%
\pgfpathcurveto{\pgfqpoint{7.521090in}{2.134433in}}{\pgfqpoint{7.523093in}{2.129595in}}{\pgfqpoint{7.526660in}{2.126029in}}%
\pgfpathcurveto{\pgfqpoint{7.530226in}{2.122462in}}{\pgfqpoint{7.535064in}{2.120459in}}{\pgfqpoint{7.540108in}{2.120459in}}%
\pgfpathclose%
\pgfusepath{fill}%
\end{pgfscope}%
\begin{pgfscope}%
\pgfpathrectangle{\pgfqpoint{6.572727in}{0.474100in}}{\pgfqpoint{4.227273in}{3.318700in}}%
\pgfusepath{clip}%
\pgfsetbuttcap%
\pgfsetroundjoin%
\definecolor{currentfill}{rgb}{0.993248,0.906157,0.143936}%
\pgfsetfillcolor{currentfill}%
\pgfsetfillopacity{0.700000}%
\pgfsetlinewidth{0.000000pt}%
\definecolor{currentstroke}{rgb}{0.000000,0.000000,0.000000}%
\pgfsetstrokecolor{currentstroke}%
\pgfsetstrokeopacity{0.700000}%
\pgfsetdash{}{0pt}%
\pgfpathmoveto{\pgfqpoint{9.661492in}{1.525717in}}%
\pgfpathcurveto{\pgfqpoint{9.666535in}{1.525717in}}{\pgfqpoint{9.671373in}{1.527721in}}{\pgfqpoint{9.674939in}{1.531287in}}%
\pgfpathcurveto{\pgfqpoint{9.678506in}{1.534854in}}{\pgfqpoint{9.680510in}{1.539691in}}{\pgfqpoint{9.680510in}{1.544735in}}%
\pgfpathcurveto{\pgfqpoint{9.680510in}{1.549779in}}{\pgfqpoint{9.678506in}{1.554617in}}{\pgfqpoint{9.674939in}{1.558183in}}%
\pgfpathcurveto{\pgfqpoint{9.671373in}{1.561749in}}{\pgfqpoint{9.666535in}{1.563753in}}{\pgfqpoint{9.661492in}{1.563753in}}%
\pgfpathcurveto{\pgfqpoint{9.656448in}{1.563753in}}{\pgfqpoint{9.651610in}{1.561749in}}{\pgfqpoint{9.648044in}{1.558183in}}%
\pgfpathcurveto{\pgfqpoint{9.644477in}{1.554617in}}{\pgfqpoint{9.642473in}{1.549779in}}{\pgfqpoint{9.642473in}{1.544735in}}%
\pgfpathcurveto{\pgfqpoint{9.642473in}{1.539691in}}{\pgfqpoint{9.644477in}{1.534854in}}{\pgfqpoint{9.648044in}{1.531287in}}%
\pgfpathcurveto{\pgfqpoint{9.651610in}{1.527721in}}{\pgfqpoint{9.656448in}{1.525717in}}{\pgfqpoint{9.661492in}{1.525717in}}%
\pgfpathclose%
\pgfusepath{fill}%
\end{pgfscope}%
\begin{pgfscope}%
\pgfpathrectangle{\pgfqpoint{6.572727in}{0.474100in}}{\pgfqpoint{4.227273in}{3.318700in}}%
\pgfusepath{clip}%
\pgfsetbuttcap%
\pgfsetroundjoin%
\definecolor{currentfill}{rgb}{0.993248,0.906157,0.143936}%
\pgfsetfillcolor{currentfill}%
\pgfsetfillopacity{0.700000}%
\pgfsetlinewidth{0.000000pt}%
\definecolor{currentstroke}{rgb}{0.000000,0.000000,0.000000}%
\pgfsetstrokecolor{currentstroke}%
\pgfsetstrokeopacity{0.700000}%
\pgfsetdash{}{0pt}%
\pgfpathmoveto{\pgfqpoint{9.978806in}{1.311540in}}%
\pgfpathcurveto{\pgfqpoint{9.983850in}{1.311540in}}{\pgfqpoint{9.988688in}{1.313544in}}{\pgfqpoint{9.992254in}{1.317110in}}%
\pgfpathcurveto{\pgfqpoint{9.995821in}{1.320677in}}{\pgfqpoint{9.997825in}{1.325515in}}{\pgfqpoint{9.997825in}{1.330558in}}%
\pgfpathcurveto{\pgfqpoint{9.997825in}{1.335602in}}{\pgfqpoint{9.995821in}{1.340440in}}{\pgfqpoint{9.992254in}{1.344006in}}%
\pgfpathcurveto{\pgfqpoint{9.988688in}{1.347573in}}{\pgfqpoint{9.983850in}{1.349576in}}{\pgfqpoint{9.978806in}{1.349576in}}%
\pgfpathcurveto{\pgfqpoint{9.973763in}{1.349576in}}{\pgfqpoint{9.968925in}{1.347573in}}{\pgfqpoint{9.965359in}{1.344006in}}%
\pgfpathcurveto{\pgfqpoint{9.961792in}{1.340440in}}{\pgfqpoint{9.959788in}{1.335602in}}{\pgfqpoint{9.959788in}{1.330558in}}%
\pgfpathcurveto{\pgfqpoint{9.959788in}{1.325515in}}{\pgfqpoint{9.961792in}{1.320677in}}{\pgfqpoint{9.965359in}{1.317110in}}%
\pgfpathcurveto{\pgfqpoint{9.968925in}{1.313544in}}{\pgfqpoint{9.973763in}{1.311540in}}{\pgfqpoint{9.978806in}{1.311540in}}%
\pgfpathclose%
\pgfusepath{fill}%
\end{pgfscope}%
\begin{pgfscope}%
\pgfpathrectangle{\pgfqpoint{6.572727in}{0.474100in}}{\pgfqpoint{4.227273in}{3.318700in}}%
\pgfusepath{clip}%
\pgfsetbuttcap%
\pgfsetroundjoin%
\definecolor{currentfill}{rgb}{0.127568,0.566949,0.550556}%
\pgfsetfillcolor{currentfill}%
\pgfsetfillopacity{0.700000}%
\pgfsetlinewidth{0.000000pt}%
\definecolor{currentstroke}{rgb}{0.000000,0.000000,0.000000}%
\pgfsetstrokecolor{currentstroke}%
\pgfsetstrokeopacity{0.700000}%
\pgfsetdash{}{0pt}%
\pgfpathmoveto{\pgfqpoint{8.092817in}{2.563308in}}%
\pgfpathcurveto{\pgfqpoint{8.097861in}{2.563308in}}{\pgfqpoint{8.102698in}{2.565312in}}{\pgfqpoint{8.106265in}{2.568878in}}%
\pgfpathcurveto{\pgfqpoint{8.109831in}{2.572445in}}{\pgfqpoint{8.111835in}{2.577283in}}{\pgfqpoint{8.111835in}{2.582326in}}%
\pgfpathcurveto{\pgfqpoint{8.111835in}{2.587370in}}{\pgfqpoint{8.109831in}{2.592208in}}{\pgfqpoint{8.106265in}{2.595774in}}%
\pgfpathcurveto{\pgfqpoint{8.102698in}{2.599340in}}{\pgfqpoint{8.097861in}{2.601344in}}{\pgfqpoint{8.092817in}{2.601344in}}%
\pgfpathcurveto{\pgfqpoint{8.087773in}{2.601344in}}{\pgfqpoint{8.082936in}{2.599340in}}{\pgfqpoint{8.079369in}{2.595774in}}%
\pgfpathcurveto{\pgfqpoint{8.075803in}{2.592208in}}{\pgfqpoint{8.073799in}{2.587370in}}{\pgfqpoint{8.073799in}{2.582326in}}%
\pgfpathcurveto{\pgfqpoint{8.073799in}{2.577283in}}{\pgfqpoint{8.075803in}{2.572445in}}{\pgfqpoint{8.079369in}{2.568878in}}%
\pgfpathcurveto{\pgfqpoint{8.082936in}{2.565312in}}{\pgfqpoint{8.087773in}{2.563308in}}{\pgfqpoint{8.092817in}{2.563308in}}%
\pgfpathclose%
\pgfusepath{fill}%
\end{pgfscope}%
\begin{pgfscope}%
\pgfpathrectangle{\pgfqpoint{6.572727in}{0.474100in}}{\pgfqpoint{4.227273in}{3.318700in}}%
\pgfusepath{clip}%
\pgfsetbuttcap%
\pgfsetroundjoin%
\definecolor{currentfill}{rgb}{0.127568,0.566949,0.550556}%
\pgfsetfillcolor{currentfill}%
\pgfsetfillopacity{0.700000}%
\pgfsetlinewidth{0.000000pt}%
\definecolor{currentstroke}{rgb}{0.000000,0.000000,0.000000}%
\pgfsetstrokecolor{currentstroke}%
\pgfsetstrokeopacity{0.700000}%
\pgfsetdash{}{0pt}%
\pgfpathmoveto{\pgfqpoint{7.165098in}{1.518740in}}%
\pgfpathcurveto{\pgfqpoint{7.170142in}{1.518740in}}{\pgfqpoint{7.174980in}{1.520744in}}{\pgfqpoint{7.178546in}{1.524310in}}%
\pgfpathcurveto{\pgfqpoint{7.182113in}{1.527877in}}{\pgfqpoint{7.184116in}{1.532714in}}{\pgfqpoint{7.184116in}{1.537758in}}%
\pgfpathcurveto{\pgfqpoint{7.184116in}{1.542802in}}{\pgfqpoint{7.182113in}{1.547639in}}{\pgfqpoint{7.178546in}{1.551206in}}%
\pgfpathcurveto{\pgfqpoint{7.174980in}{1.554772in}}{\pgfqpoint{7.170142in}{1.556776in}}{\pgfqpoint{7.165098in}{1.556776in}}%
\pgfpathcurveto{\pgfqpoint{7.160055in}{1.556776in}}{\pgfqpoint{7.155217in}{1.554772in}}{\pgfqpoint{7.151650in}{1.551206in}}%
\pgfpathcurveto{\pgfqpoint{7.148084in}{1.547639in}}{\pgfqpoint{7.146080in}{1.542802in}}{\pgfqpoint{7.146080in}{1.537758in}}%
\pgfpathcurveto{\pgfqpoint{7.146080in}{1.532714in}}{\pgfqpoint{7.148084in}{1.527877in}}{\pgfqpoint{7.151650in}{1.524310in}}%
\pgfpathcurveto{\pgfqpoint{7.155217in}{1.520744in}}{\pgfqpoint{7.160055in}{1.518740in}}{\pgfqpoint{7.165098in}{1.518740in}}%
\pgfpathclose%
\pgfusepath{fill}%
\end{pgfscope}%
\begin{pgfscope}%
\pgfpathrectangle{\pgfqpoint{6.572727in}{0.474100in}}{\pgfqpoint{4.227273in}{3.318700in}}%
\pgfusepath{clip}%
\pgfsetbuttcap%
\pgfsetroundjoin%
\definecolor{currentfill}{rgb}{0.127568,0.566949,0.550556}%
\pgfsetfillcolor{currentfill}%
\pgfsetfillopacity{0.700000}%
\pgfsetlinewidth{0.000000pt}%
\definecolor{currentstroke}{rgb}{0.000000,0.000000,0.000000}%
\pgfsetstrokecolor{currentstroke}%
\pgfsetstrokeopacity{0.700000}%
\pgfsetdash{}{0pt}%
\pgfpathmoveto{\pgfqpoint{7.917579in}{1.507975in}}%
\pgfpathcurveto{\pgfqpoint{7.922623in}{1.507975in}}{\pgfqpoint{7.927461in}{1.509979in}}{\pgfqpoint{7.931027in}{1.513546in}}%
\pgfpathcurveto{\pgfqpoint{7.934593in}{1.517112in}}{\pgfqpoint{7.936597in}{1.521950in}}{\pgfqpoint{7.936597in}{1.526994in}}%
\pgfpathcurveto{\pgfqpoint{7.936597in}{1.532037in}}{\pgfqpoint{7.934593in}{1.536875in}}{\pgfqpoint{7.931027in}{1.540441in}}%
\pgfpathcurveto{\pgfqpoint{7.927461in}{1.544008in}}{\pgfqpoint{7.922623in}{1.546012in}}{\pgfqpoint{7.917579in}{1.546012in}}%
\pgfpathcurveto{\pgfqpoint{7.912536in}{1.546012in}}{\pgfqpoint{7.907698in}{1.544008in}}{\pgfqpoint{7.904131in}{1.540441in}}%
\pgfpathcurveto{\pgfqpoint{7.900565in}{1.536875in}}{\pgfqpoint{7.898561in}{1.532037in}}{\pgfqpoint{7.898561in}{1.526994in}}%
\pgfpathcurveto{\pgfqpoint{7.898561in}{1.521950in}}{\pgfqpoint{7.900565in}{1.517112in}}{\pgfqpoint{7.904131in}{1.513546in}}%
\pgfpathcurveto{\pgfqpoint{7.907698in}{1.509979in}}{\pgfqpoint{7.912536in}{1.507975in}}{\pgfqpoint{7.917579in}{1.507975in}}%
\pgfpathclose%
\pgfusepath{fill}%
\end{pgfscope}%
\begin{pgfscope}%
\pgfpathrectangle{\pgfqpoint{6.572727in}{0.474100in}}{\pgfqpoint{4.227273in}{3.318700in}}%
\pgfusepath{clip}%
\pgfsetbuttcap%
\pgfsetroundjoin%
\definecolor{currentfill}{rgb}{0.127568,0.566949,0.550556}%
\pgfsetfillcolor{currentfill}%
\pgfsetfillopacity{0.700000}%
\pgfsetlinewidth{0.000000pt}%
\definecolor{currentstroke}{rgb}{0.000000,0.000000,0.000000}%
\pgfsetstrokecolor{currentstroke}%
\pgfsetstrokeopacity{0.700000}%
\pgfsetdash{}{0pt}%
\pgfpathmoveto{\pgfqpoint{7.869032in}{1.215507in}}%
\pgfpathcurveto{\pgfqpoint{7.874076in}{1.215507in}}{\pgfqpoint{7.878914in}{1.217511in}}{\pgfqpoint{7.882480in}{1.221077in}}%
\pgfpathcurveto{\pgfqpoint{7.886046in}{1.224643in}}{\pgfqpoint{7.888050in}{1.229481in}}{\pgfqpoint{7.888050in}{1.234525in}}%
\pgfpathcurveto{\pgfqpoint{7.888050in}{1.239569in}}{\pgfqpoint{7.886046in}{1.244406in}}{\pgfqpoint{7.882480in}{1.247973in}}%
\pgfpathcurveto{\pgfqpoint{7.878914in}{1.251539in}}{\pgfqpoint{7.874076in}{1.253543in}}{\pgfqpoint{7.869032in}{1.253543in}}%
\pgfpathcurveto{\pgfqpoint{7.863988in}{1.253543in}}{\pgfqpoint{7.859151in}{1.251539in}}{\pgfqpoint{7.855584in}{1.247973in}}%
\pgfpathcurveto{\pgfqpoint{7.852018in}{1.244406in}}{\pgfqpoint{7.850014in}{1.239569in}}{\pgfqpoint{7.850014in}{1.234525in}}%
\pgfpathcurveto{\pgfqpoint{7.850014in}{1.229481in}}{\pgfqpoint{7.852018in}{1.224643in}}{\pgfqpoint{7.855584in}{1.221077in}}%
\pgfpathcurveto{\pgfqpoint{7.859151in}{1.217511in}}{\pgfqpoint{7.863988in}{1.215507in}}{\pgfqpoint{7.869032in}{1.215507in}}%
\pgfpathclose%
\pgfusepath{fill}%
\end{pgfscope}%
\begin{pgfscope}%
\pgfpathrectangle{\pgfqpoint{6.572727in}{0.474100in}}{\pgfqpoint{4.227273in}{3.318700in}}%
\pgfusepath{clip}%
\pgfsetbuttcap%
\pgfsetroundjoin%
\definecolor{currentfill}{rgb}{0.993248,0.906157,0.143936}%
\pgfsetfillcolor{currentfill}%
\pgfsetfillopacity{0.700000}%
\pgfsetlinewidth{0.000000pt}%
\definecolor{currentstroke}{rgb}{0.000000,0.000000,0.000000}%
\pgfsetstrokecolor{currentstroke}%
\pgfsetstrokeopacity{0.700000}%
\pgfsetdash{}{0pt}%
\pgfpathmoveto{\pgfqpoint{9.679658in}{1.905726in}}%
\pgfpathcurveto{\pgfqpoint{9.684702in}{1.905726in}}{\pgfqpoint{9.689540in}{1.907730in}}{\pgfqpoint{9.693106in}{1.911297in}}%
\pgfpathcurveto{\pgfqpoint{9.696672in}{1.914863in}}{\pgfqpoint{9.698676in}{1.919701in}}{\pgfqpoint{9.698676in}{1.924745in}}%
\pgfpathcurveto{\pgfqpoint{9.698676in}{1.929788in}}{\pgfqpoint{9.696672in}{1.934626in}}{\pgfqpoint{9.693106in}{1.938192in}}%
\pgfpathcurveto{\pgfqpoint{9.689540in}{1.941759in}}{\pgfqpoint{9.684702in}{1.943763in}}{\pgfqpoint{9.679658in}{1.943763in}}%
\pgfpathcurveto{\pgfqpoint{9.674614in}{1.943763in}}{\pgfqpoint{9.669777in}{1.941759in}}{\pgfqpoint{9.666210in}{1.938192in}}%
\pgfpathcurveto{\pgfqpoint{9.662644in}{1.934626in}}{\pgfqpoint{9.660640in}{1.929788in}}{\pgfqpoint{9.660640in}{1.924745in}}%
\pgfpathcurveto{\pgfqpoint{9.660640in}{1.919701in}}{\pgfqpoint{9.662644in}{1.914863in}}{\pgfqpoint{9.666210in}{1.911297in}}%
\pgfpathcurveto{\pgfqpoint{9.669777in}{1.907730in}}{\pgfqpoint{9.674614in}{1.905726in}}{\pgfqpoint{9.679658in}{1.905726in}}%
\pgfpathclose%
\pgfusepath{fill}%
\end{pgfscope}%
\begin{pgfscope}%
\pgfpathrectangle{\pgfqpoint{6.572727in}{0.474100in}}{\pgfqpoint{4.227273in}{3.318700in}}%
\pgfusepath{clip}%
\pgfsetbuttcap%
\pgfsetroundjoin%
\definecolor{currentfill}{rgb}{0.127568,0.566949,0.550556}%
\pgfsetfillcolor{currentfill}%
\pgfsetfillopacity{0.700000}%
\pgfsetlinewidth{0.000000pt}%
\definecolor{currentstroke}{rgb}{0.000000,0.000000,0.000000}%
\pgfsetstrokecolor{currentstroke}%
\pgfsetstrokeopacity{0.700000}%
\pgfsetdash{}{0pt}%
\pgfpathmoveto{\pgfqpoint{7.533285in}{2.877259in}}%
\pgfpathcurveto{\pgfqpoint{7.538329in}{2.877259in}}{\pgfqpoint{7.543166in}{2.879263in}}{\pgfqpoint{7.546733in}{2.882829in}}%
\pgfpathcurveto{\pgfqpoint{7.550299in}{2.886395in}}{\pgfqpoint{7.552303in}{2.891233in}}{\pgfqpoint{7.552303in}{2.896277in}}%
\pgfpathcurveto{\pgfqpoint{7.552303in}{2.901320in}}{\pgfqpoint{7.550299in}{2.906158in}}{\pgfqpoint{7.546733in}{2.909725in}}%
\pgfpathcurveto{\pgfqpoint{7.543166in}{2.913291in}}{\pgfqpoint{7.538329in}{2.915295in}}{\pgfqpoint{7.533285in}{2.915295in}}%
\pgfpathcurveto{\pgfqpoint{7.528241in}{2.915295in}}{\pgfqpoint{7.523403in}{2.913291in}}{\pgfqpoint{7.519837in}{2.909725in}}%
\pgfpathcurveto{\pgfqpoint{7.516271in}{2.906158in}}{\pgfqpoint{7.514267in}{2.901320in}}{\pgfqpoint{7.514267in}{2.896277in}}%
\pgfpathcurveto{\pgfqpoint{7.514267in}{2.891233in}}{\pgfqpoint{7.516271in}{2.886395in}}{\pgfqpoint{7.519837in}{2.882829in}}%
\pgfpathcurveto{\pgfqpoint{7.523403in}{2.879263in}}{\pgfqpoint{7.528241in}{2.877259in}}{\pgfqpoint{7.533285in}{2.877259in}}%
\pgfpathclose%
\pgfusepath{fill}%
\end{pgfscope}%
\begin{pgfscope}%
\pgfpathrectangle{\pgfqpoint{6.572727in}{0.474100in}}{\pgfqpoint{4.227273in}{3.318700in}}%
\pgfusepath{clip}%
\pgfsetbuttcap%
\pgfsetroundjoin%
\definecolor{currentfill}{rgb}{0.127568,0.566949,0.550556}%
\pgfsetfillcolor{currentfill}%
\pgfsetfillopacity{0.700000}%
\pgfsetlinewidth{0.000000pt}%
\definecolor{currentstroke}{rgb}{0.000000,0.000000,0.000000}%
\pgfsetstrokecolor{currentstroke}%
\pgfsetstrokeopacity{0.700000}%
\pgfsetdash{}{0pt}%
\pgfpathmoveto{\pgfqpoint{7.858219in}{1.919263in}}%
\pgfpathcurveto{\pgfqpoint{7.863263in}{1.919263in}}{\pgfqpoint{7.868101in}{1.921267in}}{\pgfqpoint{7.871667in}{1.924833in}}%
\pgfpathcurveto{\pgfqpoint{7.875233in}{1.928400in}}{\pgfqpoint{7.877237in}{1.933237in}}{\pgfqpoint{7.877237in}{1.938281in}}%
\pgfpathcurveto{\pgfqpoint{7.877237in}{1.943325in}}{\pgfqpoint{7.875233in}{1.948162in}}{\pgfqpoint{7.871667in}{1.951729in}}%
\pgfpathcurveto{\pgfqpoint{7.868101in}{1.955295in}}{\pgfqpoint{7.863263in}{1.957299in}}{\pgfqpoint{7.858219in}{1.957299in}}%
\pgfpathcurveto{\pgfqpoint{7.853175in}{1.957299in}}{\pgfqpoint{7.848338in}{1.955295in}}{\pgfqpoint{7.844771in}{1.951729in}}%
\pgfpathcurveto{\pgfqpoint{7.841205in}{1.948162in}}{\pgfqpoint{7.839201in}{1.943325in}}{\pgfqpoint{7.839201in}{1.938281in}}%
\pgfpathcurveto{\pgfqpoint{7.839201in}{1.933237in}}{\pgfqpoint{7.841205in}{1.928400in}}{\pgfqpoint{7.844771in}{1.924833in}}%
\pgfpathcurveto{\pgfqpoint{7.848338in}{1.921267in}}{\pgfqpoint{7.853175in}{1.919263in}}{\pgfqpoint{7.858219in}{1.919263in}}%
\pgfpathclose%
\pgfusepath{fill}%
\end{pgfscope}%
\begin{pgfscope}%
\pgfpathrectangle{\pgfqpoint{6.572727in}{0.474100in}}{\pgfqpoint{4.227273in}{3.318700in}}%
\pgfusepath{clip}%
\pgfsetbuttcap%
\pgfsetroundjoin%
\definecolor{currentfill}{rgb}{0.127568,0.566949,0.550556}%
\pgfsetfillcolor{currentfill}%
\pgfsetfillopacity{0.700000}%
\pgfsetlinewidth{0.000000pt}%
\definecolor{currentstroke}{rgb}{0.000000,0.000000,0.000000}%
\pgfsetstrokecolor{currentstroke}%
\pgfsetstrokeopacity{0.700000}%
\pgfsetdash{}{0pt}%
\pgfpathmoveto{\pgfqpoint{8.060040in}{3.622932in}}%
\pgfpathcurveto{\pgfqpoint{8.065083in}{3.622932in}}{\pgfqpoint{8.069921in}{3.624936in}}{\pgfqpoint{8.073487in}{3.628502in}}%
\pgfpathcurveto{\pgfqpoint{8.077054in}{3.632069in}}{\pgfqpoint{8.079058in}{3.636906in}}{\pgfqpoint{8.079058in}{3.641950in}}%
\pgfpathcurveto{\pgfqpoint{8.079058in}{3.646994in}}{\pgfqpoint{8.077054in}{3.651831in}}{\pgfqpoint{8.073487in}{3.655398in}}%
\pgfpathcurveto{\pgfqpoint{8.069921in}{3.658964in}}{\pgfqpoint{8.065083in}{3.660968in}}{\pgfqpoint{8.060040in}{3.660968in}}%
\pgfpathcurveto{\pgfqpoint{8.054996in}{3.660968in}}{\pgfqpoint{8.050158in}{3.658964in}}{\pgfqpoint{8.046592in}{3.655398in}}%
\pgfpathcurveto{\pgfqpoint{8.043025in}{3.651831in}}{\pgfqpoint{8.041021in}{3.646994in}}{\pgfqpoint{8.041021in}{3.641950in}}%
\pgfpathcurveto{\pgfqpoint{8.041021in}{3.636906in}}{\pgfqpoint{8.043025in}{3.632069in}}{\pgfqpoint{8.046592in}{3.628502in}}%
\pgfpathcurveto{\pgfqpoint{8.050158in}{3.624936in}}{\pgfqpoint{8.054996in}{3.622932in}}{\pgfqpoint{8.060040in}{3.622932in}}%
\pgfpathclose%
\pgfusepath{fill}%
\end{pgfscope}%
\begin{pgfscope}%
\pgfpathrectangle{\pgfqpoint{6.572727in}{0.474100in}}{\pgfqpoint{4.227273in}{3.318700in}}%
\pgfusepath{clip}%
\pgfsetbuttcap%
\pgfsetroundjoin%
\definecolor{currentfill}{rgb}{0.127568,0.566949,0.550556}%
\pgfsetfillcolor{currentfill}%
\pgfsetfillopacity{0.700000}%
\pgfsetlinewidth{0.000000pt}%
\definecolor{currentstroke}{rgb}{0.000000,0.000000,0.000000}%
\pgfsetstrokecolor{currentstroke}%
\pgfsetstrokeopacity{0.700000}%
\pgfsetdash{}{0pt}%
\pgfpathmoveto{\pgfqpoint{7.588776in}{1.564973in}}%
\pgfpathcurveto{\pgfqpoint{7.593820in}{1.564973in}}{\pgfqpoint{7.598658in}{1.566977in}}{\pgfqpoint{7.602224in}{1.570543in}}%
\pgfpathcurveto{\pgfqpoint{7.605791in}{1.574109in}}{\pgfqpoint{7.607795in}{1.578947in}}{\pgfqpoint{7.607795in}{1.583991in}}%
\pgfpathcurveto{\pgfqpoint{7.607795in}{1.589035in}}{\pgfqpoint{7.605791in}{1.593872in}}{\pgfqpoint{7.602224in}{1.597439in}}%
\pgfpathcurveto{\pgfqpoint{7.598658in}{1.601005in}}{\pgfqpoint{7.593820in}{1.603009in}}{\pgfqpoint{7.588776in}{1.603009in}}%
\pgfpathcurveto{\pgfqpoint{7.583733in}{1.603009in}}{\pgfqpoint{7.578895in}{1.601005in}}{\pgfqpoint{7.575329in}{1.597439in}}%
\pgfpathcurveto{\pgfqpoint{7.571762in}{1.593872in}}{\pgfqpoint{7.569758in}{1.589035in}}{\pgfqpoint{7.569758in}{1.583991in}}%
\pgfpathcurveto{\pgfqpoint{7.569758in}{1.578947in}}{\pgfqpoint{7.571762in}{1.574109in}}{\pgfqpoint{7.575329in}{1.570543in}}%
\pgfpathcurveto{\pgfqpoint{7.578895in}{1.566977in}}{\pgfqpoint{7.583733in}{1.564973in}}{\pgfqpoint{7.588776in}{1.564973in}}%
\pgfpathclose%
\pgfusepath{fill}%
\end{pgfscope}%
\begin{pgfscope}%
\pgfpathrectangle{\pgfqpoint{6.572727in}{0.474100in}}{\pgfqpoint{4.227273in}{3.318700in}}%
\pgfusepath{clip}%
\pgfsetbuttcap%
\pgfsetroundjoin%
\definecolor{currentfill}{rgb}{0.127568,0.566949,0.550556}%
\pgfsetfillcolor{currentfill}%
\pgfsetfillopacity{0.700000}%
\pgfsetlinewidth{0.000000pt}%
\definecolor{currentstroke}{rgb}{0.000000,0.000000,0.000000}%
\pgfsetstrokecolor{currentstroke}%
\pgfsetstrokeopacity{0.700000}%
\pgfsetdash{}{0pt}%
\pgfpathmoveto{\pgfqpoint{8.196337in}{2.949685in}}%
\pgfpathcurveto{\pgfqpoint{8.201381in}{2.949685in}}{\pgfqpoint{8.206219in}{2.951689in}}{\pgfqpoint{8.209785in}{2.955255in}}%
\pgfpathcurveto{\pgfqpoint{8.213351in}{2.958822in}}{\pgfqpoint{8.215355in}{2.963660in}}{\pgfqpoint{8.215355in}{2.968703in}}%
\pgfpathcurveto{\pgfqpoint{8.215355in}{2.973747in}}{\pgfqpoint{8.213351in}{2.978585in}}{\pgfqpoint{8.209785in}{2.982151in}}%
\pgfpathcurveto{\pgfqpoint{8.206219in}{2.985718in}}{\pgfqpoint{8.201381in}{2.987721in}}{\pgfqpoint{8.196337in}{2.987721in}}%
\pgfpathcurveto{\pgfqpoint{8.191293in}{2.987721in}}{\pgfqpoint{8.186456in}{2.985718in}}{\pgfqpoint{8.182889in}{2.982151in}}%
\pgfpathcurveto{\pgfqpoint{8.179323in}{2.978585in}}{\pgfqpoint{8.177319in}{2.973747in}}{\pgfqpoint{8.177319in}{2.968703in}}%
\pgfpathcurveto{\pgfqpoint{8.177319in}{2.963660in}}{\pgfqpoint{8.179323in}{2.958822in}}{\pgfqpoint{8.182889in}{2.955255in}}%
\pgfpathcurveto{\pgfqpoint{8.186456in}{2.951689in}}{\pgfqpoint{8.191293in}{2.949685in}}{\pgfqpoint{8.196337in}{2.949685in}}%
\pgfpathclose%
\pgfusepath{fill}%
\end{pgfscope}%
\begin{pgfscope}%
\pgfpathrectangle{\pgfqpoint{6.572727in}{0.474100in}}{\pgfqpoint{4.227273in}{3.318700in}}%
\pgfusepath{clip}%
\pgfsetbuttcap%
\pgfsetroundjoin%
\definecolor{currentfill}{rgb}{0.993248,0.906157,0.143936}%
\pgfsetfillcolor{currentfill}%
\pgfsetfillopacity{0.700000}%
\pgfsetlinewidth{0.000000pt}%
\definecolor{currentstroke}{rgb}{0.000000,0.000000,0.000000}%
\pgfsetstrokecolor{currentstroke}%
\pgfsetstrokeopacity{0.700000}%
\pgfsetdash{}{0pt}%
\pgfpathmoveto{\pgfqpoint{9.253400in}{1.590459in}}%
\pgfpathcurveto{\pgfqpoint{9.258444in}{1.590459in}}{\pgfqpoint{9.263281in}{1.592463in}}{\pgfqpoint{9.266848in}{1.596029in}}%
\pgfpathcurveto{\pgfqpoint{9.270414in}{1.599595in}}{\pgfqpoint{9.272418in}{1.604433in}}{\pgfqpoint{9.272418in}{1.609477in}}%
\pgfpathcurveto{\pgfqpoint{9.272418in}{1.614521in}}{\pgfqpoint{9.270414in}{1.619358in}}{\pgfqpoint{9.266848in}{1.622925in}}%
\pgfpathcurveto{\pgfqpoint{9.263281in}{1.626491in}}{\pgfqpoint{9.258444in}{1.628495in}}{\pgfqpoint{9.253400in}{1.628495in}}%
\pgfpathcurveto{\pgfqpoint{9.248356in}{1.628495in}}{\pgfqpoint{9.243519in}{1.626491in}}{\pgfqpoint{9.239952in}{1.622925in}}%
\pgfpathcurveto{\pgfqpoint{9.236386in}{1.619358in}}{\pgfqpoint{9.234382in}{1.614521in}}{\pgfqpoint{9.234382in}{1.609477in}}%
\pgfpathcurveto{\pgfqpoint{9.234382in}{1.604433in}}{\pgfqpoint{9.236386in}{1.599595in}}{\pgfqpoint{9.239952in}{1.596029in}}%
\pgfpathcurveto{\pgfqpoint{9.243519in}{1.592463in}}{\pgfqpoint{9.248356in}{1.590459in}}{\pgfqpoint{9.253400in}{1.590459in}}%
\pgfpathclose%
\pgfusepath{fill}%
\end{pgfscope}%
\begin{pgfscope}%
\pgfpathrectangle{\pgfqpoint{6.572727in}{0.474100in}}{\pgfqpoint{4.227273in}{3.318700in}}%
\pgfusepath{clip}%
\pgfsetbuttcap%
\pgfsetroundjoin%
\definecolor{currentfill}{rgb}{0.993248,0.906157,0.143936}%
\pgfsetfillcolor{currentfill}%
\pgfsetfillopacity{0.700000}%
\pgfsetlinewidth{0.000000pt}%
\definecolor{currentstroke}{rgb}{0.000000,0.000000,0.000000}%
\pgfsetstrokecolor{currentstroke}%
\pgfsetstrokeopacity{0.700000}%
\pgfsetdash{}{0pt}%
\pgfpathmoveto{\pgfqpoint{9.654639in}{2.032378in}}%
\pgfpathcurveto{\pgfqpoint{9.659682in}{2.032378in}}{\pgfqpoint{9.664520in}{2.034381in}}{\pgfqpoint{9.668087in}{2.037948in}}%
\pgfpathcurveto{\pgfqpoint{9.671653in}{2.041514in}}{\pgfqpoint{9.673657in}{2.046352in}}{\pgfqpoint{9.673657in}{2.051396in}}%
\pgfpathcurveto{\pgfqpoint{9.673657in}{2.056439in}}{\pgfqpoint{9.671653in}{2.061277in}}{\pgfqpoint{9.668087in}{2.064844in}}%
\pgfpathcurveto{\pgfqpoint{9.664520in}{2.068410in}}{\pgfqpoint{9.659682in}{2.070414in}}{\pgfqpoint{9.654639in}{2.070414in}}%
\pgfpathcurveto{\pgfqpoint{9.649595in}{2.070414in}}{\pgfqpoint{9.644757in}{2.068410in}}{\pgfqpoint{9.641191in}{2.064844in}}%
\pgfpathcurveto{\pgfqpoint{9.637624in}{2.061277in}}{\pgfqpoint{9.635621in}{2.056439in}}{\pgfqpoint{9.635621in}{2.051396in}}%
\pgfpathcurveto{\pgfqpoint{9.635621in}{2.046352in}}{\pgfqpoint{9.637624in}{2.041514in}}{\pgfqpoint{9.641191in}{2.037948in}}%
\pgfpathcurveto{\pgfqpoint{9.644757in}{2.034381in}}{\pgfqpoint{9.649595in}{2.032378in}}{\pgfqpoint{9.654639in}{2.032378in}}%
\pgfpathclose%
\pgfusepath{fill}%
\end{pgfscope}%
\begin{pgfscope}%
\pgfpathrectangle{\pgfqpoint{6.572727in}{0.474100in}}{\pgfqpoint{4.227273in}{3.318700in}}%
\pgfusepath{clip}%
\pgfsetbuttcap%
\pgfsetroundjoin%
\definecolor{currentfill}{rgb}{0.127568,0.566949,0.550556}%
\pgfsetfillcolor{currentfill}%
\pgfsetfillopacity{0.700000}%
\pgfsetlinewidth{0.000000pt}%
\definecolor{currentstroke}{rgb}{0.000000,0.000000,0.000000}%
\pgfsetstrokecolor{currentstroke}%
\pgfsetstrokeopacity{0.700000}%
\pgfsetdash{}{0pt}%
\pgfpathmoveto{\pgfqpoint{7.898775in}{1.094336in}}%
\pgfpathcurveto{\pgfqpoint{7.903818in}{1.094336in}}{\pgfqpoint{7.908656in}{1.096340in}}{\pgfqpoint{7.912222in}{1.099906in}}%
\pgfpathcurveto{\pgfqpoint{7.915789in}{1.103472in}}{\pgfqpoint{7.917793in}{1.108310in}}{\pgfqpoint{7.917793in}{1.113354in}}%
\pgfpathcurveto{\pgfqpoint{7.917793in}{1.118397in}}{\pgfqpoint{7.915789in}{1.123235in}}{\pgfqpoint{7.912222in}{1.126802in}}%
\pgfpathcurveto{\pgfqpoint{7.908656in}{1.130368in}}{\pgfqpoint{7.903818in}{1.132372in}}{\pgfqpoint{7.898775in}{1.132372in}}%
\pgfpathcurveto{\pgfqpoint{7.893731in}{1.132372in}}{\pgfqpoint{7.888893in}{1.130368in}}{\pgfqpoint{7.885327in}{1.126802in}}%
\pgfpathcurveto{\pgfqpoint{7.881760in}{1.123235in}}{\pgfqpoint{7.879756in}{1.118397in}}{\pgfqpoint{7.879756in}{1.113354in}}%
\pgfpathcurveto{\pgfqpoint{7.879756in}{1.108310in}}{\pgfqpoint{7.881760in}{1.103472in}}{\pgfqpoint{7.885327in}{1.099906in}}%
\pgfpathcurveto{\pgfqpoint{7.888893in}{1.096340in}}{\pgfqpoint{7.893731in}{1.094336in}}{\pgfqpoint{7.898775in}{1.094336in}}%
\pgfpathclose%
\pgfusepath{fill}%
\end{pgfscope}%
\begin{pgfscope}%
\pgfpathrectangle{\pgfqpoint{6.572727in}{0.474100in}}{\pgfqpoint{4.227273in}{3.318700in}}%
\pgfusepath{clip}%
\pgfsetbuttcap%
\pgfsetroundjoin%
\definecolor{currentfill}{rgb}{0.127568,0.566949,0.550556}%
\pgfsetfillcolor{currentfill}%
\pgfsetfillopacity{0.700000}%
\pgfsetlinewidth{0.000000pt}%
\definecolor{currentstroke}{rgb}{0.000000,0.000000,0.000000}%
\pgfsetstrokecolor{currentstroke}%
\pgfsetstrokeopacity{0.700000}%
\pgfsetdash{}{0pt}%
\pgfpathmoveto{\pgfqpoint{7.254500in}{1.150518in}}%
\pgfpathcurveto{\pgfqpoint{7.259544in}{1.150518in}}{\pgfqpoint{7.264381in}{1.152522in}}{\pgfqpoint{7.267948in}{1.156088in}}%
\pgfpathcurveto{\pgfqpoint{7.271514in}{1.159654in}}{\pgfqpoint{7.273518in}{1.164492in}}{\pgfqpoint{7.273518in}{1.169536in}}%
\pgfpathcurveto{\pgfqpoint{7.273518in}{1.174579in}}{\pgfqpoint{7.271514in}{1.179417in}}{\pgfqpoint{7.267948in}{1.182984in}}%
\pgfpathcurveto{\pgfqpoint{7.264381in}{1.186550in}}{\pgfqpoint{7.259544in}{1.188554in}}{\pgfqpoint{7.254500in}{1.188554in}}%
\pgfpathcurveto{\pgfqpoint{7.249456in}{1.188554in}}{\pgfqpoint{7.244619in}{1.186550in}}{\pgfqpoint{7.241052in}{1.182984in}}%
\pgfpathcurveto{\pgfqpoint{7.237486in}{1.179417in}}{\pgfqpoint{7.235482in}{1.174579in}}{\pgfqpoint{7.235482in}{1.169536in}}%
\pgfpathcurveto{\pgfqpoint{7.235482in}{1.164492in}}{\pgfqpoint{7.237486in}{1.159654in}}{\pgfqpoint{7.241052in}{1.156088in}}%
\pgfpathcurveto{\pgfqpoint{7.244619in}{1.152522in}}{\pgfqpoint{7.249456in}{1.150518in}}{\pgfqpoint{7.254500in}{1.150518in}}%
\pgfpathclose%
\pgfusepath{fill}%
\end{pgfscope}%
\begin{pgfscope}%
\pgfpathrectangle{\pgfqpoint{6.572727in}{0.474100in}}{\pgfqpoint{4.227273in}{3.318700in}}%
\pgfusepath{clip}%
\pgfsetbuttcap%
\pgfsetroundjoin%
\definecolor{currentfill}{rgb}{0.127568,0.566949,0.550556}%
\pgfsetfillcolor{currentfill}%
\pgfsetfillopacity{0.700000}%
\pgfsetlinewidth{0.000000pt}%
\definecolor{currentstroke}{rgb}{0.000000,0.000000,0.000000}%
\pgfsetstrokecolor{currentstroke}%
\pgfsetstrokeopacity{0.700000}%
\pgfsetdash{}{0pt}%
\pgfpathmoveto{\pgfqpoint{7.777944in}{3.139026in}}%
\pgfpathcurveto{\pgfqpoint{7.782988in}{3.139026in}}{\pgfqpoint{7.787826in}{3.141030in}}{\pgfqpoint{7.791392in}{3.144597in}}%
\pgfpathcurveto{\pgfqpoint{7.794959in}{3.148163in}}{\pgfqpoint{7.796963in}{3.153001in}}{\pgfqpoint{7.796963in}{3.158044in}}%
\pgfpathcurveto{\pgfqpoint{7.796963in}{3.163088in}}{\pgfqpoint{7.794959in}{3.167926in}}{\pgfqpoint{7.791392in}{3.171492in}}%
\pgfpathcurveto{\pgfqpoint{7.787826in}{3.175059in}}{\pgfqpoint{7.782988in}{3.177063in}}{\pgfqpoint{7.777944in}{3.177063in}}%
\pgfpathcurveto{\pgfqpoint{7.772901in}{3.177063in}}{\pgfqpoint{7.768063in}{3.175059in}}{\pgfqpoint{7.764497in}{3.171492in}}%
\pgfpathcurveto{\pgfqpoint{7.760930in}{3.167926in}}{\pgfqpoint{7.758926in}{3.163088in}}{\pgfqpoint{7.758926in}{3.158044in}}%
\pgfpathcurveto{\pgfqpoint{7.758926in}{3.153001in}}{\pgfqpoint{7.760930in}{3.148163in}}{\pgfqpoint{7.764497in}{3.144597in}}%
\pgfpathcurveto{\pgfqpoint{7.768063in}{3.141030in}}{\pgfqpoint{7.772901in}{3.139026in}}{\pgfqpoint{7.777944in}{3.139026in}}%
\pgfpathclose%
\pgfusepath{fill}%
\end{pgfscope}%
\begin{pgfscope}%
\pgfpathrectangle{\pgfqpoint{6.572727in}{0.474100in}}{\pgfqpoint{4.227273in}{3.318700in}}%
\pgfusepath{clip}%
\pgfsetbuttcap%
\pgfsetroundjoin%
\definecolor{currentfill}{rgb}{0.993248,0.906157,0.143936}%
\pgfsetfillcolor{currentfill}%
\pgfsetfillopacity{0.700000}%
\pgfsetlinewidth{0.000000pt}%
\definecolor{currentstroke}{rgb}{0.000000,0.000000,0.000000}%
\pgfsetstrokecolor{currentstroke}%
\pgfsetstrokeopacity{0.700000}%
\pgfsetdash{}{0pt}%
\pgfpathmoveto{\pgfqpoint{9.963304in}{1.743597in}}%
\pgfpathcurveto{\pgfqpoint{9.968348in}{1.743597in}}{\pgfqpoint{9.973186in}{1.745601in}}{\pgfqpoint{9.976752in}{1.749167in}}%
\pgfpathcurveto{\pgfqpoint{9.980319in}{1.752734in}}{\pgfqpoint{9.982322in}{1.757572in}}{\pgfqpoint{9.982322in}{1.762615in}}%
\pgfpathcurveto{\pgfqpoint{9.982322in}{1.767659in}}{\pgfqpoint{9.980319in}{1.772497in}}{\pgfqpoint{9.976752in}{1.776063in}}%
\pgfpathcurveto{\pgfqpoint{9.973186in}{1.779630in}}{\pgfqpoint{9.968348in}{1.781633in}}{\pgfqpoint{9.963304in}{1.781633in}}%
\pgfpathcurveto{\pgfqpoint{9.958261in}{1.781633in}}{\pgfqpoint{9.953423in}{1.779630in}}{\pgfqpoint{9.949856in}{1.776063in}}%
\pgfpathcurveto{\pgfqpoint{9.946290in}{1.772497in}}{\pgfqpoint{9.944286in}{1.767659in}}{\pgfqpoint{9.944286in}{1.762615in}}%
\pgfpathcurveto{\pgfqpoint{9.944286in}{1.757572in}}{\pgfqpoint{9.946290in}{1.752734in}}{\pgfqpoint{9.949856in}{1.749167in}}%
\pgfpathcurveto{\pgfqpoint{9.953423in}{1.745601in}}{\pgfqpoint{9.958261in}{1.743597in}}{\pgfqpoint{9.963304in}{1.743597in}}%
\pgfpathclose%
\pgfusepath{fill}%
\end{pgfscope}%
\begin{pgfscope}%
\pgfpathrectangle{\pgfqpoint{6.572727in}{0.474100in}}{\pgfqpoint{4.227273in}{3.318700in}}%
\pgfusepath{clip}%
\pgfsetbuttcap%
\pgfsetroundjoin%
\definecolor{currentfill}{rgb}{0.993248,0.906157,0.143936}%
\pgfsetfillcolor{currentfill}%
\pgfsetfillopacity{0.700000}%
\pgfsetlinewidth{0.000000pt}%
\definecolor{currentstroke}{rgb}{0.000000,0.000000,0.000000}%
\pgfsetstrokecolor{currentstroke}%
\pgfsetstrokeopacity{0.700000}%
\pgfsetdash{}{0pt}%
\pgfpathmoveto{\pgfqpoint{9.563522in}{2.158193in}}%
\pgfpathcurveto{\pgfqpoint{9.568566in}{2.158193in}}{\pgfqpoint{9.573403in}{2.160197in}}{\pgfqpoint{9.576970in}{2.163763in}}%
\pgfpathcurveto{\pgfqpoint{9.580536in}{2.167329in}}{\pgfqpoint{9.582540in}{2.172167in}}{\pgfqpoint{9.582540in}{2.177211in}}%
\pgfpathcurveto{\pgfqpoint{9.582540in}{2.182254in}}{\pgfqpoint{9.580536in}{2.187092in}}{\pgfqpoint{9.576970in}{2.190659in}}%
\pgfpathcurveto{\pgfqpoint{9.573403in}{2.194225in}}{\pgfqpoint{9.568566in}{2.196229in}}{\pgfqpoint{9.563522in}{2.196229in}}%
\pgfpathcurveto{\pgfqpoint{9.558478in}{2.196229in}}{\pgfqpoint{9.553641in}{2.194225in}}{\pgfqpoint{9.550074in}{2.190659in}}%
\pgfpathcurveto{\pgfqpoint{9.546508in}{2.187092in}}{\pgfqpoint{9.544504in}{2.182254in}}{\pgfqpoint{9.544504in}{2.177211in}}%
\pgfpathcurveto{\pgfqpoint{9.544504in}{2.172167in}}{\pgfqpoint{9.546508in}{2.167329in}}{\pgfqpoint{9.550074in}{2.163763in}}%
\pgfpathcurveto{\pgfqpoint{9.553641in}{2.160197in}}{\pgfqpoint{9.558478in}{2.158193in}}{\pgfqpoint{9.563522in}{2.158193in}}%
\pgfpathclose%
\pgfusepath{fill}%
\end{pgfscope}%
\begin{pgfscope}%
\pgfpathrectangle{\pgfqpoint{6.572727in}{0.474100in}}{\pgfqpoint{4.227273in}{3.318700in}}%
\pgfusepath{clip}%
\pgfsetbuttcap%
\pgfsetroundjoin%
\definecolor{currentfill}{rgb}{0.127568,0.566949,0.550556}%
\pgfsetfillcolor{currentfill}%
\pgfsetfillopacity{0.700000}%
\pgfsetlinewidth{0.000000pt}%
\definecolor{currentstroke}{rgb}{0.000000,0.000000,0.000000}%
\pgfsetstrokecolor{currentstroke}%
\pgfsetstrokeopacity{0.700000}%
\pgfsetdash{}{0pt}%
\pgfpathmoveto{\pgfqpoint{8.235870in}{2.566468in}}%
\pgfpathcurveto{\pgfqpoint{8.240914in}{2.566468in}}{\pgfqpoint{8.245752in}{2.568472in}}{\pgfqpoint{8.249318in}{2.572039in}}%
\pgfpathcurveto{\pgfqpoint{8.252885in}{2.575605in}}{\pgfqpoint{8.254889in}{2.580443in}}{\pgfqpoint{8.254889in}{2.585486in}}%
\pgfpathcurveto{\pgfqpoint{8.254889in}{2.590530in}}{\pgfqpoint{8.252885in}{2.595368in}}{\pgfqpoint{8.249318in}{2.598934in}}%
\pgfpathcurveto{\pgfqpoint{8.245752in}{2.602501in}}{\pgfqpoint{8.240914in}{2.604505in}}{\pgfqpoint{8.235870in}{2.604505in}}%
\pgfpathcurveto{\pgfqpoint{8.230827in}{2.604505in}}{\pgfqpoint{8.225989in}{2.602501in}}{\pgfqpoint{8.222423in}{2.598934in}}%
\pgfpathcurveto{\pgfqpoint{8.218856in}{2.595368in}}{\pgfqpoint{8.216852in}{2.590530in}}{\pgfqpoint{8.216852in}{2.585486in}}%
\pgfpathcurveto{\pgfqpoint{8.216852in}{2.580443in}}{\pgfqpoint{8.218856in}{2.575605in}}{\pgfqpoint{8.222423in}{2.572039in}}%
\pgfpathcurveto{\pgfqpoint{8.225989in}{2.568472in}}{\pgfqpoint{8.230827in}{2.566468in}}{\pgfqpoint{8.235870in}{2.566468in}}%
\pgfpathclose%
\pgfusepath{fill}%
\end{pgfscope}%
\begin{pgfscope}%
\pgfpathrectangle{\pgfqpoint{6.572727in}{0.474100in}}{\pgfqpoint{4.227273in}{3.318700in}}%
\pgfusepath{clip}%
\pgfsetbuttcap%
\pgfsetroundjoin%
\definecolor{currentfill}{rgb}{0.127568,0.566949,0.550556}%
\pgfsetfillcolor{currentfill}%
\pgfsetfillopacity{0.700000}%
\pgfsetlinewidth{0.000000pt}%
\definecolor{currentstroke}{rgb}{0.000000,0.000000,0.000000}%
\pgfsetstrokecolor{currentstroke}%
\pgfsetstrokeopacity{0.700000}%
\pgfsetdash{}{0pt}%
\pgfpathmoveto{\pgfqpoint{7.442700in}{1.973380in}}%
\pgfpathcurveto{\pgfqpoint{7.447744in}{1.973380in}}{\pgfqpoint{7.452582in}{1.975384in}}{\pgfqpoint{7.456148in}{1.978951in}}%
\pgfpathcurveto{\pgfqpoint{7.459714in}{1.982517in}}{\pgfqpoint{7.461718in}{1.987355in}}{\pgfqpoint{7.461718in}{1.992399in}}%
\pgfpathcurveto{\pgfqpoint{7.461718in}{1.997442in}}{\pgfqpoint{7.459714in}{2.002280in}}{\pgfqpoint{7.456148in}{2.005846in}}%
\pgfpathcurveto{\pgfqpoint{7.452582in}{2.009413in}}{\pgfqpoint{7.447744in}{2.011417in}}{\pgfqpoint{7.442700in}{2.011417in}}%
\pgfpathcurveto{\pgfqpoint{7.437656in}{2.011417in}}{\pgfqpoint{7.432819in}{2.009413in}}{\pgfqpoint{7.429252in}{2.005846in}}%
\pgfpathcurveto{\pgfqpoint{7.425686in}{2.002280in}}{\pgfqpoint{7.423682in}{1.997442in}}{\pgfqpoint{7.423682in}{1.992399in}}%
\pgfpathcurveto{\pgfqpoint{7.423682in}{1.987355in}}{\pgfqpoint{7.425686in}{1.982517in}}{\pgfqpoint{7.429252in}{1.978951in}}%
\pgfpathcurveto{\pgfqpoint{7.432819in}{1.975384in}}{\pgfqpoint{7.437656in}{1.973380in}}{\pgfqpoint{7.442700in}{1.973380in}}%
\pgfpathclose%
\pgfusepath{fill}%
\end{pgfscope}%
\begin{pgfscope}%
\pgfpathrectangle{\pgfqpoint{6.572727in}{0.474100in}}{\pgfqpoint{4.227273in}{3.318700in}}%
\pgfusepath{clip}%
\pgfsetbuttcap%
\pgfsetroundjoin%
\definecolor{currentfill}{rgb}{0.993248,0.906157,0.143936}%
\pgfsetfillcolor{currentfill}%
\pgfsetfillopacity{0.700000}%
\pgfsetlinewidth{0.000000pt}%
\definecolor{currentstroke}{rgb}{0.000000,0.000000,0.000000}%
\pgfsetstrokecolor{currentstroke}%
\pgfsetstrokeopacity{0.700000}%
\pgfsetdash{}{0pt}%
\pgfpathmoveto{\pgfqpoint{9.637547in}{1.446546in}}%
\pgfpathcurveto{\pgfqpoint{9.642591in}{1.446546in}}{\pgfqpoint{9.647428in}{1.448550in}}{\pgfqpoint{9.650995in}{1.452117in}}%
\pgfpathcurveto{\pgfqpoint{9.654561in}{1.455683in}}{\pgfqpoint{9.656565in}{1.460521in}}{\pgfqpoint{9.656565in}{1.465564in}}%
\pgfpathcurveto{\pgfqpoint{9.656565in}{1.470608in}}{\pgfqpoint{9.654561in}{1.475446in}}{\pgfqpoint{9.650995in}{1.479012in}}%
\pgfpathcurveto{\pgfqpoint{9.647428in}{1.482579in}}{\pgfqpoint{9.642591in}{1.484583in}}{\pgfqpoint{9.637547in}{1.484583in}}%
\pgfpathcurveto{\pgfqpoint{9.632503in}{1.484583in}}{\pgfqpoint{9.627666in}{1.482579in}}{\pgfqpoint{9.624099in}{1.479012in}}%
\pgfpathcurveto{\pgfqpoint{9.620533in}{1.475446in}}{\pgfqpoint{9.618529in}{1.470608in}}{\pgfqpoint{9.618529in}{1.465564in}}%
\pgfpathcurveto{\pgfqpoint{9.618529in}{1.460521in}}{\pgfqpoint{9.620533in}{1.455683in}}{\pgfqpoint{9.624099in}{1.452117in}}%
\pgfpathcurveto{\pgfqpoint{9.627666in}{1.448550in}}{\pgfqpoint{9.632503in}{1.446546in}}{\pgfqpoint{9.637547in}{1.446546in}}%
\pgfpathclose%
\pgfusepath{fill}%
\end{pgfscope}%
\begin{pgfscope}%
\pgfpathrectangle{\pgfqpoint{6.572727in}{0.474100in}}{\pgfqpoint{4.227273in}{3.318700in}}%
\pgfusepath{clip}%
\pgfsetbuttcap%
\pgfsetroundjoin%
\definecolor{currentfill}{rgb}{0.993248,0.906157,0.143936}%
\pgfsetfillcolor{currentfill}%
\pgfsetfillopacity{0.700000}%
\pgfsetlinewidth{0.000000pt}%
\definecolor{currentstroke}{rgb}{0.000000,0.000000,0.000000}%
\pgfsetstrokecolor{currentstroke}%
\pgfsetstrokeopacity{0.700000}%
\pgfsetdash{}{0pt}%
\pgfpathmoveto{\pgfqpoint{9.799364in}{1.231788in}}%
\pgfpathcurveto{\pgfqpoint{9.804407in}{1.231788in}}{\pgfqpoint{9.809245in}{1.233792in}}{\pgfqpoint{9.812811in}{1.237358in}}%
\pgfpathcurveto{\pgfqpoint{9.816378in}{1.240925in}}{\pgfqpoint{9.818382in}{1.245763in}}{\pgfqpoint{9.818382in}{1.250806in}}%
\pgfpathcurveto{\pgfqpoint{9.818382in}{1.255850in}}{\pgfqpoint{9.816378in}{1.260688in}}{\pgfqpoint{9.812811in}{1.264254in}}%
\pgfpathcurveto{\pgfqpoint{9.809245in}{1.267820in}}{\pgfqpoint{9.804407in}{1.269824in}}{\pgfqpoint{9.799364in}{1.269824in}}%
\pgfpathcurveto{\pgfqpoint{9.794320in}{1.269824in}}{\pgfqpoint{9.789482in}{1.267820in}}{\pgfqpoint{9.785916in}{1.264254in}}%
\pgfpathcurveto{\pgfqpoint{9.782349in}{1.260688in}}{\pgfqpoint{9.780345in}{1.255850in}}{\pgfqpoint{9.780345in}{1.250806in}}%
\pgfpathcurveto{\pgfqpoint{9.780345in}{1.245763in}}{\pgfqpoint{9.782349in}{1.240925in}}{\pgfqpoint{9.785916in}{1.237358in}}%
\pgfpathcurveto{\pgfqpoint{9.789482in}{1.233792in}}{\pgfqpoint{9.794320in}{1.231788in}}{\pgfqpoint{9.799364in}{1.231788in}}%
\pgfpathclose%
\pgfusepath{fill}%
\end{pgfscope}%
\begin{pgfscope}%
\pgfpathrectangle{\pgfqpoint{6.572727in}{0.474100in}}{\pgfqpoint{4.227273in}{3.318700in}}%
\pgfusepath{clip}%
\pgfsetbuttcap%
\pgfsetroundjoin%
\definecolor{currentfill}{rgb}{0.127568,0.566949,0.550556}%
\pgfsetfillcolor{currentfill}%
\pgfsetfillopacity{0.700000}%
\pgfsetlinewidth{0.000000pt}%
\definecolor{currentstroke}{rgb}{0.000000,0.000000,0.000000}%
\pgfsetstrokecolor{currentstroke}%
\pgfsetstrokeopacity{0.700000}%
\pgfsetdash{}{0pt}%
\pgfpathmoveto{\pgfqpoint{8.002259in}{2.872913in}}%
\pgfpathcurveto{\pgfqpoint{8.007302in}{2.872913in}}{\pgfqpoint{8.012140in}{2.874917in}}{\pgfqpoint{8.015707in}{2.878483in}}%
\pgfpathcurveto{\pgfqpoint{8.019273in}{2.882050in}}{\pgfqpoint{8.021277in}{2.886887in}}{\pgfqpoint{8.021277in}{2.891931in}}%
\pgfpathcurveto{\pgfqpoint{8.021277in}{2.896975in}}{\pgfqpoint{8.019273in}{2.901812in}}{\pgfqpoint{8.015707in}{2.905379in}}%
\pgfpathcurveto{\pgfqpoint{8.012140in}{2.908945in}}{\pgfqpoint{8.007302in}{2.910949in}}{\pgfqpoint{8.002259in}{2.910949in}}%
\pgfpathcurveto{\pgfqpoint{7.997215in}{2.910949in}}{\pgfqpoint{7.992377in}{2.908945in}}{\pgfqpoint{7.988811in}{2.905379in}}%
\pgfpathcurveto{\pgfqpoint{7.985244in}{2.901812in}}{\pgfqpoint{7.983241in}{2.896975in}}{\pgfqpoint{7.983241in}{2.891931in}}%
\pgfpathcurveto{\pgfqpoint{7.983241in}{2.886887in}}{\pgfqpoint{7.985244in}{2.882050in}}{\pgfqpoint{7.988811in}{2.878483in}}%
\pgfpathcurveto{\pgfqpoint{7.992377in}{2.874917in}}{\pgfqpoint{7.997215in}{2.872913in}}{\pgfqpoint{8.002259in}{2.872913in}}%
\pgfpathclose%
\pgfusepath{fill}%
\end{pgfscope}%
\begin{pgfscope}%
\pgfpathrectangle{\pgfqpoint{6.572727in}{0.474100in}}{\pgfqpoint{4.227273in}{3.318700in}}%
\pgfusepath{clip}%
\pgfsetbuttcap%
\pgfsetroundjoin%
\definecolor{currentfill}{rgb}{0.127568,0.566949,0.550556}%
\pgfsetfillcolor{currentfill}%
\pgfsetfillopacity{0.700000}%
\pgfsetlinewidth{0.000000pt}%
\definecolor{currentstroke}{rgb}{0.000000,0.000000,0.000000}%
\pgfsetstrokecolor{currentstroke}%
\pgfsetstrokeopacity{0.700000}%
\pgfsetdash{}{0pt}%
\pgfpathmoveto{\pgfqpoint{7.437299in}{1.638424in}}%
\pgfpathcurveto{\pgfqpoint{7.442342in}{1.638424in}}{\pgfqpoint{7.447180in}{1.640427in}}{\pgfqpoint{7.450746in}{1.643994in}}%
\pgfpathcurveto{\pgfqpoint{7.454313in}{1.647560in}}{\pgfqpoint{7.456317in}{1.652398in}}{\pgfqpoint{7.456317in}{1.657442in}}%
\pgfpathcurveto{\pgfqpoint{7.456317in}{1.662485in}}{\pgfqpoint{7.454313in}{1.667323in}}{\pgfqpoint{7.450746in}{1.670890in}}%
\pgfpathcurveto{\pgfqpoint{7.447180in}{1.674456in}}{\pgfqpoint{7.442342in}{1.676460in}}{\pgfqpoint{7.437299in}{1.676460in}}%
\pgfpathcurveto{\pgfqpoint{7.432255in}{1.676460in}}{\pgfqpoint{7.427417in}{1.674456in}}{\pgfqpoint{7.423851in}{1.670890in}}%
\pgfpathcurveto{\pgfqpoint{7.420284in}{1.667323in}}{\pgfqpoint{7.418280in}{1.662485in}}{\pgfqpoint{7.418280in}{1.657442in}}%
\pgfpathcurveto{\pgfqpoint{7.418280in}{1.652398in}}{\pgfqpoint{7.420284in}{1.647560in}}{\pgfqpoint{7.423851in}{1.643994in}}%
\pgfpathcurveto{\pgfqpoint{7.427417in}{1.640427in}}{\pgfqpoint{7.432255in}{1.638424in}}{\pgfqpoint{7.437299in}{1.638424in}}%
\pgfpathclose%
\pgfusepath{fill}%
\end{pgfscope}%
\begin{pgfscope}%
\pgfpathrectangle{\pgfqpoint{6.572727in}{0.474100in}}{\pgfqpoint{4.227273in}{3.318700in}}%
\pgfusepath{clip}%
\pgfsetbuttcap%
\pgfsetroundjoin%
\definecolor{currentfill}{rgb}{0.993248,0.906157,0.143936}%
\pgfsetfillcolor{currentfill}%
\pgfsetfillopacity{0.700000}%
\pgfsetlinewidth{0.000000pt}%
\definecolor{currentstroke}{rgb}{0.000000,0.000000,0.000000}%
\pgfsetstrokecolor{currentstroke}%
\pgfsetstrokeopacity{0.700000}%
\pgfsetdash{}{0pt}%
\pgfpathmoveto{\pgfqpoint{9.747416in}{1.358730in}}%
\pgfpathcurveto{\pgfqpoint{9.752460in}{1.358730in}}{\pgfqpoint{9.757297in}{1.360734in}}{\pgfqpoint{9.760864in}{1.364300in}}%
\pgfpathcurveto{\pgfqpoint{9.764430in}{1.367867in}}{\pgfqpoint{9.766434in}{1.372704in}}{\pgfqpoint{9.766434in}{1.377748in}}%
\pgfpathcurveto{\pgfqpoint{9.766434in}{1.382792in}}{\pgfqpoint{9.764430in}{1.387630in}}{\pgfqpoint{9.760864in}{1.391196in}}%
\pgfpathcurveto{\pgfqpoint{9.757297in}{1.394762in}}{\pgfqpoint{9.752460in}{1.396766in}}{\pgfqpoint{9.747416in}{1.396766in}}%
\pgfpathcurveto{\pgfqpoint{9.742372in}{1.396766in}}{\pgfqpoint{9.737535in}{1.394762in}}{\pgfqpoint{9.733968in}{1.391196in}}%
\pgfpathcurveto{\pgfqpoint{9.730402in}{1.387630in}}{\pgfqpoint{9.728398in}{1.382792in}}{\pgfqpoint{9.728398in}{1.377748in}}%
\pgfpathcurveto{\pgfqpoint{9.728398in}{1.372704in}}{\pgfqpoint{9.730402in}{1.367867in}}{\pgfqpoint{9.733968in}{1.364300in}}%
\pgfpathcurveto{\pgfqpoint{9.737535in}{1.360734in}}{\pgfqpoint{9.742372in}{1.358730in}}{\pgfqpoint{9.747416in}{1.358730in}}%
\pgfpathclose%
\pgfusepath{fill}%
\end{pgfscope}%
\begin{pgfscope}%
\pgfpathrectangle{\pgfqpoint{6.572727in}{0.474100in}}{\pgfqpoint{4.227273in}{3.318700in}}%
\pgfusepath{clip}%
\pgfsetbuttcap%
\pgfsetroundjoin%
\definecolor{currentfill}{rgb}{0.127568,0.566949,0.550556}%
\pgfsetfillcolor{currentfill}%
\pgfsetfillopacity{0.700000}%
\pgfsetlinewidth{0.000000pt}%
\definecolor{currentstroke}{rgb}{0.000000,0.000000,0.000000}%
\pgfsetstrokecolor{currentstroke}%
\pgfsetstrokeopacity{0.700000}%
\pgfsetdash{}{0pt}%
\pgfpathmoveto{\pgfqpoint{8.350238in}{2.035781in}}%
\pgfpathcurveto{\pgfqpoint{8.355282in}{2.035781in}}{\pgfqpoint{8.360120in}{2.037785in}}{\pgfqpoint{8.363686in}{2.041352in}}%
\pgfpathcurveto{\pgfqpoint{8.367253in}{2.044918in}}{\pgfqpoint{8.369257in}{2.049756in}}{\pgfqpoint{8.369257in}{2.054800in}}%
\pgfpathcurveto{\pgfqpoint{8.369257in}{2.059843in}}{\pgfqpoint{8.367253in}{2.064681in}}{\pgfqpoint{8.363686in}{2.068247in}}%
\pgfpathcurveto{\pgfqpoint{8.360120in}{2.071814in}}{\pgfqpoint{8.355282in}{2.073818in}}{\pgfqpoint{8.350238in}{2.073818in}}%
\pgfpathcurveto{\pgfqpoint{8.345195in}{2.073818in}}{\pgfqpoint{8.340357in}{2.071814in}}{\pgfqpoint{8.336790in}{2.068247in}}%
\pgfpathcurveto{\pgfqpoint{8.333224in}{2.064681in}}{\pgfqpoint{8.331220in}{2.059843in}}{\pgfqpoint{8.331220in}{2.054800in}}%
\pgfpathcurveto{\pgfqpoint{8.331220in}{2.049756in}}{\pgfqpoint{8.333224in}{2.044918in}}{\pgfqpoint{8.336790in}{2.041352in}}%
\pgfpathcurveto{\pgfqpoint{8.340357in}{2.037785in}}{\pgfqpoint{8.345195in}{2.035781in}}{\pgfqpoint{8.350238in}{2.035781in}}%
\pgfpathclose%
\pgfusepath{fill}%
\end{pgfscope}%
\begin{pgfscope}%
\pgfpathrectangle{\pgfqpoint{6.572727in}{0.474100in}}{\pgfqpoint{4.227273in}{3.318700in}}%
\pgfusepath{clip}%
\pgfsetbuttcap%
\pgfsetroundjoin%
\definecolor{currentfill}{rgb}{0.267004,0.004874,0.329415}%
\pgfsetfillcolor{currentfill}%
\pgfsetfillopacity{0.700000}%
\pgfsetlinewidth{0.000000pt}%
\definecolor{currentstroke}{rgb}{0.000000,0.000000,0.000000}%
\pgfsetstrokecolor{currentstroke}%
\pgfsetstrokeopacity{0.700000}%
\pgfsetdash{}{0pt}%
\pgfpathmoveto{\pgfqpoint{10.537250in}{2.125169in}}%
\pgfpathcurveto{\pgfqpoint{10.542293in}{2.125169in}}{\pgfqpoint{10.547131in}{2.127173in}}{\pgfqpoint{10.550697in}{2.130740in}}%
\pgfpathcurveto{\pgfqpoint{10.554264in}{2.134306in}}{\pgfqpoint{10.556268in}{2.139144in}}{\pgfqpoint{10.556268in}{2.144187in}}%
\pgfpathcurveto{\pgfqpoint{10.556268in}{2.149231in}}{\pgfqpoint{10.554264in}{2.154069in}}{\pgfqpoint{10.550697in}{2.157635in}}%
\pgfpathcurveto{\pgfqpoint{10.547131in}{2.161202in}}{\pgfqpoint{10.542293in}{2.163206in}}{\pgfqpoint{10.537250in}{2.163206in}}%
\pgfpathcurveto{\pgfqpoint{10.532206in}{2.163206in}}{\pgfqpoint{10.527368in}{2.161202in}}{\pgfqpoint{10.523802in}{2.157635in}}%
\pgfpathcurveto{\pgfqpoint{10.520235in}{2.154069in}}{\pgfqpoint{10.518231in}{2.149231in}}{\pgfqpoint{10.518231in}{2.144187in}}%
\pgfpathcurveto{\pgfqpoint{10.518231in}{2.139144in}}{\pgfqpoint{10.520235in}{2.134306in}}{\pgfqpoint{10.523802in}{2.130740in}}%
\pgfpathcurveto{\pgfqpoint{10.527368in}{2.127173in}}{\pgfqpoint{10.532206in}{2.125169in}}{\pgfqpoint{10.537250in}{2.125169in}}%
\pgfpathclose%
\pgfusepath{fill}%
\end{pgfscope}%
\begin{pgfscope}%
\pgfpathrectangle{\pgfqpoint{6.572727in}{0.474100in}}{\pgfqpoint{4.227273in}{3.318700in}}%
\pgfusepath{clip}%
\pgfsetbuttcap%
\pgfsetroundjoin%
\definecolor{currentfill}{rgb}{0.993248,0.906157,0.143936}%
\pgfsetfillcolor{currentfill}%
\pgfsetfillopacity{0.700000}%
\pgfsetlinewidth{0.000000pt}%
\definecolor{currentstroke}{rgb}{0.000000,0.000000,0.000000}%
\pgfsetstrokecolor{currentstroke}%
\pgfsetstrokeopacity{0.700000}%
\pgfsetdash{}{0pt}%
\pgfpathmoveto{\pgfqpoint{9.727474in}{1.452709in}}%
\pgfpathcurveto{\pgfqpoint{9.732518in}{1.452709in}}{\pgfqpoint{9.737355in}{1.454713in}}{\pgfqpoint{9.740922in}{1.458279in}}%
\pgfpathcurveto{\pgfqpoint{9.744488in}{1.461845in}}{\pgfqpoint{9.746492in}{1.466683in}}{\pgfqpoint{9.746492in}{1.471727in}}%
\pgfpathcurveto{\pgfqpoint{9.746492in}{1.476771in}}{\pgfqpoint{9.744488in}{1.481608in}}{\pgfqpoint{9.740922in}{1.485175in}}%
\pgfpathcurveto{\pgfqpoint{9.737355in}{1.488741in}}{\pgfqpoint{9.732518in}{1.490745in}}{\pgfqpoint{9.727474in}{1.490745in}}%
\pgfpathcurveto{\pgfqpoint{9.722430in}{1.490745in}}{\pgfqpoint{9.717593in}{1.488741in}}{\pgfqpoint{9.714026in}{1.485175in}}%
\pgfpathcurveto{\pgfqpoint{9.710460in}{1.481608in}}{\pgfqpoint{9.708456in}{1.476771in}}{\pgfqpoint{9.708456in}{1.471727in}}%
\pgfpathcurveto{\pgfqpoint{9.708456in}{1.466683in}}{\pgfqpoint{9.710460in}{1.461845in}}{\pgfqpoint{9.714026in}{1.458279in}}%
\pgfpathcurveto{\pgfqpoint{9.717593in}{1.454713in}}{\pgfqpoint{9.722430in}{1.452709in}}{\pgfqpoint{9.727474in}{1.452709in}}%
\pgfpathclose%
\pgfusepath{fill}%
\end{pgfscope}%
\begin{pgfscope}%
\pgfpathrectangle{\pgfqpoint{6.572727in}{0.474100in}}{\pgfqpoint{4.227273in}{3.318700in}}%
\pgfusepath{clip}%
\pgfsetbuttcap%
\pgfsetroundjoin%
\definecolor{currentfill}{rgb}{0.127568,0.566949,0.550556}%
\pgfsetfillcolor{currentfill}%
\pgfsetfillopacity{0.700000}%
\pgfsetlinewidth{0.000000pt}%
\definecolor{currentstroke}{rgb}{0.000000,0.000000,0.000000}%
\pgfsetstrokecolor{currentstroke}%
\pgfsetstrokeopacity{0.700000}%
\pgfsetdash{}{0pt}%
\pgfpathmoveto{\pgfqpoint{8.011155in}{3.040786in}}%
\pgfpathcurveto{\pgfqpoint{8.016199in}{3.040786in}}{\pgfqpoint{8.021037in}{3.042790in}}{\pgfqpoint{8.024603in}{3.046356in}}%
\pgfpathcurveto{\pgfqpoint{8.028170in}{3.049922in}}{\pgfqpoint{8.030173in}{3.054760in}}{\pgfqpoint{8.030173in}{3.059804in}}%
\pgfpathcurveto{\pgfqpoint{8.030173in}{3.064848in}}{\pgfqpoint{8.028170in}{3.069685in}}{\pgfqpoint{8.024603in}{3.073252in}}%
\pgfpathcurveto{\pgfqpoint{8.021037in}{3.076818in}}{\pgfqpoint{8.016199in}{3.078822in}}{\pgfqpoint{8.011155in}{3.078822in}}%
\pgfpathcurveto{\pgfqpoint{8.006112in}{3.078822in}}{\pgfqpoint{8.001274in}{3.076818in}}{\pgfqpoint{7.997707in}{3.073252in}}%
\pgfpathcurveto{\pgfqpoint{7.994141in}{3.069685in}}{\pgfqpoint{7.992137in}{3.064848in}}{\pgfqpoint{7.992137in}{3.059804in}}%
\pgfpathcurveto{\pgfqpoint{7.992137in}{3.054760in}}{\pgfqpoint{7.994141in}{3.049922in}}{\pgfqpoint{7.997707in}{3.046356in}}%
\pgfpathcurveto{\pgfqpoint{8.001274in}{3.042790in}}{\pgfqpoint{8.006112in}{3.040786in}}{\pgfqpoint{8.011155in}{3.040786in}}%
\pgfpathclose%
\pgfusepath{fill}%
\end{pgfscope}%
\begin{pgfscope}%
\pgfpathrectangle{\pgfqpoint{6.572727in}{0.474100in}}{\pgfqpoint{4.227273in}{3.318700in}}%
\pgfusepath{clip}%
\pgfsetbuttcap%
\pgfsetroundjoin%
\definecolor{currentfill}{rgb}{0.993248,0.906157,0.143936}%
\pgfsetfillcolor{currentfill}%
\pgfsetfillopacity{0.700000}%
\pgfsetlinewidth{0.000000pt}%
\definecolor{currentstroke}{rgb}{0.000000,0.000000,0.000000}%
\pgfsetstrokecolor{currentstroke}%
\pgfsetstrokeopacity{0.700000}%
\pgfsetdash{}{0pt}%
\pgfpathmoveto{\pgfqpoint{10.145355in}{1.709488in}}%
\pgfpathcurveto{\pgfqpoint{10.150399in}{1.709488in}}{\pgfqpoint{10.155237in}{1.711491in}}{\pgfqpoint{10.158803in}{1.715058in}}%
\pgfpathcurveto{\pgfqpoint{10.162370in}{1.718624in}}{\pgfqpoint{10.164373in}{1.723462in}}{\pgfqpoint{10.164373in}{1.728506in}}%
\pgfpathcurveto{\pgfqpoint{10.164373in}{1.733549in}}{\pgfqpoint{10.162370in}{1.738387in}}{\pgfqpoint{10.158803in}{1.741954in}}%
\pgfpathcurveto{\pgfqpoint{10.155237in}{1.745520in}}{\pgfqpoint{10.150399in}{1.747524in}}{\pgfqpoint{10.145355in}{1.747524in}}%
\pgfpathcurveto{\pgfqpoint{10.140312in}{1.747524in}}{\pgfqpoint{10.135474in}{1.745520in}}{\pgfqpoint{10.131907in}{1.741954in}}%
\pgfpathcurveto{\pgfqpoint{10.128341in}{1.738387in}}{\pgfqpoint{10.126337in}{1.733549in}}{\pgfqpoint{10.126337in}{1.728506in}}%
\pgfpathcurveto{\pgfqpoint{10.126337in}{1.723462in}}{\pgfqpoint{10.128341in}{1.718624in}}{\pgfqpoint{10.131907in}{1.715058in}}%
\pgfpathcurveto{\pgfqpoint{10.135474in}{1.711491in}}{\pgfqpoint{10.140312in}{1.709488in}}{\pgfqpoint{10.145355in}{1.709488in}}%
\pgfpathclose%
\pgfusepath{fill}%
\end{pgfscope}%
\begin{pgfscope}%
\pgfpathrectangle{\pgfqpoint{6.572727in}{0.474100in}}{\pgfqpoint{4.227273in}{3.318700in}}%
\pgfusepath{clip}%
\pgfsetbuttcap%
\pgfsetroundjoin%
\definecolor{currentfill}{rgb}{0.267004,0.004874,0.329415}%
\pgfsetfillcolor{currentfill}%
\pgfsetfillopacity{0.700000}%
\pgfsetlinewidth{0.000000pt}%
\definecolor{currentstroke}{rgb}{0.000000,0.000000,0.000000}%
\pgfsetstrokecolor{currentstroke}%
\pgfsetstrokeopacity{0.700000}%
\pgfsetdash{}{0pt}%
\pgfpathmoveto{\pgfqpoint{7.498594in}{3.506249in}}%
\pgfpathcurveto{\pgfqpoint{7.503638in}{3.506249in}}{\pgfqpoint{7.508475in}{3.508253in}}{\pgfqpoint{7.512042in}{3.511819in}}%
\pgfpathcurveto{\pgfqpoint{7.515608in}{3.515386in}}{\pgfqpoint{7.517612in}{3.520223in}}{\pgfqpoint{7.517612in}{3.525267in}}%
\pgfpathcurveto{\pgfqpoint{7.517612in}{3.530311in}}{\pgfqpoint{7.515608in}{3.535148in}}{\pgfqpoint{7.512042in}{3.538715in}}%
\pgfpathcurveto{\pgfqpoint{7.508475in}{3.542281in}}{\pgfqpoint{7.503638in}{3.544285in}}{\pgfqpoint{7.498594in}{3.544285in}}%
\pgfpathcurveto{\pgfqpoint{7.493550in}{3.544285in}}{\pgfqpoint{7.488713in}{3.542281in}}{\pgfqpoint{7.485146in}{3.538715in}}%
\pgfpathcurveto{\pgfqpoint{7.481580in}{3.535148in}}{\pgfqpoint{7.479576in}{3.530311in}}{\pgfqpoint{7.479576in}{3.525267in}}%
\pgfpathcurveto{\pgfqpoint{7.479576in}{3.520223in}}{\pgfqpoint{7.481580in}{3.515386in}}{\pgfqpoint{7.485146in}{3.511819in}}%
\pgfpathcurveto{\pgfqpoint{7.488713in}{3.508253in}}{\pgfqpoint{7.493550in}{3.506249in}}{\pgfqpoint{7.498594in}{3.506249in}}%
\pgfpathclose%
\pgfusepath{fill}%
\end{pgfscope}%
\begin{pgfscope}%
\pgfpathrectangle{\pgfqpoint{6.572727in}{0.474100in}}{\pgfqpoint{4.227273in}{3.318700in}}%
\pgfusepath{clip}%
\pgfsetbuttcap%
\pgfsetroundjoin%
\definecolor{currentfill}{rgb}{0.993248,0.906157,0.143936}%
\pgfsetfillcolor{currentfill}%
\pgfsetfillopacity{0.700000}%
\pgfsetlinewidth{0.000000pt}%
\definecolor{currentstroke}{rgb}{0.000000,0.000000,0.000000}%
\pgfsetstrokecolor{currentstroke}%
\pgfsetstrokeopacity{0.700000}%
\pgfsetdash{}{0pt}%
\pgfpathmoveto{\pgfqpoint{9.475706in}{1.062734in}}%
\pgfpathcurveto{\pgfqpoint{9.480749in}{1.062734in}}{\pgfqpoint{9.485587in}{1.064738in}}{\pgfqpoint{9.489154in}{1.068304in}}%
\pgfpathcurveto{\pgfqpoint{9.492720in}{1.071871in}}{\pgfqpoint{9.494724in}{1.076708in}}{\pgfqpoint{9.494724in}{1.081752in}}%
\pgfpathcurveto{\pgfqpoint{9.494724in}{1.086796in}}{\pgfqpoint{9.492720in}{1.091634in}}{\pgfqpoint{9.489154in}{1.095200in}}%
\pgfpathcurveto{\pgfqpoint{9.485587in}{1.098766in}}{\pgfqpoint{9.480749in}{1.100770in}}{\pgfqpoint{9.475706in}{1.100770in}}%
\pgfpathcurveto{\pgfqpoint{9.470662in}{1.100770in}}{\pgfqpoint{9.465824in}{1.098766in}}{\pgfqpoint{9.462258in}{1.095200in}}%
\pgfpathcurveto{\pgfqpoint{9.458692in}{1.091634in}}{\pgfqpoint{9.456688in}{1.086796in}}{\pgfqpoint{9.456688in}{1.081752in}}%
\pgfpathcurveto{\pgfqpoint{9.456688in}{1.076708in}}{\pgfqpoint{9.458692in}{1.071871in}}{\pgfqpoint{9.462258in}{1.068304in}}%
\pgfpathcurveto{\pgfqpoint{9.465824in}{1.064738in}}{\pgfqpoint{9.470662in}{1.062734in}}{\pgfqpoint{9.475706in}{1.062734in}}%
\pgfpathclose%
\pgfusepath{fill}%
\end{pgfscope}%
\begin{pgfscope}%
\pgfpathrectangle{\pgfqpoint{6.572727in}{0.474100in}}{\pgfqpoint{4.227273in}{3.318700in}}%
\pgfusepath{clip}%
\pgfsetbuttcap%
\pgfsetroundjoin%
\definecolor{currentfill}{rgb}{0.993248,0.906157,0.143936}%
\pgfsetfillcolor{currentfill}%
\pgfsetfillopacity{0.700000}%
\pgfsetlinewidth{0.000000pt}%
\definecolor{currentstroke}{rgb}{0.000000,0.000000,0.000000}%
\pgfsetstrokecolor{currentstroke}%
\pgfsetstrokeopacity{0.700000}%
\pgfsetdash{}{0pt}%
\pgfpathmoveto{\pgfqpoint{8.989587in}{2.159229in}}%
\pgfpathcurveto{\pgfqpoint{8.994631in}{2.159229in}}{\pgfqpoint{8.999469in}{2.161233in}}{\pgfqpoint{9.003035in}{2.164799in}}%
\pgfpathcurveto{\pgfqpoint{9.006602in}{2.168366in}}{\pgfqpoint{9.008605in}{2.173203in}}{\pgfqpoint{9.008605in}{2.178247in}}%
\pgfpathcurveto{\pgfqpoint{9.008605in}{2.183291in}}{\pgfqpoint{9.006602in}{2.188129in}}{\pgfqpoint{9.003035in}{2.191695in}}%
\pgfpathcurveto{\pgfqpoint{8.999469in}{2.195261in}}{\pgfqpoint{8.994631in}{2.197265in}}{\pgfqpoint{8.989587in}{2.197265in}}%
\pgfpathcurveto{\pgfqpoint{8.984544in}{2.197265in}}{\pgfqpoint{8.979706in}{2.195261in}}{\pgfqpoint{8.976139in}{2.191695in}}%
\pgfpathcurveto{\pgfqpoint{8.972573in}{2.188129in}}{\pgfqpoint{8.970569in}{2.183291in}}{\pgfqpoint{8.970569in}{2.178247in}}%
\pgfpathcurveto{\pgfqpoint{8.970569in}{2.173203in}}{\pgfqpoint{8.972573in}{2.168366in}}{\pgfqpoint{8.976139in}{2.164799in}}%
\pgfpathcurveto{\pgfqpoint{8.979706in}{2.161233in}}{\pgfqpoint{8.984544in}{2.159229in}}{\pgfqpoint{8.989587in}{2.159229in}}%
\pgfpathclose%
\pgfusepath{fill}%
\end{pgfscope}%
\begin{pgfscope}%
\pgfpathrectangle{\pgfqpoint{6.572727in}{0.474100in}}{\pgfqpoint{4.227273in}{3.318700in}}%
\pgfusepath{clip}%
\pgfsetbuttcap%
\pgfsetroundjoin%
\definecolor{currentfill}{rgb}{0.267004,0.004874,0.329415}%
\pgfsetfillcolor{currentfill}%
\pgfsetfillopacity{0.700000}%
\pgfsetlinewidth{0.000000pt}%
\definecolor{currentstroke}{rgb}{0.000000,0.000000,0.000000}%
\pgfsetstrokecolor{currentstroke}%
\pgfsetstrokeopacity{0.700000}%
\pgfsetdash{}{0pt}%
\pgfpathmoveto{\pgfqpoint{7.544241in}{0.605932in}}%
\pgfpathcurveto{\pgfqpoint{7.549285in}{0.605932in}}{\pgfqpoint{7.554123in}{0.607936in}}{\pgfqpoint{7.557689in}{0.611502in}}%
\pgfpathcurveto{\pgfqpoint{7.561256in}{0.615069in}}{\pgfqpoint{7.563260in}{0.619906in}}{\pgfqpoint{7.563260in}{0.624950in}}%
\pgfpathcurveto{\pgfqpoint{7.563260in}{0.629994in}}{\pgfqpoint{7.561256in}{0.634831in}}{\pgfqpoint{7.557689in}{0.638398in}}%
\pgfpathcurveto{\pgfqpoint{7.554123in}{0.641964in}}{\pgfqpoint{7.549285in}{0.643968in}}{\pgfqpoint{7.544241in}{0.643968in}}%
\pgfpathcurveto{\pgfqpoint{7.539198in}{0.643968in}}{\pgfqpoint{7.534360in}{0.641964in}}{\pgfqpoint{7.530794in}{0.638398in}}%
\pgfpathcurveto{\pgfqpoint{7.527227in}{0.634831in}}{\pgfqpoint{7.525223in}{0.629994in}}{\pgfqpoint{7.525223in}{0.624950in}}%
\pgfpathcurveto{\pgfqpoint{7.525223in}{0.619906in}}{\pgfqpoint{7.527227in}{0.615069in}}{\pgfqpoint{7.530794in}{0.611502in}}%
\pgfpathcurveto{\pgfqpoint{7.534360in}{0.607936in}}{\pgfqpoint{7.539198in}{0.605932in}}{\pgfqpoint{7.544241in}{0.605932in}}%
\pgfpathclose%
\pgfusepath{fill}%
\end{pgfscope}%
\begin{pgfscope}%
\pgfpathrectangle{\pgfqpoint{6.572727in}{0.474100in}}{\pgfqpoint{4.227273in}{3.318700in}}%
\pgfusepath{clip}%
\pgfsetbuttcap%
\pgfsetroundjoin%
\definecolor{currentfill}{rgb}{0.127568,0.566949,0.550556}%
\pgfsetfillcolor{currentfill}%
\pgfsetfillopacity{0.700000}%
\pgfsetlinewidth{0.000000pt}%
\definecolor{currentstroke}{rgb}{0.000000,0.000000,0.000000}%
\pgfsetstrokecolor{currentstroke}%
\pgfsetstrokeopacity{0.700000}%
\pgfsetdash{}{0pt}%
\pgfpathmoveto{\pgfqpoint{8.273392in}{1.522490in}}%
\pgfpathcurveto{\pgfqpoint{8.278436in}{1.522490in}}{\pgfqpoint{8.283274in}{1.524494in}}{\pgfqpoint{8.286840in}{1.528061in}}%
\pgfpathcurveto{\pgfqpoint{8.290407in}{1.531627in}}{\pgfqpoint{8.292411in}{1.536465in}}{\pgfqpoint{8.292411in}{1.541509in}}%
\pgfpathcurveto{\pgfqpoint{8.292411in}{1.546552in}}{\pgfqpoint{8.290407in}{1.551390in}}{\pgfqpoint{8.286840in}{1.554956in}}%
\pgfpathcurveto{\pgfqpoint{8.283274in}{1.558523in}}{\pgfqpoint{8.278436in}{1.560527in}}{\pgfqpoint{8.273392in}{1.560527in}}%
\pgfpathcurveto{\pgfqpoint{8.268349in}{1.560527in}}{\pgfqpoint{8.263511in}{1.558523in}}{\pgfqpoint{8.259945in}{1.554956in}}%
\pgfpathcurveto{\pgfqpoint{8.256378in}{1.551390in}}{\pgfqpoint{8.254374in}{1.546552in}}{\pgfqpoint{8.254374in}{1.541509in}}%
\pgfpathcurveto{\pgfqpoint{8.254374in}{1.536465in}}{\pgfqpoint{8.256378in}{1.531627in}}{\pgfqpoint{8.259945in}{1.528061in}}%
\pgfpathcurveto{\pgfqpoint{8.263511in}{1.524494in}}{\pgfqpoint{8.268349in}{1.522490in}}{\pgfqpoint{8.273392in}{1.522490in}}%
\pgfpathclose%
\pgfusepath{fill}%
\end{pgfscope}%
\begin{pgfscope}%
\pgfpathrectangle{\pgfqpoint{6.572727in}{0.474100in}}{\pgfqpoint{4.227273in}{3.318700in}}%
\pgfusepath{clip}%
\pgfsetbuttcap%
\pgfsetroundjoin%
\definecolor{currentfill}{rgb}{0.127568,0.566949,0.550556}%
\pgfsetfillcolor{currentfill}%
\pgfsetfillopacity{0.700000}%
\pgfsetlinewidth{0.000000pt}%
\definecolor{currentstroke}{rgb}{0.000000,0.000000,0.000000}%
\pgfsetstrokecolor{currentstroke}%
\pgfsetstrokeopacity{0.700000}%
\pgfsetdash{}{0pt}%
\pgfpathmoveto{\pgfqpoint{7.494979in}{1.252205in}}%
\pgfpathcurveto{\pgfqpoint{7.500022in}{1.252205in}}{\pgfqpoint{7.504860in}{1.254209in}}{\pgfqpoint{7.508427in}{1.257775in}}%
\pgfpathcurveto{\pgfqpoint{7.511993in}{1.261342in}}{\pgfqpoint{7.513997in}{1.266180in}}{\pgfqpoint{7.513997in}{1.271223in}}%
\pgfpathcurveto{\pgfqpoint{7.513997in}{1.276267in}}{\pgfqpoint{7.511993in}{1.281105in}}{\pgfqpoint{7.508427in}{1.284671in}}%
\pgfpathcurveto{\pgfqpoint{7.504860in}{1.288238in}}{\pgfqpoint{7.500022in}{1.290241in}}{\pgfqpoint{7.494979in}{1.290241in}}%
\pgfpathcurveto{\pgfqpoint{7.489935in}{1.290241in}}{\pgfqpoint{7.485097in}{1.288238in}}{\pgfqpoint{7.481531in}{1.284671in}}%
\pgfpathcurveto{\pgfqpoint{7.477964in}{1.281105in}}{\pgfqpoint{7.475960in}{1.276267in}}{\pgfqpoint{7.475960in}{1.271223in}}%
\pgfpathcurveto{\pgfqpoint{7.475960in}{1.266180in}}{\pgfqpoint{7.477964in}{1.261342in}}{\pgfqpoint{7.481531in}{1.257775in}}%
\pgfpathcurveto{\pgfqpoint{7.485097in}{1.254209in}}{\pgfqpoint{7.489935in}{1.252205in}}{\pgfqpoint{7.494979in}{1.252205in}}%
\pgfpathclose%
\pgfusepath{fill}%
\end{pgfscope}%
\begin{pgfscope}%
\pgfpathrectangle{\pgfqpoint{6.572727in}{0.474100in}}{\pgfqpoint{4.227273in}{3.318700in}}%
\pgfusepath{clip}%
\pgfsetbuttcap%
\pgfsetroundjoin%
\definecolor{currentfill}{rgb}{0.127568,0.566949,0.550556}%
\pgfsetfillcolor{currentfill}%
\pgfsetfillopacity{0.700000}%
\pgfsetlinewidth{0.000000pt}%
\definecolor{currentstroke}{rgb}{0.000000,0.000000,0.000000}%
\pgfsetstrokecolor{currentstroke}%
\pgfsetstrokeopacity{0.700000}%
\pgfsetdash{}{0pt}%
\pgfpathmoveto{\pgfqpoint{7.871762in}{2.271436in}}%
\pgfpathcurveto{\pgfqpoint{7.876806in}{2.271436in}}{\pgfqpoint{7.881643in}{2.273440in}}{\pgfqpoint{7.885210in}{2.277007in}}%
\pgfpathcurveto{\pgfqpoint{7.888776in}{2.280573in}}{\pgfqpoint{7.890780in}{2.285411in}}{\pgfqpoint{7.890780in}{2.290455in}}%
\pgfpathcurveto{\pgfqpoint{7.890780in}{2.295498in}}{\pgfqpoint{7.888776in}{2.300336in}}{\pgfqpoint{7.885210in}{2.303902in}}%
\pgfpathcurveto{\pgfqpoint{7.881643in}{2.307469in}}{\pgfqpoint{7.876806in}{2.309473in}}{\pgfqpoint{7.871762in}{2.309473in}}%
\pgfpathcurveto{\pgfqpoint{7.866718in}{2.309473in}}{\pgfqpoint{7.861881in}{2.307469in}}{\pgfqpoint{7.858314in}{2.303902in}}%
\pgfpathcurveto{\pgfqpoint{7.854748in}{2.300336in}}{\pgfqpoint{7.852744in}{2.295498in}}{\pgfqpoint{7.852744in}{2.290455in}}%
\pgfpathcurveto{\pgfqpoint{7.852744in}{2.285411in}}{\pgfqpoint{7.854748in}{2.280573in}}{\pgfqpoint{7.858314in}{2.277007in}}%
\pgfpathcurveto{\pgfqpoint{7.861881in}{2.273440in}}{\pgfqpoint{7.866718in}{2.271436in}}{\pgfqpoint{7.871762in}{2.271436in}}%
\pgfpathclose%
\pgfusepath{fill}%
\end{pgfscope}%
\begin{pgfscope}%
\pgfpathrectangle{\pgfqpoint{6.572727in}{0.474100in}}{\pgfqpoint{4.227273in}{3.318700in}}%
\pgfusepath{clip}%
\pgfsetbuttcap%
\pgfsetroundjoin%
\definecolor{currentfill}{rgb}{0.993248,0.906157,0.143936}%
\pgfsetfillcolor{currentfill}%
\pgfsetfillopacity{0.700000}%
\pgfsetlinewidth{0.000000pt}%
\definecolor{currentstroke}{rgb}{0.000000,0.000000,0.000000}%
\pgfsetstrokecolor{currentstroke}%
\pgfsetstrokeopacity{0.700000}%
\pgfsetdash{}{0pt}%
\pgfpathmoveto{\pgfqpoint{9.449838in}{1.647931in}}%
\pgfpathcurveto{\pgfqpoint{9.454882in}{1.647931in}}{\pgfqpoint{9.459720in}{1.649935in}}{\pgfqpoint{9.463286in}{1.653502in}}%
\pgfpathcurveto{\pgfqpoint{9.466852in}{1.657068in}}{\pgfqpoint{9.468856in}{1.661906in}}{\pgfqpoint{9.468856in}{1.666949in}}%
\pgfpathcurveto{\pgfqpoint{9.468856in}{1.671993in}}{\pgfqpoint{9.466852in}{1.676831in}}{\pgfqpoint{9.463286in}{1.680397in}}%
\pgfpathcurveto{\pgfqpoint{9.459720in}{1.683964in}}{\pgfqpoint{9.454882in}{1.685968in}}{\pgfqpoint{9.449838in}{1.685968in}}%
\pgfpathcurveto{\pgfqpoint{9.444795in}{1.685968in}}{\pgfqpoint{9.439957in}{1.683964in}}{\pgfqpoint{9.436390in}{1.680397in}}%
\pgfpathcurveto{\pgfqpoint{9.432824in}{1.676831in}}{\pgfqpoint{9.430820in}{1.671993in}}{\pgfqpoint{9.430820in}{1.666949in}}%
\pgfpathcurveto{\pgfqpoint{9.430820in}{1.661906in}}{\pgfqpoint{9.432824in}{1.657068in}}{\pgfqpoint{9.436390in}{1.653502in}}%
\pgfpathcurveto{\pgfqpoint{9.439957in}{1.649935in}}{\pgfqpoint{9.444795in}{1.647931in}}{\pgfqpoint{9.449838in}{1.647931in}}%
\pgfpathclose%
\pgfusepath{fill}%
\end{pgfscope}%
\begin{pgfscope}%
\pgfpathrectangle{\pgfqpoint{6.572727in}{0.474100in}}{\pgfqpoint{4.227273in}{3.318700in}}%
\pgfusepath{clip}%
\pgfsetbuttcap%
\pgfsetroundjoin%
\definecolor{currentfill}{rgb}{0.993248,0.906157,0.143936}%
\pgfsetfillcolor{currentfill}%
\pgfsetfillopacity{0.700000}%
\pgfsetlinewidth{0.000000pt}%
\definecolor{currentstroke}{rgb}{0.000000,0.000000,0.000000}%
\pgfsetstrokecolor{currentstroke}%
\pgfsetstrokeopacity{0.700000}%
\pgfsetdash{}{0pt}%
\pgfpathmoveto{\pgfqpoint{9.454653in}{1.681134in}}%
\pgfpathcurveto{\pgfqpoint{9.459696in}{1.681134in}}{\pgfqpoint{9.464534in}{1.683138in}}{\pgfqpoint{9.468101in}{1.686705in}}%
\pgfpathcurveto{\pgfqpoint{9.471667in}{1.690271in}}{\pgfqpoint{9.473671in}{1.695109in}}{\pgfqpoint{9.473671in}{1.700152in}}%
\pgfpathcurveto{\pgfqpoint{9.473671in}{1.705196in}}{\pgfqpoint{9.471667in}{1.710034in}}{\pgfqpoint{9.468101in}{1.713600in}}%
\pgfpathcurveto{\pgfqpoint{9.464534in}{1.717167in}}{\pgfqpoint{9.459696in}{1.719171in}}{\pgfqpoint{9.454653in}{1.719171in}}%
\pgfpathcurveto{\pgfqpoint{9.449609in}{1.719171in}}{\pgfqpoint{9.444771in}{1.717167in}}{\pgfqpoint{9.441205in}{1.713600in}}%
\pgfpathcurveto{\pgfqpoint{9.437639in}{1.710034in}}{\pgfqpoint{9.435635in}{1.705196in}}{\pgfqpoint{9.435635in}{1.700152in}}%
\pgfpathcurveto{\pgfqpoint{9.435635in}{1.695109in}}{\pgfqpoint{9.437639in}{1.690271in}}{\pgfqpoint{9.441205in}{1.686705in}}%
\pgfpathcurveto{\pgfqpoint{9.444771in}{1.683138in}}{\pgfqpoint{9.449609in}{1.681134in}}{\pgfqpoint{9.454653in}{1.681134in}}%
\pgfpathclose%
\pgfusepath{fill}%
\end{pgfscope}%
\begin{pgfscope}%
\pgfpathrectangle{\pgfqpoint{6.572727in}{0.474100in}}{\pgfqpoint{4.227273in}{3.318700in}}%
\pgfusepath{clip}%
\pgfsetbuttcap%
\pgfsetroundjoin%
\definecolor{currentfill}{rgb}{0.127568,0.566949,0.550556}%
\pgfsetfillcolor{currentfill}%
\pgfsetfillopacity{0.700000}%
\pgfsetlinewidth{0.000000pt}%
\definecolor{currentstroke}{rgb}{0.000000,0.000000,0.000000}%
\pgfsetstrokecolor{currentstroke}%
\pgfsetstrokeopacity{0.700000}%
\pgfsetdash{}{0pt}%
\pgfpathmoveto{\pgfqpoint{8.102479in}{2.712583in}}%
\pgfpathcurveto{\pgfqpoint{8.107522in}{2.712583in}}{\pgfqpoint{8.112360in}{2.714586in}}{\pgfqpoint{8.115926in}{2.718153in}}%
\pgfpathcurveto{\pgfqpoint{8.119493in}{2.721719in}}{\pgfqpoint{8.121497in}{2.726557in}}{\pgfqpoint{8.121497in}{2.731601in}}%
\pgfpathcurveto{\pgfqpoint{8.121497in}{2.736644in}}{\pgfqpoint{8.119493in}{2.741482in}}{\pgfqpoint{8.115926in}{2.745049in}}%
\pgfpathcurveto{\pgfqpoint{8.112360in}{2.748615in}}{\pgfqpoint{8.107522in}{2.750619in}}{\pgfqpoint{8.102479in}{2.750619in}}%
\pgfpathcurveto{\pgfqpoint{8.097435in}{2.750619in}}{\pgfqpoint{8.092597in}{2.748615in}}{\pgfqpoint{8.089031in}{2.745049in}}%
\pgfpathcurveto{\pgfqpoint{8.085464in}{2.741482in}}{\pgfqpoint{8.083460in}{2.736644in}}{\pgfqpoint{8.083460in}{2.731601in}}%
\pgfpathcurveto{\pgfqpoint{8.083460in}{2.726557in}}{\pgfqpoint{8.085464in}{2.721719in}}{\pgfqpoint{8.089031in}{2.718153in}}%
\pgfpathcurveto{\pgfqpoint{8.092597in}{2.714586in}}{\pgfqpoint{8.097435in}{2.712583in}}{\pgfqpoint{8.102479in}{2.712583in}}%
\pgfpathclose%
\pgfusepath{fill}%
\end{pgfscope}%
\begin{pgfscope}%
\pgfpathrectangle{\pgfqpoint{6.572727in}{0.474100in}}{\pgfqpoint{4.227273in}{3.318700in}}%
\pgfusepath{clip}%
\pgfsetbuttcap%
\pgfsetroundjoin%
\definecolor{currentfill}{rgb}{0.127568,0.566949,0.550556}%
\pgfsetfillcolor{currentfill}%
\pgfsetfillopacity{0.700000}%
\pgfsetlinewidth{0.000000pt}%
\definecolor{currentstroke}{rgb}{0.000000,0.000000,0.000000}%
\pgfsetstrokecolor{currentstroke}%
\pgfsetstrokeopacity{0.700000}%
\pgfsetdash{}{0pt}%
\pgfpathmoveto{\pgfqpoint{8.361357in}{2.988342in}}%
\pgfpathcurveto{\pgfqpoint{8.366400in}{2.988342in}}{\pgfqpoint{8.371238in}{2.990346in}}{\pgfqpoint{8.374804in}{2.993912in}}%
\pgfpathcurveto{\pgfqpoint{8.378371in}{2.997479in}}{\pgfqpoint{8.380375in}{3.002316in}}{\pgfqpoint{8.380375in}{3.007360in}}%
\pgfpathcurveto{\pgfqpoint{8.380375in}{3.012404in}}{\pgfqpoint{8.378371in}{3.017242in}}{\pgfqpoint{8.374804in}{3.020808in}}%
\pgfpathcurveto{\pgfqpoint{8.371238in}{3.024374in}}{\pgfqpoint{8.366400in}{3.026378in}}{\pgfqpoint{8.361357in}{3.026378in}}%
\pgfpathcurveto{\pgfqpoint{8.356313in}{3.026378in}}{\pgfqpoint{8.351475in}{3.024374in}}{\pgfqpoint{8.347909in}{3.020808in}}%
\pgfpathcurveto{\pgfqpoint{8.344342in}{3.017242in}}{\pgfqpoint{8.342338in}{3.012404in}}{\pgfqpoint{8.342338in}{3.007360in}}%
\pgfpathcurveto{\pgfqpoint{8.342338in}{3.002316in}}{\pgfqpoint{8.344342in}{2.997479in}}{\pgfqpoint{8.347909in}{2.993912in}}%
\pgfpathcurveto{\pgfqpoint{8.351475in}{2.990346in}}{\pgfqpoint{8.356313in}{2.988342in}}{\pgfqpoint{8.361357in}{2.988342in}}%
\pgfpathclose%
\pgfusepath{fill}%
\end{pgfscope}%
\begin{pgfscope}%
\pgfpathrectangle{\pgfqpoint{6.572727in}{0.474100in}}{\pgfqpoint{4.227273in}{3.318700in}}%
\pgfusepath{clip}%
\pgfsetbuttcap%
\pgfsetroundjoin%
\definecolor{currentfill}{rgb}{0.993248,0.906157,0.143936}%
\pgfsetfillcolor{currentfill}%
\pgfsetfillopacity{0.700000}%
\pgfsetlinewidth{0.000000pt}%
\definecolor{currentstroke}{rgb}{0.000000,0.000000,0.000000}%
\pgfsetstrokecolor{currentstroke}%
\pgfsetstrokeopacity{0.700000}%
\pgfsetdash{}{0pt}%
\pgfpathmoveto{\pgfqpoint{9.025197in}{1.511243in}}%
\pgfpathcurveto{\pgfqpoint{9.030241in}{1.511243in}}{\pgfqpoint{9.035079in}{1.513247in}}{\pgfqpoint{9.038645in}{1.516813in}}%
\pgfpathcurveto{\pgfqpoint{9.042212in}{1.520380in}}{\pgfqpoint{9.044216in}{1.525217in}}{\pgfqpoint{9.044216in}{1.530261in}}%
\pgfpathcurveto{\pgfqpoint{9.044216in}{1.535305in}}{\pgfqpoint{9.042212in}{1.540143in}}{\pgfqpoint{9.038645in}{1.543709in}}%
\pgfpathcurveto{\pgfqpoint{9.035079in}{1.547275in}}{\pgfqpoint{9.030241in}{1.549279in}}{\pgfqpoint{9.025197in}{1.549279in}}%
\pgfpathcurveto{\pgfqpoint{9.020154in}{1.549279in}}{\pgfqpoint{9.015316in}{1.547275in}}{\pgfqpoint{9.011750in}{1.543709in}}%
\pgfpathcurveto{\pgfqpoint{9.008183in}{1.540143in}}{\pgfqpoint{9.006179in}{1.535305in}}{\pgfqpoint{9.006179in}{1.530261in}}%
\pgfpathcurveto{\pgfqpoint{9.006179in}{1.525217in}}{\pgfqpoint{9.008183in}{1.520380in}}{\pgfqpoint{9.011750in}{1.516813in}}%
\pgfpathcurveto{\pgfqpoint{9.015316in}{1.513247in}}{\pgfqpoint{9.020154in}{1.511243in}}{\pgfqpoint{9.025197in}{1.511243in}}%
\pgfpathclose%
\pgfusepath{fill}%
\end{pgfscope}%
\begin{pgfscope}%
\pgfpathrectangle{\pgfqpoint{6.572727in}{0.474100in}}{\pgfqpoint{4.227273in}{3.318700in}}%
\pgfusepath{clip}%
\pgfsetbuttcap%
\pgfsetroundjoin%
\definecolor{currentfill}{rgb}{0.127568,0.566949,0.550556}%
\pgfsetfillcolor{currentfill}%
\pgfsetfillopacity{0.700000}%
\pgfsetlinewidth{0.000000pt}%
\definecolor{currentstroke}{rgb}{0.000000,0.000000,0.000000}%
\pgfsetstrokecolor{currentstroke}%
\pgfsetstrokeopacity{0.700000}%
\pgfsetdash{}{0pt}%
\pgfpathmoveto{\pgfqpoint{8.583198in}{2.920835in}}%
\pgfpathcurveto{\pgfqpoint{8.588241in}{2.920835in}}{\pgfqpoint{8.593079in}{2.922839in}}{\pgfqpoint{8.596645in}{2.926405in}}%
\pgfpathcurveto{\pgfqpoint{8.600212in}{2.929972in}}{\pgfqpoint{8.602216in}{2.934809in}}{\pgfqpoint{8.602216in}{2.939853in}}%
\pgfpathcurveto{\pgfqpoint{8.602216in}{2.944897in}}{\pgfqpoint{8.600212in}{2.949735in}}{\pgfqpoint{8.596645in}{2.953301in}}%
\pgfpathcurveto{\pgfqpoint{8.593079in}{2.956867in}}{\pgfqpoint{8.588241in}{2.958871in}}{\pgfqpoint{8.583198in}{2.958871in}}%
\pgfpathcurveto{\pgfqpoint{8.578154in}{2.958871in}}{\pgfqpoint{8.573316in}{2.956867in}}{\pgfqpoint{8.569750in}{2.953301in}}%
\pgfpathcurveto{\pgfqpoint{8.566183in}{2.949735in}}{\pgfqpoint{8.564179in}{2.944897in}}{\pgfqpoint{8.564179in}{2.939853in}}%
\pgfpathcurveto{\pgfqpoint{8.564179in}{2.934809in}}{\pgfqpoint{8.566183in}{2.929972in}}{\pgfqpoint{8.569750in}{2.926405in}}%
\pgfpathcurveto{\pgfqpoint{8.573316in}{2.922839in}}{\pgfqpoint{8.578154in}{2.920835in}}{\pgfqpoint{8.583198in}{2.920835in}}%
\pgfpathclose%
\pgfusepath{fill}%
\end{pgfscope}%
\begin{pgfscope}%
\pgfpathrectangle{\pgfqpoint{6.572727in}{0.474100in}}{\pgfqpoint{4.227273in}{3.318700in}}%
\pgfusepath{clip}%
\pgfsetbuttcap%
\pgfsetroundjoin%
\definecolor{currentfill}{rgb}{0.127568,0.566949,0.550556}%
\pgfsetfillcolor{currentfill}%
\pgfsetfillopacity{0.700000}%
\pgfsetlinewidth{0.000000pt}%
\definecolor{currentstroke}{rgb}{0.000000,0.000000,0.000000}%
\pgfsetstrokecolor{currentstroke}%
\pgfsetstrokeopacity{0.700000}%
\pgfsetdash{}{0pt}%
\pgfpathmoveto{\pgfqpoint{8.106954in}{2.402770in}}%
\pgfpathcurveto{\pgfqpoint{8.111998in}{2.402770in}}{\pgfqpoint{8.116836in}{2.404774in}}{\pgfqpoint{8.120402in}{2.408340in}}%
\pgfpathcurveto{\pgfqpoint{8.123969in}{2.411907in}}{\pgfqpoint{8.125972in}{2.416744in}}{\pgfqpoint{8.125972in}{2.421788in}}%
\pgfpathcurveto{\pgfqpoint{8.125972in}{2.426832in}}{\pgfqpoint{8.123969in}{2.431670in}}{\pgfqpoint{8.120402in}{2.435236in}}%
\pgfpathcurveto{\pgfqpoint{8.116836in}{2.438802in}}{\pgfqpoint{8.111998in}{2.440806in}}{\pgfqpoint{8.106954in}{2.440806in}}%
\pgfpathcurveto{\pgfqpoint{8.101911in}{2.440806in}}{\pgfqpoint{8.097073in}{2.438802in}}{\pgfqpoint{8.093506in}{2.435236in}}%
\pgfpathcurveto{\pgfqpoint{8.089940in}{2.431670in}}{\pgfqpoint{8.087936in}{2.426832in}}{\pgfqpoint{8.087936in}{2.421788in}}%
\pgfpathcurveto{\pgfqpoint{8.087936in}{2.416744in}}{\pgfqpoint{8.089940in}{2.411907in}}{\pgfqpoint{8.093506in}{2.408340in}}%
\pgfpathcurveto{\pgfqpoint{8.097073in}{2.404774in}}{\pgfqpoint{8.101911in}{2.402770in}}{\pgfqpoint{8.106954in}{2.402770in}}%
\pgfpathclose%
\pgfusepath{fill}%
\end{pgfscope}%
\begin{pgfscope}%
\pgfpathrectangle{\pgfqpoint{6.572727in}{0.474100in}}{\pgfqpoint{4.227273in}{3.318700in}}%
\pgfusepath{clip}%
\pgfsetbuttcap%
\pgfsetroundjoin%
\definecolor{currentfill}{rgb}{0.127568,0.566949,0.550556}%
\pgfsetfillcolor{currentfill}%
\pgfsetfillopacity{0.700000}%
\pgfsetlinewidth{0.000000pt}%
\definecolor{currentstroke}{rgb}{0.000000,0.000000,0.000000}%
\pgfsetstrokecolor{currentstroke}%
\pgfsetstrokeopacity{0.700000}%
\pgfsetdash{}{0pt}%
\pgfpathmoveto{\pgfqpoint{8.002811in}{2.641515in}}%
\pgfpathcurveto{\pgfqpoint{8.007855in}{2.641515in}}{\pgfqpoint{8.012693in}{2.643519in}}{\pgfqpoint{8.016259in}{2.647085in}}%
\pgfpathcurveto{\pgfqpoint{8.019825in}{2.650652in}}{\pgfqpoint{8.021829in}{2.655489in}}{\pgfqpoint{8.021829in}{2.660533in}}%
\pgfpathcurveto{\pgfqpoint{8.021829in}{2.665577in}}{\pgfqpoint{8.019825in}{2.670414in}}{\pgfqpoint{8.016259in}{2.673981in}}%
\pgfpathcurveto{\pgfqpoint{8.012693in}{2.677547in}}{\pgfqpoint{8.007855in}{2.679551in}}{\pgfqpoint{8.002811in}{2.679551in}}%
\pgfpathcurveto{\pgfqpoint{7.997767in}{2.679551in}}{\pgfqpoint{7.992930in}{2.677547in}}{\pgfqpoint{7.989363in}{2.673981in}}%
\pgfpathcurveto{\pgfqpoint{7.985797in}{2.670414in}}{\pgfqpoint{7.983793in}{2.665577in}}{\pgfqpoint{7.983793in}{2.660533in}}%
\pgfpathcurveto{\pgfqpoint{7.983793in}{2.655489in}}{\pgfqpoint{7.985797in}{2.650652in}}{\pgfqpoint{7.989363in}{2.647085in}}%
\pgfpathcurveto{\pgfqpoint{7.992930in}{2.643519in}}{\pgfqpoint{7.997767in}{2.641515in}}{\pgfqpoint{8.002811in}{2.641515in}}%
\pgfpathclose%
\pgfusepath{fill}%
\end{pgfscope}%
\begin{pgfscope}%
\pgfpathrectangle{\pgfqpoint{6.572727in}{0.474100in}}{\pgfqpoint{4.227273in}{3.318700in}}%
\pgfusepath{clip}%
\pgfsetbuttcap%
\pgfsetroundjoin%
\definecolor{currentfill}{rgb}{0.127568,0.566949,0.550556}%
\pgfsetfillcolor{currentfill}%
\pgfsetfillopacity{0.700000}%
\pgfsetlinewidth{0.000000pt}%
\definecolor{currentstroke}{rgb}{0.000000,0.000000,0.000000}%
\pgfsetstrokecolor{currentstroke}%
\pgfsetstrokeopacity{0.700000}%
\pgfsetdash{}{0pt}%
\pgfpathmoveto{\pgfqpoint{7.933425in}{0.960938in}}%
\pgfpathcurveto{\pgfqpoint{7.938469in}{0.960938in}}{\pgfqpoint{7.943306in}{0.962942in}}{\pgfqpoint{7.946873in}{0.966509in}}%
\pgfpathcurveto{\pgfqpoint{7.950439in}{0.970075in}}{\pgfqpoint{7.952443in}{0.974913in}}{\pgfqpoint{7.952443in}{0.979957in}}%
\pgfpathcurveto{\pgfqpoint{7.952443in}{0.985000in}}{\pgfqpoint{7.950439in}{0.989838in}}{\pgfqpoint{7.946873in}{0.993404in}}%
\pgfpathcurveto{\pgfqpoint{7.943306in}{0.996971in}}{\pgfqpoint{7.938469in}{0.998975in}}{\pgfqpoint{7.933425in}{0.998975in}}%
\pgfpathcurveto{\pgfqpoint{7.928381in}{0.998975in}}{\pgfqpoint{7.923543in}{0.996971in}}{\pgfqpoint{7.919977in}{0.993404in}}%
\pgfpathcurveto{\pgfqpoint{7.916411in}{0.989838in}}{\pgfqpoint{7.914407in}{0.985000in}}{\pgfqpoint{7.914407in}{0.979957in}}%
\pgfpathcurveto{\pgfqpoint{7.914407in}{0.974913in}}{\pgfqpoint{7.916411in}{0.970075in}}{\pgfqpoint{7.919977in}{0.966509in}}%
\pgfpathcurveto{\pgfqpoint{7.923543in}{0.962942in}}{\pgfqpoint{7.928381in}{0.960938in}}{\pgfqpoint{7.933425in}{0.960938in}}%
\pgfpathclose%
\pgfusepath{fill}%
\end{pgfscope}%
\begin{pgfscope}%
\pgfpathrectangle{\pgfqpoint{6.572727in}{0.474100in}}{\pgfqpoint{4.227273in}{3.318700in}}%
\pgfusepath{clip}%
\pgfsetbuttcap%
\pgfsetroundjoin%
\definecolor{currentfill}{rgb}{0.127568,0.566949,0.550556}%
\pgfsetfillcolor{currentfill}%
\pgfsetfillopacity{0.700000}%
\pgfsetlinewidth{0.000000pt}%
\definecolor{currentstroke}{rgb}{0.000000,0.000000,0.000000}%
\pgfsetstrokecolor{currentstroke}%
\pgfsetstrokeopacity{0.700000}%
\pgfsetdash{}{0pt}%
\pgfpathmoveto{\pgfqpoint{8.204208in}{2.257066in}}%
\pgfpathcurveto{\pgfqpoint{8.209252in}{2.257066in}}{\pgfqpoint{8.214089in}{2.259070in}}{\pgfqpoint{8.217656in}{2.262637in}}%
\pgfpathcurveto{\pgfqpoint{8.221222in}{2.266203in}}{\pgfqpoint{8.223226in}{2.271041in}}{\pgfqpoint{8.223226in}{2.276084in}}%
\pgfpathcurveto{\pgfqpoint{8.223226in}{2.281128in}}{\pgfqpoint{8.221222in}{2.285966in}}{\pgfqpoint{8.217656in}{2.289532in}}%
\pgfpathcurveto{\pgfqpoint{8.214089in}{2.293099in}}{\pgfqpoint{8.209252in}{2.295103in}}{\pgfqpoint{8.204208in}{2.295103in}}%
\pgfpathcurveto{\pgfqpoint{8.199164in}{2.295103in}}{\pgfqpoint{8.194327in}{2.293099in}}{\pgfqpoint{8.190760in}{2.289532in}}%
\pgfpathcurveto{\pgfqpoint{8.187194in}{2.285966in}}{\pgfqpoint{8.185190in}{2.281128in}}{\pgfqpoint{8.185190in}{2.276084in}}%
\pgfpathcurveto{\pgfqpoint{8.185190in}{2.271041in}}{\pgfqpoint{8.187194in}{2.266203in}}{\pgfqpoint{8.190760in}{2.262637in}}%
\pgfpathcurveto{\pgfqpoint{8.194327in}{2.259070in}}{\pgfqpoint{8.199164in}{2.257066in}}{\pgfqpoint{8.204208in}{2.257066in}}%
\pgfpathclose%
\pgfusepath{fill}%
\end{pgfscope}%
\begin{pgfscope}%
\pgfpathrectangle{\pgfqpoint{6.572727in}{0.474100in}}{\pgfqpoint{4.227273in}{3.318700in}}%
\pgfusepath{clip}%
\pgfsetbuttcap%
\pgfsetroundjoin%
\definecolor{currentfill}{rgb}{0.127568,0.566949,0.550556}%
\pgfsetfillcolor{currentfill}%
\pgfsetfillopacity{0.700000}%
\pgfsetlinewidth{0.000000pt}%
\definecolor{currentstroke}{rgb}{0.000000,0.000000,0.000000}%
\pgfsetstrokecolor{currentstroke}%
\pgfsetstrokeopacity{0.700000}%
\pgfsetdash{}{0pt}%
\pgfpathmoveto{\pgfqpoint{7.617970in}{2.034693in}}%
\pgfpathcurveto{\pgfqpoint{7.623014in}{2.034693in}}{\pgfqpoint{7.627851in}{2.036696in}}{\pgfqpoint{7.631418in}{2.040263in}}%
\pgfpathcurveto{\pgfqpoint{7.634984in}{2.043829in}}{\pgfqpoint{7.636988in}{2.048667in}}{\pgfqpoint{7.636988in}{2.053711in}}%
\pgfpathcurveto{\pgfqpoint{7.636988in}{2.058754in}}{\pgfqpoint{7.634984in}{2.063592in}}{\pgfqpoint{7.631418in}{2.067159in}}%
\pgfpathcurveto{\pgfqpoint{7.627851in}{2.070725in}}{\pgfqpoint{7.623014in}{2.072729in}}{\pgfqpoint{7.617970in}{2.072729in}}%
\pgfpathcurveto{\pgfqpoint{7.612926in}{2.072729in}}{\pgfqpoint{7.608089in}{2.070725in}}{\pgfqpoint{7.604522in}{2.067159in}}%
\pgfpathcurveto{\pgfqpoint{7.600956in}{2.063592in}}{\pgfqpoint{7.598952in}{2.058754in}}{\pgfqpoint{7.598952in}{2.053711in}}%
\pgfpathcurveto{\pgfqpoint{7.598952in}{2.048667in}}{\pgfqpoint{7.600956in}{2.043829in}}{\pgfqpoint{7.604522in}{2.040263in}}%
\pgfpathcurveto{\pgfqpoint{7.608089in}{2.036696in}}{\pgfqpoint{7.612926in}{2.034693in}}{\pgfqpoint{7.617970in}{2.034693in}}%
\pgfpathclose%
\pgfusepath{fill}%
\end{pgfscope}%
\begin{pgfscope}%
\pgfpathrectangle{\pgfqpoint{6.572727in}{0.474100in}}{\pgfqpoint{4.227273in}{3.318700in}}%
\pgfusepath{clip}%
\pgfsetbuttcap%
\pgfsetroundjoin%
\definecolor{currentfill}{rgb}{0.993248,0.906157,0.143936}%
\pgfsetfillcolor{currentfill}%
\pgfsetfillopacity{0.700000}%
\pgfsetlinewidth{0.000000pt}%
\definecolor{currentstroke}{rgb}{0.000000,0.000000,0.000000}%
\pgfsetstrokecolor{currentstroke}%
\pgfsetstrokeopacity{0.700000}%
\pgfsetdash{}{0pt}%
\pgfpathmoveto{\pgfqpoint{9.077907in}{1.117363in}}%
\pgfpathcurveto{\pgfqpoint{9.082951in}{1.117363in}}{\pgfqpoint{9.087788in}{1.119367in}}{\pgfqpoint{9.091355in}{1.122934in}}%
\pgfpathcurveto{\pgfqpoint{9.094921in}{1.126500in}}{\pgfqpoint{9.096925in}{1.131338in}}{\pgfqpoint{9.096925in}{1.136382in}}%
\pgfpathcurveto{\pgfqpoint{9.096925in}{1.141425in}}{\pgfqpoint{9.094921in}{1.146263in}}{\pgfqpoint{9.091355in}{1.149829in}}%
\pgfpathcurveto{\pgfqpoint{9.087788in}{1.153396in}}{\pgfqpoint{9.082951in}{1.155400in}}{\pgfqpoint{9.077907in}{1.155400in}}%
\pgfpathcurveto{\pgfqpoint{9.072863in}{1.155400in}}{\pgfqpoint{9.068025in}{1.153396in}}{\pgfqpoint{9.064459in}{1.149829in}}%
\pgfpathcurveto{\pgfqpoint{9.060893in}{1.146263in}}{\pgfqpoint{9.058889in}{1.141425in}}{\pgfqpoint{9.058889in}{1.136382in}}%
\pgfpathcurveto{\pgfqpoint{9.058889in}{1.131338in}}{\pgfqpoint{9.060893in}{1.126500in}}{\pgfqpoint{9.064459in}{1.122934in}}%
\pgfpathcurveto{\pgfqpoint{9.068025in}{1.119367in}}{\pgfqpoint{9.072863in}{1.117363in}}{\pgfqpoint{9.077907in}{1.117363in}}%
\pgfpathclose%
\pgfusepath{fill}%
\end{pgfscope}%
\begin{pgfscope}%
\pgfpathrectangle{\pgfqpoint{6.572727in}{0.474100in}}{\pgfqpoint{4.227273in}{3.318700in}}%
\pgfusepath{clip}%
\pgfsetbuttcap%
\pgfsetroundjoin%
\definecolor{currentfill}{rgb}{0.127568,0.566949,0.550556}%
\pgfsetfillcolor{currentfill}%
\pgfsetfillopacity{0.700000}%
\pgfsetlinewidth{0.000000pt}%
\definecolor{currentstroke}{rgb}{0.000000,0.000000,0.000000}%
\pgfsetstrokecolor{currentstroke}%
\pgfsetstrokeopacity{0.700000}%
\pgfsetdash{}{0pt}%
\pgfpathmoveto{\pgfqpoint{8.162401in}{2.712600in}}%
\pgfpathcurveto{\pgfqpoint{8.167445in}{2.712600in}}{\pgfqpoint{8.172283in}{2.714604in}}{\pgfqpoint{8.175849in}{2.718171in}}%
\pgfpathcurveto{\pgfqpoint{8.179415in}{2.721737in}}{\pgfqpoint{8.181419in}{2.726575in}}{\pgfqpoint{8.181419in}{2.731618in}}%
\pgfpathcurveto{\pgfqpoint{8.181419in}{2.736662in}}{\pgfqpoint{8.179415in}{2.741500in}}{\pgfqpoint{8.175849in}{2.745066in}}%
\pgfpathcurveto{\pgfqpoint{8.172283in}{2.748633in}}{\pgfqpoint{8.167445in}{2.750637in}}{\pgfqpoint{8.162401in}{2.750637in}}%
\pgfpathcurveto{\pgfqpoint{8.157357in}{2.750637in}}{\pgfqpoint{8.152520in}{2.748633in}}{\pgfqpoint{8.148953in}{2.745066in}}%
\pgfpathcurveto{\pgfqpoint{8.145387in}{2.741500in}}{\pgfqpoint{8.143383in}{2.736662in}}{\pgfqpoint{8.143383in}{2.731618in}}%
\pgfpathcurveto{\pgfqpoint{8.143383in}{2.726575in}}{\pgfqpoint{8.145387in}{2.721737in}}{\pgfqpoint{8.148953in}{2.718171in}}%
\pgfpathcurveto{\pgfqpoint{8.152520in}{2.714604in}}{\pgfqpoint{8.157357in}{2.712600in}}{\pgfqpoint{8.162401in}{2.712600in}}%
\pgfpathclose%
\pgfusepath{fill}%
\end{pgfscope}%
\begin{pgfscope}%
\pgfpathrectangle{\pgfqpoint{6.572727in}{0.474100in}}{\pgfqpoint{4.227273in}{3.318700in}}%
\pgfusepath{clip}%
\pgfsetbuttcap%
\pgfsetroundjoin%
\definecolor{currentfill}{rgb}{0.127568,0.566949,0.550556}%
\pgfsetfillcolor{currentfill}%
\pgfsetfillopacity{0.700000}%
\pgfsetlinewidth{0.000000pt}%
\definecolor{currentstroke}{rgb}{0.000000,0.000000,0.000000}%
\pgfsetstrokecolor{currentstroke}%
\pgfsetstrokeopacity{0.700000}%
\pgfsetdash{}{0pt}%
\pgfpathmoveto{\pgfqpoint{7.568948in}{1.406068in}}%
\pgfpathcurveto{\pgfqpoint{7.573992in}{1.406068in}}{\pgfqpoint{7.578830in}{1.408072in}}{\pgfqpoint{7.582396in}{1.411638in}}%
\pgfpathcurveto{\pgfqpoint{7.585962in}{1.415205in}}{\pgfqpoint{7.587966in}{1.420043in}}{\pgfqpoint{7.587966in}{1.425086in}}%
\pgfpathcurveto{\pgfqpoint{7.587966in}{1.430130in}}{\pgfqpoint{7.585962in}{1.434968in}}{\pgfqpoint{7.582396in}{1.438534in}}%
\pgfpathcurveto{\pgfqpoint{7.578830in}{1.442100in}}{\pgfqpoint{7.573992in}{1.444104in}}{\pgfqpoint{7.568948in}{1.444104in}}%
\pgfpathcurveto{\pgfqpoint{7.563904in}{1.444104in}}{\pgfqpoint{7.559067in}{1.442100in}}{\pgfqpoint{7.555500in}{1.438534in}}%
\pgfpathcurveto{\pgfqpoint{7.551934in}{1.434968in}}{\pgfqpoint{7.549930in}{1.430130in}}{\pgfqpoint{7.549930in}{1.425086in}}%
\pgfpathcurveto{\pgfqpoint{7.549930in}{1.420043in}}{\pgfqpoint{7.551934in}{1.415205in}}{\pgfqpoint{7.555500in}{1.411638in}}%
\pgfpathcurveto{\pgfqpoint{7.559067in}{1.408072in}}{\pgfqpoint{7.563904in}{1.406068in}}{\pgfqpoint{7.568948in}{1.406068in}}%
\pgfpathclose%
\pgfusepath{fill}%
\end{pgfscope}%
\begin{pgfscope}%
\pgfpathrectangle{\pgfqpoint{6.572727in}{0.474100in}}{\pgfqpoint{4.227273in}{3.318700in}}%
\pgfusepath{clip}%
\pgfsetbuttcap%
\pgfsetroundjoin%
\definecolor{currentfill}{rgb}{0.993248,0.906157,0.143936}%
\pgfsetfillcolor{currentfill}%
\pgfsetfillopacity{0.700000}%
\pgfsetlinewidth{0.000000pt}%
\definecolor{currentstroke}{rgb}{0.000000,0.000000,0.000000}%
\pgfsetstrokecolor{currentstroke}%
\pgfsetstrokeopacity{0.700000}%
\pgfsetdash{}{0pt}%
\pgfpathmoveto{\pgfqpoint{9.632150in}{1.761138in}}%
\pgfpathcurveto{\pgfqpoint{9.637194in}{1.761138in}}{\pgfqpoint{9.642032in}{1.763142in}}{\pgfqpoint{9.645598in}{1.766709in}}%
\pgfpathcurveto{\pgfqpoint{9.649165in}{1.770275in}}{\pgfqpoint{9.651169in}{1.775113in}}{\pgfqpoint{9.651169in}{1.780157in}}%
\pgfpathcurveto{\pgfqpoint{9.651169in}{1.785200in}}{\pgfqpoint{9.649165in}{1.790038in}}{\pgfqpoint{9.645598in}{1.793604in}}%
\pgfpathcurveto{\pgfqpoint{9.642032in}{1.797171in}}{\pgfqpoint{9.637194in}{1.799175in}}{\pgfqpoint{9.632150in}{1.799175in}}%
\pgfpathcurveto{\pgfqpoint{9.627107in}{1.799175in}}{\pgfqpoint{9.622269in}{1.797171in}}{\pgfqpoint{9.618703in}{1.793604in}}%
\pgfpathcurveto{\pgfqpoint{9.615136in}{1.790038in}}{\pgfqpoint{9.613132in}{1.785200in}}{\pgfqpoint{9.613132in}{1.780157in}}%
\pgfpathcurveto{\pgfqpoint{9.613132in}{1.775113in}}{\pgfqpoint{9.615136in}{1.770275in}}{\pgfqpoint{9.618703in}{1.766709in}}%
\pgfpathcurveto{\pgfqpoint{9.622269in}{1.763142in}}{\pgfqpoint{9.627107in}{1.761138in}}{\pgfqpoint{9.632150in}{1.761138in}}%
\pgfpathclose%
\pgfusepath{fill}%
\end{pgfscope}%
\begin{pgfscope}%
\pgfpathrectangle{\pgfqpoint{6.572727in}{0.474100in}}{\pgfqpoint{4.227273in}{3.318700in}}%
\pgfusepath{clip}%
\pgfsetbuttcap%
\pgfsetroundjoin%
\definecolor{currentfill}{rgb}{0.127568,0.566949,0.550556}%
\pgfsetfillcolor{currentfill}%
\pgfsetfillopacity{0.700000}%
\pgfsetlinewidth{0.000000pt}%
\definecolor{currentstroke}{rgb}{0.000000,0.000000,0.000000}%
\pgfsetstrokecolor{currentstroke}%
\pgfsetstrokeopacity{0.700000}%
\pgfsetdash{}{0pt}%
\pgfpathmoveto{\pgfqpoint{7.644865in}{2.288230in}}%
\pgfpathcurveto{\pgfqpoint{7.649909in}{2.288230in}}{\pgfqpoint{7.654747in}{2.290233in}}{\pgfqpoint{7.658313in}{2.293800in}}%
\pgfpathcurveto{\pgfqpoint{7.661880in}{2.297366in}}{\pgfqpoint{7.663884in}{2.302204in}}{\pgfqpoint{7.663884in}{2.307248in}}%
\pgfpathcurveto{\pgfqpoint{7.663884in}{2.312291in}}{\pgfqpoint{7.661880in}{2.317129in}}{\pgfqpoint{7.658313in}{2.320696in}}%
\pgfpathcurveto{\pgfqpoint{7.654747in}{2.324262in}}{\pgfqpoint{7.649909in}{2.326266in}}{\pgfqpoint{7.644865in}{2.326266in}}%
\pgfpathcurveto{\pgfqpoint{7.639822in}{2.326266in}}{\pgfqpoint{7.634984in}{2.324262in}}{\pgfqpoint{7.631418in}{2.320696in}}%
\pgfpathcurveto{\pgfqpoint{7.627851in}{2.317129in}}{\pgfqpoint{7.625847in}{2.312291in}}{\pgfqpoint{7.625847in}{2.307248in}}%
\pgfpathcurveto{\pgfqpoint{7.625847in}{2.302204in}}{\pgfqpoint{7.627851in}{2.297366in}}{\pgfqpoint{7.631418in}{2.293800in}}%
\pgfpathcurveto{\pgfqpoint{7.634984in}{2.290233in}}{\pgfqpoint{7.639822in}{2.288230in}}{\pgfqpoint{7.644865in}{2.288230in}}%
\pgfpathclose%
\pgfusepath{fill}%
\end{pgfscope}%
\begin{pgfscope}%
\pgfpathrectangle{\pgfqpoint{6.572727in}{0.474100in}}{\pgfqpoint{4.227273in}{3.318700in}}%
\pgfusepath{clip}%
\pgfsetbuttcap%
\pgfsetroundjoin%
\definecolor{currentfill}{rgb}{0.267004,0.004874,0.329415}%
\pgfsetfillcolor{currentfill}%
\pgfsetfillopacity{0.700000}%
\pgfsetlinewidth{0.000000pt}%
\definecolor{currentstroke}{rgb}{0.000000,0.000000,0.000000}%
\pgfsetstrokecolor{currentstroke}%
\pgfsetstrokeopacity{0.700000}%
\pgfsetdash{}{0pt}%
\pgfpathmoveto{\pgfqpoint{9.867545in}{0.706532in}}%
\pgfpathcurveto{\pgfqpoint{9.872589in}{0.706532in}}{\pgfqpoint{9.877426in}{0.708536in}}{\pgfqpoint{9.880993in}{0.712102in}}%
\pgfpathcurveto{\pgfqpoint{9.884559in}{0.715669in}}{\pgfqpoint{9.886563in}{0.720506in}}{\pgfqpoint{9.886563in}{0.725550in}}%
\pgfpathcurveto{\pgfqpoint{9.886563in}{0.730594in}}{\pgfqpoint{9.884559in}{0.735432in}}{\pgfqpoint{9.880993in}{0.738998in}}%
\pgfpathcurveto{\pgfqpoint{9.877426in}{0.742564in}}{\pgfqpoint{9.872589in}{0.744568in}}{\pgfqpoint{9.867545in}{0.744568in}}%
\pgfpathcurveto{\pgfqpoint{9.862501in}{0.744568in}}{\pgfqpoint{9.857663in}{0.742564in}}{\pgfqpoint{9.854097in}{0.738998in}}%
\pgfpathcurveto{\pgfqpoint{9.850531in}{0.735432in}}{\pgfqpoint{9.848527in}{0.730594in}}{\pgfqpoint{9.848527in}{0.725550in}}%
\pgfpathcurveto{\pgfqpoint{9.848527in}{0.720506in}}{\pgfqpoint{9.850531in}{0.715669in}}{\pgfqpoint{9.854097in}{0.712102in}}%
\pgfpathcurveto{\pgfqpoint{9.857663in}{0.708536in}}{\pgfqpoint{9.862501in}{0.706532in}}{\pgfqpoint{9.867545in}{0.706532in}}%
\pgfpathclose%
\pgfusepath{fill}%
\end{pgfscope}%
\begin{pgfscope}%
\pgfpathrectangle{\pgfqpoint{6.572727in}{0.474100in}}{\pgfqpoint{4.227273in}{3.318700in}}%
\pgfusepath{clip}%
\pgfsetbuttcap%
\pgfsetroundjoin%
\definecolor{currentfill}{rgb}{0.127568,0.566949,0.550556}%
\pgfsetfillcolor{currentfill}%
\pgfsetfillopacity{0.700000}%
\pgfsetlinewidth{0.000000pt}%
\definecolor{currentstroke}{rgb}{0.000000,0.000000,0.000000}%
\pgfsetstrokecolor{currentstroke}%
\pgfsetstrokeopacity{0.700000}%
\pgfsetdash{}{0pt}%
\pgfpathmoveto{\pgfqpoint{8.457399in}{2.931206in}}%
\pgfpathcurveto{\pgfqpoint{8.462443in}{2.931206in}}{\pgfqpoint{8.467280in}{2.933210in}}{\pgfqpoint{8.470847in}{2.936776in}}%
\pgfpathcurveto{\pgfqpoint{8.474413in}{2.940342in}}{\pgfqpoint{8.476417in}{2.945180in}}{\pgfqpoint{8.476417in}{2.950224in}}%
\pgfpathcurveto{\pgfqpoint{8.476417in}{2.955268in}}{\pgfqpoint{8.474413in}{2.960105in}}{\pgfqpoint{8.470847in}{2.963672in}}%
\pgfpathcurveto{\pgfqpoint{8.467280in}{2.967238in}}{\pgfqpoint{8.462443in}{2.969242in}}{\pgfqpoint{8.457399in}{2.969242in}}%
\pgfpathcurveto{\pgfqpoint{8.452355in}{2.969242in}}{\pgfqpoint{8.447517in}{2.967238in}}{\pgfqpoint{8.443951in}{2.963672in}}%
\pgfpathcurveto{\pgfqpoint{8.440385in}{2.960105in}}{\pgfqpoint{8.438381in}{2.955268in}}{\pgfqpoint{8.438381in}{2.950224in}}%
\pgfpathcurveto{\pgfqpoint{8.438381in}{2.945180in}}{\pgfqpoint{8.440385in}{2.940342in}}{\pgfqpoint{8.443951in}{2.936776in}}%
\pgfpathcurveto{\pgfqpoint{8.447517in}{2.933210in}}{\pgfqpoint{8.452355in}{2.931206in}}{\pgfqpoint{8.457399in}{2.931206in}}%
\pgfpathclose%
\pgfusepath{fill}%
\end{pgfscope}%
\begin{pgfscope}%
\pgfpathrectangle{\pgfqpoint{6.572727in}{0.474100in}}{\pgfqpoint{4.227273in}{3.318700in}}%
\pgfusepath{clip}%
\pgfsetbuttcap%
\pgfsetroundjoin%
\definecolor{currentfill}{rgb}{0.993248,0.906157,0.143936}%
\pgfsetfillcolor{currentfill}%
\pgfsetfillopacity{0.700000}%
\pgfsetlinewidth{0.000000pt}%
\definecolor{currentstroke}{rgb}{0.000000,0.000000,0.000000}%
\pgfsetstrokecolor{currentstroke}%
\pgfsetstrokeopacity{0.700000}%
\pgfsetdash{}{0pt}%
\pgfpathmoveto{\pgfqpoint{9.993451in}{1.733375in}}%
\pgfpathcurveto{\pgfqpoint{9.998495in}{1.733375in}}{\pgfqpoint{10.003333in}{1.735379in}}{\pgfqpoint{10.006899in}{1.738945in}}%
\pgfpathcurveto{\pgfqpoint{10.010465in}{1.742511in}}{\pgfqpoint{10.012469in}{1.747349in}}{\pgfqpoint{10.012469in}{1.752393in}}%
\pgfpathcurveto{\pgfqpoint{10.012469in}{1.757436in}}{\pgfqpoint{10.010465in}{1.762274in}}{\pgfqpoint{10.006899in}{1.765841in}}%
\pgfpathcurveto{\pgfqpoint{10.003333in}{1.769407in}}{\pgfqpoint{9.998495in}{1.771411in}}{\pgfqpoint{9.993451in}{1.771411in}}%
\pgfpathcurveto{\pgfqpoint{9.988407in}{1.771411in}}{\pgfqpoint{9.983570in}{1.769407in}}{\pgfqpoint{9.980003in}{1.765841in}}%
\pgfpathcurveto{\pgfqpoint{9.976437in}{1.762274in}}{\pgfqpoint{9.974433in}{1.757436in}}{\pgfqpoint{9.974433in}{1.752393in}}%
\pgfpathcurveto{\pgfqpoint{9.974433in}{1.747349in}}{\pgfqpoint{9.976437in}{1.742511in}}{\pgfqpoint{9.980003in}{1.738945in}}%
\pgfpathcurveto{\pgfqpoint{9.983570in}{1.735379in}}{\pgfqpoint{9.988407in}{1.733375in}}{\pgfqpoint{9.993451in}{1.733375in}}%
\pgfpathclose%
\pgfusepath{fill}%
\end{pgfscope}%
\begin{pgfscope}%
\pgfpathrectangle{\pgfqpoint{6.572727in}{0.474100in}}{\pgfqpoint{4.227273in}{3.318700in}}%
\pgfusepath{clip}%
\pgfsetbuttcap%
\pgfsetroundjoin%
\definecolor{currentfill}{rgb}{0.127568,0.566949,0.550556}%
\pgfsetfillcolor{currentfill}%
\pgfsetfillopacity{0.700000}%
\pgfsetlinewidth{0.000000pt}%
\definecolor{currentstroke}{rgb}{0.000000,0.000000,0.000000}%
\pgfsetstrokecolor{currentstroke}%
\pgfsetstrokeopacity{0.700000}%
\pgfsetdash{}{0pt}%
\pgfpathmoveto{\pgfqpoint{8.398872in}{1.657721in}}%
\pgfpathcurveto{\pgfqpoint{8.403915in}{1.657721in}}{\pgfqpoint{8.408753in}{1.659725in}}{\pgfqpoint{8.412320in}{1.663291in}}%
\pgfpathcurveto{\pgfqpoint{8.415886in}{1.666857in}}{\pgfqpoint{8.417890in}{1.671695in}}{\pgfqpoint{8.417890in}{1.676739in}}%
\pgfpathcurveto{\pgfqpoint{8.417890in}{1.681782in}}{\pgfqpoint{8.415886in}{1.686620in}}{\pgfqpoint{8.412320in}{1.690187in}}%
\pgfpathcurveto{\pgfqpoint{8.408753in}{1.693753in}}{\pgfqpoint{8.403915in}{1.695757in}}{\pgfqpoint{8.398872in}{1.695757in}}%
\pgfpathcurveto{\pgfqpoint{8.393828in}{1.695757in}}{\pgfqpoint{8.388990in}{1.693753in}}{\pgfqpoint{8.385424in}{1.690187in}}%
\pgfpathcurveto{\pgfqpoint{8.381857in}{1.686620in}}{\pgfqpoint{8.379854in}{1.681782in}}{\pgfqpoint{8.379854in}{1.676739in}}%
\pgfpathcurveto{\pgfqpoint{8.379854in}{1.671695in}}{\pgfqpoint{8.381857in}{1.666857in}}{\pgfqpoint{8.385424in}{1.663291in}}%
\pgfpathcurveto{\pgfqpoint{8.388990in}{1.659725in}}{\pgfqpoint{8.393828in}{1.657721in}}{\pgfqpoint{8.398872in}{1.657721in}}%
\pgfpathclose%
\pgfusepath{fill}%
\end{pgfscope}%
\begin{pgfscope}%
\pgfpathrectangle{\pgfqpoint{6.572727in}{0.474100in}}{\pgfqpoint{4.227273in}{3.318700in}}%
\pgfusepath{clip}%
\pgfsetbuttcap%
\pgfsetroundjoin%
\definecolor{currentfill}{rgb}{0.993248,0.906157,0.143936}%
\pgfsetfillcolor{currentfill}%
\pgfsetfillopacity{0.700000}%
\pgfsetlinewidth{0.000000pt}%
\definecolor{currentstroke}{rgb}{0.000000,0.000000,0.000000}%
\pgfsetstrokecolor{currentstroke}%
\pgfsetstrokeopacity{0.700000}%
\pgfsetdash{}{0pt}%
\pgfpathmoveto{\pgfqpoint{10.287256in}{1.702873in}}%
\pgfpathcurveto{\pgfqpoint{10.292300in}{1.702873in}}{\pgfqpoint{10.297138in}{1.704877in}}{\pgfqpoint{10.300704in}{1.708443in}}%
\pgfpathcurveto{\pgfqpoint{10.304270in}{1.712010in}}{\pgfqpoint{10.306274in}{1.716848in}}{\pgfqpoint{10.306274in}{1.721891in}}%
\pgfpathcurveto{\pgfqpoint{10.306274in}{1.726935in}}{\pgfqpoint{10.304270in}{1.731773in}}{\pgfqpoint{10.300704in}{1.735339in}}%
\pgfpathcurveto{\pgfqpoint{10.297138in}{1.738906in}}{\pgfqpoint{10.292300in}{1.740909in}}{\pgfqpoint{10.287256in}{1.740909in}}%
\pgfpathcurveto{\pgfqpoint{10.282212in}{1.740909in}}{\pgfqpoint{10.277375in}{1.738906in}}{\pgfqpoint{10.273808in}{1.735339in}}%
\pgfpathcurveto{\pgfqpoint{10.270242in}{1.731773in}}{\pgfqpoint{10.268238in}{1.726935in}}{\pgfqpoint{10.268238in}{1.721891in}}%
\pgfpathcurveto{\pgfqpoint{10.268238in}{1.716848in}}{\pgfqpoint{10.270242in}{1.712010in}}{\pgfqpoint{10.273808in}{1.708443in}}%
\pgfpathcurveto{\pgfqpoint{10.277375in}{1.704877in}}{\pgfqpoint{10.282212in}{1.702873in}}{\pgfqpoint{10.287256in}{1.702873in}}%
\pgfpathclose%
\pgfusepath{fill}%
\end{pgfscope}%
\begin{pgfscope}%
\pgfpathrectangle{\pgfqpoint{6.572727in}{0.474100in}}{\pgfqpoint{4.227273in}{3.318700in}}%
\pgfusepath{clip}%
\pgfsetbuttcap%
\pgfsetroundjoin%
\definecolor{currentfill}{rgb}{0.127568,0.566949,0.550556}%
\pgfsetfillcolor{currentfill}%
\pgfsetfillopacity{0.700000}%
\pgfsetlinewidth{0.000000pt}%
\definecolor{currentstroke}{rgb}{0.000000,0.000000,0.000000}%
\pgfsetstrokecolor{currentstroke}%
\pgfsetstrokeopacity{0.700000}%
\pgfsetdash{}{0pt}%
\pgfpathmoveto{\pgfqpoint{7.906750in}{2.302888in}}%
\pgfpathcurveto{\pgfqpoint{7.911794in}{2.302888in}}{\pgfqpoint{7.916632in}{2.304892in}}{\pgfqpoint{7.920198in}{2.308458in}}%
\pgfpathcurveto{\pgfqpoint{7.923764in}{2.312024in}}{\pgfqpoint{7.925768in}{2.316862in}}{\pgfqpoint{7.925768in}{2.321906in}}%
\pgfpathcurveto{\pgfqpoint{7.925768in}{2.326949in}}{\pgfqpoint{7.923764in}{2.331787in}}{\pgfqpoint{7.920198in}{2.335354in}}%
\pgfpathcurveto{\pgfqpoint{7.916632in}{2.338920in}}{\pgfqpoint{7.911794in}{2.340924in}}{\pgfqpoint{7.906750in}{2.340924in}}%
\pgfpathcurveto{\pgfqpoint{7.901706in}{2.340924in}}{\pgfqpoint{7.896869in}{2.338920in}}{\pgfqpoint{7.893302in}{2.335354in}}%
\pgfpathcurveto{\pgfqpoint{7.889736in}{2.331787in}}{\pgfqpoint{7.887732in}{2.326949in}}{\pgfqpoint{7.887732in}{2.321906in}}%
\pgfpathcurveto{\pgfqpoint{7.887732in}{2.316862in}}{\pgfqpoint{7.889736in}{2.312024in}}{\pgfqpoint{7.893302in}{2.308458in}}%
\pgfpathcurveto{\pgfqpoint{7.896869in}{2.304892in}}{\pgfqpoint{7.901706in}{2.302888in}}{\pgfqpoint{7.906750in}{2.302888in}}%
\pgfpathclose%
\pgfusepath{fill}%
\end{pgfscope}%
\begin{pgfscope}%
\pgfpathrectangle{\pgfqpoint{6.572727in}{0.474100in}}{\pgfqpoint{4.227273in}{3.318700in}}%
\pgfusepath{clip}%
\pgfsetbuttcap%
\pgfsetroundjoin%
\definecolor{currentfill}{rgb}{0.127568,0.566949,0.550556}%
\pgfsetfillcolor{currentfill}%
\pgfsetfillopacity{0.700000}%
\pgfsetlinewidth{0.000000pt}%
\definecolor{currentstroke}{rgb}{0.000000,0.000000,0.000000}%
\pgfsetstrokecolor{currentstroke}%
\pgfsetstrokeopacity{0.700000}%
\pgfsetdash{}{0pt}%
\pgfpathmoveto{\pgfqpoint{7.907470in}{2.898625in}}%
\pgfpathcurveto{\pgfqpoint{7.912514in}{2.898625in}}{\pgfqpoint{7.917352in}{2.900629in}}{\pgfqpoint{7.920918in}{2.904196in}}%
\pgfpathcurveto{\pgfqpoint{7.924485in}{2.907762in}}{\pgfqpoint{7.926489in}{2.912600in}}{\pgfqpoint{7.926489in}{2.917644in}}%
\pgfpathcurveto{\pgfqpoint{7.926489in}{2.922687in}}{\pgfqpoint{7.924485in}{2.927525in}}{\pgfqpoint{7.920918in}{2.931091in}}%
\pgfpathcurveto{\pgfqpoint{7.917352in}{2.934658in}}{\pgfqpoint{7.912514in}{2.936662in}}{\pgfqpoint{7.907470in}{2.936662in}}%
\pgfpathcurveto{\pgfqpoint{7.902427in}{2.936662in}}{\pgfqpoint{7.897589in}{2.934658in}}{\pgfqpoint{7.894023in}{2.931091in}}%
\pgfpathcurveto{\pgfqpoint{7.890456in}{2.927525in}}{\pgfqpoint{7.888452in}{2.922687in}}{\pgfqpoint{7.888452in}{2.917644in}}%
\pgfpathcurveto{\pgfqpoint{7.888452in}{2.912600in}}{\pgfqpoint{7.890456in}{2.907762in}}{\pgfqpoint{7.894023in}{2.904196in}}%
\pgfpathcurveto{\pgfqpoint{7.897589in}{2.900629in}}{\pgfqpoint{7.902427in}{2.898625in}}{\pgfqpoint{7.907470in}{2.898625in}}%
\pgfpathclose%
\pgfusepath{fill}%
\end{pgfscope}%
\begin{pgfscope}%
\pgfpathrectangle{\pgfqpoint{6.572727in}{0.474100in}}{\pgfqpoint{4.227273in}{3.318700in}}%
\pgfusepath{clip}%
\pgfsetbuttcap%
\pgfsetroundjoin%
\definecolor{currentfill}{rgb}{0.127568,0.566949,0.550556}%
\pgfsetfillcolor{currentfill}%
\pgfsetfillopacity{0.700000}%
\pgfsetlinewidth{0.000000pt}%
\definecolor{currentstroke}{rgb}{0.000000,0.000000,0.000000}%
\pgfsetstrokecolor{currentstroke}%
\pgfsetstrokeopacity{0.700000}%
\pgfsetdash{}{0pt}%
\pgfpathmoveto{\pgfqpoint{7.630284in}{1.591568in}}%
\pgfpathcurveto{\pgfqpoint{7.635328in}{1.591568in}}{\pgfqpoint{7.640166in}{1.593572in}}{\pgfqpoint{7.643732in}{1.597138in}}%
\pgfpathcurveto{\pgfqpoint{7.647298in}{1.600704in}}{\pgfqpoint{7.649302in}{1.605542in}}{\pgfqpoint{7.649302in}{1.610586in}}%
\pgfpathcurveto{\pgfqpoint{7.649302in}{1.615629in}}{\pgfqpoint{7.647298in}{1.620467in}}{\pgfqpoint{7.643732in}{1.624034in}}%
\pgfpathcurveto{\pgfqpoint{7.640166in}{1.627600in}}{\pgfqpoint{7.635328in}{1.629604in}}{\pgfqpoint{7.630284in}{1.629604in}}%
\pgfpathcurveto{\pgfqpoint{7.625240in}{1.629604in}}{\pgfqpoint{7.620403in}{1.627600in}}{\pgfqpoint{7.616836in}{1.624034in}}%
\pgfpathcurveto{\pgfqpoint{7.613270in}{1.620467in}}{\pgfqpoint{7.611266in}{1.615629in}}{\pgfqpoint{7.611266in}{1.610586in}}%
\pgfpathcurveto{\pgfqpoint{7.611266in}{1.605542in}}{\pgfqpoint{7.613270in}{1.600704in}}{\pgfqpoint{7.616836in}{1.597138in}}%
\pgfpathcurveto{\pgfqpoint{7.620403in}{1.593572in}}{\pgfqpoint{7.625240in}{1.591568in}}{\pgfqpoint{7.630284in}{1.591568in}}%
\pgfpathclose%
\pgfusepath{fill}%
\end{pgfscope}%
\begin{pgfscope}%
\pgfpathrectangle{\pgfqpoint{6.572727in}{0.474100in}}{\pgfqpoint{4.227273in}{3.318700in}}%
\pgfusepath{clip}%
\pgfsetbuttcap%
\pgfsetroundjoin%
\definecolor{currentfill}{rgb}{0.127568,0.566949,0.550556}%
\pgfsetfillcolor{currentfill}%
\pgfsetfillopacity{0.700000}%
\pgfsetlinewidth{0.000000pt}%
\definecolor{currentstroke}{rgb}{0.000000,0.000000,0.000000}%
\pgfsetstrokecolor{currentstroke}%
\pgfsetstrokeopacity{0.700000}%
\pgfsetdash{}{0pt}%
\pgfpathmoveto{\pgfqpoint{8.717370in}{2.308608in}}%
\pgfpathcurveto{\pgfqpoint{8.722414in}{2.308608in}}{\pgfqpoint{8.727252in}{2.310611in}}{\pgfqpoint{8.730818in}{2.314178in}}%
\pgfpathcurveto{\pgfqpoint{8.734384in}{2.317744in}}{\pgfqpoint{8.736388in}{2.322582in}}{\pgfqpoint{8.736388in}{2.327626in}}%
\pgfpathcurveto{\pgfqpoint{8.736388in}{2.332669in}}{\pgfqpoint{8.734384in}{2.337507in}}{\pgfqpoint{8.730818in}{2.341074in}}%
\pgfpathcurveto{\pgfqpoint{8.727252in}{2.344640in}}{\pgfqpoint{8.722414in}{2.346644in}}{\pgfqpoint{8.717370in}{2.346644in}}%
\pgfpathcurveto{\pgfqpoint{8.712326in}{2.346644in}}{\pgfqpoint{8.707489in}{2.344640in}}{\pgfqpoint{8.703922in}{2.341074in}}%
\pgfpathcurveto{\pgfqpoint{8.700356in}{2.337507in}}{\pgfqpoint{8.698352in}{2.332669in}}{\pgfqpoint{8.698352in}{2.327626in}}%
\pgfpathcurveto{\pgfqpoint{8.698352in}{2.322582in}}{\pgfqpoint{8.700356in}{2.317744in}}{\pgfqpoint{8.703922in}{2.314178in}}%
\pgfpathcurveto{\pgfqpoint{8.707489in}{2.310611in}}{\pgfqpoint{8.712326in}{2.308608in}}{\pgfqpoint{8.717370in}{2.308608in}}%
\pgfpathclose%
\pgfusepath{fill}%
\end{pgfscope}%
\begin{pgfscope}%
\pgfpathrectangle{\pgfqpoint{6.572727in}{0.474100in}}{\pgfqpoint{4.227273in}{3.318700in}}%
\pgfusepath{clip}%
\pgfsetbuttcap%
\pgfsetroundjoin%
\definecolor{currentfill}{rgb}{0.127568,0.566949,0.550556}%
\pgfsetfillcolor{currentfill}%
\pgfsetfillopacity{0.700000}%
\pgfsetlinewidth{0.000000pt}%
\definecolor{currentstroke}{rgb}{0.000000,0.000000,0.000000}%
\pgfsetstrokecolor{currentstroke}%
\pgfsetstrokeopacity{0.700000}%
\pgfsetdash{}{0pt}%
\pgfpathmoveto{\pgfqpoint{8.369347in}{3.491679in}}%
\pgfpathcurveto{\pgfqpoint{8.374390in}{3.491679in}}{\pgfqpoint{8.379228in}{3.493683in}}{\pgfqpoint{8.382795in}{3.497249in}}%
\pgfpathcurveto{\pgfqpoint{8.386361in}{3.500816in}}{\pgfqpoint{8.388365in}{3.505653in}}{\pgfqpoint{8.388365in}{3.510697in}}%
\pgfpathcurveto{\pgfqpoint{8.388365in}{3.515741in}}{\pgfqpoint{8.386361in}{3.520578in}}{\pgfqpoint{8.382795in}{3.524145in}}%
\pgfpathcurveto{\pgfqpoint{8.379228in}{3.527711in}}{\pgfqpoint{8.374390in}{3.529715in}}{\pgfqpoint{8.369347in}{3.529715in}}%
\pgfpathcurveto{\pgfqpoint{8.364303in}{3.529715in}}{\pgfqpoint{8.359465in}{3.527711in}}{\pgfqpoint{8.355899in}{3.524145in}}%
\pgfpathcurveto{\pgfqpoint{8.352332in}{3.520578in}}{\pgfqpoint{8.350329in}{3.515741in}}{\pgfqpoint{8.350329in}{3.510697in}}%
\pgfpathcurveto{\pgfqpoint{8.350329in}{3.505653in}}{\pgfqpoint{8.352332in}{3.500816in}}{\pgfqpoint{8.355899in}{3.497249in}}%
\pgfpathcurveto{\pgfqpoint{8.359465in}{3.493683in}}{\pgfqpoint{8.364303in}{3.491679in}}{\pgfqpoint{8.369347in}{3.491679in}}%
\pgfpathclose%
\pgfusepath{fill}%
\end{pgfscope}%
\begin{pgfscope}%
\pgfpathrectangle{\pgfqpoint{6.572727in}{0.474100in}}{\pgfqpoint{4.227273in}{3.318700in}}%
\pgfusepath{clip}%
\pgfsetbuttcap%
\pgfsetroundjoin%
\definecolor{currentfill}{rgb}{0.127568,0.566949,0.550556}%
\pgfsetfillcolor{currentfill}%
\pgfsetfillopacity{0.700000}%
\pgfsetlinewidth{0.000000pt}%
\definecolor{currentstroke}{rgb}{0.000000,0.000000,0.000000}%
\pgfsetstrokecolor{currentstroke}%
\pgfsetstrokeopacity{0.700000}%
\pgfsetdash{}{0pt}%
\pgfpathmoveto{\pgfqpoint{7.475106in}{2.855443in}}%
\pgfpathcurveto{\pgfqpoint{7.480150in}{2.855443in}}{\pgfqpoint{7.484987in}{2.857447in}}{\pgfqpoint{7.488554in}{2.861013in}}%
\pgfpathcurveto{\pgfqpoint{7.492120in}{2.864580in}}{\pgfqpoint{7.494124in}{2.869417in}}{\pgfqpoint{7.494124in}{2.874461in}}%
\pgfpathcurveto{\pgfqpoint{7.494124in}{2.879505in}}{\pgfqpoint{7.492120in}{2.884343in}}{\pgfqpoint{7.488554in}{2.887909in}}%
\pgfpathcurveto{\pgfqpoint{7.484987in}{2.891475in}}{\pgfqpoint{7.480150in}{2.893479in}}{\pgfqpoint{7.475106in}{2.893479in}}%
\pgfpathcurveto{\pgfqpoint{7.470062in}{2.893479in}}{\pgfqpoint{7.465224in}{2.891475in}}{\pgfqpoint{7.461658in}{2.887909in}}%
\pgfpathcurveto{\pgfqpoint{7.458092in}{2.884343in}}{\pgfqpoint{7.456088in}{2.879505in}}{\pgfqpoint{7.456088in}{2.874461in}}%
\pgfpathcurveto{\pgfqpoint{7.456088in}{2.869417in}}{\pgfqpoint{7.458092in}{2.864580in}}{\pgfqpoint{7.461658in}{2.861013in}}%
\pgfpathcurveto{\pgfqpoint{7.465224in}{2.857447in}}{\pgfqpoint{7.470062in}{2.855443in}}{\pgfqpoint{7.475106in}{2.855443in}}%
\pgfpathclose%
\pgfusepath{fill}%
\end{pgfscope}%
\begin{pgfscope}%
\pgfpathrectangle{\pgfqpoint{6.572727in}{0.474100in}}{\pgfqpoint{4.227273in}{3.318700in}}%
\pgfusepath{clip}%
\pgfsetbuttcap%
\pgfsetroundjoin%
\definecolor{currentfill}{rgb}{0.127568,0.566949,0.550556}%
\pgfsetfillcolor{currentfill}%
\pgfsetfillopacity{0.700000}%
\pgfsetlinewidth{0.000000pt}%
\definecolor{currentstroke}{rgb}{0.000000,0.000000,0.000000}%
\pgfsetstrokecolor{currentstroke}%
\pgfsetstrokeopacity{0.700000}%
\pgfsetdash{}{0pt}%
\pgfpathmoveto{\pgfqpoint{8.010337in}{1.106604in}}%
\pgfpathcurveto{\pgfqpoint{8.015381in}{1.106604in}}{\pgfqpoint{8.020218in}{1.108608in}}{\pgfqpoint{8.023785in}{1.112175in}}%
\pgfpathcurveto{\pgfqpoint{8.027351in}{1.115741in}}{\pgfqpoint{8.029355in}{1.120579in}}{\pgfqpoint{8.029355in}{1.125623in}}%
\pgfpathcurveto{\pgfqpoint{8.029355in}{1.130666in}}{\pgfqpoint{8.027351in}{1.135504in}}{\pgfqpoint{8.023785in}{1.139070in}}%
\pgfpathcurveto{\pgfqpoint{8.020218in}{1.142637in}}{\pgfqpoint{8.015381in}{1.144641in}}{\pgfqpoint{8.010337in}{1.144641in}}%
\pgfpathcurveto{\pgfqpoint{8.005293in}{1.144641in}}{\pgfqpoint{8.000456in}{1.142637in}}{\pgfqpoint{7.996889in}{1.139070in}}%
\pgfpathcurveto{\pgfqpoint{7.993323in}{1.135504in}}{\pgfqpoint{7.991319in}{1.130666in}}{\pgfqpoint{7.991319in}{1.125623in}}%
\pgfpathcurveto{\pgfqpoint{7.991319in}{1.120579in}}{\pgfqpoint{7.993323in}{1.115741in}}{\pgfqpoint{7.996889in}{1.112175in}}%
\pgfpathcurveto{\pgfqpoint{8.000456in}{1.108608in}}{\pgfqpoint{8.005293in}{1.106604in}}{\pgfqpoint{8.010337in}{1.106604in}}%
\pgfpathclose%
\pgfusepath{fill}%
\end{pgfscope}%
\begin{pgfscope}%
\pgfpathrectangle{\pgfqpoint{6.572727in}{0.474100in}}{\pgfqpoint{4.227273in}{3.318700in}}%
\pgfusepath{clip}%
\pgfsetbuttcap%
\pgfsetroundjoin%
\definecolor{currentfill}{rgb}{0.993248,0.906157,0.143936}%
\pgfsetfillcolor{currentfill}%
\pgfsetfillopacity{0.700000}%
\pgfsetlinewidth{0.000000pt}%
\definecolor{currentstroke}{rgb}{0.000000,0.000000,0.000000}%
\pgfsetstrokecolor{currentstroke}%
\pgfsetstrokeopacity{0.700000}%
\pgfsetdash{}{0pt}%
\pgfpathmoveto{\pgfqpoint{9.609688in}{1.669131in}}%
\pgfpathcurveto{\pgfqpoint{9.614731in}{1.669131in}}{\pgfqpoint{9.619569in}{1.671135in}}{\pgfqpoint{9.623135in}{1.674701in}}%
\pgfpathcurveto{\pgfqpoint{9.626702in}{1.678268in}}{\pgfqpoint{9.628706in}{1.683105in}}{\pgfqpoint{9.628706in}{1.688149in}}%
\pgfpathcurveto{\pgfqpoint{9.628706in}{1.693193in}}{\pgfqpoint{9.626702in}{1.698030in}}{\pgfqpoint{9.623135in}{1.701597in}}%
\pgfpathcurveto{\pgfqpoint{9.619569in}{1.705163in}}{\pgfqpoint{9.614731in}{1.707167in}}{\pgfqpoint{9.609688in}{1.707167in}}%
\pgfpathcurveto{\pgfqpoint{9.604644in}{1.707167in}}{\pgfqpoint{9.599806in}{1.705163in}}{\pgfqpoint{9.596240in}{1.701597in}}%
\pgfpathcurveto{\pgfqpoint{9.592673in}{1.698030in}}{\pgfqpoint{9.590669in}{1.693193in}}{\pgfqpoint{9.590669in}{1.688149in}}%
\pgfpathcurveto{\pgfqpoint{9.590669in}{1.683105in}}{\pgfqpoint{9.592673in}{1.678268in}}{\pgfqpoint{9.596240in}{1.674701in}}%
\pgfpathcurveto{\pgfqpoint{9.599806in}{1.671135in}}{\pgfqpoint{9.604644in}{1.669131in}}{\pgfqpoint{9.609688in}{1.669131in}}%
\pgfpathclose%
\pgfusepath{fill}%
\end{pgfscope}%
\begin{pgfscope}%
\pgfpathrectangle{\pgfqpoint{6.572727in}{0.474100in}}{\pgfqpoint{4.227273in}{3.318700in}}%
\pgfusepath{clip}%
\pgfsetbuttcap%
\pgfsetroundjoin%
\definecolor{currentfill}{rgb}{0.993248,0.906157,0.143936}%
\pgfsetfillcolor{currentfill}%
\pgfsetfillopacity{0.700000}%
\pgfsetlinewidth{0.000000pt}%
\definecolor{currentstroke}{rgb}{0.000000,0.000000,0.000000}%
\pgfsetstrokecolor{currentstroke}%
\pgfsetstrokeopacity{0.700000}%
\pgfsetdash{}{0pt}%
\pgfpathmoveto{\pgfqpoint{9.953565in}{2.160806in}}%
\pgfpathcurveto{\pgfqpoint{9.958609in}{2.160806in}}{\pgfqpoint{9.963446in}{2.162810in}}{\pgfqpoint{9.967013in}{2.166376in}}%
\pgfpathcurveto{\pgfqpoint{9.970579in}{2.169943in}}{\pgfqpoint{9.972583in}{2.174781in}}{\pgfqpoint{9.972583in}{2.179824in}}%
\pgfpathcurveto{\pgfqpoint{9.972583in}{2.184868in}}{\pgfqpoint{9.970579in}{2.189706in}}{\pgfqpoint{9.967013in}{2.193272in}}%
\pgfpathcurveto{\pgfqpoint{9.963446in}{2.196838in}}{\pgfqpoint{9.958609in}{2.198842in}}{\pgfqpoint{9.953565in}{2.198842in}}%
\pgfpathcurveto{\pgfqpoint{9.948521in}{2.198842in}}{\pgfqpoint{9.943683in}{2.196838in}}{\pgfqpoint{9.940117in}{2.193272in}}%
\pgfpathcurveto{\pgfqpoint{9.936551in}{2.189706in}}{\pgfqpoint{9.934547in}{2.184868in}}{\pgfqpoint{9.934547in}{2.179824in}}%
\pgfpathcurveto{\pgfqpoint{9.934547in}{2.174781in}}{\pgfqpoint{9.936551in}{2.169943in}}{\pgfqpoint{9.940117in}{2.166376in}}%
\pgfpathcurveto{\pgfqpoint{9.943683in}{2.162810in}}{\pgfqpoint{9.948521in}{2.160806in}}{\pgfqpoint{9.953565in}{2.160806in}}%
\pgfpathclose%
\pgfusepath{fill}%
\end{pgfscope}%
\begin{pgfscope}%
\pgfpathrectangle{\pgfqpoint{6.572727in}{0.474100in}}{\pgfqpoint{4.227273in}{3.318700in}}%
\pgfusepath{clip}%
\pgfsetbuttcap%
\pgfsetroundjoin%
\definecolor{currentfill}{rgb}{0.127568,0.566949,0.550556}%
\pgfsetfillcolor{currentfill}%
\pgfsetfillopacity{0.700000}%
\pgfsetlinewidth{0.000000pt}%
\definecolor{currentstroke}{rgb}{0.000000,0.000000,0.000000}%
\pgfsetstrokecolor{currentstroke}%
\pgfsetstrokeopacity{0.700000}%
\pgfsetdash{}{0pt}%
\pgfpathmoveto{\pgfqpoint{7.889949in}{1.695751in}}%
\pgfpathcurveto{\pgfqpoint{7.894993in}{1.695751in}}{\pgfqpoint{7.899831in}{1.697755in}}{\pgfqpoint{7.903397in}{1.701322in}}%
\pgfpathcurveto{\pgfqpoint{7.906963in}{1.704888in}}{\pgfqpoint{7.908967in}{1.709726in}}{\pgfqpoint{7.908967in}{1.714769in}}%
\pgfpathcurveto{\pgfqpoint{7.908967in}{1.719813in}}{\pgfqpoint{7.906963in}{1.724651in}}{\pgfqpoint{7.903397in}{1.728217in}}%
\pgfpathcurveto{\pgfqpoint{7.899831in}{1.731784in}}{\pgfqpoint{7.894993in}{1.733788in}}{\pgfqpoint{7.889949in}{1.733788in}}%
\pgfpathcurveto{\pgfqpoint{7.884905in}{1.733788in}}{\pgfqpoint{7.880068in}{1.731784in}}{\pgfqpoint{7.876501in}{1.728217in}}%
\pgfpathcurveto{\pgfqpoint{7.872935in}{1.724651in}}{\pgfqpoint{7.870931in}{1.719813in}}{\pgfqpoint{7.870931in}{1.714769in}}%
\pgfpathcurveto{\pgfqpoint{7.870931in}{1.709726in}}{\pgfqpoint{7.872935in}{1.704888in}}{\pgfqpoint{7.876501in}{1.701322in}}%
\pgfpathcurveto{\pgfqpoint{7.880068in}{1.697755in}}{\pgfqpoint{7.884905in}{1.695751in}}{\pgfqpoint{7.889949in}{1.695751in}}%
\pgfpathclose%
\pgfusepath{fill}%
\end{pgfscope}%
\begin{pgfscope}%
\pgfpathrectangle{\pgfqpoint{6.572727in}{0.474100in}}{\pgfqpoint{4.227273in}{3.318700in}}%
\pgfusepath{clip}%
\pgfsetbuttcap%
\pgfsetroundjoin%
\definecolor{currentfill}{rgb}{0.127568,0.566949,0.550556}%
\pgfsetfillcolor{currentfill}%
\pgfsetfillopacity{0.700000}%
\pgfsetlinewidth{0.000000pt}%
\definecolor{currentstroke}{rgb}{0.000000,0.000000,0.000000}%
\pgfsetstrokecolor{currentstroke}%
\pgfsetstrokeopacity{0.700000}%
\pgfsetdash{}{0pt}%
\pgfpathmoveto{\pgfqpoint{7.507862in}{1.153650in}}%
\pgfpathcurveto{\pgfqpoint{7.512906in}{1.153650in}}{\pgfqpoint{7.517744in}{1.155654in}}{\pgfqpoint{7.521310in}{1.159220in}}%
\pgfpathcurveto{\pgfqpoint{7.524877in}{1.162787in}}{\pgfqpoint{7.526880in}{1.167625in}}{\pgfqpoint{7.526880in}{1.172668in}}%
\pgfpathcurveto{\pgfqpoint{7.526880in}{1.177712in}}{\pgfqpoint{7.524877in}{1.182550in}}{\pgfqpoint{7.521310in}{1.186116in}}%
\pgfpathcurveto{\pgfqpoint{7.517744in}{1.189683in}}{\pgfqpoint{7.512906in}{1.191686in}}{\pgfqpoint{7.507862in}{1.191686in}}%
\pgfpathcurveto{\pgfqpoint{7.502819in}{1.191686in}}{\pgfqpoint{7.497981in}{1.189683in}}{\pgfqpoint{7.494414in}{1.186116in}}%
\pgfpathcurveto{\pgfqpoint{7.490848in}{1.182550in}}{\pgfqpoint{7.488844in}{1.177712in}}{\pgfqpoint{7.488844in}{1.172668in}}%
\pgfpathcurveto{\pgfqpoint{7.488844in}{1.167625in}}{\pgfqpoint{7.490848in}{1.162787in}}{\pgfqpoint{7.494414in}{1.159220in}}%
\pgfpathcurveto{\pgfqpoint{7.497981in}{1.155654in}}{\pgfqpoint{7.502819in}{1.153650in}}{\pgfqpoint{7.507862in}{1.153650in}}%
\pgfpathclose%
\pgfusepath{fill}%
\end{pgfscope}%
\begin{pgfscope}%
\pgfpathrectangle{\pgfqpoint{6.572727in}{0.474100in}}{\pgfqpoint{4.227273in}{3.318700in}}%
\pgfusepath{clip}%
\pgfsetbuttcap%
\pgfsetroundjoin%
\definecolor{currentfill}{rgb}{0.127568,0.566949,0.550556}%
\pgfsetfillcolor{currentfill}%
\pgfsetfillopacity{0.700000}%
\pgfsetlinewidth{0.000000pt}%
\definecolor{currentstroke}{rgb}{0.000000,0.000000,0.000000}%
\pgfsetstrokecolor{currentstroke}%
\pgfsetstrokeopacity{0.700000}%
\pgfsetdash{}{0pt}%
\pgfpathmoveto{\pgfqpoint{7.851340in}{3.167548in}}%
\pgfpathcurveto{\pgfqpoint{7.856384in}{3.167548in}}{\pgfqpoint{7.861222in}{3.169552in}}{\pgfqpoint{7.864788in}{3.173118in}}%
\pgfpathcurveto{\pgfqpoint{7.868354in}{3.176685in}}{\pgfqpoint{7.870358in}{3.181523in}}{\pgfqpoint{7.870358in}{3.186566in}}%
\pgfpathcurveto{\pgfqpoint{7.870358in}{3.191610in}}{\pgfqpoint{7.868354in}{3.196448in}}{\pgfqpoint{7.864788in}{3.200014in}}%
\pgfpathcurveto{\pgfqpoint{7.861222in}{3.203580in}}{\pgfqpoint{7.856384in}{3.205584in}}{\pgfqpoint{7.851340in}{3.205584in}}%
\pgfpathcurveto{\pgfqpoint{7.846296in}{3.205584in}}{\pgfqpoint{7.841459in}{3.203580in}}{\pgfqpoint{7.837892in}{3.200014in}}%
\pgfpathcurveto{\pgfqpoint{7.834326in}{3.196448in}}{\pgfqpoint{7.832322in}{3.191610in}}{\pgfqpoint{7.832322in}{3.186566in}}%
\pgfpathcurveto{\pgfqpoint{7.832322in}{3.181523in}}{\pgfqpoint{7.834326in}{3.176685in}}{\pgfqpoint{7.837892in}{3.173118in}}%
\pgfpathcurveto{\pgfqpoint{7.841459in}{3.169552in}}{\pgfqpoint{7.846296in}{3.167548in}}{\pgfqpoint{7.851340in}{3.167548in}}%
\pgfpathclose%
\pgfusepath{fill}%
\end{pgfscope}%
\begin{pgfscope}%
\pgfpathrectangle{\pgfqpoint{6.572727in}{0.474100in}}{\pgfqpoint{4.227273in}{3.318700in}}%
\pgfusepath{clip}%
\pgfsetbuttcap%
\pgfsetroundjoin%
\definecolor{currentfill}{rgb}{0.127568,0.566949,0.550556}%
\pgfsetfillcolor{currentfill}%
\pgfsetfillopacity{0.700000}%
\pgfsetlinewidth{0.000000pt}%
\definecolor{currentstroke}{rgb}{0.000000,0.000000,0.000000}%
\pgfsetstrokecolor{currentstroke}%
\pgfsetstrokeopacity{0.700000}%
\pgfsetdash{}{0pt}%
\pgfpathmoveto{\pgfqpoint{7.773228in}{1.043123in}}%
\pgfpathcurveto{\pgfqpoint{7.778272in}{1.043123in}}{\pgfqpoint{7.783110in}{1.045127in}}{\pgfqpoint{7.786676in}{1.048694in}}%
\pgfpathcurveto{\pgfqpoint{7.790243in}{1.052260in}}{\pgfqpoint{7.792246in}{1.057098in}}{\pgfqpoint{7.792246in}{1.062142in}}%
\pgfpathcurveto{\pgfqpoint{7.792246in}{1.067185in}}{\pgfqpoint{7.790243in}{1.072023in}}{\pgfqpoint{7.786676in}{1.075590in}}%
\pgfpathcurveto{\pgfqpoint{7.783110in}{1.079156in}}{\pgfqpoint{7.778272in}{1.081160in}}{\pgfqpoint{7.773228in}{1.081160in}}%
\pgfpathcurveto{\pgfqpoint{7.768185in}{1.081160in}}{\pgfqpoint{7.763347in}{1.079156in}}{\pgfqpoint{7.759780in}{1.075590in}}%
\pgfpathcurveto{\pgfqpoint{7.756214in}{1.072023in}}{\pgfqpoint{7.754210in}{1.067185in}}{\pgfqpoint{7.754210in}{1.062142in}}%
\pgfpathcurveto{\pgfqpoint{7.754210in}{1.057098in}}{\pgfqpoint{7.756214in}{1.052260in}}{\pgfqpoint{7.759780in}{1.048694in}}%
\pgfpathcurveto{\pgfqpoint{7.763347in}{1.045127in}}{\pgfqpoint{7.768185in}{1.043123in}}{\pgfqpoint{7.773228in}{1.043123in}}%
\pgfpathclose%
\pgfusepath{fill}%
\end{pgfscope}%
\begin{pgfscope}%
\pgfpathrectangle{\pgfqpoint{6.572727in}{0.474100in}}{\pgfqpoint{4.227273in}{3.318700in}}%
\pgfusepath{clip}%
\pgfsetbuttcap%
\pgfsetroundjoin%
\definecolor{currentfill}{rgb}{0.127568,0.566949,0.550556}%
\pgfsetfillcolor{currentfill}%
\pgfsetfillopacity{0.700000}%
\pgfsetlinewidth{0.000000pt}%
\definecolor{currentstroke}{rgb}{0.000000,0.000000,0.000000}%
\pgfsetstrokecolor{currentstroke}%
\pgfsetstrokeopacity{0.700000}%
\pgfsetdash{}{0pt}%
\pgfpathmoveto{\pgfqpoint{7.470716in}{1.039507in}}%
\pgfpathcurveto{\pgfqpoint{7.475760in}{1.039507in}}{\pgfqpoint{7.480597in}{1.041511in}}{\pgfqpoint{7.484164in}{1.045078in}}%
\pgfpathcurveto{\pgfqpoint{7.487730in}{1.048644in}}{\pgfqpoint{7.489734in}{1.053482in}}{\pgfqpoint{7.489734in}{1.058525in}}%
\pgfpathcurveto{\pgfqpoint{7.489734in}{1.063569in}}{\pgfqpoint{7.487730in}{1.068407in}}{\pgfqpoint{7.484164in}{1.071973in}}%
\pgfpathcurveto{\pgfqpoint{7.480597in}{1.075540in}}{\pgfqpoint{7.475760in}{1.077544in}}{\pgfqpoint{7.470716in}{1.077544in}}%
\pgfpathcurveto{\pgfqpoint{7.465672in}{1.077544in}}{\pgfqpoint{7.460835in}{1.075540in}}{\pgfqpoint{7.457268in}{1.071973in}}%
\pgfpathcurveto{\pgfqpoint{7.453702in}{1.068407in}}{\pgfqpoint{7.451698in}{1.063569in}}{\pgfqpoint{7.451698in}{1.058525in}}%
\pgfpathcurveto{\pgfqpoint{7.451698in}{1.053482in}}{\pgfqpoint{7.453702in}{1.048644in}}{\pgfqpoint{7.457268in}{1.045078in}}%
\pgfpathcurveto{\pgfqpoint{7.460835in}{1.041511in}}{\pgfqpoint{7.465672in}{1.039507in}}{\pgfqpoint{7.470716in}{1.039507in}}%
\pgfpathclose%
\pgfusepath{fill}%
\end{pgfscope}%
\begin{pgfscope}%
\pgfpathrectangle{\pgfqpoint{6.572727in}{0.474100in}}{\pgfqpoint{4.227273in}{3.318700in}}%
\pgfusepath{clip}%
\pgfsetbuttcap%
\pgfsetroundjoin%
\definecolor{currentfill}{rgb}{0.127568,0.566949,0.550556}%
\pgfsetfillcolor{currentfill}%
\pgfsetfillopacity{0.700000}%
\pgfsetlinewidth{0.000000pt}%
\definecolor{currentstroke}{rgb}{0.000000,0.000000,0.000000}%
\pgfsetstrokecolor{currentstroke}%
\pgfsetstrokeopacity{0.700000}%
\pgfsetdash{}{0pt}%
\pgfpathmoveto{\pgfqpoint{7.406007in}{1.409997in}}%
\pgfpathcurveto{\pgfqpoint{7.411050in}{1.409997in}}{\pgfqpoint{7.415888in}{1.412001in}}{\pgfqpoint{7.419455in}{1.415567in}}%
\pgfpathcurveto{\pgfqpoint{7.423021in}{1.419134in}}{\pgfqpoint{7.425025in}{1.423972in}}{\pgfqpoint{7.425025in}{1.429015in}}%
\pgfpathcurveto{\pgfqpoint{7.425025in}{1.434059in}}{\pgfqpoint{7.423021in}{1.438897in}}{\pgfqpoint{7.419455in}{1.442463in}}%
\pgfpathcurveto{\pgfqpoint{7.415888in}{1.446029in}}{\pgfqpoint{7.411050in}{1.448033in}}{\pgfqpoint{7.406007in}{1.448033in}}%
\pgfpathcurveto{\pgfqpoint{7.400963in}{1.448033in}}{\pgfqpoint{7.396125in}{1.446029in}}{\pgfqpoint{7.392559in}{1.442463in}}%
\pgfpathcurveto{\pgfqpoint{7.388993in}{1.438897in}}{\pgfqpoint{7.386989in}{1.434059in}}{\pgfqpoint{7.386989in}{1.429015in}}%
\pgfpathcurveto{\pgfqpoint{7.386989in}{1.423972in}}{\pgfqpoint{7.388993in}{1.419134in}}{\pgfqpoint{7.392559in}{1.415567in}}%
\pgfpathcurveto{\pgfqpoint{7.396125in}{1.412001in}}{\pgfqpoint{7.400963in}{1.409997in}}{\pgfqpoint{7.406007in}{1.409997in}}%
\pgfpathclose%
\pgfusepath{fill}%
\end{pgfscope}%
\begin{pgfscope}%
\pgfpathrectangle{\pgfqpoint{6.572727in}{0.474100in}}{\pgfqpoint{4.227273in}{3.318700in}}%
\pgfusepath{clip}%
\pgfsetbuttcap%
\pgfsetroundjoin%
\definecolor{currentfill}{rgb}{0.127568,0.566949,0.550556}%
\pgfsetfillcolor{currentfill}%
\pgfsetfillopacity{0.700000}%
\pgfsetlinewidth{0.000000pt}%
\definecolor{currentstroke}{rgb}{0.000000,0.000000,0.000000}%
\pgfsetstrokecolor{currentstroke}%
\pgfsetstrokeopacity{0.700000}%
\pgfsetdash{}{0pt}%
\pgfpathmoveto{\pgfqpoint{7.992728in}{1.125700in}}%
\pgfpathcurveto{\pgfqpoint{7.997772in}{1.125700in}}{\pgfqpoint{8.002610in}{1.127704in}}{\pgfqpoint{8.006176in}{1.131270in}}%
\pgfpathcurveto{\pgfqpoint{8.009743in}{1.134837in}}{\pgfqpoint{8.011746in}{1.139674in}}{\pgfqpoint{8.011746in}{1.144718in}}%
\pgfpathcurveto{\pgfqpoint{8.011746in}{1.149762in}}{\pgfqpoint{8.009743in}{1.154599in}}{\pgfqpoint{8.006176in}{1.158166in}}%
\pgfpathcurveto{\pgfqpoint{8.002610in}{1.161732in}}{\pgfqpoint{7.997772in}{1.163736in}}{\pgfqpoint{7.992728in}{1.163736in}}%
\pgfpathcurveto{\pgfqpoint{7.987685in}{1.163736in}}{\pgfqpoint{7.982847in}{1.161732in}}{\pgfqpoint{7.979280in}{1.158166in}}%
\pgfpathcurveto{\pgfqpoint{7.975714in}{1.154599in}}{\pgfqpoint{7.973710in}{1.149762in}}{\pgfqpoint{7.973710in}{1.144718in}}%
\pgfpathcurveto{\pgfqpoint{7.973710in}{1.139674in}}{\pgfqpoint{7.975714in}{1.134837in}}{\pgfqpoint{7.979280in}{1.131270in}}%
\pgfpathcurveto{\pgfqpoint{7.982847in}{1.127704in}}{\pgfqpoint{7.987685in}{1.125700in}}{\pgfqpoint{7.992728in}{1.125700in}}%
\pgfpathclose%
\pgfusepath{fill}%
\end{pgfscope}%
\begin{pgfscope}%
\pgfpathrectangle{\pgfqpoint{6.572727in}{0.474100in}}{\pgfqpoint{4.227273in}{3.318700in}}%
\pgfusepath{clip}%
\pgfsetbuttcap%
\pgfsetroundjoin%
\definecolor{currentfill}{rgb}{0.127568,0.566949,0.550556}%
\pgfsetfillcolor{currentfill}%
\pgfsetfillopacity{0.700000}%
\pgfsetlinewidth{0.000000pt}%
\definecolor{currentstroke}{rgb}{0.000000,0.000000,0.000000}%
\pgfsetstrokecolor{currentstroke}%
\pgfsetstrokeopacity{0.700000}%
\pgfsetdash{}{0pt}%
\pgfpathmoveto{\pgfqpoint{8.638296in}{1.595718in}}%
\pgfpathcurveto{\pgfqpoint{8.643340in}{1.595718in}}{\pgfqpoint{8.648178in}{1.597722in}}{\pgfqpoint{8.651744in}{1.601288in}}%
\pgfpathcurveto{\pgfqpoint{8.655311in}{1.604855in}}{\pgfqpoint{8.657315in}{1.609692in}}{\pgfqpoint{8.657315in}{1.614736in}}%
\pgfpathcurveto{\pgfqpoint{8.657315in}{1.619780in}}{\pgfqpoint{8.655311in}{1.624618in}}{\pgfqpoint{8.651744in}{1.628184in}}%
\pgfpathcurveto{\pgfqpoint{8.648178in}{1.631750in}}{\pgfqpoint{8.643340in}{1.633754in}}{\pgfqpoint{8.638296in}{1.633754in}}%
\pgfpathcurveto{\pgfqpoint{8.633253in}{1.633754in}}{\pgfqpoint{8.628415in}{1.631750in}}{\pgfqpoint{8.624848in}{1.628184in}}%
\pgfpathcurveto{\pgfqpoint{8.621282in}{1.624618in}}{\pgfqpoint{8.619278in}{1.619780in}}{\pgfqpoint{8.619278in}{1.614736in}}%
\pgfpathcurveto{\pgfqpoint{8.619278in}{1.609692in}}{\pgfqpoint{8.621282in}{1.604855in}}{\pgfqpoint{8.624848in}{1.601288in}}%
\pgfpathcurveto{\pgfqpoint{8.628415in}{1.597722in}}{\pgfqpoint{8.633253in}{1.595718in}}{\pgfqpoint{8.638296in}{1.595718in}}%
\pgfpathclose%
\pgfusepath{fill}%
\end{pgfscope}%
\begin{pgfscope}%
\pgfpathrectangle{\pgfqpoint{6.572727in}{0.474100in}}{\pgfqpoint{4.227273in}{3.318700in}}%
\pgfusepath{clip}%
\pgfsetbuttcap%
\pgfsetroundjoin%
\definecolor{currentfill}{rgb}{0.127568,0.566949,0.550556}%
\pgfsetfillcolor{currentfill}%
\pgfsetfillopacity{0.700000}%
\pgfsetlinewidth{0.000000pt}%
\definecolor{currentstroke}{rgb}{0.000000,0.000000,0.000000}%
\pgfsetstrokecolor{currentstroke}%
\pgfsetstrokeopacity{0.700000}%
\pgfsetdash{}{0pt}%
\pgfpathmoveto{\pgfqpoint{7.577322in}{1.453274in}}%
\pgfpathcurveto{\pgfqpoint{7.582366in}{1.453274in}}{\pgfqpoint{7.587204in}{1.455278in}}{\pgfqpoint{7.590770in}{1.458845in}}%
\pgfpathcurveto{\pgfqpoint{7.594337in}{1.462411in}}{\pgfqpoint{7.596341in}{1.467249in}}{\pgfqpoint{7.596341in}{1.472292in}}%
\pgfpathcurveto{\pgfqpoint{7.596341in}{1.477336in}}{\pgfqpoint{7.594337in}{1.482174in}}{\pgfqpoint{7.590770in}{1.485740in}}%
\pgfpathcurveto{\pgfqpoint{7.587204in}{1.489307in}}{\pgfqpoint{7.582366in}{1.491311in}}{\pgfqpoint{7.577322in}{1.491311in}}%
\pgfpathcurveto{\pgfqpoint{7.572279in}{1.491311in}}{\pgfqpoint{7.567441in}{1.489307in}}{\pgfqpoint{7.563875in}{1.485740in}}%
\pgfpathcurveto{\pgfqpoint{7.560308in}{1.482174in}}{\pgfqpoint{7.558304in}{1.477336in}}{\pgfqpoint{7.558304in}{1.472292in}}%
\pgfpathcurveto{\pgfqpoint{7.558304in}{1.467249in}}{\pgfqpoint{7.560308in}{1.462411in}}{\pgfqpoint{7.563875in}{1.458845in}}%
\pgfpathcurveto{\pgfqpoint{7.567441in}{1.455278in}}{\pgfqpoint{7.572279in}{1.453274in}}{\pgfqpoint{7.577322in}{1.453274in}}%
\pgfpathclose%
\pgfusepath{fill}%
\end{pgfscope}%
\begin{pgfscope}%
\pgfpathrectangle{\pgfqpoint{6.572727in}{0.474100in}}{\pgfqpoint{4.227273in}{3.318700in}}%
\pgfusepath{clip}%
\pgfsetbuttcap%
\pgfsetroundjoin%
\definecolor{currentfill}{rgb}{0.127568,0.566949,0.550556}%
\pgfsetfillcolor{currentfill}%
\pgfsetfillopacity{0.700000}%
\pgfsetlinewidth{0.000000pt}%
\definecolor{currentstroke}{rgb}{0.000000,0.000000,0.000000}%
\pgfsetstrokecolor{currentstroke}%
\pgfsetstrokeopacity{0.700000}%
\pgfsetdash{}{0pt}%
\pgfpathmoveto{\pgfqpoint{7.253898in}{1.025726in}}%
\pgfpathcurveto{\pgfqpoint{7.258942in}{1.025726in}}{\pgfqpoint{7.263780in}{1.027730in}}{\pgfqpoint{7.267346in}{1.031296in}}%
\pgfpathcurveto{\pgfqpoint{7.270912in}{1.034862in}}{\pgfqpoint{7.272916in}{1.039700in}}{\pgfqpoint{7.272916in}{1.044744in}}%
\pgfpathcurveto{\pgfqpoint{7.272916in}{1.049787in}}{\pgfqpoint{7.270912in}{1.054625in}}{\pgfqpoint{7.267346in}{1.058192in}}%
\pgfpathcurveto{\pgfqpoint{7.263780in}{1.061758in}}{\pgfqpoint{7.258942in}{1.063762in}}{\pgfqpoint{7.253898in}{1.063762in}}%
\pgfpathcurveto{\pgfqpoint{7.248855in}{1.063762in}}{\pgfqpoint{7.244017in}{1.061758in}}{\pgfqpoint{7.240450in}{1.058192in}}%
\pgfpathcurveto{\pgfqpoint{7.236884in}{1.054625in}}{\pgfqpoint{7.234880in}{1.049787in}}{\pgfqpoint{7.234880in}{1.044744in}}%
\pgfpathcurveto{\pgfqpoint{7.234880in}{1.039700in}}{\pgfqpoint{7.236884in}{1.034862in}}{\pgfqpoint{7.240450in}{1.031296in}}%
\pgfpathcurveto{\pgfqpoint{7.244017in}{1.027730in}}{\pgfqpoint{7.248855in}{1.025726in}}{\pgfqpoint{7.253898in}{1.025726in}}%
\pgfpathclose%
\pgfusepath{fill}%
\end{pgfscope}%
\begin{pgfscope}%
\pgfpathrectangle{\pgfqpoint{6.572727in}{0.474100in}}{\pgfqpoint{4.227273in}{3.318700in}}%
\pgfusepath{clip}%
\pgfsetbuttcap%
\pgfsetroundjoin%
\definecolor{currentfill}{rgb}{0.993248,0.906157,0.143936}%
\pgfsetfillcolor{currentfill}%
\pgfsetfillopacity{0.700000}%
\pgfsetlinewidth{0.000000pt}%
\definecolor{currentstroke}{rgb}{0.000000,0.000000,0.000000}%
\pgfsetstrokecolor{currentstroke}%
\pgfsetstrokeopacity{0.700000}%
\pgfsetdash{}{0pt}%
\pgfpathmoveto{\pgfqpoint{10.270034in}{1.809327in}}%
\pgfpathcurveto{\pgfqpoint{10.275078in}{1.809327in}}{\pgfqpoint{10.279916in}{1.811331in}}{\pgfqpoint{10.283482in}{1.814897in}}%
\pgfpathcurveto{\pgfqpoint{10.287049in}{1.818464in}}{\pgfqpoint{10.289053in}{1.823302in}}{\pgfqpoint{10.289053in}{1.828345in}}%
\pgfpathcurveto{\pgfqpoint{10.289053in}{1.833389in}}{\pgfqpoint{10.287049in}{1.838227in}}{\pgfqpoint{10.283482in}{1.841793in}}%
\pgfpathcurveto{\pgfqpoint{10.279916in}{1.845360in}}{\pgfqpoint{10.275078in}{1.847363in}}{\pgfqpoint{10.270034in}{1.847363in}}%
\pgfpathcurveto{\pgfqpoint{10.264991in}{1.847363in}}{\pgfqpoint{10.260153in}{1.845360in}}{\pgfqpoint{10.256587in}{1.841793in}}%
\pgfpathcurveto{\pgfqpoint{10.253020in}{1.838227in}}{\pgfqpoint{10.251016in}{1.833389in}}{\pgfqpoint{10.251016in}{1.828345in}}%
\pgfpathcurveto{\pgfqpoint{10.251016in}{1.823302in}}{\pgfqpoint{10.253020in}{1.818464in}}{\pgfqpoint{10.256587in}{1.814897in}}%
\pgfpathcurveto{\pgfqpoint{10.260153in}{1.811331in}}{\pgfqpoint{10.264991in}{1.809327in}}{\pgfqpoint{10.270034in}{1.809327in}}%
\pgfpathclose%
\pgfusepath{fill}%
\end{pgfscope}%
\begin{pgfscope}%
\pgfpathrectangle{\pgfqpoint{6.572727in}{0.474100in}}{\pgfqpoint{4.227273in}{3.318700in}}%
\pgfusepath{clip}%
\pgfsetbuttcap%
\pgfsetroundjoin%
\definecolor{currentfill}{rgb}{0.127568,0.566949,0.550556}%
\pgfsetfillcolor{currentfill}%
\pgfsetfillopacity{0.700000}%
\pgfsetlinewidth{0.000000pt}%
\definecolor{currentstroke}{rgb}{0.000000,0.000000,0.000000}%
\pgfsetstrokecolor{currentstroke}%
\pgfsetstrokeopacity{0.700000}%
\pgfsetdash{}{0pt}%
\pgfpathmoveto{\pgfqpoint{7.981775in}{1.663051in}}%
\pgfpathcurveto{\pgfqpoint{7.986819in}{1.663051in}}{\pgfqpoint{7.991656in}{1.665055in}}{\pgfqpoint{7.995223in}{1.668621in}}%
\pgfpathcurveto{\pgfqpoint{7.998789in}{1.672188in}}{\pgfqpoint{8.000793in}{1.677026in}}{\pgfqpoint{8.000793in}{1.682069in}}%
\pgfpathcurveto{\pgfqpoint{8.000793in}{1.687113in}}{\pgfqpoint{7.998789in}{1.691951in}}{\pgfqpoint{7.995223in}{1.695517in}}%
\pgfpathcurveto{\pgfqpoint{7.991656in}{1.699083in}}{\pgfqpoint{7.986819in}{1.701087in}}{\pgfqpoint{7.981775in}{1.701087in}}%
\pgfpathcurveto{\pgfqpoint{7.976731in}{1.701087in}}{\pgfqpoint{7.971894in}{1.699083in}}{\pgfqpoint{7.968327in}{1.695517in}}%
\pgfpathcurveto{\pgfqpoint{7.964761in}{1.691951in}}{\pgfqpoint{7.962757in}{1.687113in}}{\pgfqpoint{7.962757in}{1.682069in}}%
\pgfpathcurveto{\pgfqpoint{7.962757in}{1.677026in}}{\pgfqpoint{7.964761in}{1.672188in}}{\pgfqpoint{7.968327in}{1.668621in}}%
\pgfpathcurveto{\pgfqpoint{7.971894in}{1.665055in}}{\pgfqpoint{7.976731in}{1.663051in}}{\pgfqpoint{7.981775in}{1.663051in}}%
\pgfpathclose%
\pgfusepath{fill}%
\end{pgfscope}%
\begin{pgfscope}%
\pgfpathrectangle{\pgfqpoint{6.572727in}{0.474100in}}{\pgfqpoint{4.227273in}{3.318700in}}%
\pgfusepath{clip}%
\pgfsetbuttcap%
\pgfsetroundjoin%
\definecolor{currentfill}{rgb}{0.993248,0.906157,0.143936}%
\pgfsetfillcolor{currentfill}%
\pgfsetfillopacity{0.700000}%
\pgfsetlinewidth{0.000000pt}%
\definecolor{currentstroke}{rgb}{0.000000,0.000000,0.000000}%
\pgfsetstrokecolor{currentstroke}%
\pgfsetstrokeopacity{0.700000}%
\pgfsetdash{}{0pt}%
\pgfpathmoveto{\pgfqpoint{9.923404in}{1.462924in}}%
\pgfpathcurveto{\pgfqpoint{9.928448in}{1.462924in}}{\pgfqpoint{9.933286in}{1.464928in}}{\pgfqpoint{9.936852in}{1.468494in}}%
\pgfpathcurveto{\pgfqpoint{9.940418in}{1.472061in}}{\pgfqpoint{9.942422in}{1.476899in}}{\pgfqpoint{9.942422in}{1.481942in}}%
\pgfpathcurveto{\pgfqpoint{9.942422in}{1.486986in}}{\pgfqpoint{9.940418in}{1.491824in}}{\pgfqpoint{9.936852in}{1.495390in}}%
\pgfpathcurveto{\pgfqpoint{9.933286in}{1.498957in}}{\pgfqpoint{9.928448in}{1.500960in}}{\pgfqpoint{9.923404in}{1.500960in}}%
\pgfpathcurveto{\pgfqpoint{9.918360in}{1.500960in}}{\pgfqpoint{9.913523in}{1.498957in}}{\pgfqpoint{9.909956in}{1.495390in}}%
\pgfpathcurveto{\pgfqpoint{9.906390in}{1.491824in}}{\pgfqpoint{9.904386in}{1.486986in}}{\pgfqpoint{9.904386in}{1.481942in}}%
\pgfpathcurveto{\pgfqpoint{9.904386in}{1.476899in}}{\pgfqpoint{9.906390in}{1.472061in}}{\pgfqpoint{9.909956in}{1.468494in}}%
\pgfpathcurveto{\pgfqpoint{9.913523in}{1.464928in}}{\pgfqpoint{9.918360in}{1.462924in}}{\pgfqpoint{9.923404in}{1.462924in}}%
\pgfpathclose%
\pgfusepath{fill}%
\end{pgfscope}%
\begin{pgfscope}%
\pgfpathrectangle{\pgfqpoint{6.572727in}{0.474100in}}{\pgfqpoint{4.227273in}{3.318700in}}%
\pgfusepath{clip}%
\pgfsetbuttcap%
\pgfsetroundjoin%
\definecolor{currentfill}{rgb}{0.127568,0.566949,0.550556}%
\pgfsetfillcolor{currentfill}%
\pgfsetfillopacity{0.700000}%
\pgfsetlinewidth{0.000000pt}%
\definecolor{currentstroke}{rgb}{0.000000,0.000000,0.000000}%
\pgfsetstrokecolor{currentstroke}%
\pgfsetstrokeopacity{0.700000}%
\pgfsetdash{}{0pt}%
\pgfpathmoveto{\pgfqpoint{7.723658in}{1.683179in}}%
\pgfpathcurveto{\pgfqpoint{7.728702in}{1.683179in}}{\pgfqpoint{7.733540in}{1.685183in}}{\pgfqpoint{7.737106in}{1.688750in}}%
\pgfpathcurveto{\pgfqpoint{7.740673in}{1.692316in}}{\pgfqpoint{7.742677in}{1.697154in}}{\pgfqpoint{7.742677in}{1.702198in}}%
\pgfpathcurveto{\pgfqpoint{7.742677in}{1.707241in}}{\pgfqpoint{7.740673in}{1.712079in}}{\pgfqpoint{7.737106in}{1.715645in}}%
\pgfpathcurveto{\pgfqpoint{7.733540in}{1.719212in}}{\pgfqpoint{7.728702in}{1.721216in}}{\pgfqpoint{7.723658in}{1.721216in}}%
\pgfpathcurveto{\pgfqpoint{7.718615in}{1.721216in}}{\pgfqpoint{7.713777in}{1.719212in}}{\pgfqpoint{7.710211in}{1.715645in}}%
\pgfpathcurveto{\pgfqpoint{7.706644in}{1.712079in}}{\pgfqpoint{7.704640in}{1.707241in}}{\pgfqpoint{7.704640in}{1.702198in}}%
\pgfpathcurveto{\pgfqpoint{7.704640in}{1.697154in}}{\pgfqpoint{7.706644in}{1.692316in}}{\pgfqpoint{7.710211in}{1.688750in}}%
\pgfpathcurveto{\pgfqpoint{7.713777in}{1.685183in}}{\pgfqpoint{7.718615in}{1.683179in}}{\pgfqpoint{7.723658in}{1.683179in}}%
\pgfpathclose%
\pgfusepath{fill}%
\end{pgfscope}%
\begin{pgfscope}%
\pgfpathrectangle{\pgfqpoint{6.572727in}{0.474100in}}{\pgfqpoint{4.227273in}{3.318700in}}%
\pgfusepath{clip}%
\pgfsetbuttcap%
\pgfsetroundjoin%
\definecolor{currentfill}{rgb}{0.267004,0.004874,0.329415}%
\pgfsetfillcolor{currentfill}%
\pgfsetfillopacity{0.700000}%
\pgfsetlinewidth{0.000000pt}%
\definecolor{currentstroke}{rgb}{0.000000,0.000000,0.000000}%
\pgfsetstrokecolor{currentstroke}%
\pgfsetstrokeopacity{0.700000}%
\pgfsetdash{}{0pt}%
\pgfpathmoveto{\pgfqpoint{10.607851in}{1.489116in}}%
\pgfpathcurveto{\pgfqpoint{10.612895in}{1.489116in}}{\pgfqpoint{10.617733in}{1.491120in}}{\pgfqpoint{10.621299in}{1.494686in}}%
\pgfpathcurveto{\pgfqpoint{10.624866in}{1.498252in}}{\pgfqpoint{10.626869in}{1.503090in}}{\pgfqpoint{10.626869in}{1.508134in}}%
\pgfpathcurveto{\pgfqpoint{10.626869in}{1.513177in}}{\pgfqpoint{10.624866in}{1.518015in}}{\pgfqpoint{10.621299in}{1.521582in}}%
\pgfpathcurveto{\pgfqpoint{10.617733in}{1.525148in}}{\pgfqpoint{10.612895in}{1.527152in}}{\pgfqpoint{10.607851in}{1.527152in}}%
\pgfpathcurveto{\pgfqpoint{10.602808in}{1.527152in}}{\pgfqpoint{10.597970in}{1.525148in}}{\pgfqpoint{10.594403in}{1.521582in}}%
\pgfpathcurveto{\pgfqpoint{10.590837in}{1.518015in}}{\pgfqpoint{10.588833in}{1.513177in}}{\pgfqpoint{10.588833in}{1.508134in}}%
\pgfpathcurveto{\pgfqpoint{10.588833in}{1.503090in}}{\pgfqpoint{10.590837in}{1.498252in}}{\pgfqpoint{10.594403in}{1.494686in}}%
\pgfpathcurveto{\pgfqpoint{10.597970in}{1.491120in}}{\pgfqpoint{10.602808in}{1.489116in}}{\pgfqpoint{10.607851in}{1.489116in}}%
\pgfpathclose%
\pgfusepath{fill}%
\end{pgfscope}%
\begin{pgfscope}%
\pgfpathrectangle{\pgfqpoint{6.572727in}{0.474100in}}{\pgfqpoint{4.227273in}{3.318700in}}%
\pgfusepath{clip}%
\pgfsetbuttcap%
\pgfsetroundjoin%
\definecolor{currentfill}{rgb}{0.993248,0.906157,0.143936}%
\pgfsetfillcolor{currentfill}%
\pgfsetfillopacity{0.700000}%
\pgfsetlinewidth{0.000000pt}%
\definecolor{currentstroke}{rgb}{0.000000,0.000000,0.000000}%
\pgfsetstrokecolor{currentstroke}%
\pgfsetstrokeopacity{0.700000}%
\pgfsetdash{}{0pt}%
\pgfpathmoveto{\pgfqpoint{10.345549in}{1.490650in}}%
\pgfpathcurveto{\pgfqpoint{10.350592in}{1.490650in}}{\pgfqpoint{10.355430in}{1.492654in}}{\pgfqpoint{10.358997in}{1.496220in}}%
\pgfpathcurveto{\pgfqpoint{10.362563in}{1.499786in}}{\pgfqpoint{10.364567in}{1.504624in}}{\pgfqpoint{10.364567in}{1.509668in}}%
\pgfpathcurveto{\pgfqpoint{10.364567in}{1.514711in}}{\pgfqpoint{10.362563in}{1.519549in}}{\pgfqpoint{10.358997in}{1.523116in}}%
\pgfpathcurveto{\pgfqpoint{10.355430in}{1.526682in}}{\pgfqpoint{10.350592in}{1.528686in}}{\pgfqpoint{10.345549in}{1.528686in}}%
\pgfpathcurveto{\pgfqpoint{10.340505in}{1.528686in}}{\pgfqpoint{10.335667in}{1.526682in}}{\pgfqpoint{10.332101in}{1.523116in}}%
\pgfpathcurveto{\pgfqpoint{10.328534in}{1.519549in}}{\pgfqpoint{10.326531in}{1.514711in}}{\pgfqpoint{10.326531in}{1.509668in}}%
\pgfpathcurveto{\pgfqpoint{10.326531in}{1.504624in}}{\pgfqpoint{10.328534in}{1.499786in}}{\pgfqpoint{10.332101in}{1.496220in}}%
\pgfpathcurveto{\pgfqpoint{10.335667in}{1.492654in}}{\pgfqpoint{10.340505in}{1.490650in}}{\pgfqpoint{10.345549in}{1.490650in}}%
\pgfpathclose%
\pgfusepath{fill}%
\end{pgfscope}%
\begin{pgfscope}%
\pgfpathrectangle{\pgfqpoint{6.572727in}{0.474100in}}{\pgfqpoint{4.227273in}{3.318700in}}%
\pgfusepath{clip}%
\pgfsetbuttcap%
\pgfsetroundjoin%
\definecolor{currentfill}{rgb}{0.127568,0.566949,0.550556}%
\pgfsetfillcolor{currentfill}%
\pgfsetfillopacity{0.700000}%
\pgfsetlinewidth{0.000000pt}%
\definecolor{currentstroke}{rgb}{0.000000,0.000000,0.000000}%
\pgfsetstrokecolor{currentstroke}%
\pgfsetstrokeopacity{0.700000}%
\pgfsetdash{}{0pt}%
\pgfpathmoveto{\pgfqpoint{7.657607in}{1.102617in}}%
\pgfpathcurveto{\pgfqpoint{7.662650in}{1.102617in}}{\pgfqpoint{7.667488in}{1.104620in}}{\pgfqpoint{7.671054in}{1.108187in}}%
\pgfpathcurveto{\pgfqpoint{7.674621in}{1.111753in}}{\pgfqpoint{7.676625in}{1.116591in}}{\pgfqpoint{7.676625in}{1.121635in}}%
\pgfpathcurveto{\pgfqpoint{7.676625in}{1.126678in}}{\pgfqpoint{7.674621in}{1.131516in}}{\pgfqpoint{7.671054in}{1.135083in}}%
\pgfpathcurveto{\pgfqpoint{7.667488in}{1.138649in}}{\pgfqpoint{7.662650in}{1.140653in}}{\pgfqpoint{7.657607in}{1.140653in}}%
\pgfpathcurveto{\pgfqpoint{7.652563in}{1.140653in}}{\pgfqpoint{7.647725in}{1.138649in}}{\pgfqpoint{7.644159in}{1.135083in}}%
\pgfpathcurveto{\pgfqpoint{7.640592in}{1.131516in}}{\pgfqpoint{7.638588in}{1.126678in}}{\pgfqpoint{7.638588in}{1.121635in}}%
\pgfpathcurveto{\pgfqpoint{7.638588in}{1.116591in}}{\pgfqpoint{7.640592in}{1.111753in}}{\pgfqpoint{7.644159in}{1.108187in}}%
\pgfpathcurveto{\pgfqpoint{7.647725in}{1.104620in}}{\pgfqpoint{7.652563in}{1.102617in}}{\pgfqpoint{7.657607in}{1.102617in}}%
\pgfpathclose%
\pgfusepath{fill}%
\end{pgfscope}%
\begin{pgfscope}%
\pgfpathrectangle{\pgfqpoint{6.572727in}{0.474100in}}{\pgfqpoint{4.227273in}{3.318700in}}%
\pgfusepath{clip}%
\pgfsetbuttcap%
\pgfsetroundjoin%
\definecolor{currentfill}{rgb}{0.127568,0.566949,0.550556}%
\pgfsetfillcolor{currentfill}%
\pgfsetfillopacity{0.700000}%
\pgfsetlinewidth{0.000000pt}%
\definecolor{currentstroke}{rgb}{0.000000,0.000000,0.000000}%
\pgfsetstrokecolor{currentstroke}%
\pgfsetstrokeopacity{0.700000}%
\pgfsetdash{}{0pt}%
\pgfpathmoveto{\pgfqpoint{7.594925in}{2.933160in}}%
\pgfpathcurveto{\pgfqpoint{7.599968in}{2.933160in}}{\pgfqpoint{7.604806in}{2.935164in}}{\pgfqpoint{7.608373in}{2.938730in}}%
\pgfpathcurveto{\pgfqpoint{7.611939in}{2.942297in}}{\pgfqpoint{7.613943in}{2.947134in}}{\pgfqpoint{7.613943in}{2.952178in}}%
\pgfpathcurveto{\pgfqpoint{7.613943in}{2.957222in}}{\pgfqpoint{7.611939in}{2.962059in}}{\pgfqpoint{7.608373in}{2.965626in}}%
\pgfpathcurveto{\pgfqpoint{7.604806in}{2.969192in}}{\pgfqpoint{7.599968in}{2.971196in}}{\pgfqpoint{7.594925in}{2.971196in}}%
\pgfpathcurveto{\pgfqpoint{7.589881in}{2.971196in}}{\pgfqpoint{7.585043in}{2.969192in}}{\pgfqpoint{7.581477in}{2.965626in}}%
\pgfpathcurveto{\pgfqpoint{7.577911in}{2.962059in}}{\pgfqpoint{7.575907in}{2.957222in}}{\pgfqpoint{7.575907in}{2.952178in}}%
\pgfpathcurveto{\pgfqpoint{7.575907in}{2.947134in}}{\pgfqpoint{7.577911in}{2.942297in}}{\pgfqpoint{7.581477in}{2.938730in}}%
\pgfpathcurveto{\pgfqpoint{7.585043in}{2.935164in}}{\pgfqpoint{7.589881in}{2.933160in}}{\pgfqpoint{7.594925in}{2.933160in}}%
\pgfpathclose%
\pgfusepath{fill}%
\end{pgfscope}%
\begin{pgfscope}%
\pgfpathrectangle{\pgfqpoint{6.572727in}{0.474100in}}{\pgfqpoint{4.227273in}{3.318700in}}%
\pgfusepath{clip}%
\pgfsetbuttcap%
\pgfsetroundjoin%
\definecolor{currentfill}{rgb}{0.127568,0.566949,0.550556}%
\pgfsetfillcolor{currentfill}%
\pgfsetfillopacity{0.700000}%
\pgfsetlinewidth{0.000000pt}%
\definecolor{currentstroke}{rgb}{0.000000,0.000000,0.000000}%
\pgfsetstrokecolor{currentstroke}%
\pgfsetstrokeopacity{0.700000}%
\pgfsetdash{}{0pt}%
\pgfpathmoveto{\pgfqpoint{8.084105in}{1.753381in}}%
\pgfpathcurveto{\pgfqpoint{8.089149in}{1.753381in}}{\pgfqpoint{8.093987in}{1.755385in}}{\pgfqpoint{8.097553in}{1.758952in}}%
\pgfpathcurveto{\pgfqpoint{8.101120in}{1.762518in}}{\pgfqpoint{8.103124in}{1.767356in}}{\pgfqpoint{8.103124in}{1.772399in}}%
\pgfpathcurveto{\pgfqpoint{8.103124in}{1.777443in}}{\pgfqpoint{8.101120in}{1.782281in}}{\pgfqpoint{8.097553in}{1.785847in}}%
\pgfpathcurveto{\pgfqpoint{8.093987in}{1.789414in}}{\pgfqpoint{8.089149in}{1.791418in}}{\pgfqpoint{8.084105in}{1.791418in}}%
\pgfpathcurveto{\pgfqpoint{8.079062in}{1.791418in}}{\pgfqpoint{8.074224in}{1.789414in}}{\pgfqpoint{8.070658in}{1.785847in}}%
\pgfpathcurveto{\pgfqpoint{8.067091in}{1.782281in}}{\pgfqpoint{8.065087in}{1.777443in}}{\pgfqpoint{8.065087in}{1.772399in}}%
\pgfpathcurveto{\pgfqpoint{8.065087in}{1.767356in}}{\pgfqpoint{8.067091in}{1.762518in}}{\pgfqpoint{8.070658in}{1.758952in}}%
\pgfpathcurveto{\pgfqpoint{8.074224in}{1.755385in}}{\pgfqpoint{8.079062in}{1.753381in}}{\pgfqpoint{8.084105in}{1.753381in}}%
\pgfpathclose%
\pgfusepath{fill}%
\end{pgfscope}%
\begin{pgfscope}%
\pgfpathrectangle{\pgfqpoint{6.572727in}{0.474100in}}{\pgfqpoint{4.227273in}{3.318700in}}%
\pgfusepath{clip}%
\pgfsetbuttcap%
\pgfsetroundjoin%
\definecolor{currentfill}{rgb}{0.993248,0.906157,0.143936}%
\pgfsetfillcolor{currentfill}%
\pgfsetfillopacity{0.700000}%
\pgfsetlinewidth{0.000000pt}%
\definecolor{currentstroke}{rgb}{0.000000,0.000000,0.000000}%
\pgfsetstrokecolor{currentstroke}%
\pgfsetstrokeopacity{0.700000}%
\pgfsetdash{}{0pt}%
\pgfpathmoveto{\pgfqpoint{9.382759in}{1.582782in}}%
\pgfpathcurveto{\pgfqpoint{9.387803in}{1.582782in}}{\pgfqpoint{9.392641in}{1.584786in}}{\pgfqpoint{9.396207in}{1.588352in}}%
\pgfpathcurveto{\pgfqpoint{9.399774in}{1.591919in}}{\pgfqpoint{9.401778in}{1.596756in}}{\pgfqpoint{9.401778in}{1.601800in}}%
\pgfpathcurveto{\pgfqpoint{9.401778in}{1.606844in}}{\pgfqpoint{9.399774in}{1.611681in}}{\pgfqpoint{9.396207in}{1.615248in}}%
\pgfpathcurveto{\pgfqpoint{9.392641in}{1.618814in}}{\pgfqpoint{9.387803in}{1.620818in}}{\pgfqpoint{9.382759in}{1.620818in}}%
\pgfpathcurveto{\pgfqpoint{9.377716in}{1.620818in}}{\pgfqpoint{9.372878in}{1.618814in}}{\pgfqpoint{9.369312in}{1.615248in}}%
\pgfpathcurveto{\pgfqpoint{9.365745in}{1.611681in}}{\pgfqpoint{9.363741in}{1.606844in}}{\pgfqpoint{9.363741in}{1.601800in}}%
\pgfpathcurveto{\pgfqpoint{9.363741in}{1.596756in}}{\pgfqpoint{9.365745in}{1.591919in}}{\pgfqpoint{9.369312in}{1.588352in}}%
\pgfpathcurveto{\pgfqpoint{9.372878in}{1.584786in}}{\pgfqpoint{9.377716in}{1.582782in}}{\pgfqpoint{9.382759in}{1.582782in}}%
\pgfpathclose%
\pgfusepath{fill}%
\end{pgfscope}%
\begin{pgfscope}%
\pgfpathrectangle{\pgfqpoint{6.572727in}{0.474100in}}{\pgfqpoint{4.227273in}{3.318700in}}%
\pgfusepath{clip}%
\pgfsetbuttcap%
\pgfsetroundjoin%
\definecolor{currentfill}{rgb}{0.127568,0.566949,0.550556}%
\pgfsetfillcolor{currentfill}%
\pgfsetfillopacity{0.700000}%
\pgfsetlinewidth{0.000000pt}%
\definecolor{currentstroke}{rgb}{0.000000,0.000000,0.000000}%
\pgfsetstrokecolor{currentstroke}%
\pgfsetstrokeopacity{0.700000}%
\pgfsetdash{}{0pt}%
\pgfpathmoveto{\pgfqpoint{8.072087in}{2.741277in}}%
\pgfpathcurveto{\pgfqpoint{8.077130in}{2.741277in}}{\pgfqpoint{8.081968in}{2.743281in}}{\pgfqpoint{8.085534in}{2.746848in}}%
\pgfpathcurveto{\pgfqpoint{8.089101in}{2.750414in}}{\pgfqpoint{8.091105in}{2.755252in}}{\pgfqpoint{8.091105in}{2.760296in}}%
\pgfpathcurveto{\pgfqpoint{8.091105in}{2.765339in}}{\pgfqpoint{8.089101in}{2.770177in}}{\pgfqpoint{8.085534in}{2.773743in}}%
\pgfpathcurveto{\pgfqpoint{8.081968in}{2.777310in}}{\pgfqpoint{8.077130in}{2.779314in}}{\pgfqpoint{8.072087in}{2.779314in}}%
\pgfpathcurveto{\pgfqpoint{8.067043in}{2.779314in}}{\pgfqpoint{8.062205in}{2.777310in}}{\pgfqpoint{8.058639in}{2.773743in}}%
\pgfpathcurveto{\pgfqpoint{8.055072in}{2.770177in}}{\pgfqpoint{8.053068in}{2.765339in}}{\pgfqpoint{8.053068in}{2.760296in}}%
\pgfpathcurveto{\pgfqpoint{8.053068in}{2.755252in}}{\pgfqpoint{8.055072in}{2.750414in}}{\pgfqpoint{8.058639in}{2.746848in}}%
\pgfpathcurveto{\pgfqpoint{8.062205in}{2.743281in}}{\pgfqpoint{8.067043in}{2.741277in}}{\pgfqpoint{8.072087in}{2.741277in}}%
\pgfpathclose%
\pgfusepath{fill}%
\end{pgfscope}%
\begin{pgfscope}%
\pgfpathrectangle{\pgfqpoint{6.572727in}{0.474100in}}{\pgfqpoint{4.227273in}{3.318700in}}%
\pgfusepath{clip}%
\pgfsetbuttcap%
\pgfsetroundjoin%
\definecolor{currentfill}{rgb}{0.993248,0.906157,0.143936}%
\pgfsetfillcolor{currentfill}%
\pgfsetfillopacity{0.700000}%
\pgfsetlinewidth{0.000000pt}%
\definecolor{currentstroke}{rgb}{0.000000,0.000000,0.000000}%
\pgfsetstrokecolor{currentstroke}%
\pgfsetstrokeopacity{0.700000}%
\pgfsetdash{}{0pt}%
\pgfpathmoveto{\pgfqpoint{9.928319in}{1.720087in}}%
\pgfpathcurveto{\pgfqpoint{9.933362in}{1.720087in}}{\pgfqpoint{9.938200in}{1.722091in}}{\pgfqpoint{9.941766in}{1.725657in}}%
\pgfpathcurveto{\pgfqpoint{9.945333in}{1.729224in}}{\pgfqpoint{9.947337in}{1.734062in}}{\pgfqpoint{9.947337in}{1.739105in}}%
\pgfpathcurveto{\pgfqpoint{9.947337in}{1.744149in}}{\pgfqpoint{9.945333in}{1.748987in}}{\pgfqpoint{9.941766in}{1.752553in}}%
\pgfpathcurveto{\pgfqpoint{9.938200in}{1.756120in}}{\pgfqpoint{9.933362in}{1.758123in}}{\pgfqpoint{9.928319in}{1.758123in}}%
\pgfpathcurveto{\pgfqpoint{9.923275in}{1.758123in}}{\pgfqpoint{9.918437in}{1.756120in}}{\pgfqpoint{9.914871in}{1.752553in}}%
\pgfpathcurveto{\pgfqpoint{9.911304in}{1.748987in}}{\pgfqpoint{9.909300in}{1.744149in}}{\pgfqpoint{9.909300in}{1.739105in}}%
\pgfpathcurveto{\pgfqpoint{9.909300in}{1.734062in}}{\pgfqpoint{9.911304in}{1.729224in}}{\pgfqpoint{9.914871in}{1.725657in}}%
\pgfpathcurveto{\pgfqpoint{9.918437in}{1.722091in}}{\pgfqpoint{9.923275in}{1.720087in}}{\pgfqpoint{9.928319in}{1.720087in}}%
\pgfpathclose%
\pgfusepath{fill}%
\end{pgfscope}%
\begin{pgfscope}%
\pgfpathrectangle{\pgfqpoint{6.572727in}{0.474100in}}{\pgfqpoint{4.227273in}{3.318700in}}%
\pgfusepath{clip}%
\pgfsetbuttcap%
\pgfsetroundjoin%
\definecolor{currentfill}{rgb}{0.127568,0.566949,0.550556}%
\pgfsetfillcolor{currentfill}%
\pgfsetfillopacity{0.700000}%
\pgfsetlinewidth{0.000000pt}%
\definecolor{currentstroke}{rgb}{0.000000,0.000000,0.000000}%
\pgfsetstrokecolor{currentstroke}%
\pgfsetstrokeopacity{0.700000}%
\pgfsetdash{}{0pt}%
\pgfpathmoveto{\pgfqpoint{7.964010in}{2.331815in}}%
\pgfpathcurveto{\pgfqpoint{7.969053in}{2.331815in}}{\pgfqpoint{7.973891in}{2.333819in}}{\pgfqpoint{7.977457in}{2.337385in}}%
\pgfpathcurveto{\pgfqpoint{7.981024in}{2.340952in}}{\pgfqpoint{7.983028in}{2.345789in}}{\pgfqpoint{7.983028in}{2.350833in}}%
\pgfpathcurveto{\pgfqpoint{7.983028in}{2.355877in}}{\pgfqpoint{7.981024in}{2.360715in}}{\pgfqpoint{7.977457in}{2.364281in}}%
\pgfpathcurveto{\pgfqpoint{7.973891in}{2.367847in}}{\pgfqpoint{7.969053in}{2.369851in}}{\pgfqpoint{7.964010in}{2.369851in}}%
\pgfpathcurveto{\pgfqpoint{7.958966in}{2.369851in}}{\pgfqpoint{7.954128in}{2.367847in}}{\pgfqpoint{7.950562in}{2.364281in}}%
\pgfpathcurveto{\pgfqpoint{7.946995in}{2.360715in}}{\pgfqpoint{7.944991in}{2.355877in}}{\pgfqpoint{7.944991in}{2.350833in}}%
\pgfpathcurveto{\pgfqpoint{7.944991in}{2.345789in}}{\pgfqpoint{7.946995in}{2.340952in}}{\pgfqpoint{7.950562in}{2.337385in}}%
\pgfpathcurveto{\pgfqpoint{7.954128in}{2.333819in}}{\pgfqpoint{7.958966in}{2.331815in}}{\pgfqpoint{7.964010in}{2.331815in}}%
\pgfpathclose%
\pgfusepath{fill}%
\end{pgfscope}%
\begin{pgfscope}%
\pgfpathrectangle{\pgfqpoint{6.572727in}{0.474100in}}{\pgfqpoint{4.227273in}{3.318700in}}%
\pgfusepath{clip}%
\pgfsetbuttcap%
\pgfsetroundjoin%
\definecolor{currentfill}{rgb}{0.993248,0.906157,0.143936}%
\pgfsetfillcolor{currentfill}%
\pgfsetfillopacity{0.700000}%
\pgfsetlinewidth{0.000000pt}%
\definecolor{currentstroke}{rgb}{0.000000,0.000000,0.000000}%
\pgfsetstrokecolor{currentstroke}%
\pgfsetstrokeopacity{0.700000}%
\pgfsetdash{}{0pt}%
\pgfpathmoveto{\pgfqpoint{9.534888in}{1.473695in}}%
\pgfpathcurveto{\pgfqpoint{9.539931in}{1.473695in}}{\pgfqpoint{9.544769in}{1.475699in}}{\pgfqpoint{9.548336in}{1.479266in}}%
\pgfpathcurveto{\pgfqpoint{9.551902in}{1.482832in}}{\pgfqpoint{9.553906in}{1.487670in}}{\pgfqpoint{9.553906in}{1.492713in}}%
\pgfpathcurveto{\pgfqpoint{9.553906in}{1.497757in}}{\pgfqpoint{9.551902in}{1.502595in}}{\pgfqpoint{9.548336in}{1.506161in}}%
\pgfpathcurveto{\pgfqpoint{9.544769in}{1.509728in}}{\pgfqpoint{9.539931in}{1.511732in}}{\pgfqpoint{9.534888in}{1.511732in}}%
\pgfpathcurveto{\pgfqpoint{9.529844in}{1.511732in}}{\pgfqpoint{9.525006in}{1.509728in}}{\pgfqpoint{9.521440in}{1.506161in}}%
\pgfpathcurveto{\pgfqpoint{9.517873in}{1.502595in}}{\pgfqpoint{9.515870in}{1.497757in}}{\pgfqpoint{9.515870in}{1.492713in}}%
\pgfpathcurveto{\pgfqpoint{9.515870in}{1.487670in}}{\pgfqpoint{9.517873in}{1.482832in}}{\pgfqpoint{9.521440in}{1.479266in}}%
\pgfpathcurveto{\pgfqpoint{9.525006in}{1.475699in}}{\pgfqpoint{9.529844in}{1.473695in}}{\pgfqpoint{9.534888in}{1.473695in}}%
\pgfpathclose%
\pgfusepath{fill}%
\end{pgfscope}%
\begin{pgfscope}%
\pgfpathrectangle{\pgfqpoint{6.572727in}{0.474100in}}{\pgfqpoint{4.227273in}{3.318700in}}%
\pgfusepath{clip}%
\pgfsetbuttcap%
\pgfsetroundjoin%
\definecolor{currentfill}{rgb}{0.127568,0.566949,0.550556}%
\pgfsetfillcolor{currentfill}%
\pgfsetfillopacity{0.700000}%
\pgfsetlinewidth{0.000000pt}%
\definecolor{currentstroke}{rgb}{0.000000,0.000000,0.000000}%
\pgfsetstrokecolor{currentstroke}%
\pgfsetstrokeopacity{0.700000}%
\pgfsetdash{}{0pt}%
\pgfpathmoveto{\pgfqpoint{7.980377in}{2.541331in}}%
\pgfpathcurveto{\pgfqpoint{7.985420in}{2.541331in}}{\pgfqpoint{7.990258in}{2.543335in}}{\pgfqpoint{7.993824in}{2.546901in}}%
\pgfpathcurveto{\pgfqpoint{7.997391in}{2.550468in}}{\pgfqpoint{7.999395in}{2.555305in}}{\pgfqpoint{7.999395in}{2.560349in}}%
\pgfpathcurveto{\pgfqpoint{7.999395in}{2.565393in}}{\pgfqpoint{7.997391in}{2.570230in}}{\pgfqpoint{7.993824in}{2.573797in}}%
\pgfpathcurveto{\pgfqpoint{7.990258in}{2.577363in}}{\pgfqpoint{7.985420in}{2.579367in}}{\pgfqpoint{7.980377in}{2.579367in}}%
\pgfpathcurveto{\pgfqpoint{7.975333in}{2.579367in}}{\pgfqpoint{7.970495in}{2.577363in}}{\pgfqpoint{7.966929in}{2.573797in}}%
\pgfpathcurveto{\pgfqpoint{7.963362in}{2.570230in}}{\pgfqpoint{7.961358in}{2.565393in}}{\pgfqpoint{7.961358in}{2.560349in}}%
\pgfpathcurveto{\pgfqpoint{7.961358in}{2.555305in}}{\pgfqpoint{7.963362in}{2.550468in}}{\pgfqpoint{7.966929in}{2.546901in}}%
\pgfpathcurveto{\pgfqpoint{7.970495in}{2.543335in}}{\pgfqpoint{7.975333in}{2.541331in}}{\pgfqpoint{7.980377in}{2.541331in}}%
\pgfpathclose%
\pgfusepath{fill}%
\end{pgfscope}%
\begin{pgfscope}%
\pgfpathrectangle{\pgfqpoint{6.572727in}{0.474100in}}{\pgfqpoint{4.227273in}{3.318700in}}%
\pgfusepath{clip}%
\pgfsetbuttcap%
\pgfsetroundjoin%
\definecolor{currentfill}{rgb}{0.993248,0.906157,0.143936}%
\pgfsetfillcolor{currentfill}%
\pgfsetfillopacity{0.700000}%
\pgfsetlinewidth{0.000000pt}%
\definecolor{currentstroke}{rgb}{0.000000,0.000000,0.000000}%
\pgfsetstrokecolor{currentstroke}%
\pgfsetstrokeopacity{0.700000}%
\pgfsetdash{}{0pt}%
\pgfpathmoveto{\pgfqpoint{9.356800in}{1.181004in}}%
\pgfpathcurveto{\pgfqpoint{9.361844in}{1.181004in}}{\pgfqpoint{9.366681in}{1.183008in}}{\pgfqpoint{9.370248in}{1.186575in}}%
\pgfpathcurveto{\pgfqpoint{9.373814in}{1.190141in}}{\pgfqpoint{9.375818in}{1.194979in}}{\pgfqpoint{9.375818in}{1.200023in}}%
\pgfpathcurveto{\pgfqpoint{9.375818in}{1.205066in}}{\pgfqpoint{9.373814in}{1.209904in}}{\pgfqpoint{9.370248in}{1.213470in}}%
\pgfpathcurveto{\pgfqpoint{9.366681in}{1.217037in}}{\pgfqpoint{9.361844in}{1.219041in}}{\pgfqpoint{9.356800in}{1.219041in}}%
\pgfpathcurveto{\pgfqpoint{9.351756in}{1.219041in}}{\pgfqpoint{9.346918in}{1.217037in}}{\pgfqpoint{9.343352in}{1.213470in}}%
\pgfpathcurveto{\pgfqpoint{9.339786in}{1.209904in}}{\pgfqpoint{9.337782in}{1.205066in}}{\pgfqpoint{9.337782in}{1.200023in}}%
\pgfpathcurveto{\pgfqpoint{9.337782in}{1.194979in}}{\pgfqpoint{9.339786in}{1.190141in}}{\pgfqpoint{9.343352in}{1.186575in}}%
\pgfpathcurveto{\pgfqpoint{9.346918in}{1.183008in}}{\pgfqpoint{9.351756in}{1.181004in}}{\pgfqpoint{9.356800in}{1.181004in}}%
\pgfpathclose%
\pgfusepath{fill}%
\end{pgfscope}%
\begin{pgfscope}%
\pgfpathrectangle{\pgfqpoint{6.572727in}{0.474100in}}{\pgfqpoint{4.227273in}{3.318700in}}%
\pgfusepath{clip}%
\pgfsetbuttcap%
\pgfsetroundjoin%
\definecolor{currentfill}{rgb}{0.127568,0.566949,0.550556}%
\pgfsetfillcolor{currentfill}%
\pgfsetfillopacity{0.700000}%
\pgfsetlinewidth{0.000000pt}%
\definecolor{currentstroke}{rgb}{0.000000,0.000000,0.000000}%
\pgfsetstrokecolor{currentstroke}%
\pgfsetstrokeopacity{0.700000}%
\pgfsetdash{}{0pt}%
\pgfpathmoveto{\pgfqpoint{7.915122in}{1.505223in}}%
\pgfpathcurveto{\pgfqpoint{7.920165in}{1.505223in}}{\pgfqpoint{7.925003in}{1.507227in}}{\pgfqpoint{7.928569in}{1.510793in}}%
\pgfpathcurveto{\pgfqpoint{7.932136in}{1.514360in}}{\pgfqpoint{7.934140in}{1.519198in}}{\pgfqpoint{7.934140in}{1.524241in}}%
\pgfpathcurveto{\pgfqpoint{7.934140in}{1.529285in}}{\pgfqpoint{7.932136in}{1.534123in}}{\pgfqpoint{7.928569in}{1.537689in}}%
\pgfpathcurveto{\pgfqpoint{7.925003in}{1.541255in}}{\pgfqpoint{7.920165in}{1.543259in}}{\pgfqpoint{7.915122in}{1.543259in}}%
\pgfpathcurveto{\pgfqpoint{7.910078in}{1.543259in}}{\pgfqpoint{7.905240in}{1.541255in}}{\pgfqpoint{7.901674in}{1.537689in}}%
\pgfpathcurveto{\pgfqpoint{7.898107in}{1.534123in}}{\pgfqpoint{7.896103in}{1.529285in}}{\pgfqpoint{7.896103in}{1.524241in}}%
\pgfpathcurveto{\pgfqpoint{7.896103in}{1.519198in}}{\pgfqpoint{7.898107in}{1.514360in}}{\pgfqpoint{7.901674in}{1.510793in}}%
\pgfpathcurveto{\pgfqpoint{7.905240in}{1.507227in}}{\pgfqpoint{7.910078in}{1.505223in}}{\pgfqpoint{7.915122in}{1.505223in}}%
\pgfpathclose%
\pgfusepath{fill}%
\end{pgfscope}%
\begin{pgfscope}%
\pgfpathrectangle{\pgfqpoint{6.572727in}{0.474100in}}{\pgfqpoint{4.227273in}{3.318700in}}%
\pgfusepath{clip}%
\pgfsetbuttcap%
\pgfsetroundjoin%
\definecolor{currentfill}{rgb}{0.993248,0.906157,0.143936}%
\pgfsetfillcolor{currentfill}%
\pgfsetfillopacity{0.700000}%
\pgfsetlinewidth{0.000000pt}%
\definecolor{currentstroke}{rgb}{0.000000,0.000000,0.000000}%
\pgfsetstrokecolor{currentstroke}%
\pgfsetstrokeopacity{0.700000}%
\pgfsetdash{}{0pt}%
\pgfpathmoveto{\pgfqpoint{9.906081in}{1.637996in}}%
\pgfpathcurveto{\pgfqpoint{9.911125in}{1.637996in}}{\pgfqpoint{9.915963in}{1.640000in}}{\pgfqpoint{9.919529in}{1.643566in}}%
\pgfpathcurveto{\pgfqpoint{9.923096in}{1.647133in}}{\pgfqpoint{9.925100in}{1.651970in}}{\pgfqpoint{9.925100in}{1.657014in}}%
\pgfpathcurveto{\pgfqpoint{9.925100in}{1.662058in}}{\pgfqpoint{9.923096in}{1.666896in}}{\pgfqpoint{9.919529in}{1.670462in}}%
\pgfpathcurveto{\pgfqpoint{9.915963in}{1.674028in}}{\pgfqpoint{9.911125in}{1.676032in}}{\pgfqpoint{9.906081in}{1.676032in}}%
\pgfpathcurveto{\pgfqpoint{9.901038in}{1.676032in}}{\pgfqpoint{9.896200in}{1.674028in}}{\pgfqpoint{9.892634in}{1.670462in}}%
\pgfpathcurveto{\pgfqpoint{9.889067in}{1.666896in}}{\pgfqpoint{9.887063in}{1.662058in}}{\pgfqpoint{9.887063in}{1.657014in}}%
\pgfpathcurveto{\pgfqpoint{9.887063in}{1.651970in}}{\pgfqpoint{9.889067in}{1.647133in}}{\pgfqpoint{9.892634in}{1.643566in}}%
\pgfpathcurveto{\pgfqpoint{9.896200in}{1.640000in}}{\pgfqpoint{9.901038in}{1.637996in}}{\pgfqpoint{9.906081in}{1.637996in}}%
\pgfpathclose%
\pgfusepath{fill}%
\end{pgfscope}%
\begin{pgfscope}%
\pgfpathrectangle{\pgfqpoint{6.572727in}{0.474100in}}{\pgfqpoint{4.227273in}{3.318700in}}%
\pgfusepath{clip}%
\pgfsetbuttcap%
\pgfsetroundjoin%
\definecolor{currentfill}{rgb}{0.127568,0.566949,0.550556}%
\pgfsetfillcolor{currentfill}%
\pgfsetfillopacity{0.700000}%
\pgfsetlinewidth{0.000000pt}%
\definecolor{currentstroke}{rgb}{0.000000,0.000000,0.000000}%
\pgfsetstrokecolor{currentstroke}%
\pgfsetstrokeopacity{0.700000}%
\pgfsetdash{}{0pt}%
\pgfpathmoveto{\pgfqpoint{7.620807in}{2.943823in}}%
\pgfpathcurveto{\pgfqpoint{7.625850in}{2.943823in}}{\pgfqpoint{7.630688in}{2.945827in}}{\pgfqpoint{7.634255in}{2.949393in}}%
\pgfpathcurveto{\pgfqpoint{7.637821in}{2.952960in}}{\pgfqpoint{7.639825in}{2.957798in}}{\pgfqpoint{7.639825in}{2.962841in}}%
\pgfpathcurveto{\pgfqpoint{7.639825in}{2.967885in}}{\pgfqpoint{7.637821in}{2.972723in}}{\pgfqpoint{7.634255in}{2.976289in}}%
\pgfpathcurveto{\pgfqpoint{7.630688in}{2.979855in}}{\pgfqpoint{7.625850in}{2.981859in}}{\pgfqpoint{7.620807in}{2.981859in}}%
\pgfpathcurveto{\pgfqpoint{7.615763in}{2.981859in}}{\pgfqpoint{7.610925in}{2.979855in}}{\pgfqpoint{7.607359in}{2.976289in}}%
\pgfpathcurveto{\pgfqpoint{7.603792in}{2.972723in}}{\pgfqpoint{7.601789in}{2.967885in}}{\pgfqpoint{7.601789in}{2.962841in}}%
\pgfpathcurveto{\pgfqpoint{7.601789in}{2.957798in}}{\pgfqpoint{7.603792in}{2.952960in}}{\pgfqpoint{7.607359in}{2.949393in}}%
\pgfpathcurveto{\pgfqpoint{7.610925in}{2.945827in}}{\pgfqpoint{7.615763in}{2.943823in}}{\pgfqpoint{7.620807in}{2.943823in}}%
\pgfpathclose%
\pgfusepath{fill}%
\end{pgfscope}%
\begin{pgfscope}%
\pgfpathrectangle{\pgfqpoint{6.572727in}{0.474100in}}{\pgfqpoint{4.227273in}{3.318700in}}%
\pgfusepath{clip}%
\pgfsetbuttcap%
\pgfsetroundjoin%
\definecolor{currentfill}{rgb}{0.993248,0.906157,0.143936}%
\pgfsetfillcolor{currentfill}%
\pgfsetfillopacity{0.700000}%
\pgfsetlinewidth{0.000000pt}%
\definecolor{currentstroke}{rgb}{0.000000,0.000000,0.000000}%
\pgfsetstrokecolor{currentstroke}%
\pgfsetstrokeopacity{0.700000}%
\pgfsetdash{}{0pt}%
\pgfpathmoveto{\pgfqpoint{8.829797in}{1.998608in}}%
\pgfpathcurveto{\pgfqpoint{8.834840in}{1.998608in}}{\pgfqpoint{8.839678in}{2.000612in}}{\pgfqpoint{8.843244in}{2.004178in}}%
\pgfpathcurveto{\pgfqpoint{8.846811in}{2.007745in}}{\pgfqpoint{8.848815in}{2.012583in}}{\pgfqpoint{8.848815in}{2.017626in}}%
\pgfpathcurveto{\pgfqpoint{8.848815in}{2.022670in}}{\pgfqpoint{8.846811in}{2.027508in}}{\pgfqpoint{8.843244in}{2.031074in}}%
\pgfpathcurveto{\pgfqpoint{8.839678in}{2.034640in}}{\pgfqpoint{8.834840in}{2.036644in}}{\pgfqpoint{8.829797in}{2.036644in}}%
\pgfpathcurveto{\pgfqpoint{8.824753in}{2.036644in}}{\pgfqpoint{8.819915in}{2.034640in}}{\pgfqpoint{8.816349in}{2.031074in}}%
\pgfpathcurveto{\pgfqpoint{8.812782in}{2.027508in}}{\pgfqpoint{8.810778in}{2.022670in}}{\pgfqpoint{8.810778in}{2.017626in}}%
\pgfpathcurveto{\pgfqpoint{8.810778in}{2.012583in}}{\pgfqpoint{8.812782in}{2.007745in}}{\pgfqpoint{8.816349in}{2.004178in}}%
\pgfpathcurveto{\pgfqpoint{8.819915in}{2.000612in}}{\pgfqpoint{8.824753in}{1.998608in}}{\pgfqpoint{8.829797in}{1.998608in}}%
\pgfpathclose%
\pgfusepath{fill}%
\end{pgfscope}%
\begin{pgfscope}%
\pgfpathrectangle{\pgfqpoint{6.572727in}{0.474100in}}{\pgfqpoint{4.227273in}{3.318700in}}%
\pgfusepath{clip}%
\pgfsetbuttcap%
\pgfsetroundjoin%
\definecolor{currentfill}{rgb}{0.127568,0.566949,0.550556}%
\pgfsetfillcolor{currentfill}%
\pgfsetfillopacity{0.700000}%
\pgfsetlinewidth{0.000000pt}%
\definecolor{currentstroke}{rgb}{0.000000,0.000000,0.000000}%
\pgfsetstrokecolor{currentstroke}%
\pgfsetstrokeopacity{0.700000}%
\pgfsetdash{}{0pt}%
\pgfpathmoveto{\pgfqpoint{8.213233in}{2.879108in}}%
\pgfpathcurveto{\pgfqpoint{8.218276in}{2.879108in}}{\pgfqpoint{8.223114in}{2.881112in}}{\pgfqpoint{8.226680in}{2.884678in}}%
\pgfpathcurveto{\pgfqpoint{8.230247in}{2.888245in}}{\pgfqpoint{8.232251in}{2.893082in}}{\pgfqpoint{8.232251in}{2.898126in}}%
\pgfpathcurveto{\pgfqpoint{8.232251in}{2.903170in}}{\pgfqpoint{8.230247in}{2.908007in}}{\pgfqpoint{8.226680in}{2.911574in}}%
\pgfpathcurveto{\pgfqpoint{8.223114in}{2.915140in}}{\pgfqpoint{8.218276in}{2.917144in}}{\pgfqpoint{8.213233in}{2.917144in}}%
\pgfpathcurveto{\pgfqpoint{8.208189in}{2.917144in}}{\pgfqpoint{8.203351in}{2.915140in}}{\pgfqpoint{8.199785in}{2.911574in}}%
\pgfpathcurveto{\pgfqpoint{8.196218in}{2.908007in}}{\pgfqpoint{8.194214in}{2.903170in}}{\pgfqpoint{8.194214in}{2.898126in}}%
\pgfpathcurveto{\pgfqpoint{8.194214in}{2.893082in}}{\pgfqpoint{8.196218in}{2.888245in}}{\pgfqpoint{8.199785in}{2.884678in}}%
\pgfpathcurveto{\pgfqpoint{8.203351in}{2.881112in}}{\pgfqpoint{8.208189in}{2.879108in}}{\pgfqpoint{8.213233in}{2.879108in}}%
\pgfpathclose%
\pgfusepath{fill}%
\end{pgfscope}%
\begin{pgfscope}%
\pgfpathrectangle{\pgfqpoint{6.572727in}{0.474100in}}{\pgfqpoint{4.227273in}{3.318700in}}%
\pgfusepath{clip}%
\pgfsetbuttcap%
\pgfsetroundjoin%
\definecolor{currentfill}{rgb}{0.993248,0.906157,0.143936}%
\pgfsetfillcolor{currentfill}%
\pgfsetfillopacity{0.700000}%
\pgfsetlinewidth{0.000000pt}%
\definecolor{currentstroke}{rgb}{0.000000,0.000000,0.000000}%
\pgfsetstrokecolor{currentstroke}%
\pgfsetstrokeopacity{0.700000}%
\pgfsetdash{}{0pt}%
\pgfpathmoveto{\pgfqpoint{9.332851in}{1.556872in}}%
\pgfpathcurveto{\pgfqpoint{9.337894in}{1.556872in}}{\pgfqpoint{9.342732in}{1.558876in}}{\pgfqpoint{9.346298in}{1.562442in}}%
\pgfpathcurveto{\pgfqpoint{9.349865in}{1.566009in}}{\pgfqpoint{9.351869in}{1.570847in}}{\pgfqpoint{9.351869in}{1.575890in}}%
\pgfpathcurveto{\pgfqpoint{9.351869in}{1.580934in}}{\pgfqpoint{9.349865in}{1.585772in}}{\pgfqpoint{9.346298in}{1.589338in}}%
\pgfpathcurveto{\pgfqpoint{9.342732in}{1.592905in}}{\pgfqpoint{9.337894in}{1.594908in}}{\pgfqpoint{9.332851in}{1.594908in}}%
\pgfpathcurveto{\pgfqpoint{9.327807in}{1.594908in}}{\pgfqpoint{9.322969in}{1.592905in}}{\pgfqpoint{9.319403in}{1.589338in}}%
\pgfpathcurveto{\pgfqpoint{9.315836in}{1.585772in}}{\pgfqpoint{9.313832in}{1.580934in}}{\pgfqpoint{9.313832in}{1.575890in}}%
\pgfpathcurveto{\pgfqpoint{9.313832in}{1.570847in}}{\pgfqpoint{9.315836in}{1.566009in}}{\pgfqpoint{9.319403in}{1.562442in}}%
\pgfpathcurveto{\pgfqpoint{9.322969in}{1.558876in}}{\pgfqpoint{9.327807in}{1.556872in}}{\pgfqpoint{9.332851in}{1.556872in}}%
\pgfpathclose%
\pgfusepath{fill}%
\end{pgfscope}%
\begin{pgfscope}%
\pgfpathrectangle{\pgfqpoint{6.572727in}{0.474100in}}{\pgfqpoint{4.227273in}{3.318700in}}%
\pgfusepath{clip}%
\pgfsetbuttcap%
\pgfsetroundjoin%
\definecolor{currentfill}{rgb}{0.993248,0.906157,0.143936}%
\pgfsetfillcolor{currentfill}%
\pgfsetfillopacity{0.700000}%
\pgfsetlinewidth{0.000000pt}%
\definecolor{currentstroke}{rgb}{0.000000,0.000000,0.000000}%
\pgfsetstrokecolor{currentstroke}%
\pgfsetstrokeopacity{0.700000}%
\pgfsetdash{}{0pt}%
\pgfpathmoveto{\pgfqpoint{9.693822in}{1.991496in}}%
\pgfpathcurveto{\pgfqpoint{9.698866in}{1.991496in}}{\pgfqpoint{9.703704in}{1.993500in}}{\pgfqpoint{9.707270in}{1.997066in}}%
\pgfpathcurveto{\pgfqpoint{9.710836in}{2.000633in}}{\pgfqpoint{9.712840in}{2.005470in}}{\pgfqpoint{9.712840in}{2.010514in}}%
\pgfpathcurveto{\pgfqpoint{9.712840in}{2.015558in}}{\pgfqpoint{9.710836in}{2.020396in}}{\pgfqpoint{9.707270in}{2.023962in}}%
\pgfpathcurveto{\pgfqpoint{9.703704in}{2.027528in}}{\pgfqpoint{9.698866in}{2.029532in}}{\pgfqpoint{9.693822in}{2.029532in}}%
\pgfpathcurveto{\pgfqpoint{9.688778in}{2.029532in}}{\pgfqpoint{9.683941in}{2.027528in}}{\pgfqpoint{9.680374in}{2.023962in}}%
\pgfpathcurveto{\pgfqpoint{9.676808in}{2.020396in}}{\pgfqpoint{9.674804in}{2.015558in}}{\pgfqpoint{9.674804in}{2.010514in}}%
\pgfpathcurveto{\pgfqpoint{9.674804in}{2.005470in}}{\pgfqpoint{9.676808in}{2.000633in}}{\pgfqpoint{9.680374in}{1.997066in}}%
\pgfpathcurveto{\pgfqpoint{9.683941in}{1.993500in}}{\pgfqpoint{9.688778in}{1.991496in}}{\pgfqpoint{9.693822in}{1.991496in}}%
\pgfpathclose%
\pgfusepath{fill}%
\end{pgfscope}%
\begin{pgfscope}%
\pgfpathrectangle{\pgfqpoint{6.572727in}{0.474100in}}{\pgfqpoint{4.227273in}{3.318700in}}%
\pgfusepath{clip}%
\pgfsetbuttcap%
\pgfsetroundjoin%
\definecolor{currentfill}{rgb}{0.993248,0.906157,0.143936}%
\pgfsetfillcolor{currentfill}%
\pgfsetfillopacity{0.700000}%
\pgfsetlinewidth{0.000000pt}%
\definecolor{currentstroke}{rgb}{0.000000,0.000000,0.000000}%
\pgfsetstrokecolor{currentstroke}%
\pgfsetstrokeopacity{0.700000}%
\pgfsetdash{}{0pt}%
\pgfpathmoveto{\pgfqpoint{9.584245in}{2.014994in}}%
\pgfpathcurveto{\pgfqpoint{9.589289in}{2.014994in}}{\pgfqpoint{9.594127in}{2.016998in}}{\pgfqpoint{9.597693in}{2.020564in}}%
\pgfpathcurveto{\pgfqpoint{9.601259in}{2.024131in}}{\pgfqpoint{9.603263in}{2.028968in}}{\pgfqpoint{9.603263in}{2.034012in}}%
\pgfpathcurveto{\pgfqpoint{9.603263in}{2.039056in}}{\pgfqpoint{9.601259in}{2.043893in}}{\pgfqpoint{9.597693in}{2.047460in}}%
\pgfpathcurveto{\pgfqpoint{9.594127in}{2.051026in}}{\pgfqpoint{9.589289in}{2.053030in}}{\pgfqpoint{9.584245in}{2.053030in}}%
\pgfpathcurveto{\pgfqpoint{9.579201in}{2.053030in}}{\pgfqpoint{9.574364in}{2.051026in}}{\pgfqpoint{9.570797in}{2.047460in}}%
\pgfpathcurveto{\pgfqpoint{9.567231in}{2.043893in}}{\pgfqpoint{9.565227in}{2.039056in}}{\pgfqpoint{9.565227in}{2.034012in}}%
\pgfpathcurveto{\pgfqpoint{9.565227in}{2.028968in}}{\pgfqpoint{9.567231in}{2.024131in}}{\pgfqpoint{9.570797in}{2.020564in}}%
\pgfpathcurveto{\pgfqpoint{9.574364in}{2.016998in}}{\pgfqpoint{9.579201in}{2.014994in}}{\pgfqpoint{9.584245in}{2.014994in}}%
\pgfpathclose%
\pgfusepath{fill}%
\end{pgfscope}%
\begin{pgfscope}%
\pgfpathrectangle{\pgfqpoint{6.572727in}{0.474100in}}{\pgfqpoint{4.227273in}{3.318700in}}%
\pgfusepath{clip}%
\pgfsetbuttcap%
\pgfsetroundjoin%
\definecolor{currentfill}{rgb}{0.127568,0.566949,0.550556}%
\pgfsetfillcolor{currentfill}%
\pgfsetfillopacity{0.700000}%
\pgfsetlinewidth{0.000000pt}%
\definecolor{currentstroke}{rgb}{0.000000,0.000000,0.000000}%
\pgfsetstrokecolor{currentstroke}%
\pgfsetstrokeopacity{0.700000}%
\pgfsetdash{}{0pt}%
\pgfpathmoveto{\pgfqpoint{7.763943in}{1.449971in}}%
\pgfpathcurveto{\pgfqpoint{7.768987in}{1.449971in}}{\pgfqpoint{7.773825in}{1.451974in}}{\pgfqpoint{7.777391in}{1.455541in}}%
\pgfpathcurveto{\pgfqpoint{7.780958in}{1.459107in}}{\pgfqpoint{7.782961in}{1.463945in}}{\pgfqpoint{7.782961in}{1.468989in}}%
\pgfpathcurveto{\pgfqpoint{7.782961in}{1.474032in}}{\pgfqpoint{7.780958in}{1.478870in}}{\pgfqpoint{7.777391in}{1.482437in}}%
\pgfpathcurveto{\pgfqpoint{7.773825in}{1.486003in}}{\pgfqpoint{7.768987in}{1.488007in}}{\pgfqpoint{7.763943in}{1.488007in}}%
\pgfpathcurveto{\pgfqpoint{7.758900in}{1.488007in}}{\pgfqpoint{7.754062in}{1.486003in}}{\pgfqpoint{7.750495in}{1.482437in}}%
\pgfpathcurveto{\pgfqpoint{7.746929in}{1.478870in}}{\pgfqpoint{7.744925in}{1.474032in}}{\pgfqpoint{7.744925in}{1.468989in}}%
\pgfpathcurveto{\pgfqpoint{7.744925in}{1.463945in}}{\pgfqpoint{7.746929in}{1.459107in}}{\pgfqpoint{7.750495in}{1.455541in}}%
\pgfpathcurveto{\pgfqpoint{7.754062in}{1.451974in}}{\pgfqpoint{7.758900in}{1.449971in}}{\pgfqpoint{7.763943in}{1.449971in}}%
\pgfpathclose%
\pgfusepath{fill}%
\end{pgfscope}%
\begin{pgfscope}%
\pgfpathrectangle{\pgfqpoint{6.572727in}{0.474100in}}{\pgfqpoint{4.227273in}{3.318700in}}%
\pgfusepath{clip}%
\pgfsetbuttcap%
\pgfsetroundjoin%
\definecolor{currentfill}{rgb}{0.993248,0.906157,0.143936}%
\pgfsetfillcolor{currentfill}%
\pgfsetfillopacity{0.700000}%
\pgfsetlinewidth{0.000000pt}%
\definecolor{currentstroke}{rgb}{0.000000,0.000000,0.000000}%
\pgfsetstrokecolor{currentstroke}%
\pgfsetstrokeopacity{0.700000}%
\pgfsetdash{}{0pt}%
\pgfpathmoveto{\pgfqpoint{9.068067in}{1.463435in}}%
\pgfpathcurveto{\pgfqpoint{9.073110in}{1.463435in}}{\pgfqpoint{9.077948in}{1.465439in}}{\pgfqpoint{9.081515in}{1.469005in}}%
\pgfpathcurveto{\pgfqpoint{9.085081in}{1.472572in}}{\pgfqpoint{9.087085in}{1.477409in}}{\pgfqpoint{9.087085in}{1.482453in}}%
\pgfpathcurveto{\pgfqpoint{9.087085in}{1.487497in}}{\pgfqpoint{9.085081in}{1.492334in}}{\pgfqpoint{9.081515in}{1.495901in}}%
\pgfpathcurveto{\pgfqpoint{9.077948in}{1.499467in}}{\pgfqpoint{9.073110in}{1.501471in}}{\pgfqpoint{9.068067in}{1.501471in}}%
\pgfpathcurveto{\pgfqpoint{9.063023in}{1.501471in}}{\pgfqpoint{9.058185in}{1.499467in}}{\pgfqpoint{9.054619in}{1.495901in}}%
\pgfpathcurveto{\pgfqpoint{9.051052in}{1.492334in}}{\pgfqpoint{9.049049in}{1.487497in}}{\pgfqpoint{9.049049in}{1.482453in}}%
\pgfpathcurveto{\pgfqpoint{9.049049in}{1.477409in}}{\pgfqpoint{9.051052in}{1.472572in}}{\pgfqpoint{9.054619in}{1.469005in}}%
\pgfpathcurveto{\pgfqpoint{9.058185in}{1.465439in}}{\pgfqpoint{9.063023in}{1.463435in}}{\pgfqpoint{9.068067in}{1.463435in}}%
\pgfpathclose%
\pgfusepath{fill}%
\end{pgfscope}%
\begin{pgfscope}%
\pgfpathrectangle{\pgfqpoint{6.572727in}{0.474100in}}{\pgfqpoint{4.227273in}{3.318700in}}%
\pgfusepath{clip}%
\pgfsetbuttcap%
\pgfsetroundjoin%
\definecolor{currentfill}{rgb}{0.127568,0.566949,0.550556}%
\pgfsetfillcolor{currentfill}%
\pgfsetfillopacity{0.700000}%
\pgfsetlinewidth{0.000000pt}%
\definecolor{currentstroke}{rgb}{0.000000,0.000000,0.000000}%
\pgfsetstrokecolor{currentstroke}%
\pgfsetstrokeopacity{0.700000}%
\pgfsetdash{}{0pt}%
\pgfpathmoveto{\pgfqpoint{7.934818in}{1.491675in}}%
\pgfpathcurveto{\pgfqpoint{7.939861in}{1.491675in}}{\pgfqpoint{7.944699in}{1.493679in}}{\pgfqpoint{7.948265in}{1.497245in}}%
\pgfpathcurveto{\pgfqpoint{7.951832in}{1.500812in}}{\pgfqpoint{7.953836in}{1.505649in}}{\pgfqpoint{7.953836in}{1.510693in}}%
\pgfpathcurveto{\pgfqpoint{7.953836in}{1.515737in}}{\pgfqpoint{7.951832in}{1.520574in}}{\pgfqpoint{7.948265in}{1.524141in}}%
\pgfpathcurveto{\pgfqpoint{7.944699in}{1.527707in}}{\pgfqpoint{7.939861in}{1.529711in}}{\pgfqpoint{7.934818in}{1.529711in}}%
\pgfpathcurveto{\pgfqpoint{7.929774in}{1.529711in}}{\pgfqpoint{7.924936in}{1.527707in}}{\pgfqpoint{7.921370in}{1.524141in}}%
\pgfpathcurveto{\pgfqpoint{7.917803in}{1.520574in}}{\pgfqpoint{7.915799in}{1.515737in}}{\pgfqpoint{7.915799in}{1.510693in}}%
\pgfpathcurveto{\pgfqpoint{7.915799in}{1.505649in}}{\pgfqpoint{7.917803in}{1.500812in}}{\pgfqpoint{7.921370in}{1.497245in}}%
\pgfpathcurveto{\pgfqpoint{7.924936in}{1.493679in}}{\pgfqpoint{7.929774in}{1.491675in}}{\pgfqpoint{7.934818in}{1.491675in}}%
\pgfpathclose%
\pgfusepath{fill}%
\end{pgfscope}%
\begin{pgfscope}%
\pgfpathrectangle{\pgfqpoint{6.572727in}{0.474100in}}{\pgfqpoint{4.227273in}{3.318700in}}%
\pgfusepath{clip}%
\pgfsetbuttcap%
\pgfsetroundjoin%
\definecolor{currentfill}{rgb}{0.127568,0.566949,0.550556}%
\pgfsetfillcolor{currentfill}%
\pgfsetfillopacity{0.700000}%
\pgfsetlinewidth{0.000000pt}%
\definecolor{currentstroke}{rgb}{0.000000,0.000000,0.000000}%
\pgfsetstrokecolor{currentstroke}%
\pgfsetstrokeopacity{0.700000}%
\pgfsetdash{}{0pt}%
\pgfpathmoveto{\pgfqpoint{8.053854in}{1.926856in}}%
\pgfpathcurveto{\pgfqpoint{8.058898in}{1.926856in}}{\pgfqpoint{8.063736in}{1.928860in}}{\pgfqpoint{8.067302in}{1.932426in}}%
\pgfpathcurveto{\pgfqpoint{8.070869in}{1.935993in}}{\pgfqpoint{8.072873in}{1.940830in}}{\pgfqpoint{8.072873in}{1.945874in}}%
\pgfpathcurveto{\pgfqpoint{8.072873in}{1.950918in}}{\pgfqpoint{8.070869in}{1.955755in}}{\pgfqpoint{8.067302in}{1.959322in}}%
\pgfpathcurveto{\pgfqpoint{8.063736in}{1.962888in}}{\pgfqpoint{8.058898in}{1.964892in}}{\pgfqpoint{8.053854in}{1.964892in}}%
\pgfpathcurveto{\pgfqpoint{8.048811in}{1.964892in}}{\pgfqpoint{8.043973in}{1.962888in}}{\pgfqpoint{8.040407in}{1.959322in}}%
\pgfpathcurveto{\pgfqpoint{8.036840in}{1.955755in}}{\pgfqpoint{8.034836in}{1.950918in}}{\pgfqpoint{8.034836in}{1.945874in}}%
\pgfpathcurveto{\pgfqpoint{8.034836in}{1.940830in}}{\pgfqpoint{8.036840in}{1.935993in}}{\pgfqpoint{8.040407in}{1.932426in}}%
\pgfpathcurveto{\pgfqpoint{8.043973in}{1.928860in}}{\pgfqpoint{8.048811in}{1.926856in}}{\pgfqpoint{8.053854in}{1.926856in}}%
\pgfpathclose%
\pgfusepath{fill}%
\end{pgfscope}%
\begin{pgfscope}%
\pgfpathrectangle{\pgfqpoint{6.572727in}{0.474100in}}{\pgfqpoint{4.227273in}{3.318700in}}%
\pgfusepath{clip}%
\pgfsetbuttcap%
\pgfsetroundjoin%
\definecolor{currentfill}{rgb}{0.127568,0.566949,0.550556}%
\pgfsetfillcolor{currentfill}%
\pgfsetfillopacity{0.700000}%
\pgfsetlinewidth{0.000000pt}%
\definecolor{currentstroke}{rgb}{0.000000,0.000000,0.000000}%
\pgfsetstrokecolor{currentstroke}%
\pgfsetstrokeopacity{0.700000}%
\pgfsetdash{}{0pt}%
\pgfpathmoveto{\pgfqpoint{7.732711in}{1.428347in}}%
\pgfpathcurveto{\pgfqpoint{7.737755in}{1.428347in}}{\pgfqpoint{7.742592in}{1.430351in}}{\pgfqpoint{7.746159in}{1.433918in}}%
\pgfpathcurveto{\pgfqpoint{7.749725in}{1.437484in}}{\pgfqpoint{7.751729in}{1.442322in}}{\pgfqpoint{7.751729in}{1.447366in}}%
\pgfpathcurveto{\pgfqpoint{7.751729in}{1.452409in}}{\pgfqpoint{7.749725in}{1.457247in}}{\pgfqpoint{7.746159in}{1.460813in}}%
\pgfpathcurveto{\pgfqpoint{7.742592in}{1.464380in}}{\pgfqpoint{7.737755in}{1.466384in}}{\pgfqpoint{7.732711in}{1.466384in}}%
\pgfpathcurveto{\pgfqpoint{7.727667in}{1.466384in}}{\pgfqpoint{7.722830in}{1.464380in}}{\pgfqpoint{7.719263in}{1.460813in}}%
\pgfpathcurveto{\pgfqpoint{7.715697in}{1.457247in}}{\pgfqpoint{7.713693in}{1.452409in}}{\pgfqpoint{7.713693in}{1.447366in}}%
\pgfpathcurveto{\pgfqpoint{7.713693in}{1.442322in}}{\pgfqpoint{7.715697in}{1.437484in}}{\pgfqpoint{7.719263in}{1.433918in}}%
\pgfpathcurveto{\pgfqpoint{7.722830in}{1.430351in}}{\pgfqpoint{7.727667in}{1.428347in}}{\pgfqpoint{7.732711in}{1.428347in}}%
\pgfpathclose%
\pgfusepath{fill}%
\end{pgfscope}%
\begin{pgfscope}%
\pgfpathrectangle{\pgfqpoint{6.572727in}{0.474100in}}{\pgfqpoint{4.227273in}{3.318700in}}%
\pgfusepath{clip}%
\pgfsetbuttcap%
\pgfsetroundjoin%
\definecolor{currentfill}{rgb}{0.127568,0.566949,0.550556}%
\pgfsetfillcolor{currentfill}%
\pgfsetfillopacity{0.700000}%
\pgfsetlinewidth{0.000000pt}%
\definecolor{currentstroke}{rgb}{0.000000,0.000000,0.000000}%
\pgfsetstrokecolor{currentstroke}%
\pgfsetstrokeopacity{0.700000}%
\pgfsetdash{}{0pt}%
\pgfpathmoveto{\pgfqpoint{8.593097in}{1.157935in}}%
\pgfpathcurveto{\pgfqpoint{8.598140in}{1.157935in}}{\pgfqpoint{8.602978in}{1.159939in}}{\pgfqpoint{8.606544in}{1.163505in}}%
\pgfpathcurveto{\pgfqpoint{8.610111in}{1.167071in}}{\pgfqpoint{8.612115in}{1.171909in}}{\pgfqpoint{8.612115in}{1.176953in}}%
\pgfpathcurveto{\pgfqpoint{8.612115in}{1.181997in}}{\pgfqpoint{8.610111in}{1.186834in}}{\pgfqpoint{8.606544in}{1.190401in}}%
\pgfpathcurveto{\pgfqpoint{8.602978in}{1.193967in}}{\pgfqpoint{8.598140in}{1.195971in}}{\pgfqpoint{8.593097in}{1.195971in}}%
\pgfpathcurveto{\pgfqpoint{8.588053in}{1.195971in}}{\pgfqpoint{8.583215in}{1.193967in}}{\pgfqpoint{8.579649in}{1.190401in}}%
\pgfpathcurveto{\pgfqpoint{8.576082in}{1.186834in}}{\pgfqpoint{8.574078in}{1.181997in}}{\pgfqpoint{8.574078in}{1.176953in}}%
\pgfpathcurveto{\pgfqpoint{8.574078in}{1.171909in}}{\pgfqpoint{8.576082in}{1.167071in}}{\pgfqpoint{8.579649in}{1.163505in}}%
\pgfpathcurveto{\pgfqpoint{8.583215in}{1.159939in}}{\pgfqpoint{8.588053in}{1.157935in}}{\pgfqpoint{8.593097in}{1.157935in}}%
\pgfpathclose%
\pgfusepath{fill}%
\end{pgfscope}%
\begin{pgfscope}%
\pgfpathrectangle{\pgfqpoint{6.572727in}{0.474100in}}{\pgfqpoint{4.227273in}{3.318700in}}%
\pgfusepath{clip}%
\pgfsetbuttcap%
\pgfsetroundjoin%
\definecolor{currentfill}{rgb}{0.993248,0.906157,0.143936}%
\pgfsetfillcolor{currentfill}%
\pgfsetfillopacity{0.700000}%
\pgfsetlinewidth{0.000000pt}%
\definecolor{currentstroke}{rgb}{0.000000,0.000000,0.000000}%
\pgfsetstrokecolor{currentstroke}%
\pgfsetstrokeopacity{0.700000}%
\pgfsetdash{}{0pt}%
\pgfpathmoveto{\pgfqpoint{9.435072in}{1.816246in}}%
\pgfpathcurveto{\pgfqpoint{9.440116in}{1.816246in}}{\pgfqpoint{9.444953in}{1.818250in}}{\pgfqpoint{9.448520in}{1.821816in}}%
\pgfpathcurveto{\pgfqpoint{9.452086in}{1.825383in}}{\pgfqpoint{9.454090in}{1.830220in}}{\pgfqpoint{9.454090in}{1.835264in}}%
\pgfpathcurveto{\pgfqpoint{9.454090in}{1.840308in}}{\pgfqpoint{9.452086in}{1.845146in}}{\pgfqpoint{9.448520in}{1.848712in}}%
\pgfpathcurveto{\pgfqpoint{9.444953in}{1.852278in}}{\pgfqpoint{9.440116in}{1.854282in}}{\pgfqpoint{9.435072in}{1.854282in}}%
\pgfpathcurveto{\pgfqpoint{9.430028in}{1.854282in}}{\pgfqpoint{9.425190in}{1.852278in}}{\pgfqpoint{9.421624in}{1.848712in}}%
\pgfpathcurveto{\pgfqpoint{9.418058in}{1.845146in}}{\pgfqpoint{9.416054in}{1.840308in}}{\pgfqpoint{9.416054in}{1.835264in}}%
\pgfpathcurveto{\pgfqpoint{9.416054in}{1.830220in}}{\pgfqpoint{9.418058in}{1.825383in}}{\pgfqpoint{9.421624in}{1.821816in}}%
\pgfpathcurveto{\pgfqpoint{9.425190in}{1.818250in}}{\pgfqpoint{9.430028in}{1.816246in}}{\pgfqpoint{9.435072in}{1.816246in}}%
\pgfpathclose%
\pgfusepath{fill}%
\end{pgfscope}%
\begin{pgfscope}%
\pgfpathrectangle{\pgfqpoint{6.572727in}{0.474100in}}{\pgfqpoint{4.227273in}{3.318700in}}%
\pgfusepath{clip}%
\pgfsetbuttcap%
\pgfsetroundjoin%
\definecolor{currentfill}{rgb}{0.127568,0.566949,0.550556}%
\pgfsetfillcolor{currentfill}%
\pgfsetfillopacity{0.700000}%
\pgfsetlinewidth{0.000000pt}%
\definecolor{currentstroke}{rgb}{0.000000,0.000000,0.000000}%
\pgfsetstrokecolor{currentstroke}%
\pgfsetstrokeopacity{0.700000}%
\pgfsetdash{}{0pt}%
\pgfpathmoveto{\pgfqpoint{7.853059in}{1.427186in}}%
\pgfpathcurveto{\pgfqpoint{7.858103in}{1.427186in}}{\pgfqpoint{7.862941in}{1.429190in}}{\pgfqpoint{7.866507in}{1.432756in}}%
\pgfpathcurveto{\pgfqpoint{7.870073in}{1.436322in}}{\pgfqpoint{7.872077in}{1.441160in}}{\pgfqpoint{7.872077in}{1.446204in}}%
\pgfpathcurveto{\pgfqpoint{7.872077in}{1.451248in}}{\pgfqpoint{7.870073in}{1.456085in}}{\pgfqpoint{7.866507in}{1.459652in}}%
\pgfpathcurveto{\pgfqpoint{7.862941in}{1.463218in}}{\pgfqpoint{7.858103in}{1.465222in}}{\pgfqpoint{7.853059in}{1.465222in}}%
\pgfpathcurveto{\pgfqpoint{7.848016in}{1.465222in}}{\pgfqpoint{7.843178in}{1.463218in}}{\pgfqpoint{7.839611in}{1.459652in}}%
\pgfpathcurveto{\pgfqpoint{7.836045in}{1.456085in}}{\pgfqpoint{7.834041in}{1.451248in}}{\pgfqpoint{7.834041in}{1.446204in}}%
\pgfpathcurveto{\pgfqpoint{7.834041in}{1.441160in}}{\pgfqpoint{7.836045in}{1.436322in}}{\pgfqpoint{7.839611in}{1.432756in}}%
\pgfpathcurveto{\pgfqpoint{7.843178in}{1.429190in}}{\pgfqpoint{7.848016in}{1.427186in}}{\pgfqpoint{7.853059in}{1.427186in}}%
\pgfpathclose%
\pgfusepath{fill}%
\end{pgfscope}%
\begin{pgfscope}%
\pgfpathrectangle{\pgfqpoint{6.572727in}{0.474100in}}{\pgfqpoint{4.227273in}{3.318700in}}%
\pgfusepath{clip}%
\pgfsetbuttcap%
\pgfsetroundjoin%
\definecolor{currentfill}{rgb}{0.127568,0.566949,0.550556}%
\pgfsetfillcolor{currentfill}%
\pgfsetfillopacity{0.700000}%
\pgfsetlinewidth{0.000000pt}%
\definecolor{currentstroke}{rgb}{0.000000,0.000000,0.000000}%
\pgfsetstrokecolor{currentstroke}%
\pgfsetstrokeopacity{0.700000}%
\pgfsetdash{}{0pt}%
\pgfpathmoveto{\pgfqpoint{7.529593in}{1.459074in}}%
\pgfpathcurveto{\pgfqpoint{7.534637in}{1.459074in}}{\pgfqpoint{7.539474in}{1.461077in}}{\pgfqpoint{7.543041in}{1.464644in}}%
\pgfpathcurveto{\pgfqpoint{7.546607in}{1.468210in}}{\pgfqpoint{7.548611in}{1.473048in}}{\pgfqpoint{7.548611in}{1.478092in}}%
\pgfpathcurveto{\pgfqpoint{7.548611in}{1.483135in}}{\pgfqpoint{7.546607in}{1.487973in}}{\pgfqpoint{7.543041in}{1.491540in}}%
\pgfpathcurveto{\pgfqpoint{7.539474in}{1.495106in}}{\pgfqpoint{7.534637in}{1.497110in}}{\pgfqpoint{7.529593in}{1.497110in}}%
\pgfpathcurveto{\pgfqpoint{7.524549in}{1.497110in}}{\pgfqpoint{7.519712in}{1.495106in}}{\pgfqpoint{7.516145in}{1.491540in}}%
\pgfpathcurveto{\pgfqpoint{7.512579in}{1.487973in}}{\pgfqpoint{7.510575in}{1.483135in}}{\pgfqpoint{7.510575in}{1.478092in}}%
\pgfpathcurveto{\pgfqpoint{7.510575in}{1.473048in}}{\pgfqpoint{7.512579in}{1.468210in}}{\pgfqpoint{7.516145in}{1.464644in}}%
\pgfpathcurveto{\pgfqpoint{7.519712in}{1.461077in}}{\pgfqpoint{7.524549in}{1.459074in}}{\pgfqpoint{7.529593in}{1.459074in}}%
\pgfpathclose%
\pgfusepath{fill}%
\end{pgfscope}%
\begin{pgfscope}%
\pgfpathrectangle{\pgfqpoint{6.572727in}{0.474100in}}{\pgfqpoint{4.227273in}{3.318700in}}%
\pgfusepath{clip}%
\pgfsetbuttcap%
\pgfsetroundjoin%
\definecolor{currentfill}{rgb}{0.127568,0.566949,0.550556}%
\pgfsetfillcolor{currentfill}%
\pgfsetfillopacity{0.700000}%
\pgfsetlinewidth{0.000000pt}%
\definecolor{currentstroke}{rgb}{0.000000,0.000000,0.000000}%
\pgfsetstrokecolor{currentstroke}%
\pgfsetstrokeopacity{0.700000}%
\pgfsetdash{}{0pt}%
\pgfpathmoveto{\pgfqpoint{8.803362in}{3.390591in}}%
\pgfpathcurveto{\pgfqpoint{8.808406in}{3.390591in}}{\pgfqpoint{8.813243in}{3.392595in}}{\pgfqpoint{8.816810in}{3.396161in}}%
\pgfpathcurveto{\pgfqpoint{8.820376in}{3.399728in}}{\pgfqpoint{8.822380in}{3.404566in}}{\pgfqpoint{8.822380in}{3.409609in}}%
\pgfpathcurveto{\pgfqpoint{8.822380in}{3.414653in}}{\pgfqpoint{8.820376in}{3.419491in}}{\pgfqpoint{8.816810in}{3.423057in}}%
\pgfpathcurveto{\pgfqpoint{8.813243in}{3.426624in}}{\pgfqpoint{8.808406in}{3.428627in}}{\pgfqpoint{8.803362in}{3.428627in}}%
\pgfpathcurveto{\pgfqpoint{8.798318in}{3.428627in}}{\pgfqpoint{8.793481in}{3.426624in}}{\pgfqpoint{8.789914in}{3.423057in}}%
\pgfpathcurveto{\pgfqpoint{8.786348in}{3.419491in}}{\pgfqpoint{8.784344in}{3.414653in}}{\pgfqpoint{8.784344in}{3.409609in}}%
\pgfpathcurveto{\pgfqpoint{8.784344in}{3.404566in}}{\pgfqpoint{8.786348in}{3.399728in}}{\pgfqpoint{8.789914in}{3.396161in}}%
\pgfpathcurveto{\pgfqpoint{8.793481in}{3.392595in}}{\pgfqpoint{8.798318in}{3.390591in}}{\pgfqpoint{8.803362in}{3.390591in}}%
\pgfpathclose%
\pgfusepath{fill}%
\end{pgfscope}%
\begin{pgfscope}%
\pgfpathrectangle{\pgfqpoint{6.572727in}{0.474100in}}{\pgfqpoint{4.227273in}{3.318700in}}%
\pgfusepath{clip}%
\pgfsetbuttcap%
\pgfsetroundjoin%
\definecolor{currentfill}{rgb}{0.127568,0.566949,0.550556}%
\pgfsetfillcolor{currentfill}%
\pgfsetfillopacity{0.700000}%
\pgfsetlinewidth{0.000000pt}%
\definecolor{currentstroke}{rgb}{0.000000,0.000000,0.000000}%
\pgfsetstrokecolor{currentstroke}%
\pgfsetstrokeopacity{0.700000}%
\pgfsetdash{}{0pt}%
\pgfpathmoveto{\pgfqpoint{7.813722in}{3.465717in}}%
\pgfpathcurveto{\pgfqpoint{7.818766in}{3.465717in}}{\pgfqpoint{7.823603in}{3.467720in}}{\pgfqpoint{7.827170in}{3.471287in}}%
\pgfpathcurveto{\pgfqpoint{7.830736in}{3.474853in}}{\pgfqpoint{7.832740in}{3.479691in}}{\pgfqpoint{7.832740in}{3.484735in}}%
\pgfpathcurveto{\pgfqpoint{7.832740in}{3.489778in}}{\pgfqpoint{7.830736in}{3.494616in}}{\pgfqpoint{7.827170in}{3.498183in}}%
\pgfpathcurveto{\pgfqpoint{7.823603in}{3.501749in}}{\pgfqpoint{7.818766in}{3.503753in}}{\pgfqpoint{7.813722in}{3.503753in}}%
\pgfpathcurveto{\pgfqpoint{7.808678in}{3.503753in}}{\pgfqpoint{7.803841in}{3.501749in}}{\pgfqpoint{7.800274in}{3.498183in}}%
\pgfpathcurveto{\pgfqpoint{7.796708in}{3.494616in}}{\pgfqpoint{7.794704in}{3.489778in}}{\pgfqpoint{7.794704in}{3.484735in}}%
\pgfpathcurveto{\pgfqpoint{7.794704in}{3.479691in}}{\pgfqpoint{7.796708in}{3.474853in}}{\pgfqpoint{7.800274in}{3.471287in}}%
\pgfpathcurveto{\pgfqpoint{7.803841in}{3.467720in}}{\pgfqpoint{7.808678in}{3.465717in}}{\pgfqpoint{7.813722in}{3.465717in}}%
\pgfpathclose%
\pgfusepath{fill}%
\end{pgfscope}%
\begin{pgfscope}%
\pgfpathrectangle{\pgfqpoint{6.572727in}{0.474100in}}{\pgfqpoint{4.227273in}{3.318700in}}%
\pgfusepath{clip}%
\pgfsetbuttcap%
\pgfsetroundjoin%
\definecolor{currentfill}{rgb}{0.127568,0.566949,0.550556}%
\pgfsetfillcolor{currentfill}%
\pgfsetfillopacity{0.700000}%
\pgfsetlinewidth{0.000000pt}%
\definecolor{currentstroke}{rgb}{0.000000,0.000000,0.000000}%
\pgfsetstrokecolor{currentstroke}%
\pgfsetstrokeopacity{0.700000}%
\pgfsetdash{}{0pt}%
\pgfpathmoveto{\pgfqpoint{7.181732in}{1.420284in}}%
\pgfpathcurveto{\pgfqpoint{7.186775in}{1.420284in}}{\pgfqpoint{7.191613in}{1.422288in}}{\pgfqpoint{7.195180in}{1.425854in}}%
\pgfpathcurveto{\pgfqpoint{7.198746in}{1.429421in}}{\pgfqpoint{7.200750in}{1.434258in}}{\pgfqpoint{7.200750in}{1.439302in}}%
\pgfpathcurveto{\pgfqpoint{7.200750in}{1.444346in}}{\pgfqpoint{7.198746in}{1.449184in}}{\pgfqpoint{7.195180in}{1.452750in}}%
\pgfpathcurveto{\pgfqpoint{7.191613in}{1.456316in}}{\pgfqpoint{7.186775in}{1.458320in}}{\pgfqpoint{7.181732in}{1.458320in}}%
\pgfpathcurveto{\pgfqpoint{7.176688in}{1.458320in}}{\pgfqpoint{7.171850in}{1.456316in}}{\pgfqpoint{7.168284in}{1.452750in}}%
\pgfpathcurveto{\pgfqpoint{7.164717in}{1.449184in}}{\pgfqpoint{7.162714in}{1.444346in}}{\pgfqpoint{7.162714in}{1.439302in}}%
\pgfpathcurveto{\pgfqpoint{7.162714in}{1.434258in}}{\pgfqpoint{7.164717in}{1.429421in}}{\pgfqpoint{7.168284in}{1.425854in}}%
\pgfpathcurveto{\pgfqpoint{7.171850in}{1.422288in}}{\pgfqpoint{7.176688in}{1.420284in}}{\pgfqpoint{7.181732in}{1.420284in}}%
\pgfpathclose%
\pgfusepath{fill}%
\end{pgfscope}%
\begin{pgfscope}%
\pgfpathrectangle{\pgfqpoint{6.572727in}{0.474100in}}{\pgfqpoint{4.227273in}{3.318700in}}%
\pgfusepath{clip}%
\pgfsetbuttcap%
\pgfsetroundjoin%
\definecolor{currentfill}{rgb}{0.127568,0.566949,0.550556}%
\pgfsetfillcolor{currentfill}%
\pgfsetfillopacity{0.700000}%
\pgfsetlinewidth{0.000000pt}%
\definecolor{currentstroke}{rgb}{0.000000,0.000000,0.000000}%
\pgfsetstrokecolor{currentstroke}%
\pgfsetstrokeopacity{0.700000}%
\pgfsetdash{}{0pt}%
\pgfpathmoveto{\pgfqpoint{8.454971in}{1.531085in}}%
\pgfpathcurveto{\pgfqpoint{8.460015in}{1.531085in}}{\pgfqpoint{8.464853in}{1.533089in}}{\pgfqpoint{8.468419in}{1.536656in}}%
\pgfpathcurveto{\pgfqpoint{8.471985in}{1.540222in}}{\pgfqpoint{8.473989in}{1.545060in}}{\pgfqpoint{8.473989in}{1.550103in}}%
\pgfpathcurveto{\pgfqpoint{8.473989in}{1.555147in}}{\pgfqpoint{8.471985in}{1.559985in}}{\pgfqpoint{8.468419in}{1.563551in}}%
\pgfpathcurveto{\pgfqpoint{8.464853in}{1.567118in}}{\pgfqpoint{8.460015in}{1.569122in}}{\pgfqpoint{8.454971in}{1.569122in}}%
\pgfpathcurveto{\pgfqpoint{8.449928in}{1.569122in}}{\pgfqpoint{8.445090in}{1.567118in}}{\pgfqpoint{8.441523in}{1.563551in}}%
\pgfpathcurveto{\pgfqpoint{8.437957in}{1.559985in}}{\pgfqpoint{8.435953in}{1.555147in}}{\pgfqpoint{8.435953in}{1.550103in}}%
\pgfpathcurveto{\pgfqpoint{8.435953in}{1.545060in}}{\pgfqpoint{8.437957in}{1.540222in}}{\pgfqpoint{8.441523in}{1.536656in}}%
\pgfpathcurveto{\pgfqpoint{8.445090in}{1.533089in}}{\pgfqpoint{8.449928in}{1.531085in}}{\pgfqpoint{8.454971in}{1.531085in}}%
\pgfpathclose%
\pgfusepath{fill}%
\end{pgfscope}%
\begin{pgfscope}%
\pgfpathrectangle{\pgfqpoint{6.572727in}{0.474100in}}{\pgfqpoint{4.227273in}{3.318700in}}%
\pgfusepath{clip}%
\pgfsetbuttcap%
\pgfsetroundjoin%
\definecolor{currentfill}{rgb}{0.993248,0.906157,0.143936}%
\pgfsetfillcolor{currentfill}%
\pgfsetfillopacity{0.700000}%
\pgfsetlinewidth{0.000000pt}%
\definecolor{currentstroke}{rgb}{0.000000,0.000000,0.000000}%
\pgfsetstrokecolor{currentstroke}%
\pgfsetstrokeopacity{0.700000}%
\pgfsetdash{}{0pt}%
\pgfpathmoveto{\pgfqpoint{9.700941in}{2.075225in}}%
\pgfpathcurveto{\pgfqpoint{9.705985in}{2.075225in}}{\pgfqpoint{9.710823in}{2.077229in}}{\pgfqpoint{9.714389in}{2.080796in}}%
\pgfpathcurveto{\pgfqpoint{9.717956in}{2.084362in}}{\pgfqpoint{9.719959in}{2.089200in}}{\pgfqpoint{9.719959in}{2.094243in}}%
\pgfpathcurveto{\pgfqpoint{9.719959in}{2.099287in}}{\pgfqpoint{9.717956in}{2.104125in}}{\pgfqpoint{9.714389in}{2.107691in}}%
\pgfpathcurveto{\pgfqpoint{9.710823in}{2.111258in}}{\pgfqpoint{9.705985in}{2.113262in}}{\pgfqpoint{9.700941in}{2.113262in}}%
\pgfpathcurveto{\pgfqpoint{9.695898in}{2.113262in}}{\pgfqpoint{9.691060in}{2.111258in}}{\pgfqpoint{9.687493in}{2.107691in}}%
\pgfpathcurveto{\pgfqpoint{9.683927in}{2.104125in}}{\pgfqpoint{9.681923in}{2.099287in}}{\pgfqpoint{9.681923in}{2.094243in}}%
\pgfpathcurveto{\pgfqpoint{9.681923in}{2.089200in}}{\pgfqpoint{9.683927in}{2.084362in}}{\pgfqpoint{9.687493in}{2.080796in}}%
\pgfpathcurveto{\pgfqpoint{9.691060in}{2.077229in}}{\pgfqpoint{9.695898in}{2.075225in}}{\pgfqpoint{9.700941in}{2.075225in}}%
\pgfpathclose%
\pgfusepath{fill}%
\end{pgfscope}%
\begin{pgfscope}%
\pgfpathrectangle{\pgfqpoint{6.572727in}{0.474100in}}{\pgfqpoint{4.227273in}{3.318700in}}%
\pgfusepath{clip}%
\pgfsetbuttcap%
\pgfsetroundjoin%
\definecolor{currentfill}{rgb}{0.127568,0.566949,0.550556}%
\pgfsetfillcolor{currentfill}%
\pgfsetfillopacity{0.700000}%
\pgfsetlinewidth{0.000000pt}%
\definecolor{currentstroke}{rgb}{0.000000,0.000000,0.000000}%
\pgfsetstrokecolor{currentstroke}%
\pgfsetstrokeopacity{0.700000}%
\pgfsetdash{}{0pt}%
\pgfpathmoveto{\pgfqpoint{7.978174in}{2.350066in}}%
\pgfpathcurveto{\pgfqpoint{7.983218in}{2.350066in}}{\pgfqpoint{7.988056in}{2.352070in}}{\pgfqpoint{7.991622in}{2.355637in}}%
\pgfpathcurveto{\pgfqpoint{7.995189in}{2.359203in}}{\pgfqpoint{7.997192in}{2.364041in}}{\pgfqpoint{7.997192in}{2.369084in}}%
\pgfpathcurveto{\pgfqpoint{7.997192in}{2.374128in}}{\pgfqpoint{7.995189in}{2.378966in}}{\pgfqpoint{7.991622in}{2.382532in}}%
\pgfpathcurveto{\pgfqpoint{7.988056in}{2.386099in}}{\pgfqpoint{7.983218in}{2.388103in}}{\pgfqpoint{7.978174in}{2.388103in}}%
\pgfpathcurveto{\pgfqpoint{7.973131in}{2.388103in}}{\pgfqpoint{7.968293in}{2.386099in}}{\pgfqpoint{7.964726in}{2.382532in}}%
\pgfpathcurveto{\pgfqpoint{7.961160in}{2.378966in}}{\pgfqpoint{7.959156in}{2.374128in}}{\pgfqpoint{7.959156in}{2.369084in}}%
\pgfpathcurveto{\pgfqpoint{7.959156in}{2.364041in}}{\pgfqpoint{7.961160in}{2.359203in}}{\pgfqpoint{7.964726in}{2.355637in}}%
\pgfpathcurveto{\pgfqpoint{7.968293in}{2.352070in}}{\pgfqpoint{7.973131in}{2.350066in}}{\pgfqpoint{7.978174in}{2.350066in}}%
\pgfpathclose%
\pgfusepath{fill}%
\end{pgfscope}%
\begin{pgfscope}%
\pgfpathrectangle{\pgfqpoint{6.572727in}{0.474100in}}{\pgfqpoint{4.227273in}{3.318700in}}%
\pgfusepath{clip}%
\pgfsetbuttcap%
\pgfsetroundjoin%
\definecolor{currentfill}{rgb}{0.267004,0.004874,0.329415}%
\pgfsetfillcolor{currentfill}%
\pgfsetfillopacity{0.700000}%
\pgfsetlinewidth{0.000000pt}%
\definecolor{currentstroke}{rgb}{0.000000,0.000000,0.000000}%
\pgfsetstrokecolor{currentstroke}%
\pgfsetstrokeopacity{0.700000}%
\pgfsetdash{}{0pt}%
\pgfpathmoveto{\pgfqpoint{9.235591in}{2.789165in}}%
\pgfpathcurveto{\pgfqpoint{9.240635in}{2.789165in}}{\pgfqpoint{9.245472in}{2.791169in}}{\pgfqpoint{9.249039in}{2.794736in}}%
\pgfpathcurveto{\pgfqpoint{9.252605in}{2.798302in}}{\pgfqpoint{9.254609in}{2.803140in}}{\pgfqpoint{9.254609in}{2.808184in}}%
\pgfpathcurveto{\pgfqpoint{9.254609in}{2.813227in}}{\pgfqpoint{9.252605in}{2.818065in}}{\pgfqpoint{9.249039in}{2.821631in}}%
\pgfpathcurveto{\pgfqpoint{9.245472in}{2.825198in}}{\pgfqpoint{9.240635in}{2.827202in}}{\pgfqpoint{9.235591in}{2.827202in}}%
\pgfpathcurveto{\pgfqpoint{9.230547in}{2.827202in}}{\pgfqpoint{9.225709in}{2.825198in}}{\pgfqpoint{9.222143in}{2.821631in}}%
\pgfpathcurveto{\pgfqpoint{9.218577in}{2.818065in}}{\pgfqpoint{9.216573in}{2.813227in}}{\pgfqpoint{9.216573in}{2.808184in}}%
\pgfpathcurveto{\pgfqpoint{9.216573in}{2.803140in}}{\pgfqpoint{9.218577in}{2.798302in}}{\pgfqpoint{9.222143in}{2.794736in}}%
\pgfpathcurveto{\pgfqpoint{9.225709in}{2.791169in}}{\pgfqpoint{9.230547in}{2.789165in}}{\pgfqpoint{9.235591in}{2.789165in}}%
\pgfpathclose%
\pgfusepath{fill}%
\end{pgfscope}%
\begin{pgfscope}%
\pgfpathrectangle{\pgfqpoint{6.572727in}{0.474100in}}{\pgfqpoint{4.227273in}{3.318700in}}%
\pgfusepath{clip}%
\pgfsetbuttcap%
\pgfsetroundjoin%
\definecolor{currentfill}{rgb}{0.127568,0.566949,0.550556}%
\pgfsetfillcolor{currentfill}%
\pgfsetfillopacity{0.700000}%
\pgfsetlinewidth{0.000000pt}%
\definecolor{currentstroke}{rgb}{0.000000,0.000000,0.000000}%
\pgfsetstrokecolor{currentstroke}%
\pgfsetstrokeopacity{0.700000}%
\pgfsetdash{}{0pt}%
\pgfpathmoveto{\pgfqpoint{8.167309in}{1.717110in}}%
\pgfpathcurveto{\pgfqpoint{8.172353in}{1.717110in}}{\pgfqpoint{8.177190in}{1.719114in}}{\pgfqpoint{8.180757in}{1.722681in}}%
\pgfpathcurveto{\pgfqpoint{8.184323in}{1.726247in}}{\pgfqpoint{8.186327in}{1.731085in}}{\pgfqpoint{8.186327in}{1.736128in}}%
\pgfpathcurveto{\pgfqpoint{8.186327in}{1.741172in}}{\pgfqpoint{8.184323in}{1.746010in}}{\pgfqpoint{8.180757in}{1.749576in}}%
\pgfpathcurveto{\pgfqpoint{8.177190in}{1.753143in}}{\pgfqpoint{8.172353in}{1.755147in}}{\pgfqpoint{8.167309in}{1.755147in}}%
\pgfpathcurveto{\pgfqpoint{8.162265in}{1.755147in}}{\pgfqpoint{8.157427in}{1.753143in}}{\pgfqpoint{8.153861in}{1.749576in}}%
\pgfpathcurveto{\pgfqpoint{8.150295in}{1.746010in}}{\pgfqpoint{8.148291in}{1.741172in}}{\pgfqpoint{8.148291in}{1.736128in}}%
\pgfpathcurveto{\pgfqpoint{8.148291in}{1.731085in}}{\pgfqpoint{8.150295in}{1.726247in}}{\pgfqpoint{8.153861in}{1.722681in}}%
\pgfpathcurveto{\pgfqpoint{8.157427in}{1.719114in}}{\pgfqpoint{8.162265in}{1.717110in}}{\pgfqpoint{8.167309in}{1.717110in}}%
\pgfpathclose%
\pgfusepath{fill}%
\end{pgfscope}%
\begin{pgfscope}%
\pgfpathrectangle{\pgfqpoint{6.572727in}{0.474100in}}{\pgfqpoint{4.227273in}{3.318700in}}%
\pgfusepath{clip}%
\pgfsetbuttcap%
\pgfsetroundjoin%
\definecolor{currentfill}{rgb}{0.993248,0.906157,0.143936}%
\pgfsetfillcolor{currentfill}%
\pgfsetfillopacity{0.700000}%
\pgfsetlinewidth{0.000000pt}%
\definecolor{currentstroke}{rgb}{0.000000,0.000000,0.000000}%
\pgfsetstrokecolor{currentstroke}%
\pgfsetstrokeopacity{0.700000}%
\pgfsetdash{}{0pt}%
\pgfpathmoveto{\pgfqpoint{9.014533in}{1.607976in}}%
\pgfpathcurveto{\pgfqpoint{9.019576in}{1.607976in}}{\pgfqpoint{9.024414in}{1.609980in}}{\pgfqpoint{9.027981in}{1.613546in}}%
\pgfpathcurveto{\pgfqpoint{9.031547in}{1.617112in}}{\pgfqpoint{9.033551in}{1.621950in}}{\pgfqpoint{9.033551in}{1.626994in}}%
\pgfpathcurveto{\pgfqpoint{9.033551in}{1.632037in}}{\pgfqpoint{9.031547in}{1.636875in}}{\pgfqpoint{9.027981in}{1.640442in}}%
\pgfpathcurveto{\pgfqpoint{9.024414in}{1.644008in}}{\pgfqpoint{9.019576in}{1.646012in}}{\pgfqpoint{9.014533in}{1.646012in}}%
\pgfpathcurveto{\pgfqpoint{9.009489in}{1.646012in}}{\pgfqpoint{9.004651in}{1.644008in}}{\pgfqpoint{9.001085in}{1.640442in}}%
\pgfpathcurveto{\pgfqpoint{8.997518in}{1.636875in}}{\pgfqpoint{8.995515in}{1.632037in}}{\pgfqpoint{8.995515in}{1.626994in}}%
\pgfpathcurveto{\pgfqpoint{8.995515in}{1.621950in}}{\pgfqpoint{8.997518in}{1.617112in}}{\pgfqpoint{9.001085in}{1.613546in}}%
\pgfpathcurveto{\pgfqpoint{9.004651in}{1.609980in}}{\pgfqpoint{9.009489in}{1.607976in}}{\pgfqpoint{9.014533in}{1.607976in}}%
\pgfpathclose%
\pgfusepath{fill}%
\end{pgfscope}%
\begin{pgfscope}%
\pgfpathrectangle{\pgfqpoint{6.572727in}{0.474100in}}{\pgfqpoint{4.227273in}{3.318700in}}%
\pgfusepath{clip}%
\pgfsetbuttcap%
\pgfsetroundjoin%
\definecolor{currentfill}{rgb}{0.127568,0.566949,0.550556}%
\pgfsetfillcolor{currentfill}%
\pgfsetfillopacity{0.700000}%
\pgfsetlinewidth{0.000000pt}%
\definecolor{currentstroke}{rgb}{0.000000,0.000000,0.000000}%
\pgfsetstrokecolor{currentstroke}%
\pgfsetstrokeopacity{0.700000}%
\pgfsetdash{}{0pt}%
\pgfpathmoveto{\pgfqpoint{7.703718in}{1.268429in}}%
\pgfpathcurveto{\pgfqpoint{7.708762in}{1.268429in}}{\pgfqpoint{7.713600in}{1.270433in}}{\pgfqpoint{7.717166in}{1.273999in}}%
\pgfpathcurveto{\pgfqpoint{7.720733in}{1.277565in}}{\pgfqpoint{7.722737in}{1.282403in}}{\pgfqpoint{7.722737in}{1.287447in}}%
\pgfpathcurveto{\pgfqpoint{7.722737in}{1.292491in}}{\pgfqpoint{7.720733in}{1.297328in}}{\pgfqpoint{7.717166in}{1.300895in}}%
\pgfpathcurveto{\pgfqpoint{7.713600in}{1.304461in}}{\pgfqpoint{7.708762in}{1.306465in}}{\pgfqpoint{7.703718in}{1.306465in}}%
\pgfpathcurveto{\pgfqpoint{7.698675in}{1.306465in}}{\pgfqpoint{7.693837in}{1.304461in}}{\pgfqpoint{7.690271in}{1.300895in}}%
\pgfpathcurveto{\pgfqpoint{7.686704in}{1.297328in}}{\pgfqpoint{7.684700in}{1.292491in}}{\pgfqpoint{7.684700in}{1.287447in}}%
\pgfpathcurveto{\pgfqpoint{7.684700in}{1.282403in}}{\pgfqpoint{7.686704in}{1.277565in}}{\pgfqpoint{7.690271in}{1.273999in}}%
\pgfpathcurveto{\pgfqpoint{7.693837in}{1.270433in}}{\pgfqpoint{7.698675in}{1.268429in}}{\pgfqpoint{7.703718in}{1.268429in}}%
\pgfpathclose%
\pgfusepath{fill}%
\end{pgfscope}%
\begin{pgfscope}%
\pgfpathrectangle{\pgfqpoint{6.572727in}{0.474100in}}{\pgfqpoint{4.227273in}{3.318700in}}%
\pgfusepath{clip}%
\pgfsetbuttcap%
\pgfsetroundjoin%
\definecolor{currentfill}{rgb}{0.127568,0.566949,0.550556}%
\pgfsetfillcolor{currentfill}%
\pgfsetfillopacity{0.700000}%
\pgfsetlinewidth{0.000000pt}%
\definecolor{currentstroke}{rgb}{0.000000,0.000000,0.000000}%
\pgfsetstrokecolor{currentstroke}%
\pgfsetstrokeopacity{0.700000}%
\pgfsetdash{}{0pt}%
\pgfpathmoveto{\pgfqpoint{8.528957in}{2.416960in}}%
\pgfpathcurveto{\pgfqpoint{8.534000in}{2.416960in}}{\pgfqpoint{8.538838in}{2.418964in}}{\pgfqpoint{8.542405in}{2.422530in}}%
\pgfpathcurveto{\pgfqpoint{8.545971in}{2.426096in}}{\pgfqpoint{8.547975in}{2.430934in}}{\pgfqpoint{8.547975in}{2.435978in}}%
\pgfpathcurveto{\pgfqpoint{8.547975in}{2.441022in}}{\pgfqpoint{8.545971in}{2.445859in}}{\pgfqpoint{8.542405in}{2.449426in}}%
\pgfpathcurveto{\pgfqpoint{8.538838in}{2.452992in}}{\pgfqpoint{8.534000in}{2.454996in}}{\pgfqpoint{8.528957in}{2.454996in}}%
\pgfpathcurveto{\pgfqpoint{8.523913in}{2.454996in}}{\pgfqpoint{8.519075in}{2.452992in}}{\pgfqpoint{8.515509in}{2.449426in}}%
\pgfpathcurveto{\pgfqpoint{8.511943in}{2.445859in}}{\pgfqpoint{8.509939in}{2.441022in}}{\pgfqpoint{8.509939in}{2.435978in}}%
\pgfpathcurveto{\pgfqpoint{8.509939in}{2.430934in}}{\pgfqpoint{8.511943in}{2.426096in}}{\pgfqpoint{8.515509in}{2.422530in}}%
\pgfpathcurveto{\pgfqpoint{8.519075in}{2.418964in}}{\pgfqpoint{8.523913in}{2.416960in}}{\pgfqpoint{8.528957in}{2.416960in}}%
\pgfpathclose%
\pgfusepath{fill}%
\end{pgfscope}%
\begin{pgfscope}%
\pgfpathrectangle{\pgfqpoint{6.572727in}{0.474100in}}{\pgfqpoint{4.227273in}{3.318700in}}%
\pgfusepath{clip}%
\pgfsetbuttcap%
\pgfsetroundjoin%
\definecolor{currentfill}{rgb}{0.993248,0.906157,0.143936}%
\pgfsetfillcolor{currentfill}%
\pgfsetfillopacity{0.700000}%
\pgfsetlinewidth{0.000000pt}%
\definecolor{currentstroke}{rgb}{0.000000,0.000000,0.000000}%
\pgfsetstrokecolor{currentstroke}%
\pgfsetstrokeopacity{0.700000}%
\pgfsetdash{}{0pt}%
\pgfpathmoveto{\pgfqpoint{10.242300in}{1.003808in}}%
\pgfpathcurveto{\pgfqpoint{10.247344in}{1.003808in}}{\pgfqpoint{10.252181in}{1.005812in}}{\pgfqpoint{10.255748in}{1.009378in}}%
\pgfpathcurveto{\pgfqpoint{10.259314in}{1.012945in}}{\pgfqpoint{10.261318in}{1.017783in}}{\pgfqpoint{10.261318in}{1.022826in}}%
\pgfpathcurveto{\pgfqpoint{10.261318in}{1.027870in}}{\pgfqpoint{10.259314in}{1.032708in}}{\pgfqpoint{10.255748in}{1.036274in}}%
\pgfpathcurveto{\pgfqpoint{10.252181in}{1.039840in}}{\pgfqpoint{10.247344in}{1.041844in}}{\pgfqpoint{10.242300in}{1.041844in}}%
\pgfpathcurveto{\pgfqpoint{10.237256in}{1.041844in}}{\pgfqpoint{10.232419in}{1.039840in}}{\pgfqpoint{10.228852in}{1.036274in}}%
\pgfpathcurveto{\pgfqpoint{10.225286in}{1.032708in}}{\pgfqpoint{10.223282in}{1.027870in}}{\pgfqpoint{10.223282in}{1.022826in}}%
\pgfpathcurveto{\pgfqpoint{10.223282in}{1.017783in}}{\pgfqpoint{10.225286in}{1.012945in}}{\pgfqpoint{10.228852in}{1.009378in}}%
\pgfpathcurveto{\pgfqpoint{10.232419in}{1.005812in}}{\pgfqpoint{10.237256in}{1.003808in}}{\pgfqpoint{10.242300in}{1.003808in}}%
\pgfpathclose%
\pgfusepath{fill}%
\end{pgfscope}%
\begin{pgfscope}%
\pgfpathrectangle{\pgfqpoint{6.572727in}{0.474100in}}{\pgfqpoint{4.227273in}{3.318700in}}%
\pgfusepath{clip}%
\pgfsetbuttcap%
\pgfsetroundjoin%
\definecolor{currentfill}{rgb}{0.127568,0.566949,0.550556}%
\pgfsetfillcolor{currentfill}%
\pgfsetfillopacity{0.700000}%
\pgfsetlinewidth{0.000000pt}%
\definecolor{currentstroke}{rgb}{0.000000,0.000000,0.000000}%
\pgfsetstrokecolor{currentstroke}%
\pgfsetstrokeopacity{0.700000}%
\pgfsetdash{}{0pt}%
\pgfpathmoveto{\pgfqpoint{8.063184in}{1.375664in}}%
\pgfpathcurveto{\pgfqpoint{8.068228in}{1.375664in}}{\pgfqpoint{8.073065in}{1.377668in}}{\pgfqpoint{8.076632in}{1.381234in}}%
\pgfpathcurveto{\pgfqpoint{8.080198in}{1.384801in}}{\pgfqpoint{8.082202in}{1.389639in}}{\pgfqpoint{8.082202in}{1.394682in}}%
\pgfpathcurveto{\pgfqpoint{8.082202in}{1.399726in}}{\pgfqpoint{8.080198in}{1.404564in}}{\pgfqpoint{8.076632in}{1.408130in}}%
\pgfpathcurveto{\pgfqpoint{8.073065in}{1.411697in}}{\pgfqpoint{8.068228in}{1.413700in}}{\pgfqpoint{8.063184in}{1.413700in}}%
\pgfpathcurveto{\pgfqpoint{8.058140in}{1.413700in}}{\pgfqpoint{8.053303in}{1.411697in}}{\pgfqpoint{8.049736in}{1.408130in}}%
\pgfpathcurveto{\pgfqpoint{8.046170in}{1.404564in}}{\pgfqpoint{8.044166in}{1.399726in}}{\pgfqpoint{8.044166in}{1.394682in}}%
\pgfpathcurveto{\pgfqpoint{8.044166in}{1.389639in}}{\pgfqpoint{8.046170in}{1.384801in}}{\pgfqpoint{8.049736in}{1.381234in}}%
\pgfpathcurveto{\pgfqpoint{8.053303in}{1.377668in}}{\pgfqpoint{8.058140in}{1.375664in}}{\pgfqpoint{8.063184in}{1.375664in}}%
\pgfpathclose%
\pgfusepath{fill}%
\end{pgfscope}%
\begin{pgfscope}%
\pgfpathrectangle{\pgfqpoint{6.572727in}{0.474100in}}{\pgfqpoint{4.227273in}{3.318700in}}%
\pgfusepath{clip}%
\pgfsetbuttcap%
\pgfsetroundjoin%
\definecolor{currentfill}{rgb}{0.993248,0.906157,0.143936}%
\pgfsetfillcolor{currentfill}%
\pgfsetfillopacity{0.700000}%
\pgfsetlinewidth{0.000000pt}%
\definecolor{currentstroke}{rgb}{0.000000,0.000000,0.000000}%
\pgfsetstrokecolor{currentstroke}%
\pgfsetstrokeopacity{0.700000}%
\pgfsetdash{}{0pt}%
\pgfpathmoveto{\pgfqpoint{9.215122in}{1.372852in}}%
\pgfpathcurveto{\pgfqpoint{9.220166in}{1.372852in}}{\pgfqpoint{9.225004in}{1.374856in}}{\pgfqpoint{9.228570in}{1.378423in}}%
\pgfpathcurveto{\pgfqpoint{9.232137in}{1.381989in}}{\pgfqpoint{9.234140in}{1.386827in}}{\pgfqpoint{9.234140in}{1.391870in}}%
\pgfpathcurveto{\pgfqpoint{9.234140in}{1.396914in}}{\pgfqpoint{9.232137in}{1.401752in}}{\pgfqpoint{9.228570in}{1.405318in}}%
\pgfpathcurveto{\pgfqpoint{9.225004in}{1.408885in}}{\pgfqpoint{9.220166in}{1.410889in}}{\pgfqpoint{9.215122in}{1.410889in}}%
\pgfpathcurveto{\pgfqpoint{9.210079in}{1.410889in}}{\pgfqpoint{9.205241in}{1.408885in}}{\pgfqpoint{9.201674in}{1.405318in}}%
\pgfpathcurveto{\pgfqpoint{9.198108in}{1.401752in}}{\pgfqpoint{9.196104in}{1.396914in}}{\pgfqpoint{9.196104in}{1.391870in}}%
\pgfpathcurveto{\pgfqpoint{9.196104in}{1.386827in}}{\pgfqpoint{9.198108in}{1.381989in}}{\pgfqpoint{9.201674in}{1.378423in}}%
\pgfpathcurveto{\pgfqpoint{9.205241in}{1.374856in}}{\pgfqpoint{9.210079in}{1.372852in}}{\pgfqpoint{9.215122in}{1.372852in}}%
\pgfpathclose%
\pgfusepath{fill}%
\end{pgfscope}%
\begin{pgfscope}%
\pgfpathrectangle{\pgfqpoint{6.572727in}{0.474100in}}{\pgfqpoint{4.227273in}{3.318700in}}%
\pgfusepath{clip}%
\pgfsetbuttcap%
\pgfsetroundjoin%
\definecolor{currentfill}{rgb}{0.993248,0.906157,0.143936}%
\pgfsetfillcolor{currentfill}%
\pgfsetfillopacity{0.700000}%
\pgfsetlinewidth{0.000000pt}%
\definecolor{currentstroke}{rgb}{0.000000,0.000000,0.000000}%
\pgfsetstrokecolor{currentstroke}%
\pgfsetstrokeopacity{0.700000}%
\pgfsetdash{}{0pt}%
\pgfpathmoveto{\pgfqpoint{10.168488in}{0.844242in}}%
\pgfpathcurveto{\pgfqpoint{10.173531in}{0.844242in}}{\pgfqpoint{10.178369in}{0.846245in}}{\pgfqpoint{10.181935in}{0.849812in}}%
\pgfpathcurveto{\pgfqpoint{10.185502in}{0.853378in}}{\pgfqpoint{10.187506in}{0.858216in}}{\pgfqpoint{10.187506in}{0.863260in}}%
\pgfpathcurveto{\pgfqpoint{10.187506in}{0.868303in}}{\pgfqpoint{10.185502in}{0.873141in}}{\pgfqpoint{10.181935in}{0.876708in}}%
\pgfpathcurveto{\pgfqpoint{10.178369in}{0.880274in}}{\pgfqpoint{10.173531in}{0.882278in}}{\pgfqpoint{10.168488in}{0.882278in}}%
\pgfpathcurveto{\pgfqpoint{10.163444in}{0.882278in}}{\pgfqpoint{10.158606in}{0.880274in}}{\pgfqpoint{10.155040in}{0.876708in}}%
\pgfpathcurveto{\pgfqpoint{10.151473in}{0.873141in}}{\pgfqpoint{10.149469in}{0.868303in}}{\pgfqpoint{10.149469in}{0.863260in}}%
\pgfpathcurveto{\pgfqpoint{10.149469in}{0.858216in}}{\pgfqpoint{10.151473in}{0.853378in}}{\pgfqpoint{10.155040in}{0.849812in}}%
\pgfpathcurveto{\pgfqpoint{10.158606in}{0.846245in}}{\pgfqpoint{10.163444in}{0.844242in}}{\pgfqpoint{10.168488in}{0.844242in}}%
\pgfpathclose%
\pgfusepath{fill}%
\end{pgfscope}%
\begin{pgfscope}%
\pgfpathrectangle{\pgfqpoint{6.572727in}{0.474100in}}{\pgfqpoint{4.227273in}{3.318700in}}%
\pgfusepath{clip}%
\pgfsetbuttcap%
\pgfsetroundjoin%
\definecolor{currentfill}{rgb}{0.127568,0.566949,0.550556}%
\pgfsetfillcolor{currentfill}%
\pgfsetfillopacity{0.700000}%
\pgfsetlinewidth{0.000000pt}%
\definecolor{currentstroke}{rgb}{0.000000,0.000000,0.000000}%
\pgfsetstrokecolor{currentstroke}%
\pgfsetstrokeopacity{0.700000}%
\pgfsetdash{}{0pt}%
\pgfpathmoveto{\pgfqpoint{8.381077in}{2.861593in}}%
\pgfpathcurveto{\pgfqpoint{8.386120in}{2.861593in}}{\pgfqpoint{8.390958in}{2.863597in}}{\pgfqpoint{8.394524in}{2.867163in}}%
\pgfpathcurveto{\pgfqpoint{8.398091in}{2.870730in}}{\pgfqpoint{8.400095in}{2.875567in}}{\pgfqpoint{8.400095in}{2.880611in}}%
\pgfpathcurveto{\pgfqpoint{8.400095in}{2.885655in}}{\pgfqpoint{8.398091in}{2.890493in}}{\pgfqpoint{8.394524in}{2.894059in}}%
\pgfpathcurveto{\pgfqpoint{8.390958in}{2.897625in}}{\pgfqpoint{8.386120in}{2.899629in}}{\pgfqpoint{8.381077in}{2.899629in}}%
\pgfpathcurveto{\pgfqpoint{8.376033in}{2.899629in}}{\pgfqpoint{8.371195in}{2.897625in}}{\pgfqpoint{8.367629in}{2.894059in}}%
\pgfpathcurveto{\pgfqpoint{8.364062in}{2.890493in}}{\pgfqpoint{8.362058in}{2.885655in}}{\pgfqpoint{8.362058in}{2.880611in}}%
\pgfpathcurveto{\pgfqpoint{8.362058in}{2.875567in}}{\pgfqpoint{8.364062in}{2.870730in}}{\pgfqpoint{8.367629in}{2.867163in}}%
\pgfpathcurveto{\pgfqpoint{8.371195in}{2.863597in}}{\pgfqpoint{8.376033in}{2.861593in}}{\pgfqpoint{8.381077in}{2.861593in}}%
\pgfpathclose%
\pgfusepath{fill}%
\end{pgfscope}%
\begin{pgfscope}%
\pgfpathrectangle{\pgfqpoint{6.572727in}{0.474100in}}{\pgfqpoint{4.227273in}{3.318700in}}%
\pgfusepath{clip}%
\pgfsetbuttcap%
\pgfsetroundjoin%
\definecolor{currentfill}{rgb}{0.127568,0.566949,0.550556}%
\pgfsetfillcolor{currentfill}%
\pgfsetfillopacity{0.700000}%
\pgfsetlinewidth{0.000000pt}%
\definecolor{currentstroke}{rgb}{0.000000,0.000000,0.000000}%
\pgfsetstrokecolor{currentstroke}%
\pgfsetstrokeopacity{0.700000}%
\pgfsetdash{}{0pt}%
\pgfpathmoveto{\pgfqpoint{8.544121in}{2.833715in}}%
\pgfpathcurveto{\pgfqpoint{8.549165in}{2.833715in}}{\pgfqpoint{8.554003in}{2.835718in}}{\pgfqpoint{8.557569in}{2.839285in}}%
\pgfpathcurveto{\pgfqpoint{8.561135in}{2.842851in}}{\pgfqpoint{8.563139in}{2.847689in}}{\pgfqpoint{8.563139in}{2.852733in}}%
\pgfpathcurveto{\pgfqpoint{8.563139in}{2.857776in}}{\pgfqpoint{8.561135in}{2.862614in}}{\pgfqpoint{8.557569in}{2.866181in}}%
\pgfpathcurveto{\pgfqpoint{8.554003in}{2.869747in}}{\pgfqpoint{8.549165in}{2.871751in}}{\pgfqpoint{8.544121in}{2.871751in}}%
\pgfpathcurveto{\pgfqpoint{8.539077in}{2.871751in}}{\pgfqpoint{8.534240in}{2.869747in}}{\pgfqpoint{8.530673in}{2.866181in}}%
\pgfpathcurveto{\pgfqpoint{8.527107in}{2.862614in}}{\pgfqpoint{8.525103in}{2.857776in}}{\pgfqpoint{8.525103in}{2.852733in}}%
\pgfpathcurveto{\pgfqpoint{8.525103in}{2.847689in}}{\pgfqpoint{8.527107in}{2.842851in}}{\pgfqpoint{8.530673in}{2.839285in}}%
\pgfpathcurveto{\pgfqpoint{8.534240in}{2.835718in}}{\pgfqpoint{8.539077in}{2.833715in}}{\pgfqpoint{8.544121in}{2.833715in}}%
\pgfpathclose%
\pgfusepath{fill}%
\end{pgfscope}%
\begin{pgfscope}%
\pgfpathrectangle{\pgfqpoint{6.572727in}{0.474100in}}{\pgfqpoint{4.227273in}{3.318700in}}%
\pgfusepath{clip}%
\pgfsetbuttcap%
\pgfsetroundjoin%
\definecolor{currentfill}{rgb}{0.127568,0.566949,0.550556}%
\pgfsetfillcolor{currentfill}%
\pgfsetfillopacity{0.700000}%
\pgfsetlinewidth{0.000000pt}%
\definecolor{currentstroke}{rgb}{0.000000,0.000000,0.000000}%
\pgfsetstrokecolor{currentstroke}%
\pgfsetstrokeopacity{0.700000}%
\pgfsetdash{}{0pt}%
\pgfpathmoveto{\pgfqpoint{8.456689in}{2.744105in}}%
\pgfpathcurveto{\pgfqpoint{8.461733in}{2.744105in}}{\pgfqpoint{8.466571in}{2.746109in}}{\pgfqpoint{8.470137in}{2.749676in}}%
\pgfpathcurveto{\pgfqpoint{8.473704in}{2.753242in}}{\pgfqpoint{8.475708in}{2.758080in}}{\pgfqpoint{8.475708in}{2.763123in}}%
\pgfpathcurveto{\pgfqpoint{8.475708in}{2.768167in}}{\pgfqpoint{8.473704in}{2.773005in}}{\pgfqpoint{8.470137in}{2.776571in}}%
\pgfpathcurveto{\pgfqpoint{8.466571in}{2.780138in}}{\pgfqpoint{8.461733in}{2.782142in}}{\pgfqpoint{8.456689in}{2.782142in}}%
\pgfpathcurveto{\pgfqpoint{8.451646in}{2.782142in}}{\pgfqpoint{8.446808in}{2.780138in}}{\pgfqpoint{8.443242in}{2.776571in}}%
\pgfpathcurveto{\pgfqpoint{8.439675in}{2.773005in}}{\pgfqpoint{8.437671in}{2.768167in}}{\pgfqpoint{8.437671in}{2.763123in}}%
\pgfpathcurveto{\pgfqpoint{8.437671in}{2.758080in}}{\pgfqpoint{8.439675in}{2.753242in}}{\pgfqpoint{8.443242in}{2.749676in}}%
\pgfpathcurveto{\pgfqpoint{8.446808in}{2.746109in}}{\pgfqpoint{8.451646in}{2.744105in}}{\pgfqpoint{8.456689in}{2.744105in}}%
\pgfpathclose%
\pgfusepath{fill}%
\end{pgfscope}%
\begin{pgfscope}%
\pgfpathrectangle{\pgfqpoint{6.572727in}{0.474100in}}{\pgfqpoint{4.227273in}{3.318700in}}%
\pgfusepath{clip}%
\pgfsetbuttcap%
\pgfsetroundjoin%
\definecolor{currentfill}{rgb}{0.127568,0.566949,0.550556}%
\pgfsetfillcolor{currentfill}%
\pgfsetfillopacity{0.700000}%
\pgfsetlinewidth{0.000000pt}%
\definecolor{currentstroke}{rgb}{0.000000,0.000000,0.000000}%
\pgfsetstrokecolor{currentstroke}%
\pgfsetstrokeopacity{0.700000}%
\pgfsetdash{}{0pt}%
\pgfpathmoveto{\pgfqpoint{8.560071in}{3.196605in}}%
\pgfpathcurveto{\pgfqpoint{8.565115in}{3.196605in}}{\pgfqpoint{8.569952in}{3.198609in}}{\pgfqpoint{8.573519in}{3.202175in}}%
\pgfpathcurveto{\pgfqpoint{8.577085in}{3.205741in}}{\pgfqpoint{8.579089in}{3.210579in}}{\pgfqpoint{8.579089in}{3.215623in}}%
\pgfpathcurveto{\pgfqpoint{8.579089in}{3.220667in}}{\pgfqpoint{8.577085in}{3.225504in}}{\pgfqpoint{8.573519in}{3.229071in}}%
\pgfpathcurveto{\pgfqpoint{8.569952in}{3.232637in}}{\pgfqpoint{8.565115in}{3.234641in}}{\pgfqpoint{8.560071in}{3.234641in}}%
\pgfpathcurveto{\pgfqpoint{8.555027in}{3.234641in}}{\pgfqpoint{8.550189in}{3.232637in}}{\pgfqpoint{8.546623in}{3.229071in}}%
\pgfpathcurveto{\pgfqpoint{8.543057in}{3.225504in}}{\pgfqpoint{8.541053in}{3.220667in}}{\pgfqpoint{8.541053in}{3.215623in}}%
\pgfpathcurveto{\pgfqpoint{8.541053in}{3.210579in}}{\pgfqpoint{8.543057in}{3.205741in}}{\pgfqpoint{8.546623in}{3.202175in}}%
\pgfpathcurveto{\pgfqpoint{8.550189in}{3.198609in}}{\pgfqpoint{8.555027in}{3.196605in}}{\pgfqpoint{8.560071in}{3.196605in}}%
\pgfpathclose%
\pgfusepath{fill}%
\end{pgfscope}%
\begin{pgfscope}%
\pgfpathrectangle{\pgfqpoint{6.572727in}{0.474100in}}{\pgfqpoint{4.227273in}{3.318700in}}%
\pgfusepath{clip}%
\pgfsetbuttcap%
\pgfsetroundjoin%
\definecolor{currentfill}{rgb}{0.127568,0.566949,0.550556}%
\pgfsetfillcolor{currentfill}%
\pgfsetfillopacity{0.700000}%
\pgfsetlinewidth{0.000000pt}%
\definecolor{currentstroke}{rgb}{0.000000,0.000000,0.000000}%
\pgfsetstrokecolor{currentstroke}%
\pgfsetstrokeopacity{0.700000}%
\pgfsetdash{}{0pt}%
\pgfpathmoveto{\pgfqpoint{7.861425in}{1.734530in}}%
\pgfpathcurveto{\pgfqpoint{7.866468in}{1.734530in}}{\pgfqpoint{7.871306in}{1.736534in}}{\pgfqpoint{7.874873in}{1.740100in}}%
\pgfpathcurveto{\pgfqpoint{7.878439in}{1.743667in}}{\pgfqpoint{7.880443in}{1.748504in}}{\pgfqpoint{7.880443in}{1.753548in}}%
\pgfpathcurveto{\pgfqpoint{7.880443in}{1.758592in}}{\pgfqpoint{7.878439in}{1.763430in}}{\pgfqpoint{7.874873in}{1.766996in}}%
\pgfpathcurveto{\pgfqpoint{7.871306in}{1.770562in}}{\pgfqpoint{7.866468in}{1.772566in}}{\pgfqpoint{7.861425in}{1.772566in}}%
\pgfpathcurveto{\pgfqpoint{7.856381in}{1.772566in}}{\pgfqpoint{7.851543in}{1.770562in}}{\pgfqpoint{7.847977in}{1.766996in}}%
\pgfpathcurveto{\pgfqpoint{7.844410in}{1.763430in}}{\pgfqpoint{7.842407in}{1.758592in}}{\pgfqpoint{7.842407in}{1.753548in}}%
\pgfpathcurveto{\pgfqpoint{7.842407in}{1.748504in}}{\pgfqpoint{7.844410in}{1.743667in}}{\pgfqpoint{7.847977in}{1.740100in}}%
\pgfpathcurveto{\pgfqpoint{7.851543in}{1.736534in}}{\pgfqpoint{7.856381in}{1.734530in}}{\pgfqpoint{7.861425in}{1.734530in}}%
\pgfpathclose%
\pgfusepath{fill}%
\end{pgfscope}%
\begin{pgfscope}%
\pgfpathrectangle{\pgfqpoint{6.572727in}{0.474100in}}{\pgfqpoint{4.227273in}{3.318700in}}%
\pgfusepath{clip}%
\pgfsetbuttcap%
\pgfsetroundjoin%
\definecolor{currentfill}{rgb}{0.993248,0.906157,0.143936}%
\pgfsetfillcolor{currentfill}%
\pgfsetfillopacity{0.700000}%
\pgfsetlinewidth{0.000000pt}%
\definecolor{currentstroke}{rgb}{0.000000,0.000000,0.000000}%
\pgfsetstrokecolor{currentstroke}%
\pgfsetstrokeopacity{0.700000}%
\pgfsetdash{}{0pt}%
\pgfpathmoveto{\pgfqpoint{8.986997in}{1.753879in}}%
\pgfpathcurveto{\pgfqpoint{8.992041in}{1.753879in}}{\pgfqpoint{8.996879in}{1.755883in}}{\pgfqpoint{9.000445in}{1.759449in}}%
\pgfpathcurveto{\pgfqpoint{9.004012in}{1.763015in}}{\pgfqpoint{9.006016in}{1.767853in}}{\pgfqpoint{9.006016in}{1.772897in}}%
\pgfpathcurveto{\pgfqpoint{9.006016in}{1.777940in}}{\pgfqpoint{9.004012in}{1.782778in}}{\pgfqpoint{9.000445in}{1.786345in}}%
\pgfpathcurveto{\pgfqpoint{8.996879in}{1.789911in}}{\pgfqpoint{8.992041in}{1.791915in}}{\pgfqpoint{8.986997in}{1.791915in}}%
\pgfpathcurveto{\pgfqpoint{8.981954in}{1.791915in}}{\pgfqpoint{8.977116in}{1.789911in}}{\pgfqpoint{8.973550in}{1.786345in}}%
\pgfpathcurveto{\pgfqpoint{8.969983in}{1.782778in}}{\pgfqpoint{8.967979in}{1.777940in}}{\pgfqpoint{8.967979in}{1.772897in}}%
\pgfpathcurveto{\pgfqpoint{8.967979in}{1.767853in}}{\pgfqpoint{8.969983in}{1.763015in}}{\pgfqpoint{8.973550in}{1.759449in}}%
\pgfpathcurveto{\pgfqpoint{8.977116in}{1.755883in}}{\pgfqpoint{8.981954in}{1.753879in}}{\pgfqpoint{8.986997in}{1.753879in}}%
\pgfpathclose%
\pgfusepath{fill}%
\end{pgfscope}%
\begin{pgfscope}%
\pgfpathrectangle{\pgfqpoint{6.572727in}{0.474100in}}{\pgfqpoint{4.227273in}{3.318700in}}%
\pgfusepath{clip}%
\pgfsetbuttcap%
\pgfsetroundjoin%
\definecolor{currentfill}{rgb}{0.127568,0.566949,0.550556}%
\pgfsetfillcolor{currentfill}%
\pgfsetfillopacity{0.700000}%
\pgfsetlinewidth{0.000000pt}%
\definecolor{currentstroke}{rgb}{0.000000,0.000000,0.000000}%
\pgfsetstrokecolor{currentstroke}%
\pgfsetstrokeopacity{0.700000}%
\pgfsetdash{}{0pt}%
\pgfpathmoveto{\pgfqpoint{7.945923in}{2.322213in}}%
\pgfpathcurveto{\pgfqpoint{7.950966in}{2.322213in}}{\pgfqpoint{7.955804in}{2.324217in}}{\pgfqpoint{7.959370in}{2.327784in}}%
\pgfpathcurveto{\pgfqpoint{7.962937in}{2.331350in}}{\pgfqpoint{7.964941in}{2.336188in}}{\pgfqpoint{7.964941in}{2.341232in}}%
\pgfpathcurveto{\pgfqpoint{7.964941in}{2.346275in}}{\pgfqpoint{7.962937in}{2.351113in}}{\pgfqpoint{7.959370in}{2.354679in}}%
\pgfpathcurveto{\pgfqpoint{7.955804in}{2.358246in}}{\pgfqpoint{7.950966in}{2.360250in}}{\pgfqpoint{7.945923in}{2.360250in}}%
\pgfpathcurveto{\pgfqpoint{7.940879in}{2.360250in}}{\pgfqpoint{7.936041in}{2.358246in}}{\pgfqpoint{7.932475in}{2.354679in}}%
\pgfpathcurveto{\pgfqpoint{7.928908in}{2.351113in}}{\pgfqpoint{7.926904in}{2.346275in}}{\pgfqpoint{7.926904in}{2.341232in}}%
\pgfpathcurveto{\pgfqpoint{7.926904in}{2.336188in}}{\pgfqpoint{7.928908in}{2.331350in}}{\pgfqpoint{7.932475in}{2.327784in}}%
\pgfpathcurveto{\pgfqpoint{7.936041in}{2.324217in}}{\pgfqpoint{7.940879in}{2.322213in}}{\pgfqpoint{7.945923in}{2.322213in}}%
\pgfpathclose%
\pgfusepath{fill}%
\end{pgfscope}%
\begin{pgfscope}%
\pgfpathrectangle{\pgfqpoint{6.572727in}{0.474100in}}{\pgfqpoint{4.227273in}{3.318700in}}%
\pgfusepath{clip}%
\pgfsetbuttcap%
\pgfsetroundjoin%
\definecolor{currentfill}{rgb}{0.127568,0.566949,0.550556}%
\pgfsetfillcolor{currentfill}%
\pgfsetfillopacity{0.700000}%
\pgfsetlinewidth{0.000000pt}%
\definecolor{currentstroke}{rgb}{0.000000,0.000000,0.000000}%
\pgfsetstrokecolor{currentstroke}%
\pgfsetstrokeopacity{0.700000}%
\pgfsetdash{}{0pt}%
\pgfpathmoveto{\pgfqpoint{7.784445in}{3.074674in}}%
\pgfpathcurveto{\pgfqpoint{7.789489in}{3.074674in}}{\pgfqpoint{7.794326in}{3.076677in}}{\pgfqpoint{7.797893in}{3.080244in}}%
\pgfpathcurveto{\pgfqpoint{7.801459in}{3.083810in}}{\pgfqpoint{7.803463in}{3.088648in}}{\pgfqpoint{7.803463in}{3.093692in}}%
\pgfpathcurveto{\pgfqpoint{7.803463in}{3.098735in}}{\pgfqpoint{7.801459in}{3.103573in}}{\pgfqpoint{7.797893in}{3.107140in}}%
\pgfpathcurveto{\pgfqpoint{7.794326in}{3.110706in}}{\pgfqpoint{7.789489in}{3.112710in}}{\pgfqpoint{7.784445in}{3.112710in}}%
\pgfpathcurveto{\pgfqpoint{7.779401in}{3.112710in}}{\pgfqpoint{7.774563in}{3.110706in}}{\pgfqpoint{7.770997in}{3.107140in}}%
\pgfpathcurveto{\pgfqpoint{7.767431in}{3.103573in}}{\pgfqpoint{7.765427in}{3.098735in}}{\pgfqpoint{7.765427in}{3.093692in}}%
\pgfpathcurveto{\pgfqpoint{7.765427in}{3.088648in}}{\pgfqpoint{7.767431in}{3.083810in}}{\pgfqpoint{7.770997in}{3.080244in}}%
\pgfpathcurveto{\pgfqpoint{7.774563in}{3.076677in}}{\pgfqpoint{7.779401in}{3.074674in}}{\pgfqpoint{7.784445in}{3.074674in}}%
\pgfpathclose%
\pgfusepath{fill}%
\end{pgfscope}%
\begin{pgfscope}%
\pgfpathrectangle{\pgfqpoint{6.572727in}{0.474100in}}{\pgfqpoint{4.227273in}{3.318700in}}%
\pgfusepath{clip}%
\pgfsetbuttcap%
\pgfsetroundjoin%
\definecolor{currentfill}{rgb}{0.127568,0.566949,0.550556}%
\pgfsetfillcolor{currentfill}%
\pgfsetfillopacity{0.700000}%
\pgfsetlinewidth{0.000000pt}%
\definecolor{currentstroke}{rgb}{0.000000,0.000000,0.000000}%
\pgfsetstrokecolor{currentstroke}%
\pgfsetstrokeopacity{0.700000}%
\pgfsetdash{}{0pt}%
\pgfpathmoveto{\pgfqpoint{7.712566in}{1.588039in}}%
\pgfpathcurveto{\pgfqpoint{7.717610in}{1.588039in}}{\pgfqpoint{7.722448in}{1.590043in}}{\pgfqpoint{7.726014in}{1.593610in}}%
\pgfpathcurveto{\pgfqpoint{7.729581in}{1.597176in}}{\pgfqpoint{7.731585in}{1.602014in}}{\pgfqpoint{7.731585in}{1.607057in}}%
\pgfpathcurveto{\pgfqpoint{7.731585in}{1.612101in}}{\pgfqpoint{7.729581in}{1.616939in}}{\pgfqpoint{7.726014in}{1.620505in}}%
\pgfpathcurveto{\pgfqpoint{7.722448in}{1.624072in}}{\pgfqpoint{7.717610in}{1.626076in}}{\pgfqpoint{7.712566in}{1.626076in}}%
\pgfpathcurveto{\pgfqpoint{7.707523in}{1.626076in}}{\pgfqpoint{7.702685in}{1.624072in}}{\pgfqpoint{7.699119in}{1.620505in}}%
\pgfpathcurveto{\pgfqpoint{7.695552in}{1.616939in}}{\pgfqpoint{7.693548in}{1.612101in}}{\pgfqpoint{7.693548in}{1.607057in}}%
\pgfpathcurveto{\pgfqpoint{7.693548in}{1.602014in}}{\pgfqpoint{7.695552in}{1.597176in}}{\pgfqpoint{7.699119in}{1.593610in}}%
\pgfpathcurveto{\pgfqpoint{7.702685in}{1.590043in}}{\pgfqpoint{7.707523in}{1.588039in}}{\pgfqpoint{7.712566in}{1.588039in}}%
\pgfpathclose%
\pgfusepath{fill}%
\end{pgfscope}%
\begin{pgfscope}%
\pgfpathrectangle{\pgfqpoint{6.572727in}{0.474100in}}{\pgfqpoint{4.227273in}{3.318700in}}%
\pgfusepath{clip}%
\pgfsetbuttcap%
\pgfsetroundjoin%
\definecolor{currentfill}{rgb}{0.127568,0.566949,0.550556}%
\pgfsetfillcolor{currentfill}%
\pgfsetfillopacity{0.700000}%
\pgfsetlinewidth{0.000000pt}%
\definecolor{currentstroke}{rgb}{0.000000,0.000000,0.000000}%
\pgfsetstrokecolor{currentstroke}%
\pgfsetstrokeopacity{0.700000}%
\pgfsetdash{}{0pt}%
\pgfpathmoveto{\pgfqpoint{8.301600in}{1.605560in}}%
\pgfpathcurveto{\pgfqpoint{8.306644in}{1.605560in}}{\pgfqpoint{8.311482in}{1.607564in}}{\pgfqpoint{8.315048in}{1.611130in}}%
\pgfpathcurveto{\pgfqpoint{8.318615in}{1.614697in}}{\pgfqpoint{8.320619in}{1.619534in}}{\pgfqpoint{8.320619in}{1.624578in}}%
\pgfpathcurveto{\pgfqpoint{8.320619in}{1.629622in}}{\pgfqpoint{8.318615in}{1.634460in}}{\pgfqpoint{8.315048in}{1.638026in}}%
\pgfpathcurveto{\pgfqpoint{8.311482in}{1.641592in}}{\pgfqpoint{8.306644in}{1.643596in}}{\pgfqpoint{8.301600in}{1.643596in}}%
\pgfpathcurveto{\pgfqpoint{8.296557in}{1.643596in}}{\pgfqpoint{8.291719in}{1.641592in}}{\pgfqpoint{8.288153in}{1.638026in}}%
\pgfpathcurveto{\pgfqpoint{8.284586in}{1.634460in}}{\pgfqpoint{8.282582in}{1.629622in}}{\pgfqpoint{8.282582in}{1.624578in}}%
\pgfpathcurveto{\pgfqpoint{8.282582in}{1.619534in}}{\pgfqpoint{8.284586in}{1.614697in}}{\pgfqpoint{8.288153in}{1.611130in}}%
\pgfpathcurveto{\pgfqpoint{8.291719in}{1.607564in}}{\pgfqpoint{8.296557in}{1.605560in}}{\pgfqpoint{8.301600in}{1.605560in}}%
\pgfpathclose%
\pgfusepath{fill}%
\end{pgfscope}%
\begin{pgfscope}%
\pgfpathrectangle{\pgfqpoint{6.572727in}{0.474100in}}{\pgfqpoint{4.227273in}{3.318700in}}%
\pgfusepath{clip}%
\pgfsetbuttcap%
\pgfsetroundjoin%
\definecolor{currentfill}{rgb}{0.127568,0.566949,0.550556}%
\pgfsetfillcolor{currentfill}%
\pgfsetfillopacity{0.700000}%
\pgfsetlinewidth{0.000000pt}%
\definecolor{currentstroke}{rgb}{0.000000,0.000000,0.000000}%
\pgfsetstrokecolor{currentstroke}%
\pgfsetstrokeopacity{0.700000}%
\pgfsetdash{}{0pt}%
\pgfpathmoveto{\pgfqpoint{8.559669in}{2.595644in}}%
\pgfpathcurveto{\pgfqpoint{8.564713in}{2.595644in}}{\pgfqpoint{8.569551in}{2.597648in}}{\pgfqpoint{8.573117in}{2.601214in}}%
\pgfpathcurveto{\pgfqpoint{8.576683in}{2.604781in}}{\pgfqpoint{8.578687in}{2.609619in}}{\pgfqpoint{8.578687in}{2.614662in}}%
\pgfpathcurveto{\pgfqpoint{8.578687in}{2.619706in}}{\pgfqpoint{8.576683in}{2.624544in}}{\pgfqpoint{8.573117in}{2.628110in}}%
\pgfpathcurveto{\pgfqpoint{8.569551in}{2.631676in}}{\pgfqpoint{8.564713in}{2.633680in}}{\pgfqpoint{8.559669in}{2.633680in}}%
\pgfpathcurveto{\pgfqpoint{8.554625in}{2.633680in}}{\pgfqpoint{8.549788in}{2.631676in}}{\pgfqpoint{8.546221in}{2.628110in}}%
\pgfpathcurveto{\pgfqpoint{8.542655in}{2.624544in}}{\pgfqpoint{8.540651in}{2.619706in}}{\pgfqpoint{8.540651in}{2.614662in}}%
\pgfpathcurveto{\pgfqpoint{8.540651in}{2.609619in}}{\pgfqpoint{8.542655in}{2.604781in}}{\pgfqpoint{8.546221in}{2.601214in}}%
\pgfpathcurveto{\pgfqpoint{8.549788in}{2.597648in}}{\pgfqpoint{8.554625in}{2.595644in}}{\pgfqpoint{8.559669in}{2.595644in}}%
\pgfpathclose%
\pgfusepath{fill}%
\end{pgfscope}%
\begin{pgfscope}%
\pgfpathrectangle{\pgfqpoint{6.572727in}{0.474100in}}{\pgfqpoint{4.227273in}{3.318700in}}%
\pgfusepath{clip}%
\pgfsetbuttcap%
\pgfsetroundjoin%
\definecolor{currentfill}{rgb}{0.127568,0.566949,0.550556}%
\pgfsetfillcolor{currentfill}%
\pgfsetfillopacity{0.700000}%
\pgfsetlinewidth{0.000000pt}%
\definecolor{currentstroke}{rgb}{0.000000,0.000000,0.000000}%
\pgfsetstrokecolor{currentstroke}%
\pgfsetstrokeopacity{0.700000}%
\pgfsetdash{}{0pt}%
\pgfpathmoveto{\pgfqpoint{7.904852in}{1.890821in}}%
\pgfpathcurveto{\pgfqpoint{7.909896in}{1.890821in}}{\pgfqpoint{7.914733in}{1.892825in}}{\pgfqpoint{7.918300in}{1.896391in}}%
\pgfpathcurveto{\pgfqpoint{7.921866in}{1.899957in}}{\pgfqpoint{7.923870in}{1.904795in}}{\pgfqpoint{7.923870in}{1.909839in}}%
\pgfpathcurveto{\pgfqpoint{7.923870in}{1.914883in}}{\pgfqpoint{7.921866in}{1.919720in}}{\pgfqpoint{7.918300in}{1.923287in}}%
\pgfpathcurveto{\pgfqpoint{7.914733in}{1.926853in}}{\pgfqpoint{7.909896in}{1.928857in}}{\pgfqpoint{7.904852in}{1.928857in}}%
\pgfpathcurveto{\pgfqpoint{7.899808in}{1.928857in}}{\pgfqpoint{7.894970in}{1.926853in}}{\pgfqpoint{7.891404in}{1.923287in}}%
\pgfpathcurveto{\pgfqpoint{7.887838in}{1.919720in}}{\pgfqpoint{7.885834in}{1.914883in}}{\pgfqpoint{7.885834in}{1.909839in}}%
\pgfpathcurveto{\pgfqpoint{7.885834in}{1.904795in}}{\pgfqpoint{7.887838in}{1.899957in}}{\pgfqpoint{7.891404in}{1.896391in}}%
\pgfpathcurveto{\pgfqpoint{7.894970in}{1.892825in}}{\pgfqpoint{7.899808in}{1.890821in}}{\pgfqpoint{7.904852in}{1.890821in}}%
\pgfpathclose%
\pgfusepath{fill}%
\end{pgfscope}%
\begin{pgfscope}%
\pgfpathrectangle{\pgfqpoint{6.572727in}{0.474100in}}{\pgfqpoint{4.227273in}{3.318700in}}%
\pgfusepath{clip}%
\pgfsetbuttcap%
\pgfsetroundjoin%
\definecolor{currentfill}{rgb}{0.993248,0.906157,0.143936}%
\pgfsetfillcolor{currentfill}%
\pgfsetfillopacity{0.700000}%
\pgfsetlinewidth{0.000000pt}%
\definecolor{currentstroke}{rgb}{0.000000,0.000000,0.000000}%
\pgfsetstrokecolor{currentstroke}%
\pgfsetstrokeopacity{0.700000}%
\pgfsetdash{}{0pt}%
\pgfpathmoveto{\pgfqpoint{10.120906in}{1.422941in}}%
\pgfpathcurveto{\pgfqpoint{10.125950in}{1.422941in}}{\pgfqpoint{10.130788in}{1.424945in}}{\pgfqpoint{10.134354in}{1.428511in}}%
\pgfpathcurveto{\pgfqpoint{10.137921in}{1.432078in}}{\pgfqpoint{10.139925in}{1.436915in}}{\pgfqpoint{10.139925in}{1.441959in}}%
\pgfpathcurveto{\pgfqpoint{10.139925in}{1.447003in}}{\pgfqpoint{10.137921in}{1.451841in}}{\pgfqpoint{10.134354in}{1.455407in}}%
\pgfpathcurveto{\pgfqpoint{10.130788in}{1.458973in}}{\pgfqpoint{10.125950in}{1.460977in}}{\pgfqpoint{10.120906in}{1.460977in}}%
\pgfpathcurveto{\pgfqpoint{10.115863in}{1.460977in}}{\pgfqpoint{10.111025in}{1.458973in}}{\pgfqpoint{10.107459in}{1.455407in}}%
\pgfpathcurveto{\pgfqpoint{10.103892in}{1.451841in}}{\pgfqpoint{10.101888in}{1.447003in}}{\pgfqpoint{10.101888in}{1.441959in}}%
\pgfpathcurveto{\pgfqpoint{10.101888in}{1.436915in}}{\pgfqpoint{10.103892in}{1.432078in}}{\pgfqpoint{10.107459in}{1.428511in}}%
\pgfpathcurveto{\pgfqpoint{10.111025in}{1.424945in}}{\pgfqpoint{10.115863in}{1.422941in}}{\pgfqpoint{10.120906in}{1.422941in}}%
\pgfpathclose%
\pgfusepath{fill}%
\end{pgfscope}%
\begin{pgfscope}%
\pgfpathrectangle{\pgfqpoint{6.572727in}{0.474100in}}{\pgfqpoint{4.227273in}{3.318700in}}%
\pgfusepath{clip}%
\pgfsetbuttcap%
\pgfsetroundjoin%
\definecolor{currentfill}{rgb}{0.993248,0.906157,0.143936}%
\pgfsetfillcolor{currentfill}%
\pgfsetfillopacity{0.700000}%
\pgfsetlinewidth{0.000000pt}%
\definecolor{currentstroke}{rgb}{0.000000,0.000000,0.000000}%
\pgfsetstrokecolor{currentstroke}%
\pgfsetstrokeopacity{0.700000}%
\pgfsetdash{}{0pt}%
\pgfpathmoveto{\pgfqpoint{9.162532in}{1.359841in}}%
\pgfpathcurveto{\pgfqpoint{9.167575in}{1.359841in}}{\pgfqpoint{9.172413in}{1.361845in}}{\pgfqpoint{9.175979in}{1.365411in}}%
\pgfpathcurveto{\pgfqpoint{9.179546in}{1.368977in}}{\pgfqpoint{9.181550in}{1.373815in}}{\pgfqpoint{9.181550in}{1.378859in}}%
\pgfpathcurveto{\pgfqpoint{9.181550in}{1.383902in}}{\pgfqpoint{9.179546in}{1.388740in}}{\pgfqpoint{9.175979in}{1.392307in}}%
\pgfpathcurveto{\pgfqpoint{9.172413in}{1.395873in}}{\pgfqpoint{9.167575in}{1.397877in}}{\pgfqpoint{9.162532in}{1.397877in}}%
\pgfpathcurveto{\pgfqpoint{9.157488in}{1.397877in}}{\pgfqpoint{9.152650in}{1.395873in}}{\pgfqpoint{9.149084in}{1.392307in}}%
\pgfpathcurveto{\pgfqpoint{9.145517in}{1.388740in}}{\pgfqpoint{9.143513in}{1.383902in}}{\pgfqpoint{9.143513in}{1.378859in}}%
\pgfpathcurveto{\pgfqpoint{9.143513in}{1.373815in}}{\pgfqpoint{9.145517in}{1.368977in}}{\pgfqpoint{9.149084in}{1.365411in}}%
\pgfpathcurveto{\pgfqpoint{9.152650in}{1.361845in}}{\pgfqpoint{9.157488in}{1.359841in}}{\pgfqpoint{9.162532in}{1.359841in}}%
\pgfpathclose%
\pgfusepath{fill}%
\end{pgfscope}%
\begin{pgfscope}%
\pgfpathrectangle{\pgfqpoint{6.572727in}{0.474100in}}{\pgfqpoint{4.227273in}{3.318700in}}%
\pgfusepath{clip}%
\pgfsetbuttcap%
\pgfsetroundjoin%
\definecolor{currentfill}{rgb}{0.127568,0.566949,0.550556}%
\pgfsetfillcolor{currentfill}%
\pgfsetfillopacity{0.700000}%
\pgfsetlinewidth{0.000000pt}%
\definecolor{currentstroke}{rgb}{0.000000,0.000000,0.000000}%
\pgfsetstrokecolor{currentstroke}%
\pgfsetstrokeopacity{0.700000}%
\pgfsetdash{}{0pt}%
\pgfpathmoveto{\pgfqpoint{8.793988in}{2.119819in}}%
\pgfpathcurveto{\pgfqpoint{8.799031in}{2.119819in}}{\pgfqpoint{8.803869in}{2.121823in}}{\pgfqpoint{8.807436in}{2.125389in}}%
\pgfpathcurveto{\pgfqpoint{8.811002in}{2.128956in}}{\pgfqpoint{8.813006in}{2.133794in}}{\pgfqpoint{8.813006in}{2.138837in}}%
\pgfpathcurveto{\pgfqpoint{8.813006in}{2.143881in}}{\pgfqpoint{8.811002in}{2.148719in}}{\pgfqpoint{8.807436in}{2.152285in}}%
\pgfpathcurveto{\pgfqpoint{8.803869in}{2.155852in}}{\pgfqpoint{8.799031in}{2.157855in}}{\pgfqpoint{8.793988in}{2.157855in}}%
\pgfpathcurveto{\pgfqpoint{8.788944in}{2.157855in}}{\pgfqpoint{8.784106in}{2.155852in}}{\pgfqpoint{8.780540in}{2.152285in}}%
\pgfpathcurveto{\pgfqpoint{8.776973in}{2.148719in}}{\pgfqpoint{8.774970in}{2.143881in}}{\pgfqpoint{8.774970in}{2.138837in}}%
\pgfpathcurveto{\pgfqpoint{8.774970in}{2.133794in}}{\pgfqpoint{8.776973in}{2.128956in}}{\pgfqpoint{8.780540in}{2.125389in}}%
\pgfpathcurveto{\pgfqpoint{8.784106in}{2.121823in}}{\pgfqpoint{8.788944in}{2.119819in}}{\pgfqpoint{8.793988in}{2.119819in}}%
\pgfpathclose%
\pgfusepath{fill}%
\end{pgfscope}%
\begin{pgfscope}%
\pgfpathrectangle{\pgfqpoint{6.572727in}{0.474100in}}{\pgfqpoint{4.227273in}{3.318700in}}%
\pgfusepath{clip}%
\pgfsetbuttcap%
\pgfsetroundjoin%
\definecolor{currentfill}{rgb}{0.993248,0.906157,0.143936}%
\pgfsetfillcolor{currentfill}%
\pgfsetfillopacity{0.700000}%
\pgfsetlinewidth{0.000000pt}%
\definecolor{currentstroke}{rgb}{0.000000,0.000000,0.000000}%
\pgfsetstrokecolor{currentstroke}%
\pgfsetstrokeopacity{0.700000}%
\pgfsetdash{}{0pt}%
\pgfpathmoveto{\pgfqpoint{9.683772in}{2.002159in}}%
\pgfpathcurveto{\pgfqpoint{9.688815in}{2.002159in}}{\pgfqpoint{9.693653in}{2.004163in}}{\pgfqpoint{9.697220in}{2.007730in}}%
\pgfpathcurveto{\pgfqpoint{9.700786in}{2.011296in}}{\pgfqpoint{9.702790in}{2.016134in}}{\pgfqpoint{9.702790in}{2.021177in}}%
\pgfpathcurveto{\pgfqpoint{9.702790in}{2.026221in}}{\pgfqpoint{9.700786in}{2.031059in}}{\pgfqpoint{9.697220in}{2.034625in}}%
\pgfpathcurveto{\pgfqpoint{9.693653in}{2.038192in}}{\pgfqpoint{9.688815in}{2.040196in}}{\pgfqpoint{9.683772in}{2.040196in}}%
\pgfpathcurveto{\pgfqpoint{9.678728in}{2.040196in}}{\pgfqpoint{9.673890in}{2.038192in}}{\pgfqpoint{9.670324in}{2.034625in}}%
\pgfpathcurveto{\pgfqpoint{9.666758in}{2.031059in}}{\pgfqpoint{9.664754in}{2.026221in}}{\pgfqpoint{9.664754in}{2.021177in}}%
\pgfpathcurveto{\pgfqpoint{9.664754in}{2.016134in}}{\pgfqpoint{9.666758in}{2.011296in}}{\pgfqpoint{9.670324in}{2.007730in}}%
\pgfpathcurveto{\pgfqpoint{9.673890in}{2.004163in}}{\pgfqpoint{9.678728in}{2.002159in}}{\pgfqpoint{9.683772in}{2.002159in}}%
\pgfpathclose%
\pgfusepath{fill}%
\end{pgfscope}%
\begin{pgfscope}%
\pgfpathrectangle{\pgfqpoint{6.572727in}{0.474100in}}{\pgfqpoint{4.227273in}{3.318700in}}%
\pgfusepath{clip}%
\pgfsetbuttcap%
\pgfsetroundjoin%
\definecolor{currentfill}{rgb}{0.127568,0.566949,0.550556}%
\pgfsetfillcolor{currentfill}%
\pgfsetfillopacity{0.700000}%
\pgfsetlinewidth{0.000000pt}%
\definecolor{currentstroke}{rgb}{0.000000,0.000000,0.000000}%
\pgfsetstrokecolor{currentstroke}%
\pgfsetstrokeopacity{0.700000}%
\pgfsetdash{}{0pt}%
\pgfpathmoveto{\pgfqpoint{8.239047in}{2.867919in}}%
\pgfpathcurveto{\pgfqpoint{8.244091in}{2.867919in}}{\pgfqpoint{8.248928in}{2.869923in}}{\pgfqpoint{8.252495in}{2.873490in}}%
\pgfpathcurveto{\pgfqpoint{8.256061in}{2.877056in}}{\pgfqpoint{8.258065in}{2.881894in}}{\pgfqpoint{8.258065in}{2.886938in}}%
\pgfpathcurveto{\pgfqpoint{8.258065in}{2.891981in}}{\pgfqpoint{8.256061in}{2.896819in}}{\pgfqpoint{8.252495in}{2.900385in}}%
\pgfpathcurveto{\pgfqpoint{8.248928in}{2.903952in}}{\pgfqpoint{8.244091in}{2.905956in}}{\pgfqpoint{8.239047in}{2.905956in}}%
\pgfpathcurveto{\pgfqpoint{8.234003in}{2.905956in}}{\pgfqpoint{8.229165in}{2.903952in}}{\pgfqpoint{8.225599in}{2.900385in}}%
\pgfpathcurveto{\pgfqpoint{8.222033in}{2.896819in}}{\pgfqpoint{8.220029in}{2.891981in}}{\pgfqpoint{8.220029in}{2.886938in}}%
\pgfpathcurveto{\pgfqpoint{8.220029in}{2.881894in}}{\pgfqpoint{8.222033in}{2.877056in}}{\pgfqpoint{8.225599in}{2.873490in}}%
\pgfpathcurveto{\pgfqpoint{8.229165in}{2.869923in}}{\pgfqpoint{8.234003in}{2.867919in}}{\pgfqpoint{8.239047in}{2.867919in}}%
\pgfpathclose%
\pgfusepath{fill}%
\end{pgfscope}%
\begin{pgfscope}%
\pgfpathrectangle{\pgfqpoint{6.572727in}{0.474100in}}{\pgfqpoint{4.227273in}{3.318700in}}%
\pgfusepath{clip}%
\pgfsetbuttcap%
\pgfsetroundjoin%
\definecolor{currentfill}{rgb}{0.127568,0.566949,0.550556}%
\pgfsetfillcolor{currentfill}%
\pgfsetfillopacity{0.700000}%
\pgfsetlinewidth{0.000000pt}%
\definecolor{currentstroke}{rgb}{0.000000,0.000000,0.000000}%
\pgfsetstrokecolor{currentstroke}%
\pgfsetstrokeopacity{0.700000}%
\pgfsetdash{}{0pt}%
\pgfpathmoveto{\pgfqpoint{8.050045in}{1.186612in}}%
\pgfpathcurveto{\pgfqpoint{8.055088in}{1.186612in}}{\pgfqpoint{8.059926in}{1.188616in}}{\pgfqpoint{8.063492in}{1.192182in}}%
\pgfpathcurveto{\pgfqpoint{8.067059in}{1.195749in}}{\pgfqpoint{8.069063in}{1.200586in}}{\pgfqpoint{8.069063in}{1.205630in}}%
\pgfpathcurveto{\pgfqpoint{8.069063in}{1.210674in}}{\pgfqpoint{8.067059in}{1.215511in}}{\pgfqpoint{8.063492in}{1.219078in}}%
\pgfpathcurveto{\pgfqpoint{8.059926in}{1.222644in}}{\pgfqpoint{8.055088in}{1.224648in}}{\pgfqpoint{8.050045in}{1.224648in}}%
\pgfpathcurveto{\pgfqpoint{8.045001in}{1.224648in}}{\pgfqpoint{8.040163in}{1.222644in}}{\pgfqpoint{8.036597in}{1.219078in}}%
\pgfpathcurveto{\pgfqpoint{8.033030in}{1.215511in}}{\pgfqpoint{8.031026in}{1.210674in}}{\pgfqpoint{8.031026in}{1.205630in}}%
\pgfpathcurveto{\pgfqpoint{8.031026in}{1.200586in}}{\pgfqpoint{8.033030in}{1.195749in}}{\pgfqpoint{8.036597in}{1.192182in}}%
\pgfpathcurveto{\pgfqpoint{8.040163in}{1.188616in}}{\pgfqpoint{8.045001in}{1.186612in}}{\pgfqpoint{8.050045in}{1.186612in}}%
\pgfpathclose%
\pgfusepath{fill}%
\end{pgfscope}%
\begin{pgfscope}%
\pgfpathrectangle{\pgfqpoint{6.572727in}{0.474100in}}{\pgfqpoint{4.227273in}{3.318700in}}%
\pgfusepath{clip}%
\pgfsetbuttcap%
\pgfsetroundjoin%
\definecolor{currentfill}{rgb}{0.993248,0.906157,0.143936}%
\pgfsetfillcolor{currentfill}%
\pgfsetfillopacity{0.700000}%
\pgfsetlinewidth{0.000000pt}%
\definecolor{currentstroke}{rgb}{0.000000,0.000000,0.000000}%
\pgfsetstrokecolor{currentstroke}%
\pgfsetstrokeopacity{0.700000}%
\pgfsetdash{}{0pt}%
\pgfpathmoveto{\pgfqpoint{9.711457in}{1.934637in}}%
\pgfpathcurveto{\pgfqpoint{9.716501in}{1.934637in}}{\pgfqpoint{9.721339in}{1.936641in}}{\pgfqpoint{9.724905in}{1.940208in}}%
\pgfpathcurveto{\pgfqpoint{9.728471in}{1.943774in}}{\pgfqpoint{9.730475in}{1.948612in}}{\pgfqpoint{9.730475in}{1.953655in}}%
\pgfpathcurveto{\pgfqpoint{9.730475in}{1.958699in}}{\pgfqpoint{9.728471in}{1.963537in}}{\pgfqpoint{9.724905in}{1.967103in}}%
\pgfpathcurveto{\pgfqpoint{9.721339in}{1.970670in}}{\pgfqpoint{9.716501in}{1.972674in}}{\pgfqpoint{9.711457in}{1.972674in}}%
\pgfpathcurveto{\pgfqpoint{9.706413in}{1.972674in}}{\pgfqpoint{9.701576in}{1.970670in}}{\pgfqpoint{9.698009in}{1.967103in}}%
\pgfpathcurveto{\pgfqpoint{9.694443in}{1.963537in}}{\pgfqpoint{9.692439in}{1.958699in}}{\pgfqpoint{9.692439in}{1.953655in}}%
\pgfpathcurveto{\pgfqpoint{9.692439in}{1.948612in}}{\pgfqpoint{9.694443in}{1.943774in}}{\pgfqpoint{9.698009in}{1.940208in}}%
\pgfpathcurveto{\pgfqpoint{9.701576in}{1.936641in}}{\pgfqpoint{9.706413in}{1.934637in}}{\pgfqpoint{9.711457in}{1.934637in}}%
\pgfpathclose%
\pgfusepath{fill}%
\end{pgfscope}%
\begin{pgfscope}%
\pgfpathrectangle{\pgfqpoint{6.572727in}{0.474100in}}{\pgfqpoint{4.227273in}{3.318700in}}%
\pgfusepath{clip}%
\pgfsetbuttcap%
\pgfsetroundjoin%
\definecolor{currentfill}{rgb}{0.127568,0.566949,0.550556}%
\pgfsetfillcolor{currentfill}%
\pgfsetfillopacity{0.700000}%
\pgfsetlinewidth{0.000000pt}%
\definecolor{currentstroke}{rgb}{0.000000,0.000000,0.000000}%
\pgfsetstrokecolor{currentstroke}%
\pgfsetstrokeopacity{0.700000}%
\pgfsetdash{}{0pt}%
\pgfpathmoveto{\pgfqpoint{7.683891in}{1.469409in}}%
\pgfpathcurveto{\pgfqpoint{7.688935in}{1.469409in}}{\pgfqpoint{7.693773in}{1.471412in}}{\pgfqpoint{7.697339in}{1.474979in}}%
\pgfpathcurveto{\pgfqpoint{7.700906in}{1.478545in}}{\pgfqpoint{7.702909in}{1.483383in}}{\pgfqpoint{7.702909in}{1.488427in}}%
\pgfpathcurveto{\pgfqpoint{7.702909in}{1.493470in}}{\pgfqpoint{7.700906in}{1.498308in}}{\pgfqpoint{7.697339in}{1.501875in}}%
\pgfpathcurveto{\pgfqpoint{7.693773in}{1.505441in}}{\pgfqpoint{7.688935in}{1.507445in}}{\pgfqpoint{7.683891in}{1.507445in}}%
\pgfpathcurveto{\pgfqpoint{7.678848in}{1.507445in}}{\pgfqpoint{7.674010in}{1.505441in}}{\pgfqpoint{7.670443in}{1.501875in}}%
\pgfpathcurveto{\pgfqpoint{7.666877in}{1.498308in}}{\pgfqpoint{7.664873in}{1.493470in}}{\pgfqpoint{7.664873in}{1.488427in}}%
\pgfpathcurveto{\pgfqpoint{7.664873in}{1.483383in}}{\pgfqpoint{7.666877in}{1.478545in}}{\pgfqpoint{7.670443in}{1.474979in}}%
\pgfpathcurveto{\pgfqpoint{7.674010in}{1.471412in}}{\pgfqpoint{7.678848in}{1.469409in}}{\pgfqpoint{7.683891in}{1.469409in}}%
\pgfpathclose%
\pgfusepath{fill}%
\end{pgfscope}%
\begin{pgfscope}%
\pgfpathrectangle{\pgfqpoint{6.572727in}{0.474100in}}{\pgfqpoint{4.227273in}{3.318700in}}%
\pgfusepath{clip}%
\pgfsetbuttcap%
\pgfsetroundjoin%
\definecolor{currentfill}{rgb}{0.127568,0.566949,0.550556}%
\pgfsetfillcolor{currentfill}%
\pgfsetfillopacity{0.700000}%
\pgfsetlinewidth{0.000000pt}%
\definecolor{currentstroke}{rgb}{0.000000,0.000000,0.000000}%
\pgfsetstrokecolor{currentstroke}%
\pgfsetstrokeopacity{0.700000}%
\pgfsetdash{}{0pt}%
\pgfpathmoveto{\pgfqpoint{8.415927in}{2.298801in}}%
\pgfpathcurveto{\pgfqpoint{8.420970in}{2.298801in}}{\pgfqpoint{8.425808in}{2.300804in}}{\pgfqpoint{8.429374in}{2.304371in}}%
\pgfpathcurveto{\pgfqpoint{8.432941in}{2.307937in}}{\pgfqpoint{8.434945in}{2.312775in}}{\pgfqpoint{8.434945in}{2.317819in}}%
\pgfpathcurveto{\pgfqpoint{8.434945in}{2.322862in}}{\pgfqpoint{8.432941in}{2.327700in}}{\pgfqpoint{8.429374in}{2.331267in}}%
\pgfpathcurveto{\pgfqpoint{8.425808in}{2.334833in}}{\pgfqpoint{8.420970in}{2.336837in}}{\pgfqpoint{8.415927in}{2.336837in}}%
\pgfpathcurveto{\pgfqpoint{8.410883in}{2.336837in}}{\pgfqpoint{8.406045in}{2.334833in}}{\pgfqpoint{8.402479in}{2.331267in}}%
\pgfpathcurveto{\pgfqpoint{8.398912in}{2.327700in}}{\pgfqpoint{8.396908in}{2.322862in}}{\pgfqpoint{8.396908in}{2.317819in}}%
\pgfpathcurveto{\pgfqpoint{8.396908in}{2.312775in}}{\pgfqpoint{8.398912in}{2.307937in}}{\pgfqpoint{8.402479in}{2.304371in}}%
\pgfpathcurveto{\pgfqpoint{8.406045in}{2.300804in}}{\pgfqpoint{8.410883in}{2.298801in}}{\pgfqpoint{8.415927in}{2.298801in}}%
\pgfpathclose%
\pgfusepath{fill}%
\end{pgfscope}%
\begin{pgfscope}%
\pgfpathrectangle{\pgfqpoint{6.572727in}{0.474100in}}{\pgfqpoint{4.227273in}{3.318700in}}%
\pgfusepath{clip}%
\pgfsetbuttcap%
\pgfsetroundjoin%
\definecolor{currentfill}{rgb}{0.127568,0.566949,0.550556}%
\pgfsetfillcolor{currentfill}%
\pgfsetfillopacity{0.700000}%
\pgfsetlinewidth{0.000000pt}%
\definecolor{currentstroke}{rgb}{0.000000,0.000000,0.000000}%
\pgfsetstrokecolor{currentstroke}%
\pgfsetstrokeopacity{0.700000}%
\pgfsetdash{}{0pt}%
\pgfpathmoveto{\pgfqpoint{7.992050in}{2.406834in}}%
\pgfpathcurveto{\pgfqpoint{7.997094in}{2.406834in}}{\pgfqpoint{8.001931in}{2.408838in}}{\pgfqpoint{8.005498in}{2.412404in}}%
\pgfpathcurveto{\pgfqpoint{8.009064in}{2.415970in}}{\pgfqpoint{8.011068in}{2.420808in}}{\pgfqpoint{8.011068in}{2.425852in}}%
\pgfpathcurveto{\pgfqpoint{8.011068in}{2.430896in}}{\pgfqpoint{8.009064in}{2.435733in}}{\pgfqpoint{8.005498in}{2.439300in}}%
\pgfpathcurveto{\pgfqpoint{8.001931in}{2.442866in}}{\pgfqpoint{7.997094in}{2.444870in}}{\pgfqpoint{7.992050in}{2.444870in}}%
\pgfpathcurveto{\pgfqpoint{7.987006in}{2.444870in}}{\pgfqpoint{7.982168in}{2.442866in}}{\pgfqpoint{7.978602in}{2.439300in}}%
\pgfpathcurveto{\pgfqpoint{7.975036in}{2.435733in}}{\pgfqpoint{7.973032in}{2.430896in}}{\pgfqpoint{7.973032in}{2.425852in}}%
\pgfpathcurveto{\pgfqpoint{7.973032in}{2.420808in}}{\pgfqpoint{7.975036in}{2.415970in}}{\pgfqpoint{7.978602in}{2.412404in}}%
\pgfpathcurveto{\pgfqpoint{7.982168in}{2.408838in}}{\pgfqpoint{7.987006in}{2.406834in}}{\pgfqpoint{7.992050in}{2.406834in}}%
\pgfpathclose%
\pgfusepath{fill}%
\end{pgfscope}%
\begin{pgfscope}%
\pgfpathrectangle{\pgfqpoint{6.572727in}{0.474100in}}{\pgfqpoint{4.227273in}{3.318700in}}%
\pgfusepath{clip}%
\pgfsetbuttcap%
\pgfsetroundjoin%
\definecolor{currentfill}{rgb}{0.993248,0.906157,0.143936}%
\pgfsetfillcolor{currentfill}%
\pgfsetfillopacity{0.700000}%
\pgfsetlinewidth{0.000000pt}%
\definecolor{currentstroke}{rgb}{0.000000,0.000000,0.000000}%
\pgfsetstrokecolor{currentstroke}%
\pgfsetstrokeopacity{0.700000}%
\pgfsetdash{}{0pt}%
\pgfpathmoveto{\pgfqpoint{9.851579in}{1.344794in}}%
\pgfpathcurveto{\pgfqpoint{9.856623in}{1.344794in}}{\pgfqpoint{9.861460in}{1.346797in}}{\pgfqpoint{9.865027in}{1.350364in}}%
\pgfpathcurveto{\pgfqpoint{9.868593in}{1.353930in}}{\pgfqpoint{9.870597in}{1.358768in}}{\pgfqpoint{9.870597in}{1.363812in}}%
\pgfpathcurveto{\pgfqpoint{9.870597in}{1.368855in}}{\pgfqpoint{9.868593in}{1.373693in}}{\pgfqpoint{9.865027in}{1.377260in}}%
\pgfpathcurveto{\pgfqpoint{9.861460in}{1.380826in}}{\pgfqpoint{9.856623in}{1.382830in}}{\pgfqpoint{9.851579in}{1.382830in}}%
\pgfpathcurveto{\pgfqpoint{9.846535in}{1.382830in}}{\pgfqpoint{9.841698in}{1.380826in}}{\pgfqpoint{9.838131in}{1.377260in}}%
\pgfpathcurveto{\pgfqpoint{9.834565in}{1.373693in}}{\pgfqpoint{9.832561in}{1.368855in}}{\pgfqpoint{9.832561in}{1.363812in}}%
\pgfpathcurveto{\pgfqpoint{9.832561in}{1.358768in}}{\pgfqpoint{9.834565in}{1.353930in}}{\pgfqpoint{9.838131in}{1.350364in}}%
\pgfpathcurveto{\pgfqpoint{9.841698in}{1.346797in}}{\pgfqpoint{9.846535in}{1.344794in}}{\pgfqpoint{9.851579in}{1.344794in}}%
\pgfpathclose%
\pgfusepath{fill}%
\end{pgfscope}%
\begin{pgfscope}%
\pgfpathrectangle{\pgfqpoint{6.572727in}{0.474100in}}{\pgfqpoint{4.227273in}{3.318700in}}%
\pgfusepath{clip}%
\pgfsetbuttcap%
\pgfsetroundjoin%
\definecolor{currentfill}{rgb}{0.127568,0.566949,0.550556}%
\pgfsetfillcolor{currentfill}%
\pgfsetfillopacity{0.700000}%
\pgfsetlinewidth{0.000000pt}%
\definecolor{currentstroke}{rgb}{0.000000,0.000000,0.000000}%
\pgfsetstrokecolor{currentstroke}%
\pgfsetstrokeopacity{0.700000}%
\pgfsetdash{}{0pt}%
\pgfpathmoveto{\pgfqpoint{8.376556in}{1.368137in}}%
\pgfpathcurveto{\pgfqpoint{8.381599in}{1.368137in}}{\pgfqpoint{8.386437in}{1.370141in}}{\pgfqpoint{8.390004in}{1.373708in}}%
\pgfpathcurveto{\pgfqpoint{8.393570in}{1.377274in}}{\pgfqpoint{8.395574in}{1.382112in}}{\pgfqpoint{8.395574in}{1.387155in}}%
\pgfpathcurveto{\pgfqpoint{8.395574in}{1.392199in}}{\pgfqpoint{8.393570in}{1.397037in}}{\pgfqpoint{8.390004in}{1.400603in}}%
\pgfpathcurveto{\pgfqpoint{8.386437in}{1.404170in}}{\pgfqpoint{8.381599in}{1.406174in}}{\pgfqpoint{8.376556in}{1.406174in}}%
\pgfpathcurveto{\pgfqpoint{8.371512in}{1.406174in}}{\pgfqpoint{8.366674in}{1.404170in}}{\pgfqpoint{8.363108in}{1.400603in}}%
\pgfpathcurveto{\pgfqpoint{8.359541in}{1.397037in}}{\pgfqpoint{8.357538in}{1.392199in}}{\pgfqpoint{8.357538in}{1.387155in}}%
\pgfpathcurveto{\pgfqpoint{8.357538in}{1.382112in}}{\pgfqpoint{8.359541in}{1.377274in}}{\pgfqpoint{8.363108in}{1.373708in}}%
\pgfpathcurveto{\pgfqpoint{8.366674in}{1.370141in}}{\pgfqpoint{8.371512in}{1.368137in}}{\pgfqpoint{8.376556in}{1.368137in}}%
\pgfpathclose%
\pgfusepath{fill}%
\end{pgfscope}%
\begin{pgfscope}%
\pgfpathrectangle{\pgfqpoint{6.572727in}{0.474100in}}{\pgfqpoint{4.227273in}{3.318700in}}%
\pgfusepath{clip}%
\pgfsetbuttcap%
\pgfsetroundjoin%
\definecolor{currentfill}{rgb}{0.127568,0.566949,0.550556}%
\pgfsetfillcolor{currentfill}%
\pgfsetfillopacity{0.700000}%
\pgfsetlinewidth{0.000000pt}%
\definecolor{currentstroke}{rgb}{0.000000,0.000000,0.000000}%
\pgfsetstrokecolor{currentstroke}%
\pgfsetstrokeopacity{0.700000}%
\pgfsetdash{}{0pt}%
\pgfpathmoveto{\pgfqpoint{8.663505in}{2.399876in}}%
\pgfpathcurveto{\pgfqpoint{8.668549in}{2.399876in}}{\pgfqpoint{8.673387in}{2.401880in}}{\pgfqpoint{8.676953in}{2.405446in}}%
\pgfpathcurveto{\pgfqpoint{8.680519in}{2.409013in}}{\pgfqpoint{8.682523in}{2.413851in}}{\pgfqpoint{8.682523in}{2.418894in}}%
\pgfpathcurveto{\pgfqpoint{8.682523in}{2.423938in}}{\pgfqpoint{8.680519in}{2.428776in}}{\pgfqpoint{8.676953in}{2.432342in}}%
\pgfpathcurveto{\pgfqpoint{8.673387in}{2.435909in}}{\pgfqpoint{8.668549in}{2.437912in}}{\pgfqpoint{8.663505in}{2.437912in}}%
\pgfpathcurveto{\pgfqpoint{8.658461in}{2.437912in}}{\pgfqpoint{8.653624in}{2.435909in}}{\pgfqpoint{8.650057in}{2.432342in}}%
\pgfpathcurveto{\pgfqpoint{8.646491in}{2.428776in}}{\pgfqpoint{8.644487in}{2.423938in}}{\pgfqpoint{8.644487in}{2.418894in}}%
\pgfpathcurveto{\pgfqpoint{8.644487in}{2.413851in}}{\pgfqpoint{8.646491in}{2.409013in}}{\pgfqpoint{8.650057in}{2.405446in}}%
\pgfpathcurveto{\pgfqpoint{8.653624in}{2.401880in}}{\pgfqpoint{8.658461in}{2.399876in}}{\pgfqpoint{8.663505in}{2.399876in}}%
\pgfpathclose%
\pgfusepath{fill}%
\end{pgfscope}%
\begin{pgfscope}%
\pgfpathrectangle{\pgfqpoint{6.572727in}{0.474100in}}{\pgfqpoint{4.227273in}{3.318700in}}%
\pgfusepath{clip}%
\pgfsetbuttcap%
\pgfsetroundjoin%
\definecolor{currentfill}{rgb}{0.993248,0.906157,0.143936}%
\pgfsetfillcolor{currentfill}%
\pgfsetfillopacity{0.700000}%
\pgfsetlinewidth{0.000000pt}%
\definecolor{currentstroke}{rgb}{0.000000,0.000000,0.000000}%
\pgfsetstrokecolor{currentstroke}%
\pgfsetstrokeopacity{0.700000}%
\pgfsetdash{}{0pt}%
\pgfpathmoveto{\pgfqpoint{9.500784in}{1.314533in}}%
\pgfpathcurveto{\pgfqpoint{9.505828in}{1.314533in}}{\pgfqpoint{9.510666in}{1.316537in}}{\pgfqpoint{9.514232in}{1.320104in}}%
\pgfpathcurveto{\pgfqpoint{9.517799in}{1.323670in}}{\pgfqpoint{9.519802in}{1.328508in}}{\pgfqpoint{9.519802in}{1.333551in}}%
\pgfpathcurveto{\pgfqpoint{9.519802in}{1.338595in}}{\pgfqpoint{9.517799in}{1.343433in}}{\pgfqpoint{9.514232in}{1.346999in}}%
\pgfpathcurveto{\pgfqpoint{9.510666in}{1.350566in}}{\pgfqpoint{9.505828in}{1.352570in}}{\pgfqpoint{9.500784in}{1.352570in}}%
\pgfpathcurveto{\pgfqpoint{9.495741in}{1.352570in}}{\pgfqpoint{9.490903in}{1.350566in}}{\pgfqpoint{9.487336in}{1.346999in}}%
\pgfpathcurveto{\pgfqpoint{9.483770in}{1.343433in}}{\pgfqpoint{9.481766in}{1.338595in}}{\pgfqpoint{9.481766in}{1.333551in}}%
\pgfpathcurveto{\pgfqpoint{9.481766in}{1.328508in}}{\pgfqpoint{9.483770in}{1.323670in}}{\pgfqpoint{9.487336in}{1.320104in}}%
\pgfpathcurveto{\pgfqpoint{9.490903in}{1.316537in}}{\pgfqpoint{9.495741in}{1.314533in}}{\pgfqpoint{9.500784in}{1.314533in}}%
\pgfpathclose%
\pgfusepath{fill}%
\end{pgfscope}%
\begin{pgfscope}%
\pgfpathrectangle{\pgfqpoint{6.572727in}{0.474100in}}{\pgfqpoint{4.227273in}{3.318700in}}%
\pgfusepath{clip}%
\pgfsetbuttcap%
\pgfsetroundjoin%
\definecolor{currentfill}{rgb}{0.127568,0.566949,0.550556}%
\pgfsetfillcolor{currentfill}%
\pgfsetfillopacity{0.700000}%
\pgfsetlinewidth{0.000000pt}%
\definecolor{currentstroke}{rgb}{0.000000,0.000000,0.000000}%
\pgfsetstrokecolor{currentstroke}%
\pgfsetstrokeopacity{0.700000}%
\pgfsetdash{}{0pt}%
\pgfpathmoveto{\pgfqpoint{7.629421in}{1.278138in}}%
\pgfpathcurveto{\pgfqpoint{7.634465in}{1.278138in}}{\pgfqpoint{7.639303in}{1.280142in}}{\pgfqpoint{7.642869in}{1.283709in}}%
\pgfpathcurveto{\pgfqpoint{7.646435in}{1.287275in}}{\pgfqpoint{7.648439in}{1.292113in}}{\pgfqpoint{7.648439in}{1.297157in}}%
\pgfpathcurveto{\pgfqpoint{7.648439in}{1.302200in}}{\pgfqpoint{7.646435in}{1.307038in}}{\pgfqpoint{7.642869in}{1.310604in}}%
\pgfpathcurveto{\pgfqpoint{7.639303in}{1.314171in}}{\pgfqpoint{7.634465in}{1.316175in}}{\pgfqpoint{7.629421in}{1.316175in}}%
\pgfpathcurveto{\pgfqpoint{7.624377in}{1.316175in}}{\pgfqpoint{7.619540in}{1.314171in}}{\pgfqpoint{7.615973in}{1.310604in}}%
\pgfpathcurveto{\pgfqpoint{7.612407in}{1.307038in}}{\pgfqpoint{7.610403in}{1.302200in}}{\pgfqpoint{7.610403in}{1.297157in}}%
\pgfpathcurveto{\pgfqpoint{7.610403in}{1.292113in}}{\pgfqpoint{7.612407in}{1.287275in}}{\pgfqpoint{7.615973in}{1.283709in}}%
\pgfpathcurveto{\pgfqpoint{7.619540in}{1.280142in}}{\pgfqpoint{7.624377in}{1.278138in}}{\pgfqpoint{7.629421in}{1.278138in}}%
\pgfpathclose%
\pgfusepath{fill}%
\end{pgfscope}%
\begin{pgfscope}%
\pgfpathrectangle{\pgfqpoint{6.572727in}{0.474100in}}{\pgfqpoint{4.227273in}{3.318700in}}%
\pgfusepath{clip}%
\pgfsetbuttcap%
\pgfsetroundjoin%
\definecolor{currentfill}{rgb}{0.127568,0.566949,0.550556}%
\pgfsetfillcolor{currentfill}%
\pgfsetfillopacity{0.700000}%
\pgfsetlinewidth{0.000000pt}%
\definecolor{currentstroke}{rgb}{0.000000,0.000000,0.000000}%
\pgfsetstrokecolor{currentstroke}%
\pgfsetstrokeopacity{0.700000}%
\pgfsetdash{}{0pt}%
\pgfpathmoveto{\pgfqpoint{8.269253in}{2.915384in}}%
\pgfpathcurveto{\pgfqpoint{8.274296in}{2.915384in}}{\pgfqpoint{8.279134in}{2.917388in}}{\pgfqpoint{8.282701in}{2.920954in}}%
\pgfpathcurveto{\pgfqpoint{8.286267in}{2.924521in}}{\pgfqpoint{8.288271in}{2.929358in}}{\pgfqpoint{8.288271in}{2.934402in}}%
\pgfpathcurveto{\pgfqpoint{8.288271in}{2.939446in}}{\pgfqpoint{8.286267in}{2.944283in}}{\pgfqpoint{8.282701in}{2.947850in}}%
\pgfpathcurveto{\pgfqpoint{8.279134in}{2.951416in}}{\pgfqpoint{8.274296in}{2.953420in}}{\pgfqpoint{8.269253in}{2.953420in}}%
\pgfpathcurveto{\pgfqpoint{8.264209in}{2.953420in}}{\pgfqpoint{8.259371in}{2.951416in}}{\pgfqpoint{8.255805in}{2.947850in}}%
\pgfpathcurveto{\pgfqpoint{8.252238in}{2.944283in}}{\pgfqpoint{8.250235in}{2.939446in}}{\pgfqpoint{8.250235in}{2.934402in}}%
\pgfpathcurveto{\pgfqpoint{8.250235in}{2.929358in}}{\pgfqpoint{8.252238in}{2.924521in}}{\pgfqpoint{8.255805in}{2.920954in}}%
\pgfpathcurveto{\pgfqpoint{8.259371in}{2.917388in}}{\pgfqpoint{8.264209in}{2.915384in}}{\pgfqpoint{8.269253in}{2.915384in}}%
\pgfpathclose%
\pgfusepath{fill}%
\end{pgfscope}%
\begin{pgfscope}%
\pgfpathrectangle{\pgfqpoint{6.572727in}{0.474100in}}{\pgfqpoint{4.227273in}{3.318700in}}%
\pgfusepath{clip}%
\pgfsetbuttcap%
\pgfsetroundjoin%
\definecolor{currentfill}{rgb}{0.127568,0.566949,0.550556}%
\pgfsetfillcolor{currentfill}%
\pgfsetfillopacity{0.700000}%
\pgfsetlinewidth{0.000000pt}%
\definecolor{currentstroke}{rgb}{0.000000,0.000000,0.000000}%
\pgfsetstrokecolor{currentstroke}%
\pgfsetstrokeopacity{0.700000}%
\pgfsetdash{}{0pt}%
\pgfpathmoveto{\pgfqpoint{7.947538in}{1.616409in}}%
\pgfpathcurveto{\pgfqpoint{7.952582in}{1.616409in}}{\pgfqpoint{7.957420in}{1.618413in}}{\pgfqpoint{7.960986in}{1.621979in}}%
\pgfpathcurveto{\pgfqpoint{7.964553in}{1.625545in}}{\pgfqpoint{7.966557in}{1.630383in}}{\pgfqpoint{7.966557in}{1.635427in}}%
\pgfpathcurveto{\pgfqpoint{7.966557in}{1.640471in}}{\pgfqpoint{7.964553in}{1.645308in}}{\pgfqpoint{7.960986in}{1.648875in}}%
\pgfpathcurveto{\pgfqpoint{7.957420in}{1.652441in}}{\pgfqpoint{7.952582in}{1.654445in}}{\pgfqpoint{7.947538in}{1.654445in}}%
\pgfpathcurveto{\pgfqpoint{7.942495in}{1.654445in}}{\pgfqpoint{7.937657in}{1.652441in}}{\pgfqpoint{7.934091in}{1.648875in}}%
\pgfpathcurveto{\pgfqpoint{7.930524in}{1.645308in}}{\pgfqpoint{7.928520in}{1.640471in}}{\pgfqpoint{7.928520in}{1.635427in}}%
\pgfpathcurveto{\pgfqpoint{7.928520in}{1.630383in}}{\pgfqpoint{7.930524in}{1.625545in}}{\pgfqpoint{7.934091in}{1.621979in}}%
\pgfpathcurveto{\pgfqpoint{7.937657in}{1.618413in}}{\pgfqpoint{7.942495in}{1.616409in}}{\pgfqpoint{7.947538in}{1.616409in}}%
\pgfpathclose%
\pgfusepath{fill}%
\end{pgfscope}%
\begin{pgfscope}%
\pgfpathrectangle{\pgfqpoint{6.572727in}{0.474100in}}{\pgfqpoint{4.227273in}{3.318700in}}%
\pgfusepath{clip}%
\pgfsetbuttcap%
\pgfsetroundjoin%
\definecolor{currentfill}{rgb}{0.127568,0.566949,0.550556}%
\pgfsetfillcolor{currentfill}%
\pgfsetfillopacity{0.700000}%
\pgfsetlinewidth{0.000000pt}%
\definecolor{currentstroke}{rgb}{0.000000,0.000000,0.000000}%
\pgfsetstrokecolor{currentstroke}%
\pgfsetstrokeopacity{0.700000}%
\pgfsetdash{}{0pt}%
\pgfpathmoveto{\pgfqpoint{8.072516in}{2.075737in}}%
\pgfpathcurveto{\pgfqpoint{8.077560in}{2.075737in}}{\pgfqpoint{8.082398in}{2.077741in}}{\pgfqpoint{8.085964in}{2.081308in}}%
\pgfpathcurveto{\pgfqpoint{8.089530in}{2.084874in}}{\pgfqpoint{8.091534in}{2.089712in}}{\pgfqpoint{8.091534in}{2.094756in}}%
\pgfpathcurveto{\pgfqpoint{8.091534in}{2.099799in}}{\pgfqpoint{8.089530in}{2.104637in}}{\pgfqpoint{8.085964in}{2.108203in}}%
\pgfpathcurveto{\pgfqpoint{8.082398in}{2.111770in}}{\pgfqpoint{8.077560in}{2.113774in}}{\pgfqpoint{8.072516in}{2.113774in}}%
\pgfpathcurveto{\pgfqpoint{8.067472in}{2.113774in}}{\pgfqpoint{8.062635in}{2.111770in}}{\pgfqpoint{8.059068in}{2.108203in}}%
\pgfpathcurveto{\pgfqpoint{8.055502in}{2.104637in}}{\pgfqpoint{8.053498in}{2.099799in}}{\pgfqpoint{8.053498in}{2.094756in}}%
\pgfpathcurveto{\pgfqpoint{8.053498in}{2.089712in}}{\pgfqpoint{8.055502in}{2.084874in}}{\pgfqpoint{8.059068in}{2.081308in}}%
\pgfpathcurveto{\pgfqpoint{8.062635in}{2.077741in}}{\pgfqpoint{8.067472in}{2.075737in}}{\pgfqpoint{8.072516in}{2.075737in}}%
\pgfpathclose%
\pgfusepath{fill}%
\end{pgfscope}%
\begin{pgfscope}%
\pgfpathrectangle{\pgfqpoint{6.572727in}{0.474100in}}{\pgfqpoint{4.227273in}{3.318700in}}%
\pgfusepath{clip}%
\pgfsetbuttcap%
\pgfsetroundjoin%
\definecolor{currentfill}{rgb}{0.127568,0.566949,0.550556}%
\pgfsetfillcolor{currentfill}%
\pgfsetfillopacity{0.700000}%
\pgfsetlinewidth{0.000000pt}%
\definecolor{currentstroke}{rgb}{0.000000,0.000000,0.000000}%
\pgfsetstrokecolor{currentstroke}%
\pgfsetstrokeopacity{0.700000}%
\pgfsetdash{}{0pt}%
\pgfpathmoveto{\pgfqpoint{7.909375in}{3.188528in}}%
\pgfpathcurveto{\pgfqpoint{7.914418in}{3.188528in}}{\pgfqpoint{7.919256in}{3.190532in}}{\pgfqpoint{7.922823in}{3.194098in}}%
\pgfpathcurveto{\pgfqpoint{7.926389in}{3.197665in}}{\pgfqpoint{7.928393in}{3.202503in}}{\pgfqpoint{7.928393in}{3.207546in}}%
\pgfpathcurveto{\pgfqpoint{7.928393in}{3.212590in}}{\pgfqpoint{7.926389in}{3.217428in}}{\pgfqpoint{7.922823in}{3.220994in}}%
\pgfpathcurveto{\pgfqpoint{7.919256in}{3.224560in}}{\pgfqpoint{7.914418in}{3.226564in}}{\pgfqpoint{7.909375in}{3.226564in}}%
\pgfpathcurveto{\pgfqpoint{7.904331in}{3.226564in}}{\pgfqpoint{7.899493in}{3.224560in}}{\pgfqpoint{7.895927in}{3.220994in}}%
\pgfpathcurveto{\pgfqpoint{7.892361in}{3.217428in}}{\pgfqpoint{7.890357in}{3.212590in}}{\pgfqpoint{7.890357in}{3.207546in}}%
\pgfpathcurveto{\pgfqpoint{7.890357in}{3.202503in}}{\pgfqpoint{7.892361in}{3.197665in}}{\pgfqpoint{7.895927in}{3.194098in}}%
\pgfpathcurveto{\pgfqpoint{7.899493in}{3.190532in}}{\pgfqpoint{7.904331in}{3.188528in}}{\pgfqpoint{7.909375in}{3.188528in}}%
\pgfpathclose%
\pgfusepath{fill}%
\end{pgfscope}%
\begin{pgfscope}%
\pgfpathrectangle{\pgfqpoint{6.572727in}{0.474100in}}{\pgfqpoint{4.227273in}{3.318700in}}%
\pgfusepath{clip}%
\pgfsetbuttcap%
\pgfsetroundjoin%
\definecolor{currentfill}{rgb}{0.127568,0.566949,0.550556}%
\pgfsetfillcolor{currentfill}%
\pgfsetfillopacity{0.700000}%
\pgfsetlinewidth{0.000000pt}%
\definecolor{currentstroke}{rgb}{0.000000,0.000000,0.000000}%
\pgfsetstrokecolor{currentstroke}%
\pgfsetstrokeopacity{0.700000}%
\pgfsetdash{}{0pt}%
\pgfpathmoveto{\pgfqpoint{8.131219in}{2.735666in}}%
\pgfpathcurveto{\pgfqpoint{8.136263in}{2.735666in}}{\pgfqpoint{8.141101in}{2.737669in}}{\pgfqpoint{8.144667in}{2.741236in}}%
\pgfpathcurveto{\pgfqpoint{8.148234in}{2.744802in}}{\pgfqpoint{8.150237in}{2.749640in}}{\pgfqpoint{8.150237in}{2.754684in}}%
\pgfpathcurveto{\pgfqpoint{8.150237in}{2.759727in}}{\pgfqpoint{8.148234in}{2.764565in}}{\pgfqpoint{8.144667in}{2.768132in}}%
\pgfpathcurveto{\pgfqpoint{8.141101in}{2.771698in}}{\pgfqpoint{8.136263in}{2.773702in}}{\pgfqpoint{8.131219in}{2.773702in}}%
\pgfpathcurveto{\pgfqpoint{8.126176in}{2.773702in}}{\pgfqpoint{8.121338in}{2.771698in}}{\pgfqpoint{8.117771in}{2.768132in}}%
\pgfpathcurveto{\pgfqpoint{8.114205in}{2.764565in}}{\pgfqpoint{8.112201in}{2.759727in}}{\pgfqpoint{8.112201in}{2.754684in}}%
\pgfpathcurveto{\pgfqpoint{8.112201in}{2.749640in}}{\pgfqpoint{8.114205in}{2.744802in}}{\pgfqpoint{8.117771in}{2.741236in}}%
\pgfpathcurveto{\pgfqpoint{8.121338in}{2.737669in}}{\pgfqpoint{8.126176in}{2.735666in}}{\pgfqpoint{8.131219in}{2.735666in}}%
\pgfpathclose%
\pgfusepath{fill}%
\end{pgfscope}%
\begin{pgfscope}%
\pgfpathrectangle{\pgfqpoint{6.572727in}{0.474100in}}{\pgfqpoint{4.227273in}{3.318700in}}%
\pgfusepath{clip}%
\pgfsetbuttcap%
\pgfsetroundjoin%
\definecolor{currentfill}{rgb}{0.127568,0.566949,0.550556}%
\pgfsetfillcolor{currentfill}%
\pgfsetfillopacity{0.700000}%
\pgfsetlinewidth{0.000000pt}%
\definecolor{currentstroke}{rgb}{0.000000,0.000000,0.000000}%
\pgfsetstrokecolor{currentstroke}%
\pgfsetstrokeopacity{0.700000}%
\pgfsetdash{}{0pt}%
\pgfpathmoveto{\pgfqpoint{7.992248in}{2.774562in}}%
\pgfpathcurveto{\pgfqpoint{7.997291in}{2.774562in}}{\pgfqpoint{8.002129in}{2.776566in}}{\pgfqpoint{8.005696in}{2.780132in}}%
\pgfpathcurveto{\pgfqpoint{8.009262in}{2.783698in}}{\pgfqpoint{8.011266in}{2.788536in}}{\pgfqpoint{8.011266in}{2.793580in}}%
\pgfpathcurveto{\pgfqpoint{8.011266in}{2.798623in}}{\pgfqpoint{8.009262in}{2.803461in}}{\pgfqpoint{8.005696in}{2.807028in}}%
\pgfpathcurveto{\pgfqpoint{8.002129in}{2.810594in}}{\pgfqpoint{7.997291in}{2.812598in}}{\pgfqpoint{7.992248in}{2.812598in}}%
\pgfpathcurveto{\pgfqpoint{7.987204in}{2.812598in}}{\pgfqpoint{7.982366in}{2.810594in}}{\pgfqpoint{7.978800in}{2.807028in}}%
\pgfpathcurveto{\pgfqpoint{7.975234in}{2.803461in}}{\pgfqpoint{7.973230in}{2.798623in}}{\pgfqpoint{7.973230in}{2.793580in}}%
\pgfpathcurveto{\pgfqpoint{7.973230in}{2.788536in}}{\pgfqpoint{7.975234in}{2.783698in}}{\pgfqpoint{7.978800in}{2.780132in}}%
\pgfpathcurveto{\pgfqpoint{7.982366in}{2.776566in}}{\pgfqpoint{7.987204in}{2.774562in}}{\pgfqpoint{7.992248in}{2.774562in}}%
\pgfpathclose%
\pgfusepath{fill}%
\end{pgfscope}%
\begin{pgfscope}%
\pgfpathrectangle{\pgfqpoint{6.572727in}{0.474100in}}{\pgfqpoint{4.227273in}{3.318700in}}%
\pgfusepath{clip}%
\pgfsetbuttcap%
\pgfsetroundjoin%
\definecolor{currentfill}{rgb}{0.127568,0.566949,0.550556}%
\pgfsetfillcolor{currentfill}%
\pgfsetfillopacity{0.700000}%
\pgfsetlinewidth{0.000000pt}%
\definecolor{currentstroke}{rgb}{0.000000,0.000000,0.000000}%
\pgfsetstrokecolor{currentstroke}%
\pgfsetstrokeopacity{0.700000}%
\pgfsetdash{}{0pt}%
\pgfpathmoveto{\pgfqpoint{8.465141in}{1.484636in}}%
\pgfpathcurveto{\pgfqpoint{8.470185in}{1.484636in}}{\pgfqpoint{8.475023in}{1.486640in}}{\pgfqpoint{8.478589in}{1.490206in}}%
\pgfpathcurveto{\pgfqpoint{8.482156in}{1.493773in}}{\pgfqpoint{8.484159in}{1.498611in}}{\pgfqpoint{8.484159in}{1.503654in}}%
\pgfpathcurveto{\pgfqpoint{8.484159in}{1.508698in}}{\pgfqpoint{8.482156in}{1.513536in}}{\pgfqpoint{8.478589in}{1.517102in}}%
\pgfpathcurveto{\pgfqpoint{8.475023in}{1.520669in}}{\pgfqpoint{8.470185in}{1.522672in}}{\pgfqpoint{8.465141in}{1.522672in}}%
\pgfpathcurveto{\pgfqpoint{8.460098in}{1.522672in}}{\pgfqpoint{8.455260in}{1.520669in}}{\pgfqpoint{8.451693in}{1.517102in}}%
\pgfpathcurveto{\pgfqpoint{8.448127in}{1.513536in}}{\pgfqpoint{8.446123in}{1.508698in}}{\pgfqpoint{8.446123in}{1.503654in}}%
\pgfpathcurveto{\pgfqpoint{8.446123in}{1.498611in}}{\pgfqpoint{8.448127in}{1.493773in}}{\pgfqpoint{8.451693in}{1.490206in}}%
\pgfpathcurveto{\pgfqpoint{8.455260in}{1.486640in}}{\pgfqpoint{8.460098in}{1.484636in}}{\pgfqpoint{8.465141in}{1.484636in}}%
\pgfpathclose%
\pgfusepath{fill}%
\end{pgfscope}%
\begin{pgfscope}%
\pgfpathrectangle{\pgfqpoint{6.572727in}{0.474100in}}{\pgfqpoint{4.227273in}{3.318700in}}%
\pgfusepath{clip}%
\pgfsetbuttcap%
\pgfsetroundjoin%
\definecolor{currentfill}{rgb}{0.127568,0.566949,0.550556}%
\pgfsetfillcolor{currentfill}%
\pgfsetfillopacity{0.700000}%
\pgfsetlinewidth{0.000000pt}%
\definecolor{currentstroke}{rgb}{0.000000,0.000000,0.000000}%
\pgfsetstrokecolor{currentstroke}%
\pgfsetstrokeopacity{0.700000}%
\pgfsetdash{}{0pt}%
\pgfpathmoveto{\pgfqpoint{7.924818in}{2.619847in}}%
\pgfpathcurveto{\pgfqpoint{7.929861in}{2.619847in}}{\pgfqpoint{7.934699in}{2.621851in}}{\pgfqpoint{7.938266in}{2.625417in}}%
\pgfpathcurveto{\pgfqpoint{7.941832in}{2.628984in}}{\pgfqpoint{7.943836in}{2.633822in}}{\pgfqpoint{7.943836in}{2.638865in}}%
\pgfpathcurveto{\pgfqpoint{7.943836in}{2.643909in}}{\pgfqpoint{7.941832in}{2.648747in}}{\pgfqpoint{7.938266in}{2.652313in}}%
\pgfpathcurveto{\pgfqpoint{7.934699in}{2.655879in}}{\pgfqpoint{7.929861in}{2.657883in}}{\pgfqpoint{7.924818in}{2.657883in}}%
\pgfpathcurveto{\pgfqpoint{7.919774in}{2.657883in}}{\pgfqpoint{7.914936in}{2.655879in}}{\pgfqpoint{7.911370in}{2.652313in}}%
\pgfpathcurveto{\pgfqpoint{7.907803in}{2.648747in}}{\pgfqpoint{7.905800in}{2.643909in}}{\pgfqpoint{7.905800in}{2.638865in}}%
\pgfpathcurveto{\pgfqpoint{7.905800in}{2.633822in}}{\pgfqpoint{7.907803in}{2.628984in}}{\pgfqpoint{7.911370in}{2.625417in}}%
\pgfpathcurveto{\pgfqpoint{7.914936in}{2.621851in}}{\pgfqpoint{7.919774in}{2.619847in}}{\pgfqpoint{7.924818in}{2.619847in}}%
\pgfpathclose%
\pgfusepath{fill}%
\end{pgfscope}%
\begin{pgfscope}%
\pgfpathrectangle{\pgfqpoint{6.572727in}{0.474100in}}{\pgfqpoint{4.227273in}{3.318700in}}%
\pgfusepath{clip}%
\pgfsetbuttcap%
\pgfsetroundjoin%
\definecolor{currentfill}{rgb}{0.993248,0.906157,0.143936}%
\pgfsetfillcolor{currentfill}%
\pgfsetfillopacity{0.700000}%
\pgfsetlinewidth{0.000000pt}%
\definecolor{currentstroke}{rgb}{0.000000,0.000000,0.000000}%
\pgfsetstrokecolor{currentstroke}%
\pgfsetstrokeopacity{0.700000}%
\pgfsetdash{}{0pt}%
\pgfpathmoveto{\pgfqpoint{9.627687in}{1.515561in}}%
\pgfpathcurveto{\pgfqpoint{9.632731in}{1.515561in}}{\pgfqpoint{9.637569in}{1.517565in}}{\pgfqpoint{9.641135in}{1.521132in}}%
\pgfpathcurveto{\pgfqpoint{9.644701in}{1.524698in}}{\pgfqpoint{9.646705in}{1.529536in}}{\pgfqpoint{9.646705in}{1.534579in}}%
\pgfpathcurveto{\pgfqpoint{9.646705in}{1.539623in}}{\pgfqpoint{9.644701in}{1.544461in}}{\pgfqpoint{9.641135in}{1.548027in}}%
\pgfpathcurveto{\pgfqpoint{9.637569in}{1.551594in}}{\pgfqpoint{9.632731in}{1.553598in}}{\pgfqpoint{9.627687in}{1.553598in}}%
\pgfpathcurveto{\pgfqpoint{9.622644in}{1.553598in}}{\pgfqpoint{9.617806in}{1.551594in}}{\pgfqpoint{9.614239in}{1.548027in}}%
\pgfpathcurveto{\pgfqpoint{9.610673in}{1.544461in}}{\pgfqpoint{9.608669in}{1.539623in}}{\pgfqpoint{9.608669in}{1.534579in}}%
\pgfpathcurveto{\pgfqpoint{9.608669in}{1.529536in}}{\pgfqpoint{9.610673in}{1.524698in}}{\pgfqpoint{9.614239in}{1.521132in}}%
\pgfpathcurveto{\pgfqpoint{9.617806in}{1.517565in}}{\pgfqpoint{9.622644in}{1.515561in}}{\pgfqpoint{9.627687in}{1.515561in}}%
\pgfpathclose%
\pgfusepath{fill}%
\end{pgfscope}%
\begin{pgfscope}%
\pgfpathrectangle{\pgfqpoint{6.572727in}{0.474100in}}{\pgfqpoint{4.227273in}{3.318700in}}%
\pgfusepath{clip}%
\pgfsetbuttcap%
\pgfsetroundjoin%
\definecolor{currentfill}{rgb}{0.127568,0.566949,0.550556}%
\pgfsetfillcolor{currentfill}%
\pgfsetfillopacity{0.700000}%
\pgfsetlinewidth{0.000000pt}%
\definecolor{currentstroke}{rgb}{0.000000,0.000000,0.000000}%
\pgfsetstrokecolor{currentstroke}%
\pgfsetstrokeopacity{0.700000}%
\pgfsetdash{}{0pt}%
\pgfpathmoveto{\pgfqpoint{7.472487in}{2.167674in}}%
\pgfpathcurveto{\pgfqpoint{7.477531in}{2.167674in}}{\pgfqpoint{7.482369in}{2.169678in}}{\pgfqpoint{7.485935in}{2.173244in}}%
\pgfpathcurveto{\pgfqpoint{7.489501in}{2.176811in}}{\pgfqpoint{7.491505in}{2.181649in}}{\pgfqpoint{7.491505in}{2.186692in}}%
\pgfpathcurveto{\pgfqpoint{7.491505in}{2.191736in}}{\pgfqpoint{7.489501in}{2.196574in}}{\pgfqpoint{7.485935in}{2.200140in}}%
\pgfpathcurveto{\pgfqpoint{7.482369in}{2.203707in}}{\pgfqpoint{7.477531in}{2.205710in}}{\pgfqpoint{7.472487in}{2.205710in}}%
\pgfpathcurveto{\pgfqpoint{7.467444in}{2.205710in}}{\pgfqpoint{7.462606in}{2.203707in}}{\pgfqpoint{7.459039in}{2.200140in}}%
\pgfpathcurveto{\pgfqpoint{7.455473in}{2.196574in}}{\pgfqpoint{7.453469in}{2.191736in}}{\pgfqpoint{7.453469in}{2.186692in}}%
\pgfpathcurveto{\pgfqpoint{7.453469in}{2.181649in}}{\pgfqpoint{7.455473in}{2.176811in}}{\pgfqpoint{7.459039in}{2.173244in}}%
\pgfpathcurveto{\pgfqpoint{7.462606in}{2.169678in}}{\pgfqpoint{7.467444in}{2.167674in}}{\pgfqpoint{7.472487in}{2.167674in}}%
\pgfpathclose%
\pgfusepath{fill}%
\end{pgfscope}%
\begin{pgfscope}%
\pgfpathrectangle{\pgfqpoint{6.572727in}{0.474100in}}{\pgfqpoint{4.227273in}{3.318700in}}%
\pgfusepath{clip}%
\pgfsetbuttcap%
\pgfsetroundjoin%
\definecolor{currentfill}{rgb}{0.127568,0.566949,0.550556}%
\pgfsetfillcolor{currentfill}%
\pgfsetfillopacity{0.700000}%
\pgfsetlinewidth{0.000000pt}%
\definecolor{currentstroke}{rgb}{0.000000,0.000000,0.000000}%
\pgfsetstrokecolor{currentstroke}%
\pgfsetstrokeopacity{0.700000}%
\pgfsetdash{}{0pt}%
\pgfpathmoveto{\pgfqpoint{8.528278in}{3.159160in}}%
\pgfpathcurveto{\pgfqpoint{8.533322in}{3.159160in}}{\pgfqpoint{8.538160in}{3.161164in}}{\pgfqpoint{8.541726in}{3.164731in}}%
\pgfpathcurveto{\pgfqpoint{8.545292in}{3.168297in}}{\pgfqpoint{8.547296in}{3.173135in}}{\pgfqpoint{8.547296in}{3.178179in}}%
\pgfpathcurveto{\pgfqpoint{8.547296in}{3.183222in}}{\pgfqpoint{8.545292in}{3.188060in}}{\pgfqpoint{8.541726in}{3.191626in}}%
\pgfpathcurveto{\pgfqpoint{8.538160in}{3.195193in}}{\pgfqpoint{8.533322in}{3.197197in}}{\pgfqpoint{8.528278in}{3.197197in}}%
\pgfpathcurveto{\pgfqpoint{8.523234in}{3.197197in}}{\pgfqpoint{8.518397in}{3.195193in}}{\pgfqpoint{8.514830in}{3.191626in}}%
\pgfpathcurveto{\pgfqpoint{8.511264in}{3.188060in}}{\pgfqpoint{8.509260in}{3.183222in}}{\pgfqpoint{8.509260in}{3.178179in}}%
\pgfpathcurveto{\pgfqpoint{8.509260in}{3.173135in}}{\pgfqpoint{8.511264in}{3.168297in}}{\pgfqpoint{8.514830in}{3.164731in}}%
\pgfpathcurveto{\pgfqpoint{8.518397in}{3.161164in}}{\pgfqpoint{8.523234in}{3.159160in}}{\pgfqpoint{8.528278in}{3.159160in}}%
\pgfpathclose%
\pgfusepath{fill}%
\end{pgfscope}%
\begin{pgfscope}%
\pgfpathrectangle{\pgfqpoint{6.572727in}{0.474100in}}{\pgfqpoint{4.227273in}{3.318700in}}%
\pgfusepath{clip}%
\pgfsetbuttcap%
\pgfsetroundjoin%
\definecolor{currentfill}{rgb}{0.127568,0.566949,0.550556}%
\pgfsetfillcolor{currentfill}%
\pgfsetfillopacity{0.700000}%
\pgfsetlinewidth{0.000000pt}%
\definecolor{currentstroke}{rgb}{0.000000,0.000000,0.000000}%
\pgfsetstrokecolor{currentstroke}%
\pgfsetstrokeopacity{0.700000}%
\pgfsetdash{}{0pt}%
\pgfpathmoveto{\pgfqpoint{8.062541in}{1.913690in}}%
\pgfpathcurveto{\pgfqpoint{8.067585in}{1.913690in}}{\pgfqpoint{8.072423in}{1.915694in}}{\pgfqpoint{8.075989in}{1.919260in}}%
\pgfpathcurveto{\pgfqpoint{8.079556in}{1.922826in}}{\pgfqpoint{8.081559in}{1.927664in}}{\pgfqpoint{8.081559in}{1.932708in}}%
\pgfpathcurveto{\pgfqpoint{8.081559in}{1.937751in}}{\pgfqpoint{8.079556in}{1.942589in}}{\pgfqpoint{8.075989in}{1.946156in}}%
\pgfpathcurveto{\pgfqpoint{8.072423in}{1.949722in}}{\pgfqpoint{8.067585in}{1.951726in}}{\pgfqpoint{8.062541in}{1.951726in}}%
\pgfpathcurveto{\pgfqpoint{8.057498in}{1.951726in}}{\pgfqpoint{8.052660in}{1.949722in}}{\pgfqpoint{8.049093in}{1.946156in}}%
\pgfpathcurveto{\pgfqpoint{8.045527in}{1.942589in}}{\pgfqpoint{8.043523in}{1.937751in}}{\pgfqpoint{8.043523in}{1.932708in}}%
\pgfpathcurveto{\pgfqpoint{8.043523in}{1.927664in}}{\pgfqpoint{8.045527in}{1.922826in}}{\pgfqpoint{8.049093in}{1.919260in}}%
\pgfpathcurveto{\pgfqpoint{8.052660in}{1.915694in}}{\pgfqpoint{8.057498in}{1.913690in}}{\pgfqpoint{8.062541in}{1.913690in}}%
\pgfpathclose%
\pgfusepath{fill}%
\end{pgfscope}%
\begin{pgfscope}%
\pgfpathrectangle{\pgfqpoint{6.572727in}{0.474100in}}{\pgfqpoint{4.227273in}{3.318700in}}%
\pgfusepath{clip}%
\pgfsetbuttcap%
\pgfsetroundjoin%
\definecolor{currentfill}{rgb}{0.993248,0.906157,0.143936}%
\pgfsetfillcolor{currentfill}%
\pgfsetfillopacity{0.700000}%
\pgfsetlinewidth{0.000000pt}%
\definecolor{currentstroke}{rgb}{0.000000,0.000000,0.000000}%
\pgfsetstrokecolor{currentstroke}%
\pgfsetstrokeopacity{0.700000}%
\pgfsetdash{}{0pt}%
\pgfpathmoveto{\pgfqpoint{9.647632in}{1.817210in}}%
\pgfpathcurveto{\pgfqpoint{9.652676in}{1.817210in}}{\pgfqpoint{9.657514in}{1.819214in}}{\pgfqpoint{9.661080in}{1.822781in}}%
\pgfpathcurveto{\pgfqpoint{9.664647in}{1.826347in}}{\pgfqpoint{9.666651in}{1.831185in}}{\pgfqpoint{9.666651in}{1.836229in}}%
\pgfpathcurveto{\pgfqpoint{9.666651in}{1.841272in}}{\pgfqpoint{9.664647in}{1.846110in}}{\pgfqpoint{9.661080in}{1.849676in}}%
\pgfpathcurveto{\pgfqpoint{9.657514in}{1.853243in}}{\pgfqpoint{9.652676in}{1.855247in}}{\pgfqpoint{9.647632in}{1.855247in}}%
\pgfpathcurveto{\pgfqpoint{9.642589in}{1.855247in}}{\pgfqpoint{9.637751in}{1.853243in}}{\pgfqpoint{9.634185in}{1.849676in}}%
\pgfpathcurveto{\pgfqpoint{9.630618in}{1.846110in}}{\pgfqpoint{9.628614in}{1.841272in}}{\pgfqpoint{9.628614in}{1.836229in}}%
\pgfpathcurveto{\pgfqpoint{9.628614in}{1.831185in}}{\pgfqpoint{9.630618in}{1.826347in}}{\pgfqpoint{9.634185in}{1.822781in}}%
\pgfpathcurveto{\pgfqpoint{9.637751in}{1.819214in}}{\pgfqpoint{9.642589in}{1.817210in}}{\pgfqpoint{9.647632in}{1.817210in}}%
\pgfpathclose%
\pgfusepath{fill}%
\end{pgfscope}%
\begin{pgfscope}%
\pgfpathrectangle{\pgfqpoint{6.572727in}{0.474100in}}{\pgfqpoint{4.227273in}{3.318700in}}%
\pgfusepath{clip}%
\pgfsetbuttcap%
\pgfsetroundjoin%
\definecolor{currentfill}{rgb}{0.127568,0.566949,0.550556}%
\pgfsetfillcolor{currentfill}%
\pgfsetfillopacity{0.700000}%
\pgfsetlinewidth{0.000000pt}%
\definecolor{currentstroke}{rgb}{0.000000,0.000000,0.000000}%
\pgfsetstrokecolor{currentstroke}%
\pgfsetstrokeopacity{0.700000}%
\pgfsetdash{}{0pt}%
\pgfpathmoveto{\pgfqpoint{8.206322in}{1.672645in}}%
\pgfpathcurveto{\pgfqpoint{8.211365in}{1.672645in}}{\pgfqpoint{8.216203in}{1.674649in}}{\pgfqpoint{8.219769in}{1.678215in}}%
\pgfpathcurveto{\pgfqpoint{8.223336in}{1.681781in}}{\pgfqpoint{8.225340in}{1.686619in}}{\pgfqpoint{8.225340in}{1.691663in}}%
\pgfpathcurveto{\pgfqpoint{8.225340in}{1.696707in}}{\pgfqpoint{8.223336in}{1.701544in}}{\pgfqpoint{8.219769in}{1.705111in}}%
\pgfpathcurveto{\pgfqpoint{8.216203in}{1.708677in}}{\pgfqpoint{8.211365in}{1.710681in}}{\pgfqpoint{8.206322in}{1.710681in}}%
\pgfpathcurveto{\pgfqpoint{8.201278in}{1.710681in}}{\pgfqpoint{8.196440in}{1.708677in}}{\pgfqpoint{8.192874in}{1.705111in}}%
\pgfpathcurveto{\pgfqpoint{8.189307in}{1.701544in}}{\pgfqpoint{8.187303in}{1.696707in}}{\pgfqpoint{8.187303in}{1.691663in}}%
\pgfpathcurveto{\pgfqpoint{8.187303in}{1.686619in}}{\pgfqpoint{8.189307in}{1.681781in}}{\pgfqpoint{8.192874in}{1.678215in}}%
\pgfpathcurveto{\pgfqpoint{8.196440in}{1.674649in}}{\pgfqpoint{8.201278in}{1.672645in}}{\pgfqpoint{8.206322in}{1.672645in}}%
\pgfpathclose%
\pgfusepath{fill}%
\end{pgfscope}%
\begin{pgfscope}%
\pgfpathrectangle{\pgfqpoint{6.572727in}{0.474100in}}{\pgfqpoint{4.227273in}{3.318700in}}%
\pgfusepath{clip}%
\pgfsetbuttcap%
\pgfsetroundjoin%
\definecolor{currentfill}{rgb}{0.127568,0.566949,0.550556}%
\pgfsetfillcolor{currentfill}%
\pgfsetfillopacity{0.700000}%
\pgfsetlinewidth{0.000000pt}%
\definecolor{currentstroke}{rgb}{0.000000,0.000000,0.000000}%
\pgfsetstrokecolor{currentstroke}%
\pgfsetstrokeopacity{0.700000}%
\pgfsetdash{}{0pt}%
\pgfpathmoveto{\pgfqpoint{7.334725in}{1.880871in}}%
\pgfpathcurveto{\pgfqpoint{7.339769in}{1.880871in}}{\pgfqpoint{7.344607in}{1.882875in}}{\pgfqpoint{7.348173in}{1.886441in}}%
\pgfpathcurveto{\pgfqpoint{7.351740in}{1.890007in}}{\pgfqpoint{7.353744in}{1.894845in}}{\pgfqpoint{7.353744in}{1.899889in}}%
\pgfpathcurveto{\pgfqpoint{7.353744in}{1.904933in}}{\pgfqpoint{7.351740in}{1.909770in}}{\pgfqpoint{7.348173in}{1.913337in}}%
\pgfpathcurveto{\pgfqpoint{7.344607in}{1.916903in}}{\pgfqpoint{7.339769in}{1.918907in}}{\pgfqpoint{7.334725in}{1.918907in}}%
\pgfpathcurveto{\pgfqpoint{7.329682in}{1.918907in}}{\pgfqpoint{7.324844in}{1.916903in}}{\pgfqpoint{7.321278in}{1.913337in}}%
\pgfpathcurveto{\pgfqpoint{7.317711in}{1.909770in}}{\pgfqpoint{7.315707in}{1.904933in}}{\pgfqpoint{7.315707in}{1.899889in}}%
\pgfpathcurveto{\pgfqpoint{7.315707in}{1.894845in}}{\pgfqpoint{7.317711in}{1.890007in}}{\pgfqpoint{7.321278in}{1.886441in}}%
\pgfpathcurveto{\pgfqpoint{7.324844in}{1.882875in}}{\pgfqpoint{7.329682in}{1.880871in}}{\pgfqpoint{7.334725in}{1.880871in}}%
\pgfpathclose%
\pgfusepath{fill}%
\end{pgfscope}%
\begin{pgfscope}%
\pgfpathrectangle{\pgfqpoint{6.572727in}{0.474100in}}{\pgfqpoint{4.227273in}{3.318700in}}%
\pgfusepath{clip}%
\pgfsetbuttcap%
\pgfsetroundjoin%
\definecolor{currentfill}{rgb}{0.127568,0.566949,0.550556}%
\pgfsetfillcolor{currentfill}%
\pgfsetfillopacity{0.700000}%
\pgfsetlinewidth{0.000000pt}%
\definecolor{currentstroke}{rgb}{0.000000,0.000000,0.000000}%
\pgfsetstrokecolor{currentstroke}%
\pgfsetstrokeopacity{0.700000}%
\pgfsetdash{}{0pt}%
\pgfpathmoveto{\pgfqpoint{7.490342in}{1.400745in}}%
\pgfpathcurveto{\pgfqpoint{7.495385in}{1.400745in}}{\pgfqpoint{7.500223in}{1.402749in}}{\pgfqpoint{7.503790in}{1.406316in}}%
\pgfpathcurveto{\pgfqpoint{7.507356in}{1.409882in}}{\pgfqpoint{7.509360in}{1.414720in}}{\pgfqpoint{7.509360in}{1.419764in}}%
\pgfpathcurveto{\pgfqpoint{7.509360in}{1.424807in}}{\pgfqpoint{7.507356in}{1.429645in}}{\pgfqpoint{7.503790in}{1.433211in}}%
\pgfpathcurveto{\pgfqpoint{7.500223in}{1.436778in}}{\pgfqpoint{7.495385in}{1.438782in}}{\pgfqpoint{7.490342in}{1.438782in}}%
\pgfpathcurveto{\pgfqpoint{7.485298in}{1.438782in}}{\pgfqpoint{7.480460in}{1.436778in}}{\pgfqpoint{7.476894in}{1.433211in}}%
\pgfpathcurveto{\pgfqpoint{7.473327in}{1.429645in}}{\pgfqpoint{7.471324in}{1.424807in}}{\pgfqpoint{7.471324in}{1.419764in}}%
\pgfpathcurveto{\pgfqpoint{7.471324in}{1.414720in}}{\pgfqpoint{7.473327in}{1.409882in}}{\pgfqpoint{7.476894in}{1.406316in}}%
\pgfpathcurveto{\pgfqpoint{7.480460in}{1.402749in}}{\pgfqpoint{7.485298in}{1.400745in}}{\pgfqpoint{7.490342in}{1.400745in}}%
\pgfpathclose%
\pgfusepath{fill}%
\end{pgfscope}%
\begin{pgfscope}%
\pgfpathrectangle{\pgfqpoint{6.572727in}{0.474100in}}{\pgfqpoint{4.227273in}{3.318700in}}%
\pgfusepath{clip}%
\pgfsetbuttcap%
\pgfsetroundjoin%
\definecolor{currentfill}{rgb}{0.127568,0.566949,0.550556}%
\pgfsetfillcolor{currentfill}%
\pgfsetfillopacity{0.700000}%
\pgfsetlinewidth{0.000000pt}%
\definecolor{currentstroke}{rgb}{0.000000,0.000000,0.000000}%
\pgfsetstrokecolor{currentstroke}%
\pgfsetstrokeopacity{0.700000}%
\pgfsetdash{}{0pt}%
\pgfpathmoveto{\pgfqpoint{7.807482in}{1.485187in}}%
\pgfpathcurveto{\pgfqpoint{7.812525in}{1.485187in}}{\pgfqpoint{7.817363in}{1.487191in}}{\pgfqpoint{7.820929in}{1.490758in}}%
\pgfpathcurveto{\pgfqpoint{7.824496in}{1.494324in}}{\pgfqpoint{7.826500in}{1.499162in}}{\pgfqpoint{7.826500in}{1.504205in}}%
\pgfpathcurveto{\pgfqpoint{7.826500in}{1.509249in}}{\pgfqpoint{7.824496in}{1.514087in}}{\pgfqpoint{7.820929in}{1.517653in}}%
\pgfpathcurveto{\pgfqpoint{7.817363in}{1.521220in}}{\pgfqpoint{7.812525in}{1.523224in}}{\pgfqpoint{7.807482in}{1.523224in}}%
\pgfpathcurveto{\pgfqpoint{7.802438in}{1.523224in}}{\pgfqpoint{7.797600in}{1.521220in}}{\pgfqpoint{7.794034in}{1.517653in}}%
\pgfpathcurveto{\pgfqpoint{7.790467in}{1.514087in}}{\pgfqpoint{7.788463in}{1.509249in}}{\pgfqpoint{7.788463in}{1.504205in}}%
\pgfpathcurveto{\pgfqpoint{7.788463in}{1.499162in}}{\pgfqpoint{7.790467in}{1.494324in}}{\pgfqpoint{7.794034in}{1.490758in}}%
\pgfpathcurveto{\pgfqpoint{7.797600in}{1.487191in}}{\pgfqpoint{7.802438in}{1.485187in}}{\pgfqpoint{7.807482in}{1.485187in}}%
\pgfpathclose%
\pgfusepath{fill}%
\end{pgfscope}%
\begin{pgfscope}%
\pgfpathrectangle{\pgfqpoint{6.572727in}{0.474100in}}{\pgfqpoint{4.227273in}{3.318700in}}%
\pgfusepath{clip}%
\pgfsetbuttcap%
\pgfsetroundjoin%
\definecolor{currentfill}{rgb}{0.127568,0.566949,0.550556}%
\pgfsetfillcolor{currentfill}%
\pgfsetfillopacity{0.700000}%
\pgfsetlinewidth{0.000000pt}%
\definecolor{currentstroke}{rgb}{0.000000,0.000000,0.000000}%
\pgfsetstrokecolor{currentstroke}%
\pgfsetstrokeopacity{0.700000}%
\pgfsetdash{}{0pt}%
\pgfpathmoveto{\pgfqpoint{7.499513in}{2.138853in}}%
\pgfpathcurveto{\pgfqpoint{7.504557in}{2.138853in}}{\pgfqpoint{7.509394in}{2.140857in}}{\pgfqpoint{7.512961in}{2.144423in}}%
\pgfpathcurveto{\pgfqpoint{7.516527in}{2.147989in}}{\pgfqpoint{7.518531in}{2.152827in}}{\pgfqpoint{7.518531in}{2.157871in}}%
\pgfpathcurveto{\pgfqpoint{7.518531in}{2.162914in}}{\pgfqpoint{7.516527in}{2.167752in}}{\pgfqpoint{7.512961in}{2.171319in}}%
\pgfpathcurveto{\pgfqpoint{7.509394in}{2.174885in}}{\pgfqpoint{7.504557in}{2.176889in}}{\pgfqpoint{7.499513in}{2.176889in}}%
\pgfpathcurveto{\pgfqpoint{7.494469in}{2.176889in}}{\pgfqpoint{7.489632in}{2.174885in}}{\pgfqpoint{7.486065in}{2.171319in}}%
\pgfpathcurveto{\pgfqpoint{7.482499in}{2.167752in}}{\pgfqpoint{7.480495in}{2.162914in}}{\pgfqpoint{7.480495in}{2.157871in}}%
\pgfpathcurveto{\pgfqpoint{7.480495in}{2.152827in}}{\pgfqpoint{7.482499in}{2.147989in}}{\pgfqpoint{7.486065in}{2.144423in}}%
\pgfpathcurveto{\pgfqpoint{7.489632in}{2.140857in}}{\pgfqpoint{7.494469in}{2.138853in}}{\pgfqpoint{7.499513in}{2.138853in}}%
\pgfpathclose%
\pgfusepath{fill}%
\end{pgfscope}%
\begin{pgfscope}%
\pgfpathrectangle{\pgfqpoint{6.572727in}{0.474100in}}{\pgfqpoint{4.227273in}{3.318700in}}%
\pgfusepath{clip}%
\pgfsetbuttcap%
\pgfsetroundjoin%
\definecolor{currentfill}{rgb}{0.127568,0.566949,0.550556}%
\pgfsetfillcolor{currentfill}%
\pgfsetfillopacity{0.700000}%
\pgfsetlinewidth{0.000000pt}%
\definecolor{currentstroke}{rgb}{0.000000,0.000000,0.000000}%
\pgfsetstrokecolor{currentstroke}%
\pgfsetstrokeopacity{0.700000}%
\pgfsetdash{}{0pt}%
\pgfpathmoveto{\pgfqpoint{7.911405in}{2.831374in}}%
\pgfpathcurveto{\pgfqpoint{7.916448in}{2.831374in}}{\pgfqpoint{7.921286in}{2.833378in}}{\pgfqpoint{7.924853in}{2.836944in}}%
\pgfpathcurveto{\pgfqpoint{7.928419in}{2.840511in}}{\pgfqpoint{7.930423in}{2.845348in}}{\pgfqpoint{7.930423in}{2.850392in}}%
\pgfpathcurveto{\pgfqpoint{7.930423in}{2.855436in}}{\pgfqpoint{7.928419in}{2.860273in}}{\pgfqpoint{7.924853in}{2.863840in}}%
\pgfpathcurveto{\pgfqpoint{7.921286in}{2.867406in}}{\pgfqpoint{7.916448in}{2.869410in}}{\pgfqpoint{7.911405in}{2.869410in}}%
\pgfpathcurveto{\pgfqpoint{7.906361in}{2.869410in}}{\pgfqpoint{7.901523in}{2.867406in}}{\pgfqpoint{7.897957in}{2.863840in}}%
\pgfpathcurveto{\pgfqpoint{7.894390in}{2.860273in}}{\pgfqpoint{7.892387in}{2.855436in}}{\pgfqpoint{7.892387in}{2.850392in}}%
\pgfpathcurveto{\pgfqpoint{7.892387in}{2.845348in}}{\pgfqpoint{7.894390in}{2.840511in}}{\pgfqpoint{7.897957in}{2.836944in}}%
\pgfpathcurveto{\pgfqpoint{7.901523in}{2.833378in}}{\pgfqpoint{7.906361in}{2.831374in}}{\pgfqpoint{7.911405in}{2.831374in}}%
\pgfpathclose%
\pgfusepath{fill}%
\end{pgfscope}%
\begin{pgfscope}%
\pgfpathrectangle{\pgfqpoint{6.572727in}{0.474100in}}{\pgfqpoint{4.227273in}{3.318700in}}%
\pgfusepath{clip}%
\pgfsetbuttcap%
\pgfsetroundjoin%
\definecolor{currentfill}{rgb}{0.127568,0.566949,0.550556}%
\pgfsetfillcolor{currentfill}%
\pgfsetfillopacity{0.700000}%
\pgfsetlinewidth{0.000000pt}%
\definecolor{currentstroke}{rgb}{0.000000,0.000000,0.000000}%
\pgfsetstrokecolor{currentstroke}%
\pgfsetstrokeopacity{0.700000}%
\pgfsetdash{}{0pt}%
\pgfpathmoveto{\pgfqpoint{7.651751in}{2.742616in}}%
\pgfpathcurveto{\pgfqpoint{7.656794in}{2.742616in}}{\pgfqpoint{7.661632in}{2.744620in}}{\pgfqpoint{7.665198in}{2.748186in}}%
\pgfpathcurveto{\pgfqpoint{7.668765in}{2.751753in}}{\pgfqpoint{7.670769in}{2.756590in}}{\pgfqpoint{7.670769in}{2.761634in}}%
\pgfpathcurveto{\pgfqpoint{7.670769in}{2.766678in}}{\pgfqpoint{7.668765in}{2.771515in}}{\pgfqpoint{7.665198in}{2.775082in}}%
\pgfpathcurveto{\pgfqpoint{7.661632in}{2.778648in}}{\pgfqpoint{7.656794in}{2.780652in}}{\pgfqpoint{7.651751in}{2.780652in}}%
\pgfpathcurveto{\pgfqpoint{7.646707in}{2.780652in}}{\pgfqpoint{7.641869in}{2.778648in}}{\pgfqpoint{7.638303in}{2.775082in}}%
\pgfpathcurveto{\pgfqpoint{7.634736in}{2.771515in}}{\pgfqpoint{7.632732in}{2.766678in}}{\pgfqpoint{7.632732in}{2.761634in}}%
\pgfpathcurveto{\pgfqpoint{7.632732in}{2.756590in}}{\pgfqpoint{7.634736in}{2.751753in}}{\pgfqpoint{7.638303in}{2.748186in}}%
\pgfpathcurveto{\pgfqpoint{7.641869in}{2.744620in}}{\pgfqpoint{7.646707in}{2.742616in}}{\pgfqpoint{7.651751in}{2.742616in}}%
\pgfpathclose%
\pgfusepath{fill}%
\end{pgfscope}%
\begin{pgfscope}%
\pgfpathrectangle{\pgfqpoint{6.572727in}{0.474100in}}{\pgfqpoint{4.227273in}{3.318700in}}%
\pgfusepath{clip}%
\pgfsetbuttcap%
\pgfsetroundjoin%
\definecolor{currentfill}{rgb}{0.993248,0.906157,0.143936}%
\pgfsetfillcolor{currentfill}%
\pgfsetfillopacity{0.700000}%
\pgfsetlinewidth{0.000000pt}%
\definecolor{currentstroke}{rgb}{0.000000,0.000000,0.000000}%
\pgfsetstrokecolor{currentstroke}%
\pgfsetstrokeopacity{0.700000}%
\pgfsetdash{}{0pt}%
\pgfpathmoveto{\pgfqpoint{9.465287in}{1.342564in}}%
\pgfpathcurveto{\pgfqpoint{9.470330in}{1.342564in}}{\pgfqpoint{9.475168in}{1.344568in}}{\pgfqpoint{9.478735in}{1.348134in}}%
\pgfpathcurveto{\pgfqpoint{9.482301in}{1.351701in}}{\pgfqpoint{9.484305in}{1.356539in}}{\pgfqpoint{9.484305in}{1.361582in}}%
\pgfpathcurveto{\pgfqpoint{9.484305in}{1.366626in}}{\pgfqpoint{9.482301in}{1.371464in}}{\pgfqpoint{9.478735in}{1.375030in}}%
\pgfpathcurveto{\pgfqpoint{9.475168in}{1.378596in}}{\pgfqpoint{9.470330in}{1.380600in}}{\pgfqpoint{9.465287in}{1.380600in}}%
\pgfpathcurveto{\pgfqpoint{9.460243in}{1.380600in}}{\pgfqpoint{9.455405in}{1.378596in}}{\pgfqpoint{9.451839in}{1.375030in}}%
\pgfpathcurveto{\pgfqpoint{9.448273in}{1.371464in}}{\pgfqpoint{9.446269in}{1.366626in}}{\pgfqpoint{9.446269in}{1.361582in}}%
\pgfpathcurveto{\pgfqpoint{9.446269in}{1.356539in}}{\pgfqpoint{9.448273in}{1.351701in}}{\pgfqpoint{9.451839in}{1.348134in}}%
\pgfpathcurveto{\pgfqpoint{9.455405in}{1.344568in}}{\pgfqpoint{9.460243in}{1.342564in}}{\pgfqpoint{9.465287in}{1.342564in}}%
\pgfpathclose%
\pgfusepath{fill}%
\end{pgfscope}%
\begin{pgfscope}%
\pgfpathrectangle{\pgfqpoint{6.572727in}{0.474100in}}{\pgfqpoint{4.227273in}{3.318700in}}%
\pgfusepath{clip}%
\pgfsetbuttcap%
\pgfsetroundjoin%
\definecolor{currentfill}{rgb}{0.127568,0.566949,0.550556}%
\pgfsetfillcolor{currentfill}%
\pgfsetfillopacity{0.700000}%
\pgfsetlinewidth{0.000000pt}%
\definecolor{currentstroke}{rgb}{0.000000,0.000000,0.000000}%
\pgfsetstrokecolor{currentstroke}%
\pgfsetstrokeopacity{0.700000}%
\pgfsetdash{}{0pt}%
\pgfpathmoveto{\pgfqpoint{8.266369in}{1.547802in}}%
\pgfpathcurveto{\pgfqpoint{8.271412in}{1.547802in}}{\pgfqpoint{8.276250in}{1.549806in}}{\pgfqpoint{8.279816in}{1.553373in}}%
\pgfpathcurveto{\pgfqpoint{8.283383in}{1.556939in}}{\pgfqpoint{8.285387in}{1.561777in}}{\pgfqpoint{8.285387in}{1.566821in}}%
\pgfpathcurveto{\pgfqpoint{8.285387in}{1.571864in}}{\pgfqpoint{8.283383in}{1.576702in}}{\pgfqpoint{8.279816in}{1.580268in}}%
\pgfpathcurveto{\pgfqpoint{8.276250in}{1.583835in}}{\pgfqpoint{8.271412in}{1.585839in}}{\pgfqpoint{8.266369in}{1.585839in}}%
\pgfpathcurveto{\pgfqpoint{8.261325in}{1.585839in}}{\pgfqpoint{8.256487in}{1.583835in}}{\pgfqpoint{8.252921in}{1.580268in}}%
\pgfpathcurveto{\pgfqpoint{8.249354in}{1.576702in}}{\pgfqpoint{8.247350in}{1.571864in}}{\pgfqpoint{8.247350in}{1.566821in}}%
\pgfpathcurveto{\pgfqpoint{8.247350in}{1.561777in}}{\pgfqpoint{8.249354in}{1.556939in}}{\pgfqpoint{8.252921in}{1.553373in}}%
\pgfpathcurveto{\pgfqpoint{8.256487in}{1.549806in}}{\pgfqpoint{8.261325in}{1.547802in}}{\pgfqpoint{8.266369in}{1.547802in}}%
\pgfpathclose%
\pgfusepath{fill}%
\end{pgfscope}%
\begin{pgfscope}%
\pgfpathrectangle{\pgfqpoint{6.572727in}{0.474100in}}{\pgfqpoint{4.227273in}{3.318700in}}%
\pgfusepath{clip}%
\pgfsetbuttcap%
\pgfsetroundjoin%
\definecolor{currentfill}{rgb}{0.127568,0.566949,0.550556}%
\pgfsetfillcolor{currentfill}%
\pgfsetfillopacity{0.700000}%
\pgfsetlinewidth{0.000000pt}%
\definecolor{currentstroke}{rgb}{0.000000,0.000000,0.000000}%
\pgfsetstrokecolor{currentstroke}%
\pgfsetstrokeopacity{0.700000}%
\pgfsetdash{}{0pt}%
\pgfpathmoveto{\pgfqpoint{7.877706in}{1.910149in}}%
\pgfpathcurveto{\pgfqpoint{7.882750in}{1.910149in}}{\pgfqpoint{7.887588in}{1.912153in}}{\pgfqpoint{7.891154in}{1.915720in}}%
\pgfpathcurveto{\pgfqpoint{7.894721in}{1.919286in}}{\pgfqpoint{7.896725in}{1.924124in}}{\pgfqpoint{7.896725in}{1.929168in}}%
\pgfpathcurveto{\pgfqpoint{7.896725in}{1.934211in}}{\pgfqpoint{7.894721in}{1.939049in}}{\pgfqpoint{7.891154in}{1.942615in}}%
\pgfpathcurveto{\pgfqpoint{7.887588in}{1.946182in}}{\pgfqpoint{7.882750in}{1.948186in}}{\pgfqpoint{7.877706in}{1.948186in}}%
\pgfpathcurveto{\pgfqpoint{7.872663in}{1.948186in}}{\pgfqpoint{7.867825in}{1.946182in}}{\pgfqpoint{7.864259in}{1.942615in}}%
\pgfpathcurveto{\pgfqpoint{7.860692in}{1.939049in}}{\pgfqpoint{7.858688in}{1.934211in}}{\pgfqpoint{7.858688in}{1.929168in}}%
\pgfpathcurveto{\pgfqpoint{7.858688in}{1.924124in}}{\pgfqpoint{7.860692in}{1.919286in}}{\pgfqpoint{7.864259in}{1.915720in}}%
\pgfpathcurveto{\pgfqpoint{7.867825in}{1.912153in}}{\pgfqpoint{7.872663in}{1.910149in}}{\pgfqpoint{7.877706in}{1.910149in}}%
\pgfpathclose%
\pgfusepath{fill}%
\end{pgfscope}%
\begin{pgfscope}%
\pgfpathrectangle{\pgfqpoint{6.572727in}{0.474100in}}{\pgfqpoint{4.227273in}{3.318700in}}%
\pgfusepath{clip}%
\pgfsetbuttcap%
\pgfsetroundjoin%
\definecolor{currentfill}{rgb}{0.127568,0.566949,0.550556}%
\pgfsetfillcolor{currentfill}%
\pgfsetfillopacity{0.700000}%
\pgfsetlinewidth{0.000000pt}%
\definecolor{currentstroke}{rgb}{0.000000,0.000000,0.000000}%
\pgfsetstrokecolor{currentstroke}%
\pgfsetstrokeopacity{0.700000}%
\pgfsetdash{}{0pt}%
\pgfpathmoveto{\pgfqpoint{8.063057in}{1.847324in}}%
\pgfpathcurveto{\pgfqpoint{8.068101in}{1.847324in}}{\pgfqpoint{8.072939in}{1.849328in}}{\pgfqpoint{8.076505in}{1.852895in}}%
\pgfpathcurveto{\pgfqpoint{8.080072in}{1.856461in}}{\pgfqpoint{8.082075in}{1.861299in}}{\pgfqpoint{8.082075in}{1.866343in}}%
\pgfpathcurveto{\pgfqpoint{8.082075in}{1.871386in}}{\pgfqpoint{8.080072in}{1.876224in}}{\pgfqpoint{8.076505in}{1.879790in}}%
\pgfpathcurveto{\pgfqpoint{8.072939in}{1.883357in}}{\pgfqpoint{8.068101in}{1.885361in}}{\pgfqpoint{8.063057in}{1.885361in}}%
\pgfpathcurveto{\pgfqpoint{8.058014in}{1.885361in}}{\pgfqpoint{8.053176in}{1.883357in}}{\pgfqpoint{8.049609in}{1.879790in}}%
\pgfpathcurveto{\pgfqpoint{8.046043in}{1.876224in}}{\pgfqpoint{8.044039in}{1.871386in}}{\pgfqpoint{8.044039in}{1.866343in}}%
\pgfpathcurveto{\pgfqpoint{8.044039in}{1.861299in}}{\pgfqpoint{8.046043in}{1.856461in}}{\pgfqpoint{8.049609in}{1.852895in}}%
\pgfpathcurveto{\pgfqpoint{8.053176in}{1.849328in}}{\pgfqpoint{8.058014in}{1.847324in}}{\pgfqpoint{8.063057in}{1.847324in}}%
\pgfpathclose%
\pgfusepath{fill}%
\end{pgfscope}%
\begin{pgfscope}%
\pgfpathrectangle{\pgfqpoint{6.572727in}{0.474100in}}{\pgfqpoint{4.227273in}{3.318700in}}%
\pgfusepath{clip}%
\pgfsetbuttcap%
\pgfsetroundjoin%
\definecolor{currentfill}{rgb}{0.127568,0.566949,0.550556}%
\pgfsetfillcolor{currentfill}%
\pgfsetfillopacity{0.700000}%
\pgfsetlinewidth{0.000000pt}%
\definecolor{currentstroke}{rgb}{0.000000,0.000000,0.000000}%
\pgfsetstrokecolor{currentstroke}%
\pgfsetstrokeopacity{0.700000}%
\pgfsetdash{}{0pt}%
\pgfpathmoveto{\pgfqpoint{7.997701in}{2.693385in}}%
\pgfpathcurveto{\pgfqpoint{8.002744in}{2.693385in}}{\pgfqpoint{8.007582in}{2.695388in}}{\pgfqpoint{8.011148in}{2.698955in}}%
\pgfpathcurveto{\pgfqpoint{8.014715in}{2.702521in}}{\pgfqpoint{8.016719in}{2.707359in}}{\pgfqpoint{8.016719in}{2.712403in}}%
\pgfpathcurveto{\pgfqpoint{8.016719in}{2.717446in}}{\pgfqpoint{8.014715in}{2.722284in}}{\pgfqpoint{8.011148in}{2.725851in}}%
\pgfpathcurveto{\pgfqpoint{8.007582in}{2.729417in}}{\pgfqpoint{8.002744in}{2.731421in}}{\pgfqpoint{7.997701in}{2.731421in}}%
\pgfpathcurveto{\pgfqpoint{7.992657in}{2.731421in}}{\pgfqpoint{7.987819in}{2.729417in}}{\pgfqpoint{7.984253in}{2.725851in}}%
\pgfpathcurveto{\pgfqpoint{7.980686in}{2.722284in}}{\pgfqpoint{7.978682in}{2.717446in}}{\pgfqpoint{7.978682in}{2.712403in}}%
\pgfpathcurveto{\pgfqpoint{7.978682in}{2.707359in}}{\pgfqpoint{7.980686in}{2.702521in}}{\pgfqpoint{7.984253in}{2.698955in}}%
\pgfpathcurveto{\pgfqpoint{7.987819in}{2.695388in}}{\pgfqpoint{7.992657in}{2.693385in}}{\pgfqpoint{7.997701in}{2.693385in}}%
\pgfpathclose%
\pgfusepath{fill}%
\end{pgfscope}%
\begin{pgfscope}%
\pgfpathrectangle{\pgfqpoint{6.572727in}{0.474100in}}{\pgfqpoint{4.227273in}{3.318700in}}%
\pgfusepath{clip}%
\pgfsetbuttcap%
\pgfsetroundjoin%
\definecolor{currentfill}{rgb}{0.993248,0.906157,0.143936}%
\pgfsetfillcolor{currentfill}%
\pgfsetfillopacity{0.700000}%
\pgfsetlinewidth{0.000000pt}%
\definecolor{currentstroke}{rgb}{0.000000,0.000000,0.000000}%
\pgfsetstrokecolor{currentstroke}%
\pgfsetstrokeopacity{0.700000}%
\pgfsetdash{}{0pt}%
\pgfpathmoveto{\pgfqpoint{9.206261in}{1.890701in}}%
\pgfpathcurveto{\pgfqpoint{9.211304in}{1.890701in}}{\pgfqpoint{9.216142in}{1.892704in}}{\pgfqpoint{9.219708in}{1.896271in}}%
\pgfpathcurveto{\pgfqpoint{9.223275in}{1.899837in}}{\pgfqpoint{9.225279in}{1.904675in}}{\pgfqpoint{9.225279in}{1.909719in}}%
\pgfpathcurveto{\pgfqpoint{9.225279in}{1.914762in}}{\pgfqpoint{9.223275in}{1.919600in}}{\pgfqpoint{9.219708in}{1.923167in}}%
\pgfpathcurveto{\pgfqpoint{9.216142in}{1.926733in}}{\pgfqpoint{9.211304in}{1.928737in}}{\pgfqpoint{9.206261in}{1.928737in}}%
\pgfpathcurveto{\pgfqpoint{9.201217in}{1.928737in}}{\pgfqpoint{9.196379in}{1.926733in}}{\pgfqpoint{9.192813in}{1.923167in}}%
\pgfpathcurveto{\pgfqpoint{9.189246in}{1.919600in}}{\pgfqpoint{9.187242in}{1.914762in}}{\pgfqpoint{9.187242in}{1.909719in}}%
\pgfpathcurveto{\pgfqpoint{9.187242in}{1.904675in}}{\pgfqpoint{9.189246in}{1.899837in}}{\pgfqpoint{9.192813in}{1.896271in}}%
\pgfpathcurveto{\pgfqpoint{9.196379in}{1.892704in}}{\pgfqpoint{9.201217in}{1.890701in}}{\pgfqpoint{9.206261in}{1.890701in}}%
\pgfpathclose%
\pgfusepath{fill}%
\end{pgfscope}%
\begin{pgfscope}%
\pgfpathrectangle{\pgfqpoint{6.572727in}{0.474100in}}{\pgfqpoint{4.227273in}{3.318700in}}%
\pgfusepath{clip}%
\pgfsetbuttcap%
\pgfsetroundjoin%
\definecolor{currentfill}{rgb}{0.127568,0.566949,0.550556}%
\pgfsetfillcolor{currentfill}%
\pgfsetfillopacity{0.700000}%
\pgfsetlinewidth{0.000000pt}%
\definecolor{currentstroke}{rgb}{0.000000,0.000000,0.000000}%
\pgfsetstrokecolor{currentstroke}%
\pgfsetstrokeopacity{0.700000}%
\pgfsetdash{}{0pt}%
\pgfpathmoveto{\pgfqpoint{7.603169in}{1.531369in}}%
\pgfpathcurveto{\pgfqpoint{7.608213in}{1.531369in}}{\pgfqpoint{7.613051in}{1.533373in}}{\pgfqpoint{7.616617in}{1.536939in}}%
\pgfpathcurveto{\pgfqpoint{7.620184in}{1.540506in}}{\pgfqpoint{7.622188in}{1.545343in}}{\pgfqpoint{7.622188in}{1.550387in}}%
\pgfpathcurveto{\pgfqpoint{7.622188in}{1.555431in}}{\pgfqpoint{7.620184in}{1.560268in}}{\pgfqpoint{7.616617in}{1.563835in}}%
\pgfpathcurveto{\pgfqpoint{7.613051in}{1.567401in}}{\pgfqpoint{7.608213in}{1.569405in}}{\pgfqpoint{7.603169in}{1.569405in}}%
\pgfpathcurveto{\pgfqpoint{7.598126in}{1.569405in}}{\pgfqpoint{7.593288in}{1.567401in}}{\pgfqpoint{7.589722in}{1.563835in}}%
\pgfpathcurveto{\pgfqpoint{7.586155in}{1.560268in}}{\pgfqpoint{7.584151in}{1.555431in}}{\pgfqpoint{7.584151in}{1.550387in}}%
\pgfpathcurveto{\pgfqpoint{7.584151in}{1.545343in}}{\pgfqpoint{7.586155in}{1.540506in}}{\pgfqpoint{7.589722in}{1.536939in}}%
\pgfpathcurveto{\pgfqpoint{7.593288in}{1.533373in}}{\pgfqpoint{7.598126in}{1.531369in}}{\pgfqpoint{7.603169in}{1.531369in}}%
\pgfpathclose%
\pgfusepath{fill}%
\end{pgfscope}%
\begin{pgfscope}%
\pgfpathrectangle{\pgfqpoint{6.572727in}{0.474100in}}{\pgfqpoint{4.227273in}{3.318700in}}%
\pgfusepath{clip}%
\pgfsetbuttcap%
\pgfsetroundjoin%
\definecolor{currentfill}{rgb}{0.127568,0.566949,0.550556}%
\pgfsetfillcolor{currentfill}%
\pgfsetfillopacity{0.700000}%
\pgfsetlinewidth{0.000000pt}%
\definecolor{currentstroke}{rgb}{0.000000,0.000000,0.000000}%
\pgfsetstrokecolor{currentstroke}%
\pgfsetstrokeopacity{0.700000}%
\pgfsetdash{}{0pt}%
\pgfpathmoveto{\pgfqpoint{8.624924in}{2.479891in}}%
\pgfpathcurveto{\pgfqpoint{8.629968in}{2.479891in}}{\pgfqpoint{8.634806in}{2.481895in}}{\pgfqpoint{8.638372in}{2.485461in}}%
\pgfpathcurveto{\pgfqpoint{8.641938in}{2.489028in}}{\pgfqpoint{8.643942in}{2.493866in}}{\pgfqpoint{8.643942in}{2.498909in}}%
\pgfpathcurveto{\pgfqpoint{8.643942in}{2.503953in}}{\pgfqpoint{8.641938in}{2.508791in}}{\pgfqpoint{8.638372in}{2.512357in}}%
\pgfpathcurveto{\pgfqpoint{8.634806in}{2.515924in}}{\pgfqpoint{8.629968in}{2.517927in}}{\pgfqpoint{8.624924in}{2.517927in}}%
\pgfpathcurveto{\pgfqpoint{8.619880in}{2.517927in}}{\pgfqpoint{8.615043in}{2.515924in}}{\pgfqpoint{8.611476in}{2.512357in}}%
\pgfpathcurveto{\pgfqpoint{8.607910in}{2.508791in}}{\pgfqpoint{8.605906in}{2.503953in}}{\pgfqpoint{8.605906in}{2.498909in}}%
\pgfpathcurveto{\pgfqpoint{8.605906in}{2.493866in}}{\pgfqpoint{8.607910in}{2.489028in}}{\pgfqpoint{8.611476in}{2.485461in}}%
\pgfpathcurveto{\pgfqpoint{8.615043in}{2.481895in}}{\pgfqpoint{8.619880in}{2.479891in}}{\pgfqpoint{8.624924in}{2.479891in}}%
\pgfpathclose%
\pgfusepath{fill}%
\end{pgfscope}%
\begin{pgfscope}%
\pgfpathrectangle{\pgfqpoint{6.572727in}{0.474100in}}{\pgfqpoint{4.227273in}{3.318700in}}%
\pgfusepath{clip}%
\pgfsetbuttcap%
\pgfsetroundjoin%
\definecolor{currentfill}{rgb}{0.993248,0.906157,0.143936}%
\pgfsetfillcolor{currentfill}%
\pgfsetfillopacity{0.700000}%
\pgfsetlinewidth{0.000000pt}%
\definecolor{currentstroke}{rgb}{0.000000,0.000000,0.000000}%
\pgfsetstrokecolor{currentstroke}%
\pgfsetstrokeopacity{0.700000}%
\pgfsetdash{}{0pt}%
\pgfpathmoveto{\pgfqpoint{9.678243in}{1.470786in}}%
\pgfpathcurveto{\pgfqpoint{9.683287in}{1.470786in}}{\pgfqpoint{9.688125in}{1.472789in}}{\pgfqpoint{9.691691in}{1.476356in}}%
\pgfpathcurveto{\pgfqpoint{9.695257in}{1.479922in}}{\pgfqpoint{9.697261in}{1.484760in}}{\pgfqpoint{9.697261in}{1.489804in}}%
\pgfpathcurveto{\pgfqpoint{9.697261in}{1.494847in}}{\pgfqpoint{9.695257in}{1.499685in}}{\pgfqpoint{9.691691in}{1.503252in}}%
\pgfpathcurveto{\pgfqpoint{9.688125in}{1.506818in}}{\pgfqpoint{9.683287in}{1.508822in}}{\pgfqpoint{9.678243in}{1.508822in}}%
\pgfpathcurveto{\pgfqpoint{9.673199in}{1.508822in}}{\pgfqpoint{9.668362in}{1.506818in}}{\pgfqpoint{9.664795in}{1.503252in}}%
\pgfpathcurveto{\pgfqpoint{9.661229in}{1.499685in}}{\pgfqpoint{9.659225in}{1.494847in}}{\pgfqpoint{9.659225in}{1.489804in}}%
\pgfpathcurveto{\pgfqpoint{9.659225in}{1.484760in}}{\pgfqpoint{9.661229in}{1.479922in}}{\pgfqpoint{9.664795in}{1.476356in}}%
\pgfpathcurveto{\pgfqpoint{9.668362in}{1.472789in}}{\pgfqpoint{9.673199in}{1.470786in}}{\pgfqpoint{9.678243in}{1.470786in}}%
\pgfpathclose%
\pgfusepath{fill}%
\end{pgfscope}%
\begin{pgfscope}%
\pgfpathrectangle{\pgfqpoint{6.572727in}{0.474100in}}{\pgfqpoint{4.227273in}{3.318700in}}%
\pgfusepath{clip}%
\pgfsetbuttcap%
\pgfsetroundjoin%
\definecolor{currentfill}{rgb}{0.993248,0.906157,0.143936}%
\pgfsetfillcolor{currentfill}%
\pgfsetfillopacity{0.700000}%
\pgfsetlinewidth{0.000000pt}%
\definecolor{currentstroke}{rgb}{0.000000,0.000000,0.000000}%
\pgfsetstrokecolor{currentstroke}%
\pgfsetstrokeopacity{0.700000}%
\pgfsetdash{}{0pt}%
\pgfpathmoveto{\pgfqpoint{10.026416in}{1.510231in}}%
\pgfpathcurveto{\pgfqpoint{10.031459in}{1.510231in}}{\pgfqpoint{10.036297in}{1.512235in}}{\pgfqpoint{10.039863in}{1.515802in}}%
\pgfpathcurveto{\pgfqpoint{10.043430in}{1.519368in}}{\pgfqpoint{10.045434in}{1.524206in}}{\pgfqpoint{10.045434in}{1.529250in}}%
\pgfpathcurveto{\pgfqpoint{10.045434in}{1.534293in}}{\pgfqpoint{10.043430in}{1.539131in}}{\pgfqpoint{10.039863in}{1.542697in}}%
\pgfpathcurveto{\pgfqpoint{10.036297in}{1.546264in}}{\pgfqpoint{10.031459in}{1.548268in}}{\pgfqpoint{10.026416in}{1.548268in}}%
\pgfpathcurveto{\pgfqpoint{10.021372in}{1.548268in}}{\pgfqpoint{10.016534in}{1.546264in}}{\pgfqpoint{10.012968in}{1.542697in}}%
\pgfpathcurveto{\pgfqpoint{10.009401in}{1.539131in}}{\pgfqpoint{10.007397in}{1.534293in}}{\pgfqpoint{10.007397in}{1.529250in}}%
\pgfpathcurveto{\pgfqpoint{10.007397in}{1.524206in}}{\pgfqpoint{10.009401in}{1.519368in}}{\pgfqpoint{10.012968in}{1.515802in}}%
\pgfpathcurveto{\pgfqpoint{10.016534in}{1.512235in}}{\pgfqpoint{10.021372in}{1.510231in}}{\pgfqpoint{10.026416in}{1.510231in}}%
\pgfpathclose%
\pgfusepath{fill}%
\end{pgfscope}%
\begin{pgfscope}%
\pgfpathrectangle{\pgfqpoint{6.572727in}{0.474100in}}{\pgfqpoint{4.227273in}{3.318700in}}%
\pgfusepath{clip}%
\pgfsetbuttcap%
\pgfsetroundjoin%
\definecolor{currentfill}{rgb}{0.993248,0.906157,0.143936}%
\pgfsetfillcolor{currentfill}%
\pgfsetfillopacity{0.700000}%
\pgfsetlinewidth{0.000000pt}%
\definecolor{currentstroke}{rgb}{0.000000,0.000000,0.000000}%
\pgfsetstrokecolor{currentstroke}%
\pgfsetstrokeopacity{0.700000}%
\pgfsetdash{}{0pt}%
\pgfpathmoveto{\pgfqpoint{9.879733in}{1.456269in}}%
\pgfpathcurveto{\pgfqpoint{9.884777in}{1.456269in}}{\pgfqpoint{9.889615in}{1.458273in}}{\pgfqpoint{9.893181in}{1.461840in}}%
\pgfpathcurveto{\pgfqpoint{9.896748in}{1.465406in}}{\pgfqpoint{9.898752in}{1.470244in}}{\pgfqpoint{9.898752in}{1.475287in}}%
\pgfpathcurveto{\pgfqpoint{9.898752in}{1.480331in}}{\pgfqpoint{9.896748in}{1.485169in}}{\pgfqpoint{9.893181in}{1.488735in}}%
\pgfpathcurveto{\pgfqpoint{9.889615in}{1.492302in}}{\pgfqpoint{9.884777in}{1.494306in}}{\pgfqpoint{9.879733in}{1.494306in}}%
\pgfpathcurveto{\pgfqpoint{9.874690in}{1.494306in}}{\pgfqpoint{9.869852in}{1.492302in}}{\pgfqpoint{9.866286in}{1.488735in}}%
\pgfpathcurveto{\pgfqpoint{9.862719in}{1.485169in}}{\pgfqpoint{9.860715in}{1.480331in}}{\pgfqpoint{9.860715in}{1.475287in}}%
\pgfpathcurveto{\pgfqpoint{9.860715in}{1.470244in}}{\pgfqpoint{9.862719in}{1.465406in}}{\pgfqpoint{9.866286in}{1.461840in}}%
\pgfpathcurveto{\pgfqpoint{9.869852in}{1.458273in}}{\pgfqpoint{9.874690in}{1.456269in}}{\pgfqpoint{9.879733in}{1.456269in}}%
\pgfpathclose%
\pgfusepath{fill}%
\end{pgfscope}%
\begin{pgfscope}%
\pgfpathrectangle{\pgfqpoint{6.572727in}{0.474100in}}{\pgfqpoint{4.227273in}{3.318700in}}%
\pgfusepath{clip}%
\pgfsetbuttcap%
\pgfsetroundjoin%
\definecolor{currentfill}{rgb}{0.127568,0.566949,0.550556}%
\pgfsetfillcolor{currentfill}%
\pgfsetfillopacity{0.700000}%
\pgfsetlinewidth{0.000000pt}%
\definecolor{currentstroke}{rgb}{0.000000,0.000000,0.000000}%
\pgfsetstrokecolor{currentstroke}%
\pgfsetstrokeopacity{0.700000}%
\pgfsetdash{}{0pt}%
\pgfpathmoveto{\pgfqpoint{8.054616in}{1.430629in}}%
\pgfpathcurveto{\pgfqpoint{8.059660in}{1.430629in}}{\pgfqpoint{8.064497in}{1.432633in}}{\pgfqpoint{8.068064in}{1.436199in}}%
\pgfpathcurveto{\pgfqpoint{8.071630in}{1.439765in}}{\pgfqpoint{8.073634in}{1.444603in}}{\pgfqpoint{8.073634in}{1.449647in}}%
\pgfpathcurveto{\pgfqpoint{8.073634in}{1.454691in}}{\pgfqpoint{8.071630in}{1.459528in}}{\pgfqpoint{8.068064in}{1.463095in}}%
\pgfpathcurveto{\pgfqpoint{8.064497in}{1.466661in}}{\pgfqpoint{8.059660in}{1.468665in}}{\pgfqpoint{8.054616in}{1.468665in}}%
\pgfpathcurveto{\pgfqpoint{8.049572in}{1.468665in}}{\pgfqpoint{8.044734in}{1.466661in}}{\pgfqpoint{8.041168in}{1.463095in}}%
\pgfpathcurveto{\pgfqpoint{8.037602in}{1.459528in}}{\pgfqpoint{8.035598in}{1.454691in}}{\pgfqpoint{8.035598in}{1.449647in}}%
\pgfpathcurveto{\pgfqpoint{8.035598in}{1.444603in}}{\pgfqpoint{8.037602in}{1.439765in}}{\pgfqpoint{8.041168in}{1.436199in}}%
\pgfpathcurveto{\pgfqpoint{8.044734in}{1.432633in}}{\pgfqpoint{8.049572in}{1.430629in}}{\pgfqpoint{8.054616in}{1.430629in}}%
\pgfpathclose%
\pgfusepath{fill}%
\end{pgfscope}%
\begin{pgfscope}%
\pgfpathrectangle{\pgfqpoint{6.572727in}{0.474100in}}{\pgfqpoint{4.227273in}{3.318700in}}%
\pgfusepath{clip}%
\pgfsetbuttcap%
\pgfsetroundjoin%
\definecolor{currentfill}{rgb}{0.127568,0.566949,0.550556}%
\pgfsetfillcolor{currentfill}%
\pgfsetfillopacity{0.700000}%
\pgfsetlinewidth{0.000000pt}%
\definecolor{currentstroke}{rgb}{0.000000,0.000000,0.000000}%
\pgfsetstrokecolor{currentstroke}%
\pgfsetstrokeopacity{0.700000}%
\pgfsetdash{}{0pt}%
\pgfpathmoveto{\pgfqpoint{8.951433in}{3.310421in}}%
\pgfpathcurveto{\pgfqpoint{8.956477in}{3.310421in}}{\pgfqpoint{8.961314in}{3.312425in}}{\pgfqpoint{8.964881in}{3.315992in}}%
\pgfpathcurveto{\pgfqpoint{8.968447in}{3.319558in}}{\pgfqpoint{8.970451in}{3.324396in}}{\pgfqpoint{8.970451in}{3.329439in}}%
\pgfpathcurveto{\pgfqpoint{8.970451in}{3.334483in}}{\pgfqpoint{8.968447in}{3.339321in}}{\pgfqpoint{8.964881in}{3.342887in}}%
\pgfpathcurveto{\pgfqpoint{8.961314in}{3.346454in}}{\pgfqpoint{8.956477in}{3.348458in}}{\pgfqpoint{8.951433in}{3.348458in}}%
\pgfpathcurveto{\pgfqpoint{8.946389in}{3.348458in}}{\pgfqpoint{8.941552in}{3.346454in}}{\pgfqpoint{8.937985in}{3.342887in}}%
\pgfpathcurveto{\pgfqpoint{8.934419in}{3.339321in}}{\pgfqpoint{8.932415in}{3.334483in}}{\pgfqpoint{8.932415in}{3.329439in}}%
\pgfpathcurveto{\pgfqpoint{8.932415in}{3.324396in}}{\pgfqpoint{8.934419in}{3.319558in}}{\pgfqpoint{8.937985in}{3.315992in}}%
\pgfpathcurveto{\pgfqpoint{8.941552in}{3.312425in}}{\pgfqpoint{8.946389in}{3.310421in}}{\pgfqpoint{8.951433in}{3.310421in}}%
\pgfpathclose%
\pgfusepath{fill}%
\end{pgfscope}%
\begin{pgfscope}%
\pgfpathrectangle{\pgfqpoint{6.572727in}{0.474100in}}{\pgfqpoint{4.227273in}{3.318700in}}%
\pgfusepath{clip}%
\pgfsetbuttcap%
\pgfsetroundjoin%
\definecolor{currentfill}{rgb}{0.127568,0.566949,0.550556}%
\pgfsetfillcolor{currentfill}%
\pgfsetfillopacity{0.700000}%
\pgfsetlinewidth{0.000000pt}%
\definecolor{currentstroke}{rgb}{0.000000,0.000000,0.000000}%
\pgfsetstrokecolor{currentstroke}%
\pgfsetstrokeopacity{0.700000}%
\pgfsetdash{}{0pt}%
\pgfpathmoveto{\pgfqpoint{8.088472in}{1.734575in}}%
\pgfpathcurveto{\pgfqpoint{8.093516in}{1.734575in}}{\pgfqpoint{8.098354in}{1.736579in}}{\pgfqpoint{8.101920in}{1.740146in}}%
\pgfpathcurveto{\pgfqpoint{8.105486in}{1.743712in}}{\pgfqpoint{8.107490in}{1.748550in}}{\pgfqpoint{8.107490in}{1.753594in}}%
\pgfpathcurveto{\pgfqpoint{8.107490in}{1.758637in}}{\pgfqpoint{8.105486in}{1.763475in}}{\pgfqpoint{8.101920in}{1.767041in}}%
\pgfpathcurveto{\pgfqpoint{8.098354in}{1.770608in}}{\pgfqpoint{8.093516in}{1.772612in}}{\pgfqpoint{8.088472in}{1.772612in}}%
\pgfpathcurveto{\pgfqpoint{8.083428in}{1.772612in}}{\pgfqpoint{8.078591in}{1.770608in}}{\pgfqpoint{8.075024in}{1.767041in}}%
\pgfpathcurveto{\pgfqpoint{8.071458in}{1.763475in}}{\pgfqpoint{8.069454in}{1.758637in}}{\pgfqpoint{8.069454in}{1.753594in}}%
\pgfpathcurveto{\pgfqpoint{8.069454in}{1.748550in}}{\pgfqpoint{8.071458in}{1.743712in}}{\pgfqpoint{8.075024in}{1.740146in}}%
\pgfpathcurveto{\pgfqpoint{8.078591in}{1.736579in}}{\pgfqpoint{8.083428in}{1.734575in}}{\pgfqpoint{8.088472in}{1.734575in}}%
\pgfpathclose%
\pgfusepath{fill}%
\end{pgfscope}%
\begin{pgfscope}%
\pgfpathrectangle{\pgfqpoint{6.572727in}{0.474100in}}{\pgfqpoint{4.227273in}{3.318700in}}%
\pgfusepath{clip}%
\pgfsetbuttcap%
\pgfsetroundjoin%
\definecolor{currentfill}{rgb}{0.993248,0.906157,0.143936}%
\pgfsetfillcolor{currentfill}%
\pgfsetfillopacity{0.700000}%
\pgfsetlinewidth{0.000000pt}%
\definecolor{currentstroke}{rgb}{0.000000,0.000000,0.000000}%
\pgfsetstrokecolor{currentstroke}%
\pgfsetstrokeopacity{0.700000}%
\pgfsetdash{}{0pt}%
\pgfpathmoveto{\pgfqpoint{9.027756in}{1.385084in}}%
\pgfpathcurveto{\pgfqpoint{9.032799in}{1.385084in}}{\pgfqpoint{9.037637in}{1.387088in}}{\pgfqpoint{9.041204in}{1.390655in}}%
\pgfpathcurveto{\pgfqpoint{9.044770in}{1.394221in}}{\pgfqpoint{9.046774in}{1.399059in}}{\pgfqpoint{9.046774in}{1.404103in}}%
\pgfpathcurveto{\pgfqpoint{9.046774in}{1.409146in}}{\pgfqpoint{9.044770in}{1.413984in}}{\pgfqpoint{9.041204in}{1.417550in}}%
\pgfpathcurveto{\pgfqpoint{9.037637in}{1.421117in}}{\pgfqpoint{9.032799in}{1.423121in}}{\pgfqpoint{9.027756in}{1.423121in}}%
\pgfpathcurveto{\pgfqpoint{9.022712in}{1.423121in}}{\pgfqpoint{9.017874in}{1.421117in}}{\pgfqpoint{9.014308in}{1.417550in}}%
\pgfpathcurveto{\pgfqpoint{9.010742in}{1.413984in}}{\pgfqpoint{9.008738in}{1.409146in}}{\pgfqpoint{9.008738in}{1.404103in}}%
\pgfpathcurveto{\pgfqpoint{9.008738in}{1.399059in}}{\pgfqpoint{9.010742in}{1.394221in}}{\pgfqpoint{9.014308in}{1.390655in}}%
\pgfpathcurveto{\pgfqpoint{9.017874in}{1.387088in}}{\pgfqpoint{9.022712in}{1.385084in}}{\pgfqpoint{9.027756in}{1.385084in}}%
\pgfpathclose%
\pgfusepath{fill}%
\end{pgfscope}%
\begin{pgfscope}%
\pgfpathrectangle{\pgfqpoint{6.572727in}{0.474100in}}{\pgfqpoint{4.227273in}{3.318700in}}%
\pgfusepath{clip}%
\pgfsetbuttcap%
\pgfsetroundjoin%
\definecolor{currentfill}{rgb}{0.993248,0.906157,0.143936}%
\pgfsetfillcolor{currentfill}%
\pgfsetfillopacity{0.700000}%
\pgfsetlinewidth{0.000000pt}%
\definecolor{currentstroke}{rgb}{0.000000,0.000000,0.000000}%
\pgfsetstrokecolor{currentstroke}%
\pgfsetstrokeopacity{0.700000}%
\pgfsetdash{}{0pt}%
\pgfpathmoveto{\pgfqpoint{9.303541in}{1.776563in}}%
\pgfpathcurveto{\pgfqpoint{9.308585in}{1.776563in}}{\pgfqpoint{9.313423in}{1.778567in}}{\pgfqpoint{9.316989in}{1.782134in}}%
\pgfpathcurveto{\pgfqpoint{9.320555in}{1.785700in}}{\pgfqpoint{9.322559in}{1.790538in}}{\pgfqpoint{9.322559in}{1.795582in}}%
\pgfpathcurveto{\pgfqpoint{9.322559in}{1.800625in}}{\pgfqpoint{9.320555in}{1.805463in}}{\pgfqpoint{9.316989in}{1.809029in}}%
\pgfpathcurveto{\pgfqpoint{9.313423in}{1.812596in}}{\pgfqpoint{9.308585in}{1.814600in}}{\pgfqpoint{9.303541in}{1.814600in}}%
\pgfpathcurveto{\pgfqpoint{9.298497in}{1.814600in}}{\pgfqpoint{9.293660in}{1.812596in}}{\pgfqpoint{9.290093in}{1.809029in}}%
\pgfpathcurveto{\pgfqpoint{9.286527in}{1.805463in}}{\pgfqpoint{9.284523in}{1.800625in}}{\pgfqpoint{9.284523in}{1.795582in}}%
\pgfpathcurveto{\pgfqpoint{9.284523in}{1.790538in}}{\pgfqpoint{9.286527in}{1.785700in}}{\pgfqpoint{9.290093in}{1.782134in}}%
\pgfpathcurveto{\pgfqpoint{9.293660in}{1.778567in}}{\pgfqpoint{9.298497in}{1.776563in}}{\pgfqpoint{9.303541in}{1.776563in}}%
\pgfpathclose%
\pgfusepath{fill}%
\end{pgfscope}%
\begin{pgfscope}%
\pgfpathrectangle{\pgfqpoint{6.572727in}{0.474100in}}{\pgfqpoint{4.227273in}{3.318700in}}%
\pgfusepath{clip}%
\pgfsetbuttcap%
\pgfsetroundjoin%
\definecolor{currentfill}{rgb}{0.993248,0.906157,0.143936}%
\pgfsetfillcolor{currentfill}%
\pgfsetfillopacity{0.700000}%
\pgfsetlinewidth{0.000000pt}%
\definecolor{currentstroke}{rgb}{0.000000,0.000000,0.000000}%
\pgfsetstrokecolor{currentstroke}%
\pgfsetstrokeopacity{0.700000}%
\pgfsetdash{}{0pt}%
\pgfpathmoveto{\pgfqpoint{9.887278in}{1.473150in}}%
\pgfpathcurveto{\pgfqpoint{9.892322in}{1.473150in}}{\pgfqpoint{9.897160in}{1.475154in}}{\pgfqpoint{9.900726in}{1.478720in}}%
\pgfpathcurveto{\pgfqpoint{9.904293in}{1.482287in}}{\pgfqpoint{9.906297in}{1.487125in}}{\pgfqpoint{9.906297in}{1.492168in}}%
\pgfpathcurveto{\pgfqpoint{9.906297in}{1.497212in}}{\pgfqpoint{9.904293in}{1.502050in}}{\pgfqpoint{9.900726in}{1.505616in}}%
\pgfpathcurveto{\pgfqpoint{9.897160in}{1.509183in}}{\pgfqpoint{9.892322in}{1.511186in}}{\pgfqpoint{9.887278in}{1.511186in}}%
\pgfpathcurveto{\pgfqpoint{9.882235in}{1.511186in}}{\pgfqpoint{9.877397in}{1.509183in}}{\pgfqpoint{9.873831in}{1.505616in}}%
\pgfpathcurveto{\pgfqpoint{9.870264in}{1.502050in}}{\pgfqpoint{9.868260in}{1.497212in}}{\pgfqpoint{9.868260in}{1.492168in}}%
\pgfpathcurveto{\pgfqpoint{9.868260in}{1.487125in}}{\pgfqpoint{9.870264in}{1.482287in}}{\pgfqpoint{9.873831in}{1.478720in}}%
\pgfpathcurveto{\pgfqpoint{9.877397in}{1.475154in}}{\pgfqpoint{9.882235in}{1.473150in}}{\pgfqpoint{9.887278in}{1.473150in}}%
\pgfpathclose%
\pgfusepath{fill}%
\end{pgfscope}%
\begin{pgfscope}%
\pgfpathrectangle{\pgfqpoint{6.572727in}{0.474100in}}{\pgfqpoint{4.227273in}{3.318700in}}%
\pgfusepath{clip}%
\pgfsetbuttcap%
\pgfsetroundjoin%
\definecolor{currentfill}{rgb}{0.993248,0.906157,0.143936}%
\pgfsetfillcolor{currentfill}%
\pgfsetfillopacity{0.700000}%
\pgfsetlinewidth{0.000000pt}%
\definecolor{currentstroke}{rgb}{0.000000,0.000000,0.000000}%
\pgfsetstrokecolor{currentstroke}%
\pgfsetstrokeopacity{0.700000}%
\pgfsetdash{}{0pt}%
\pgfpathmoveto{\pgfqpoint{9.792150in}{1.421989in}}%
\pgfpathcurveto{\pgfqpoint{9.797194in}{1.421989in}}{\pgfqpoint{9.802031in}{1.423993in}}{\pgfqpoint{9.805598in}{1.427560in}}%
\pgfpathcurveto{\pgfqpoint{9.809164in}{1.431126in}}{\pgfqpoint{9.811168in}{1.435964in}}{\pgfqpoint{9.811168in}{1.441007in}}%
\pgfpathcurveto{\pgfqpoint{9.811168in}{1.446051in}}{\pgfqpoint{9.809164in}{1.450889in}}{\pgfqpoint{9.805598in}{1.454455in}}%
\pgfpathcurveto{\pgfqpoint{9.802031in}{1.458022in}}{\pgfqpoint{9.797194in}{1.460026in}}{\pgfqpoint{9.792150in}{1.460026in}}%
\pgfpathcurveto{\pgfqpoint{9.787106in}{1.460026in}}{\pgfqpoint{9.782269in}{1.458022in}}{\pgfqpoint{9.778702in}{1.454455in}}%
\pgfpathcurveto{\pgfqpoint{9.775136in}{1.450889in}}{\pgfqpoint{9.773132in}{1.446051in}}{\pgfqpoint{9.773132in}{1.441007in}}%
\pgfpathcurveto{\pgfqpoint{9.773132in}{1.435964in}}{\pgfqpoint{9.775136in}{1.431126in}}{\pgfqpoint{9.778702in}{1.427560in}}%
\pgfpathcurveto{\pgfqpoint{9.782269in}{1.423993in}}{\pgfqpoint{9.787106in}{1.421989in}}{\pgfqpoint{9.792150in}{1.421989in}}%
\pgfpathclose%
\pgfusepath{fill}%
\end{pgfscope}%
\begin{pgfscope}%
\pgfpathrectangle{\pgfqpoint{6.572727in}{0.474100in}}{\pgfqpoint{4.227273in}{3.318700in}}%
\pgfusepath{clip}%
\pgfsetbuttcap%
\pgfsetroundjoin%
\definecolor{currentfill}{rgb}{0.127568,0.566949,0.550556}%
\pgfsetfillcolor{currentfill}%
\pgfsetfillopacity{0.700000}%
\pgfsetlinewidth{0.000000pt}%
\definecolor{currentstroke}{rgb}{0.000000,0.000000,0.000000}%
\pgfsetstrokecolor{currentstroke}%
\pgfsetstrokeopacity{0.700000}%
\pgfsetdash{}{0pt}%
\pgfpathmoveto{\pgfqpoint{8.204189in}{3.156053in}}%
\pgfpathcurveto{\pgfqpoint{8.209233in}{3.156053in}}{\pgfqpoint{8.214071in}{3.158057in}}{\pgfqpoint{8.217637in}{3.161623in}}%
\pgfpathcurveto{\pgfqpoint{8.221203in}{3.165190in}}{\pgfqpoint{8.223207in}{3.170028in}}{\pgfqpoint{8.223207in}{3.175071in}}%
\pgfpathcurveto{\pgfqpoint{8.223207in}{3.180115in}}{\pgfqpoint{8.221203in}{3.184953in}}{\pgfqpoint{8.217637in}{3.188519in}}%
\pgfpathcurveto{\pgfqpoint{8.214071in}{3.192086in}}{\pgfqpoint{8.209233in}{3.194089in}}{\pgfqpoint{8.204189in}{3.194089in}}%
\pgfpathcurveto{\pgfqpoint{8.199146in}{3.194089in}}{\pgfqpoint{8.194308in}{3.192086in}}{\pgfqpoint{8.190741in}{3.188519in}}%
\pgfpathcurveto{\pgfqpoint{8.187175in}{3.184953in}}{\pgfqpoint{8.185171in}{3.180115in}}{\pgfqpoint{8.185171in}{3.175071in}}%
\pgfpathcurveto{\pgfqpoint{8.185171in}{3.170028in}}{\pgfqpoint{8.187175in}{3.165190in}}{\pgfqpoint{8.190741in}{3.161623in}}%
\pgfpathcurveto{\pgfqpoint{8.194308in}{3.158057in}}{\pgfqpoint{8.199146in}{3.156053in}}{\pgfqpoint{8.204189in}{3.156053in}}%
\pgfpathclose%
\pgfusepath{fill}%
\end{pgfscope}%
\begin{pgfscope}%
\pgfpathrectangle{\pgfqpoint{6.572727in}{0.474100in}}{\pgfqpoint{4.227273in}{3.318700in}}%
\pgfusepath{clip}%
\pgfsetbuttcap%
\pgfsetroundjoin%
\definecolor{currentfill}{rgb}{0.993248,0.906157,0.143936}%
\pgfsetfillcolor{currentfill}%
\pgfsetfillopacity{0.700000}%
\pgfsetlinewidth{0.000000pt}%
\definecolor{currentstroke}{rgb}{0.000000,0.000000,0.000000}%
\pgfsetstrokecolor{currentstroke}%
\pgfsetstrokeopacity{0.700000}%
\pgfsetdash{}{0pt}%
\pgfpathmoveto{\pgfqpoint{9.905322in}{1.415168in}}%
\pgfpathcurveto{\pgfqpoint{9.910366in}{1.415168in}}{\pgfqpoint{9.915204in}{1.417172in}}{\pgfqpoint{9.918770in}{1.420738in}}%
\pgfpathcurveto{\pgfqpoint{9.922336in}{1.424305in}}{\pgfqpoint{9.924340in}{1.429143in}}{\pgfqpoint{9.924340in}{1.434186in}}%
\pgfpathcurveto{\pgfqpoint{9.924340in}{1.439230in}}{\pgfqpoint{9.922336in}{1.444068in}}{\pgfqpoint{9.918770in}{1.447634in}}%
\pgfpathcurveto{\pgfqpoint{9.915204in}{1.451201in}}{\pgfqpoint{9.910366in}{1.453204in}}{\pgfqpoint{9.905322in}{1.453204in}}%
\pgfpathcurveto{\pgfqpoint{9.900278in}{1.453204in}}{\pgfqpoint{9.895441in}{1.451201in}}{\pgfqpoint{9.891874in}{1.447634in}}%
\pgfpathcurveto{\pgfqpoint{9.888308in}{1.444068in}}{\pgfqpoint{9.886304in}{1.439230in}}{\pgfqpoint{9.886304in}{1.434186in}}%
\pgfpathcurveto{\pgfqpoint{9.886304in}{1.429143in}}{\pgfqpoint{9.888308in}{1.424305in}}{\pgfqpoint{9.891874in}{1.420738in}}%
\pgfpathcurveto{\pgfqpoint{9.895441in}{1.417172in}}{\pgfqpoint{9.900278in}{1.415168in}}{\pgfqpoint{9.905322in}{1.415168in}}%
\pgfpathclose%
\pgfusepath{fill}%
\end{pgfscope}%
\begin{pgfscope}%
\pgfpathrectangle{\pgfqpoint{6.572727in}{0.474100in}}{\pgfqpoint{4.227273in}{3.318700in}}%
\pgfusepath{clip}%
\pgfsetbuttcap%
\pgfsetroundjoin%
\definecolor{currentfill}{rgb}{0.127568,0.566949,0.550556}%
\pgfsetfillcolor{currentfill}%
\pgfsetfillopacity{0.700000}%
\pgfsetlinewidth{0.000000pt}%
\definecolor{currentstroke}{rgb}{0.000000,0.000000,0.000000}%
\pgfsetstrokecolor{currentstroke}%
\pgfsetstrokeopacity{0.700000}%
\pgfsetdash{}{0pt}%
\pgfpathmoveto{\pgfqpoint{8.158699in}{1.791536in}}%
\pgfpathcurveto{\pgfqpoint{8.163742in}{1.791536in}}{\pgfqpoint{8.168580in}{1.793540in}}{\pgfqpoint{8.172146in}{1.797107in}}%
\pgfpathcurveto{\pgfqpoint{8.175713in}{1.800673in}}{\pgfqpoint{8.177717in}{1.805511in}}{\pgfqpoint{8.177717in}{1.810554in}}%
\pgfpathcurveto{\pgfqpoint{8.177717in}{1.815598in}}{\pgfqpoint{8.175713in}{1.820436in}}{\pgfqpoint{8.172146in}{1.824002in}}%
\pgfpathcurveto{\pgfqpoint{8.168580in}{1.827569in}}{\pgfqpoint{8.163742in}{1.829573in}}{\pgfqpoint{8.158699in}{1.829573in}}%
\pgfpathcurveto{\pgfqpoint{8.153655in}{1.829573in}}{\pgfqpoint{8.148817in}{1.827569in}}{\pgfqpoint{8.145251in}{1.824002in}}%
\pgfpathcurveto{\pgfqpoint{8.141684in}{1.820436in}}{\pgfqpoint{8.139680in}{1.815598in}}{\pgfqpoint{8.139680in}{1.810554in}}%
\pgfpathcurveto{\pgfqpoint{8.139680in}{1.805511in}}{\pgfqpoint{8.141684in}{1.800673in}}{\pgfqpoint{8.145251in}{1.797107in}}%
\pgfpathcurveto{\pgfqpoint{8.148817in}{1.793540in}}{\pgfqpoint{8.153655in}{1.791536in}}{\pgfqpoint{8.158699in}{1.791536in}}%
\pgfpathclose%
\pgfusepath{fill}%
\end{pgfscope}%
\begin{pgfscope}%
\pgfpathrectangle{\pgfqpoint{6.572727in}{0.474100in}}{\pgfqpoint{4.227273in}{3.318700in}}%
\pgfusepath{clip}%
\pgfsetbuttcap%
\pgfsetroundjoin%
\definecolor{currentfill}{rgb}{0.127568,0.566949,0.550556}%
\pgfsetfillcolor{currentfill}%
\pgfsetfillopacity{0.700000}%
\pgfsetlinewidth{0.000000pt}%
\definecolor{currentstroke}{rgb}{0.000000,0.000000,0.000000}%
\pgfsetstrokecolor{currentstroke}%
\pgfsetstrokeopacity{0.700000}%
\pgfsetdash{}{0pt}%
\pgfpathmoveto{\pgfqpoint{7.893528in}{2.697936in}}%
\pgfpathcurveto{\pgfqpoint{7.898572in}{2.697936in}}{\pgfqpoint{7.903410in}{2.699939in}}{\pgfqpoint{7.906976in}{2.703506in}}%
\pgfpathcurveto{\pgfqpoint{7.910543in}{2.707072in}}{\pgfqpoint{7.912547in}{2.711910in}}{\pgfqpoint{7.912547in}{2.716954in}}%
\pgfpathcurveto{\pgfqpoint{7.912547in}{2.721997in}}{\pgfqpoint{7.910543in}{2.726835in}}{\pgfqpoint{7.906976in}{2.730402in}}%
\pgfpathcurveto{\pgfqpoint{7.903410in}{2.733968in}}{\pgfqpoint{7.898572in}{2.735972in}}{\pgfqpoint{7.893528in}{2.735972in}}%
\pgfpathcurveto{\pgfqpoint{7.888485in}{2.735972in}}{\pgfqpoint{7.883647in}{2.733968in}}{\pgfqpoint{7.880081in}{2.730402in}}%
\pgfpathcurveto{\pgfqpoint{7.876514in}{2.726835in}}{\pgfqpoint{7.874510in}{2.721997in}}{\pgfqpoint{7.874510in}{2.716954in}}%
\pgfpathcurveto{\pgfqpoint{7.874510in}{2.711910in}}{\pgfqpoint{7.876514in}{2.707072in}}{\pgfqpoint{7.880081in}{2.703506in}}%
\pgfpathcurveto{\pgfqpoint{7.883647in}{2.699939in}}{\pgfqpoint{7.888485in}{2.697936in}}{\pgfqpoint{7.893528in}{2.697936in}}%
\pgfpathclose%
\pgfusepath{fill}%
\end{pgfscope}%
\begin{pgfscope}%
\pgfpathrectangle{\pgfqpoint{6.572727in}{0.474100in}}{\pgfqpoint{4.227273in}{3.318700in}}%
\pgfusepath{clip}%
\pgfsetbuttcap%
\pgfsetroundjoin%
\definecolor{currentfill}{rgb}{0.993248,0.906157,0.143936}%
\pgfsetfillcolor{currentfill}%
\pgfsetfillopacity{0.700000}%
\pgfsetlinewidth{0.000000pt}%
\definecolor{currentstroke}{rgb}{0.000000,0.000000,0.000000}%
\pgfsetstrokecolor{currentstroke}%
\pgfsetstrokeopacity{0.700000}%
\pgfsetdash{}{0pt}%
\pgfpathmoveto{\pgfqpoint{9.565851in}{1.578606in}}%
\pgfpathcurveto{\pgfqpoint{9.570895in}{1.578606in}}{\pgfqpoint{9.575732in}{1.580610in}}{\pgfqpoint{9.579299in}{1.584176in}}%
\pgfpathcurveto{\pgfqpoint{9.582865in}{1.587742in}}{\pgfqpoint{9.584869in}{1.592580in}}{\pgfqpoint{9.584869in}{1.597624in}}%
\pgfpathcurveto{\pgfqpoint{9.584869in}{1.602668in}}{\pgfqpoint{9.582865in}{1.607505in}}{\pgfqpoint{9.579299in}{1.611072in}}%
\pgfpathcurveto{\pgfqpoint{9.575732in}{1.614638in}}{\pgfqpoint{9.570895in}{1.616642in}}{\pgfqpoint{9.565851in}{1.616642in}}%
\pgfpathcurveto{\pgfqpoint{9.560807in}{1.616642in}}{\pgfqpoint{9.555970in}{1.614638in}}{\pgfqpoint{9.552403in}{1.611072in}}%
\pgfpathcurveto{\pgfqpoint{9.548837in}{1.607505in}}{\pgfqpoint{9.546833in}{1.602668in}}{\pgfqpoint{9.546833in}{1.597624in}}%
\pgfpathcurveto{\pgfqpoint{9.546833in}{1.592580in}}{\pgfqpoint{9.548837in}{1.587742in}}{\pgfqpoint{9.552403in}{1.584176in}}%
\pgfpathcurveto{\pgfqpoint{9.555970in}{1.580610in}}{\pgfqpoint{9.560807in}{1.578606in}}{\pgfqpoint{9.565851in}{1.578606in}}%
\pgfpathclose%
\pgfusepath{fill}%
\end{pgfscope}%
\begin{pgfscope}%
\pgfpathrectangle{\pgfqpoint{6.572727in}{0.474100in}}{\pgfqpoint{4.227273in}{3.318700in}}%
\pgfusepath{clip}%
\pgfsetbuttcap%
\pgfsetroundjoin%
\definecolor{currentfill}{rgb}{0.127568,0.566949,0.550556}%
\pgfsetfillcolor{currentfill}%
\pgfsetfillopacity{0.700000}%
\pgfsetlinewidth{0.000000pt}%
\definecolor{currentstroke}{rgb}{0.000000,0.000000,0.000000}%
\pgfsetstrokecolor{currentstroke}%
\pgfsetstrokeopacity{0.700000}%
\pgfsetdash{}{0pt}%
\pgfpathmoveto{\pgfqpoint{7.945522in}{1.355513in}}%
\pgfpathcurveto{\pgfqpoint{7.950566in}{1.355513in}}{\pgfqpoint{7.955404in}{1.357517in}}{\pgfqpoint{7.958970in}{1.361084in}}%
\pgfpathcurveto{\pgfqpoint{7.962537in}{1.364650in}}{\pgfqpoint{7.964540in}{1.369488in}}{\pgfqpoint{7.964540in}{1.374532in}}%
\pgfpathcurveto{\pgfqpoint{7.964540in}{1.379575in}}{\pgfqpoint{7.962537in}{1.384413in}}{\pgfqpoint{7.958970in}{1.387979in}}%
\pgfpathcurveto{\pgfqpoint{7.955404in}{1.391546in}}{\pgfqpoint{7.950566in}{1.393550in}}{\pgfqpoint{7.945522in}{1.393550in}}%
\pgfpathcurveto{\pgfqpoint{7.940479in}{1.393550in}}{\pgfqpoint{7.935641in}{1.391546in}}{\pgfqpoint{7.932074in}{1.387979in}}%
\pgfpathcurveto{\pgfqpoint{7.928508in}{1.384413in}}{\pgfqpoint{7.926504in}{1.379575in}}{\pgfqpoint{7.926504in}{1.374532in}}%
\pgfpathcurveto{\pgfqpoint{7.926504in}{1.369488in}}{\pgfqpoint{7.928508in}{1.364650in}}{\pgfqpoint{7.932074in}{1.361084in}}%
\pgfpathcurveto{\pgfqpoint{7.935641in}{1.357517in}}{\pgfqpoint{7.940479in}{1.355513in}}{\pgfqpoint{7.945522in}{1.355513in}}%
\pgfpathclose%
\pgfusepath{fill}%
\end{pgfscope}%
\begin{pgfscope}%
\pgfpathrectangle{\pgfqpoint{6.572727in}{0.474100in}}{\pgfqpoint{4.227273in}{3.318700in}}%
\pgfusepath{clip}%
\pgfsetbuttcap%
\pgfsetroundjoin%
\definecolor{currentfill}{rgb}{0.127568,0.566949,0.550556}%
\pgfsetfillcolor{currentfill}%
\pgfsetfillopacity{0.700000}%
\pgfsetlinewidth{0.000000pt}%
\definecolor{currentstroke}{rgb}{0.000000,0.000000,0.000000}%
\pgfsetstrokecolor{currentstroke}%
\pgfsetstrokeopacity{0.700000}%
\pgfsetdash{}{0pt}%
\pgfpathmoveto{\pgfqpoint{7.508508in}{1.602626in}}%
\pgfpathcurveto{\pgfqpoint{7.513552in}{1.602626in}}{\pgfqpoint{7.518390in}{1.604629in}}{\pgfqpoint{7.521956in}{1.608196in}}%
\pgfpathcurveto{\pgfqpoint{7.525523in}{1.611762in}}{\pgfqpoint{7.527526in}{1.616600in}}{\pgfqpoint{7.527526in}{1.621644in}}%
\pgfpathcurveto{\pgfqpoint{7.527526in}{1.626687in}}{\pgfqpoint{7.525523in}{1.631525in}}{\pgfqpoint{7.521956in}{1.635092in}}%
\pgfpathcurveto{\pgfqpoint{7.518390in}{1.638658in}}{\pgfqpoint{7.513552in}{1.640662in}}{\pgfqpoint{7.508508in}{1.640662in}}%
\pgfpathcurveto{\pgfqpoint{7.503465in}{1.640662in}}{\pgfqpoint{7.498627in}{1.638658in}}{\pgfqpoint{7.495060in}{1.635092in}}%
\pgfpathcurveto{\pgfqpoint{7.491494in}{1.631525in}}{\pgfqpoint{7.489490in}{1.626687in}}{\pgfqpoint{7.489490in}{1.621644in}}%
\pgfpathcurveto{\pgfqpoint{7.489490in}{1.616600in}}{\pgfqpoint{7.491494in}{1.611762in}}{\pgfqpoint{7.495060in}{1.608196in}}%
\pgfpathcurveto{\pgfqpoint{7.498627in}{1.604629in}}{\pgfqpoint{7.503465in}{1.602626in}}{\pgfqpoint{7.508508in}{1.602626in}}%
\pgfpathclose%
\pgfusepath{fill}%
\end{pgfscope}%
\begin{pgfscope}%
\pgfpathrectangle{\pgfqpoint{6.572727in}{0.474100in}}{\pgfqpoint{4.227273in}{3.318700in}}%
\pgfusepath{clip}%
\pgfsetbuttcap%
\pgfsetroundjoin%
\definecolor{currentfill}{rgb}{0.127568,0.566949,0.550556}%
\pgfsetfillcolor{currentfill}%
\pgfsetfillopacity{0.700000}%
\pgfsetlinewidth{0.000000pt}%
\definecolor{currentstroke}{rgb}{0.000000,0.000000,0.000000}%
\pgfsetstrokecolor{currentstroke}%
\pgfsetstrokeopacity{0.700000}%
\pgfsetdash{}{0pt}%
\pgfpathmoveto{\pgfqpoint{8.140730in}{1.726826in}}%
\pgfpathcurveto{\pgfqpoint{8.145774in}{1.726826in}}{\pgfqpoint{8.150612in}{1.728830in}}{\pgfqpoint{8.154178in}{1.732396in}}%
\pgfpathcurveto{\pgfqpoint{8.157745in}{1.735962in}}{\pgfqpoint{8.159748in}{1.740800in}}{\pgfqpoint{8.159748in}{1.745844in}}%
\pgfpathcurveto{\pgfqpoint{8.159748in}{1.750888in}}{\pgfqpoint{8.157745in}{1.755725in}}{\pgfqpoint{8.154178in}{1.759292in}}%
\pgfpathcurveto{\pgfqpoint{8.150612in}{1.762858in}}{\pgfqpoint{8.145774in}{1.764862in}}{\pgfqpoint{8.140730in}{1.764862in}}%
\pgfpathcurveto{\pgfqpoint{8.135687in}{1.764862in}}{\pgfqpoint{8.130849in}{1.762858in}}{\pgfqpoint{8.127282in}{1.759292in}}%
\pgfpathcurveto{\pgfqpoint{8.123716in}{1.755725in}}{\pgfqpoint{8.121712in}{1.750888in}}{\pgfqpoint{8.121712in}{1.745844in}}%
\pgfpathcurveto{\pgfqpoint{8.121712in}{1.740800in}}{\pgfqpoint{8.123716in}{1.735962in}}{\pgfqpoint{8.127282in}{1.732396in}}%
\pgfpathcurveto{\pgfqpoint{8.130849in}{1.728830in}}{\pgfqpoint{8.135687in}{1.726826in}}{\pgfqpoint{8.140730in}{1.726826in}}%
\pgfpathclose%
\pgfusepath{fill}%
\end{pgfscope}%
\begin{pgfscope}%
\pgfpathrectangle{\pgfqpoint{6.572727in}{0.474100in}}{\pgfqpoint{4.227273in}{3.318700in}}%
\pgfusepath{clip}%
\pgfsetbuttcap%
\pgfsetroundjoin%
\definecolor{currentfill}{rgb}{0.127568,0.566949,0.550556}%
\pgfsetfillcolor{currentfill}%
\pgfsetfillopacity{0.700000}%
\pgfsetlinewidth{0.000000pt}%
\definecolor{currentstroke}{rgb}{0.000000,0.000000,0.000000}%
\pgfsetstrokecolor{currentstroke}%
\pgfsetstrokeopacity{0.700000}%
\pgfsetdash{}{0pt}%
\pgfpathmoveto{\pgfqpoint{7.789967in}{2.082679in}}%
\pgfpathcurveto{\pgfqpoint{7.795011in}{2.082679in}}{\pgfqpoint{7.799848in}{2.084683in}}{\pgfqpoint{7.803415in}{2.088249in}}%
\pgfpathcurveto{\pgfqpoint{7.806981in}{2.091816in}}{\pgfqpoint{7.808985in}{2.096654in}}{\pgfqpoint{7.808985in}{2.101697in}}%
\pgfpathcurveto{\pgfqpoint{7.808985in}{2.106741in}}{\pgfqpoint{7.806981in}{2.111579in}}{\pgfqpoint{7.803415in}{2.115145in}}%
\pgfpathcurveto{\pgfqpoint{7.799848in}{2.118712in}}{\pgfqpoint{7.795011in}{2.120715in}}{\pgfqpoint{7.789967in}{2.120715in}}%
\pgfpathcurveto{\pgfqpoint{7.784923in}{2.120715in}}{\pgfqpoint{7.780086in}{2.118712in}}{\pgfqpoint{7.776519in}{2.115145in}}%
\pgfpathcurveto{\pgfqpoint{7.772953in}{2.111579in}}{\pgfqpoint{7.770949in}{2.106741in}}{\pgfqpoint{7.770949in}{2.101697in}}%
\pgfpathcurveto{\pgfqpoint{7.770949in}{2.096654in}}{\pgfqpoint{7.772953in}{2.091816in}}{\pgfqpoint{7.776519in}{2.088249in}}%
\pgfpathcurveto{\pgfqpoint{7.780086in}{2.084683in}}{\pgfqpoint{7.784923in}{2.082679in}}{\pgfqpoint{7.789967in}{2.082679in}}%
\pgfpathclose%
\pgfusepath{fill}%
\end{pgfscope}%
\begin{pgfscope}%
\pgfpathrectangle{\pgfqpoint{6.572727in}{0.474100in}}{\pgfqpoint{4.227273in}{3.318700in}}%
\pgfusepath{clip}%
\pgfsetbuttcap%
\pgfsetroundjoin%
\definecolor{currentfill}{rgb}{0.127568,0.566949,0.550556}%
\pgfsetfillcolor{currentfill}%
\pgfsetfillopacity{0.700000}%
\pgfsetlinewidth{0.000000pt}%
\definecolor{currentstroke}{rgb}{0.000000,0.000000,0.000000}%
\pgfsetstrokecolor{currentstroke}%
\pgfsetstrokeopacity{0.700000}%
\pgfsetdash{}{0pt}%
\pgfpathmoveto{\pgfqpoint{7.241159in}{1.129722in}}%
\pgfpathcurveto{\pgfqpoint{7.246202in}{1.129722in}}{\pgfqpoint{7.251040in}{1.131726in}}{\pgfqpoint{7.254607in}{1.135292in}}%
\pgfpathcurveto{\pgfqpoint{7.258173in}{1.138859in}}{\pgfqpoint{7.260177in}{1.143696in}}{\pgfqpoint{7.260177in}{1.148740in}}%
\pgfpathcurveto{\pgfqpoint{7.260177in}{1.153784in}}{\pgfqpoint{7.258173in}{1.158621in}}{\pgfqpoint{7.254607in}{1.162188in}}%
\pgfpathcurveto{\pgfqpoint{7.251040in}{1.165754in}}{\pgfqpoint{7.246202in}{1.167758in}}{\pgfqpoint{7.241159in}{1.167758in}}%
\pgfpathcurveto{\pgfqpoint{7.236115in}{1.167758in}}{\pgfqpoint{7.231277in}{1.165754in}}{\pgfqpoint{7.227711in}{1.162188in}}%
\pgfpathcurveto{\pgfqpoint{7.224144in}{1.158621in}}{\pgfqpoint{7.222141in}{1.153784in}}{\pgfqpoint{7.222141in}{1.148740in}}%
\pgfpathcurveto{\pgfqpoint{7.222141in}{1.143696in}}{\pgfqpoint{7.224144in}{1.138859in}}{\pgfqpoint{7.227711in}{1.135292in}}%
\pgfpathcurveto{\pgfqpoint{7.231277in}{1.131726in}}{\pgfqpoint{7.236115in}{1.129722in}}{\pgfqpoint{7.241159in}{1.129722in}}%
\pgfpathclose%
\pgfusepath{fill}%
\end{pgfscope}%
\begin{pgfscope}%
\pgfpathrectangle{\pgfqpoint{6.572727in}{0.474100in}}{\pgfqpoint{4.227273in}{3.318700in}}%
\pgfusepath{clip}%
\pgfsetbuttcap%
\pgfsetroundjoin%
\definecolor{currentfill}{rgb}{0.127568,0.566949,0.550556}%
\pgfsetfillcolor{currentfill}%
\pgfsetfillopacity{0.700000}%
\pgfsetlinewidth{0.000000pt}%
\definecolor{currentstroke}{rgb}{0.000000,0.000000,0.000000}%
\pgfsetstrokecolor{currentstroke}%
\pgfsetstrokeopacity{0.700000}%
\pgfsetdash{}{0pt}%
\pgfpathmoveto{\pgfqpoint{7.509983in}{1.567455in}}%
\pgfpathcurveto{\pgfqpoint{7.515027in}{1.567455in}}{\pgfqpoint{7.519865in}{1.569459in}}{\pgfqpoint{7.523431in}{1.573026in}}%
\pgfpathcurveto{\pgfqpoint{7.526998in}{1.576592in}}{\pgfqpoint{7.529002in}{1.581430in}}{\pgfqpoint{7.529002in}{1.586474in}}%
\pgfpathcurveto{\pgfqpoint{7.529002in}{1.591517in}}{\pgfqpoint{7.526998in}{1.596355in}}{\pgfqpoint{7.523431in}{1.599921in}}%
\pgfpathcurveto{\pgfqpoint{7.519865in}{1.603488in}}{\pgfqpoint{7.515027in}{1.605492in}}{\pgfqpoint{7.509983in}{1.605492in}}%
\pgfpathcurveto{\pgfqpoint{7.504940in}{1.605492in}}{\pgfqpoint{7.500102in}{1.603488in}}{\pgfqpoint{7.496536in}{1.599921in}}%
\pgfpathcurveto{\pgfqpoint{7.492969in}{1.596355in}}{\pgfqpoint{7.490965in}{1.591517in}}{\pgfqpoint{7.490965in}{1.586474in}}%
\pgfpathcurveto{\pgfqpoint{7.490965in}{1.581430in}}{\pgfqpoint{7.492969in}{1.576592in}}{\pgfqpoint{7.496536in}{1.573026in}}%
\pgfpathcurveto{\pgfqpoint{7.500102in}{1.569459in}}{\pgfqpoint{7.504940in}{1.567455in}}{\pgfqpoint{7.509983in}{1.567455in}}%
\pgfpathclose%
\pgfusepath{fill}%
\end{pgfscope}%
\begin{pgfscope}%
\pgfpathrectangle{\pgfqpoint{6.572727in}{0.474100in}}{\pgfqpoint{4.227273in}{3.318700in}}%
\pgfusepath{clip}%
\pgfsetbuttcap%
\pgfsetroundjoin%
\definecolor{currentfill}{rgb}{0.127568,0.566949,0.550556}%
\pgfsetfillcolor{currentfill}%
\pgfsetfillopacity{0.700000}%
\pgfsetlinewidth{0.000000pt}%
\definecolor{currentstroke}{rgb}{0.000000,0.000000,0.000000}%
\pgfsetstrokecolor{currentstroke}%
\pgfsetstrokeopacity{0.700000}%
\pgfsetdash{}{0pt}%
\pgfpathmoveto{\pgfqpoint{8.349274in}{3.398330in}}%
\pgfpathcurveto{\pgfqpoint{8.354317in}{3.398330in}}{\pgfqpoint{8.359155in}{3.400334in}}{\pgfqpoint{8.362722in}{3.403901in}}%
\pgfpathcurveto{\pgfqpoint{8.366288in}{3.407467in}}{\pgfqpoint{8.368292in}{3.412305in}}{\pgfqpoint{8.368292in}{3.417349in}}%
\pgfpathcurveto{\pgfqpoint{8.368292in}{3.422392in}}{\pgfqpoint{8.366288in}{3.427230in}}{\pgfqpoint{8.362722in}{3.430796in}}%
\pgfpathcurveto{\pgfqpoint{8.359155in}{3.434363in}}{\pgfqpoint{8.354317in}{3.436367in}}{\pgfqpoint{8.349274in}{3.436367in}}%
\pgfpathcurveto{\pgfqpoint{8.344230in}{3.436367in}}{\pgfqpoint{8.339392in}{3.434363in}}{\pgfqpoint{8.335826in}{3.430796in}}%
\pgfpathcurveto{\pgfqpoint{8.332260in}{3.427230in}}{\pgfqpoint{8.330256in}{3.422392in}}{\pgfqpoint{8.330256in}{3.417349in}}%
\pgfpathcurveto{\pgfqpoint{8.330256in}{3.412305in}}{\pgfqpoint{8.332260in}{3.407467in}}{\pgfqpoint{8.335826in}{3.403901in}}%
\pgfpathcurveto{\pgfqpoint{8.339392in}{3.400334in}}{\pgfqpoint{8.344230in}{3.398330in}}{\pgfqpoint{8.349274in}{3.398330in}}%
\pgfpathclose%
\pgfusepath{fill}%
\end{pgfscope}%
\begin{pgfscope}%
\pgfpathrectangle{\pgfqpoint{6.572727in}{0.474100in}}{\pgfqpoint{4.227273in}{3.318700in}}%
\pgfusepath{clip}%
\pgfsetbuttcap%
\pgfsetroundjoin%
\definecolor{currentfill}{rgb}{0.127568,0.566949,0.550556}%
\pgfsetfillcolor{currentfill}%
\pgfsetfillopacity{0.700000}%
\pgfsetlinewidth{0.000000pt}%
\definecolor{currentstroke}{rgb}{0.000000,0.000000,0.000000}%
\pgfsetstrokecolor{currentstroke}%
\pgfsetstrokeopacity{0.700000}%
\pgfsetdash{}{0pt}%
\pgfpathmoveto{\pgfqpoint{7.844840in}{2.080716in}}%
\pgfpathcurveto{\pgfqpoint{7.849884in}{2.080716in}}{\pgfqpoint{7.854722in}{2.082720in}}{\pgfqpoint{7.858288in}{2.086286in}}%
\pgfpathcurveto{\pgfqpoint{7.861855in}{2.089852in}}{\pgfqpoint{7.863858in}{2.094690in}}{\pgfqpoint{7.863858in}{2.099734in}}%
\pgfpathcurveto{\pgfqpoint{7.863858in}{2.104778in}}{\pgfqpoint{7.861855in}{2.109615in}}{\pgfqpoint{7.858288in}{2.113182in}}%
\pgfpathcurveto{\pgfqpoint{7.854722in}{2.116748in}}{\pgfqpoint{7.849884in}{2.118752in}}{\pgfqpoint{7.844840in}{2.118752in}}%
\pgfpathcurveto{\pgfqpoint{7.839797in}{2.118752in}}{\pgfqpoint{7.834959in}{2.116748in}}{\pgfqpoint{7.831392in}{2.113182in}}%
\pgfpathcurveto{\pgfqpoint{7.827826in}{2.109615in}}{\pgfqpoint{7.825822in}{2.104778in}}{\pgfqpoint{7.825822in}{2.099734in}}%
\pgfpathcurveto{\pgfqpoint{7.825822in}{2.094690in}}{\pgfqpoint{7.827826in}{2.089852in}}{\pgfqpoint{7.831392in}{2.086286in}}%
\pgfpathcurveto{\pgfqpoint{7.834959in}{2.082720in}}{\pgfqpoint{7.839797in}{2.080716in}}{\pgfqpoint{7.844840in}{2.080716in}}%
\pgfpathclose%
\pgfusepath{fill}%
\end{pgfscope}%
\begin{pgfscope}%
\pgfpathrectangle{\pgfqpoint{6.572727in}{0.474100in}}{\pgfqpoint{4.227273in}{3.318700in}}%
\pgfusepath{clip}%
\pgfsetbuttcap%
\pgfsetroundjoin%
\definecolor{currentfill}{rgb}{0.127568,0.566949,0.550556}%
\pgfsetfillcolor{currentfill}%
\pgfsetfillopacity{0.700000}%
\pgfsetlinewidth{0.000000pt}%
\definecolor{currentstroke}{rgb}{0.000000,0.000000,0.000000}%
\pgfsetstrokecolor{currentstroke}%
\pgfsetstrokeopacity{0.700000}%
\pgfsetdash{}{0pt}%
\pgfpathmoveto{\pgfqpoint{8.142850in}{2.272011in}}%
\pgfpathcurveto{\pgfqpoint{8.147893in}{2.272011in}}{\pgfqpoint{8.152731in}{2.274015in}}{\pgfqpoint{8.156298in}{2.277582in}}%
\pgfpathcurveto{\pgfqpoint{8.159864in}{2.281148in}}{\pgfqpoint{8.161868in}{2.285986in}}{\pgfqpoint{8.161868in}{2.291030in}}%
\pgfpathcurveto{\pgfqpoint{8.161868in}{2.296073in}}{\pgfqpoint{8.159864in}{2.300911in}}{\pgfqpoint{8.156298in}{2.304477in}}%
\pgfpathcurveto{\pgfqpoint{8.152731in}{2.308044in}}{\pgfqpoint{8.147893in}{2.310048in}}{\pgfqpoint{8.142850in}{2.310048in}}%
\pgfpathcurveto{\pgfqpoint{8.137806in}{2.310048in}}{\pgfqpoint{8.132968in}{2.308044in}}{\pgfqpoint{8.129402in}{2.304477in}}%
\pgfpathcurveto{\pgfqpoint{8.125836in}{2.300911in}}{\pgfqpoint{8.123832in}{2.296073in}}{\pgfqpoint{8.123832in}{2.291030in}}%
\pgfpathcurveto{\pgfqpoint{8.123832in}{2.285986in}}{\pgfqpoint{8.125836in}{2.281148in}}{\pgfqpoint{8.129402in}{2.277582in}}%
\pgfpathcurveto{\pgfqpoint{8.132968in}{2.274015in}}{\pgfqpoint{8.137806in}{2.272011in}}{\pgfqpoint{8.142850in}{2.272011in}}%
\pgfpathclose%
\pgfusepath{fill}%
\end{pgfscope}%
\begin{pgfscope}%
\pgfpathrectangle{\pgfqpoint{6.572727in}{0.474100in}}{\pgfqpoint{4.227273in}{3.318700in}}%
\pgfusepath{clip}%
\pgfsetbuttcap%
\pgfsetroundjoin%
\definecolor{currentfill}{rgb}{0.127568,0.566949,0.550556}%
\pgfsetfillcolor{currentfill}%
\pgfsetfillopacity{0.700000}%
\pgfsetlinewidth{0.000000pt}%
\definecolor{currentstroke}{rgb}{0.000000,0.000000,0.000000}%
\pgfsetstrokecolor{currentstroke}%
\pgfsetstrokeopacity{0.700000}%
\pgfsetdash{}{0pt}%
\pgfpathmoveto{\pgfqpoint{8.365658in}{1.289695in}}%
\pgfpathcurveto{\pgfqpoint{8.370701in}{1.289695in}}{\pgfqpoint{8.375539in}{1.291699in}}{\pgfqpoint{8.379106in}{1.295266in}}%
\pgfpathcurveto{\pgfqpoint{8.382672in}{1.298832in}}{\pgfqpoint{8.384676in}{1.303670in}}{\pgfqpoint{8.384676in}{1.308713in}}%
\pgfpathcurveto{\pgfqpoint{8.384676in}{1.313757in}}{\pgfqpoint{8.382672in}{1.318595in}}{\pgfqpoint{8.379106in}{1.322161in}}%
\pgfpathcurveto{\pgfqpoint{8.375539in}{1.325728in}}{\pgfqpoint{8.370701in}{1.327732in}}{\pgfqpoint{8.365658in}{1.327732in}}%
\pgfpathcurveto{\pgfqpoint{8.360614in}{1.327732in}}{\pgfqpoint{8.355776in}{1.325728in}}{\pgfqpoint{8.352210in}{1.322161in}}%
\pgfpathcurveto{\pgfqpoint{8.348643in}{1.318595in}}{\pgfqpoint{8.346640in}{1.313757in}}{\pgfqpoint{8.346640in}{1.308713in}}%
\pgfpathcurveto{\pgfqpoint{8.346640in}{1.303670in}}{\pgfqpoint{8.348643in}{1.298832in}}{\pgfqpoint{8.352210in}{1.295266in}}%
\pgfpathcurveto{\pgfqpoint{8.355776in}{1.291699in}}{\pgfqpoint{8.360614in}{1.289695in}}{\pgfqpoint{8.365658in}{1.289695in}}%
\pgfpathclose%
\pgfusepath{fill}%
\end{pgfscope}%
\begin{pgfscope}%
\pgfpathrectangle{\pgfqpoint{6.572727in}{0.474100in}}{\pgfqpoint{4.227273in}{3.318700in}}%
\pgfusepath{clip}%
\pgfsetbuttcap%
\pgfsetroundjoin%
\definecolor{currentfill}{rgb}{0.127568,0.566949,0.550556}%
\pgfsetfillcolor{currentfill}%
\pgfsetfillopacity{0.700000}%
\pgfsetlinewidth{0.000000pt}%
\definecolor{currentstroke}{rgb}{0.000000,0.000000,0.000000}%
\pgfsetstrokecolor{currentstroke}%
\pgfsetstrokeopacity{0.700000}%
\pgfsetdash{}{0pt}%
\pgfpathmoveto{\pgfqpoint{7.651325in}{0.955761in}}%
\pgfpathcurveto{\pgfqpoint{7.656369in}{0.955761in}}{\pgfqpoint{7.661207in}{0.957765in}}{\pgfqpoint{7.664773in}{0.961331in}}%
\pgfpathcurveto{\pgfqpoint{7.668340in}{0.964898in}}{\pgfqpoint{7.670343in}{0.969735in}}{\pgfqpoint{7.670343in}{0.974779in}}%
\pgfpathcurveto{\pgfqpoint{7.670343in}{0.979823in}}{\pgfqpoint{7.668340in}{0.984661in}}{\pgfqpoint{7.664773in}{0.988227in}}%
\pgfpathcurveto{\pgfqpoint{7.661207in}{0.991793in}}{\pgfqpoint{7.656369in}{0.993797in}}{\pgfqpoint{7.651325in}{0.993797in}}%
\pgfpathcurveto{\pgfqpoint{7.646282in}{0.993797in}}{\pgfqpoint{7.641444in}{0.991793in}}{\pgfqpoint{7.637877in}{0.988227in}}%
\pgfpathcurveto{\pgfqpoint{7.634311in}{0.984661in}}{\pgfqpoint{7.632307in}{0.979823in}}{\pgfqpoint{7.632307in}{0.974779in}}%
\pgfpathcurveto{\pgfqpoint{7.632307in}{0.969735in}}{\pgfqpoint{7.634311in}{0.964898in}}{\pgfqpoint{7.637877in}{0.961331in}}%
\pgfpathcurveto{\pgfqpoint{7.641444in}{0.957765in}}{\pgfqpoint{7.646282in}{0.955761in}}{\pgfqpoint{7.651325in}{0.955761in}}%
\pgfpathclose%
\pgfusepath{fill}%
\end{pgfscope}%
\begin{pgfscope}%
\pgfpathrectangle{\pgfqpoint{6.572727in}{0.474100in}}{\pgfqpoint{4.227273in}{3.318700in}}%
\pgfusepath{clip}%
\pgfsetbuttcap%
\pgfsetroundjoin%
\definecolor{currentfill}{rgb}{0.993248,0.906157,0.143936}%
\pgfsetfillcolor{currentfill}%
\pgfsetfillopacity{0.700000}%
\pgfsetlinewidth{0.000000pt}%
\definecolor{currentstroke}{rgb}{0.000000,0.000000,0.000000}%
\pgfsetstrokecolor{currentstroke}%
\pgfsetstrokeopacity{0.700000}%
\pgfsetdash{}{0pt}%
\pgfpathmoveto{\pgfqpoint{9.634312in}{1.525565in}}%
\pgfpathcurveto{\pgfqpoint{9.639356in}{1.525565in}}{\pgfqpoint{9.644193in}{1.527569in}}{\pgfqpoint{9.647760in}{1.531135in}}%
\pgfpathcurveto{\pgfqpoint{9.651326in}{1.534702in}}{\pgfqpoint{9.653330in}{1.539540in}}{\pgfqpoint{9.653330in}{1.544583in}}%
\pgfpathcurveto{\pgfqpoint{9.653330in}{1.549627in}}{\pgfqpoint{9.651326in}{1.554465in}}{\pgfqpoint{9.647760in}{1.558031in}}%
\pgfpathcurveto{\pgfqpoint{9.644193in}{1.561598in}}{\pgfqpoint{9.639356in}{1.563601in}}{\pgfqpoint{9.634312in}{1.563601in}}%
\pgfpathcurveto{\pgfqpoint{9.629268in}{1.563601in}}{\pgfqpoint{9.624430in}{1.561598in}}{\pgfqpoint{9.620864in}{1.558031in}}%
\pgfpathcurveto{\pgfqpoint{9.617298in}{1.554465in}}{\pgfqpoint{9.615294in}{1.549627in}}{\pgfqpoint{9.615294in}{1.544583in}}%
\pgfpathcurveto{\pgfqpoint{9.615294in}{1.539540in}}{\pgfqpoint{9.617298in}{1.534702in}}{\pgfqpoint{9.620864in}{1.531135in}}%
\pgfpathcurveto{\pgfqpoint{9.624430in}{1.527569in}}{\pgfqpoint{9.629268in}{1.525565in}}{\pgfqpoint{9.634312in}{1.525565in}}%
\pgfpathclose%
\pgfusepath{fill}%
\end{pgfscope}%
\begin{pgfscope}%
\pgfpathrectangle{\pgfqpoint{6.572727in}{0.474100in}}{\pgfqpoint{4.227273in}{3.318700in}}%
\pgfusepath{clip}%
\pgfsetbuttcap%
\pgfsetroundjoin%
\definecolor{currentfill}{rgb}{0.127568,0.566949,0.550556}%
\pgfsetfillcolor{currentfill}%
\pgfsetfillopacity{0.700000}%
\pgfsetlinewidth{0.000000pt}%
\definecolor{currentstroke}{rgb}{0.000000,0.000000,0.000000}%
\pgfsetstrokecolor{currentstroke}%
\pgfsetstrokeopacity{0.700000}%
\pgfsetdash{}{0pt}%
\pgfpathmoveto{\pgfqpoint{7.958342in}{1.660892in}}%
\pgfpathcurveto{\pgfqpoint{7.963386in}{1.660892in}}{\pgfqpoint{7.968224in}{1.662896in}}{\pgfqpoint{7.971790in}{1.666462in}}%
\pgfpathcurveto{\pgfqpoint{7.975356in}{1.670029in}}{\pgfqpoint{7.977360in}{1.674867in}}{\pgfqpoint{7.977360in}{1.679910in}}%
\pgfpathcurveto{\pgfqpoint{7.977360in}{1.684954in}}{\pgfqpoint{7.975356in}{1.689792in}}{\pgfqpoint{7.971790in}{1.693358in}}%
\pgfpathcurveto{\pgfqpoint{7.968224in}{1.696925in}}{\pgfqpoint{7.963386in}{1.698928in}}{\pgfqpoint{7.958342in}{1.698928in}}%
\pgfpathcurveto{\pgfqpoint{7.953298in}{1.698928in}}{\pgfqpoint{7.948461in}{1.696925in}}{\pgfqpoint{7.944894in}{1.693358in}}%
\pgfpathcurveto{\pgfqpoint{7.941328in}{1.689792in}}{\pgfqpoint{7.939324in}{1.684954in}}{\pgfqpoint{7.939324in}{1.679910in}}%
\pgfpathcurveto{\pgfqpoint{7.939324in}{1.674867in}}{\pgfqpoint{7.941328in}{1.670029in}}{\pgfqpoint{7.944894in}{1.666462in}}%
\pgfpathcurveto{\pgfqpoint{7.948461in}{1.662896in}}{\pgfqpoint{7.953298in}{1.660892in}}{\pgfqpoint{7.958342in}{1.660892in}}%
\pgfpathclose%
\pgfusepath{fill}%
\end{pgfscope}%
\begin{pgfscope}%
\pgfpathrectangle{\pgfqpoint{6.572727in}{0.474100in}}{\pgfqpoint{4.227273in}{3.318700in}}%
\pgfusepath{clip}%
\pgfsetbuttcap%
\pgfsetroundjoin%
\definecolor{currentfill}{rgb}{0.993248,0.906157,0.143936}%
\pgfsetfillcolor{currentfill}%
\pgfsetfillopacity{0.700000}%
\pgfsetlinewidth{0.000000pt}%
\definecolor{currentstroke}{rgb}{0.000000,0.000000,0.000000}%
\pgfsetstrokecolor{currentstroke}%
\pgfsetstrokeopacity{0.700000}%
\pgfsetdash{}{0pt}%
\pgfpathmoveto{\pgfqpoint{9.283712in}{1.069771in}}%
\pgfpathcurveto{\pgfqpoint{9.288756in}{1.069771in}}{\pgfqpoint{9.293594in}{1.071775in}}{\pgfqpoint{9.297160in}{1.075341in}}%
\pgfpathcurveto{\pgfqpoint{9.300726in}{1.078907in}}{\pgfqpoint{9.302730in}{1.083745in}}{\pgfqpoint{9.302730in}{1.088789in}}%
\pgfpathcurveto{\pgfqpoint{9.302730in}{1.093833in}}{\pgfqpoint{9.300726in}{1.098670in}}{\pgfqpoint{9.297160in}{1.102237in}}%
\pgfpathcurveto{\pgfqpoint{9.293594in}{1.105803in}}{\pgfqpoint{9.288756in}{1.107807in}}{\pgfqpoint{9.283712in}{1.107807in}}%
\pgfpathcurveto{\pgfqpoint{9.278668in}{1.107807in}}{\pgfqpoint{9.273831in}{1.105803in}}{\pgfqpoint{9.270264in}{1.102237in}}%
\pgfpathcurveto{\pgfqpoint{9.266698in}{1.098670in}}{\pgfqpoint{9.264694in}{1.093833in}}{\pgfqpoint{9.264694in}{1.088789in}}%
\pgfpathcurveto{\pgfqpoint{9.264694in}{1.083745in}}{\pgfqpoint{9.266698in}{1.078907in}}{\pgfqpoint{9.270264in}{1.075341in}}%
\pgfpathcurveto{\pgfqpoint{9.273831in}{1.071775in}}{\pgfqpoint{9.278668in}{1.069771in}}{\pgfqpoint{9.283712in}{1.069771in}}%
\pgfpathclose%
\pgfusepath{fill}%
\end{pgfscope}%
\begin{pgfscope}%
\pgfpathrectangle{\pgfqpoint{6.572727in}{0.474100in}}{\pgfqpoint{4.227273in}{3.318700in}}%
\pgfusepath{clip}%
\pgfsetbuttcap%
\pgfsetroundjoin%
\definecolor{currentfill}{rgb}{0.993248,0.906157,0.143936}%
\pgfsetfillcolor{currentfill}%
\pgfsetfillopacity{0.700000}%
\pgfsetlinewidth{0.000000pt}%
\definecolor{currentstroke}{rgb}{0.000000,0.000000,0.000000}%
\pgfsetstrokecolor{currentstroke}%
\pgfsetstrokeopacity{0.700000}%
\pgfsetdash{}{0pt}%
\pgfpathmoveto{\pgfqpoint{9.519123in}{1.168069in}}%
\pgfpathcurveto{\pgfqpoint{9.524167in}{1.168069in}}{\pgfqpoint{9.529005in}{1.170073in}}{\pgfqpoint{9.532571in}{1.173639in}}%
\pgfpathcurveto{\pgfqpoint{9.536138in}{1.177206in}}{\pgfqpoint{9.538142in}{1.182043in}}{\pgfqpoint{9.538142in}{1.187087in}}%
\pgfpathcurveto{\pgfqpoint{9.538142in}{1.192131in}}{\pgfqpoint{9.536138in}{1.196969in}}{\pgfqpoint{9.532571in}{1.200535in}}%
\pgfpathcurveto{\pgfqpoint{9.529005in}{1.204101in}}{\pgfqpoint{9.524167in}{1.206105in}}{\pgfqpoint{9.519123in}{1.206105in}}%
\pgfpathcurveto{\pgfqpoint{9.514080in}{1.206105in}}{\pgfqpoint{9.509242in}{1.204101in}}{\pgfqpoint{9.505676in}{1.200535in}}%
\pgfpathcurveto{\pgfqpoint{9.502109in}{1.196969in}}{\pgfqpoint{9.500105in}{1.192131in}}{\pgfqpoint{9.500105in}{1.187087in}}%
\pgfpathcurveto{\pgfqpoint{9.500105in}{1.182043in}}{\pgfqpoint{9.502109in}{1.177206in}}{\pgfqpoint{9.505676in}{1.173639in}}%
\pgfpathcurveto{\pgfqpoint{9.509242in}{1.170073in}}{\pgfqpoint{9.514080in}{1.168069in}}{\pgfqpoint{9.519123in}{1.168069in}}%
\pgfpathclose%
\pgfusepath{fill}%
\end{pgfscope}%
\begin{pgfscope}%
\pgfpathrectangle{\pgfqpoint{6.572727in}{0.474100in}}{\pgfqpoint{4.227273in}{3.318700in}}%
\pgfusepath{clip}%
\pgfsetbuttcap%
\pgfsetroundjoin%
\definecolor{currentfill}{rgb}{0.993248,0.906157,0.143936}%
\pgfsetfillcolor{currentfill}%
\pgfsetfillopacity{0.700000}%
\pgfsetlinewidth{0.000000pt}%
\definecolor{currentstroke}{rgb}{0.000000,0.000000,0.000000}%
\pgfsetstrokecolor{currentstroke}%
\pgfsetstrokeopacity{0.700000}%
\pgfsetdash{}{0pt}%
\pgfpathmoveto{\pgfqpoint{9.659079in}{1.622575in}}%
\pgfpathcurveto{\pgfqpoint{9.664123in}{1.622575in}}{\pgfqpoint{9.668961in}{1.624579in}}{\pgfqpoint{9.672527in}{1.628145in}}%
\pgfpathcurveto{\pgfqpoint{9.676094in}{1.631712in}}{\pgfqpoint{9.678097in}{1.636549in}}{\pgfqpoint{9.678097in}{1.641593in}}%
\pgfpathcurveto{\pgfqpoint{9.678097in}{1.646637in}}{\pgfqpoint{9.676094in}{1.651475in}}{\pgfqpoint{9.672527in}{1.655041in}}%
\pgfpathcurveto{\pgfqpoint{9.668961in}{1.658607in}}{\pgfqpoint{9.664123in}{1.660611in}}{\pgfqpoint{9.659079in}{1.660611in}}%
\pgfpathcurveto{\pgfqpoint{9.654036in}{1.660611in}}{\pgfqpoint{9.649198in}{1.658607in}}{\pgfqpoint{9.645631in}{1.655041in}}%
\pgfpathcurveto{\pgfqpoint{9.642065in}{1.651475in}}{\pgfqpoint{9.640061in}{1.646637in}}{\pgfqpoint{9.640061in}{1.641593in}}%
\pgfpathcurveto{\pgfqpoint{9.640061in}{1.636549in}}{\pgfqpoint{9.642065in}{1.631712in}}{\pgfqpoint{9.645631in}{1.628145in}}%
\pgfpathcurveto{\pgfqpoint{9.649198in}{1.624579in}}{\pgfqpoint{9.654036in}{1.622575in}}{\pgfqpoint{9.659079in}{1.622575in}}%
\pgfpathclose%
\pgfusepath{fill}%
\end{pgfscope}%
\begin{pgfscope}%
\pgfpathrectangle{\pgfqpoint{6.572727in}{0.474100in}}{\pgfqpoint{4.227273in}{3.318700in}}%
\pgfusepath{clip}%
\pgfsetbuttcap%
\pgfsetroundjoin%
\definecolor{currentfill}{rgb}{0.993248,0.906157,0.143936}%
\pgfsetfillcolor{currentfill}%
\pgfsetfillopacity{0.700000}%
\pgfsetlinewidth{0.000000pt}%
\definecolor{currentstroke}{rgb}{0.000000,0.000000,0.000000}%
\pgfsetstrokecolor{currentstroke}%
\pgfsetstrokeopacity{0.700000}%
\pgfsetdash{}{0pt}%
\pgfpathmoveto{\pgfqpoint{9.654489in}{2.291135in}}%
\pgfpathcurveto{\pgfqpoint{9.659533in}{2.291135in}}{\pgfqpoint{9.664371in}{2.293139in}}{\pgfqpoint{9.667937in}{2.296706in}}%
\pgfpathcurveto{\pgfqpoint{9.671504in}{2.300272in}}{\pgfqpoint{9.673508in}{2.305110in}}{\pgfqpoint{9.673508in}{2.310154in}}%
\pgfpathcurveto{\pgfqpoint{9.673508in}{2.315197in}}{\pgfqpoint{9.671504in}{2.320035in}}{\pgfqpoint{9.667937in}{2.323601in}}%
\pgfpathcurveto{\pgfqpoint{9.664371in}{2.327168in}}{\pgfqpoint{9.659533in}{2.329172in}}{\pgfqpoint{9.654489in}{2.329172in}}%
\pgfpathcurveto{\pgfqpoint{9.649446in}{2.329172in}}{\pgfqpoint{9.644608in}{2.327168in}}{\pgfqpoint{9.641041in}{2.323601in}}%
\pgfpathcurveto{\pgfqpoint{9.637475in}{2.320035in}}{\pgfqpoint{9.635471in}{2.315197in}}{\pgfqpoint{9.635471in}{2.310154in}}%
\pgfpathcurveto{\pgfqpoint{9.635471in}{2.305110in}}{\pgfqpoint{9.637475in}{2.300272in}}{\pgfqpoint{9.641041in}{2.296706in}}%
\pgfpathcurveto{\pgfqpoint{9.644608in}{2.293139in}}{\pgfqpoint{9.649446in}{2.291135in}}{\pgfqpoint{9.654489in}{2.291135in}}%
\pgfpathclose%
\pgfusepath{fill}%
\end{pgfscope}%
\begin{pgfscope}%
\pgfpathrectangle{\pgfqpoint{6.572727in}{0.474100in}}{\pgfqpoint{4.227273in}{3.318700in}}%
\pgfusepath{clip}%
\pgfsetbuttcap%
\pgfsetroundjoin%
\definecolor{currentfill}{rgb}{0.993248,0.906157,0.143936}%
\pgfsetfillcolor{currentfill}%
\pgfsetfillopacity{0.700000}%
\pgfsetlinewidth{0.000000pt}%
\definecolor{currentstroke}{rgb}{0.000000,0.000000,0.000000}%
\pgfsetstrokecolor{currentstroke}%
\pgfsetstrokeopacity{0.700000}%
\pgfsetdash{}{0pt}%
\pgfpathmoveto{\pgfqpoint{9.831714in}{1.444850in}}%
\pgfpathcurveto{\pgfqpoint{9.836757in}{1.444850in}}{\pgfqpoint{9.841595in}{1.446854in}}{\pgfqpoint{9.845161in}{1.450421in}}%
\pgfpathcurveto{\pgfqpoint{9.848728in}{1.453987in}}{\pgfqpoint{9.850732in}{1.458825in}}{\pgfqpoint{9.850732in}{1.463869in}}%
\pgfpathcurveto{\pgfqpoint{9.850732in}{1.468912in}}{\pgfqpoint{9.848728in}{1.473750in}}{\pgfqpoint{9.845161in}{1.477316in}}%
\pgfpathcurveto{\pgfqpoint{9.841595in}{1.480883in}}{\pgfqpoint{9.836757in}{1.482887in}}{\pgfqpoint{9.831714in}{1.482887in}}%
\pgfpathcurveto{\pgfqpoint{9.826670in}{1.482887in}}{\pgfqpoint{9.821832in}{1.480883in}}{\pgfqpoint{9.818266in}{1.477316in}}%
\pgfpathcurveto{\pgfqpoint{9.814699in}{1.473750in}}{\pgfqpoint{9.812695in}{1.468912in}}{\pgfqpoint{9.812695in}{1.463869in}}%
\pgfpathcurveto{\pgfqpoint{9.812695in}{1.458825in}}{\pgfqpoint{9.814699in}{1.453987in}}{\pgfqpoint{9.818266in}{1.450421in}}%
\pgfpathcurveto{\pgfqpoint{9.821832in}{1.446854in}}{\pgfqpoint{9.826670in}{1.444850in}}{\pgfqpoint{9.831714in}{1.444850in}}%
\pgfpathclose%
\pgfusepath{fill}%
\end{pgfscope}%
\begin{pgfscope}%
\pgfpathrectangle{\pgfqpoint{6.572727in}{0.474100in}}{\pgfqpoint{4.227273in}{3.318700in}}%
\pgfusepath{clip}%
\pgfsetbuttcap%
\pgfsetroundjoin%
\definecolor{currentfill}{rgb}{0.993248,0.906157,0.143936}%
\pgfsetfillcolor{currentfill}%
\pgfsetfillopacity{0.700000}%
\pgfsetlinewidth{0.000000pt}%
\definecolor{currentstroke}{rgb}{0.000000,0.000000,0.000000}%
\pgfsetstrokecolor{currentstroke}%
\pgfsetstrokeopacity{0.700000}%
\pgfsetdash{}{0pt}%
\pgfpathmoveto{\pgfqpoint{9.821301in}{2.324524in}}%
\pgfpathcurveto{\pgfqpoint{9.826345in}{2.324524in}}{\pgfqpoint{9.831182in}{2.326528in}}{\pgfqpoint{9.834749in}{2.330095in}}%
\pgfpathcurveto{\pgfqpoint{9.838315in}{2.333661in}}{\pgfqpoint{9.840319in}{2.338499in}}{\pgfqpoint{9.840319in}{2.343542in}}%
\pgfpathcurveto{\pgfqpoint{9.840319in}{2.348586in}}{\pgfqpoint{9.838315in}{2.353424in}}{\pgfqpoint{9.834749in}{2.356990in}}%
\pgfpathcurveto{\pgfqpoint{9.831182in}{2.360557in}}{\pgfqpoint{9.826345in}{2.362561in}}{\pgfqpoint{9.821301in}{2.362561in}}%
\pgfpathcurveto{\pgfqpoint{9.816257in}{2.362561in}}{\pgfqpoint{9.811420in}{2.360557in}}{\pgfqpoint{9.807853in}{2.356990in}}%
\pgfpathcurveto{\pgfqpoint{9.804287in}{2.353424in}}{\pgfqpoint{9.802283in}{2.348586in}}{\pgfqpoint{9.802283in}{2.343542in}}%
\pgfpathcurveto{\pgfqpoint{9.802283in}{2.338499in}}{\pgfqpoint{9.804287in}{2.333661in}}{\pgfqpoint{9.807853in}{2.330095in}}%
\pgfpathcurveto{\pgfqpoint{9.811420in}{2.326528in}}{\pgfqpoint{9.816257in}{2.324524in}}{\pgfqpoint{9.821301in}{2.324524in}}%
\pgfpathclose%
\pgfusepath{fill}%
\end{pgfscope}%
\begin{pgfscope}%
\pgfpathrectangle{\pgfqpoint{6.572727in}{0.474100in}}{\pgfqpoint{4.227273in}{3.318700in}}%
\pgfusepath{clip}%
\pgfsetbuttcap%
\pgfsetroundjoin%
\definecolor{currentfill}{rgb}{0.127568,0.566949,0.550556}%
\pgfsetfillcolor{currentfill}%
\pgfsetfillopacity{0.700000}%
\pgfsetlinewidth{0.000000pt}%
\definecolor{currentstroke}{rgb}{0.000000,0.000000,0.000000}%
\pgfsetstrokecolor{currentstroke}%
\pgfsetstrokeopacity{0.700000}%
\pgfsetdash{}{0pt}%
\pgfpathmoveto{\pgfqpoint{8.129570in}{1.380188in}}%
\pgfpathcurveto{\pgfqpoint{8.134614in}{1.380188in}}{\pgfqpoint{8.139452in}{1.382192in}}{\pgfqpoint{8.143018in}{1.385759in}}%
\pgfpathcurveto{\pgfqpoint{8.146585in}{1.389325in}}{\pgfqpoint{8.148589in}{1.394163in}}{\pgfqpoint{8.148589in}{1.399207in}}%
\pgfpathcurveto{\pgfqpoint{8.148589in}{1.404250in}}{\pgfqpoint{8.146585in}{1.409088in}}{\pgfqpoint{8.143018in}{1.412654in}}%
\pgfpathcurveto{\pgfqpoint{8.139452in}{1.416221in}}{\pgfqpoint{8.134614in}{1.418225in}}{\pgfqpoint{8.129570in}{1.418225in}}%
\pgfpathcurveto{\pgfqpoint{8.124527in}{1.418225in}}{\pgfqpoint{8.119689in}{1.416221in}}{\pgfqpoint{8.116123in}{1.412654in}}%
\pgfpathcurveto{\pgfqpoint{8.112556in}{1.409088in}}{\pgfqpoint{8.110552in}{1.404250in}}{\pgfqpoint{8.110552in}{1.399207in}}%
\pgfpathcurveto{\pgfqpoint{8.110552in}{1.394163in}}{\pgfqpoint{8.112556in}{1.389325in}}{\pgfqpoint{8.116123in}{1.385759in}}%
\pgfpathcurveto{\pgfqpoint{8.119689in}{1.382192in}}{\pgfqpoint{8.124527in}{1.380188in}}{\pgfqpoint{8.129570in}{1.380188in}}%
\pgfpathclose%
\pgfusepath{fill}%
\end{pgfscope}%
\begin{pgfscope}%
\pgfpathrectangle{\pgfqpoint{6.572727in}{0.474100in}}{\pgfqpoint{4.227273in}{3.318700in}}%
\pgfusepath{clip}%
\pgfsetbuttcap%
\pgfsetroundjoin%
\definecolor{currentfill}{rgb}{0.993248,0.906157,0.143936}%
\pgfsetfillcolor{currentfill}%
\pgfsetfillopacity{0.700000}%
\pgfsetlinewidth{0.000000pt}%
\definecolor{currentstroke}{rgb}{0.000000,0.000000,0.000000}%
\pgfsetstrokecolor{currentstroke}%
\pgfsetstrokeopacity{0.700000}%
\pgfsetdash{}{0pt}%
\pgfpathmoveto{\pgfqpoint{9.330031in}{2.075572in}}%
\pgfpathcurveto{\pgfqpoint{9.335075in}{2.075572in}}{\pgfqpoint{9.339912in}{2.077576in}}{\pgfqpoint{9.343479in}{2.081143in}}%
\pgfpathcurveto{\pgfqpoint{9.347045in}{2.084709in}}{\pgfqpoint{9.349049in}{2.089547in}}{\pgfqpoint{9.349049in}{2.094591in}}%
\pgfpathcurveto{\pgfqpoint{9.349049in}{2.099634in}}{\pgfqpoint{9.347045in}{2.104472in}}{\pgfqpoint{9.343479in}{2.108038in}}%
\pgfpathcurveto{\pgfqpoint{9.339912in}{2.111605in}}{\pgfqpoint{9.335075in}{2.113609in}}{\pgfqpoint{9.330031in}{2.113609in}}%
\pgfpathcurveto{\pgfqpoint{9.324987in}{2.113609in}}{\pgfqpoint{9.320150in}{2.111605in}}{\pgfqpoint{9.316583in}{2.108038in}}%
\pgfpathcurveto{\pgfqpoint{9.313017in}{2.104472in}}{\pgfqpoint{9.311013in}{2.099634in}}{\pgfqpoint{9.311013in}{2.094591in}}%
\pgfpathcurveto{\pgfqpoint{9.311013in}{2.089547in}}{\pgfqpoint{9.313017in}{2.084709in}}{\pgfqpoint{9.316583in}{2.081143in}}%
\pgfpathcurveto{\pgfqpoint{9.320150in}{2.077576in}}{\pgfqpoint{9.324987in}{2.075572in}}{\pgfqpoint{9.330031in}{2.075572in}}%
\pgfpathclose%
\pgfusepath{fill}%
\end{pgfscope}%
\begin{pgfscope}%
\pgfpathrectangle{\pgfqpoint{6.572727in}{0.474100in}}{\pgfqpoint{4.227273in}{3.318700in}}%
\pgfusepath{clip}%
\pgfsetbuttcap%
\pgfsetroundjoin%
\definecolor{currentfill}{rgb}{0.127568,0.566949,0.550556}%
\pgfsetfillcolor{currentfill}%
\pgfsetfillopacity{0.700000}%
\pgfsetlinewidth{0.000000pt}%
\definecolor{currentstroke}{rgb}{0.000000,0.000000,0.000000}%
\pgfsetstrokecolor{currentstroke}%
\pgfsetstrokeopacity{0.700000}%
\pgfsetdash{}{0pt}%
\pgfpathmoveto{\pgfqpoint{8.165087in}{2.963310in}}%
\pgfpathcurveto{\pgfqpoint{8.170131in}{2.963310in}}{\pgfqpoint{8.174969in}{2.965314in}}{\pgfqpoint{8.178535in}{2.968880in}}%
\pgfpathcurveto{\pgfqpoint{8.182101in}{2.972447in}}{\pgfqpoint{8.184105in}{2.977285in}}{\pgfqpoint{8.184105in}{2.982328in}}%
\pgfpathcurveto{\pgfqpoint{8.184105in}{2.987372in}}{\pgfqpoint{8.182101in}{2.992210in}}{\pgfqpoint{8.178535in}{2.995776in}}%
\pgfpathcurveto{\pgfqpoint{8.174969in}{2.999343in}}{\pgfqpoint{8.170131in}{3.001347in}}{\pgfqpoint{8.165087in}{3.001347in}}%
\pgfpathcurveto{\pgfqpoint{8.160043in}{3.001347in}}{\pgfqpoint{8.155206in}{2.999343in}}{\pgfqpoint{8.151639in}{2.995776in}}%
\pgfpathcurveto{\pgfqpoint{8.148073in}{2.992210in}}{\pgfqpoint{8.146069in}{2.987372in}}{\pgfqpoint{8.146069in}{2.982328in}}%
\pgfpathcurveto{\pgfqpoint{8.146069in}{2.977285in}}{\pgfqpoint{8.148073in}{2.972447in}}{\pgfqpoint{8.151639in}{2.968880in}}%
\pgfpathcurveto{\pgfqpoint{8.155206in}{2.965314in}}{\pgfqpoint{8.160043in}{2.963310in}}{\pgfqpoint{8.165087in}{2.963310in}}%
\pgfpathclose%
\pgfusepath{fill}%
\end{pgfscope}%
\begin{pgfscope}%
\pgfpathrectangle{\pgfqpoint{6.572727in}{0.474100in}}{\pgfqpoint{4.227273in}{3.318700in}}%
\pgfusepath{clip}%
\pgfsetbuttcap%
\pgfsetroundjoin%
\definecolor{currentfill}{rgb}{0.127568,0.566949,0.550556}%
\pgfsetfillcolor{currentfill}%
\pgfsetfillopacity{0.700000}%
\pgfsetlinewidth{0.000000pt}%
\definecolor{currentstroke}{rgb}{0.000000,0.000000,0.000000}%
\pgfsetstrokecolor{currentstroke}%
\pgfsetstrokeopacity{0.700000}%
\pgfsetdash{}{0pt}%
\pgfpathmoveto{\pgfqpoint{7.494331in}{1.378188in}}%
\pgfpathcurveto{\pgfqpoint{7.499374in}{1.378188in}}{\pgfqpoint{7.504212in}{1.380192in}}{\pgfqpoint{7.507778in}{1.383758in}}%
\pgfpathcurveto{\pgfqpoint{7.511345in}{1.387325in}}{\pgfqpoint{7.513349in}{1.392163in}}{\pgfqpoint{7.513349in}{1.397206in}}%
\pgfpathcurveto{\pgfqpoint{7.513349in}{1.402250in}}{\pgfqpoint{7.511345in}{1.407088in}}{\pgfqpoint{7.507778in}{1.410654in}}%
\pgfpathcurveto{\pgfqpoint{7.504212in}{1.414221in}}{\pgfqpoint{7.499374in}{1.416224in}}{\pgfqpoint{7.494331in}{1.416224in}}%
\pgfpathcurveto{\pgfqpoint{7.489287in}{1.416224in}}{\pgfqpoint{7.484449in}{1.414221in}}{\pgfqpoint{7.480883in}{1.410654in}}%
\pgfpathcurveto{\pgfqpoint{7.477316in}{1.407088in}}{\pgfqpoint{7.475312in}{1.402250in}}{\pgfqpoint{7.475312in}{1.397206in}}%
\pgfpathcurveto{\pgfqpoint{7.475312in}{1.392163in}}{\pgfqpoint{7.477316in}{1.387325in}}{\pgfqpoint{7.480883in}{1.383758in}}%
\pgfpathcurveto{\pgfqpoint{7.484449in}{1.380192in}}{\pgfqpoint{7.489287in}{1.378188in}}{\pgfqpoint{7.494331in}{1.378188in}}%
\pgfpathclose%
\pgfusepath{fill}%
\end{pgfscope}%
\begin{pgfscope}%
\pgfpathrectangle{\pgfqpoint{6.572727in}{0.474100in}}{\pgfqpoint{4.227273in}{3.318700in}}%
\pgfusepath{clip}%
\pgfsetbuttcap%
\pgfsetroundjoin%
\definecolor{currentfill}{rgb}{0.127568,0.566949,0.550556}%
\pgfsetfillcolor{currentfill}%
\pgfsetfillopacity{0.700000}%
\pgfsetlinewidth{0.000000pt}%
\definecolor{currentstroke}{rgb}{0.000000,0.000000,0.000000}%
\pgfsetstrokecolor{currentstroke}%
\pgfsetstrokeopacity{0.700000}%
\pgfsetdash{}{0pt}%
\pgfpathmoveto{\pgfqpoint{7.507483in}{1.657561in}}%
\pgfpathcurveto{\pgfqpoint{7.512527in}{1.657561in}}{\pgfqpoint{7.517365in}{1.659565in}}{\pgfqpoint{7.520931in}{1.663132in}}%
\pgfpathcurveto{\pgfqpoint{7.524498in}{1.666698in}}{\pgfqpoint{7.526502in}{1.671536in}}{\pgfqpoint{7.526502in}{1.676579in}}%
\pgfpathcurveto{\pgfqpoint{7.526502in}{1.681623in}}{\pgfqpoint{7.524498in}{1.686461in}}{\pgfqpoint{7.520931in}{1.690027in}}%
\pgfpathcurveto{\pgfqpoint{7.517365in}{1.693594in}}{\pgfqpoint{7.512527in}{1.695598in}}{\pgfqpoint{7.507483in}{1.695598in}}%
\pgfpathcurveto{\pgfqpoint{7.502440in}{1.695598in}}{\pgfqpoint{7.497602in}{1.693594in}}{\pgfqpoint{7.494036in}{1.690027in}}%
\pgfpathcurveto{\pgfqpoint{7.490469in}{1.686461in}}{\pgfqpoint{7.488465in}{1.681623in}}{\pgfqpoint{7.488465in}{1.676579in}}%
\pgfpathcurveto{\pgfqpoint{7.488465in}{1.671536in}}{\pgfqpoint{7.490469in}{1.666698in}}{\pgfqpoint{7.494036in}{1.663132in}}%
\pgfpathcurveto{\pgfqpoint{7.497602in}{1.659565in}}{\pgfqpoint{7.502440in}{1.657561in}}{\pgfqpoint{7.507483in}{1.657561in}}%
\pgfpathclose%
\pgfusepath{fill}%
\end{pgfscope}%
\begin{pgfscope}%
\pgfpathrectangle{\pgfqpoint{6.572727in}{0.474100in}}{\pgfqpoint{4.227273in}{3.318700in}}%
\pgfusepath{clip}%
\pgfsetbuttcap%
\pgfsetroundjoin%
\definecolor{currentfill}{rgb}{0.993248,0.906157,0.143936}%
\pgfsetfillcolor{currentfill}%
\pgfsetfillopacity{0.700000}%
\pgfsetlinewidth{0.000000pt}%
\definecolor{currentstroke}{rgb}{0.000000,0.000000,0.000000}%
\pgfsetstrokecolor{currentstroke}%
\pgfsetstrokeopacity{0.700000}%
\pgfsetdash{}{0pt}%
\pgfpathmoveto{\pgfqpoint{10.420660in}{1.778469in}}%
\pgfpathcurveto{\pgfqpoint{10.425704in}{1.778469in}}{\pgfqpoint{10.430541in}{1.780473in}}{\pgfqpoint{10.434108in}{1.784040in}}%
\pgfpathcurveto{\pgfqpoint{10.437674in}{1.787606in}}{\pgfqpoint{10.439678in}{1.792444in}}{\pgfqpoint{10.439678in}{1.797487in}}%
\pgfpathcurveto{\pgfqpoint{10.439678in}{1.802531in}}{\pgfqpoint{10.437674in}{1.807369in}}{\pgfqpoint{10.434108in}{1.810935in}}%
\pgfpathcurveto{\pgfqpoint{10.430541in}{1.814502in}}{\pgfqpoint{10.425704in}{1.816506in}}{\pgfqpoint{10.420660in}{1.816506in}}%
\pgfpathcurveto{\pgfqpoint{10.415616in}{1.816506in}}{\pgfqpoint{10.410778in}{1.814502in}}{\pgfqpoint{10.407212in}{1.810935in}}%
\pgfpathcurveto{\pgfqpoint{10.403646in}{1.807369in}}{\pgfqpoint{10.401642in}{1.802531in}}{\pgfqpoint{10.401642in}{1.797487in}}%
\pgfpathcurveto{\pgfqpoint{10.401642in}{1.792444in}}{\pgfqpoint{10.403646in}{1.787606in}}{\pgfqpoint{10.407212in}{1.784040in}}%
\pgfpathcurveto{\pgfqpoint{10.410778in}{1.780473in}}{\pgfqpoint{10.415616in}{1.778469in}}{\pgfqpoint{10.420660in}{1.778469in}}%
\pgfpathclose%
\pgfusepath{fill}%
\end{pgfscope}%
\begin{pgfscope}%
\pgfpathrectangle{\pgfqpoint{6.572727in}{0.474100in}}{\pgfqpoint{4.227273in}{3.318700in}}%
\pgfusepath{clip}%
\pgfsetbuttcap%
\pgfsetroundjoin%
\definecolor{currentfill}{rgb}{0.993248,0.906157,0.143936}%
\pgfsetfillcolor{currentfill}%
\pgfsetfillopacity{0.700000}%
\pgfsetlinewidth{0.000000pt}%
\definecolor{currentstroke}{rgb}{0.000000,0.000000,0.000000}%
\pgfsetstrokecolor{currentstroke}%
\pgfsetstrokeopacity{0.700000}%
\pgfsetdash{}{0pt}%
\pgfpathmoveto{\pgfqpoint{9.917552in}{0.937249in}}%
\pgfpathcurveto{\pgfqpoint{9.922596in}{0.937249in}}{\pgfqpoint{9.927433in}{0.939253in}}{\pgfqpoint{9.931000in}{0.942819in}}%
\pgfpathcurveto{\pgfqpoint{9.934566in}{0.946386in}}{\pgfqpoint{9.936570in}{0.951224in}}{\pgfqpoint{9.936570in}{0.956267in}}%
\pgfpathcurveto{\pgfqpoint{9.936570in}{0.961311in}}{\pgfqpoint{9.934566in}{0.966149in}}{\pgfqpoint{9.931000in}{0.969715in}}%
\pgfpathcurveto{\pgfqpoint{9.927433in}{0.973282in}}{\pgfqpoint{9.922596in}{0.975285in}}{\pgfqpoint{9.917552in}{0.975285in}}%
\pgfpathcurveto{\pgfqpoint{9.912508in}{0.975285in}}{\pgfqpoint{9.907670in}{0.973282in}}{\pgfqpoint{9.904104in}{0.969715in}}%
\pgfpathcurveto{\pgfqpoint{9.900538in}{0.966149in}}{\pgfqpoint{9.898534in}{0.961311in}}{\pgfqpoint{9.898534in}{0.956267in}}%
\pgfpathcurveto{\pgfqpoint{9.898534in}{0.951224in}}{\pgfqpoint{9.900538in}{0.946386in}}{\pgfqpoint{9.904104in}{0.942819in}}%
\pgfpathcurveto{\pgfqpoint{9.907670in}{0.939253in}}{\pgfqpoint{9.912508in}{0.937249in}}{\pgfqpoint{9.917552in}{0.937249in}}%
\pgfpathclose%
\pgfusepath{fill}%
\end{pgfscope}%
\begin{pgfscope}%
\pgfpathrectangle{\pgfqpoint{6.572727in}{0.474100in}}{\pgfqpoint{4.227273in}{3.318700in}}%
\pgfusepath{clip}%
\pgfsetbuttcap%
\pgfsetroundjoin%
\definecolor{currentfill}{rgb}{0.993248,0.906157,0.143936}%
\pgfsetfillcolor{currentfill}%
\pgfsetfillopacity{0.700000}%
\pgfsetlinewidth{0.000000pt}%
\definecolor{currentstroke}{rgb}{0.000000,0.000000,0.000000}%
\pgfsetstrokecolor{currentstroke}%
\pgfsetstrokeopacity{0.700000}%
\pgfsetdash{}{0pt}%
\pgfpathmoveto{\pgfqpoint{9.715546in}{1.425406in}}%
\pgfpathcurveto{\pgfqpoint{9.720590in}{1.425406in}}{\pgfqpoint{9.725428in}{1.427410in}}{\pgfqpoint{9.728994in}{1.430976in}}%
\pgfpathcurveto{\pgfqpoint{9.732560in}{1.434542in}}{\pgfqpoint{9.734564in}{1.439380in}}{\pgfqpoint{9.734564in}{1.444424in}}%
\pgfpathcurveto{\pgfqpoint{9.734564in}{1.449468in}}{\pgfqpoint{9.732560in}{1.454305in}}{\pgfqpoint{9.728994in}{1.457872in}}%
\pgfpathcurveto{\pgfqpoint{9.725428in}{1.461438in}}{\pgfqpoint{9.720590in}{1.463442in}}{\pgfqpoint{9.715546in}{1.463442in}}%
\pgfpathcurveto{\pgfqpoint{9.710503in}{1.463442in}}{\pgfqpoint{9.705665in}{1.461438in}}{\pgfqpoint{9.702098in}{1.457872in}}%
\pgfpathcurveto{\pgfqpoint{9.698532in}{1.454305in}}{\pgfqpoint{9.696528in}{1.449468in}}{\pgfqpoint{9.696528in}{1.444424in}}%
\pgfpathcurveto{\pgfqpoint{9.696528in}{1.439380in}}{\pgfqpoint{9.698532in}{1.434542in}}{\pgfqpoint{9.702098in}{1.430976in}}%
\pgfpathcurveto{\pgfqpoint{9.705665in}{1.427410in}}{\pgfqpoint{9.710503in}{1.425406in}}{\pgfqpoint{9.715546in}{1.425406in}}%
\pgfpathclose%
\pgfusepath{fill}%
\end{pgfscope}%
\begin{pgfscope}%
\pgfpathrectangle{\pgfqpoint{6.572727in}{0.474100in}}{\pgfqpoint{4.227273in}{3.318700in}}%
\pgfusepath{clip}%
\pgfsetbuttcap%
\pgfsetroundjoin%
\definecolor{currentfill}{rgb}{0.993248,0.906157,0.143936}%
\pgfsetfillcolor{currentfill}%
\pgfsetfillopacity{0.700000}%
\pgfsetlinewidth{0.000000pt}%
\definecolor{currentstroke}{rgb}{0.000000,0.000000,0.000000}%
\pgfsetstrokecolor{currentstroke}%
\pgfsetstrokeopacity{0.700000}%
\pgfsetdash{}{0pt}%
\pgfpathmoveto{\pgfqpoint{9.740088in}{1.659490in}}%
\pgfpathcurveto{\pgfqpoint{9.745132in}{1.659490in}}{\pgfqpoint{9.749970in}{1.661494in}}{\pgfqpoint{9.753536in}{1.665060in}}%
\pgfpathcurveto{\pgfqpoint{9.757103in}{1.668626in}}{\pgfqpoint{9.759107in}{1.673464in}}{\pgfqpoint{9.759107in}{1.678508in}}%
\pgfpathcurveto{\pgfqpoint{9.759107in}{1.683552in}}{\pgfqpoint{9.757103in}{1.688389in}}{\pgfqpoint{9.753536in}{1.691956in}}%
\pgfpathcurveto{\pgfqpoint{9.749970in}{1.695522in}}{\pgfqpoint{9.745132in}{1.697526in}}{\pgfqpoint{9.740088in}{1.697526in}}%
\pgfpathcurveto{\pgfqpoint{9.735045in}{1.697526in}}{\pgfqpoint{9.730207in}{1.695522in}}{\pgfqpoint{9.726641in}{1.691956in}}%
\pgfpathcurveto{\pgfqpoint{9.723074in}{1.688389in}}{\pgfqpoint{9.721070in}{1.683552in}}{\pgfqpoint{9.721070in}{1.678508in}}%
\pgfpathcurveto{\pgfqpoint{9.721070in}{1.673464in}}{\pgfqpoint{9.723074in}{1.668626in}}{\pgfqpoint{9.726641in}{1.665060in}}%
\pgfpathcurveto{\pgfqpoint{9.730207in}{1.661494in}}{\pgfqpoint{9.735045in}{1.659490in}}{\pgfqpoint{9.740088in}{1.659490in}}%
\pgfpathclose%
\pgfusepath{fill}%
\end{pgfscope}%
\begin{pgfscope}%
\pgfpathrectangle{\pgfqpoint{6.572727in}{0.474100in}}{\pgfqpoint{4.227273in}{3.318700in}}%
\pgfusepath{clip}%
\pgfsetbuttcap%
\pgfsetroundjoin%
\definecolor{currentfill}{rgb}{0.127568,0.566949,0.550556}%
\pgfsetfillcolor{currentfill}%
\pgfsetfillopacity{0.700000}%
\pgfsetlinewidth{0.000000pt}%
\definecolor{currentstroke}{rgb}{0.000000,0.000000,0.000000}%
\pgfsetstrokecolor{currentstroke}%
\pgfsetstrokeopacity{0.700000}%
\pgfsetdash{}{0pt}%
\pgfpathmoveto{\pgfqpoint{7.619632in}{1.645548in}}%
\pgfpathcurveto{\pgfqpoint{7.624675in}{1.645548in}}{\pgfqpoint{7.629513in}{1.647551in}}{\pgfqpoint{7.633080in}{1.651118in}}%
\pgfpathcurveto{\pgfqpoint{7.636646in}{1.654684in}}{\pgfqpoint{7.638650in}{1.659522in}}{\pgfqpoint{7.638650in}{1.664566in}}%
\pgfpathcurveto{\pgfqpoint{7.638650in}{1.669609in}}{\pgfqpoint{7.636646in}{1.674447in}}{\pgfqpoint{7.633080in}{1.678014in}}%
\pgfpathcurveto{\pgfqpoint{7.629513in}{1.681580in}}{\pgfqpoint{7.624675in}{1.683584in}}{\pgfqpoint{7.619632in}{1.683584in}}%
\pgfpathcurveto{\pgfqpoint{7.614588in}{1.683584in}}{\pgfqpoint{7.609750in}{1.681580in}}{\pgfqpoint{7.606184in}{1.678014in}}%
\pgfpathcurveto{\pgfqpoint{7.602618in}{1.674447in}}{\pgfqpoint{7.600614in}{1.669609in}}{\pgfqpoint{7.600614in}{1.664566in}}%
\pgfpathcurveto{\pgfqpoint{7.600614in}{1.659522in}}{\pgfqpoint{7.602618in}{1.654684in}}{\pgfqpoint{7.606184in}{1.651118in}}%
\pgfpathcurveto{\pgfqpoint{7.609750in}{1.647551in}}{\pgfqpoint{7.614588in}{1.645548in}}{\pgfqpoint{7.619632in}{1.645548in}}%
\pgfpathclose%
\pgfusepath{fill}%
\end{pgfscope}%
\begin{pgfscope}%
\pgfpathrectangle{\pgfqpoint{6.572727in}{0.474100in}}{\pgfqpoint{4.227273in}{3.318700in}}%
\pgfusepath{clip}%
\pgfsetbuttcap%
\pgfsetroundjoin%
\definecolor{currentfill}{rgb}{0.993248,0.906157,0.143936}%
\pgfsetfillcolor{currentfill}%
\pgfsetfillopacity{0.700000}%
\pgfsetlinewidth{0.000000pt}%
\definecolor{currentstroke}{rgb}{0.000000,0.000000,0.000000}%
\pgfsetstrokecolor{currentstroke}%
\pgfsetstrokeopacity{0.700000}%
\pgfsetdash{}{0pt}%
\pgfpathmoveto{\pgfqpoint{9.385049in}{1.613877in}}%
\pgfpathcurveto{\pgfqpoint{9.390093in}{1.613877in}}{\pgfqpoint{9.394931in}{1.615881in}}{\pgfqpoint{9.398497in}{1.619448in}}%
\pgfpathcurveto{\pgfqpoint{9.402063in}{1.623014in}}{\pgfqpoint{9.404067in}{1.627852in}}{\pgfqpoint{9.404067in}{1.632895in}}%
\pgfpathcurveto{\pgfqpoint{9.404067in}{1.637939in}}{\pgfqpoint{9.402063in}{1.642777in}}{\pgfqpoint{9.398497in}{1.646343in}}%
\pgfpathcurveto{\pgfqpoint{9.394931in}{1.649910in}}{\pgfqpoint{9.390093in}{1.651914in}}{\pgfqpoint{9.385049in}{1.651914in}}%
\pgfpathcurveto{\pgfqpoint{9.380006in}{1.651914in}}{\pgfqpoint{9.375168in}{1.649910in}}{\pgfqpoint{9.371601in}{1.646343in}}%
\pgfpathcurveto{\pgfqpoint{9.368035in}{1.642777in}}{\pgfqpoint{9.366031in}{1.637939in}}{\pgfqpoint{9.366031in}{1.632895in}}%
\pgfpathcurveto{\pgfqpoint{9.366031in}{1.627852in}}{\pgfqpoint{9.368035in}{1.623014in}}{\pgfqpoint{9.371601in}{1.619448in}}%
\pgfpathcurveto{\pgfqpoint{9.375168in}{1.615881in}}{\pgfqpoint{9.380006in}{1.613877in}}{\pgfqpoint{9.385049in}{1.613877in}}%
\pgfpathclose%
\pgfusepath{fill}%
\end{pgfscope}%
\begin{pgfscope}%
\pgfpathrectangle{\pgfqpoint{6.572727in}{0.474100in}}{\pgfqpoint{4.227273in}{3.318700in}}%
\pgfusepath{clip}%
\pgfsetbuttcap%
\pgfsetroundjoin%
\definecolor{currentfill}{rgb}{0.267004,0.004874,0.329415}%
\pgfsetfillcolor{currentfill}%
\pgfsetfillopacity{0.700000}%
\pgfsetlinewidth{0.000000pt}%
\definecolor{currentstroke}{rgb}{0.000000,0.000000,0.000000}%
\pgfsetstrokecolor{currentstroke}%
\pgfsetstrokeopacity{0.700000}%
\pgfsetdash{}{0pt}%
\pgfpathmoveto{\pgfqpoint{6.836372in}{1.169495in}}%
\pgfpathcurveto{\pgfqpoint{6.841416in}{1.169495in}}{\pgfqpoint{6.846253in}{1.171499in}}{\pgfqpoint{6.849820in}{1.175065in}}%
\pgfpathcurveto{\pgfqpoint{6.853386in}{1.178632in}}{\pgfqpoint{6.855390in}{1.183470in}}{\pgfqpoint{6.855390in}{1.188513in}}%
\pgfpathcurveto{\pgfqpoint{6.855390in}{1.193557in}}{\pgfqpoint{6.853386in}{1.198395in}}{\pgfqpoint{6.849820in}{1.201961in}}%
\pgfpathcurveto{\pgfqpoint{6.846253in}{1.205527in}}{\pgfqpoint{6.841416in}{1.207531in}}{\pgfqpoint{6.836372in}{1.207531in}}%
\pgfpathcurveto{\pgfqpoint{6.831328in}{1.207531in}}{\pgfqpoint{6.826490in}{1.205527in}}{\pgfqpoint{6.822924in}{1.201961in}}%
\pgfpathcurveto{\pgfqpoint{6.819358in}{1.198395in}}{\pgfqpoint{6.817354in}{1.193557in}}{\pgfqpoint{6.817354in}{1.188513in}}%
\pgfpathcurveto{\pgfqpoint{6.817354in}{1.183470in}}{\pgfqpoint{6.819358in}{1.178632in}}{\pgfqpoint{6.822924in}{1.175065in}}%
\pgfpathcurveto{\pgfqpoint{6.826490in}{1.171499in}}{\pgfqpoint{6.831328in}{1.169495in}}{\pgfqpoint{6.836372in}{1.169495in}}%
\pgfpathclose%
\pgfusepath{fill}%
\end{pgfscope}%
\begin{pgfscope}%
\pgfpathrectangle{\pgfqpoint{6.572727in}{0.474100in}}{\pgfqpoint{4.227273in}{3.318700in}}%
\pgfusepath{clip}%
\pgfsetbuttcap%
\pgfsetroundjoin%
\definecolor{currentfill}{rgb}{0.993248,0.906157,0.143936}%
\pgfsetfillcolor{currentfill}%
\pgfsetfillopacity{0.700000}%
\pgfsetlinewidth{0.000000pt}%
\definecolor{currentstroke}{rgb}{0.000000,0.000000,0.000000}%
\pgfsetstrokecolor{currentstroke}%
\pgfsetstrokeopacity{0.700000}%
\pgfsetdash{}{0pt}%
\pgfpathmoveto{\pgfqpoint{9.283312in}{2.029996in}}%
\pgfpathcurveto{\pgfqpoint{9.288356in}{2.029996in}}{\pgfqpoint{9.293193in}{2.031999in}}{\pgfqpoint{9.296760in}{2.035566in}}%
\pgfpathcurveto{\pgfqpoint{9.300326in}{2.039132in}}{\pgfqpoint{9.302330in}{2.043970in}}{\pgfqpoint{9.302330in}{2.049014in}}%
\pgfpathcurveto{\pgfqpoint{9.302330in}{2.054057in}}{\pgfqpoint{9.300326in}{2.058895in}}{\pgfqpoint{9.296760in}{2.062462in}}%
\pgfpathcurveto{\pgfqpoint{9.293193in}{2.066028in}}{\pgfqpoint{9.288356in}{2.068032in}}{\pgfqpoint{9.283312in}{2.068032in}}%
\pgfpathcurveto{\pgfqpoint{9.278268in}{2.068032in}}{\pgfqpoint{9.273431in}{2.066028in}}{\pgfqpoint{9.269864in}{2.062462in}}%
\pgfpathcurveto{\pgfqpoint{9.266298in}{2.058895in}}{\pgfqpoint{9.264294in}{2.054057in}}{\pgfqpoint{9.264294in}{2.049014in}}%
\pgfpathcurveto{\pgfqpoint{9.264294in}{2.043970in}}{\pgfqpoint{9.266298in}{2.039132in}}{\pgfqpoint{9.269864in}{2.035566in}}%
\pgfpathcurveto{\pgfqpoint{9.273431in}{2.031999in}}{\pgfqpoint{9.278268in}{2.029996in}}{\pgfqpoint{9.283312in}{2.029996in}}%
\pgfpathclose%
\pgfusepath{fill}%
\end{pgfscope}%
\begin{pgfscope}%
\pgfpathrectangle{\pgfqpoint{6.572727in}{0.474100in}}{\pgfqpoint{4.227273in}{3.318700in}}%
\pgfusepath{clip}%
\pgfsetbuttcap%
\pgfsetroundjoin%
\definecolor{currentfill}{rgb}{0.127568,0.566949,0.550556}%
\pgfsetfillcolor{currentfill}%
\pgfsetfillopacity{0.700000}%
\pgfsetlinewidth{0.000000pt}%
\definecolor{currentstroke}{rgb}{0.000000,0.000000,0.000000}%
\pgfsetstrokecolor{currentstroke}%
\pgfsetstrokeopacity{0.700000}%
\pgfsetdash{}{0pt}%
\pgfpathmoveto{\pgfqpoint{7.758034in}{1.643210in}}%
\pgfpathcurveto{\pgfqpoint{7.763078in}{1.643210in}}{\pgfqpoint{7.767915in}{1.645214in}}{\pgfqpoint{7.771482in}{1.648780in}}%
\pgfpathcurveto{\pgfqpoint{7.775048in}{1.652347in}}{\pgfqpoint{7.777052in}{1.657184in}}{\pgfqpoint{7.777052in}{1.662228in}}%
\pgfpathcurveto{\pgfqpoint{7.777052in}{1.667272in}}{\pgfqpoint{7.775048in}{1.672109in}}{\pgfqpoint{7.771482in}{1.675676in}}%
\pgfpathcurveto{\pgfqpoint{7.767915in}{1.679242in}}{\pgfqpoint{7.763078in}{1.681246in}}{\pgfqpoint{7.758034in}{1.681246in}}%
\pgfpathcurveto{\pgfqpoint{7.752990in}{1.681246in}}{\pgfqpoint{7.748152in}{1.679242in}}{\pgfqpoint{7.744586in}{1.675676in}}%
\pgfpathcurveto{\pgfqpoint{7.741020in}{1.672109in}}{\pgfqpoint{7.739016in}{1.667272in}}{\pgfqpoint{7.739016in}{1.662228in}}%
\pgfpathcurveto{\pgfqpoint{7.739016in}{1.657184in}}{\pgfqpoint{7.741020in}{1.652347in}}{\pgfqpoint{7.744586in}{1.648780in}}%
\pgfpathcurveto{\pgfqpoint{7.748152in}{1.645214in}}{\pgfqpoint{7.752990in}{1.643210in}}{\pgfqpoint{7.758034in}{1.643210in}}%
\pgfpathclose%
\pgfusepath{fill}%
\end{pgfscope}%
\begin{pgfscope}%
\pgfpathrectangle{\pgfqpoint{6.572727in}{0.474100in}}{\pgfqpoint{4.227273in}{3.318700in}}%
\pgfusepath{clip}%
\pgfsetbuttcap%
\pgfsetroundjoin%
\definecolor{currentfill}{rgb}{0.993248,0.906157,0.143936}%
\pgfsetfillcolor{currentfill}%
\pgfsetfillopacity{0.700000}%
\pgfsetlinewidth{0.000000pt}%
\definecolor{currentstroke}{rgb}{0.000000,0.000000,0.000000}%
\pgfsetstrokecolor{currentstroke}%
\pgfsetstrokeopacity{0.700000}%
\pgfsetdash{}{0pt}%
\pgfpathmoveto{\pgfqpoint{9.315670in}{1.240538in}}%
\pgfpathcurveto{\pgfqpoint{9.320714in}{1.240538in}}{\pgfqpoint{9.325551in}{1.242542in}}{\pgfqpoint{9.329118in}{1.246108in}}%
\pgfpathcurveto{\pgfqpoint{9.332684in}{1.249674in}}{\pgfqpoint{9.334688in}{1.254512in}}{\pgfqpoint{9.334688in}{1.259556in}}%
\pgfpathcurveto{\pgfqpoint{9.334688in}{1.264600in}}{\pgfqpoint{9.332684in}{1.269437in}}{\pgfqpoint{9.329118in}{1.273004in}}%
\pgfpathcurveto{\pgfqpoint{9.325551in}{1.276570in}}{\pgfqpoint{9.320714in}{1.278574in}}{\pgfqpoint{9.315670in}{1.278574in}}%
\pgfpathcurveto{\pgfqpoint{9.310626in}{1.278574in}}{\pgfqpoint{9.305789in}{1.276570in}}{\pgfqpoint{9.302222in}{1.273004in}}%
\pgfpathcurveto{\pgfqpoint{9.298656in}{1.269437in}}{\pgfqpoint{9.296652in}{1.264600in}}{\pgfqpoint{9.296652in}{1.259556in}}%
\pgfpathcurveto{\pgfqpoint{9.296652in}{1.254512in}}{\pgfqpoint{9.298656in}{1.249674in}}{\pgfqpoint{9.302222in}{1.246108in}}%
\pgfpathcurveto{\pgfqpoint{9.305789in}{1.242542in}}{\pgfqpoint{9.310626in}{1.240538in}}{\pgfqpoint{9.315670in}{1.240538in}}%
\pgfpathclose%
\pgfusepath{fill}%
\end{pgfscope}%
\begin{pgfscope}%
\pgfpathrectangle{\pgfqpoint{6.572727in}{0.474100in}}{\pgfqpoint{4.227273in}{3.318700in}}%
\pgfusepath{clip}%
\pgfsetbuttcap%
\pgfsetroundjoin%
\definecolor{currentfill}{rgb}{0.127568,0.566949,0.550556}%
\pgfsetfillcolor{currentfill}%
\pgfsetfillopacity{0.700000}%
\pgfsetlinewidth{0.000000pt}%
\definecolor{currentstroke}{rgb}{0.000000,0.000000,0.000000}%
\pgfsetstrokecolor{currentstroke}%
\pgfsetstrokeopacity{0.700000}%
\pgfsetdash{}{0pt}%
\pgfpathmoveto{\pgfqpoint{7.663127in}{3.018636in}}%
\pgfpathcurveto{\pgfqpoint{7.668171in}{3.018636in}}{\pgfqpoint{7.673009in}{3.020639in}}{\pgfqpoint{7.676575in}{3.024206in}}%
\pgfpathcurveto{\pgfqpoint{7.680142in}{3.027772in}}{\pgfqpoint{7.682145in}{3.032610in}}{\pgfqpoint{7.682145in}{3.037654in}}%
\pgfpathcurveto{\pgfqpoint{7.682145in}{3.042697in}}{\pgfqpoint{7.680142in}{3.047535in}}{\pgfqpoint{7.676575in}{3.051102in}}%
\pgfpathcurveto{\pgfqpoint{7.673009in}{3.054668in}}{\pgfqpoint{7.668171in}{3.056672in}}{\pgfqpoint{7.663127in}{3.056672in}}%
\pgfpathcurveto{\pgfqpoint{7.658084in}{3.056672in}}{\pgfqpoint{7.653246in}{3.054668in}}{\pgfqpoint{7.649679in}{3.051102in}}%
\pgfpathcurveto{\pgfqpoint{7.646113in}{3.047535in}}{\pgfqpoint{7.644109in}{3.042697in}}{\pgfqpoint{7.644109in}{3.037654in}}%
\pgfpathcurveto{\pgfqpoint{7.644109in}{3.032610in}}{\pgfqpoint{7.646113in}{3.027772in}}{\pgfqpoint{7.649679in}{3.024206in}}%
\pgfpathcurveto{\pgfqpoint{7.653246in}{3.020639in}}{\pgfqpoint{7.658084in}{3.018636in}}{\pgfqpoint{7.663127in}{3.018636in}}%
\pgfpathclose%
\pgfusepath{fill}%
\end{pgfscope}%
\begin{pgfscope}%
\pgfpathrectangle{\pgfqpoint{6.572727in}{0.474100in}}{\pgfqpoint{4.227273in}{3.318700in}}%
\pgfusepath{clip}%
\pgfsetbuttcap%
\pgfsetroundjoin%
\definecolor{currentfill}{rgb}{0.127568,0.566949,0.550556}%
\pgfsetfillcolor{currentfill}%
\pgfsetfillopacity{0.700000}%
\pgfsetlinewidth{0.000000pt}%
\definecolor{currentstroke}{rgb}{0.000000,0.000000,0.000000}%
\pgfsetstrokecolor{currentstroke}%
\pgfsetstrokeopacity{0.700000}%
\pgfsetdash{}{0pt}%
\pgfpathmoveto{\pgfqpoint{7.816652in}{3.116032in}}%
\pgfpathcurveto{\pgfqpoint{7.821696in}{3.116032in}}{\pgfqpoint{7.826533in}{3.118036in}}{\pgfqpoint{7.830100in}{3.121603in}}%
\pgfpathcurveto{\pgfqpoint{7.833666in}{3.125169in}}{\pgfqpoint{7.835670in}{3.130007in}}{\pgfqpoint{7.835670in}{3.135050in}}%
\pgfpathcurveto{\pgfqpoint{7.835670in}{3.140094in}}{\pgfqpoint{7.833666in}{3.144932in}}{\pgfqpoint{7.830100in}{3.148498in}}%
\pgfpathcurveto{\pgfqpoint{7.826533in}{3.152065in}}{\pgfqpoint{7.821696in}{3.154069in}}{\pgfqpoint{7.816652in}{3.154069in}}%
\pgfpathcurveto{\pgfqpoint{7.811608in}{3.154069in}}{\pgfqpoint{7.806771in}{3.152065in}}{\pgfqpoint{7.803204in}{3.148498in}}%
\pgfpathcurveto{\pgfqpoint{7.799638in}{3.144932in}}{\pgfqpoint{7.797634in}{3.140094in}}{\pgfqpoint{7.797634in}{3.135050in}}%
\pgfpathcurveto{\pgfqpoint{7.797634in}{3.130007in}}{\pgfqpoint{7.799638in}{3.125169in}}{\pgfqpoint{7.803204in}{3.121603in}}%
\pgfpathcurveto{\pgfqpoint{7.806771in}{3.118036in}}{\pgfqpoint{7.811608in}{3.116032in}}{\pgfqpoint{7.816652in}{3.116032in}}%
\pgfpathclose%
\pgfusepath{fill}%
\end{pgfscope}%
\begin{pgfscope}%
\pgfpathrectangle{\pgfqpoint{6.572727in}{0.474100in}}{\pgfqpoint{4.227273in}{3.318700in}}%
\pgfusepath{clip}%
\pgfsetbuttcap%
\pgfsetroundjoin%
\definecolor{currentfill}{rgb}{0.993248,0.906157,0.143936}%
\pgfsetfillcolor{currentfill}%
\pgfsetfillopacity{0.700000}%
\pgfsetlinewidth{0.000000pt}%
\definecolor{currentstroke}{rgb}{0.000000,0.000000,0.000000}%
\pgfsetstrokecolor{currentstroke}%
\pgfsetstrokeopacity{0.700000}%
\pgfsetdash{}{0pt}%
\pgfpathmoveto{\pgfqpoint{9.033129in}{2.045610in}}%
\pgfpathcurveto{\pgfqpoint{9.038172in}{2.045610in}}{\pgfqpoint{9.043010in}{2.047614in}}{\pgfqpoint{9.046577in}{2.051180in}}%
\pgfpathcurveto{\pgfqpoint{9.050143in}{2.054746in}}{\pgfqpoint{9.052147in}{2.059584in}}{\pgfqpoint{9.052147in}{2.064628in}}%
\pgfpathcurveto{\pgfqpoint{9.052147in}{2.069672in}}{\pgfqpoint{9.050143in}{2.074509in}}{\pgfqpoint{9.046577in}{2.078076in}}%
\pgfpathcurveto{\pgfqpoint{9.043010in}{2.081642in}}{\pgfqpoint{9.038172in}{2.083646in}}{\pgfqpoint{9.033129in}{2.083646in}}%
\pgfpathcurveto{\pgfqpoint{9.028085in}{2.083646in}}{\pgfqpoint{9.023247in}{2.081642in}}{\pgfqpoint{9.019681in}{2.078076in}}%
\pgfpathcurveto{\pgfqpoint{9.016114in}{2.074509in}}{\pgfqpoint{9.014111in}{2.069672in}}{\pgfqpoint{9.014111in}{2.064628in}}%
\pgfpathcurveto{\pgfqpoint{9.014111in}{2.059584in}}{\pgfqpoint{9.016114in}{2.054746in}}{\pgfqpoint{9.019681in}{2.051180in}}%
\pgfpathcurveto{\pgfqpoint{9.023247in}{2.047614in}}{\pgfqpoint{9.028085in}{2.045610in}}{\pgfqpoint{9.033129in}{2.045610in}}%
\pgfpathclose%
\pgfusepath{fill}%
\end{pgfscope}%
\begin{pgfscope}%
\pgfpathrectangle{\pgfqpoint{6.572727in}{0.474100in}}{\pgfqpoint{4.227273in}{3.318700in}}%
\pgfusepath{clip}%
\pgfsetbuttcap%
\pgfsetroundjoin%
\definecolor{currentfill}{rgb}{0.127568,0.566949,0.550556}%
\pgfsetfillcolor{currentfill}%
\pgfsetfillopacity{0.700000}%
\pgfsetlinewidth{0.000000pt}%
\definecolor{currentstroke}{rgb}{0.000000,0.000000,0.000000}%
\pgfsetstrokecolor{currentstroke}%
\pgfsetstrokeopacity{0.700000}%
\pgfsetdash{}{0pt}%
\pgfpathmoveto{\pgfqpoint{7.749487in}{2.558574in}}%
\pgfpathcurveto{\pgfqpoint{7.754531in}{2.558574in}}{\pgfqpoint{7.759369in}{2.560578in}}{\pgfqpoint{7.762935in}{2.564144in}}%
\pgfpathcurveto{\pgfqpoint{7.766502in}{2.567710in}}{\pgfqpoint{7.768506in}{2.572548in}}{\pgfqpoint{7.768506in}{2.577592in}}%
\pgfpathcurveto{\pgfqpoint{7.768506in}{2.582635in}}{\pgfqpoint{7.766502in}{2.587473in}}{\pgfqpoint{7.762935in}{2.591040in}}%
\pgfpathcurveto{\pgfqpoint{7.759369in}{2.594606in}}{\pgfqpoint{7.754531in}{2.596610in}}{\pgfqpoint{7.749487in}{2.596610in}}%
\pgfpathcurveto{\pgfqpoint{7.744444in}{2.596610in}}{\pgfqpoint{7.739606in}{2.594606in}}{\pgfqpoint{7.736040in}{2.591040in}}%
\pgfpathcurveto{\pgfqpoint{7.732473in}{2.587473in}}{\pgfqpoint{7.730469in}{2.582635in}}{\pgfqpoint{7.730469in}{2.577592in}}%
\pgfpathcurveto{\pgfqpoint{7.730469in}{2.572548in}}{\pgfqpoint{7.732473in}{2.567710in}}{\pgfqpoint{7.736040in}{2.564144in}}%
\pgfpathcurveto{\pgfqpoint{7.739606in}{2.560578in}}{\pgfqpoint{7.744444in}{2.558574in}}{\pgfqpoint{7.749487in}{2.558574in}}%
\pgfpathclose%
\pgfusepath{fill}%
\end{pgfscope}%
\begin{pgfscope}%
\pgfpathrectangle{\pgfqpoint{6.572727in}{0.474100in}}{\pgfqpoint{4.227273in}{3.318700in}}%
\pgfusepath{clip}%
\pgfsetbuttcap%
\pgfsetroundjoin%
\definecolor{currentfill}{rgb}{0.993248,0.906157,0.143936}%
\pgfsetfillcolor{currentfill}%
\pgfsetfillopacity{0.700000}%
\pgfsetlinewidth{0.000000pt}%
\definecolor{currentstroke}{rgb}{0.000000,0.000000,0.000000}%
\pgfsetstrokecolor{currentstroke}%
\pgfsetstrokeopacity{0.700000}%
\pgfsetdash{}{0pt}%
\pgfpathmoveto{\pgfqpoint{9.647676in}{1.401187in}}%
\pgfpathcurveto{\pgfqpoint{9.652719in}{1.401187in}}{\pgfqpoint{9.657557in}{1.403191in}}{\pgfqpoint{9.661123in}{1.406757in}}%
\pgfpathcurveto{\pgfqpoint{9.664690in}{1.410323in}}{\pgfqpoint{9.666694in}{1.415161in}}{\pgfqpoint{9.666694in}{1.420205in}}%
\pgfpathcurveto{\pgfqpoint{9.666694in}{1.425249in}}{\pgfqpoint{9.664690in}{1.430086in}}{\pgfqpoint{9.661123in}{1.433653in}}%
\pgfpathcurveto{\pgfqpoint{9.657557in}{1.437219in}}{\pgfqpoint{9.652719in}{1.439223in}}{\pgfqpoint{9.647676in}{1.439223in}}%
\pgfpathcurveto{\pgfqpoint{9.642632in}{1.439223in}}{\pgfqpoint{9.637794in}{1.437219in}}{\pgfqpoint{9.634228in}{1.433653in}}%
\pgfpathcurveto{\pgfqpoint{9.630661in}{1.430086in}}{\pgfqpoint{9.628657in}{1.425249in}}{\pgfqpoint{9.628657in}{1.420205in}}%
\pgfpathcurveto{\pgfqpoint{9.628657in}{1.415161in}}{\pgfqpoint{9.630661in}{1.410323in}}{\pgfqpoint{9.634228in}{1.406757in}}%
\pgfpathcurveto{\pgfqpoint{9.637794in}{1.403191in}}{\pgfqpoint{9.642632in}{1.401187in}}{\pgfqpoint{9.647676in}{1.401187in}}%
\pgfpathclose%
\pgfusepath{fill}%
\end{pgfscope}%
\begin{pgfscope}%
\pgfpathrectangle{\pgfqpoint{6.572727in}{0.474100in}}{\pgfqpoint{4.227273in}{3.318700in}}%
\pgfusepath{clip}%
\pgfsetbuttcap%
\pgfsetroundjoin%
\definecolor{currentfill}{rgb}{0.127568,0.566949,0.550556}%
\pgfsetfillcolor{currentfill}%
\pgfsetfillopacity{0.700000}%
\pgfsetlinewidth{0.000000pt}%
\definecolor{currentstroke}{rgb}{0.000000,0.000000,0.000000}%
\pgfsetstrokecolor{currentstroke}%
\pgfsetstrokeopacity{0.700000}%
\pgfsetdash{}{0pt}%
\pgfpathmoveto{\pgfqpoint{8.130323in}{2.832789in}}%
\pgfpathcurveto{\pgfqpoint{8.135367in}{2.832789in}}{\pgfqpoint{8.140205in}{2.834793in}}{\pgfqpoint{8.143771in}{2.838359in}}%
\pgfpathcurveto{\pgfqpoint{8.147338in}{2.841926in}}{\pgfqpoint{8.149342in}{2.846763in}}{\pgfqpoint{8.149342in}{2.851807in}}%
\pgfpathcurveto{\pgfqpoint{8.149342in}{2.856851in}}{\pgfqpoint{8.147338in}{2.861689in}}{\pgfqpoint{8.143771in}{2.865255in}}%
\pgfpathcurveto{\pgfqpoint{8.140205in}{2.868821in}}{\pgfqpoint{8.135367in}{2.870825in}}{\pgfqpoint{8.130323in}{2.870825in}}%
\pgfpathcurveto{\pgfqpoint{8.125280in}{2.870825in}}{\pgfqpoint{8.120442in}{2.868821in}}{\pgfqpoint{8.116876in}{2.865255in}}%
\pgfpathcurveto{\pgfqpoint{8.113309in}{2.861689in}}{\pgfqpoint{8.111305in}{2.856851in}}{\pgfqpoint{8.111305in}{2.851807in}}%
\pgfpathcurveto{\pgfqpoint{8.111305in}{2.846763in}}{\pgfqpoint{8.113309in}{2.841926in}}{\pgfqpoint{8.116876in}{2.838359in}}%
\pgfpathcurveto{\pgfqpoint{8.120442in}{2.834793in}}{\pgfqpoint{8.125280in}{2.832789in}}{\pgfqpoint{8.130323in}{2.832789in}}%
\pgfpathclose%
\pgfusepath{fill}%
\end{pgfscope}%
\begin{pgfscope}%
\pgfpathrectangle{\pgfqpoint{6.572727in}{0.474100in}}{\pgfqpoint{4.227273in}{3.318700in}}%
\pgfusepath{clip}%
\pgfsetbuttcap%
\pgfsetroundjoin%
\definecolor{currentfill}{rgb}{0.127568,0.566949,0.550556}%
\pgfsetfillcolor{currentfill}%
\pgfsetfillopacity{0.700000}%
\pgfsetlinewidth{0.000000pt}%
\definecolor{currentstroke}{rgb}{0.000000,0.000000,0.000000}%
\pgfsetstrokecolor{currentstroke}%
\pgfsetstrokeopacity{0.700000}%
\pgfsetdash{}{0pt}%
\pgfpathmoveto{\pgfqpoint{8.963549in}{2.944766in}}%
\pgfpathcurveto{\pgfqpoint{8.968592in}{2.944766in}}{\pgfqpoint{8.973430in}{2.946770in}}{\pgfqpoint{8.976996in}{2.950337in}}%
\pgfpathcurveto{\pgfqpoint{8.980563in}{2.953903in}}{\pgfqpoint{8.982567in}{2.958741in}}{\pgfqpoint{8.982567in}{2.963784in}}%
\pgfpathcurveto{\pgfqpoint{8.982567in}{2.968828in}}{\pgfqpoint{8.980563in}{2.973666in}}{\pgfqpoint{8.976996in}{2.977232in}}%
\pgfpathcurveto{\pgfqpoint{8.973430in}{2.980799in}}{\pgfqpoint{8.968592in}{2.982803in}}{\pgfqpoint{8.963549in}{2.982803in}}%
\pgfpathcurveto{\pgfqpoint{8.958505in}{2.982803in}}{\pgfqpoint{8.953667in}{2.980799in}}{\pgfqpoint{8.950101in}{2.977232in}}%
\pgfpathcurveto{\pgfqpoint{8.946534in}{2.973666in}}{\pgfqpoint{8.944530in}{2.968828in}}{\pgfqpoint{8.944530in}{2.963784in}}%
\pgfpathcurveto{\pgfqpoint{8.944530in}{2.958741in}}{\pgfqpoint{8.946534in}{2.953903in}}{\pgfqpoint{8.950101in}{2.950337in}}%
\pgfpathcurveto{\pgfqpoint{8.953667in}{2.946770in}}{\pgfqpoint{8.958505in}{2.944766in}}{\pgfqpoint{8.963549in}{2.944766in}}%
\pgfpathclose%
\pgfusepath{fill}%
\end{pgfscope}%
\begin{pgfscope}%
\pgfpathrectangle{\pgfqpoint{6.572727in}{0.474100in}}{\pgfqpoint{4.227273in}{3.318700in}}%
\pgfusepath{clip}%
\pgfsetbuttcap%
\pgfsetroundjoin%
\definecolor{currentfill}{rgb}{0.127568,0.566949,0.550556}%
\pgfsetfillcolor{currentfill}%
\pgfsetfillopacity{0.700000}%
\pgfsetlinewidth{0.000000pt}%
\definecolor{currentstroke}{rgb}{0.000000,0.000000,0.000000}%
\pgfsetstrokecolor{currentstroke}%
\pgfsetstrokeopacity{0.700000}%
\pgfsetdash{}{0pt}%
\pgfpathmoveto{\pgfqpoint{8.296973in}{2.706873in}}%
\pgfpathcurveto{\pgfqpoint{8.302017in}{2.706873in}}{\pgfqpoint{8.306855in}{2.708877in}}{\pgfqpoint{8.310421in}{2.712444in}}%
\pgfpathcurveto{\pgfqpoint{8.313987in}{2.716010in}}{\pgfqpoint{8.315991in}{2.720848in}}{\pgfqpoint{8.315991in}{2.725891in}}%
\pgfpathcurveto{\pgfqpoint{8.315991in}{2.730935in}}{\pgfqpoint{8.313987in}{2.735773in}}{\pgfqpoint{8.310421in}{2.739339in}}%
\pgfpathcurveto{\pgfqpoint{8.306855in}{2.742906in}}{\pgfqpoint{8.302017in}{2.744910in}}{\pgfqpoint{8.296973in}{2.744910in}}%
\pgfpathcurveto{\pgfqpoint{8.291929in}{2.744910in}}{\pgfqpoint{8.287092in}{2.742906in}}{\pgfqpoint{8.283525in}{2.739339in}}%
\pgfpathcurveto{\pgfqpoint{8.279959in}{2.735773in}}{\pgfqpoint{8.277955in}{2.730935in}}{\pgfqpoint{8.277955in}{2.725891in}}%
\pgfpathcurveto{\pgfqpoint{8.277955in}{2.720848in}}{\pgfqpoint{8.279959in}{2.716010in}}{\pgfqpoint{8.283525in}{2.712444in}}%
\pgfpathcurveto{\pgfqpoint{8.287092in}{2.708877in}}{\pgfqpoint{8.291929in}{2.706873in}}{\pgfqpoint{8.296973in}{2.706873in}}%
\pgfpathclose%
\pgfusepath{fill}%
\end{pgfscope}%
\begin{pgfscope}%
\pgfpathrectangle{\pgfqpoint{6.572727in}{0.474100in}}{\pgfqpoint{4.227273in}{3.318700in}}%
\pgfusepath{clip}%
\pgfsetbuttcap%
\pgfsetroundjoin%
\definecolor{currentfill}{rgb}{0.993248,0.906157,0.143936}%
\pgfsetfillcolor{currentfill}%
\pgfsetfillopacity{0.700000}%
\pgfsetlinewidth{0.000000pt}%
\definecolor{currentstroke}{rgb}{0.000000,0.000000,0.000000}%
\pgfsetstrokecolor{currentstroke}%
\pgfsetstrokeopacity{0.700000}%
\pgfsetdash{}{0pt}%
\pgfpathmoveto{\pgfqpoint{10.308772in}{1.810458in}}%
\pgfpathcurveto{\pgfqpoint{10.313816in}{1.810458in}}{\pgfqpoint{10.318653in}{1.812462in}}{\pgfqpoint{10.322220in}{1.816028in}}%
\pgfpathcurveto{\pgfqpoint{10.325786in}{1.819595in}}{\pgfqpoint{10.327790in}{1.824433in}}{\pgfqpoint{10.327790in}{1.829476in}}%
\pgfpathcurveto{\pgfqpoint{10.327790in}{1.834520in}}{\pgfqpoint{10.325786in}{1.839358in}}{\pgfqpoint{10.322220in}{1.842924in}}%
\pgfpathcurveto{\pgfqpoint{10.318653in}{1.846491in}}{\pgfqpoint{10.313816in}{1.848494in}}{\pgfqpoint{10.308772in}{1.848494in}}%
\pgfpathcurveto{\pgfqpoint{10.303728in}{1.848494in}}{\pgfqpoint{10.298891in}{1.846491in}}{\pgfqpoint{10.295324in}{1.842924in}}%
\pgfpathcurveto{\pgfqpoint{10.291758in}{1.839358in}}{\pgfqpoint{10.289754in}{1.834520in}}{\pgfqpoint{10.289754in}{1.829476in}}%
\pgfpathcurveto{\pgfqpoint{10.289754in}{1.824433in}}{\pgfqpoint{10.291758in}{1.819595in}}{\pgfqpoint{10.295324in}{1.816028in}}%
\pgfpathcurveto{\pgfqpoint{10.298891in}{1.812462in}}{\pgfqpoint{10.303728in}{1.810458in}}{\pgfqpoint{10.308772in}{1.810458in}}%
\pgfpathclose%
\pgfusepath{fill}%
\end{pgfscope}%
\begin{pgfscope}%
\pgfpathrectangle{\pgfqpoint{6.572727in}{0.474100in}}{\pgfqpoint{4.227273in}{3.318700in}}%
\pgfusepath{clip}%
\pgfsetbuttcap%
\pgfsetroundjoin%
\definecolor{currentfill}{rgb}{0.993248,0.906157,0.143936}%
\pgfsetfillcolor{currentfill}%
\pgfsetfillopacity{0.700000}%
\pgfsetlinewidth{0.000000pt}%
\definecolor{currentstroke}{rgb}{0.000000,0.000000,0.000000}%
\pgfsetstrokecolor{currentstroke}%
\pgfsetstrokeopacity{0.700000}%
\pgfsetdash{}{0pt}%
\pgfpathmoveto{\pgfqpoint{9.307642in}{1.601876in}}%
\pgfpathcurveto{\pgfqpoint{9.312686in}{1.601876in}}{\pgfqpoint{9.317523in}{1.603880in}}{\pgfqpoint{9.321090in}{1.607446in}}%
\pgfpathcurveto{\pgfqpoint{9.324656in}{1.611013in}}{\pgfqpoint{9.326660in}{1.615850in}}{\pgfqpoint{9.326660in}{1.620894in}}%
\pgfpathcurveto{\pgfqpoint{9.326660in}{1.625938in}}{\pgfqpoint{9.324656in}{1.630776in}}{\pgfqpoint{9.321090in}{1.634342in}}%
\pgfpathcurveto{\pgfqpoint{9.317523in}{1.637908in}}{\pgfqpoint{9.312686in}{1.639912in}}{\pgfqpoint{9.307642in}{1.639912in}}%
\pgfpathcurveto{\pgfqpoint{9.302598in}{1.639912in}}{\pgfqpoint{9.297761in}{1.637908in}}{\pgfqpoint{9.294194in}{1.634342in}}%
\pgfpathcurveto{\pgfqpoint{9.290628in}{1.630776in}}{\pgfqpoint{9.288624in}{1.625938in}}{\pgfqpoint{9.288624in}{1.620894in}}%
\pgfpathcurveto{\pgfqpoint{9.288624in}{1.615850in}}{\pgfqpoint{9.290628in}{1.611013in}}{\pgfqpoint{9.294194in}{1.607446in}}%
\pgfpathcurveto{\pgfqpoint{9.297761in}{1.603880in}}{\pgfqpoint{9.302598in}{1.601876in}}{\pgfqpoint{9.307642in}{1.601876in}}%
\pgfpathclose%
\pgfusepath{fill}%
\end{pgfscope}%
\begin{pgfscope}%
\pgfpathrectangle{\pgfqpoint{6.572727in}{0.474100in}}{\pgfqpoint{4.227273in}{3.318700in}}%
\pgfusepath{clip}%
\pgfsetbuttcap%
\pgfsetroundjoin%
\definecolor{currentfill}{rgb}{0.127568,0.566949,0.550556}%
\pgfsetfillcolor{currentfill}%
\pgfsetfillopacity{0.700000}%
\pgfsetlinewidth{0.000000pt}%
\definecolor{currentstroke}{rgb}{0.000000,0.000000,0.000000}%
\pgfsetstrokecolor{currentstroke}%
\pgfsetstrokeopacity{0.700000}%
\pgfsetdash{}{0pt}%
\pgfpathmoveto{\pgfqpoint{7.812880in}{1.195266in}}%
\pgfpathcurveto{\pgfqpoint{7.817924in}{1.195266in}}{\pgfqpoint{7.822762in}{1.197270in}}{\pgfqpoint{7.826328in}{1.200836in}}%
\pgfpathcurveto{\pgfqpoint{7.829895in}{1.204403in}}{\pgfqpoint{7.831899in}{1.209240in}}{\pgfqpoint{7.831899in}{1.214284in}}%
\pgfpathcurveto{\pgfqpoint{7.831899in}{1.219328in}}{\pgfqpoint{7.829895in}{1.224165in}}{\pgfqpoint{7.826328in}{1.227732in}}%
\pgfpathcurveto{\pgfqpoint{7.822762in}{1.231298in}}{\pgfqpoint{7.817924in}{1.233302in}}{\pgfqpoint{7.812880in}{1.233302in}}%
\pgfpathcurveto{\pgfqpoint{7.807837in}{1.233302in}}{\pgfqpoint{7.802999in}{1.231298in}}{\pgfqpoint{7.799433in}{1.227732in}}%
\pgfpathcurveto{\pgfqpoint{7.795866in}{1.224165in}}{\pgfqpoint{7.793862in}{1.219328in}}{\pgfqpoint{7.793862in}{1.214284in}}%
\pgfpathcurveto{\pgfqpoint{7.793862in}{1.209240in}}{\pgfqpoint{7.795866in}{1.204403in}}{\pgfqpoint{7.799433in}{1.200836in}}%
\pgfpathcurveto{\pgfqpoint{7.802999in}{1.197270in}}{\pgfqpoint{7.807837in}{1.195266in}}{\pgfqpoint{7.812880in}{1.195266in}}%
\pgfpathclose%
\pgfusepath{fill}%
\end{pgfscope}%
\begin{pgfscope}%
\pgfpathrectangle{\pgfqpoint{6.572727in}{0.474100in}}{\pgfqpoint{4.227273in}{3.318700in}}%
\pgfusepath{clip}%
\pgfsetbuttcap%
\pgfsetroundjoin%
\definecolor{currentfill}{rgb}{0.267004,0.004874,0.329415}%
\pgfsetfillcolor{currentfill}%
\pgfsetfillopacity{0.700000}%
\pgfsetlinewidth{0.000000pt}%
\definecolor{currentstroke}{rgb}{0.000000,0.000000,0.000000}%
\pgfsetstrokecolor{currentstroke}%
\pgfsetstrokeopacity{0.700000}%
\pgfsetdash{}{0pt}%
\pgfpathmoveto{\pgfqpoint{8.647794in}{1.127508in}}%
\pgfpathcurveto{\pgfqpoint{8.652838in}{1.127508in}}{\pgfqpoint{8.657676in}{1.129512in}}{\pgfqpoint{8.661242in}{1.133078in}}%
\pgfpathcurveto{\pgfqpoint{8.664809in}{1.136644in}}{\pgfqpoint{8.666813in}{1.141482in}}{\pgfqpoint{8.666813in}{1.146526in}}%
\pgfpathcurveto{\pgfqpoint{8.666813in}{1.151569in}}{\pgfqpoint{8.664809in}{1.156407in}}{\pgfqpoint{8.661242in}{1.159974in}}%
\pgfpathcurveto{\pgfqpoint{8.657676in}{1.163540in}}{\pgfqpoint{8.652838in}{1.165544in}}{\pgfqpoint{8.647794in}{1.165544in}}%
\pgfpathcurveto{\pgfqpoint{8.642751in}{1.165544in}}{\pgfqpoint{8.637913in}{1.163540in}}{\pgfqpoint{8.634347in}{1.159974in}}%
\pgfpathcurveto{\pgfqpoint{8.630780in}{1.156407in}}{\pgfqpoint{8.628776in}{1.151569in}}{\pgfqpoint{8.628776in}{1.146526in}}%
\pgfpathcurveto{\pgfqpoint{8.628776in}{1.141482in}}{\pgfqpoint{8.630780in}{1.136644in}}{\pgfqpoint{8.634347in}{1.133078in}}%
\pgfpathcurveto{\pgfqpoint{8.637913in}{1.129512in}}{\pgfqpoint{8.642751in}{1.127508in}}{\pgfqpoint{8.647794in}{1.127508in}}%
\pgfpathclose%
\pgfusepath{fill}%
\end{pgfscope}%
\begin{pgfscope}%
\pgfpathrectangle{\pgfqpoint{6.572727in}{0.474100in}}{\pgfqpoint{4.227273in}{3.318700in}}%
\pgfusepath{clip}%
\pgfsetbuttcap%
\pgfsetroundjoin%
\definecolor{currentfill}{rgb}{0.127568,0.566949,0.550556}%
\pgfsetfillcolor{currentfill}%
\pgfsetfillopacity{0.700000}%
\pgfsetlinewidth{0.000000pt}%
\definecolor{currentstroke}{rgb}{0.000000,0.000000,0.000000}%
\pgfsetstrokecolor{currentstroke}%
\pgfsetstrokeopacity{0.700000}%
\pgfsetdash{}{0pt}%
\pgfpathmoveto{\pgfqpoint{8.008567in}{1.204967in}}%
\pgfpathcurveto{\pgfqpoint{8.013611in}{1.204967in}}{\pgfqpoint{8.018449in}{1.206971in}}{\pgfqpoint{8.022015in}{1.210537in}}%
\pgfpathcurveto{\pgfqpoint{8.025582in}{1.214103in}}{\pgfqpoint{8.027586in}{1.218941in}}{\pgfqpoint{8.027586in}{1.223985in}}%
\pgfpathcurveto{\pgfqpoint{8.027586in}{1.229028in}}{\pgfqpoint{8.025582in}{1.233866in}}{\pgfqpoint{8.022015in}{1.237433in}}%
\pgfpathcurveto{\pgfqpoint{8.018449in}{1.240999in}}{\pgfqpoint{8.013611in}{1.243003in}}{\pgfqpoint{8.008567in}{1.243003in}}%
\pgfpathcurveto{\pgfqpoint{8.003524in}{1.243003in}}{\pgfqpoint{7.998686in}{1.240999in}}{\pgfqpoint{7.995120in}{1.237433in}}%
\pgfpathcurveto{\pgfqpoint{7.991553in}{1.233866in}}{\pgfqpoint{7.989549in}{1.229028in}}{\pgfqpoint{7.989549in}{1.223985in}}%
\pgfpathcurveto{\pgfqpoint{7.989549in}{1.218941in}}{\pgfqpoint{7.991553in}{1.214103in}}{\pgfqpoint{7.995120in}{1.210537in}}%
\pgfpathcurveto{\pgfqpoint{7.998686in}{1.206971in}}{\pgfqpoint{8.003524in}{1.204967in}}{\pgfqpoint{8.008567in}{1.204967in}}%
\pgfpathclose%
\pgfusepath{fill}%
\end{pgfscope}%
\begin{pgfscope}%
\pgfpathrectangle{\pgfqpoint{6.572727in}{0.474100in}}{\pgfqpoint{4.227273in}{3.318700in}}%
\pgfusepath{clip}%
\pgfsetbuttcap%
\pgfsetroundjoin%
\definecolor{currentfill}{rgb}{0.127568,0.566949,0.550556}%
\pgfsetfillcolor{currentfill}%
\pgfsetfillopacity{0.700000}%
\pgfsetlinewidth{0.000000pt}%
\definecolor{currentstroke}{rgb}{0.000000,0.000000,0.000000}%
\pgfsetstrokecolor{currentstroke}%
\pgfsetstrokeopacity{0.700000}%
\pgfsetdash{}{0pt}%
\pgfpathmoveto{\pgfqpoint{8.385035in}{1.683129in}}%
\pgfpathcurveto{\pgfqpoint{8.390079in}{1.683129in}}{\pgfqpoint{8.394917in}{1.685133in}}{\pgfqpoint{8.398483in}{1.688699in}}%
\pgfpathcurveto{\pgfqpoint{8.402049in}{1.692266in}}{\pgfqpoint{8.404053in}{1.697103in}}{\pgfqpoint{8.404053in}{1.702147in}}%
\pgfpathcurveto{\pgfqpoint{8.404053in}{1.707191in}}{\pgfqpoint{8.402049in}{1.712028in}}{\pgfqpoint{8.398483in}{1.715595in}}%
\pgfpathcurveto{\pgfqpoint{8.394917in}{1.719161in}}{\pgfqpoint{8.390079in}{1.721165in}}{\pgfqpoint{8.385035in}{1.721165in}}%
\pgfpathcurveto{\pgfqpoint{8.379991in}{1.721165in}}{\pgfqpoint{8.375154in}{1.719161in}}{\pgfqpoint{8.371587in}{1.715595in}}%
\pgfpathcurveto{\pgfqpoint{8.368021in}{1.712028in}}{\pgfqpoint{8.366017in}{1.707191in}}{\pgfqpoint{8.366017in}{1.702147in}}%
\pgfpathcurveto{\pgfqpoint{8.366017in}{1.697103in}}{\pgfqpoint{8.368021in}{1.692266in}}{\pgfqpoint{8.371587in}{1.688699in}}%
\pgfpathcurveto{\pgfqpoint{8.375154in}{1.685133in}}{\pgfqpoint{8.379991in}{1.683129in}}{\pgfqpoint{8.385035in}{1.683129in}}%
\pgfpathclose%
\pgfusepath{fill}%
\end{pgfscope}%
\begin{pgfscope}%
\pgfpathrectangle{\pgfqpoint{6.572727in}{0.474100in}}{\pgfqpoint{4.227273in}{3.318700in}}%
\pgfusepath{clip}%
\pgfsetbuttcap%
\pgfsetroundjoin%
\definecolor{currentfill}{rgb}{0.993248,0.906157,0.143936}%
\pgfsetfillcolor{currentfill}%
\pgfsetfillopacity{0.700000}%
\pgfsetlinewidth{0.000000pt}%
\definecolor{currentstroke}{rgb}{0.000000,0.000000,0.000000}%
\pgfsetstrokecolor{currentstroke}%
\pgfsetstrokeopacity{0.700000}%
\pgfsetdash{}{0pt}%
\pgfpathmoveto{\pgfqpoint{9.265493in}{2.007887in}}%
\pgfpathcurveto{\pgfqpoint{9.270537in}{2.007887in}}{\pgfqpoint{9.275375in}{2.009891in}}{\pgfqpoint{9.278941in}{2.013457in}}%
\pgfpathcurveto{\pgfqpoint{9.282507in}{2.017023in}}{\pgfqpoint{9.284511in}{2.021861in}}{\pgfqpoint{9.284511in}{2.026905in}}%
\pgfpathcurveto{\pgfqpoint{9.284511in}{2.031949in}}{\pgfqpoint{9.282507in}{2.036786in}}{\pgfqpoint{9.278941in}{2.040353in}}%
\pgfpathcurveto{\pgfqpoint{9.275375in}{2.043919in}}{\pgfqpoint{9.270537in}{2.045923in}}{\pgfqpoint{9.265493in}{2.045923in}}%
\pgfpathcurveto{\pgfqpoint{9.260449in}{2.045923in}}{\pgfqpoint{9.255612in}{2.043919in}}{\pgfqpoint{9.252045in}{2.040353in}}%
\pgfpathcurveto{\pgfqpoint{9.248479in}{2.036786in}}{\pgfqpoint{9.246475in}{2.031949in}}{\pgfqpoint{9.246475in}{2.026905in}}%
\pgfpathcurveto{\pgfqpoint{9.246475in}{2.021861in}}{\pgfqpoint{9.248479in}{2.017023in}}{\pgfqpoint{9.252045in}{2.013457in}}%
\pgfpathcurveto{\pgfqpoint{9.255612in}{2.009891in}}{\pgfqpoint{9.260449in}{2.007887in}}{\pgfqpoint{9.265493in}{2.007887in}}%
\pgfpathclose%
\pgfusepath{fill}%
\end{pgfscope}%
\begin{pgfscope}%
\pgfpathrectangle{\pgfqpoint{6.572727in}{0.474100in}}{\pgfqpoint{4.227273in}{3.318700in}}%
\pgfusepath{clip}%
\pgfsetbuttcap%
\pgfsetroundjoin%
\definecolor{currentfill}{rgb}{1.000000,1.000000,1.000000}%
\pgfsetfillcolor{currentfill}%
\pgfsetlinewidth{1.003750pt}%
\definecolor{currentstroke}{rgb}{0.000000,0.000000,0.000000}%
\pgfsetstrokecolor{currentstroke}%
\pgfsetdash{}{0pt}%
\pgfsys@defobject{currentmarker}{\pgfqpoint{-0.098209in}{-0.098209in}}{\pgfqpoint{0.098209in}{0.098209in}}{%
\pgfpathmoveto{\pgfqpoint{0.000000in}{-0.098209in}}%
\pgfpathcurveto{\pgfqpoint{0.026045in}{-0.098209in}}{\pgfqpoint{0.051028in}{-0.087861in}}{\pgfqpoint{0.069444in}{-0.069444in}}%
\pgfpathcurveto{\pgfqpoint{0.087861in}{-0.051028in}}{\pgfqpoint{0.098209in}{-0.026045in}}{\pgfqpoint{0.098209in}{0.000000in}}%
\pgfpathcurveto{\pgfqpoint{0.098209in}{0.026045in}}{\pgfqpoint{0.087861in}{0.051028in}}{\pgfqpoint{0.069444in}{0.069444in}}%
\pgfpathcurveto{\pgfqpoint{0.051028in}{0.087861in}}{\pgfqpoint{0.026045in}{0.098209in}}{\pgfqpoint{0.000000in}{0.098209in}}%
\pgfpathcurveto{\pgfqpoint{-0.026045in}{0.098209in}}{\pgfqpoint{-0.051028in}{0.087861in}}{\pgfqpoint{-0.069444in}{0.069444in}}%
\pgfpathcurveto{\pgfqpoint{-0.087861in}{0.051028in}}{\pgfqpoint{-0.098209in}{0.026045in}}{\pgfqpoint{-0.098209in}{0.000000in}}%
\pgfpathcurveto{\pgfqpoint{-0.098209in}{-0.026045in}}{\pgfqpoint{-0.087861in}{-0.051028in}}{\pgfqpoint{-0.069444in}{-0.069444in}}%
\pgfpathcurveto{\pgfqpoint{-0.051028in}{-0.087861in}}{\pgfqpoint{-0.026045in}{-0.098209in}}{\pgfqpoint{0.000000in}{-0.098209in}}%
\pgfpathclose%
\pgfusepath{stroke,fill}%
}%
\begin{pgfscope}%
\pgfsys@transformshift{8.011679in}{2.194312in}%
\pgfsys@useobject{currentmarker}{}%
\end{pgfscope}%
\begin{pgfscope}%
\pgfsys@transformshift{9.602347in}{1.598507in}%
\pgfsys@useobject{currentmarker}{}%
\end{pgfscope}%
\end{pgfscope}%
\begin{pgfscope}%
\pgfpathrectangle{\pgfqpoint{6.572727in}{0.474100in}}{\pgfqpoint{4.227273in}{3.318700in}}%
\pgfusepath{clip}%
\pgfsetbuttcap%
\pgfsetroundjoin%
\definecolor{currentfill}{rgb}{0.121569,0.466667,0.705882}%
\pgfsetfillcolor{currentfill}%
\pgfsetlinewidth{1.003750pt}%
\definecolor{currentstroke}{rgb}{0.000000,0.000000,0.000000}%
\pgfsetstrokecolor{currentstroke}%
\pgfsetdash{}{0pt}%
\pgfsys@defobject{currentmarker}{\pgfqpoint{-0.028432in}{-0.049105in}}{\pgfqpoint{0.036993in}{0.049105in}}{%
\pgfpathmoveto{\pgfqpoint{0.004270in}{0.038961in}}%
\pgfpathquadraticcurveto{\pgfqpoint{-0.005609in}{0.038961in}}{\pgfqpoint{-0.010600in}{0.029223in}}%
\pgfpathquadraticcurveto{\pgfqpoint{-0.015570in}{0.019506in}}{\pgfqpoint{-0.015570in}{-0.000030in}}%
\pgfpathquadraticcurveto{\pgfqpoint{-0.015570in}{-0.019486in}}{\pgfqpoint{-0.010600in}{-0.029223in}}%
\pgfpathquadraticcurveto{\pgfqpoint{-0.005609in}{-0.038961in}}{\pgfqpoint{0.004270in}{-0.038961in}}%
\pgfpathquadraticcurveto{\pgfqpoint{0.014231in}{-0.038961in}}{\pgfqpoint{0.019202in}{-0.029223in}}%
\pgfpathquadraticcurveto{\pgfqpoint{0.024192in}{-0.019486in}}{\pgfqpoint{0.024192in}{-0.000030in}}%
\pgfpathquadraticcurveto{\pgfqpoint{0.024192in}{0.019506in}}{\pgfqpoint{0.019202in}{0.029223in}}%
\pgfpathquadraticcurveto{\pgfqpoint{0.014231in}{0.038961in}}{\pgfqpoint{0.004270in}{0.038961in}}%
\pgfpathclose%
\pgfpathmoveto{\pgfqpoint{0.004270in}{0.049105in}}%
\pgfpathquadraticcurveto{\pgfqpoint{0.020196in}{0.049105in}}{\pgfqpoint{0.028594in}{0.036506in}}%
\pgfpathquadraticcurveto{\pgfqpoint{0.036993in}{0.023928in}}{\pgfqpoint{0.036993in}{-0.000030in}}%
\pgfpathquadraticcurveto{\pgfqpoint{0.036993in}{-0.023928in}}{\pgfqpoint{0.028594in}{-0.036527in}}%
\pgfpathquadraticcurveto{\pgfqpoint{0.020196in}{-0.049105in}}{\pgfqpoint{0.004270in}{-0.049105in}}%
\pgfpathquadraticcurveto{\pgfqpoint{-0.011635in}{-0.049105in}}{\pgfqpoint{-0.020033in}{-0.036527in}}%
\pgfpathquadraticcurveto{\pgfqpoint{-0.028432in}{-0.023928in}}{\pgfqpoint{-0.028432in}{-0.000030in}}%
\pgfpathquadraticcurveto{\pgfqpoint{-0.028432in}{0.023928in}}{\pgfqpoint{-0.020033in}{0.036506in}}%
\pgfpathquadraticcurveto{\pgfqpoint{-0.011635in}{0.049105in}}{\pgfqpoint{0.004270in}{0.049105in}}%
\pgfpathclose%
\pgfusepath{stroke,fill}%
}%
\begin{pgfscope}%
\pgfsys@transformshift{8.011679in}{2.194312in}%
\pgfsys@useobject{currentmarker}{}%
\end{pgfscope}%
\end{pgfscope}%
\begin{pgfscope}%
\pgfpathrectangle{\pgfqpoint{6.572727in}{0.474100in}}{\pgfqpoint{4.227273in}{3.318700in}}%
\pgfusepath{clip}%
\pgfsetbuttcap%
\pgfsetroundjoin%
\definecolor{currentfill}{rgb}{1.000000,0.498039,0.054902}%
\pgfsetfillcolor{currentfill}%
\pgfsetlinewidth{1.003750pt}%
\definecolor{currentstroke}{rgb}{0.000000,0.000000,0.000000}%
\pgfsetstrokecolor{currentstroke}%
\pgfsetdash{}{0pt}%
\pgfsys@defobject{currentmarker}{\pgfqpoint{-0.021837in}{-0.049105in}}{\pgfqpoint{0.036634in}{0.049105in}}{%
\pgfpathmoveto{\pgfqpoint{-0.019922in}{-0.037928in}}%
\pgfpathlineto{\pgfqpoint{0.001779in}{-0.037928in}}%
\pgfpathlineto{\pgfqpoint{0.001779in}{0.037002in}}%
\pgfpathlineto{\pgfqpoint{-0.021837in}{0.032266in}}%
\pgfpathlineto{\pgfqpoint{-0.021837in}{0.044369in}}%
\pgfpathlineto{\pgfqpoint{0.001652in}{0.049105in}}%
\pgfpathlineto{\pgfqpoint{0.014933in}{0.049105in}}%
\pgfpathlineto{\pgfqpoint{0.014933in}{-0.037928in}}%
\pgfpathlineto{\pgfqpoint{0.036634in}{-0.037928in}}%
\pgfpathlineto{\pgfqpoint{0.036634in}{-0.049105in}}%
\pgfpathlineto{\pgfqpoint{-0.019922in}{-0.049105in}}%
\pgfpathlineto{\pgfqpoint{-0.019922in}{-0.037928in}}%
\pgfpathclose%
\pgfusepath{stroke,fill}%
}%
\begin{pgfscope}%
\pgfsys@transformshift{9.602347in}{1.598507in}%
\pgfsys@useobject{currentmarker}{}%
\end{pgfscope}%
\end{pgfscope}%
\begin{pgfscope}%
\pgfsetbuttcap%
\pgfsetroundjoin%
\definecolor{currentfill}{rgb}{0.000000,0.000000,0.000000}%
\pgfsetfillcolor{currentfill}%
\pgfsetlinewidth{0.803000pt}%
\definecolor{currentstroke}{rgb}{0.000000,0.000000,0.000000}%
\pgfsetstrokecolor{currentstroke}%
\pgfsetdash{}{0pt}%
\pgfsys@defobject{currentmarker}{\pgfqpoint{0.000000in}{-0.048611in}}{\pgfqpoint{0.000000in}{0.000000in}}{%
\pgfpathmoveto{\pgfqpoint{0.000000in}{0.000000in}}%
\pgfpathlineto{\pgfqpoint{0.000000in}{-0.048611in}}%
\pgfusepath{stroke,fill}%
}%
\begin{pgfscope}%
\pgfsys@transformshift{8.796015in}{0.474100in}%
\pgfsys@useobject{currentmarker}{}%
\end{pgfscope}%
\end{pgfscope}%
\begin{pgfscope}%
\definecolor{textcolor}{rgb}{0.000000,0.000000,0.000000}%
\pgfsetstrokecolor{textcolor}%
\pgfsetfillcolor{textcolor}%
\pgftext[x=8.796015in,y=0.376878in,,top]{\color{textcolor}\sffamily\fontsize{10.000000}{12.000000}\selectfont 0.10}%
\end{pgfscope}%
\begin{pgfscope}%
\pgfsetbuttcap%
\pgfsetroundjoin%
\definecolor{currentfill}{rgb}{0.000000,0.000000,0.000000}%
\pgfsetfillcolor{currentfill}%
\pgfsetlinewidth{0.803000pt}%
\definecolor{currentstroke}{rgb}{0.000000,0.000000,0.000000}%
\pgfsetstrokecolor{currentstroke}%
\pgfsetdash{}{0pt}%
\pgfsys@defobject{currentmarker}{\pgfqpoint{0.000000in}{-0.048611in}}{\pgfqpoint{0.000000in}{0.000000in}}{%
\pgfpathmoveto{\pgfqpoint{0.000000in}{0.000000in}}%
\pgfpathlineto{\pgfqpoint{0.000000in}{-0.048611in}}%
\pgfusepath{stroke,fill}%
}%
\begin{pgfscope}%
\pgfsys@transformshift{8.805201in}{0.474100in}%
\pgfsys@useobject{currentmarker}{}%
\end{pgfscope}%
\end{pgfscope}%
\begin{pgfscope}%
\definecolor{textcolor}{rgb}{0.000000,0.000000,0.000000}%
\pgfsetstrokecolor{textcolor}%
\pgfsetfillcolor{textcolor}%
\pgftext[x=8.805201in,y=0.376878in,,top]{\color{textcolor}\sffamily\fontsize{10.000000}{12.000000}\selectfont 0.11}%
\end{pgfscope}%
\begin{pgfscope}%
\pgfsetbuttcap%
\pgfsetroundjoin%
\definecolor{currentfill}{rgb}{0.000000,0.000000,0.000000}%
\pgfsetfillcolor{currentfill}%
\pgfsetlinewidth{0.803000pt}%
\definecolor{currentstroke}{rgb}{0.000000,0.000000,0.000000}%
\pgfsetstrokecolor{currentstroke}%
\pgfsetdash{}{0pt}%
\pgfsys@defobject{currentmarker}{\pgfqpoint{0.000000in}{-0.048611in}}{\pgfqpoint{0.000000in}{0.000000in}}{%
\pgfpathmoveto{\pgfqpoint{0.000000in}{0.000000in}}%
\pgfpathlineto{\pgfqpoint{0.000000in}{-0.048611in}}%
\pgfusepath{stroke,fill}%
}%
\begin{pgfscope}%
\pgfsys@transformshift{8.814388in}{0.474100in}%
\pgfsys@useobject{currentmarker}{}%
\end{pgfscope}%
\end{pgfscope}%
\begin{pgfscope}%
\definecolor{textcolor}{rgb}{0.000000,0.000000,0.000000}%
\pgfsetstrokecolor{textcolor}%
\pgfsetfillcolor{textcolor}%
\pgftext[x=8.814388in,y=0.376878in,,top]{\color{textcolor}\sffamily\fontsize{10.000000}{12.000000}\selectfont 0.12}%
\end{pgfscope}%
\begin{pgfscope}%
\pgfsetbuttcap%
\pgfsetroundjoin%
\definecolor{currentfill}{rgb}{0.000000,0.000000,0.000000}%
\pgfsetfillcolor{currentfill}%
\pgfsetlinewidth{0.803000pt}%
\definecolor{currentstroke}{rgb}{0.000000,0.000000,0.000000}%
\pgfsetstrokecolor{currentstroke}%
\pgfsetdash{}{0pt}%
\pgfsys@defobject{currentmarker}{\pgfqpoint{0.000000in}{-0.048611in}}{\pgfqpoint{0.000000in}{0.000000in}}{%
\pgfpathmoveto{\pgfqpoint{0.000000in}{0.000000in}}%
\pgfpathlineto{\pgfqpoint{0.000000in}{-0.048611in}}%
\pgfusepath{stroke,fill}%
}%
\begin{pgfscope}%
\pgfsys@transformshift{8.823575in}{0.474100in}%
\pgfsys@useobject{currentmarker}{}%
\end{pgfscope}%
\end{pgfscope}%
\begin{pgfscope}%
\definecolor{textcolor}{rgb}{0.000000,0.000000,0.000000}%
\pgfsetstrokecolor{textcolor}%
\pgfsetfillcolor{textcolor}%
\pgftext[x=8.823575in,y=0.376878in,,top]{\color{textcolor}\sffamily\fontsize{10.000000}{12.000000}\selectfont 0.13}%
\end{pgfscope}%
\begin{pgfscope}%
\pgfsetbuttcap%
\pgfsetroundjoin%
\definecolor{currentfill}{rgb}{0.000000,0.000000,0.000000}%
\pgfsetfillcolor{currentfill}%
\pgfsetlinewidth{0.803000pt}%
\definecolor{currentstroke}{rgb}{0.000000,0.000000,0.000000}%
\pgfsetstrokecolor{currentstroke}%
\pgfsetdash{}{0pt}%
\pgfsys@defobject{currentmarker}{\pgfqpoint{0.000000in}{-0.048611in}}{\pgfqpoint{0.000000in}{0.000000in}}{%
\pgfpathmoveto{\pgfqpoint{0.000000in}{0.000000in}}%
\pgfpathlineto{\pgfqpoint{0.000000in}{-0.048611in}}%
\pgfusepath{stroke,fill}%
}%
\begin{pgfscope}%
\pgfsys@transformshift{8.832762in}{0.474100in}%
\pgfsys@useobject{currentmarker}{}%
\end{pgfscope}%
\end{pgfscope}%
\begin{pgfscope}%
\definecolor{textcolor}{rgb}{0.000000,0.000000,0.000000}%
\pgfsetstrokecolor{textcolor}%
\pgfsetfillcolor{textcolor}%
\pgftext[x=8.832762in,y=0.376878in,,top]{\color{textcolor}\sffamily\fontsize{10.000000}{12.000000}\selectfont 0.14}%
\end{pgfscope}%
\begin{pgfscope}%
\pgfsetbuttcap%
\pgfsetroundjoin%
\definecolor{currentfill}{rgb}{0.000000,0.000000,0.000000}%
\pgfsetfillcolor{currentfill}%
\pgfsetlinewidth{0.803000pt}%
\definecolor{currentstroke}{rgb}{0.000000,0.000000,0.000000}%
\pgfsetstrokecolor{currentstroke}%
\pgfsetdash{}{0pt}%
\pgfsys@defobject{currentmarker}{\pgfqpoint{0.000000in}{-0.048611in}}{\pgfqpoint{0.000000in}{0.000000in}}{%
\pgfpathmoveto{\pgfqpoint{0.000000in}{0.000000in}}%
\pgfpathlineto{\pgfqpoint{0.000000in}{-0.048611in}}%
\pgfusepath{stroke,fill}%
}%
\begin{pgfscope}%
\pgfsys@transformshift{8.841948in}{0.474100in}%
\pgfsys@useobject{currentmarker}{}%
\end{pgfscope}%
\end{pgfscope}%
\begin{pgfscope}%
\definecolor{textcolor}{rgb}{0.000000,0.000000,0.000000}%
\pgfsetstrokecolor{textcolor}%
\pgfsetfillcolor{textcolor}%
\pgftext[x=8.841948in,y=0.376878in,,top]{\color{textcolor}\sffamily\fontsize{10.000000}{12.000000}\selectfont 0.15}%
\end{pgfscope}%
\begin{pgfscope}%
\pgfsetbuttcap%
\pgfsetroundjoin%
\definecolor{currentfill}{rgb}{0.000000,0.000000,0.000000}%
\pgfsetfillcolor{currentfill}%
\pgfsetlinewidth{0.803000pt}%
\definecolor{currentstroke}{rgb}{0.000000,0.000000,0.000000}%
\pgfsetstrokecolor{currentstroke}%
\pgfsetdash{}{0pt}%
\pgfsys@defobject{currentmarker}{\pgfqpoint{0.000000in}{-0.048611in}}{\pgfqpoint{0.000000in}{0.000000in}}{%
\pgfpathmoveto{\pgfqpoint{0.000000in}{0.000000in}}%
\pgfpathlineto{\pgfqpoint{0.000000in}{-0.048611in}}%
\pgfusepath{stroke,fill}%
}%
\begin{pgfscope}%
\pgfsys@transformshift{8.851135in}{0.474100in}%
\pgfsys@useobject{currentmarker}{}%
\end{pgfscope}%
\end{pgfscope}%
\begin{pgfscope}%
\definecolor{textcolor}{rgb}{0.000000,0.000000,0.000000}%
\pgfsetstrokecolor{textcolor}%
\pgfsetfillcolor{textcolor}%
\pgftext[x=8.851135in,y=0.376878in,,top]{\color{textcolor}\sffamily\fontsize{10.000000}{12.000000}\selectfont 0.16}%
\end{pgfscope}%
\begin{pgfscope}%
\pgfsetbuttcap%
\pgfsetroundjoin%
\definecolor{currentfill}{rgb}{0.000000,0.000000,0.000000}%
\pgfsetfillcolor{currentfill}%
\pgfsetlinewidth{0.803000pt}%
\definecolor{currentstroke}{rgb}{0.000000,0.000000,0.000000}%
\pgfsetstrokecolor{currentstroke}%
\pgfsetdash{}{0pt}%
\pgfsys@defobject{currentmarker}{\pgfqpoint{0.000000in}{-0.048611in}}{\pgfqpoint{0.000000in}{0.000000in}}{%
\pgfpathmoveto{\pgfqpoint{0.000000in}{0.000000in}}%
\pgfpathlineto{\pgfqpoint{0.000000in}{-0.048611in}}%
\pgfusepath{stroke,fill}%
}%
\begin{pgfscope}%
\pgfsys@transformshift{8.860322in}{0.474100in}%
\pgfsys@useobject{currentmarker}{}%
\end{pgfscope}%
\end{pgfscope}%
\begin{pgfscope}%
\definecolor{textcolor}{rgb}{0.000000,0.000000,0.000000}%
\pgfsetstrokecolor{textcolor}%
\pgfsetfillcolor{textcolor}%
\pgftext[x=8.860322in,y=0.376878in,,top]{\color{textcolor}\sffamily\fontsize{10.000000}{12.000000}\selectfont 0.17}%
\end{pgfscope}%
\begin{pgfscope}%
\pgfsetbuttcap%
\pgfsetroundjoin%
\definecolor{currentfill}{rgb}{0.000000,0.000000,0.000000}%
\pgfsetfillcolor{currentfill}%
\pgfsetlinewidth{0.803000pt}%
\definecolor{currentstroke}{rgb}{0.000000,0.000000,0.000000}%
\pgfsetstrokecolor{currentstroke}%
\pgfsetdash{}{0pt}%
\pgfsys@defobject{currentmarker}{\pgfqpoint{0.000000in}{-0.048611in}}{\pgfqpoint{0.000000in}{0.000000in}}{%
\pgfpathmoveto{\pgfqpoint{0.000000in}{0.000000in}}%
\pgfpathlineto{\pgfqpoint{0.000000in}{-0.048611in}}%
\pgfusepath{stroke,fill}%
}%
\begin{pgfscope}%
\pgfsys@transformshift{8.869509in}{0.474100in}%
\pgfsys@useobject{currentmarker}{}%
\end{pgfscope}%
\end{pgfscope}%
\begin{pgfscope}%
\definecolor{textcolor}{rgb}{0.000000,0.000000,0.000000}%
\pgfsetstrokecolor{textcolor}%
\pgfsetfillcolor{textcolor}%
\pgftext[x=8.869509in,y=0.376878in,,top]{\color{textcolor}\sffamily\fontsize{10.000000}{12.000000}\selectfont 0.18}%
\end{pgfscope}%
\begin{pgfscope}%
\pgfsetbuttcap%
\pgfsetroundjoin%
\definecolor{currentfill}{rgb}{0.000000,0.000000,0.000000}%
\pgfsetfillcolor{currentfill}%
\pgfsetlinewidth{0.803000pt}%
\definecolor{currentstroke}{rgb}{0.000000,0.000000,0.000000}%
\pgfsetstrokecolor{currentstroke}%
\pgfsetdash{}{0pt}%
\pgfsys@defobject{currentmarker}{\pgfqpoint{0.000000in}{-0.048611in}}{\pgfqpoint{0.000000in}{0.000000in}}{%
\pgfpathmoveto{\pgfqpoint{0.000000in}{0.000000in}}%
\pgfpathlineto{\pgfqpoint{0.000000in}{-0.048611in}}%
\pgfusepath{stroke,fill}%
}%
\begin{pgfscope}%
\pgfsys@transformshift{8.878695in}{0.474100in}%
\pgfsys@useobject{currentmarker}{}%
\end{pgfscope}%
\end{pgfscope}%
\begin{pgfscope}%
\definecolor{textcolor}{rgb}{0.000000,0.000000,0.000000}%
\pgfsetstrokecolor{textcolor}%
\pgfsetfillcolor{textcolor}%
\pgftext[x=8.878695in,y=0.376878in,,top]{\color{textcolor}\sffamily\fontsize{10.000000}{12.000000}\selectfont 0.19}%
\end{pgfscope}%
\begin{pgfscope}%
\pgfsetbuttcap%
\pgfsetroundjoin%
\definecolor{currentfill}{rgb}{0.000000,0.000000,0.000000}%
\pgfsetfillcolor{currentfill}%
\pgfsetlinewidth{0.803000pt}%
\definecolor{currentstroke}{rgb}{0.000000,0.000000,0.000000}%
\pgfsetstrokecolor{currentstroke}%
\pgfsetdash{}{0pt}%
\pgfsys@defobject{currentmarker}{\pgfqpoint{0.000000in}{-0.048611in}}{\pgfqpoint{0.000000in}{0.000000in}}{%
\pgfpathmoveto{\pgfqpoint{0.000000in}{0.000000in}}%
\pgfpathlineto{\pgfqpoint{0.000000in}{-0.048611in}}%
\pgfusepath{stroke,fill}%
}%
\begin{pgfscope}%
\pgfsys@transformshift{8.887882in}{0.474100in}%
\pgfsys@useobject{currentmarker}{}%
\end{pgfscope}%
\end{pgfscope}%
\begin{pgfscope}%
\definecolor{textcolor}{rgb}{0.000000,0.000000,0.000000}%
\pgfsetstrokecolor{textcolor}%
\pgfsetfillcolor{textcolor}%
\pgftext[x=8.887882in,y=0.376878in,,top]{\color{textcolor}\sffamily\fontsize{10.000000}{12.000000}\selectfont 0.20}%
\end{pgfscope}%
\begin{pgfscope}%
\pgfsetbuttcap%
\pgfsetroundjoin%
\definecolor{currentfill}{rgb}{0.000000,0.000000,0.000000}%
\pgfsetfillcolor{currentfill}%
\pgfsetlinewidth{0.803000pt}%
\definecolor{currentstroke}{rgb}{0.000000,0.000000,0.000000}%
\pgfsetstrokecolor{currentstroke}%
\pgfsetdash{}{0pt}%
\pgfsys@defobject{currentmarker}{\pgfqpoint{0.000000in}{-0.048611in}}{\pgfqpoint{0.000000in}{0.000000in}}{%
\pgfpathmoveto{\pgfqpoint{0.000000in}{0.000000in}}%
\pgfpathlineto{\pgfqpoint{0.000000in}{-0.048611in}}%
\pgfusepath{stroke,fill}%
}%
\begin{pgfscope}%
\pgfsys@transformshift{8.897069in}{0.474100in}%
\pgfsys@useobject{currentmarker}{}%
\end{pgfscope}%
\end{pgfscope}%
\begin{pgfscope}%
\definecolor{textcolor}{rgb}{0.000000,0.000000,0.000000}%
\pgfsetstrokecolor{textcolor}%
\pgfsetfillcolor{textcolor}%
\pgftext[x=8.897069in,y=0.376878in,,top]{\color{textcolor}\sffamily\fontsize{10.000000}{12.000000}\selectfont 0.21}%
\end{pgfscope}%
\begin{pgfscope}%
\pgfsetbuttcap%
\pgfsetroundjoin%
\definecolor{currentfill}{rgb}{0.000000,0.000000,0.000000}%
\pgfsetfillcolor{currentfill}%
\pgfsetlinewidth{0.803000pt}%
\definecolor{currentstroke}{rgb}{0.000000,0.000000,0.000000}%
\pgfsetstrokecolor{currentstroke}%
\pgfsetdash{}{0pt}%
\pgfsys@defobject{currentmarker}{\pgfqpoint{0.000000in}{-0.048611in}}{\pgfqpoint{0.000000in}{0.000000in}}{%
\pgfpathmoveto{\pgfqpoint{0.000000in}{0.000000in}}%
\pgfpathlineto{\pgfqpoint{0.000000in}{-0.048611in}}%
\pgfusepath{stroke,fill}%
}%
\begin{pgfscope}%
\pgfsys@transformshift{8.906256in}{0.474100in}%
\pgfsys@useobject{currentmarker}{}%
\end{pgfscope}%
\end{pgfscope}%
\begin{pgfscope}%
\definecolor{textcolor}{rgb}{0.000000,0.000000,0.000000}%
\pgfsetstrokecolor{textcolor}%
\pgfsetfillcolor{textcolor}%
\pgftext[x=8.906256in,y=0.376878in,,top]{\color{textcolor}\sffamily\fontsize{10.000000}{12.000000}\selectfont 0.22}%
\end{pgfscope}%
\begin{pgfscope}%
\pgfsetbuttcap%
\pgfsetroundjoin%
\definecolor{currentfill}{rgb}{0.000000,0.000000,0.000000}%
\pgfsetfillcolor{currentfill}%
\pgfsetlinewidth{0.803000pt}%
\definecolor{currentstroke}{rgb}{0.000000,0.000000,0.000000}%
\pgfsetstrokecolor{currentstroke}%
\pgfsetdash{}{0pt}%
\pgfsys@defobject{currentmarker}{\pgfqpoint{0.000000in}{-0.048611in}}{\pgfqpoint{0.000000in}{0.000000in}}{%
\pgfpathmoveto{\pgfqpoint{0.000000in}{0.000000in}}%
\pgfpathlineto{\pgfqpoint{0.000000in}{-0.048611in}}%
\pgfusepath{stroke,fill}%
}%
\begin{pgfscope}%
\pgfsys@transformshift{8.915442in}{0.474100in}%
\pgfsys@useobject{currentmarker}{}%
\end{pgfscope}%
\end{pgfscope}%
\begin{pgfscope}%
\definecolor{textcolor}{rgb}{0.000000,0.000000,0.000000}%
\pgfsetstrokecolor{textcolor}%
\pgfsetfillcolor{textcolor}%
\pgftext[x=8.915442in,y=0.376878in,,top]{\color{textcolor}\sffamily\fontsize{10.000000}{12.000000}\selectfont 0.23}%
\end{pgfscope}%
\begin{pgfscope}%
\pgfsetbuttcap%
\pgfsetroundjoin%
\definecolor{currentfill}{rgb}{0.000000,0.000000,0.000000}%
\pgfsetfillcolor{currentfill}%
\pgfsetlinewidth{0.803000pt}%
\definecolor{currentstroke}{rgb}{0.000000,0.000000,0.000000}%
\pgfsetstrokecolor{currentstroke}%
\pgfsetdash{}{0pt}%
\pgfsys@defobject{currentmarker}{\pgfqpoint{0.000000in}{-0.048611in}}{\pgfqpoint{0.000000in}{0.000000in}}{%
\pgfpathmoveto{\pgfqpoint{0.000000in}{0.000000in}}%
\pgfpathlineto{\pgfqpoint{0.000000in}{-0.048611in}}%
\pgfusepath{stroke,fill}%
}%
\begin{pgfscope}%
\pgfsys@transformshift{8.924629in}{0.474100in}%
\pgfsys@useobject{currentmarker}{}%
\end{pgfscope}%
\end{pgfscope}%
\begin{pgfscope}%
\definecolor{textcolor}{rgb}{0.000000,0.000000,0.000000}%
\pgfsetstrokecolor{textcolor}%
\pgfsetfillcolor{textcolor}%
\pgftext[x=8.924629in,y=0.376878in,,top]{\color{textcolor}\sffamily\fontsize{10.000000}{12.000000}\selectfont 0.24}%
\end{pgfscope}%
\begin{pgfscope}%
\pgfsetbuttcap%
\pgfsetroundjoin%
\definecolor{currentfill}{rgb}{0.000000,0.000000,0.000000}%
\pgfsetfillcolor{currentfill}%
\pgfsetlinewidth{0.803000pt}%
\definecolor{currentstroke}{rgb}{0.000000,0.000000,0.000000}%
\pgfsetstrokecolor{currentstroke}%
\pgfsetdash{}{0pt}%
\pgfsys@defobject{currentmarker}{\pgfqpoint{0.000000in}{-0.048611in}}{\pgfqpoint{0.000000in}{0.000000in}}{%
\pgfpathmoveto{\pgfqpoint{0.000000in}{0.000000in}}%
\pgfpathlineto{\pgfqpoint{0.000000in}{-0.048611in}}%
\pgfusepath{stroke,fill}%
}%
\begin{pgfscope}%
\pgfsys@transformshift{8.933816in}{0.474100in}%
\pgfsys@useobject{currentmarker}{}%
\end{pgfscope}%
\end{pgfscope}%
\begin{pgfscope}%
\definecolor{textcolor}{rgb}{0.000000,0.000000,0.000000}%
\pgfsetstrokecolor{textcolor}%
\pgfsetfillcolor{textcolor}%
\pgftext[x=8.933816in,y=0.376878in,,top]{\color{textcolor}\sffamily\fontsize{10.000000}{12.000000}\selectfont 0.25}%
\end{pgfscope}%
\begin{pgfscope}%
\pgfsetbuttcap%
\pgfsetroundjoin%
\definecolor{currentfill}{rgb}{0.000000,0.000000,0.000000}%
\pgfsetfillcolor{currentfill}%
\pgfsetlinewidth{0.803000pt}%
\definecolor{currentstroke}{rgb}{0.000000,0.000000,0.000000}%
\pgfsetstrokecolor{currentstroke}%
\pgfsetdash{}{0pt}%
\pgfsys@defobject{currentmarker}{\pgfqpoint{0.000000in}{-0.048611in}}{\pgfqpoint{0.000000in}{0.000000in}}{%
\pgfpathmoveto{\pgfqpoint{0.000000in}{0.000000in}}%
\pgfpathlineto{\pgfqpoint{0.000000in}{-0.048611in}}%
\pgfusepath{stroke,fill}%
}%
\begin{pgfscope}%
\pgfsys@transformshift{8.943003in}{0.474100in}%
\pgfsys@useobject{currentmarker}{}%
\end{pgfscope}%
\end{pgfscope}%
\begin{pgfscope}%
\definecolor{textcolor}{rgb}{0.000000,0.000000,0.000000}%
\pgfsetstrokecolor{textcolor}%
\pgfsetfillcolor{textcolor}%
\pgftext[x=8.943003in,y=0.376878in,,top]{\color{textcolor}\sffamily\fontsize{10.000000}{12.000000}\selectfont 0.26}%
\end{pgfscope}%
\begin{pgfscope}%
\pgfsetbuttcap%
\pgfsetroundjoin%
\definecolor{currentfill}{rgb}{0.000000,0.000000,0.000000}%
\pgfsetfillcolor{currentfill}%
\pgfsetlinewidth{0.803000pt}%
\definecolor{currentstroke}{rgb}{0.000000,0.000000,0.000000}%
\pgfsetstrokecolor{currentstroke}%
\pgfsetdash{}{0pt}%
\pgfsys@defobject{currentmarker}{\pgfqpoint{0.000000in}{-0.048611in}}{\pgfqpoint{0.000000in}{0.000000in}}{%
\pgfpathmoveto{\pgfqpoint{0.000000in}{0.000000in}}%
\pgfpathlineto{\pgfqpoint{0.000000in}{-0.048611in}}%
\pgfusepath{stroke,fill}%
}%
\begin{pgfscope}%
\pgfsys@transformshift{8.952189in}{0.474100in}%
\pgfsys@useobject{currentmarker}{}%
\end{pgfscope}%
\end{pgfscope}%
\begin{pgfscope}%
\definecolor{textcolor}{rgb}{0.000000,0.000000,0.000000}%
\pgfsetstrokecolor{textcolor}%
\pgfsetfillcolor{textcolor}%
\pgftext[x=8.952189in,y=0.376878in,,top]{\color{textcolor}\sffamily\fontsize{10.000000}{12.000000}\selectfont 0.27}%
\end{pgfscope}%
\begin{pgfscope}%
\pgfsetbuttcap%
\pgfsetroundjoin%
\definecolor{currentfill}{rgb}{0.000000,0.000000,0.000000}%
\pgfsetfillcolor{currentfill}%
\pgfsetlinewidth{0.803000pt}%
\definecolor{currentstroke}{rgb}{0.000000,0.000000,0.000000}%
\pgfsetstrokecolor{currentstroke}%
\pgfsetdash{}{0pt}%
\pgfsys@defobject{currentmarker}{\pgfqpoint{0.000000in}{-0.048611in}}{\pgfqpoint{0.000000in}{0.000000in}}{%
\pgfpathmoveto{\pgfqpoint{0.000000in}{0.000000in}}%
\pgfpathlineto{\pgfqpoint{0.000000in}{-0.048611in}}%
\pgfusepath{stroke,fill}%
}%
\begin{pgfscope}%
\pgfsys@transformshift{8.961376in}{0.474100in}%
\pgfsys@useobject{currentmarker}{}%
\end{pgfscope}%
\end{pgfscope}%
\begin{pgfscope}%
\definecolor{textcolor}{rgb}{0.000000,0.000000,0.000000}%
\pgfsetstrokecolor{textcolor}%
\pgfsetfillcolor{textcolor}%
\pgftext[x=8.961376in,y=0.376878in,,top]{\color{textcolor}\sffamily\fontsize{10.000000}{12.000000}\selectfont 0.28}%
\end{pgfscope}%
\begin{pgfscope}%
\pgfsetbuttcap%
\pgfsetroundjoin%
\definecolor{currentfill}{rgb}{0.000000,0.000000,0.000000}%
\pgfsetfillcolor{currentfill}%
\pgfsetlinewidth{0.803000pt}%
\definecolor{currentstroke}{rgb}{0.000000,0.000000,0.000000}%
\pgfsetstrokecolor{currentstroke}%
\pgfsetdash{}{0pt}%
\pgfsys@defobject{currentmarker}{\pgfqpoint{0.000000in}{-0.048611in}}{\pgfqpoint{0.000000in}{0.000000in}}{%
\pgfpathmoveto{\pgfqpoint{0.000000in}{0.000000in}}%
\pgfpathlineto{\pgfqpoint{0.000000in}{-0.048611in}}%
\pgfusepath{stroke,fill}%
}%
\begin{pgfscope}%
\pgfsys@transformshift{8.970563in}{0.474100in}%
\pgfsys@useobject{currentmarker}{}%
\end{pgfscope}%
\end{pgfscope}%
\begin{pgfscope}%
\definecolor{textcolor}{rgb}{0.000000,0.000000,0.000000}%
\pgfsetstrokecolor{textcolor}%
\pgfsetfillcolor{textcolor}%
\pgftext[x=8.970563in,y=0.376878in,,top]{\color{textcolor}\sffamily\fontsize{10.000000}{12.000000}\selectfont 0.29}%
\end{pgfscope}%
\begin{pgfscope}%
\pgfsetbuttcap%
\pgfsetroundjoin%
\definecolor{currentfill}{rgb}{0.000000,0.000000,0.000000}%
\pgfsetfillcolor{currentfill}%
\pgfsetlinewidth{0.803000pt}%
\definecolor{currentstroke}{rgb}{0.000000,0.000000,0.000000}%
\pgfsetstrokecolor{currentstroke}%
\pgfsetdash{}{0pt}%
\pgfsys@defobject{currentmarker}{\pgfqpoint{0.000000in}{-0.048611in}}{\pgfqpoint{0.000000in}{0.000000in}}{%
\pgfpathmoveto{\pgfqpoint{0.000000in}{0.000000in}}%
\pgfpathlineto{\pgfqpoint{0.000000in}{-0.048611in}}%
\pgfusepath{stroke,fill}%
}%
\begin{pgfscope}%
\pgfsys@transformshift{8.979750in}{0.474100in}%
\pgfsys@useobject{currentmarker}{}%
\end{pgfscope}%
\end{pgfscope}%
\begin{pgfscope}%
\definecolor{textcolor}{rgb}{0.000000,0.000000,0.000000}%
\pgfsetstrokecolor{textcolor}%
\pgfsetfillcolor{textcolor}%
\pgftext[x=8.979750in,y=0.376878in,,top]{\color{textcolor}\sffamily\fontsize{10.000000}{12.000000}\selectfont 0.30}%
\end{pgfscope}%
\begin{pgfscope}%
\pgfsetbuttcap%
\pgfsetroundjoin%
\definecolor{currentfill}{rgb}{0.000000,0.000000,0.000000}%
\pgfsetfillcolor{currentfill}%
\pgfsetlinewidth{0.803000pt}%
\definecolor{currentstroke}{rgb}{0.000000,0.000000,0.000000}%
\pgfsetstrokecolor{currentstroke}%
\pgfsetdash{}{0pt}%
\pgfsys@defobject{currentmarker}{\pgfqpoint{0.000000in}{-0.048611in}}{\pgfqpoint{0.000000in}{0.000000in}}{%
\pgfpathmoveto{\pgfqpoint{0.000000in}{0.000000in}}%
\pgfpathlineto{\pgfqpoint{0.000000in}{-0.048611in}}%
\pgfusepath{stroke,fill}%
}%
\begin{pgfscope}%
\pgfsys@transformshift{8.988936in}{0.474100in}%
\pgfsys@useobject{currentmarker}{}%
\end{pgfscope}%
\end{pgfscope}%
\begin{pgfscope}%
\definecolor{textcolor}{rgb}{0.000000,0.000000,0.000000}%
\pgfsetstrokecolor{textcolor}%
\pgfsetfillcolor{textcolor}%
\pgftext[x=8.988936in,y=0.376878in,,top]{\color{textcolor}\sffamily\fontsize{10.000000}{12.000000}\selectfont 0.31}%
\end{pgfscope}%
\begin{pgfscope}%
\pgfsetbuttcap%
\pgfsetroundjoin%
\definecolor{currentfill}{rgb}{0.000000,0.000000,0.000000}%
\pgfsetfillcolor{currentfill}%
\pgfsetlinewidth{0.803000pt}%
\definecolor{currentstroke}{rgb}{0.000000,0.000000,0.000000}%
\pgfsetstrokecolor{currentstroke}%
\pgfsetdash{}{0pt}%
\pgfsys@defobject{currentmarker}{\pgfqpoint{0.000000in}{-0.048611in}}{\pgfqpoint{0.000000in}{0.000000in}}{%
\pgfpathmoveto{\pgfqpoint{0.000000in}{0.000000in}}%
\pgfpathlineto{\pgfqpoint{0.000000in}{-0.048611in}}%
\pgfusepath{stroke,fill}%
}%
\begin{pgfscope}%
\pgfsys@transformshift{8.998123in}{0.474100in}%
\pgfsys@useobject{currentmarker}{}%
\end{pgfscope}%
\end{pgfscope}%
\begin{pgfscope}%
\definecolor{textcolor}{rgb}{0.000000,0.000000,0.000000}%
\pgfsetstrokecolor{textcolor}%
\pgfsetfillcolor{textcolor}%
\pgftext[x=8.998123in,y=0.376878in,,top]{\color{textcolor}\sffamily\fontsize{10.000000}{12.000000}\selectfont 0.32}%
\end{pgfscope}%
\begin{pgfscope}%
\pgfsetbuttcap%
\pgfsetroundjoin%
\definecolor{currentfill}{rgb}{0.000000,0.000000,0.000000}%
\pgfsetfillcolor{currentfill}%
\pgfsetlinewidth{0.803000pt}%
\definecolor{currentstroke}{rgb}{0.000000,0.000000,0.000000}%
\pgfsetstrokecolor{currentstroke}%
\pgfsetdash{}{0pt}%
\pgfsys@defobject{currentmarker}{\pgfqpoint{0.000000in}{-0.048611in}}{\pgfqpoint{0.000000in}{0.000000in}}{%
\pgfpathmoveto{\pgfqpoint{0.000000in}{0.000000in}}%
\pgfpathlineto{\pgfqpoint{0.000000in}{-0.048611in}}%
\pgfusepath{stroke,fill}%
}%
\begin{pgfscope}%
\pgfsys@transformshift{9.007310in}{0.474100in}%
\pgfsys@useobject{currentmarker}{}%
\end{pgfscope}%
\end{pgfscope}%
\begin{pgfscope}%
\definecolor{textcolor}{rgb}{0.000000,0.000000,0.000000}%
\pgfsetstrokecolor{textcolor}%
\pgfsetfillcolor{textcolor}%
\pgftext[x=9.007310in,y=0.376878in,,top]{\color{textcolor}\sffamily\fontsize{10.000000}{12.000000}\selectfont 0.33}%
\end{pgfscope}%
\begin{pgfscope}%
\pgfsetbuttcap%
\pgfsetroundjoin%
\definecolor{currentfill}{rgb}{0.000000,0.000000,0.000000}%
\pgfsetfillcolor{currentfill}%
\pgfsetlinewidth{0.803000pt}%
\definecolor{currentstroke}{rgb}{0.000000,0.000000,0.000000}%
\pgfsetstrokecolor{currentstroke}%
\pgfsetdash{}{0pt}%
\pgfsys@defobject{currentmarker}{\pgfqpoint{0.000000in}{-0.048611in}}{\pgfqpoint{0.000000in}{0.000000in}}{%
\pgfpathmoveto{\pgfqpoint{0.000000in}{0.000000in}}%
\pgfpathlineto{\pgfqpoint{0.000000in}{-0.048611in}}%
\pgfusepath{stroke,fill}%
}%
\begin{pgfscope}%
\pgfsys@transformshift{9.016497in}{0.474100in}%
\pgfsys@useobject{currentmarker}{}%
\end{pgfscope}%
\end{pgfscope}%
\begin{pgfscope}%
\definecolor{textcolor}{rgb}{0.000000,0.000000,0.000000}%
\pgfsetstrokecolor{textcolor}%
\pgfsetfillcolor{textcolor}%
\pgftext[x=9.016497in,y=0.376878in,,top]{\color{textcolor}\sffamily\fontsize{10.000000}{12.000000}\selectfont 0.34}%
\end{pgfscope}%
\begin{pgfscope}%
\pgfsetbuttcap%
\pgfsetroundjoin%
\definecolor{currentfill}{rgb}{0.000000,0.000000,0.000000}%
\pgfsetfillcolor{currentfill}%
\pgfsetlinewidth{0.803000pt}%
\definecolor{currentstroke}{rgb}{0.000000,0.000000,0.000000}%
\pgfsetstrokecolor{currentstroke}%
\pgfsetdash{}{0pt}%
\pgfsys@defobject{currentmarker}{\pgfqpoint{0.000000in}{-0.048611in}}{\pgfqpoint{0.000000in}{0.000000in}}{%
\pgfpathmoveto{\pgfqpoint{0.000000in}{0.000000in}}%
\pgfpathlineto{\pgfqpoint{0.000000in}{-0.048611in}}%
\pgfusepath{stroke,fill}%
}%
\begin{pgfscope}%
\pgfsys@transformshift{9.025683in}{0.474100in}%
\pgfsys@useobject{currentmarker}{}%
\end{pgfscope}%
\end{pgfscope}%
\begin{pgfscope}%
\definecolor{textcolor}{rgb}{0.000000,0.000000,0.000000}%
\pgfsetstrokecolor{textcolor}%
\pgfsetfillcolor{textcolor}%
\pgftext[x=9.025683in,y=0.376878in,,top]{\color{textcolor}\sffamily\fontsize{10.000000}{12.000000}\selectfont 0.35}%
\end{pgfscope}%
\begin{pgfscope}%
\pgfsetbuttcap%
\pgfsetroundjoin%
\definecolor{currentfill}{rgb}{0.000000,0.000000,0.000000}%
\pgfsetfillcolor{currentfill}%
\pgfsetlinewidth{0.803000pt}%
\definecolor{currentstroke}{rgb}{0.000000,0.000000,0.000000}%
\pgfsetstrokecolor{currentstroke}%
\pgfsetdash{}{0pt}%
\pgfsys@defobject{currentmarker}{\pgfqpoint{0.000000in}{-0.048611in}}{\pgfqpoint{0.000000in}{0.000000in}}{%
\pgfpathmoveto{\pgfqpoint{0.000000in}{0.000000in}}%
\pgfpathlineto{\pgfqpoint{0.000000in}{-0.048611in}}%
\pgfusepath{stroke,fill}%
}%
\begin{pgfscope}%
\pgfsys@transformshift{9.034870in}{0.474100in}%
\pgfsys@useobject{currentmarker}{}%
\end{pgfscope}%
\end{pgfscope}%
\begin{pgfscope}%
\definecolor{textcolor}{rgb}{0.000000,0.000000,0.000000}%
\pgfsetstrokecolor{textcolor}%
\pgfsetfillcolor{textcolor}%
\pgftext[x=9.034870in,y=0.376878in,,top]{\color{textcolor}\sffamily\fontsize{10.000000}{12.000000}\selectfont 0.36}%
\end{pgfscope}%
\begin{pgfscope}%
\pgfsetbuttcap%
\pgfsetroundjoin%
\definecolor{currentfill}{rgb}{0.000000,0.000000,0.000000}%
\pgfsetfillcolor{currentfill}%
\pgfsetlinewidth{0.803000pt}%
\definecolor{currentstroke}{rgb}{0.000000,0.000000,0.000000}%
\pgfsetstrokecolor{currentstroke}%
\pgfsetdash{}{0pt}%
\pgfsys@defobject{currentmarker}{\pgfqpoint{0.000000in}{-0.048611in}}{\pgfqpoint{0.000000in}{0.000000in}}{%
\pgfpathmoveto{\pgfqpoint{0.000000in}{0.000000in}}%
\pgfpathlineto{\pgfqpoint{0.000000in}{-0.048611in}}%
\pgfusepath{stroke,fill}%
}%
\begin{pgfscope}%
\pgfsys@transformshift{9.044057in}{0.474100in}%
\pgfsys@useobject{currentmarker}{}%
\end{pgfscope}%
\end{pgfscope}%
\begin{pgfscope}%
\definecolor{textcolor}{rgb}{0.000000,0.000000,0.000000}%
\pgfsetstrokecolor{textcolor}%
\pgfsetfillcolor{textcolor}%
\pgftext[x=9.044057in,y=0.376878in,,top]{\color{textcolor}\sffamily\fontsize{10.000000}{12.000000}\selectfont 0.37}%
\end{pgfscope}%
\begin{pgfscope}%
\pgfsetbuttcap%
\pgfsetroundjoin%
\definecolor{currentfill}{rgb}{0.000000,0.000000,0.000000}%
\pgfsetfillcolor{currentfill}%
\pgfsetlinewidth{0.803000pt}%
\definecolor{currentstroke}{rgb}{0.000000,0.000000,0.000000}%
\pgfsetstrokecolor{currentstroke}%
\pgfsetdash{}{0pt}%
\pgfsys@defobject{currentmarker}{\pgfqpoint{0.000000in}{-0.048611in}}{\pgfqpoint{0.000000in}{0.000000in}}{%
\pgfpathmoveto{\pgfqpoint{0.000000in}{0.000000in}}%
\pgfpathlineto{\pgfqpoint{0.000000in}{-0.048611in}}%
\pgfusepath{stroke,fill}%
}%
\begin{pgfscope}%
\pgfsys@transformshift{9.053244in}{0.474100in}%
\pgfsys@useobject{currentmarker}{}%
\end{pgfscope}%
\end{pgfscope}%
\begin{pgfscope}%
\definecolor{textcolor}{rgb}{0.000000,0.000000,0.000000}%
\pgfsetstrokecolor{textcolor}%
\pgfsetfillcolor{textcolor}%
\pgftext[x=9.053244in,y=0.376878in,,top]{\color{textcolor}\sffamily\fontsize{10.000000}{12.000000}\selectfont 0.38}%
\end{pgfscope}%
\begin{pgfscope}%
\pgfsetbuttcap%
\pgfsetroundjoin%
\definecolor{currentfill}{rgb}{0.000000,0.000000,0.000000}%
\pgfsetfillcolor{currentfill}%
\pgfsetlinewidth{0.803000pt}%
\definecolor{currentstroke}{rgb}{0.000000,0.000000,0.000000}%
\pgfsetstrokecolor{currentstroke}%
\pgfsetdash{}{0pt}%
\pgfsys@defobject{currentmarker}{\pgfqpoint{0.000000in}{-0.048611in}}{\pgfqpoint{0.000000in}{0.000000in}}{%
\pgfpathmoveto{\pgfqpoint{0.000000in}{0.000000in}}%
\pgfpathlineto{\pgfqpoint{0.000000in}{-0.048611in}}%
\pgfusepath{stroke,fill}%
}%
\begin{pgfscope}%
\pgfsys@transformshift{9.062430in}{0.474100in}%
\pgfsys@useobject{currentmarker}{}%
\end{pgfscope}%
\end{pgfscope}%
\begin{pgfscope}%
\definecolor{textcolor}{rgb}{0.000000,0.000000,0.000000}%
\pgfsetstrokecolor{textcolor}%
\pgfsetfillcolor{textcolor}%
\pgftext[x=9.062430in,y=0.376878in,,top]{\color{textcolor}\sffamily\fontsize{10.000000}{12.000000}\selectfont 0.39}%
\end{pgfscope}%
\begin{pgfscope}%
\pgfsetbuttcap%
\pgfsetroundjoin%
\definecolor{currentfill}{rgb}{0.000000,0.000000,0.000000}%
\pgfsetfillcolor{currentfill}%
\pgfsetlinewidth{0.803000pt}%
\definecolor{currentstroke}{rgb}{0.000000,0.000000,0.000000}%
\pgfsetstrokecolor{currentstroke}%
\pgfsetdash{}{0pt}%
\pgfsys@defobject{currentmarker}{\pgfqpoint{0.000000in}{-0.048611in}}{\pgfqpoint{0.000000in}{0.000000in}}{%
\pgfpathmoveto{\pgfqpoint{0.000000in}{0.000000in}}%
\pgfpathlineto{\pgfqpoint{0.000000in}{-0.048611in}}%
\pgfusepath{stroke,fill}%
}%
\begin{pgfscope}%
\pgfsys@transformshift{9.071617in}{0.474100in}%
\pgfsys@useobject{currentmarker}{}%
\end{pgfscope}%
\end{pgfscope}%
\begin{pgfscope}%
\definecolor{textcolor}{rgb}{0.000000,0.000000,0.000000}%
\pgfsetstrokecolor{textcolor}%
\pgfsetfillcolor{textcolor}%
\pgftext[x=9.071617in,y=0.376878in,,top]{\color{textcolor}\sffamily\fontsize{10.000000}{12.000000}\selectfont 0.40}%
\end{pgfscope}%
\begin{pgfscope}%
\pgfsetbuttcap%
\pgfsetroundjoin%
\definecolor{currentfill}{rgb}{0.000000,0.000000,0.000000}%
\pgfsetfillcolor{currentfill}%
\pgfsetlinewidth{0.803000pt}%
\definecolor{currentstroke}{rgb}{0.000000,0.000000,0.000000}%
\pgfsetstrokecolor{currentstroke}%
\pgfsetdash{}{0pt}%
\pgfsys@defobject{currentmarker}{\pgfqpoint{-0.048611in}{0.000000in}}{\pgfqpoint{-0.000000in}{0.000000in}}{%
\pgfpathmoveto{\pgfqpoint{-0.000000in}{0.000000in}}%
\pgfpathlineto{\pgfqpoint{-0.048611in}{0.000000in}}%
\pgfusepath{stroke,fill}%
}%
\begin{pgfscope}%
\pgfsys@transformshift{6.572727in}{0.737137in}%
\pgfsys@useobject{currentmarker}{}%
\end{pgfscope}%
\end{pgfscope}%
\begin{pgfscope}%
\definecolor{textcolor}{rgb}{0.000000,0.000000,0.000000}%
\pgfsetstrokecolor{textcolor}%
\pgfsetfillcolor{textcolor}%
\pgftext[x=6.146601in, y=0.684376in, left, base]{\color{textcolor}\sffamily\fontsize{10.000000}{12.000000}\selectfont \ensuremath{-}2.0}%
\end{pgfscope}%
\begin{pgfscope}%
\pgfsetbuttcap%
\pgfsetroundjoin%
\definecolor{currentfill}{rgb}{0.000000,0.000000,0.000000}%
\pgfsetfillcolor{currentfill}%
\pgfsetlinewidth{0.803000pt}%
\definecolor{currentstroke}{rgb}{0.000000,0.000000,0.000000}%
\pgfsetstrokecolor{currentstroke}%
\pgfsetdash{}{0pt}%
\pgfsys@defobject{currentmarker}{\pgfqpoint{-0.048611in}{0.000000in}}{\pgfqpoint{-0.000000in}{0.000000in}}{%
\pgfpathmoveto{\pgfqpoint{-0.000000in}{0.000000in}}%
\pgfpathlineto{\pgfqpoint{-0.048611in}{0.000000in}}%
\pgfusepath{stroke,fill}%
}%
\begin{pgfscope}%
\pgfsys@transformshift{6.572727in}{1.156305in}%
\pgfsys@useobject{currentmarker}{}%
\end{pgfscope}%
\end{pgfscope}%
\begin{pgfscope}%
\definecolor{textcolor}{rgb}{0.000000,0.000000,0.000000}%
\pgfsetstrokecolor{textcolor}%
\pgfsetfillcolor{textcolor}%
\pgftext[x=6.146601in, y=1.103543in, left, base]{\color{textcolor}\sffamily\fontsize{10.000000}{12.000000}\selectfont \ensuremath{-}1.5}%
\end{pgfscope}%
\begin{pgfscope}%
\pgfsetbuttcap%
\pgfsetroundjoin%
\definecolor{currentfill}{rgb}{0.000000,0.000000,0.000000}%
\pgfsetfillcolor{currentfill}%
\pgfsetlinewidth{0.803000pt}%
\definecolor{currentstroke}{rgb}{0.000000,0.000000,0.000000}%
\pgfsetstrokecolor{currentstroke}%
\pgfsetdash{}{0pt}%
\pgfsys@defobject{currentmarker}{\pgfqpoint{-0.048611in}{0.000000in}}{\pgfqpoint{-0.000000in}{0.000000in}}{%
\pgfpathmoveto{\pgfqpoint{-0.000000in}{0.000000in}}%
\pgfpathlineto{\pgfqpoint{-0.048611in}{0.000000in}}%
\pgfusepath{stroke,fill}%
}%
\begin{pgfscope}%
\pgfsys@transformshift{6.572727in}{1.575472in}%
\pgfsys@useobject{currentmarker}{}%
\end{pgfscope}%
\end{pgfscope}%
\begin{pgfscope}%
\definecolor{textcolor}{rgb}{0.000000,0.000000,0.000000}%
\pgfsetstrokecolor{textcolor}%
\pgfsetfillcolor{textcolor}%
\pgftext[x=6.146601in, y=1.522710in, left, base]{\color{textcolor}\sffamily\fontsize{10.000000}{12.000000}\selectfont \ensuremath{-}1.0}%
\end{pgfscope}%
\begin{pgfscope}%
\pgfsetbuttcap%
\pgfsetroundjoin%
\definecolor{currentfill}{rgb}{0.000000,0.000000,0.000000}%
\pgfsetfillcolor{currentfill}%
\pgfsetlinewidth{0.803000pt}%
\definecolor{currentstroke}{rgb}{0.000000,0.000000,0.000000}%
\pgfsetstrokecolor{currentstroke}%
\pgfsetdash{}{0pt}%
\pgfsys@defobject{currentmarker}{\pgfqpoint{-0.048611in}{0.000000in}}{\pgfqpoint{-0.000000in}{0.000000in}}{%
\pgfpathmoveto{\pgfqpoint{-0.000000in}{0.000000in}}%
\pgfpathlineto{\pgfqpoint{-0.048611in}{0.000000in}}%
\pgfusepath{stroke,fill}%
}%
\begin{pgfscope}%
\pgfsys@transformshift{6.572727in}{1.994639in}%
\pgfsys@useobject{currentmarker}{}%
\end{pgfscope}%
\end{pgfscope}%
\begin{pgfscope}%
\definecolor{textcolor}{rgb}{0.000000,0.000000,0.000000}%
\pgfsetstrokecolor{textcolor}%
\pgfsetfillcolor{textcolor}%
\pgftext[x=6.146601in, y=1.941877in, left, base]{\color{textcolor}\sffamily\fontsize{10.000000}{12.000000}\selectfont \ensuremath{-}0.5}%
\end{pgfscope}%
\begin{pgfscope}%
\pgfsetbuttcap%
\pgfsetroundjoin%
\definecolor{currentfill}{rgb}{0.000000,0.000000,0.000000}%
\pgfsetfillcolor{currentfill}%
\pgfsetlinewidth{0.803000pt}%
\definecolor{currentstroke}{rgb}{0.000000,0.000000,0.000000}%
\pgfsetstrokecolor{currentstroke}%
\pgfsetdash{}{0pt}%
\pgfsys@defobject{currentmarker}{\pgfqpoint{-0.048611in}{0.000000in}}{\pgfqpoint{-0.000000in}{0.000000in}}{%
\pgfpathmoveto{\pgfqpoint{-0.000000in}{0.000000in}}%
\pgfpathlineto{\pgfqpoint{-0.048611in}{0.000000in}}%
\pgfusepath{stroke,fill}%
}%
\begin{pgfscope}%
\pgfsys@transformshift{6.572727in}{2.413806in}%
\pgfsys@useobject{currentmarker}{}%
\end{pgfscope}%
\end{pgfscope}%
\begin{pgfscope}%
\definecolor{textcolor}{rgb}{0.000000,0.000000,0.000000}%
\pgfsetstrokecolor{textcolor}%
\pgfsetfillcolor{textcolor}%
\pgftext[x=6.254626in, y=2.361044in, left, base]{\color{textcolor}\sffamily\fontsize{10.000000}{12.000000}\selectfont 0.0}%
\end{pgfscope}%
\begin{pgfscope}%
\pgfsetbuttcap%
\pgfsetroundjoin%
\definecolor{currentfill}{rgb}{0.000000,0.000000,0.000000}%
\pgfsetfillcolor{currentfill}%
\pgfsetlinewidth{0.803000pt}%
\definecolor{currentstroke}{rgb}{0.000000,0.000000,0.000000}%
\pgfsetstrokecolor{currentstroke}%
\pgfsetdash{}{0pt}%
\pgfsys@defobject{currentmarker}{\pgfqpoint{-0.048611in}{0.000000in}}{\pgfqpoint{-0.000000in}{0.000000in}}{%
\pgfpathmoveto{\pgfqpoint{-0.000000in}{0.000000in}}%
\pgfpathlineto{\pgfqpoint{-0.048611in}{0.000000in}}%
\pgfusepath{stroke,fill}%
}%
\begin{pgfscope}%
\pgfsys@transformshift{6.572727in}{2.832973in}%
\pgfsys@useobject{currentmarker}{}%
\end{pgfscope}%
\end{pgfscope}%
\begin{pgfscope}%
\definecolor{textcolor}{rgb}{0.000000,0.000000,0.000000}%
\pgfsetstrokecolor{textcolor}%
\pgfsetfillcolor{textcolor}%
\pgftext[x=6.254626in, y=2.780211in, left, base]{\color{textcolor}\sffamily\fontsize{10.000000}{12.000000}\selectfont 0.5}%
\end{pgfscope}%
\begin{pgfscope}%
\pgfsetbuttcap%
\pgfsetroundjoin%
\definecolor{currentfill}{rgb}{0.000000,0.000000,0.000000}%
\pgfsetfillcolor{currentfill}%
\pgfsetlinewidth{0.803000pt}%
\definecolor{currentstroke}{rgb}{0.000000,0.000000,0.000000}%
\pgfsetstrokecolor{currentstroke}%
\pgfsetdash{}{0pt}%
\pgfsys@defobject{currentmarker}{\pgfqpoint{-0.048611in}{0.000000in}}{\pgfqpoint{-0.000000in}{0.000000in}}{%
\pgfpathmoveto{\pgfqpoint{-0.000000in}{0.000000in}}%
\pgfpathlineto{\pgfqpoint{-0.048611in}{0.000000in}}%
\pgfusepath{stroke,fill}%
}%
\begin{pgfscope}%
\pgfsys@transformshift{6.572727in}{3.252140in}%
\pgfsys@useobject{currentmarker}{}%
\end{pgfscope}%
\end{pgfscope}%
\begin{pgfscope}%
\definecolor{textcolor}{rgb}{0.000000,0.000000,0.000000}%
\pgfsetstrokecolor{textcolor}%
\pgfsetfillcolor{textcolor}%
\pgftext[x=6.254626in, y=3.199379in, left, base]{\color{textcolor}\sffamily\fontsize{10.000000}{12.000000}\selectfont 1.0}%
\end{pgfscope}%
\begin{pgfscope}%
\pgfsetbuttcap%
\pgfsetroundjoin%
\definecolor{currentfill}{rgb}{0.000000,0.000000,0.000000}%
\pgfsetfillcolor{currentfill}%
\pgfsetlinewidth{0.803000pt}%
\definecolor{currentstroke}{rgb}{0.000000,0.000000,0.000000}%
\pgfsetstrokecolor{currentstroke}%
\pgfsetdash{}{0pt}%
\pgfsys@defobject{currentmarker}{\pgfqpoint{-0.048611in}{0.000000in}}{\pgfqpoint{-0.000000in}{0.000000in}}{%
\pgfpathmoveto{\pgfqpoint{-0.000000in}{0.000000in}}%
\pgfpathlineto{\pgfqpoint{-0.048611in}{0.000000in}}%
\pgfusepath{stroke,fill}%
}%
\begin{pgfscope}%
\pgfsys@transformshift{6.572727in}{3.671307in}%
\pgfsys@useobject{currentmarker}{}%
\end{pgfscope}%
\end{pgfscope}%
\begin{pgfscope}%
\definecolor{textcolor}{rgb}{0.000000,0.000000,0.000000}%
\pgfsetstrokecolor{textcolor}%
\pgfsetfillcolor{textcolor}%
\pgftext[x=6.254626in, y=3.618546in, left, base]{\color{textcolor}\sffamily\fontsize{10.000000}{12.000000}\selectfont 1.5}%
\end{pgfscope}%
\begin{pgfscope}%
\pgfsetrectcap%
\pgfsetmiterjoin%
\pgfsetlinewidth{0.803000pt}%
\definecolor{currentstroke}{rgb}{0.000000,0.000000,0.000000}%
\pgfsetstrokecolor{currentstroke}%
\pgfsetdash{}{0pt}%
\pgfpathmoveto{\pgfqpoint{6.572727in}{0.474100in}}%
\pgfpathlineto{\pgfqpoint{6.572727in}{3.792800in}}%
\pgfusepath{stroke}%
\end{pgfscope}%
\begin{pgfscope}%
\pgfsetrectcap%
\pgfsetmiterjoin%
\pgfsetlinewidth{0.803000pt}%
\definecolor{currentstroke}{rgb}{0.000000,0.000000,0.000000}%
\pgfsetstrokecolor{currentstroke}%
\pgfsetdash{}{0pt}%
\pgfpathmoveto{\pgfqpoint{10.800000in}{0.474100in}}%
\pgfpathlineto{\pgfqpoint{10.800000in}{3.792800in}}%
\pgfusepath{stroke}%
\end{pgfscope}%
\begin{pgfscope}%
\pgfsetrectcap%
\pgfsetmiterjoin%
\pgfsetlinewidth{0.803000pt}%
\definecolor{currentstroke}{rgb}{0.000000,0.000000,0.000000}%
\pgfsetstrokecolor{currentstroke}%
\pgfsetdash{}{0pt}%
\pgfpathmoveto{\pgfqpoint{6.572727in}{0.474100in}}%
\pgfpathlineto{\pgfqpoint{10.800000in}{0.474100in}}%
\pgfusepath{stroke}%
\end{pgfscope}%
\begin{pgfscope}%
\pgfsetrectcap%
\pgfsetmiterjoin%
\pgfsetlinewidth{0.803000pt}%
\definecolor{currentstroke}{rgb}{0.000000,0.000000,0.000000}%
\pgfsetstrokecolor{currentstroke}%
\pgfsetdash{}{0pt}%
\pgfpathmoveto{\pgfqpoint{6.572727in}{3.792800in}}%
\pgfpathlineto{\pgfqpoint{10.800000in}{3.792800in}}%
\pgfusepath{stroke}%
\end{pgfscope}%
\begin{pgfscope}%
\definecolor{textcolor}{rgb}{0.000000,0.000000,0.000000}%
\pgfsetstrokecolor{textcolor}%
\pgfsetfillcolor{textcolor}%
\pgftext[x=8.686364in,y=3.876133in,,base]{\color{textcolor}\sffamily\fontsize{12.000000}{14.400000}\selectfont ε = 0.28}%
\end{pgfscope}%
\begin{pgfscope}%
\definecolor{textcolor}{rgb}{0.000000,0.000000,0.000000}%
\pgfsetstrokecolor{textcolor}%
\pgfsetfillcolor{textcolor}%
%\pgftext[x=6.000000in,y=4.223800in,,top]{\color{textcolor}\sffamily\fontsize{12.000000}{14.400000}\bfseries\selectfont DBSCAN (métrica euclidean)}%
\end{pgfscope}%
\end{pgfpicture}%
\makeatother%
\endgroup%
}}
\end{center}
\subsubsection{DBSCAN con métrica de Manhattan}
\vspace{-1em}
\begin{center}
    \makebox[\textwidth][c]{\scalebox{0.65}{%% Creator: Matplotlib, PGF backend
%%
%% To include the figure in your LaTeX document, write
%%   \input{<filename>.pgf}
%%
%% Make sure the required packages are loaded in your preamble
%%   \usepackage{pgf}
%%
%% Figures using additional raster images can only be included by \input if
%% they are in the same directory as the main LaTeX file. For loading figures
%% from other directories you can use the `import` package
%%   \usepackage{import}
%%
%% and then include the figures with
%%   \import{<path to file>}{<filename>.pgf}
%%
%% Matplotlib used the following preamble
%%   \usepackage{fontspec}
%%   \setmainfont{DejaVuSerif.ttf}[Path=\detokenize{/nix/store/zl80nl46sadml2lln6v1xgbhqks16lz2-python3.8-matplotlib-3.4.3/lib/python3.8/site-packages/matplotlib/mpl-data/fonts/ttf/}]
%%   \setsansfont{DejaVuSans.ttf}[Path=\detokenize{/nix/store/zl80nl46sadml2lln6v1xgbhqks16lz2-python3.8-matplotlib-3.4.3/lib/python3.8/site-packages/matplotlib/mpl-data/fonts/ttf/}]
%%   \setmonofont{DejaVuSansMono.ttf}[Path=\detokenize{/nix/store/zl80nl46sadml2lln6v1xgbhqks16lz2-python3.8-matplotlib-3.4.3/lib/python3.8/site-packages/matplotlib/mpl-data/fonts/ttf/}]
%%
\begingroup%
\makeatletter%
\begin{pgfpicture}%
\pgfpathrectangle{\pgfpointorigin}{\pgfqpoint{12.000000in}{4.300000in}}%
\pgfusepath{use as bounding box, clip}%
\begin{pgfscope}%
\pgfsetbuttcap%
\pgfsetmiterjoin%
\definecolor{currentfill}{rgb}{1.000000,1.000000,1.000000}%
\pgfsetfillcolor{currentfill}%
\pgfsetlinewidth{0.000000pt}%
\definecolor{currentstroke}{rgb}{1.000000,1.000000,1.000000}%
\pgfsetstrokecolor{currentstroke}%
\pgfsetdash{}{0pt}%
\pgfpathmoveto{\pgfqpoint{0.000000in}{0.000000in}}%
\pgfpathlineto{\pgfqpoint{12.000000in}{0.000000in}}%
\pgfpathlineto{\pgfqpoint{12.000000in}{4.300000in}}%
\pgfpathlineto{\pgfqpoint{0.000000in}{4.300000in}}%
\pgfpathclose%
\pgfusepath{fill}%
\end{pgfscope}%
\begin{pgfscope}%
\pgfsetbuttcap%
\pgfsetmiterjoin%
\definecolor{currentfill}{rgb}{1.000000,1.000000,1.000000}%
\pgfsetfillcolor{currentfill}%
\pgfsetlinewidth{0.000000pt}%
\definecolor{currentstroke}{rgb}{0.000000,0.000000,0.000000}%
\pgfsetstrokecolor{currentstroke}%
\pgfsetstrokeopacity{0.000000}%
\pgfsetdash{}{0pt}%
\pgfpathmoveto{\pgfqpoint{1.500000in}{0.473000in}}%
\pgfpathlineto{\pgfqpoint{5.727273in}{0.473000in}}%
\pgfpathlineto{\pgfqpoint{5.727273in}{3.784000in}}%
\pgfpathlineto{\pgfqpoint{1.500000in}{3.784000in}}%
\pgfpathclose%
\pgfusepath{fill}%
\end{pgfscope}%
\begin{pgfscope}%
\pgfpathrectangle{\pgfqpoint{1.500000in}{0.473000in}}{\pgfqpoint{4.227273in}{3.311000in}}%
\pgfusepath{clip}%
\pgfsetbuttcap%
\pgfsetmiterjoin%
\definecolor{currentfill}{rgb}{0.121569,0.466667,0.705882}%
\pgfsetfillcolor{currentfill}%
\pgfsetlinewidth{0.000000pt}%
\definecolor{currentstroke}{rgb}{0.000000,0.000000,0.000000}%
\pgfsetstrokecolor{currentstroke}%
\pgfsetstrokeopacity{0.000000}%
\pgfsetdash{}{0pt}%
\pgfpathmoveto{\pgfqpoint{1.692149in}{2.252476in}}%
\pgfpathlineto{\pgfqpoint{1.804080in}{2.252476in}}%
\pgfpathlineto{\pgfqpoint{1.804080in}{0.623500in}}%
\pgfpathlineto{\pgfqpoint{1.692149in}{0.623500in}}%
\pgfpathclose%
\pgfusepath{fill}%
\end{pgfscope}%
\begin{pgfscope}%
\pgfpathrectangle{\pgfqpoint{1.500000in}{0.473000in}}{\pgfqpoint{4.227273in}{3.311000in}}%
\pgfusepath{clip}%
\pgfsetbuttcap%
\pgfsetmiterjoin%
\definecolor{currentfill}{rgb}{0.121569,0.466667,0.705882}%
\pgfsetfillcolor{currentfill}%
\pgfsetlinewidth{0.000000pt}%
\definecolor{currentstroke}{rgb}{0.000000,0.000000,0.000000}%
\pgfsetstrokecolor{currentstroke}%
\pgfsetstrokeopacity{0.000000}%
\pgfsetdash{}{0pt}%
\pgfpathmoveto{\pgfqpoint{1.816517in}{2.252476in}}%
\pgfpathlineto{\pgfqpoint{1.928448in}{2.252476in}}%
\pgfpathlineto{\pgfqpoint{1.928448in}{0.957937in}}%
\pgfpathlineto{\pgfqpoint{1.816517in}{0.957937in}}%
\pgfpathclose%
\pgfusepath{fill}%
\end{pgfscope}%
\begin{pgfscope}%
\pgfpathrectangle{\pgfqpoint{1.500000in}{0.473000in}}{\pgfqpoint{4.227273in}{3.311000in}}%
\pgfusepath{clip}%
\pgfsetbuttcap%
\pgfsetmiterjoin%
\definecolor{currentfill}{rgb}{0.121569,0.466667,0.705882}%
\pgfsetfillcolor{currentfill}%
\pgfsetlinewidth{0.000000pt}%
\definecolor{currentstroke}{rgb}{0.000000,0.000000,0.000000}%
\pgfsetstrokecolor{currentstroke}%
\pgfsetstrokeopacity{0.000000}%
\pgfsetdash{}{0pt}%
\pgfpathmoveto{\pgfqpoint{1.940885in}{2.252476in}}%
\pgfpathlineto{\pgfqpoint{2.052816in}{2.252476in}}%
\pgfpathlineto{\pgfqpoint{2.052816in}{1.123423in}}%
\pgfpathlineto{\pgfqpoint{1.940885in}{1.123423in}}%
\pgfpathclose%
\pgfusepath{fill}%
\end{pgfscope}%
\begin{pgfscope}%
\pgfpathrectangle{\pgfqpoint{1.500000in}{0.473000in}}{\pgfqpoint{4.227273in}{3.311000in}}%
\pgfusepath{clip}%
\pgfsetbuttcap%
\pgfsetmiterjoin%
\definecolor{currentfill}{rgb}{0.121569,0.466667,0.705882}%
\pgfsetfillcolor{currentfill}%
\pgfsetlinewidth{0.000000pt}%
\definecolor{currentstroke}{rgb}{0.000000,0.000000,0.000000}%
\pgfsetstrokecolor{currentstroke}%
\pgfsetstrokeopacity{0.000000}%
\pgfsetdash{}{0pt}%
\pgfpathmoveto{\pgfqpoint{2.065253in}{2.252476in}}%
\pgfpathlineto{\pgfqpoint{2.177184in}{2.252476in}}%
\pgfpathlineto{\pgfqpoint{2.177184in}{1.468442in}}%
\pgfpathlineto{\pgfqpoint{2.065253in}{1.468442in}}%
\pgfpathclose%
\pgfusepath{fill}%
\end{pgfscope}%
\begin{pgfscope}%
\pgfpathrectangle{\pgfqpoint{1.500000in}{0.473000in}}{\pgfqpoint{4.227273in}{3.311000in}}%
\pgfusepath{clip}%
\pgfsetbuttcap%
\pgfsetmiterjoin%
\definecolor{currentfill}{rgb}{0.121569,0.466667,0.705882}%
\pgfsetfillcolor{currentfill}%
\pgfsetlinewidth{0.000000pt}%
\definecolor{currentstroke}{rgb}{0.000000,0.000000,0.000000}%
\pgfsetstrokecolor{currentstroke}%
\pgfsetstrokeopacity{0.000000}%
\pgfsetdash{}{0pt}%
\pgfpathmoveto{\pgfqpoint{2.189621in}{2.252476in}}%
\pgfpathlineto{\pgfqpoint{2.301553in}{2.252476in}}%
\pgfpathlineto{\pgfqpoint{2.301553in}{2.042913in}}%
\pgfpathlineto{\pgfqpoint{2.189621in}{2.042913in}}%
\pgfpathclose%
\pgfusepath{fill}%
\end{pgfscope}%
\begin{pgfscope}%
\pgfpathrectangle{\pgfqpoint{1.500000in}{0.473000in}}{\pgfqpoint{4.227273in}{3.311000in}}%
\pgfusepath{clip}%
\pgfsetbuttcap%
\pgfsetmiterjoin%
\definecolor{currentfill}{rgb}{0.121569,0.466667,0.705882}%
\pgfsetfillcolor{currentfill}%
\pgfsetlinewidth{0.000000pt}%
\definecolor{currentstroke}{rgb}{0.000000,0.000000,0.000000}%
\pgfsetstrokecolor{currentstroke}%
\pgfsetstrokeopacity{0.000000}%
\pgfsetdash{}{0pt}%
\pgfpathmoveto{\pgfqpoint{2.313989in}{2.252476in}}%
\pgfpathlineto{\pgfqpoint{2.425921in}{2.252476in}}%
\pgfpathlineto{\pgfqpoint{2.425921in}{2.369867in}}%
\pgfpathlineto{\pgfqpoint{2.313989in}{2.369867in}}%
\pgfpathclose%
\pgfusepath{fill}%
\end{pgfscope}%
\begin{pgfscope}%
\pgfpathrectangle{\pgfqpoint{1.500000in}{0.473000in}}{\pgfqpoint{4.227273in}{3.311000in}}%
\pgfusepath{clip}%
\pgfsetbuttcap%
\pgfsetmiterjoin%
\definecolor{currentfill}{rgb}{0.121569,0.466667,0.705882}%
\pgfsetfillcolor{currentfill}%
\pgfsetlinewidth{0.000000pt}%
\definecolor{currentstroke}{rgb}{0.000000,0.000000,0.000000}%
\pgfsetstrokecolor{currentstroke}%
\pgfsetstrokeopacity{0.000000}%
\pgfsetdash{}{0pt}%
\pgfpathmoveto{\pgfqpoint{2.438358in}{2.252476in}}%
\pgfpathlineto{\pgfqpoint{2.550289in}{2.252476in}}%
\pgfpathlineto{\pgfqpoint{2.550289in}{2.492095in}}%
\pgfpathlineto{\pgfqpoint{2.438358in}{2.492095in}}%
\pgfpathclose%
\pgfusepath{fill}%
\end{pgfscope}%
\begin{pgfscope}%
\pgfpathrectangle{\pgfqpoint{1.500000in}{0.473000in}}{\pgfqpoint{4.227273in}{3.311000in}}%
\pgfusepath{clip}%
\pgfsetbuttcap%
\pgfsetmiterjoin%
\definecolor{currentfill}{rgb}{0.121569,0.466667,0.705882}%
\pgfsetfillcolor{currentfill}%
\pgfsetlinewidth{0.000000pt}%
\definecolor{currentstroke}{rgb}{0.000000,0.000000,0.000000}%
\pgfsetstrokecolor{currentstroke}%
\pgfsetstrokeopacity{0.000000}%
\pgfsetdash{}{0pt}%
\pgfpathmoveto{\pgfqpoint{2.562726in}{2.252476in}}%
\pgfpathlineto{\pgfqpoint{2.674657in}{2.252476in}}%
\pgfpathlineto{\pgfqpoint{2.674657in}{2.693551in}}%
\pgfpathlineto{\pgfqpoint{2.562726in}{2.693551in}}%
\pgfpathclose%
\pgfusepath{fill}%
\end{pgfscope}%
\begin{pgfscope}%
\pgfpathrectangle{\pgfqpoint{1.500000in}{0.473000in}}{\pgfqpoint{4.227273in}{3.311000in}}%
\pgfusepath{clip}%
\pgfsetbuttcap%
\pgfsetmiterjoin%
\definecolor{currentfill}{rgb}{0.121569,0.466667,0.705882}%
\pgfsetfillcolor{currentfill}%
\pgfsetlinewidth{0.000000pt}%
\definecolor{currentstroke}{rgb}{0.000000,0.000000,0.000000}%
\pgfsetstrokecolor{currentstroke}%
\pgfsetstrokeopacity{0.000000}%
\pgfsetdash{}{0pt}%
\pgfpathmoveto{\pgfqpoint{2.687094in}{2.252476in}}%
\pgfpathlineto{\pgfqpoint{2.799025in}{2.252476in}}%
\pgfpathlineto{\pgfqpoint{2.799025in}{3.442891in}}%
\pgfpathlineto{\pgfqpoint{2.687094in}{3.442891in}}%
\pgfpathclose%
\pgfusepath{fill}%
\end{pgfscope}%
\begin{pgfscope}%
\pgfpathrectangle{\pgfqpoint{1.500000in}{0.473000in}}{\pgfqpoint{4.227273in}{3.311000in}}%
\pgfusepath{clip}%
\pgfsetbuttcap%
\pgfsetmiterjoin%
\definecolor{currentfill}{rgb}{0.121569,0.466667,0.705882}%
\pgfsetfillcolor{currentfill}%
\pgfsetlinewidth{0.000000pt}%
\definecolor{currentstroke}{rgb}{0.000000,0.000000,0.000000}%
\pgfsetstrokecolor{currentstroke}%
\pgfsetstrokeopacity{0.000000}%
\pgfsetdash{}{0pt}%
\pgfpathmoveto{\pgfqpoint{2.811462in}{2.252476in}}%
\pgfpathlineto{\pgfqpoint{2.923393in}{2.252476in}}%
\pgfpathlineto{\pgfqpoint{2.923393in}{3.536968in}}%
\pgfpathlineto{\pgfqpoint{2.811462in}{3.536968in}}%
\pgfpathclose%
\pgfusepath{fill}%
\end{pgfscope}%
\begin{pgfscope}%
\pgfpathrectangle{\pgfqpoint{1.500000in}{0.473000in}}{\pgfqpoint{4.227273in}{3.311000in}}%
\pgfusepath{clip}%
\pgfsetbuttcap%
\pgfsetmiterjoin%
\definecolor{currentfill}{rgb}{0.121569,0.466667,0.705882}%
\pgfsetfillcolor{currentfill}%
\pgfsetlinewidth{0.000000pt}%
\definecolor{currentstroke}{rgb}{0.000000,0.000000,0.000000}%
\pgfsetstrokecolor{currentstroke}%
\pgfsetstrokeopacity{0.000000}%
\pgfsetdash{}{0pt}%
\pgfpathmoveto{\pgfqpoint{2.935830in}{2.252476in}}%
\pgfpathlineto{\pgfqpoint{3.047761in}{2.252476in}}%
\pgfpathlineto{\pgfqpoint{3.047761in}{2.597500in}}%
\pgfpathlineto{\pgfqpoint{2.935830in}{2.597500in}}%
\pgfpathclose%
\pgfusepath{fill}%
\end{pgfscope}%
\begin{pgfscope}%
\pgfpathrectangle{\pgfqpoint{1.500000in}{0.473000in}}{\pgfqpoint{4.227273in}{3.311000in}}%
\pgfusepath{clip}%
\pgfsetbuttcap%
\pgfsetmiterjoin%
\definecolor{currentfill}{rgb}{0.121569,0.466667,0.705882}%
\pgfsetfillcolor{currentfill}%
\pgfsetlinewidth{0.000000pt}%
\definecolor{currentstroke}{rgb}{0.000000,0.000000,0.000000}%
\pgfsetstrokecolor{currentstroke}%
\pgfsetstrokeopacity{0.000000}%
\pgfsetdash{}{0pt}%
\pgfpathmoveto{\pgfqpoint{3.060198in}{2.252476in}}%
\pgfpathlineto{\pgfqpoint{3.172130in}{2.252476in}}%
\pgfpathlineto{\pgfqpoint{3.172130in}{2.704198in}}%
\pgfpathlineto{\pgfqpoint{3.060198in}{2.704198in}}%
\pgfpathclose%
\pgfusepath{fill}%
\end{pgfscope}%
\begin{pgfscope}%
\pgfpathrectangle{\pgfqpoint{1.500000in}{0.473000in}}{\pgfqpoint{4.227273in}{3.311000in}}%
\pgfusepath{clip}%
\pgfsetbuttcap%
\pgfsetmiterjoin%
\definecolor{currentfill}{rgb}{0.121569,0.466667,0.705882}%
\pgfsetfillcolor{currentfill}%
\pgfsetlinewidth{0.000000pt}%
\definecolor{currentstroke}{rgb}{0.000000,0.000000,0.000000}%
\pgfsetstrokecolor{currentstroke}%
\pgfsetstrokeopacity{0.000000}%
\pgfsetdash{}{0pt}%
\pgfpathmoveto{\pgfqpoint{3.184566in}{2.252476in}}%
\pgfpathlineto{\pgfqpoint{3.296498in}{2.252476in}}%
\pgfpathlineto{\pgfqpoint{3.296498in}{3.365428in}}%
\pgfpathlineto{\pgfqpoint{3.184566in}{3.365428in}}%
\pgfpathclose%
\pgfusepath{fill}%
\end{pgfscope}%
\begin{pgfscope}%
\pgfpathrectangle{\pgfqpoint{1.500000in}{0.473000in}}{\pgfqpoint{4.227273in}{3.311000in}}%
\pgfusepath{clip}%
\pgfsetbuttcap%
\pgfsetmiterjoin%
\definecolor{currentfill}{rgb}{0.121569,0.466667,0.705882}%
\pgfsetfillcolor{currentfill}%
\pgfsetlinewidth{0.000000pt}%
\definecolor{currentstroke}{rgb}{0.000000,0.000000,0.000000}%
\pgfsetstrokecolor{currentstroke}%
\pgfsetstrokeopacity{0.000000}%
\pgfsetdash{}{0pt}%
\pgfpathmoveto{\pgfqpoint{3.308934in}{2.252476in}}%
\pgfpathlineto{\pgfqpoint{3.420866in}{2.252476in}}%
\pgfpathlineto{\pgfqpoint{3.420866in}{3.385433in}}%
\pgfpathlineto{\pgfqpoint{3.308934in}{3.385433in}}%
\pgfpathclose%
\pgfusepath{fill}%
\end{pgfscope}%
\begin{pgfscope}%
\pgfpathrectangle{\pgfqpoint{1.500000in}{0.473000in}}{\pgfqpoint{4.227273in}{3.311000in}}%
\pgfusepath{clip}%
\pgfsetbuttcap%
\pgfsetmiterjoin%
\definecolor{currentfill}{rgb}{0.121569,0.466667,0.705882}%
\pgfsetfillcolor{currentfill}%
\pgfsetlinewidth{0.000000pt}%
\definecolor{currentstroke}{rgb}{0.000000,0.000000,0.000000}%
\pgfsetstrokecolor{currentstroke}%
\pgfsetstrokeopacity{0.000000}%
\pgfsetdash{}{0pt}%
\pgfpathmoveto{\pgfqpoint{3.433303in}{2.252476in}}%
\pgfpathlineto{\pgfqpoint{3.545234in}{2.252476in}}%
\pgfpathlineto{\pgfqpoint{3.545234in}{3.433286in}}%
\pgfpathlineto{\pgfqpoint{3.433303in}{3.433286in}}%
\pgfpathclose%
\pgfusepath{fill}%
\end{pgfscope}%
\begin{pgfscope}%
\pgfpathrectangle{\pgfqpoint{1.500000in}{0.473000in}}{\pgfqpoint{4.227273in}{3.311000in}}%
\pgfusepath{clip}%
\pgfsetbuttcap%
\pgfsetmiterjoin%
\definecolor{currentfill}{rgb}{0.121569,0.466667,0.705882}%
\pgfsetfillcolor{currentfill}%
\pgfsetlinewidth{0.000000pt}%
\definecolor{currentstroke}{rgb}{0.000000,0.000000,0.000000}%
\pgfsetstrokecolor{currentstroke}%
\pgfsetstrokeopacity{0.000000}%
\pgfsetdash{}{0pt}%
\pgfpathmoveto{\pgfqpoint{3.557671in}{2.252476in}}%
\pgfpathlineto{\pgfqpoint{3.669602in}{2.252476in}}%
\pgfpathlineto{\pgfqpoint{3.669602in}{3.498877in}}%
\pgfpathlineto{\pgfqpoint{3.557671in}{3.498877in}}%
\pgfpathclose%
\pgfusepath{fill}%
\end{pgfscope}%
\begin{pgfscope}%
\pgfpathrectangle{\pgfqpoint{1.500000in}{0.473000in}}{\pgfqpoint{4.227273in}{3.311000in}}%
\pgfusepath{clip}%
\pgfsetbuttcap%
\pgfsetmiterjoin%
\definecolor{currentfill}{rgb}{0.121569,0.466667,0.705882}%
\pgfsetfillcolor{currentfill}%
\pgfsetlinewidth{0.000000pt}%
\definecolor{currentstroke}{rgb}{0.000000,0.000000,0.000000}%
\pgfsetstrokecolor{currentstroke}%
\pgfsetstrokeopacity{0.000000}%
\pgfsetdash{}{0pt}%
\pgfpathmoveto{\pgfqpoint{3.682039in}{2.252476in}}%
\pgfpathlineto{\pgfqpoint{3.793970in}{2.252476in}}%
\pgfpathlineto{\pgfqpoint{3.793970in}{3.518352in}}%
\pgfpathlineto{\pgfqpoint{3.682039in}{3.518352in}}%
\pgfpathclose%
\pgfusepath{fill}%
\end{pgfscope}%
\begin{pgfscope}%
\pgfpathrectangle{\pgfqpoint{1.500000in}{0.473000in}}{\pgfqpoint{4.227273in}{3.311000in}}%
\pgfusepath{clip}%
\pgfsetbuttcap%
\pgfsetmiterjoin%
\definecolor{currentfill}{rgb}{0.121569,0.466667,0.705882}%
\pgfsetfillcolor{currentfill}%
\pgfsetlinewidth{0.000000pt}%
\definecolor{currentstroke}{rgb}{0.000000,0.000000,0.000000}%
\pgfsetstrokecolor{currentstroke}%
\pgfsetstrokeopacity{0.000000}%
\pgfsetdash{}{0pt}%
\pgfpathmoveto{\pgfqpoint{3.806407in}{2.252476in}}%
\pgfpathlineto{\pgfqpoint{3.918338in}{2.252476in}}%
\pgfpathlineto{\pgfqpoint{3.918338in}{3.505580in}}%
\pgfpathlineto{\pgfqpoint{3.806407in}{3.505580in}}%
\pgfpathclose%
\pgfusepath{fill}%
\end{pgfscope}%
\begin{pgfscope}%
\pgfpathrectangle{\pgfqpoint{1.500000in}{0.473000in}}{\pgfqpoint{4.227273in}{3.311000in}}%
\pgfusepath{clip}%
\pgfsetbuttcap%
\pgfsetmiterjoin%
\definecolor{currentfill}{rgb}{0.121569,0.466667,0.705882}%
\pgfsetfillcolor{currentfill}%
\pgfsetlinewidth{0.000000pt}%
\definecolor{currentstroke}{rgb}{0.000000,0.000000,0.000000}%
\pgfsetstrokecolor{currentstroke}%
\pgfsetstrokeopacity{0.000000}%
\pgfsetdash{}{0pt}%
\pgfpathmoveto{\pgfqpoint{3.930775in}{2.252476in}}%
\pgfpathlineto{\pgfqpoint{4.042706in}{2.252476in}}%
\pgfpathlineto{\pgfqpoint{4.042706in}{3.524693in}}%
\pgfpathlineto{\pgfqpoint{3.930775in}{3.524693in}}%
\pgfpathclose%
\pgfusepath{fill}%
\end{pgfscope}%
\begin{pgfscope}%
\pgfpathrectangle{\pgfqpoint{1.500000in}{0.473000in}}{\pgfqpoint{4.227273in}{3.311000in}}%
\pgfusepath{clip}%
\pgfsetbuttcap%
\pgfsetmiterjoin%
\definecolor{currentfill}{rgb}{0.121569,0.466667,0.705882}%
\pgfsetfillcolor{currentfill}%
\pgfsetlinewidth{0.000000pt}%
\definecolor{currentstroke}{rgb}{0.000000,0.000000,0.000000}%
\pgfsetstrokecolor{currentstroke}%
\pgfsetstrokeopacity{0.000000}%
\pgfsetdash{}{0pt}%
\pgfpathmoveto{\pgfqpoint{4.055143in}{2.252476in}}%
\pgfpathlineto{\pgfqpoint{4.167075in}{2.252476in}}%
\pgfpathlineto{\pgfqpoint{4.167075in}{3.561078in}}%
\pgfpathlineto{\pgfqpoint{4.055143in}{3.561078in}}%
\pgfpathclose%
\pgfusepath{fill}%
\end{pgfscope}%
\begin{pgfscope}%
\pgfpathrectangle{\pgfqpoint{1.500000in}{0.473000in}}{\pgfqpoint{4.227273in}{3.311000in}}%
\pgfusepath{clip}%
\pgfsetbuttcap%
\pgfsetmiterjoin%
\definecolor{currentfill}{rgb}{0.121569,0.466667,0.705882}%
\pgfsetfillcolor{currentfill}%
\pgfsetlinewidth{0.000000pt}%
\definecolor{currentstroke}{rgb}{0.000000,0.000000,0.000000}%
\pgfsetstrokecolor{currentstroke}%
\pgfsetstrokeopacity{0.000000}%
\pgfsetdash{}{0pt}%
\pgfpathmoveto{\pgfqpoint{4.179511in}{2.252476in}}%
\pgfpathlineto{\pgfqpoint{4.291443in}{2.252476in}}%
\pgfpathlineto{\pgfqpoint{4.291443in}{3.567353in}}%
\pgfpathlineto{\pgfqpoint{4.179511in}{3.567353in}}%
\pgfpathclose%
\pgfusepath{fill}%
\end{pgfscope}%
\begin{pgfscope}%
\pgfpathrectangle{\pgfqpoint{1.500000in}{0.473000in}}{\pgfqpoint{4.227273in}{3.311000in}}%
\pgfusepath{clip}%
\pgfsetbuttcap%
\pgfsetmiterjoin%
\definecolor{currentfill}{rgb}{0.121569,0.466667,0.705882}%
\pgfsetfillcolor{currentfill}%
\pgfsetlinewidth{0.000000pt}%
\definecolor{currentstroke}{rgb}{0.000000,0.000000,0.000000}%
\pgfsetstrokecolor{currentstroke}%
\pgfsetstrokeopacity{0.000000}%
\pgfsetdash{}{0pt}%
\pgfpathmoveto{\pgfqpoint{4.303879in}{2.252476in}}%
\pgfpathlineto{\pgfqpoint{4.415811in}{2.252476in}}%
\pgfpathlineto{\pgfqpoint{4.415811in}{3.616329in}}%
\pgfpathlineto{\pgfqpoint{4.303879in}{3.616329in}}%
\pgfpathclose%
\pgfusepath{fill}%
\end{pgfscope}%
\begin{pgfscope}%
\pgfpathrectangle{\pgfqpoint{1.500000in}{0.473000in}}{\pgfqpoint{4.227273in}{3.311000in}}%
\pgfusepath{clip}%
\pgfsetbuttcap%
\pgfsetmiterjoin%
\definecolor{currentfill}{rgb}{0.121569,0.466667,0.705882}%
\pgfsetfillcolor{currentfill}%
\pgfsetlinewidth{0.000000pt}%
\definecolor{currentstroke}{rgb}{0.000000,0.000000,0.000000}%
\pgfsetstrokecolor{currentstroke}%
\pgfsetstrokeopacity{0.000000}%
\pgfsetdash{}{0pt}%
\pgfpathmoveto{\pgfqpoint{4.428248in}{2.252476in}}%
\pgfpathlineto{\pgfqpoint{4.540179in}{2.252476in}}%
\pgfpathlineto{\pgfqpoint{4.540179in}{3.596951in}}%
\pgfpathlineto{\pgfqpoint{4.428248in}{3.596951in}}%
\pgfpathclose%
\pgfusepath{fill}%
\end{pgfscope}%
\begin{pgfscope}%
\pgfpathrectangle{\pgfqpoint{1.500000in}{0.473000in}}{\pgfqpoint{4.227273in}{3.311000in}}%
\pgfusepath{clip}%
\pgfsetbuttcap%
\pgfsetmiterjoin%
\definecolor{currentfill}{rgb}{0.121569,0.466667,0.705882}%
\pgfsetfillcolor{currentfill}%
\pgfsetlinewidth{0.000000pt}%
\definecolor{currentstroke}{rgb}{0.000000,0.000000,0.000000}%
\pgfsetstrokecolor{currentstroke}%
\pgfsetstrokeopacity{0.000000}%
\pgfsetdash{}{0pt}%
\pgfpathmoveto{\pgfqpoint{4.552616in}{2.252476in}}%
\pgfpathlineto{\pgfqpoint{4.664547in}{2.252476in}}%
\pgfpathlineto{\pgfqpoint{4.664547in}{3.568454in}}%
\pgfpathlineto{\pgfqpoint{4.552616in}{3.568454in}}%
\pgfpathclose%
\pgfusepath{fill}%
\end{pgfscope}%
\begin{pgfscope}%
\pgfpathrectangle{\pgfqpoint{1.500000in}{0.473000in}}{\pgfqpoint{4.227273in}{3.311000in}}%
\pgfusepath{clip}%
\pgfsetbuttcap%
\pgfsetmiterjoin%
\definecolor{currentfill}{rgb}{0.121569,0.466667,0.705882}%
\pgfsetfillcolor{currentfill}%
\pgfsetlinewidth{0.000000pt}%
\definecolor{currentstroke}{rgb}{0.000000,0.000000,0.000000}%
\pgfsetstrokecolor{currentstroke}%
\pgfsetstrokeopacity{0.000000}%
\pgfsetdash{}{0pt}%
\pgfpathmoveto{\pgfqpoint{4.676984in}{2.252476in}}%
\pgfpathlineto{\pgfqpoint{4.788915in}{2.252476in}}%
\pgfpathlineto{\pgfqpoint{4.788915in}{3.579554in}}%
\pgfpathlineto{\pgfqpoint{4.676984in}{3.579554in}}%
\pgfpathclose%
\pgfusepath{fill}%
\end{pgfscope}%
\begin{pgfscope}%
\pgfpathrectangle{\pgfqpoint{1.500000in}{0.473000in}}{\pgfqpoint{4.227273in}{3.311000in}}%
\pgfusepath{clip}%
\pgfsetbuttcap%
\pgfsetmiterjoin%
\definecolor{currentfill}{rgb}{0.121569,0.466667,0.705882}%
\pgfsetfillcolor{currentfill}%
\pgfsetlinewidth{0.000000pt}%
\definecolor{currentstroke}{rgb}{0.000000,0.000000,0.000000}%
\pgfsetstrokecolor{currentstroke}%
\pgfsetstrokeopacity{0.000000}%
\pgfsetdash{}{0pt}%
\pgfpathmoveto{\pgfqpoint{4.801352in}{2.252476in}}%
\pgfpathlineto{\pgfqpoint{4.913283in}{2.252476in}}%
\pgfpathlineto{\pgfqpoint{4.913283in}{3.580456in}}%
\pgfpathlineto{\pgfqpoint{4.801352in}{3.580456in}}%
\pgfpathclose%
\pgfusepath{fill}%
\end{pgfscope}%
\begin{pgfscope}%
\pgfpathrectangle{\pgfqpoint{1.500000in}{0.473000in}}{\pgfqpoint{4.227273in}{3.311000in}}%
\pgfusepath{clip}%
\pgfsetbuttcap%
\pgfsetmiterjoin%
\definecolor{currentfill}{rgb}{1.000000,0.000000,0.000000}%
\pgfsetfillcolor{currentfill}%
\pgfsetlinewidth{1.003750pt}%
\definecolor{currentstroke}{rgb}{1.000000,0.000000,0.000000}%
\pgfsetstrokecolor{currentstroke}%
\pgfsetdash{}{0pt}%
\pgfpathmoveto{\pgfqpoint{4.925720in}{2.252476in}}%
\pgfpathlineto{\pgfqpoint{5.037651in}{2.252476in}}%
\pgfpathlineto{\pgfqpoint{5.037651in}{3.633500in}}%
\pgfpathlineto{\pgfqpoint{4.925720in}{3.633500in}}%
\pgfpathclose%
\pgfusepath{stroke,fill}%
\end{pgfscope}%
\begin{pgfscope}%
\pgfpathrectangle{\pgfqpoint{1.500000in}{0.473000in}}{\pgfqpoint{4.227273in}{3.311000in}}%
\pgfusepath{clip}%
\pgfsetbuttcap%
\pgfsetmiterjoin%
\definecolor{currentfill}{rgb}{0.121569,0.466667,0.705882}%
\pgfsetfillcolor{currentfill}%
\pgfsetlinewidth{0.000000pt}%
\definecolor{currentstroke}{rgb}{0.000000,0.000000,0.000000}%
\pgfsetstrokecolor{currentstroke}%
\pgfsetstrokeopacity{0.000000}%
\pgfsetdash{}{0pt}%
\pgfpathmoveto{\pgfqpoint{5.050088in}{2.252476in}}%
\pgfpathlineto{\pgfqpoint{5.162020in}{2.252476in}}%
\pgfpathlineto{\pgfqpoint{5.162020in}{2.918627in}}%
\pgfpathlineto{\pgfqpoint{5.050088in}{2.918627in}}%
\pgfpathclose%
\pgfusepath{fill}%
\end{pgfscope}%
\begin{pgfscope}%
\pgfpathrectangle{\pgfqpoint{1.500000in}{0.473000in}}{\pgfqpoint{4.227273in}{3.311000in}}%
\pgfusepath{clip}%
\pgfsetbuttcap%
\pgfsetmiterjoin%
\definecolor{currentfill}{rgb}{0.121569,0.466667,0.705882}%
\pgfsetfillcolor{currentfill}%
\pgfsetlinewidth{0.000000pt}%
\definecolor{currentstroke}{rgb}{0.000000,0.000000,0.000000}%
\pgfsetstrokecolor{currentstroke}%
\pgfsetstrokeopacity{0.000000}%
\pgfsetdash{}{0pt}%
\pgfpathmoveto{\pgfqpoint{5.174456in}{2.252476in}}%
\pgfpathlineto{\pgfqpoint{5.286388in}{2.252476in}}%
\pgfpathlineto{\pgfqpoint{5.286388in}{2.972741in}}%
\pgfpathlineto{\pgfqpoint{5.174456in}{2.972741in}}%
\pgfpathclose%
\pgfusepath{fill}%
\end{pgfscope}%
\begin{pgfscope}%
\pgfpathrectangle{\pgfqpoint{1.500000in}{0.473000in}}{\pgfqpoint{4.227273in}{3.311000in}}%
\pgfusepath{clip}%
\pgfsetbuttcap%
\pgfsetmiterjoin%
\definecolor{currentfill}{rgb}{0.121569,0.466667,0.705882}%
\pgfsetfillcolor{currentfill}%
\pgfsetlinewidth{0.000000pt}%
\definecolor{currentstroke}{rgb}{0.000000,0.000000,0.000000}%
\pgfsetstrokecolor{currentstroke}%
\pgfsetstrokeopacity{0.000000}%
\pgfsetdash{}{0pt}%
\pgfpathmoveto{\pgfqpoint{5.298825in}{2.252476in}}%
\pgfpathlineto{\pgfqpoint{5.410756in}{2.252476in}}%
\pgfpathlineto{\pgfqpoint{5.410756in}{2.946536in}}%
\pgfpathlineto{\pgfqpoint{5.298825in}{2.946536in}}%
\pgfpathclose%
\pgfusepath{fill}%
\end{pgfscope}%
\begin{pgfscope}%
\pgfpathrectangle{\pgfqpoint{1.500000in}{0.473000in}}{\pgfqpoint{4.227273in}{3.311000in}}%
\pgfusepath{clip}%
\pgfsetbuttcap%
\pgfsetmiterjoin%
\definecolor{currentfill}{rgb}{0.121569,0.466667,0.705882}%
\pgfsetfillcolor{currentfill}%
\pgfsetlinewidth{0.000000pt}%
\definecolor{currentstroke}{rgb}{0.000000,0.000000,0.000000}%
\pgfsetstrokecolor{currentstroke}%
\pgfsetstrokeopacity{0.000000}%
\pgfsetdash{}{0pt}%
\pgfpathmoveto{\pgfqpoint{5.423193in}{2.252476in}}%
\pgfpathlineto{\pgfqpoint{5.535124in}{2.252476in}}%
\pgfpathlineto{\pgfqpoint{5.535124in}{3.027604in}}%
\pgfpathlineto{\pgfqpoint{5.423193in}{3.027604in}}%
\pgfpathclose%
\pgfusepath{fill}%
\end{pgfscope}%
\begin{pgfscope}%
\pgfsetbuttcap%
\pgfsetroundjoin%
\definecolor{currentfill}{rgb}{0.000000,0.000000,0.000000}%
\pgfsetfillcolor{currentfill}%
\pgfsetlinewidth{0.803000pt}%
\definecolor{currentstroke}{rgb}{0.000000,0.000000,0.000000}%
\pgfsetstrokecolor{currentstroke}%
\pgfsetdash{}{0pt}%
\pgfsys@defobject{currentmarker}{\pgfqpoint{0.000000in}{-0.048611in}}{\pgfqpoint{0.000000in}{0.000000in}}{%
\pgfpathmoveto{\pgfqpoint{0.000000in}{0.000000in}}%
\pgfpathlineto{\pgfqpoint{0.000000in}{-0.048611in}}%
\pgfusepath{stroke,fill}%
}%
\begin{pgfscope}%
\pgfsys@transformshift{1.748114in}{0.473000in}%
\pgfsys@useobject{currentmarker}{}%
\end{pgfscope}%
\end{pgfscope}%
\begin{pgfscope}%
\definecolor{textcolor}{rgb}{0.000000,0.000000,0.000000}%
\pgfsetstrokecolor{textcolor}%
\pgfsetfillcolor{textcolor}%
\pgftext[x=1.748114in,y=0.375778in,,top]{\color{textcolor}\sffamily\fontsize{10.000000}{12.000000}\selectfont 0.10}%
\end{pgfscope}%
\begin{pgfscope}%
\pgfsetbuttcap%
\pgfsetroundjoin%
\definecolor{currentfill}{rgb}{0.000000,0.000000,0.000000}%
\pgfsetfillcolor{currentfill}%
\pgfsetlinewidth{0.803000pt}%
\definecolor{currentstroke}{rgb}{0.000000,0.000000,0.000000}%
\pgfsetstrokecolor{currentstroke}%
\pgfsetdash{}{0pt}%
\pgfsys@defobject{currentmarker}{\pgfqpoint{0.000000in}{-0.048611in}}{\pgfqpoint{0.000000in}{0.000000in}}{%
\pgfpathmoveto{\pgfqpoint{0.000000in}{0.000000in}}%
\pgfpathlineto{\pgfqpoint{0.000000in}{-0.048611in}}%
\pgfusepath{stroke,fill}%
}%
\begin{pgfscope}%
\pgfsys@transformshift{2.369955in}{0.473000in}%
\pgfsys@useobject{currentmarker}{}%
\end{pgfscope}%
\end{pgfscope}%
\begin{pgfscope}%
\definecolor{textcolor}{rgb}{0.000000,0.000000,0.000000}%
\pgfsetstrokecolor{textcolor}%
\pgfsetfillcolor{textcolor}%
\pgftext[x=2.369955in,y=0.375778in,,top]{\color{textcolor}\sffamily\fontsize{10.000000}{12.000000}\selectfont 0.15}%
\end{pgfscope}%
\begin{pgfscope}%
\pgfsetbuttcap%
\pgfsetroundjoin%
\definecolor{currentfill}{rgb}{0.000000,0.000000,0.000000}%
\pgfsetfillcolor{currentfill}%
\pgfsetlinewidth{0.803000pt}%
\definecolor{currentstroke}{rgb}{0.000000,0.000000,0.000000}%
\pgfsetstrokecolor{currentstroke}%
\pgfsetdash{}{0pt}%
\pgfsys@defobject{currentmarker}{\pgfqpoint{0.000000in}{-0.048611in}}{\pgfqpoint{0.000000in}{0.000000in}}{%
\pgfpathmoveto{\pgfqpoint{0.000000in}{0.000000in}}%
\pgfpathlineto{\pgfqpoint{0.000000in}{-0.048611in}}%
\pgfusepath{stroke,fill}%
}%
\begin{pgfscope}%
\pgfsys@transformshift{2.991796in}{0.473000in}%
\pgfsys@useobject{currentmarker}{}%
\end{pgfscope}%
\end{pgfscope}%
\begin{pgfscope}%
\definecolor{textcolor}{rgb}{0.000000,0.000000,0.000000}%
\pgfsetstrokecolor{textcolor}%
\pgfsetfillcolor{textcolor}%
\pgftext[x=2.991796in,y=0.375778in,,top]{\color{textcolor}\sffamily\fontsize{10.000000}{12.000000}\selectfont 0.20}%
\end{pgfscope}%
\begin{pgfscope}%
\pgfsetbuttcap%
\pgfsetroundjoin%
\definecolor{currentfill}{rgb}{0.000000,0.000000,0.000000}%
\pgfsetfillcolor{currentfill}%
\pgfsetlinewidth{0.803000pt}%
\definecolor{currentstroke}{rgb}{0.000000,0.000000,0.000000}%
\pgfsetstrokecolor{currentstroke}%
\pgfsetdash{}{0pt}%
\pgfsys@defobject{currentmarker}{\pgfqpoint{0.000000in}{-0.048611in}}{\pgfqpoint{0.000000in}{0.000000in}}{%
\pgfpathmoveto{\pgfqpoint{0.000000in}{0.000000in}}%
\pgfpathlineto{\pgfqpoint{0.000000in}{-0.048611in}}%
\pgfusepath{stroke,fill}%
}%
\begin{pgfscope}%
\pgfsys@transformshift{3.613636in}{0.473000in}%
\pgfsys@useobject{currentmarker}{}%
\end{pgfscope}%
\end{pgfscope}%
\begin{pgfscope}%
\definecolor{textcolor}{rgb}{0.000000,0.000000,0.000000}%
\pgfsetstrokecolor{textcolor}%
\pgfsetfillcolor{textcolor}%
\pgftext[x=3.613636in,y=0.375778in,,top]{\color{textcolor}\sffamily\fontsize{10.000000}{12.000000}\selectfont 0.25}%
\end{pgfscope}%
\begin{pgfscope}%
\pgfsetbuttcap%
\pgfsetroundjoin%
\definecolor{currentfill}{rgb}{0.000000,0.000000,0.000000}%
\pgfsetfillcolor{currentfill}%
\pgfsetlinewidth{0.803000pt}%
\definecolor{currentstroke}{rgb}{0.000000,0.000000,0.000000}%
\pgfsetstrokecolor{currentstroke}%
\pgfsetdash{}{0pt}%
\pgfsys@defobject{currentmarker}{\pgfqpoint{0.000000in}{-0.048611in}}{\pgfqpoint{0.000000in}{0.000000in}}{%
\pgfpathmoveto{\pgfqpoint{0.000000in}{0.000000in}}%
\pgfpathlineto{\pgfqpoint{0.000000in}{-0.048611in}}%
\pgfusepath{stroke,fill}%
}%
\begin{pgfscope}%
\pgfsys@transformshift{4.235477in}{0.473000in}%
\pgfsys@useobject{currentmarker}{}%
\end{pgfscope}%
\end{pgfscope}%
\begin{pgfscope}%
\definecolor{textcolor}{rgb}{0.000000,0.000000,0.000000}%
\pgfsetstrokecolor{textcolor}%
\pgfsetfillcolor{textcolor}%
\pgftext[x=4.235477in,y=0.375778in,,top]{\color{textcolor}\sffamily\fontsize{10.000000}{12.000000}\selectfont 0.30}%
\end{pgfscope}%
\begin{pgfscope}%
\pgfsetbuttcap%
\pgfsetroundjoin%
\definecolor{currentfill}{rgb}{0.000000,0.000000,0.000000}%
\pgfsetfillcolor{currentfill}%
\pgfsetlinewidth{0.803000pt}%
\definecolor{currentstroke}{rgb}{0.000000,0.000000,0.000000}%
\pgfsetstrokecolor{currentstroke}%
\pgfsetdash{}{0pt}%
\pgfsys@defobject{currentmarker}{\pgfqpoint{0.000000in}{-0.048611in}}{\pgfqpoint{0.000000in}{0.000000in}}{%
\pgfpathmoveto{\pgfqpoint{0.000000in}{0.000000in}}%
\pgfpathlineto{\pgfqpoint{0.000000in}{-0.048611in}}%
\pgfusepath{stroke,fill}%
}%
\begin{pgfscope}%
\pgfsys@transformshift{4.857318in}{0.473000in}%
\pgfsys@useobject{currentmarker}{}%
\end{pgfscope}%
\end{pgfscope}%
\begin{pgfscope}%
\definecolor{textcolor}{rgb}{0.000000,0.000000,0.000000}%
\pgfsetstrokecolor{textcolor}%
\pgfsetfillcolor{textcolor}%
\pgftext[x=4.857318in,y=0.375778in,,top]{\color{textcolor}\sffamily\fontsize{10.000000}{12.000000}\selectfont 0.35}%
\end{pgfscope}%
\begin{pgfscope}%
\pgfsetbuttcap%
\pgfsetroundjoin%
\definecolor{currentfill}{rgb}{0.000000,0.000000,0.000000}%
\pgfsetfillcolor{currentfill}%
\pgfsetlinewidth{0.803000pt}%
\definecolor{currentstroke}{rgb}{0.000000,0.000000,0.000000}%
\pgfsetstrokecolor{currentstroke}%
\pgfsetdash{}{0pt}%
\pgfsys@defobject{currentmarker}{\pgfqpoint{0.000000in}{-0.048611in}}{\pgfqpoint{0.000000in}{0.000000in}}{%
\pgfpathmoveto{\pgfqpoint{0.000000in}{0.000000in}}%
\pgfpathlineto{\pgfqpoint{0.000000in}{-0.048611in}}%
\pgfusepath{stroke,fill}%
}%
\begin{pgfscope}%
\pgfsys@transformshift{5.479158in}{0.473000in}%
\pgfsys@useobject{currentmarker}{}%
\end{pgfscope}%
\end{pgfscope}%
\begin{pgfscope}%
\definecolor{textcolor}{rgb}{0.000000,0.000000,0.000000}%
\pgfsetstrokecolor{textcolor}%
\pgfsetfillcolor{textcolor}%
\pgftext[x=5.479158in,y=0.375778in,,top]{\color{textcolor}\sffamily\fontsize{10.000000}{12.000000}\selectfont 0.40}%
\end{pgfscope}%
\begin{pgfscope}%
\definecolor{textcolor}{rgb}{0.000000,0.000000,0.000000}%
\pgfsetstrokecolor{textcolor}%
\pgfsetfillcolor{textcolor}%
\pgftext[x=3.613636in,y=0.185809in,,top]{\color{textcolor}\sffamily\fontsize{10.000000}{12.000000}\selectfont Umbral de distancia ε}%
\end{pgfscope}%
\begin{pgfscope}%
\pgfsetbuttcap%
\pgfsetroundjoin%
\definecolor{currentfill}{rgb}{0.000000,0.000000,0.000000}%
\pgfsetfillcolor{currentfill}%
\pgfsetlinewidth{0.803000pt}%
\definecolor{currentstroke}{rgb}{0.000000,0.000000,0.000000}%
\pgfsetstrokecolor{currentstroke}%
\pgfsetdash{}{0pt}%
\pgfsys@defobject{currentmarker}{\pgfqpoint{-0.048611in}{0.000000in}}{\pgfqpoint{-0.000000in}{0.000000in}}{%
\pgfpathmoveto{\pgfqpoint{-0.000000in}{0.000000in}}%
\pgfpathlineto{\pgfqpoint{-0.048611in}{0.000000in}}%
\pgfusepath{stroke,fill}%
}%
\begin{pgfscope}%
\pgfsys@transformshift{1.500000in}{1.031388in}%
\pgfsys@useobject{currentmarker}{}%
\end{pgfscope}%
\end{pgfscope}%
\begin{pgfscope}%
\definecolor{textcolor}{rgb}{0.000000,0.000000,0.000000}%
\pgfsetstrokecolor{textcolor}%
\pgfsetfillcolor{textcolor}%
\pgftext[x=1.073873in, y=0.978626in, left, base]{\color{textcolor}\sffamily\fontsize{10.000000}{12.000000}\selectfont \ensuremath{-}0.4}%
\end{pgfscope}%
\begin{pgfscope}%
\pgfsetbuttcap%
\pgfsetroundjoin%
\definecolor{currentfill}{rgb}{0.000000,0.000000,0.000000}%
\pgfsetfillcolor{currentfill}%
\pgfsetlinewidth{0.803000pt}%
\definecolor{currentstroke}{rgb}{0.000000,0.000000,0.000000}%
\pgfsetstrokecolor{currentstroke}%
\pgfsetdash{}{0pt}%
\pgfsys@defobject{currentmarker}{\pgfqpoint{-0.048611in}{0.000000in}}{\pgfqpoint{-0.000000in}{0.000000in}}{%
\pgfpathmoveto{\pgfqpoint{-0.000000in}{0.000000in}}%
\pgfpathlineto{\pgfqpoint{-0.048611in}{0.000000in}}%
\pgfusepath{stroke,fill}%
}%
\begin{pgfscope}%
\pgfsys@transformshift{1.500000in}{1.641932in}%
\pgfsys@useobject{currentmarker}{}%
\end{pgfscope}%
\end{pgfscope}%
\begin{pgfscope}%
\definecolor{textcolor}{rgb}{0.000000,0.000000,0.000000}%
\pgfsetstrokecolor{textcolor}%
\pgfsetfillcolor{textcolor}%
\pgftext[x=1.073873in, y=1.589170in, left, base]{\color{textcolor}\sffamily\fontsize{10.000000}{12.000000}\selectfont \ensuremath{-}0.2}%
\end{pgfscope}%
\begin{pgfscope}%
\pgfsetbuttcap%
\pgfsetroundjoin%
\definecolor{currentfill}{rgb}{0.000000,0.000000,0.000000}%
\pgfsetfillcolor{currentfill}%
\pgfsetlinewidth{0.803000pt}%
\definecolor{currentstroke}{rgb}{0.000000,0.000000,0.000000}%
\pgfsetstrokecolor{currentstroke}%
\pgfsetdash{}{0pt}%
\pgfsys@defobject{currentmarker}{\pgfqpoint{-0.048611in}{0.000000in}}{\pgfqpoint{-0.000000in}{0.000000in}}{%
\pgfpathmoveto{\pgfqpoint{-0.000000in}{0.000000in}}%
\pgfpathlineto{\pgfqpoint{-0.048611in}{0.000000in}}%
\pgfusepath{stroke,fill}%
}%
\begin{pgfscope}%
\pgfsys@transformshift{1.500000in}{2.252476in}%
\pgfsys@useobject{currentmarker}{}%
\end{pgfscope}%
\end{pgfscope}%
\begin{pgfscope}%
\definecolor{textcolor}{rgb}{0.000000,0.000000,0.000000}%
\pgfsetstrokecolor{textcolor}%
\pgfsetfillcolor{textcolor}%
\pgftext[x=1.181898in, y=2.199715in, left, base]{\color{textcolor}\sffamily\fontsize{10.000000}{12.000000}\selectfont 0.0}%
\end{pgfscope}%
\begin{pgfscope}%
\pgfsetbuttcap%
\pgfsetroundjoin%
\definecolor{currentfill}{rgb}{0.000000,0.000000,0.000000}%
\pgfsetfillcolor{currentfill}%
\pgfsetlinewidth{0.803000pt}%
\definecolor{currentstroke}{rgb}{0.000000,0.000000,0.000000}%
\pgfsetstrokecolor{currentstroke}%
\pgfsetdash{}{0pt}%
\pgfsys@defobject{currentmarker}{\pgfqpoint{-0.048611in}{0.000000in}}{\pgfqpoint{-0.000000in}{0.000000in}}{%
\pgfpathmoveto{\pgfqpoint{-0.000000in}{0.000000in}}%
\pgfpathlineto{\pgfqpoint{-0.048611in}{0.000000in}}%
\pgfusepath{stroke,fill}%
}%
\begin{pgfscope}%
\pgfsys@transformshift{1.500000in}{2.863020in}%
\pgfsys@useobject{currentmarker}{}%
\end{pgfscope}%
\end{pgfscope}%
\begin{pgfscope}%
\definecolor{textcolor}{rgb}{0.000000,0.000000,0.000000}%
\pgfsetstrokecolor{textcolor}%
\pgfsetfillcolor{textcolor}%
\pgftext[x=1.181898in, y=2.810259in, left, base]{\color{textcolor}\sffamily\fontsize{10.000000}{12.000000}\selectfont 0.2}%
\end{pgfscope}%
\begin{pgfscope}%
\pgfsetbuttcap%
\pgfsetroundjoin%
\definecolor{currentfill}{rgb}{0.000000,0.000000,0.000000}%
\pgfsetfillcolor{currentfill}%
\pgfsetlinewidth{0.803000pt}%
\definecolor{currentstroke}{rgb}{0.000000,0.000000,0.000000}%
\pgfsetstrokecolor{currentstroke}%
\pgfsetdash{}{0pt}%
\pgfsys@defobject{currentmarker}{\pgfqpoint{-0.048611in}{0.000000in}}{\pgfqpoint{-0.000000in}{0.000000in}}{%
\pgfpathmoveto{\pgfqpoint{-0.000000in}{0.000000in}}%
\pgfpathlineto{\pgfqpoint{-0.048611in}{0.000000in}}%
\pgfusepath{stroke,fill}%
}%
\begin{pgfscope}%
\pgfsys@transformshift{1.500000in}{3.473565in}%
\pgfsys@useobject{currentmarker}{}%
\end{pgfscope}%
\end{pgfscope}%
\begin{pgfscope}%
\definecolor{textcolor}{rgb}{0.000000,0.000000,0.000000}%
\pgfsetstrokecolor{textcolor}%
\pgfsetfillcolor{textcolor}%
\pgftext[x=1.181898in, y=3.420803in, left, base]{\color{textcolor}\sffamily\fontsize{10.000000}{12.000000}\selectfont 0.4}%
\end{pgfscope}%
\begin{pgfscope}%
\definecolor{textcolor}{rgb}{0.000000,0.000000,0.000000}%
\pgfsetstrokecolor{textcolor}%
\pgfsetfillcolor{textcolor}%
\pgftext[x=1.018318in,y=2.128500in,,bottom,rotate=90.000000]{\color{textcolor}\sffamily\fontsize{10.000000}{12.000000}\selectfont Valor medio del coeficiente de Silhouette}%
\end{pgfscope}%
\begin{pgfscope}%
\pgfsetrectcap%
\pgfsetmiterjoin%
\pgfsetlinewidth{0.803000pt}%
\definecolor{currentstroke}{rgb}{0.000000,0.000000,0.000000}%
\pgfsetstrokecolor{currentstroke}%
\pgfsetdash{}{0pt}%
\pgfpathmoveto{\pgfqpoint{1.500000in}{0.473000in}}%
\pgfpathlineto{\pgfqpoint{1.500000in}{3.784000in}}%
\pgfusepath{stroke}%
\end{pgfscope}%
\begin{pgfscope}%
\pgfsetrectcap%
\pgfsetmiterjoin%
\pgfsetlinewidth{0.803000pt}%
\definecolor{currentstroke}{rgb}{0.000000,0.000000,0.000000}%
\pgfsetstrokecolor{currentstroke}%
\pgfsetdash{}{0pt}%
\pgfpathmoveto{\pgfqpoint{5.727273in}{0.473000in}}%
\pgfpathlineto{\pgfqpoint{5.727273in}{3.784000in}}%
\pgfusepath{stroke}%
\end{pgfscope}%
\begin{pgfscope}%
\pgfsetrectcap%
\pgfsetmiterjoin%
\pgfsetlinewidth{0.803000pt}%
\definecolor{currentstroke}{rgb}{0.000000,0.000000,0.000000}%
\pgfsetstrokecolor{currentstroke}%
\pgfsetdash{}{0pt}%
\pgfpathmoveto{\pgfqpoint{1.500000in}{0.473000in}}%
\pgfpathlineto{\pgfqpoint{5.727273in}{0.473000in}}%
\pgfusepath{stroke}%
\end{pgfscope}%
\begin{pgfscope}%
\pgfsetrectcap%
\pgfsetmiterjoin%
\pgfsetlinewidth{0.803000pt}%
\definecolor{currentstroke}{rgb}{0.000000,0.000000,0.000000}%
\pgfsetstrokecolor{currentstroke}%
\pgfsetdash{}{0pt}%
\pgfpathmoveto{\pgfqpoint{1.500000in}{3.784000in}}%
\pgfpathlineto{\pgfqpoint{5.727273in}{3.784000in}}%
\pgfusepath{stroke}%
\end{pgfscope}%
\begin{pgfscope}%
\pgfsetbuttcap%
\pgfsetmiterjoin%
\definecolor{currentfill}{rgb}{1.000000,1.000000,1.000000}%
\pgfsetfillcolor{currentfill}%
\pgfsetlinewidth{0.000000pt}%
\definecolor{currentstroke}{rgb}{0.000000,0.000000,0.000000}%
\pgfsetstrokecolor{currentstroke}%
\pgfsetstrokeopacity{0.000000}%
\pgfsetdash{}{0pt}%
\pgfpathmoveto{\pgfqpoint{6.572727in}{0.473000in}}%
\pgfpathlineto{\pgfqpoint{10.800000in}{0.473000in}}%
\pgfpathlineto{\pgfqpoint{10.800000in}{3.784000in}}%
\pgfpathlineto{\pgfqpoint{6.572727in}{3.784000in}}%
\pgfpathclose%
\pgfusepath{fill}%
\end{pgfscope}%
\begin{pgfscope}%
\pgfpathrectangle{\pgfqpoint{6.572727in}{0.473000in}}{\pgfqpoint{4.227273in}{3.311000in}}%
\pgfusepath{clip}%
\pgfsetbuttcap%
\pgfsetroundjoin%
\definecolor{currentfill}{rgb}{0.127568,0.566949,0.550556}%
\pgfsetfillcolor{currentfill}%
\pgfsetfillopacity{0.700000}%
\pgfsetlinewidth{0.000000pt}%
\definecolor{currentstroke}{rgb}{0.000000,0.000000,0.000000}%
\pgfsetstrokecolor{currentstroke}%
\pgfsetstrokeopacity{0.700000}%
\pgfsetdash{}{0pt}%
\pgfpathmoveto{\pgfqpoint{7.859498in}{2.332307in}}%
\pgfpathcurveto{\pgfqpoint{7.864542in}{2.332307in}}{\pgfqpoint{7.869379in}{2.334311in}}{\pgfqpoint{7.872946in}{2.337877in}}%
\pgfpathcurveto{\pgfqpoint{7.876512in}{2.341443in}}{\pgfqpoint{7.878516in}{2.346281in}}{\pgfqpoint{7.878516in}{2.351325in}}%
\pgfpathcurveto{\pgfqpoint{7.878516in}{2.356369in}}{\pgfqpoint{7.876512in}{2.361206in}}{\pgfqpoint{7.872946in}{2.364773in}}%
\pgfpathcurveto{\pgfqpoint{7.869379in}{2.368339in}}{\pgfqpoint{7.864542in}{2.370343in}}{\pgfqpoint{7.859498in}{2.370343in}}%
\pgfpathcurveto{\pgfqpoint{7.854454in}{2.370343in}}{\pgfqpoint{7.849616in}{2.368339in}}{\pgfqpoint{7.846050in}{2.364773in}}%
\pgfpathcurveto{\pgfqpoint{7.842484in}{2.361206in}}{\pgfqpoint{7.840480in}{2.356369in}}{\pgfqpoint{7.840480in}{2.351325in}}%
\pgfpathcurveto{\pgfqpoint{7.840480in}{2.346281in}}{\pgfqpoint{7.842484in}{2.341443in}}{\pgfqpoint{7.846050in}{2.337877in}}%
\pgfpathcurveto{\pgfqpoint{7.849616in}{2.334311in}}{\pgfqpoint{7.854454in}{2.332307in}}{\pgfqpoint{7.859498in}{2.332307in}}%
\pgfpathclose%
\pgfusepath{fill}%
\end{pgfscope}%
\begin{pgfscope}%
\pgfpathrectangle{\pgfqpoint{6.572727in}{0.473000in}}{\pgfqpoint{4.227273in}{3.311000in}}%
\pgfusepath{clip}%
\pgfsetbuttcap%
\pgfsetroundjoin%
\definecolor{currentfill}{rgb}{0.127568,0.566949,0.550556}%
\pgfsetfillcolor{currentfill}%
\pgfsetfillopacity{0.700000}%
\pgfsetlinewidth{0.000000pt}%
\definecolor{currentstroke}{rgb}{0.000000,0.000000,0.000000}%
\pgfsetstrokecolor{currentstroke}%
\pgfsetstrokeopacity{0.700000}%
\pgfsetdash{}{0pt}%
\pgfpathmoveto{\pgfqpoint{8.893046in}{2.941257in}}%
\pgfpathcurveto{\pgfqpoint{8.898090in}{2.941257in}}{\pgfqpoint{8.902927in}{2.943261in}}{\pgfqpoint{8.906494in}{2.946827in}}%
\pgfpathcurveto{\pgfqpoint{8.910060in}{2.950393in}}{\pgfqpoint{8.912064in}{2.955231in}}{\pgfqpoint{8.912064in}{2.960275in}}%
\pgfpathcurveto{\pgfqpoint{8.912064in}{2.965318in}}{\pgfqpoint{8.910060in}{2.970156in}}{\pgfqpoint{8.906494in}{2.973723in}}%
\pgfpathcurveto{\pgfqpoint{8.902927in}{2.977289in}}{\pgfqpoint{8.898090in}{2.979293in}}{\pgfqpoint{8.893046in}{2.979293in}}%
\pgfpathcurveto{\pgfqpoint{8.888002in}{2.979293in}}{\pgfqpoint{8.883164in}{2.977289in}}{\pgfqpoint{8.879598in}{2.973723in}}%
\pgfpathcurveto{\pgfqpoint{8.876032in}{2.970156in}}{\pgfqpoint{8.874028in}{2.965318in}}{\pgfqpoint{8.874028in}{2.960275in}}%
\pgfpathcurveto{\pgfqpoint{8.874028in}{2.955231in}}{\pgfqpoint{8.876032in}{2.950393in}}{\pgfqpoint{8.879598in}{2.946827in}}%
\pgfpathcurveto{\pgfqpoint{8.883164in}{2.943261in}}{\pgfqpoint{8.888002in}{2.941257in}}{\pgfqpoint{8.893046in}{2.941257in}}%
\pgfpathclose%
\pgfusepath{fill}%
\end{pgfscope}%
\begin{pgfscope}%
\pgfpathrectangle{\pgfqpoint{6.572727in}{0.473000in}}{\pgfqpoint{4.227273in}{3.311000in}}%
\pgfusepath{clip}%
\pgfsetbuttcap%
\pgfsetroundjoin%
\definecolor{currentfill}{rgb}{0.993248,0.906157,0.143936}%
\pgfsetfillcolor{currentfill}%
\pgfsetfillopacity{0.700000}%
\pgfsetlinewidth{0.000000pt}%
\definecolor{currentstroke}{rgb}{0.000000,0.000000,0.000000}%
\pgfsetstrokecolor{currentstroke}%
\pgfsetstrokeopacity{0.700000}%
\pgfsetdash{}{0pt}%
\pgfpathmoveto{\pgfqpoint{9.440632in}{1.959811in}}%
\pgfpathcurveto{\pgfqpoint{9.445676in}{1.959811in}}{\pgfqpoint{9.450513in}{1.961815in}}{\pgfqpoint{9.454080in}{1.965381in}}%
\pgfpathcurveto{\pgfqpoint{9.457646in}{1.968948in}}{\pgfqpoint{9.459650in}{1.973786in}}{\pgfqpoint{9.459650in}{1.978829in}}%
\pgfpathcurveto{\pgfqpoint{9.459650in}{1.983873in}}{\pgfqpoint{9.457646in}{1.988711in}}{\pgfqpoint{9.454080in}{1.992277in}}%
\pgfpathcurveto{\pgfqpoint{9.450513in}{1.995843in}}{\pgfqpoint{9.445676in}{1.997847in}}{\pgfqpoint{9.440632in}{1.997847in}}%
\pgfpathcurveto{\pgfqpoint{9.435588in}{1.997847in}}{\pgfqpoint{9.430751in}{1.995843in}}{\pgfqpoint{9.427184in}{1.992277in}}%
\pgfpathcurveto{\pgfqpoint{9.423618in}{1.988711in}}{\pgfqpoint{9.421614in}{1.983873in}}{\pgfqpoint{9.421614in}{1.978829in}}%
\pgfpathcurveto{\pgfqpoint{9.421614in}{1.973786in}}{\pgfqpoint{9.423618in}{1.968948in}}{\pgfqpoint{9.427184in}{1.965381in}}%
\pgfpathcurveto{\pgfqpoint{9.430751in}{1.961815in}}{\pgfqpoint{9.435588in}{1.959811in}}{\pgfqpoint{9.440632in}{1.959811in}}%
\pgfpathclose%
\pgfusepath{fill}%
\end{pgfscope}%
\begin{pgfscope}%
\pgfpathrectangle{\pgfqpoint{6.572727in}{0.473000in}}{\pgfqpoint{4.227273in}{3.311000in}}%
\pgfusepath{clip}%
\pgfsetbuttcap%
\pgfsetroundjoin%
\definecolor{currentfill}{rgb}{0.267004,0.004874,0.329415}%
\pgfsetfillcolor{currentfill}%
\pgfsetfillopacity{0.700000}%
\pgfsetlinewidth{0.000000pt}%
\definecolor{currentstroke}{rgb}{0.000000,0.000000,0.000000}%
\pgfsetstrokecolor{currentstroke}%
\pgfsetstrokeopacity{0.700000}%
\pgfsetdash{}{0pt}%
\pgfpathmoveto{\pgfqpoint{8.563329in}{1.013870in}}%
\pgfpathcurveto{\pgfqpoint{8.568373in}{1.013870in}}{\pgfqpoint{8.573211in}{1.015873in}}{\pgfqpoint{8.576777in}{1.019440in}}%
\pgfpathcurveto{\pgfqpoint{8.580344in}{1.023006in}}{\pgfqpoint{8.582347in}{1.027844in}}{\pgfqpoint{8.582347in}{1.032888in}}%
\pgfpathcurveto{\pgfqpoint{8.582347in}{1.037931in}}{\pgfqpoint{8.580344in}{1.042769in}}{\pgfqpoint{8.576777in}{1.046336in}}%
\pgfpathcurveto{\pgfqpoint{8.573211in}{1.049902in}}{\pgfqpoint{8.568373in}{1.051906in}}{\pgfqpoint{8.563329in}{1.051906in}}%
\pgfpathcurveto{\pgfqpoint{8.558286in}{1.051906in}}{\pgfqpoint{8.553448in}{1.049902in}}{\pgfqpoint{8.549881in}{1.046336in}}%
\pgfpathcurveto{\pgfqpoint{8.546315in}{1.042769in}}{\pgfqpoint{8.544311in}{1.037931in}}{\pgfqpoint{8.544311in}{1.032888in}}%
\pgfpathcurveto{\pgfqpoint{8.544311in}{1.027844in}}{\pgfqpoint{8.546315in}{1.023006in}}{\pgfqpoint{8.549881in}{1.019440in}}%
\pgfpathcurveto{\pgfqpoint{8.553448in}{1.015873in}}{\pgfqpoint{8.558286in}{1.013870in}}{\pgfqpoint{8.563329in}{1.013870in}}%
\pgfpathclose%
\pgfusepath{fill}%
\end{pgfscope}%
\begin{pgfscope}%
\pgfpathrectangle{\pgfqpoint{6.572727in}{0.473000in}}{\pgfqpoint{4.227273in}{3.311000in}}%
\pgfusepath{clip}%
\pgfsetbuttcap%
\pgfsetroundjoin%
\definecolor{currentfill}{rgb}{0.993248,0.906157,0.143936}%
\pgfsetfillcolor{currentfill}%
\pgfsetfillopacity{0.700000}%
\pgfsetlinewidth{0.000000pt}%
\definecolor{currentstroke}{rgb}{0.000000,0.000000,0.000000}%
\pgfsetstrokecolor{currentstroke}%
\pgfsetstrokeopacity{0.700000}%
\pgfsetdash{}{0pt}%
\pgfpathmoveto{\pgfqpoint{9.354248in}{1.003557in}}%
\pgfpathcurveto{\pgfqpoint{9.359292in}{1.003557in}}{\pgfqpoint{9.364129in}{1.005561in}}{\pgfqpoint{9.367696in}{1.009127in}}%
\pgfpathcurveto{\pgfqpoint{9.371262in}{1.012694in}}{\pgfqpoint{9.373266in}{1.017531in}}{\pgfqpoint{9.373266in}{1.022575in}}%
\pgfpathcurveto{\pgfqpoint{9.373266in}{1.027619in}}{\pgfqpoint{9.371262in}{1.032457in}}{\pgfqpoint{9.367696in}{1.036023in}}%
\pgfpathcurveto{\pgfqpoint{9.364129in}{1.039589in}}{\pgfqpoint{9.359292in}{1.041593in}}{\pgfqpoint{9.354248in}{1.041593in}}%
\pgfpathcurveto{\pgfqpoint{9.349204in}{1.041593in}}{\pgfqpoint{9.344366in}{1.039589in}}{\pgfqpoint{9.340800in}{1.036023in}}%
\pgfpathcurveto{\pgfqpoint{9.337234in}{1.032457in}}{\pgfqpoint{9.335230in}{1.027619in}}{\pgfqpoint{9.335230in}{1.022575in}}%
\pgfpathcurveto{\pgfqpoint{9.335230in}{1.017531in}}{\pgfqpoint{9.337234in}{1.012694in}}{\pgfqpoint{9.340800in}{1.009127in}}%
\pgfpathcurveto{\pgfqpoint{9.344366in}{1.005561in}}{\pgfqpoint{9.349204in}{1.003557in}}{\pgfqpoint{9.354248in}{1.003557in}}%
\pgfpathclose%
\pgfusepath{fill}%
\end{pgfscope}%
\begin{pgfscope}%
\pgfpathrectangle{\pgfqpoint{6.572727in}{0.473000in}}{\pgfqpoint{4.227273in}{3.311000in}}%
\pgfusepath{clip}%
\pgfsetbuttcap%
\pgfsetroundjoin%
\definecolor{currentfill}{rgb}{0.127568,0.566949,0.550556}%
\pgfsetfillcolor{currentfill}%
\pgfsetfillopacity{0.700000}%
\pgfsetlinewidth{0.000000pt}%
\definecolor{currentstroke}{rgb}{0.000000,0.000000,0.000000}%
\pgfsetstrokecolor{currentstroke}%
\pgfsetstrokeopacity{0.700000}%
\pgfsetdash{}{0pt}%
\pgfpathmoveto{\pgfqpoint{8.124062in}{1.706966in}}%
\pgfpathcurveto{\pgfqpoint{8.129106in}{1.706966in}}{\pgfqpoint{8.133943in}{1.708970in}}{\pgfqpoint{8.137510in}{1.712536in}}%
\pgfpathcurveto{\pgfqpoint{8.141076in}{1.716103in}}{\pgfqpoint{8.143080in}{1.720941in}}{\pgfqpoint{8.143080in}{1.725984in}}%
\pgfpathcurveto{\pgfqpoint{8.143080in}{1.731028in}}{\pgfqpoint{8.141076in}{1.735866in}}{\pgfqpoint{8.137510in}{1.739432in}}%
\pgfpathcurveto{\pgfqpoint{8.133943in}{1.742999in}}{\pgfqpoint{8.129106in}{1.745002in}}{\pgfqpoint{8.124062in}{1.745002in}}%
\pgfpathcurveto{\pgfqpoint{8.119018in}{1.745002in}}{\pgfqpoint{8.114181in}{1.742999in}}{\pgfqpoint{8.110614in}{1.739432in}}%
\pgfpathcurveto{\pgfqpoint{8.107048in}{1.735866in}}{\pgfqpoint{8.105044in}{1.731028in}}{\pgfqpoint{8.105044in}{1.725984in}}%
\pgfpathcurveto{\pgfqpoint{8.105044in}{1.720941in}}{\pgfqpoint{8.107048in}{1.716103in}}{\pgfqpoint{8.110614in}{1.712536in}}%
\pgfpathcurveto{\pgfqpoint{8.114181in}{1.708970in}}{\pgfqpoint{8.119018in}{1.706966in}}{\pgfqpoint{8.124062in}{1.706966in}}%
\pgfpathclose%
\pgfusepath{fill}%
\end{pgfscope}%
\begin{pgfscope}%
\pgfpathrectangle{\pgfqpoint{6.572727in}{0.473000in}}{\pgfqpoint{4.227273in}{3.311000in}}%
\pgfusepath{clip}%
\pgfsetbuttcap%
\pgfsetroundjoin%
\definecolor{currentfill}{rgb}{0.127568,0.566949,0.550556}%
\pgfsetfillcolor{currentfill}%
\pgfsetfillopacity{0.700000}%
\pgfsetlinewidth{0.000000pt}%
\definecolor{currentstroke}{rgb}{0.000000,0.000000,0.000000}%
\pgfsetstrokecolor{currentstroke}%
\pgfsetstrokeopacity{0.700000}%
\pgfsetdash{}{0pt}%
\pgfpathmoveto{\pgfqpoint{9.075038in}{2.793244in}}%
\pgfpathcurveto{\pgfqpoint{9.080081in}{2.793244in}}{\pgfqpoint{9.084919in}{2.795248in}}{\pgfqpoint{9.088486in}{2.798815in}}%
\pgfpathcurveto{\pgfqpoint{9.092052in}{2.802381in}}{\pgfqpoint{9.094056in}{2.807219in}}{\pgfqpoint{9.094056in}{2.812263in}}%
\pgfpathcurveto{\pgfqpoint{9.094056in}{2.817306in}}{\pgfqpoint{9.092052in}{2.822144in}}{\pgfqpoint{9.088486in}{2.825710in}}%
\pgfpathcurveto{\pgfqpoint{9.084919in}{2.829277in}}{\pgfqpoint{9.080081in}{2.831281in}}{\pgfqpoint{9.075038in}{2.831281in}}%
\pgfpathcurveto{\pgfqpoint{9.069994in}{2.831281in}}{\pgfqpoint{9.065156in}{2.829277in}}{\pgfqpoint{9.061590in}{2.825710in}}%
\pgfpathcurveto{\pgfqpoint{9.058023in}{2.822144in}}{\pgfqpoint{9.056020in}{2.817306in}}{\pgfqpoint{9.056020in}{2.812263in}}%
\pgfpathcurveto{\pgfqpoint{9.056020in}{2.807219in}}{\pgfqpoint{9.058023in}{2.802381in}}{\pgfqpoint{9.061590in}{2.798815in}}%
\pgfpathcurveto{\pgfqpoint{9.065156in}{2.795248in}}{\pgfqpoint{9.069994in}{2.793244in}}{\pgfqpoint{9.075038in}{2.793244in}}%
\pgfpathclose%
\pgfusepath{fill}%
\end{pgfscope}%
\begin{pgfscope}%
\pgfpathrectangle{\pgfqpoint{6.572727in}{0.473000in}}{\pgfqpoint{4.227273in}{3.311000in}}%
\pgfusepath{clip}%
\pgfsetbuttcap%
\pgfsetroundjoin%
\definecolor{currentfill}{rgb}{0.127568,0.566949,0.550556}%
\pgfsetfillcolor{currentfill}%
\pgfsetfillopacity{0.700000}%
\pgfsetlinewidth{0.000000pt}%
\definecolor{currentstroke}{rgb}{0.000000,0.000000,0.000000}%
\pgfsetstrokecolor{currentstroke}%
\pgfsetstrokeopacity{0.700000}%
\pgfsetdash{}{0pt}%
\pgfpathmoveto{\pgfqpoint{7.421822in}{2.941706in}}%
\pgfpathcurveto{\pgfqpoint{7.426866in}{2.941706in}}{\pgfqpoint{7.431704in}{2.943709in}}{\pgfqpoint{7.435270in}{2.947276in}}%
\pgfpathcurveto{\pgfqpoint{7.438837in}{2.950842in}}{\pgfqpoint{7.440840in}{2.955680in}}{\pgfqpoint{7.440840in}{2.960724in}}%
\pgfpathcurveto{\pgfqpoint{7.440840in}{2.965767in}}{\pgfqpoint{7.438837in}{2.970605in}}{\pgfqpoint{7.435270in}{2.974172in}}%
\pgfpathcurveto{\pgfqpoint{7.431704in}{2.977738in}}{\pgfqpoint{7.426866in}{2.979742in}}{\pgfqpoint{7.421822in}{2.979742in}}%
\pgfpathcurveto{\pgfqpoint{7.416779in}{2.979742in}}{\pgfqpoint{7.411941in}{2.977738in}}{\pgfqpoint{7.408374in}{2.974172in}}%
\pgfpathcurveto{\pgfqpoint{7.404808in}{2.970605in}}{\pgfqpoint{7.402804in}{2.965767in}}{\pgfqpoint{7.402804in}{2.960724in}}%
\pgfpathcurveto{\pgfqpoint{7.402804in}{2.955680in}}{\pgfqpoint{7.404808in}{2.950842in}}{\pgfqpoint{7.408374in}{2.947276in}}%
\pgfpathcurveto{\pgfqpoint{7.411941in}{2.943709in}}{\pgfqpoint{7.416779in}{2.941706in}}{\pgfqpoint{7.421822in}{2.941706in}}%
\pgfpathclose%
\pgfusepath{fill}%
\end{pgfscope}%
\begin{pgfscope}%
\pgfpathrectangle{\pgfqpoint{6.572727in}{0.473000in}}{\pgfqpoint{4.227273in}{3.311000in}}%
\pgfusepath{clip}%
\pgfsetbuttcap%
\pgfsetroundjoin%
\definecolor{currentfill}{rgb}{0.127568,0.566949,0.550556}%
\pgfsetfillcolor{currentfill}%
\pgfsetfillopacity{0.700000}%
\pgfsetlinewidth{0.000000pt}%
\definecolor{currentstroke}{rgb}{0.000000,0.000000,0.000000}%
\pgfsetstrokecolor{currentstroke}%
\pgfsetstrokeopacity{0.700000}%
\pgfsetdash{}{0pt}%
\pgfpathmoveto{\pgfqpoint{8.522392in}{1.847521in}}%
\pgfpathcurveto{\pgfqpoint{8.527435in}{1.847521in}}{\pgfqpoint{8.532273in}{1.849525in}}{\pgfqpoint{8.535840in}{1.853091in}}%
\pgfpathcurveto{\pgfqpoint{8.539406in}{1.856657in}}{\pgfqpoint{8.541410in}{1.861495in}}{\pgfqpoint{8.541410in}{1.866539in}}%
\pgfpathcurveto{\pgfqpoint{8.541410in}{1.871583in}}{\pgfqpoint{8.539406in}{1.876420in}}{\pgfqpoint{8.535840in}{1.879987in}}%
\pgfpathcurveto{\pgfqpoint{8.532273in}{1.883553in}}{\pgfqpoint{8.527435in}{1.885557in}}{\pgfqpoint{8.522392in}{1.885557in}}%
\pgfpathcurveto{\pgfqpoint{8.517348in}{1.885557in}}{\pgfqpoint{8.512510in}{1.883553in}}{\pgfqpoint{8.508944in}{1.879987in}}%
\pgfpathcurveto{\pgfqpoint{8.505377in}{1.876420in}}{\pgfqpoint{8.503374in}{1.871583in}}{\pgfqpoint{8.503374in}{1.866539in}}%
\pgfpathcurveto{\pgfqpoint{8.503374in}{1.861495in}}{\pgfqpoint{8.505377in}{1.856657in}}{\pgfqpoint{8.508944in}{1.853091in}}%
\pgfpathcurveto{\pgfqpoint{8.512510in}{1.849525in}}{\pgfqpoint{8.517348in}{1.847521in}}{\pgfqpoint{8.522392in}{1.847521in}}%
\pgfpathclose%
\pgfusepath{fill}%
\end{pgfscope}%
\begin{pgfscope}%
\pgfpathrectangle{\pgfqpoint{6.572727in}{0.473000in}}{\pgfqpoint{4.227273in}{3.311000in}}%
\pgfusepath{clip}%
\pgfsetbuttcap%
\pgfsetroundjoin%
\definecolor{currentfill}{rgb}{0.127568,0.566949,0.550556}%
\pgfsetfillcolor{currentfill}%
\pgfsetfillopacity{0.700000}%
\pgfsetlinewidth{0.000000pt}%
\definecolor{currentstroke}{rgb}{0.000000,0.000000,0.000000}%
\pgfsetstrokecolor{currentstroke}%
\pgfsetstrokeopacity{0.700000}%
\pgfsetdash{}{0pt}%
\pgfpathmoveto{\pgfqpoint{7.738347in}{1.461875in}}%
\pgfpathcurveto{\pgfqpoint{7.743391in}{1.461875in}}{\pgfqpoint{7.748228in}{1.463879in}}{\pgfqpoint{7.751795in}{1.467446in}}%
\pgfpathcurveto{\pgfqpoint{7.755361in}{1.471012in}}{\pgfqpoint{7.757365in}{1.475850in}}{\pgfqpoint{7.757365in}{1.480894in}}%
\pgfpathcurveto{\pgfqpoint{7.757365in}{1.485937in}}{\pgfqpoint{7.755361in}{1.490775in}}{\pgfqpoint{7.751795in}{1.494341in}}%
\pgfpathcurveto{\pgfqpoint{7.748228in}{1.497908in}}{\pgfqpoint{7.743391in}{1.499912in}}{\pgfqpoint{7.738347in}{1.499912in}}%
\pgfpathcurveto{\pgfqpoint{7.733303in}{1.499912in}}{\pgfqpoint{7.728466in}{1.497908in}}{\pgfqpoint{7.724899in}{1.494341in}}%
\pgfpathcurveto{\pgfqpoint{7.721333in}{1.490775in}}{\pgfqpoint{7.719329in}{1.485937in}}{\pgfqpoint{7.719329in}{1.480894in}}%
\pgfpathcurveto{\pgfqpoint{7.719329in}{1.475850in}}{\pgfqpoint{7.721333in}{1.471012in}}{\pgfqpoint{7.724899in}{1.467446in}}%
\pgfpathcurveto{\pgfqpoint{7.728466in}{1.463879in}}{\pgfqpoint{7.733303in}{1.461875in}}{\pgfqpoint{7.738347in}{1.461875in}}%
\pgfpathclose%
\pgfusepath{fill}%
\end{pgfscope}%
\begin{pgfscope}%
\pgfpathrectangle{\pgfqpoint{6.572727in}{0.473000in}}{\pgfqpoint{4.227273in}{3.311000in}}%
\pgfusepath{clip}%
\pgfsetbuttcap%
\pgfsetroundjoin%
\definecolor{currentfill}{rgb}{0.127568,0.566949,0.550556}%
\pgfsetfillcolor{currentfill}%
\pgfsetfillopacity{0.700000}%
\pgfsetlinewidth{0.000000pt}%
\definecolor{currentstroke}{rgb}{0.000000,0.000000,0.000000}%
\pgfsetstrokecolor{currentstroke}%
\pgfsetstrokeopacity{0.700000}%
\pgfsetdash{}{0pt}%
\pgfpathmoveto{\pgfqpoint{8.209600in}{1.610510in}}%
\pgfpathcurveto{\pgfqpoint{8.214644in}{1.610510in}}{\pgfqpoint{8.219482in}{1.612514in}}{\pgfqpoint{8.223048in}{1.616081in}}%
\pgfpathcurveto{\pgfqpoint{8.226614in}{1.619647in}}{\pgfqpoint{8.228618in}{1.624485in}}{\pgfqpoint{8.228618in}{1.629529in}}%
\pgfpathcurveto{\pgfqpoint{8.228618in}{1.634572in}}{\pgfqpoint{8.226614in}{1.639410in}}{\pgfqpoint{8.223048in}{1.642976in}}%
\pgfpathcurveto{\pgfqpoint{8.219482in}{1.646543in}}{\pgfqpoint{8.214644in}{1.648547in}}{\pgfqpoint{8.209600in}{1.648547in}}%
\pgfpathcurveto{\pgfqpoint{8.204556in}{1.648547in}}{\pgfqpoint{8.199719in}{1.646543in}}{\pgfqpoint{8.196152in}{1.642976in}}%
\pgfpathcurveto{\pgfqpoint{8.192586in}{1.639410in}}{\pgfqpoint{8.190582in}{1.634572in}}{\pgfqpoint{8.190582in}{1.629529in}}%
\pgfpathcurveto{\pgfqpoint{8.190582in}{1.624485in}}{\pgfqpoint{8.192586in}{1.619647in}}{\pgfqpoint{8.196152in}{1.616081in}}%
\pgfpathcurveto{\pgfqpoint{8.199719in}{1.612514in}}{\pgfqpoint{8.204556in}{1.610510in}}{\pgfqpoint{8.209600in}{1.610510in}}%
\pgfpathclose%
\pgfusepath{fill}%
\end{pgfscope}%
\begin{pgfscope}%
\pgfpathrectangle{\pgfqpoint{6.572727in}{0.473000in}}{\pgfqpoint{4.227273in}{3.311000in}}%
\pgfusepath{clip}%
\pgfsetbuttcap%
\pgfsetroundjoin%
\definecolor{currentfill}{rgb}{0.127568,0.566949,0.550556}%
\pgfsetfillcolor{currentfill}%
\pgfsetfillopacity{0.700000}%
\pgfsetlinewidth{0.000000pt}%
\definecolor{currentstroke}{rgb}{0.000000,0.000000,0.000000}%
\pgfsetstrokecolor{currentstroke}%
\pgfsetstrokeopacity{0.700000}%
\pgfsetdash{}{0pt}%
\pgfpathmoveto{\pgfqpoint{8.211597in}{3.264919in}}%
\pgfpathcurveto{\pgfqpoint{8.216641in}{3.264919in}}{\pgfqpoint{8.221478in}{3.266923in}}{\pgfqpoint{8.225045in}{3.270490in}}%
\pgfpathcurveto{\pgfqpoint{8.228611in}{3.274056in}}{\pgfqpoint{8.230615in}{3.278894in}}{\pgfqpoint{8.230615in}{3.283937in}}%
\pgfpathcurveto{\pgfqpoint{8.230615in}{3.288981in}}{\pgfqpoint{8.228611in}{3.293819in}}{\pgfqpoint{8.225045in}{3.297385in}}%
\pgfpathcurveto{\pgfqpoint{8.221478in}{3.300952in}}{\pgfqpoint{8.216641in}{3.302956in}}{\pgfqpoint{8.211597in}{3.302956in}}%
\pgfpathcurveto{\pgfqpoint{8.206553in}{3.302956in}}{\pgfqpoint{8.201715in}{3.300952in}}{\pgfqpoint{8.198149in}{3.297385in}}%
\pgfpathcurveto{\pgfqpoint{8.194583in}{3.293819in}}{\pgfqpoint{8.192579in}{3.288981in}}{\pgfqpoint{8.192579in}{3.283937in}}%
\pgfpathcurveto{\pgfqpoint{8.192579in}{3.278894in}}{\pgfqpoint{8.194583in}{3.274056in}}{\pgfqpoint{8.198149in}{3.270490in}}%
\pgfpathcurveto{\pgfqpoint{8.201715in}{3.266923in}}{\pgfqpoint{8.206553in}{3.264919in}}{\pgfqpoint{8.211597in}{3.264919in}}%
\pgfpathclose%
\pgfusepath{fill}%
\end{pgfscope}%
\begin{pgfscope}%
\pgfpathrectangle{\pgfqpoint{6.572727in}{0.473000in}}{\pgfqpoint{4.227273in}{3.311000in}}%
\pgfusepath{clip}%
\pgfsetbuttcap%
\pgfsetroundjoin%
\definecolor{currentfill}{rgb}{0.993248,0.906157,0.143936}%
\pgfsetfillcolor{currentfill}%
\pgfsetfillopacity{0.700000}%
\pgfsetlinewidth{0.000000pt}%
\definecolor{currentstroke}{rgb}{0.000000,0.000000,0.000000}%
\pgfsetstrokecolor{currentstroke}%
\pgfsetstrokeopacity{0.700000}%
\pgfsetdash{}{0pt}%
\pgfpathmoveto{\pgfqpoint{10.157788in}{1.571363in}}%
\pgfpathcurveto{\pgfqpoint{10.162831in}{1.571363in}}{\pgfqpoint{10.167669in}{1.573367in}}{\pgfqpoint{10.171236in}{1.576934in}}%
\pgfpathcurveto{\pgfqpoint{10.174802in}{1.580500in}}{\pgfqpoint{10.176806in}{1.585338in}}{\pgfqpoint{10.176806in}{1.590381in}}%
\pgfpathcurveto{\pgfqpoint{10.176806in}{1.595425in}}{\pgfqpoint{10.174802in}{1.600263in}}{\pgfqpoint{10.171236in}{1.603829in}}%
\pgfpathcurveto{\pgfqpoint{10.167669in}{1.607396in}}{\pgfqpoint{10.162831in}{1.609400in}}{\pgfqpoint{10.157788in}{1.609400in}}%
\pgfpathcurveto{\pgfqpoint{10.152744in}{1.609400in}}{\pgfqpoint{10.147906in}{1.607396in}}{\pgfqpoint{10.144340in}{1.603829in}}%
\pgfpathcurveto{\pgfqpoint{10.140773in}{1.600263in}}{\pgfqpoint{10.138770in}{1.595425in}}{\pgfqpoint{10.138770in}{1.590381in}}%
\pgfpathcurveto{\pgfqpoint{10.138770in}{1.585338in}}{\pgfqpoint{10.140773in}{1.580500in}}{\pgfqpoint{10.144340in}{1.576934in}}%
\pgfpathcurveto{\pgfqpoint{10.147906in}{1.573367in}}{\pgfqpoint{10.152744in}{1.571363in}}{\pgfqpoint{10.157788in}{1.571363in}}%
\pgfpathclose%
\pgfusepath{fill}%
\end{pgfscope}%
\begin{pgfscope}%
\pgfpathrectangle{\pgfqpoint{6.572727in}{0.473000in}}{\pgfqpoint{4.227273in}{3.311000in}}%
\pgfusepath{clip}%
\pgfsetbuttcap%
\pgfsetroundjoin%
\definecolor{currentfill}{rgb}{0.127568,0.566949,0.550556}%
\pgfsetfillcolor{currentfill}%
\pgfsetfillopacity{0.700000}%
\pgfsetlinewidth{0.000000pt}%
\definecolor{currentstroke}{rgb}{0.000000,0.000000,0.000000}%
\pgfsetstrokecolor{currentstroke}%
\pgfsetstrokeopacity{0.700000}%
\pgfsetdash{}{0pt}%
\pgfpathmoveto{\pgfqpoint{8.554084in}{2.723924in}}%
\pgfpathcurveto{\pgfqpoint{8.559127in}{2.723924in}}{\pgfqpoint{8.563965in}{2.725928in}}{\pgfqpoint{8.567532in}{2.729494in}}%
\pgfpathcurveto{\pgfqpoint{8.571098in}{2.733061in}}{\pgfqpoint{8.573102in}{2.737899in}}{\pgfqpoint{8.573102in}{2.742942in}}%
\pgfpathcurveto{\pgfqpoint{8.573102in}{2.747986in}}{\pgfqpoint{8.571098in}{2.752824in}}{\pgfqpoint{8.567532in}{2.756390in}}%
\pgfpathcurveto{\pgfqpoint{8.563965in}{2.759956in}}{\pgfqpoint{8.559127in}{2.761960in}}{\pgfqpoint{8.554084in}{2.761960in}}%
\pgfpathcurveto{\pgfqpoint{8.549040in}{2.761960in}}{\pgfqpoint{8.544202in}{2.759956in}}{\pgfqpoint{8.540636in}{2.756390in}}%
\pgfpathcurveto{\pgfqpoint{8.537070in}{2.752824in}}{\pgfqpoint{8.535066in}{2.747986in}}{\pgfqpoint{8.535066in}{2.742942in}}%
\pgfpathcurveto{\pgfqpoint{8.535066in}{2.737899in}}{\pgfqpoint{8.537070in}{2.733061in}}{\pgfqpoint{8.540636in}{2.729494in}}%
\pgfpathcurveto{\pgfqpoint{8.544202in}{2.725928in}}{\pgfqpoint{8.549040in}{2.723924in}}{\pgfqpoint{8.554084in}{2.723924in}}%
\pgfpathclose%
\pgfusepath{fill}%
\end{pgfscope}%
\begin{pgfscope}%
\pgfpathrectangle{\pgfqpoint{6.572727in}{0.473000in}}{\pgfqpoint{4.227273in}{3.311000in}}%
\pgfusepath{clip}%
\pgfsetbuttcap%
\pgfsetroundjoin%
\definecolor{currentfill}{rgb}{0.993248,0.906157,0.143936}%
\pgfsetfillcolor{currentfill}%
\pgfsetfillopacity{0.700000}%
\pgfsetlinewidth{0.000000pt}%
\definecolor{currentstroke}{rgb}{0.000000,0.000000,0.000000}%
\pgfsetstrokecolor{currentstroke}%
\pgfsetstrokeopacity{0.700000}%
\pgfsetdash{}{0pt}%
\pgfpathmoveto{\pgfqpoint{9.476601in}{1.814614in}}%
\pgfpathcurveto{\pgfqpoint{9.481645in}{1.814614in}}{\pgfqpoint{9.486483in}{1.816618in}}{\pgfqpoint{9.490049in}{1.820185in}}%
\pgfpathcurveto{\pgfqpoint{9.493615in}{1.823751in}}{\pgfqpoint{9.495619in}{1.828589in}}{\pgfqpoint{9.495619in}{1.833632in}}%
\pgfpathcurveto{\pgfqpoint{9.495619in}{1.838676in}}{\pgfqpoint{9.493615in}{1.843514in}}{\pgfqpoint{9.490049in}{1.847080in}}%
\pgfpathcurveto{\pgfqpoint{9.486483in}{1.850647in}}{\pgfqpoint{9.481645in}{1.852651in}}{\pgfqpoint{9.476601in}{1.852651in}}%
\pgfpathcurveto{\pgfqpoint{9.471558in}{1.852651in}}{\pgfqpoint{9.466720in}{1.850647in}}{\pgfqpoint{9.463153in}{1.847080in}}%
\pgfpathcurveto{\pgfqpoint{9.459587in}{1.843514in}}{\pgfqpoint{9.457583in}{1.838676in}}{\pgfqpoint{9.457583in}{1.833632in}}%
\pgfpathcurveto{\pgfqpoint{9.457583in}{1.828589in}}{\pgfqpoint{9.459587in}{1.823751in}}{\pgfqpoint{9.463153in}{1.820185in}}%
\pgfpathcurveto{\pgfqpoint{9.466720in}{1.816618in}}{\pgfqpoint{9.471558in}{1.814614in}}{\pgfqpoint{9.476601in}{1.814614in}}%
\pgfpathclose%
\pgfusepath{fill}%
\end{pgfscope}%
\begin{pgfscope}%
\pgfpathrectangle{\pgfqpoint{6.572727in}{0.473000in}}{\pgfqpoint{4.227273in}{3.311000in}}%
\pgfusepath{clip}%
\pgfsetbuttcap%
\pgfsetroundjoin%
\definecolor{currentfill}{rgb}{0.127568,0.566949,0.550556}%
\pgfsetfillcolor{currentfill}%
\pgfsetfillopacity{0.700000}%
\pgfsetlinewidth{0.000000pt}%
\definecolor{currentstroke}{rgb}{0.000000,0.000000,0.000000}%
\pgfsetstrokecolor{currentstroke}%
\pgfsetstrokeopacity{0.700000}%
\pgfsetdash{}{0pt}%
\pgfpathmoveto{\pgfqpoint{8.696904in}{3.209645in}}%
\pgfpathcurveto{\pgfqpoint{8.701948in}{3.209645in}}{\pgfqpoint{8.706786in}{3.211649in}}{\pgfqpoint{8.710352in}{3.215215in}}%
\pgfpathcurveto{\pgfqpoint{8.713919in}{3.218782in}}{\pgfqpoint{8.715923in}{3.223619in}}{\pgfqpoint{8.715923in}{3.228663in}}%
\pgfpathcurveto{\pgfqpoint{8.715923in}{3.233707in}}{\pgfqpoint{8.713919in}{3.238544in}}{\pgfqpoint{8.710352in}{3.242111in}}%
\pgfpathcurveto{\pgfqpoint{8.706786in}{3.245677in}}{\pgfqpoint{8.701948in}{3.247681in}}{\pgfqpoint{8.696904in}{3.247681in}}%
\pgfpathcurveto{\pgfqpoint{8.691861in}{3.247681in}}{\pgfqpoint{8.687023in}{3.245677in}}{\pgfqpoint{8.683457in}{3.242111in}}%
\pgfpathcurveto{\pgfqpoint{8.679890in}{3.238544in}}{\pgfqpoint{8.677886in}{3.233707in}}{\pgfqpoint{8.677886in}{3.228663in}}%
\pgfpathcurveto{\pgfqpoint{8.677886in}{3.223619in}}{\pgfqpoint{8.679890in}{3.218782in}}{\pgfqpoint{8.683457in}{3.215215in}}%
\pgfpathcurveto{\pgfqpoint{8.687023in}{3.211649in}}{\pgfqpoint{8.691861in}{3.209645in}}{\pgfqpoint{8.696904in}{3.209645in}}%
\pgfpathclose%
\pgfusepath{fill}%
\end{pgfscope}%
\begin{pgfscope}%
\pgfpathrectangle{\pgfqpoint{6.572727in}{0.473000in}}{\pgfqpoint{4.227273in}{3.311000in}}%
\pgfusepath{clip}%
\pgfsetbuttcap%
\pgfsetroundjoin%
\definecolor{currentfill}{rgb}{0.127568,0.566949,0.550556}%
\pgfsetfillcolor{currentfill}%
\pgfsetfillopacity{0.700000}%
\pgfsetlinewidth{0.000000pt}%
\definecolor{currentstroke}{rgb}{0.000000,0.000000,0.000000}%
\pgfsetstrokecolor{currentstroke}%
\pgfsetstrokeopacity{0.700000}%
\pgfsetdash{}{0pt}%
\pgfpathmoveto{\pgfqpoint{7.504100in}{1.738740in}}%
\pgfpathcurveto{\pgfqpoint{7.509143in}{1.738740in}}{\pgfqpoint{7.513981in}{1.740744in}}{\pgfqpoint{7.517547in}{1.744310in}}%
\pgfpathcurveto{\pgfqpoint{7.521114in}{1.747876in}}{\pgfqpoint{7.523118in}{1.752714in}}{\pgfqpoint{7.523118in}{1.757758in}}%
\pgfpathcurveto{\pgfqpoint{7.523118in}{1.762802in}}{\pgfqpoint{7.521114in}{1.767639in}}{\pgfqpoint{7.517547in}{1.771206in}}%
\pgfpathcurveto{\pgfqpoint{7.513981in}{1.774772in}}{\pgfqpoint{7.509143in}{1.776776in}}{\pgfqpoint{7.504100in}{1.776776in}}%
\pgfpathcurveto{\pgfqpoint{7.499056in}{1.776776in}}{\pgfqpoint{7.494218in}{1.774772in}}{\pgfqpoint{7.490652in}{1.771206in}}%
\pgfpathcurveto{\pgfqpoint{7.487085in}{1.767639in}}{\pgfqpoint{7.485081in}{1.762802in}}{\pgfqpoint{7.485081in}{1.757758in}}%
\pgfpathcurveto{\pgfqpoint{7.485081in}{1.752714in}}{\pgfqpoint{7.487085in}{1.747876in}}{\pgfqpoint{7.490652in}{1.744310in}}%
\pgfpathcurveto{\pgfqpoint{7.494218in}{1.740744in}}{\pgfqpoint{7.499056in}{1.738740in}}{\pgfqpoint{7.504100in}{1.738740in}}%
\pgfpathclose%
\pgfusepath{fill}%
\end{pgfscope}%
\begin{pgfscope}%
\pgfpathrectangle{\pgfqpoint{6.572727in}{0.473000in}}{\pgfqpoint{4.227273in}{3.311000in}}%
\pgfusepath{clip}%
\pgfsetbuttcap%
\pgfsetroundjoin%
\definecolor{currentfill}{rgb}{0.993248,0.906157,0.143936}%
\pgfsetfillcolor{currentfill}%
\pgfsetfillopacity{0.700000}%
\pgfsetlinewidth{0.000000pt}%
\definecolor{currentstroke}{rgb}{0.000000,0.000000,0.000000}%
\pgfsetstrokecolor{currentstroke}%
\pgfsetstrokeopacity{0.700000}%
\pgfsetdash{}{0pt}%
\pgfpathmoveto{\pgfqpoint{9.212377in}{1.540989in}}%
\pgfpathcurveto{\pgfqpoint{9.217420in}{1.540989in}}{\pgfqpoint{9.222258in}{1.542993in}}{\pgfqpoint{9.225824in}{1.546560in}}%
\pgfpathcurveto{\pgfqpoint{9.229391in}{1.550126in}}{\pgfqpoint{9.231395in}{1.554964in}}{\pgfqpoint{9.231395in}{1.560008in}}%
\pgfpathcurveto{\pgfqpoint{9.231395in}{1.565051in}}{\pgfqpoint{9.229391in}{1.569889in}}{\pgfqpoint{9.225824in}{1.573455in}}%
\pgfpathcurveto{\pgfqpoint{9.222258in}{1.577022in}}{\pgfqpoint{9.217420in}{1.579026in}}{\pgfqpoint{9.212377in}{1.579026in}}%
\pgfpathcurveto{\pgfqpoint{9.207333in}{1.579026in}}{\pgfqpoint{9.202495in}{1.577022in}}{\pgfqpoint{9.198929in}{1.573455in}}%
\pgfpathcurveto{\pgfqpoint{9.195362in}{1.569889in}}{\pgfqpoint{9.193358in}{1.565051in}}{\pgfqpoint{9.193358in}{1.560008in}}%
\pgfpathcurveto{\pgfqpoint{9.193358in}{1.554964in}}{\pgfqpoint{9.195362in}{1.550126in}}{\pgfqpoint{9.198929in}{1.546560in}}%
\pgfpathcurveto{\pgfqpoint{9.202495in}{1.542993in}}{\pgfqpoint{9.207333in}{1.540989in}}{\pgfqpoint{9.212377in}{1.540989in}}%
\pgfpathclose%
\pgfusepath{fill}%
\end{pgfscope}%
\begin{pgfscope}%
\pgfpathrectangle{\pgfqpoint{6.572727in}{0.473000in}}{\pgfqpoint{4.227273in}{3.311000in}}%
\pgfusepath{clip}%
\pgfsetbuttcap%
\pgfsetroundjoin%
\definecolor{currentfill}{rgb}{0.993248,0.906157,0.143936}%
\pgfsetfillcolor{currentfill}%
\pgfsetfillopacity{0.700000}%
\pgfsetlinewidth{0.000000pt}%
\definecolor{currentstroke}{rgb}{0.000000,0.000000,0.000000}%
\pgfsetstrokecolor{currentstroke}%
\pgfsetstrokeopacity{0.700000}%
\pgfsetdash{}{0pt}%
\pgfpathmoveto{\pgfqpoint{9.314282in}{1.095044in}}%
\pgfpathcurveto{\pgfqpoint{9.319326in}{1.095044in}}{\pgfqpoint{9.324164in}{1.097048in}}{\pgfqpoint{9.327730in}{1.100614in}}%
\pgfpathcurveto{\pgfqpoint{9.331297in}{1.104181in}}{\pgfqpoint{9.333300in}{1.109018in}}{\pgfqpoint{9.333300in}{1.114062in}}%
\pgfpathcurveto{\pgfqpoint{9.333300in}{1.119106in}}{\pgfqpoint{9.331297in}{1.123944in}}{\pgfqpoint{9.327730in}{1.127510in}}%
\pgfpathcurveto{\pgfqpoint{9.324164in}{1.131076in}}{\pgfqpoint{9.319326in}{1.133080in}}{\pgfqpoint{9.314282in}{1.133080in}}%
\pgfpathcurveto{\pgfqpoint{9.309239in}{1.133080in}}{\pgfqpoint{9.304401in}{1.131076in}}{\pgfqpoint{9.300834in}{1.127510in}}%
\pgfpathcurveto{\pgfqpoint{9.297268in}{1.123944in}}{\pgfqpoint{9.295264in}{1.119106in}}{\pgfqpoint{9.295264in}{1.114062in}}%
\pgfpathcurveto{\pgfqpoint{9.295264in}{1.109018in}}{\pgfqpoint{9.297268in}{1.104181in}}{\pgfqpoint{9.300834in}{1.100614in}}%
\pgfpathcurveto{\pgfqpoint{9.304401in}{1.097048in}}{\pgfqpoint{9.309239in}{1.095044in}}{\pgfqpoint{9.314282in}{1.095044in}}%
\pgfpathclose%
\pgfusepath{fill}%
\end{pgfscope}%
\begin{pgfscope}%
\pgfpathrectangle{\pgfqpoint{6.572727in}{0.473000in}}{\pgfqpoint{4.227273in}{3.311000in}}%
\pgfusepath{clip}%
\pgfsetbuttcap%
\pgfsetroundjoin%
\definecolor{currentfill}{rgb}{0.993248,0.906157,0.143936}%
\pgfsetfillcolor{currentfill}%
\pgfsetfillopacity{0.700000}%
\pgfsetlinewidth{0.000000pt}%
\definecolor{currentstroke}{rgb}{0.000000,0.000000,0.000000}%
\pgfsetstrokecolor{currentstroke}%
\pgfsetstrokeopacity{0.700000}%
\pgfsetdash{}{0pt}%
\pgfpathmoveto{\pgfqpoint{9.338851in}{1.847895in}}%
\pgfpathcurveto{\pgfqpoint{9.343895in}{1.847895in}}{\pgfqpoint{9.348733in}{1.849899in}}{\pgfqpoint{9.352299in}{1.853465in}}%
\pgfpathcurveto{\pgfqpoint{9.355866in}{1.857032in}}{\pgfqpoint{9.357870in}{1.861870in}}{\pgfqpoint{9.357870in}{1.866913in}}%
\pgfpathcurveto{\pgfqpoint{9.357870in}{1.871957in}}{\pgfqpoint{9.355866in}{1.876795in}}{\pgfqpoint{9.352299in}{1.880361in}}%
\pgfpathcurveto{\pgfqpoint{9.348733in}{1.883928in}}{\pgfqpoint{9.343895in}{1.885931in}}{\pgfqpoint{9.338851in}{1.885931in}}%
\pgfpathcurveto{\pgfqpoint{9.333808in}{1.885931in}}{\pgfqpoint{9.328970in}{1.883928in}}{\pgfqpoint{9.325404in}{1.880361in}}%
\pgfpathcurveto{\pgfqpoint{9.321837in}{1.876795in}}{\pgfqpoint{9.319833in}{1.871957in}}{\pgfqpoint{9.319833in}{1.866913in}}%
\pgfpathcurveto{\pgfqpoint{9.319833in}{1.861870in}}{\pgfqpoint{9.321837in}{1.857032in}}{\pgfqpoint{9.325404in}{1.853465in}}%
\pgfpathcurveto{\pgfqpoint{9.328970in}{1.849899in}}{\pgfqpoint{9.333808in}{1.847895in}}{\pgfqpoint{9.338851in}{1.847895in}}%
\pgfpathclose%
\pgfusepath{fill}%
\end{pgfscope}%
\begin{pgfscope}%
\pgfpathrectangle{\pgfqpoint{6.572727in}{0.473000in}}{\pgfqpoint{4.227273in}{3.311000in}}%
\pgfusepath{clip}%
\pgfsetbuttcap%
\pgfsetroundjoin%
\definecolor{currentfill}{rgb}{0.993248,0.906157,0.143936}%
\pgfsetfillcolor{currentfill}%
\pgfsetfillopacity{0.700000}%
\pgfsetlinewidth{0.000000pt}%
\definecolor{currentstroke}{rgb}{0.000000,0.000000,0.000000}%
\pgfsetstrokecolor{currentstroke}%
\pgfsetstrokeopacity{0.700000}%
\pgfsetdash{}{0pt}%
\pgfpathmoveto{\pgfqpoint{9.753532in}{1.393119in}}%
\pgfpathcurveto{\pgfqpoint{9.758576in}{1.393119in}}{\pgfqpoint{9.763414in}{1.395123in}}{\pgfqpoint{9.766980in}{1.398690in}}%
\pgfpathcurveto{\pgfqpoint{9.770547in}{1.402256in}}{\pgfqpoint{9.772550in}{1.407094in}}{\pgfqpoint{9.772550in}{1.412137in}}%
\pgfpathcurveto{\pgfqpoint{9.772550in}{1.417181in}}{\pgfqpoint{9.770547in}{1.422019in}}{\pgfqpoint{9.766980in}{1.425585in}}%
\pgfpathcurveto{\pgfqpoint{9.763414in}{1.429152in}}{\pgfqpoint{9.758576in}{1.431156in}}{\pgfqpoint{9.753532in}{1.431156in}}%
\pgfpathcurveto{\pgfqpoint{9.748489in}{1.431156in}}{\pgfqpoint{9.743651in}{1.429152in}}{\pgfqpoint{9.740084in}{1.425585in}}%
\pgfpathcurveto{\pgfqpoint{9.736518in}{1.422019in}}{\pgfqpoint{9.734514in}{1.417181in}}{\pgfqpoint{9.734514in}{1.412137in}}%
\pgfpathcurveto{\pgfqpoint{9.734514in}{1.407094in}}{\pgfqpoint{9.736518in}{1.402256in}}{\pgfqpoint{9.740084in}{1.398690in}}%
\pgfpathcurveto{\pgfqpoint{9.743651in}{1.395123in}}{\pgfqpoint{9.748489in}{1.393119in}}{\pgfqpoint{9.753532in}{1.393119in}}%
\pgfpathclose%
\pgfusepath{fill}%
\end{pgfscope}%
\begin{pgfscope}%
\pgfpathrectangle{\pgfqpoint{6.572727in}{0.473000in}}{\pgfqpoint{4.227273in}{3.311000in}}%
\pgfusepath{clip}%
\pgfsetbuttcap%
\pgfsetroundjoin%
\definecolor{currentfill}{rgb}{0.993248,0.906157,0.143936}%
\pgfsetfillcolor{currentfill}%
\pgfsetfillopacity{0.700000}%
\pgfsetlinewidth{0.000000pt}%
\definecolor{currentstroke}{rgb}{0.000000,0.000000,0.000000}%
\pgfsetstrokecolor{currentstroke}%
\pgfsetstrokeopacity{0.700000}%
\pgfsetdash{}{0pt}%
\pgfpathmoveto{\pgfqpoint{9.121166in}{2.204211in}}%
\pgfpathcurveto{\pgfqpoint{9.126209in}{2.204211in}}{\pgfqpoint{9.131047in}{2.206215in}}{\pgfqpoint{9.134613in}{2.209781in}}%
\pgfpathcurveto{\pgfqpoint{9.138180in}{2.213348in}}{\pgfqpoint{9.140184in}{2.218185in}}{\pgfqpoint{9.140184in}{2.223229in}}%
\pgfpathcurveto{\pgfqpoint{9.140184in}{2.228273in}}{\pgfqpoint{9.138180in}{2.233111in}}{\pgfqpoint{9.134613in}{2.236677in}}%
\pgfpathcurveto{\pgfqpoint{9.131047in}{2.240243in}}{\pgfqpoint{9.126209in}{2.242247in}}{\pgfqpoint{9.121166in}{2.242247in}}%
\pgfpathcurveto{\pgfqpoint{9.116122in}{2.242247in}}{\pgfqpoint{9.111284in}{2.240243in}}{\pgfqpoint{9.107718in}{2.236677in}}%
\pgfpathcurveto{\pgfqpoint{9.104151in}{2.233111in}}{\pgfqpoint{9.102147in}{2.228273in}}{\pgfqpoint{9.102147in}{2.223229in}}%
\pgfpathcurveto{\pgfqpoint{9.102147in}{2.218185in}}{\pgfqpoint{9.104151in}{2.213348in}}{\pgfqpoint{9.107718in}{2.209781in}}%
\pgfpathcurveto{\pgfqpoint{9.111284in}{2.206215in}}{\pgfqpoint{9.116122in}{2.204211in}}{\pgfqpoint{9.121166in}{2.204211in}}%
\pgfpathclose%
\pgfusepath{fill}%
\end{pgfscope}%
\begin{pgfscope}%
\pgfpathrectangle{\pgfqpoint{6.572727in}{0.473000in}}{\pgfqpoint{4.227273in}{3.311000in}}%
\pgfusepath{clip}%
\pgfsetbuttcap%
\pgfsetroundjoin%
\definecolor{currentfill}{rgb}{0.127568,0.566949,0.550556}%
\pgfsetfillcolor{currentfill}%
\pgfsetfillopacity{0.700000}%
\pgfsetlinewidth{0.000000pt}%
\definecolor{currentstroke}{rgb}{0.000000,0.000000,0.000000}%
\pgfsetstrokecolor{currentstroke}%
\pgfsetstrokeopacity{0.700000}%
\pgfsetdash{}{0pt}%
\pgfpathmoveto{\pgfqpoint{7.761253in}{2.652956in}}%
\pgfpathcurveto{\pgfqpoint{7.766297in}{2.652956in}}{\pgfqpoint{7.771135in}{2.654960in}}{\pgfqpoint{7.774701in}{2.658526in}}%
\pgfpathcurveto{\pgfqpoint{7.778267in}{2.662093in}}{\pgfqpoint{7.780271in}{2.666931in}}{\pgfqpoint{7.780271in}{2.671974in}}%
\pgfpathcurveto{\pgfqpoint{7.780271in}{2.677018in}}{\pgfqpoint{7.778267in}{2.681856in}}{\pgfqpoint{7.774701in}{2.685422in}}%
\pgfpathcurveto{\pgfqpoint{7.771135in}{2.688989in}}{\pgfqpoint{7.766297in}{2.690992in}}{\pgfqpoint{7.761253in}{2.690992in}}%
\pgfpathcurveto{\pgfqpoint{7.756209in}{2.690992in}}{\pgfqpoint{7.751372in}{2.688989in}}{\pgfqpoint{7.747805in}{2.685422in}}%
\pgfpathcurveto{\pgfqpoint{7.744239in}{2.681856in}}{\pgfqpoint{7.742235in}{2.677018in}}{\pgfqpoint{7.742235in}{2.671974in}}%
\pgfpathcurveto{\pgfqpoint{7.742235in}{2.666931in}}{\pgfqpoint{7.744239in}{2.662093in}}{\pgfqpoint{7.747805in}{2.658526in}}%
\pgfpathcurveto{\pgfqpoint{7.751372in}{2.654960in}}{\pgfqpoint{7.756209in}{2.652956in}}{\pgfqpoint{7.761253in}{2.652956in}}%
\pgfpathclose%
\pgfusepath{fill}%
\end{pgfscope}%
\begin{pgfscope}%
\pgfpathrectangle{\pgfqpoint{6.572727in}{0.473000in}}{\pgfqpoint{4.227273in}{3.311000in}}%
\pgfusepath{clip}%
\pgfsetbuttcap%
\pgfsetroundjoin%
\definecolor{currentfill}{rgb}{0.993248,0.906157,0.143936}%
\pgfsetfillcolor{currentfill}%
\pgfsetfillopacity{0.700000}%
\pgfsetlinewidth{0.000000pt}%
\definecolor{currentstroke}{rgb}{0.000000,0.000000,0.000000}%
\pgfsetstrokecolor{currentstroke}%
\pgfsetstrokeopacity{0.700000}%
\pgfsetdash{}{0pt}%
\pgfpathmoveto{\pgfqpoint{9.003812in}{1.822639in}}%
\pgfpathcurveto{\pgfqpoint{9.008856in}{1.822639in}}{\pgfqpoint{9.013694in}{1.824642in}}{\pgfqpoint{9.017260in}{1.828209in}}%
\pgfpathcurveto{\pgfqpoint{9.020827in}{1.831775in}}{\pgfqpoint{9.022831in}{1.836613in}}{\pgfqpoint{9.022831in}{1.841657in}}%
\pgfpathcurveto{\pgfqpoint{9.022831in}{1.846700in}}{\pgfqpoint{9.020827in}{1.851538in}}{\pgfqpoint{9.017260in}{1.855105in}}%
\pgfpathcurveto{\pgfqpoint{9.013694in}{1.858671in}}{\pgfqpoint{9.008856in}{1.860675in}}{\pgfqpoint{9.003812in}{1.860675in}}%
\pgfpathcurveto{\pgfqpoint{8.998769in}{1.860675in}}{\pgfqpoint{8.993931in}{1.858671in}}{\pgfqpoint{8.990365in}{1.855105in}}%
\pgfpathcurveto{\pgfqpoint{8.986798in}{1.851538in}}{\pgfqpoint{8.984794in}{1.846700in}}{\pgfqpoint{8.984794in}{1.841657in}}%
\pgfpathcurveto{\pgfqpoint{8.984794in}{1.836613in}}{\pgfqpoint{8.986798in}{1.831775in}}{\pgfqpoint{8.990365in}{1.828209in}}%
\pgfpathcurveto{\pgfqpoint{8.993931in}{1.824642in}}{\pgfqpoint{8.998769in}{1.822639in}}{\pgfqpoint{9.003812in}{1.822639in}}%
\pgfpathclose%
\pgfusepath{fill}%
\end{pgfscope}%
\begin{pgfscope}%
\pgfpathrectangle{\pgfqpoint{6.572727in}{0.473000in}}{\pgfqpoint{4.227273in}{3.311000in}}%
\pgfusepath{clip}%
\pgfsetbuttcap%
\pgfsetroundjoin%
\definecolor{currentfill}{rgb}{0.127568,0.566949,0.550556}%
\pgfsetfillcolor{currentfill}%
\pgfsetfillopacity{0.700000}%
\pgfsetlinewidth{0.000000pt}%
\definecolor{currentstroke}{rgb}{0.000000,0.000000,0.000000}%
\pgfsetstrokecolor{currentstroke}%
\pgfsetstrokeopacity{0.700000}%
\pgfsetdash{}{0pt}%
\pgfpathmoveto{\pgfqpoint{8.194852in}{3.187735in}}%
\pgfpathcurveto{\pgfqpoint{8.199896in}{3.187735in}}{\pgfqpoint{8.204734in}{3.189739in}}{\pgfqpoint{8.208300in}{3.193306in}}%
\pgfpathcurveto{\pgfqpoint{8.211866in}{3.196872in}}{\pgfqpoint{8.213870in}{3.201710in}}{\pgfqpoint{8.213870in}{3.206753in}}%
\pgfpathcurveto{\pgfqpoint{8.213870in}{3.211797in}}{\pgfqpoint{8.211866in}{3.216635in}}{\pgfqpoint{8.208300in}{3.220201in}}%
\pgfpathcurveto{\pgfqpoint{8.204734in}{3.223768in}}{\pgfqpoint{8.199896in}{3.225772in}}{\pgfqpoint{8.194852in}{3.225772in}}%
\pgfpathcurveto{\pgfqpoint{8.189808in}{3.225772in}}{\pgfqpoint{8.184971in}{3.223768in}}{\pgfqpoint{8.181404in}{3.220201in}}%
\pgfpathcurveto{\pgfqpoint{8.177838in}{3.216635in}}{\pgfqpoint{8.175834in}{3.211797in}}{\pgfqpoint{8.175834in}{3.206753in}}%
\pgfpathcurveto{\pgfqpoint{8.175834in}{3.201710in}}{\pgfqpoint{8.177838in}{3.196872in}}{\pgfqpoint{8.181404in}{3.193306in}}%
\pgfpathcurveto{\pgfqpoint{8.184971in}{3.189739in}}{\pgfqpoint{8.189808in}{3.187735in}}{\pgfqpoint{8.194852in}{3.187735in}}%
\pgfpathclose%
\pgfusepath{fill}%
\end{pgfscope}%
\begin{pgfscope}%
\pgfpathrectangle{\pgfqpoint{6.572727in}{0.473000in}}{\pgfqpoint{4.227273in}{3.311000in}}%
\pgfusepath{clip}%
\pgfsetbuttcap%
\pgfsetroundjoin%
\definecolor{currentfill}{rgb}{0.127568,0.566949,0.550556}%
\pgfsetfillcolor{currentfill}%
\pgfsetfillopacity{0.700000}%
\pgfsetlinewidth{0.000000pt}%
\definecolor{currentstroke}{rgb}{0.000000,0.000000,0.000000}%
\pgfsetstrokecolor{currentstroke}%
\pgfsetstrokeopacity{0.700000}%
\pgfsetdash{}{0pt}%
\pgfpathmoveto{\pgfqpoint{7.894513in}{1.830475in}}%
\pgfpathcurveto{\pgfqpoint{7.899556in}{1.830475in}}{\pgfqpoint{7.904394in}{1.832479in}}{\pgfqpoint{7.907961in}{1.836045in}}%
\pgfpathcurveto{\pgfqpoint{7.911527in}{1.839611in}}{\pgfqpoint{7.913531in}{1.844449in}}{\pgfqpoint{7.913531in}{1.849493in}}%
\pgfpathcurveto{\pgfqpoint{7.913531in}{1.854536in}}{\pgfqpoint{7.911527in}{1.859374in}}{\pgfqpoint{7.907961in}{1.862941in}}%
\pgfpathcurveto{\pgfqpoint{7.904394in}{1.866507in}}{\pgfqpoint{7.899556in}{1.868511in}}{\pgfqpoint{7.894513in}{1.868511in}}%
\pgfpathcurveto{\pgfqpoint{7.889469in}{1.868511in}}{\pgfqpoint{7.884631in}{1.866507in}}{\pgfqpoint{7.881065in}{1.862941in}}%
\pgfpathcurveto{\pgfqpoint{7.877498in}{1.859374in}}{\pgfqpoint{7.875494in}{1.854536in}}{\pgfqpoint{7.875494in}{1.849493in}}%
\pgfpathcurveto{\pgfqpoint{7.875494in}{1.844449in}}{\pgfqpoint{7.877498in}{1.839611in}}{\pgfqpoint{7.881065in}{1.836045in}}%
\pgfpathcurveto{\pgfqpoint{7.884631in}{1.832479in}}{\pgfqpoint{7.889469in}{1.830475in}}{\pgfqpoint{7.894513in}{1.830475in}}%
\pgfpathclose%
\pgfusepath{fill}%
\end{pgfscope}%
\begin{pgfscope}%
\pgfpathrectangle{\pgfqpoint{6.572727in}{0.473000in}}{\pgfqpoint{4.227273in}{3.311000in}}%
\pgfusepath{clip}%
\pgfsetbuttcap%
\pgfsetroundjoin%
\definecolor{currentfill}{rgb}{0.127568,0.566949,0.550556}%
\pgfsetfillcolor{currentfill}%
\pgfsetfillopacity{0.700000}%
\pgfsetlinewidth{0.000000pt}%
\definecolor{currentstroke}{rgb}{0.000000,0.000000,0.000000}%
\pgfsetstrokecolor{currentstroke}%
\pgfsetstrokeopacity{0.700000}%
\pgfsetdash{}{0pt}%
\pgfpathmoveto{\pgfqpoint{7.919408in}{2.045037in}}%
\pgfpathcurveto{\pgfqpoint{7.924452in}{2.045037in}}{\pgfqpoint{7.929289in}{2.047041in}}{\pgfqpoint{7.932856in}{2.050607in}}%
\pgfpathcurveto{\pgfqpoint{7.936422in}{2.054174in}}{\pgfqpoint{7.938426in}{2.059012in}}{\pgfqpoint{7.938426in}{2.064055in}}%
\pgfpathcurveto{\pgfqpoint{7.938426in}{2.069099in}}{\pgfqpoint{7.936422in}{2.073937in}}{\pgfqpoint{7.932856in}{2.077503in}}%
\pgfpathcurveto{\pgfqpoint{7.929289in}{2.081070in}}{\pgfqpoint{7.924452in}{2.083073in}}{\pgfqpoint{7.919408in}{2.083073in}}%
\pgfpathcurveto{\pgfqpoint{7.914364in}{2.083073in}}{\pgfqpoint{7.909527in}{2.081070in}}{\pgfqpoint{7.905960in}{2.077503in}}%
\pgfpathcurveto{\pgfqpoint{7.902394in}{2.073937in}}{\pgfqpoint{7.900390in}{2.069099in}}{\pgfqpoint{7.900390in}{2.064055in}}%
\pgfpathcurveto{\pgfqpoint{7.900390in}{2.059012in}}{\pgfqpoint{7.902394in}{2.054174in}}{\pgfqpoint{7.905960in}{2.050607in}}%
\pgfpathcurveto{\pgfqpoint{7.909527in}{2.047041in}}{\pgfqpoint{7.914364in}{2.045037in}}{\pgfqpoint{7.919408in}{2.045037in}}%
\pgfpathclose%
\pgfusepath{fill}%
\end{pgfscope}%
\begin{pgfscope}%
\pgfpathrectangle{\pgfqpoint{6.572727in}{0.473000in}}{\pgfqpoint{4.227273in}{3.311000in}}%
\pgfusepath{clip}%
\pgfsetbuttcap%
\pgfsetroundjoin%
\definecolor{currentfill}{rgb}{0.127568,0.566949,0.550556}%
\pgfsetfillcolor{currentfill}%
\pgfsetfillopacity{0.700000}%
\pgfsetlinewidth{0.000000pt}%
\definecolor{currentstroke}{rgb}{0.000000,0.000000,0.000000}%
\pgfsetstrokecolor{currentstroke}%
\pgfsetstrokeopacity{0.700000}%
\pgfsetdash{}{0pt}%
\pgfpathmoveto{\pgfqpoint{7.527580in}{2.586611in}}%
\pgfpathcurveto{\pgfqpoint{7.532624in}{2.586611in}}{\pgfqpoint{7.537462in}{2.588615in}}{\pgfqpoint{7.541028in}{2.592182in}}%
\pgfpathcurveto{\pgfqpoint{7.544594in}{2.595748in}}{\pgfqpoint{7.546598in}{2.600586in}}{\pgfqpoint{7.546598in}{2.605629in}}%
\pgfpathcurveto{\pgfqpoint{7.546598in}{2.610673in}}{\pgfqpoint{7.544594in}{2.615511in}}{\pgfqpoint{7.541028in}{2.619077in}}%
\pgfpathcurveto{\pgfqpoint{7.537462in}{2.622644in}}{\pgfqpoint{7.532624in}{2.624648in}}{\pgfqpoint{7.527580in}{2.624648in}}%
\pgfpathcurveto{\pgfqpoint{7.522537in}{2.624648in}}{\pgfqpoint{7.517699in}{2.622644in}}{\pgfqpoint{7.514132in}{2.619077in}}%
\pgfpathcurveto{\pgfqpoint{7.510566in}{2.615511in}}{\pgfqpoint{7.508562in}{2.610673in}}{\pgfqpoint{7.508562in}{2.605629in}}%
\pgfpathcurveto{\pgfqpoint{7.508562in}{2.600586in}}{\pgfqpoint{7.510566in}{2.595748in}}{\pgfqpoint{7.514132in}{2.592182in}}%
\pgfpathcurveto{\pgfqpoint{7.517699in}{2.588615in}}{\pgfqpoint{7.522537in}{2.586611in}}{\pgfqpoint{7.527580in}{2.586611in}}%
\pgfpathclose%
\pgfusepath{fill}%
\end{pgfscope}%
\begin{pgfscope}%
\pgfpathrectangle{\pgfqpoint{6.572727in}{0.473000in}}{\pgfqpoint{4.227273in}{3.311000in}}%
\pgfusepath{clip}%
\pgfsetbuttcap%
\pgfsetroundjoin%
\definecolor{currentfill}{rgb}{0.127568,0.566949,0.550556}%
\pgfsetfillcolor{currentfill}%
\pgfsetfillopacity{0.700000}%
\pgfsetlinewidth{0.000000pt}%
\definecolor{currentstroke}{rgb}{0.000000,0.000000,0.000000}%
\pgfsetstrokecolor{currentstroke}%
\pgfsetstrokeopacity{0.700000}%
\pgfsetdash{}{0pt}%
\pgfpathmoveto{\pgfqpoint{8.538993in}{2.833564in}}%
\pgfpathcurveto{\pgfqpoint{8.544037in}{2.833564in}}{\pgfqpoint{8.548875in}{2.835568in}}{\pgfqpoint{8.552441in}{2.839134in}}%
\pgfpathcurveto{\pgfqpoint{8.556007in}{2.842701in}}{\pgfqpoint{8.558011in}{2.847539in}}{\pgfqpoint{8.558011in}{2.852582in}}%
\pgfpathcurveto{\pgfqpoint{8.558011in}{2.857626in}}{\pgfqpoint{8.556007in}{2.862464in}}{\pgfqpoint{8.552441in}{2.866030in}}%
\pgfpathcurveto{\pgfqpoint{8.548875in}{2.869597in}}{\pgfqpoint{8.544037in}{2.871600in}}{\pgfqpoint{8.538993in}{2.871600in}}%
\pgfpathcurveto{\pgfqpoint{8.533949in}{2.871600in}}{\pgfqpoint{8.529112in}{2.869597in}}{\pgfqpoint{8.525545in}{2.866030in}}%
\pgfpathcurveto{\pgfqpoint{8.521979in}{2.862464in}}{\pgfqpoint{8.519975in}{2.857626in}}{\pgfqpoint{8.519975in}{2.852582in}}%
\pgfpathcurveto{\pgfqpoint{8.519975in}{2.847539in}}{\pgfqpoint{8.521979in}{2.842701in}}{\pgfqpoint{8.525545in}{2.839134in}}%
\pgfpathcurveto{\pgfqpoint{8.529112in}{2.835568in}}{\pgfqpoint{8.533949in}{2.833564in}}{\pgfqpoint{8.538993in}{2.833564in}}%
\pgfpathclose%
\pgfusepath{fill}%
\end{pgfscope}%
\begin{pgfscope}%
\pgfpathrectangle{\pgfqpoint{6.572727in}{0.473000in}}{\pgfqpoint{4.227273in}{3.311000in}}%
\pgfusepath{clip}%
\pgfsetbuttcap%
\pgfsetroundjoin%
\definecolor{currentfill}{rgb}{0.127568,0.566949,0.550556}%
\pgfsetfillcolor{currentfill}%
\pgfsetfillopacity{0.700000}%
\pgfsetlinewidth{0.000000pt}%
\definecolor{currentstroke}{rgb}{0.000000,0.000000,0.000000}%
\pgfsetstrokecolor{currentstroke}%
\pgfsetstrokeopacity{0.700000}%
\pgfsetdash{}{0pt}%
\pgfpathmoveto{\pgfqpoint{7.185606in}{1.530338in}}%
\pgfpathcurveto{\pgfqpoint{7.190649in}{1.530338in}}{\pgfqpoint{7.195487in}{1.532342in}}{\pgfqpoint{7.199053in}{1.535908in}}%
\pgfpathcurveto{\pgfqpoint{7.202620in}{1.539475in}}{\pgfqpoint{7.204624in}{1.544312in}}{\pgfqpoint{7.204624in}{1.549356in}}%
\pgfpathcurveto{\pgfqpoint{7.204624in}{1.554400in}}{\pgfqpoint{7.202620in}{1.559238in}}{\pgfqpoint{7.199053in}{1.562804in}}%
\pgfpathcurveto{\pgfqpoint{7.195487in}{1.566370in}}{\pgfqpoint{7.190649in}{1.568374in}}{\pgfqpoint{7.185606in}{1.568374in}}%
\pgfpathcurveto{\pgfqpoint{7.180562in}{1.568374in}}{\pgfqpoint{7.175724in}{1.566370in}}{\pgfqpoint{7.172158in}{1.562804in}}%
\pgfpathcurveto{\pgfqpoint{7.168591in}{1.559238in}}{\pgfqpoint{7.166587in}{1.554400in}}{\pgfqpoint{7.166587in}{1.549356in}}%
\pgfpathcurveto{\pgfqpoint{7.166587in}{1.544312in}}{\pgfqpoint{7.168591in}{1.539475in}}{\pgfqpoint{7.172158in}{1.535908in}}%
\pgfpathcurveto{\pgfqpoint{7.175724in}{1.532342in}}{\pgfqpoint{7.180562in}{1.530338in}}{\pgfqpoint{7.185606in}{1.530338in}}%
\pgfpathclose%
\pgfusepath{fill}%
\end{pgfscope}%
\begin{pgfscope}%
\pgfpathrectangle{\pgfqpoint{6.572727in}{0.473000in}}{\pgfqpoint{4.227273in}{3.311000in}}%
\pgfusepath{clip}%
\pgfsetbuttcap%
\pgfsetroundjoin%
\definecolor{currentfill}{rgb}{0.127568,0.566949,0.550556}%
\pgfsetfillcolor{currentfill}%
\pgfsetfillopacity{0.700000}%
\pgfsetlinewidth{0.000000pt}%
\definecolor{currentstroke}{rgb}{0.000000,0.000000,0.000000}%
\pgfsetstrokecolor{currentstroke}%
\pgfsetstrokeopacity{0.700000}%
\pgfsetdash{}{0pt}%
\pgfpathmoveto{\pgfqpoint{8.430377in}{1.508306in}}%
\pgfpathcurveto{\pgfqpoint{8.435420in}{1.508306in}}{\pgfqpoint{8.440258in}{1.510310in}}{\pgfqpoint{8.443825in}{1.513876in}}%
\pgfpathcurveto{\pgfqpoint{8.447391in}{1.517443in}}{\pgfqpoint{8.449395in}{1.522281in}}{\pgfqpoint{8.449395in}{1.527324in}}%
\pgfpathcurveto{\pgfqpoint{8.449395in}{1.532368in}}{\pgfqpoint{8.447391in}{1.537206in}}{\pgfqpoint{8.443825in}{1.540772in}}%
\pgfpathcurveto{\pgfqpoint{8.440258in}{1.544338in}}{\pgfqpoint{8.435420in}{1.546342in}}{\pgfqpoint{8.430377in}{1.546342in}}%
\pgfpathcurveto{\pgfqpoint{8.425333in}{1.546342in}}{\pgfqpoint{8.420495in}{1.544338in}}{\pgfqpoint{8.416929in}{1.540772in}}%
\pgfpathcurveto{\pgfqpoint{8.413363in}{1.537206in}}{\pgfqpoint{8.411359in}{1.532368in}}{\pgfqpoint{8.411359in}{1.527324in}}%
\pgfpathcurveto{\pgfqpoint{8.411359in}{1.522281in}}{\pgfqpoint{8.413363in}{1.517443in}}{\pgfqpoint{8.416929in}{1.513876in}}%
\pgfpathcurveto{\pgfqpoint{8.420495in}{1.510310in}}{\pgfqpoint{8.425333in}{1.508306in}}{\pgfqpoint{8.430377in}{1.508306in}}%
\pgfpathclose%
\pgfusepath{fill}%
\end{pgfscope}%
\begin{pgfscope}%
\pgfpathrectangle{\pgfqpoint{6.572727in}{0.473000in}}{\pgfqpoint{4.227273in}{3.311000in}}%
\pgfusepath{clip}%
\pgfsetbuttcap%
\pgfsetroundjoin%
\definecolor{currentfill}{rgb}{0.993248,0.906157,0.143936}%
\pgfsetfillcolor{currentfill}%
\pgfsetfillopacity{0.700000}%
\pgfsetlinewidth{0.000000pt}%
\definecolor{currentstroke}{rgb}{0.000000,0.000000,0.000000}%
\pgfsetstrokecolor{currentstroke}%
\pgfsetstrokeopacity{0.700000}%
\pgfsetdash{}{0pt}%
\pgfpathmoveto{\pgfqpoint{9.990781in}{1.752599in}}%
\pgfpathcurveto{\pgfqpoint{9.995825in}{1.752599in}}{\pgfqpoint{10.000662in}{1.754603in}}{\pgfqpoint{10.004229in}{1.758169in}}%
\pgfpathcurveto{\pgfqpoint{10.007795in}{1.761736in}}{\pgfqpoint{10.009799in}{1.766573in}}{\pgfqpoint{10.009799in}{1.771617in}}%
\pgfpathcurveto{\pgfqpoint{10.009799in}{1.776661in}}{\pgfqpoint{10.007795in}{1.781499in}}{\pgfqpoint{10.004229in}{1.785065in}}%
\pgfpathcurveto{\pgfqpoint{10.000662in}{1.788631in}}{\pgfqpoint{9.995825in}{1.790635in}}{\pgfqpoint{9.990781in}{1.790635in}}%
\pgfpathcurveto{\pgfqpoint{9.985737in}{1.790635in}}{\pgfqpoint{9.980900in}{1.788631in}}{\pgfqpoint{9.977333in}{1.785065in}}%
\pgfpathcurveto{\pgfqpoint{9.973767in}{1.781499in}}{\pgfqpoint{9.971763in}{1.776661in}}{\pgfqpoint{9.971763in}{1.771617in}}%
\pgfpathcurveto{\pgfqpoint{9.971763in}{1.766573in}}{\pgfqpoint{9.973767in}{1.761736in}}{\pgfqpoint{9.977333in}{1.758169in}}%
\pgfpathcurveto{\pgfqpoint{9.980900in}{1.754603in}}{\pgfqpoint{9.985737in}{1.752599in}}{\pgfqpoint{9.990781in}{1.752599in}}%
\pgfpathclose%
\pgfusepath{fill}%
\end{pgfscope}%
\begin{pgfscope}%
\pgfpathrectangle{\pgfqpoint{6.572727in}{0.473000in}}{\pgfqpoint{4.227273in}{3.311000in}}%
\pgfusepath{clip}%
\pgfsetbuttcap%
\pgfsetroundjoin%
\definecolor{currentfill}{rgb}{0.993248,0.906157,0.143936}%
\pgfsetfillcolor{currentfill}%
\pgfsetfillopacity{0.700000}%
\pgfsetlinewidth{0.000000pt}%
\definecolor{currentstroke}{rgb}{0.000000,0.000000,0.000000}%
\pgfsetstrokecolor{currentstroke}%
\pgfsetstrokeopacity{0.700000}%
\pgfsetdash{}{0pt}%
\pgfpathmoveto{\pgfqpoint{9.653179in}{1.277949in}}%
\pgfpathcurveto{\pgfqpoint{9.658223in}{1.277949in}}{\pgfqpoint{9.663060in}{1.279953in}}{\pgfqpoint{9.666627in}{1.283519in}}%
\pgfpathcurveto{\pgfqpoint{9.670193in}{1.287086in}}{\pgfqpoint{9.672197in}{1.291924in}}{\pgfqpoint{9.672197in}{1.296967in}}%
\pgfpathcurveto{\pgfqpoint{9.672197in}{1.302011in}}{\pgfqpoint{9.670193in}{1.306849in}}{\pgfqpoint{9.666627in}{1.310415in}}%
\pgfpathcurveto{\pgfqpoint{9.663060in}{1.313982in}}{\pgfqpoint{9.658223in}{1.315985in}}{\pgfqpoint{9.653179in}{1.315985in}}%
\pgfpathcurveto{\pgfqpoint{9.648135in}{1.315985in}}{\pgfqpoint{9.643298in}{1.313982in}}{\pgfqpoint{9.639731in}{1.310415in}}%
\pgfpathcurveto{\pgfqpoint{9.636165in}{1.306849in}}{\pgfqpoint{9.634161in}{1.302011in}}{\pgfqpoint{9.634161in}{1.296967in}}%
\pgfpathcurveto{\pgfqpoint{9.634161in}{1.291924in}}{\pgfqpoint{9.636165in}{1.287086in}}{\pgfqpoint{9.639731in}{1.283519in}}%
\pgfpathcurveto{\pgfqpoint{9.643298in}{1.279953in}}{\pgfqpoint{9.648135in}{1.277949in}}{\pgfqpoint{9.653179in}{1.277949in}}%
\pgfpathclose%
\pgfusepath{fill}%
\end{pgfscope}%
\begin{pgfscope}%
\pgfpathrectangle{\pgfqpoint{6.572727in}{0.473000in}}{\pgfqpoint{4.227273in}{3.311000in}}%
\pgfusepath{clip}%
\pgfsetbuttcap%
\pgfsetroundjoin%
\definecolor{currentfill}{rgb}{0.993248,0.906157,0.143936}%
\pgfsetfillcolor{currentfill}%
\pgfsetfillopacity{0.700000}%
\pgfsetlinewidth{0.000000pt}%
\definecolor{currentstroke}{rgb}{0.000000,0.000000,0.000000}%
\pgfsetstrokecolor{currentstroke}%
\pgfsetstrokeopacity{0.700000}%
\pgfsetdash{}{0pt}%
\pgfpathmoveto{\pgfqpoint{9.630854in}{1.567030in}}%
\pgfpathcurveto{\pgfqpoint{9.635898in}{1.567030in}}{\pgfqpoint{9.640735in}{1.569034in}}{\pgfqpoint{9.644302in}{1.572600in}}%
\pgfpathcurveto{\pgfqpoint{9.647868in}{1.576166in}}{\pgfqpoint{9.649872in}{1.581004in}}{\pgfqpoint{9.649872in}{1.586048in}}%
\pgfpathcurveto{\pgfqpoint{9.649872in}{1.591092in}}{\pgfqpoint{9.647868in}{1.595929in}}{\pgfqpoint{9.644302in}{1.599496in}}%
\pgfpathcurveto{\pgfqpoint{9.640735in}{1.603062in}}{\pgfqpoint{9.635898in}{1.605066in}}{\pgfqpoint{9.630854in}{1.605066in}}%
\pgfpathcurveto{\pgfqpoint{9.625810in}{1.605066in}}{\pgfqpoint{9.620972in}{1.603062in}}{\pgfqpoint{9.617406in}{1.599496in}}%
\pgfpathcurveto{\pgfqpoint{9.613840in}{1.595929in}}{\pgfqpoint{9.611836in}{1.591092in}}{\pgfqpoint{9.611836in}{1.586048in}}%
\pgfpathcurveto{\pgfqpoint{9.611836in}{1.581004in}}{\pgfqpoint{9.613840in}{1.576166in}}{\pgfqpoint{9.617406in}{1.572600in}}%
\pgfpathcurveto{\pgfqpoint{9.620972in}{1.569034in}}{\pgfqpoint{9.625810in}{1.567030in}}{\pgfqpoint{9.630854in}{1.567030in}}%
\pgfpathclose%
\pgfusepath{fill}%
\end{pgfscope}%
\begin{pgfscope}%
\pgfpathrectangle{\pgfqpoint{6.572727in}{0.473000in}}{\pgfqpoint{4.227273in}{3.311000in}}%
\pgfusepath{clip}%
\pgfsetbuttcap%
\pgfsetroundjoin%
\definecolor{currentfill}{rgb}{0.127568,0.566949,0.550556}%
\pgfsetfillcolor{currentfill}%
\pgfsetfillopacity{0.700000}%
\pgfsetlinewidth{0.000000pt}%
\definecolor{currentstroke}{rgb}{0.000000,0.000000,0.000000}%
\pgfsetstrokecolor{currentstroke}%
\pgfsetstrokeopacity{0.700000}%
\pgfsetdash{}{0pt}%
\pgfpathmoveto{\pgfqpoint{7.593807in}{1.156455in}}%
\pgfpathcurveto{\pgfqpoint{7.598851in}{1.156455in}}{\pgfqpoint{7.603689in}{1.158458in}}{\pgfqpoint{7.607255in}{1.162025in}}%
\pgfpathcurveto{\pgfqpoint{7.610822in}{1.165591in}}{\pgfqpoint{7.612825in}{1.170429in}}{\pgfqpoint{7.612825in}{1.175473in}}%
\pgfpathcurveto{\pgfqpoint{7.612825in}{1.180516in}}{\pgfqpoint{7.610822in}{1.185354in}}{\pgfqpoint{7.607255in}{1.188921in}}%
\pgfpathcurveto{\pgfqpoint{7.603689in}{1.192487in}}{\pgfqpoint{7.598851in}{1.194491in}}{\pgfqpoint{7.593807in}{1.194491in}}%
\pgfpathcurveto{\pgfqpoint{7.588764in}{1.194491in}}{\pgfqpoint{7.583926in}{1.192487in}}{\pgfqpoint{7.580359in}{1.188921in}}%
\pgfpathcurveto{\pgfqpoint{7.576793in}{1.185354in}}{\pgfqpoint{7.574789in}{1.180516in}}{\pgfqpoint{7.574789in}{1.175473in}}%
\pgfpathcurveto{\pgfqpoint{7.574789in}{1.170429in}}{\pgfqpoint{7.576793in}{1.165591in}}{\pgfqpoint{7.580359in}{1.162025in}}%
\pgfpathcurveto{\pgfqpoint{7.583926in}{1.158458in}}{\pgfqpoint{7.588764in}{1.156455in}}{\pgfqpoint{7.593807in}{1.156455in}}%
\pgfpathclose%
\pgfusepath{fill}%
\end{pgfscope}%
\begin{pgfscope}%
\pgfpathrectangle{\pgfqpoint{6.572727in}{0.473000in}}{\pgfqpoint{4.227273in}{3.311000in}}%
\pgfusepath{clip}%
\pgfsetbuttcap%
\pgfsetroundjoin%
\definecolor{currentfill}{rgb}{0.127568,0.566949,0.550556}%
\pgfsetfillcolor{currentfill}%
\pgfsetfillopacity{0.700000}%
\pgfsetlinewidth{0.000000pt}%
\definecolor{currentstroke}{rgb}{0.000000,0.000000,0.000000}%
\pgfsetstrokecolor{currentstroke}%
\pgfsetstrokeopacity{0.700000}%
\pgfsetdash{}{0pt}%
\pgfpathmoveto{\pgfqpoint{7.796717in}{1.529499in}}%
\pgfpathcurveto{\pgfqpoint{7.801761in}{1.529499in}}{\pgfqpoint{7.806599in}{1.531503in}}{\pgfqpoint{7.810165in}{1.535069in}}%
\pgfpathcurveto{\pgfqpoint{7.813732in}{1.538636in}}{\pgfqpoint{7.815736in}{1.543474in}}{\pgfqpoint{7.815736in}{1.548517in}}%
\pgfpathcurveto{\pgfqpoint{7.815736in}{1.553561in}}{\pgfqpoint{7.813732in}{1.558399in}}{\pgfqpoint{7.810165in}{1.561965in}}%
\pgfpathcurveto{\pgfqpoint{7.806599in}{1.565532in}}{\pgfqpoint{7.801761in}{1.567535in}}{\pgfqpoint{7.796717in}{1.567535in}}%
\pgfpathcurveto{\pgfqpoint{7.791674in}{1.567535in}}{\pgfqpoint{7.786836in}{1.565532in}}{\pgfqpoint{7.783270in}{1.561965in}}%
\pgfpathcurveto{\pgfqpoint{7.779703in}{1.558399in}}{\pgfqpoint{7.777699in}{1.553561in}}{\pgfqpoint{7.777699in}{1.548517in}}%
\pgfpathcurveto{\pgfqpoint{7.777699in}{1.543474in}}{\pgfqpoint{7.779703in}{1.538636in}}{\pgfqpoint{7.783270in}{1.535069in}}%
\pgfpathcurveto{\pgfqpoint{7.786836in}{1.531503in}}{\pgfqpoint{7.791674in}{1.529499in}}{\pgfqpoint{7.796717in}{1.529499in}}%
\pgfpathclose%
\pgfusepath{fill}%
\end{pgfscope}%
\begin{pgfscope}%
\pgfpathrectangle{\pgfqpoint{6.572727in}{0.473000in}}{\pgfqpoint{4.227273in}{3.311000in}}%
\pgfusepath{clip}%
\pgfsetbuttcap%
\pgfsetroundjoin%
\definecolor{currentfill}{rgb}{0.993248,0.906157,0.143936}%
\pgfsetfillcolor{currentfill}%
\pgfsetfillopacity{0.700000}%
\pgfsetlinewidth{0.000000pt}%
\definecolor{currentstroke}{rgb}{0.000000,0.000000,0.000000}%
\pgfsetstrokecolor{currentstroke}%
\pgfsetstrokeopacity{0.700000}%
\pgfsetdash{}{0pt}%
\pgfpathmoveto{\pgfqpoint{9.555610in}{1.852643in}}%
\pgfpathcurveto{\pgfqpoint{9.560654in}{1.852643in}}{\pgfqpoint{9.565491in}{1.854647in}}{\pgfqpoint{9.569058in}{1.858213in}}%
\pgfpathcurveto{\pgfqpoint{9.572624in}{1.861780in}}{\pgfqpoint{9.574628in}{1.866618in}}{\pgfqpoint{9.574628in}{1.871661in}}%
\pgfpathcurveto{\pgfqpoint{9.574628in}{1.876705in}}{\pgfqpoint{9.572624in}{1.881543in}}{\pgfqpoint{9.569058in}{1.885109in}}%
\pgfpathcurveto{\pgfqpoint{9.565491in}{1.888676in}}{\pgfqpoint{9.560654in}{1.890679in}}{\pgfqpoint{9.555610in}{1.890679in}}%
\pgfpathcurveto{\pgfqpoint{9.550566in}{1.890679in}}{\pgfqpoint{9.545728in}{1.888676in}}{\pgfqpoint{9.542162in}{1.885109in}}%
\pgfpathcurveto{\pgfqpoint{9.538596in}{1.881543in}}{\pgfqpoint{9.536592in}{1.876705in}}{\pgfqpoint{9.536592in}{1.871661in}}%
\pgfpathcurveto{\pgfqpoint{9.536592in}{1.866618in}}{\pgfqpoint{9.538596in}{1.861780in}}{\pgfqpoint{9.542162in}{1.858213in}}%
\pgfpathcurveto{\pgfqpoint{9.545728in}{1.854647in}}{\pgfqpoint{9.550566in}{1.852643in}}{\pgfqpoint{9.555610in}{1.852643in}}%
\pgfpathclose%
\pgfusepath{fill}%
\end{pgfscope}%
\begin{pgfscope}%
\pgfpathrectangle{\pgfqpoint{6.572727in}{0.473000in}}{\pgfqpoint{4.227273in}{3.311000in}}%
\pgfusepath{clip}%
\pgfsetbuttcap%
\pgfsetroundjoin%
\definecolor{currentfill}{rgb}{0.127568,0.566949,0.550556}%
\pgfsetfillcolor{currentfill}%
\pgfsetfillopacity{0.700000}%
\pgfsetlinewidth{0.000000pt}%
\definecolor{currentstroke}{rgb}{0.000000,0.000000,0.000000}%
\pgfsetstrokecolor{currentstroke}%
\pgfsetstrokeopacity{0.700000}%
\pgfsetdash{}{0pt}%
\pgfpathmoveto{\pgfqpoint{7.679271in}{0.884142in}}%
\pgfpathcurveto{\pgfqpoint{7.684315in}{0.884142in}}{\pgfqpoint{7.689153in}{0.886146in}}{\pgfqpoint{7.692719in}{0.889712in}}%
\pgfpathcurveto{\pgfqpoint{7.696286in}{0.893279in}}{\pgfqpoint{7.698290in}{0.898116in}}{\pgfqpoint{7.698290in}{0.903160in}}%
\pgfpathcurveto{\pgfqpoint{7.698290in}{0.908204in}}{\pgfqpoint{7.696286in}{0.913042in}}{\pgfqpoint{7.692719in}{0.916608in}}%
\pgfpathcurveto{\pgfqpoint{7.689153in}{0.920174in}}{\pgfqpoint{7.684315in}{0.922178in}}{\pgfqpoint{7.679271in}{0.922178in}}%
\pgfpathcurveto{\pgfqpoint{7.674228in}{0.922178in}}{\pgfqpoint{7.669390in}{0.920174in}}{\pgfqpoint{7.665824in}{0.916608in}}%
\pgfpathcurveto{\pgfqpoint{7.662257in}{0.913042in}}{\pgfqpoint{7.660253in}{0.908204in}}{\pgfqpoint{7.660253in}{0.903160in}}%
\pgfpathcurveto{\pgfqpoint{7.660253in}{0.898116in}}{\pgfqpoint{7.662257in}{0.893279in}}{\pgfqpoint{7.665824in}{0.889712in}}%
\pgfpathcurveto{\pgfqpoint{7.669390in}{0.886146in}}{\pgfqpoint{7.674228in}{0.884142in}}{\pgfqpoint{7.679271in}{0.884142in}}%
\pgfpathclose%
\pgfusepath{fill}%
\end{pgfscope}%
\begin{pgfscope}%
\pgfpathrectangle{\pgfqpoint{6.572727in}{0.473000in}}{\pgfqpoint{4.227273in}{3.311000in}}%
\pgfusepath{clip}%
\pgfsetbuttcap%
\pgfsetroundjoin%
\definecolor{currentfill}{rgb}{0.127568,0.566949,0.550556}%
\pgfsetfillcolor{currentfill}%
\pgfsetfillopacity{0.700000}%
\pgfsetlinewidth{0.000000pt}%
\definecolor{currentstroke}{rgb}{0.000000,0.000000,0.000000}%
\pgfsetstrokecolor{currentstroke}%
\pgfsetstrokeopacity{0.700000}%
\pgfsetdash{}{0pt}%
\pgfpathmoveto{\pgfqpoint{8.637455in}{2.655699in}}%
\pgfpathcurveto{\pgfqpoint{8.642498in}{2.655699in}}{\pgfqpoint{8.647336in}{2.657703in}}{\pgfqpoint{8.650903in}{2.661269in}}%
\pgfpathcurveto{\pgfqpoint{8.654469in}{2.664836in}}{\pgfqpoint{8.656473in}{2.669674in}}{\pgfqpoint{8.656473in}{2.674717in}}%
\pgfpathcurveto{\pgfqpoint{8.656473in}{2.679761in}}{\pgfqpoint{8.654469in}{2.684599in}}{\pgfqpoint{8.650903in}{2.688165in}}%
\pgfpathcurveto{\pgfqpoint{8.647336in}{2.691731in}}{\pgfqpoint{8.642498in}{2.693735in}}{\pgfqpoint{8.637455in}{2.693735in}}%
\pgfpathcurveto{\pgfqpoint{8.632411in}{2.693735in}}{\pgfqpoint{8.627573in}{2.691731in}}{\pgfqpoint{8.624007in}{2.688165in}}%
\pgfpathcurveto{\pgfqpoint{8.620440in}{2.684599in}}{\pgfqpoint{8.618437in}{2.679761in}}{\pgfqpoint{8.618437in}{2.674717in}}%
\pgfpathcurveto{\pgfqpoint{8.618437in}{2.669674in}}{\pgfqpoint{8.620440in}{2.664836in}}{\pgfqpoint{8.624007in}{2.661269in}}%
\pgfpathcurveto{\pgfqpoint{8.627573in}{2.657703in}}{\pgfqpoint{8.632411in}{2.655699in}}{\pgfqpoint{8.637455in}{2.655699in}}%
\pgfpathclose%
\pgfusepath{fill}%
\end{pgfscope}%
\begin{pgfscope}%
\pgfpathrectangle{\pgfqpoint{6.572727in}{0.473000in}}{\pgfqpoint{4.227273in}{3.311000in}}%
\pgfusepath{clip}%
\pgfsetbuttcap%
\pgfsetroundjoin%
\definecolor{currentfill}{rgb}{0.993248,0.906157,0.143936}%
\pgfsetfillcolor{currentfill}%
\pgfsetfillopacity{0.700000}%
\pgfsetlinewidth{0.000000pt}%
\definecolor{currentstroke}{rgb}{0.000000,0.000000,0.000000}%
\pgfsetstrokecolor{currentstroke}%
\pgfsetstrokeopacity{0.700000}%
\pgfsetdash{}{0pt}%
\pgfpathmoveto{\pgfqpoint{9.117483in}{1.720408in}}%
\pgfpathcurveto{\pgfqpoint{9.122527in}{1.720408in}}{\pgfqpoint{9.127364in}{1.722412in}}{\pgfqpoint{9.130931in}{1.725978in}}%
\pgfpathcurveto{\pgfqpoint{9.134497in}{1.729545in}}{\pgfqpoint{9.136501in}{1.734382in}}{\pgfqpoint{9.136501in}{1.739426in}}%
\pgfpathcurveto{\pgfqpoint{9.136501in}{1.744470in}}{\pgfqpoint{9.134497in}{1.749307in}}{\pgfqpoint{9.130931in}{1.752874in}}%
\pgfpathcurveto{\pgfqpoint{9.127364in}{1.756440in}}{\pgfqpoint{9.122527in}{1.758444in}}{\pgfqpoint{9.117483in}{1.758444in}}%
\pgfpathcurveto{\pgfqpoint{9.112439in}{1.758444in}}{\pgfqpoint{9.107602in}{1.756440in}}{\pgfqpoint{9.104035in}{1.752874in}}%
\pgfpathcurveto{\pgfqpoint{9.100469in}{1.749307in}}{\pgfqpoint{9.098465in}{1.744470in}}{\pgfqpoint{9.098465in}{1.739426in}}%
\pgfpathcurveto{\pgfqpoint{9.098465in}{1.734382in}}{\pgfqpoint{9.100469in}{1.729545in}}{\pgfqpoint{9.104035in}{1.725978in}}%
\pgfpathcurveto{\pgfqpoint{9.107602in}{1.722412in}}{\pgfqpoint{9.112439in}{1.720408in}}{\pgfqpoint{9.117483in}{1.720408in}}%
\pgfpathclose%
\pgfusepath{fill}%
\end{pgfscope}%
\begin{pgfscope}%
\pgfpathrectangle{\pgfqpoint{6.572727in}{0.473000in}}{\pgfqpoint{4.227273in}{3.311000in}}%
\pgfusepath{clip}%
\pgfsetbuttcap%
\pgfsetroundjoin%
\definecolor{currentfill}{rgb}{0.993248,0.906157,0.143936}%
\pgfsetfillcolor{currentfill}%
\pgfsetfillopacity{0.700000}%
\pgfsetlinewidth{0.000000pt}%
\definecolor{currentstroke}{rgb}{0.000000,0.000000,0.000000}%
\pgfsetstrokecolor{currentstroke}%
\pgfsetstrokeopacity{0.700000}%
\pgfsetdash{}{0pt}%
\pgfpathmoveto{\pgfqpoint{9.693644in}{1.785796in}}%
\pgfpathcurveto{\pgfqpoint{9.698688in}{1.785796in}}{\pgfqpoint{9.703526in}{1.787799in}}{\pgfqpoint{9.707092in}{1.791366in}}%
\pgfpathcurveto{\pgfqpoint{9.710658in}{1.794932in}}{\pgfqpoint{9.712662in}{1.799770in}}{\pgfqpoint{9.712662in}{1.804814in}}%
\pgfpathcurveto{\pgfqpoint{9.712662in}{1.809857in}}{\pgfqpoint{9.710658in}{1.814695in}}{\pgfqpoint{9.707092in}{1.818262in}}%
\pgfpathcurveto{\pgfqpoint{9.703526in}{1.821828in}}{\pgfqpoint{9.698688in}{1.823832in}}{\pgfqpoint{9.693644in}{1.823832in}}%
\pgfpathcurveto{\pgfqpoint{9.688600in}{1.823832in}}{\pgfqpoint{9.683763in}{1.821828in}}{\pgfqpoint{9.680196in}{1.818262in}}%
\pgfpathcurveto{\pgfqpoint{9.676630in}{1.814695in}}{\pgfqpoint{9.674626in}{1.809857in}}{\pgfqpoint{9.674626in}{1.804814in}}%
\pgfpathcurveto{\pgfqpoint{9.674626in}{1.799770in}}{\pgfqpoint{9.676630in}{1.794932in}}{\pgfqpoint{9.680196in}{1.791366in}}%
\pgfpathcurveto{\pgfqpoint{9.683763in}{1.787799in}}{\pgfqpoint{9.688600in}{1.785796in}}{\pgfqpoint{9.693644in}{1.785796in}}%
\pgfpathclose%
\pgfusepath{fill}%
\end{pgfscope}%
\begin{pgfscope}%
\pgfpathrectangle{\pgfqpoint{6.572727in}{0.473000in}}{\pgfqpoint{4.227273in}{3.311000in}}%
\pgfusepath{clip}%
\pgfsetbuttcap%
\pgfsetroundjoin%
\definecolor{currentfill}{rgb}{0.993248,0.906157,0.143936}%
\pgfsetfillcolor{currentfill}%
\pgfsetfillopacity{0.700000}%
\pgfsetlinewidth{0.000000pt}%
\definecolor{currentstroke}{rgb}{0.000000,0.000000,0.000000}%
\pgfsetstrokecolor{currentstroke}%
\pgfsetstrokeopacity{0.700000}%
\pgfsetdash{}{0pt}%
\pgfpathmoveto{\pgfqpoint{9.619127in}{1.966980in}}%
\pgfpathcurveto{\pgfqpoint{9.624170in}{1.966980in}}{\pgfqpoint{9.629008in}{1.968984in}}{\pgfqpoint{9.632574in}{1.972551in}}%
\pgfpathcurveto{\pgfqpoint{9.636141in}{1.976117in}}{\pgfqpoint{9.638145in}{1.980955in}}{\pgfqpoint{9.638145in}{1.985998in}}%
\pgfpathcurveto{\pgfqpoint{9.638145in}{1.991042in}}{\pgfqpoint{9.636141in}{1.995880in}}{\pgfqpoint{9.632574in}{1.999446in}}%
\pgfpathcurveto{\pgfqpoint{9.629008in}{2.003013in}}{\pgfqpoint{9.624170in}{2.005017in}}{\pgfqpoint{9.619127in}{2.005017in}}%
\pgfpathcurveto{\pgfqpoint{9.614083in}{2.005017in}}{\pgfqpoint{9.609245in}{2.003013in}}{\pgfqpoint{9.605679in}{1.999446in}}%
\pgfpathcurveto{\pgfqpoint{9.602112in}{1.995880in}}{\pgfqpoint{9.600108in}{1.991042in}}{\pgfqpoint{9.600108in}{1.985998in}}%
\pgfpathcurveto{\pgfqpoint{9.600108in}{1.980955in}}{\pgfqpoint{9.602112in}{1.976117in}}{\pgfqpoint{9.605679in}{1.972551in}}%
\pgfpathcurveto{\pgfqpoint{9.609245in}{1.968984in}}{\pgfqpoint{9.614083in}{1.966980in}}{\pgfqpoint{9.619127in}{1.966980in}}%
\pgfpathclose%
\pgfusepath{fill}%
\end{pgfscope}%
\begin{pgfscope}%
\pgfpathrectangle{\pgfqpoint{6.572727in}{0.473000in}}{\pgfqpoint{4.227273in}{3.311000in}}%
\pgfusepath{clip}%
\pgfsetbuttcap%
\pgfsetroundjoin%
\definecolor{currentfill}{rgb}{0.127568,0.566949,0.550556}%
\pgfsetfillcolor{currentfill}%
\pgfsetfillopacity{0.700000}%
\pgfsetlinewidth{0.000000pt}%
\definecolor{currentstroke}{rgb}{0.000000,0.000000,0.000000}%
\pgfsetstrokecolor{currentstroke}%
\pgfsetstrokeopacity{0.700000}%
\pgfsetdash{}{0pt}%
\pgfpathmoveto{\pgfqpoint{7.627012in}{2.509971in}}%
\pgfpathcurveto{\pgfqpoint{7.632056in}{2.509971in}}{\pgfqpoint{7.636894in}{2.511975in}}{\pgfqpoint{7.640460in}{2.515542in}}%
\pgfpathcurveto{\pgfqpoint{7.644027in}{2.519108in}}{\pgfqpoint{7.646031in}{2.523946in}}{\pgfqpoint{7.646031in}{2.528990in}}%
\pgfpathcurveto{\pgfqpoint{7.646031in}{2.534033in}}{\pgfqpoint{7.644027in}{2.538871in}}{\pgfqpoint{7.640460in}{2.542437in}}%
\pgfpathcurveto{\pgfqpoint{7.636894in}{2.546004in}}{\pgfqpoint{7.632056in}{2.548008in}}{\pgfqpoint{7.627012in}{2.548008in}}%
\pgfpathcurveto{\pgfqpoint{7.621969in}{2.548008in}}{\pgfqpoint{7.617131in}{2.546004in}}{\pgfqpoint{7.613565in}{2.542437in}}%
\pgfpathcurveto{\pgfqpoint{7.609998in}{2.538871in}}{\pgfqpoint{7.607994in}{2.534033in}}{\pgfqpoint{7.607994in}{2.528990in}}%
\pgfpathcurveto{\pgfqpoint{7.607994in}{2.523946in}}{\pgfqpoint{7.609998in}{2.519108in}}{\pgfqpoint{7.613565in}{2.515542in}}%
\pgfpathcurveto{\pgfqpoint{7.617131in}{2.511975in}}{\pgfqpoint{7.621969in}{2.509971in}}{\pgfqpoint{7.627012in}{2.509971in}}%
\pgfpathclose%
\pgfusepath{fill}%
\end{pgfscope}%
\begin{pgfscope}%
\pgfpathrectangle{\pgfqpoint{6.572727in}{0.473000in}}{\pgfqpoint{4.227273in}{3.311000in}}%
\pgfusepath{clip}%
\pgfsetbuttcap%
\pgfsetroundjoin%
\definecolor{currentfill}{rgb}{0.127568,0.566949,0.550556}%
\pgfsetfillcolor{currentfill}%
\pgfsetfillopacity{0.700000}%
\pgfsetlinewidth{0.000000pt}%
\definecolor{currentstroke}{rgb}{0.000000,0.000000,0.000000}%
\pgfsetstrokecolor{currentstroke}%
\pgfsetstrokeopacity{0.700000}%
\pgfsetdash{}{0pt}%
\pgfpathmoveto{\pgfqpoint{7.374058in}{3.096455in}}%
\pgfpathcurveto{\pgfqpoint{7.379101in}{3.096455in}}{\pgfqpoint{7.383939in}{3.098459in}}{\pgfqpoint{7.387505in}{3.102026in}}%
\pgfpathcurveto{\pgfqpoint{7.391072in}{3.105592in}}{\pgfqpoint{7.393076in}{3.110430in}}{\pgfqpoint{7.393076in}{3.115474in}}%
\pgfpathcurveto{\pgfqpoint{7.393076in}{3.120517in}}{\pgfqpoint{7.391072in}{3.125355in}}{\pgfqpoint{7.387505in}{3.128921in}}%
\pgfpathcurveto{\pgfqpoint{7.383939in}{3.132488in}}{\pgfqpoint{7.379101in}{3.134492in}}{\pgfqpoint{7.374058in}{3.134492in}}%
\pgfpathcurveto{\pgfqpoint{7.369014in}{3.134492in}}{\pgfqpoint{7.364176in}{3.132488in}}{\pgfqpoint{7.360610in}{3.128921in}}%
\pgfpathcurveto{\pgfqpoint{7.357043in}{3.125355in}}{\pgfqpoint{7.355039in}{3.120517in}}{\pgfqpoint{7.355039in}{3.115474in}}%
\pgfpathcurveto{\pgfqpoint{7.355039in}{3.110430in}}{\pgfqpoint{7.357043in}{3.105592in}}{\pgfqpoint{7.360610in}{3.102026in}}%
\pgfpathcurveto{\pgfqpoint{7.364176in}{3.098459in}}{\pgfqpoint{7.369014in}{3.096455in}}{\pgfqpoint{7.374058in}{3.096455in}}%
\pgfpathclose%
\pgfusepath{fill}%
\end{pgfscope}%
\begin{pgfscope}%
\pgfpathrectangle{\pgfqpoint{6.572727in}{0.473000in}}{\pgfqpoint{4.227273in}{3.311000in}}%
\pgfusepath{clip}%
\pgfsetbuttcap%
\pgfsetroundjoin%
\definecolor{currentfill}{rgb}{0.993248,0.906157,0.143936}%
\pgfsetfillcolor{currentfill}%
\pgfsetfillopacity{0.700000}%
\pgfsetlinewidth{0.000000pt}%
\definecolor{currentstroke}{rgb}{0.000000,0.000000,0.000000}%
\pgfsetstrokecolor{currentstroke}%
\pgfsetstrokeopacity{0.700000}%
\pgfsetdash{}{0pt}%
\pgfpathmoveto{\pgfqpoint{9.408185in}{1.664910in}}%
\pgfpathcurveto{\pgfqpoint{9.413229in}{1.664910in}}{\pgfqpoint{9.418067in}{1.666913in}}{\pgfqpoint{9.421633in}{1.670480in}}%
\pgfpathcurveto{\pgfqpoint{9.425199in}{1.674046in}}{\pgfqpoint{9.427203in}{1.678884in}}{\pgfqpoint{9.427203in}{1.683928in}}%
\pgfpathcurveto{\pgfqpoint{9.427203in}{1.688971in}}{\pgfqpoint{9.425199in}{1.693809in}}{\pgfqpoint{9.421633in}{1.697376in}}%
\pgfpathcurveto{\pgfqpoint{9.418067in}{1.700942in}}{\pgfqpoint{9.413229in}{1.702946in}}{\pgfqpoint{9.408185in}{1.702946in}}%
\pgfpathcurveto{\pgfqpoint{9.403141in}{1.702946in}}{\pgfqpoint{9.398304in}{1.700942in}}{\pgfqpoint{9.394737in}{1.697376in}}%
\pgfpathcurveto{\pgfqpoint{9.391171in}{1.693809in}}{\pgfqpoint{9.389167in}{1.688971in}}{\pgfqpoint{9.389167in}{1.683928in}}%
\pgfpathcurveto{\pgfqpoint{9.389167in}{1.678884in}}{\pgfqpoint{9.391171in}{1.674046in}}{\pgfqpoint{9.394737in}{1.670480in}}%
\pgfpathcurveto{\pgfqpoint{9.398304in}{1.666913in}}{\pgfqpoint{9.403141in}{1.664910in}}{\pgfqpoint{9.408185in}{1.664910in}}%
\pgfpathclose%
\pgfusepath{fill}%
\end{pgfscope}%
\begin{pgfscope}%
\pgfpathrectangle{\pgfqpoint{6.572727in}{0.473000in}}{\pgfqpoint{4.227273in}{3.311000in}}%
\pgfusepath{clip}%
\pgfsetbuttcap%
\pgfsetroundjoin%
\definecolor{currentfill}{rgb}{0.993248,0.906157,0.143936}%
\pgfsetfillcolor{currentfill}%
\pgfsetfillopacity{0.700000}%
\pgfsetlinewidth{0.000000pt}%
\definecolor{currentstroke}{rgb}{0.000000,0.000000,0.000000}%
\pgfsetstrokecolor{currentstroke}%
\pgfsetstrokeopacity{0.700000}%
\pgfsetdash{}{0pt}%
\pgfpathmoveto{\pgfqpoint{9.627874in}{0.935643in}}%
\pgfpathcurveto{\pgfqpoint{9.632918in}{0.935643in}}{\pgfqpoint{9.637756in}{0.937647in}}{\pgfqpoint{9.641322in}{0.941213in}}%
\pgfpathcurveto{\pgfqpoint{9.644888in}{0.944780in}}{\pgfqpoint{9.646892in}{0.949617in}}{\pgfqpoint{9.646892in}{0.954661in}}%
\pgfpathcurveto{\pgfqpoint{9.646892in}{0.959705in}}{\pgfqpoint{9.644888in}{0.964542in}}{\pgfqpoint{9.641322in}{0.968109in}}%
\pgfpathcurveto{\pgfqpoint{9.637756in}{0.971675in}}{\pgfqpoint{9.632918in}{0.973679in}}{\pgfqpoint{9.627874in}{0.973679in}}%
\pgfpathcurveto{\pgfqpoint{9.622830in}{0.973679in}}{\pgfqpoint{9.617993in}{0.971675in}}{\pgfqpoint{9.614426in}{0.968109in}}%
\pgfpathcurveto{\pgfqpoint{9.610860in}{0.964542in}}{\pgfqpoint{9.608856in}{0.959705in}}{\pgfqpoint{9.608856in}{0.954661in}}%
\pgfpathcurveto{\pgfqpoint{9.608856in}{0.949617in}}{\pgfqpoint{9.610860in}{0.944780in}}{\pgfqpoint{9.614426in}{0.941213in}}%
\pgfpathcurveto{\pgfqpoint{9.617993in}{0.937647in}}{\pgfqpoint{9.622830in}{0.935643in}}{\pgfqpoint{9.627874in}{0.935643in}}%
\pgfpathclose%
\pgfusepath{fill}%
\end{pgfscope}%
\begin{pgfscope}%
\pgfpathrectangle{\pgfqpoint{6.572727in}{0.473000in}}{\pgfqpoint{4.227273in}{3.311000in}}%
\pgfusepath{clip}%
\pgfsetbuttcap%
\pgfsetroundjoin%
\definecolor{currentfill}{rgb}{0.993248,0.906157,0.143936}%
\pgfsetfillcolor{currentfill}%
\pgfsetfillopacity{0.700000}%
\pgfsetlinewidth{0.000000pt}%
\definecolor{currentstroke}{rgb}{0.000000,0.000000,0.000000}%
\pgfsetstrokecolor{currentstroke}%
\pgfsetstrokeopacity{0.700000}%
\pgfsetdash{}{0pt}%
\pgfpathmoveto{\pgfqpoint{9.445242in}{1.553658in}}%
\pgfpathcurveto{\pgfqpoint{9.450286in}{1.553658in}}{\pgfqpoint{9.455124in}{1.555662in}}{\pgfqpoint{9.458690in}{1.559229in}}%
\pgfpathcurveto{\pgfqpoint{9.462257in}{1.562795in}}{\pgfqpoint{9.464260in}{1.567633in}}{\pgfqpoint{9.464260in}{1.572676in}}%
\pgfpathcurveto{\pgfqpoint{9.464260in}{1.577720in}}{\pgfqpoint{9.462257in}{1.582558in}}{\pgfqpoint{9.458690in}{1.586124in}}%
\pgfpathcurveto{\pgfqpoint{9.455124in}{1.589691in}}{\pgfqpoint{9.450286in}{1.591695in}}{\pgfqpoint{9.445242in}{1.591695in}}%
\pgfpathcurveto{\pgfqpoint{9.440199in}{1.591695in}}{\pgfqpoint{9.435361in}{1.589691in}}{\pgfqpoint{9.431794in}{1.586124in}}%
\pgfpathcurveto{\pgfqpoint{9.428228in}{1.582558in}}{\pgfqpoint{9.426224in}{1.577720in}}{\pgfqpoint{9.426224in}{1.572676in}}%
\pgfpathcurveto{\pgfqpoint{9.426224in}{1.567633in}}{\pgfqpoint{9.428228in}{1.562795in}}{\pgfqpoint{9.431794in}{1.559229in}}%
\pgfpathcurveto{\pgfqpoint{9.435361in}{1.555662in}}{\pgfqpoint{9.440199in}{1.553658in}}{\pgfqpoint{9.445242in}{1.553658in}}%
\pgfpathclose%
\pgfusepath{fill}%
\end{pgfscope}%
\begin{pgfscope}%
\pgfpathrectangle{\pgfqpoint{6.572727in}{0.473000in}}{\pgfqpoint{4.227273in}{3.311000in}}%
\pgfusepath{clip}%
\pgfsetbuttcap%
\pgfsetroundjoin%
\definecolor{currentfill}{rgb}{0.127568,0.566949,0.550556}%
\pgfsetfillcolor{currentfill}%
\pgfsetfillopacity{0.700000}%
\pgfsetlinewidth{0.000000pt}%
\definecolor{currentstroke}{rgb}{0.000000,0.000000,0.000000}%
\pgfsetstrokecolor{currentstroke}%
\pgfsetstrokeopacity{0.700000}%
\pgfsetdash{}{0pt}%
\pgfpathmoveto{\pgfqpoint{8.011715in}{2.686025in}}%
\pgfpathcurveto{\pgfqpoint{8.016759in}{2.686025in}}{\pgfqpoint{8.021597in}{2.688029in}}{\pgfqpoint{8.025163in}{2.691595in}}%
\pgfpathcurveto{\pgfqpoint{8.028730in}{2.695161in}}{\pgfqpoint{8.030733in}{2.699999in}}{\pgfqpoint{8.030733in}{2.705043in}}%
\pgfpathcurveto{\pgfqpoint{8.030733in}{2.710087in}}{\pgfqpoint{8.028730in}{2.714924in}}{\pgfqpoint{8.025163in}{2.718491in}}%
\pgfpathcurveto{\pgfqpoint{8.021597in}{2.722057in}}{\pgfqpoint{8.016759in}{2.724061in}}{\pgfqpoint{8.011715in}{2.724061in}}%
\pgfpathcurveto{\pgfqpoint{8.006672in}{2.724061in}}{\pgfqpoint{8.001834in}{2.722057in}}{\pgfqpoint{7.998267in}{2.718491in}}%
\pgfpathcurveto{\pgfqpoint{7.994701in}{2.714924in}}{\pgfqpoint{7.992697in}{2.710087in}}{\pgfqpoint{7.992697in}{2.705043in}}%
\pgfpathcurveto{\pgfqpoint{7.992697in}{2.699999in}}{\pgfqpoint{7.994701in}{2.695161in}}{\pgfqpoint{7.998267in}{2.691595in}}%
\pgfpathcurveto{\pgfqpoint{8.001834in}{2.688029in}}{\pgfqpoint{8.006672in}{2.686025in}}{\pgfqpoint{8.011715in}{2.686025in}}%
\pgfpathclose%
\pgfusepath{fill}%
\end{pgfscope}%
\begin{pgfscope}%
\pgfpathrectangle{\pgfqpoint{6.572727in}{0.473000in}}{\pgfqpoint{4.227273in}{3.311000in}}%
\pgfusepath{clip}%
\pgfsetbuttcap%
\pgfsetroundjoin%
\definecolor{currentfill}{rgb}{0.127568,0.566949,0.550556}%
\pgfsetfillcolor{currentfill}%
\pgfsetfillopacity{0.700000}%
\pgfsetlinewidth{0.000000pt}%
\definecolor{currentstroke}{rgb}{0.000000,0.000000,0.000000}%
\pgfsetstrokecolor{currentstroke}%
\pgfsetstrokeopacity{0.700000}%
\pgfsetdash{}{0pt}%
\pgfpathmoveto{\pgfqpoint{8.222164in}{2.771482in}}%
\pgfpathcurveto{\pgfqpoint{8.227207in}{2.771482in}}{\pgfqpoint{8.232045in}{2.773486in}}{\pgfqpoint{8.235612in}{2.777053in}}%
\pgfpathcurveto{\pgfqpoint{8.239178in}{2.780619in}}{\pgfqpoint{8.241182in}{2.785457in}}{\pgfqpoint{8.241182in}{2.790500in}}%
\pgfpathcurveto{\pgfqpoint{8.241182in}{2.795544in}}{\pgfqpoint{8.239178in}{2.800382in}}{\pgfqpoint{8.235612in}{2.803948in}}%
\pgfpathcurveto{\pgfqpoint{8.232045in}{2.807515in}}{\pgfqpoint{8.227207in}{2.809519in}}{\pgfqpoint{8.222164in}{2.809519in}}%
\pgfpathcurveto{\pgfqpoint{8.217120in}{2.809519in}}{\pgfqpoint{8.212282in}{2.807515in}}{\pgfqpoint{8.208716in}{2.803948in}}%
\pgfpathcurveto{\pgfqpoint{8.205149in}{2.800382in}}{\pgfqpoint{8.203146in}{2.795544in}}{\pgfqpoint{8.203146in}{2.790500in}}%
\pgfpathcurveto{\pgfqpoint{8.203146in}{2.785457in}}{\pgfqpoint{8.205149in}{2.780619in}}{\pgfqpoint{8.208716in}{2.777053in}}%
\pgfpathcurveto{\pgfqpoint{8.212282in}{2.773486in}}{\pgfqpoint{8.217120in}{2.771482in}}{\pgfqpoint{8.222164in}{2.771482in}}%
\pgfpathclose%
\pgfusepath{fill}%
\end{pgfscope}%
\begin{pgfscope}%
\pgfpathrectangle{\pgfqpoint{6.572727in}{0.473000in}}{\pgfqpoint{4.227273in}{3.311000in}}%
\pgfusepath{clip}%
\pgfsetbuttcap%
\pgfsetroundjoin%
\definecolor{currentfill}{rgb}{0.127568,0.566949,0.550556}%
\pgfsetfillcolor{currentfill}%
\pgfsetfillopacity{0.700000}%
\pgfsetlinewidth{0.000000pt}%
\definecolor{currentstroke}{rgb}{0.000000,0.000000,0.000000}%
\pgfsetstrokecolor{currentstroke}%
\pgfsetstrokeopacity{0.700000}%
\pgfsetdash{}{0pt}%
\pgfpathmoveto{\pgfqpoint{7.855825in}{1.847378in}}%
\pgfpathcurveto{\pgfqpoint{7.860868in}{1.847378in}}{\pgfqpoint{7.865706in}{1.849382in}}{\pgfqpoint{7.869272in}{1.852948in}}%
\pgfpathcurveto{\pgfqpoint{7.872839in}{1.856515in}}{\pgfqpoint{7.874843in}{1.861353in}}{\pgfqpoint{7.874843in}{1.866396in}}%
\pgfpathcurveto{\pgfqpoint{7.874843in}{1.871440in}}{\pgfqpoint{7.872839in}{1.876278in}}{\pgfqpoint{7.869272in}{1.879844in}}%
\pgfpathcurveto{\pgfqpoint{7.865706in}{1.883411in}}{\pgfqpoint{7.860868in}{1.885414in}}{\pgfqpoint{7.855825in}{1.885414in}}%
\pgfpathcurveto{\pgfqpoint{7.850781in}{1.885414in}}{\pgfqpoint{7.845943in}{1.883411in}}{\pgfqpoint{7.842377in}{1.879844in}}%
\pgfpathcurveto{\pgfqpoint{7.838810in}{1.876278in}}{\pgfqpoint{7.836806in}{1.871440in}}{\pgfqpoint{7.836806in}{1.866396in}}%
\pgfpathcurveto{\pgfqpoint{7.836806in}{1.861353in}}{\pgfqpoint{7.838810in}{1.856515in}}{\pgfqpoint{7.842377in}{1.852948in}}%
\pgfpathcurveto{\pgfqpoint{7.845943in}{1.849382in}}{\pgfqpoint{7.850781in}{1.847378in}}{\pgfqpoint{7.855825in}{1.847378in}}%
\pgfpathclose%
\pgfusepath{fill}%
\end{pgfscope}%
\begin{pgfscope}%
\pgfpathrectangle{\pgfqpoint{6.572727in}{0.473000in}}{\pgfqpoint{4.227273in}{3.311000in}}%
\pgfusepath{clip}%
\pgfsetbuttcap%
\pgfsetroundjoin%
\definecolor{currentfill}{rgb}{0.127568,0.566949,0.550556}%
\pgfsetfillcolor{currentfill}%
\pgfsetfillopacity{0.700000}%
\pgfsetlinewidth{0.000000pt}%
\definecolor{currentstroke}{rgb}{0.000000,0.000000,0.000000}%
\pgfsetstrokecolor{currentstroke}%
\pgfsetstrokeopacity{0.700000}%
\pgfsetdash{}{0pt}%
\pgfpathmoveto{\pgfqpoint{8.054335in}{1.530889in}}%
\pgfpathcurveto{\pgfqpoint{8.059378in}{1.530889in}}{\pgfqpoint{8.064216in}{1.532892in}}{\pgfqpoint{8.067783in}{1.536459in}}%
\pgfpathcurveto{\pgfqpoint{8.071349in}{1.540025in}}{\pgfqpoint{8.073353in}{1.544863in}}{\pgfqpoint{8.073353in}{1.549907in}}%
\pgfpathcurveto{\pgfqpoint{8.073353in}{1.554950in}}{\pgfqpoint{8.071349in}{1.559788in}}{\pgfqpoint{8.067783in}{1.563355in}}%
\pgfpathcurveto{\pgfqpoint{8.064216in}{1.566921in}}{\pgfqpoint{8.059378in}{1.568925in}}{\pgfqpoint{8.054335in}{1.568925in}}%
\pgfpathcurveto{\pgfqpoint{8.049291in}{1.568925in}}{\pgfqpoint{8.044453in}{1.566921in}}{\pgfqpoint{8.040887in}{1.563355in}}%
\pgfpathcurveto{\pgfqpoint{8.037321in}{1.559788in}}{\pgfqpoint{8.035317in}{1.554950in}}{\pgfqpoint{8.035317in}{1.549907in}}%
\pgfpathcurveto{\pgfqpoint{8.035317in}{1.544863in}}{\pgfqpoint{8.037321in}{1.540025in}}{\pgfqpoint{8.040887in}{1.536459in}}%
\pgfpathcurveto{\pgfqpoint{8.044453in}{1.532892in}}{\pgfqpoint{8.049291in}{1.530889in}}{\pgfqpoint{8.054335in}{1.530889in}}%
\pgfpathclose%
\pgfusepath{fill}%
\end{pgfscope}%
\begin{pgfscope}%
\pgfpathrectangle{\pgfqpoint{6.572727in}{0.473000in}}{\pgfqpoint{4.227273in}{3.311000in}}%
\pgfusepath{clip}%
\pgfsetbuttcap%
\pgfsetroundjoin%
\definecolor{currentfill}{rgb}{0.127568,0.566949,0.550556}%
\pgfsetfillcolor{currentfill}%
\pgfsetfillopacity{0.700000}%
\pgfsetlinewidth{0.000000pt}%
\definecolor{currentstroke}{rgb}{0.000000,0.000000,0.000000}%
\pgfsetstrokecolor{currentstroke}%
\pgfsetstrokeopacity{0.700000}%
\pgfsetdash{}{0pt}%
\pgfpathmoveto{\pgfqpoint{8.437098in}{2.749990in}}%
\pgfpathcurveto{\pgfqpoint{8.442142in}{2.749990in}}{\pgfqpoint{8.446979in}{2.751994in}}{\pgfqpoint{8.450546in}{2.755560in}}%
\pgfpathcurveto{\pgfqpoint{8.454112in}{2.759127in}}{\pgfqpoint{8.456116in}{2.763965in}}{\pgfqpoint{8.456116in}{2.769008in}}%
\pgfpathcurveto{\pgfqpoint{8.456116in}{2.774052in}}{\pgfqpoint{8.454112in}{2.778890in}}{\pgfqpoint{8.450546in}{2.782456in}}%
\pgfpathcurveto{\pgfqpoint{8.446979in}{2.786022in}}{\pgfqpoint{8.442142in}{2.788026in}}{\pgfqpoint{8.437098in}{2.788026in}}%
\pgfpathcurveto{\pgfqpoint{8.432054in}{2.788026in}}{\pgfqpoint{8.427217in}{2.786022in}}{\pgfqpoint{8.423650in}{2.782456in}}%
\pgfpathcurveto{\pgfqpoint{8.420084in}{2.778890in}}{\pgfqpoint{8.418080in}{2.774052in}}{\pgfqpoint{8.418080in}{2.769008in}}%
\pgfpathcurveto{\pgfqpoint{8.418080in}{2.763965in}}{\pgfqpoint{8.420084in}{2.759127in}}{\pgfqpoint{8.423650in}{2.755560in}}%
\pgfpathcurveto{\pgfqpoint{8.427217in}{2.751994in}}{\pgfqpoint{8.432054in}{2.749990in}}{\pgfqpoint{8.437098in}{2.749990in}}%
\pgfpathclose%
\pgfusepath{fill}%
\end{pgfscope}%
\begin{pgfscope}%
\pgfpathrectangle{\pgfqpoint{6.572727in}{0.473000in}}{\pgfqpoint{4.227273in}{3.311000in}}%
\pgfusepath{clip}%
\pgfsetbuttcap%
\pgfsetroundjoin%
\definecolor{currentfill}{rgb}{0.127568,0.566949,0.550556}%
\pgfsetfillcolor{currentfill}%
\pgfsetfillopacity{0.700000}%
\pgfsetlinewidth{0.000000pt}%
\definecolor{currentstroke}{rgb}{0.000000,0.000000,0.000000}%
\pgfsetstrokecolor{currentstroke}%
\pgfsetstrokeopacity{0.700000}%
\pgfsetdash{}{0pt}%
\pgfpathmoveto{\pgfqpoint{7.403810in}{2.891517in}}%
\pgfpathcurveto{\pgfqpoint{7.408854in}{2.891517in}}{\pgfqpoint{7.413692in}{2.893521in}}{\pgfqpoint{7.417258in}{2.897088in}}%
\pgfpathcurveto{\pgfqpoint{7.420825in}{2.900654in}}{\pgfqpoint{7.422829in}{2.905492in}}{\pgfqpoint{7.422829in}{2.910536in}}%
\pgfpathcurveto{\pgfqpoint{7.422829in}{2.915579in}}{\pgfqpoint{7.420825in}{2.920417in}}{\pgfqpoint{7.417258in}{2.923983in}}%
\pgfpathcurveto{\pgfqpoint{7.413692in}{2.927550in}}{\pgfqpoint{7.408854in}{2.929554in}}{\pgfqpoint{7.403810in}{2.929554in}}%
\pgfpathcurveto{\pgfqpoint{7.398767in}{2.929554in}}{\pgfqpoint{7.393929in}{2.927550in}}{\pgfqpoint{7.390363in}{2.923983in}}%
\pgfpathcurveto{\pgfqpoint{7.386796in}{2.920417in}}{\pgfqpoint{7.384792in}{2.915579in}}{\pgfqpoint{7.384792in}{2.910536in}}%
\pgfpathcurveto{\pgfqpoint{7.384792in}{2.905492in}}{\pgfqpoint{7.386796in}{2.900654in}}{\pgfqpoint{7.390363in}{2.897088in}}%
\pgfpathcurveto{\pgfqpoint{7.393929in}{2.893521in}}{\pgfqpoint{7.398767in}{2.891517in}}{\pgfqpoint{7.403810in}{2.891517in}}%
\pgfpathclose%
\pgfusepath{fill}%
\end{pgfscope}%
\begin{pgfscope}%
\pgfpathrectangle{\pgfqpoint{6.572727in}{0.473000in}}{\pgfqpoint{4.227273in}{3.311000in}}%
\pgfusepath{clip}%
\pgfsetbuttcap%
\pgfsetroundjoin%
\definecolor{currentfill}{rgb}{0.127568,0.566949,0.550556}%
\pgfsetfillcolor{currentfill}%
\pgfsetfillopacity{0.700000}%
\pgfsetlinewidth{0.000000pt}%
\definecolor{currentstroke}{rgb}{0.000000,0.000000,0.000000}%
\pgfsetstrokecolor{currentstroke}%
\pgfsetstrokeopacity{0.700000}%
\pgfsetdash{}{0pt}%
\pgfpathmoveto{\pgfqpoint{7.807939in}{2.637824in}}%
\pgfpathcurveto{\pgfqpoint{7.812983in}{2.637824in}}{\pgfqpoint{7.817821in}{2.639827in}}{\pgfqpoint{7.821387in}{2.643394in}}%
\pgfpathcurveto{\pgfqpoint{7.824954in}{2.646960in}}{\pgfqpoint{7.826958in}{2.651798in}}{\pgfqpoint{7.826958in}{2.656842in}}%
\pgfpathcurveto{\pgfqpoint{7.826958in}{2.661885in}}{\pgfqpoint{7.824954in}{2.666723in}}{\pgfqpoint{7.821387in}{2.670290in}}%
\pgfpathcurveto{\pgfqpoint{7.817821in}{2.673856in}}{\pgfqpoint{7.812983in}{2.675860in}}{\pgfqpoint{7.807939in}{2.675860in}}%
\pgfpathcurveto{\pgfqpoint{7.802896in}{2.675860in}}{\pgfqpoint{7.798058in}{2.673856in}}{\pgfqpoint{7.794492in}{2.670290in}}%
\pgfpathcurveto{\pgfqpoint{7.790925in}{2.666723in}}{\pgfqpoint{7.788921in}{2.661885in}}{\pgfqpoint{7.788921in}{2.656842in}}%
\pgfpathcurveto{\pgfqpoint{7.788921in}{2.651798in}}{\pgfqpoint{7.790925in}{2.646960in}}{\pgfqpoint{7.794492in}{2.643394in}}%
\pgfpathcurveto{\pgfqpoint{7.798058in}{2.639827in}}{\pgfqpoint{7.802896in}{2.637824in}}{\pgfqpoint{7.807939in}{2.637824in}}%
\pgfpathclose%
\pgfusepath{fill}%
\end{pgfscope}%
\begin{pgfscope}%
\pgfpathrectangle{\pgfqpoint{6.572727in}{0.473000in}}{\pgfqpoint{4.227273in}{3.311000in}}%
\pgfusepath{clip}%
\pgfsetbuttcap%
\pgfsetroundjoin%
\definecolor{currentfill}{rgb}{0.127568,0.566949,0.550556}%
\pgfsetfillcolor{currentfill}%
\pgfsetfillopacity{0.700000}%
\pgfsetlinewidth{0.000000pt}%
\definecolor{currentstroke}{rgb}{0.000000,0.000000,0.000000}%
\pgfsetstrokecolor{currentstroke}%
\pgfsetstrokeopacity{0.700000}%
\pgfsetdash{}{0pt}%
\pgfpathmoveto{\pgfqpoint{8.068671in}{1.509159in}}%
\pgfpathcurveto{\pgfqpoint{8.073715in}{1.509159in}}{\pgfqpoint{8.078553in}{1.511163in}}{\pgfqpoint{8.082119in}{1.514729in}}%
\pgfpathcurveto{\pgfqpoint{8.085686in}{1.518295in}}{\pgfqpoint{8.087689in}{1.523133in}}{\pgfqpoint{8.087689in}{1.528177in}}%
\pgfpathcurveto{\pgfqpoint{8.087689in}{1.533221in}}{\pgfqpoint{8.085686in}{1.538058in}}{\pgfqpoint{8.082119in}{1.541625in}}%
\pgfpathcurveto{\pgfqpoint{8.078553in}{1.545191in}}{\pgfqpoint{8.073715in}{1.547195in}}{\pgfqpoint{8.068671in}{1.547195in}}%
\pgfpathcurveto{\pgfqpoint{8.063628in}{1.547195in}}{\pgfqpoint{8.058790in}{1.545191in}}{\pgfqpoint{8.055223in}{1.541625in}}%
\pgfpathcurveto{\pgfqpoint{8.051657in}{1.538058in}}{\pgfqpoint{8.049653in}{1.533221in}}{\pgfqpoint{8.049653in}{1.528177in}}%
\pgfpathcurveto{\pgfqpoint{8.049653in}{1.523133in}}{\pgfqpoint{8.051657in}{1.518295in}}{\pgfqpoint{8.055223in}{1.514729in}}%
\pgfpathcurveto{\pgfqpoint{8.058790in}{1.511163in}}{\pgfqpoint{8.063628in}{1.509159in}}{\pgfqpoint{8.068671in}{1.509159in}}%
\pgfpathclose%
\pgfusepath{fill}%
\end{pgfscope}%
\begin{pgfscope}%
\pgfpathrectangle{\pgfqpoint{6.572727in}{0.473000in}}{\pgfqpoint{4.227273in}{3.311000in}}%
\pgfusepath{clip}%
\pgfsetbuttcap%
\pgfsetroundjoin%
\definecolor{currentfill}{rgb}{0.993248,0.906157,0.143936}%
\pgfsetfillcolor{currentfill}%
\pgfsetfillopacity{0.700000}%
\pgfsetlinewidth{0.000000pt}%
\definecolor{currentstroke}{rgb}{0.000000,0.000000,0.000000}%
\pgfsetstrokecolor{currentstroke}%
\pgfsetstrokeopacity{0.700000}%
\pgfsetdash{}{0pt}%
\pgfpathmoveto{\pgfqpoint{9.265946in}{2.029980in}}%
\pgfpathcurveto{\pgfqpoint{9.270990in}{2.029980in}}{\pgfqpoint{9.275827in}{2.031984in}}{\pgfqpoint{9.279394in}{2.035551in}}%
\pgfpathcurveto{\pgfqpoint{9.282960in}{2.039117in}}{\pgfqpoint{9.284964in}{2.043955in}}{\pgfqpoint{9.284964in}{2.048999in}}%
\pgfpathcurveto{\pgfqpoint{9.284964in}{2.054042in}}{\pgfqpoint{9.282960in}{2.058880in}}{\pgfqpoint{9.279394in}{2.062446in}}%
\pgfpathcurveto{\pgfqpoint{9.275827in}{2.066013in}}{\pgfqpoint{9.270990in}{2.068017in}}{\pgfqpoint{9.265946in}{2.068017in}}%
\pgfpathcurveto{\pgfqpoint{9.260902in}{2.068017in}}{\pgfqpoint{9.256064in}{2.066013in}}{\pgfqpoint{9.252498in}{2.062446in}}%
\pgfpathcurveto{\pgfqpoint{9.248932in}{2.058880in}}{\pgfqpoint{9.246928in}{2.054042in}}{\pgfqpoint{9.246928in}{2.048999in}}%
\pgfpathcurveto{\pgfqpoint{9.246928in}{2.043955in}}{\pgfqpoint{9.248932in}{2.039117in}}{\pgfqpoint{9.252498in}{2.035551in}}%
\pgfpathcurveto{\pgfqpoint{9.256064in}{2.031984in}}{\pgfqpoint{9.260902in}{2.029980in}}{\pgfqpoint{9.265946in}{2.029980in}}%
\pgfpathclose%
\pgfusepath{fill}%
\end{pgfscope}%
\begin{pgfscope}%
\pgfpathrectangle{\pgfqpoint{6.572727in}{0.473000in}}{\pgfqpoint{4.227273in}{3.311000in}}%
\pgfusepath{clip}%
\pgfsetbuttcap%
\pgfsetroundjoin%
\definecolor{currentfill}{rgb}{0.127568,0.566949,0.550556}%
\pgfsetfillcolor{currentfill}%
\pgfsetfillopacity{0.700000}%
\pgfsetlinewidth{0.000000pt}%
\definecolor{currentstroke}{rgb}{0.000000,0.000000,0.000000}%
\pgfsetstrokecolor{currentstroke}%
\pgfsetstrokeopacity{0.700000}%
\pgfsetdash{}{0pt}%
\pgfpathmoveto{\pgfqpoint{7.660278in}{1.130196in}}%
\pgfpathcurveto{\pgfqpoint{7.665322in}{1.130196in}}{\pgfqpoint{7.670159in}{1.132200in}}{\pgfqpoint{7.673726in}{1.135767in}}%
\pgfpathcurveto{\pgfqpoint{7.677292in}{1.139333in}}{\pgfqpoint{7.679296in}{1.144171in}}{\pgfqpoint{7.679296in}{1.149215in}}%
\pgfpathcurveto{\pgfqpoint{7.679296in}{1.154258in}}{\pgfqpoint{7.677292in}{1.159096in}}{\pgfqpoint{7.673726in}{1.162662in}}%
\pgfpathcurveto{\pgfqpoint{7.670159in}{1.166229in}}{\pgfqpoint{7.665322in}{1.168233in}}{\pgfqpoint{7.660278in}{1.168233in}}%
\pgfpathcurveto{\pgfqpoint{7.655234in}{1.168233in}}{\pgfqpoint{7.650396in}{1.166229in}}{\pgfqpoint{7.646830in}{1.162662in}}%
\pgfpathcurveto{\pgfqpoint{7.643264in}{1.159096in}}{\pgfqpoint{7.641260in}{1.154258in}}{\pgfqpoint{7.641260in}{1.149215in}}%
\pgfpathcurveto{\pgfqpoint{7.641260in}{1.144171in}}{\pgfqpoint{7.643264in}{1.139333in}}{\pgfqpoint{7.646830in}{1.135767in}}%
\pgfpathcurveto{\pgfqpoint{7.650396in}{1.132200in}}{\pgfqpoint{7.655234in}{1.130196in}}{\pgfqpoint{7.660278in}{1.130196in}}%
\pgfpathclose%
\pgfusepath{fill}%
\end{pgfscope}%
\begin{pgfscope}%
\pgfpathrectangle{\pgfqpoint{6.572727in}{0.473000in}}{\pgfqpoint{4.227273in}{3.311000in}}%
\pgfusepath{clip}%
\pgfsetbuttcap%
\pgfsetroundjoin%
\definecolor{currentfill}{rgb}{0.993248,0.906157,0.143936}%
\pgfsetfillcolor{currentfill}%
\pgfsetfillopacity{0.700000}%
\pgfsetlinewidth{0.000000pt}%
\definecolor{currentstroke}{rgb}{0.000000,0.000000,0.000000}%
\pgfsetstrokecolor{currentstroke}%
\pgfsetstrokeopacity{0.700000}%
\pgfsetdash{}{0pt}%
\pgfpathmoveto{\pgfqpoint{9.000927in}{1.796805in}}%
\pgfpathcurveto{\pgfqpoint{9.005970in}{1.796805in}}{\pgfqpoint{9.010808in}{1.798809in}}{\pgfqpoint{9.014375in}{1.802375in}}%
\pgfpathcurveto{\pgfqpoint{9.017941in}{1.805942in}}{\pgfqpoint{9.019945in}{1.810780in}}{\pgfqpoint{9.019945in}{1.815823in}}%
\pgfpathcurveto{\pgfqpoint{9.019945in}{1.820867in}}{\pgfqpoint{9.017941in}{1.825705in}}{\pgfqpoint{9.014375in}{1.829271in}}%
\pgfpathcurveto{\pgfqpoint{9.010808in}{1.832837in}}{\pgfqpoint{9.005970in}{1.834841in}}{\pgfqpoint{9.000927in}{1.834841in}}%
\pgfpathcurveto{\pgfqpoint{8.995883in}{1.834841in}}{\pgfqpoint{8.991045in}{1.832837in}}{\pgfqpoint{8.987479in}{1.829271in}}%
\pgfpathcurveto{\pgfqpoint{8.983912in}{1.825705in}}{\pgfqpoint{8.981908in}{1.820867in}}{\pgfqpoint{8.981908in}{1.815823in}}%
\pgfpathcurveto{\pgfqpoint{8.981908in}{1.810780in}}{\pgfqpoint{8.983912in}{1.805942in}}{\pgfqpoint{8.987479in}{1.802375in}}%
\pgfpathcurveto{\pgfqpoint{8.991045in}{1.798809in}}{\pgfqpoint{8.995883in}{1.796805in}}{\pgfqpoint{9.000927in}{1.796805in}}%
\pgfpathclose%
\pgfusepath{fill}%
\end{pgfscope}%
\begin{pgfscope}%
\pgfpathrectangle{\pgfqpoint{6.572727in}{0.473000in}}{\pgfqpoint{4.227273in}{3.311000in}}%
\pgfusepath{clip}%
\pgfsetbuttcap%
\pgfsetroundjoin%
\definecolor{currentfill}{rgb}{0.127568,0.566949,0.550556}%
\pgfsetfillcolor{currentfill}%
\pgfsetfillopacity{0.700000}%
\pgfsetlinewidth{0.000000pt}%
\definecolor{currentstroke}{rgb}{0.000000,0.000000,0.000000}%
\pgfsetstrokecolor{currentstroke}%
\pgfsetstrokeopacity{0.700000}%
\pgfsetdash{}{0pt}%
\pgfpathmoveto{\pgfqpoint{8.412896in}{2.484665in}}%
\pgfpathcurveto{\pgfqpoint{8.417940in}{2.484665in}}{\pgfqpoint{8.422778in}{2.486669in}}{\pgfqpoint{8.426344in}{2.490236in}}%
\pgfpathcurveto{\pgfqpoint{8.429910in}{2.493802in}}{\pgfqpoint{8.431914in}{2.498640in}}{\pgfqpoint{8.431914in}{2.503684in}}%
\pgfpathcurveto{\pgfqpoint{8.431914in}{2.508727in}}{\pgfqpoint{8.429910in}{2.513565in}}{\pgfqpoint{8.426344in}{2.517131in}}%
\pgfpathcurveto{\pgfqpoint{8.422778in}{2.520698in}}{\pgfqpoint{8.417940in}{2.522702in}}{\pgfqpoint{8.412896in}{2.522702in}}%
\pgfpathcurveto{\pgfqpoint{8.407852in}{2.522702in}}{\pgfqpoint{8.403015in}{2.520698in}}{\pgfqpoint{8.399448in}{2.517131in}}%
\pgfpathcurveto{\pgfqpoint{8.395882in}{2.513565in}}{\pgfqpoint{8.393878in}{2.508727in}}{\pgfqpoint{8.393878in}{2.503684in}}%
\pgfpathcurveto{\pgfqpoint{8.393878in}{2.498640in}}{\pgfqpoint{8.395882in}{2.493802in}}{\pgfqpoint{8.399448in}{2.490236in}}%
\pgfpathcurveto{\pgfqpoint{8.403015in}{2.486669in}}{\pgfqpoint{8.407852in}{2.484665in}}{\pgfqpoint{8.412896in}{2.484665in}}%
\pgfpathclose%
\pgfusepath{fill}%
\end{pgfscope}%
\begin{pgfscope}%
\pgfpathrectangle{\pgfqpoint{6.572727in}{0.473000in}}{\pgfqpoint{4.227273in}{3.311000in}}%
\pgfusepath{clip}%
\pgfsetbuttcap%
\pgfsetroundjoin%
\definecolor{currentfill}{rgb}{0.127568,0.566949,0.550556}%
\pgfsetfillcolor{currentfill}%
\pgfsetfillopacity{0.700000}%
\pgfsetlinewidth{0.000000pt}%
\definecolor{currentstroke}{rgb}{0.000000,0.000000,0.000000}%
\pgfsetstrokecolor{currentstroke}%
\pgfsetstrokeopacity{0.700000}%
\pgfsetdash{}{0pt}%
\pgfpathmoveto{\pgfqpoint{8.091446in}{1.662220in}}%
\pgfpathcurveto{\pgfqpoint{8.096490in}{1.662220in}}{\pgfqpoint{8.101328in}{1.664224in}}{\pgfqpoint{8.104894in}{1.667790in}}%
\pgfpathcurveto{\pgfqpoint{8.108461in}{1.671357in}}{\pgfqpoint{8.110464in}{1.676194in}}{\pgfqpoint{8.110464in}{1.681238in}}%
\pgfpathcurveto{\pgfqpoint{8.110464in}{1.686282in}}{\pgfqpoint{8.108461in}{1.691120in}}{\pgfqpoint{8.104894in}{1.694686in}}%
\pgfpathcurveto{\pgfqpoint{8.101328in}{1.698252in}}{\pgfqpoint{8.096490in}{1.700256in}}{\pgfqpoint{8.091446in}{1.700256in}}%
\pgfpathcurveto{\pgfqpoint{8.086403in}{1.700256in}}{\pgfqpoint{8.081565in}{1.698252in}}{\pgfqpoint{8.077998in}{1.694686in}}%
\pgfpathcurveto{\pgfqpoint{8.074432in}{1.691120in}}{\pgfqpoint{8.072428in}{1.686282in}}{\pgfqpoint{8.072428in}{1.681238in}}%
\pgfpathcurveto{\pgfqpoint{8.072428in}{1.676194in}}{\pgfqpoint{8.074432in}{1.671357in}}{\pgfqpoint{8.077998in}{1.667790in}}%
\pgfpathcurveto{\pgfqpoint{8.081565in}{1.664224in}}{\pgfqpoint{8.086403in}{1.662220in}}{\pgfqpoint{8.091446in}{1.662220in}}%
\pgfpathclose%
\pgfusepath{fill}%
\end{pgfscope}%
\begin{pgfscope}%
\pgfpathrectangle{\pgfqpoint{6.572727in}{0.473000in}}{\pgfqpoint{4.227273in}{3.311000in}}%
\pgfusepath{clip}%
\pgfsetbuttcap%
\pgfsetroundjoin%
\definecolor{currentfill}{rgb}{0.127568,0.566949,0.550556}%
\pgfsetfillcolor{currentfill}%
\pgfsetfillopacity{0.700000}%
\pgfsetlinewidth{0.000000pt}%
\definecolor{currentstroke}{rgb}{0.000000,0.000000,0.000000}%
\pgfsetstrokecolor{currentstroke}%
\pgfsetstrokeopacity{0.700000}%
\pgfsetdash{}{0pt}%
\pgfpathmoveto{\pgfqpoint{8.080122in}{2.590142in}}%
\pgfpathcurveto{\pgfqpoint{8.085166in}{2.590142in}}{\pgfqpoint{8.090004in}{2.592146in}}{\pgfqpoint{8.093570in}{2.595712in}}%
\pgfpathcurveto{\pgfqpoint{8.097137in}{2.599279in}}{\pgfqpoint{8.099140in}{2.604117in}}{\pgfqpoint{8.099140in}{2.609160in}}%
\pgfpathcurveto{\pgfqpoint{8.099140in}{2.614204in}}{\pgfqpoint{8.097137in}{2.619042in}}{\pgfqpoint{8.093570in}{2.622608in}}%
\pgfpathcurveto{\pgfqpoint{8.090004in}{2.626175in}}{\pgfqpoint{8.085166in}{2.628178in}}{\pgfqpoint{8.080122in}{2.628178in}}%
\pgfpathcurveto{\pgfqpoint{8.075079in}{2.628178in}}{\pgfqpoint{8.070241in}{2.626175in}}{\pgfqpoint{8.066674in}{2.622608in}}%
\pgfpathcurveto{\pgfqpoint{8.063108in}{2.619042in}}{\pgfqpoint{8.061104in}{2.614204in}}{\pgfqpoint{8.061104in}{2.609160in}}%
\pgfpathcurveto{\pgfqpoint{8.061104in}{2.604117in}}{\pgfqpoint{8.063108in}{2.599279in}}{\pgfqpoint{8.066674in}{2.595712in}}%
\pgfpathcurveto{\pgfqpoint{8.070241in}{2.592146in}}{\pgfqpoint{8.075079in}{2.590142in}}{\pgfqpoint{8.080122in}{2.590142in}}%
\pgfpathclose%
\pgfusepath{fill}%
\end{pgfscope}%
\begin{pgfscope}%
\pgfpathrectangle{\pgfqpoint{6.572727in}{0.473000in}}{\pgfqpoint{4.227273in}{3.311000in}}%
\pgfusepath{clip}%
\pgfsetbuttcap%
\pgfsetroundjoin%
\definecolor{currentfill}{rgb}{0.127568,0.566949,0.550556}%
\pgfsetfillcolor{currentfill}%
\pgfsetfillopacity{0.700000}%
\pgfsetlinewidth{0.000000pt}%
\definecolor{currentstroke}{rgb}{0.000000,0.000000,0.000000}%
\pgfsetstrokecolor{currentstroke}%
\pgfsetstrokeopacity{0.700000}%
\pgfsetdash{}{0pt}%
\pgfpathmoveto{\pgfqpoint{7.656938in}{2.229762in}}%
\pgfpathcurveto{\pgfqpoint{7.661981in}{2.229762in}}{\pgfqpoint{7.666819in}{2.231766in}}{\pgfqpoint{7.670386in}{2.235332in}}%
\pgfpathcurveto{\pgfqpoint{7.673952in}{2.238899in}}{\pgfqpoint{7.675956in}{2.243737in}}{\pgfqpoint{7.675956in}{2.248780in}}%
\pgfpathcurveto{\pgfqpoint{7.675956in}{2.253824in}}{\pgfqpoint{7.673952in}{2.258662in}}{\pgfqpoint{7.670386in}{2.262228in}}%
\pgfpathcurveto{\pgfqpoint{7.666819in}{2.265795in}}{\pgfqpoint{7.661981in}{2.267798in}}{\pgfqpoint{7.656938in}{2.267798in}}%
\pgfpathcurveto{\pgfqpoint{7.651894in}{2.267798in}}{\pgfqpoint{7.647056in}{2.265795in}}{\pgfqpoint{7.643490in}{2.262228in}}%
\pgfpathcurveto{\pgfqpoint{7.639923in}{2.258662in}}{\pgfqpoint{7.637920in}{2.253824in}}{\pgfqpoint{7.637920in}{2.248780in}}%
\pgfpathcurveto{\pgfqpoint{7.637920in}{2.243737in}}{\pgfqpoint{7.639923in}{2.238899in}}{\pgfqpoint{7.643490in}{2.235332in}}%
\pgfpathcurveto{\pgfqpoint{7.647056in}{2.231766in}}{\pgfqpoint{7.651894in}{2.229762in}}{\pgfqpoint{7.656938in}{2.229762in}}%
\pgfpathclose%
\pgfusepath{fill}%
\end{pgfscope}%
\begin{pgfscope}%
\pgfpathrectangle{\pgfqpoint{6.572727in}{0.473000in}}{\pgfqpoint{4.227273in}{3.311000in}}%
\pgfusepath{clip}%
\pgfsetbuttcap%
\pgfsetroundjoin%
\definecolor{currentfill}{rgb}{0.127568,0.566949,0.550556}%
\pgfsetfillcolor{currentfill}%
\pgfsetfillopacity{0.700000}%
\pgfsetlinewidth{0.000000pt}%
\definecolor{currentstroke}{rgb}{0.000000,0.000000,0.000000}%
\pgfsetstrokecolor{currentstroke}%
\pgfsetstrokeopacity{0.700000}%
\pgfsetdash{}{0pt}%
\pgfpathmoveto{\pgfqpoint{7.769723in}{2.896725in}}%
\pgfpathcurveto{\pgfqpoint{7.774767in}{2.896725in}}{\pgfqpoint{7.779605in}{2.898729in}}{\pgfqpoint{7.783171in}{2.902295in}}%
\pgfpathcurveto{\pgfqpoint{7.786738in}{2.905862in}}{\pgfqpoint{7.788742in}{2.910700in}}{\pgfqpoint{7.788742in}{2.915743in}}%
\pgfpathcurveto{\pgfqpoint{7.788742in}{2.920787in}}{\pgfqpoint{7.786738in}{2.925625in}}{\pgfqpoint{7.783171in}{2.929191in}}%
\pgfpathcurveto{\pgfqpoint{7.779605in}{2.932758in}}{\pgfqpoint{7.774767in}{2.934761in}}{\pgfqpoint{7.769723in}{2.934761in}}%
\pgfpathcurveto{\pgfqpoint{7.764680in}{2.934761in}}{\pgfqpoint{7.759842in}{2.932758in}}{\pgfqpoint{7.756276in}{2.929191in}}%
\pgfpathcurveto{\pgfqpoint{7.752709in}{2.925625in}}{\pgfqpoint{7.750705in}{2.920787in}}{\pgfqpoint{7.750705in}{2.915743in}}%
\pgfpathcurveto{\pgfqpoint{7.750705in}{2.910700in}}{\pgfqpoint{7.752709in}{2.905862in}}{\pgfqpoint{7.756276in}{2.902295in}}%
\pgfpathcurveto{\pgfqpoint{7.759842in}{2.898729in}}{\pgfqpoint{7.764680in}{2.896725in}}{\pgfqpoint{7.769723in}{2.896725in}}%
\pgfpathclose%
\pgfusepath{fill}%
\end{pgfscope}%
\begin{pgfscope}%
\pgfpathrectangle{\pgfqpoint{6.572727in}{0.473000in}}{\pgfqpoint{4.227273in}{3.311000in}}%
\pgfusepath{clip}%
\pgfsetbuttcap%
\pgfsetroundjoin%
\definecolor{currentfill}{rgb}{0.127568,0.566949,0.550556}%
\pgfsetfillcolor{currentfill}%
\pgfsetfillopacity{0.700000}%
\pgfsetlinewidth{0.000000pt}%
\definecolor{currentstroke}{rgb}{0.000000,0.000000,0.000000}%
\pgfsetstrokecolor{currentstroke}%
\pgfsetstrokeopacity{0.700000}%
\pgfsetdash{}{0pt}%
\pgfpathmoveto{\pgfqpoint{8.519379in}{2.409614in}}%
\pgfpathcurveto{\pgfqpoint{8.524423in}{2.409614in}}{\pgfqpoint{8.529260in}{2.411617in}}{\pgfqpoint{8.532827in}{2.415184in}}%
\pgfpathcurveto{\pgfqpoint{8.536393in}{2.418750in}}{\pgfqpoint{8.538397in}{2.423588in}}{\pgfqpoint{8.538397in}{2.428632in}}%
\pgfpathcurveto{\pgfqpoint{8.538397in}{2.433675in}}{\pgfqpoint{8.536393in}{2.438513in}}{\pgfqpoint{8.532827in}{2.442080in}}%
\pgfpathcurveto{\pgfqpoint{8.529260in}{2.445646in}}{\pgfqpoint{8.524423in}{2.447650in}}{\pgfqpoint{8.519379in}{2.447650in}}%
\pgfpathcurveto{\pgfqpoint{8.514335in}{2.447650in}}{\pgfqpoint{8.509497in}{2.445646in}}{\pgfqpoint{8.505931in}{2.442080in}}%
\pgfpathcurveto{\pgfqpoint{8.502365in}{2.438513in}}{\pgfqpoint{8.500361in}{2.433675in}}{\pgfqpoint{8.500361in}{2.428632in}}%
\pgfpathcurveto{\pgfqpoint{8.500361in}{2.423588in}}{\pgfqpoint{8.502365in}{2.418750in}}{\pgfqpoint{8.505931in}{2.415184in}}%
\pgfpathcurveto{\pgfqpoint{8.509497in}{2.411617in}}{\pgfqpoint{8.514335in}{2.409614in}}{\pgfqpoint{8.519379in}{2.409614in}}%
\pgfpathclose%
\pgfusepath{fill}%
\end{pgfscope}%
\begin{pgfscope}%
\pgfpathrectangle{\pgfqpoint{6.572727in}{0.473000in}}{\pgfqpoint{4.227273in}{3.311000in}}%
\pgfusepath{clip}%
\pgfsetbuttcap%
\pgfsetroundjoin%
\definecolor{currentfill}{rgb}{0.127568,0.566949,0.550556}%
\pgfsetfillcolor{currentfill}%
\pgfsetfillopacity{0.700000}%
\pgfsetlinewidth{0.000000pt}%
\definecolor{currentstroke}{rgb}{0.000000,0.000000,0.000000}%
\pgfsetstrokecolor{currentstroke}%
\pgfsetstrokeopacity{0.700000}%
\pgfsetdash{}{0pt}%
\pgfpathmoveto{\pgfqpoint{8.308376in}{1.685311in}}%
\pgfpathcurveto{\pgfqpoint{8.313419in}{1.685311in}}{\pgfqpoint{8.318257in}{1.687315in}}{\pgfqpoint{8.321824in}{1.690881in}}%
\pgfpathcurveto{\pgfqpoint{8.325390in}{1.694448in}}{\pgfqpoint{8.327394in}{1.699286in}}{\pgfqpoint{8.327394in}{1.704329in}}%
\pgfpathcurveto{\pgfqpoint{8.327394in}{1.709373in}}{\pgfqpoint{8.325390in}{1.714211in}}{\pgfqpoint{8.321824in}{1.717777in}}%
\pgfpathcurveto{\pgfqpoint{8.318257in}{1.721344in}}{\pgfqpoint{8.313419in}{1.723347in}}{\pgfqpoint{8.308376in}{1.723347in}}%
\pgfpathcurveto{\pgfqpoint{8.303332in}{1.723347in}}{\pgfqpoint{8.298494in}{1.721344in}}{\pgfqpoint{8.294928in}{1.717777in}}%
\pgfpathcurveto{\pgfqpoint{8.291362in}{1.714211in}}{\pgfqpoint{8.289358in}{1.709373in}}{\pgfqpoint{8.289358in}{1.704329in}}%
\pgfpathcurveto{\pgfqpoint{8.289358in}{1.699286in}}{\pgfqpoint{8.291362in}{1.694448in}}{\pgfqpoint{8.294928in}{1.690881in}}%
\pgfpathcurveto{\pgfqpoint{8.298494in}{1.687315in}}{\pgfqpoint{8.303332in}{1.685311in}}{\pgfqpoint{8.308376in}{1.685311in}}%
\pgfpathclose%
\pgfusepath{fill}%
\end{pgfscope}%
\begin{pgfscope}%
\pgfpathrectangle{\pgfqpoint{6.572727in}{0.473000in}}{\pgfqpoint{4.227273in}{3.311000in}}%
\pgfusepath{clip}%
\pgfsetbuttcap%
\pgfsetroundjoin%
\definecolor{currentfill}{rgb}{0.993248,0.906157,0.143936}%
\pgfsetfillcolor{currentfill}%
\pgfsetfillopacity{0.700000}%
\pgfsetlinewidth{0.000000pt}%
\definecolor{currentstroke}{rgb}{0.000000,0.000000,0.000000}%
\pgfsetstrokecolor{currentstroke}%
\pgfsetstrokeopacity{0.700000}%
\pgfsetdash{}{0pt}%
\pgfpathmoveto{\pgfqpoint{8.950710in}{1.616196in}}%
\pgfpathcurveto{\pgfqpoint{8.955754in}{1.616196in}}{\pgfqpoint{8.960592in}{1.618200in}}{\pgfqpoint{8.964158in}{1.621767in}}%
\pgfpathcurveto{\pgfqpoint{8.967724in}{1.625333in}}{\pgfqpoint{8.969728in}{1.630171in}}{\pgfqpoint{8.969728in}{1.635214in}}%
\pgfpathcurveto{\pgfqpoint{8.969728in}{1.640258in}}{\pgfqpoint{8.967724in}{1.645096in}}{\pgfqpoint{8.964158in}{1.648662in}}%
\pgfpathcurveto{\pgfqpoint{8.960592in}{1.652229in}}{\pgfqpoint{8.955754in}{1.654233in}}{\pgfqpoint{8.950710in}{1.654233in}}%
\pgfpathcurveto{\pgfqpoint{8.945666in}{1.654233in}}{\pgfqpoint{8.940829in}{1.652229in}}{\pgfqpoint{8.937262in}{1.648662in}}%
\pgfpathcurveto{\pgfqpoint{8.933696in}{1.645096in}}{\pgfqpoint{8.931692in}{1.640258in}}{\pgfqpoint{8.931692in}{1.635214in}}%
\pgfpathcurveto{\pgfqpoint{8.931692in}{1.630171in}}{\pgfqpoint{8.933696in}{1.625333in}}{\pgfqpoint{8.937262in}{1.621767in}}%
\pgfpathcurveto{\pgfqpoint{8.940829in}{1.618200in}}{\pgfqpoint{8.945666in}{1.616196in}}{\pgfqpoint{8.950710in}{1.616196in}}%
\pgfpathclose%
\pgfusepath{fill}%
\end{pgfscope}%
\begin{pgfscope}%
\pgfpathrectangle{\pgfqpoint{6.572727in}{0.473000in}}{\pgfqpoint{4.227273in}{3.311000in}}%
\pgfusepath{clip}%
\pgfsetbuttcap%
\pgfsetroundjoin%
\definecolor{currentfill}{rgb}{0.127568,0.566949,0.550556}%
\pgfsetfillcolor{currentfill}%
\pgfsetfillopacity{0.700000}%
\pgfsetlinewidth{0.000000pt}%
\definecolor{currentstroke}{rgb}{0.000000,0.000000,0.000000}%
\pgfsetstrokecolor{currentstroke}%
\pgfsetstrokeopacity{0.700000}%
\pgfsetdash{}{0pt}%
\pgfpathmoveto{\pgfqpoint{8.641015in}{1.951192in}}%
\pgfpathcurveto{\pgfqpoint{8.646058in}{1.951192in}}{\pgfqpoint{8.650896in}{1.953196in}}{\pgfqpoint{8.654463in}{1.956762in}}%
\pgfpathcurveto{\pgfqpoint{8.658029in}{1.960329in}}{\pgfqpoint{8.660033in}{1.965167in}}{\pgfqpoint{8.660033in}{1.970210in}}%
\pgfpathcurveto{\pgfqpoint{8.660033in}{1.975254in}}{\pgfqpoint{8.658029in}{1.980092in}}{\pgfqpoint{8.654463in}{1.983658in}}%
\pgfpathcurveto{\pgfqpoint{8.650896in}{1.987225in}}{\pgfqpoint{8.646058in}{1.989228in}}{\pgfqpoint{8.641015in}{1.989228in}}%
\pgfpathcurveto{\pgfqpoint{8.635971in}{1.989228in}}{\pgfqpoint{8.631133in}{1.987225in}}{\pgfqpoint{8.627567in}{1.983658in}}%
\pgfpathcurveto{\pgfqpoint{8.624000in}{1.980092in}}{\pgfqpoint{8.621997in}{1.975254in}}{\pgfqpoint{8.621997in}{1.970210in}}%
\pgfpathcurveto{\pgfqpoint{8.621997in}{1.965167in}}{\pgfqpoint{8.624000in}{1.960329in}}{\pgfqpoint{8.627567in}{1.956762in}}%
\pgfpathcurveto{\pgfqpoint{8.631133in}{1.953196in}}{\pgfqpoint{8.635971in}{1.951192in}}{\pgfqpoint{8.641015in}{1.951192in}}%
\pgfpathclose%
\pgfusepath{fill}%
\end{pgfscope}%
\begin{pgfscope}%
\pgfpathrectangle{\pgfqpoint{6.572727in}{0.473000in}}{\pgfqpoint{4.227273in}{3.311000in}}%
\pgfusepath{clip}%
\pgfsetbuttcap%
\pgfsetroundjoin%
\definecolor{currentfill}{rgb}{0.993248,0.906157,0.143936}%
\pgfsetfillcolor{currentfill}%
\pgfsetfillopacity{0.700000}%
\pgfsetlinewidth{0.000000pt}%
\definecolor{currentstroke}{rgb}{0.000000,0.000000,0.000000}%
\pgfsetstrokecolor{currentstroke}%
\pgfsetstrokeopacity{0.700000}%
\pgfsetdash{}{0pt}%
\pgfpathmoveto{\pgfqpoint{9.948102in}{2.045854in}}%
\pgfpathcurveto{\pgfqpoint{9.953145in}{2.045854in}}{\pgfqpoint{9.957983in}{2.047858in}}{\pgfqpoint{9.961550in}{2.051425in}}%
\pgfpathcurveto{\pgfqpoint{9.965116in}{2.054991in}}{\pgfqpoint{9.967120in}{2.059829in}}{\pgfqpoint{9.967120in}{2.064873in}}%
\pgfpathcurveto{\pgfqpoint{9.967120in}{2.069916in}}{\pgfqpoint{9.965116in}{2.074754in}}{\pgfqpoint{9.961550in}{2.078320in}}%
\pgfpathcurveto{\pgfqpoint{9.957983in}{2.081887in}}{\pgfqpoint{9.953145in}{2.083891in}}{\pgfqpoint{9.948102in}{2.083891in}}%
\pgfpathcurveto{\pgfqpoint{9.943058in}{2.083891in}}{\pgfqpoint{9.938220in}{2.081887in}}{\pgfqpoint{9.934654in}{2.078320in}}%
\pgfpathcurveto{\pgfqpoint{9.931088in}{2.074754in}}{\pgfqpoint{9.929084in}{2.069916in}}{\pgfqpoint{9.929084in}{2.064873in}}%
\pgfpathcurveto{\pgfqpoint{9.929084in}{2.059829in}}{\pgfqpoint{9.931088in}{2.054991in}}{\pgfqpoint{9.934654in}{2.051425in}}%
\pgfpathcurveto{\pgfqpoint{9.938220in}{2.047858in}}{\pgfqpoint{9.943058in}{2.045854in}}{\pgfqpoint{9.948102in}{2.045854in}}%
\pgfpathclose%
\pgfusepath{fill}%
\end{pgfscope}%
\begin{pgfscope}%
\pgfpathrectangle{\pgfqpoint{6.572727in}{0.473000in}}{\pgfqpoint{4.227273in}{3.311000in}}%
\pgfusepath{clip}%
\pgfsetbuttcap%
\pgfsetroundjoin%
\definecolor{currentfill}{rgb}{0.127568,0.566949,0.550556}%
\pgfsetfillcolor{currentfill}%
\pgfsetfillopacity{0.700000}%
\pgfsetlinewidth{0.000000pt}%
\definecolor{currentstroke}{rgb}{0.000000,0.000000,0.000000}%
\pgfsetstrokecolor{currentstroke}%
\pgfsetstrokeopacity{0.700000}%
\pgfsetdash{}{0pt}%
\pgfpathmoveto{\pgfqpoint{8.229313in}{2.711402in}}%
\pgfpathcurveto{\pgfqpoint{8.234357in}{2.711402in}}{\pgfqpoint{8.239194in}{2.713406in}}{\pgfqpoint{8.242761in}{2.716972in}}%
\pgfpathcurveto{\pgfqpoint{8.246327in}{2.720539in}}{\pgfqpoint{8.248331in}{2.725377in}}{\pgfqpoint{8.248331in}{2.730420in}}%
\pgfpathcurveto{\pgfqpoint{8.248331in}{2.735464in}}{\pgfqpoint{8.246327in}{2.740302in}}{\pgfqpoint{8.242761in}{2.743868in}}%
\pgfpathcurveto{\pgfqpoint{8.239194in}{2.747435in}}{\pgfqpoint{8.234357in}{2.749438in}}{\pgfqpoint{8.229313in}{2.749438in}}%
\pgfpathcurveto{\pgfqpoint{8.224269in}{2.749438in}}{\pgfqpoint{8.219431in}{2.747435in}}{\pgfqpoint{8.215865in}{2.743868in}}%
\pgfpathcurveto{\pgfqpoint{8.212299in}{2.740302in}}{\pgfqpoint{8.210295in}{2.735464in}}{\pgfqpoint{8.210295in}{2.730420in}}%
\pgfpathcurveto{\pgfqpoint{8.210295in}{2.725377in}}{\pgfqpoint{8.212299in}{2.720539in}}{\pgfqpoint{8.215865in}{2.716972in}}%
\pgfpathcurveto{\pgfqpoint{8.219431in}{2.713406in}}{\pgfqpoint{8.224269in}{2.711402in}}{\pgfqpoint{8.229313in}{2.711402in}}%
\pgfpathclose%
\pgfusepath{fill}%
\end{pgfscope}%
\begin{pgfscope}%
\pgfpathrectangle{\pgfqpoint{6.572727in}{0.473000in}}{\pgfqpoint{4.227273in}{3.311000in}}%
\pgfusepath{clip}%
\pgfsetbuttcap%
\pgfsetroundjoin%
\definecolor{currentfill}{rgb}{0.127568,0.566949,0.550556}%
\pgfsetfillcolor{currentfill}%
\pgfsetfillopacity{0.700000}%
\pgfsetlinewidth{0.000000pt}%
\definecolor{currentstroke}{rgb}{0.000000,0.000000,0.000000}%
\pgfsetstrokecolor{currentstroke}%
\pgfsetstrokeopacity{0.700000}%
\pgfsetdash{}{0pt}%
\pgfpathmoveto{\pgfqpoint{8.259005in}{2.253118in}}%
\pgfpathcurveto{\pgfqpoint{8.264049in}{2.253118in}}{\pgfqpoint{8.268887in}{2.255122in}}{\pgfqpoint{8.272453in}{2.258689in}}%
\pgfpathcurveto{\pgfqpoint{8.276019in}{2.262255in}}{\pgfqpoint{8.278023in}{2.267093in}}{\pgfqpoint{8.278023in}{2.272137in}}%
\pgfpathcurveto{\pgfqpoint{8.278023in}{2.277180in}}{\pgfqpoint{8.276019in}{2.282018in}}{\pgfqpoint{8.272453in}{2.285584in}}%
\pgfpathcurveto{\pgfqpoint{8.268887in}{2.289151in}}{\pgfqpoint{8.264049in}{2.291155in}}{\pgfqpoint{8.259005in}{2.291155in}}%
\pgfpathcurveto{\pgfqpoint{8.253961in}{2.291155in}}{\pgfqpoint{8.249124in}{2.289151in}}{\pgfqpoint{8.245557in}{2.285584in}}%
\pgfpathcurveto{\pgfqpoint{8.241991in}{2.282018in}}{\pgfqpoint{8.239987in}{2.277180in}}{\pgfqpoint{8.239987in}{2.272137in}}%
\pgfpathcurveto{\pgfqpoint{8.239987in}{2.267093in}}{\pgfqpoint{8.241991in}{2.262255in}}{\pgfqpoint{8.245557in}{2.258689in}}%
\pgfpathcurveto{\pgfqpoint{8.249124in}{2.255122in}}{\pgfqpoint{8.253961in}{2.253118in}}{\pgfqpoint{8.259005in}{2.253118in}}%
\pgfpathclose%
\pgfusepath{fill}%
\end{pgfscope}%
\begin{pgfscope}%
\pgfpathrectangle{\pgfqpoint{6.572727in}{0.473000in}}{\pgfqpoint{4.227273in}{3.311000in}}%
\pgfusepath{clip}%
\pgfsetbuttcap%
\pgfsetroundjoin%
\definecolor{currentfill}{rgb}{0.127568,0.566949,0.550556}%
\pgfsetfillcolor{currentfill}%
\pgfsetfillopacity{0.700000}%
\pgfsetlinewidth{0.000000pt}%
\definecolor{currentstroke}{rgb}{0.000000,0.000000,0.000000}%
\pgfsetstrokecolor{currentstroke}%
\pgfsetstrokeopacity{0.700000}%
\pgfsetdash{}{0pt}%
\pgfpathmoveto{\pgfqpoint{8.883793in}{3.140062in}}%
\pgfpathcurveto{\pgfqpoint{8.888837in}{3.140062in}}{\pgfqpoint{8.893675in}{3.142066in}}{\pgfqpoint{8.897241in}{3.145632in}}%
\pgfpathcurveto{\pgfqpoint{8.900807in}{3.149199in}}{\pgfqpoint{8.902811in}{3.154037in}}{\pgfqpoint{8.902811in}{3.159080in}}%
\pgfpathcurveto{\pgfqpoint{8.902811in}{3.164124in}}{\pgfqpoint{8.900807in}{3.168962in}}{\pgfqpoint{8.897241in}{3.172528in}}%
\pgfpathcurveto{\pgfqpoint{8.893675in}{3.176095in}}{\pgfqpoint{8.888837in}{3.178098in}}{\pgfqpoint{8.883793in}{3.178098in}}%
\pgfpathcurveto{\pgfqpoint{8.878749in}{3.178098in}}{\pgfqpoint{8.873912in}{3.176095in}}{\pgfqpoint{8.870345in}{3.172528in}}%
\pgfpathcurveto{\pgfqpoint{8.866779in}{3.168962in}}{\pgfqpoint{8.864775in}{3.164124in}}{\pgfqpoint{8.864775in}{3.159080in}}%
\pgfpathcurveto{\pgfqpoint{8.864775in}{3.154037in}}{\pgfqpoint{8.866779in}{3.149199in}}{\pgfqpoint{8.870345in}{3.145632in}}%
\pgfpathcurveto{\pgfqpoint{8.873912in}{3.142066in}}{\pgfqpoint{8.878749in}{3.140062in}}{\pgfqpoint{8.883793in}{3.140062in}}%
\pgfpathclose%
\pgfusepath{fill}%
\end{pgfscope}%
\begin{pgfscope}%
\pgfpathrectangle{\pgfqpoint{6.572727in}{0.473000in}}{\pgfqpoint{4.227273in}{3.311000in}}%
\pgfusepath{clip}%
\pgfsetbuttcap%
\pgfsetroundjoin%
\definecolor{currentfill}{rgb}{0.127568,0.566949,0.550556}%
\pgfsetfillcolor{currentfill}%
\pgfsetfillopacity{0.700000}%
\pgfsetlinewidth{0.000000pt}%
\definecolor{currentstroke}{rgb}{0.000000,0.000000,0.000000}%
\pgfsetstrokecolor{currentstroke}%
\pgfsetstrokeopacity{0.700000}%
\pgfsetdash{}{0pt}%
\pgfpathmoveto{\pgfqpoint{7.810273in}{1.558264in}}%
\pgfpathcurveto{\pgfqpoint{7.815317in}{1.558264in}}{\pgfqpoint{7.820155in}{1.560268in}}{\pgfqpoint{7.823721in}{1.563834in}}%
\pgfpathcurveto{\pgfqpoint{7.827288in}{1.567401in}}{\pgfqpoint{7.829292in}{1.572239in}}{\pgfqpoint{7.829292in}{1.577282in}}%
\pgfpathcurveto{\pgfqpoint{7.829292in}{1.582326in}}{\pgfqpoint{7.827288in}{1.587164in}}{\pgfqpoint{7.823721in}{1.590730in}}%
\pgfpathcurveto{\pgfqpoint{7.820155in}{1.594297in}}{\pgfqpoint{7.815317in}{1.596300in}}{\pgfqpoint{7.810273in}{1.596300in}}%
\pgfpathcurveto{\pgfqpoint{7.805230in}{1.596300in}}{\pgfqpoint{7.800392in}{1.594297in}}{\pgfqpoint{7.796826in}{1.590730in}}%
\pgfpathcurveto{\pgfqpoint{7.793259in}{1.587164in}}{\pgfqpoint{7.791255in}{1.582326in}}{\pgfqpoint{7.791255in}{1.577282in}}%
\pgfpathcurveto{\pgfqpoint{7.791255in}{1.572239in}}{\pgfqpoint{7.793259in}{1.567401in}}{\pgfqpoint{7.796826in}{1.563834in}}%
\pgfpathcurveto{\pgfqpoint{7.800392in}{1.560268in}}{\pgfqpoint{7.805230in}{1.558264in}}{\pgfqpoint{7.810273in}{1.558264in}}%
\pgfpathclose%
\pgfusepath{fill}%
\end{pgfscope}%
\begin{pgfscope}%
\pgfpathrectangle{\pgfqpoint{6.572727in}{0.473000in}}{\pgfqpoint{4.227273in}{3.311000in}}%
\pgfusepath{clip}%
\pgfsetbuttcap%
\pgfsetroundjoin%
\definecolor{currentfill}{rgb}{0.127568,0.566949,0.550556}%
\pgfsetfillcolor{currentfill}%
\pgfsetfillopacity{0.700000}%
\pgfsetlinewidth{0.000000pt}%
\definecolor{currentstroke}{rgb}{0.000000,0.000000,0.000000}%
\pgfsetstrokecolor{currentstroke}%
\pgfsetstrokeopacity{0.700000}%
\pgfsetdash{}{0pt}%
\pgfpathmoveto{\pgfqpoint{8.219733in}{3.380590in}}%
\pgfpathcurveto{\pgfqpoint{8.224777in}{3.380590in}}{\pgfqpoint{8.229614in}{3.382594in}}{\pgfqpoint{8.233181in}{3.386161in}}%
\pgfpathcurveto{\pgfqpoint{8.236747in}{3.389727in}}{\pgfqpoint{8.238751in}{3.394565in}}{\pgfqpoint{8.238751in}{3.399608in}}%
\pgfpathcurveto{\pgfqpoint{8.238751in}{3.404652in}}{\pgfqpoint{8.236747in}{3.409490in}}{\pgfqpoint{8.233181in}{3.413056in}}%
\pgfpathcurveto{\pgfqpoint{8.229614in}{3.416623in}}{\pgfqpoint{8.224777in}{3.418627in}}{\pgfqpoint{8.219733in}{3.418627in}}%
\pgfpathcurveto{\pgfqpoint{8.214689in}{3.418627in}}{\pgfqpoint{8.209851in}{3.416623in}}{\pgfqpoint{8.206285in}{3.413056in}}%
\pgfpathcurveto{\pgfqpoint{8.202719in}{3.409490in}}{\pgfqpoint{8.200715in}{3.404652in}}{\pgfqpoint{8.200715in}{3.399608in}}%
\pgfpathcurveto{\pgfqpoint{8.200715in}{3.394565in}}{\pgfqpoint{8.202719in}{3.389727in}}{\pgfqpoint{8.206285in}{3.386161in}}%
\pgfpathcurveto{\pgfqpoint{8.209851in}{3.382594in}}{\pgfqpoint{8.214689in}{3.380590in}}{\pgfqpoint{8.219733in}{3.380590in}}%
\pgfpathclose%
\pgfusepath{fill}%
\end{pgfscope}%
\begin{pgfscope}%
\pgfpathrectangle{\pgfqpoint{6.572727in}{0.473000in}}{\pgfqpoint{4.227273in}{3.311000in}}%
\pgfusepath{clip}%
\pgfsetbuttcap%
\pgfsetroundjoin%
\definecolor{currentfill}{rgb}{0.127568,0.566949,0.550556}%
\pgfsetfillcolor{currentfill}%
\pgfsetfillopacity{0.700000}%
\pgfsetlinewidth{0.000000pt}%
\definecolor{currentstroke}{rgb}{0.000000,0.000000,0.000000}%
\pgfsetstrokecolor{currentstroke}%
\pgfsetstrokeopacity{0.700000}%
\pgfsetdash{}{0pt}%
\pgfpathmoveto{\pgfqpoint{8.327065in}{2.917659in}}%
\pgfpathcurveto{\pgfqpoint{8.332109in}{2.917659in}}{\pgfqpoint{8.336947in}{2.919663in}}{\pgfqpoint{8.340513in}{2.923229in}}%
\pgfpathcurveto{\pgfqpoint{8.344080in}{2.926796in}}{\pgfqpoint{8.346084in}{2.931634in}}{\pgfqpoint{8.346084in}{2.936677in}}%
\pgfpathcurveto{\pgfqpoint{8.346084in}{2.941721in}}{\pgfqpoint{8.344080in}{2.946559in}}{\pgfqpoint{8.340513in}{2.950125in}}%
\pgfpathcurveto{\pgfqpoint{8.336947in}{2.953691in}}{\pgfqpoint{8.332109in}{2.955695in}}{\pgfqpoint{8.327065in}{2.955695in}}%
\pgfpathcurveto{\pgfqpoint{8.322022in}{2.955695in}}{\pgfqpoint{8.317184in}{2.953691in}}{\pgfqpoint{8.313618in}{2.950125in}}%
\pgfpathcurveto{\pgfqpoint{8.310051in}{2.946559in}}{\pgfqpoint{8.308047in}{2.941721in}}{\pgfqpoint{8.308047in}{2.936677in}}%
\pgfpathcurveto{\pgfqpoint{8.308047in}{2.931634in}}{\pgfqpoint{8.310051in}{2.926796in}}{\pgfqpoint{8.313618in}{2.923229in}}%
\pgfpathcurveto{\pgfqpoint{8.317184in}{2.919663in}}{\pgfqpoint{8.322022in}{2.917659in}}{\pgfqpoint{8.327065in}{2.917659in}}%
\pgfpathclose%
\pgfusepath{fill}%
\end{pgfscope}%
\begin{pgfscope}%
\pgfpathrectangle{\pgfqpoint{6.572727in}{0.473000in}}{\pgfqpoint{4.227273in}{3.311000in}}%
\pgfusepath{clip}%
\pgfsetbuttcap%
\pgfsetroundjoin%
\definecolor{currentfill}{rgb}{0.127568,0.566949,0.550556}%
\pgfsetfillcolor{currentfill}%
\pgfsetfillopacity{0.700000}%
\pgfsetlinewidth{0.000000pt}%
\definecolor{currentstroke}{rgb}{0.000000,0.000000,0.000000}%
\pgfsetstrokecolor{currentstroke}%
\pgfsetstrokeopacity{0.700000}%
\pgfsetdash{}{0pt}%
\pgfpathmoveto{\pgfqpoint{7.352729in}{1.171339in}}%
\pgfpathcurveto{\pgfqpoint{7.357773in}{1.171339in}}{\pgfqpoint{7.362610in}{1.173343in}}{\pgfqpoint{7.366177in}{1.176909in}}%
\pgfpathcurveto{\pgfqpoint{7.369743in}{1.180476in}}{\pgfqpoint{7.371747in}{1.185313in}}{\pgfqpoint{7.371747in}{1.190357in}}%
\pgfpathcurveto{\pgfqpoint{7.371747in}{1.195401in}}{\pgfqpoint{7.369743in}{1.200239in}}{\pgfqpoint{7.366177in}{1.203805in}}%
\pgfpathcurveto{\pgfqpoint{7.362610in}{1.207371in}}{\pgfqpoint{7.357773in}{1.209375in}}{\pgfqpoint{7.352729in}{1.209375in}}%
\pgfpathcurveto{\pgfqpoint{7.347685in}{1.209375in}}{\pgfqpoint{7.342847in}{1.207371in}}{\pgfqpoint{7.339281in}{1.203805in}}%
\pgfpathcurveto{\pgfqpoint{7.335715in}{1.200239in}}{\pgfqpoint{7.333711in}{1.195401in}}{\pgfqpoint{7.333711in}{1.190357in}}%
\pgfpathcurveto{\pgfqpoint{7.333711in}{1.185313in}}{\pgfqpoint{7.335715in}{1.180476in}}{\pgfqpoint{7.339281in}{1.176909in}}%
\pgfpathcurveto{\pgfqpoint{7.342847in}{1.173343in}}{\pgfqpoint{7.347685in}{1.171339in}}{\pgfqpoint{7.352729in}{1.171339in}}%
\pgfpathclose%
\pgfusepath{fill}%
\end{pgfscope}%
\begin{pgfscope}%
\pgfpathrectangle{\pgfqpoint{6.572727in}{0.473000in}}{\pgfqpoint{4.227273in}{3.311000in}}%
\pgfusepath{clip}%
\pgfsetbuttcap%
\pgfsetroundjoin%
\definecolor{currentfill}{rgb}{0.127568,0.566949,0.550556}%
\pgfsetfillcolor{currentfill}%
\pgfsetfillopacity{0.700000}%
\pgfsetlinewidth{0.000000pt}%
\definecolor{currentstroke}{rgb}{0.000000,0.000000,0.000000}%
\pgfsetstrokecolor{currentstroke}%
\pgfsetstrokeopacity{0.700000}%
\pgfsetdash{}{0pt}%
\pgfpathmoveto{\pgfqpoint{8.524468in}{2.848089in}}%
\pgfpathcurveto{\pgfqpoint{8.529512in}{2.848089in}}{\pgfqpoint{8.534350in}{2.850093in}}{\pgfqpoint{8.537916in}{2.853659in}}%
\pgfpathcurveto{\pgfqpoint{8.541482in}{2.857226in}}{\pgfqpoint{8.543486in}{2.862063in}}{\pgfqpoint{8.543486in}{2.867107in}}%
\pgfpathcurveto{\pgfqpoint{8.543486in}{2.872151in}}{\pgfqpoint{8.541482in}{2.876988in}}{\pgfqpoint{8.537916in}{2.880555in}}%
\pgfpathcurveto{\pgfqpoint{8.534350in}{2.884121in}}{\pgfqpoint{8.529512in}{2.886125in}}{\pgfqpoint{8.524468in}{2.886125in}}%
\pgfpathcurveto{\pgfqpoint{8.519424in}{2.886125in}}{\pgfqpoint{8.514587in}{2.884121in}}{\pgfqpoint{8.511020in}{2.880555in}}%
\pgfpathcurveto{\pgfqpoint{8.507454in}{2.876988in}}{\pgfqpoint{8.505450in}{2.872151in}}{\pgfqpoint{8.505450in}{2.867107in}}%
\pgfpathcurveto{\pgfqpoint{8.505450in}{2.862063in}}{\pgfqpoint{8.507454in}{2.857226in}}{\pgfqpoint{8.511020in}{2.853659in}}%
\pgfpathcurveto{\pgfqpoint{8.514587in}{2.850093in}}{\pgfqpoint{8.519424in}{2.848089in}}{\pgfqpoint{8.524468in}{2.848089in}}%
\pgfpathclose%
\pgfusepath{fill}%
\end{pgfscope}%
\begin{pgfscope}%
\pgfpathrectangle{\pgfqpoint{6.572727in}{0.473000in}}{\pgfqpoint{4.227273in}{3.311000in}}%
\pgfusepath{clip}%
\pgfsetbuttcap%
\pgfsetroundjoin%
\definecolor{currentfill}{rgb}{0.127568,0.566949,0.550556}%
\pgfsetfillcolor{currentfill}%
\pgfsetfillopacity{0.700000}%
\pgfsetlinewidth{0.000000pt}%
\definecolor{currentstroke}{rgb}{0.000000,0.000000,0.000000}%
\pgfsetstrokecolor{currentstroke}%
\pgfsetstrokeopacity{0.700000}%
\pgfsetdash{}{0pt}%
\pgfpathmoveto{\pgfqpoint{8.186872in}{3.562381in}}%
\pgfpathcurveto{\pgfqpoint{8.191915in}{3.562381in}}{\pgfqpoint{8.196753in}{3.564385in}}{\pgfqpoint{8.200320in}{3.567952in}}%
\pgfpathcurveto{\pgfqpoint{8.203886in}{3.571518in}}{\pgfqpoint{8.205890in}{3.576356in}}{\pgfqpoint{8.205890in}{3.581400in}}%
\pgfpathcurveto{\pgfqpoint{8.205890in}{3.586443in}}{\pgfqpoint{8.203886in}{3.591281in}}{\pgfqpoint{8.200320in}{3.594847in}}%
\pgfpathcurveto{\pgfqpoint{8.196753in}{3.598414in}}{\pgfqpoint{8.191915in}{3.600418in}}{\pgfqpoint{8.186872in}{3.600418in}}%
\pgfpathcurveto{\pgfqpoint{8.181828in}{3.600418in}}{\pgfqpoint{8.176990in}{3.598414in}}{\pgfqpoint{8.173424in}{3.594847in}}%
\pgfpathcurveto{\pgfqpoint{8.169857in}{3.591281in}}{\pgfqpoint{8.167854in}{3.586443in}}{\pgfqpoint{8.167854in}{3.581400in}}%
\pgfpathcurveto{\pgfqpoint{8.167854in}{3.576356in}}{\pgfqpoint{8.169857in}{3.571518in}}{\pgfqpoint{8.173424in}{3.567952in}}%
\pgfpathcurveto{\pgfqpoint{8.176990in}{3.564385in}}{\pgfqpoint{8.181828in}{3.562381in}}{\pgfqpoint{8.186872in}{3.562381in}}%
\pgfpathclose%
\pgfusepath{fill}%
\end{pgfscope}%
\begin{pgfscope}%
\pgfpathrectangle{\pgfqpoint{6.572727in}{0.473000in}}{\pgfqpoint{4.227273in}{3.311000in}}%
\pgfusepath{clip}%
\pgfsetbuttcap%
\pgfsetroundjoin%
\definecolor{currentfill}{rgb}{0.993248,0.906157,0.143936}%
\pgfsetfillcolor{currentfill}%
\pgfsetfillopacity{0.700000}%
\pgfsetlinewidth{0.000000pt}%
\definecolor{currentstroke}{rgb}{0.000000,0.000000,0.000000}%
\pgfsetstrokecolor{currentstroke}%
\pgfsetstrokeopacity{0.700000}%
\pgfsetdash{}{0pt}%
\pgfpathmoveto{\pgfqpoint{9.933332in}{1.326774in}}%
\pgfpathcurveto{\pgfqpoint{9.938376in}{1.326774in}}{\pgfqpoint{9.943214in}{1.328778in}}{\pgfqpoint{9.946780in}{1.332345in}}%
\pgfpathcurveto{\pgfqpoint{9.950347in}{1.335911in}}{\pgfqpoint{9.952351in}{1.340749in}}{\pgfqpoint{9.952351in}{1.345793in}}%
\pgfpathcurveto{\pgfqpoint{9.952351in}{1.350836in}}{\pgfqpoint{9.950347in}{1.355674in}}{\pgfqpoint{9.946780in}{1.359240in}}%
\pgfpathcurveto{\pgfqpoint{9.943214in}{1.362807in}}{\pgfqpoint{9.938376in}{1.364811in}}{\pgfqpoint{9.933332in}{1.364811in}}%
\pgfpathcurveto{\pgfqpoint{9.928289in}{1.364811in}}{\pgfqpoint{9.923451in}{1.362807in}}{\pgfqpoint{9.919885in}{1.359240in}}%
\pgfpathcurveto{\pgfqpoint{9.916318in}{1.355674in}}{\pgfqpoint{9.914314in}{1.350836in}}{\pgfqpoint{9.914314in}{1.345793in}}%
\pgfpathcurveto{\pgfqpoint{9.914314in}{1.340749in}}{\pgfqpoint{9.916318in}{1.335911in}}{\pgfqpoint{9.919885in}{1.332345in}}%
\pgfpathcurveto{\pgfqpoint{9.923451in}{1.328778in}}{\pgfqpoint{9.928289in}{1.326774in}}{\pgfqpoint{9.933332in}{1.326774in}}%
\pgfpathclose%
\pgfusepath{fill}%
\end{pgfscope}%
\begin{pgfscope}%
\pgfpathrectangle{\pgfqpoint{6.572727in}{0.473000in}}{\pgfqpoint{4.227273in}{3.311000in}}%
\pgfusepath{clip}%
\pgfsetbuttcap%
\pgfsetroundjoin%
\definecolor{currentfill}{rgb}{0.127568,0.566949,0.550556}%
\pgfsetfillcolor{currentfill}%
\pgfsetfillopacity{0.700000}%
\pgfsetlinewidth{0.000000pt}%
\definecolor{currentstroke}{rgb}{0.000000,0.000000,0.000000}%
\pgfsetstrokecolor{currentstroke}%
\pgfsetstrokeopacity{0.700000}%
\pgfsetdash{}{0pt}%
\pgfpathmoveto{\pgfqpoint{8.230374in}{2.416590in}}%
\pgfpathcurveto{\pgfqpoint{8.235418in}{2.416590in}}{\pgfqpoint{8.240256in}{2.418593in}}{\pgfqpoint{8.243822in}{2.422160in}}%
\pgfpathcurveto{\pgfqpoint{8.247389in}{2.425726in}}{\pgfqpoint{8.249392in}{2.430564in}}{\pgfqpoint{8.249392in}{2.435608in}}%
\pgfpathcurveto{\pgfqpoint{8.249392in}{2.440651in}}{\pgfqpoint{8.247389in}{2.445489in}}{\pgfqpoint{8.243822in}{2.449056in}}%
\pgfpathcurveto{\pgfqpoint{8.240256in}{2.452622in}}{\pgfqpoint{8.235418in}{2.454626in}}{\pgfqpoint{8.230374in}{2.454626in}}%
\pgfpathcurveto{\pgfqpoint{8.225331in}{2.454626in}}{\pgfqpoint{8.220493in}{2.452622in}}{\pgfqpoint{8.216926in}{2.449056in}}%
\pgfpathcurveto{\pgfqpoint{8.213360in}{2.445489in}}{\pgfqpoint{8.211356in}{2.440651in}}{\pgfqpoint{8.211356in}{2.435608in}}%
\pgfpathcurveto{\pgfqpoint{8.211356in}{2.430564in}}{\pgfqpoint{8.213360in}{2.425726in}}{\pgfqpoint{8.216926in}{2.422160in}}%
\pgfpathcurveto{\pgfqpoint{8.220493in}{2.418593in}}{\pgfqpoint{8.225331in}{2.416590in}}{\pgfqpoint{8.230374in}{2.416590in}}%
\pgfpathclose%
\pgfusepath{fill}%
\end{pgfscope}%
\begin{pgfscope}%
\pgfpathrectangle{\pgfqpoint{6.572727in}{0.473000in}}{\pgfqpoint{4.227273in}{3.311000in}}%
\pgfusepath{clip}%
\pgfsetbuttcap%
\pgfsetroundjoin%
\definecolor{currentfill}{rgb}{0.127568,0.566949,0.550556}%
\pgfsetfillcolor{currentfill}%
\pgfsetfillopacity{0.700000}%
\pgfsetlinewidth{0.000000pt}%
\definecolor{currentstroke}{rgb}{0.000000,0.000000,0.000000}%
\pgfsetstrokecolor{currentstroke}%
\pgfsetstrokeopacity{0.700000}%
\pgfsetdash{}{0pt}%
\pgfpathmoveto{\pgfqpoint{8.234453in}{2.950683in}}%
\pgfpathcurveto{\pgfqpoint{8.239497in}{2.950683in}}{\pgfqpoint{8.244335in}{2.952687in}}{\pgfqpoint{8.247901in}{2.956253in}}%
\pgfpathcurveto{\pgfqpoint{8.251468in}{2.959819in}}{\pgfqpoint{8.253472in}{2.964657in}}{\pgfqpoint{8.253472in}{2.969701in}}%
\pgfpathcurveto{\pgfqpoint{8.253472in}{2.974744in}}{\pgfqpoint{8.251468in}{2.979582in}}{\pgfqpoint{8.247901in}{2.983149in}}%
\pgfpathcurveto{\pgfqpoint{8.244335in}{2.986715in}}{\pgfqpoint{8.239497in}{2.988719in}}{\pgfqpoint{8.234453in}{2.988719in}}%
\pgfpathcurveto{\pgfqpoint{8.229410in}{2.988719in}}{\pgfqpoint{8.224572in}{2.986715in}}{\pgfqpoint{8.221006in}{2.983149in}}%
\pgfpathcurveto{\pgfqpoint{8.217439in}{2.979582in}}{\pgfqpoint{8.215435in}{2.974744in}}{\pgfqpoint{8.215435in}{2.969701in}}%
\pgfpathcurveto{\pgfqpoint{8.215435in}{2.964657in}}{\pgfqpoint{8.217439in}{2.959819in}}{\pgfqpoint{8.221006in}{2.956253in}}%
\pgfpathcurveto{\pgfqpoint{8.224572in}{2.952687in}}{\pgfqpoint{8.229410in}{2.950683in}}{\pgfqpoint{8.234453in}{2.950683in}}%
\pgfpathclose%
\pgfusepath{fill}%
\end{pgfscope}%
\begin{pgfscope}%
\pgfpathrectangle{\pgfqpoint{6.572727in}{0.473000in}}{\pgfqpoint{4.227273in}{3.311000in}}%
\pgfusepath{clip}%
\pgfsetbuttcap%
\pgfsetroundjoin%
\definecolor{currentfill}{rgb}{0.127568,0.566949,0.550556}%
\pgfsetfillcolor{currentfill}%
\pgfsetfillopacity{0.700000}%
\pgfsetlinewidth{0.000000pt}%
\definecolor{currentstroke}{rgb}{0.000000,0.000000,0.000000}%
\pgfsetstrokecolor{currentstroke}%
\pgfsetstrokeopacity{0.700000}%
\pgfsetdash{}{0pt}%
\pgfpathmoveto{\pgfqpoint{8.201388in}{2.579824in}}%
\pgfpathcurveto{\pgfqpoint{8.206432in}{2.579824in}}{\pgfqpoint{8.211269in}{2.581828in}}{\pgfqpoint{8.214836in}{2.585395in}}%
\pgfpathcurveto{\pgfqpoint{8.218402in}{2.588961in}}{\pgfqpoint{8.220406in}{2.593799in}}{\pgfqpoint{8.220406in}{2.598843in}}%
\pgfpathcurveto{\pgfqpoint{8.220406in}{2.603886in}}{\pgfqpoint{8.218402in}{2.608724in}}{\pgfqpoint{8.214836in}{2.612290in}}%
\pgfpathcurveto{\pgfqpoint{8.211269in}{2.615857in}}{\pgfqpoint{8.206432in}{2.617861in}}{\pgfqpoint{8.201388in}{2.617861in}}%
\pgfpathcurveto{\pgfqpoint{8.196344in}{2.617861in}}{\pgfqpoint{8.191506in}{2.615857in}}{\pgfqpoint{8.187940in}{2.612290in}}%
\pgfpathcurveto{\pgfqpoint{8.184374in}{2.608724in}}{\pgfqpoint{8.182370in}{2.603886in}}{\pgfqpoint{8.182370in}{2.598843in}}%
\pgfpathcurveto{\pgfqpoint{8.182370in}{2.593799in}}{\pgfqpoint{8.184374in}{2.588961in}}{\pgfqpoint{8.187940in}{2.585395in}}%
\pgfpathcurveto{\pgfqpoint{8.191506in}{2.581828in}}{\pgfqpoint{8.196344in}{2.579824in}}{\pgfqpoint{8.201388in}{2.579824in}}%
\pgfpathclose%
\pgfusepath{fill}%
\end{pgfscope}%
\begin{pgfscope}%
\pgfpathrectangle{\pgfqpoint{6.572727in}{0.473000in}}{\pgfqpoint{4.227273in}{3.311000in}}%
\pgfusepath{clip}%
\pgfsetbuttcap%
\pgfsetroundjoin%
\definecolor{currentfill}{rgb}{0.127568,0.566949,0.550556}%
\pgfsetfillcolor{currentfill}%
\pgfsetfillopacity{0.700000}%
\pgfsetlinewidth{0.000000pt}%
\definecolor{currentstroke}{rgb}{0.000000,0.000000,0.000000}%
\pgfsetstrokecolor{currentstroke}%
\pgfsetstrokeopacity{0.700000}%
\pgfsetdash{}{0pt}%
\pgfpathmoveto{\pgfqpoint{7.506977in}{1.536042in}}%
\pgfpathcurveto{\pgfqpoint{7.512021in}{1.536042in}}{\pgfqpoint{7.516859in}{1.538046in}}{\pgfqpoint{7.520425in}{1.541612in}}%
\pgfpathcurveto{\pgfqpoint{7.523992in}{1.545179in}}{\pgfqpoint{7.525995in}{1.550017in}}{\pgfqpoint{7.525995in}{1.555060in}}%
\pgfpathcurveto{\pgfqpoint{7.525995in}{1.560104in}}{\pgfqpoint{7.523992in}{1.564942in}}{\pgfqpoint{7.520425in}{1.568508in}}%
\pgfpathcurveto{\pgfqpoint{7.516859in}{1.572075in}}{\pgfqpoint{7.512021in}{1.574078in}}{\pgfqpoint{7.506977in}{1.574078in}}%
\pgfpathcurveto{\pgfqpoint{7.501934in}{1.574078in}}{\pgfqpoint{7.497096in}{1.572075in}}{\pgfqpoint{7.493529in}{1.568508in}}%
\pgfpathcurveto{\pgfqpoint{7.489963in}{1.564942in}}{\pgfqpoint{7.487959in}{1.560104in}}{\pgfqpoint{7.487959in}{1.555060in}}%
\pgfpathcurveto{\pgfqpoint{7.487959in}{1.550017in}}{\pgfqpoint{7.489963in}{1.545179in}}{\pgfqpoint{7.493529in}{1.541612in}}%
\pgfpathcurveto{\pgfqpoint{7.497096in}{1.538046in}}{\pgfqpoint{7.501934in}{1.536042in}}{\pgfqpoint{7.506977in}{1.536042in}}%
\pgfpathclose%
\pgfusepath{fill}%
\end{pgfscope}%
\begin{pgfscope}%
\pgfpathrectangle{\pgfqpoint{6.572727in}{0.473000in}}{\pgfqpoint{4.227273in}{3.311000in}}%
\pgfusepath{clip}%
\pgfsetbuttcap%
\pgfsetroundjoin%
\definecolor{currentfill}{rgb}{0.993248,0.906157,0.143936}%
\pgfsetfillcolor{currentfill}%
\pgfsetfillopacity{0.700000}%
\pgfsetlinewidth{0.000000pt}%
\definecolor{currentstroke}{rgb}{0.000000,0.000000,0.000000}%
\pgfsetstrokecolor{currentstroke}%
\pgfsetstrokeopacity{0.700000}%
\pgfsetdash{}{0pt}%
\pgfpathmoveto{\pgfqpoint{9.828348in}{1.196933in}}%
\pgfpathcurveto{\pgfqpoint{9.833391in}{1.196933in}}{\pgfqpoint{9.838229in}{1.198937in}}{\pgfqpoint{9.841795in}{1.202503in}}%
\pgfpathcurveto{\pgfqpoint{9.845362in}{1.206069in}}{\pgfqpoint{9.847366in}{1.210907in}}{\pgfqpoint{9.847366in}{1.215951in}}%
\pgfpathcurveto{\pgfqpoint{9.847366in}{1.220995in}}{\pgfqpoint{9.845362in}{1.225832in}}{\pgfqpoint{9.841795in}{1.229399in}}%
\pgfpathcurveto{\pgfqpoint{9.838229in}{1.232965in}}{\pgfqpoint{9.833391in}{1.234969in}}{\pgfqpoint{9.828348in}{1.234969in}}%
\pgfpathcurveto{\pgfqpoint{9.823304in}{1.234969in}}{\pgfqpoint{9.818466in}{1.232965in}}{\pgfqpoint{9.814900in}{1.229399in}}%
\pgfpathcurveto{\pgfqpoint{9.811333in}{1.225832in}}{\pgfqpoint{9.809329in}{1.220995in}}{\pgfqpoint{9.809329in}{1.215951in}}%
\pgfpathcurveto{\pgfqpoint{9.809329in}{1.210907in}}{\pgfqpoint{9.811333in}{1.206069in}}{\pgfqpoint{9.814900in}{1.202503in}}%
\pgfpathcurveto{\pgfqpoint{9.818466in}{1.198937in}}{\pgfqpoint{9.823304in}{1.196933in}}{\pgfqpoint{9.828348in}{1.196933in}}%
\pgfpathclose%
\pgfusepath{fill}%
\end{pgfscope}%
\begin{pgfscope}%
\pgfpathrectangle{\pgfqpoint{6.572727in}{0.473000in}}{\pgfqpoint{4.227273in}{3.311000in}}%
\pgfusepath{clip}%
\pgfsetbuttcap%
\pgfsetroundjoin%
\definecolor{currentfill}{rgb}{0.127568,0.566949,0.550556}%
\pgfsetfillcolor{currentfill}%
\pgfsetfillopacity{0.700000}%
\pgfsetlinewidth{0.000000pt}%
\definecolor{currentstroke}{rgb}{0.000000,0.000000,0.000000}%
\pgfsetstrokecolor{currentstroke}%
\pgfsetstrokeopacity{0.700000}%
\pgfsetdash{}{0pt}%
\pgfpathmoveto{\pgfqpoint{8.108587in}{2.491021in}}%
\pgfpathcurveto{\pgfqpoint{8.113631in}{2.491021in}}{\pgfqpoint{8.118468in}{2.493025in}}{\pgfqpoint{8.122035in}{2.496591in}}%
\pgfpathcurveto{\pgfqpoint{8.125601in}{2.500157in}}{\pgfqpoint{8.127605in}{2.504995in}}{\pgfqpoint{8.127605in}{2.510039in}}%
\pgfpathcurveto{\pgfqpoint{8.127605in}{2.515083in}}{\pgfqpoint{8.125601in}{2.519920in}}{\pgfqpoint{8.122035in}{2.523487in}}%
\pgfpathcurveto{\pgfqpoint{8.118468in}{2.527053in}}{\pgfqpoint{8.113631in}{2.529057in}}{\pgfqpoint{8.108587in}{2.529057in}}%
\pgfpathcurveto{\pgfqpoint{8.103543in}{2.529057in}}{\pgfqpoint{8.098706in}{2.527053in}}{\pgfqpoint{8.095139in}{2.523487in}}%
\pgfpathcurveto{\pgfqpoint{8.091573in}{2.519920in}}{\pgfqpoint{8.089569in}{2.515083in}}{\pgfqpoint{8.089569in}{2.510039in}}%
\pgfpathcurveto{\pgfqpoint{8.089569in}{2.504995in}}{\pgfqpoint{8.091573in}{2.500157in}}{\pgfqpoint{8.095139in}{2.496591in}}%
\pgfpathcurveto{\pgfqpoint{8.098706in}{2.493025in}}{\pgfqpoint{8.103543in}{2.491021in}}{\pgfqpoint{8.108587in}{2.491021in}}%
\pgfpathclose%
\pgfusepath{fill}%
\end{pgfscope}%
\begin{pgfscope}%
\pgfpathrectangle{\pgfqpoint{6.572727in}{0.473000in}}{\pgfqpoint{4.227273in}{3.311000in}}%
\pgfusepath{clip}%
\pgfsetbuttcap%
\pgfsetroundjoin%
\definecolor{currentfill}{rgb}{0.993248,0.906157,0.143936}%
\pgfsetfillcolor{currentfill}%
\pgfsetfillopacity{0.700000}%
\pgfsetlinewidth{0.000000pt}%
\definecolor{currentstroke}{rgb}{0.000000,0.000000,0.000000}%
\pgfsetstrokecolor{currentstroke}%
\pgfsetstrokeopacity{0.700000}%
\pgfsetdash{}{0pt}%
\pgfpathmoveto{\pgfqpoint{10.210872in}{1.489622in}}%
\pgfpathcurveto{\pgfqpoint{10.215916in}{1.489622in}}{\pgfqpoint{10.220754in}{1.491626in}}{\pgfqpoint{10.224320in}{1.495193in}}%
\pgfpathcurveto{\pgfqpoint{10.227887in}{1.498759in}}{\pgfqpoint{10.229890in}{1.503597in}}{\pgfqpoint{10.229890in}{1.508641in}}%
\pgfpathcurveto{\pgfqpoint{10.229890in}{1.513684in}}{\pgfqpoint{10.227887in}{1.518522in}}{\pgfqpoint{10.224320in}{1.522088in}}%
\pgfpathcurveto{\pgfqpoint{10.220754in}{1.525655in}}{\pgfqpoint{10.215916in}{1.527659in}}{\pgfqpoint{10.210872in}{1.527659in}}%
\pgfpathcurveto{\pgfqpoint{10.205829in}{1.527659in}}{\pgfqpoint{10.200991in}{1.525655in}}{\pgfqpoint{10.197424in}{1.522088in}}%
\pgfpathcurveto{\pgfqpoint{10.193858in}{1.518522in}}{\pgfqpoint{10.191854in}{1.513684in}}{\pgfqpoint{10.191854in}{1.508641in}}%
\pgfpathcurveto{\pgfqpoint{10.191854in}{1.503597in}}{\pgfqpoint{10.193858in}{1.498759in}}{\pgfqpoint{10.197424in}{1.495193in}}%
\pgfpathcurveto{\pgfqpoint{10.200991in}{1.491626in}}{\pgfqpoint{10.205829in}{1.489622in}}{\pgfqpoint{10.210872in}{1.489622in}}%
\pgfpathclose%
\pgfusepath{fill}%
\end{pgfscope}%
\begin{pgfscope}%
\pgfpathrectangle{\pgfqpoint{6.572727in}{0.473000in}}{\pgfqpoint{4.227273in}{3.311000in}}%
\pgfusepath{clip}%
\pgfsetbuttcap%
\pgfsetroundjoin%
\definecolor{currentfill}{rgb}{0.127568,0.566949,0.550556}%
\pgfsetfillcolor{currentfill}%
\pgfsetfillopacity{0.700000}%
\pgfsetlinewidth{0.000000pt}%
\definecolor{currentstroke}{rgb}{0.000000,0.000000,0.000000}%
\pgfsetstrokecolor{currentstroke}%
\pgfsetstrokeopacity{0.700000}%
\pgfsetdash{}{0pt}%
\pgfpathmoveto{\pgfqpoint{7.602098in}{1.559941in}}%
\pgfpathcurveto{\pgfqpoint{7.607142in}{1.559941in}}{\pgfqpoint{7.611980in}{1.561945in}}{\pgfqpoint{7.615546in}{1.565512in}}%
\pgfpathcurveto{\pgfqpoint{7.619113in}{1.569078in}}{\pgfqpoint{7.621117in}{1.573916in}}{\pgfqpoint{7.621117in}{1.578959in}}%
\pgfpathcurveto{\pgfqpoint{7.621117in}{1.584003in}}{\pgfqpoint{7.619113in}{1.588841in}}{\pgfqpoint{7.615546in}{1.592407in}}%
\pgfpathcurveto{\pgfqpoint{7.611980in}{1.595974in}}{\pgfqpoint{7.607142in}{1.597978in}}{\pgfqpoint{7.602098in}{1.597978in}}%
\pgfpathcurveto{\pgfqpoint{7.597055in}{1.597978in}}{\pgfqpoint{7.592217in}{1.595974in}}{\pgfqpoint{7.588651in}{1.592407in}}%
\pgfpathcurveto{\pgfqpoint{7.585084in}{1.588841in}}{\pgfqpoint{7.583080in}{1.584003in}}{\pgfqpoint{7.583080in}{1.578959in}}%
\pgfpathcurveto{\pgfqpoint{7.583080in}{1.573916in}}{\pgfqpoint{7.585084in}{1.569078in}}{\pgfqpoint{7.588651in}{1.565512in}}%
\pgfpathcurveto{\pgfqpoint{7.592217in}{1.561945in}}{\pgfqpoint{7.597055in}{1.559941in}}{\pgfqpoint{7.602098in}{1.559941in}}%
\pgfpathclose%
\pgfusepath{fill}%
\end{pgfscope}%
\begin{pgfscope}%
\pgfpathrectangle{\pgfqpoint{6.572727in}{0.473000in}}{\pgfqpoint{4.227273in}{3.311000in}}%
\pgfusepath{clip}%
\pgfsetbuttcap%
\pgfsetroundjoin%
\definecolor{currentfill}{rgb}{0.127568,0.566949,0.550556}%
\pgfsetfillcolor{currentfill}%
\pgfsetfillopacity{0.700000}%
\pgfsetlinewidth{0.000000pt}%
\definecolor{currentstroke}{rgb}{0.000000,0.000000,0.000000}%
\pgfsetstrokecolor{currentstroke}%
\pgfsetstrokeopacity{0.700000}%
\pgfsetdash{}{0pt}%
\pgfpathmoveto{\pgfqpoint{8.272134in}{1.495187in}}%
\pgfpathcurveto{\pgfqpoint{8.277178in}{1.495187in}}{\pgfqpoint{8.282016in}{1.497191in}}{\pgfqpoint{8.285582in}{1.500758in}}%
\pgfpathcurveto{\pgfqpoint{8.289148in}{1.504324in}}{\pgfqpoint{8.291152in}{1.509162in}}{\pgfqpoint{8.291152in}{1.514206in}}%
\pgfpathcurveto{\pgfqpoint{8.291152in}{1.519249in}}{\pgfqpoint{8.289148in}{1.524087in}}{\pgfqpoint{8.285582in}{1.527653in}}%
\pgfpathcurveto{\pgfqpoint{8.282016in}{1.531220in}}{\pgfqpoint{8.277178in}{1.533224in}}{\pgfqpoint{8.272134in}{1.533224in}}%
\pgfpathcurveto{\pgfqpoint{8.267091in}{1.533224in}}{\pgfqpoint{8.262253in}{1.531220in}}{\pgfqpoint{8.258686in}{1.527653in}}%
\pgfpathcurveto{\pgfqpoint{8.255120in}{1.524087in}}{\pgfqpoint{8.253116in}{1.519249in}}{\pgfqpoint{8.253116in}{1.514206in}}%
\pgfpathcurveto{\pgfqpoint{8.253116in}{1.509162in}}{\pgfqpoint{8.255120in}{1.504324in}}{\pgfqpoint{8.258686in}{1.500758in}}%
\pgfpathcurveto{\pgfqpoint{8.262253in}{1.497191in}}{\pgfqpoint{8.267091in}{1.495187in}}{\pgfqpoint{8.272134in}{1.495187in}}%
\pgfpathclose%
\pgfusepath{fill}%
\end{pgfscope}%
\begin{pgfscope}%
\pgfpathrectangle{\pgfqpoint{6.572727in}{0.473000in}}{\pgfqpoint{4.227273in}{3.311000in}}%
\pgfusepath{clip}%
\pgfsetbuttcap%
\pgfsetroundjoin%
\definecolor{currentfill}{rgb}{0.127568,0.566949,0.550556}%
\pgfsetfillcolor{currentfill}%
\pgfsetfillopacity{0.700000}%
\pgfsetlinewidth{0.000000pt}%
\definecolor{currentstroke}{rgb}{0.000000,0.000000,0.000000}%
\pgfsetstrokecolor{currentstroke}%
\pgfsetstrokeopacity{0.700000}%
\pgfsetdash{}{0pt}%
\pgfpathmoveto{\pgfqpoint{8.312181in}{3.366405in}}%
\pgfpathcurveto{\pgfqpoint{8.317225in}{3.366405in}}{\pgfqpoint{8.322063in}{3.368409in}}{\pgfqpoint{8.325629in}{3.371976in}}%
\pgfpathcurveto{\pgfqpoint{8.329195in}{3.375542in}}{\pgfqpoint{8.331199in}{3.380380in}}{\pgfqpoint{8.331199in}{3.385423in}}%
\pgfpathcurveto{\pgfqpoint{8.331199in}{3.390467in}}{\pgfqpoint{8.329195in}{3.395305in}}{\pgfqpoint{8.325629in}{3.398871in}}%
\pgfpathcurveto{\pgfqpoint{8.322063in}{3.402438in}}{\pgfqpoint{8.317225in}{3.404442in}}{\pgfqpoint{8.312181in}{3.404442in}}%
\pgfpathcurveto{\pgfqpoint{8.307138in}{3.404442in}}{\pgfqpoint{8.302300in}{3.402438in}}{\pgfqpoint{8.298733in}{3.398871in}}%
\pgfpathcurveto{\pgfqpoint{8.295167in}{3.395305in}}{\pgfqpoint{8.293163in}{3.390467in}}{\pgfqpoint{8.293163in}{3.385423in}}%
\pgfpathcurveto{\pgfqpoint{8.293163in}{3.380380in}}{\pgfqpoint{8.295167in}{3.375542in}}{\pgfqpoint{8.298733in}{3.371976in}}%
\pgfpathcurveto{\pgfqpoint{8.302300in}{3.368409in}}{\pgfqpoint{8.307138in}{3.366405in}}{\pgfqpoint{8.312181in}{3.366405in}}%
\pgfpathclose%
\pgfusepath{fill}%
\end{pgfscope}%
\begin{pgfscope}%
\pgfpathrectangle{\pgfqpoint{6.572727in}{0.473000in}}{\pgfqpoint{4.227273in}{3.311000in}}%
\pgfusepath{clip}%
\pgfsetbuttcap%
\pgfsetroundjoin%
\definecolor{currentfill}{rgb}{0.127568,0.566949,0.550556}%
\pgfsetfillcolor{currentfill}%
\pgfsetfillopacity{0.700000}%
\pgfsetlinewidth{0.000000pt}%
\definecolor{currentstroke}{rgb}{0.000000,0.000000,0.000000}%
\pgfsetstrokecolor{currentstroke}%
\pgfsetstrokeopacity{0.700000}%
\pgfsetdash{}{0pt}%
\pgfpathmoveto{\pgfqpoint{7.728138in}{1.903778in}}%
\pgfpathcurveto{\pgfqpoint{7.733182in}{1.903778in}}{\pgfqpoint{7.738019in}{1.905782in}}{\pgfqpoint{7.741586in}{1.909348in}}%
\pgfpathcurveto{\pgfqpoint{7.745152in}{1.912915in}}{\pgfqpoint{7.747156in}{1.917753in}}{\pgfqpoint{7.747156in}{1.922796in}}%
\pgfpathcurveto{\pgfqpoint{7.747156in}{1.927840in}}{\pgfqpoint{7.745152in}{1.932678in}}{\pgfqpoint{7.741586in}{1.936244in}}%
\pgfpathcurveto{\pgfqpoint{7.738019in}{1.939811in}}{\pgfqpoint{7.733182in}{1.941814in}}{\pgfqpoint{7.728138in}{1.941814in}}%
\pgfpathcurveto{\pgfqpoint{7.723094in}{1.941814in}}{\pgfqpoint{7.718257in}{1.939811in}}{\pgfqpoint{7.714690in}{1.936244in}}%
\pgfpathcurveto{\pgfqpoint{7.711124in}{1.932678in}}{\pgfqpoint{7.709120in}{1.927840in}}{\pgfqpoint{7.709120in}{1.922796in}}%
\pgfpathcurveto{\pgfqpoint{7.709120in}{1.917753in}}{\pgfqpoint{7.711124in}{1.912915in}}{\pgfqpoint{7.714690in}{1.909348in}}%
\pgfpathcurveto{\pgfqpoint{7.718257in}{1.905782in}}{\pgfqpoint{7.723094in}{1.903778in}}{\pgfqpoint{7.728138in}{1.903778in}}%
\pgfpathclose%
\pgfusepath{fill}%
\end{pgfscope}%
\begin{pgfscope}%
\pgfpathrectangle{\pgfqpoint{6.572727in}{0.473000in}}{\pgfqpoint{4.227273in}{3.311000in}}%
\pgfusepath{clip}%
\pgfsetbuttcap%
\pgfsetroundjoin%
\definecolor{currentfill}{rgb}{0.127568,0.566949,0.550556}%
\pgfsetfillcolor{currentfill}%
\pgfsetfillopacity{0.700000}%
\pgfsetlinewidth{0.000000pt}%
\definecolor{currentstroke}{rgb}{0.000000,0.000000,0.000000}%
\pgfsetstrokecolor{currentstroke}%
\pgfsetstrokeopacity{0.700000}%
\pgfsetdash{}{0pt}%
\pgfpathmoveto{\pgfqpoint{8.401447in}{3.033845in}}%
\pgfpathcurveto{\pgfqpoint{8.406491in}{3.033845in}}{\pgfqpoint{8.411328in}{3.035849in}}{\pgfqpoint{8.414895in}{3.039416in}}%
\pgfpathcurveto{\pgfqpoint{8.418461in}{3.042982in}}{\pgfqpoint{8.420465in}{3.047820in}}{\pgfqpoint{8.420465in}{3.052863in}}%
\pgfpathcurveto{\pgfqpoint{8.420465in}{3.057907in}}{\pgfqpoint{8.418461in}{3.062745in}}{\pgfqpoint{8.414895in}{3.066311in}}%
\pgfpathcurveto{\pgfqpoint{8.411328in}{3.069878in}}{\pgfqpoint{8.406491in}{3.071882in}}{\pgfqpoint{8.401447in}{3.071882in}}%
\pgfpathcurveto{\pgfqpoint{8.396403in}{3.071882in}}{\pgfqpoint{8.391565in}{3.069878in}}{\pgfqpoint{8.387999in}{3.066311in}}%
\pgfpathcurveto{\pgfqpoint{8.384433in}{3.062745in}}{\pgfqpoint{8.382429in}{3.057907in}}{\pgfqpoint{8.382429in}{3.052863in}}%
\pgfpathcurveto{\pgfqpoint{8.382429in}{3.047820in}}{\pgfqpoint{8.384433in}{3.042982in}}{\pgfqpoint{8.387999in}{3.039416in}}%
\pgfpathcurveto{\pgfqpoint{8.391565in}{3.035849in}}{\pgfqpoint{8.396403in}{3.033845in}}{\pgfqpoint{8.401447in}{3.033845in}}%
\pgfpathclose%
\pgfusepath{fill}%
\end{pgfscope}%
\begin{pgfscope}%
\pgfpathrectangle{\pgfqpoint{6.572727in}{0.473000in}}{\pgfqpoint{4.227273in}{3.311000in}}%
\pgfusepath{clip}%
\pgfsetbuttcap%
\pgfsetroundjoin%
\definecolor{currentfill}{rgb}{0.993248,0.906157,0.143936}%
\pgfsetfillcolor{currentfill}%
\pgfsetfillopacity{0.700000}%
\pgfsetlinewidth{0.000000pt}%
\definecolor{currentstroke}{rgb}{0.000000,0.000000,0.000000}%
\pgfsetstrokecolor{currentstroke}%
\pgfsetstrokeopacity{0.700000}%
\pgfsetdash{}{0pt}%
\pgfpathmoveto{\pgfqpoint{9.217785in}{1.544585in}}%
\pgfpathcurveto{\pgfqpoint{9.222829in}{1.544585in}}{\pgfqpoint{9.227667in}{1.546589in}}{\pgfqpoint{9.231233in}{1.550155in}}%
\pgfpathcurveto{\pgfqpoint{9.234800in}{1.553722in}}{\pgfqpoint{9.236804in}{1.558559in}}{\pgfqpoint{9.236804in}{1.563603in}}%
\pgfpathcurveto{\pgfqpoint{9.236804in}{1.568647in}}{\pgfqpoint{9.234800in}{1.573485in}}{\pgfqpoint{9.231233in}{1.577051in}}%
\pgfpathcurveto{\pgfqpoint{9.227667in}{1.580617in}}{\pgfqpoint{9.222829in}{1.582621in}}{\pgfqpoint{9.217785in}{1.582621in}}%
\pgfpathcurveto{\pgfqpoint{9.212742in}{1.582621in}}{\pgfqpoint{9.207904in}{1.580617in}}{\pgfqpoint{9.204338in}{1.577051in}}%
\pgfpathcurveto{\pgfqpoint{9.200771in}{1.573485in}}{\pgfqpoint{9.198767in}{1.568647in}}{\pgfqpoint{9.198767in}{1.563603in}}%
\pgfpathcurveto{\pgfqpoint{9.198767in}{1.558559in}}{\pgfqpoint{9.200771in}{1.553722in}}{\pgfqpoint{9.204338in}{1.550155in}}%
\pgfpathcurveto{\pgfqpoint{9.207904in}{1.546589in}}{\pgfqpoint{9.212742in}{1.544585in}}{\pgfqpoint{9.217785in}{1.544585in}}%
\pgfpathclose%
\pgfusepath{fill}%
\end{pgfscope}%
\begin{pgfscope}%
\pgfpathrectangle{\pgfqpoint{6.572727in}{0.473000in}}{\pgfqpoint{4.227273in}{3.311000in}}%
\pgfusepath{clip}%
\pgfsetbuttcap%
\pgfsetroundjoin%
\definecolor{currentfill}{rgb}{0.267004,0.004874,0.329415}%
\pgfsetfillcolor{currentfill}%
\pgfsetfillopacity{0.700000}%
\pgfsetlinewidth{0.000000pt}%
\definecolor{currentstroke}{rgb}{0.000000,0.000000,0.000000}%
\pgfsetstrokecolor{currentstroke}%
\pgfsetstrokeopacity{0.700000}%
\pgfsetdash{}{0pt}%
\pgfpathmoveto{\pgfqpoint{6.764876in}{1.938117in}}%
\pgfpathcurveto{\pgfqpoint{6.769920in}{1.938117in}}{\pgfqpoint{6.774757in}{1.940121in}}{\pgfqpoint{6.778324in}{1.943688in}}%
\pgfpathcurveto{\pgfqpoint{6.781890in}{1.947254in}}{\pgfqpoint{6.783894in}{1.952092in}}{\pgfqpoint{6.783894in}{1.957135in}}%
\pgfpathcurveto{\pgfqpoint{6.783894in}{1.962179in}}{\pgfqpoint{6.781890in}{1.967017in}}{\pgfqpoint{6.778324in}{1.970583in}}%
\pgfpathcurveto{\pgfqpoint{6.774757in}{1.974150in}}{\pgfqpoint{6.769920in}{1.976154in}}{\pgfqpoint{6.764876in}{1.976154in}}%
\pgfpathcurveto{\pgfqpoint{6.759832in}{1.976154in}}{\pgfqpoint{6.754995in}{1.974150in}}{\pgfqpoint{6.751428in}{1.970583in}}%
\pgfpathcurveto{\pgfqpoint{6.747862in}{1.967017in}}{\pgfqpoint{6.745858in}{1.962179in}}{\pgfqpoint{6.745858in}{1.957135in}}%
\pgfpathcurveto{\pgfqpoint{6.745858in}{1.952092in}}{\pgfqpoint{6.747862in}{1.947254in}}{\pgfqpoint{6.751428in}{1.943688in}}%
\pgfpathcurveto{\pgfqpoint{6.754995in}{1.940121in}}{\pgfqpoint{6.759832in}{1.938117in}}{\pgfqpoint{6.764876in}{1.938117in}}%
\pgfpathclose%
\pgfusepath{fill}%
\end{pgfscope}%
\begin{pgfscope}%
\pgfpathrectangle{\pgfqpoint{6.572727in}{0.473000in}}{\pgfqpoint{4.227273in}{3.311000in}}%
\pgfusepath{clip}%
\pgfsetbuttcap%
\pgfsetroundjoin%
\definecolor{currentfill}{rgb}{0.127568,0.566949,0.550556}%
\pgfsetfillcolor{currentfill}%
\pgfsetfillopacity{0.700000}%
\pgfsetlinewidth{0.000000pt}%
\definecolor{currentstroke}{rgb}{0.000000,0.000000,0.000000}%
\pgfsetstrokecolor{currentstroke}%
\pgfsetstrokeopacity{0.700000}%
\pgfsetdash{}{0pt}%
\pgfpathmoveto{\pgfqpoint{7.804196in}{0.952529in}}%
\pgfpathcurveto{\pgfqpoint{7.809240in}{0.952529in}}{\pgfqpoint{7.814078in}{0.954532in}}{\pgfqpoint{7.817644in}{0.958099in}}%
\pgfpathcurveto{\pgfqpoint{7.821211in}{0.961665in}}{\pgfqpoint{7.823214in}{0.966503in}}{\pgfqpoint{7.823214in}{0.971547in}}%
\pgfpathcurveto{\pgfqpoint{7.823214in}{0.976590in}}{\pgfqpoint{7.821211in}{0.981428in}}{\pgfqpoint{7.817644in}{0.984995in}}%
\pgfpathcurveto{\pgfqpoint{7.814078in}{0.988561in}}{\pgfqpoint{7.809240in}{0.990565in}}{\pgfqpoint{7.804196in}{0.990565in}}%
\pgfpathcurveto{\pgfqpoint{7.799153in}{0.990565in}}{\pgfqpoint{7.794315in}{0.988561in}}{\pgfqpoint{7.790748in}{0.984995in}}%
\pgfpathcurveto{\pgfqpoint{7.787182in}{0.981428in}}{\pgfqpoint{7.785178in}{0.976590in}}{\pgfqpoint{7.785178in}{0.971547in}}%
\pgfpathcurveto{\pgfqpoint{7.785178in}{0.966503in}}{\pgfqpoint{7.787182in}{0.961665in}}{\pgfqpoint{7.790748in}{0.958099in}}%
\pgfpathcurveto{\pgfqpoint{7.794315in}{0.954532in}}{\pgfqpoint{7.799153in}{0.952529in}}{\pgfqpoint{7.804196in}{0.952529in}}%
\pgfpathclose%
\pgfusepath{fill}%
\end{pgfscope}%
\begin{pgfscope}%
\pgfpathrectangle{\pgfqpoint{6.572727in}{0.473000in}}{\pgfqpoint{4.227273in}{3.311000in}}%
\pgfusepath{clip}%
\pgfsetbuttcap%
\pgfsetroundjoin%
\definecolor{currentfill}{rgb}{0.127568,0.566949,0.550556}%
\pgfsetfillcolor{currentfill}%
\pgfsetfillopacity{0.700000}%
\pgfsetlinewidth{0.000000pt}%
\definecolor{currentstroke}{rgb}{0.000000,0.000000,0.000000}%
\pgfsetstrokecolor{currentstroke}%
\pgfsetstrokeopacity{0.700000}%
\pgfsetdash{}{0pt}%
\pgfpathmoveto{\pgfqpoint{8.931081in}{3.110504in}}%
\pgfpathcurveto{\pgfqpoint{8.936125in}{3.110504in}}{\pgfqpoint{8.940963in}{3.112508in}}{\pgfqpoint{8.944529in}{3.116074in}}%
\pgfpathcurveto{\pgfqpoint{8.948096in}{3.119641in}}{\pgfqpoint{8.950100in}{3.124479in}}{\pgfqpoint{8.950100in}{3.129522in}}%
\pgfpathcurveto{\pgfqpoint{8.950100in}{3.134566in}}{\pgfqpoint{8.948096in}{3.139404in}}{\pgfqpoint{8.944529in}{3.142970in}}%
\pgfpathcurveto{\pgfqpoint{8.940963in}{3.146537in}}{\pgfqpoint{8.936125in}{3.148540in}}{\pgfqpoint{8.931081in}{3.148540in}}%
\pgfpathcurveto{\pgfqpoint{8.926038in}{3.148540in}}{\pgfqpoint{8.921200in}{3.146537in}}{\pgfqpoint{8.917634in}{3.142970in}}%
\pgfpathcurveto{\pgfqpoint{8.914067in}{3.139404in}}{\pgfqpoint{8.912063in}{3.134566in}}{\pgfqpoint{8.912063in}{3.129522in}}%
\pgfpathcurveto{\pgfqpoint{8.912063in}{3.124479in}}{\pgfqpoint{8.914067in}{3.119641in}}{\pgfqpoint{8.917634in}{3.116074in}}%
\pgfpathcurveto{\pgfqpoint{8.921200in}{3.112508in}}{\pgfqpoint{8.926038in}{3.110504in}}{\pgfqpoint{8.931081in}{3.110504in}}%
\pgfpathclose%
\pgfusepath{fill}%
\end{pgfscope}%
\begin{pgfscope}%
\pgfpathrectangle{\pgfqpoint{6.572727in}{0.473000in}}{\pgfqpoint{4.227273in}{3.311000in}}%
\pgfusepath{clip}%
\pgfsetbuttcap%
\pgfsetroundjoin%
\definecolor{currentfill}{rgb}{0.127568,0.566949,0.550556}%
\pgfsetfillcolor{currentfill}%
\pgfsetfillopacity{0.700000}%
\pgfsetlinewidth{0.000000pt}%
\definecolor{currentstroke}{rgb}{0.000000,0.000000,0.000000}%
\pgfsetstrokecolor{currentstroke}%
\pgfsetstrokeopacity{0.700000}%
\pgfsetdash{}{0pt}%
\pgfpathmoveto{\pgfqpoint{8.546253in}{3.262360in}}%
\pgfpathcurveto{\pgfqpoint{8.551297in}{3.262360in}}{\pgfqpoint{8.556135in}{3.264364in}}{\pgfqpoint{8.559701in}{3.267931in}}%
\pgfpathcurveto{\pgfqpoint{8.563267in}{3.271497in}}{\pgfqpoint{8.565271in}{3.276335in}}{\pgfqpoint{8.565271in}{3.281378in}}%
\pgfpathcurveto{\pgfqpoint{8.565271in}{3.286422in}}{\pgfqpoint{8.563267in}{3.291260in}}{\pgfqpoint{8.559701in}{3.294826in}}%
\pgfpathcurveto{\pgfqpoint{8.556135in}{3.298393in}}{\pgfqpoint{8.551297in}{3.300397in}}{\pgfqpoint{8.546253in}{3.300397in}}%
\pgfpathcurveto{\pgfqpoint{8.541210in}{3.300397in}}{\pgfqpoint{8.536372in}{3.298393in}}{\pgfqpoint{8.532805in}{3.294826in}}%
\pgfpathcurveto{\pgfqpoint{8.529239in}{3.291260in}}{\pgfqpoint{8.527235in}{3.286422in}}{\pgfqpoint{8.527235in}{3.281378in}}%
\pgfpathcurveto{\pgfqpoint{8.527235in}{3.276335in}}{\pgfqpoint{8.529239in}{3.271497in}}{\pgfqpoint{8.532805in}{3.267931in}}%
\pgfpathcurveto{\pgfqpoint{8.536372in}{3.264364in}}{\pgfqpoint{8.541210in}{3.262360in}}{\pgfqpoint{8.546253in}{3.262360in}}%
\pgfpathclose%
\pgfusepath{fill}%
\end{pgfscope}%
\begin{pgfscope}%
\pgfpathrectangle{\pgfqpoint{6.572727in}{0.473000in}}{\pgfqpoint{4.227273in}{3.311000in}}%
\pgfusepath{clip}%
\pgfsetbuttcap%
\pgfsetroundjoin%
\definecolor{currentfill}{rgb}{0.127568,0.566949,0.550556}%
\pgfsetfillcolor{currentfill}%
\pgfsetfillopacity{0.700000}%
\pgfsetlinewidth{0.000000pt}%
\definecolor{currentstroke}{rgb}{0.000000,0.000000,0.000000}%
\pgfsetstrokecolor{currentstroke}%
\pgfsetstrokeopacity{0.700000}%
\pgfsetdash{}{0pt}%
\pgfpathmoveto{\pgfqpoint{8.474563in}{2.271404in}}%
\pgfpathcurveto{\pgfqpoint{8.479607in}{2.271404in}}{\pgfqpoint{8.484445in}{2.273408in}}{\pgfqpoint{8.488011in}{2.276975in}}%
\pgfpathcurveto{\pgfqpoint{8.491578in}{2.280541in}}{\pgfqpoint{8.493581in}{2.285379in}}{\pgfqpoint{8.493581in}{2.290423in}}%
\pgfpathcurveto{\pgfqpoint{8.493581in}{2.295466in}}{\pgfqpoint{8.491578in}{2.300304in}}{\pgfqpoint{8.488011in}{2.303870in}}%
\pgfpathcurveto{\pgfqpoint{8.484445in}{2.307437in}}{\pgfqpoint{8.479607in}{2.309441in}}{\pgfqpoint{8.474563in}{2.309441in}}%
\pgfpathcurveto{\pgfqpoint{8.469520in}{2.309441in}}{\pgfqpoint{8.464682in}{2.307437in}}{\pgfqpoint{8.461115in}{2.303870in}}%
\pgfpathcurveto{\pgfqpoint{8.457549in}{2.300304in}}{\pgfqpoint{8.455545in}{2.295466in}}{\pgfqpoint{8.455545in}{2.290423in}}%
\pgfpathcurveto{\pgfqpoint{8.455545in}{2.285379in}}{\pgfqpoint{8.457549in}{2.280541in}}{\pgfqpoint{8.461115in}{2.276975in}}%
\pgfpathcurveto{\pgfqpoint{8.464682in}{2.273408in}}{\pgfqpoint{8.469520in}{2.271404in}}{\pgfqpoint{8.474563in}{2.271404in}}%
\pgfpathclose%
\pgfusepath{fill}%
\end{pgfscope}%
\begin{pgfscope}%
\pgfpathrectangle{\pgfqpoint{6.572727in}{0.473000in}}{\pgfqpoint{4.227273in}{3.311000in}}%
\pgfusepath{clip}%
\pgfsetbuttcap%
\pgfsetroundjoin%
\definecolor{currentfill}{rgb}{0.993248,0.906157,0.143936}%
\pgfsetfillcolor{currentfill}%
\pgfsetfillopacity{0.700000}%
\pgfsetlinewidth{0.000000pt}%
\definecolor{currentstroke}{rgb}{0.000000,0.000000,0.000000}%
\pgfsetstrokecolor{currentstroke}%
\pgfsetstrokeopacity{0.700000}%
\pgfsetdash{}{0pt}%
\pgfpathmoveto{\pgfqpoint{9.510394in}{1.406104in}}%
\pgfpathcurveto{\pgfqpoint{9.515437in}{1.406104in}}{\pgfqpoint{9.520275in}{1.408108in}}{\pgfqpoint{9.523841in}{1.411674in}}%
\pgfpathcurveto{\pgfqpoint{9.527408in}{1.415241in}}{\pgfqpoint{9.529412in}{1.420078in}}{\pgfqpoint{9.529412in}{1.425122in}}%
\pgfpathcurveto{\pgfqpoint{9.529412in}{1.430166in}}{\pgfqpoint{9.527408in}{1.435004in}}{\pgfqpoint{9.523841in}{1.438570in}}%
\pgfpathcurveto{\pgfqpoint{9.520275in}{1.442136in}}{\pgfqpoint{9.515437in}{1.444140in}}{\pgfqpoint{9.510394in}{1.444140in}}%
\pgfpathcurveto{\pgfqpoint{9.505350in}{1.444140in}}{\pgfqpoint{9.500512in}{1.442136in}}{\pgfqpoint{9.496946in}{1.438570in}}%
\pgfpathcurveto{\pgfqpoint{9.493379in}{1.435004in}}{\pgfqpoint{9.491375in}{1.430166in}}{\pgfqpoint{9.491375in}{1.425122in}}%
\pgfpathcurveto{\pgfqpoint{9.491375in}{1.420078in}}{\pgfqpoint{9.493379in}{1.415241in}}{\pgfqpoint{9.496946in}{1.411674in}}%
\pgfpathcurveto{\pgfqpoint{9.500512in}{1.408108in}}{\pgfqpoint{9.505350in}{1.406104in}}{\pgfqpoint{9.510394in}{1.406104in}}%
\pgfpathclose%
\pgfusepath{fill}%
\end{pgfscope}%
\begin{pgfscope}%
\pgfpathrectangle{\pgfqpoint{6.572727in}{0.473000in}}{\pgfqpoint{4.227273in}{3.311000in}}%
\pgfusepath{clip}%
\pgfsetbuttcap%
\pgfsetroundjoin%
\definecolor{currentfill}{rgb}{0.127568,0.566949,0.550556}%
\pgfsetfillcolor{currentfill}%
\pgfsetfillopacity{0.700000}%
\pgfsetlinewidth{0.000000pt}%
\definecolor{currentstroke}{rgb}{0.000000,0.000000,0.000000}%
\pgfsetstrokecolor{currentstroke}%
\pgfsetstrokeopacity{0.700000}%
\pgfsetdash{}{0pt}%
\pgfpathmoveto{\pgfqpoint{7.520916in}{1.253848in}}%
\pgfpathcurveto{\pgfqpoint{7.525959in}{1.253848in}}{\pgfqpoint{7.530797in}{1.255852in}}{\pgfqpoint{7.534363in}{1.259418in}}%
\pgfpathcurveto{\pgfqpoint{7.537930in}{1.262984in}}{\pgfqpoint{7.539934in}{1.267822in}}{\pgfqpoint{7.539934in}{1.272866in}}%
\pgfpathcurveto{\pgfqpoint{7.539934in}{1.277910in}}{\pgfqpoint{7.537930in}{1.282747in}}{\pgfqpoint{7.534363in}{1.286314in}}%
\pgfpathcurveto{\pgfqpoint{7.530797in}{1.289880in}}{\pgfqpoint{7.525959in}{1.291884in}}{\pgfqpoint{7.520916in}{1.291884in}}%
\pgfpathcurveto{\pgfqpoint{7.515872in}{1.291884in}}{\pgfqpoint{7.511034in}{1.289880in}}{\pgfqpoint{7.507468in}{1.286314in}}%
\pgfpathcurveto{\pgfqpoint{7.503901in}{1.282747in}}{\pgfqpoint{7.501897in}{1.277910in}}{\pgfqpoint{7.501897in}{1.272866in}}%
\pgfpathcurveto{\pgfqpoint{7.501897in}{1.267822in}}{\pgfqpoint{7.503901in}{1.262984in}}{\pgfqpoint{7.507468in}{1.259418in}}%
\pgfpathcurveto{\pgfqpoint{7.511034in}{1.255852in}}{\pgfqpoint{7.515872in}{1.253848in}}{\pgfqpoint{7.520916in}{1.253848in}}%
\pgfpathclose%
\pgfusepath{fill}%
\end{pgfscope}%
\begin{pgfscope}%
\pgfpathrectangle{\pgfqpoint{6.572727in}{0.473000in}}{\pgfqpoint{4.227273in}{3.311000in}}%
\pgfusepath{clip}%
\pgfsetbuttcap%
\pgfsetroundjoin%
\definecolor{currentfill}{rgb}{0.127568,0.566949,0.550556}%
\pgfsetfillcolor{currentfill}%
\pgfsetfillopacity{0.700000}%
\pgfsetlinewidth{0.000000pt}%
\definecolor{currentstroke}{rgb}{0.000000,0.000000,0.000000}%
\pgfsetstrokecolor{currentstroke}%
\pgfsetstrokeopacity{0.700000}%
\pgfsetdash{}{0pt}%
\pgfpathmoveto{\pgfqpoint{7.748101in}{1.559248in}}%
\pgfpathcurveto{\pgfqpoint{7.753145in}{1.559248in}}{\pgfqpoint{7.757983in}{1.561252in}}{\pgfqpoint{7.761549in}{1.564818in}}%
\pgfpathcurveto{\pgfqpoint{7.765116in}{1.568385in}}{\pgfqpoint{7.767120in}{1.573223in}}{\pgfqpoint{7.767120in}{1.578266in}}%
\pgfpathcurveto{\pgfqpoint{7.767120in}{1.583310in}}{\pgfqpoint{7.765116in}{1.588148in}}{\pgfqpoint{7.761549in}{1.591714in}}%
\pgfpathcurveto{\pgfqpoint{7.757983in}{1.595281in}}{\pgfqpoint{7.753145in}{1.597284in}}{\pgfqpoint{7.748101in}{1.597284in}}%
\pgfpathcurveto{\pgfqpoint{7.743058in}{1.597284in}}{\pgfqpoint{7.738220in}{1.595281in}}{\pgfqpoint{7.734654in}{1.591714in}}%
\pgfpathcurveto{\pgfqpoint{7.731087in}{1.588148in}}{\pgfqpoint{7.729083in}{1.583310in}}{\pgfqpoint{7.729083in}{1.578266in}}%
\pgfpathcurveto{\pgfqpoint{7.729083in}{1.573223in}}{\pgfqpoint{7.731087in}{1.568385in}}{\pgfqpoint{7.734654in}{1.564818in}}%
\pgfpathcurveto{\pgfqpoint{7.738220in}{1.561252in}}{\pgfqpoint{7.743058in}{1.559248in}}{\pgfqpoint{7.748101in}{1.559248in}}%
\pgfpathclose%
\pgfusepath{fill}%
\end{pgfscope}%
\begin{pgfscope}%
\pgfpathrectangle{\pgfqpoint{6.572727in}{0.473000in}}{\pgfqpoint{4.227273in}{3.311000in}}%
\pgfusepath{clip}%
\pgfsetbuttcap%
\pgfsetroundjoin%
\definecolor{currentfill}{rgb}{0.127568,0.566949,0.550556}%
\pgfsetfillcolor{currentfill}%
\pgfsetfillopacity{0.700000}%
\pgfsetlinewidth{0.000000pt}%
\definecolor{currentstroke}{rgb}{0.000000,0.000000,0.000000}%
\pgfsetstrokecolor{currentstroke}%
\pgfsetstrokeopacity{0.700000}%
\pgfsetdash{}{0pt}%
\pgfpathmoveto{\pgfqpoint{8.681375in}{2.913417in}}%
\pgfpathcurveto{\pgfqpoint{8.686419in}{2.913417in}}{\pgfqpoint{8.691256in}{2.915421in}}{\pgfqpoint{8.694823in}{2.918987in}}%
\pgfpathcurveto{\pgfqpoint{8.698389in}{2.922553in}}{\pgfqpoint{8.700393in}{2.927391in}}{\pgfqpoint{8.700393in}{2.932435in}}%
\pgfpathcurveto{\pgfqpoint{8.700393in}{2.937479in}}{\pgfqpoint{8.698389in}{2.942316in}}{\pgfqpoint{8.694823in}{2.945883in}}%
\pgfpathcurveto{\pgfqpoint{8.691256in}{2.949449in}}{\pgfqpoint{8.686419in}{2.951453in}}{\pgfqpoint{8.681375in}{2.951453in}}%
\pgfpathcurveto{\pgfqpoint{8.676331in}{2.951453in}}{\pgfqpoint{8.671493in}{2.949449in}}{\pgfqpoint{8.667927in}{2.945883in}}%
\pgfpathcurveto{\pgfqpoint{8.664361in}{2.942316in}}{\pgfqpoint{8.662357in}{2.937479in}}{\pgfqpoint{8.662357in}{2.932435in}}%
\pgfpathcurveto{\pgfqpoint{8.662357in}{2.927391in}}{\pgfqpoint{8.664361in}{2.922553in}}{\pgfqpoint{8.667927in}{2.918987in}}%
\pgfpathcurveto{\pgfqpoint{8.671493in}{2.915421in}}{\pgfqpoint{8.676331in}{2.913417in}}{\pgfqpoint{8.681375in}{2.913417in}}%
\pgfpathclose%
\pgfusepath{fill}%
\end{pgfscope}%
\begin{pgfscope}%
\pgfpathrectangle{\pgfqpoint{6.572727in}{0.473000in}}{\pgfqpoint{4.227273in}{3.311000in}}%
\pgfusepath{clip}%
\pgfsetbuttcap%
\pgfsetroundjoin%
\definecolor{currentfill}{rgb}{0.993248,0.906157,0.143936}%
\pgfsetfillcolor{currentfill}%
\pgfsetfillopacity{0.700000}%
\pgfsetlinewidth{0.000000pt}%
\definecolor{currentstroke}{rgb}{0.000000,0.000000,0.000000}%
\pgfsetstrokecolor{currentstroke}%
\pgfsetstrokeopacity{0.700000}%
\pgfsetdash{}{0pt}%
\pgfpathmoveto{\pgfqpoint{9.373092in}{1.364002in}}%
\pgfpathcurveto{\pgfqpoint{9.378136in}{1.364002in}}{\pgfqpoint{9.382973in}{1.366006in}}{\pgfqpoint{9.386540in}{1.369573in}}%
\pgfpathcurveto{\pgfqpoint{9.390106in}{1.373139in}}{\pgfqpoint{9.392110in}{1.377977in}}{\pgfqpoint{9.392110in}{1.383020in}}%
\pgfpathcurveto{\pgfqpoint{9.392110in}{1.388064in}}{\pgfqpoint{9.390106in}{1.392902in}}{\pgfqpoint{9.386540in}{1.396468in}}%
\pgfpathcurveto{\pgfqpoint{9.382973in}{1.400035in}}{\pgfqpoint{9.378136in}{1.402039in}}{\pgfqpoint{9.373092in}{1.402039in}}%
\pgfpathcurveto{\pgfqpoint{9.368048in}{1.402039in}}{\pgfqpoint{9.363210in}{1.400035in}}{\pgfqpoint{9.359644in}{1.396468in}}%
\pgfpathcurveto{\pgfqpoint{9.356078in}{1.392902in}}{\pgfqpoint{9.354074in}{1.388064in}}{\pgfqpoint{9.354074in}{1.383020in}}%
\pgfpathcurveto{\pgfqpoint{9.354074in}{1.377977in}}{\pgfqpoint{9.356078in}{1.373139in}}{\pgfqpoint{9.359644in}{1.369573in}}%
\pgfpathcurveto{\pgfqpoint{9.363210in}{1.366006in}}{\pgfqpoint{9.368048in}{1.364002in}}{\pgfqpoint{9.373092in}{1.364002in}}%
\pgfpathclose%
\pgfusepath{fill}%
\end{pgfscope}%
\begin{pgfscope}%
\pgfpathrectangle{\pgfqpoint{6.572727in}{0.473000in}}{\pgfqpoint{4.227273in}{3.311000in}}%
\pgfusepath{clip}%
\pgfsetbuttcap%
\pgfsetroundjoin%
\definecolor{currentfill}{rgb}{0.127568,0.566949,0.550556}%
\pgfsetfillcolor{currentfill}%
\pgfsetfillopacity{0.700000}%
\pgfsetlinewidth{0.000000pt}%
\definecolor{currentstroke}{rgb}{0.000000,0.000000,0.000000}%
\pgfsetstrokecolor{currentstroke}%
\pgfsetstrokeopacity{0.700000}%
\pgfsetdash{}{0pt}%
\pgfpathmoveto{\pgfqpoint{8.099575in}{2.419491in}}%
\pgfpathcurveto{\pgfqpoint{8.104618in}{2.419491in}}{\pgfqpoint{8.109456in}{2.421495in}}{\pgfqpoint{8.113023in}{2.425061in}}%
\pgfpathcurveto{\pgfqpoint{8.116589in}{2.428628in}}{\pgfqpoint{8.118593in}{2.433466in}}{\pgfqpoint{8.118593in}{2.438509in}}%
\pgfpathcurveto{\pgfqpoint{8.118593in}{2.443553in}}{\pgfqpoint{8.116589in}{2.448391in}}{\pgfqpoint{8.113023in}{2.451957in}}%
\pgfpathcurveto{\pgfqpoint{8.109456in}{2.455524in}}{\pgfqpoint{8.104618in}{2.457527in}}{\pgfqpoint{8.099575in}{2.457527in}}%
\pgfpathcurveto{\pgfqpoint{8.094531in}{2.457527in}}{\pgfqpoint{8.089693in}{2.455524in}}{\pgfqpoint{8.086127in}{2.451957in}}%
\pgfpathcurveto{\pgfqpoint{8.082561in}{2.448391in}}{\pgfqpoint{8.080557in}{2.443553in}}{\pgfqpoint{8.080557in}{2.438509in}}%
\pgfpathcurveto{\pgfqpoint{8.080557in}{2.433466in}}{\pgfqpoint{8.082561in}{2.428628in}}{\pgfqpoint{8.086127in}{2.425061in}}%
\pgfpathcurveto{\pgfqpoint{8.089693in}{2.421495in}}{\pgfqpoint{8.094531in}{2.419491in}}{\pgfqpoint{8.099575in}{2.419491in}}%
\pgfpathclose%
\pgfusepath{fill}%
\end{pgfscope}%
\begin{pgfscope}%
\pgfpathrectangle{\pgfqpoint{6.572727in}{0.473000in}}{\pgfqpoint{4.227273in}{3.311000in}}%
\pgfusepath{clip}%
\pgfsetbuttcap%
\pgfsetroundjoin%
\definecolor{currentfill}{rgb}{0.127568,0.566949,0.550556}%
\pgfsetfillcolor{currentfill}%
\pgfsetfillopacity{0.700000}%
\pgfsetlinewidth{0.000000pt}%
\definecolor{currentstroke}{rgb}{0.000000,0.000000,0.000000}%
\pgfsetstrokecolor{currentstroke}%
\pgfsetstrokeopacity{0.700000}%
\pgfsetdash{}{0pt}%
\pgfpathmoveto{\pgfqpoint{7.860710in}{3.212577in}}%
\pgfpathcurveto{\pgfqpoint{7.865754in}{3.212577in}}{\pgfqpoint{7.870592in}{3.214581in}}{\pgfqpoint{7.874158in}{3.218148in}}%
\pgfpathcurveto{\pgfqpoint{7.877725in}{3.221714in}}{\pgfqpoint{7.879728in}{3.226552in}}{\pgfqpoint{7.879728in}{3.231595in}}%
\pgfpathcurveto{\pgfqpoint{7.879728in}{3.236639in}}{\pgfqpoint{7.877725in}{3.241477in}}{\pgfqpoint{7.874158in}{3.245043in}}%
\pgfpathcurveto{\pgfqpoint{7.870592in}{3.248610in}}{\pgfqpoint{7.865754in}{3.250614in}}{\pgfqpoint{7.860710in}{3.250614in}}%
\pgfpathcurveto{\pgfqpoint{7.855667in}{3.250614in}}{\pgfqpoint{7.850829in}{3.248610in}}{\pgfqpoint{7.847262in}{3.245043in}}%
\pgfpathcurveto{\pgfqpoint{7.843696in}{3.241477in}}{\pgfqpoint{7.841692in}{3.236639in}}{\pgfqpoint{7.841692in}{3.231595in}}%
\pgfpathcurveto{\pgfqpoint{7.841692in}{3.226552in}}{\pgfqpoint{7.843696in}{3.221714in}}{\pgfqpoint{7.847262in}{3.218148in}}%
\pgfpathcurveto{\pgfqpoint{7.850829in}{3.214581in}}{\pgfqpoint{7.855667in}{3.212577in}}{\pgfqpoint{7.860710in}{3.212577in}}%
\pgfpathclose%
\pgfusepath{fill}%
\end{pgfscope}%
\begin{pgfscope}%
\pgfpathrectangle{\pgfqpoint{6.572727in}{0.473000in}}{\pgfqpoint{4.227273in}{3.311000in}}%
\pgfusepath{clip}%
\pgfsetbuttcap%
\pgfsetroundjoin%
\definecolor{currentfill}{rgb}{0.127568,0.566949,0.550556}%
\pgfsetfillcolor{currentfill}%
\pgfsetfillopacity{0.700000}%
\pgfsetlinewidth{0.000000pt}%
\definecolor{currentstroke}{rgb}{0.000000,0.000000,0.000000}%
\pgfsetstrokecolor{currentstroke}%
\pgfsetstrokeopacity{0.700000}%
\pgfsetdash{}{0pt}%
\pgfpathmoveto{\pgfqpoint{7.780012in}{1.827604in}}%
\pgfpathcurveto{\pgfqpoint{7.785056in}{1.827604in}}{\pgfqpoint{7.789894in}{1.829608in}}{\pgfqpoint{7.793460in}{1.833174in}}%
\pgfpathcurveto{\pgfqpoint{7.797027in}{1.836741in}}{\pgfqpoint{7.799030in}{1.841579in}}{\pgfqpoint{7.799030in}{1.846622in}}%
\pgfpathcurveto{\pgfqpoint{7.799030in}{1.851666in}}{\pgfqpoint{7.797027in}{1.856504in}}{\pgfqpoint{7.793460in}{1.860070in}}%
\pgfpathcurveto{\pgfqpoint{7.789894in}{1.863637in}}{\pgfqpoint{7.785056in}{1.865640in}}{\pgfqpoint{7.780012in}{1.865640in}}%
\pgfpathcurveto{\pgfqpoint{7.774969in}{1.865640in}}{\pgfqpoint{7.770131in}{1.863637in}}{\pgfqpoint{7.766564in}{1.860070in}}%
\pgfpathcurveto{\pgfqpoint{7.762998in}{1.856504in}}{\pgfqpoint{7.760994in}{1.851666in}}{\pgfqpoint{7.760994in}{1.846622in}}%
\pgfpathcurveto{\pgfqpoint{7.760994in}{1.841579in}}{\pgfqpoint{7.762998in}{1.836741in}}{\pgfqpoint{7.766564in}{1.833174in}}%
\pgfpathcurveto{\pgfqpoint{7.770131in}{1.829608in}}{\pgfqpoint{7.774969in}{1.827604in}}{\pgfqpoint{7.780012in}{1.827604in}}%
\pgfpathclose%
\pgfusepath{fill}%
\end{pgfscope}%
\begin{pgfscope}%
\pgfpathrectangle{\pgfqpoint{6.572727in}{0.473000in}}{\pgfqpoint{4.227273in}{3.311000in}}%
\pgfusepath{clip}%
\pgfsetbuttcap%
\pgfsetroundjoin%
\definecolor{currentfill}{rgb}{0.993248,0.906157,0.143936}%
\pgfsetfillcolor{currentfill}%
\pgfsetfillopacity{0.700000}%
\pgfsetlinewidth{0.000000pt}%
\definecolor{currentstroke}{rgb}{0.000000,0.000000,0.000000}%
\pgfsetstrokecolor{currentstroke}%
\pgfsetstrokeopacity{0.700000}%
\pgfsetdash{}{0pt}%
\pgfpathmoveto{\pgfqpoint{9.546283in}{2.079513in}}%
\pgfpathcurveto{\pgfqpoint{9.551327in}{2.079513in}}{\pgfqpoint{9.556165in}{2.081517in}}{\pgfqpoint{9.559731in}{2.085083in}}%
\pgfpathcurveto{\pgfqpoint{9.563298in}{2.088650in}}{\pgfqpoint{9.565302in}{2.093487in}}{\pgfqpoint{9.565302in}{2.098531in}}%
\pgfpathcurveto{\pgfqpoint{9.565302in}{2.103575in}}{\pgfqpoint{9.563298in}{2.108413in}}{\pgfqpoint{9.559731in}{2.111979in}}%
\pgfpathcurveto{\pgfqpoint{9.556165in}{2.115545in}}{\pgfqpoint{9.551327in}{2.117549in}}{\pgfqpoint{9.546283in}{2.117549in}}%
\pgfpathcurveto{\pgfqpoint{9.541240in}{2.117549in}}{\pgfqpoint{9.536402in}{2.115545in}}{\pgfqpoint{9.532836in}{2.111979in}}%
\pgfpathcurveto{\pgfqpoint{9.529269in}{2.108413in}}{\pgfqpoint{9.527265in}{2.103575in}}{\pgfqpoint{9.527265in}{2.098531in}}%
\pgfpathcurveto{\pgfqpoint{9.527265in}{2.093487in}}{\pgfqpoint{9.529269in}{2.088650in}}{\pgfqpoint{9.532836in}{2.085083in}}%
\pgfpathcurveto{\pgfqpoint{9.536402in}{2.081517in}}{\pgfqpoint{9.541240in}{2.079513in}}{\pgfqpoint{9.546283in}{2.079513in}}%
\pgfpathclose%
\pgfusepath{fill}%
\end{pgfscope}%
\begin{pgfscope}%
\pgfpathrectangle{\pgfqpoint{6.572727in}{0.473000in}}{\pgfqpoint{4.227273in}{3.311000in}}%
\pgfusepath{clip}%
\pgfsetbuttcap%
\pgfsetroundjoin%
\definecolor{currentfill}{rgb}{0.127568,0.566949,0.550556}%
\pgfsetfillcolor{currentfill}%
\pgfsetfillopacity{0.700000}%
\pgfsetlinewidth{0.000000pt}%
\definecolor{currentstroke}{rgb}{0.000000,0.000000,0.000000}%
\pgfsetstrokecolor{currentstroke}%
\pgfsetstrokeopacity{0.700000}%
\pgfsetdash{}{0pt}%
\pgfpathmoveto{\pgfqpoint{8.109146in}{2.727296in}}%
\pgfpathcurveto{\pgfqpoint{8.114190in}{2.727296in}}{\pgfqpoint{8.119028in}{2.729300in}}{\pgfqpoint{8.122594in}{2.732866in}}%
\pgfpathcurveto{\pgfqpoint{8.126161in}{2.736433in}}{\pgfqpoint{8.128164in}{2.741271in}}{\pgfqpoint{8.128164in}{2.746314in}}%
\pgfpathcurveto{\pgfqpoint{8.128164in}{2.751358in}}{\pgfqpoint{8.126161in}{2.756196in}}{\pgfqpoint{8.122594in}{2.759762in}}%
\pgfpathcurveto{\pgfqpoint{8.119028in}{2.763329in}}{\pgfqpoint{8.114190in}{2.765332in}}{\pgfqpoint{8.109146in}{2.765332in}}%
\pgfpathcurveto{\pgfqpoint{8.104103in}{2.765332in}}{\pgfqpoint{8.099265in}{2.763329in}}{\pgfqpoint{8.095698in}{2.759762in}}%
\pgfpathcurveto{\pgfqpoint{8.092132in}{2.756196in}}{\pgfqpoint{8.090128in}{2.751358in}}{\pgfqpoint{8.090128in}{2.746314in}}%
\pgfpathcurveto{\pgfqpoint{8.090128in}{2.741271in}}{\pgfqpoint{8.092132in}{2.736433in}}{\pgfqpoint{8.095698in}{2.732866in}}%
\pgfpathcurveto{\pgfqpoint{8.099265in}{2.729300in}}{\pgfqpoint{8.104103in}{2.727296in}}{\pgfqpoint{8.109146in}{2.727296in}}%
\pgfpathclose%
\pgfusepath{fill}%
\end{pgfscope}%
\begin{pgfscope}%
\pgfpathrectangle{\pgfqpoint{6.572727in}{0.473000in}}{\pgfqpoint{4.227273in}{3.311000in}}%
\pgfusepath{clip}%
\pgfsetbuttcap%
\pgfsetroundjoin%
\definecolor{currentfill}{rgb}{0.127568,0.566949,0.550556}%
\pgfsetfillcolor{currentfill}%
\pgfsetfillopacity{0.700000}%
\pgfsetlinewidth{0.000000pt}%
\definecolor{currentstroke}{rgb}{0.000000,0.000000,0.000000}%
\pgfsetstrokecolor{currentstroke}%
\pgfsetstrokeopacity{0.700000}%
\pgfsetdash{}{0pt}%
\pgfpathmoveto{\pgfqpoint{7.675055in}{1.434021in}}%
\pgfpathcurveto{\pgfqpoint{7.680099in}{1.434021in}}{\pgfqpoint{7.684937in}{1.436025in}}{\pgfqpoint{7.688503in}{1.439592in}}%
\pgfpathcurveto{\pgfqpoint{7.692069in}{1.443158in}}{\pgfqpoint{7.694073in}{1.447996in}}{\pgfqpoint{7.694073in}{1.453039in}}%
\pgfpathcurveto{\pgfqpoint{7.694073in}{1.458083in}}{\pgfqpoint{7.692069in}{1.462921in}}{\pgfqpoint{7.688503in}{1.466487in}}%
\pgfpathcurveto{\pgfqpoint{7.684937in}{1.470054in}}{\pgfqpoint{7.680099in}{1.472058in}}{\pgfqpoint{7.675055in}{1.472058in}}%
\pgfpathcurveto{\pgfqpoint{7.670012in}{1.472058in}}{\pgfqpoint{7.665174in}{1.470054in}}{\pgfqpoint{7.661607in}{1.466487in}}%
\pgfpathcurveto{\pgfqpoint{7.658041in}{1.462921in}}{\pgfqpoint{7.656037in}{1.458083in}}{\pgfqpoint{7.656037in}{1.453039in}}%
\pgfpathcurveto{\pgfqpoint{7.656037in}{1.447996in}}{\pgfqpoint{7.658041in}{1.443158in}}{\pgfqpoint{7.661607in}{1.439592in}}%
\pgfpathcurveto{\pgfqpoint{7.665174in}{1.436025in}}{\pgfqpoint{7.670012in}{1.434021in}}{\pgfqpoint{7.675055in}{1.434021in}}%
\pgfpathclose%
\pgfusepath{fill}%
\end{pgfscope}%
\begin{pgfscope}%
\pgfpathrectangle{\pgfqpoint{6.572727in}{0.473000in}}{\pgfqpoint{4.227273in}{3.311000in}}%
\pgfusepath{clip}%
\pgfsetbuttcap%
\pgfsetroundjoin%
\definecolor{currentfill}{rgb}{0.993248,0.906157,0.143936}%
\pgfsetfillcolor{currentfill}%
\pgfsetfillopacity{0.700000}%
\pgfsetlinewidth{0.000000pt}%
\definecolor{currentstroke}{rgb}{0.000000,0.000000,0.000000}%
\pgfsetstrokecolor{currentstroke}%
\pgfsetstrokeopacity{0.700000}%
\pgfsetdash{}{0pt}%
\pgfpathmoveto{\pgfqpoint{9.871804in}{1.990124in}}%
\pgfpathcurveto{\pgfqpoint{9.876847in}{1.990124in}}{\pgfqpoint{9.881685in}{1.992128in}}{\pgfqpoint{9.885252in}{1.995694in}}%
\pgfpathcurveto{\pgfqpoint{9.888818in}{1.999261in}}{\pgfqpoint{9.890822in}{2.004098in}}{\pgfqpoint{9.890822in}{2.009142in}}%
\pgfpathcurveto{\pgfqpoint{9.890822in}{2.014186in}}{\pgfqpoint{9.888818in}{2.019023in}}{\pgfqpoint{9.885252in}{2.022590in}}%
\pgfpathcurveto{\pgfqpoint{9.881685in}{2.026156in}}{\pgfqpoint{9.876847in}{2.028160in}}{\pgfqpoint{9.871804in}{2.028160in}}%
\pgfpathcurveto{\pgfqpoint{9.866760in}{2.028160in}}{\pgfqpoint{9.861922in}{2.026156in}}{\pgfqpoint{9.858356in}{2.022590in}}%
\pgfpathcurveto{\pgfqpoint{9.854790in}{2.019023in}}{\pgfqpoint{9.852786in}{2.014186in}}{\pgfqpoint{9.852786in}{2.009142in}}%
\pgfpathcurveto{\pgfqpoint{9.852786in}{2.004098in}}{\pgfqpoint{9.854790in}{1.999261in}}{\pgfqpoint{9.858356in}{1.995694in}}%
\pgfpathcurveto{\pgfqpoint{9.861922in}{1.992128in}}{\pgfqpoint{9.866760in}{1.990124in}}{\pgfqpoint{9.871804in}{1.990124in}}%
\pgfpathclose%
\pgfusepath{fill}%
\end{pgfscope}%
\begin{pgfscope}%
\pgfpathrectangle{\pgfqpoint{6.572727in}{0.473000in}}{\pgfqpoint{4.227273in}{3.311000in}}%
\pgfusepath{clip}%
\pgfsetbuttcap%
\pgfsetroundjoin%
\definecolor{currentfill}{rgb}{0.127568,0.566949,0.550556}%
\pgfsetfillcolor{currentfill}%
\pgfsetfillopacity{0.700000}%
\pgfsetlinewidth{0.000000pt}%
\definecolor{currentstroke}{rgb}{0.000000,0.000000,0.000000}%
\pgfsetstrokecolor{currentstroke}%
\pgfsetstrokeopacity{0.700000}%
\pgfsetdash{}{0pt}%
\pgfpathmoveto{\pgfqpoint{8.783733in}{3.092608in}}%
\pgfpathcurveto{\pgfqpoint{8.788777in}{3.092608in}}{\pgfqpoint{8.793615in}{3.094612in}}{\pgfqpoint{8.797181in}{3.098178in}}%
\pgfpathcurveto{\pgfqpoint{8.800747in}{3.101745in}}{\pgfqpoint{8.802751in}{3.106582in}}{\pgfqpoint{8.802751in}{3.111626in}}%
\pgfpathcurveto{\pgfqpoint{8.802751in}{3.116670in}}{\pgfqpoint{8.800747in}{3.121507in}}{\pgfqpoint{8.797181in}{3.125074in}}%
\pgfpathcurveto{\pgfqpoint{8.793615in}{3.128640in}}{\pgfqpoint{8.788777in}{3.130644in}}{\pgfqpoint{8.783733in}{3.130644in}}%
\pgfpathcurveto{\pgfqpoint{8.778690in}{3.130644in}}{\pgfqpoint{8.773852in}{3.128640in}}{\pgfqpoint{8.770285in}{3.125074in}}%
\pgfpathcurveto{\pgfqpoint{8.766719in}{3.121507in}}{\pgfqpoint{8.764715in}{3.116670in}}{\pgfqpoint{8.764715in}{3.111626in}}%
\pgfpathcurveto{\pgfqpoint{8.764715in}{3.106582in}}{\pgfqpoint{8.766719in}{3.101745in}}{\pgfqpoint{8.770285in}{3.098178in}}%
\pgfpathcurveto{\pgfqpoint{8.773852in}{3.094612in}}{\pgfqpoint{8.778690in}{3.092608in}}{\pgfqpoint{8.783733in}{3.092608in}}%
\pgfpathclose%
\pgfusepath{fill}%
\end{pgfscope}%
\begin{pgfscope}%
\pgfpathrectangle{\pgfqpoint{6.572727in}{0.473000in}}{\pgfqpoint{4.227273in}{3.311000in}}%
\pgfusepath{clip}%
\pgfsetbuttcap%
\pgfsetroundjoin%
\definecolor{currentfill}{rgb}{0.127568,0.566949,0.550556}%
\pgfsetfillcolor{currentfill}%
\pgfsetfillopacity{0.700000}%
\pgfsetlinewidth{0.000000pt}%
\definecolor{currentstroke}{rgb}{0.000000,0.000000,0.000000}%
\pgfsetstrokecolor{currentstroke}%
\pgfsetstrokeopacity{0.700000}%
\pgfsetdash{}{0pt}%
\pgfpathmoveto{\pgfqpoint{7.678613in}{2.828546in}}%
\pgfpathcurveto{\pgfqpoint{7.683657in}{2.828546in}}{\pgfqpoint{7.688494in}{2.830550in}}{\pgfqpoint{7.692061in}{2.834117in}}%
\pgfpathcurveto{\pgfqpoint{7.695627in}{2.837683in}}{\pgfqpoint{7.697631in}{2.842521in}}{\pgfqpoint{7.697631in}{2.847565in}}%
\pgfpathcurveto{\pgfqpoint{7.697631in}{2.852608in}}{\pgfqpoint{7.695627in}{2.857446in}}{\pgfqpoint{7.692061in}{2.861012in}}%
\pgfpathcurveto{\pgfqpoint{7.688494in}{2.864579in}}{\pgfqpoint{7.683657in}{2.866583in}}{\pgfqpoint{7.678613in}{2.866583in}}%
\pgfpathcurveto{\pgfqpoint{7.673569in}{2.866583in}}{\pgfqpoint{7.668731in}{2.864579in}}{\pgfqpoint{7.665165in}{2.861012in}}%
\pgfpathcurveto{\pgfqpoint{7.661599in}{2.857446in}}{\pgfqpoint{7.659595in}{2.852608in}}{\pgfqpoint{7.659595in}{2.847565in}}%
\pgfpathcurveto{\pgfqpoint{7.659595in}{2.842521in}}{\pgfqpoint{7.661599in}{2.837683in}}{\pgfqpoint{7.665165in}{2.834117in}}%
\pgfpathcurveto{\pgfqpoint{7.668731in}{2.830550in}}{\pgfqpoint{7.673569in}{2.828546in}}{\pgfqpoint{7.678613in}{2.828546in}}%
\pgfpathclose%
\pgfusepath{fill}%
\end{pgfscope}%
\begin{pgfscope}%
\pgfpathrectangle{\pgfqpoint{6.572727in}{0.473000in}}{\pgfqpoint{4.227273in}{3.311000in}}%
\pgfusepath{clip}%
\pgfsetbuttcap%
\pgfsetroundjoin%
\definecolor{currentfill}{rgb}{0.993248,0.906157,0.143936}%
\pgfsetfillcolor{currentfill}%
\pgfsetfillopacity{0.700000}%
\pgfsetlinewidth{0.000000pt}%
\definecolor{currentstroke}{rgb}{0.000000,0.000000,0.000000}%
\pgfsetstrokecolor{currentstroke}%
\pgfsetstrokeopacity{0.700000}%
\pgfsetdash{}{0pt}%
\pgfpathmoveto{\pgfqpoint{9.317019in}{1.610455in}}%
\pgfpathcurveto{\pgfqpoint{9.322062in}{1.610455in}}{\pgfqpoint{9.326900in}{1.612459in}}{\pgfqpoint{9.330466in}{1.616025in}}%
\pgfpathcurveto{\pgfqpoint{9.334033in}{1.619592in}}{\pgfqpoint{9.336037in}{1.624430in}}{\pgfqpoint{9.336037in}{1.629473in}}%
\pgfpathcurveto{\pgfqpoint{9.336037in}{1.634517in}}{\pgfqpoint{9.334033in}{1.639355in}}{\pgfqpoint{9.330466in}{1.642921in}}%
\pgfpathcurveto{\pgfqpoint{9.326900in}{1.646488in}}{\pgfqpoint{9.322062in}{1.648491in}}{\pgfqpoint{9.317019in}{1.648491in}}%
\pgfpathcurveto{\pgfqpoint{9.311975in}{1.648491in}}{\pgfqpoint{9.307137in}{1.646488in}}{\pgfqpoint{9.303571in}{1.642921in}}%
\pgfpathcurveto{\pgfqpoint{9.300004in}{1.639355in}}{\pgfqpoint{9.298000in}{1.634517in}}{\pgfqpoint{9.298000in}{1.629473in}}%
\pgfpathcurveto{\pgfqpoint{9.298000in}{1.624430in}}{\pgfqpoint{9.300004in}{1.619592in}}{\pgfqpoint{9.303571in}{1.616025in}}%
\pgfpathcurveto{\pgfqpoint{9.307137in}{1.612459in}}{\pgfqpoint{9.311975in}{1.610455in}}{\pgfqpoint{9.317019in}{1.610455in}}%
\pgfpathclose%
\pgfusepath{fill}%
\end{pgfscope}%
\begin{pgfscope}%
\pgfpathrectangle{\pgfqpoint{6.572727in}{0.473000in}}{\pgfqpoint{4.227273in}{3.311000in}}%
\pgfusepath{clip}%
\pgfsetbuttcap%
\pgfsetroundjoin%
\definecolor{currentfill}{rgb}{0.127568,0.566949,0.550556}%
\pgfsetfillcolor{currentfill}%
\pgfsetfillopacity{0.700000}%
\pgfsetlinewidth{0.000000pt}%
\definecolor{currentstroke}{rgb}{0.000000,0.000000,0.000000}%
\pgfsetstrokecolor{currentstroke}%
\pgfsetstrokeopacity{0.700000}%
\pgfsetdash{}{0pt}%
\pgfpathmoveto{\pgfqpoint{8.354403in}{3.251132in}}%
\pgfpathcurveto{\pgfqpoint{8.359447in}{3.251132in}}{\pgfqpoint{8.364285in}{3.253136in}}{\pgfqpoint{8.367851in}{3.256702in}}%
\pgfpathcurveto{\pgfqpoint{8.371417in}{3.260268in}}{\pgfqpoint{8.373421in}{3.265106in}}{\pgfqpoint{8.373421in}{3.270150in}}%
\pgfpathcurveto{\pgfqpoint{8.373421in}{3.275194in}}{\pgfqpoint{8.371417in}{3.280031in}}{\pgfqpoint{8.367851in}{3.283598in}}%
\pgfpathcurveto{\pgfqpoint{8.364285in}{3.287164in}}{\pgfqpoint{8.359447in}{3.289168in}}{\pgfqpoint{8.354403in}{3.289168in}}%
\pgfpathcurveto{\pgfqpoint{8.349359in}{3.289168in}}{\pgfqpoint{8.344522in}{3.287164in}}{\pgfqpoint{8.340955in}{3.283598in}}%
\pgfpathcurveto{\pgfqpoint{8.337389in}{3.280031in}}{\pgfqpoint{8.335385in}{3.275194in}}{\pgfqpoint{8.335385in}{3.270150in}}%
\pgfpathcurveto{\pgfqpoint{8.335385in}{3.265106in}}{\pgfqpoint{8.337389in}{3.260268in}}{\pgfqpoint{8.340955in}{3.256702in}}%
\pgfpathcurveto{\pgfqpoint{8.344522in}{3.253136in}}{\pgfqpoint{8.349359in}{3.251132in}}{\pgfqpoint{8.354403in}{3.251132in}}%
\pgfpathclose%
\pgfusepath{fill}%
\end{pgfscope}%
\begin{pgfscope}%
\pgfpathrectangle{\pgfqpoint{6.572727in}{0.473000in}}{\pgfqpoint{4.227273in}{3.311000in}}%
\pgfusepath{clip}%
\pgfsetbuttcap%
\pgfsetroundjoin%
\definecolor{currentfill}{rgb}{0.993248,0.906157,0.143936}%
\pgfsetfillcolor{currentfill}%
\pgfsetfillopacity{0.700000}%
\pgfsetlinewidth{0.000000pt}%
\definecolor{currentstroke}{rgb}{0.000000,0.000000,0.000000}%
\pgfsetstrokecolor{currentstroke}%
\pgfsetstrokeopacity{0.700000}%
\pgfsetdash{}{0pt}%
\pgfpathmoveto{\pgfqpoint{9.600668in}{1.645832in}}%
\pgfpathcurveto{\pgfqpoint{9.605712in}{1.645832in}}{\pgfqpoint{9.610550in}{1.647836in}}{\pgfqpoint{9.614116in}{1.651402in}}%
\pgfpathcurveto{\pgfqpoint{9.617682in}{1.654968in}}{\pgfqpoint{9.619686in}{1.659806in}}{\pgfqpoint{9.619686in}{1.664850in}}%
\pgfpathcurveto{\pgfqpoint{9.619686in}{1.669894in}}{\pgfqpoint{9.617682in}{1.674731in}}{\pgfqpoint{9.614116in}{1.678298in}}%
\pgfpathcurveto{\pgfqpoint{9.610550in}{1.681864in}}{\pgfqpoint{9.605712in}{1.683868in}}{\pgfqpoint{9.600668in}{1.683868in}}%
\pgfpathcurveto{\pgfqpoint{9.595624in}{1.683868in}}{\pgfqpoint{9.590787in}{1.681864in}}{\pgfqpoint{9.587220in}{1.678298in}}%
\pgfpathcurveto{\pgfqpoint{9.583654in}{1.674731in}}{\pgfqpoint{9.581650in}{1.669894in}}{\pgfqpoint{9.581650in}{1.664850in}}%
\pgfpathcurveto{\pgfqpoint{9.581650in}{1.659806in}}{\pgfqpoint{9.583654in}{1.654968in}}{\pgfqpoint{9.587220in}{1.651402in}}%
\pgfpathcurveto{\pgfqpoint{9.590787in}{1.647836in}}{\pgfqpoint{9.595624in}{1.645832in}}{\pgfqpoint{9.600668in}{1.645832in}}%
\pgfpathclose%
\pgfusepath{fill}%
\end{pgfscope}%
\begin{pgfscope}%
\pgfpathrectangle{\pgfqpoint{6.572727in}{0.473000in}}{\pgfqpoint{4.227273in}{3.311000in}}%
\pgfusepath{clip}%
\pgfsetbuttcap%
\pgfsetroundjoin%
\definecolor{currentfill}{rgb}{0.127568,0.566949,0.550556}%
\pgfsetfillcolor{currentfill}%
\pgfsetfillopacity{0.700000}%
\pgfsetlinewidth{0.000000pt}%
\definecolor{currentstroke}{rgb}{0.000000,0.000000,0.000000}%
\pgfsetstrokecolor{currentstroke}%
\pgfsetstrokeopacity{0.700000}%
\pgfsetdash{}{0pt}%
\pgfpathmoveto{\pgfqpoint{8.470465in}{3.115911in}}%
\pgfpathcurveto{\pgfqpoint{8.475509in}{3.115911in}}{\pgfqpoint{8.480347in}{3.117915in}}{\pgfqpoint{8.483913in}{3.121482in}}%
\pgfpathcurveto{\pgfqpoint{8.487480in}{3.125048in}}{\pgfqpoint{8.489483in}{3.129886in}}{\pgfqpoint{8.489483in}{3.134929in}}%
\pgfpathcurveto{\pgfqpoint{8.489483in}{3.139973in}}{\pgfqpoint{8.487480in}{3.144811in}}{\pgfqpoint{8.483913in}{3.148377in}}%
\pgfpathcurveto{\pgfqpoint{8.480347in}{3.151944in}}{\pgfqpoint{8.475509in}{3.153948in}}{\pgfqpoint{8.470465in}{3.153948in}}%
\pgfpathcurveto{\pgfqpoint{8.465422in}{3.153948in}}{\pgfqpoint{8.460584in}{3.151944in}}{\pgfqpoint{8.457017in}{3.148377in}}%
\pgfpathcurveto{\pgfqpoint{8.453451in}{3.144811in}}{\pgfqpoint{8.451447in}{3.139973in}}{\pgfqpoint{8.451447in}{3.134929in}}%
\pgfpathcurveto{\pgfqpoint{8.451447in}{3.129886in}}{\pgfqpoint{8.453451in}{3.125048in}}{\pgfqpoint{8.457017in}{3.121482in}}%
\pgfpathcurveto{\pgfqpoint{8.460584in}{3.117915in}}{\pgfqpoint{8.465422in}{3.115911in}}{\pgfqpoint{8.470465in}{3.115911in}}%
\pgfpathclose%
\pgfusepath{fill}%
\end{pgfscope}%
\begin{pgfscope}%
\pgfpathrectangle{\pgfqpoint{6.572727in}{0.473000in}}{\pgfqpoint{4.227273in}{3.311000in}}%
\pgfusepath{clip}%
\pgfsetbuttcap%
\pgfsetroundjoin%
\definecolor{currentfill}{rgb}{0.993248,0.906157,0.143936}%
\pgfsetfillcolor{currentfill}%
\pgfsetfillopacity{0.700000}%
\pgfsetlinewidth{0.000000pt}%
\definecolor{currentstroke}{rgb}{0.000000,0.000000,0.000000}%
\pgfsetstrokecolor{currentstroke}%
\pgfsetstrokeopacity{0.700000}%
\pgfsetdash{}{0pt}%
\pgfpathmoveto{\pgfqpoint{9.776351in}{1.137877in}}%
\pgfpathcurveto{\pgfqpoint{9.781395in}{1.137877in}}{\pgfqpoint{9.786233in}{1.139881in}}{\pgfqpoint{9.789799in}{1.143447in}}%
\pgfpathcurveto{\pgfqpoint{9.793365in}{1.147013in}}{\pgfqpoint{9.795369in}{1.151851in}}{\pgfqpoint{9.795369in}{1.156895in}}%
\pgfpathcurveto{\pgfqpoint{9.795369in}{1.161939in}}{\pgfqpoint{9.793365in}{1.166776in}}{\pgfqpoint{9.789799in}{1.170343in}}%
\pgfpathcurveto{\pgfqpoint{9.786233in}{1.173909in}}{\pgfqpoint{9.781395in}{1.175913in}}{\pgfqpoint{9.776351in}{1.175913in}}%
\pgfpathcurveto{\pgfqpoint{9.771307in}{1.175913in}}{\pgfqpoint{9.766470in}{1.173909in}}{\pgfqpoint{9.762903in}{1.170343in}}%
\pgfpathcurveto{\pgfqpoint{9.759337in}{1.166776in}}{\pgfqpoint{9.757333in}{1.161939in}}{\pgfqpoint{9.757333in}{1.156895in}}%
\pgfpathcurveto{\pgfqpoint{9.757333in}{1.151851in}}{\pgfqpoint{9.759337in}{1.147013in}}{\pgfqpoint{9.762903in}{1.143447in}}%
\pgfpathcurveto{\pgfqpoint{9.766470in}{1.139881in}}{\pgfqpoint{9.771307in}{1.137877in}}{\pgfqpoint{9.776351in}{1.137877in}}%
\pgfpathclose%
\pgfusepath{fill}%
\end{pgfscope}%
\begin{pgfscope}%
\pgfpathrectangle{\pgfqpoint{6.572727in}{0.473000in}}{\pgfqpoint{4.227273in}{3.311000in}}%
\pgfusepath{clip}%
\pgfsetbuttcap%
\pgfsetroundjoin%
\definecolor{currentfill}{rgb}{0.993248,0.906157,0.143936}%
\pgfsetfillcolor{currentfill}%
\pgfsetfillopacity{0.700000}%
\pgfsetlinewidth{0.000000pt}%
\definecolor{currentstroke}{rgb}{0.000000,0.000000,0.000000}%
\pgfsetstrokecolor{currentstroke}%
\pgfsetstrokeopacity{0.700000}%
\pgfsetdash{}{0pt}%
\pgfpathmoveto{\pgfqpoint{9.637120in}{1.762419in}}%
\pgfpathcurveto{\pgfqpoint{9.642163in}{1.762419in}}{\pgfqpoint{9.647001in}{1.764423in}}{\pgfqpoint{9.650568in}{1.767989in}}%
\pgfpathcurveto{\pgfqpoint{9.654134in}{1.771555in}}{\pgfqpoint{9.656138in}{1.776393in}}{\pgfqpoint{9.656138in}{1.781437in}}%
\pgfpathcurveto{\pgfqpoint{9.656138in}{1.786481in}}{\pgfqpoint{9.654134in}{1.791318in}}{\pgfqpoint{9.650568in}{1.794885in}}%
\pgfpathcurveto{\pgfqpoint{9.647001in}{1.798451in}}{\pgfqpoint{9.642163in}{1.800455in}}{\pgfqpoint{9.637120in}{1.800455in}}%
\pgfpathcurveto{\pgfqpoint{9.632076in}{1.800455in}}{\pgfqpoint{9.627238in}{1.798451in}}{\pgfqpoint{9.623672in}{1.794885in}}%
\pgfpathcurveto{\pgfqpoint{9.620105in}{1.791318in}}{\pgfqpoint{9.618102in}{1.786481in}}{\pgfqpoint{9.618102in}{1.781437in}}%
\pgfpathcurveto{\pgfqpoint{9.618102in}{1.776393in}}{\pgfqpoint{9.620105in}{1.771555in}}{\pgfqpoint{9.623672in}{1.767989in}}%
\pgfpathcurveto{\pgfqpoint{9.627238in}{1.764423in}}{\pgfqpoint{9.632076in}{1.762419in}}{\pgfqpoint{9.637120in}{1.762419in}}%
\pgfpathclose%
\pgfusepath{fill}%
\end{pgfscope}%
\begin{pgfscope}%
\pgfpathrectangle{\pgfqpoint{6.572727in}{0.473000in}}{\pgfqpoint{4.227273in}{3.311000in}}%
\pgfusepath{clip}%
\pgfsetbuttcap%
\pgfsetroundjoin%
\definecolor{currentfill}{rgb}{0.127568,0.566949,0.550556}%
\pgfsetfillcolor{currentfill}%
\pgfsetfillopacity{0.700000}%
\pgfsetlinewidth{0.000000pt}%
\definecolor{currentstroke}{rgb}{0.000000,0.000000,0.000000}%
\pgfsetstrokecolor{currentstroke}%
\pgfsetstrokeopacity{0.700000}%
\pgfsetdash{}{0pt}%
\pgfpathmoveto{\pgfqpoint{7.540105in}{1.662185in}}%
\pgfpathcurveto{\pgfqpoint{7.545149in}{1.662185in}}{\pgfqpoint{7.549986in}{1.664189in}}{\pgfqpoint{7.553553in}{1.667756in}}%
\pgfpathcurveto{\pgfqpoint{7.557119in}{1.671322in}}{\pgfqpoint{7.559123in}{1.676160in}}{\pgfqpoint{7.559123in}{1.681203in}}%
\pgfpathcurveto{\pgfqpoint{7.559123in}{1.686247in}}{\pgfqpoint{7.557119in}{1.691085in}}{\pgfqpoint{7.553553in}{1.694651in}}%
\pgfpathcurveto{\pgfqpoint{7.549986in}{1.698218in}}{\pgfqpoint{7.545149in}{1.700222in}}{\pgfqpoint{7.540105in}{1.700222in}}%
\pgfpathcurveto{\pgfqpoint{7.535061in}{1.700222in}}{\pgfqpoint{7.530224in}{1.698218in}}{\pgfqpoint{7.526657in}{1.694651in}}%
\pgfpathcurveto{\pgfqpoint{7.523091in}{1.691085in}}{\pgfqpoint{7.521087in}{1.686247in}}{\pgfqpoint{7.521087in}{1.681203in}}%
\pgfpathcurveto{\pgfqpoint{7.521087in}{1.676160in}}{\pgfqpoint{7.523091in}{1.671322in}}{\pgfqpoint{7.526657in}{1.667756in}}%
\pgfpathcurveto{\pgfqpoint{7.530224in}{1.664189in}}{\pgfqpoint{7.535061in}{1.662185in}}{\pgfqpoint{7.540105in}{1.662185in}}%
\pgfpathclose%
\pgfusepath{fill}%
\end{pgfscope}%
\begin{pgfscope}%
\pgfpathrectangle{\pgfqpoint{6.572727in}{0.473000in}}{\pgfqpoint{4.227273in}{3.311000in}}%
\pgfusepath{clip}%
\pgfsetbuttcap%
\pgfsetroundjoin%
\definecolor{currentfill}{rgb}{0.127568,0.566949,0.550556}%
\pgfsetfillcolor{currentfill}%
\pgfsetfillopacity{0.700000}%
\pgfsetlinewidth{0.000000pt}%
\definecolor{currentstroke}{rgb}{0.000000,0.000000,0.000000}%
\pgfsetstrokecolor{currentstroke}%
\pgfsetstrokeopacity{0.700000}%
\pgfsetdash{}{0pt}%
\pgfpathmoveto{\pgfqpoint{7.233386in}{0.958070in}}%
\pgfpathcurveto{\pgfqpoint{7.238430in}{0.958070in}}{\pgfqpoint{7.243268in}{0.960073in}}{\pgfqpoint{7.246834in}{0.963640in}}%
\pgfpathcurveto{\pgfqpoint{7.250401in}{0.967206in}}{\pgfqpoint{7.252405in}{0.972044in}}{\pgfqpoint{7.252405in}{0.977088in}}%
\pgfpathcurveto{\pgfqpoint{7.252405in}{0.982131in}}{\pgfqpoint{7.250401in}{0.986969in}}{\pgfqpoint{7.246834in}{0.990536in}}%
\pgfpathcurveto{\pgfqpoint{7.243268in}{0.994102in}}{\pgfqpoint{7.238430in}{0.996106in}}{\pgfqpoint{7.233386in}{0.996106in}}%
\pgfpathcurveto{\pgfqpoint{7.228343in}{0.996106in}}{\pgfqpoint{7.223505in}{0.994102in}}{\pgfqpoint{7.219939in}{0.990536in}}%
\pgfpathcurveto{\pgfqpoint{7.216372in}{0.986969in}}{\pgfqpoint{7.214368in}{0.982131in}}{\pgfqpoint{7.214368in}{0.977088in}}%
\pgfpathcurveto{\pgfqpoint{7.214368in}{0.972044in}}{\pgfqpoint{7.216372in}{0.967206in}}{\pgfqpoint{7.219939in}{0.963640in}}%
\pgfpathcurveto{\pgfqpoint{7.223505in}{0.960073in}}{\pgfqpoint{7.228343in}{0.958070in}}{\pgfqpoint{7.233386in}{0.958070in}}%
\pgfpathclose%
\pgfusepath{fill}%
\end{pgfscope}%
\begin{pgfscope}%
\pgfpathrectangle{\pgfqpoint{6.572727in}{0.473000in}}{\pgfqpoint{4.227273in}{3.311000in}}%
\pgfusepath{clip}%
\pgfsetbuttcap%
\pgfsetroundjoin%
\definecolor{currentfill}{rgb}{0.127568,0.566949,0.550556}%
\pgfsetfillcolor{currentfill}%
\pgfsetfillopacity{0.700000}%
\pgfsetlinewidth{0.000000pt}%
\definecolor{currentstroke}{rgb}{0.000000,0.000000,0.000000}%
\pgfsetstrokecolor{currentstroke}%
\pgfsetstrokeopacity{0.700000}%
\pgfsetdash{}{0pt}%
\pgfpathmoveto{\pgfqpoint{8.024822in}{2.434052in}}%
\pgfpathcurveto{\pgfqpoint{8.029866in}{2.434052in}}{\pgfqpoint{8.034704in}{2.436056in}}{\pgfqpoint{8.038270in}{2.439622in}}%
\pgfpathcurveto{\pgfqpoint{8.041837in}{2.443189in}}{\pgfqpoint{8.043840in}{2.448027in}}{\pgfqpoint{8.043840in}{2.453070in}}%
\pgfpathcurveto{\pgfqpoint{8.043840in}{2.458114in}}{\pgfqpoint{8.041837in}{2.462952in}}{\pgfqpoint{8.038270in}{2.466518in}}%
\pgfpathcurveto{\pgfqpoint{8.034704in}{2.470085in}}{\pgfqpoint{8.029866in}{2.472088in}}{\pgfqpoint{8.024822in}{2.472088in}}%
\pgfpathcurveto{\pgfqpoint{8.019779in}{2.472088in}}{\pgfqpoint{8.014941in}{2.470085in}}{\pgfqpoint{8.011374in}{2.466518in}}%
\pgfpathcurveto{\pgfqpoint{8.007808in}{2.462952in}}{\pgfqpoint{8.005804in}{2.458114in}}{\pgfqpoint{8.005804in}{2.453070in}}%
\pgfpathcurveto{\pgfqpoint{8.005804in}{2.448027in}}{\pgfqpoint{8.007808in}{2.443189in}}{\pgfqpoint{8.011374in}{2.439622in}}%
\pgfpathcurveto{\pgfqpoint{8.014941in}{2.436056in}}{\pgfqpoint{8.019779in}{2.434052in}}{\pgfqpoint{8.024822in}{2.434052in}}%
\pgfpathclose%
\pgfusepath{fill}%
\end{pgfscope}%
\begin{pgfscope}%
\pgfpathrectangle{\pgfqpoint{6.572727in}{0.473000in}}{\pgfqpoint{4.227273in}{3.311000in}}%
\pgfusepath{clip}%
\pgfsetbuttcap%
\pgfsetroundjoin%
\definecolor{currentfill}{rgb}{0.127568,0.566949,0.550556}%
\pgfsetfillcolor{currentfill}%
\pgfsetfillopacity{0.700000}%
\pgfsetlinewidth{0.000000pt}%
\definecolor{currentstroke}{rgb}{0.000000,0.000000,0.000000}%
\pgfsetstrokecolor{currentstroke}%
\pgfsetstrokeopacity{0.700000}%
\pgfsetdash{}{0pt}%
\pgfpathmoveto{\pgfqpoint{8.562464in}{2.559086in}}%
\pgfpathcurveto{\pgfqpoint{8.567508in}{2.559086in}}{\pgfqpoint{8.572345in}{2.561090in}}{\pgfqpoint{8.575912in}{2.564657in}}%
\pgfpathcurveto{\pgfqpoint{8.579478in}{2.568223in}}{\pgfqpoint{8.581482in}{2.573061in}}{\pgfqpoint{8.581482in}{2.578104in}}%
\pgfpathcurveto{\pgfqpoint{8.581482in}{2.583148in}}{\pgfqpoint{8.579478in}{2.587986in}}{\pgfqpoint{8.575912in}{2.591552in}}%
\pgfpathcurveto{\pgfqpoint{8.572345in}{2.595119in}}{\pgfqpoint{8.567508in}{2.597123in}}{\pgfqpoint{8.562464in}{2.597123in}}%
\pgfpathcurveto{\pgfqpoint{8.557420in}{2.597123in}}{\pgfqpoint{8.552582in}{2.595119in}}{\pgfqpoint{8.549016in}{2.591552in}}%
\pgfpathcurveto{\pgfqpoint{8.545450in}{2.587986in}}{\pgfqpoint{8.543446in}{2.583148in}}{\pgfqpoint{8.543446in}{2.578104in}}%
\pgfpathcurveto{\pgfqpoint{8.543446in}{2.573061in}}{\pgfqpoint{8.545450in}{2.568223in}}{\pgfqpoint{8.549016in}{2.564657in}}%
\pgfpathcurveto{\pgfqpoint{8.552582in}{2.561090in}}{\pgfqpoint{8.557420in}{2.559086in}}{\pgfqpoint{8.562464in}{2.559086in}}%
\pgfpathclose%
\pgfusepath{fill}%
\end{pgfscope}%
\begin{pgfscope}%
\pgfpathrectangle{\pgfqpoint{6.572727in}{0.473000in}}{\pgfqpoint{4.227273in}{3.311000in}}%
\pgfusepath{clip}%
\pgfsetbuttcap%
\pgfsetroundjoin%
\definecolor{currentfill}{rgb}{0.993248,0.906157,0.143936}%
\pgfsetfillcolor{currentfill}%
\pgfsetfillopacity{0.700000}%
\pgfsetlinewidth{0.000000pt}%
\definecolor{currentstroke}{rgb}{0.000000,0.000000,0.000000}%
\pgfsetstrokecolor{currentstroke}%
\pgfsetstrokeopacity{0.700000}%
\pgfsetdash{}{0pt}%
\pgfpathmoveto{\pgfqpoint{9.774988in}{1.383100in}}%
\pgfpathcurveto{\pgfqpoint{9.780032in}{1.383100in}}{\pgfqpoint{9.784870in}{1.385104in}}{\pgfqpoint{9.788436in}{1.388670in}}%
\pgfpathcurveto{\pgfqpoint{9.792002in}{1.392237in}}{\pgfqpoint{9.794006in}{1.397074in}}{\pgfqpoint{9.794006in}{1.402118in}}%
\pgfpathcurveto{\pgfqpoint{9.794006in}{1.407162in}}{\pgfqpoint{9.792002in}{1.412000in}}{\pgfqpoint{9.788436in}{1.415566in}}%
\pgfpathcurveto{\pgfqpoint{9.784870in}{1.419132in}}{\pgfqpoint{9.780032in}{1.421136in}}{\pgfqpoint{9.774988in}{1.421136in}}%
\pgfpathcurveto{\pgfqpoint{9.769944in}{1.421136in}}{\pgfqpoint{9.765107in}{1.419132in}}{\pgfqpoint{9.761540in}{1.415566in}}%
\pgfpathcurveto{\pgfqpoint{9.757974in}{1.412000in}}{\pgfqpoint{9.755970in}{1.407162in}}{\pgfqpoint{9.755970in}{1.402118in}}%
\pgfpathcurveto{\pgfqpoint{9.755970in}{1.397074in}}{\pgfqpoint{9.757974in}{1.392237in}}{\pgfqpoint{9.761540in}{1.388670in}}%
\pgfpathcurveto{\pgfqpoint{9.765107in}{1.385104in}}{\pgfqpoint{9.769944in}{1.383100in}}{\pgfqpoint{9.774988in}{1.383100in}}%
\pgfpathclose%
\pgfusepath{fill}%
\end{pgfscope}%
\begin{pgfscope}%
\pgfpathrectangle{\pgfqpoint{6.572727in}{0.473000in}}{\pgfqpoint{4.227273in}{3.311000in}}%
\pgfusepath{clip}%
\pgfsetbuttcap%
\pgfsetroundjoin%
\definecolor{currentfill}{rgb}{0.993248,0.906157,0.143936}%
\pgfsetfillcolor{currentfill}%
\pgfsetfillopacity{0.700000}%
\pgfsetlinewidth{0.000000pt}%
\definecolor{currentstroke}{rgb}{0.000000,0.000000,0.000000}%
\pgfsetstrokecolor{currentstroke}%
\pgfsetstrokeopacity{0.700000}%
\pgfsetdash{}{0pt}%
\pgfpathmoveto{\pgfqpoint{9.605057in}{1.349354in}}%
\pgfpathcurveto{\pgfqpoint{9.610100in}{1.349354in}}{\pgfqpoint{9.614938in}{1.351358in}}{\pgfqpoint{9.618505in}{1.354925in}}%
\pgfpathcurveto{\pgfqpoint{9.622071in}{1.358491in}}{\pgfqpoint{9.624075in}{1.363329in}}{\pgfqpoint{9.624075in}{1.368373in}}%
\pgfpathcurveto{\pgfqpoint{9.624075in}{1.373416in}}{\pgfqpoint{9.622071in}{1.378254in}}{\pgfqpoint{9.618505in}{1.381820in}}%
\pgfpathcurveto{\pgfqpoint{9.614938in}{1.385387in}}{\pgfqpoint{9.610100in}{1.387391in}}{\pgfqpoint{9.605057in}{1.387391in}}%
\pgfpathcurveto{\pgfqpoint{9.600013in}{1.387391in}}{\pgfqpoint{9.595175in}{1.385387in}}{\pgfqpoint{9.591609in}{1.381820in}}%
\pgfpathcurveto{\pgfqpoint{9.588042in}{1.378254in}}{\pgfqpoint{9.586039in}{1.373416in}}{\pgfqpoint{9.586039in}{1.368373in}}%
\pgfpathcurveto{\pgfqpoint{9.586039in}{1.363329in}}{\pgfqpoint{9.588042in}{1.358491in}}{\pgfqpoint{9.591609in}{1.354925in}}%
\pgfpathcurveto{\pgfqpoint{9.595175in}{1.351358in}}{\pgfqpoint{9.600013in}{1.349354in}}{\pgfqpoint{9.605057in}{1.349354in}}%
\pgfpathclose%
\pgfusepath{fill}%
\end{pgfscope}%
\begin{pgfscope}%
\pgfpathrectangle{\pgfqpoint{6.572727in}{0.473000in}}{\pgfqpoint{4.227273in}{3.311000in}}%
\pgfusepath{clip}%
\pgfsetbuttcap%
\pgfsetroundjoin%
\definecolor{currentfill}{rgb}{0.127568,0.566949,0.550556}%
\pgfsetfillcolor{currentfill}%
\pgfsetfillopacity{0.700000}%
\pgfsetlinewidth{0.000000pt}%
\definecolor{currentstroke}{rgb}{0.000000,0.000000,0.000000}%
\pgfsetstrokecolor{currentstroke}%
\pgfsetstrokeopacity{0.700000}%
\pgfsetdash{}{0pt}%
\pgfpathmoveto{\pgfqpoint{7.976908in}{2.873141in}}%
\pgfpathcurveto{\pgfqpoint{7.981951in}{2.873141in}}{\pgfqpoint{7.986789in}{2.875145in}}{\pgfqpoint{7.990355in}{2.878712in}}%
\pgfpathcurveto{\pgfqpoint{7.993922in}{2.882278in}}{\pgfqpoint{7.995926in}{2.887116in}}{\pgfqpoint{7.995926in}{2.892159in}}%
\pgfpathcurveto{\pgfqpoint{7.995926in}{2.897203in}}{\pgfqpoint{7.993922in}{2.902041in}}{\pgfqpoint{7.990355in}{2.905607in}}%
\pgfpathcurveto{\pgfqpoint{7.986789in}{2.909174in}}{\pgfqpoint{7.981951in}{2.911178in}}{\pgfqpoint{7.976908in}{2.911178in}}%
\pgfpathcurveto{\pgfqpoint{7.971864in}{2.911178in}}{\pgfqpoint{7.967026in}{2.909174in}}{\pgfqpoint{7.963460in}{2.905607in}}%
\pgfpathcurveto{\pgfqpoint{7.959893in}{2.902041in}}{\pgfqpoint{7.957889in}{2.897203in}}{\pgfqpoint{7.957889in}{2.892159in}}%
\pgfpathcurveto{\pgfqpoint{7.957889in}{2.887116in}}{\pgfqpoint{7.959893in}{2.882278in}}{\pgfqpoint{7.963460in}{2.878712in}}%
\pgfpathcurveto{\pgfqpoint{7.967026in}{2.875145in}}{\pgfqpoint{7.971864in}{2.873141in}}{\pgfqpoint{7.976908in}{2.873141in}}%
\pgfpathclose%
\pgfusepath{fill}%
\end{pgfscope}%
\begin{pgfscope}%
\pgfpathrectangle{\pgfqpoint{6.572727in}{0.473000in}}{\pgfqpoint{4.227273in}{3.311000in}}%
\pgfusepath{clip}%
\pgfsetbuttcap%
\pgfsetroundjoin%
\definecolor{currentfill}{rgb}{0.127568,0.566949,0.550556}%
\pgfsetfillcolor{currentfill}%
\pgfsetfillopacity{0.700000}%
\pgfsetlinewidth{0.000000pt}%
\definecolor{currentstroke}{rgb}{0.000000,0.000000,0.000000}%
\pgfsetstrokecolor{currentstroke}%
\pgfsetstrokeopacity{0.700000}%
\pgfsetdash{}{0pt}%
\pgfpathmoveto{\pgfqpoint{8.196999in}{2.838814in}}%
\pgfpathcurveto{\pgfqpoint{8.202043in}{2.838814in}}{\pgfqpoint{8.206881in}{2.840818in}}{\pgfqpoint{8.210447in}{2.844385in}}%
\pgfpathcurveto{\pgfqpoint{8.214013in}{2.847951in}}{\pgfqpoint{8.216017in}{2.852789in}}{\pgfqpoint{8.216017in}{2.857833in}}%
\pgfpathcurveto{\pgfqpoint{8.216017in}{2.862876in}}{\pgfqpoint{8.214013in}{2.867714in}}{\pgfqpoint{8.210447in}{2.871280in}}%
\pgfpathcurveto{\pgfqpoint{8.206881in}{2.874847in}}{\pgfqpoint{8.202043in}{2.876851in}}{\pgfqpoint{8.196999in}{2.876851in}}%
\pgfpathcurveto{\pgfqpoint{8.191956in}{2.876851in}}{\pgfqpoint{8.187118in}{2.874847in}}{\pgfqpoint{8.183551in}{2.871280in}}%
\pgfpathcurveto{\pgfqpoint{8.179985in}{2.867714in}}{\pgfqpoint{8.177981in}{2.862876in}}{\pgfqpoint{8.177981in}{2.857833in}}%
\pgfpathcurveto{\pgfqpoint{8.177981in}{2.852789in}}{\pgfqpoint{8.179985in}{2.847951in}}{\pgfqpoint{8.183551in}{2.844385in}}%
\pgfpathcurveto{\pgfqpoint{8.187118in}{2.840818in}}{\pgfqpoint{8.191956in}{2.838814in}}{\pgfqpoint{8.196999in}{2.838814in}}%
\pgfpathclose%
\pgfusepath{fill}%
\end{pgfscope}%
\begin{pgfscope}%
\pgfpathrectangle{\pgfqpoint{6.572727in}{0.473000in}}{\pgfqpoint{4.227273in}{3.311000in}}%
\pgfusepath{clip}%
\pgfsetbuttcap%
\pgfsetroundjoin%
\definecolor{currentfill}{rgb}{0.127568,0.566949,0.550556}%
\pgfsetfillcolor{currentfill}%
\pgfsetfillopacity{0.700000}%
\pgfsetlinewidth{0.000000pt}%
\definecolor{currentstroke}{rgb}{0.000000,0.000000,0.000000}%
\pgfsetstrokecolor{currentstroke}%
\pgfsetstrokeopacity{0.700000}%
\pgfsetdash{}{0pt}%
\pgfpathmoveto{\pgfqpoint{7.797632in}{1.267034in}}%
\pgfpathcurveto{\pgfqpoint{7.802675in}{1.267034in}}{\pgfqpoint{7.807513in}{1.269037in}}{\pgfqpoint{7.811079in}{1.272604in}}%
\pgfpathcurveto{\pgfqpoint{7.814646in}{1.276170in}}{\pgfqpoint{7.816650in}{1.281008in}}{\pgfqpoint{7.816650in}{1.286052in}}%
\pgfpathcurveto{\pgfqpoint{7.816650in}{1.291095in}}{\pgfqpoint{7.814646in}{1.295933in}}{\pgfqpoint{7.811079in}{1.299500in}}%
\pgfpathcurveto{\pgfqpoint{7.807513in}{1.303066in}}{\pgfqpoint{7.802675in}{1.305070in}}{\pgfqpoint{7.797632in}{1.305070in}}%
\pgfpathcurveto{\pgfqpoint{7.792588in}{1.305070in}}{\pgfqpoint{7.787750in}{1.303066in}}{\pgfqpoint{7.784184in}{1.299500in}}%
\pgfpathcurveto{\pgfqpoint{7.780617in}{1.295933in}}{\pgfqpoint{7.778613in}{1.291095in}}{\pgfqpoint{7.778613in}{1.286052in}}%
\pgfpathcurveto{\pgfqpoint{7.778613in}{1.281008in}}{\pgfqpoint{7.780617in}{1.276170in}}{\pgfqpoint{7.784184in}{1.272604in}}%
\pgfpathcurveto{\pgfqpoint{7.787750in}{1.269037in}}{\pgfqpoint{7.792588in}{1.267034in}}{\pgfqpoint{7.797632in}{1.267034in}}%
\pgfpathclose%
\pgfusepath{fill}%
\end{pgfscope}%
\begin{pgfscope}%
\pgfpathrectangle{\pgfqpoint{6.572727in}{0.473000in}}{\pgfqpoint{4.227273in}{3.311000in}}%
\pgfusepath{clip}%
\pgfsetbuttcap%
\pgfsetroundjoin%
\definecolor{currentfill}{rgb}{0.127568,0.566949,0.550556}%
\pgfsetfillcolor{currentfill}%
\pgfsetfillopacity{0.700000}%
\pgfsetlinewidth{0.000000pt}%
\definecolor{currentstroke}{rgb}{0.000000,0.000000,0.000000}%
\pgfsetstrokecolor{currentstroke}%
\pgfsetstrokeopacity{0.700000}%
\pgfsetdash{}{0pt}%
\pgfpathmoveto{\pgfqpoint{8.678931in}{2.595975in}}%
\pgfpathcurveto{\pgfqpoint{8.683974in}{2.595975in}}{\pgfqpoint{8.688812in}{2.597979in}}{\pgfqpoint{8.692379in}{2.601545in}}%
\pgfpathcurveto{\pgfqpoint{8.695945in}{2.605112in}}{\pgfqpoint{8.697949in}{2.609949in}}{\pgfqpoint{8.697949in}{2.614993in}}%
\pgfpathcurveto{\pgfqpoint{8.697949in}{2.620037in}}{\pgfqpoint{8.695945in}{2.624874in}}{\pgfqpoint{8.692379in}{2.628441in}}%
\pgfpathcurveto{\pgfqpoint{8.688812in}{2.632007in}}{\pgfqpoint{8.683974in}{2.634011in}}{\pgfqpoint{8.678931in}{2.634011in}}%
\pgfpathcurveto{\pgfqpoint{8.673887in}{2.634011in}}{\pgfqpoint{8.669049in}{2.632007in}}{\pgfqpoint{8.665483in}{2.628441in}}%
\pgfpathcurveto{\pgfqpoint{8.661916in}{2.624874in}}{\pgfqpoint{8.659913in}{2.620037in}}{\pgfqpoint{8.659913in}{2.614993in}}%
\pgfpathcurveto{\pgfqpoint{8.659913in}{2.609949in}}{\pgfqpoint{8.661916in}{2.605112in}}{\pgfqpoint{8.665483in}{2.601545in}}%
\pgfpathcurveto{\pgfqpoint{8.669049in}{2.597979in}}{\pgfqpoint{8.673887in}{2.595975in}}{\pgfqpoint{8.678931in}{2.595975in}}%
\pgfpathclose%
\pgfusepath{fill}%
\end{pgfscope}%
\begin{pgfscope}%
\pgfpathrectangle{\pgfqpoint{6.572727in}{0.473000in}}{\pgfqpoint{4.227273in}{3.311000in}}%
\pgfusepath{clip}%
\pgfsetbuttcap%
\pgfsetroundjoin%
\definecolor{currentfill}{rgb}{0.127568,0.566949,0.550556}%
\pgfsetfillcolor{currentfill}%
\pgfsetfillopacity{0.700000}%
\pgfsetlinewidth{0.000000pt}%
\definecolor{currentstroke}{rgb}{0.000000,0.000000,0.000000}%
\pgfsetstrokecolor{currentstroke}%
\pgfsetstrokeopacity{0.700000}%
\pgfsetdash{}{0pt}%
\pgfpathmoveto{\pgfqpoint{8.480039in}{2.266420in}}%
\pgfpathcurveto{\pgfqpoint{8.485082in}{2.266420in}}{\pgfqpoint{8.489920in}{2.268424in}}{\pgfqpoint{8.493487in}{2.271990in}}%
\pgfpathcurveto{\pgfqpoint{8.497053in}{2.275557in}}{\pgfqpoint{8.499057in}{2.280395in}}{\pgfqpoint{8.499057in}{2.285438in}}%
\pgfpathcurveto{\pgfqpoint{8.499057in}{2.290482in}}{\pgfqpoint{8.497053in}{2.295320in}}{\pgfqpoint{8.493487in}{2.298886in}}%
\pgfpathcurveto{\pgfqpoint{8.489920in}{2.302452in}}{\pgfqpoint{8.485082in}{2.304456in}}{\pgfqpoint{8.480039in}{2.304456in}}%
\pgfpathcurveto{\pgfqpoint{8.474995in}{2.304456in}}{\pgfqpoint{8.470157in}{2.302452in}}{\pgfqpoint{8.466591in}{2.298886in}}%
\pgfpathcurveto{\pgfqpoint{8.463024in}{2.295320in}}{\pgfqpoint{8.461021in}{2.290482in}}{\pgfqpoint{8.461021in}{2.285438in}}%
\pgfpathcurveto{\pgfqpoint{8.461021in}{2.280395in}}{\pgfqpoint{8.463024in}{2.275557in}}{\pgfqpoint{8.466591in}{2.271990in}}%
\pgfpathcurveto{\pgfqpoint{8.470157in}{2.268424in}}{\pgfqpoint{8.474995in}{2.266420in}}{\pgfqpoint{8.480039in}{2.266420in}}%
\pgfpathclose%
\pgfusepath{fill}%
\end{pgfscope}%
\begin{pgfscope}%
\pgfpathrectangle{\pgfqpoint{6.572727in}{0.473000in}}{\pgfqpoint{4.227273in}{3.311000in}}%
\pgfusepath{clip}%
\pgfsetbuttcap%
\pgfsetroundjoin%
\definecolor{currentfill}{rgb}{0.127568,0.566949,0.550556}%
\pgfsetfillcolor{currentfill}%
\pgfsetfillopacity{0.700000}%
\pgfsetlinewidth{0.000000pt}%
\definecolor{currentstroke}{rgb}{0.000000,0.000000,0.000000}%
\pgfsetstrokecolor{currentstroke}%
\pgfsetstrokeopacity{0.700000}%
\pgfsetdash{}{0pt}%
\pgfpathmoveto{\pgfqpoint{8.791698in}{3.441662in}}%
\pgfpathcurveto{\pgfqpoint{8.796741in}{3.441662in}}{\pgfqpoint{8.801579in}{3.443666in}}{\pgfqpoint{8.805145in}{3.447233in}}%
\pgfpathcurveto{\pgfqpoint{8.808712in}{3.450799in}}{\pgfqpoint{8.810716in}{3.455637in}}{\pgfqpoint{8.810716in}{3.460680in}}%
\pgfpathcurveto{\pgfqpoint{8.810716in}{3.465724in}}{\pgfqpoint{8.808712in}{3.470562in}}{\pgfqpoint{8.805145in}{3.474128in}}%
\pgfpathcurveto{\pgfqpoint{8.801579in}{3.477695in}}{\pgfqpoint{8.796741in}{3.479699in}}{\pgfqpoint{8.791698in}{3.479699in}}%
\pgfpathcurveto{\pgfqpoint{8.786654in}{3.479699in}}{\pgfqpoint{8.781816in}{3.477695in}}{\pgfqpoint{8.778250in}{3.474128in}}%
\pgfpathcurveto{\pgfqpoint{8.774683in}{3.470562in}}{\pgfqpoint{8.772679in}{3.465724in}}{\pgfqpoint{8.772679in}{3.460680in}}%
\pgfpathcurveto{\pgfqpoint{8.772679in}{3.455637in}}{\pgfqpoint{8.774683in}{3.450799in}}{\pgfqpoint{8.778250in}{3.447233in}}%
\pgfpathcurveto{\pgfqpoint{8.781816in}{3.443666in}}{\pgfqpoint{8.786654in}{3.441662in}}{\pgfqpoint{8.791698in}{3.441662in}}%
\pgfpathclose%
\pgfusepath{fill}%
\end{pgfscope}%
\begin{pgfscope}%
\pgfpathrectangle{\pgfqpoint{6.572727in}{0.473000in}}{\pgfqpoint{4.227273in}{3.311000in}}%
\pgfusepath{clip}%
\pgfsetbuttcap%
\pgfsetroundjoin%
\definecolor{currentfill}{rgb}{0.127568,0.566949,0.550556}%
\pgfsetfillcolor{currentfill}%
\pgfsetfillopacity{0.700000}%
\pgfsetlinewidth{0.000000pt}%
\definecolor{currentstroke}{rgb}{0.000000,0.000000,0.000000}%
\pgfsetstrokecolor{currentstroke}%
\pgfsetstrokeopacity{0.700000}%
\pgfsetdash{}{0pt}%
\pgfpathmoveto{\pgfqpoint{8.713837in}{2.144935in}}%
\pgfpathcurveto{\pgfqpoint{8.718880in}{2.144935in}}{\pgfqpoint{8.723718in}{2.146939in}}{\pgfqpoint{8.727284in}{2.150506in}}%
\pgfpathcurveto{\pgfqpoint{8.730851in}{2.154072in}}{\pgfqpoint{8.732855in}{2.158910in}}{\pgfqpoint{8.732855in}{2.163953in}}%
\pgfpathcurveto{\pgfqpoint{8.732855in}{2.168997in}}{\pgfqpoint{8.730851in}{2.173835in}}{\pgfqpoint{8.727284in}{2.177401in}}%
\pgfpathcurveto{\pgfqpoint{8.723718in}{2.180968in}}{\pgfqpoint{8.718880in}{2.182972in}}{\pgfqpoint{8.713837in}{2.182972in}}%
\pgfpathcurveto{\pgfqpoint{8.708793in}{2.182972in}}{\pgfqpoint{8.703955in}{2.180968in}}{\pgfqpoint{8.700389in}{2.177401in}}%
\pgfpathcurveto{\pgfqpoint{8.696822in}{2.173835in}}{\pgfqpoint{8.694818in}{2.168997in}}{\pgfqpoint{8.694818in}{2.163953in}}%
\pgfpathcurveto{\pgfqpoint{8.694818in}{2.158910in}}{\pgfqpoint{8.696822in}{2.154072in}}{\pgfqpoint{8.700389in}{2.150506in}}%
\pgfpathcurveto{\pgfqpoint{8.703955in}{2.146939in}}{\pgfqpoint{8.708793in}{2.144935in}}{\pgfqpoint{8.713837in}{2.144935in}}%
\pgfpathclose%
\pgfusepath{fill}%
\end{pgfscope}%
\begin{pgfscope}%
\pgfpathrectangle{\pgfqpoint{6.572727in}{0.473000in}}{\pgfqpoint{4.227273in}{3.311000in}}%
\pgfusepath{clip}%
\pgfsetbuttcap%
\pgfsetroundjoin%
\definecolor{currentfill}{rgb}{0.127568,0.566949,0.550556}%
\pgfsetfillcolor{currentfill}%
\pgfsetfillopacity{0.700000}%
\pgfsetlinewidth{0.000000pt}%
\definecolor{currentstroke}{rgb}{0.000000,0.000000,0.000000}%
\pgfsetstrokecolor{currentstroke}%
\pgfsetstrokeopacity{0.700000}%
\pgfsetdash{}{0pt}%
\pgfpathmoveto{\pgfqpoint{8.528419in}{3.082890in}}%
\pgfpathcurveto{\pgfqpoint{8.533463in}{3.082890in}}{\pgfqpoint{8.538301in}{3.084894in}}{\pgfqpoint{8.541867in}{3.088460in}}%
\pgfpathcurveto{\pgfqpoint{8.545434in}{3.092026in}}{\pgfqpoint{8.547438in}{3.096864in}}{\pgfqpoint{8.547438in}{3.101908in}}%
\pgfpathcurveto{\pgfqpoint{8.547438in}{3.106952in}}{\pgfqpoint{8.545434in}{3.111789in}}{\pgfqpoint{8.541867in}{3.115356in}}%
\pgfpathcurveto{\pgfqpoint{8.538301in}{3.118922in}}{\pgfqpoint{8.533463in}{3.120926in}}{\pgfqpoint{8.528419in}{3.120926in}}%
\pgfpathcurveto{\pgfqpoint{8.523376in}{3.120926in}}{\pgfqpoint{8.518538in}{3.118922in}}{\pgfqpoint{8.514972in}{3.115356in}}%
\pgfpathcurveto{\pgfqpoint{8.511405in}{3.111789in}}{\pgfqpoint{8.509401in}{3.106952in}}{\pgfqpoint{8.509401in}{3.101908in}}%
\pgfpathcurveto{\pgfqpoint{8.509401in}{3.096864in}}{\pgfqpoint{8.511405in}{3.092026in}}{\pgfqpoint{8.514972in}{3.088460in}}%
\pgfpathcurveto{\pgfqpoint{8.518538in}{3.084894in}}{\pgfqpoint{8.523376in}{3.082890in}}{\pgfqpoint{8.528419in}{3.082890in}}%
\pgfpathclose%
\pgfusepath{fill}%
\end{pgfscope}%
\begin{pgfscope}%
\pgfpathrectangle{\pgfqpoint{6.572727in}{0.473000in}}{\pgfqpoint{4.227273in}{3.311000in}}%
\pgfusepath{clip}%
\pgfsetbuttcap%
\pgfsetroundjoin%
\definecolor{currentfill}{rgb}{0.127568,0.566949,0.550556}%
\pgfsetfillcolor{currentfill}%
\pgfsetfillopacity{0.700000}%
\pgfsetlinewidth{0.000000pt}%
\definecolor{currentstroke}{rgb}{0.000000,0.000000,0.000000}%
\pgfsetstrokecolor{currentstroke}%
\pgfsetstrokeopacity{0.700000}%
\pgfsetdash{}{0pt}%
\pgfpathmoveto{\pgfqpoint{7.823149in}{2.660907in}}%
\pgfpathcurveto{\pgfqpoint{7.828193in}{2.660907in}}{\pgfqpoint{7.833031in}{2.662911in}}{\pgfqpoint{7.836597in}{2.666477in}}%
\pgfpathcurveto{\pgfqpoint{7.840164in}{2.670043in}}{\pgfqpoint{7.842167in}{2.674881in}}{\pgfqpoint{7.842167in}{2.679925in}}%
\pgfpathcurveto{\pgfqpoint{7.842167in}{2.684968in}}{\pgfqpoint{7.840164in}{2.689806in}}{\pgfqpoint{7.836597in}{2.693373in}}%
\pgfpathcurveto{\pgfqpoint{7.833031in}{2.696939in}}{\pgfqpoint{7.828193in}{2.698943in}}{\pgfqpoint{7.823149in}{2.698943in}}%
\pgfpathcurveto{\pgfqpoint{7.818106in}{2.698943in}}{\pgfqpoint{7.813268in}{2.696939in}}{\pgfqpoint{7.809701in}{2.693373in}}%
\pgfpathcurveto{\pgfqpoint{7.806135in}{2.689806in}}{\pgfqpoint{7.804131in}{2.684968in}}{\pgfqpoint{7.804131in}{2.679925in}}%
\pgfpathcurveto{\pgfqpoint{7.804131in}{2.674881in}}{\pgfqpoint{7.806135in}{2.670043in}}{\pgfqpoint{7.809701in}{2.666477in}}%
\pgfpathcurveto{\pgfqpoint{7.813268in}{2.662911in}}{\pgfqpoint{7.818106in}{2.660907in}}{\pgfqpoint{7.823149in}{2.660907in}}%
\pgfpathclose%
\pgfusepath{fill}%
\end{pgfscope}%
\begin{pgfscope}%
\pgfpathrectangle{\pgfqpoint{6.572727in}{0.473000in}}{\pgfqpoint{4.227273in}{3.311000in}}%
\pgfusepath{clip}%
\pgfsetbuttcap%
\pgfsetroundjoin%
\definecolor{currentfill}{rgb}{0.127568,0.566949,0.550556}%
\pgfsetfillcolor{currentfill}%
\pgfsetfillopacity{0.700000}%
\pgfsetlinewidth{0.000000pt}%
\definecolor{currentstroke}{rgb}{0.000000,0.000000,0.000000}%
\pgfsetstrokecolor{currentstroke}%
\pgfsetstrokeopacity{0.700000}%
\pgfsetdash{}{0pt}%
\pgfpathmoveto{\pgfqpoint{8.131231in}{1.688361in}}%
\pgfpathcurveto{\pgfqpoint{8.136275in}{1.688361in}}{\pgfqpoint{8.141113in}{1.690365in}}{\pgfqpoint{8.144679in}{1.693932in}}%
\pgfpathcurveto{\pgfqpoint{8.148245in}{1.697498in}}{\pgfqpoint{8.150249in}{1.702336in}}{\pgfqpoint{8.150249in}{1.707380in}}%
\pgfpathcurveto{\pgfqpoint{8.150249in}{1.712423in}}{\pgfqpoint{8.148245in}{1.717261in}}{\pgfqpoint{8.144679in}{1.720827in}}%
\pgfpathcurveto{\pgfqpoint{8.141113in}{1.724394in}}{\pgfqpoint{8.136275in}{1.726398in}}{\pgfqpoint{8.131231in}{1.726398in}}%
\pgfpathcurveto{\pgfqpoint{8.126187in}{1.726398in}}{\pgfqpoint{8.121350in}{1.724394in}}{\pgfqpoint{8.117783in}{1.720827in}}%
\pgfpathcurveto{\pgfqpoint{8.114217in}{1.717261in}}{\pgfqpoint{8.112213in}{1.712423in}}{\pgfqpoint{8.112213in}{1.707380in}}%
\pgfpathcurveto{\pgfqpoint{8.112213in}{1.702336in}}{\pgfqpoint{8.114217in}{1.697498in}}{\pgfqpoint{8.117783in}{1.693932in}}%
\pgfpathcurveto{\pgfqpoint{8.121350in}{1.690365in}}{\pgfqpoint{8.126187in}{1.688361in}}{\pgfqpoint{8.131231in}{1.688361in}}%
\pgfpathclose%
\pgfusepath{fill}%
\end{pgfscope}%
\begin{pgfscope}%
\pgfpathrectangle{\pgfqpoint{6.572727in}{0.473000in}}{\pgfqpoint{4.227273in}{3.311000in}}%
\pgfusepath{clip}%
\pgfsetbuttcap%
\pgfsetroundjoin%
\definecolor{currentfill}{rgb}{0.993248,0.906157,0.143936}%
\pgfsetfillcolor{currentfill}%
\pgfsetfillopacity{0.700000}%
\pgfsetlinewidth{0.000000pt}%
\definecolor{currentstroke}{rgb}{0.000000,0.000000,0.000000}%
\pgfsetstrokecolor{currentstroke}%
\pgfsetstrokeopacity{0.700000}%
\pgfsetdash{}{0pt}%
\pgfpathmoveto{\pgfqpoint{9.491682in}{1.735351in}}%
\pgfpathcurveto{\pgfqpoint{9.496726in}{1.735351in}}{\pgfqpoint{9.501564in}{1.737355in}}{\pgfqpoint{9.505130in}{1.740921in}}%
\pgfpathcurveto{\pgfqpoint{9.508697in}{1.744488in}}{\pgfqpoint{9.510701in}{1.749325in}}{\pgfqpoint{9.510701in}{1.754369in}}%
\pgfpathcurveto{\pgfqpoint{9.510701in}{1.759413in}}{\pgfqpoint{9.508697in}{1.764250in}}{\pgfqpoint{9.505130in}{1.767817in}}%
\pgfpathcurveto{\pgfqpoint{9.501564in}{1.771383in}}{\pgfqpoint{9.496726in}{1.773387in}}{\pgfqpoint{9.491682in}{1.773387in}}%
\pgfpathcurveto{\pgfqpoint{9.486639in}{1.773387in}}{\pgfqpoint{9.481801in}{1.771383in}}{\pgfqpoint{9.478235in}{1.767817in}}%
\pgfpathcurveto{\pgfqpoint{9.474668in}{1.764250in}}{\pgfqpoint{9.472664in}{1.759413in}}{\pgfqpoint{9.472664in}{1.754369in}}%
\pgfpathcurveto{\pgfqpoint{9.472664in}{1.749325in}}{\pgfqpoint{9.474668in}{1.744488in}}{\pgfqpoint{9.478235in}{1.740921in}}%
\pgfpathcurveto{\pgfqpoint{9.481801in}{1.737355in}}{\pgfqpoint{9.486639in}{1.735351in}}{\pgfqpoint{9.491682in}{1.735351in}}%
\pgfpathclose%
\pgfusepath{fill}%
\end{pgfscope}%
\begin{pgfscope}%
\pgfpathrectangle{\pgfqpoint{6.572727in}{0.473000in}}{\pgfqpoint{4.227273in}{3.311000in}}%
\pgfusepath{clip}%
\pgfsetbuttcap%
\pgfsetroundjoin%
\definecolor{currentfill}{rgb}{0.993248,0.906157,0.143936}%
\pgfsetfillcolor{currentfill}%
\pgfsetfillopacity{0.700000}%
\pgfsetlinewidth{0.000000pt}%
\definecolor{currentstroke}{rgb}{0.000000,0.000000,0.000000}%
\pgfsetstrokecolor{currentstroke}%
\pgfsetstrokeopacity{0.700000}%
\pgfsetdash{}{0pt}%
\pgfpathmoveto{\pgfqpoint{8.950341in}{1.320003in}}%
\pgfpathcurveto{\pgfqpoint{8.955385in}{1.320003in}}{\pgfqpoint{8.960223in}{1.322006in}}{\pgfqpoint{8.963789in}{1.325573in}}%
\pgfpathcurveto{\pgfqpoint{8.967356in}{1.329139in}}{\pgfqpoint{8.969360in}{1.333977in}}{\pgfqpoint{8.969360in}{1.339021in}}%
\pgfpathcurveto{\pgfqpoint{8.969360in}{1.344064in}}{\pgfqpoint{8.967356in}{1.348902in}}{\pgfqpoint{8.963789in}{1.352469in}}%
\pgfpathcurveto{\pgfqpoint{8.960223in}{1.356035in}}{\pgfqpoint{8.955385in}{1.358039in}}{\pgfqpoint{8.950341in}{1.358039in}}%
\pgfpathcurveto{\pgfqpoint{8.945298in}{1.358039in}}{\pgfqpoint{8.940460in}{1.356035in}}{\pgfqpoint{8.936894in}{1.352469in}}%
\pgfpathcurveto{\pgfqpoint{8.933327in}{1.348902in}}{\pgfqpoint{8.931323in}{1.344064in}}{\pgfqpoint{8.931323in}{1.339021in}}%
\pgfpathcurveto{\pgfqpoint{8.931323in}{1.333977in}}{\pgfqpoint{8.933327in}{1.329139in}}{\pgfqpoint{8.936894in}{1.325573in}}%
\pgfpathcurveto{\pgfqpoint{8.940460in}{1.322006in}}{\pgfqpoint{8.945298in}{1.320003in}}{\pgfqpoint{8.950341in}{1.320003in}}%
\pgfpathclose%
\pgfusepath{fill}%
\end{pgfscope}%
\begin{pgfscope}%
\pgfpathrectangle{\pgfqpoint{6.572727in}{0.473000in}}{\pgfqpoint{4.227273in}{3.311000in}}%
\pgfusepath{clip}%
\pgfsetbuttcap%
\pgfsetroundjoin%
\definecolor{currentfill}{rgb}{0.127568,0.566949,0.550556}%
\pgfsetfillcolor{currentfill}%
\pgfsetfillopacity{0.700000}%
\pgfsetlinewidth{0.000000pt}%
\definecolor{currentstroke}{rgb}{0.000000,0.000000,0.000000}%
\pgfsetstrokecolor{currentstroke}%
\pgfsetstrokeopacity{0.700000}%
\pgfsetdash{}{0pt}%
\pgfpathmoveto{\pgfqpoint{8.587236in}{2.921119in}}%
\pgfpathcurveto{\pgfqpoint{8.592280in}{2.921119in}}{\pgfqpoint{8.597117in}{2.923123in}}{\pgfqpoint{8.600684in}{2.926689in}}%
\pgfpathcurveto{\pgfqpoint{8.604250in}{2.930256in}}{\pgfqpoint{8.606254in}{2.935094in}}{\pgfqpoint{8.606254in}{2.940137in}}%
\pgfpathcurveto{\pgfqpoint{8.606254in}{2.945181in}}{\pgfqpoint{8.604250in}{2.950019in}}{\pgfqpoint{8.600684in}{2.953585in}}%
\pgfpathcurveto{\pgfqpoint{8.597117in}{2.957151in}}{\pgfqpoint{8.592280in}{2.959155in}}{\pgfqpoint{8.587236in}{2.959155in}}%
\pgfpathcurveto{\pgfqpoint{8.582192in}{2.959155in}}{\pgfqpoint{8.577354in}{2.957151in}}{\pgfqpoint{8.573788in}{2.953585in}}%
\pgfpathcurveto{\pgfqpoint{8.570222in}{2.950019in}}{\pgfqpoint{8.568218in}{2.945181in}}{\pgfqpoint{8.568218in}{2.940137in}}%
\pgfpathcurveto{\pgfqpoint{8.568218in}{2.935094in}}{\pgfqpoint{8.570222in}{2.930256in}}{\pgfqpoint{8.573788in}{2.926689in}}%
\pgfpathcurveto{\pgfqpoint{8.577354in}{2.923123in}}{\pgfqpoint{8.582192in}{2.921119in}}{\pgfqpoint{8.587236in}{2.921119in}}%
\pgfpathclose%
\pgfusepath{fill}%
\end{pgfscope}%
\begin{pgfscope}%
\pgfpathrectangle{\pgfqpoint{6.572727in}{0.473000in}}{\pgfqpoint{4.227273in}{3.311000in}}%
\pgfusepath{clip}%
\pgfsetbuttcap%
\pgfsetroundjoin%
\definecolor{currentfill}{rgb}{0.267004,0.004874,0.329415}%
\pgfsetfillcolor{currentfill}%
\pgfsetfillopacity{0.700000}%
\pgfsetlinewidth{0.000000pt}%
\definecolor{currentstroke}{rgb}{0.000000,0.000000,0.000000}%
\pgfsetstrokecolor{currentstroke}%
\pgfsetstrokeopacity{0.700000}%
\pgfsetdash{}{0pt}%
\pgfpathmoveto{\pgfqpoint{10.585804in}{1.302727in}}%
\pgfpathcurveto{\pgfqpoint{10.590848in}{1.302727in}}{\pgfqpoint{10.595686in}{1.304731in}}{\pgfqpoint{10.599252in}{1.308297in}}%
\pgfpathcurveto{\pgfqpoint{10.602818in}{1.311863in}}{\pgfqpoint{10.604822in}{1.316701in}}{\pgfqpoint{10.604822in}{1.321745in}}%
\pgfpathcurveto{\pgfqpoint{10.604822in}{1.326789in}}{\pgfqpoint{10.602818in}{1.331626in}}{\pgfqpoint{10.599252in}{1.335193in}}%
\pgfpathcurveto{\pgfqpoint{10.595686in}{1.338759in}}{\pgfqpoint{10.590848in}{1.340763in}}{\pgfqpoint{10.585804in}{1.340763in}}%
\pgfpathcurveto{\pgfqpoint{10.580760in}{1.340763in}}{\pgfqpoint{10.575923in}{1.338759in}}{\pgfqpoint{10.572356in}{1.335193in}}%
\pgfpathcurveto{\pgfqpoint{10.568790in}{1.331626in}}{\pgfqpoint{10.566786in}{1.326789in}}{\pgfqpoint{10.566786in}{1.321745in}}%
\pgfpathcurveto{\pgfqpoint{10.566786in}{1.316701in}}{\pgfqpoint{10.568790in}{1.311863in}}{\pgfqpoint{10.572356in}{1.308297in}}%
\pgfpathcurveto{\pgfqpoint{10.575923in}{1.304731in}}{\pgfqpoint{10.580760in}{1.302727in}}{\pgfqpoint{10.585804in}{1.302727in}}%
\pgfpathclose%
\pgfusepath{fill}%
\end{pgfscope}%
\begin{pgfscope}%
\pgfpathrectangle{\pgfqpoint{6.572727in}{0.473000in}}{\pgfqpoint{4.227273in}{3.311000in}}%
\pgfusepath{clip}%
\pgfsetbuttcap%
\pgfsetroundjoin%
\definecolor{currentfill}{rgb}{0.127568,0.566949,0.550556}%
\pgfsetfillcolor{currentfill}%
\pgfsetfillopacity{0.700000}%
\pgfsetlinewidth{0.000000pt}%
\definecolor{currentstroke}{rgb}{0.000000,0.000000,0.000000}%
\pgfsetstrokecolor{currentstroke}%
\pgfsetstrokeopacity{0.700000}%
\pgfsetdash{}{0pt}%
\pgfpathmoveto{\pgfqpoint{7.754195in}{1.364008in}}%
\pgfpathcurveto{\pgfqpoint{7.759238in}{1.364008in}}{\pgfqpoint{7.764076in}{1.366012in}}{\pgfqpoint{7.767643in}{1.369578in}}%
\pgfpathcurveto{\pgfqpoint{7.771209in}{1.373145in}}{\pgfqpoint{7.773213in}{1.377982in}}{\pgfqpoint{7.773213in}{1.383026in}}%
\pgfpathcurveto{\pgfqpoint{7.773213in}{1.388070in}}{\pgfqpoint{7.771209in}{1.392908in}}{\pgfqpoint{7.767643in}{1.396474in}}%
\pgfpathcurveto{\pgfqpoint{7.764076in}{1.400040in}}{\pgfqpoint{7.759238in}{1.402044in}}{\pgfqpoint{7.754195in}{1.402044in}}%
\pgfpathcurveto{\pgfqpoint{7.749151in}{1.402044in}}{\pgfqpoint{7.744313in}{1.400040in}}{\pgfqpoint{7.740747in}{1.396474in}}%
\pgfpathcurveto{\pgfqpoint{7.737180in}{1.392908in}}{\pgfqpoint{7.735177in}{1.388070in}}{\pgfqpoint{7.735177in}{1.383026in}}%
\pgfpathcurveto{\pgfqpoint{7.735177in}{1.377982in}}{\pgfqpoint{7.737180in}{1.373145in}}{\pgfqpoint{7.740747in}{1.369578in}}%
\pgfpathcurveto{\pgfqpoint{7.744313in}{1.366012in}}{\pgfqpoint{7.749151in}{1.364008in}}{\pgfqpoint{7.754195in}{1.364008in}}%
\pgfpathclose%
\pgfusepath{fill}%
\end{pgfscope}%
\begin{pgfscope}%
\pgfpathrectangle{\pgfqpoint{6.572727in}{0.473000in}}{\pgfqpoint{4.227273in}{3.311000in}}%
\pgfusepath{clip}%
\pgfsetbuttcap%
\pgfsetroundjoin%
\definecolor{currentfill}{rgb}{0.127568,0.566949,0.550556}%
\pgfsetfillcolor{currentfill}%
\pgfsetfillopacity{0.700000}%
\pgfsetlinewidth{0.000000pt}%
\definecolor{currentstroke}{rgb}{0.000000,0.000000,0.000000}%
\pgfsetstrokecolor{currentstroke}%
\pgfsetstrokeopacity{0.700000}%
\pgfsetdash{}{0pt}%
\pgfpathmoveto{\pgfqpoint{8.103378in}{2.838948in}}%
\pgfpathcurveto{\pgfqpoint{8.108421in}{2.838948in}}{\pgfqpoint{8.113259in}{2.840952in}}{\pgfqpoint{8.116826in}{2.844518in}}%
\pgfpathcurveto{\pgfqpoint{8.120392in}{2.848084in}}{\pgfqpoint{8.122396in}{2.852922in}}{\pgfqpoint{8.122396in}{2.857966in}}%
\pgfpathcurveto{\pgfqpoint{8.122396in}{2.863009in}}{\pgfqpoint{8.120392in}{2.867847in}}{\pgfqpoint{8.116826in}{2.871414in}}%
\pgfpathcurveto{\pgfqpoint{8.113259in}{2.874980in}}{\pgfqpoint{8.108421in}{2.876984in}}{\pgfqpoint{8.103378in}{2.876984in}}%
\pgfpathcurveto{\pgfqpoint{8.098334in}{2.876984in}}{\pgfqpoint{8.093496in}{2.874980in}}{\pgfqpoint{8.089930in}{2.871414in}}%
\pgfpathcurveto{\pgfqpoint{8.086364in}{2.867847in}}{\pgfqpoint{8.084360in}{2.863009in}}{\pgfqpoint{8.084360in}{2.857966in}}%
\pgfpathcurveto{\pgfqpoint{8.084360in}{2.852922in}}{\pgfqpoint{8.086364in}{2.848084in}}{\pgfqpoint{8.089930in}{2.844518in}}%
\pgfpathcurveto{\pgfqpoint{8.093496in}{2.840952in}}{\pgfqpoint{8.098334in}{2.838948in}}{\pgfqpoint{8.103378in}{2.838948in}}%
\pgfpathclose%
\pgfusepath{fill}%
\end{pgfscope}%
\begin{pgfscope}%
\pgfpathrectangle{\pgfqpoint{6.572727in}{0.473000in}}{\pgfqpoint{4.227273in}{3.311000in}}%
\pgfusepath{clip}%
\pgfsetbuttcap%
\pgfsetroundjoin%
\definecolor{currentfill}{rgb}{0.993248,0.906157,0.143936}%
\pgfsetfillcolor{currentfill}%
\pgfsetfillopacity{0.700000}%
\pgfsetlinewidth{0.000000pt}%
\definecolor{currentstroke}{rgb}{0.000000,0.000000,0.000000}%
\pgfsetstrokecolor{currentstroke}%
\pgfsetstrokeopacity{0.700000}%
\pgfsetdash{}{0pt}%
\pgfpathmoveto{\pgfqpoint{10.005022in}{1.607881in}}%
\pgfpathcurveto{\pgfqpoint{10.010066in}{1.607881in}}{\pgfqpoint{10.014904in}{1.609884in}}{\pgfqpoint{10.018470in}{1.613451in}}%
\pgfpathcurveto{\pgfqpoint{10.022037in}{1.617017in}}{\pgfqpoint{10.024041in}{1.621855in}}{\pgfqpoint{10.024041in}{1.626899in}}%
\pgfpathcurveto{\pgfqpoint{10.024041in}{1.631942in}}{\pgfqpoint{10.022037in}{1.636780in}}{\pgfqpoint{10.018470in}{1.640347in}}%
\pgfpathcurveto{\pgfqpoint{10.014904in}{1.643913in}}{\pgfqpoint{10.010066in}{1.645917in}}{\pgfqpoint{10.005022in}{1.645917in}}%
\pgfpathcurveto{\pgfqpoint{9.999979in}{1.645917in}}{\pgfqpoint{9.995141in}{1.643913in}}{\pgfqpoint{9.991575in}{1.640347in}}%
\pgfpathcurveto{\pgfqpoint{9.988008in}{1.636780in}}{\pgfqpoint{9.986004in}{1.631942in}}{\pgfqpoint{9.986004in}{1.626899in}}%
\pgfpathcurveto{\pgfqpoint{9.986004in}{1.621855in}}{\pgfqpoint{9.988008in}{1.617017in}}{\pgfqpoint{9.991575in}{1.613451in}}%
\pgfpathcurveto{\pgfqpoint{9.995141in}{1.609884in}}{\pgfqpoint{9.999979in}{1.607881in}}{\pgfqpoint{10.005022in}{1.607881in}}%
\pgfpathclose%
\pgfusepath{fill}%
\end{pgfscope}%
\begin{pgfscope}%
\pgfpathrectangle{\pgfqpoint{6.572727in}{0.473000in}}{\pgfqpoint{4.227273in}{3.311000in}}%
\pgfusepath{clip}%
\pgfsetbuttcap%
\pgfsetroundjoin%
\definecolor{currentfill}{rgb}{0.127568,0.566949,0.550556}%
\pgfsetfillcolor{currentfill}%
\pgfsetfillopacity{0.700000}%
\pgfsetlinewidth{0.000000pt}%
\definecolor{currentstroke}{rgb}{0.000000,0.000000,0.000000}%
\pgfsetstrokecolor{currentstroke}%
\pgfsetstrokeopacity{0.700000}%
\pgfsetdash{}{0pt}%
\pgfpathmoveto{\pgfqpoint{9.120544in}{3.123363in}}%
\pgfpathcurveto{\pgfqpoint{9.125587in}{3.123363in}}{\pgfqpoint{9.130425in}{3.125367in}}{\pgfqpoint{9.133992in}{3.128933in}}%
\pgfpathcurveto{\pgfqpoint{9.137558in}{3.132499in}}{\pgfqpoint{9.139562in}{3.137337in}}{\pgfqpoint{9.139562in}{3.142381in}}%
\pgfpathcurveto{\pgfqpoint{9.139562in}{3.147425in}}{\pgfqpoint{9.137558in}{3.152262in}}{\pgfqpoint{9.133992in}{3.155829in}}%
\pgfpathcurveto{\pgfqpoint{9.130425in}{3.159395in}}{\pgfqpoint{9.125587in}{3.161399in}}{\pgfqpoint{9.120544in}{3.161399in}}%
\pgfpathcurveto{\pgfqpoint{9.115500in}{3.161399in}}{\pgfqpoint{9.110662in}{3.159395in}}{\pgfqpoint{9.107096in}{3.155829in}}%
\pgfpathcurveto{\pgfqpoint{9.103529in}{3.152262in}}{\pgfqpoint{9.101526in}{3.147425in}}{\pgfqpoint{9.101526in}{3.142381in}}%
\pgfpathcurveto{\pgfqpoint{9.101526in}{3.137337in}}{\pgfqpoint{9.103529in}{3.132499in}}{\pgfqpoint{9.107096in}{3.128933in}}%
\pgfpathcurveto{\pgfqpoint{9.110662in}{3.125367in}}{\pgfqpoint{9.115500in}{3.123363in}}{\pgfqpoint{9.120544in}{3.123363in}}%
\pgfpathclose%
\pgfusepath{fill}%
\end{pgfscope}%
\begin{pgfscope}%
\pgfpathrectangle{\pgfqpoint{6.572727in}{0.473000in}}{\pgfqpoint{4.227273in}{3.311000in}}%
\pgfusepath{clip}%
\pgfsetbuttcap%
\pgfsetroundjoin%
\definecolor{currentfill}{rgb}{0.127568,0.566949,0.550556}%
\pgfsetfillcolor{currentfill}%
\pgfsetfillopacity{0.700000}%
\pgfsetlinewidth{0.000000pt}%
\definecolor{currentstroke}{rgb}{0.000000,0.000000,0.000000}%
\pgfsetstrokecolor{currentstroke}%
\pgfsetstrokeopacity{0.700000}%
\pgfsetdash{}{0pt}%
\pgfpathmoveto{\pgfqpoint{7.971187in}{1.672987in}}%
\pgfpathcurveto{\pgfqpoint{7.976230in}{1.672987in}}{\pgfqpoint{7.981068in}{1.674991in}}{\pgfqpoint{7.984634in}{1.678557in}}%
\pgfpathcurveto{\pgfqpoint{7.988201in}{1.682124in}}{\pgfqpoint{7.990205in}{1.686961in}}{\pgfqpoint{7.990205in}{1.692005in}}%
\pgfpathcurveto{\pgfqpoint{7.990205in}{1.697049in}}{\pgfqpoint{7.988201in}{1.701887in}}{\pgfqpoint{7.984634in}{1.705453in}}%
\pgfpathcurveto{\pgfqpoint{7.981068in}{1.709019in}}{\pgfqpoint{7.976230in}{1.711023in}}{\pgfqpoint{7.971187in}{1.711023in}}%
\pgfpathcurveto{\pgfqpoint{7.966143in}{1.711023in}}{\pgfqpoint{7.961305in}{1.709019in}}{\pgfqpoint{7.957739in}{1.705453in}}%
\pgfpathcurveto{\pgfqpoint{7.954172in}{1.701887in}}{\pgfqpoint{7.952168in}{1.697049in}}{\pgfqpoint{7.952168in}{1.692005in}}%
\pgfpathcurveto{\pgfqpoint{7.952168in}{1.686961in}}{\pgfqpoint{7.954172in}{1.682124in}}{\pgfqpoint{7.957739in}{1.678557in}}%
\pgfpathcurveto{\pgfqpoint{7.961305in}{1.674991in}}{\pgfqpoint{7.966143in}{1.672987in}}{\pgfqpoint{7.971187in}{1.672987in}}%
\pgfpathclose%
\pgfusepath{fill}%
\end{pgfscope}%
\begin{pgfscope}%
\pgfpathrectangle{\pgfqpoint{6.572727in}{0.473000in}}{\pgfqpoint{4.227273in}{3.311000in}}%
\pgfusepath{clip}%
\pgfsetbuttcap%
\pgfsetroundjoin%
\definecolor{currentfill}{rgb}{0.993248,0.906157,0.143936}%
\pgfsetfillcolor{currentfill}%
\pgfsetfillopacity{0.700000}%
\pgfsetlinewidth{0.000000pt}%
\definecolor{currentstroke}{rgb}{0.000000,0.000000,0.000000}%
\pgfsetstrokecolor{currentstroke}%
\pgfsetstrokeopacity{0.700000}%
\pgfsetdash{}{0pt}%
\pgfpathmoveto{\pgfqpoint{9.790444in}{1.433168in}}%
\pgfpathcurveto{\pgfqpoint{9.795488in}{1.433168in}}{\pgfqpoint{9.800326in}{1.435172in}}{\pgfqpoint{9.803892in}{1.438738in}}%
\pgfpathcurveto{\pgfqpoint{9.807459in}{1.442305in}}{\pgfqpoint{9.809463in}{1.447142in}}{\pgfqpoint{9.809463in}{1.452186in}}%
\pgfpathcurveto{\pgfqpoint{9.809463in}{1.457230in}}{\pgfqpoint{9.807459in}{1.462068in}}{\pgfqpoint{9.803892in}{1.465634in}}%
\pgfpathcurveto{\pgfqpoint{9.800326in}{1.469200in}}{\pgfqpoint{9.795488in}{1.471204in}}{\pgfqpoint{9.790444in}{1.471204in}}%
\pgfpathcurveto{\pgfqpoint{9.785401in}{1.471204in}}{\pgfqpoint{9.780563in}{1.469200in}}{\pgfqpoint{9.776997in}{1.465634in}}%
\pgfpathcurveto{\pgfqpoint{9.773430in}{1.462068in}}{\pgfqpoint{9.771426in}{1.457230in}}{\pgfqpoint{9.771426in}{1.452186in}}%
\pgfpathcurveto{\pgfqpoint{9.771426in}{1.447142in}}{\pgfqpoint{9.773430in}{1.442305in}}{\pgfqpoint{9.776997in}{1.438738in}}%
\pgfpathcurveto{\pgfqpoint{9.780563in}{1.435172in}}{\pgfqpoint{9.785401in}{1.433168in}}{\pgfqpoint{9.790444in}{1.433168in}}%
\pgfpathclose%
\pgfusepath{fill}%
\end{pgfscope}%
\begin{pgfscope}%
\pgfpathrectangle{\pgfqpoint{6.572727in}{0.473000in}}{\pgfqpoint{4.227273in}{3.311000in}}%
\pgfusepath{clip}%
\pgfsetbuttcap%
\pgfsetroundjoin%
\definecolor{currentfill}{rgb}{0.993248,0.906157,0.143936}%
\pgfsetfillcolor{currentfill}%
\pgfsetfillopacity{0.700000}%
\pgfsetlinewidth{0.000000pt}%
\definecolor{currentstroke}{rgb}{0.000000,0.000000,0.000000}%
\pgfsetstrokecolor{currentstroke}%
\pgfsetstrokeopacity{0.700000}%
\pgfsetdash{}{0pt}%
\pgfpathmoveto{\pgfqpoint{9.639419in}{2.174853in}}%
\pgfpathcurveto{\pgfqpoint{9.644463in}{2.174853in}}{\pgfqpoint{9.649300in}{2.176857in}}{\pgfqpoint{9.652867in}{2.180423in}}%
\pgfpathcurveto{\pgfqpoint{9.656433in}{2.183990in}}{\pgfqpoint{9.658437in}{2.188827in}}{\pgfqpoint{9.658437in}{2.193871in}}%
\pgfpathcurveto{\pgfqpoint{9.658437in}{2.198915in}}{\pgfqpoint{9.656433in}{2.203753in}}{\pgfqpoint{9.652867in}{2.207319in}}%
\pgfpathcurveto{\pgfqpoint{9.649300in}{2.210885in}}{\pgfqpoint{9.644463in}{2.212889in}}{\pgfqpoint{9.639419in}{2.212889in}}%
\pgfpathcurveto{\pgfqpoint{9.634375in}{2.212889in}}{\pgfqpoint{9.629538in}{2.210885in}}{\pgfqpoint{9.625971in}{2.207319in}}%
\pgfpathcurveto{\pgfqpoint{9.622405in}{2.203753in}}{\pgfqpoint{9.620401in}{2.198915in}}{\pgfqpoint{9.620401in}{2.193871in}}%
\pgfpathcurveto{\pgfqpoint{9.620401in}{2.188827in}}{\pgfqpoint{9.622405in}{2.183990in}}{\pgfqpoint{9.625971in}{2.180423in}}%
\pgfpathcurveto{\pgfqpoint{9.629538in}{2.176857in}}{\pgfqpoint{9.634375in}{2.174853in}}{\pgfqpoint{9.639419in}{2.174853in}}%
\pgfpathclose%
\pgfusepath{fill}%
\end{pgfscope}%
\begin{pgfscope}%
\pgfpathrectangle{\pgfqpoint{6.572727in}{0.473000in}}{\pgfqpoint{4.227273in}{3.311000in}}%
\pgfusepath{clip}%
\pgfsetbuttcap%
\pgfsetroundjoin%
\definecolor{currentfill}{rgb}{0.127568,0.566949,0.550556}%
\pgfsetfillcolor{currentfill}%
\pgfsetfillopacity{0.700000}%
\pgfsetlinewidth{0.000000pt}%
\definecolor{currentstroke}{rgb}{0.000000,0.000000,0.000000}%
\pgfsetstrokecolor{currentstroke}%
\pgfsetstrokeopacity{0.700000}%
\pgfsetdash{}{0pt}%
\pgfpathmoveto{\pgfqpoint{7.990502in}{3.321385in}}%
\pgfpathcurveto{\pgfqpoint{7.995546in}{3.321385in}}{\pgfqpoint{8.000383in}{3.323389in}}{\pgfqpoint{8.003950in}{3.326956in}}%
\pgfpathcurveto{\pgfqpoint{8.007516in}{3.330522in}}{\pgfqpoint{8.009520in}{3.335360in}}{\pgfqpoint{8.009520in}{3.340404in}}%
\pgfpathcurveto{\pgfqpoint{8.009520in}{3.345447in}}{\pgfqpoint{8.007516in}{3.350285in}}{\pgfqpoint{8.003950in}{3.353851in}}%
\pgfpathcurveto{\pgfqpoint{8.000383in}{3.357418in}}{\pgfqpoint{7.995546in}{3.359422in}}{\pgfqpoint{7.990502in}{3.359422in}}%
\pgfpathcurveto{\pgfqpoint{7.985458in}{3.359422in}}{\pgfqpoint{7.980621in}{3.357418in}}{\pgfqpoint{7.977054in}{3.353851in}}%
\pgfpathcurveto{\pgfqpoint{7.973488in}{3.350285in}}{\pgfqpoint{7.971484in}{3.345447in}}{\pgfqpoint{7.971484in}{3.340404in}}%
\pgfpathcurveto{\pgfqpoint{7.971484in}{3.335360in}}{\pgfqpoint{7.973488in}{3.330522in}}{\pgfqpoint{7.977054in}{3.326956in}}%
\pgfpathcurveto{\pgfqpoint{7.980621in}{3.323389in}}{\pgfqpoint{7.985458in}{3.321385in}}{\pgfqpoint{7.990502in}{3.321385in}}%
\pgfpathclose%
\pgfusepath{fill}%
\end{pgfscope}%
\begin{pgfscope}%
\pgfpathrectangle{\pgfqpoint{6.572727in}{0.473000in}}{\pgfqpoint{4.227273in}{3.311000in}}%
\pgfusepath{clip}%
\pgfsetbuttcap%
\pgfsetroundjoin%
\definecolor{currentfill}{rgb}{0.993248,0.906157,0.143936}%
\pgfsetfillcolor{currentfill}%
\pgfsetfillopacity{0.700000}%
\pgfsetlinewidth{0.000000pt}%
\definecolor{currentstroke}{rgb}{0.000000,0.000000,0.000000}%
\pgfsetstrokecolor{currentstroke}%
\pgfsetstrokeopacity{0.700000}%
\pgfsetdash{}{0pt}%
\pgfpathmoveto{\pgfqpoint{9.177129in}{1.647735in}}%
\pgfpathcurveto{\pgfqpoint{9.182173in}{1.647735in}}{\pgfqpoint{9.187011in}{1.649739in}}{\pgfqpoint{9.190577in}{1.653305in}}%
\pgfpathcurveto{\pgfqpoint{9.194144in}{1.656872in}}{\pgfqpoint{9.196148in}{1.661709in}}{\pgfqpoint{9.196148in}{1.666753in}}%
\pgfpathcurveto{\pgfqpoint{9.196148in}{1.671797in}}{\pgfqpoint{9.194144in}{1.676634in}}{\pgfqpoint{9.190577in}{1.680201in}}%
\pgfpathcurveto{\pgfqpoint{9.187011in}{1.683767in}}{\pgfqpoint{9.182173in}{1.685771in}}{\pgfqpoint{9.177129in}{1.685771in}}%
\pgfpathcurveto{\pgfqpoint{9.172086in}{1.685771in}}{\pgfqpoint{9.167248in}{1.683767in}}{\pgfqpoint{9.163682in}{1.680201in}}%
\pgfpathcurveto{\pgfqpoint{9.160115in}{1.676634in}}{\pgfqpoint{9.158111in}{1.671797in}}{\pgfqpoint{9.158111in}{1.666753in}}%
\pgfpathcurveto{\pgfqpoint{9.158111in}{1.661709in}}{\pgfqpoint{9.160115in}{1.656872in}}{\pgfqpoint{9.163682in}{1.653305in}}%
\pgfpathcurveto{\pgfqpoint{9.167248in}{1.649739in}}{\pgfqpoint{9.172086in}{1.647735in}}{\pgfqpoint{9.177129in}{1.647735in}}%
\pgfpathclose%
\pgfusepath{fill}%
\end{pgfscope}%
\begin{pgfscope}%
\pgfpathrectangle{\pgfqpoint{6.572727in}{0.473000in}}{\pgfqpoint{4.227273in}{3.311000in}}%
\pgfusepath{clip}%
\pgfsetbuttcap%
\pgfsetroundjoin%
\definecolor{currentfill}{rgb}{0.993248,0.906157,0.143936}%
\pgfsetfillcolor{currentfill}%
\pgfsetfillopacity{0.700000}%
\pgfsetlinewidth{0.000000pt}%
\definecolor{currentstroke}{rgb}{0.000000,0.000000,0.000000}%
\pgfsetstrokecolor{currentstroke}%
\pgfsetstrokeopacity{0.700000}%
\pgfsetdash{}{0pt}%
\pgfpathmoveto{\pgfqpoint{9.986951in}{1.490334in}}%
\pgfpathcurveto{\pgfqpoint{9.991995in}{1.490334in}}{\pgfqpoint{9.996833in}{1.492337in}}{\pgfqpoint{10.000399in}{1.495904in}}%
\pgfpathcurveto{\pgfqpoint{10.003966in}{1.499470in}}{\pgfqpoint{10.005969in}{1.504308in}}{\pgfqpoint{10.005969in}{1.509352in}}%
\pgfpathcurveto{\pgfqpoint{10.005969in}{1.514395in}}{\pgfqpoint{10.003966in}{1.519233in}}{\pgfqpoint{10.000399in}{1.522800in}}%
\pgfpathcurveto{\pgfqpoint{9.996833in}{1.526366in}}{\pgfqpoint{9.991995in}{1.528370in}}{\pgfqpoint{9.986951in}{1.528370in}}%
\pgfpathcurveto{\pgfqpoint{9.981908in}{1.528370in}}{\pgfqpoint{9.977070in}{1.526366in}}{\pgfqpoint{9.973503in}{1.522800in}}%
\pgfpathcurveto{\pgfqpoint{9.969937in}{1.519233in}}{\pgfqpoint{9.967933in}{1.514395in}}{\pgfqpoint{9.967933in}{1.509352in}}%
\pgfpathcurveto{\pgfqpoint{9.967933in}{1.504308in}}{\pgfqpoint{9.969937in}{1.499470in}}{\pgfqpoint{9.973503in}{1.495904in}}%
\pgfpathcurveto{\pgfqpoint{9.977070in}{1.492337in}}{\pgfqpoint{9.981908in}{1.490334in}}{\pgfqpoint{9.986951in}{1.490334in}}%
\pgfpathclose%
\pgfusepath{fill}%
\end{pgfscope}%
\begin{pgfscope}%
\pgfpathrectangle{\pgfqpoint{6.572727in}{0.473000in}}{\pgfqpoint{4.227273in}{3.311000in}}%
\pgfusepath{clip}%
\pgfsetbuttcap%
\pgfsetroundjoin%
\definecolor{currentfill}{rgb}{0.993248,0.906157,0.143936}%
\pgfsetfillcolor{currentfill}%
\pgfsetfillopacity{0.700000}%
\pgfsetlinewidth{0.000000pt}%
\definecolor{currentstroke}{rgb}{0.000000,0.000000,0.000000}%
\pgfsetstrokecolor{currentstroke}%
\pgfsetstrokeopacity{0.700000}%
\pgfsetdash{}{0pt}%
\pgfpathmoveto{\pgfqpoint{9.387374in}{1.173308in}}%
\pgfpathcurveto{\pgfqpoint{9.392418in}{1.173308in}}{\pgfqpoint{9.397256in}{1.175312in}}{\pgfqpoint{9.400822in}{1.178878in}}%
\pgfpathcurveto{\pgfqpoint{9.404389in}{1.182444in}}{\pgfqpoint{9.406393in}{1.187282in}}{\pgfqpoint{9.406393in}{1.192326in}}%
\pgfpathcurveto{\pgfqpoint{9.406393in}{1.197370in}}{\pgfqpoint{9.404389in}{1.202207in}}{\pgfqpoint{9.400822in}{1.205774in}}%
\pgfpathcurveto{\pgfqpoint{9.397256in}{1.209340in}}{\pgfqpoint{9.392418in}{1.211344in}}{\pgfqpoint{9.387374in}{1.211344in}}%
\pgfpathcurveto{\pgfqpoint{9.382331in}{1.211344in}}{\pgfqpoint{9.377493in}{1.209340in}}{\pgfqpoint{9.373927in}{1.205774in}}%
\pgfpathcurveto{\pgfqpoint{9.370360in}{1.202207in}}{\pgfqpoint{9.368356in}{1.197370in}}{\pgfqpoint{9.368356in}{1.192326in}}%
\pgfpathcurveto{\pgfqpoint{9.368356in}{1.187282in}}{\pgfqpoint{9.370360in}{1.182444in}}{\pgfqpoint{9.373927in}{1.178878in}}%
\pgfpathcurveto{\pgfqpoint{9.377493in}{1.175312in}}{\pgfqpoint{9.382331in}{1.173308in}}{\pgfqpoint{9.387374in}{1.173308in}}%
\pgfpathclose%
\pgfusepath{fill}%
\end{pgfscope}%
\begin{pgfscope}%
\pgfpathrectangle{\pgfqpoint{6.572727in}{0.473000in}}{\pgfqpoint{4.227273in}{3.311000in}}%
\pgfusepath{clip}%
\pgfsetbuttcap%
\pgfsetroundjoin%
\definecolor{currentfill}{rgb}{0.127568,0.566949,0.550556}%
\pgfsetfillcolor{currentfill}%
\pgfsetfillopacity{0.700000}%
\pgfsetlinewidth{0.000000pt}%
\definecolor{currentstroke}{rgb}{0.000000,0.000000,0.000000}%
\pgfsetstrokecolor{currentstroke}%
\pgfsetstrokeopacity{0.700000}%
\pgfsetdash{}{0pt}%
\pgfpathmoveto{\pgfqpoint{8.347673in}{2.774545in}}%
\pgfpathcurveto{\pgfqpoint{8.352717in}{2.774545in}}{\pgfqpoint{8.357555in}{2.776549in}}{\pgfqpoint{8.361121in}{2.780115in}}%
\pgfpathcurveto{\pgfqpoint{8.364688in}{2.783682in}}{\pgfqpoint{8.366692in}{2.788520in}}{\pgfqpoint{8.366692in}{2.793563in}}%
\pgfpathcurveto{\pgfqpoint{8.366692in}{2.798607in}}{\pgfqpoint{8.364688in}{2.803445in}}{\pgfqpoint{8.361121in}{2.807011in}}%
\pgfpathcurveto{\pgfqpoint{8.357555in}{2.810577in}}{\pgfqpoint{8.352717in}{2.812581in}}{\pgfqpoint{8.347673in}{2.812581in}}%
\pgfpathcurveto{\pgfqpoint{8.342630in}{2.812581in}}{\pgfqpoint{8.337792in}{2.810577in}}{\pgfqpoint{8.334226in}{2.807011in}}%
\pgfpathcurveto{\pgfqpoint{8.330659in}{2.803445in}}{\pgfqpoint{8.328655in}{2.798607in}}{\pgfqpoint{8.328655in}{2.793563in}}%
\pgfpathcurveto{\pgfqpoint{8.328655in}{2.788520in}}{\pgfqpoint{8.330659in}{2.783682in}}{\pgfqpoint{8.334226in}{2.780115in}}%
\pgfpathcurveto{\pgfqpoint{8.337792in}{2.776549in}}{\pgfqpoint{8.342630in}{2.774545in}}{\pgfqpoint{8.347673in}{2.774545in}}%
\pgfpathclose%
\pgfusepath{fill}%
\end{pgfscope}%
\begin{pgfscope}%
\pgfpathrectangle{\pgfqpoint{6.572727in}{0.473000in}}{\pgfqpoint{4.227273in}{3.311000in}}%
\pgfusepath{clip}%
\pgfsetbuttcap%
\pgfsetroundjoin%
\definecolor{currentfill}{rgb}{0.993248,0.906157,0.143936}%
\pgfsetfillcolor{currentfill}%
\pgfsetfillopacity{0.700000}%
\pgfsetlinewidth{0.000000pt}%
\definecolor{currentstroke}{rgb}{0.000000,0.000000,0.000000}%
\pgfsetstrokecolor{currentstroke}%
\pgfsetstrokeopacity{0.700000}%
\pgfsetdash{}{0pt}%
\pgfpathmoveto{\pgfqpoint{9.489269in}{1.813434in}}%
\pgfpathcurveto{\pgfqpoint{9.494313in}{1.813434in}}{\pgfqpoint{9.499151in}{1.815438in}}{\pgfqpoint{9.502717in}{1.819004in}}%
\pgfpathcurveto{\pgfqpoint{9.506284in}{1.822571in}}{\pgfqpoint{9.508288in}{1.827409in}}{\pgfqpoint{9.508288in}{1.832452in}}%
\pgfpathcurveto{\pgfqpoint{9.508288in}{1.837496in}}{\pgfqpoint{9.506284in}{1.842334in}}{\pgfqpoint{9.502717in}{1.845900in}}%
\pgfpathcurveto{\pgfqpoint{9.499151in}{1.849467in}}{\pgfqpoint{9.494313in}{1.851470in}}{\pgfqpoint{9.489269in}{1.851470in}}%
\pgfpathcurveto{\pgfqpoint{9.484226in}{1.851470in}}{\pgfqpoint{9.479388in}{1.849467in}}{\pgfqpoint{9.475822in}{1.845900in}}%
\pgfpathcurveto{\pgfqpoint{9.472255in}{1.842334in}}{\pgfqpoint{9.470251in}{1.837496in}}{\pgfqpoint{9.470251in}{1.832452in}}%
\pgfpathcurveto{\pgfqpoint{9.470251in}{1.827409in}}{\pgfqpoint{9.472255in}{1.822571in}}{\pgfqpoint{9.475822in}{1.819004in}}%
\pgfpathcurveto{\pgfqpoint{9.479388in}{1.815438in}}{\pgfqpoint{9.484226in}{1.813434in}}{\pgfqpoint{9.489269in}{1.813434in}}%
\pgfpathclose%
\pgfusepath{fill}%
\end{pgfscope}%
\begin{pgfscope}%
\pgfpathrectangle{\pgfqpoint{6.572727in}{0.473000in}}{\pgfqpoint{4.227273in}{3.311000in}}%
\pgfusepath{clip}%
\pgfsetbuttcap%
\pgfsetroundjoin%
\definecolor{currentfill}{rgb}{0.127568,0.566949,0.550556}%
\pgfsetfillcolor{currentfill}%
\pgfsetfillopacity{0.700000}%
\pgfsetlinewidth{0.000000pt}%
\definecolor{currentstroke}{rgb}{0.000000,0.000000,0.000000}%
\pgfsetstrokecolor{currentstroke}%
\pgfsetstrokeopacity{0.700000}%
\pgfsetdash{}{0pt}%
\pgfpathmoveto{\pgfqpoint{7.928335in}{3.446405in}}%
\pgfpathcurveto{\pgfqpoint{7.933379in}{3.446405in}}{\pgfqpoint{7.938216in}{3.448409in}}{\pgfqpoint{7.941783in}{3.451975in}}%
\pgfpathcurveto{\pgfqpoint{7.945349in}{3.455542in}}{\pgfqpoint{7.947353in}{3.460379in}}{\pgfqpoint{7.947353in}{3.465423in}}%
\pgfpathcurveto{\pgfqpoint{7.947353in}{3.470467in}}{\pgfqpoint{7.945349in}{3.475304in}}{\pgfqpoint{7.941783in}{3.478871in}}%
\pgfpathcurveto{\pgfqpoint{7.938216in}{3.482437in}}{\pgfqpoint{7.933379in}{3.484441in}}{\pgfqpoint{7.928335in}{3.484441in}}%
\pgfpathcurveto{\pgfqpoint{7.923291in}{3.484441in}}{\pgfqpoint{7.918454in}{3.482437in}}{\pgfqpoint{7.914887in}{3.478871in}}%
\pgfpathcurveto{\pgfqpoint{7.911321in}{3.475304in}}{\pgfqpoint{7.909317in}{3.470467in}}{\pgfqpoint{7.909317in}{3.465423in}}%
\pgfpathcurveto{\pgfqpoint{7.909317in}{3.460379in}}{\pgfqpoint{7.911321in}{3.455542in}}{\pgfqpoint{7.914887in}{3.451975in}}%
\pgfpathcurveto{\pgfqpoint{7.918454in}{3.448409in}}{\pgfqpoint{7.923291in}{3.446405in}}{\pgfqpoint{7.928335in}{3.446405in}}%
\pgfpathclose%
\pgfusepath{fill}%
\end{pgfscope}%
\begin{pgfscope}%
\pgfpathrectangle{\pgfqpoint{6.572727in}{0.473000in}}{\pgfqpoint{4.227273in}{3.311000in}}%
\pgfusepath{clip}%
\pgfsetbuttcap%
\pgfsetroundjoin%
\definecolor{currentfill}{rgb}{0.127568,0.566949,0.550556}%
\pgfsetfillcolor{currentfill}%
\pgfsetfillopacity{0.700000}%
\pgfsetlinewidth{0.000000pt}%
\definecolor{currentstroke}{rgb}{0.000000,0.000000,0.000000}%
\pgfsetstrokecolor{currentstroke}%
\pgfsetstrokeopacity{0.700000}%
\pgfsetdash{}{0pt}%
\pgfpathmoveto{\pgfqpoint{8.035604in}{2.897680in}}%
\pgfpathcurveto{\pgfqpoint{8.040648in}{2.897680in}}{\pgfqpoint{8.045486in}{2.899684in}}{\pgfqpoint{8.049052in}{2.903250in}}%
\pgfpathcurveto{\pgfqpoint{8.052619in}{2.906817in}}{\pgfqpoint{8.054623in}{2.911654in}}{\pgfqpoint{8.054623in}{2.916698in}}%
\pgfpathcurveto{\pgfqpoint{8.054623in}{2.921742in}}{\pgfqpoint{8.052619in}{2.926579in}}{\pgfqpoint{8.049052in}{2.930146in}}%
\pgfpathcurveto{\pgfqpoint{8.045486in}{2.933712in}}{\pgfqpoint{8.040648in}{2.935716in}}{\pgfqpoint{8.035604in}{2.935716in}}%
\pgfpathcurveto{\pgfqpoint{8.030561in}{2.935716in}}{\pgfqpoint{8.025723in}{2.933712in}}{\pgfqpoint{8.022157in}{2.930146in}}%
\pgfpathcurveto{\pgfqpoint{8.018590in}{2.926579in}}{\pgfqpoint{8.016586in}{2.921742in}}{\pgfqpoint{8.016586in}{2.916698in}}%
\pgfpathcurveto{\pgfqpoint{8.016586in}{2.911654in}}{\pgfqpoint{8.018590in}{2.906817in}}{\pgfqpoint{8.022157in}{2.903250in}}%
\pgfpathcurveto{\pgfqpoint{8.025723in}{2.899684in}}{\pgfqpoint{8.030561in}{2.897680in}}{\pgfqpoint{8.035604in}{2.897680in}}%
\pgfpathclose%
\pgfusepath{fill}%
\end{pgfscope}%
\begin{pgfscope}%
\pgfpathrectangle{\pgfqpoint{6.572727in}{0.473000in}}{\pgfqpoint{4.227273in}{3.311000in}}%
\pgfusepath{clip}%
\pgfsetbuttcap%
\pgfsetroundjoin%
\definecolor{currentfill}{rgb}{0.127568,0.566949,0.550556}%
\pgfsetfillcolor{currentfill}%
\pgfsetfillopacity{0.700000}%
\pgfsetlinewidth{0.000000pt}%
\definecolor{currentstroke}{rgb}{0.000000,0.000000,0.000000}%
\pgfsetstrokecolor{currentstroke}%
\pgfsetstrokeopacity{0.700000}%
\pgfsetdash{}{0pt}%
\pgfpathmoveto{\pgfqpoint{7.868367in}{2.017545in}}%
\pgfpathcurveto{\pgfqpoint{7.873410in}{2.017545in}}{\pgfqpoint{7.878248in}{2.019548in}}{\pgfqpoint{7.881814in}{2.023115in}}%
\pgfpathcurveto{\pgfqpoint{7.885381in}{2.026681in}}{\pgfqpoint{7.887385in}{2.031519in}}{\pgfqpoint{7.887385in}{2.036563in}}%
\pgfpathcurveto{\pgfqpoint{7.887385in}{2.041606in}}{\pgfqpoint{7.885381in}{2.046444in}}{\pgfqpoint{7.881814in}{2.050011in}}%
\pgfpathcurveto{\pgfqpoint{7.878248in}{2.053577in}}{\pgfqpoint{7.873410in}{2.055581in}}{\pgfqpoint{7.868367in}{2.055581in}}%
\pgfpathcurveto{\pgfqpoint{7.863323in}{2.055581in}}{\pgfqpoint{7.858485in}{2.053577in}}{\pgfqpoint{7.854919in}{2.050011in}}%
\pgfpathcurveto{\pgfqpoint{7.851352in}{2.046444in}}{\pgfqpoint{7.849348in}{2.041606in}}{\pgfqpoint{7.849348in}{2.036563in}}%
\pgfpathcurveto{\pgfqpoint{7.849348in}{2.031519in}}{\pgfqpoint{7.851352in}{2.026681in}}{\pgfqpoint{7.854919in}{2.023115in}}%
\pgfpathcurveto{\pgfqpoint{7.858485in}{2.019548in}}{\pgfqpoint{7.863323in}{2.017545in}}{\pgfqpoint{7.868367in}{2.017545in}}%
\pgfpathclose%
\pgfusepath{fill}%
\end{pgfscope}%
\begin{pgfscope}%
\pgfpathrectangle{\pgfqpoint{6.572727in}{0.473000in}}{\pgfqpoint{4.227273in}{3.311000in}}%
\pgfusepath{clip}%
\pgfsetbuttcap%
\pgfsetroundjoin%
\definecolor{currentfill}{rgb}{0.127568,0.566949,0.550556}%
\pgfsetfillcolor{currentfill}%
\pgfsetfillopacity{0.700000}%
\pgfsetlinewidth{0.000000pt}%
\definecolor{currentstroke}{rgb}{0.000000,0.000000,0.000000}%
\pgfsetstrokecolor{currentstroke}%
\pgfsetstrokeopacity{0.700000}%
\pgfsetdash{}{0pt}%
\pgfpathmoveto{\pgfqpoint{7.754510in}{2.356830in}}%
\pgfpathcurveto{\pgfqpoint{7.759553in}{2.356830in}}{\pgfqpoint{7.764391in}{2.358834in}}{\pgfqpoint{7.767958in}{2.362400in}}%
\pgfpathcurveto{\pgfqpoint{7.771524in}{2.365967in}}{\pgfqpoint{7.773528in}{2.370804in}}{\pgfqpoint{7.773528in}{2.375848in}}%
\pgfpathcurveto{\pgfqpoint{7.773528in}{2.380892in}}{\pgfqpoint{7.771524in}{2.385729in}}{\pgfqpoint{7.767958in}{2.389296in}}%
\pgfpathcurveto{\pgfqpoint{7.764391in}{2.392862in}}{\pgfqpoint{7.759553in}{2.394866in}}{\pgfqpoint{7.754510in}{2.394866in}}%
\pgfpathcurveto{\pgfqpoint{7.749466in}{2.394866in}}{\pgfqpoint{7.744628in}{2.392862in}}{\pgfqpoint{7.741062in}{2.389296in}}%
\pgfpathcurveto{\pgfqpoint{7.737495in}{2.385729in}}{\pgfqpoint{7.735492in}{2.380892in}}{\pgfqpoint{7.735492in}{2.375848in}}%
\pgfpathcurveto{\pgfqpoint{7.735492in}{2.370804in}}{\pgfqpoint{7.737495in}{2.365967in}}{\pgfqpoint{7.741062in}{2.362400in}}%
\pgfpathcurveto{\pgfqpoint{7.744628in}{2.358834in}}{\pgfqpoint{7.749466in}{2.356830in}}{\pgfqpoint{7.754510in}{2.356830in}}%
\pgfpathclose%
\pgfusepath{fill}%
\end{pgfscope}%
\begin{pgfscope}%
\pgfpathrectangle{\pgfqpoint{6.572727in}{0.473000in}}{\pgfqpoint{4.227273in}{3.311000in}}%
\pgfusepath{clip}%
\pgfsetbuttcap%
\pgfsetroundjoin%
\definecolor{currentfill}{rgb}{0.127568,0.566949,0.550556}%
\pgfsetfillcolor{currentfill}%
\pgfsetfillopacity{0.700000}%
\pgfsetlinewidth{0.000000pt}%
\definecolor{currentstroke}{rgb}{0.000000,0.000000,0.000000}%
\pgfsetstrokecolor{currentstroke}%
\pgfsetstrokeopacity{0.700000}%
\pgfsetdash{}{0pt}%
\pgfpathmoveto{\pgfqpoint{8.084904in}{3.426063in}}%
\pgfpathcurveto{\pgfqpoint{8.089947in}{3.426063in}}{\pgfqpoint{8.094785in}{3.428067in}}{\pgfqpoint{8.098352in}{3.431634in}}%
\pgfpathcurveto{\pgfqpoint{8.101918in}{3.435200in}}{\pgfqpoint{8.103922in}{3.440038in}}{\pgfqpoint{8.103922in}{3.445082in}}%
\pgfpathcurveto{\pgfqpoint{8.103922in}{3.450125in}}{\pgfqpoint{8.101918in}{3.454963in}}{\pgfqpoint{8.098352in}{3.458529in}}%
\pgfpathcurveto{\pgfqpoint{8.094785in}{3.462096in}}{\pgfqpoint{8.089947in}{3.464100in}}{\pgfqpoint{8.084904in}{3.464100in}}%
\pgfpathcurveto{\pgfqpoint{8.079860in}{3.464100in}}{\pgfqpoint{8.075022in}{3.462096in}}{\pgfqpoint{8.071456in}{3.458529in}}%
\pgfpathcurveto{\pgfqpoint{8.067889in}{3.454963in}}{\pgfqpoint{8.065886in}{3.450125in}}{\pgfqpoint{8.065886in}{3.445082in}}%
\pgfpathcurveto{\pgfqpoint{8.065886in}{3.440038in}}{\pgfqpoint{8.067889in}{3.435200in}}{\pgfqpoint{8.071456in}{3.431634in}}%
\pgfpathcurveto{\pgfqpoint{8.075022in}{3.428067in}}{\pgfqpoint{8.079860in}{3.426063in}}{\pgfqpoint{8.084904in}{3.426063in}}%
\pgfpathclose%
\pgfusepath{fill}%
\end{pgfscope}%
\begin{pgfscope}%
\pgfpathrectangle{\pgfqpoint{6.572727in}{0.473000in}}{\pgfqpoint{4.227273in}{3.311000in}}%
\pgfusepath{clip}%
\pgfsetbuttcap%
\pgfsetroundjoin%
\definecolor{currentfill}{rgb}{0.127568,0.566949,0.550556}%
\pgfsetfillcolor{currentfill}%
\pgfsetfillopacity{0.700000}%
\pgfsetlinewidth{0.000000pt}%
\definecolor{currentstroke}{rgb}{0.000000,0.000000,0.000000}%
\pgfsetstrokecolor{currentstroke}%
\pgfsetstrokeopacity{0.700000}%
\pgfsetdash{}{0pt}%
\pgfpathmoveto{\pgfqpoint{7.821623in}{1.680403in}}%
\pgfpathcurveto{\pgfqpoint{7.826667in}{1.680403in}}{\pgfqpoint{7.831504in}{1.682407in}}{\pgfqpoint{7.835071in}{1.685973in}}%
\pgfpathcurveto{\pgfqpoint{7.838637in}{1.689540in}}{\pgfqpoint{7.840641in}{1.694378in}}{\pgfqpoint{7.840641in}{1.699421in}}%
\pgfpathcurveto{\pgfqpoint{7.840641in}{1.704465in}}{\pgfqpoint{7.838637in}{1.709303in}}{\pgfqpoint{7.835071in}{1.712869in}}%
\pgfpathcurveto{\pgfqpoint{7.831504in}{1.716435in}}{\pgfqpoint{7.826667in}{1.718439in}}{\pgfqpoint{7.821623in}{1.718439in}}%
\pgfpathcurveto{\pgfqpoint{7.816579in}{1.718439in}}{\pgfqpoint{7.811742in}{1.716435in}}{\pgfqpoint{7.808175in}{1.712869in}}%
\pgfpathcurveto{\pgfqpoint{7.804609in}{1.709303in}}{\pgfqpoint{7.802605in}{1.704465in}}{\pgfqpoint{7.802605in}{1.699421in}}%
\pgfpathcurveto{\pgfqpoint{7.802605in}{1.694378in}}{\pgfqpoint{7.804609in}{1.689540in}}{\pgfqpoint{7.808175in}{1.685973in}}%
\pgfpathcurveto{\pgfqpoint{7.811742in}{1.682407in}}{\pgfqpoint{7.816579in}{1.680403in}}{\pgfqpoint{7.821623in}{1.680403in}}%
\pgfpathclose%
\pgfusepath{fill}%
\end{pgfscope}%
\begin{pgfscope}%
\pgfpathrectangle{\pgfqpoint{6.572727in}{0.473000in}}{\pgfqpoint{4.227273in}{3.311000in}}%
\pgfusepath{clip}%
\pgfsetbuttcap%
\pgfsetroundjoin%
\definecolor{currentfill}{rgb}{0.127568,0.566949,0.550556}%
\pgfsetfillcolor{currentfill}%
\pgfsetfillopacity{0.700000}%
\pgfsetlinewidth{0.000000pt}%
\definecolor{currentstroke}{rgb}{0.000000,0.000000,0.000000}%
\pgfsetstrokecolor{currentstroke}%
\pgfsetstrokeopacity{0.700000}%
\pgfsetdash{}{0pt}%
\pgfpathmoveto{\pgfqpoint{7.861638in}{1.655752in}}%
\pgfpathcurveto{\pgfqpoint{7.866682in}{1.655752in}}{\pgfqpoint{7.871519in}{1.657756in}}{\pgfqpoint{7.875086in}{1.661322in}}%
\pgfpathcurveto{\pgfqpoint{7.878652in}{1.664888in}}{\pgfqpoint{7.880656in}{1.669726in}}{\pgfqpoint{7.880656in}{1.674770in}}%
\pgfpathcurveto{\pgfqpoint{7.880656in}{1.679814in}}{\pgfqpoint{7.878652in}{1.684651in}}{\pgfqpoint{7.875086in}{1.688218in}}%
\pgfpathcurveto{\pgfqpoint{7.871519in}{1.691784in}}{\pgfqpoint{7.866682in}{1.693788in}}{\pgfqpoint{7.861638in}{1.693788in}}%
\pgfpathcurveto{\pgfqpoint{7.856594in}{1.693788in}}{\pgfqpoint{7.851756in}{1.691784in}}{\pgfqpoint{7.848190in}{1.688218in}}%
\pgfpathcurveto{\pgfqpoint{7.844624in}{1.684651in}}{\pgfqpoint{7.842620in}{1.679814in}}{\pgfqpoint{7.842620in}{1.674770in}}%
\pgfpathcurveto{\pgfqpoint{7.842620in}{1.669726in}}{\pgfqpoint{7.844624in}{1.664888in}}{\pgfqpoint{7.848190in}{1.661322in}}%
\pgfpathcurveto{\pgfqpoint{7.851756in}{1.657756in}}{\pgfqpoint{7.856594in}{1.655752in}}{\pgfqpoint{7.861638in}{1.655752in}}%
\pgfpathclose%
\pgfusepath{fill}%
\end{pgfscope}%
\begin{pgfscope}%
\pgfpathrectangle{\pgfqpoint{6.572727in}{0.473000in}}{\pgfqpoint{4.227273in}{3.311000in}}%
\pgfusepath{clip}%
\pgfsetbuttcap%
\pgfsetroundjoin%
\definecolor{currentfill}{rgb}{0.993248,0.906157,0.143936}%
\pgfsetfillcolor{currentfill}%
\pgfsetfillopacity{0.700000}%
\pgfsetlinewidth{0.000000pt}%
\definecolor{currentstroke}{rgb}{0.000000,0.000000,0.000000}%
\pgfsetstrokecolor{currentstroke}%
\pgfsetstrokeopacity{0.700000}%
\pgfsetdash{}{0pt}%
\pgfpathmoveto{\pgfqpoint{9.457914in}{0.969270in}}%
\pgfpathcurveto{\pgfqpoint{9.462957in}{0.969270in}}{\pgfqpoint{9.467795in}{0.971274in}}{\pgfqpoint{9.471361in}{0.974841in}}%
\pgfpathcurveto{\pgfqpoint{9.474928in}{0.978407in}}{\pgfqpoint{9.476932in}{0.983245in}}{\pgfqpoint{9.476932in}{0.988288in}}%
\pgfpathcurveto{\pgfqpoint{9.476932in}{0.993332in}}{\pgfqpoint{9.474928in}{0.998170in}}{\pgfqpoint{9.471361in}{1.001736in}}%
\pgfpathcurveto{\pgfqpoint{9.467795in}{1.005303in}}{\pgfqpoint{9.462957in}{1.007307in}}{\pgfqpoint{9.457914in}{1.007307in}}%
\pgfpathcurveto{\pgfqpoint{9.452870in}{1.007307in}}{\pgfqpoint{9.448032in}{1.005303in}}{\pgfqpoint{9.444466in}{1.001736in}}%
\pgfpathcurveto{\pgfqpoint{9.440899in}{0.998170in}}{\pgfqpoint{9.438895in}{0.993332in}}{\pgfqpoint{9.438895in}{0.988288in}}%
\pgfpathcurveto{\pgfqpoint{9.438895in}{0.983245in}}{\pgfqpoint{9.440899in}{0.978407in}}{\pgfqpoint{9.444466in}{0.974841in}}%
\pgfpathcurveto{\pgfqpoint{9.448032in}{0.971274in}}{\pgfqpoint{9.452870in}{0.969270in}}{\pgfqpoint{9.457914in}{0.969270in}}%
\pgfpathclose%
\pgfusepath{fill}%
\end{pgfscope}%
\begin{pgfscope}%
\pgfpathrectangle{\pgfqpoint{6.572727in}{0.473000in}}{\pgfqpoint{4.227273in}{3.311000in}}%
\pgfusepath{clip}%
\pgfsetbuttcap%
\pgfsetroundjoin%
\definecolor{currentfill}{rgb}{0.127568,0.566949,0.550556}%
\pgfsetfillcolor{currentfill}%
\pgfsetfillopacity{0.700000}%
\pgfsetlinewidth{0.000000pt}%
\definecolor{currentstroke}{rgb}{0.000000,0.000000,0.000000}%
\pgfsetstrokecolor{currentstroke}%
\pgfsetstrokeopacity{0.700000}%
\pgfsetdash{}{0pt}%
\pgfpathmoveto{\pgfqpoint{8.546718in}{2.809153in}}%
\pgfpathcurveto{\pgfqpoint{8.551761in}{2.809153in}}{\pgfqpoint{8.556599in}{2.811156in}}{\pgfqpoint{8.560166in}{2.814723in}}%
\pgfpathcurveto{\pgfqpoint{8.563732in}{2.818289in}}{\pgfqpoint{8.565736in}{2.823127in}}{\pgfqpoint{8.565736in}{2.828171in}}%
\pgfpathcurveto{\pgfqpoint{8.565736in}{2.833214in}}{\pgfqpoint{8.563732in}{2.838052in}}{\pgfqpoint{8.560166in}{2.841619in}}%
\pgfpathcurveto{\pgfqpoint{8.556599in}{2.845185in}}{\pgfqpoint{8.551761in}{2.847189in}}{\pgfqpoint{8.546718in}{2.847189in}}%
\pgfpathcurveto{\pgfqpoint{8.541674in}{2.847189in}}{\pgfqpoint{8.536836in}{2.845185in}}{\pgfqpoint{8.533270in}{2.841619in}}%
\pgfpathcurveto{\pgfqpoint{8.529703in}{2.838052in}}{\pgfqpoint{8.527700in}{2.833214in}}{\pgfqpoint{8.527700in}{2.828171in}}%
\pgfpathcurveto{\pgfqpoint{8.527700in}{2.823127in}}{\pgfqpoint{8.529703in}{2.818289in}}{\pgfqpoint{8.533270in}{2.814723in}}%
\pgfpathcurveto{\pgfqpoint{8.536836in}{2.811156in}}{\pgfqpoint{8.541674in}{2.809153in}}{\pgfqpoint{8.546718in}{2.809153in}}%
\pgfpathclose%
\pgfusepath{fill}%
\end{pgfscope}%
\begin{pgfscope}%
\pgfpathrectangle{\pgfqpoint{6.572727in}{0.473000in}}{\pgfqpoint{4.227273in}{3.311000in}}%
\pgfusepath{clip}%
\pgfsetbuttcap%
\pgfsetroundjoin%
\definecolor{currentfill}{rgb}{0.127568,0.566949,0.550556}%
\pgfsetfillcolor{currentfill}%
\pgfsetfillopacity{0.700000}%
\pgfsetlinewidth{0.000000pt}%
\definecolor{currentstroke}{rgb}{0.000000,0.000000,0.000000}%
\pgfsetstrokecolor{currentstroke}%
\pgfsetstrokeopacity{0.700000}%
\pgfsetdash{}{0pt}%
\pgfpathmoveto{\pgfqpoint{8.045324in}{2.946574in}}%
\pgfpathcurveto{\pgfqpoint{8.050368in}{2.946574in}}{\pgfqpoint{8.055206in}{2.948578in}}{\pgfqpoint{8.058772in}{2.952144in}}%
\pgfpathcurveto{\pgfqpoint{8.062339in}{2.955710in}}{\pgfqpoint{8.064342in}{2.960548in}}{\pgfqpoint{8.064342in}{2.965592in}}%
\pgfpathcurveto{\pgfqpoint{8.064342in}{2.970636in}}{\pgfqpoint{8.062339in}{2.975473in}}{\pgfqpoint{8.058772in}{2.979040in}}%
\pgfpathcurveto{\pgfqpoint{8.055206in}{2.982606in}}{\pgfqpoint{8.050368in}{2.984610in}}{\pgfqpoint{8.045324in}{2.984610in}}%
\pgfpathcurveto{\pgfqpoint{8.040281in}{2.984610in}}{\pgfqpoint{8.035443in}{2.982606in}}{\pgfqpoint{8.031876in}{2.979040in}}%
\pgfpathcurveto{\pgfqpoint{8.028310in}{2.975473in}}{\pgfqpoint{8.026306in}{2.970636in}}{\pgfqpoint{8.026306in}{2.965592in}}%
\pgfpathcurveto{\pgfqpoint{8.026306in}{2.960548in}}{\pgfqpoint{8.028310in}{2.955710in}}{\pgfqpoint{8.031876in}{2.952144in}}%
\pgfpathcurveto{\pgfqpoint{8.035443in}{2.948578in}}{\pgfqpoint{8.040281in}{2.946574in}}{\pgfqpoint{8.045324in}{2.946574in}}%
\pgfpathclose%
\pgfusepath{fill}%
\end{pgfscope}%
\begin{pgfscope}%
\pgfpathrectangle{\pgfqpoint{6.572727in}{0.473000in}}{\pgfqpoint{4.227273in}{3.311000in}}%
\pgfusepath{clip}%
\pgfsetbuttcap%
\pgfsetroundjoin%
\definecolor{currentfill}{rgb}{0.127568,0.566949,0.550556}%
\pgfsetfillcolor{currentfill}%
\pgfsetfillopacity{0.700000}%
\pgfsetlinewidth{0.000000pt}%
\definecolor{currentstroke}{rgb}{0.000000,0.000000,0.000000}%
\pgfsetstrokecolor{currentstroke}%
\pgfsetstrokeopacity{0.700000}%
\pgfsetdash{}{0pt}%
\pgfpathmoveto{\pgfqpoint{8.499750in}{2.753996in}}%
\pgfpathcurveto{\pgfqpoint{8.504794in}{2.753996in}}{\pgfqpoint{8.509632in}{2.756000in}}{\pgfqpoint{8.513198in}{2.759566in}}%
\pgfpathcurveto{\pgfqpoint{8.516765in}{2.763132in}}{\pgfqpoint{8.518769in}{2.767970in}}{\pgfqpoint{8.518769in}{2.773014in}}%
\pgfpathcurveto{\pgfqpoint{8.518769in}{2.778057in}}{\pgfqpoint{8.516765in}{2.782895in}}{\pgfqpoint{8.513198in}{2.786462in}}%
\pgfpathcurveto{\pgfqpoint{8.509632in}{2.790028in}}{\pgfqpoint{8.504794in}{2.792032in}}{\pgfqpoint{8.499750in}{2.792032in}}%
\pgfpathcurveto{\pgfqpoint{8.494707in}{2.792032in}}{\pgfqpoint{8.489869in}{2.790028in}}{\pgfqpoint{8.486303in}{2.786462in}}%
\pgfpathcurveto{\pgfqpoint{8.482736in}{2.782895in}}{\pgfqpoint{8.480732in}{2.778057in}}{\pgfqpoint{8.480732in}{2.773014in}}%
\pgfpathcurveto{\pgfqpoint{8.480732in}{2.767970in}}{\pgfqpoint{8.482736in}{2.763132in}}{\pgfqpoint{8.486303in}{2.759566in}}%
\pgfpathcurveto{\pgfqpoint{8.489869in}{2.756000in}}{\pgfqpoint{8.494707in}{2.753996in}}{\pgfqpoint{8.499750in}{2.753996in}}%
\pgfpathclose%
\pgfusepath{fill}%
\end{pgfscope}%
\begin{pgfscope}%
\pgfpathrectangle{\pgfqpoint{6.572727in}{0.473000in}}{\pgfqpoint{4.227273in}{3.311000in}}%
\pgfusepath{clip}%
\pgfsetbuttcap%
\pgfsetroundjoin%
\definecolor{currentfill}{rgb}{0.127568,0.566949,0.550556}%
\pgfsetfillcolor{currentfill}%
\pgfsetfillopacity{0.700000}%
\pgfsetlinewidth{0.000000pt}%
\definecolor{currentstroke}{rgb}{0.000000,0.000000,0.000000}%
\pgfsetstrokecolor{currentstroke}%
\pgfsetstrokeopacity{0.700000}%
\pgfsetdash{}{0pt}%
\pgfpathmoveto{\pgfqpoint{8.114727in}{2.347384in}}%
\pgfpathcurveto{\pgfqpoint{8.119771in}{2.347384in}}{\pgfqpoint{8.124609in}{2.349388in}}{\pgfqpoint{8.128175in}{2.352954in}}%
\pgfpathcurveto{\pgfqpoint{8.131742in}{2.356521in}}{\pgfqpoint{8.133746in}{2.361359in}}{\pgfqpoint{8.133746in}{2.366402in}}%
\pgfpathcurveto{\pgfqpoint{8.133746in}{2.371446in}}{\pgfqpoint{8.131742in}{2.376284in}}{\pgfqpoint{8.128175in}{2.379850in}}%
\pgfpathcurveto{\pgfqpoint{8.124609in}{2.383417in}}{\pgfqpoint{8.119771in}{2.385420in}}{\pgfqpoint{8.114727in}{2.385420in}}%
\pgfpathcurveto{\pgfqpoint{8.109684in}{2.385420in}}{\pgfqpoint{8.104846in}{2.383417in}}{\pgfqpoint{8.101280in}{2.379850in}}%
\pgfpathcurveto{\pgfqpoint{8.097713in}{2.376284in}}{\pgfqpoint{8.095709in}{2.371446in}}{\pgfqpoint{8.095709in}{2.366402in}}%
\pgfpathcurveto{\pgfqpoint{8.095709in}{2.361359in}}{\pgfqpoint{8.097713in}{2.356521in}}{\pgfqpoint{8.101280in}{2.352954in}}%
\pgfpathcurveto{\pgfqpoint{8.104846in}{2.349388in}}{\pgfqpoint{8.109684in}{2.347384in}}{\pgfqpoint{8.114727in}{2.347384in}}%
\pgfpathclose%
\pgfusepath{fill}%
\end{pgfscope}%
\begin{pgfscope}%
\pgfpathrectangle{\pgfqpoint{6.572727in}{0.473000in}}{\pgfqpoint{4.227273in}{3.311000in}}%
\pgfusepath{clip}%
\pgfsetbuttcap%
\pgfsetroundjoin%
\definecolor{currentfill}{rgb}{0.127568,0.566949,0.550556}%
\pgfsetfillcolor{currentfill}%
\pgfsetfillopacity{0.700000}%
\pgfsetlinewidth{0.000000pt}%
\definecolor{currentstroke}{rgb}{0.000000,0.000000,0.000000}%
\pgfsetstrokecolor{currentstroke}%
\pgfsetstrokeopacity{0.700000}%
\pgfsetdash{}{0pt}%
\pgfpathmoveto{\pgfqpoint{7.721510in}{2.642339in}}%
\pgfpathcurveto{\pgfqpoint{7.726554in}{2.642339in}}{\pgfqpoint{7.731391in}{2.644343in}}{\pgfqpoint{7.734958in}{2.647909in}}%
\pgfpathcurveto{\pgfqpoint{7.738524in}{2.651476in}}{\pgfqpoint{7.740528in}{2.656313in}}{\pgfqpoint{7.740528in}{2.661357in}}%
\pgfpathcurveto{\pgfqpoint{7.740528in}{2.666401in}}{\pgfqpoint{7.738524in}{2.671238in}}{\pgfqpoint{7.734958in}{2.674805in}}%
\pgfpathcurveto{\pgfqpoint{7.731391in}{2.678371in}}{\pgfqpoint{7.726554in}{2.680375in}}{\pgfqpoint{7.721510in}{2.680375in}}%
\pgfpathcurveto{\pgfqpoint{7.716466in}{2.680375in}}{\pgfqpoint{7.711629in}{2.678371in}}{\pgfqpoint{7.708062in}{2.674805in}}%
\pgfpathcurveto{\pgfqpoint{7.704496in}{2.671238in}}{\pgfqpoint{7.702492in}{2.666401in}}{\pgfqpoint{7.702492in}{2.661357in}}%
\pgfpathcurveto{\pgfqpoint{7.702492in}{2.656313in}}{\pgfqpoint{7.704496in}{2.651476in}}{\pgfqpoint{7.708062in}{2.647909in}}%
\pgfpathcurveto{\pgfqpoint{7.711629in}{2.644343in}}{\pgfqpoint{7.716466in}{2.642339in}}{\pgfqpoint{7.721510in}{2.642339in}}%
\pgfpathclose%
\pgfusepath{fill}%
\end{pgfscope}%
\begin{pgfscope}%
\pgfpathrectangle{\pgfqpoint{6.572727in}{0.473000in}}{\pgfqpoint{4.227273in}{3.311000in}}%
\pgfusepath{clip}%
\pgfsetbuttcap%
\pgfsetroundjoin%
\definecolor{currentfill}{rgb}{0.993248,0.906157,0.143936}%
\pgfsetfillcolor{currentfill}%
\pgfsetfillopacity{0.700000}%
\pgfsetlinewidth{0.000000pt}%
\definecolor{currentstroke}{rgb}{0.000000,0.000000,0.000000}%
\pgfsetstrokecolor{currentstroke}%
\pgfsetstrokeopacity{0.700000}%
\pgfsetdash{}{0pt}%
\pgfpathmoveto{\pgfqpoint{9.075194in}{2.053340in}}%
\pgfpathcurveto{\pgfqpoint{9.080238in}{2.053340in}}{\pgfqpoint{9.085076in}{2.055344in}}{\pgfqpoint{9.088642in}{2.058910in}}%
\pgfpathcurveto{\pgfqpoint{9.092208in}{2.062477in}}{\pgfqpoint{9.094212in}{2.067315in}}{\pgfqpoint{9.094212in}{2.072358in}}%
\pgfpathcurveto{\pgfqpoint{9.094212in}{2.077402in}}{\pgfqpoint{9.092208in}{2.082240in}}{\pgfqpoint{9.088642in}{2.085806in}}%
\pgfpathcurveto{\pgfqpoint{9.085076in}{2.089373in}}{\pgfqpoint{9.080238in}{2.091376in}}{\pgfqpoint{9.075194in}{2.091376in}}%
\pgfpathcurveto{\pgfqpoint{9.070151in}{2.091376in}}{\pgfqpoint{9.065313in}{2.089373in}}{\pgfqpoint{9.061746in}{2.085806in}}%
\pgfpathcurveto{\pgfqpoint{9.058180in}{2.082240in}}{\pgfqpoint{9.056176in}{2.077402in}}{\pgfqpoint{9.056176in}{2.072358in}}%
\pgfpathcurveto{\pgfqpoint{9.056176in}{2.067315in}}{\pgfqpoint{9.058180in}{2.062477in}}{\pgfqpoint{9.061746in}{2.058910in}}%
\pgfpathcurveto{\pgfqpoint{9.065313in}{2.055344in}}{\pgfqpoint{9.070151in}{2.053340in}}{\pgfqpoint{9.075194in}{2.053340in}}%
\pgfpathclose%
\pgfusepath{fill}%
\end{pgfscope}%
\begin{pgfscope}%
\pgfpathrectangle{\pgfqpoint{6.572727in}{0.473000in}}{\pgfqpoint{4.227273in}{3.311000in}}%
\pgfusepath{clip}%
\pgfsetbuttcap%
\pgfsetroundjoin%
\definecolor{currentfill}{rgb}{0.993248,0.906157,0.143936}%
\pgfsetfillcolor{currentfill}%
\pgfsetfillopacity{0.700000}%
\pgfsetlinewidth{0.000000pt}%
\definecolor{currentstroke}{rgb}{0.000000,0.000000,0.000000}%
\pgfsetstrokecolor{currentstroke}%
\pgfsetstrokeopacity{0.700000}%
\pgfsetdash{}{0pt}%
\pgfpathmoveto{\pgfqpoint{10.148880in}{2.041990in}}%
\pgfpathcurveto{\pgfqpoint{10.153924in}{2.041990in}}{\pgfqpoint{10.158762in}{2.043994in}}{\pgfqpoint{10.162328in}{2.047560in}}%
\pgfpathcurveto{\pgfqpoint{10.165894in}{2.051127in}}{\pgfqpoint{10.167898in}{2.055965in}}{\pgfqpoint{10.167898in}{2.061008in}}%
\pgfpathcurveto{\pgfqpoint{10.167898in}{2.066052in}}{\pgfqpoint{10.165894in}{2.070890in}}{\pgfqpoint{10.162328in}{2.074456in}}%
\pgfpathcurveto{\pgfqpoint{10.158762in}{2.078023in}}{\pgfqpoint{10.153924in}{2.080026in}}{\pgfqpoint{10.148880in}{2.080026in}}%
\pgfpathcurveto{\pgfqpoint{10.143836in}{2.080026in}}{\pgfqpoint{10.138999in}{2.078023in}}{\pgfqpoint{10.135432in}{2.074456in}}%
\pgfpathcurveto{\pgfqpoint{10.131866in}{2.070890in}}{\pgfqpoint{10.129862in}{2.066052in}}{\pgfqpoint{10.129862in}{2.061008in}}%
\pgfpathcurveto{\pgfqpoint{10.129862in}{2.055965in}}{\pgfqpoint{10.131866in}{2.051127in}}{\pgfqpoint{10.135432in}{2.047560in}}%
\pgfpathcurveto{\pgfqpoint{10.138999in}{2.043994in}}{\pgfqpoint{10.143836in}{2.041990in}}{\pgfqpoint{10.148880in}{2.041990in}}%
\pgfpathclose%
\pgfusepath{fill}%
\end{pgfscope}%
\begin{pgfscope}%
\pgfpathrectangle{\pgfqpoint{6.572727in}{0.473000in}}{\pgfqpoint{4.227273in}{3.311000in}}%
\pgfusepath{clip}%
\pgfsetbuttcap%
\pgfsetroundjoin%
\definecolor{currentfill}{rgb}{0.127568,0.566949,0.550556}%
\pgfsetfillcolor{currentfill}%
\pgfsetfillopacity{0.700000}%
\pgfsetlinewidth{0.000000pt}%
\definecolor{currentstroke}{rgb}{0.000000,0.000000,0.000000}%
\pgfsetstrokecolor{currentstroke}%
\pgfsetstrokeopacity{0.700000}%
\pgfsetdash{}{0pt}%
\pgfpathmoveto{\pgfqpoint{8.426245in}{2.768539in}}%
\pgfpathcurveto{\pgfqpoint{8.431289in}{2.768539in}}{\pgfqpoint{8.436126in}{2.770542in}}{\pgfqpoint{8.439693in}{2.774109in}}%
\pgfpathcurveto{\pgfqpoint{8.443259in}{2.777675in}}{\pgfqpoint{8.445263in}{2.782513in}}{\pgfqpoint{8.445263in}{2.787557in}}%
\pgfpathcurveto{\pgfqpoint{8.445263in}{2.792600in}}{\pgfqpoint{8.443259in}{2.797438in}}{\pgfqpoint{8.439693in}{2.801005in}}%
\pgfpathcurveto{\pgfqpoint{8.436126in}{2.804571in}}{\pgfqpoint{8.431289in}{2.806575in}}{\pgfqpoint{8.426245in}{2.806575in}}%
\pgfpathcurveto{\pgfqpoint{8.421201in}{2.806575in}}{\pgfqpoint{8.416363in}{2.804571in}}{\pgfqpoint{8.412797in}{2.801005in}}%
\pgfpathcurveto{\pgfqpoint{8.409231in}{2.797438in}}{\pgfqpoint{8.407227in}{2.792600in}}{\pgfqpoint{8.407227in}{2.787557in}}%
\pgfpathcurveto{\pgfqpoint{8.407227in}{2.782513in}}{\pgfqpoint{8.409231in}{2.777675in}}{\pgfqpoint{8.412797in}{2.774109in}}%
\pgfpathcurveto{\pgfqpoint{8.416363in}{2.770542in}}{\pgfqpoint{8.421201in}{2.768539in}}{\pgfqpoint{8.426245in}{2.768539in}}%
\pgfpathclose%
\pgfusepath{fill}%
\end{pgfscope}%
\begin{pgfscope}%
\pgfpathrectangle{\pgfqpoint{6.572727in}{0.473000in}}{\pgfqpoint{4.227273in}{3.311000in}}%
\pgfusepath{clip}%
\pgfsetbuttcap%
\pgfsetroundjoin%
\definecolor{currentfill}{rgb}{0.127568,0.566949,0.550556}%
\pgfsetfillcolor{currentfill}%
\pgfsetfillopacity{0.700000}%
\pgfsetlinewidth{0.000000pt}%
\definecolor{currentstroke}{rgb}{0.000000,0.000000,0.000000}%
\pgfsetstrokecolor{currentstroke}%
\pgfsetstrokeopacity{0.700000}%
\pgfsetdash{}{0pt}%
\pgfpathmoveto{\pgfqpoint{8.220088in}{2.366305in}}%
\pgfpathcurveto{\pgfqpoint{8.225131in}{2.366305in}}{\pgfqpoint{8.229969in}{2.368309in}}{\pgfqpoint{8.233536in}{2.371875in}}%
\pgfpathcurveto{\pgfqpoint{8.237102in}{2.375442in}}{\pgfqpoint{8.239106in}{2.380280in}}{\pgfqpoint{8.239106in}{2.385323in}}%
\pgfpathcurveto{\pgfqpoint{8.239106in}{2.390367in}}{\pgfqpoint{8.237102in}{2.395205in}}{\pgfqpoint{8.233536in}{2.398771in}}%
\pgfpathcurveto{\pgfqpoint{8.229969in}{2.402337in}}{\pgfqpoint{8.225131in}{2.404341in}}{\pgfqpoint{8.220088in}{2.404341in}}%
\pgfpathcurveto{\pgfqpoint{8.215044in}{2.404341in}}{\pgfqpoint{8.210206in}{2.402337in}}{\pgfqpoint{8.206640in}{2.398771in}}%
\pgfpathcurveto{\pgfqpoint{8.203074in}{2.395205in}}{\pgfqpoint{8.201070in}{2.390367in}}{\pgfqpoint{8.201070in}{2.385323in}}%
\pgfpathcurveto{\pgfqpoint{8.201070in}{2.380280in}}{\pgfqpoint{8.203074in}{2.375442in}}{\pgfqpoint{8.206640in}{2.371875in}}%
\pgfpathcurveto{\pgfqpoint{8.210206in}{2.368309in}}{\pgfqpoint{8.215044in}{2.366305in}}{\pgfqpoint{8.220088in}{2.366305in}}%
\pgfpathclose%
\pgfusepath{fill}%
\end{pgfscope}%
\begin{pgfscope}%
\pgfpathrectangle{\pgfqpoint{6.572727in}{0.473000in}}{\pgfqpoint{4.227273in}{3.311000in}}%
\pgfusepath{clip}%
\pgfsetbuttcap%
\pgfsetroundjoin%
\definecolor{currentfill}{rgb}{0.127568,0.566949,0.550556}%
\pgfsetfillcolor{currentfill}%
\pgfsetfillopacity{0.700000}%
\pgfsetlinewidth{0.000000pt}%
\definecolor{currentstroke}{rgb}{0.000000,0.000000,0.000000}%
\pgfsetstrokecolor{currentstroke}%
\pgfsetstrokeopacity{0.700000}%
\pgfsetdash{}{0pt}%
\pgfpathmoveto{\pgfqpoint{7.902854in}{1.506703in}}%
\pgfpathcurveto{\pgfqpoint{7.907897in}{1.506703in}}{\pgfqpoint{7.912735in}{1.508707in}}{\pgfqpoint{7.916302in}{1.512274in}}%
\pgfpathcurveto{\pgfqpoint{7.919868in}{1.515840in}}{\pgfqpoint{7.921872in}{1.520678in}}{\pgfqpoint{7.921872in}{1.525721in}}%
\pgfpathcurveto{\pgfqpoint{7.921872in}{1.530765in}}{\pgfqpoint{7.919868in}{1.535603in}}{\pgfqpoint{7.916302in}{1.539169in}}%
\pgfpathcurveto{\pgfqpoint{7.912735in}{1.542736in}}{\pgfqpoint{7.907897in}{1.544740in}}{\pgfqpoint{7.902854in}{1.544740in}}%
\pgfpathcurveto{\pgfqpoint{7.897810in}{1.544740in}}{\pgfqpoint{7.892972in}{1.542736in}}{\pgfqpoint{7.889406in}{1.539169in}}%
\pgfpathcurveto{\pgfqpoint{7.885839in}{1.535603in}}{\pgfqpoint{7.883836in}{1.530765in}}{\pgfqpoint{7.883836in}{1.525721in}}%
\pgfpathcurveto{\pgfqpoint{7.883836in}{1.520678in}}{\pgfqpoint{7.885839in}{1.515840in}}{\pgfqpoint{7.889406in}{1.512274in}}%
\pgfpathcurveto{\pgfqpoint{7.892972in}{1.508707in}}{\pgfqpoint{7.897810in}{1.506703in}}{\pgfqpoint{7.902854in}{1.506703in}}%
\pgfpathclose%
\pgfusepath{fill}%
\end{pgfscope}%
\begin{pgfscope}%
\pgfpathrectangle{\pgfqpoint{6.572727in}{0.473000in}}{\pgfqpoint{4.227273in}{3.311000in}}%
\pgfusepath{clip}%
\pgfsetbuttcap%
\pgfsetroundjoin%
\definecolor{currentfill}{rgb}{0.127568,0.566949,0.550556}%
\pgfsetfillcolor{currentfill}%
\pgfsetfillopacity{0.700000}%
\pgfsetlinewidth{0.000000pt}%
\definecolor{currentstroke}{rgb}{0.000000,0.000000,0.000000}%
\pgfsetstrokecolor{currentstroke}%
\pgfsetstrokeopacity{0.700000}%
\pgfsetdash{}{0pt}%
\pgfpathmoveto{\pgfqpoint{8.066402in}{1.463012in}}%
\pgfpathcurveto{\pgfqpoint{8.071446in}{1.463012in}}{\pgfqpoint{8.076283in}{1.465016in}}{\pgfqpoint{8.079850in}{1.468583in}}%
\pgfpathcurveto{\pgfqpoint{8.083416in}{1.472149in}}{\pgfqpoint{8.085420in}{1.476987in}}{\pgfqpoint{8.085420in}{1.482031in}}%
\pgfpathcurveto{\pgfqpoint{8.085420in}{1.487074in}}{\pgfqpoint{8.083416in}{1.491912in}}{\pgfqpoint{8.079850in}{1.495478in}}%
\pgfpathcurveto{\pgfqpoint{8.076283in}{1.499045in}}{\pgfqpoint{8.071446in}{1.501049in}}{\pgfqpoint{8.066402in}{1.501049in}}%
\pgfpathcurveto{\pgfqpoint{8.061358in}{1.501049in}}{\pgfqpoint{8.056521in}{1.499045in}}{\pgfqpoint{8.052954in}{1.495478in}}%
\pgfpathcurveto{\pgfqpoint{8.049388in}{1.491912in}}{\pgfqpoint{8.047384in}{1.487074in}}{\pgfqpoint{8.047384in}{1.482031in}}%
\pgfpathcurveto{\pgfqpoint{8.047384in}{1.476987in}}{\pgfqpoint{8.049388in}{1.472149in}}{\pgfqpoint{8.052954in}{1.468583in}}%
\pgfpathcurveto{\pgfqpoint{8.056521in}{1.465016in}}{\pgfqpoint{8.061358in}{1.463012in}}{\pgfqpoint{8.066402in}{1.463012in}}%
\pgfpathclose%
\pgfusepath{fill}%
\end{pgfscope}%
\begin{pgfscope}%
\pgfpathrectangle{\pgfqpoint{6.572727in}{0.473000in}}{\pgfqpoint{4.227273in}{3.311000in}}%
\pgfusepath{clip}%
\pgfsetbuttcap%
\pgfsetroundjoin%
\definecolor{currentfill}{rgb}{0.127568,0.566949,0.550556}%
\pgfsetfillcolor{currentfill}%
\pgfsetfillopacity{0.700000}%
\pgfsetlinewidth{0.000000pt}%
\definecolor{currentstroke}{rgb}{0.000000,0.000000,0.000000}%
\pgfsetstrokecolor{currentstroke}%
\pgfsetstrokeopacity{0.700000}%
\pgfsetdash{}{0pt}%
\pgfpathmoveto{\pgfqpoint{8.305048in}{2.839607in}}%
\pgfpathcurveto{\pgfqpoint{8.310092in}{2.839607in}}{\pgfqpoint{8.314930in}{2.841611in}}{\pgfqpoint{8.318496in}{2.845177in}}%
\pgfpathcurveto{\pgfqpoint{8.322063in}{2.848743in}}{\pgfqpoint{8.324067in}{2.853581in}}{\pgfqpoint{8.324067in}{2.858625in}}%
\pgfpathcurveto{\pgfqpoint{8.324067in}{2.863668in}}{\pgfqpoint{8.322063in}{2.868506in}}{\pgfqpoint{8.318496in}{2.872073in}}%
\pgfpathcurveto{\pgfqpoint{8.314930in}{2.875639in}}{\pgfqpoint{8.310092in}{2.877643in}}{\pgfqpoint{8.305048in}{2.877643in}}%
\pgfpathcurveto{\pgfqpoint{8.300005in}{2.877643in}}{\pgfqpoint{8.295167in}{2.875639in}}{\pgfqpoint{8.291601in}{2.872073in}}%
\pgfpathcurveto{\pgfqpoint{8.288034in}{2.868506in}}{\pgfqpoint{8.286030in}{2.863668in}}{\pgfqpoint{8.286030in}{2.858625in}}%
\pgfpathcurveto{\pgfqpoint{8.286030in}{2.853581in}}{\pgfqpoint{8.288034in}{2.848743in}}{\pgfqpoint{8.291601in}{2.845177in}}%
\pgfpathcurveto{\pgfqpoint{8.295167in}{2.841611in}}{\pgfqpoint{8.300005in}{2.839607in}}{\pgfqpoint{8.305048in}{2.839607in}}%
\pgfpathclose%
\pgfusepath{fill}%
\end{pgfscope}%
\begin{pgfscope}%
\pgfpathrectangle{\pgfqpoint{6.572727in}{0.473000in}}{\pgfqpoint{4.227273in}{3.311000in}}%
\pgfusepath{clip}%
\pgfsetbuttcap%
\pgfsetroundjoin%
\definecolor{currentfill}{rgb}{0.127568,0.566949,0.550556}%
\pgfsetfillcolor{currentfill}%
\pgfsetfillopacity{0.700000}%
\pgfsetlinewidth{0.000000pt}%
\definecolor{currentstroke}{rgb}{0.000000,0.000000,0.000000}%
\pgfsetstrokecolor{currentstroke}%
\pgfsetstrokeopacity{0.700000}%
\pgfsetdash{}{0pt}%
\pgfpathmoveto{\pgfqpoint{7.743296in}{1.656093in}}%
\pgfpathcurveto{\pgfqpoint{7.748339in}{1.656093in}}{\pgfqpoint{7.753177in}{1.658096in}}{\pgfqpoint{7.756743in}{1.661663in}}%
\pgfpathcurveto{\pgfqpoint{7.760310in}{1.665229in}}{\pgfqpoint{7.762314in}{1.670067in}}{\pgfqpoint{7.762314in}{1.675111in}}%
\pgfpathcurveto{\pgfqpoint{7.762314in}{1.680154in}}{\pgfqpoint{7.760310in}{1.684992in}}{\pgfqpoint{7.756743in}{1.688559in}}%
\pgfpathcurveto{\pgfqpoint{7.753177in}{1.692125in}}{\pgfqpoint{7.748339in}{1.694129in}}{\pgfqpoint{7.743296in}{1.694129in}}%
\pgfpathcurveto{\pgfqpoint{7.738252in}{1.694129in}}{\pgfqpoint{7.733414in}{1.692125in}}{\pgfqpoint{7.729848in}{1.688559in}}%
\pgfpathcurveto{\pgfqpoint{7.726281in}{1.684992in}}{\pgfqpoint{7.724277in}{1.680154in}}{\pgfqpoint{7.724277in}{1.675111in}}%
\pgfpathcurveto{\pgfqpoint{7.724277in}{1.670067in}}{\pgfqpoint{7.726281in}{1.665229in}}{\pgfqpoint{7.729848in}{1.661663in}}%
\pgfpathcurveto{\pgfqpoint{7.733414in}{1.658096in}}{\pgfqpoint{7.738252in}{1.656093in}}{\pgfqpoint{7.743296in}{1.656093in}}%
\pgfpathclose%
\pgfusepath{fill}%
\end{pgfscope}%
\begin{pgfscope}%
\pgfpathrectangle{\pgfqpoint{6.572727in}{0.473000in}}{\pgfqpoint{4.227273in}{3.311000in}}%
\pgfusepath{clip}%
\pgfsetbuttcap%
\pgfsetroundjoin%
\definecolor{currentfill}{rgb}{0.993248,0.906157,0.143936}%
\pgfsetfillcolor{currentfill}%
\pgfsetfillopacity{0.700000}%
\pgfsetlinewidth{0.000000pt}%
\definecolor{currentstroke}{rgb}{0.000000,0.000000,0.000000}%
\pgfsetstrokecolor{currentstroke}%
\pgfsetstrokeopacity{0.700000}%
\pgfsetdash{}{0pt}%
\pgfpathmoveto{\pgfqpoint{9.271095in}{1.892028in}}%
\pgfpathcurveto{\pgfqpoint{9.276139in}{1.892028in}}{\pgfqpoint{9.280977in}{1.894032in}}{\pgfqpoint{9.284543in}{1.897599in}}%
\pgfpathcurveto{\pgfqpoint{9.288109in}{1.901165in}}{\pgfqpoint{9.290113in}{1.906003in}}{\pgfqpoint{9.290113in}{1.911046in}}%
\pgfpathcurveto{\pgfqpoint{9.290113in}{1.916090in}}{\pgfqpoint{9.288109in}{1.920928in}}{\pgfqpoint{9.284543in}{1.924494in}}%
\pgfpathcurveto{\pgfqpoint{9.280977in}{1.928061in}}{\pgfqpoint{9.276139in}{1.930065in}}{\pgfqpoint{9.271095in}{1.930065in}}%
\pgfpathcurveto{\pgfqpoint{9.266051in}{1.930065in}}{\pgfqpoint{9.261214in}{1.928061in}}{\pgfqpoint{9.257647in}{1.924494in}}%
\pgfpathcurveto{\pgfqpoint{9.254081in}{1.920928in}}{\pgfqpoint{9.252077in}{1.916090in}}{\pgfqpoint{9.252077in}{1.911046in}}%
\pgfpathcurveto{\pgfqpoint{9.252077in}{1.906003in}}{\pgfqpoint{9.254081in}{1.901165in}}{\pgfqpoint{9.257647in}{1.897599in}}%
\pgfpathcurveto{\pgfqpoint{9.261214in}{1.894032in}}{\pgfqpoint{9.266051in}{1.892028in}}{\pgfqpoint{9.271095in}{1.892028in}}%
\pgfpathclose%
\pgfusepath{fill}%
\end{pgfscope}%
\begin{pgfscope}%
\pgfpathrectangle{\pgfqpoint{6.572727in}{0.473000in}}{\pgfqpoint{4.227273in}{3.311000in}}%
\pgfusepath{clip}%
\pgfsetbuttcap%
\pgfsetroundjoin%
\definecolor{currentfill}{rgb}{0.993248,0.906157,0.143936}%
\pgfsetfillcolor{currentfill}%
\pgfsetfillopacity{0.700000}%
\pgfsetlinewidth{0.000000pt}%
\definecolor{currentstroke}{rgb}{0.000000,0.000000,0.000000}%
\pgfsetstrokecolor{currentstroke}%
\pgfsetstrokeopacity{0.700000}%
\pgfsetdash{}{0pt}%
\pgfpathmoveto{\pgfqpoint{9.844824in}{1.007612in}}%
\pgfpathcurveto{\pgfqpoint{9.849868in}{1.007612in}}{\pgfqpoint{9.854706in}{1.009616in}}{\pgfqpoint{9.858272in}{1.013182in}}%
\pgfpathcurveto{\pgfqpoint{9.861838in}{1.016748in}}{\pgfqpoint{9.863842in}{1.021586in}}{\pgfqpoint{9.863842in}{1.026630in}}%
\pgfpathcurveto{\pgfqpoint{9.863842in}{1.031674in}}{\pgfqpoint{9.861838in}{1.036511in}}{\pgfqpoint{9.858272in}{1.040078in}}%
\pgfpathcurveto{\pgfqpoint{9.854706in}{1.043644in}}{\pgfqpoint{9.849868in}{1.045648in}}{\pgfqpoint{9.844824in}{1.045648in}}%
\pgfpathcurveto{\pgfqpoint{9.839780in}{1.045648in}}{\pgfqpoint{9.834943in}{1.043644in}}{\pgfqpoint{9.831376in}{1.040078in}}%
\pgfpathcurveto{\pgfqpoint{9.827810in}{1.036511in}}{\pgfqpoint{9.825806in}{1.031674in}}{\pgfqpoint{9.825806in}{1.026630in}}%
\pgfpathcurveto{\pgfqpoint{9.825806in}{1.021586in}}{\pgfqpoint{9.827810in}{1.016748in}}{\pgfqpoint{9.831376in}{1.013182in}}%
\pgfpathcurveto{\pgfqpoint{9.834943in}{1.009616in}}{\pgfqpoint{9.839780in}{1.007612in}}{\pgfqpoint{9.844824in}{1.007612in}}%
\pgfpathclose%
\pgfusepath{fill}%
\end{pgfscope}%
\begin{pgfscope}%
\pgfpathrectangle{\pgfqpoint{6.572727in}{0.473000in}}{\pgfqpoint{4.227273in}{3.311000in}}%
\pgfusepath{clip}%
\pgfsetbuttcap%
\pgfsetroundjoin%
\definecolor{currentfill}{rgb}{0.127568,0.566949,0.550556}%
\pgfsetfillcolor{currentfill}%
\pgfsetfillopacity{0.700000}%
\pgfsetlinewidth{0.000000pt}%
\definecolor{currentstroke}{rgb}{0.000000,0.000000,0.000000}%
\pgfsetstrokecolor{currentstroke}%
\pgfsetstrokeopacity{0.700000}%
\pgfsetdash{}{0pt}%
\pgfpathmoveto{\pgfqpoint{8.068550in}{3.014926in}}%
\pgfpathcurveto{\pgfqpoint{8.073594in}{3.014926in}}{\pgfqpoint{8.078432in}{3.016930in}}{\pgfqpoint{8.081998in}{3.020496in}}%
\pgfpathcurveto{\pgfqpoint{8.085565in}{3.024063in}}{\pgfqpoint{8.087569in}{3.028901in}}{\pgfqpoint{8.087569in}{3.033944in}}%
\pgfpathcurveto{\pgfqpoint{8.087569in}{3.038988in}}{\pgfqpoint{8.085565in}{3.043826in}}{\pgfqpoint{8.081998in}{3.047392in}}%
\pgfpathcurveto{\pgfqpoint{8.078432in}{3.050959in}}{\pgfqpoint{8.073594in}{3.052962in}}{\pgfqpoint{8.068550in}{3.052962in}}%
\pgfpathcurveto{\pgfqpoint{8.063507in}{3.052962in}}{\pgfqpoint{8.058669in}{3.050959in}}{\pgfqpoint{8.055103in}{3.047392in}}%
\pgfpathcurveto{\pgfqpoint{8.051536in}{3.043826in}}{\pgfqpoint{8.049532in}{3.038988in}}{\pgfqpoint{8.049532in}{3.033944in}}%
\pgfpathcurveto{\pgfqpoint{8.049532in}{3.028901in}}{\pgfqpoint{8.051536in}{3.024063in}}{\pgfqpoint{8.055103in}{3.020496in}}%
\pgfpathcurveto{\pgfqpoint{8.058669in}{3.016930in}}{\pgfqpoint{8.063507in}{3.014926in}}{\pgfqpoint{8.068550in}{3.014926in}}%
\pgfpathclose%
\pgfusepath{fill}%
\end{pgfscope}%
\begin{pgfscope}%
\pgfpathrectangle{\pgfqpoint{6.572727in}{0.473000in}}{\pgfqpoint{4.227273in}{3.311000in}}%
\pgfusepath{clip}%
\pgfsetbuttcap%
\pgfsetroundjoin%
\definecolor{currentfill}{rgb}{0.993248,0.906157,0.143936}%
\pgfsetfillcolor{currentfill}%
\pgfsetfillopacity{0.700000}%
\pgfsetlinewidth{0.000000pt}%
\definecolor{currentstroke}{rgb}{0.000000,0.000000,0.000000}%
\pgfsetstrokecolor{currentstroke}%
\pgfsetstrokeopacity{0.700000}%
\pgfsetdash{}{0pt}%
\pgfpathmoveto{\pgfqpoint{9.180485in}{1.692322in}}%
\pgfpathcurveto{\pgfqpoint{9.185529in}{1.692322in}}{\pgfqpoint{9.190367in}{1.694326in}}{\pgfqpoint{9.193933in}{1.697893in}}%
\pgfpathcurveto{\pgfqpoint{9.197500in}{1.701459in}}{\pgfqpoint{9.199504in}{1.706297in}}{\pgfqpoint{9.199504in}{1.711341in}}%
\pgfpathcurveto{\pgfqpoint{9.199504in}{1.716384in}}{\pgfqpoint{9.197500in}{1.721222in}}{\pgfqpoint{9.193933in}{1.724788in}}%
\pgfpathcurveto{\pgfqpoint{9.190367in}{1.728355in}}{\pgfqpoint{9.185529in}{1.730359in}}{\pgfqpoint{9.180485in}{1.730359in}}%
\pgfpathcurveto{\pgfqpoint{9.175442in}{1.730359in}}{\pgfqpoint{9.170604in}{1.728355in}}{\pgfqpoint{9.167038in}{1.724788in}}%
\pgfpathcurveto{\pgfqpoint{9.163471in}{1.721222in}}{\pgfqpoint{9.161467in}{1.716384in}}{\pgfqpoint{9.161467in}{1.711341in}}%
\pgfpathcurveto{\pgfqpoint{9.161467in}{1.706297in}}{\pgfqpoint{9.163471in}{1.701459in}}{\pgfqpoint{9.167038in}{1.697893in}}%
\pgfpathcurveto{\pgfqpoint{9.170604in}{1.694326in}}{\pgfqpoint{9.175442in}{1.692322in}}{\pgfqpoint{9.180485in}{1.692322in}}%
\pgfpathclose%
\pgfusepath{fill}%
\end{pgfscope}%
\begin{pgfscope}%
\pgfpathrectangle{\pgfqpoint{6.572727in}{0.473000in}}{\pgfqpoint{4.227273in}{3.311000in}}%
\pgfusepath{clip}%
\pgfsetbuttcap%
\pgfsetroundjoin%
\definecolor{currentfill}{rgb}{0.127568,0.566949,0.550556}%
\pgfsetfillcolor{currentfill}%
\pgfsetfillopacity{0.700000}%
\pgfsetlinewidth{0.000000pt}%
\definecolor{currentstroke}{rgb}{0.000000,0.000000,0.000000}%
\pgfsetstrokecolor{currentstroke}%
\pgfsetstrokeopacity{0.700000}%
\pgfsetdash{}{0pt}%
\pgfpathmoveto{\pgfqpoint{7.364351in}{1.568544in}}%
\pgfpathcurveto{\pgfqpoint{7.369395in}{1.568544in}}{\pgfqpoint{7.374233in}{1.570548in}}{\pgfqpoint{7.377799in}{1.574115in}}%
\pgfpathcurveto{\pgfqpoint{7.381366in}{1.577681in}}{\pgfqpoint{7.383370in}{1.582519in}}{\pgfqpoint{7.383370in}{1.587562in}}%
\pgfpathcurveto{\pgfqpoint{7.383370in}{1.592606in}}{\pgfqpoint{7.381366in}{1.597444in}}{\pgfqpoint{7.377799in}{1.601010in}}%
\pgfpathcurveto{\pgfqpoint{7.374233in}{1.604577in}}{\pgfqpoint{7.369395in}{1.606581in}}{\pgfqpoint{7.364351in}{1.606581in}}%
\pgfpathcurveto{\pgfqpoint{7.359308in}{1.606581in}}{\pgfqpoint{7.354470in}{1.604577in}}{\pgfqpoint{7.350904in}{1.601010in}}%
\pgfpathcurveto{\pgfqpoint{7.347337in}{1.597444in}}{\pgfqpoint{7.345333in}{1.592606in}}{\pgfqpoint{7.345333in}{1.587562in}}%
\pgfpathcurveto{\pgfqpoint{7.345333in}{1.582519in}}{\pgfqpoint{7.347337in}{1.577681in}}{\pgfqpoint{7.350904in}{1.574115in}}%
\pgfpathcurveto{\pgfqpoint{7.354470in}{1.570548in}}{\pgfqpoint{7.359308in}{1.568544in}}{\pgfqpoint{7.364351in}{1.568544in}}%
\pgfpathclose%
\pgfusepath{fill}%
\end{pgfscope}%
\begin{pgfscope}%
\pgfpathrectangle{\pgfqpoint{6.572727in}{0.473000in}}{\pgfqpoint{4.227273in}{3.311000in}}%
\pgfusepath{clip}%
\pgfsetbuttcap%
\pgfsetroundjoin%
\definecolor{currentfill}{rgb}{0.127568,0.566949,0.550556}%
\pgfsetfillcolor{currentfill}%
\pgfsetfillopacity{0.700000}%
\pgfsetlinewidth{0.000000pt}%
\definecolor{currentstroke}{rgb}{0.000000,0.000000,0.000000}%
\pgfsetstrokecolor{currentstroke}%
\pgfsetstrokeopacity{0.700000}%
\pgfsetdash{}{0pt}%
\pgfpathmoveto{\pgfqpoint{8.498929in}{3.039814in}}%
\pgfpathcurveto{\pgfqpoint{8.503973in}{3.039814in}}{\pgfqpoint{8.508811in}{3.041818in}}{\pgfqpoint{8.512377in}{3.045384in}}%
\pgfpathcurveto{\pgfqpoint{8.515944in}{3.048951in}}{\pgfqpoint{8.517947in}{3.053789in}}{\pgfqpoint{8.517947in}{3.058832in}}%
\pgfpathcurveto{\pgfqpoint{8.517947in}{3.063876in}}{\pgfqpoint{8.515944in}{3.068714in}}{\pgfqpoint{8.512377in}{3.072280in}}%
\pgfpathcurveto{\pgfqpoint{8.508811in}{3.075847in}}{\pgfqpoint{8.503973in}{3.077850in}}{\pgfqpoint{8.498929in}{3.077850in}}%
\pgfpathcurveto{\pgfqpoint{8.493886in}{3.077850in}}{\pgfqpoint{8.489048in}{3.075847in}}{\pgfqpoint{8.485481in}{3.072280in}}%
\pgfpathcurveto{\pgfqpoint{8.481915in}{3.068714in}}{\pgfqpoint{8.479911in}{3.063876in}}{\pgfqpoint{8.479911in}{3.058832in}}%
\pgfpathcurveto{\pgfqpoint{8.479911in}{3.053789in}}{\pgfqpoint{8.481915in}{3.048951in}}{\pgfqpoint{8.485481in}{3.045384in}}%
\pgfpathcurveto{\pgfqpoint{8.489048in}{3.041818in}}{\pgfqpoint{8.493886in}{3.039814in}}{\pgfqpoint{8.498929in}{3.039814in}}%
\pgfpathclose%
\pgfusepath{fill}%
\end{pgfscope}%
\begin{pgfscope}%
\pgfpathrectangle{\pgfqpoint{6.572727in}{0.473000in}}{\pgfqpoint{4.227273in}{3.311000in}}%
\pgfusepath{clip}%
\pgfsetbuttcap%
\pgfsetroundjoin%
\definecolor{currentfill}{rgb}{0.127568,0.566949,0.550556}%
\pgfsetfillcolor{currentfill}%
\pgfsetfillopacity{0.700000}%
\pgfsetlinewidth{0.000000pt}%
\definecolor{currentstroke}{rgb}{0.000000,0.000000,0.000000}%
\pgfsetstrokecolor{currentstroke}%
\pgfsetstrokeopacity{0.700000}%
\pgfsetdash{}{0pt}%
\pgfpathmoveto{\pgfqpoint{7.473740in}{1.443844in}}%
\pgfpathcurveto{\pgfqpoint{7.478783in}{1.443844in}}{\pgfqpoint{7.483621in}{1.445847in}}{\pgfqpoint{7.487188in}{1.449414in}}%
\pgfpathcurveto{\pgfqpoint{7.490754in}{1.452980in}}{\pgfqpoint{7.492758in}{1.457818in}}{\pgfqpoint{7.492758in}{1.462862in}}%
\pgfpathcurveto{\pgfqpoint{7.492758in}{1.467905in}}{\pgfqpoint{7.490754in}{1.472743in}}{\pgfqpoint{7.487188in}{1.476310in}}%
\pgfpathcurveto{\pgfqpoint{7.483621in}{1.479876in}}{\pgfqpoint{7.478783in}{1.481880in}}{\pgfqpoint{7.473740in}{1.481880in}}%
\pgfpathcurveto{\pgfqpoint{7.468696in}{1.481880in}}{\pgfqpoint{7.463858in}{1.479876in}}{\pgfqpoint{7.460292in}{1.476310in}}%
\pgfpathcurveto{\pgfqpoint{7.456726in}{1.472743in}}{\pgfqpoint{7.454722in}{1.467905in}}{\pgfqpoint{7.454722in}{1.462862in}}%
\pgfpathcurveto{\pgfqpoint{7.454722in}{1.457818in}}{\pgfqpoint{7.456726in}{1.452980in}}{\pgfqpoint{7.460292in}{1.449414in}}%
\pgfpathcurveto{\pgfqpoint{7.463858in}{1.445847in}}{\pgfqpoint{7.468696in}{1.443844in}}{\pgfqpoint{7.473740in}{1.443844in}}%
\pgfpathclose%
\pgfusepath{fill}%
\end{pgfscope}%
\begin{pgfscope}%
\pgfpathrectangle{\pgfqpoint{6.572727in}{0.473000in}}{\pgfqpoint{4.227273in}{3.311000in}}%
\pgfusepath{clip}%
\pgfsetbuttcap%
\pgfsetroundjoin%
\definecolor{currentfill}{rgb}{0.127568,0.566949,0.550556}%
\pgfsetfillcolor{currentfill}%
\pgfsetfillopacity{0.700000}%
\pgfsetlinewidth{0.000000pt}%
\definecolor{currentstroke}{rgb}{0.000000,0.000000,0.000000}%
\pgfsetstrokecolor{currentstroke}%
\pgfsetstrokeopacity{0.700000}%
\pgfsetdash{}{0pt}%
\pgfpathmoveto{\pgfqpoint{8.014006in}{2.646451in}}%
\pgfpathcurveto{\pgfqpoint{8.019050in}{2.646451in}}{\pgfqpoint{8.023888in}{2.648455in}}{\pgfqpoint{8.027454in}{2.652022in}}%
\pgfpathcurveto{\pgfqpoint{8.031021in}{2.655588in}}{\pgfqpoint{8.033024in}{2.660426in}}{\pgfqpoint{8.033024in}{2.665470in}}%
\pgfpathcurveto{\pgfqpoint{8.033024in}{2.670513in}}{\pgfqpoint{8.031021in}{2.675351in}}{\pgfqpoint{8.027454in}{2.678917in}}%
\pgfpathcurveto{\pgfqpoint{8.023888in}{2.682484in}}{\pgfqpoint{8.019050in}{2.684488in}}{\pgfqpoint{8.014006in}{2.684488in}}%
\pgfpathcurveto{\pgfqpoint{8.008963in}{2.684488in}}{\pgfqpoint{8.004125in}{2.682484in}}{\pgfqpoint{8.000558in}{2.678917in}}%
\pgfpathcurveto{\pgfqpoint{7.996992in}{2.675351in}}{\pgfqpoint{7.994988in}{2.670513in}}{\pgfqpoint{7.994988in}{2.665470in}}%
\pgfpathcurveto{\pgfqpoint{7.994988in}{2.660426in}}{\pgfqpoint{7.996992in}{2.655588in}}{\pgfqpoint{8.000558in}{2.652022in}}%
\pgfpathcurveto{\pgfqpoint{8.004125in}{2.648455in}}{\pgfqpoint{8.008963in}{2.646451in}}{\pgfqpoint{8.014006in}{2.646451in}}%
\pgfpathclose%
\pgfusepath{fill}%
\end{pgfscope}%
\begin{pgfscope}%
\pgfpathrectangle{\pgfqpoint{6.572727in}{0.473000in}}{\pgfqpoint{4.227273in}{3.311000in}}%
\pgfusepath{clip}%
\pgfsetbuttcap%
\pgfsetroundjoin%
\definecolor{currentfill}{rgb}{0.127568,0.566949,0.550556}%
\pgfsetfillcolor{currentfill}%
\pgfsetfillopacity{0.700000}%
\pgfsetlinewidth{0.000000pt}%
\definecolor{currentstroke}{rgb}{0.000000,0.000000,0.000000}%
\pgfsetstrokecolor{currentstroke}%
\pgfsetstrokeopacity{0.700000}%
\pgfsetdash{}{0pt}%
\pgfpathmoveto{\pgfqpoint{8.168779in}{3.101851in}}%
\pgfpathcurveto{\pgfqpoint{8.173822in}{3.101851in}}{\pgfqpoint{8.178660in}{3.103855in}}{\pgfqpoint{8.182227in}{3.107421in}}%
\pgfpathcurveto{\pgfqpoint{8.185793in}{3.110987in}}{\pgfqpoint{8.187797in}{3.115825in}}{\pgfqpoint{8.187797in}{3.120869in}}%
\pgfpathcurveto{\pgfqpoint{8.187797in}{3.125912in}}{\pgfqpoint{8.185793in}{3.130750in}}{\pgfqpoint{8.182227in}{3.134317in}}%
\pgfpathcurveto{\pgfqpoint{8.178660in}{3.137883in}}{\pgfqpoint{8.173822in}{3.139887in}}{\pgfqpoint{8.168779in}{3.139887in}}%
\pgfpathcurveto{\pgfqpoint{8.163735in}{3.139887in}}{\pgfqpoint{8.158897in}{3.137883in}}{\pgfqpoint{8.155331in}{3.134317in}}%
\pgfpathcurveto{\pgfqpoint{8.151764in}{3.130750in}}{\pgfqpoint{8.149761in}{3.125912in}}{\pgfqpoint{8.149761in}{3.120869in}}%
\pgfpathcurveto{\pgfqpoint{8.149761in}{3.115825in}}{\pgfqpoint{8.151764in}{3.110987in}}{\pgfqpoint{8.155331in}{3.107421in}}%
\pgfpathcurveto{\pgfqpoint{8.158897in}{3.103855in}}{\pgfqpoint{8.163735in}{3.101851in}}{\pgfqpoint{8.168779in}{3.101851in}}%
\pgfpathclose%
\pgfusepath{fill}%
\end{pgfscope}%
\begin{pgfscope}%
\pgfpathrectangle{\pgfqpoint{6.572727in}{0.473000in}}{\pgfqpoint{4.227273in}{3.311000in}}%
\pgfusepath{clip}%
\pgfsetbuttcap%
\pgfsetroundjoin%
\definecolor{currentfill}{rgb}{0.127568,0.566949,0.550556}%
\pgfsetfillcolor{currentfill}%
\pgfsetfillopacity{0.700000}%
\pgfsetlinewidth{0.000000pt}%
\definecolor{currentstroke}{rgb}{0.000000,0.000000,0.000000}%
\pgfsetstrokecolor{currentstroke}%
\pgfsetstrokeopacity{0.700000}%
\pgfsetdash{}{0pt}%
\pgfpathmoveto{\pgfqpoint{8.098024in}{2.931186in}}%
\pgfpathcurveto{\pgfqpoint{8.103068in}{2.931186in}}{\pgfqpoint{8.107905in}{2.933190in}}{\pgfqpoint{8.111472in}{2.936756in}}%
\pgfpathcurveto{\pgfqpoint{8.115038in}{2.940323in}}{\pgfqpoint{8.117042in}{2.945161in}}{\pgfqpoint{8.117042in}{2.950204in}}%
\pgfpathcurveto{\pgfqpoint{8.117042in}{2.955248in}}{\pgfqpoint{8.115038in}{2.960086in}}{\pgfqpoint{8.111472in}{2.963652in}}%
\pgfpathcurveto{\pgfqpoint{8.107905in}{2.967219in}}{\pgfqpoint{8.103068in}{2.969222in}}{\pgfqpoint{8.098024in}{2.969222in}}%
\pgfpathcurveto{\pgfqpoint{8.092980in}{2.969222in}}{\pgfqpoint{8.088143in}{2.967219in}}{\pgfqpoint{8.084576in}{2.963652in}}%
\pgfpathcurveto{\pgfqpoint{8.081010in}{2.960086in}}{\pgfqpoint{8.079006in}{2.955248in}}{\pgfqpoint{8.079006in}{2.950204in}}%
\pgfpathcurveto{\pgfqpoint{8.079006in}{2.945161in}}{\pgfqpoint{8.081010in}{2.940323in}}{\pgfqpoint{8.084576in}{2.936756in}}%
\pgfpathcurveto{\pgfqpoint{8.088143in}{2.933190in}}{\pgfqpoint{8.092980in}{2.931186in}}{\pgfqpoint{8.098024in}{2.931186in}}%
\pgfpathclose%
\pgfusepath{fill}%
\end{pgfscope}%
\begin{pgfscope}%
\pgfpathrectangle{\pgfqpoint{6.572727in}{0.473000in}}{\pgfqpoint{4.227273in}{3.311000in}}%
\pgfusepath{clip}%
\pgfsetbuttcap%
\pgfsetroundjoin%
\definecolor{currentfill}{rgb}{0.127568,0.566949,0.550556}%
\pgfsetfillcolor{currentfill}%
\pgfsetfillopacity{0.700000}%
\pgfsetlinewidth{0.000000pt}%
\definecolor{currentstroke}{rgb}{0.000000,0.000000,0.000000}%
\pgfsetstrokecolor{currentstroke}%
\pgfsetstrokeopacity{0.700000}%
\pgfsetdash{}{0pt}%
\pgfpathmoveto{\pgfqpoint{7.048983in}{1.678884in}}%
\pgfpathcurveto{\pgfqpoint{7.054027in}{1.678884in}}{\pgfqpoint{7.058864in}{1.680888in}}{\pgfqpoint{7.062431in}{1.684455in}}%
\pgfpathcurveto{\pgfqpoint{7.065997in}{1.688021in}}{\pgfqpoint{7.068001in}{1.692859in}}{\pgfqpoint{7.068001in}{1.697902in}}%
\pgfpathcurveto{\pgfqpoint{7.068001in}{1.702946in}}{\pgfqpoint{7.065997in}{1.707784in}}{\pgfqpoint{7.062431in}{1.711350in}}%
\pgfpathcurveto{\pgfqpoint{7.058864in}{1.714917in}}{\pgfqpoint{7.054027in}{1.716921in}}{\pgfqpoint{7.048983in}{1.716921in}}%
\pgfpathcurveto{\pgfqpoint{7.043939in}{1.716921in}}{\pgfqpoint{7.039102in}{1.714917in}}{\pgfqpoint{7.035535in}{1.711350in}}%
\pgfpathcurveto{\pgfqpoint{7.031969in}{1.707784in}}{\pgfqpoint{7.029965in}{1.702946in}}{\pgfqpoint{7.029965in}{1.697902in}}%
\pgfpathcurveto{\pgfqpoint{7.029965in}{1.692859in}}{\pgfqpoint{7.031969in}{1.688021in}}{\pgfqpoint{7.035535in}{1.684455in}}%
\pgfpathcurveto{\pgfqpoint{7.039102in}{1.680888in}}{\pgfqpoint{7.043939in}{1.678884in}}{\pgfqpoint{7.048983in}{1.678884in}}%
\pgfpathclose%
\pgfusepath{fill}%
\end{pgfscope}%
\begin{pgfscope}%
\pgfpathrectangle{\pgfqpoint{6.572727in}{0.473000in}}{\pgfqpoint{4.227273in}{3.311000in}}%
\pgfusepath{clip}%
\pgfsetbuttcap%
\pgfsetroundjoin%
\definecolor{currentfill}{rgb}{0.127568,0.566949,0.550556}%
\pgfsetfillcolor{currentfill}%
\pgfsetfillopacity{0.700000}%
\pgfsetlinewidth{0.000000pt}%
\definecolor{currentstroke}{rgb}{0.000000,0.000000,0.000000}%
\pgfsetstrokecolor{currentstroke}%
\pgfsetstrokeopacity{0.700000}%
\pgfsetdash{}{0pt}%
\pgfpathmoveto{\pgfqpoint{7.721729in}{1.382165in}}%
\pgfpathcurveto{\pgfqpoint{7.726773in}{1.382165in}}{\pgfqpoint{7.731611in}{1.384169in}}{\pgfqpoint{7.735177in}{1.387735in}}%
\pgfpathcurveto{\pgfqpoint{7.738744in}{1.391302in}}{\pgfqpoint{7.740747in}{1.396139in}}{\pgfqpoint{7.740747in}{1.401183in}}%
\pgfpathcurveto{\pgfqpoint{7.740747in}{1.406227in}}{\pgfqpoint{7.738744in}{1.411064in}}{\pgfqpoint{7.735177in}{1.414631in}}%
\pgfpathcurveto{\pgfqpoint{7.731611in}{1.418197in}}{\pgfqpoint{7.726773in}{1.420201in}}{\pgfqpoint{7.721729in}{1.420201in}}%
\pgfpathcurveto{\pgfqpoint{7.716686in}{1.420201in}}{\pgfqpoint{7.711848in}{1.418197in}}{\pgfqpoint{7.708281in}{1.414631in}}%
\pgfpathcurveto{\pgfqpoint{7.704715in}{1.411064in}}{\pgfqpoint{7.702711in}{1.406227in}}{\pgfqpoint{7.702711in}{1.401183in}}%
\pgfpathcurveto{\pgfqpoint{7.702711in}{1.396139in}}{\pgfqpoint{7.704715in}{1.391302in}}{\pgfqpoint{7.708281in}{1.387735in}}%
\pgfpathcurveto{\pgfqpoint{7.711848in}{1.384169in}}{\pgfqpoint{7.716686in}{1.382165in}}{\pgfqpoint{7.721729in}{1.382165in}}%
\pgfpathclose%
\pgfusepath{fill}%
\end{pgfscope}%
\begin{pgfscope}%
\pgfpathrectangle{\pgfqpoint{6.572727in}{0.473000in}}{\pgfqpoint{4.227273in}{3.311000in}}%
\pgfusepath{clip}%
\pgfsetbuttcap%
\pgfsetroundjoin%
\definecolor{currentfill}{rgb}{0.127568,0.566949,0.550556}%
\pgfsetfillcolor{currentfill}%
\pgfsetfillopacity{0.700000}%
\pgfsetlinewidth{0.000000pt}%
\definecolor{currentstroke}{rgb}{0.000000,0.000000,0.000000}%
\pgfsetstrokecolor{currentstroke}%
\pgfsetstrokeopacity{0.700000}%
\pgfsetdash{}{0pt}%
\pgfpathmoveto{\pgfqpoint{7.758603in}{1.118752in}}%
\pgfpathcurveto{\pgfqpoint{7.763647in}{1.118752in}}{\pgfqpoint{7.768484in}{1.120756in}}{\pgfqpoint{7.772051in}{1.124323in}}%
\pgfpathcurveto{\pgfqpoint{7.775617in}{1.127889in}}{\pgfqpoint{7.777621in}{1.132727in}}{\pgfqpoint{7.777621in}{1.137771in}}%
\pgfpathcurveto{\pgfqpoint{7.777621in}{1.142814in}}{\pgfqpoint{7.775617in}{1.147652in}}{\pgfqpoint{7.772051in}{1.151218in}}%
\pgfpathcurveto{\pgfqpoint{7.768484in}{1.154785in}}{\pgfqpoint{7.763647in}{1.156789in}}{\pgfqpoint{7.758603in}{1.156789in}}%
\pgfpathcurveto{\pgfqpoint{7.753559in}{1.156789in}}{\pgfqpoint{7.748721in}{1.154785in}}{\pgfqpoint{7.745155in}{1.151218in}}%
\pgfpathcurveto{\pgfqpoint{7.741589in}{1.147652in}}{\pgfqpoint{7.739585in}{1.142814in}}{\pgfqpoint{7.739585in}{1.137771in}}%
\pgfpathcurveto{\pgfqpoint{7.739585in}{1.132727in}}{\pgfqpoint{7.741589in}{1.127889in}}{\pgfqpoint{7.745155in}{1.124323in}}%
\pgfpathcurveto{\pgfqpoint{7.748721in}{1.120756in}}{\pgfqpoint{7.753559in}{1.118752in}}{\pgfqpoint{7.758603in}{1.118752in}}%
\pgfpathclose%
\pgfusepath{fill}%
\end{pgfscope}%
\begin{pgfscope}%
\pgfpathrectangle{\pgfqpoint{6.572727in}{0.473000in}}{\pgfqpoint{4.227273in}{3.311000in}}%
\pgfusepath{clip}%
\pgfsetbuttcap%
\pgfsetroundjoin%
\definecolor{currentfill}{rgb}{0.993248,0.906157,0.143936}%
\pgfsetfillcolor{currentfill}%
\pgfsetfillopacity{0.700000}%
\pgfsetlinewidth{0.000000pt}%
\definecolor{currentstroke}{rgb}{0.000000,0.000000,0.000000}%
\pgfsetstrokecolor{currentstroke}%
\pgfsetstrokeopacity{0.700000}%
\pgfsetdash{}{0pt}%
\pgfpathmoveto{\pgfqpoint{9.794202in}{1.521219in}}%
\pgfpathcurveto{\pgfqpoint{9.799246in}{1.521219in}}{\pgfqpoint{9.804084in}{1.523222in}}{\pgfqpoint{9.807650in}{1.526789in}}%
\pgfpathcurveto{\pgfqpoint{9.811217in}{1.530355in}}{\pgfqpoint{9.813221in}{1.535193in}}{\pgfqpoint{9.813221in}{1.540237in}}%
\pgfpathcurveto{\pgfqpoint{9.813221in}{1.545280in}}{\pgfqpoint{9.811217in}{1.550118in}}{\pgfqpoint{9.807650in}{1.553685in}}%
\pgfpathcurveto{\pgfqpoint{9.804084in}{1.557251in}}{\pgfqpoint{9.799246in}{1.559255in}}{\pgfqpoint{9.794202in}{1.559255in}}%
\pgfpathcurveto{\pgfqpoint{9.789159in}{1.559255in}}{\pgfqpoint{9.784321in}{1.557251in}}{\pgfqpoint{9.780755in}{1.553685in}}%
\pgfpathcurveto{\pgfqpoint{9.777188in}{1.550118in}}{\pgfqpoint{9.775184in}{1.545280in}}{\pgfqpoint{9.775184in}{1.540237in}}%
\pgfpathcurveto{\pgfqpoint{9.775184in}{1.535193in}}{\pgfqpoint{9.777188in}{1.530355in}}{\pgfqpoint{9.780755in}{1.526789in}}%
\pgfpathcurveto{\pgfqpoint{9.784321in}{1.523222in}}{\pgfqpoint{9.789159in}{1.521219in}}{\pgfqpoint{9.794202in}{1.521219in}}%
\pgfpathclose%
\pgfusepath{fill}%
\end{pgfscope}%
\begin{pgfscope}%
\pgfpathrectangle{\pgfqpoint{6.572727in}{0.473000in}}{\pgfqpoint{4.227273in}{3.311000in}}%
\pgfusepath{clip}%
\pgfsetbuttcap%
\pgfsetroundjoin%
\definecolor{currentfill}{rgb}{0.127568,0.566949,0.550556}%
\pgfsetfillcolor{currentfill}%
\pgfsetfillopacity{0.700000}%
\pgfsetlinewidth{0.000000pt}%
\definecolor{currentstroke}{rgb}{0.000000,0.000000,0.000000}%
\pgfsetstrokecolor{currentstroke}%
\pgfsetstrokeopacity{0.700000}%
\pgfsetdash{}{0pt}%
\pgfpathmoveto{\pgfqpoint{7.971867in}{1.513913in}}%
\pgfpathcurveto{\pgfqpoint{7.976911in}{1.513913in}}{\pgfqpoint{7.981749in}{1.515917in}}{\pgfqpoint{7.985315in}{1.519483in}}%
\pgfpathcurveto{\pgfqpoint{7.988881in}{1.523050in}}{\pgfqpoint{7.990885in}{1.527887in}}{\pgfqpoint{7.990885in}{1.532931in}}%
\pgfpathcurveto{\pgfqpoint{7.990885in}{1.537975in}}{\pgfqpoint{7.988881in}{1.542812in}}{\pgfqpoint{7.985315in}{1.546379in}}%
\pgfpathcurveto{\pgfqpoint{7.981749in}{1.549945in}}{\pgfqpoint{7.976911in}{1.551949in}}{\pgfqpoint{7.971867in}{1.551949in}}%
\pgfpathcurveto{\pgfqpoint{7.966824in}{1.551949in}}{\pgfqpoint{7.961986in}{1.549945in}}{\pgfqpoint{7.958419in}{1.546379in}}%
\pgfpathcurveto{\pgfqpoint{7.954853in}{1.542812in}}{\pgfqpoint{7.952849in}{1.537975in}}{\pgfqpoint{7.952849in}{1.532931in}}%
\pgfpathcurveto{\pgfqpoint{7.952849in}{1.527887in}}{\pgfqpoint{7.954853in}{1.523050in}}{\pgfqpoint{7.958419in}{1.519483in}}%
\pgfpathcurveto{\pgfqpoint{7.961986in}{1.515917in}}{\pgfqpoint{7.966824in}{1.513913in}}{\pgfqpoint{7.971867in}{1.513913in}}%
\pgfpathclose%
\pgfusepath{fill}%
\end{pgfscope}%
\begin{pgfscope}%
\pgfpathrectangle{\pgfqpoint{6.572727in}{0.473000in}}{\pgfqpoint{4.227273in}{3.311000in}}%
\pgfusepath{clip}%
\pgfsetbuttcap%
\pgfsetroundjoin%
\definecolor{currentfill}{rgb}{0.993248,0.906157,0.143936}%
\pgfsetfillcolor{currentfill}%
\pgfsetfillopacity{0.700000}%
\pgfsetlinewidth{0.000000pt}%
\definecolor{currentstroke}{rgb}{0.000000,0.000000,0.000000}%
\pgfsetstrokecolor{currentstroke}%
\pgfsetstrokeopacity{0.700000}%
\pgfsetdash{}{0pt}%
\pgfpathmoveto{\pgfqpoint{9.284878in}{2.437007in}}%
\pgfpathcurveto{\pgfqpoint{9.289922in}{2.437007in}}{\pgfqpoint{9.294759in}{2.439011in}}{\pgfqpoint{9.298326in}{2.442577in}}%
\pgfpathcurveto{\pgfqpoint{9.301892in}{2.446144in}}{\pgfqpoint{9.303896in}{2.450982in}}{\pgfqpoint{9.303896in}{2.456025in}}%
\pgfpathcurveto{\pgfqpoint{9.303896in}{2.461069in}}{\pgfqpoint{9.301892in}{2.465907in}}{\pgfqpoint{9.298326in}{2.469473in}}%
\pgfpathcurveto{\pgfqpoint{9.294759in}{2.473040in}}{\pgfqpoint{9.289922in}{2.475043in}}{\pgfqpoint{9.284878in}{2.475043in}}%
\pgfpathcurveto{\pgfqpoint{9.279834in}{2.475043in}}{\pgfqpoint{9.274997in}{2.473040in}}{\pgfqpoint{9.271430in}{2.469473in}}%
\pgfpathcurveto{\pgfqpoint{9.267864in}{2.465907in}}{\pgfqpoint{9.265860in}{2.461069in}}{\pgfqpoint{9.265860in}{2.456025in}}%
\pgfpathcurveto{\pgfqpoint{9.265860in}{2.450982in}}{\pgfqpoint{9.267864in}{2.446144in}}{\pgfqpoint{9.271430in}{2.442577in}}%
\pgfpathcurveto{\pgfqpoint{9.274997in}{2.439011in}}{\pgfqpoint{9.279834in}{2.437007in}}{\pgfqpoint{9.284878in}{2.437007in}}%
\pgfpathclose%
\pgfusepath{fill}%
\end{pgfscope}%
\begin{pgfscope}%
\pgfpathrectangle{\pgfqpoint{6.572727in}{0.473000in}}{\pgfqpoint{4.227273in}{3.311000in}}%
\pgfusepath{clip}%
\pgfsetbuttcap%
\pgfsetroundjoin%
\definecolor{currentfill}{rgb}{0.127568,0.566949,0.550556}%
\pgfsetfillcolor{currentfill}%
\pgfsetfillopacity{0.700000}%
\pgfsetlinewidth{0.000000pt}%
\definecolor{currentstroke}{rgb}{0.000000,0.000000,0.000000}%
\pgfsetstrokecolor{currentstroke}%
\pgfsetstrokeopacity{0.700000}%
\pgfsetdash{}{0pt}%
\pgfpathmoveto{\pgfqpoint{7.469351in}{1.662633in}}%
\pgfpathcurveto{\pgfqpoint{7.474395in}{1.662633in}}{\pgfqpoint{7.479233in}{1.664637in}}{\pgfqpoint{7.482799in}{1.668203in}}%
\pgfpathcurveto{\pgfqpoint{7.486366in}{1.671770in}}{\pgfqpoint{7.488370in}{1.676608in}}{\pgfqpoint{7.488370in}{1.681651in}}%
\pgfpathcurveto{\pgfqpoint{7.488370in}{1.686695in}}{\pgfqpoint{7.486366in}{1.691533in}}{\pgfqpoint{7.482799in}{1.695099in}}%
\pgfpathcurveto{\pgfqpoint{7.479233in}{1.698666in}}{\pgfqpoint{7.474395in}{1.700669in}}{\pgfqpoint{7.469351in}{1.700669in}}%
\pgfpathcurveto{\pgfqpoint{7.464308in}{1.700669in}}{\pgfqpoint{7.459470in}{1.698666in}}{\pgfqpoint{7.455904in}{1.695099in}}%
\pgfpathcurveto{\pgfqpoint{7.452337in}{1.691533in}}{\pgfqpoint{7.450333in}{1.686695in}}{\pgfqpoint{7.450333in}{1.681651in}}%
\pgfpathcurveto{\pgfqpoint{7.450333in}{1.676608in}}{\pgfqpoint{7.452337in}{1.671770in}}{\pgfqpoint{7.455904in}{1.668203in}}%
\pgfpathcurveto{\pgfqpoint{7.459470in}{1.664637in}}{\pgfqpoint{7.464308in}{1.662633in}}{\pgfqpoint{7.469351in}{1.662633in}}%
\pgfpathclose%
\pgfusepath{fill}%
\end{pgfscope}%
\begin{pgfscope}%
\pgfpathrectangle{\pgfqpoint{6.572727in}{0.473000in}}{\pgfqpoint{4.227273in}{3.311000in}}%
\pgfusepath{clip}%
\pgfsetbuttcap%
\pgfsetroundjoin%
\definecolor{currentfill}{rgb}{0.127568,0.566949,0.550556}%
\pgfsetfillcolor{currentfill}%
\pgfsetfillopacity{0.700000}%
\pgfsetlinewidth{0.000000pt}%
\definecolor{currentstroke}{rgb}{0.000000,0.000000,0.000000}%
\pgfsetstrokecolor{currentstroke}%
\pgfsetstrokeopacity{0.700000}%
\pgfsetdash{}{0pt}%
\pgfpathmoveto{\pgfqpoint{7.865279in}{1.578964in}}%
\pgfpathcurveto{\pgfqpoint{7.870323in}{1.578964in}}{\pgfqpoint{7.875161in}{1.580968in}}{\pgfqpoint{7.878727in}{1.584534in}}%
\pgfpathcurveto{\pgfqpoint{7.882293in}{1.588101in}}{\pgfqpoint{7.884297in}{1.592939in}}{\pgfqpoint{7.884297in}{1.597982in}}%
\pgfpathcurveto{\pgfqpoint{7.884297in}{1.603026in}}{\pgfqpoint{7.882293in}{1.607864in}}{\pgfqpoint{7.878727in}{1.611430in}}%
\pgfpathcurveto{\pgfqpoint{7.875161in}{1.614997in}}{\pgfqpoint{7.870323in}{1.617000in}}{\pgfqpoint{7.865279in}{1.617000in}}%
\pgfpathcurveto{\pgfqpoint{7.860235in}{1.617000in}}{\pgfqpoint{7.855398in}{1.614997in}}{\pgfqpoint{7.851831in}{1.611430in}}%
\pgfpathcurveto{\pgfqpoint{7.848265in}{1.607864in}}{\pgfqpoint{7.846261in}{1.603026in}}{\pgfqpoint{7.846261in}{1.597982in}}%
\pgfpathcurveto{\pgfqpoint{7.846261in}{1.592939in}}{\pgfqpoint{7.848265in}{1.588101in}}{\pgfqpoint{7.851831in}{1.584534in}}%
\pgfpathcurveto{\pgfqpoint{7.855398in}{1.580968in}}{\pgfqpoint{7.860235in}{1.578964in}}{\pgfqpoint{7.865279in}{1.578964in}}%
\pgfpathclose%
\pgfusepath{fill}%
\end{pgfscope}%
\begin{pgfscope}%
\pgfpathrectangle{\pgfqpoint{6.572727in}{0.473000in}}{\pgfqpoint{4.227273in}{3.311000in}}%
\pgfusepath{clip}%
\pgfsetbuttcap%
\pgfsetroundjoin%
\definecolor{currentfill}{rgb}{0.993248,0.906157,0.143936}%
\pgfsetfillcolor{currentfill}%
\pgfsetfillopacity{0.700000}%
\pgfsetlinewidth{0.000000pt}%
\definecolor{currentstroke}{rgb}{0.000000,0.000000,0.000000}%
\pgfsetstrokecolor{currentstroke}%
\pgfsetstrokeopacity{0.700000}%
\pgfsetdash{}{0pt}%
\pgfpathmoveto{\pgfqpoint{9.115555in}{1.373505in}}%
\pgfpathcurveto{\pgfqpoint{9.120599in}{1.373505in}}{\pgfqpoint{9.125437in}{1.375509in}}{\pgfqpoint{9.129003in}{1.379075in}}%
\pgfpathcurveto{\pgfqpoint{9.132569in}{1.382642in}}{\pgfqpoint{9.134573in}{1.387479in}}{\pgfqpoint{9.134573in}{1.392523in}}%
\pgfpathcurveto{\pgfqpoint{9.134573in}{1.397567in}}{\pgfqpoint{9.132569in}{1.402404in}}{\pgfqpoint{9.129003in}{1.405971in}}%
\pgfpathcurveto{\pgfqpoint{9.125437in}{1.409537in}}{\pgfqpoint{9.120599in}{1.411541in}}{\pgfqpoint{9.115555in}{1.411541in}}%
\pgfpathcurveto{\pgfqpoint{9.110511in}{1.411541in}}{\pgfqpoint{9.105674in}{1.409537in}}{\pgfqpoint{9.102107in}{1.405971in}}%
\pgfpathcurveto{\pgfqpoint{9.098541in}{1.402404in}}{\pgfqpoint{9.096537in}{1.397567in}}{\pgfqpoint{9.096537in}{1.392523in}}%
\pgfpathcurveto{\pgfqpoint{9.096537in}{1.387479in}}{\pgfqpoint{9.098541in}{1.382642in}}{\pgfqpoint{9.102107in}{1.379075in}}%
\pgfpathcurveto{\pgfqpoint{9.105674in}{1.375509in}}{\pgfqpoint{9.110511in}{1.373505in}}{\pgfqpoint{9.115555in}{1.373505in}}%
\pgfpathclose%
\pgfusepath{fill}%
\end{pgfscope}%
\begin{pgfscope}%
\pgfpathrectangle{\pgfqpoint{6.572727in}{0.473000in}}{\pgfqpoint{4.227273in}{3.311000in}}%
\pgfusepath{clip}%
\pgfsetbuttcap%
\pgfsetroundjoin%
\definecolor{currentfill}{rgb}{0.127568,0.566949,0.550556}%
\pgfsetfillcolor{currentfill}%
\pgfsetfillopacity{0.700000}%
\pgfsetlinewidth{0.000000pt}%
\definecolor{currentstroke}{rgb}{0.000000,0.000000,0.000000}%
\pgfsetstrokecolor{currentstroke}%
\pgfsetstrokeopacity{0.700000}%
\pgfsetdash{}{0pt}%
\pgfpathmoveto{\pgfqpoint{7.285794in}{1.538947in}}%
\pgfpathcurveto{\pgfqpoint{7.290837in}{1.538947in}}{\pgfqpoint{7.295675in}{1.540951in}}{\pgfqpoint{7.299241in}{1.544518in}}%
\pgfpathcurveto{\pgfqpoint{7.302808in}{1.548084in}}{\pgfqpoint{7.304812in}{1.552922in}}{\pgfqpoint{7.304812in}{1.557965in}}%
\pgfpathcurveto{\pgfqpoint{7.304812in}{1.563009in}}{\pgfqpoint{7.302808in}{1.567847in}}{\pgfqpoint{7.299241in}{1.571413in}}%
\pgfpathcurveto{\pgfqpoint{7.295675in}{1.574980in}}{\pgfqpoint{7.290837in}{1.576984in}}{\pgfqpoint{7.285794in}{1.576984in}}%
\pgfpathcurveto{\pgfqpoint{7.280750in}{1.576984in}}{\pgfqpoint{7.275912in}{1.574980in}}{\pgfqpoint{7.272346in}{1.571413in}}%
\pgfpathcurveto{\pgfqpoint{7.268779in}{1.567847in}}{\pgfqpoint{7.266775in}{1.563009in}}{\pgfqpoint{7.266775in}{1.557965in}}%
\pgfpathcurveto{\pgfqpoint{7.266775in}{1.552922in}}{\pgfqpoint{7.268779in}{1.548084in}}{\pgfqpoint{7.272346in}{1.544518in}}%
\pgfpathcurveto{\pgfqpoint{7.275912in}{1.540951in}}{\pgfqpoint{7.280750in}{1.538947in}}{\pgfqpoint{7.285794in}{1.538947in}}%
\pgfpathclose%
\pgfusepath{fill}%
\end{pgfscope}%
\begin{pgfscope}%
\pgfpathrectangle{\pgfqpoint{6.572727in}{0.473000in}}{\pgfqpoint{4.227273in}{3.311000in}}%
\pgfusepath{clip}%
\pgfsetbuttcap%
\pgfsetroundjoin%
\definecolor{currentfill}{rgb}{0.127568,0.566949,0.550556}%
\pgfsetfillcolor{currentfill}%
\pgfsetfillopacity{0.700000}%
\pgfsetlinewidth{0.000000pt}%
\definecolor{currentstroke}{rgb}{0.000000,0.000000,0.000000}%
\pgfsetstrokecolor{currentstroke}%
\pgfsetstrokeopacity{0.700000}%
\pgfsetdash{}{0pt}%
\pgfpathmoveto{\pgfqpoint{7.825966in}{3.069994in}}%
\pgfpathcurveto{\pgfqpoint{7.831010in}{3.069994in}}{\pgfqpoint{7.835848in}{3.071998in}}{\pgfqpoint{7.839414in}{3.075564in}}%
\pgfpathcurveto{\pgfqpoint{7.842981in}{3.079131in}}{\pgfqpoint{7.844985in}{3.083968in}}{\pgfqpoint{7.844985in}{3.089012in}}%
\pgfpathcurveto{\pgfqpoint{7.844985in}{3.094056in}}{\pgfqpoint{7.842981in}{3.098893in}}{\pgfqpoint{7.839414in}{3.102460in}}%
\pgfpathcurveto{\pgfqpoint{7.835848in}{3.106026in}}{\pgfqpoint{7.831010in}{3.108030in}}{\pgfqpoint{7.825966in}{3.108030in}}%
\pgfpathcurveto{\pgfqpoint{7.820923in}{3.108030in}}{\pgfqpoint{7.816085in}{3.106026in}}{\pgfqpoint{7.812519in}{3.102460in}}%
\pgfpathcurveto{\pgfqpoint{7.808952in}{3.098893in}}{\pgfqpoint{7.806948in}{3.094056in}}{\pgfqpoint{7.806948in}{3.089012in}}%
\pgfpathcurveto{\pgfqpoint{7.806948in}{3.083968in}}{\pgfqpoint{7.808952in}{3.079131in}}{\pgfqpoint{7.812519in}{3.075564in}}%
\pgfpathcurveto{\pgfqpoint{7.816085in}{3.071998in}}{\pgfqpoint{7.820923in}{3.069994in}}{\pgfqpoint{7.825966in}{3.069994in}}%
\pgfpathclose%
\pgfusepath{fill}%
\end{pgfscope}%
\begin{pgfscope}%
\pgfpathrectangle{\pgfqpoint{6.572727in}{0.473000in}}{\pgfqpoint{4.227273in}{3.311000in}}%
\pgfusepath{clip}%
\pgfsetbuttcap%
\pgfsetroundjoin%
\definecolor{currentfill}{rgb}{0.993248,0.906157,0.143936}%
\pgfsetfillcolor{currentfill}%
\pgfsetfillopacity{0.700000}%
\pgfsetlinewidth{0.000000pt}%
\definecolor{currentstroke}{rgb}{0.000000,0.000000,0.000000}%
\pgfsetstrokecolor{currentstroke}%
\pgfsetstrokeopacity{0.700000}%
\pgfsetdash{}{0pt}%
\pgfpathmoveto{\pgfqpoint{9.959873in}{1.533714in}}%
\pgfpathcurveto{\pgfqpoint{9.964917in}{1.533714in}}{\pgfqpoint{9.969755in}{1.535718in}}{\pgfqpoint{9.973321in}{1.539284in}}%
\pgfpathcurveto{\pgfqpoint{9.976888in}{1.542851in}}{\pgfqpoint{9.978891in}{1.547689in}}{\pgfqpoint{9.978891in}{1.552732in}}%
\pgfpathcurveto{\pgfqpoint{9.978891in}{1.557776in}}{\pgfqpoint{9.976888in}{1.562614in}}{\pgfqpoint{9.973321in}{1.566180in}}%
\pgfpathcurveto{\pgfqpoint{9.969755in}{1.569747in}}{\pgfqpoint{9.964917in}{1.571750in}}{\pgfqpoint{9.959873in}{1.571750in}}%
\pgfpathcurveto{\pgfqpoint{9.954830in}{1.571750in}}{\pgfqpoint{9.949992in}{1.569747in}}{\pgfqpoint{9.946425in}{1.566180in}}%
\pgfpathcurveto{\pgfqpoint{9.942859in}{1.562614in}}{\pgfqpoint{9.940855in}{1.557776in}}{\pgfqpoint{9.940855in}{1.552732in}}%
\pgfpathcurveto{\pgfqpoint{9.940855in}{1.547689in}}{\pgfqpoint{9.942859in}{1.542851in}}{\pgfqpoint{9.946425in}{1.539284in}}%
\pgfpathcurveto{\pgfqpoint{9.949992in}{1.535718in}}{\pgfqpoint{9.954830in}{1.533714in}}{\pgfqpoint{9.959873in}{1.533714in}}%
\pgfpathclose%
\pgfusepath{fill}%
\end{pgfscope}%
\begin{pgfscope}%
\pgfpathrectangle{\pgfqpoint{6.572727in}{0.473000in}}{\pgfqpoint{4.227273in}{3.311000in}}%
\pgfusepath{clip}%
\pgfsetbuttcap%
\pgfsetroundjoin%
\definecolor{currentfill}{rgb}{0.127568,0.566949,0.550556}%
\pgfsetfillcolor{currentfill}%
\pgfsetfillopacity{0.700000}%
\pgfsetlinewidth{0.000000pt}%
\definecolor{currentstroke}{rgb}{0.000000,0.000000,0.000000}%
\pgfsetstrokecolor{currentstroke}%
\pgfsetstrokeopacity{0.700000}%
\pgfsetdash{}{0pt}%
\pgfpathmoveto{\pgfqpoint{7.498734in}{1.643500in}}%
\pgfpathcurveto{\pgfqpoint{7.503777in}{1.643500in}}{\pgfqpoint{7.508615in}{1.645504in}}{\pgfqpoint{7.512182in}{1.649071in}}%
\pgfpathcurveto{\pgfqpoint{7.515748in}{1.652637in}}{\pgfqpoint{7.517752in}{1.657475in}}{\pgfqpoint{7.517752in}{1.662519in}}%
\pgfpathcurveto{\pgfqpoint{7.517752in}{1.667562in}}{\pgfqpoint{7.515748in}{1.672400in}}{\pgfqpoint{7.512182in}{1.675966in}}%
\pgfpathcurveto{\pgfqpoint{7.508615in}{1.679533in}}{\pgfqpoint{7.503777in}{1.681537in}}{\pgfqpoint{7.498734in}{1.681537in}}%
\pgfpathcurveto{\pgfqpoint{7.493690in}{1.681537in}}{\pgfqpoint{7.488852in}{1.679533in}}{\pgfqpoint{7.485286in}{1.675966in}}%
\pgfpathcurveto{\pgfqpoint{7.481719in}{1.672400in}}{\pgfqpoint{7.479716in}{1.667562in}}{\pgfqpoint{7.479716in}{1.662519in}}%
\pgfpathcurveto{\pgfqpoint{7.479716in}{1.657475in}}{\pgfqpoint{7.481719in}{1.652637in}}{\pgfqpoint{7.485286in}{1.649071in}}%
\pgfpathcurveto{\pgfqpoint{7.488852in}{1.645504in}}{\pgfqpoint{7.493690in}{1.643500in}}{\pgfqpoint{7.498734in}{1.643500in}}%
\pgfpathclose%
\pgfusepath{fill}%
\end{pgfscope}%
\begin{pgfscope}%
\pgfpathrectangle{\pgfqpoint{6.572727in}{0.473000in}}{\pgfqpoint{4.227273in}{3.311000in}}%
\pgfusepath{clip}%
\pgfsetbuttcap%
\pgfsetroundjoin%
\definecolor{currentfill}{rgb}{0.127568,0.566949,0.550556}%
\pgfsetfillcolor{currentfill}%
\pgfsetfillopacity{0.700000}%
\pgfsetlinewidth{0.000000pt}%
\definecolor{currentstroke}{rgb}{0.000000,0.000000,0.000000}%
\pgfsetstrokecolor{currentstroke}%
\pgfsetstrokeopacity{0.700000}%
\pgfsetdash{}{0pt}%
\pgfpathmoveto{\pgfqpoint{8.024086in}{3.326694in}}%
\pgfpathcurveto{\pgfqpoint{8.029130in}{3.326694in}}{\pgfqpoint{8.033967in}{3.328697in}}{\pgfqpoint{8.037534in}{3.332264in}}%
\pgfpathcurveto{\pgfqpoint{8.041100in}{3.335830in}}{\pgfqpoint{8.043104in}{3.340668in}}{\pgfqpoint{8.043104in}{3.345712in}}%
\pgfpathcurveto{\pgfqpoint{8.043104in}{3.350755in}}{\pgfqpoint{8.041100in}{3.355593in}}{\pgfqpoint{8.037534in}{3.359160in}}%
\pgfpathcurveto{\pgfqpoint{8.033967in}{3.362726in}}{\pgfqpoint{8.029130in}{3.364730in}}{\pgfqpoint{8.024086in}{3.364730in}}%
\pgfpathcurveto{\pgfqpoint{8.019042in}{3.364730in}}{\pgfqpoint{8.014205in}{3.362726in}}{\pgfqpoint{8.010638in}{3.359160in}}%
\pgfpathcurveto{\pgfqpoint{8.007072in}{3.355593in}}{\pgfqpoint{8.005068in}{3.350755in}}{\pgfqpoint{8.005068in}{3.345712in}}%
\pgfpathcurveto{\pgfqpoint{8.005068in}{3.340668in}}{\pgfqpoint{8.007072in}{3.335830in}}{\pgfqpoint{8.010638in}{3.332264in}}%
\pgfpathcurveto{\pgfqpoint{8.014205in}{3.328697in}}{\pgfqpoint{8.019042in}{3.326694in}}{\pgfqpoint{8.024086in}{3.326694in}}%
\pgfpathclose%
\pgfusepath{fill}%
\end{pgfscope}%
\begin{pgfscope}%
\pgfpathrectangle{\pgfqpoint{6.572727in}{0.473000in}}{\pgfqpoint{4.227273in}{3.311000in}}%
\pgfusepath{clip}%
\pgfsetbuttcap%
\pgfsetroundjoin%
\definecolor{currentfill}{rgb}{0.127568,0.566949,0.550556}%
\pgfsetfillcolor{currentfill}%
\pgfsetfillopacity{0.700000}%
\pgfsetlinewidth{0.000000pt}%
\definecolor{currentstroke}{rgb}{0.000000,0.000000,0.000000}%
\pgfsetstrokecolor{currentstroke}%
\pgfsetstrokeopacity{0.700000}%
\pgfsetdash{}{0pt}%
\pgfpathmoveto{\pgfqpoint{8.437606in}{1.421266in}}%
\pgfpathcurveto{\pgfqpoint{8.442650in}{1.421266in}}{\pgfqpoint{8.447487in}{1.423270in}}{\pgfqpoint{8.451054in}{1.426837in}}%
\pgfpathcurveto{\pgfqpoint{8.454620in}{1.430403in}}{\pgfqpoint{8.456624in}{1.435241in}}{\pgfqpoint{8.456624in}{1.440285in}}%
\pgfpathcurveto{\pgfqpoint{8.456624in}{1.445328in}}{\pgfqpoint{8.454620in}{1.450166in}}{\pgfqpoint{8.451054in}{1.453732in}}%
\pgfpathcurveto{\pgfqpoint{8.447487in}{1.457299in}}{\pgfqpoint{8.442650in}{1.459303in}}{\pgfqpoint{8.437606in}{1.459303in}}%
\pgfpathcurveto{\pgfqpoint{8.432562in}{1.459303in}}{\pgfqpoint{8.427724in}{1.457299in}}{\pgfqpoint{8.424158in}{1.453732in}}%
\pgfpathcurveto{\pgfqpoint{8.420592in}{1.450166in}}{\pgfqpoint{8.418588in}{1.445328in}}{\pgfqpoint{8.418588in}{1.440285in}}%
\pgfpathcurveto{\pgfqpoint{8.418588in}{1.435241in}}{\pgfqpoint{8.420592in}{1.430403in}}{\pgfqpoint{8.424158in}{1.426837in}}%
\pgfpathcurveto{\pgfqpoint{8.427724in}{1.423270in}}{\pgfqpoint{8.432562in}{1.421266in}}{\pgfqpoint{8.437606in}{1.421266in}}%
\pgfpathclose%
\pgfusepath{fill}%
\end{pgfscope}%
\begin{pgfscope}%
\pgfpathrectangle{\pgfqpoint{6.572727in}{0.473000in}}{\pgfqpoint{4.227273in}{3.311000in}}%
\pgfusepath{clip}%
\pgfsetbuttcap%
\pgfsetroundjoin%
\definecolor{currentfill}{rgb}{0.127568,0.566949,0.550556}%
\pgfsetfillcolor{currentfill}%
\pgfsetfillopacity{0.700000}%
\pgfsetlinewidth{0.000000pt}%
\definecolor{currentstroke}{rgb}{0.000000,0.000000,0.000000}%
\pgfsetstrokecolor{currentstroke}%
\pgfsetstrokeopacity{0.700000}%
\pgfsetdash{}{0pt}%
\pgfpathmoveto{\pgfqpoint{7.658236in}{1.286948in}}%
\pgfpathcurveto{\pgfqpoint{7.663280in}{1.286948in}}{\pgfqpoint{7.668117in}{1.288952in}}{\pgfqpoint{7.671684in}{1.292518in}}%
\pgfpathcurveto{\pgfqpoint{7.675250in}{1.296085in}}{\pgfqpoint{7.677254in}{1.300923in}}{\pgfqpoint{7.677254in}{1.305966in}}%
\pgfpathcurveto{\pgfqpoint{7.677254in}{1.311010in}}{\pgfqpoint{7.675250in}{1.315848in}}{\pgfqpoint{7.671684in}{1.319414in}}%
\pgfpathcurveto{\pgfqpoint{7.668117in}{1.322980in}}{\pgfqpoint{7.663280in}{1.324984in}}{\pgfqpoint{7.658236in}{1.324984in}}%
\pgfpathcurveto{\pgfqpoint{7.653192in}{1.324984in}}{\pgfqpoint{7.648354in}{1.322980in}}{\pgfqpoint{7.644788in}{1.319414in}}%
\pgfpathcurveto{\pgfqpoint{7.641222in}{1.315848in}}{\pgfqpoint{7.639218in}{1.311010in}}{\pgfqpoint{7.639218in}{1.305966in}}%
\pgfpathcurveto{\pgfqpoint{7.639218in}{1.300923in}}{\pgfqpoint{7.641222in}{1.296085in}}{\pgfqpoint{7.644788in}{1.292518in}}%
\pgfpathcurveto{\pgfqpoint{7.648354in}{1.288952in}}{\pgfqpoint{7.653192in}{1.286948in}}{\pgfqpoint{7.658236in}{1.286948in}}%
\pgfpathclose%
\pgfusepath{fill}%
\end{pgfscope}%
\begin{pgfscope}%
\pgfpathrectangle{\pgfqpoint{6.572727in}{0.473000in}}{\pgfqpoint{4.227273in}{3.311000in}}%
\pgfusepath{clip}%
\pgfsetbuttcap%
\pgfsetroundjoin%
\definecolor{currentfill}{rgb}{0.127568,0.566949,0.550556}%
\pgfsetfillcolor{currentfill}%
\pgfsetfillopacity{0.700000}%
\pgfsetlinewidth{0.000000pt}%
\definecolor{currentstroke}{rgb}{0.000000,0.000000,0.000000}%
\pgfsetstrokecolor{currentstroke}%
\pgfsetstrokeopacity{0.700000}%
\pgfsetdash{}{0pt}%
\pgfpathmoveto{\pgfqpoint{7.115113in}{1.654371in}}%
\pgfpathcurveto{\pgfqpoint{7.120156in}{1.654371in}}{\pgfqpoint{7.124994in}{1.656374in}}{\pgfqpoint{7.128560in}{1.659941in}}%
\pgfpathcurveto{\pgfqpoint{7.132127in}{1.663507in}}{\pgfqpoint{7.134131in}{1.668345in}}{\pgfqpoint{7.134131in}{1.673389in}}%
\pgfpathcurveto{\pgfqpoint{7.134131in}{1.678432in}}{\pgfqpoint{7.132127in}{1.683270in}}{\pgfqpoint{7.128560in}{1.686837in}}%
\pgfpathcurveto{\pgfqpoint{7.124994in}{1.690403in}}{\pgfqpoint{7.120156in}{1.692407in}}{\pgfqpoint{7.115113in}{1.692407in}}%
\pgfpathcurveto{\pgfqpoint{7.110069in}{1.692407in}}{\pgfqpoint{7.105231in}{1.690403in}}{\pgfqpoint{7.101665in}{1.686837in}}%
\pgfpathcurveto{\pgfqpoint{7.098098in}{1.683270in}}{\pgfqpoint{7.096094in}{1.678432in}}{\pgfqpoint{7.096094in}{1.673389in}}%
\pgfpathcurveto{\pgfqpoint{7.096094in}{1.668345in}}{\pgfqpoint{7.098098in}{1.663507in}}{\pgfqpoint{7.101665in}{1.659941in}}%
\pgfpathcurveto{\pgfqpoint{7.105231in}{1.656374in}}{\pgfqpoint{7.110069in}{1.654371in}}{\pgfqpoint{7.115113in}{1.654371in}}%
\pgfpathclose%
\pgfusepath{fill}%
\end{pgfscope}%
\begin{pgfscope}%
\pgfpathrectangle{\pgfqpoint{6.572727in}{0.473000in}}{\pgfqpoint{4.227273in}{3.311000in}}%
\pgfusepath{clip}%
\pgfsetbuttcap%
\pgfsetroundjoin%
\definecolor{currentfill}{rgb}{0.127568,0.566949,0.550556}%
\pgfsetfillcolor{currentfill}%
\pgfsetfillopacity{0.700000}%
\pgfsetlinewidth{0.000000pt}%
\definecolor{currentstroke}{rgb}{0.000000,0.000000,0.000000}%
\pgfsetstrokecolor{currentstroke}%
\pgfsetstrokeopacity{0.700000}%
\pgfsetdash{}{0pt}%
\pgfpathmoveto{\pgfqpoint{8.668257in}{3.168574in}}%
\pgfpathcurveto{\pgfqpoint{8.673300in}{3.168574in}}{\pgfqpoint{8.678138in}{3.170578in}}{\pgfqpoint{8.681705in}{3.174145in}}%
\pgfpathcurveto{\pgfqpoint{8.685271in}{3.177711in}}{\pgfqpoint{8.687275in}{3.182549in}}{\pgfqpoint{8.687275in}{3.187592in}}%
\pgfpathcurveto{\pgfqpoint{8.687275in}{3.192636in}}{\pgfqpoint{8.685271in}{3.197474in}}{\pgfqpoint{8.681705in}{3.201040in}}%
\pgfpathcurveto{\pgfqpoint{8.678138in}{3.204607in}}{\pgfqpoint{8.673300in}{3.206611in}}{\pgfqpoint{8.668257in}{3.206611in}}%
\pgfpathcurveto{\pgfqpoint{8.663213in}{3.206611in}}{\pgfqpoint{8.658375in}{3.204607in}}{\pgfqpoint{8.654809in}{3.201040in}}%
\pgfpathcurveto{\pgfqpoint{8.651243in}{3.197474in}}{\pgfqpoint{8.649239in}{3.192636in}}{\pgfqpoint{8.649239in}{3.187592in}}%
\pgfpathcurveto{\pgfqpoint{8.649239in}{3.182549in}}{\pgfqpoint{8.651243in}{3.177711in}}{\pgfqpoint{8.654809in}{3.174145in}}%
\pgfpathcurveto{\pgfqpoint{8.658375in}{3.170578in}}{\pgfqpoint{8.663213in}{3.168574in}}{\pgfqpoint{8.668257in}{3.168574in}}%
\pgfpathclose%
\pgfusepath{fill}%
\end{pgfscope}%
\begin{pgfscope}%
\pgfpathrectangle{\pgfqpoint{6.572727in}{0.473000in}}{\pgfqpoint{4.227273in}{3.311000in}}%
\pgfusepath{clip}%
\pgfsetbuttcap%
\pgfsetroundjoin%
\definecolor{currentfill}{rgb}{0.993248,0.906157,0.143936}%
\pgfsetfillcolor{currentfill}%
\pgfsetfillopacity{0.700000}%
\pgfsetlinewidth{0.000000pt}%
\definecolor{currentstroke}{rgb}{0.000000,0.000000,0.000000}%
\pgfsetstrokecolor{currentstroke}%
\pgfsetstrokeopacity{0.700000}%
\pgfsetdash{}{0pt}%
\pgfpathmoveto{\pgfqpoint{9.451032in}{1.264246in}}%
\pgfpathcurveto{\pgfqpoint{9.456076in}{1.264246in}}{\pgfqpoint{9.460914in}{1.266250in}}{\pgfqpoint{9.464480in}{1.269816in}}%
\pgfpathcurveto{\pgfqpoint{9.468047in}{1.273383in}}{\pgfqpoint{9.470050in}{1.278221in}}{\pgfqpoint{9.470050in}{1.283264in}}%
\pgfpathcurveto{\pgfqpoint{9.470050in}{1.288308in}}{\pgfqpoint{9.468047in}{1.293146in}}{\pgfqpoint{9.464480in}{1.296712in}}%
\pgfpathcurveto{\pgfqpoint{9.460914in}{1.300279in}}{\pgfqpoint{9.456076in}{1.302282in}}{\pgfqpoint{9.451032in}{1.302282in}}%
\pgfpathcurveto{\pgfqpoint{9.445989in}{1.302282in}}{\pgfqpoint{9.441151in}{1.300279in}}{\pgfqpoint{9.437584in}{1.296712in}}%
\pgfpathcurveto{\pgfqpoint{9.434018in}{1.293146in}}{\pgfqpoint{9.432014in}{1.288308in}}{\pgfqpoint{9.432014in}{1.283264in}}%
\pgfpathcurveto{\pgfqpoint{9.432014in}{1.278221in}}{\pgfqpoint{9.434018in}{1.273383in}}{\pgfqpoint{9.437584in}{1.269816in}}%
\pgfpathcurveto{\pgfqpoint{9.441151in}{1.266250in}}{\pgfqpoint{9.445989in}{1.264246in}}{\pgfqpoint{9.451032in}{1.264246in}}%
\pgfpathclose%
\pgfusepath{fill}%
\end{pgfscope}%
\begin{pgfscope}%
\pgfpathrectangle{\pgfqpoint{6.572727in}{0.473000in}}{\pgfqpoint{4.227273in}{3.311000in}}%
\pgfusepath{clip}%
\pgfsetbuttcap%
\pgfsetroundjoin%
\definecolor{currentfill}{rgb}{0.127568,0.566949,0.550556}%
\pgfsetfillcolor{currentfill}%
\pgfsetfillopacity{0.700000}%
\pgfsetlinewidth{0.000000pt}%
\definecolor{currentstroke}{rgb}{0.000000,0.000000,0.000000}%
\pgfsetstrokecolor{currentstroke}%
\pgfsetstrokeopacity{0.700000}%
\pgfsetdash{}{0pt}%
\pgfpathmoveto{\pgfqpoint{7.725862in}{1.810215in}}%
\pgfpathcurveto{\pgfqpoint{7.730905in}{1.810215in}}{\pgfqpoint{7.735743in}{1.812219in}}{\pgfqpoint{7.739310in}{1.815786in}}%
\pgfpathcurveto{\pgfqpoint{7.742876in}{1.819352in}}{\pgfqpoint{7.744880in}{1.824190in}}{\pgfqpoint{7.744880in}{1.829233in}}%
\pgfpathcurveto{\pgfqpoint{7.744880in}{1.834277in}}{\pgfqpoint{7.742876in}{1.839115in}}{\pgfqpoint{7.739310in}{1.842681in}}%
\pgfpathcurveto{\pgfqpoint{7.735743in}{1.846248in}}{\pgfqpoint{7.730905in}{1.848252in}}{\pgfqpoint{7.725862in}{1.848252in}}%
\pgfpathcurveto{\pgfqpoint{7.720818in}{1.848252in}}{\pgfqpoint{7.715980in}{1.846248in}}{\pgfqpoint{7.712414in}{1.842681in}}%
\pgfpathcurveto{\pgfqpoint{7.708847in}{1.839115in}}{\pgfqpoint{7.706844in}{1.834277in}}{\pgfqpoint{7.706844in}{1.829233in}}%
\pgfpathcurveto{\pgfqpoint{7.706844in}{1.824190in}}{\pgfqpoint{7.708847in}{1.819352in}}{\pgfqpoint{7.712414in}{1.815786in}}%
\pgfpathcurveto{\pgfqpoint{7.715980in}{1.812219in}}{\pgfqpoint{7.720818in}{1.810215in}}{\pgfqpoint{7.725862in}{1.810215in}}%
\pgfpathclose%
\pgfusepath{fill}%
\end{pgfscope}%
\begin{pgfscope}%
\pgfpathrectangle{\pgfqpoint{6.572727in}{0.473000in}}{\pgfqpoint{4.227273in}{3.311000in}}%
\pgfusepath{clip}%
\pgfsetbuttcap%
\pgfsetroundjoin%
\definecolor{currentfill}{rgb}{0.993248,0.906157,0.143936}%
\pgfsetfillcolor{currentfill}%
\pgfsetfillopacity{0.700000}%
\pgfsetlinewidth{0.000000pt}%
\definecolor{currentstroke}{rgb}{0.000000,0.000000,0.000000}%
\pgfsetstrokecolor{currentstroke}%
\pgfsetstrokeopacity{0.700000}%
\pgfsetdash{}{0pt}%
\pgfpathmoveto{\pgfqpoint{9.831010in}{1.601745in}}%
\pgfpathcurveto{\pgfqpoint{9.836054in}{1.601745in}}{\pgfqpoint{9.840892in}{1.603749in}}{\pgfqpoint{9.844458in}{1.607315in}}%
\pgfpathcurveto{\pgfqpoint{9.848025in}{1.610882in}}{\pgfqpoint{9.850028in}{1.615720in}}{\pgfqpoint{9.850028in}{1.620763in}}%
\pgfpathcurveto{\pgfqpoint{9.850028in}{1.625807in}}{\pgfqpoint{9.848025in}{1.630645in}}{\pgfqpoint{9.844458in}{1.634211in}}%
\pgfpathcurveto{\pgfqpoint{9.840892in}{1.637778in}}{\pgfqpoint{9.836054in}{1.639781in}}{\pgfqpoint{9.831010in}{1.639781in}}%
\pgfpathcurveto{\pgfqpoint{9.825967in}{1.639781in}}{\pgfqpoint{9.821129in}{1.637778in}}{\pgfqpoint{9.817562in}{1.634211in}}%
\pgfpathcurveto{\pgfqpoint{9.813996in}{1.630645in}}{\pgfqpoint{9.811992in}{1.625807in}}{\pgfqpoint{9.811992in}{1.620763in}}%
\pgfpathcurveto{\pgfqpoint{9.811992in}{1.615720in}}{\pgfqpoint{9.813996in}{1.610882in}}{\pgfqpoint{9.817562in}{1.607315in}}%
\pgfpathcurveto{\pgfqpoint{9.821129in}{1.603749in}}{\pgfqpoint{9.825967in}{1.601745in}}{\pgfqpoint{9.831010in}{1.601745in}}%
\pgfpathclose%
\pgfusepath{fill}%
\end{pgfscope}%
\begin{pgfscope}%
\pgfpathrectangle{\pgfqpoint{6.572727in}{0.473000in}}{\pgfqpoint{4.227273in}{3.311000in}}%
\pgfusepath{clip}%
\pgfsetbuttcap%
\pgfsetroundjoin%
\definecolor{currentfill}{rgb}{0.127568,0.566949,0.550556}%
\pgfsetfillcolor{currentfill}%
\pgfsetfillopacity{0.700000}%
\pgfsetlinewidth{0.000000pt}%
\definecolor{currentstroke}{rgb}{0.000000,0.000000,0.000000}%
\pgfsetstrokecolor{currentstroke}%
\pgfsetstrokeopacity{0.700000}%
\pgfsetdash{}{0pt}%
\pgfpathmoveto{\pgfqpoint{8.677976in}{2.747187in}}%
\pgfpathcurveto{\pgfqpoint{8.683019in}{2.747187in}}{\pgfqpoint{8.687857in}{2.749191in}}{\pgfqpoint{8.691424in}{2.752757in}}%
\pgfpathcurveto{\pgfqpoint{8.694990in}{2.756324in}}{\pgfqpoint{8.696994in}{2.761161in}}{\pgfqpoint{8.696994in}{2.766205in}}%
\pgfpathcurveto{\pgfqpoint{8.696994in}{2.771249in}}{\pgfqpoint{8.694990in}{2.776087in}}{\pgfqpoint{8.691424in}{2.779653in}}%
\pgfpathcurveto{\pgfqpoint{8.687857in}{2.783219in}}{\pgfqpoint{8.683019in}{2.785223in}}{\pgfqpoint{8.677976in}{2.785223in}}%
\pgfpathcurveto{\pgfqpoint{8.672932in}{2.785223in}}{\pgfqpoint{8.668094in}{2.783219in}}{\pgfqpoint{8.664528in}{2.779653in}}%
\pgfpathcurveto{\pgfqpoint{8.660961in}{2.776087in}}{\pgfqpoint{8.658958in}{2.771249in}}{\pgfqpoint{8.658958in}{2.766205in}}%
\pgfpathcurveto{\pgfqpoint{8.658958in}{2.761161in}}{\pgfqpoint{8.660961in}{2.756324in}}{\pgfqpoint{8.664528in}{2.752757in}}%
\pgfpathcurveto{\pgfqpoint{8.668094in}{2.749191in}}{\pgfqpoint{8.672932in}{2.747187in}}{\pgfqpoint{8.677976in}{2.747187in}}%
\pgfpathclose%
\pgfusepath{fill}%
\end{pgfscope}%
\begin{pgfscope}%
\pgfpathrectangle{\pgfqpoint{6.572727in}{0.473000in}}{\pgfqpoint{4.227273in}{3.311000in}}%
\pgfusepath{clip}%
\pgfsetbuttcap%
\pgfsetroundjoin%
\definecolor{currentfill}{rgb}{0.993248,0.906157,0.143936}%
\pgfsetfillcolor{currentfill}%
\pgfsetfillopacity{0.700000}%
\pgfsetlinewidth{0.000000pt}%
\definecolor{currentstroke}{rgb}{0.000000,0.000000,0.000000}%
\pgfsetstrokecolor{currentstroke}%
\pgfsetstrokeopacity{0.700000}%
\pgfsetdash{}{0pt}%
\pgfpathmoveto{\pgfqpoint{9.400929in}{1.168208in}}%
\pgfpathcurveto{\pgfqpoint{9.405973in}{1.168208in}}{\pgfqpoint{9.410810in}{1.170212in}}{\pgfqpoint{9.414377in}{1.173779in}}%
\pgfpathcurveto{\pgfqpoint{9.417943in}{1.177345in}}{\pgfqpoint{9.419947in}{1.182183in}}{\pgfqpoint{9.419947in}{1.187226in}}%
\pgfpathcurveto{\pgfqpoint{9.419947in}{1.192270in}}{\pgfqpoint{9.417943in}{1.197108in}}{\pgfqpoint{9.414377in}{1.200674in}}%
\pgfpathcurveto{\pgfqpoint{9.410810in}{1.204241in}}{\pgfqpoint{9.405973in}{1.206245in}}{\pgfqpoint{9.400929in}{1.206245in}}%
\pgfpathcurveto{\pgfqpoint{9.395885in}{1.206245in}}{\pgfqpoint{9.391048in}{1.204241in}}{\pgfqpoint{9.387481in}{1.200674in}}%
\pgfpathcurveto{\pgfqpoint{9.383915in}{1.197108in}}{\pgfqpoint{9.381911in}{1.192270in}}{\pgfqpoint{9.381911in}{1.187226in}}%
\pgfpathcurveto{\pgfqpoint{9.381911in}{1.182183in}}{\pgfqpoint{9.383915in}{1.177345in}}{\pgfqpoint{9.387481in}{1.173779in}}%
\pgfpathcurveto{\pgfqpoint{9.391048in}{1.170212in}}{\pgfqpoint{9.395885in}{1.168208in}}{\pgfqpoint{9.400929in}{1.168208in}}%
\pgfpathclose%
\pgfusepath{fill}%
\end{pgfscope}%
\begin{pgfscope}%
\pgfpathrectangle{\pgfqpoint{6.572727in}{0.473000in}}{\pgfqpoint{4.227273in}{3.311000in}}%
\pgfusepath{clip}%
\pgfsetbuttcap%
\pgfsetroundjoin%
\definecolor{currentfill}{rgb}{0.127568,0.566949,0.550556}%
\pgfsetfillcolor{currentfill}%
\pgfsetfillopacity{0.700000}%
\pgfsetlinewidth{0.000000pt}%
\definecolor{currentstroke}{rgb}{0.000000,0.000000,0.000000}%
\pgfsetstrokecolor{currentstroke}%
\pgfsetstrokeopacity{0.700000}%
\pgfsetdash{}{0pt}%
\pgfpathmoveto{\pgfqpoint{8.138448in}{1.628354in}}%
\pgfpathcurveto{\pgfqpoint{8.143492in}{1.628354in}}{\pgfqpoint{8.148329in}{1.630358in}}{\pgfqpoint{8.151896in}{1.633925in}}%
\pgfpathcurveto{\pgfqpoint{8.155462in}{1.637491in}}{\pgfqpoint{8.157466in}{1.642329in}}{\pgfqpoint{8.157466in}{1.647372in}}%
\pgfpathcurveto{\pgfqpoint{8.157466in}{1.652416in}}{\pgfqpoint{8.155462in}{1.657254in}}{\pgfqpoint{8.151896in}{1.660820in}}%
\pgfpathcurveto{\pgfqpoint{8.148329in}{1.664387in}}{\pgfqpoint{8.143492in}{1.666391in}}{\pgfqpoint{8.138448in}{1.666391in}}%
\pgfpathcurveto{\pgfqpoint{8.133404in}{1.666391in}}{\pgfqpoint{8.128567in}{1.664387in}}{\pgfqpoint{8.125000in}{1.660820in}}%
\pgfpathcurveto{\pgfqpoint{8.121434in}{1.657254in}}{\pgfqpoint{8.119430in}{1.652416in}}{\pgfqpoint{8.119430in}{1.647372in}}%
\pgfpathcurveto{\pgfqpoint{8.119430in}{1.642329in}}{\pgfqpoint{8.121434in}{1.637491in}}{\pgfqpoint{8.125000in}{1.633925in}}%
\pgfpathcurveto{\pgfqpoint{8.128567in}{1.630358in}}{\pgfqpoint{8.133404in}{1.628354in}}{\pgfqpoint{8.138448in}{1.628354in}}%
\pgfpathclose%
\pgfusepath{fill}%
\end{pgfscope}%
\begin{pgfscope}%
\pgfpathrectangle{\pgfqpoint{6.572727in}{0.473000in}}{\pgfqpoint{4.227273in}{3.311000in}}%
\pgfusepath{clip}%
\pgfsetbuttcap%
\pgfsetroundjoin%
\definecolor{currentfill}{rgb}{0.993248,0.906157,0.143936}%
\pgfsetfillcolor{currentfill}%
\pgfsetfillopacity{0.700000}%
\pgfsetlinewidth{0.000000pt}%
\definecolor{currentstroke}{rgb}{0.000000,0.000000,0.000000}%
\pgfsetstrokecolor{currentstroke}%
\pgfsetstrokeopacity{0.700000}%
\pgfsetdash{}{0pt}%
\pgfpathmoveto{\pgfqpoint{9.633024in}{1.410431in}}%
\pgfpathcurveto{\pgfqpoint{9.638067in}{1.410431in}}{\pgfqpoint{9.642905in}{1.412434in}}{\pgfqpoint{9.646472in}{1.416001in}}%
\pgfpathcurveto{\pgfqpoint{9.650038in}{1.419567in}}{\pgfqpoint{9.652042in}{1.424405in}}{\pgfqpoint{9.652042in}{1.429449in}}%
\pgfpathcurveto{\pgfqpoint{9.652042in}{1.434492in}}{\pgfqpoint{9.650038in}{1.439330in}}{\pgfqpoint{9.646472in}{1.442897in}}%
\pgfpathcurveto{\pgfqpoint{9.642905in}{1.446463in}}{\pgfqpoint{9.638067in}{1.448467in}}{\pgfqpoint{9.633024in}{1.448467in}}%
\pgfpathcurveto{\pgfqpoint{9.627980in}{1.448467in}}{\pgfqpoint{9.623142in}{1.446463in}}{\pgfqpoint{9.619576in}{1.442897in}}%
\pgfpathcurveto{\pgfqpoint{9.616009in}{1.439330in}}{\pgfqpoint{9.614006in}{1.434492in}}{\pgfqpoint{9.614006in}{1.429449in}}%
\pgfpathcurveto{\pgfqpoint{9.614006in}{1.424405in}}{\pgfqpoint{9.616009in}{1.419567in}}{\pgfqpoint{9.619576in}{1.416001in}}%
\pgfpathcurveto{\pgfqpoint{9.623142in}{1.412434in}}{\pgfqpoint{9.627980in}{1.410431in}}{\pgfqpoint{9.633024in}{1.410431in}}%
\pgfpathclose%
\pgfusepath{fill}%
\end{pgfscope}%
\begin{pgfscope}%
\pgfpathrectangle{\pgfqpoint{6.572727in}{0.473000in}}{\pgfqpoint{4.227273in}{3.311000in}}%
\pgfusepath{clip}%
\pgfsetbuttcap%
\pgfsetroundjoin%
\definecolor{currentfill}{rgb}{0.127568,0.566949,0.550556}%
\pgfsetfillcolor{currentfill}%
\pgfsetfillopacity{0.700000}%
\pgfsetlinewidth{0.000000pt}%
\definecolor{currentstroke}{rgb}{0.000000,0.000000,0.000000}%
\pgfsetstrokecolor{currentstroke}%
\pgfsetstrokeopacity{0.700000}%
\pgfsetdash{}{0pt}%
\pgfpathmoveto{\pgfqpoint{8.248668in}{3.404855in}}%
\pgfpathcurveto{\pgfqpoint{8.253712in}{3.404855in}}{\pgfqpoint{8.258549in}{3.406859in}}{\pgfqpoint{8.262116in}{3.410425in}}%
\pgfpathcurveto{\pgfqpoint{8.265682in}{3.413992in}}{\pgfqpoint{8.267686in}{3.418829in}}{\pgfqpoint{8.267686in}{3.423873in}}%
\pgfpathcurveto{\pgfqpoint{8.267686in}{3.428917in}}{\pgfqpoint{8.265682in}{3.433754in}}{\pgfqpoint{8.262116in}{3.437321in}}%
\pgfpathcurveto{\pgfqpoint{8.258549in}{3.440887in}}{\pgfqpoint{8.253712in}{3.442891in}}{\pgfqpoint{8.248668in}{3.442891in}}%
\pgfpathcurveto{\pgfqpoint{8.243624in}{3.442891in}}{\pgfqpoint{8.238787in}{3.440887in}}{\pgfqpoint{8.235220in}{3.437321in}}%
\pgfpathcurveto{\pgfqpoint{8.231654in}{3.433754in}}{\pgfqpoint{8.229650in}{3.428917in}}{\pgfqpoint{8.229650in}{3.423873in}}%
\pgfpathcurveto{\pgfqpoint{8.229650in}{3.418829in}}{\pgfqpoint{8.231654in}{3.413992in}}{\pgfqpoint{8.235220in}{3.410425in}}%
\pgfpathcurveto{\pgfqpoint{8.238787in}{3.406859in}}{\pgfqpoint{8.243624in}{3.404855in}}{\pgfqpoint{8.248668in}{3.404855in}}%
\pgfpathclose%
\pgfusepath{fill}%
\end{pgfscope}%
\begin{pgfscope}%
\pgfpathrectangle{\pgfqpoint{6.572727in}{0.473000in}}{\pgfqpoint{4.227273in}{3.311000in}}%
\pgfusepath{clip}%
\pgfsetbuttcap%
\pgfsetroundjoin%
\definecolor{currentfill}{rgb}{0.127568,0.566949,0.550556}%
\pgfsetfillcolor{currentfill}%
\pgfsetfillopacity{0.700000}%
\pgfsetlinewidth{0.000000pt}%
\definecolor{currentstroke}{rgb}{0.000000,0.000000,0.000000}%
\pgfsetstrokecolor{currentstroke}%
\pgfsetstrokeopacity{0.700000}%
\pgfsetdash{}{0pt}%
\pgfpathmoveto{\pgfqpoint{7.897783in}{2.887041in}}%
\pgfpathcurveto{\pgfqpoint{7.902826in}{2.887041in}}{\pgfqpoint{7.907664in}{2.889044in}}{\pgfqpoint{7.911230in}{2.892611in}}%
\pgfpathcurveto{\pgfqpoint{7.914797in}{2.896177in}}{\pgfqpoint{7.916801in}{2.901015in}}{\pgfqpoint{7.916801in}{2.906059in}}%
\pgfpathcurveto{\pgfqpoint{7.916801in}{2.911102in}}{\pgfqpoint{7.914797in}{2.915940in}}{\pgfqpoint{7.911230in}{2.919507in}}%
\pgfpathcurveto{\pgfqpoint{7.907664in}{2.923073in}}{\pgfqpoint{7.902826in}{2.925077in}}{\pgfqpoint{7.897783in}{2.925077in}}%
\pgfpathcurveto{\pgfqpoint{7.892739in}{2.925077in}}{\pgfqpoint{7.887901in}{2.923073in}}{\pgfqpoint{7.884335in}{2.919507in}}%
\pgfpathcurveto{\pgfqpoint{7.880768in}{2.915940in}}{\pgfqpoint{7.878764in}{2.911102in}}{\pgfqpoint{7.878764in}{2.906059in}}%
\pgfpathcurveto{\pgfqpoint{7.878764in}{2.901015in}}{\pgfqpoint{7.880768in}{2.896177in}}{\pgfqpoint{7.884335in}{2.892611in}}%
\pgfpathcurveto{\pgfqpoint{7.887901in}{2.889044in}}{\pgfqpoint{7.892739in}{2.887041in}}{\pgfqpoint{7.897783in}{2.887041in}}%
\pgfpathclose%
\pgfusepath{fill}%
\end{pgfscope}%
\begin{pgfscope}%
\pgfpathrectangle{\pgfqpoint{6.572727in}{0.473000in}}{\pgfqpoint{4.227273in}{3.311000in}}%
\pgfusepath{clip}%
\pgfsetbuttcap%
\pgfsetroundjoin%
\definecolor{currentfill}{rgb}{0.993248,0.906157,0.143936}%
\pgfsetfillcolor{currentfill}%
\pgfsetfillopacity{0.700000}%
\pgfsetlinewidth{0.000000pt}%
\definecolor{currentstroke}{rgb}{0.000000,0.000000,0.000000}%
\pgfsetstrokecolor{currentstroke}%
\pgfsetstrokeopacity{0.700000}%
\pgfsetdash{}{0pt}%
\pgfpathmoveto{\pgfqpoint{10.106096in}{1.444584in}}%
\pgfpathcurveto{\pgfqpoint{10.111139in}{1.444584in}}{\pgfqpoint{10.115977in}{1.446587in}}{\pgfqpoint{10.119543in}{1.450154in}}%
\pgfpathcurveto{\pgfqpoint{10.123110in}{1.453720in}}{\pgfqpoint{10.125114in}{1.458558in}}{\pgfqpoint{10.125114in}{1.463602in}}%
\pgfpathcurveto{\pgfqpoint{10.125114in}{1.468645in}}{\pgfqpoint{10.123110in}{1.473483in}}{\pgfqpoint{10.119543in}{1.477050in}}%
\pgfpathcurveto{\pgfqpoint{10.115977in}{1.480616in}}{\pgfqpoint{10.111139in}{1.482620in}}{\pgfqpoint{10.106096in}{1.482620in}}%
\pgfpathcurveto{\pgfqpoint{10.101052in}{1.482620in}}{\pgfqpoint{10.096214in}{1.480616in}}{\pgfqpoint{10.092648in}{1.477050in}}%
\pgfpathcurveto{\pgfqpoint{10.089081in}{1.473483in}}{\pgfqpoint{10.087077in}{1.468645in}}{\pgfqpoint{10.087077in}{1.463602in}}%
\pgfpathcurveto{\pgfqpoint{10.087077in}{1.458558in}}{\pgfqpoint{10.089081in}{1.453720in}}{\pgfqpoint{10.092648in}{1.450154in}}%
\pgfpathcurveto{\pgfqpoint{10.096214in}{1.446587in}}{\pgfqpoint{10.101052in}{1.444584in}}{\pgfqpoint{10.106096in}{1.444584in}}%
\pgfpathclose%
\pgfusepath{fill}%
\end{pgfscope}%
\begin{pgfscope}%
\pgfpathrectangle{\pgfqpoint{6.572727in}{0.473000in}}{\pgfqpoint{4.227273in}{3.311000in}}%
\pgfusepath{clip}%
\pgfsetbuttcap%
\pgfsetroundjoin%
\definecolor{currentfill}{rgb}{0.127568,0.566949,0.550556}%
\pgfsetfillcolor{currentfill}%
\pgfsetfillopacity{0.700000}%
\pgfsetlinewidth{0.000000pt}%
\definecolor{currentstroke}{rgb}{0.000000,0.000000,0.000000}%
\pgfsetstrokecolor{currentstroke}%
\pgfsetstrokeopacity{0.700000}%
\pgfsetdash{}{0pt}%
\pgfpathmoveto{\pgfqpoint{7.713252in}{3.075126in}}%
\pgfpathcurveto{\pgfqpoint{7.718295in}{3.075126in}}{\pgfqpoint{7.723133in}{3.077130in}}{\pgfqpoint{7.726699in}{3.080696in}}%
\pgfpathcurveto{\pgfqpoint{7.730266in}{3.084263in}}{\pgfqpoint{7.732270in}{3.089101in}}{\pgfqpoint{7.732270in}{3.094144in}}%
\pgfpathcurveto{\pgfqpoint{7.732270in}{3.099188in}}{\pgfqpoint{7.730266in}{3.104026in}}{\pgfqpoint{7.726699in}{3.107592in}}%
\pgfpathcurveto{\pgfqpoint{7.723133in}{3.111158in}}{\pgfqpoint{7.718295in}{3.113162in}}{\pgfqpoint{7.713252in}{3.113162in}}%
\pgfpathcurveto{\pgfqpoint{7.708208in}{3.113162in}}{\pgfqpoint{7.703370in}{3.111158in}}{\pgfqpoint{7.699804in}{3.107592in}}%
\pgfpathcurveto{\pgfqpoint{7.696237in}{3.104026in}}{\pgfqpoint{7.694233in}{3.099188in}}{\pgfqpoint{7.694233in}{3.094144in}}%
\pgfpathcurveto{\pgfqpoint{7.694233in}{3.089101in}}{\pgfqpoint{7.696237in}{3.084263in}}{\pgfqpoint{7.699804in}{3.080696in}}%
\pgfpathcurveto{\pgfqpoint{7.703370in}{3.077130in}}{\pgfqpoint{7.708208in}{3.075126in}}{\pgfqpoint{7.713252in}{3.075126in}}%
\pgfpathclose%
\pgfusepath{fill}%
\end{pgfscope}%
\begin{pgfscope}%
\pgfpathrectangle{\pgfqpoint{6.572727in}{0.473000in}}{\pgfqpoint{4.227273in}{3.311000in}}%
\pgfusepath{clip}%
\pgfsetbuttcap%
\pgfsetroundjoin%
\definecolor{currentfill}{rgb}{0.127568,0.566949,0.550556}%
\pgfsetfillcolor{currentfill}%
\pgfsetfillopacity{0.700000}%
\pgfsetlinewidth{0.000000pt}%
\definecolor{currentstroke}{rgb}{0.000000,0.000000,0.000000}%
\pgfsetstrokecolor{currentstroke}%
\pgfsetstrokeopacity{0.700000}%
\pgfsetdash{}{0pt}%
\pgfpathmoveto{\pgfqpoint{7.902055in}{2.384815in}}%
\pgfpathcurveto{\pgfqpoint{7.907099in}{2.384815in}}{\pgfqpoint{7.911937in}{2.386819in}}{\pgfqpoint{7.915503in}{2.390385in}}%
\pgfpathcurveto{\pgfqpoint{7.919070in}{2.393952in}}{\pgfqpoint{7.921073in}{2.398790in}}{\pgfqpoint{7.921073in}{2.403833in}}%
\pgfpathcurveto{\pgfqpoint{7.921073in}{2.408877in}}{\pgfqpoint{7.919070in}{2.413715in}}{\pgfqpoint{7.915503in}{2.417281in}}%
\pgfpathcurveto{\pgfqpoint{7.911937in}{2.420848in}}{\pgfqpoint{7.907099in}{2.422851in}}{\pgfqpoint{7.902055in}{2.422851in}}%
\pgfpathcurveto{\pgfqpoint{7.897012in}{2.422851in}}{\pgfqpoint{7.892174in}{2.420848in}}{\pgfqpoint{7.888607in}{2.417281in}}%
\pgfpathcurveto{\pgfqpoint{7.885041in}{2.413715in}}{\pgfqpoint{7.883037in}{2.408877in}}{\pgfqpoint{7.883037in}{2.403833in}}%
\pgfpathcurveto{\pgfqpoint{7.883037in}{2.398790in}}{\pgfqpoint{7.885041in}{2.393952in}}{\pgfqpoint{7.888607in}{2.390385in}}%
\pgfpathcurveto{\pgfqpoint{7.892174in}{2.386819in}}{\pgfqpoint{7.897012in}{2.384815in}}{\pgfqpoint{7.902055in}{2.384815in}}%
\pgfpathclose%
\pgfusepath{fill}%
\end{pgfscope}%
\begin{pgfscope}%
\pgfpathrectangle{\pgfqpoint{6.572727in}{0.473000in}}{\pgfqpoint{4.227273in}{3.311000in}}%
\pgfusepath{clip}%
\pgfsetbuttcap%
\pgfsetroundjoin%
\definecolor{currentfill}{rgb}{0.127568,0.566949,0.550556}%
\pgfsetfillcolor{currentfill}%
\pgfsetfillopacity{0.700000}%
\pgfsetlinewidth{0.000000pt}%
\definecolor{currentstroke}{rgb}{0.000000,0.000000,0.000000}%
\pgfsetstrokecolor{currentstroke}%
\pgfsetstrokeopacity{0.700000}%
\pgfsetdash{}{0pt}%
\pgfpathmoveto{\pgfqpoint{8.729553in}{2.512164in}}%
\pgfpathcurveto{\pgfqpoint{8.734596in}{2.512164in}}{\pgfqpoint{8.739434in}{2.514167in}}{\pgfqpoint{8.743001in}{2.517734in}}%
\pgfpathcurveto{\pgfqpoint{8.746567in}{2.521300in}}{\pgfqpoint{8.748571in}{2.526138in}}{\pgfqpoint{8.748571in}{2.531182in}}%
\pgfpathcurveto{\pgfqpoint{8.748571in}{2.536225in}}{\pgfqpoint{8.746567in}{2.541063in}}{\pgfqpoint{8.743001in}{2.544630in}}%
\pgfpathcurveto{\pgfqpoint{8.739434in}{2.548196in}}{\pgfqpoint{8.734596in}{2.550200in}}{\pgfqpoint{8.729553in}{2.550200in}}%
\pgfpathcurveto{\pgfqpoint{8.724509in}{2.550200in}}{\pgfqpoint{8.719671in}{2.548196in}}{\pgfqpoint{8.716105in}{2.544630in}}%
\pgfpathcurveto{\pgfqpoint{8.712538in}{2.541063in}}{\pgfqpoint{8.710535in}{2.536225in}}{\pgfqpoint{8.710535in}{2.531182in}}%
\pgfpathcurveto{\pgfqpoint{8.710535in}{2.526138in}}{\pgfqpoint{8.712538in}{2.521300in}}{\pgfqpoint{8.716105in}{2.517734in}}%
\pgfpathcurveto{\pgfqpoint{8.719671in}{2.514167in}}{\pgfqpoint{8.724509in}{2.512164in}}{\pgfqpoint{8.729553in}{2.512164in}}%
\pgfpathclose%
\pgfusepath{fill}%
\end{pgfscope}%
\begin{pgfscope}%
\pgfpathrectangle{\pgfqpoint{6.572727in}{0.473000in}}{\pgfqpoint{4.227273in}{3.311000in}}%
\pgfusepath{clip}%
\pgfsetbuttcap%
\pgfsetroundjoin%
\definecolor{currentfill}{rgb}{0.127568,0.566949,0.550556}%
\pgfsetfillcolor{currentfill}%
\pgfsetfillopacity{0.700000}%
\pgfsetlinewidth{0.000000pt}%
\definecolor{currentstroke}{rgb}{0.000000,0.000000,0.000000}%
\pgfsetstrokecolor{currentstroke}%
\pgfsetstrokeopacity{0.700000}%
\pgfsetdash{}{0pt}%
\pgfpathmoveto{\pgfqpoint{8.533761in}{2.651339in}}%
\pgfpathcurveto{\pgfqpoint{8.538805in}{2.651339in}}{\pgfqpoint{8.543643in}{2.653343in}}{\pgfqpoint{8.547209in}{2.656909in}}%
\pgfpathcurveto{\pgfqpoint{8.550776in}{2.660475in}}{\pgfqpoint{8.552780in}{2.665313in}}{\pgfqpoint{8.552780in}{2.670357in}}%
\pgfpathcurveto{\pgfqpoint{8.552780in}{2.675400in}}{\pgfqpoint{8.550776in}{2.680238in}}{\pgfqpoint{8.547209in}{2.683805in}}%
\pgfpathcurveto{\pgfqpoint{8.543643in}{2.687371in}}{\pgfqpoint{8.538805in}{2.689375in}}{\pgfqpoint{8.533761in}{2.689375in}}%
\pgfpathcurveto{\pgfqpoint{8.528718in}{2.689375in}}{\pgfqpoint{8.523880in}{2.687371in}}{\pgfqpoint{8.520314in}{2.683805in}}%
\pgfpathcurveto{\pgfqpoint{8.516747in}{2.680238in}}{\pgfqpoint{8.514743in}{2.675400in}}{\pgfqpoint{8.514743in}{2.670357in}}%
\pgfpathcurveto{\pgfqpoint{8.514743in}{2.665313in}}{\pgfqpoint{8.516747in}{2.660475in}}{\pgfqpoint{8.520314in}{2.656909in}}%
\pgfpathcurveto{\pgfqpoint{8.523880in}{2.653343in}}{\pgfqpoint{8.528718in}{2.651339in}}{\pgfqpoint{8.533761in}{2.651339in}}%
\pgfpathclose%
\pgfusepath{fill}%
\end{pgfscope}%
\begin{pgfscope}%
\pgfpathrectangle{\pgfqpoint{6.572727in}{0.473000in}}{\pgfqpoint{4.227273in}{3.311000in}}%
\pgfusepath{clip}%
\pgfsetbuttcap%
\pgfsetroundjoin%
\definecolor{currentfill}{rgb}{0.127568,0.566949,0.550556}%
\pgfsetfillcolor{currentfill}%
\pgfsetfillopacity{0.700000}%
\pgfsetlinewidth{0.000000pt}%
\definecolor{currentstroke}{rgb}{0.000000,0.000000,0.000000}%
\pgfsetstrokecolor{currentstroke}%
\pgfsetstrokeopacity{0.700000}%
\pgfsetdash{}{0pt}%
\pgfpathmoveto{\pgfqpoint{7.503841in}{2.626904in}}%
\pgfpathcurveto{\pgfqpoint{7.508884in}{2.626904in}}{\pgfqpoint{7.513722in}{2.628908in}}{\pgfqpoint{7.517289in}{2.632475in}}%
\pgfpathcurveto{\pgfqpoint{7.520855in}{2.636041in}}{\pgfqpoint{7.522859in}{2.640879in}}{\pgfqpoint{7.522859in}{2.645922in}}%
\pgfpathcurveto{\pgfqpoint{7.522859in}{2.650966in}}{\pgfqpoint{7.520855in}{2.655804in}}{\pgfqpoint{7.517289in}{2.659370in}}%
\pgfpathcurveto{\pgfqpoint{7.513722in}{2.662937in}}{\pgfqpoint{7.508884in}{2.664941in}}{\pgfqpoint{7.503841in}{2.664941in}}%
\pgfpathcurveto{\pgfqpoint{7.498797in}{2.664941in}}{\pgfqpoint{7.493959in}{2.662937in}}{\pgfqpoint{7.490393in}{2.659370in}}%
\pgfpathcurveto{\pgfqpoint{7.486826in}{2.655804in}}{\pgfqpoint{7.484823in}{2.650966in}}{\pgfqpoint{7.484823in}{2.645922in}}%
\pgfpathcurveto{\pgfqpoint{7.484823in}{2.640879in}}{\pgfqpoint{7.486826in}{2.636041in}}{\pgfqpoint{7.490393in}{2.632475in}}%
\pgfpathcurveto{\pgfqpoint{7.493959in}{2.628908in}}{\pgfqpoint{7.498797in}{2.626904in}}{\pgfqpoint{7.503841in}{2.626904in}}%
\pgfpathclose%
\pgfusepath{fill}%
\end{pgfscope}%
\begin{pgfscope}%
\pgfpathrectangle{\pgfqpoint{6.572727in}{0.473000in}}{\pgfqpoint{4.227273in}{3.311000in}}%
\pgfusepath{clip}%
\pgfsetbuttcap%
\pgfsetroundjoin%
\definecolor{currentfill}{rgb}{0.127568,0.566949,0.550556}%
\pgfsetfillcolor{currentfill}%
\pgfsetfillopacity{0.700000}%
\pgfsetlinewidth{0.000000pt}%
\definecolor{currentstroke}{rgb}{0.000000,0.000000,0.000000}%
\pgfsetstrokecolor{currentstroke}%
\pgfsetstrokeopacity{0.700000}%
\pgfsetdash{}{0pt}%
\pgfpathmoveto{\pgfqpoint{8.262997in}{2.972598in}}%
\pgfpathcurveto{\pgfqpoint{8.268041in}{2.972598in}}{\pgfqpoint{8.272879in}{2.974602in}}{\pgfqpoint{8.276445in}{2.978168in}}%
\pgfpathcurveto{\pgfqpoint{8.280012in}{2.981734in}}{\pgfqpoint{8.282016in}{2.986572in}}{\pgfqpoint{8.282016in}{2.991616in}}%
\pgfpathcurveto{\pgfqpoint{8.282016in}{2.996660in}}{\pgfqpoint{8.280012in}{3.001497in}}{\pgfqpoint{8.276445in}{3.005064in}}%
\pgfpathcurveto{\pgfqpoint{8.272879in}{3.008630in}}{\pgfqpoint{8.268041in}{3.010634in}}{\pgfqpoint{8.262997in}{3.010634in}}%
\pgfpathcurveto{\pgfqpoint{8.257954in}{3.010634in}}{\pgfqpoint{8.253116in}{3.008630in}}{\pgfqpoint{8.249550in}{3.005064in}}%
\pgfpathcurveto{\pgfqpoint{8.245983in}{3.001497in}}{\pgfqpoint{8.243979in}{2.996660in}}{\pgfqpoint{8.243979in}{2.991616in}}%
\pgfpathcurveto{\pgfqpoint{8.243979in}{2.986572in}}{\pgfqpoint{8.245983in}{2.981734in}}{\pgfqpoint{8.249550in}{2.978168in}}%
\pgfpathcurveto{\pgfqpoint{8.253116in}{2.974602in}}{\pgfqpoint{8.257954in}{2.972598in}}{\pgfqpoint{8.262997in}{2.972598in}}%
\pgfpathclose%
\pgfusepath{fill}%
\end{pgfscope}%
\begin{pgfscope}%
\pgfpathrectangle{\pgfqpoint{6.572727in}{0.473000in}}{\pgfqpoint{4.227273in}{3.311000in}}%
\pgfusepath{clip}%
\pgfsetbuttcap%
\pgfsetroundjoin%
\definecolor{currentfill}{rgb}{0.993248,0.906157,0.143936}%
\pgfsetfillcolor{currentfill}%
\pgfsetfillopacity{0.700000}%
\pgfsetlinewidth{0.000000pt}%
\definecolor{currentstroke}{rgb}{0.000000,0.000000,0.000000}%
\pgfsetstrokecolor{currentstroke}%
\pgfsetstrokeopacity{0.700000}%
\pgfsetdash{}{0pt}%
\pgfpathmoveto{\pgfqpoint{9.564049in}{1.274462in}}%
\pgfpathcurveto{\pgfqpoint{9.569093in}{1.274462in}}{\pgfqpoint{9.573930in}{1.276466in}}{\pgfqpoint{9.577497in}{1.280032in}}%
\pgfpathcurveto{\pgfqpoint{9.581063in}{1.283599in}}{\pgfqpoint{9.583067in}{1.288436in}}{\pgfqpoint{9.583067in}{1.293480in}}%
\pgfpathcurveto{\pgfqpoint{9.583067in}{1.298524in}}{\pgfqpoint{9.581063in}{1.303362in}}{\pgfqpoint{9.577497in}{1.306928in}}%
\pgfpathcurveto{\pgfqpoint{9.573930in}{1.310494in}}{\pgfqpoint{9.569093in}{1.312498in}}{\pgfqpoint{9.564049in}{1.312498in}}%
\pgfpathcurveto{\pgfqpoint{9.559005in}{1.312498in}}{\pgfqpoint{9.554167in}{1.310494in}}{\pgfqpoint{9.550601in}{1.306928in}}%
\pgfpathcurveto{\pgfqpoint{9.547035in}{1.303362in}}{\pgfqpoint{9.545031in}{1.298524in}}{\pgfqpoint{9.545031in}{1.293480in}}%
\pgfpathcurveto{\pgfqpoint{9.545031in}{1.288436in}}{\pgfqpoint{9.547035in}{1.283599in}}{\pgfqpoint{9.550601in}{1.280032in}}%
\pgfpathcurveto{\pgfqpoint{9.554167in}{1.276466in}}{\pgfqpoint{9.559005in}{1.274462in}}{\pgfqpoint{9.564049in}{1.274462in}}%
\pgfpathclose%
\pgfusepath{fill}%
\end{pgfscope}%
\begin{pgfscope}%
\pgfpathrectangle{\pgfqpoint{6.572727in}{0.473000in}}{\pgfqpoint{4.227273in}{3.311000in}}%
\pgfusepath{clip}%
\pgfsetbuttcap%
\pgfsetroundjoin%
\definecolor{currentfill}{rgb}{0.993248,0.906157,0.143936}%
\pgfsetfillcolor{currentfill}%
\pgfsetfillopacity{0.700000}%
\pgfsetlinewidth{0.000000pt}%
\definecolor{currentstroke}{rgb}{0.000000,0.000000,0.000000}%
\pgfsetstrokecolor{currentstroke}%
\pgfsetstrokeopacity{0.700000}%
\pgfsetdash{}{0pt}%
\pgfpathmoveto{\pgfqpoint{9.713501in}{2.063316in}}%
\pgfpathcurveto{\pgfqpoint{9.718545in}{2.063316in}}{\pgfqpoint{9.723383in}{2.065320in}}{\pgfqpoint{9.726949in}{2.068886in}}%
\pgfpathcurveto{\pgfqpoint{9.730515in}{2.072453in}}{\pgfqpoint{9.732519in}{2.077290in}}{\pgfqpoint{9.732519in}{2.082334in}}%
\pgfpathcurveto{\pgfqpoint{9.732519in}{2.087378in}}{\pgfqpoint{9.730515in}{2.092215in}}{\pgfqpoint{9.726949in}{2.095782in}}%
\pgfpathcurveto{\pgfqpoint{9.723383in}{2.099348in}}{\pgfqpoint{9.718545in}{2.101352in}}{\pgfqpoint{9.713501in}{2.101352in}}%
\pgfpathcurveto{\pgfqpoint{9.708457in}{2.101352in}}{\pgfqpoint{9.703620in}{2.099348in}}{\pgfqpoint{9.700053in}{2.095782in}}%
\pgfpathcurveto{\pgfqpoint{9.696487in}{2.092215in}}{\pgfqpoint{9.694483in}{2.087378in}}{\pgfqpoint{9.694483in}{2.082334in}}%
\pgfpathcurveto{\pgfqpoint{9.694483in}{2.077290in}}{\pgfqpoint{9.696487in}{2.072453in}}{\pgfqpoint{9.700053in}{2.068886in}}%
\pgfpathcurveto{\pgfqpoint{9.703620in}{2.065320in}}{\pgfqpoint{9.708457in}{2.063316in}}{\pgfqpoint{9.713501in}{2.063316in}}%
\pgfpathclose%
\pgfusepath{fill}%
\end{pgfscope}%
\begin{pgfscope}%
\pgfpathrectangle{\pgfqpoint{6.572727in}{0.473000in}}{\pgfqpoint{4.227273in}{3.311000in}}%
\pgfusepath{clip}%
\pgfsetbuttcap%
\pgfsetroundjoin%
\definecolor{currentfill}{rgb}{0.127568,0.566949,0.550556}%
\pgfsetfillcolor{currentfill}%
\pgfsetfillopacity{0.700000}%
\pgfsetlinewidth{0.000000pt}%
\definecolor{currentstroke}{rgb}{0.000000,0.000000,0.000000}%
\pgfsetstrokecolor{currentstroke}%
\pgfsetstrokeopacity{0.700000}%
\pgfsetdash{}{0pt}%
\pgfpathmoveto{\pgfqpoint{7.960718in}{1.602463in}}%
\pgfpathcurveto{\pgfqpoint{7.965762in}{1.602463in}}{\pgfqpoint{7.970599in}{1.604467in}}{\pgfqpoint{7.974166in}{1.608033in}}%
\pgfpathcurveto{\pgfqpoint{7.977732in}{1.611599in}}{\pgfqpoint{7.979736in}{1.616437in}}{\pgfqpoint{7.979736in}{1.621481in}}%
\pgfpathcurveto{\pgfqpoint{7.979736in}{1.626525in}}{\pgfqpoint{7.977732in}{1.631362in}}{\pgfqpoint{7.974166in}{1.634929in}}%
\pgfpathcurveto{\pgfqpoint{7.970599in}{1.638495in}}{\pgfqpoint{7.965762in}{1.640499in}}{\pgfqpoint{7.960718in}{1.640499in}}%
\pgfpathcurveto{\pgfqpoint{7.955674in}{1.640499in}}{\pgfqpoint{7.950836in}{1.638495in}}{\pgfqpoint{7.947270in}{1.634929in}}%
\pgfpathcurveto{\pgfqpoint{7.943704in}{1.631362in}}{\pgfqpoint{7.941700in}{1.626525in}}{\pgfqpoint{7.941700in}{1.621481in}}%
\pgfpathcurveto{\pgfqpoint{7.941700in}{1.616437in}}{\pgfqpoint{7.943704in}{1.611599in}}{\pgfqpoint{7.947270in}{1.608033in}}%
\pgfpathcurveto{\pgfqpoint{7.950836in}{1.604467in}}{\pgfqpoint{7.955674in}{1.602463in}}{\pgfqpoint{7.960718in}{1.602463in}}%
\pgfpathclose%
\pgfusepath{fill}%
\end{pgfscope}%
\begin{pgfscope}%
\pgfpathrectangle{\pgfqpoint{6.572727in}{0.473000in}}{\pgfqpoint{4.227273in}{3.311000in}}%
\pgfusepath{clip}%
\pgfsetbuttcap%
\pgfsetroundjoin%
\definecolor{currentfill}{rgb}{0.993248,0.906157,0.143936}%
\pgfsetfillcolor{currentfill}%
\pgfsetfillopacity{0.700000}%
\pgfsetlinewidth{0.000000pt}%
\definecolor{currentstroke}{rgb}{0.000000,0.000000,0.000000}%
\pgfsetstrokecolor{currentstroke}%
\pgfsetstrokeopacity{0.700000}%
\pgfsetdash{}{0pt}%
\pgfpathmoveto{\pgfqpoint{10.033390in}{1.029585in}}%
\pgfpathcurveto{\pgfqpoint{10.038434in}{1.029585in}}{\pgfqpoint{10.043272in}{1.031589in}}{\pgfqpoint{10.046838in}{1.035155in}}%
\pgfpathcurveto{\pgfqpoint{10.050404in}{1.038722in}}{\pgfqpoint{10.052408in}{1.043559in}}{\pgfqpoint{10.052408in}{1.048603in}}%
\pgfpathcurveto{\pgfqpoint{10.052408in}{1.053647in}}{\pgfqpoint{10.050404in}{1.058485in}}{\pgfqpoint{10.046838in}{1.062051in}}%
\pgfpathcurveto{\pgfqpoint{10.043272in}{1.065617in}}{\pgfqpoint{10.038434in}{1.067621in}}{\pgfqpoint{10.033390in}{1.067621in}}%
\pgfpathcurveto{\pgfqpoint{10.028346in}{1.067621in}}{\pgfqpoint{10.023509in}{1.065617in}}{\pgfqpoint{10.019942in}{1.062051in}}%
\pgfpathcurveto{\pgfqpoint{10.016376in}{1.058485in}}{\pgfqpoint{10.014372in}{1.053647in}}{\pgfqpoint{10.014372in}{1.048603in}}%
\pgfpathcurveto{\pgfqpoint{10.014372in}{1.043559in}}{\pgfqpoint{10.016376in}{1.038722in}}{\pgfqpoint{10.019942in}{1.035155in}}%
\pgfpathcurveto{\pgfqpoint{10.023509in}{1.031589in}}{\pgfqpoint{10.028346in}{1.029585in}}{\pgfqpoint{10.033390in}{1.029585in}}%
\pgfpathclose%
\pgfusepath{fill}%
\end{pgfscope}%
\begin{pgfscope}%
\pgfpathrectangle{\pgfqpoint{6.572727in}{0.473000in}}{\pgfqpoint{4.227273in}{3.311000in}}%
\pgfusepath{clip}%
\pgfsetbuttcap%
\pgfsetroundjoin%
\definecolor{currentfill}{rgb}{0.127568,0.566949,0.550556}%
\pgfsetfillcolor{currentfill}%
\pgfsetfillopacity{0.700000}%
\pgfsetlinewidth{0.000000pt}%
\definecolor{currentstroke}{rgb}{0.000000,0.000000,0.000000}%
\pgfsetstrokecolor{currentstroke}%
\pgfsetstrokeopacity{0.700000}%
\pgfsetdash{}{0pt}%
\pgfpathmoveto{\pgfqpoint{7.636301in}{1.795140in}}%
\pgfpathcurveto{\pgfqpoint{7.641344in}{1.795140in}}{\pgfqpoint{7.646182in}{1.797144in}}{\pgfqpoint{7.649749in}{1.800710in}}%
\pgfpathcurveto{\pgfqpoint{7.653315in}{1.804276in}}{\pgfqpoint{7.655319in}{1.809114in}}{\pgfqpoint{7.655319in}{1.814158in}}%
\pgfpathcurveto{\pgfqpoint{7.655319in}{1.819202in}}{\pgfqpoint{7.653315in}{1.824039in}}{\pgfqpoint{7.649749in}{1.827606in}}%
\pgfpathcurveto{\pgfqpoint{7.646182in}{1.831172in}}{\pgfqpoint{7.641344in}{1.833176in}}{\pgfqpoint{7.636301in}{1.833176in}}%
\pgfpathcurveto{\pgfqpoint{7.631257in}{1.833176in}}{\pgfqpoint{7.626419in}{1.831172in}}{\pgfqpoint{7.622853in}{1.827606in}}%
\pgfpathcurveto{\pgfqpoint{7.619286in}{1.824039in}}{\pgfqpoint{7.617283in}{1.819202in}}{\pgfqpoint{7.617283in}{1.814158in}}%
\pgfpathcurveto{\pgfqpoint{7.617283in}{1.809114in}}{\pgfqpoint{7.619286in}{1.804276in}}{\pgfqpoint{7.622853in}{1.800710in}}%
\pgfpathcurveto{\pgfqpoint{7.626419in}{1.797144in}}{\pgfqpoint{7.631257in}{1.795140in}}{\pgfqpoint{7.636301in}{1.795140in}}%
\pgfpathclose%
\pgfusepath{fill}%
\end{pgfscope}%
\begin{pgfscope}%
\pgfpathrectangle{\pgfqpoint{6.572727in}{0.473000in}}{\pgfqpoint{4.227273in}{3.311000in}}%
\pgfusepath{clip}%
\pgfsetbuttcap%
\pgfsetroundjoin%
\definecolor{currentfill}{rgb}{0.127568,0.566949,0.550556}%
\pgfsetfillcolor{currentfill}%
\pgfsetfillopacity{0.700000}%
\pgfsetlinewidth{0.000000pt}%
\definecolor{currentstroke}{rgb}{0.000000,0.000000,0.000000}%
\pgfsetstrokecolor{currentstroke}%
\pgfsetstrokeopacity{0.700000}%
\pgfsetdash{}{0pt}%
\pgfpathmoveto{\pgfqpoint{8.489685in}{2.653245in}}%
\pgfpathcurveto{\pgfqpoint{8.494729in}{2.653245in}}{\pgfqpoint{8.499567in}{2.655249in}}{\pgfqpoint{8.503133in}{2.658816in}}%
\pgfpathcurveto{\pgfqpoint{8.506700in}{2.662382in}}{\pgfqpoint{8.508703in}{2.667220in}}{\pgfqpoint{8.508703in}{2.672264in}}%
\pgfpathcurveto{\pgfqpoint{8.508703in}{2.677307in}}{\pgfqpoint{8.506700in}{2.682145in}}{\pgfqpoint{8.503133in}{2.685711in}}%
\pgfpathcurveto{\pgfqpoint{8.499567in}{2.689278in}}{\pgfqpoint{8.494729in}{2.691282in}}{\pgfqpoint{8.489685in}{2.691282in}}%
\pgfpathcurveto{\pgfqpoint{8.484642in}{2.691282in}}{\pgfqpoint{8.479804in}{2.689278in}}{\pgfqpoint{8.476237in}{2.685711in}}%
\pgfpathcurveto{\pgfqpoint{8.472671in}{2.682145in}}{\pgfqpoint{8.470667in}{2.677307in}}{\pgfqpoint{8.470667in}{2.672264in}}%
\pgfpathcurveto{\pgfqpoint{8.470667in}{2.667220in}}{\pgfqpoint{8.472671in}{2.662382in}}{\pgfqpoint{8.476237in}{2.658816in}}%
\pgfpathcurveto{\pgfqpoint{8.479804in}{2.655249in}}{\pgfqpoint{8.484642in}{2.653245in}}{\pgfqpoint{8.489685in}{2.653245in}}%
\pgfpathclose%
\pgfusepath{fill}%
\end{pgfscope}%
\begin{pgfscope}%
\pgfpathrectangle{\pgfqpoint{6.572727in}{0.473000in}}{\pgfqpoint{4.227273in}{3.311000in}}%
\pgfusepath{clip}%
\pgfsetbuttcap%
\pgfsetroundjoin%
\definecolor{currentfill}{rgb}{0.127568,0.566949,0.550556}%
\pgfsetfillcolor{currentfill}%
\pgfsetfillopacity{0.700000}%
\pgfsetlinewidth{0.000000pt}%
\definecolor{currentstroke}{rgb}{0.000000,0.000000,0.000000}%
\pgfsetstrokecolor{currentstroke}%
\pgfsetstrokeopacity{0.700000}%
\pgfsetdash{}{0pt}%
\pgfpathmoveto{\pgfqpoint{8.525579in}{1.450168in}}%
\pgfpathcurveto{\pgfqpoint{8.530622in}{1.450168in}}{\pgfqpoint{8.535460in}{1.452172in}}{\pgfqpoint{8.539027in}{1.455738in}}%
\pgfpathcurveto{\pgfqpoint{8.542593in}{1.459305in}}{\pgfqpoint{8.544597in}{1.464142in}}{\pgfqpoint{8.544597in}{1.469186in}}%
\pgfpathcurveto{\pgfqpoint{8.544597in}{1.474230in}}{\pgfqpoint{8.542593in}{1.479067in}}{\pgfqpoint{8.539027in}{1.482634in}}%
\pgfpathcurveto{\pgfqpoint{8.535460in}{1.486200in}}{\pgfqpoint{8.530622in}{1.488204in}}{\pgfqpoint{8.525579in}{1.488204in}}%
\pgfpathcurveto{\pgfqpoint{8.520535in}{1.488204in}}{\pgfqpoint{8.515697in}{1.486200in}}{\pgfqpoint{8.512131in}{1.482634in}}%
\pgfpathcurveto{\pgfqpoint{8.508564in}{1.479067in}}{\pgfqpoint{8.506561in}{1.474230in}}{\pgfqpoint{8.506561in}{1.469186in}}%
\pgfpathcurveto{\pgfqpoint{8.506561in}{1.464142in}}{\pgfqpoint{8.508564in}{1.459305in}}{\pgfqpoint{8.512131in}{1.455738in}}%
\pgfpathcurveto{\pgfqpoint{8.515697in}{1.452172in}}{\pgfqpoint{8.520535in}{1.450168in}}{\pgfqpoint{8.525579in}{1.450168in}}%
\pgfpathclose%
\pgfusepath{fill}%
\end{pgfscope}%
\begin{pgfscope}%
\pgfpathrectangle{\pgfqpoint{6.572727in}{0.473000in}}{\pgfqpoint{4.227273in}{3.311000in}}%
\pgfusepath{clip}%
\pgfsetbuttcap%
\pgfsetroundjoin%
\definecolor{currentfill}{rgb}{0.127568,0.566949,0.550556}%
\pgfsetfillcolor{currentfill}%
\pgfsetfillopacity{0.700000}%
\pgfsetlinewidth{0.000000pt}%
\definecolor{currentstroke}{rgb}{0.000000,0.000000,0.000000}%
\pgfsetstrokecolor{currentstroke}%
\pgfsetstrokeopacity{0.700000}%
\pgfsetdash{}{0pt}%
\pgfpathmoveto{\pgfqpoint{7.617807in}{3.460025in}}%
\pgfpathcurveto{\pgfqpoint{7.622850in}{3.460025in}}{\pgfqpoint{7.627688in}{3.462029in}}{\pgfqpoint{7.631254in}{3.465595in}}%
\pgfpathcurveto{\pgfqpoint{7.634821in}{3.469161in}}{\pgfqpoint{7.636825in}{3.473999in}}{\pgfqpoint{7.636825in}{3.479043in}}%
\pgfpathcurveto{\pgfqpoint{7.636825in}{3.484087in}}{\pgfqpoint{7.634821in}{3.488924in}}{\pgfqpoint{7.631254in}{3.492491in}}%
\pgfpathcurveto{\pgfqpoint{7.627688in}{3.496057in}}{\pgfqpoint{7.622850in}{3.498061in}}{\pgfqpoint{7.617807in}{3.498061in}}%
\pgfpathcurveto{\pgfqpoint{7.612763in}{3.498061in}}{\pgfqpoint{7.607925in}{3.496057in}}{\pgfqpoint{7.604359in}{3.492491in}}%
\pgfpathcurveto{\pgfqpoint{7.600792in}{3.488924in}}{\pgfqpoint{7.598788in}{3.484087in}}{\pgfqpoint{7.598788in}{3.479043in}}%
\pgfpathcurveto{\pgfqpoint{7.598788in}{3.473999in}}{\pgfqpoint{7.600792in}{3.469161in}}{\pgfqpoint{7.604359in}{3.465595in}}%
\pgfpathcurveto{\pgfqpoint{7.607925in}{3.462029in}}{\pgfqpoint{7.612763in}{3.460025in}}{\pgfqpoint{7.617807in}{3.460025in}}%
\pgfpathclose%
\pgfusepath{fill}%
\end{pgfscope}%
\begin{pgfscope}%
\pgfpathrectangle{\pgfqpoint{6.572727in}{0.473000in}}{\pgfqpoint{4.227273in}{3.311000in}}%
\pgfusepath{clip}%
\pgfsetbuttcap%
\pgfsetroundjoin%
\definecolor{currentfill}{rgb}{0.127568,0.566949,0.550556}%
\pgfsetfillcolor{currentfill}%
\pgfsetfillopacity{0.700000}%
\pgfsetlinewidth{0.000000pt}%
\definecolor{currentstroke}{rgb}{0.000000,0.000000,0.000000}%
\pgfsetstrokecolor{currentstroke}%
\pgfsetstrokeopacity{0.700000}%
\pgfsetdash{}{0pt}%
\pgfpathmoveto{\pgfqpoint{8.793839in}{2.738745in}}%
\pgfpathcurveto{\pgfqpoint{8.798882in}{2.738745in}}{\pgfqpoint{8.803720in}{2.740749in}}{\pgfqpoint{8.807287in}{2.744315in}}%
\pgfpathcurveto{\pgfqpoint{8.810853in}{2.747882in}}{\pgfqpoint{8.812857in}{2.752719in}}{\pgfqpoint{8.812857in}{2.757763in}}%
\pgfpathcurveto{\pgfqpoint{8.812857in}{2.762807in}}{\pgfqpoint{8.810853in}{2.767645in}}{\pgfqpoint{8.807287in}{2.771211in}}%
\pgfpathcurveto{\pgfqpoint{8.803720in}{2.774777in}}{\pgfqpoint{8.798882in}{2.776781in}}{\pgfqpoint{8.793839in}{2.776781in}}%
\pgfpathcurveto{\pgfqpoint{8.788795in}{2.776781in}}{\pgfqpoint{8.783957in}{2.774777in}}{\pgfqpoint{8.780391in}{2.771211in}}%
\pgfpathcurveto{\pgfqpoint{8.776824in}{2.767645in}}{\pgfqpoint{8.774821in}{2.762807in}}{\pgfqpoint{8.774821in}{2.757763in}}%
\pgfpathcurveto{\pgfqpoint{8.774821in}{2.752719in}}{\pgfqpoint{8.776824in}{2.747882in}}{\pgfqpoint{8.780391in}{2.744315in}}%
\pgfpathcurveto{\pgfqpoint{8.783957in}{2.740749in}}{\pgfqpoint{8.788795in}{2.738745in}}{\pgfqpoint{8.793839in}{2.738745in}}%
\pgfpathclose%
\pgfusepath{fill}%
\end{pgfscope}%
\begin{pgfscope}%
\pgfpathrectangle{\pgfqpoint{6.572727in}{0.473000in}}{\pgfqpoint{4.227273in}{3.311000in}}%
\pgfusepath{clip}%
\pgfsetbuttcap%
\pgfsetroundjoin%
\definecolor{currentfill}{rgb}{0.127568,0.566949,0.550556}%
\pgfsetfillcolor{currentfill}%
\pgfsetfillopacity{0.700000}%
\pgfsetlinewidth{0.000000pt}%
\definecolor{currentstroke}{rgb}{0.000000,0.000000,0.000000}%
\pgfsetstrokecolor{currentstroke}%
\pgfsetstrokeopacity{0.700000}%
\pgfsetdash{}{0pt}%
\pgfpathmoveto{\pgfqpoint{7.728160in}{2.609975in}}%
\pgfpathcurveto{\pgfqpoint{7.733204in}{2.609975in}}{\pgfqpoint{7.738042in}{2.611979in}}{\pgfqpoint{7.741608in}{2.615545in}}%
\pgfpathcurveto{\pgfqpoint{7.745175in}{2.619111in}}{\pgfqpoint{7.747179in}{2.623949in}}{\pgfqpoint{7.747179in}{2.628993in}}%
\pgfpathcurveto{\pgfqpoint{7.747179in}{2.634037in}}{\pgfqpoint{7.745175in}{2.638874in}}{\pgfqpoint{7.741608in}{2.642441in}}%
\pgfpathcurveto{\pgfqpoint{7.738042in}{2.646007in}}{\pgfqpoint{7.733204in}{2.648011in}}{\pgfqpoint{7.728160in}{2.648011in}}%
\pgfpathcurveto{\pgfqpoint{7.723117in}{2.648011in}}{\pgfqpoint{7.718279in}{2.646007in}}{\pgfqpoint{7.714713in}{2.642441in}}%
\pgfpathcurveto{\pgfqpoint{7.711146in}{2.638874in}}{\pgfqpoint{7.709142in}{2.634037in}}{\pgfqpoint{7.709142in}{2.628993in}}%
\pgfpathcurveto{\pgfqpoint{7.709142in}{2.623949in}}{\pgfqpoint{7.711146in}{2.619111in}}{\pgfqpoint{7.714713in}{2.615545in}}%
\pgfpathcurveto{\pgfqpoint{7.718279in}{2.611979in}}{\pgfqpoint{7.723117in}{2.609975in}}{\pgfqpoint{7.728160in}{2.609975in}}%
\pgfpathclose%
\pgfusepath{fill}%
\end{pgfscope}%
\begin{pgfscope}%
\pgfpathrectangle{\pgfqpoint{6.572727in}{0.473000in}}{\pgfqpoint{4.227273in}{3.311000in}}%
\pgfusepath{clip}%
\pgfsetbuttcap%
\pgfsetroundjoin%
\definecolor{currentfill}{rgb}{0.993248,0.906157,0.143936}%
\pgfsetfillcolor{currentfill}%
\pgfsetfillopacity{0.700000}%
\pgfsetlinewidth{0.000000pt}%
\definecolor{currentstroke}{rgb}{0.000000,0.000000,0.000000}%
\pgfsetstrokecolor{currentstroke}%
\pgfsetstrokeopacity{0.700000}%
\pgfsetdash{}{0pt}%
\pgfpathmoveto{\pgfqpoint{9.415018in}{1.712057in}}%
\pgfpathcurveto{\pgfqpoint{9.420062in}{1.712057in}}{\pgfqpoint{9.424900in}{1.714061in}}{\pgfqpoint{9.428466in}{1.717627in}}%
\pgfpathcurveto{\pgfqpoint{9.432033in}{1.721194in}}{\pgfqpoint{9.434037in}{1.726032in}}{\pgfqpoint{9.434037in}{1.731075in}}%
\pgfpathcurveto{\pgfqpoint{9.434037in}{1.736119in}}{\pgfqpoint{9.432033in}{1.740957in}}{\pgfqpoint{9.428466in}{1.744523in}}%
\pgfpathcurveto{\pgfqpoint{9.424900in}{1.748090in}}{\pgfqpoint{9.420062in}{1.750093in}}{\pgfqpoint{9.415018in}{1.750093in}}%
\pgfpathcurveto{\pgfqpoint{9.409975in}{1.750093in}}{\pgfqpoint{9.405137in}{1.748090in}}{\pgfqpoint{9.401571in}{1.744523in}}%
\pgfpathcurveto{\pgfqpoint{9.398004in}{1.740957in}}{\pgfqpoint{9.396000in}{1.736119in}}{\pgfqpoint{9.396000in}{1.731075in}}%
\pgfpathcurveto{\pgfqpoint{9.396000in}{1.726032in}}{\pgfqpoint{9.398004in}{1.721194in}}{\pgfqpoint{9.401571in}{1.717627in}}%
\pgfpathcurveto{\pgfqpoint{9.405137in}{1.714061in}}{\pgfqpoint{9.409975in}{1.712057in}}{\pgfqpoint{9.415018in}{1.712057in}}%
\pgfpathclose%
\pgfusepath{fill}%
\end{pgfscope}%
\begin{pgfscope}%
\pgfpathrectangle{\pgfqpoint{6.572727in}{0.473000in}}{\pgfqpoint{4.227273in}{3.311000in}}%
\pgfusepath{clip}%
\pgfsetbuttcap%
\pgfsetroundjoin%
\definecolor{currentfill}{rgb}{0.993248,0.906157,0.143936}%
\pgfsetfillcolor{currentfill}%
\pgfsetfillopacity{0.700000}%
\pgfsetlinewidth{0.000000pt}%
\definecolor{currentstroke}{rgb}{0.000000,0.000000,0.000000}%
\pgfsetstrokecolor{currentstroke}%
\pgfsetstrokeopacity{0.700000}%
\pgfsetdash{}{0pt}%
\pgfpathmoveto{\pgfqpoint{9.629667in}{1.905528in}}%
\pgfpathcurveto{\pgfqpoint{9.634711in}{1.905528in}}{\pgfqpoint{9.639548in}{1.907532in}}{\pgfqpoint{9.643115in}{1.911099in}}%
\pgfpathcurveto{\pgfqpoint{9.646681in}{1.914665in}}{\pgfqpoint{9.648685in}{1.919503in}}{\pgfqpoint{9.648685in}{1.924547in}}%
\pgfpathcurveto{\pgfqpoint{9.648685in}{1.929590in}}{\pgfqpoint{9.646681in}{1.934428in}}{\pgfqpoint{9.643115in}{1.937994in}}%
\pgfpathcurveto{\pgfqpoint{9.639548in}{1.941561in}}{\pgfqpoint{9.634711in}{1.943565in}}{\pgfqpoint{9.629667in}{1.943565in}}%
\pgfpathcurveto{\pgfqpoint{9.624623in}{1.943565in}}{\pgfqpoint{9.619786in}{1.941561in}}{\pgfqpoint{9.616219in}{1.937994in}}%
\pgfpathcurveto{\pgfqpoint{9.612653in}{1.934428in}}{\pgfqpoint{9.610649in}{1.929590in}}{\pgfqpoint{9.610649in}{1.924547in}}%
\pgfpathcurveto{\pgfqpoint{9.610649in}{1.919503in}}{\pgfqpoint{9.612653in}{1.914665in}}{\pgfqpoint{9.616219in}{1.911099in}}%
\pgfpathcurveto{\pgfqpoint{9.619786in}{1.907532in}}{\pgfqpoint{9.624623in}{1.905528in}}{\pgfqpoint{9.629667in}{1.905528in}}%
\pgfpathclose%
\pgfusepath{fill}%
\end{pgfscope}%
\begin{pgfscope}%
\pgfpathrectangle{\pgfqpoint{6.572727in}{0.473000in}}{\pgfqpoint{4.227273in}{3.311000in}}%
\pgfusepath{clip}%
\pgfsetbuttcap%
\pgfsetroundjoin%
\definecolor{currentfill}{rgb}{0.127568,0.566949,0.550556}%
\pgfsetfillcolor{currentfill}%
\pgfsetfillopacity{0.700000}%
\pgfsetlinewidth{0.000000pt}%
\definecolor{currentstroke}{rgb}{0.000000,0.000000,0.000000}%
\pgfsetstrokecolor{currentstroke}%
\pgfsetstrokeopacity{0.700000}%
\pgfsetdash{}{0pt}%
\pgfpathmoveto{\pgfqpoint{8.727090in}{1.925802in}}%
\pgfpathcurveto{\pgfqpoint{8.732134in}{1.925802in}}{\pgfqpoint{8.736972in}{1.927806in}}{\pgfqpoint{8.740538in}{1.931373in}}%
\pgfpathcurveto{\pgfqpoint{8.744105in}{1.934939in}}{\pgfqpoint{8.746108in}{1.939777in}}{\pgfqpoint{8.746108in}{1.944821in}}%
\pgfpathcurveto{\pgfqpoint{8.746108in}{1.949864in}}{\pgfqpoint{8.744105in}{1.954702in}}{\pgfqpoint{8.740538in}{1.958268in}}%
\pgfpathcurveto{\pgfqpoint{8.736972in}{1.961835in}}{\pgfqpoint{8.732134in}{1.963839in}}{\pgfqpoint{8.727090in}{1.963839in}}%
\pgfpathcurveto{\pgfqpoint{8.722047in}{1.963839in}}{\pgfqpoint{8.717209in}{1.961835in}}{\pgfqpoint{8.713642in}{1.958268in}}%
\pgfpathcurveto{\pgfqpoint{8.710076in}{1.954702in}}{\pgfqpoint{8.708072in}{1.949864in}}{\pgfqpoint{8.708072in}{1.944821in}}%
\pgfpathcurveto{\pgfqpoint{8.708072in}{1.939777in}}{\pgfqpoint{8.710076in}{1.934939in}}{\pgfqpoint{8.713642in}{1.931373in}}%
\pgfpathcurveto{\pgfqpoint{8.717209in}{1.927806in}}{\pgfqpoint{8.722047in}{1.925802in}}{\pgfqpoint{8.727090in}{1.925802in}}%
\pgfpathclose%
\pgfusepath{fill}%
\end{pgfscope}%
\begin{pgfscope}%
\pgfpathrectangle{\pgfqpoint{6.572727in}{0.473000in}}{\pgfqpoint{4.227273in}{3.311000in}}%
\pgfusepath{clip}%
\pgfsetbuttcap%
\pgfsetroundjoin%
\definecolor{currentfill}{rgb}{0.127568,0.566949,0.550556}%
\pgfsetfillcolor{currentfill}%
\pgfsetfillopacity{0.700000}%
\pgfsetlinewidth{0.000000pt}%
\definecolor{currentstroke}{rgb}{0.000000,0.000000,0.000000}%
\pgfsetstrokecolor{currentstroke}%
\pgfsetstrokeopacity{0.700000}%
\pgfsetdash{}{0pt}%
\pgfpathmoveto{\pgfqpoint{7.412554in}{1.351285in}}%
\pgfpathcurveto{\pgfqpoint{7.417598in}{1.351285in}}{\pgfqpoint{7.422436in}{1.353288in}}{\pgfqpoint{7.426002in}{1.356855in}}%
\pgfpathcurveto{\pgfqpoint{7.429568in}{1.360421in}}{\pgfqpoint{7.431572in}{1.365259in}}{\pgfqpoint{7.431572in}{1.370303in}}%
\pgfpathcurveto{\pgfqpoint{7.431572in}{1.375346in}}{\pgfqpoint{7.429568in}{1.380184in}}{\pgfqpoint{7.426002in}{1.383751in}}%
\pgfpathcurveto{\pgfqpoint{7.422436in}{1.387317in}}{\pgfqpoint{7.417598in}{1.389321in}}{\pgfqpoint{7.412554in}{1.389321in}}%
\pgfpathcurveto{\pgfqpoint{7.407510in}{1.389321in}}{\pgfqpoint{7.402673in}{1.387317in}}{\pgfqpoint{7.399106in}{1.383751in}}%
\pgfpathcurveto{\pgfqpoint{7.395540in}{1.380184in}}{\pgfqpoint{7.393536in}{1.375346in}}{\pgfqpoint{7.393536in}{1.370303in}}%
\pgfpathcurveto{\pgfqpoint{7.393536in}{1.365259in}}{\pgfqpoint{7.395540in}{1.360421in}}{\pgfqpoint{7.399106in}{1.356855in}}%
\pgfpathcurveto{\pgfqpoint{7.402673in}{1.353288in}}{\pgfqpoint{7.407510in}{1.351285in}}{\pgfqpoint{7.412554in}{1.351285in}}%
\pgfpathclose%
\pgfusepath{fill}%
\end{pgfscope}%
\begin{pgfscope}%
\pgfpathrectangle{\pgfqpoint{6.572727in}{0.473000in}}{\pgfqpoint{4.227273in}{3.311000in}}%
\pgfusepath{clip}%
\pgfsetbuttcap%
\pgfsetroundjoin%
\definecolor{currentfill}{rgb}{0.127568,0.566949,0.550556}%
\pgfsetfillcolor{currentfill}%
\pgfsetfillopacity{0.700000}%
\pgfsetlinewidth{0.000000pt}%
\definecolor{currentstroke}{rgb}{0.000000,0.000000,0.000000}%
\pgfsetstrokecolor{currentstroke}%
\pgfsetstrokeopacity{0.700000}%
\pgfsetdash{}{0pt}%
\pgfpathmoveto{\pgfqpoint{7.819516in}{2.702903in}}%
\pgfpathcurveto{\pgfqpoint{7.824560in}{2.702903in}}{\pgfqpoint{7.829398in}{2.704907in}}{\pgfqpoint{7.832964in}{2.708473in}}%
\pgfpathcurveto{\pgfqpoint{7.836531in}{2.712039in}}{\pgfqpoint{7.838534in}{2.716877in}}{\pgfqpoint{7.838534in}{2.721921in}}%
\pgfpathcurveto{\pgfqpoint{7.838534in}{2.726965in}}{\pgfqpoint{7.836531in}{2.731802in}}{\pgfqpoint{7.832964in}{2.735369in}}%
\pgfpathcurveto{\pgfqpoint{7.829398in}{2.738935in}}{\pgfqpoint{7.824560in}{2.740939in}}{\pgfqpoint{7.819516in}{2.740939in}}%
\pgfpathcurveto{\pgfqpoint{7.814473in}{2.740939in}}{\pgfqpoint{7.809635in}{2.738935in}}{\pgfqpoint{7.806068in}{2.735369in}}%
\pgfpathcurveto{\pgfqpoint{7.802502in}{2.731802in}}{\pgfqpoint{7.800498in}{2.726965in}}{\pgfqpoint{7.800498in}{2.721921in}}%
\pgfpathcurveto{\pgfqpoint{7.800498in}{2.716877in}}{\pgfqpoint{7.802502in}{2.712039in}}{\pgfqpoint{7.806068in}{2.708473in}}%
\pgfpathcurveto{\pgfqpoint{7.809635in}{2.704907in}}{\pgfqpoint{7.814473in}{2.702903in}}{\pgfqpoint{7.819516in}{2.702903in}}%
\pgfpathclose%
\pgfusepath{fill}%
\end{pgfscope}%
\begin{pgfscope}%
\pgfpathrectangle{\pgfqpoint{6.572727in}{0.473000in}}{\pgfqpoint{4.227273in}{3.311000in}}%
\pgfusepath{clip}%
\pgfsetbuttcap%
\pgfsetroundjoin%
\definecolor{currentfill}{rgb}{0.993248,0.906157,0.143936}%
\pgfsetfillcolor{currentfill}%
\pgfsetfillopacity{0.700000}%
\pgfsetlinewidth{0.000000pt}%
\definecolor{currentstroke}{rgb}{0.000000,0.000000,0.000000}%
\pgfsetstrokecolor{currentstroke}%
\pgfsetstrokeopacity{0.700000}%
\pgfsetdash{}{0pt}%
\pgfpathmoveto{\pgfqpoint{9.634047in}{1.540459in}}%
\pgfpathcurveto{\pgfqpoint{9.639091in}{1.540459in}}{\pgfqpoint{9.643928in}{1.542462in}}{\pgfqpoint{9.647495in}{1.546029in}}%
\pgfpathcurveto{\pgfqpoint{9.651061in}{1.549595in}}{\pgfqpoint{9.653065in}{1.554433in}}{\pgfqpoint{9.653065in}{1.559477in}}%
\pgfpathcurveto{\pgfqpoint{9.653065in}{1.564520in}}{\pgfqpoint{9.651061in}{1.569358in}}{\pgfqpoint{9.647495in}{1.572925in}}%
\pgfpathcurveto{\pgfqpoint{9.643928in}{1.576491in}}{\pgfqpoint{9.639091in}{1.578495in}}{\pgfqpoint{9.634047in}{1.578495in}}%
\pgfpathcurveto{\pgfqpoint{9.629003in}{1.578495in}}{\pgfqpoint{9.624165in}{1.576491in}}{\pgfqpoint{9.620599in}{1.572925in}}%
\pgfpathcurveto{\pgfqpoint{9.617033in}{1.569358in}}{\pgfqpoint{9.615029in}{1.564520in}}{\pgfqpoint{9.615029in}{1.559477in}}%
\pgfpathcurveto{\pgfqpoint{9.615029in}{1.554433in}}{\pgfqpoint{9.617033in}{1.549595in}}{\pgfqpoint{9.620599in}{1.546029in}}%
\pgfpathcurveto{\pgfqpoint{9.624165in}{1.542462in}}{\pgfqpoint{9.629003in}{1.540459in}}{\pgfqpoint{9.634047in}{1.540459in}}%
\pgfpathclose%
\pgfusepath{fill}%
\end{pgfscope}%
\begin{pgfscope}%
\pgfpathrectangle{\pgfqpoint{6.572727in}{0.473000in}}{\pgfqpoint{4.227273in}{3.311000in}}%
\pgfusepath{clip}%
\pgfsetbuttcap%
\pgfsetroundjoin%
\definecolor{currentfill}{rgb}{0.127568,0.566949,0.550556}%
\pgfsetfillcolor{currentfill}%
\pgfsetfillopacity{0.700000}%
\pgfsetlinewidth{0.000000pt}%
\definecolor{currentstroke}{rgb}{0.000000,0.000000,0.000000}%
\pgfsetstrokecolor{currentstroke}%
\pgfsetstrokeopacity{0.700000}%
\pgfsetdash{}{0pt}%
\pgfpathmoveto{\pgfqpoint{7.652595in}{1.908961in}}%
\pgfpathcurveto{\pgfqpoint{7.657639in}{1.908961in}}{\pgfqpoint{7.662477in}{1.910965in}}{\pgfqpoint{7.666043in}{1.914531in}}%
\pgfpathcurveto{\pgfqpoint{7.669610in}{1.918098in}}{\pgfqpoint{7.671613in}{1.922936in}}{\pgfqpoint{7.671613in}{1.927979in}}%
\pgfpathcurveto{\pgfqpoint{7.671613in}{1.933023in}}{\pgfqpoint{7.669610in}{1.937861in}}{\pgfqpoint{7.666043in}{1.941427in}}%
\pgfpathcurveto{\pgfqpoint{7.662477in}{1.944994in}}{\pgfqpoint{7.657639in}{1.946997in}}{\pgfqpoint{7.652595in}{1.946997in}}%
\pgfpathcurveto{\pgfqpoint{7.647552in}{1.946997in}}{\pgfqpoint{7.642714in}{1.944994in}}{\pgfqpoint{7.639147in}{1.941427in}}%
\pgfpathcurveto{\pgfqpoint{7.635581in}{1.937861in}}{\pgfqpoint{7.633577in}{1.933023in}}{\pgfqpoint{7.633577in}{1.927979in}}%
\pgfpathcurveto{\pgfqpoint{7.633577in}{1.922936in}}{\pgfqpoint{7.635581in}{1.918098in}}{\pgfqpoint{7.639147in}{1.914531in}}%
\pgfpathcurveto{\pgfqpoint{7.642714in}{1.910965in}}{\pgfqpoint{7.647552in}{1.908961in}}{\pgfqpoint{7.652595in}{1.908961in}}%
\pgfpathclose%
\pgfusepath{fill}%
\end{pgfscope}%
\begin{pgfscope}%
\pgfpathrectangle{\pgfqpoint{6.572727in}{0.473000in}}{\pgfqpoint{4.227273in}{3.311000in}}%
\pgfusepath{clip}%
\pgfsetbuttcap%
\pgfsetroundjoin%
\definecolor{currentfill}{rgb}{0.127568,0.566949,0.550556}%
\pgfsetfillcolor{currentfill}%
\pgfsetfillopacity{0.700000}%
\pgfsetlinewidth{0.000000pt}%
\definecolor{currentstroke}{rgb}{0.000000,0.000000,0.000000}%
\pgfsetstrokecolor{currentstroke}%
\pgfsetstrokeopacity{0.700000}%
\pgfsetdash{}{0pt}%
\pgfpathmoveto{\pgfqpoint{7.366921in}{1.146513in}}%
\pgfpathcurveto{\pgfqpoint{7.371964in}{1.146513in}}{\pgfqpoint{7.376802in}{1.148517in}}{\pgfqpoint{7.380369in}{1.152084in}}%
\pgfpathcurveto{\pgfqpoint{7.383935in}{1.155650in}}{\pgfqpoint{7.385939in}{1.160488in}}{\pgfqpoint{7.385939in}{1.165531in}}%
\pgfpathcurveto{\pgfqpoint{7.385939in}{1.170575in}}{\pgfqpoint{7.383935in}{1.175413in}}{\pgfqpoint{7.380369in}{1.178979in}}%
\pgfpathcurveto{\pgfqpoint{7.376802in}{1.182546in}}{\pgfqpoint{7.371964in}{1.184550in}}{\pgfqpoint{7.366921in}{1.184550in}}%
\pgfpathcurveto{\pgfqpoint{7.361877in}{1.184550in}}{\pgfqpoint{7.357039in}{1.182546in}}{\pgfqpoint{7.353473in}{1.178979in}}%
\pgfpathcurveto{\pgfqpoint{7.349907in}{1.175413in}}{\pgfqpoint{7.347903in}{1.170575in}}{\pgfqpoint{7.347903in}{1.165531in}}%
\pgfpathcurveto{\pgfqpoint{7.347903in}{1.160488in}}{\pgfqpoint{7.349907in}{1.155650in}}{\pgfqpoint{7.353473in}{1.152084in}}%
\pgfpathcurveto{\pgfqpoint{7.357039in}{1.148517in}}{\pgfqpoint{7.361877in}{1.146513in}}{\pgfqpoint{7.366921in}{1.146513in}}%
\pgfpathclose%
\pgfusepath{fill}%
\end{pgfscope}%
\begin{pgfscope}%
\pgfpathrectangle{\pgfqpoint{6.572727in}{0.473000in}}{\pgfqpoint{4.227273in}{3.311000in}}%
\pgfusepath{clip}%
\pgfsetbuttcap%
\pgfsetroundjoin%
\definecolor{currentfill}{rgb}{0.993248,0.906157,0.143936}%
\pgfsetfillcolor{currentfill}%
\pgfsetfillopacity{0.700000}%
\pgfsetlinewidth{0.000000pt}%
\definecolor{currentstroke}{rgb}{0.000000,0.000000,0.000000}%
\pgfsetstrokecolor{currentstroke}%
\pgfsetstrokeopacity{0.700000}%
\pgfsetdash{}{0pt}%
\pgfpathmoveto{\pgfqpoint{9.702031in}{1.742828in}}%
\pgfpathcurveto{\pgfqpoint{9.707075in}{1.742828in}}{\pgfqpoint{9.711913in}{1.744832in}}{\pgfqpoint{9.715479in}{1.748398in}}%
\pgfpathcurveto{\pgfqpoint{9.719046in}{1.751965in}}{\pgfqpoint{9.721049in}{1.756803in}}{\pgfqpoint{9.721049in}{1.761846in}}%
\pgfpathcurveto{\pgfqpoint{9.721049in}{1.766890in}}{\pgfqpoint{9.719046in}{1.771728in}}{\pgfqpoint{9.715479in}{1.775294in}}%
\pgfpathcurveto{\pgfqpoint{9.711913in}{1.778861in}}{\pgfqpoint{9.707075in}{1.780864in}}{\pgfqpoint{9.702031in}{1.780864in}}%
\pgfpathcurveto{\pgfqpoint{9.696988in}{1.780864in}}{\pgfqpoint{9.692150in}{1.778861in}}{\pgfqpoint{9.688583in}{1.775294in}}%
\pgfpathcurveto{\pgfqpoint{9.685017in}{1.771728in}}{\pgfqpoint{9.683013in}{1.766890in}}{\pgfqpoint{9.683013in}{1.761846in}}%
\pgfpathcurveto{\pgfqpoint{9.683013in}{1.756803in}}{\pgfqpoint{9.685017in}{1.751965in}}{\pgfqpoint{9.688583in}{1.748398in}}%
\pgfpathcurveto{\pgfqpoint{9.692150in}{1.744832in}}{\pgfqpoint{9.696988in}{1.742828in}}{\pgfqpoint{9.702031in}{1.742828in}}%
\pgfpathclose%
\pgfusepath{fill}%
\end{pgfscope}%
\begin{pgfscope}%
\pgfpathrectangle{\pgfqpoint{6.572727in}{0.473000in}}{\pgfqpoint{4.227273in}{3.311000in}}%
\pgfusepath{clip}%
\pgfsetbuttcap%
\pgfsetroundjoin%
\definecolor{currentfill}{rgb}{0.993248,0.906157,0.143936}%
\pgfsetfillcolor{currentfill}%
\pgfsetfillopacity{0.700000}%
\pgfsetlinewidth{0.000000pt}%
\definecolor{currentstroke}{rgb}{0.000000,0.000000,0.000000}%
\pgfsetstrokecolor{currentstroke}%
\pgfsetstrokeopacity{0.700000}%
\pgfsetdash{}{0pt}%
\pgfpathmoveto{\pgfqpoint{9.625621in}{1.849504in}}%
\pgfpathcurveto{\pgfqpoint{9.630665in}{1.849504in}}{\pgfqpoint{9.635503in}{1.851508in}}{\pgfqpoint{9.639069in}{1.855074in}}%
\pgfpathcurveto{\pgfqpoint{9.642635in}{1.858640in}}{\pgfqpoint{9.644639in}{1.863478in}}{\pgfqpoint{9.644639in}{1.868522in}}%
\pgfpathcurveto{\pgfqpoint{9.644639in}{1.873566in}}{\pgfqpoint{9.642635in}{1.878403in}}{\pgfqpoint{9.639069in}{1.881970in}}%
\pgfpathcurveto{\pgfqpoint{9.635503in}{1.885536in}}{\pgfqpoint{9.630665in}{1.887540in}}{\pgfqpoint{9.625621in}{1.887540in}}%
\pgfpathcurveto{\pgfqpoint{9.620577in}{1.887540in}}{\pgfqpoint{9.615740in}{1.885536in}}{\pgfqpoint{9.612173in}{1.881970in}}%
\pgfpathcurveto{\pgfqpoint{9.608607in}{1.878403in}}{\pgfqpoint{9.606603in}{1.873566in}}{\pgfqpoint{9.606603in}{1.868522in}}%
\pgfpathcurveto{\pgfqpoint{9.606603in}{1.863478in}}{\pgfqpoint{9.608607in}{1.858640in}}{\pgfqpoint{9.612173in}{1.855074in}}%
\pgfpathcurveto{\pgfqpoint{9.615740in}{1.851508in}}{\pgfqpoint{9.620577in}{1.849504in}}{\pgfqpoint{9.625621in}{1.849504in}}%
\pgfpathclose%
\pgfusepath{fill}%
\end{pgfscope}%
\begin{pgfscope}%
\pgfpathrectangle{\pgfqpoint{6.572727in}{0.473000in}}{\pgfqpoint{4.227273in}{3.311000in}}%
\pgfusepath{clip}%
\pgfsetbuttcap%
\pgfsetroundjoin%
\definecolor{currentfill}{rgb}{0.127568,0.566949,0.550556}%
\pgfsetfillcolor{currentfill}%
\pgfsetfillopacity{0.700000}%
\pgfsetlinewidth{0.000000pt}%
\definecolor{currentstroke}{rgb}{0.000000,0.000000,0.000000}%
\pgfsetstrokecolor{currentstroke}%
\pgfsetstrokeopacity{0.700000}%
\pgfsetdash{}{0pt}%
\pgfpathmoveto{\pgfqpoint{7.811504in}{3.242782in}}%
\pgfpathcurveto{\pgfqpoint{7.816548in}{3.242782in}}{\pgfqpoint{7.821386in}{3.244786in}}{\pgfqpoint{7.824952in}{3.248352in}}%
\pgfpathcurveto{\pgfqpoint{7.828519in}{3.251919in}}{\pgfqpoint{7.830523in}{3.256756in}}{\pgfqpoint{7.830523in}{3.261800in}}%
\pgfpathcurveto{\pgfqpoint{7.830523in}{3.266844in}}{\pgfqpoint{7.828519in}{3.271682in}}{\pgfqpoint{7.824952in}{3.275248in}}%
\pgfpathcurveto{\pgfqpoint{7.821386in}{3.278814in}}{\pgfqpoint{7.816548in}{3.280818in}}{\pgfqpoint{7.811504in}{3.280818in}}%
\pgfpathcurveto{\pgfqpoint{7.806461in}{3.280818in}}{\pgfqpoint{7.801623in}{3.278814in}}{\pgfqpoint{7.798057in}{3.275248in}}%
\pgfpathcurveto{\pgfqpoint{7.794490in}{3.271682in}}{\pgfqpoint{7.792486in}{3.266844in}}{\pgfqpoint{7.792486in}{3.261800in}}%
\pgfpathcurveto{\pgfqpoint{7.792486in}{3.256756in}}{\pgfqpoint{7.794490in}{3.251919in}}{\pgfqpoint{7.798057in}{3.248352in}}%
\pgfpathcurveto{\pgfqpoint{7.801623in}{3.244786in}}{\pgfqpoint{7.806461in}{3.242782in}}{\pgfqpoint{7.811504in}{3.242782in}}%
\pgfpathclose%
\pgfusepath{fill}%
\end{pgfscope}%
\begin{pgfscope}%
\pgfpathrectangle{\pgfqpoint{6.572727in}{0.473000in}}{\pgfqpoint{4.227273in}{3.311000in}}%
\pgfusepath{clip}%
\pgfsetbuttcap%
\pgfsetroundjoin%
\definecolor{currentfill}{rgb}{0.127568,0.566949,0.550556}%
\pgfsetfillcolor{currentfill}%
\pgfsetfillopacity{0.700000}%
\pgfsetlinewidth{0.000000pt}%
\definecolor{currentstroke}{rgb}{0.000000,0.000000,0.000000}%
\pgfsetstrokecolor{currentstroke}%
\pgfsetstrokeopacity{0.700000}%
\pgfsetdash{}{0pt}%
\pgfpathmoveto{\pgfqpoint{7.262638in}{2.195923in}}%
\pgfpathcurveto{\pgfqpoint{7.267681in}{2.195923in}}{\pgfqpoint{7.272519in}{2.197927in}}{\pgfqpoint{7.276086in}{2.201493in}}%
\pgfpathcurveto{\pgfqpoint{7.279652in}{2.205059in}}{\pgfqpoint{7.281656in}{2.209897in}}{\pgfqpoint{7.281656in}{2.214941in}}%
\pgfpathcurveto{\pgfqpoint{7.281656in}{2.219985in}}{\pgfqpoint{7.279652in}{2.224822in}}{\pgfqpoint{7.276086in}{2.228389in}}%
\pgfpathcurveto{\pgfqpoint{7.272519in}{2.231955in}}{\pgfqpoint{7.267681in}{2.233959in}}{\pgfqpoint{7.262638in}{2.233959in}}%
\pgfpathcurveto{\pgfqpoint{7.257594in}{2.233959in}}{\pgfqpoint{7.252756in}{2.231955in}}{\pgfqpoint{7.249190in}{2.228389in}}%
\pgfpathcurveto{\pgfqpoint{7.245623in}{2.224822in}}{\pgfqpoint{7.243620in}{2.219985in}}{\pgfqpoint{7.243620in}{2.214941in}}%
\pgfpathcurveto{\pgfqpoint{7.243620in}{2.209897in}}{\pgfqpoint{7.245623in}{2.205059in}}{\pgfqpoint{7.249190in}{2.201493in}}%
\pgfpathcurveto{\pgfqpoint{7.252756in}{2.197927in}}{\pgfqpoint{7.257594in}{2.195923in}}{\pgfqpoint{7.262638in}{2.195923in}}%
\pgfpathclose%
\pgfusepath{fill}%
\end{pgfscope}%
\begin{pgfscope}%
\pgfpathrectangle{\pgfqpoint{6.572727in}{0.473000in}}{\pgfqpoint{4.227273in}{3.311000in}}%
\pgfusepath{clip}%
\pgfsetbuttcap%
\pgfsetroundjoin%
\definecolor{currentfill}{rgb}{0.993248,0.906157,0.143936}%
\pgfsetfillcolor{currentfill}%
\pgfsetfillopacity{0.700000}%
\pgfsetlinewidth{0.000000pt}%
\definecolor{currentstroke}{rgb}{0.000000,0.000000,0.000000}%
\pgfsetstrokecolor{currentstroke}%
\pgfsetstrokeopacity{0.700000}%
\pgfsetdash{}{0pt}%
\pgfpathmoveto{\pgfqpoint{9.632931in}{2.299532in}}%
\pgfpathcurveto{\pgfqpoint{9.637975in}{2.299532in}}{\pgfqpoint{9.642812in}{2.301536in}}{\pgfqpoint{9.646379in}{2.305102in}}%
\pgfpathcurveto{\pgfqpoint{9.649945in}{2.308669in}}{\pgfqpoint{9.651949in}{2.313506in}}{\pgfqpoint{9.651949in}{2.318550in}}%
\pgfpathcurveto{\pgfqpoint{9.651949in}{2.323594in}}{\pgfqpoint{9.649945in}{2.328431in}}{\pgfqpoint{9.646379in}{2.331998in}}%
\pgfpathcurveto{\pgfqpoint{9.642812in}{2.335564in}}{\pgfqpoint{9.637975in}{2.337568in}}{\pgfqpoint{9.632931in}{2.337568in}}%
\pgfpathcurveto{\pgfqpoint{9.627887in}{2.337568in}}{\pgfqpoint{9.623050in}{2.335564in}}{\pgfqpoint{9.619483in}{2.331998in}}%
\pgfpathcurveto{\pgfqpoint{9.615917in}{2.328431in}}{\pgfqpoint{9.613913in}{2.323594in}}{\pgfqpoint{9.613913in}{2.318550in}}%
\pgfpathcurveto{\pgfqpoint{9.613913in}{2.313506in}}{\pgfqpoint{9.615917in}{2.308669in}}{\pgfqpoint{9.619483in}{2.305102in}}%
\pgfpathcurveto{\pgfqpoint{9.623050in}{2.301536in}}{\pgfqpoint{9.627887in}{2.299532in}}{\pgfqpoint{9.632931in}{2.299532in}}%
\pgfpathclose%
\pgfusepath{fill}%
\end{pgfscope}%
\begin{pgfscope}%
\pgfpathrectangle{\pgfqpoint{6.572727in}{0.473000in}}{\pgfqpoint{4.227273in}{3.311000in}}%
\pgfusepath{clip}%
\pgfsetbuttcap%
\pgfsetroundjoin%
\definecolor{currentfill}{rgb}{0.993248,0.906157,0.143936}%
\pgfsetfillcolor{currentfill}%
\pgfsetfillopacity{0.700000}%
\pgfsetlinewidth{0.000000pt}%
\definecolor{currentstroke}{rgb}{0.000000,0.000000,0.000000}%
\pgfsetstrokecolor{currentstroke}%
\pgfsetstrokeopacity{0.700000}%
\pgfsetdash{}{0pt}%
\pgfpathmoveto{\pgfqpoint{9.032504in}{1.093435in}}%
\pgfpathcurveto{\pgfqpoint{9.037547in}{1.093435in}}{\pgfqpoint{9.042385in}{1.095439in}}{\pgfqpoint{9.045951in}{1.099006in}}%
\pgfpathcurveto{\pgfqpoint{9.049518in}{1.102572in}}{\pgfqpoint{9.051522in}{1.107410in}}{\pgfqpoint{9.051522in}{1.112453in}}%
\pgfpathcurveto{\pgfqpoint{9.051522in}{1.117497in}}{\pgfqpoint{9.049518in}{1.122335in}}{\pgfqpoint{9.045951in}{1.125901in}}%
\pgfpathcurveto{\pgfqpoint{9.042385in}{1.129468in}}{\pgfqpoint{9.037547in}{1.131472in}}{\pgfqpoint{9.032504in}{1.131472in}}%
\pgfpathcurveto{\pgfqpoint{9.027460in}{1.131472in}}{\pgfqpoint{9.022622in}{1.129468in}}{\pgfqpoint{9.019056in}{1.125901in}}%
\pgfpathcurveto{\pgfqpoint{9.015489in}{1.122335in}}{\pgfqpoint{9.013485in}{1.117497in}}{\pgfqpoint{9.013485in}{1.112453in}}%
\pgfpathcurveto{\pgfqpoint{9.013485in}{1.107410in}}{\pgfqpoint{9.015489in}{1.102572in}}{\pgfqpoint{9.019056in}{1.099006in}}%
\pgfpathcurveto{\pgfqpoint{9.022622in}{1.095439in}}{\pgfqpoint{9.027460in}{1.093435in}}{\pgfqpoint{9.032504in}{1.093435in}}%
\pgfpathclose%
\pgfusepath{fill}%
\end{pgfscope}%
\begin{pgfscope}%
\pgfpathrectangle{\pgfqpoint{6.572727in}{0.473000in}}{\pgfqpoint{4.227273in}{3.311000in}}%
\pgfusepath{clip}%
\pgfsetbuttcap%
\pgfsetroundjoin%
\definecolor{currentfill}{rgb}{0.127568,0.566949,0.550556}%
\pgfsetfillcolor{currentfill}%
\pgfsetfillopacity{0.700000}%
\pgfsetlinewidth{0.000000pt}%
\definecolor{currentstroke}{rgb}{0.000000,0.000000,0.000000}%
\pgfsetstrokecolor{currentstroke}%
\pgfsetstrokeopacity{0.700000}%
\pgfsetdash{}{0pt}%
\pgfpathmoveto{\pgfqpoint{8.096654in}{3.216358in}}%
\pgfpathcurveto{\pgfqpoint{8.101698in}{3.216358in}}{\pgfqpoint{8.106536in}{3.218362in}}{\pgfqpoint{8.110102in}{3.221928in}}%
\pgfpathcurveto{\pgfqpoint{8.113668in}{3.225494in}}{\pgfqpoint{8.115672in}{3.230332in}}{\pgfqpoint{8.115672in}{3.235376in}}%
\pgfpathcurveto{\pgfqpoint{8.115672in}{3.240420in}}{\pgfqpoint{8.113668in}{3.245257in}}{\pgfqpoint{8.110102in}{3.248824in}}%
\pgfpathcurveto{\pgfqpoint{8.106536in}{3.252390in}}{\pgfqpoint{8.101698in}{3.254394in}}{\pgfqpoint{8.096654in}{3.254394in}}%
\pgfpathcurveto{\pgfqpoint{8.091610in}{3.254394in}}{\pgfqpoint{8.086773in}{3.252390in}}{\pgfqpoint{8.083206in}{3.248824in}}%
\pgfpathcurveto{\pgfqpoint{8.079640in}{3.245257in}}{\pgfqpoint{8.077636in}{3.240420in}}{\pgfqpoint{8.077636in}{3.235376in}}%
\pgfpathcurveto{\pgfqpoint{8.077636in}{3.230332in}}{\pgfqpoint{8.079640in}{3.225494in}}{\pgfqpoint{8.083206in}{3.221928in}}%
\pgfpathcurveto{\pgfqpoint{8.086773in}{3.218362in}}{\pgfqpoint{8.091610in}{3.216358in}}{\pgfqpoint{8.096654in}{3.216358in}}%
\pgfpathclose%
\pgfusepath{fill}%
\end{pgfscope}%
\begin{pgfscope}%
\pgfpathrectangle{\pgfqpoint{6.572727in}{0.473000in}}{\pgfqpoint{4.227273in}{3.311000in}}%
\pgfusepath{clip}%
\pgfsetbuttcap%
\pgfsetroundjoin%
\definecolor{currentfill}{rgb}{0.127568,0.566949,0.550556}%
\pgfsetfillcolor{currentfill}%
\pgfsetfillopacity{0.700000}%
\pgfsetlinewidth{0.000000pt}%
\definecolor{currentstroke}{rgb}{0.000000,0.000000,0.000000}%
\pgfsetstrokecolor{currentstroke}%
\pgfsetstrokeopacity{0.700000}%
\pgfsetdash{}{0pt}%
\pgfpathmoveto{\pgfqpoint{8.467173in}{3.106995in}}%
\pgfpathcurveto{\pgfqpoint{8.472217in}{3.106995in}}{\pgfqpoint{8.477054in}{3.108999in}}{\pgfqpoint{8.480621in}{3.112565in}}%
\pgfpathcurveto{\pgfqpoint{8.484187in}{3.116132in}}{\pgfqpoint{8.486191in}{3.120970in}}{\pgfqpoint{8.486191in}{3.126013in}}%
\pgfpathcurveto{\pgfqpoint{8.486191in}{3.131057in}}{\pgfqpoint{8.484187in}{3.135895in}}{\pgfqpoint{8.480621in}{3.139461in}}%
\pgfpathcurveto{\pgfqpoint{8.477054in}{3.143028in}}{\pgfqpoint{8.472217in}{3.145031in}}{\pgfqpoint{8.467173in}{3.145031in}}%
\pgfpathcurveto{\pgfqpoint{8.462129in}{3.145031in}}{\pgfqpoint{8.457292in}{3.143028in}}{\pgfqpoint{8.453725in}{3.139461in}}%
\pgfpathcurveto{\pgfqpoint{8.450159in}{3.135895in}}{\pgfqpoint{8.448155in}{3.131057in}}{\pgfqpoint{8.448155in}{3.126013in}}%
\pgfpathcurveto{\pgfqpoint{8.448155in}{3.120970in}}{\pgfqpoint{8.450159in}{3.116132in}}{\pgfqpoint{8.453725in}{3.112565in}}%
\pgfpathcurveto{\pgfqpoint{8.457292in}{3.108999in}}{\pgfqpoint{8.462129in}{3.106995in}}{\pgfqpoint{8.467173in}{3.106995in}}%
\pgfpathclose%
\pgfusepath{fill}%
\end{pgfscope}%
\begin{pgfscope}%
\pgfpathrectangle{\pgfqpoint{6.572727in}{0.473000in}}{\pgfqpoint{4.227273in}{3.311000in}}%
\pgfusepath{clip}%
\pgfsetbuttcap%
\pgfsetroundjoin%
\definecolor{currentfill}{rgb}{0.993248,0.906157,0.143936}%
\pgfsetfillcolor{currentfill}%
\pgfsetfillopacity{0.700000}%
\pgfsetlinewidth{0.000000pt}%
\definecolor{currentstroke}{rgb}{0.000000,0.000000,0.000000}%
\pgfsetstrokecolor{currentstroke}%
\pgfsetstrokeopacity{0.700000}%
\pgfsetdash{}{0pt}%
\pgfpathmoveto{\pgfqpoint{9.584894in}{1.201049in}}%
\pgfpathcurveto{\pgfqpoint{9.589938in}{1.201049in}}{\pgfqpoint{9.594776in}{1.203053in}}{\pgfqpoint{9.598342in}{1.206620in}}%
\pgfpathcurveto{\pgfqpoint{9.601909in}{1.210186in}}{\pgfqpoint{9.603913in}{1.215024in}}{\pgfqpoint{9.603913in}{1.220068in}}%
\pgfpathcurveto{\pgfqpoint{9.603913in}{1.225111in}}{\pgfqpoint{9.601909in}{1.229949in}}{\pgfqpoint{9.598342in}{1.233516in}}%
\pgfpathcurveto{\pgfqpoint{9.594776in}{1.237082in}}{\pgfqpoint{9.589938in}{1.239086in}}{\pgfqpoint{9.584894in}{1.239086in}}%
\pgfpathcurveto{\pgfqpoint{9.579851in}{1.239086in}}{\pgfqpoint{9.575013in}{1.237082in}}{\pgfqpoint{9.571447in}{1.233516in}}%
\pgfpathcurveto{\pgfqpoint{9.567880in}{1.229949in}}{\pgfqpoint{9.565876in}{1.225111in}}{\pgfqpoint{9.565876in}{1.220068in}}%
\pgfpathcurveto{\pgfqpoint{9.565876in}{1.215024in}}{\pgfqpoint{9.567880in}{1.210186in}}{\pgfqpoint{9.571447in}{1.206620in}}%
\pgfpathcurveto{\pgfqpoint{9.575013in}{1.203053in}}{\pgfqpoint{9.579851in}{1.201049in}}{\pgfqpoint{9.584894in}{1.201049in}}%
\pgfpathclose%
\pgfusepath{fill}%
\end{pgfscope}%
\begin{pgfscope}%
\pgfpathrectangle{\pgfqpoint{6.572727in}{0.473000in}}{\pgfqpoint{4.227273in}{3.311000in}}%
\pgfusepath{clip}%
\pgfsetbuttcap%
\pgfsetroundjoin%
\definecolor{currentfill}{rgb}{0.993248,0.906157,0.143936}%
\pgfsetfillcolor{currentfill}%
\pgfsetfillopacity{0.700000}%
\pgfsetlinewidth{0.000000pt}%
\definecolor{currentstroke}{rgb}{0.000000,0.000000,0.000000}%
\pgfsetstrokecolor{currentstroke}%
\pgfsetstrokeopacity{0.700000}%
\pgfsetdash{}{0pt}%
\pgfpathmoveto{\pgfqpoint{9.776371in}{1.539410in}}%
\pgfpathcurveto{\pgfqpoint{9.781414in}{1.539410in}}{\pgfqpoint{9.786252in}{1.541413in}}{\pgfqpoint{9.789819in}{1.544980in}}%
\pgfpathcurveto{\pgfqpoint{9.793385in}{1.548546in}}{\pgfqpoint{9.795389in}{1.553384in}}{\pgfqpoint{9.795389in}{1.558428in}}%
\pgfpathcurveto{\pgfqpoint{9.795389in}{1.563471in}}{\pgfqpoint{9.793385in}{1.568309in}}{\pgfqpoint{9.789819in}{1.571876in}}%
\pgfpathcurveto{\pgfqpoint{9.786252in}{1.575442in}}{\pgfqpoint{9.781414in}{1.577446in}}{\pgfqpoint{9.776371in}{1.577446in}}%
\pgfpathcurveto{\pgfqpoint{9.771327in}{1.577446in}}{\pgfqpoint{9.766489in}{1.575442in}}{\pgfqpoint{9.762923in}{1.571876in}}%
\pgfpathcurveto{\pgfqpoint{9.759356in}{1.568309in}}{\pgfqpoint{9.757352in}{1.563471in}}{\pgfqpoint{9.757352in}{1.558428in}}%
\pgfpathcurveto{\pgfqpoint{9.757352in}{1.553384in}}{\pgfqpoint{9.759356in}{1.548546in}}{\pgfqpoint{9.762923in}{1.544980in}}%
\pgfpathcurveto{\pgfqpoint{9.766489in}{1.541413in}}{\pgfqpoint{9.771327in}{1.539410in}}{\pgfqpoint{9.776371in}{1.539410in}}%
\pgfpathclose%
\pgfusepath{fill}%
\end{pgfscope}%
\begin{pgfscope}%
\pgfpathrectangle{\pgfqpoint{6.572727in}{0.473000in}}{\pgfqpoint{4.227273in}{3.311000in}}%
\pgfusepath{clip}%
\pgfsetbuttcap%
\pgfsetroundjoin%
\definecolor{currentfill}{rgb}{0.127568,0.566949,0.550556}%
\pgfsetfillcolor{currentfill}%
\pgfsetfillopacity{0.700000}%
\pgfsetlinewidth{0.000000pt}%
\definecolor{currentstroke}{rgb}{0.000000,0.000000,0.000000}%
\pgfsetstrokecolor{currentstroke}%
\pgfsetstrokeopacity{0.700000}%
\pgfsetdash{}{0pt}%
\pgfpathmoveto{\pgfqpoint{8.333781in}{2.510055in}}%
\pgfpathcurveto{\pgfqpoint{8.338824in}{2.510055in}}{\pgfqpoint{8.343662in}{2.512059in}}{\pgfqpoint{8.347229in}{2.515626in}}%
\pgfpathcurveto{\pgfqpoint{8.350795in}{2.519192in}}{\pgfqpoint{8.352799in}{2.524030in}}{\pgfqpoint{8.352799in}{2.529074in}}%
\pgfpathcurveto{\pgfqpoint{8.352799in}{2.534117in}}{\pgfqpoint{8.350795in}{2.538955in}}{\pgfqpoint{8.347229in}{2.542521in}}%
\pgfpathcurveto{\pgfqpoint{8.343662in}{2.546088in}}{\pgfqpoint{8.338824in}{2.548092in}}{\pgfqpoint{8.333781in}{2.548092in}}%
\pgfpathcurveto{\pgfqpoint{8.328737in}{2.548092in}}{\pgfqpoint{8.323899in}{2.546088in}}{\pgfqpoint{8.320333in}{2.542521in}}%
\pgfpathcurveto{\pgfqpoint{8.316766in}{2.538955in}}{\pgfqpoint{8.314763in}{2.534117in}}{\pgfqpoint{8.314763in}{2.529074in}}%
\pgfpathcurveto{\pgfqpoint{8.314763in}{2.524030in}}{\pgfqpoint{8.316766in}{2.519192in}}{\pgfqpoint{8.320333in}{2.515626in}}%
\pgfpathcurveto{\pgfqpoint{8.323899in}{2.512059in}}{\pgfqpoint{8.328737in}{2.510055in}}{\pgfqpoint{8.333781in}{2.510055in}}%
\pgfpathclose%
\pgfusepath{fill}%
\end{pgfscope}%
\begin{pgfscope}%
\pgfpathrectangle{\pgfqpoint{6.572727in}{0.473000in}}{\pgfqpoint{4.227273in}{3.311000in}}%
\pgfusepath{clip}%
\pgfsetbuttcap%
\pgfsetroundjoin%
\definecolor{currentfill}{rgb}{0.993248,0.906157,0.143936}%
\pgfsetfillcolor{currentfill}%
\pgfsetfillopacity{0.700000}%
\pgfsetlinewidth{0.000000pt}%
\definecolor{currentstroke}{rgb}{0.000000,0.000000,0.000000}%
\pgfsetstrokecolor{currentstroke}%
\pgfsetstrokeopacity{0.700000}%
\pgfsetdash{}{0pt}%
\pgfpathmoveto{\pgfqpoint{9.943738in}{1.954177in}}%
\pgfpathcurveto{\pgfqpoint{9.948782in}{1.954177in}}{\pgfqpoint{9.953619in}{1.956181in}}{\pgfqpoint{9.957186in}{1.959747in}}%
\pgfpathcurveto{\pgfqpoint{9.960752in}{1.963313in}}{\pgfqpoint{9.962756in}{1.968151in}}{\pgfqpoint{9.962756in}{1.973195in}}%
\pgfpathcurveto{\pgfqpoint{9.962756in}{1.978238in}}{\pgfqpoint{9.960752in}{1.983076in}}{\pgfqpoint{9.957186in}{1.986643in}}%
\pgfpathcurveto{\pgfqpoint{9.953619in}{1.990209in}}{\pgfqpoint{9.948782in}{1.992213in}}{\pgfqpoint{9.943738in}{1.992213in}}%
\pgfpathcurveto{\pgfqpoint{9.938694in}{1.992213in}}{\pgfqpoint{9.933856in}{1.990209in}}{\pgfqpoint{9.930290in}{1.986643in}}%
\pgfpathcurveto{\pgfqpoint{9.926724in}{1.983076in}}{\pgfqpoint{9.924720in}{1.978238in}}{\pgfqpoint{9.924720in}{1.973195in}}%
\pgfpathcurveto{\pgfqpoint{9.924720in}{1.968151in}}{\pgfqpoint{9.926724in}{1.963313in}}{\pgfqpoint{9.930290in}{1.959747in}}%
\pgfpathcurveto{\pgfqpoint{9.933856in}{1.956181in}}{\pgfqpoint{9.938694in}{1.954177in}}{\pgfqpoint{9.943738in}{1.954177in}}%
\pgfpathclose%
\pgfusepath{fill}%
\end{pgfscope}%
\begin{pgfscope}%
\pgfpathrectangle{\pgfqpoint{6.572727in}{0.473000in}}{\pgfqpoint{4.227273in}{3.311000in}}%
\pgfusepath{clip}%
\pgfsetbuttcap%
\pgfsetroundjoin%
\definecolor{currentfill}{rgb}{0.993248,0.906157,0.143936}%
\pgfsetfillcolor{currentfill}%
\pgfsetfillopacity{0.700000}%
\pgfsetlinewidth{0.000000pt}%
\definecolor{currentstroke}{rgb}{0.000000,0.000000,0.000000}%
\pgfsetstrokecolor{currentstroke}%
\pgfsetstrokeopacity{0.700000}%
\pgfsetdash{}{0pt}%
\pgfpathmoveto{\pgfqpoint{9.140389in}{2.026152in}}%
\pgfpathcurveto{\pgfqpoint{9.145433in}{2.026152in}}{\pgfqpoint{9.150271in}{2.028156in}}{\pgfqpoint{9.153837in}{2.031723in}}%
\pgfpathcurveto{\pgfqpoint{9.157403in}{2.035289in}}{\pgfqpoint{9.159407in}{2.040127in}}{\pgfqpoint{9.159407in}{2.045170in}}%
\pgfpathcurveto{\pgfqpoint{9.159407in}{2.050214in}}{\pgfqpoint{9.157403in}{2.055052in}}{\pgfqpoint{9.153837in}{2.058618in}}%
\pgfpathcurveto{\pgfqpoint{9.150271in}{2.062185in}}{\pgfqpoint{9.145433in}{2.064189in}}{\pgfqpoint{9.140389in}{2.064189in}}%
\pgfpathcurveto{\pgfqpoint{9.135345in}{2.064189in}}{\pgfqpoint{9.130508in}{2.062185in}}{\pgfqpoint{9.126941in}{2.058618in}}%
\pgfpathcurveto{\pgfqpoint{9.123375in}{2.055052in}}{\pgfqpoint{9.121371in}{2.050214in}}{\pgfqpoint{9.121371in}{2.045170in}}%
\pgfpathcurveto{\pgfqpoint{9.121371in}{2.040127in}}{\pgfqpoint{9.123375in}{2.035289in}}{\pgfqpoint{9.126941in}{2.031723in}}%
\pgfpathcurveto{\pgfqpoint{9.130508in}{2.028156in}}{\pgfqpoint{9.135345in}{2.026152in}}{\pgfqpoint{9.140389in}{2.026152in}}%
\pgfpathclose%
\pgfusepath{fill}%
\end{pgfscope}%
\begin{pgfscope}%
\pgfpathrectangle{\pgfqpoint{6.572727in}{0.473000in}}{\pgfqpoint{4.227273in}{3.311000in}}%
\pgfusepath{clip}%
\pgfsetbuttcap%
\pgfsetroundjoin%
\definecolor{currentfill}{rgb}{0.127568,0.566949,0.550556}%
\pgfsetfillcolor{currentfill}%
\pgfsetfillopacity{0.700000}%
\pgfsetlinewidth{0.000000pt}%
\definecolor{currentstroke}{rgb}{0.000000,0.000000,0.000000}%
\pgfsetstrokecolor{currentstroke}%
\pgfsetstrokeopacity{0.700000}%
\pgfsetdash{}{0pt}%
\pgfpathmoveto{\pgfqpoint{7.638007in}{1.446859in}}%
\pgfpathcurveto{\pgfqpoint{7.643051in}{1.446859in}}{\pgfqpoint{7.647889in}{1.448863in}}{\pgfqpoint{7.651455in}{1.452430in}}%
\pgfpathcurveto{\pgfqpoint{7.655021in}{1.455996in}}{\pgfqpoint{7.657025in}{1.460834in}}{\pgfqpoint{7.657025in}{1.465877in}}%
\pgfpathcurveto{\pgfqpoint{7.657025in}{1.470921in}}{\pgfqpoint{7.655021in}{1.475759in}}{\pgfqpoint{7.651455in}{1.479325in}}%
\pgfpathcurveto{\pgfqpoint{7.647889in}{1.482892in}}{\pgfqpoint{7.643051in}{1.484896in}}{\pgfqpoint{7.638007in}{1.484896in}}%
\pgfpathcurveto{\pgfqpoint{7.632964in}{1.484896in}}{\pgfqpoint{7.628126in}{1.482892in}}{\pgfqpoint{7.624559in}{1.479325in}}%
\pgfpathcurveto{\pgfqpoint{7.620993in}{1.475759in}}{\pgfqpoint{7.618989in}{1.470921in}}{\pgfqpoint{7.618989in}{1.465877in}}%
\pgfpathcurveto{\pgfqpoint{7.618989in}{1.460834in}}{\pgfqpoint{7.620993in}{1.455996in}}{\pgfqpoint{7.624559in}{1.452430in}}%
\pgfpathcurveto{\pgfqpoint{7.628126in}{1.448863in}}{\pgfqpoint{7.632964in}{1.446859in}}{\pgfqpoint{7.638007in}{1.446859in}}%
\pgfpathclose%
\pgfusepath{fill}%
\end{pgfscope}%
\begin{pgfscope}%
\pgfpathrectangle{\pgfqpoint{6.572727in}{0.473000in}}{\pgfqpoint{4.227273in}{3.311000in}}%
\pgfusepath{clip}%
\pgfsetbuttcap%
\pgfsetroundjoin%
\definecolor{currentfill}{rgb}{0.993248,0.906157,0.143936}%
\pgfsetfillcolor{currentfill}%
\pgfsetfillopacity{0.700000}%
\pgfsetlinewidth{0.000000pt}%
\definecolor{currentstroke}{rgb}{0.000000,0.000000,0.000000}%
\pgfsetstrokecolor{currentstroke}%
\pgfsetstrokeopacity{0.700000}%
\pgfsetdash{}{0pt}%
\pgfpathmoveto{\pgfqpoint{9.578576in}{1.057885in}}%
\pgfpathcurveto{\pgfqpoint{9.583620in}{1.057885in}}{\pgfqpoint{9.588458in}{1.059888in}}{\pgfqpoint{9.592024in}{1.063455in}}%
\pgfpathcurveto{\pgfqpoint{9.595590in}{1.067021in}}{\pgfqpoint{9.597594in}{1.071859in}}{\pgfqpoint{9.597594in}{1.076903in}}%
\pgfpathcurveto{\pgfqpoint{9.597594in}{1.081946in}}{\pgfqpoint{9.595590in}{1.086784in}}{\pgfqpoint{9.592024in}{1.090351in}}%
\pgfpathcurveto{\pgfqpoint{9.588458in}{1.093917in}}{\pgfqpoint{9.583620in}{1.095921in}}{\pgfqpoint{9.578576in}{1.095921in}}%
\pgfpathcurveto{\pgfqpoint{9.573532in}{1.095921in}}{\pgfqpoint{9.568695in}{1.093917in}}{\pgfqpoint{9.565128in}{1.090351in}}%
\pgfpathcurveto{\pgfqpoint{9.561562in}{1.086784in}}{\pgfqpoint{9.559558in}{1.081946in}}{\pgfqpoint{9.559558in}{1.076903in}}%
\pgfpathcurveto{\pgfqpoint{9.559558in}{1.071859in}}{\pgfqpoint{9.561562in}{1.067021in}}{\pgfqpoint{9.565128in}{1.063455in}}%
\pgfpathcurveto{\pgfqpoint{9.568695in}{1.059888in}}{\pgfqpoint{9.573532in}{1.057885in}}{\pgfqpoint{9.578576in}{1.057885in}}%
\pgfpathclose%
\pgfusepath{fill}%
\end{pgfscope}%
\begin{pgfscope}%
\pgfpathrectangle{\pgfqpoint{6.572727in}{0.473000in}}{\pgfqpoint{4.227273in}{3.311000in}}%
\pgfusepath{clip}%
\pgfsetbuttcap%
\pgfsetroundjoin%
\definecolor{currentfill}{rgb}{0.993248,0.906157,0.143936}%
\pgfsetfillcolor{currentfill}%
\pgfsetfillopacity{0.700000}%
\pgfsetlinewidth{0.000000pt}%
\definecolor{currentstroke}{rgb}{0.000000,0.000000,0.000000}%
\pgfsetstrokecolor{currentstroke}%
\pgfsetstrokeopacity{0.700000}%
\pgfsetdash{}{0pt}%
\pgfpathmoveto{\pgfqpoint{9.907620in}{1.697606in}}%
\pgfpathcurveto{\pgfqpoint{9.912663in}{1.697606in}}{\pgfqpoint{9.917501in}{1.699610in}}{\pgfqpoint{9.921068in}{1.703177in}}%
\pgfpathcurveto{\pgfqpoint{9.924634in}{1.706743in}}{\pgfqpoint{9.926638in}{1.711581in}}{\pgfqpoint{9.926638in}{1.716625in}}%
\pgfpathcurveto{\pgfqpoint{9.926638in}{1.721668in}}{\pgfqpoint{9.924634in}{1.726506in}}{\pgfqpoint{9.921068in}{1.730072in}}%
\pgfpathcurveto{\pgfqpoint{9.917501in}{1.733639in}}{\pgfqpoint{9.912663in}{1.735643in}}{\pgfqpoint{9.907620in}{1.735643in}}%
\pgfpathcurveto{\pgfqpoint{9.902576in}{1.735643in}}{\pgfqpoint{9.897738in}{1.733639in}}{\pgfqpoint{9.894172in}{1.730072in}}%
\pgfpathcurveto{\pgfqpoint{9.890605in}{1.726506in}}{\pgfqpoint{9.888602in}{1.721668in}}{\pgfqpoint{9.888602in}{1.716625in}}%
\pgfpathcurveto{\pgfqpoint{9.888602in}{1.711581in}}{\pgfqpoint{9.890605in}{1.706743in}}{\pgfqpoint{9.894172in}{1.703177in}}%
\pgfpathcurveto{\pgfqpoint{9.897738in}{1.699610in}}{\pgfqpoint{9.902576in}{1.697606in}}{\pgfqpoint{9.907620in}{1.697606in}}%
\pgfpathclose%
\pgfusepath{fill}%
\end{pgfscope}%
\begin{pgfscope}%
\pgfpathrectangle{\pgfqpoint{6.572727in}{0.473000in}}{\pgfqpoint{4.227273in}{3.311000in}}%
\pgfusepath{clip}%
\pgfsetbuttcap%
\pgfsetroundjoin%
\definecolor{currentfill}{rgb}{0.993248,0.906157,0.143936}%
\pgfsetfillcolor{currentfill}%
\pgfsetfillopacity{0.700000}%
\pgfsetlinewidth{0.000000pt}%
\definecolor{currentstroke}{rgb}{0.000000,0.000000,0.000000}%
\pgfsetstrokecolor{currentstroke}%
\pgfsetstrokeopacity{0.700000}%
\pgfsetdash{}{0pt}%
\pgfpathmoveto{\pgfqpoint{9.089290in}{1.680666in}}%
\pgfpathcurveto{\pgfqpoint{9.094334in}{1.680666in}}{\pgfqpoint{9.099172in}{1.682670in}}{\pgfqpoint{9.102738in}{1.686236in}}%
\pgfpathcurveto{\pgfqpoint{9.106305in}{1.689803in}}{\pgfqpoint{9.108308in}{1.694641in}}{\pgfqpoint{9.108308in}{1.699684in}}%
\pgfpathcurveto{\pgfqpoint{9.108308in}{1.704728in}}{\pgfqpoint{9.106305in}{1.709566in}}{\pgfqpoint{9.102738in}{1.713132in}}%
\pgfpathcurveto{\pgfqpoint{9.099172in}{1.716699in}}{\pgfqpoint{9.094334in}{1.718702in}}{\pgfqpoint{9.089290in}{1.718702in}}%
\pgfpathcurveto{\pgfqpoint{9.084247in}{1.718702in}}{\pgfqpoint{9.079409in}{1.716699in}}{\pgfqpoint{9.075842in}{1.713132in}}%
\pgfpathcurveto{\pgfqpoint{9.072276in}{1.709566in}}{\pgfqpoint{9.070272in}{1.704728in}}{\pgfqpoint{9.070272in}{1.699684in}}%
\pgfpathcurveto{\pgfqpoint{9.070272in}{1.694641in}}{\pgfqpoint{9.072276in}{1.689803in}}{\pgfqpoint{9.075842in}{1.686236in}}%
\pgfpathcurveto{\pgfqpoint{9.079409in}{1.682670in}}{\pgfqpoint{9.084247in}{1.680666in}}{\pgfqpoint{9.089290in}{1.680666in}}%
\pgfpathclose%
\pgfusepath{fill}%
\end{pgfscope}%
\begin{pgfscope}%
\pgfpathrectangle{\pgfqpoint{6.572727in}{0.473000in}}{\pgfqpoint{4.227273in}{3.311000in}}%
\pgfusepath{clip}%
\pgfsetbuttcap%
\pgfsetroundjoin%
\definecolor{currentfill}{rgb}{0.993248,0.906157,0.143936}%
\pgfsetfillcolor{currentfill}%
\pgfsetfillopacity{0.700000}%
\pgfsetlinewidth{0.000000pt}%
\definecolor{currentstroke}{rgb}{0.000000,0.000000,0.000000}%
\pgfsetstrokecolor{currentstroke}%
\pgfsetstrokeopacity{0.700000}%
\pgfsetdash{}{0pt}%
\pgfpathmoveto{\pgfqpoint{9.091430in}{2.452748in}}%
\pgfpathcurveto{\pgfqpoint{9.096473in}{2.452748in}}{\pgfqpoint{9.101311in}{2.454751in}}{\pgfqpoint{9.104878in}{2.458318in}}%
\pgfpathcurveto{\pgfqpoint{9.108444in}{2.461884in}}{\pgfqpoint{9.110448in}{2.466722in}}{\pgfqpoint{9.110448in}{2.471766in}}%
\pgfpathcurveto{\pgfqpoint{9.110448in}{2.476809in}}{\pgfqpoint{9.108444in}{2.481647in}}{\pgfqpoint{9.104878in}{2.485214in}}%
\pgfpathcurveto{\pgfqpoint{9.101311in}{2.488780in}}{\pgfqpoint{9.096473in}{2.490784in}}{\pgfqpoint{9.091430in}{2.490784in}}%
\pgfpathcurveto{\pgfqpoint{9.086386in}{2.490784in}}{\pgfqpoint{9.081548in}{2.488780in}}{\pgfqpoint{9.077982in}{2.485214in}}%
\pgfpathcurveto{\pgfqpoint{9.074415in}{2.481647in}}{\pgfqpoint{9.072412in}{2.476809in}}{\pgfqpoint{9.072412in}{2.471766in}}%
\pgfpathcurveto{\pgfqpoint{9.072412in}{2.466722in}}{\pgfqpoint{9.074415in}{2.461884in}}{\pgfqpoint{9.077982in}{2.458318in}}%
\pgfpathcurveto{\pgfqpoint{9.081548in}{2.454751in}}{\pgfqpoint{9.086386in}{2.452748in}}{\pgfqpoint{9.091430in}{2.452748in}}%
\pgfpathclose%
\pgfusepath{fill}%
\end{pgfscope}%
\begin{pgfscope}%
\pgfpathrectangle{\pgfqpoint{6.572727in}{0.473000in}}{\pgfqpoint{4.227273in}{3.311000in}}%
\pgfusepath{clip}%
\pgfsetbuttcap%
\pgfsetroundjoin%
\definecolor{currentfill}{rgb}{0.127568,0.566949,0.550556}%
\pgfsetfillcolor{currentfill}%
\pgfsetfillopacity{0.700000}%
\pgfsetlinewidth{0.000000pt}%
\definecolor{currentstroke}{rgb}{0.000000,0.000000,0.000000}%
\pgfsetstrokecolor{currentstroke}%
\pgfsetstrokeopacity{0.700000}%
\pgfsetdash{}{0pt}%
\pgfpathmoveto{\pgfqpoint{7.440829in}{1.697697in}}%
\pgfpathcurveto{\pgfqpoint{7.445873in}{1.697697in}}{\pgfqpoint{7.450711in}{1.699701in}}{\pgfqpoint{7.454277in}{1.703267in}}%
\pgfpathcurveto{\pgfqpoint{7.457844in}{1.706834in}}{\pgfqpoint{7.459847in}{1.711671in}}{\pgfqpoint{7.459847in}{1.716715in}}%
\pgfpathcurveto{\pgfqpoint{7.459847in}{1.721759in}}{\pgfqpoint{7.457844in}{1.726596in}}{\pgfqpoint{7.454277in}{1.730163in}}%
\pgfpathcurveto{\pgfqpoint{7.450711in}{1.733729in}}{\pgfqpoint{7.445873in}{1.735733in}}{\pgfqpoint{7.440829in}{1.735733in}}%
\pgfpathcurveto{\pgfqpoint{7.435786in}{1.735733in}}{\pgfqpoint{7.430948in}{1.733729in}}{\pgfqpoint{7.427381in}{1.730163in}}%
\pgfpathcurveto{\pgfqpoint{7.423815in}{1.726596in}}{\pgfqpoint{7.421811in}{1.721759in}}{\pgfqpoint{7.421811in}{1.716715in}}%
\pgfpathcurveto{\pgfqpoint{7.421811in}{1.711671in}}{\pgfqpoint{7.423815in}{1.706834in}}{\pgfqpoint{7.427381in}{1.703267in}}%
\pgfpathcurveto{\pgfqpoint{7.430948in}{1.699701in}}{\pgfqpoint{7.435786in}{1.697697in}}{\pgfqpoint{7.440829in}{1.697697in}}%
\pgfpathclose%
\pgfusepath{fill}%
\end{pgfscope}%
\begin{pgfscope}%
\pgfpathrectangle{\pgfqpoint{6.572727in}{0.473000in}}{\pgfqpoint{4.227273in}{3.311000in}}%
\pgfusepath{clip}%
\pgfsetbuttcap%
\pgfsetroundjoin%
\definecolor{currentfill}{rgb}{0.127568,0.566949,0.550556}%
\pgfsetfillcolor{currentfill}%
\pgfsetfillopacity{0.700000}%
\pgfsetlinewidth{0.000000pt}%
\definecolor{currentstroke}{rgb}{0.000000,0.000000,0.000000}%
\pgfsetstrokecolor{currentstroke}%
\pgfsetstrokeopacity{0.700000}%
\pgfsetdash{}{0pt}%
\pgfpathmoveto{\pgfqpoint{8.116962in}{2.859689in}}%
\pgfpathcurveto{\pgfqpoint{8.122006in}{2.859689in}}{\pgfqpoint{8.126844in}{2.861693in}}{\pgfqpoint{8.130410in}{2.865260in}}%
\pgfpathcurveto{\pgfqpoint{8.133977in}{2.868826in}}{\pgfqpoint{8.135980in}{2.873664in}}{\pgfqpoint{8.135980in}{2.878707in}}%
\pgfpathcurveto{\pgfqpoint{8.135980in}{2.883751in}}{\pgfqpoint{8.133977in}{2.888589in}}{\pgfqpoint{8.130410in}{2.892155in}}%
\pgfpathcurveto{\pgfqpoint{8.126844in}{2.895722in}}{\pgfqpoint{8.122006in}{2.897726in}}{\pgfqpoint{8.116962in}{2.897726in}}%
\pgfpathcurveto{\pgfqpoint{8.111919in}{2.897726in}}{\pgfqpoint{8.107081in}{2.895722in}}{\pgfqpoint{8.103514in}{2.892155in}}%
\pgfpathcurveto{\pgfqpoint{8.099948in}{2.888589in}}{\pgfqpoint{8.097944in}{2.883751in}}{\pgfqpoint{8.097944in}{2.878707in}}%
\pgfpathcurveto{\pgfqpoint{8.097944in}{2.873664in}}{\pgfqpoint{8.099948in}{2.868826in}}{\pgfqpoint{8.103514in}{2.865260in}}%
\pgfpathcurveto{\pgfqpoint{8.107081in}{2.861693in}}{\pgfqpoint{8.111919in}{2.859689in}}{\pgfqpoint{8.116962in}{2.859689in}}%
\pgfpathclose%
\pgfusepath{fill}%
\end{pgfscope}%
\begin{pgfscope}%
\pgfpathrectangle{\pgfqpoint{6.572727in}{0.473000in}}{\pgfqpoint{4.227273in}{3.311000in}}%
\pgfusepath{clip}%
\pgfsetbuttcap%
\pgfsetroundjoin%
\definecolor{currentfill}{rgb}{0.993248,0.906157,0.143936}%
\pgfsetfillcolor{currentfill}%
\pgfsetfillopacity{0.700000}%
\pgfsetlinewidth{0.000000pt}%
\definecolor{currentstroke}{rgb}{0.000000,0.000000,0.000000}%
\pgfsetstrokecolor{currentstroke}%
\pgfsetstrokeopacity{0.700000}%
\pgfsetdash{}{0pt}%
\pgfpathmoveto{\pgfqpoint{9.851922in}{1.850777in}}%
\pgfpathcurveto{\pgfqpoint{9.856966in}{1.850777in}}{\pgfqpoint{9.861804in}{1.852781in}}{\pgfqpoint{9.865370in}{1.856348in}}%
\pgfpathcurveto{\pgfqpoint{9.868937in}{1.859914in}}{\pgfqpoint{9.870941in}{1.864752in}}{\pgfqpoint{9.870941in}{1.869796in}}%
\pgfpathcurveto{\pgfqpoint{9.870941in}{1.874839in}}{\pgfqpoint{9.868937in}{1.879677in}}{\pgfqpoint{9.865370in}{1.883243in}}%
\pgfpathcurveto{\pgfqpoint{9.861804in}{1.886810in}}{\pgfqpoint{9.856966in}{1.888814in}}{\pgfqpoint{9.851922in}{1.888814in}}%
\pgfpathcurveto{\pgfqpoint{9.846879in}{1.888814in}}{\pgfqpoint{9.842041in}{1.886810in}}{\pgfqpoint{9.838475in}{1.883243in}}%
\pgfpathcurveto{\pgfqpoint{9.834908in}{1.879677in}}{\pgfqpoint{9.832904in}{1.874839in}}{\pgfqpoint{9.832904in}{1.869796in}}%
\pgfpathcurveto{\pgfqpoint{9.832904in}{1.864752in}}{\pgfqpoint{9.834908in}{1.859914in}}{\pgfqpoint{9.838475in}{1.856348in}}%
\pgfpathcurveto{\pgfqpoint{9.842041in}{1.852781in}}{\pgfqpoint{9.846879in}{1.850777in}}{\pgfqpoint{9.851922in}{1.850777in}}%
\pgfpathclose%
\pgfusepath{fill}%
\end{pgfscope}%
\begin{pgfscope}%
\pgfpathrectangle{\pgfqpoint{6.572727in}{0.473000in}}{\pgfqpoint{4.227273in}{3.311000in}}%
\pgfusepath{clip}%
\pgfsetbuttcap%
\pgfsetroundjoin%
\definecolor{currentfill}{rgb}{0.993248,0.906157,0.143936}%
\pgfsetfillcolor{currentfill}%
\pgfsetfillopacity{0.700000}%
\pgfsetlinewidth{0.000000pt}%
\definecolor{currentstroke}{rgb}{0.000000,0.000000,0.000000}%
\pgfsetstrokecolor{currentstroke}%
\pgfsetstrokeopacity{0.700000}%
\pgfsetdash{}{0pt}%
\pgfpathmoveto{\pgfqpoint{9.680322in}{1.480419in}}%
\pgfpathcurveto{\pgfqpoint{9.685366in}{1.480419in}}{\pgfqpoint{9.690204in}{1.482423in}}{\pgfqpoint{9.693770in}{1.485989in}}%
\pgfpathcurveto{\pgfqpoint{9.697337in}{1.489556in}}{\pgfqpoint{9.699340in}{1.494393in}}{\pgfqpoint{9.699340in}{1.499437in}}%
\pgfpathcurveto{\pgfqpoint{9.699340in}{1.504481in}}{\pgfqpoint{9.697337in}{1.509319in}}{\pgfqpoint{9.693770in}{1.512885in}}%
\pgfpathcurveto{\pgfqpoint{9.690204in}{1.516451in}}{\pgfqpoint{9.685366in}{1.518455in}}{\pgfqpoint{9.680322in}{1.518455in}}%
\pgfpathcurveto{\pgfqpoint{9.675279in}{1.518455in}}{\pgfqpoint{9.670441in}{1.516451in}}{\pgfqpoint{9.666874in}{1.512885in}}%
\pgfpathcurveto{\pgfqpoint{9.663308in}{1.509319in}}{\pgfqpoint{9.661304in}{1.504481in}}{\pgfqpoint{9.661304in}{1.499437in}}%
\pgfpathcurveto{\pgfqpoint{9.661304in}{1.494393in}}{\pgfqpoint{9.663308in}{1.489556in}}{\pgfqpoint{9.666874in}{1.485989in}}%
\pgfpathcurveto{\pgfqpoint{9.670441in}{1.482423in}}{\pgfqpoint{9.675279in}{1.480419in}}{\pgfqpoint{9.680322in}{1.480419in}}%
\pgfpathclose%
\pgfusepath{fill}%
\end{pgfscope}%
\begin{pgfscope}%
\pgfpathrectangle{\pgfqpoint{6.572727in}{0.473000in}}{\pgfqpoint{4.227273in}{3.311000in}}%
\pgfusepath{clip}%
\pgfsetbuttcap%
\pgfsetroundjoin%
\definecolor{currentfill}{rgb}{0.993248,0.906157,0.143936}%
\pgfsetfillcolor{currentfill}%
\pgfsetfillopacity{0.700000}%
\pgfsetlinewidth{0.000000pt}%
\definecolor{currentstroke}{rgb}{0.000000,0.000000,0.000000}%
\pgfsetstrokecolor{currentstroke}%
\pgfsetstrokeopacity{0.700000}%
\pgfsetdash{}{0pt}%
\pgfpathmoveto{\pgfqpoint{9.586439in}{1.201296in}}%
\pgfpathcurveto{\pgfqpoint{9.591483in}{1.201296in}}{\pgfqpoint{9.596321in}{1.203300in}}{\pgfqpoint{9.599887in}{1.206866in}}%
\pgfpathcurveto{\pgfqpoint{9.603454in}{1.210433in}}{\pgfqpoint{9.605458in}{1.215270in}}{\pgfqpoint{9.605458in}{1.220314in}}%
\pgfpathcurveto{\pgfqpoint{9.605458in}{1.225358in}}{\pgfqpoint{9.603454in}{1.230196in}}{\pgfqpoint{9.599887in}{1.233762in}}%
\pgfpathcurveto{\pgfqpoint{9.596321in}{1.237328in}}{\pgfqpoint{9.591483in}{1.239332in}}{\pgfqpoint{9.586439in}{1.239332in}}%
\pgfpathcurveto{\pgfqpoint{9.581396in}{1.239332in}}{\pgfqpoint{9.576558in}{1.237328in}}{\pgfqpoint{9.572992in}{1.233762in}}%
\pgfpathcurveto{\pgfqpoint{9.569425in}{1.230196in}}{\pgfqpoint{9.567421in}{1.225358in}}{\pgfqpoint{9.567421in}{1.220314in}}%
\pgfpathcurveto{\pgfqpoint{9.567421in}{1.215270in}}{\pgfqpoint{9.569425in}{1.210433in}}{\pgfqpoint{9.572992in}{1.206866in}}%
\pgfpathcurveto{\pgfqpoint{9.576558in}{1.203300in}}{\pgfqpoint{9.581396in}{1.201296in}}{\pgfqpoint{9.586439in}{1.201296in}}%
\pgfpathclose%
\pgfusepath{fill}%
\end{pgfscope}%
\begin{pgfscope}%
\pgfpathrectangle{\pgfqpoint{6.572727in}{0.473000in}}{\pgfqpoint{4.227273in}{3.311000in}}%
\pgfusepath{clip}%
\pgfsetbuttcap%
\pgfsetroundjoin%
\definecolor{currentfill}{rgb}{0.127568,0.566949,0.550556}%
\pgfsetfillcolor{currentfill}%
\pgfsetfillopacity{0.700000}%
\pgfsetlinewidth{0.000000pt}%
\definecolor{currentstroke}{rgb}{0.000000,0.000000,0.000000}%
\pgfsetstrokecolor{currentstroke}%
\pgfsetstrokeopacity{0.700000}%
\pgfsetdash{}{0pt}%
\pgfpathmoveto{\pgfqpoint{7.819947in}{3.068736in}}%
\pgfpathcurveto{\pgfqpoint{7.824991in}{3.068736in}}{\pgfqpoint{7.829829in}{3.070740in}}{\pgfqpoint{7.833395in}{3.074306in}}%
\pgfpathcurveto{\pgfqpoint{7.836962in}{3.077873in}}{\pgfqpoint{7.838965in}{3.082711in}}{\pgfqpoint{7.838965in}{3.087754in}}%
\pgfpathcurveto{\pgfqpoint{7.838965in}{3.092798in}}{\pgfqpoint{7.836962in}{3.097636in}}{\pgfqpoint{7.833395in}{3.101202in}}%
\pgfpathcurveto{\pgfqpoint{7.829829in}{3.104768in}}{\pgfqpoint{7.824991in}{3.106772in}}{\pgfqpoint{7.819947in}{3.106772in}}%
\pgfpathcurveto{\pgfqpoint{7.814904in}{3.106772in}}{\pgfqpoint{7.810066in}{3.104768in}}{\pgfqpoint{7.806499in}{3.101202in}}%
\pgfpathcurveto{\pgfqpoint{7.802933in}{3.097636in}}{\pgfqpoint{7.800929in}{3.092798in}}{\pgfqpoint{7.800929in}{3.087754in}}%
\pgfpathcurveto{\pgfqpoint{7.800929in}{3.082711in}}{\pgfqpoint{7.802933in}{3.077873in}}{\pgfqpoint{7.806499in}{3.074306in}}%
\pgfpathcurveto{\pgfqpoint{7.810066in}{3.070740in}}{\pgfqpoint{7.814904in}{3.068736in}}{\pgfqpoint{7.819947in}{3.068736in}}%
\pgfpathclose%
\pgfusepath{fill}%
\end{pgfscope}%
\begin{pgfscope}%
\pgfpathrectangle{\pgfqpoint{6.572727in}{0.473000in}}{\pgfqpoint{4.227273in}{3.311000in}}%
\pgfusepath{clip}%
\pgfsetbuttcap%
\pgfsetroundjoin%
\definecolor{currentfill}{rgb}{0.993248,0.906157,0.143936}%
\pgfsetfillcolor{currentfill}%
\pgfsetfillopacity{0.700000}%
\pgfsetlinewidth{0.000000pt}%
\definecolor{currentstroke}{rgb}{0.000000,0.000000,0.000000}%
\pgfsetstrokecolor{currentstroke}%
\pgfsetstrokeopacity{0.700000}%
\pgfsetdash{}{0pt}%
\pgfpathmoveto{\pgfqpoint{9.253764in}{1.218847in}}%
\pgfpathcurveto{\pgfqpoint{9.258807in}{1.218847in}}{\pgfqpoint{9.263645in}{1.220851in}}{\pgfqpoint{9.267211in}{1.224418in}}%
\pgfpathcurveto{\pgfqpoint{9.270778in}{1.227984in}}{\pgfqpoint{9.272782in}{1.232822in}}{\pgfqpoint{9.272782in}{1.237866in}}%
\pgfpathcurveto{\pgfqpoint{9.272782in}{1.242909in}}{\pgfqpoint{9.270778in}{1.247747in}}{\pgfqpoint{9.267211in}{1.251313in}}%
\pgfpathcurveto{\pgfqpoint{9.263645in}{1.254880in}}{\pgfqpoint{9.258807in}{1.256884in}}{\pgfqpoint{9.253764in}{1.256884in}}%
\pgfpathcurveto{\pgfqpoint{9.248720in}{1.256884in}}{\pgfqpoint{9.243882in}{1.254880in}}{\pgfqpoint{9.240316in}{1.251313in}}%
\pgfpathcurveto{\pgfqpoint{9.236749in}{1.247747in}}{\pgfqpoint{9.234745in}{1.242909in}}{\pgfqpoint{9.234745in}{1.237866in}}%
\pgfpathcurveto{\pgfqpoint{9.234745in}{1.232822in}}{\pgfqpoint{9.236749in}{1.227984in}}{\pgfqpoint{9.240316in}{1.224418in}}%
\pgfpathcurveto{\pgfqpoint{9.243882in}{1.220851in}}{\pgfqpoint{9.248720in}{1.218847in}}{\pgfqpoint{9.253764in}{1.218847in}}%
\pgfpathclose%
\pgfusepath{fill}%
\end{pgfscope}%
\begin{pgfscope}%
\pgfpathrectangle{\pgfqpoint{6.572727in}{0.473000in}}{\pgfqpoint{4.227273in}{3.311000in}}%
\pgfusepath{clip}%
\pgfsetbuttcap%
\pgfsetroundjoin%
\definecolor{currentfill}{rgb}{0.127568,0.566949,0.550556}%
\pgfsetfillcolor{currentfill}%
\pgfsetfillopacity{0.700000}%
\pgfsetlinewidth{0.000000pt}%
\definecolor{currentstroke}{rgb}{0.000000,0.000000,0.000000}%
\pgfsetstrokecolor{currentstroke}%
\pgfsetstrokeopacity{0.700000}%
\pgfsetdash{}{0pt}%
\pgfpathmoveto{\pgfqpoint{7.869275in}{1.620204in}}%
\pgfpathcurveto{\pgfqpoint{7.874319in}{1.620204in}}{\pgfqpoint{7.879156in}{1.622208in}}{\pgfqpoint{7.882723in}{1.625775in}}%
\pgfpathcurveto{\pgfqpoint{7.886289in}{1.629341in}}{\pgfqpoint{7.888293in}{1.634179in}}{\pgfqpoint{7.888293in}{1.639223in}}%
\pgfpathcurveto{\pgfqpoint{7.888293in}{1.644266in}}{\pgfqpoint{7.886289in}{1.649104in}}{\pgfqpoint{7.882723in}{1.652670in}}%
\pgfpathcurveto{\pgfqpoint{7.879156in}{1.656237in}}{\pgfqpoint{7.874319in}{1.658241in}}{\pgfqpoint{7.869275in}{1.658241in}}%
\pgfpathcurveto{\pgfqpoint{7.864231in}{1.658241in}}{\pgfqpoint{7.859393in}{1.656237in}}{\pgfqpoint{7.855827in}{1.652670in}}%
\pgfpathcurveto{\pgfqpoint{7.852261in}{1.649104in}}{\pgfqpoint{7.850257in}{1.644266in}}{\pgfqpoint{7.850257in}{1.639223in}}%
\pgfpathcurveto{\pgfqpoint{7.850257in}{1.634179in}}{\pgfqpoint{7.852261in}{1.629341in}}{\pgfqpoint{7.855827in}{1.625775in}}%
\pgfpathcurveto{\pgfqpoint{7.859393in}{1.622208in}}{\pgfqpoint{7.864231in}{1.620204in}}{\pgfqpoint{7.869275in}{1.620204in}}%
\pgfpathclose%
\pgfusepath{fill}%
\end{pgfscope}%
\begin{pgfscope}%
\pgfpathrectangle{\pgfqpoint{6.572727in}{0.473000in}}{\pgfqpoint{4.227273in}{3.311000in}}%
\pgfusepath{clip}%
\pgfsetbuttcap%
\pgfsetroundjoin%
\definecolor{currentfill}{rgb}{0.127568,0.566949,0.550556}%
\pgfsetfillcolor{currentfill}%
\pgfsetfillopacity{0.700000}%
\pgfsetlinewidth{0.000000pt}%
\definecolor{currentstroke}{rgb}{0.000000,0.000000,0.000000}%
\pgfsetstrokecolor{currentstroke}%
\pgfsetstrokeopacity{0.700000}%
\pgfsetdash{}{0pt}%
\pgfpathmoveto{\pgfqpoint{7.867221in}{1.523232in}}%
\pgfpathcurveto{\pgfqpoint{7.872264in}{1.523232in}}{\pgfqpoint{7.877102in}{1.525235in}}{\pgfqpoint{7.880669in}{1.528802in}}%
\pgfpathcurveto{\pgfqpoint{7.884235in}{1.532368in}}{\pgfqpoint{7.886239in}{1.537206in}}{\pgfqpoint{7.886239in}{1.542250in}}%
\pgfpathcurveto{\pgfqpoint{7.886239in}{1.547293in}}{\pgfqpoint{7.884235in}{1.552131in}}{\pgfqpoint{7.880669in}{1.555698in}}%
\pgfpathcurveto{\pgfqpoint{7.877102in}{1.559264in}}{\pgfqpoint{7.872264in}{1.561268in}}{\pgfqpoint{7.867221in}{1.561268in}}%
\pgfpathcurveto{\pgfqpoint{7.862177in}{1.561268in}}{\pgfqpoint{7.857339in}{1.559264in}}{\pgfqpoint{7.853773in}{1.555698in}}%
\pgfpathcurveto{\pgfqpoint{7.850206in}{1.552131in}}{\pgfqpoint{7.848202in}{1.547293in}}{\pgfqpoint{7.848202in}{1.542250in}}%
\pgfpathcurveto{\pgfqpoint{7.848202in}{1.537206in}}{\pgfqpoint{7.850206in}{1.532368in}}{\pgfqpoint{7.853773in}{1.528802in}}%
\pgfpathcurveto{\pgfqpoint{7.857339in}{1.525235in}}{\pgfqpoint{7.862177in}{1.523232in}}{\pgfqpoint{7.867221in}{1.523232in}}%
\pgfpathclose%
\pgfusepath{fill}%
\end{pgfscope}%
\begin{pgfscope}%
\pgfpathrectangle{\pgfqpoint{6.572727in}{0.473000in}}{\pgfqpoint{4.227273in}{3.311000in}}%
\pgfusepath{clip}%
\pgfsetbuttcap%
\pgfsetroundjoin%
\definecolor{currentfill}{rgb}{0.127568,0.566949,0.550556}%
\pgfsetfillcolor{currentfill}%
\pgfsetfillopacity{0.700000}%
\pgfsetlinewidth{0.000000pt}%
\definecolor{currentstroke}{rgb}{0.000000,0.000000,0.000000}%
\pgfsetstrokecolor{currentstroke}%
\pgfsetstrokeopacity{0.700000}%
\pgfsetdash{}{0pt}%
\pgfpathmoveto{\pgfqpoint{8.437324in}{2.836964in}}%
\pgfpathcurveto{\pgfqpoint{8.442367in}{2.836964in}}{\pgfqpoint{8.447205in}{2.838968in}}{\pgfqpoint{8.450772in}{2.842534in}}%
\pgfpathcurveto{\pgfqpoint{8.454338in}{2.846101in}}{\pgfqpoint{8.456342in}{2.850938in}}{\pgfqpoint{8.456342in}{2.855982in}}%
\pgfpathcurveto{\pgfqpoint{8.456342in}{2.861026in}}{\pgfqpoint{8.454338in}{2.865864in}}{\pgfqpoint{8.450772in}{2.869430in}}%
\pgfpathcurveto{\pgfqpoint{8.447205in}{2.872996in}}{\pgfqpoint{8.442367in}{2.875000in}}{\pgfqpoint{8.437324in}{2.875000in}}%
\pgfpathcurveto{\pgfqpoint{8.432280in}{2.875000in}}{\pgfqpoint{8.427442in}{2.872996in}}{\pgfqpoint{8.423876in}{2.869430in}}%
\pgfpathcurveto{\pgfqpoint{8.420310in}{2.865864in}}{\pgfqpoint{8.418306in}{2.861026in}}{\pgfqpoint{8.418306in}{2.855982in}}%
\pgfpathcurveto{\pgfqpoint{8.418306in}{2.850938in}}{\pgfqpoint{8.420310in}{2.846101in}}{\pgfqpoint{8.423876in}{2.842534in}}%
\pgfpathcurveto{\pgfqpoint{8.427442in}{2.838968in}}{\pgfqpoint{8.432280in}{2.836964in}}{\pgfqpoint{8.437324in}{2.836964in}}%
\pgfpathclose%
\pgfusepath{fill}%
\end{pgfscope}%
\begin{pgfscope}%
\pgfpathrectangle{\pgfqpoint{6.572727in}{0.473000in}}{\pgfqpoint{4.227273in}{3.311000in}}%
\pgfusepath{clip}%
\pgfsetbuttcap%
\pgfsetroundjoin%
\definecolor{currentfill}{rgb}{0.993248,0.906157,0.143936}%
\pgfsetfillcolor{currentfill}%
\pgfsetfillopacity{0.700000}%
\pgfsetlinewidth{0.000000pt}%
\definecolor{currentstroke}{rgb}{0.000000,0.000000,0.000000}%
\pgfsetstrokecolor{currentstroke}%
\pgfsetstrokeopacity{0.700000}%
\pgfsetdash{}{0pt}%
\pgfpathmoveto{\pgfqpoint{9.695518in}{1.585501in}}%
\pgfpathcurveto{\pgfqpoint{9.700562in}{1.585501in}}{\pgfqpoint{9.705399in}{1.587505in}}{\pgfqpoint{9.708966in}{1.591071in}}%
\pgfpathcurveto{\pgfqpoint{9.712532in}{1.594638in}}{\pgfqpoint{9.714536in}{1.599476in}}{\pgfqpoint{9.714536in}{1.604519in}}%
\pgfpathcurveto{\pgfqpoint{9.714536in}{1.609563in}}{\pgfqpoint{9.712532in}{1.614401in}}{\pgfqpoint{9.708966in}{1.617967in}}%
\pgfpathcurveto{\pgfqpoint{9.705399in}{1.621534in}}{\pgfqpoint{9.700562in}{1.623537in}}{\pgfqpoint{9.695518in}{1.623537in}}%
\pgfpathcurveto{\pgfqpoint{9.690474in}{1.623537in}}{\pgfqpoint{9.685636in}{1.621534in}}{\pgfqpoint{9.682070in}{1.617967in}}%
\pgfpathcurveto{\pgfqpoint{9.678504in}{1.614401in}}{\pgfqpoint{9.676500in}{1.609563in}}{\pgfqpoint{9.676500in}{1.604519in}}%
\pgfpathcurveto{\pgfqpoint{9.676500in}{1.599476in}}{\pgfqpoint{9.678504in}{1.594638in}}{\pgfqpoint{9.682070in}{1.591071in}}%
\pgfpathcurveto{\pgfqpoint{9.685636in}{1.587505in}}{\pgfqpoint{9.690474in}{1.585501in}}{\pgfqpoint{9.695518in}{1.585501in}}%
\pgfpathclose%
\pgfusepath{fill}%
\end{pgfscope}%
\begin{pgfscope}%
\pgfpathrectangle{\pgfqpoint{6.572727in}{0.473000in}}{\pgfqpoint{4.227273in}{3.311000in}}%
\pgfusepath{clip}%
\pgfsetbuttcap%
\pgfsetroundjoin%
\definecolor{currentfill}{rgb}{0.993248,0.906157,0.143936}%
\pgfsetfillcolor{currentfill}%
\pgfsetfillopacity{0.700000}%
\pgfsetlinewidth{0.000000pt}%
\definecolor{currentstroke}{rgb}{0.000000,0.000000,0.000000}%
\pgfsetstrokecolor{currentstroke}%
\pgfsetstrokeopacity{0.700000}%
\pgfsetdash{}{0pt}%
\pgfpathmoveto{\pgfqpoint{9.966105in}{1.087262in}}%
\pgfpathcurveto{\pgfqpoint{9.971149in}{1.087262in}}{\pgfqpoint{9.975986in}{1.089266in}}{\pgfqpoint{9.979553in}{1.092832in}}%
\pgfpathcurveto{\pgfqpoint{9.983119in}{1.096399in}}{\pgfqpoint{9.985123in}{1.101237in}}{\pgfqpoint{9.985123in}{1.106280in}}%
\pgfpathcurveto{\pgfqpoint{9.985123in}{1.111324in}}{\pgfqpoint{9.983119in}{1.116162in}}{\pgfqpoint{9.979553in}{1.119728in}}%
\pgfpathcurveto{\pgfqpoint{9.975986in}{1.123295in}}{\pgfqpoint{9.971149in}{1.125298in}}{\pgfqpoint{9.966105in}{1.125298in}}%
\pgfpathcurveto{\pgfqpoint{9.961061in}{1.125298in}}{\pgfqpoint{9.956224in}{1.123295in}}{\pgfqpoint{9.952657in}{1.119728in}}%
\pgfpathcurveto{\pgfqpoint{9.949091in}{1.116162in}}{\pgfqpoint{9.947087in}{1.111324in}}{\pgfqpoint{9.947087in}{1.106280in}}%
\pgfpathcurveto{\pgfqpoint{9.947087in}{1.101237in}}{\pgfqpoint{9.949091in}{1.096399in}}{\pgfqpoint{9.952657in}{1.092832in}}%
\pgfpathcurveto{\pgfqpoint{9.956224in}{1.089266in}}{\pgfqpoint{9.961061in}{1.087262in}}{\pgfqpoint{9.966105in}{1.087262in}}%
\pgfpathclose%
\pgfusepath{fill}%
\end{pgfscope}%
\begin{pgfscope}%
\pgfpathrectangle{\pgfqpoint{6.572727in}{0.473000in}}{\pgfqpoint{4.227273in}{3.311000in}}%
\pgfusepath{clip}%
\pgfsetbuttcap%
\pgfsetroundjoin%
\definecolor{currentfill}{rgb}{0.993248,0.906157,0.143936}%
\pgfsetfillcolor{currentfill}%
\pgfsetfillopacity{0.700000}%
\pgfsetlinewidth{0.000000pt}%
\definecolor{currentstroke}{rgb}{0.000000,0.000000,0.000000}%
\pgfsetstrokecolor{currentstroke}%
\pgfsetstrokeopacity{0.700000}%
\pgfsetdash{}{0pt}%
\pgfpathmoveto{\pgfqpoint{9.973458in}{1.734845in}}%
\pgfpathcurveto{\pgfqpoint{9.978502in}{1.734845in}}{\pgfqpoint{9.983339in}{1.736849in}}{\pgfqpoint{9.986906in}{1.740415in}}%
\pgfpathcurveto{\pgfqpoint{9.990472in}{1.743982in}}{\pgfqpoint{9.992476in}{1.748819in}}{\pgfqpoint{9.992476in}{1.753863in}}%
\pgfpathcurveto{\pgfqpoint{9.992476in}{1.758907in}}{\pgfqpoint{9.990472in}{1.763744in}}{\pgfqpoint{9.986906in}{1.767311in}}%
\pgfpathcurveto{\pgfqpoint{9.983339in}{1.770877in}}{\pgfqpoint{9.978502in}{1.772881in}}{\pgfqpoint{9.973458in}{1.772881in}}%
\pgfpathcurveto{\pgfqpoint{9.968414in}{1.772881in}}{\pgfqpoint{9.963577in}{1.770877in}}{\pgfqpoint{9.960010in}{1.767311in}}%
\pgfpathcurveto{\pgfqpoint{9.956444in}{1.763744in}}{\pgfqpoint{9.954440in}{1.758907in}}{\pgfqpoint{9.954440in}{1.753863in}}%
\pgfpathcurveto{\pgfqpoint{9.954440in}{1.748819in}}{\pgfqpoint{9.956444in}{1.743982in}}{\pgfqpoint{9.960010in}{1.740415in}}%
\pgfpathcurveto{\pgfqpoint{9.963577in}{1.736849in}}{\pgfqpoint{9.968414in}{1.734845in}}{\pgfqpoint{9.973458in}{1.734845in}}%
\pgfpathclose%
\pgfusepath{fill}%
\end{pgfscope}%
\begin{pgfscope}%
\pgfpathrectangle{\pgfqpoint{6.572727in}{0.473000in}}{\pgfqpoint{4.227273in}{3.311000in}}%
\pgfusepath{clip}%
\pgfsetbuttcap%
\pgfsetroundjoin%
\definecolor{currentfill}{rgb}{0.127568,0.566949,0.550556}%
\pgfsetfillcolor{currentfill}%
\pgfsetfillopacity{0.700000}%
\pgfsetlinewidth{0.000000pt}%
\definecolor{currentstroke}{rgb}{0.000000,0.000000,0.000000}%
\pgfsetstrokecolor{currentstroke}%
\pgfsetstrokeopacity{0.700000}%
\pgfsetdash{}{0pt}%
\pgfpathmoveto{\pgfqpoint{7.477651in}{1.355014in}}%
\pgfpathcurveto{\pgfqpoint{7.482694in}{1.355014in}}{\pgfqpoint{7.487532in}{1.357018in}}{\pgfqpoint{7.491098in}{1.360585in}}%
\pgfpathcurveto{\pgfqpoint{7.494665in}{1.364151in}}{\pgfqpoint{7.496669in}{1.368989in}}{\pgfqpoint{7.496669in}{1.374033in}}%
\pgfpathcurveto{\pgfqpoint{7.496669in}{1.379076in}}{\pgfqpoint{7.494665in}{1.383914in}}{\pgfqpoint{7.491098in}{1.387480in}}%
\pgfpathcurveto{\pgfqpoint{7.487532in}{1.391047in}}{\pgfqpoint{7.482694in}{1.393051in}}{\pgfqpoint{7.477651in}{1.393051in}}%
\pgfpathcurveto{\pgfqpoint{7.472607in}{1.393051in}}{\pgfqpoint{7.467769in}{1.391047in}}{\pgfqpoint{7.464203in}{1.387480in}}%
\pgfpathcurveto{\pgfqpoint{7.460636in}{1.383914in}}{\pgfqpoint{7.458632in}{1.379076in}}{\pgfqpoint{7.458632in}{1.374033in}}%
\pgfpathcurveto{\pgfqpoint{7.458632in}{1.368989in}}{\pgfqpoint{7.460636in}{1.364151in}}{\pgfqpoint{7.464203in}{1.360585in}}%
\pgfpathcurveto{\pgfqpoint{7.467769in}{1.357018in}}{\pgfqpoint{7.472607in}{1.355014in}}{\pgfqpoint{7.477651in}{1.355014in}}%
\pgfpathclose%
\pgfusepath{fill}%
\end{pgfscope}%
\begin{pgfscope}%
\pgfpathrectangle{\pgfqpoint{6.572727in}{0.473000in}}{\pgfqpoint{4.227273in}{3.311000in}}%
\pgfusepath{clip}%
\pgfsetbuttcap%
\pgfsetroundjoin%
\definecolor{currentfill}{rgb}{0.127568,0.566949,0.550556}%
\pgfsetfillcolor{currentfill}%
\pgfsetfillopacity{0.700000}%
\pgfsetlinewidth{0.000000pt}%
\definecolor{currentstroke}{rgb}{0.000000,0.000000,0.000000}%
\pgfsetstrokecolor{currentstroke}%
\pgfsetstrokeopacity{0.700000}%
\pgfsetdash{}{0pt}%
\pgfpathmoveto{\pgfqpoint{7.599573in}{2.958245in}}%
\pgfpathcurveto{\pgfqpoint{7.604616in}{2.958245in}}{\pgfqpoint{7.609454in}{2.960248in}}{\pgfqpoint{7.613020in}{2.963815in}}%
\pgfpathcurveto{\pgfqpoint{7.616587in}{2.967381in}}{\pgfqpoint{7.618591in}{2.972219in}}{\pgfqpoint{7.618591in}{2.977263in}}%
\pgfpathcurveto{\pgfqpoint{7.618591in}{2.982306in}}{\pgfqpoint{7.616587in}{2.987144in}}{\pgfqpoint{7.613020in}{2.990711in}}%
\pgfpathcurveto{\pgfqpoint{7.609454in}{2.994277in}}{\pgfqpoint{7.604616in}{2.996281in}}{\pgfqpoint{7.599573in}{2.996281in}}%
\pgfpathcurveto{\pgfqpoint{7.594529in}{2.996281in}}{\pgfqpoint{7.589691in}{2.994277in}}{\pgfqpoint{7.586125in}{2.990711in}}%
\pgfpathcurveto{\pgfqpoint{7.582558in}{2.987144in}}{\pgfqpoint{7.580554in}{2.982306in}}{\pgfqpoint{7.580554in}{2.977263in}}%
\pgfpathcurveto{\pgfqpoint{7.580554in}{2.972219in}}{\pgfqpoint{7.582558in}{2.967381in}}{\pgfqpoint{7.586125in}{2.963815in}}%
\pgfpathcurveto{\pgfqpoint{7.589691in}{2.960248in}}{\pgfqpoint{7.594529in}{2.958245in}}{\pgfqpoint{7.599573in}{2.958245in}}%
\pgfpathclose%
\pgfusepath{fill}%
\end{pgfscope}%
\begin{pgfscope}%
\pgfpathrectangle{\pgfqpoint{6.572727in}{0.473000in}}{\pgfqpoint{4.227273in}{3.311000in}}%
\pgfusepath{clip}%
\pgfsetbuttcap%
\pgfsetroundjoin%
\definecolor{currentfill}{rgb}{0.993248,0.906157,0.143936}%
\pgfsetfillcolor{currentfill}%
\pgfsetfillopacity{0.700000}%
\pgfsetlinewidth{0.000000pt}%
\definecolor{currentstroke}{rgb}{0.000000,0.000000,0.000000}%
\pgfsetstrokecolor{currentstroke}%
\pgfsetstrokeopacity{0.700000}%
\pgfsetdash{}{0pt}%
\pgfpathmoveto{\pgfqpoint{9.608392in}{1.235327in}}%
\pgfpathcurveto{\pgfqpoint{9.613436in}{1.235327in}}{\pgfqpoint{9.618274in}{1.237331in}}{\pgfqpoint{9.621840in}{1.240898in}}%
\pgfpathcurveto{\pgfqpoint{9.625406in}{1.244464in}}{\pgfqpoint{9.627410in}{1.249302in}}{\pgfqpoint{9.627410in}{1.254345in}}%
\pgfpathcurveto{\pgfqpoint{9.627410in}{1.259389in}}{\pgfqpoint{9.625406in}{1.264227in}}{\pgfqpoint{9.621840in}{1.267793in}}%
\pgfpathcurveto{\pgfqpoint{9.618274in}{1.271360in}}{\pgfqpoint{9.613436in}{1.273364in}}{\pgfqpoint{9.608392in}{1.273364in}}%
\pgfpathcurveto{\pgfqpoint{9.603348in}{1.273364in}}{\pgfqpoint{9.598511in}{1.271360in}}{\pgfqpoint{9.594944in}{1.267793in}}%
\pgfpathcurveto{\pgfqpoint{9.591378in}{1.264227in}}{\pgfqpoint{9.589374in}{1.259389in}}{\pgfqpoint{9.589374in}{1.254345in}}%
\pgfpathcurveto{\pgfqpoint{9.589374in}{1.249302in}}{\pgfqpoint{9.591378in}{1.244464in}}{\pgfqpoint{9.594944in}{1.240898in}}%
\pgfpathcurveto{\pgfqpoint{9.598511in}{1.237331in}}{\pgfqpoint{9.603348in}{1.235327in}}{\pgfqpoint{9.608392in}{1.235327in}}%
\pgfpathclose%
\pgfusepath{fill}%
\end{pgfscope}%
\begin{pgfscope}%
\pgfpathrectangle{\pgfqpoint{6.572727in}{0.473000in}}{\pgfqpoint{4.227273in}{3.311000in}}%
\pgfusepath{clip}%
\pgfsetbuttcap%
\pgfsetroundjoin%
\definecolor{currentfill}{rgb}{0.993248,0.906157,0.143936}%
\pgfsetfillcolor{currentfill}%
\pgfsetfillopacity{0.700000}%
\pgfsetlinewidth{0.000000pt}%
\definecolor{currentstroke}{rgb}{0.000000,0.000000,0.000000}%
\pgfsetstrokecolor{currentstroke}%
\pgfsetstrokeopacity{0.700000}%
\pgfsetdash{}{0pt}%
\pgfpathmoveto{\pgfqpoint{9.267081in}{1.712404in}}%
\pgfpathcurveto{\pgfqpoint{9.272125in}{1.712404in}}{\pgfqpoint{9.276962in}{1.714408in}}{\pgfqpoint{9.280529in}{1.717974in}}%
\pgfpathcurveto{\pgfqpoint{9.284095in}{1.721541in}}{\pgfqpoint{9.286099in}{1.726378in}}{\pgfqpoint{9.286099in}{1.731422in}}%
\pgfpathcurveto{\pgfqpoint{9.286099in}{1.736466in}}{\pgfqpoint{9.284095in}{1.741304in}}{\pgfqpoint{9.280529in}{1.744870in}}%
\pgfpathcurveto{\pgfqpoint{9.276962in}{1.748436in}}{\pgfqpoint{9.272125in}{1.750440in}}{\pgfqpoint{9.267081in}{1.750440in}}%
\pgfpathcurveto{\pgfqpoint{9.262037in}{1.750440in}}{\pgfqpoint{9.257199in}{1.748436in}}{\pgfqpoint{9.253633in}{1.744870in}}%
\pgfpathcurveto{\pgfqpoint{9.250067in}{1.741304in}}{\pgfqpoint{9.248063in}{1.736466in}}{\pgfqpoint{9.248063in}{1.731422in}}%
\pgfpathcurveto{\pgfqpoint{9.248063in}{1.726378in}}{\pgfqpoint{9.250067in}{1.721541in}}{\pgfqpoint{9.253633in}{1.717974in}}%
\pgfpathcurveto{\pgfqpoint{9.257199in}{1.714408in}}{\pgfqpoint{9.262037in}{1.712404in}}{\pgfqpoint{9.267081in}{1.712404in}}%
\pgfpathclose%
\pgfusepath{fill}%
\end{pgfscope}%
\begin{pgfscope}%
\pgfpathrectangle{\pgfqpoint{6.572727in}{0.473000in}}{\pgfqpoint{4.227273in}{3.311000in}}%
\pgfusepath{clip}%
\pgfsetbuttcap%
\pgfsetroundjoin%
\definecolor{currentfill}{rgb}{0.993248,0.906157,0.143936}%
\pgfsetfillcolor{currentfill}%
\pgfsetfillopacity{0.700000}%
\pgfsetlinewidth{0.000000pt}%
\definecolor{currentstroke}{rgb}{0.000000,0.000000,0.000000}%
\pgfsetstrokecolor{currentstroke}%
\pgfsetstrokeopacity{0.700000}%
\pgfsetdash{}{0pt}%
\pgfpathmoveto{\pgfqpoint{9.427330in}{1.232876in}}%
\pgfpathcurveto{\pgfqpoint{9.432374in}{1.232876in}}{\pgfqpoint{9.437212in}{1.234880in}}{\pgfqpoint{9.440778in}{1.238447in}}%
\pgfpathcurveto{\pgfqpoint{9.444344in}{1.242013in}}{\pgfqpoint{9.446348in}{1.246851in}}{\pgfqpoint{9.446348in}{1.251895in}}%
\pgfpathcurveto{\pgfqpoint{9.446348in}{1.256938in}}{\pgfqpoint{9.444344in}{1.261776in}}{\pgfqpoint{9.440778in}{1.265342in}}%
\pgfpathcurveto{\pgfqpoint{9.437212in}{1.268909in}}{\pgfqpoint{9.432374in}{1.270913in}}{\pgfqpoint{9.427330in}{1.270913in}}%
\pgfpathcurveto{\pgfqpoint{9.422287in}{1.270913in}}{\pgfqpoint{9.417449in}{1.268909in}}{\pgfqpoint{9.413882in}{1.265342in}}%
\pgfpathcurveto{\pgfqpoint{9.410316in}{1.261776in}}{\pgfqpoint{9.408312in}{1.256938in}}{\pgfqpoint{9.408312in}{1.251895in}}%
\pgfpathcurveto{\pgfqpoint{9.408312in}{1.246851in}}{\pgfqpoint{9.410316in}{1.242013in}}{\pgfqpoint{9.413882in}{1.238447in}}%
\pgfpathcurveto{\pgfqpoint{9.417449in}{1.234880in}}{\pgfqpoint{9.422287in}{1.232876in}}{\pgfqpoint{9.427330in}{1.232876in}}%
\pgfpathclose%
\pgfusepath{fill}%
\end{pgfscope}%
\begin{pgfscope}%
\pgfpathrectangle{\pgfqpoint{6.572727in}{0.473000in}}{\pgfqpoint{4.227273in}{3.311000in}}%
\pgfusepath{clip}%
\pgfsetbuttcap%
\pgfsetroundjoin%
\definecolor{currentfill}{rgb}{0.993248,0.906157,0.143936}%
\pgfsetfillcolor{currentfill}%
\pgfsetfillopacity{0.700000}%
\pgfsetlinewidth{0.000000pt}%
\definecolor{currentstroke}{rgb}{0.000000,0.000000,0.000000}%
\pgfsetstrokecolor{currentstroke}%
\pgfsetstrokeopacity{0.700000}%
\pgfsetdash{}{0pt}%
\pgfpathmoveto{\pgfqpoint{9.562831in}{1.523874in}}%
\pgfpathcurveto{\pgfqpoint{9.567874in}{1.523874in}}{\pgfqpoint{9.572712in}{1.525878in}}{\pgfqpoint{9.576279in}{1.529444in}}%
\pgfpathcurveto{\pgfqpoint{9.579845in}{1.533011in}}{\pgfqpoint{9.581849in}{1.537848in}}{\pgfqpoint{9.581849in}{1.542892in}}%
\pgfpathcurveto{\pgfqpoint{9.581849in}{1.547936in}}{\pgfqpoint{9.579845in}{1.552774in}}{\pgfqpoint{9.576279in}{1.556340in}}%
\pgfpathcurveto{\pgfqpoint{9.572712in}{1.559906in}}{\pgfqpoint{9.567874in}{1.561910in}}{\pgfqpoint{9.562831in}{1.561910in}}%
\pgfpathcurveto{\pgfqpoint{9.557787in}{1.561910in}}{\pgfqpoint{9.552949in}{1.559906in}}{\pgfqpoint{9.549383in}{1.556340in}}%
\pgfpathcurveto{\pgfqpoint{9.545816in}{1.552774in}}{\pgfqpoint{9.543813in}{1.547936in}}{\pgfqpoint{9.543813in}{1.542892in}}%
\pgfpathcurveto{\pgfqpoint{9.543813in}{1.537848in}}{\pgfqpoint{9.545816in}{1.533011in}}{\pgfqpoint{9.549383in}{1.529444in}}%
\pgfpathcurveto{\pgfqpoint{9.552949in}{1.525878in}}{\pgfqpoint{9.557787in}{1.523874in}}{\pgfqpoint{9.562831in}{1.523874in}}%
\pgfpathclose%
\pgfusepath{fill}%
\end{pgfscope}%
\begin{pgfscope}%
\pgfpathrectangle{\pgfqpoint{6.572727in}{0.473000in}}{\pgfqpoint{4.227273in}{3.311000in}}%
\pgfusepath{clip}%
\pgfsetbuttcap%
\pgfsetroundjoin%
\definecolor{currentfill}{rgb}{0.993248,0.906157,0.143936}%
\pgfsetfillcolor{currentfill}%
\pgfsetfillopacity{0.700000}%
\pgfsetlinewidth{0.000000pt}%
\definecolor{currentstroke}{rgb}{0.000000,0.000000,0.000000}%
\pgfsetstrokecolor{currentstroke}%
\pgfsetstrokeopacity{0.700000}%
\pgfsetdash{}{0pt}%
\pgfpathmoveto{\pgfqpoint{9.380313in}{1.624064in}}%
\pgfpathcurveto{\pgfqpoint{9.385357in}{1.624064in}}{\pgfqpoint{9.390194in}{1.626068in}}{\pgfqpoint{9.393761in}{1.629634in}}%
\pgfpathcurveto{\pgfqpoint{9.397327in}{1.633201in}}{\pgfqpoint{9.399331in}{1.638039in}}{\pgfqpoint{9.399331in}{1.643082in}}%
\pgfpathcurveto{\pgfqpoint{9.399331in}{1.648126in}}{\pgfqpoint{9.397327in}{1.652964in}}{\pgfqpoint{9.393761in}{1.656530in}}%
\pgfpathcurveto{\pgfqpoint{9.390194in}{1.660097in}}{\pgfqpoint{9.385357in}{1.662100in}}{\pgfqpoint{9.380313in}{1.662100in}}%
\pgfpathcurveto{\pgfqpoint{9.375269in}{1.662100in}}{\pgfqpoint{9.370431in}{1.660097in}}{\pgfqpoint{9.366865in}{1.656530in}}%
\pgfpathcurveto{\pgfqpoint{9.363299in}{1.652964in}}{\pgfqpoint{9.361295in}{1.648126in}}{\pgfqpoint{9.361295in}{1.643082in}}%
\pgfpathcurveto{\pgfqpoint{9.361295in}{1.638039in}}{\pgfqpoint{9.363299in}{1.633201in}}{\pgfqpoint{9.366865in}{1.629634in}}%
\pgfpathcurveto{\pgfqpoint{9.370431in}{1.626068in}}{\pgfqpoint{9.375269in}{1.624064in}}{\pgfqpoint{9.380313in}{1.624064in}}%
\pgfpathclose%
\pgfusepath{fill}%
\end{pgfscope}%
\begin{pgfscope}%
\pgfpathrectangle{\pgfqpoint{6.572727in}{0.473000in}}{\pgfqpoint{4.227273in}{3.311000in}}%
\pgfusepath{clip}%
\pgfsetbuttcap%
\pgfsetroundjoin%
\definecolor{currentfill}{rgb}{0.127568,0.566949,0.550556}%
\pgfsetfillcolor{currentfill}%
\pgfsetfillopacity{0.700000}%
\pgfsetlinewidth{0.000000pt}%
\definecolor{currentstroke}{rgb}{0.000000,0.000000,0.000000}%
\pgfsetstrokecolor{currentstroke}%
\pgfsetstrokeopacity{0.700000}%
\pgfsetdash{}{0pt}%
\pgfpathmoveto{\pgfqpoint{8.069414in}{0.996901in}}%
\pgfpathcurveto{\pgfqpoint{8.074458in}{0.996901in}}{\pgfqpoint{8.079296in}{0.998905in}}{\pgfqpoint{8.082862in}{1.002471in}}%
\pgfpathcurveto{\pgfqpoint{8.086429in}{1.006038in}}{\pgfqpoint{8.088432in}{1.010875in}}{\pgfqpoint{8.088432in}{1.015919in}}%
\pgfpathcurveto{\pgfqpoint{8.088432in}{1.020963in}}{\pgfqpoint{8.086429in}{1.025801in}}{\pgfqpoint{8.082862in}{1.029367in}}%
\pgfpathcurveto{\pgfqpoint{8.079296in}{1.032933in}}{\pgfqpoint{8.074458in}{1.034937in}}{\pgfqpoint{8.069414in}{1.034937in}}%
\pgfpathcurveto{\pgfqpoint{8.064371in}{1.034937in}}{\pgfqpoint{8.059533in}{1.032933in}}{\pgfqpoint{8.055966in}{1.029367in}}%
\pgfpathcurveto{\pgfqpoint{8.052400in}{1.025801in}}{\pgfqpoint{8.050396in}{1.020963in}}{\pgfqpoint{8.050396in}{1.015919in}}%
\pgfpathcurveto{\pgfqpoint{8.050396in}{1.010875in}}{\pgfqpoint{8.052400in}{1.006038in}}{\pgfqpoint{8.055966in}{1.002471in}}%
\pgfpathcurveto{\pgfqpoint{8.059533in}{0.998905in}}{\pgfqpoint{8.064371in}{0.996901in}}{\pgfqpoint{8.069414in}{0.996901in}}%
\pgfpathclose%
\pgfusepath{fill}%
\end{pgfscope}%
\begin{pgfscope}%
\pgfpathrectangle{\pgfqpoint{6.572727in}{0.473000in}}{\pgfqpoint{4.227273in}{3.311000in}}%
\pgfusepath{clip}%
\pgfsetbuttcap%
\pgfsetroundjoin%
\definecolor{currentfill}{rgb}{0.127568,0.566949,0.550556}%
\pgfsetfillcolor{currentfill}%
\pgfsetfillopacity{0.700000}%
\pgfsetlinewidth{0.000000pt}%
\definecolor{currentstroke}{rgb}{0.000000,0.000000,0.000000}%
\pgfsetstrokecolor{currentstroke}%
\pgfsetstrokeopacity{0.700000}%
\pgfsetdash{}{0pt}%
\pgfpathmoveto{\pgfqpoint{8.367846in}{1.769579in}}%
\pgfpathcurveto{\pgfqpoint{8.372890in}{1.769579in}}{\pgfqpoint{8.377728in}{1.771583in}}{\pgfqpoint{8.381294in}{1.775150in}}%
\pgfpathcurveto{\pgfqpoint{8.384860in}{1.778716in}}{\pgfqpoint{8.386864in}{1.783554in}}{\pgfqpoint{8.386864in}{1.788598in}}%
\pgfpathcurveto{\pgfqpoint{8.386864in}{1.793641in}}{\pgfqpoint{8.384860in}{1.798479in}}{\pgfqpoint{8.381294in}{1.802045in}}%
\pgfpathcurveto{\pgfqpoint{8.377728in}{1.805612in}}{\pgfqpoint{8.372890in}{1.807616in}}{\pgfqpoint{8.367846in}{1.807616in}}%
\pgfpathcurveto{\pgfqpoint{8.362802in}{1.807616in}}{\pgfqpoint{8.357965in}{1.805612in}}{\pgfqpoint{8.354398in}{1.802045in}}%
\pgfpathcurveto{\pgfqpoint{8.350832in}{1.798479in}}{\pgfqpoint{8.348828in}{1.793641in}}{\pgfqpoint{8.348828in}{1.788598in}}%
\pgfpathcurveto{\pgfqpoint{8.348828in}{1.783554in}}{\pgfqpoint{8.350832in}{1.778716in}}{\pgfqpoint{8.354398in}{1.775150in}}%
\pgfpathcurveto{\pgfqpoint{8.357965in}{1.771583in}}{\pgfqpoint{8.362802in}{1.769579in}}{\pgfqpoint{8.367846in}{1.769579in}}%
\pgfpathclose%
\pgfusepath{fill}%
\end{pgfscope}%
\begin{pgfscope}%
\pgfpathrectangle{\pgfqpoint{6.572727in}{0.473000in}}{\pgfqpoint{4.227273in}{3.311000in}}%
\pgfusepath{clip}%
\pgfsetbuttcap%
\pgfsetroundjoin%
\definecolor{currentfill}{rgb}{0.127568,0.566949,0.550556}%
\pgfsetfillcolor{currentfill}%
\pgfsetfillopacity{0.700000}%
\pgfsetlinewidth{0.000000pt}%
\definecolor{currentstroke}{rgb}{0.000000,0.000000,0.000000}%
\pgfsetstrokecolor{currentstroke}%
\pgfsetstrokeopacity{0.700000}%
\pgfsetdash{}{0pt}%
\pgfpathmoveto{\pgfqpoint{8.586375in}{3.002168in}}%
\pgfpathcurveto{\pgfqpoint{8.591418in}{3.002168in}}{\pgfqpoint{8.596256in}{3.004172in}}{\pgfqpoint{8.599823in}{3.007739in}}%
\pgfpathcurveto{\pgfqpoint{8.603389in}{3.011305in}}{\pgfqpoint{8.605393in}{3.016143in}}{\pgfqpoint{8.605393in}{3.021187in}}%
\pgfpathcurveto{\pgfqpoint{8.605393in}{3.026230in}}{\pgfqpoint{8.603389in}{3.031068in}}{\pgfqpoint{8.599823in}{3.034634in}}%
\pgfpathcurveto{\pgfqpoint{8.596256in}{3.038201in}}{\pgfqpoint{8.591418in}{3.040205in}}{\pgfqpoint{8.586375in}{3.040205in}}%
\pgfpathcurveto{\pgfqpoint{8.581331in}{3.040205in}}{\pgfqpoint{8.576493in}{3.038201in}}{\pgfqpoint{8.572927in}{3.034634in}}%
\pgfpathcurveto{\pgfqpoint{8.569361in}{3.031068in}}{\pgfqpoint{8.567357in}{3.026230in}}{\pgfqpoint{8.567357in}{3.021187in}}%
\pgfpathcurveto{\pgfqpoint{8.567357in}{3.016143in}}{\pgfqpoint{8.569361in}{3.011305in}}{\pgfqpoint{8.572927in}{3.007739in}}%
\pgfpathcurveto{\pgfqpoint{8.576493in}{3.004172in}}{\pgfqpoint{8.581331in}{3.002168in}}{\pgfqpoint{8.586375in}{3.002168in}}%
\pgfpathclose%
\pgfusepath{fill}%
\end{pgfscope}%
\begin{pgfscope}%
\pgfpathrectangle{\pgfqpoint{6.572727in}{0.473000in}}{\pgfqpoint{4.227273in}{3.311000in}}%
\pgfusepath{clip}%
\pgfsetbuttcap%
\pgfsetroundjoin%
\definecolor{currentfill}{rgb}{0.127568,0.566949,0.550556}%
\pgfsetfillcolor{currentfill}%
\pgfsetfillopacity{0.700000}%
\pgfsetlinewidth{0.000000pt}%
\definecolor{currentstroke}{rgb}{0.000000,0.000000,0.000000}%
\pgfsetstrokecolor{currentstroke}%
\pgfsetstrokeopacity{0.700000}%
\pgfsetdash{}{0pt}%
\pgfpathmoveto{\pgfqpoint{7.440425in}{1.893048in}}%
\pgfpathcurveto{\pgfqpoint{7.445468in}{1.893048in}}{\pgfqpoint{7.450306in}{1.895052in}}{\pgfqpoint{7.453872in}{1.898619in}}%
\pgfpathcurveto{\pgfqpoint{7.457439in}{1.902185in}}{\pgfqpoint{7.459443in}{1.907023in}}{\pgfqpoint{7.459443in}{1.912066in}}%
\pgfpathcurveto{\pgfqpoint{7.459443in}{1.917110in}}{\pgfqpoint{7.457439in}{1.921948in}}{\pgfqpoint{7.453872in}{1.925514in}}%
\pgfpathcurveto{\pgfqpoint{7.450306in}{1.929081in}}{\pgfqpoint{7.445468in}{1.931085in}}{\pgfqpoint{7.440425in}{1.931085in}}%
\pgfpathcurveto{\pgfqpoint{7.435381in}{1.931085in}}{\pgfqpoint{7.430543in}{1.929081in}}{\pgfqpoint{7.426977in}{1.925514in}}%
\pgfpathcurveto{\pgfqpoint{7.423410in}{1.921948in}}{\pgfqpoint{7.421406in}{1.917110in}}{\pgfqpoint{7.421406in}{1.912066in}}%
\pgfpathcurveto{\pgfqpoint{7.421406in}{1.907023in}}{\pgfqpoint{7.423410in}{1.902185in}}{\pgfqpoint{7.426977in}{1.898619in}}%
\pgfpathcurveto{\pgfqpoint{7.430543in}{1.895052in}}{\pgfqpoint{7.435381in}{1.893048in}}{\pgfqpoint{7.440425in}{1.893048in}}%
\pgfpathclose%
\pgfusepath{fill}%
\end{pgfscope}%
\begin{pgfscope}%
\pgfpathrectangle{\pgfqpoint{6.572727in}{0.473000in}}{\pgfqpoint{4.227273in}{3.311000in}}%
\pgfusepath{clip}%
\pgfsetbuttcap%
\pgfsetroundjoin%
\definecolor{currentfill}{rgb}{0.127568,0.566949,0.550556}%
\pgfsetfillcolor{currentfill}%
\pgfsetfillopacity{0.700000}%
\pgfsetlinewidth{0.000000pt}%
\definecolor{currentstroke}{rgb}{0.000000,0.000000,0.000000}%
\pgfsetstrokecolor{currentstroke}%
\pgfsetstrokeopacity{0.700000}%
\pgfsetdash{}{0pt}%
\pgfpathmoveto{\pgfqpoint{7.878370in}{1.416294in}}%
\pgfpathcurveto{\pgfqpoint{7.883413in}{1.416294in}}{\pgfqpoint{7.888251in}{1.418298in}}{\pgfqpoint{7.891818in}{1.421865in}}%
\pgfpathcurveto{\pgfqpoint{7.895384in}{1.425431in}}{\pgfqpoint{7.897388in}{1.430269in}}{\pgfqpoint{7.897388in}{1.435313in}}%
\pgfpathcurveto{\pgfqpoint{7.897388in}{1.440356in}}{\pgfqpoint{7.895384in}{1.445194in}}{\pgfqpoint{7.891818in}{1.448760in}}%
\pgfpathcurveto{\pgfqpoint{7.888251in}{1.452327in}}{\pgfqpoint{7.883413in}{1.454331in}}{\pgfqpoint{7.878370in}{1.454331in}}%
\pgfpathcurveto{\pgfqpoint{7.873326in}{1.454331in}}{\pgfqpoint{7.868488in}{1.452327in}}{\pgfqpoint{7.864922in}{1.448760in}}%
\pgfpathcurveto{\pgfqpoint{7.861356in}{1.445194in}}{\pgfqpoint{7.859352in}{1.440356in}}{\pgfqpoint{7.859352in}{1.435313in}}%
\pgfpathcurveto{\pgfqpoint{7.859352in}{1.430269in}}{\pgfqpoint{7.861356in}{1.425431in}}{\pgfqpoint{7.864922in}{1.421865in}}%
\pgfpathcurveto{\pgfqpoint{7.868488in}{1.418298in}}{\pgfqpoint{7.873326in}{1.416294in}}{\pgfqpoint{7.878370in}{1.416294in}}%
\pgfpathclose%
\pgfusepath{fill}%
\end{pgfscope}%
\begin{pgfscope}%
\pgfpathrectangle{\pgfqpoint{6.572727in}{0.473000in}}{\pgfqpoint{4.227273in}{3.311000in}}%
\pgfusepath{clip}%
\pgfsetbuttcap%
\pgfsetroundjoin%
\definecolor{currentfill}{rgb}{0.127568,0.566949,0.550556}%
\pgfsetfillcolor{currentfill}%
\pgfsetfillopacity{0.700000}%
\pgfsetlinewidth{0.000000pt}%
\definecolor{currentstroke}{rgb}{0.000000,0.000000,0.000000}%
\pgfsetstrokecolor{currentstroke}%
\pgfsetstrokeopacity{0.700000}%
\pgfsetdash{}{0pt}%
\pgfpathmoveto{\pgfqpoint{8.461573in}{2.685651in}}%
\pgfpathcurveto{\pgfqpoint{8.466616in}{2.685651in}}{\pgfqpoint{8.471454in}{2.687655in}}{\pgfqpoint{8.475021in}{2.691221in}}%
\pgfpathcurveto{\pgfqpoint{8.478587in}{2.694788in}}{\pgfqpoint{8.480591in}{2.699625in}}{\pgfqpoint{8.480591in}{2.704669in}}%
\pgfpathcurveto{\pgfqpoint{8.480591in}{2.709713in}}{\pgfqpoint{8.478587in}{2.714550in}}{\pgfqpoint{8.475021in}{2.718117in}}%
\pgfpathcurveto{\pgfqpoint{8.471454in}{2.721683in}}{\pgfqpoint{8.466616in}{2.723687in}}{\pgfqpoint{8.461573in}{2.723687in}}%
\pgfpathcurveto{\pgfqpoint{8.456529in}{2.723687in}}{\pgfqpoint{8.451691in}{2.721683in}}{\pgfqpoint{8.448125in}{2.718117in}}%
\pgfpathcurveto{\pgfqpoint{8.444558in}{2.714550in}}{\pgfqpoint{8.442555in}{2.709713in}}{\pgfqpoint{8.442555in}{2.704669in}}%
\pgfpathcurveto{\pgfqpoint{8.442555in}{2.699625in}}{\pgfqpoint{8.444558in}{2.694788in}}{\pgfqpoint{8.448125in}{2.691221in}}%
\pgfpathcurveto{\pgfqpoint{8.451691in}{2.687655in}}{\pgfqpoint{8.456529in}{2.685651in}}{\pgfqpoint{8.461573in}{2.685651in}}%
\pgfpathclose%
\pgfusepath{fill}%
\end{pgfscope}%
\begin{pgfscope}%
\pgfpathrectangle{\pgfqpoint{6.572727in}{0.473000in}}{\pgfqpoint{4.227273in}{3.311000in}}%
\pgfusepath{clip}%
\pgfsetbuttcap%
\pgfsetroundjoin%
\definecolor{currentfill}{rgb}{0.127568,0.566949,0.550556}%
\pgfsetfillcolor{currentfill}%
\pgfsetfillopacity{0.700000}%
\pgfsetlinewidth{0.000000pt}%
\definecolor{currentstroke}{rgb}{0.000000,0.000000,0.000000}%
\pgfsetstrokecolor{currentstroke}%
\pgfsetstrokeopacity{0.700000}%
\pgfsetdash{}{0pt}%
\pgfpathmoveto{\pgfqpoint{8.746234in}{2.576723in}}%
\pgfpathcurveto{\pgfqpoint{8.751278in}{2.576723in}}{\pgfqpoint{8.756115in}{2.578727in}}{\pgfqpoint{8.759682in}{2.582293in}}%
\pgfpathcurveto{\pgfqpoint{8.763248in}{2.585859in}}{\pgfqpoint{8.765252in}{2.590697in}}{\pgfqpoint{8.765252in}{2.595741in}}%
\pgfpathcurveto{\pgfqpoint{8.765252in}{2.600785in}}{\pgfqpoint{8.763248in}{2.605622in}}{\pgfqpoint{8.759682in}{2.609189in}}%
\pgfpathcurveto{\pgfqpoint{8.756115in}{2.612755in}}{\pgfqpoint{8.751278in}{2.614759in}}{\pgfqpoint{8.746234in}{2.614759in}}%
\pgfpathcurveto{\pgfqpoint{8.741190in}{2.614759in}}{\pgfqpoint{8.736353in}{2.612755in}}{\pgfqpoint{8.732786in}{2.609189in}}%
\pgfpathcurveto{\pgfqpoint{8.729220in}{2.605622in}}{\pgfqpoint{8.727216in}{2.600785in}}{\pgfqpoint{8.727216in}{2.595741in}}%
\pgfpathcurveto{\pgfqpoint{8.727216in}{2.590697in}}{\pgfqpoint{8.729220in}{2.585859in}}{\pgfqpoint{8.732786in}{2.582293in}}%
\pgfpathcurveto{\pgfqpoint{8.736353in}{2.578727in}}{\pgfqpoint{8.741190in}{2.576723in}}{\pgfqpoint{8.746234in}{2.576723in}}%
\pgfpathclose%
\pgfusepath{fill}%
\end{pgfscope}%
\begin{pgfscope}%
\pgfpathrectangle{\pgfqpoint{6.572727in}{0.473000in}}{\pgfqpoint{4.227273in}{3.311000in}}%
\pgfusepath{clip}%
\pgfsetbuttcap%
\pgfsetroundjoin%
\definecolor{currentfill}{rgb}{0.127568,0.566949,0.550556}%
\pgfsetfillcolor{currentfill}%
\pgfsetfillopacity{0.700000}%
\pgfsetlinewidth{0.000000pt}%
\definecolor{currentstroke}{rgb}{0.000000,0.000000,0.000000}%
\pgfsetstrokecolor{currentstroke}%
\pgfsetstrokeopacity{0.700000}%
\pgfsetdash{}{0pt}%
\pgfpathmoveto{\pgfqpoint{7.289971in}{1.976005in}}%
\pgfpathcurveto{\pgfqpoint{7.295014in}{1.976005in}}{\pgfqpoint{7.299852in}{1.978009in}}{\pgfqpoint{7.303419in}{1.981576in}}%
\pgfpathcurveto{\pgfqpoint{7.306985in}{1.985142in}}{\pgfqpoint{7.308989in}{1.989980in}}{\pgfqpoint{7.308989in}{1.995024in}}%
\pgfpathcurveto{\pgfqpoint{7.308989in}{2.000067in}}{\pgfqpoint{7.306985in}{2.004905in}}{\pgfqpoint{7.303419in}{2.008471in}}%
\pgfpathcurveto{\pgfqpoint{7.299852in}{2.012038in}}{\pgfqpoint{7.295014in}{2.014042in}}{\pgfqpoint{7.289971in}{2.014042in}}%
\pgfpathcurveto{\pgfqpoint{7.284927in}{2.014042in}}{\pgfqpoint{7.280089in}{2.012038in}}{\pgfqpoint{7.276523in}{2.008471in}}%
\pgfpathcurveto{\pgfqpoint{7.272957in}{2.004905in}}{\pgfqpoint{7.270953in}{2.000067in}}{\pgfqpoint{7.270953in}{1.995024in}}%
\pgfpathcurveto{\pgfqpoint{7.270953in}{1.989980in}}{\pgfqpoint{7.272957in}{1.985142in}}{\pgfqpoint{7.276523in}{1.981576in}}%
\pgfpathcurveto{\pgfqpoint{7.280089in}{1.978009in}}{\pgfqpoint{7.284927in}{1.976005in}}{\pgfqpoint{7.289971in}{1.976005in}}%
\pgfpathclose%
\pgfusepath{fill}%
\end{pgfscope}%
\begin{pgfscope}%
\pgfpathrectangle{\pgfqpoint{6.572727in}{0.473000in}}{\pgfqpoint{4.227273in}{3.311000in}}%
\pgfusepath{clip}%
\pgfsetbuttcap%
\pgfsetroundjoin%
\definecolor{currentfill}{rgb}{0.127568,0.566949,0.550556}%
\pgfsetfillcolor{currentfill}%
\pgfsetfillopacity{0.700000}%
\pgfsetlinewidth{0.000000pt}%
\definecolor{currentstroke}{rgb}{0.000000,0.000000,0.000000}%
\pgfsetstrokecolor{currentstroke}%
\pgfsetstrokeopacity{0.700000}%
\pgfsetdash{}{0pt}%
\pgfpathmoveto{\pgfqpoint{7.497976in}{1.761935in}}%
\pgfpathcurveto{\pgfqpoint{7.503019in}{1.761935in}}{\pgfqpoint{7.507857in}{1.763939in}}{\pgfqpoint{7.511424in}{1.767505in}}%
\pgfpathcurveto{\pgfqpoint{7.514990in}{1.771072in}}{\pgfqpoint{7.516994in}{1.775910in}}{\pgfqpoint{7.516994in}{1.780953in}}%
\pgfpathcurveto{\pgfqpoint{7.516994in}{1.785997in}}{\pgfqpoint{7.514990in}{1.790835in}}{\pgfqpoint{7.511424in}{1.794401in}}%
\pgfpathcurveto{\pgfqpoint{7.507857in}{1.797968in}}{\pgfqpoint{7.503019in}{1.799971in}}{\pgfqpoint{7.497976in}{1.799971in}}%
\pgfpathcurveto{\pgfqpoint{7.492932in}{1.799971in}}{\pgfqpoint{7.488094in}{1.797968in}}{\pgfqpoint{7.484528in}{1.794401in}}%
\pgfpathcurveto{\pgfqpoint{7.480961in}{1.790835in}}{\pgfqpoint{7.478958in}{1.785997in}}{\pgfqpoint{7.478958in}{1.780953in}}%
\pgfpathcurveto{\pgfqpoint{7.478958in}{1.775910in}}{\pgfqpoint{7.480961in}{1.771072in}}{\pgfqpoint{7.484528in}{1.767505in}}%
\pgfpathcurveto{\pgfqpoint{7.488094in}{1.763939in}}{\pgfqpoint{7.492932in}{1.761935in}}{\pgfqpoint{7.497976in}{1.761935in}}%
\pgfpathclose%
\pgfusepath{fill}%
\end{pgfscope}%
\begin{pgfscope}%
\pgfpathrectangle{\pgfqpoint{6.572727in}{0.473000in}}{\pgfqpoint{4.227273in}{3.311000in}}%
\pgfusepath{clip}%
\pgfsetbuttcap%
\pgfsetroundjoin%
\definecolor{currentfill}{rgb}{0.127568,0.566949,0.550556}%
\pgfsetfillcolor{currentfill}%
\pgfsetfillopacity{0.700000}%
\pgfsetlinewidth{0.000000pt}%
\definecolor{currentstroke}{rgb}{0.000000,0.000000,0.000000}%
\pgfsetstrokecolor{currentstroke}%
\pgfsetstrokeopacity{0.700000}%
\pgfsetdash{}{0pt}%
\pgfpathmoveto{\pgfqpoint{8.130568in}{2.839968in}}%
\pgfpathcurveto{\pgfqpoint{8.135612in}{2.839968in}}{\pgfqpoint{8.140450in}{2.841971in}}{\pgfqpoint{8.144016in}{2.845538in}}%
\pgfpathcurveto{\pgfqpoint{8.147583in}{2.849104in}}{\pgfqpoint{8.149586in}{2.853942in}}{\pgfqpoint{8.149586in}{2.858986in}}%
\pgfpathcurveto{\pgfqpoint{8.149586in}{2.864029in}}{\pgfqpoint{8.147583in}{2.868867in}}{\pgfqpoint{8.144016in}{2.872434in}}%
\pgfpathcurveto{\pgfqpoint{8.140450in}{2.876000in}}{\pgfqpoint{8.135612in}{2.878004in}}{\pgfqpoint{8.130568in}{2.878004in}}%
\pgfpathcurveto{\pgfqpoint{8.125525in}{2.878004in}}{\pgfqpoint{8.120687in}{2.876000in}}{\pgfqpoint{8.117120in}{2.872434in}}%
\pgfpathcurveto{\pgfqpoint{8.113554in}{2.868867in}}{\pgfqpoint{8.111550in}{2.864029in}}{\pgfqpoint{8.111550in}{2.858986in}}%
\pgfpathcurveto{\pgfqpoint{8.111550in}{2.853942in}}{\pgfqpoint{8.113554in}{2.849104in}}{\pgfqpoint{8.117120in}{2.845538in}}%
\pgfpathcurveto{\pgfqpoint{8.120687in}{2.841971in}}{\pgfqpoint{8.125525in}{2.839968in}}{\pgfqpoint{8.130568in}{2.839968in}}%
\pgfpathclose%
\pgfusepath{fill}%
\end{pgfscope}%
\begin{pgfscope}%
\pgfpathrectangle{\pgfqpoint{6.572727in}{0.473000in}}{\pgfqpoint{4.227273in}{3.311000in}}%
\pgfusepath{clip}%
\pgfsetbuttcap%
\pgfsetroundjoin%
\definecolor{currentfill}{rgb}{0.993248,0.906157,0.143936}%
\pgfsetfillcolor{currentfill}%
\pgfsetfillopacity{0.700000}%
\pgfsetlinewidth{0.000000pt}%
\definecolor{currentstroke}{rgb}{0.000000,0.000000,0.000000}%
\pgfsetstrokecolor{currentstroke}%
\pgfsetstrokeopacity{0.700000}%
\pgfsetdash{}{0pt}%
\pgfpathmoveto{\pgfqpoint{9.747110in}{1.770855in}}%
\pgfpathcurveto{\pgfqpoint{9.752153in}{1.770855in}}{\pgfqpoint{9.756991in}{1.772859in}}{\pgfqpoint{9.760558in}{1.776426in}}%
\pgfpathcurveto{\pgfqpoint{9.764124in}{1.779992in}}{\pgfqpoint{9.766128in}{1.784830in}}{\pgfqpoint{9.766128in}{1.789873in}}%
\pgfpathcurveto{\pgfqpoint{9.766128in}{1.794917in}}{\pgfqpoint{9.764124in}{1.799755in}}{\pgfqpoint{9.760558in}{1.803321in}}%
\pgfpathcurveto{\pgfqpoint{9.756991in}{1.806888in}}{\pgfqpoint{9.752153in}{1.808892in}}{\pgfqpoint{9.747110in}{1.808892in}}%
\pgfpathcurveto{\pgfqpoint{9.742066in}{1.808892in}}{\pgfqpoint{9.737228in}{1.806888in}}{\pgfqpoint{9.733662in}{1.803321in}}%
\pgfpathcurveto{\pgfqpoint{9.730095in}{1.799755in}}{\pgfqpoint{9.728092in}{1.794917in}}{\pgfqpoint{9.728092in}{1.789873in}}%
\pgfpathcurveto{\pgfqpoint{9.728092in}{1.784830in}}{\pgfqpoint{9.730095in}{1.779992in}}{\pgfqpoint{9.733662in}{1.776426in}}%
\pgfpathcurveto{\pgfqpoint{9.737228in}{1.772859in}}{\pgfqpoint{9.742066in}{1.770855in}}{\pgfqpoint{9.747110in}{1.770855in}}%
\pgfpathclose%
\pgfusepath{fill}%
\end{pgfscope}%
\begin{pgfscope}%
\pgfpathrectangle{\pgfqpoint{6.572727in}{0.473000in}}{\pgfqpoint{4.227273in}{3.311000in}}%
\pgfusepath{clip}%
\pgfsetbuttcap%
\pgfsetroundjoin%
\definecolor{currentfill}{rgb}{0.127568,0.566949,0.550556}%
\pgfsetfillcolor{currentfill}%
\pgfsetfillopacity{0.700000}%
\pgfsetlinewidth{0.000000pt}%
\definecolor{currentstroke}{rgb}{0.000000,0.000000,0.000000}%
\pgfsetstrokecolor{currentstroke}%
\pgfsetstrokeopacity{0.700000}%
\pgfsetdash{}{0pt}%
\pgfpathmoveto{\pgfqpoint{7.462407in}{1.475482in}}%
\pgfpathcurveto{\pgfqpoint{7.467451in}{1.475482in}}{\pgfqpoint{7.472288in}{1.477486in}}{\pgfqpoint{7.475855in}{1.481052in}}%
\pgfpathcurveto{\pgfqpoint{7.479421in}{1.484618in}}{\pgfqpoint{7.481425in}{1.489456in}}{\pgfqpoint{7.481425in}{1.494500in}}%
\pgfpathcurveto{\pgfqpoint{7.481425in}{1.499544in}}{\pgfqpoint{7.479421in}{1.504381in}}{\pgfqpoint{7.475855in}{1.507948in}}%
\pgfpathcurveto{\pgfqpoint{7.472288in}{1.511514in}}{\pgfqpoint{7.467451in}{1.513518in}}{\pgfqpoint{7.462407in}{1.513518in}}%
\pgfpathcurveto{\pgfqpoint{7.457363in}{1.513518in}}{\pgfqpoint{7.452526in}{1.511514in}}{\pgfqpoint{7.448959in}{1.507948in}}%
\pgfpathcurveto{\pgfqpoint{7.445393in}{1.504381in}}{\pgfqpoint{7.443389in}{1.499544in}}{\pgfqpoint{7.443389in}{1.494500in}}%
\pgfpathcurveto{\pgfqpoint{7.443389in}{1.489456in}}{\pgfqpoint{7.445393in}{1.484618in}}{\pgfqpoint{7.448959in}{1.481052in}}%
\pgfpathcurveto{\pgfqpoint{7.452526in}{1.477486in}}{\pgfqpoint{7.457363in}{1.475482in}}{\pgfqpoint{7.462407in}{1.475482in}}%
\pgfpathclose%
\pgfusepath{fill}%
\end{pgfscope}%
\begin{pgfscope}%
\pgfpathrectangle{\pgfqpoint{6.572727in}{0.473000in}}{\pgfqpoint{4.227273in}{3.311000in}}%
\pgfusepath{clip}%
\pgfsetbuttcap%
\pgfsetroundjoin%
\definecolor{currentfill}{rgb}{0.267004,0.004874,0.329415}%
\pgfsetfillcolor{currentfill}%
\pgfsetfillopacity{0.700000}%
\pgfsetlinewidth{0.000000pt}%
\definecolor{currentstroke}{rgb}{0.000000,0.000000,0.000000}%
\pgfsetstrokecolor{currentstroke}%
\pgfsetstrokeopacity{0.700000}%
\pgfsetdash{}{0pt}%
\pgfpathmoveto{\pgfqpoint{9.811303in}{0.702612in}}%
\pgfpathcurveto{\pgfqpoint{9.816347in}{0.702612in}}{\pgfqpoint{9.821185in}{0.704616in}}{\pgfqpoint{9.824751in}{0.708183in}}%
\pgfpathcurveto{\pgfqpoint{9.828318in}{0.711749in}}{\pgfqpoint{9.830322in}{0.716587in}}{\pgfqpoint{9.830322in}{0.721630in}}%
\pgfpathcurveto{\pgfqpoint{9.830322in}{0.726674in}}{\pgfqpoint{9.828318in}{0.731512in}}{\pgfqpoint{9.824751in}{0.735078in}}%
\pgfpathcurveto{\pgfqpoint{9.821185in}{0.738645in}}{\pgfqpoint{9.816347in}{0.740649in}}{\pgfqpoint{9.811303in}{0.740649in}}%
\pgfpathcurveto{\pgfqpoint{9.806260in}{0.740649in}}{\pgfqpoint{9.801422in}{0.738645in}}{\pgfqpoint{9.797856in}{0.735078in}}%
\pgfpathcurveto{\pgfqpoint{9.794289in}{0.731512in}}{\pgfqpoint{9.792285in}{0.726674in}}{\pgfqpoint{9.792285in}{0.721630in}}%
\pgfpathcurveto{\pgfqpoint{9.792285in}{0.716587in}}{\pgfqpoint{9.794289in}{0.711749in}}{\pgfqpoint{9.797856in}{0.708183in}}%
\pgfpathcurveto{\pgfqpoint{9.801422in}{0.704616in}}{\pgfqpoint{9.806260in}{0.702612in}}{\pgfqpoint{9.811303in}{0.702612in}}%
\pgfpathclose%
\pgfusepath{fill}%
\end{pgfscope}%
\begin{pgfscope}%
\pgfpathrectangle{\pgfqpoint{6.572727in}{0.473000in}}{\pgfqpoint{4.227273in}{3.311000in}}%
\pgfusepath{clip}%
\pgfsetbuttcap%
\pgfsetroundjoin%
\definecolor{currentfill}{rgb}{0.127568,0.566949,0.550556}%
\pgfsetfillcolor{currentfill}%
\pgfsetfillopacity{0.700000}%
\pgfsetlinewidth{0.000000pt}%
\definecolor{currentstroke}{rgb}{0.000000,0.000000,0.000000}%
\pgfsetstrokecolor{currentstroke}%
\pgfsetstrokeopacity{0.700000}%
\pgfsetdash{}{0pt}%
\pgfpathmoveto{\pgfqpoint{7.547652in}{1.710791in}}%
\pgfpathcurveto{\pgfqpoint{7.552696in}{1.710791in}}{\pgfqpoint{7.557534in}{1.712795in}}{\pgfqpoint{7.561100in}{1.716361in}}%
\pgfpathcurveto{\pgfqpoint{7.564667in}{1.719928in}}{\pgfqpoint{7.566671in}{1.724766in}}{\pgfqpoint{7.566671in}{1.729809in}}%
\pgfpathcurveto{\pgfqpoint{7.566671in}{1.734853in}}{\pgfqpoint{7.564667in}{1.739691in}}{\pgfqpoint{7.561100in}{1.743257in}}%
\pgfpathcurveto{\pgfqpoint{7.557534in}{1.746823in}}{\pgfqpoint{7.552696in}{1.748827in}}{\pgfqpoint{7.547652in}{1.748827in}}%
\pgfpathcurveto{\pgfqpoint{7.542609in}{1.748827in}}{\pgfqpoint{7.537771in}{1.746823in}}{\pgfqpoint{7.534205in}{1.743257in}}%
\pgfpathcurveto{\pgfqpoint{7.530638in}{1.739691in}}{\pgfqpoint{7.528634in}{1.734853in}}{\pgfqpoint{7.528634in}{1.729809in}}%
\pgfpathcurveto{\pgfqpoint{7.528634in}{1.724766in}}{\pgfqpoint{7.530638in}{1.719928in}}{\pgfqpoint{7.534205in}{1.716361in}}%
\pgfpathcurveto{\pgfqpoint{7.537771in}{1.712795in}}{\pgfqpoint{7.542609in}{1.710791in}}{\pgfqpoint{7.547652in}{1.710791in}}%
\pgfpathclose%
\pgfusepath{fill}%
\end{pgfscope}%
\begin{pgfscope}%
\pgfpathrectangle{\pgfqpoint{6.572727in}{0.473000in}}{\pgfqpoint{4.227273in}{3.311000in}}%
\pgfusepath{clip}%
\pgfsetbuttcap%
\pgfsetroundjoin%
\definecolor{currentfill}{rgb}{0.127568,0.566949,0.550556}%
\pgfsetfillcolor{currentfill}%
\pgfsetfillopacity{0.700000}%
\pgfsetlinewidth{0.000000pt}%
\definecolor{currentstroke}{rgb}{0.000000,0.000000,0.000000}%
\pgfsetstrokecolor{currentstroke}%
\pgfsetstrokeopacity{0.700000}%
\pgfsetdash{}{0pt}%
\pgfpathmoveto{\pgfqpoint{8.850450in}{2.862325in}}%
\pgfpathcurveto{\pgfqpoint{8.855493in}{2.862325in}}{\pgfqpoint{8.860331in}{2.864329in}}{\pgfqpoint{8.863898in}{2.867895in}}%
\pgfpathcurveto{\pgfqpoint{8.867464in}{2.871462in}}{\pgfqpoint{8.869468in}{2.876300in}}{\pgfqpoint{8.869468in}{2.881343in}}%
\pgfpathcurveto{\pgfqpoint{8.869468in}{2.886387in}}{\pgfqpoint{8.867464in}{2.891225in}}{\pgfqpoint{8.863898in}{2.894791in}}%
\pgfpathcurveto{\pgfqpoint{8.860331in}{2.898358in}}{\pgfqpoint{8.855493in}{2.900361in}}{\pgfqpoint{8.850450in}{2.900361in}}%
\pgfpathcurveto{\pgfqpoint{8.845406in}{2.900361in}}{\pgfqpoint{8.840568in}{2.898358in}}{\pgfqpoint{8.837002in}{2.894791in}}%
\pgfpathcurveto{\pgfqpoint{8.833435in}{2.891225in}}{\pgfqpoint{8.831432in}{2.886387in}}{\pgfqpoint{8.831432in}{2.881343in}}%
\pgfpathcurveto{\pgfqpoint{8.831432in}{2.876300in}}{\pgfqpoint{8.833435in}{2.871462in}}{\pgfqpoint{8.837002in}{2.867895in}}%
\pgfpathcurveto{\pgfqpoint{8.840568in}{2.864329in}}{\pgfqpoint{8.845406in}{2.862325in}}{\pgfqpoint{8.850450in}{2.862325in}}%
\pgfpathclose%
\pgfusepath{fill}%
\end{pgfscope}%
\begin{pgfscope}%
\pgfpathrectangle{\pgfqpoint{6.572727in}{0.473000in}}{\pgfqpoint{4.227273in}{3.311000in}}%
\pgfusepath{clip}%
\pgfsetbuttcap%
\pgfsetroundjoin%
\definecolor{currentfill}{rgb}{0.993248,0.906157,0.143936}%
\pgfsetfillcolor{currentfill}%
\pgfsetfillopacity{0.700000}%
\pgfsetlinewidth{0.000000pt}%
\definecolor{currentstroke}{rgb}{0.000000,0.000000,0.000000}%
\pgfsetstrokecolor{currentstroke}%
\pgfsetstrokeopacity{0.700000}%
\pgfsetdash{}{0pt}%
\pgfpathmoveto{\pgfqpoint{9.659783in}{1.733178in}}%
\pgfpathcurveto{\pgfqpoint{9.664826in}{1.733178in}}{\pgfqpoint{9.669664in}{1.735181in}}{\pgfqpoint{9.673230in}{1.738748in}}%
\pgfpathcurveto{\pgfqpoint{9.676797in}{1.742314in}}{\pgfqpoint{9.678801in}{1.747152in}}{\pgfqpoint{9.678801in}{1.752196in}}%
\pgfpathcurveto{\pgfqpoint{9.678801in}{1.757239in}}{\pgfqpoint{9.676797in}{1.762077in}}{\pgfqpoint{9.673230in}{1.765644in}}%
\pgfpathcurveto{\pgfqpoint{9.669664in}{1.769210in}}{\pgfqpoint{9.664826in}{1.771214in}}{\pgfqpoint{9.659783in}{1.771214in}}%
\pgfpathcurveto{\pgfqpoint{9.654739in}{1.771214in}}{\pgfqpoint{9.649901in}{1.769210in}}{\pgfqpoint{9.646335in}{1.765644in}}%
\pgfpathcurveto{\pgfqpoint{9.642768in}{1.762077in}}{\pgfqpoint{9.640764in}{1.757239in}}{\pgfqpoint{9.640764in}{1.752196in}}%
\pgfpathcurveto{\pgfqpoint{9.640764in}{1.747152in}}{\pgfqpoint{9.642768in}{1.742314in}}{\pgfqpoint{9.646335in}{1.738748in}}%
\pgfpathcurveto{\pgfqpoint{9.649901in}{1.735181in}}{\pgfqpoint{9.654739in}{1.733178in}}{\pgfqpoint{9.659783in}{1.733178in}}%
\pgfpathclose%
\pgfusepath{fill}%
\end{pgfscope}%
\begin{pgfscope}%
\pgfpathrectangle{\pgfqpoint{6.572727in}{0.473000in}}{\pgfqpoint{4.227273in}{3.311000in}}%
\pgfusepath{clip}%
\pgfsetbuttcap%
\pgfsetroundjoin%
\definecolor{currentfill}{rgb}{0.993248,0.906157,0.143936}%
\pgfsetfillcolor{currentfill}%
\pgfsetfillopacity{0.700000}%
\pgfsetlinewidth{0.000000pt}%
\definecolor{currentstroke}{rgb}{0.000000,0.000000,0.000000}%
\pgfsetstrokecolor{currentstroke}%
\pgfsetstrokeopacity{0.700000}%
\pgfsetdash{}{0pt}%
\pgfpathmoveto{\pgfqpoint{9.671296in}{0.890189in}}%
\pgfpathcurveto{\pgfqpoint{9.676339in}{0.890189in}}{\pgfqpoint{9.681177in}{0.892193in}}{\pgfqpoint{9.684743in}{0.895759in}}%
\pgfpathcurveto{\pgfqpoint{9.688310in}{0.899325in}}{\pgfqpoint{9.690314in}{0.904163in}}{\pgfqpoint{9.690314in}{0.909207in}}%
\pgfpathcurveto{\pgfqpoint{9.690314in}{0.914250in}}{\pgfqpoint{9.688310in}{0.919088in}}{\pgfqpoint{9.684743in}{0.922655in}}%
\pgfpathcurveto{\pgfqpoint{9.681177in}{0.926221in}}{\pgfqpoint{9.676339in}{0.928225in}}{\pgfqpoint{9.671296in}{0.928225in}}%
\pgfpathcurveto{\pgfqpoint{9.666252in}{0.928225in}}{\pgfqpoint{9.661414in}{0.926221in}}{\pgfqpoint{9.657848in}{0.922655in}}%
\pgfpathcurveto{\pgfqpoint{9.654281in}{0.919088in}}{\pgfqpoint{9.652277in}{0.914250in}}{\pgfqpoint{9.652277in}{0.909207in}}%
\pgfpathcurveto{\pgfqpoint{9.652277in}{0.904163in}}{\pgfqpoint{9.654281in}{0.899325in}}{\pgfqpoint{9.657848in}{0.895759in}}%
\pgfpathcurveto{\pgfqpoint{9.661414in}{0.892193in}}{\pgfqpoint{9.666252in}{0.890189in}}{\pgfqpoint{9.671296in}{0.890189in}}%
\pgfpathclose%
\pgfusepath{fill}%
\end{pgfscope}%
\begin{pgfscope}%
\pgfpathrectangle{\pgfqpoint{6.572727in}{0.473000in}}{\pgfqpoint{4.227273in}{3.311000in}}%
\pgfusepath{clip}%
\pgfsetbuttcap%
\pgfsetroundjoin%
\definecolor{currentfill}{rgb}{0.127568,0.566949,0.550556}%
\pgfsetfillcolor{currentfill}%
\pgfsetfillopacity{0.700000}%
\pgfsetlinewidth{0.000000pt}%
\definecolor{currentstroke}{rgb}{0.000000,0.000000,0.000000}%
\pgfsetstrokecolor{currentstroke}%
\pgfsetstrokeopacity{0.700000}%
\pgfsetdash{}{0pt}%
\pgfpathmoveto{\pgfqpoint{7.262959in}{2.221226in}}%
\pgfpathcurveto{\pgfqpoint{7.268003in}{2.221226in}}{\pgfqpoint{7.272841in}{2.223229in}}{\pgfqpoint{7.276407in}{2.226796in}}%
\pgfpathcurveto{\pgfqpoint{7.279974in}{2.230362in}}{\pgfqpoint{7.281977in}{2.235200in}}{\pgfqpoint{7.281977in}{2.240244in}}%
\pgfpathcurveto{\pgfqpoint{7.281977in}{2.245287in}}{\pgfqpoint{7.279974in}{2.250125in}}{\pgfqpoint{7.276407in}{2.253692in}}%
\pgfpathcurveto{\pgfqpoint{7.272841in}{2.257258in}}{\pgfqpoint{7.268003in}{2.259262in}}{\pgfqpoint{7.262959in}{2.259262in}}%
\pgfpathcurveto{\pgfqpoint{7.257916in}{2.259262in}}{\pgfqpoint{7.253078in}{2.257258in}}{\pgfqpoint{7.249511in}{2.253692in}}%
\pgfpathcurveto{\pgfqpoint{7.245945in}{2.250125in}}{\pgfqpoint{7.243941in}{2.245287in}}{\pgfqpoint{7.243941in}{2.240244in}}%
\pgfpathcurveto{\pgfqpoint{7.243941in}{2.235200in}}{\pgfqpoint{7.245945in}{2.230362in}}{\pgfqpoint{7.249511in}{2.226796in}}%
\pgfpathcurveto{\pgfqpoint{7.253078in}{2.223229in}}{\pgfqpoint{7.257916in}{2.221226in}}{\pgfqpoint{7.262959in}{2.221226in}}%
\pgfpathclose%
\pgfusepath{fill}%
\end{pgfscope}%
\begin{pgfscope}%
\pgfpathrectangle{\pgfqpoint{6.572727in}{0.473000in}}{\pgfqpoint{4.227273in}{3.311000in}}%
\pgfusepath{clip}%
\pgfsetbuttcap%
\pgfsetroundjoin%
\definecolor{currentfill}{rgb}{0.127568,0.566949,0.550556}%
\pgfsetfillcolor{currentfill}%
\pgfsetfillopacity{0.700000}%
\pgfsetlinewidth{0.000000pt}%
\definecolor{currentstroke}{rgb}{0.000000,0.000000,0.000000}%
\pgfsetstrokecolor{currentstroke}%
\pgfsetstrokeopacity{0.700000}%
\pgfsetdash{}{0pt}%
\pgfpathmoveto{\pgfqpoint{8.289269in}{1.382227in}}%
\pgfpathcurveto{\pgfqpoint{8.294313in}{1.382227in}}{\pgfqpoint{8.299151in}{1.384231in}}{\pgfqpoint{8.302717in}{1.387797in}}%
\pgfpathcurveto{\pgfqpoint{8.306284in}{1.391364in}}{\pgfqpoint{8.308288in}{1.396202in}}{\pgfqpoint{8.308288in}{1.401245in}}%
\pgfpathcurveto{\pgfqpoint{8.308288in}{1.406289in}}{\pgfqpoint{8.306284in}{1.411127in}}{\pgfqpoint{8.302717in}{1.414693in}}%
\pgfpathcurveto{\pgfqpoint{8.299151in}{1.418260in}}{\pgfqpoint{8.294313in}{1.420263in}}{\pgfqpoint{8.289269in}{1.420263in}}%
\pgfpathcurveto{\pgfqpoint{8.284226in}{1.420263in}}{\pgfqpoint{8.279388in}{1.418260in}}{\pgfqpoint{8.275822in}{1.414693in}}%
\pgfpathcurveto{\pgfqpoint{8.272255in}{1.411127in}}{\pgfqpoint{8.270251in}{1.406289in}}{\pgfqpoint{8.270251in}{1.401245in}}%
\pgfpathcurveto{\pgfqpoint{8.270251in}{1.396202in}}{\pgfqpoint{8.272255in}{1.391364in}}{\pgfqpoint{8.275822in}{1.387797in}}%
\pgfpathcurveto{\pgfqpoint{8.279388in}{1.384231in}}{\pgfqpoint{8.284226in}{1.382227in}}{\pgfqpoint{8.289269in}{1.382227in}}%
\pgfpathclose%
\pgfusepath{fill}%
\end{pgfscope}%
\begin{pgfscope}%
\pgfpathrectangle{\pgfqpoint{6.572727in}{0.473000in}}{\pgfqpoint{4.227273in}{3.311000in}}%
\pgfusepath{clip}%
\pgfsetbuttcap%
\pgfsetroundjoin%
\definecolor{currentfill}{rgb}{0.127568,0.566949,0.550556}%
\pgfsetfillcolor{currentfill}%
\pgfsetfillopacity{0.700000}%
\pgfsetlinewidth{0.000000pt}%
\definecolor{currentstroke}{rgb}{0.000000,0.000000,0.000000}%
\pgfsetstrokecolor{currentstroke}%
\pgfsetstrokeopacity{0.700000}%
\pgfsetdash{}{0pt}%
\pgfpathmoveto{\pgfqpoint{8.163009in}{2.501906in}}%
\pgfpathcurveto{\pgfqpoint{8.168052in}{2.501906in}}{\pgfqpoint{8.172890in}{2.503910in}}{\pgfqpoint{8.176457in}{2.507476in}}%
\pgfpathcurveto{\pgfqpoint{8.180023in}{2.511043in}}{\pgfqpoint{8.182027in}{2.515880in}}{\pgfqpoint{8.182027in}{2.520924in}}%
\pgfpathcurveto{\pgfqpoint{8.182027in}{2.525968in}}{\pgfqpoint{8.180023in}{2.530806in}}{\pgfqpoint{8.176457in}{2.534372in}}%
\pgfpathcurveto{\pgfqpoint{8.172890in}{2.537938in}}{\pgfqpoint{8.168052in}{2.539942in}}{\pgfqpoint{8.163009in}{2.539942in}}%
\pgfpathcurveto{\pgfqpoint{8.157965in}{2.539942in}}{\pgfqpoint{8.153127in}{2.537938in}}{\pgfqpoint{8.149561in}{2.534372in}}%
\pgfpathcurveto{\pgfqpoint{8.145994in}{2.530806in}}{\pgfqpoint{8.143991in}{2.525968in}}{\pgfqpoint{8.143991in}{2.520924in}}%
\pgfpathcurveto{\pgfqpoint{8.143991in}{2.515880in}}{\pgfqpoint{8.145994in}{2.511043in}}{\pgfqpoint{8.149561in}{2.507476in}}%
\pgfpathcurveto{\pgfqpoint{8.153127in}{2.503910in}}{\pgfqpoint{8.157965in}{2.501906in}}{\pgfqpoint{8.163009in}{2.501906in}}%
\pgfpathclose%
\pgfusepath{fill}%
\end{pgfscope}%
\begin{pgfscope}%
\pgfpathrectangle{\pgfqpoint{6.572727in}{0.473000in}}{\pgfqpoint{4.227273in}{3.311000in}}%
\pgfusepath{clip}%
\pgfsetbuttcap%
\pgfsetroundjoin%
\definecolor{currentfill}{rgb}{0.127568,0.566949,0.550556}%
\pgfsetfillcolor{currentfill}%
\pgfsetfillopacity{0.700000}%
\pgfsetlinewidth{0.000000pt}%
\definecolor{currentstroke}{rgb}{0.000000,0.000000,0.000000}%
\pgfsetstrokecolor{currentstroke}%
\pgfsetstrokeopacity{0.700000}%
\pgfsetdash{}{0pt}%
\pgfpathmoveto{\pgfqpoint{8.005111in}{2.633015in}}%
\pgfpathcurveto{\pgfqpoint{8.010155in}{2.633015in}}{\pgfqpoint{8.014993in}{2.635019in}}{\pgfqpoint{8.018559in}{2.638585in}}%
\pgfpathcurveto{\pgfqpoint{8.022126in}{2.642152in}}{\pgfqpoint{8.024129in}{2.646989in}}{\pgfqpoint{8.024129in}{2.652033in}}%
\pgfpathcurveto{\pgfqpoint{8.024129in}{2.657077in}}{\pgfqpoint{8.022126in}{2.661915in}}{\pgfqpoint{8.018559in}{2.665481in}}%
\pgfpathcurveto{\pgfqpoint{8.014993in}{2.669047in}}{\pgfqpoint{8.010155in}{2.671051in}}{\pgfqpoint{8.005111in}{2.671051in}}%
\pgfpathcurveto{\pgfqpoint{8.000068in}{2.671051in}}{\pgfqpoint{7.995230in}{2.669047in}}{\pgfqpoint{7.991663in}{2.665481in}}%
\pgfpathcurveto{\pgfqpoint{7.988097in}{2.661915in}}{\pgfqpoint{7.986093in}{2.657077in}}{\pgfqpoint{7.986093in}{2.652033in}}%
\pgfpathcurveto{\pgfqpoint{7.986093in}{2.646989in}}{\pgfqpoint{7.988097in}{2.642152in}}{\pgfqpoint{7.991663in}{2.638585in}}%
\pgfpathcurveto{\pgfqpoint{7.995230in}{2.635019in}}{\pgfqpoint{8.000068in}{2.633015in}}{\pgfqpoint{8.005111in}{2.633015in}}%
\pgfpathclose%
\pgfusepath{fill}%
\end{pgfscope}%
\begin{pgfscope}%
\pgfpathrectangle{\pgfqpoint{6.572727in}{0.473000in}}{\pgfqpoint{4.227273in}{3.311000in}}%
\pgfusepath{clip}%
\pgfsetbuttcap%
\pgfsetroundjoin%
\definecolor{currentfill}{rgb}{0.127568,0.566949,0.550556}%
\pgfsetfillcolor{currentfill}%
\pgfsetfillopacity{0.700000}%
\pgfsetlinewidth{0.000000pt}%
\definecolor{currentstroke}{rgb}{0.000000,0.000000,0.000000}%
\pgfsetstrokecolor{currentstroke}%
\pgfsetstrokeopacity{0.700000}%
\pgfsetdash{}{0pt}%
\pgfpathmoveto{\pgfqpoint{7.308244in}{1.979716in}}%
\pgfpathcurveto{\pgfqpoint{7.313287in}{1.979716in}}{\pgfqpoint{7.318125in}{1.981720in}}{\pgfqpoint{7.321692in}{1.985287in}}%
\pgfpathcurveto{\pgfqpoint{7.325258in}{1.988853in}}{\pgfqpoint{7.327262in}{1.993691in}}{\pgfqpoint{7.327262in}{1.998734in}}%
\pgfpathcurveto{\pgfqpoint{7.327262in}{2.003778in}}{\pgfqpoint{7.325258in}{2.008616in}}{\pgfqpoint{7.321692in}{2.012182in}}%
\pgfpathcurveto{\pgfqpoint{7.318125in}{2.015749in}}{\pgfqpoint{7.313287in}{2.017753in}}{\pgfqpoint{7.308244in}{2.017753in}}%
\pgfpathcurveto{\pgfqpoint{7.303200in}{2.017753in}}{\pgfqpoint{7.298362in}{2.015749in}}{\pgfqpoint{7.294796in}{2.012182in}}%
\pgfpathcurveto{\pgfqpoint{7.291229in}{2.008616in}}{\pgfqpoint{7.289226in}{2.003778in}}{\pgfqpoint{7.289226in}{1.998734in}}%
\pgfpathcurveto{\pgfqpoint{7.289226in}{1.993691in}}{\pgfqpoint{7.291229in}{1.988853in}}{\pgfqpoint{7.294796in}{1.985287in}}%
\pgfpathcurveto{\pgfqpoint{7.298362in}{1.981720in}}{\pgfqpoint{7.303200in}{1.979716in}}{\pgfqpoint{7.308244in}{1.979716in}}%
\pgfpathclose%
\pgfusepath{fill}%
\end{pgfscope}%
\begin{pgfscope}%
\pgfpathrectangle{\pgfqpoint{6.572727in}{0.473000in}}{\pgfqpoint{4.227273in}{3.311000in}}%
\pgfusepath{clip}%
\pgfsetbuttcap%
\pgfsetroundjoin%
\definecolor{currentfill}{rgb}{0.127568,0.566949,0.550556}%
\pgfsetfillcolor{currentfill}%
\pgfsetfillopacity{0.700000}%
\pgfsetlinewidth{0.000000pt}%
\definecolor{currentstroke}{rgb}{0.000000,0.000000,0.000000}%
\pgfsetstrokecolor{currentstroke}%
\pgfsetstrokeopacity{0.700000}%
\pgfsetdash{}{0pt}%
\pgfpathmoveto{\pgfqpoint{7.376047in}{2.814185in}}%
\pgfpathcurveto{\pgfqpoint{7.381090in}{2.814185in}}{\pgfqpoint{7.385928in}{2.816189in}}{\pgfqpoint{7.389495in}{2.819755in}}%
\pgfpathcurveto{\pgfqpoint{7.393061in}{2.823321in}}{\pgfqpoint{7.395065in}{2.828159in}}{\pgfqpoint{7.395065in}{2.833203in}}%
\pgfpathcurveto{\pgfqpoint{7.395065in}{2.838247in}}{\pgfqpoint{7.393061in}{2.843084in}}{\pgfqpoint{7.389495in}{2.846651in}}%
\pgfpathcurveto{\pgfqpoint{7.385928in}{2.850217in}}{\pgfqpoint{7.381090in}{2.852221in}}{\pgfqpoint{7.376047in}{2.852221in}}%
\pgfpathcurveto{\pgfqpoint{7.371003in}{2.852221in}}{\pgfqpoint{7.366165in}{2.850217in}}{\pgfqpoint{7.362599in}{2.846651in}}%
\pgfpathcurveto{\pgfqpoint{7.359032in}{2.843084in}}{\pgfqpoint{7.357029in}{2.838247in}}{\pgfqpoint{7.357029in}{2.833203in}}%
\pgfpathcurveto{\pgfqpoint{7.357029in}{2.828159in}}{\pgfqpoint{7.359032in}{2.823321in}}{\pgfqpoint{7.362599in}{2.819755in}}%
\pgfpathcurveto{\pgfqpoint{7.366165in}{2.816189in}}{\pgfqpoint{7.371003in}{2.814185in}}{\pgfqpoint{7.376047in}{2.814185in}}%
\pgfpathclose%
\pgfusepath{fill}%
\end{pgfscope}%
\begin{pgfscope}%
\pgfpathrectangle{\pgfqpoint{6.572727in}{0.473000in}}{\pgfqpoint{4.227273in}{3.311000in}}%
\pgfusepath{clip}%
\pgfsetbuttcap%
\pgfsetroundjoin%
\definecolor{currentfill}{rgb}{0.993248,0.906157,0.143936}%
\pgfsetfillcolor{currentfill}%
\pgfsetfillopacity{0.700000}%
\pgfsetlinewidth{0.000000pt}%
\definecolor{currentstroke}{rgb}{0.000000,0.000000,0.000000}%
\pgfsetstrokecolor{currentstroke}%
\pgfsetstrokeopacity{0.700000}%
\pgfsetdash{}{0pt}%
\pgfpathmoveto{\pgfqpoint{9.555297in}{1.828679in}}%
\pgfpathcurveto{\pgfqpoint{9.560341in}{1.828679in}}{\pgfqpoint{9.565179in}{1.830683in}}{\pgfqpoint{9.568745in}{1.834249in}}%
\pgfpathcurveto{\pgfqpoint{9.572312in}{1.837815in}}{\pgfqpoint{9.574315in}{1.842653in}}{\pgfqpoint{9.574315in}{1.847697in}}%
\pgfpathcurveto{\pgfqpoint{9.574315in}{1.852741in}}{\pgfqpoint{9.572312in}{1.857578in}}{\pgfqpoint{9.568745in}{1.861145in}}%
\pgfpathcurveto{\pgfqpoint{9.565179in}{1.864711in}}{\pgfqpoint{9.560341in}{1.866715in}}{\pgfqpoint{9.555297in}{1.866715in}}%
\pgfpathcurveto{\pgfqpoint{9.550254in}{1.866715in}}{\pgfqpoint{9.545416in}{1.864711in}}{\pgfqpoint{9.541849in}{1.861145in}}%
\pgfpathcurveto{\pgfqpoint{9.538283in}{1.857578in}}{\pgfqpoint{9.536279in}{1.852741in}}{\pgfqpoint{9.536279in}{1.847697in}}%
\pgfpathcurveto{\pgfqpoint{9.536279in}{1.842653in}}{\pgfqpoint{9.538283in}{1.837815in}}{\pgfqpoint{9.541849in}{1.834249in}}%
\pgfpathcurveto{\pgfqpoint{9.545416in}{1.830683in}}{\pgfqpoint{9.550254in}{1.828679in}}{\pgfqpoint{9.555297in}{1.828679in}}%
\pgfpathclose%
\pgfusepath{fill}%
\end{pgfscope}%
\begin{pgfscope}%
\pgfpathrectangle{\pgfqpoint{6.572727in}{0.473000in}}{\pgfqpoint{4.227273in}{3.311000in}}%
\pgfusepath{clip}%
\pgfsetbuttcap%
\pgfsetroundjoin%
\definecolor{currentfill}{rgb}{0.993248,0.906157,0.143936}%
\pgfsetfillcolor{currentfill}%
\pgfsetfillopacity{0.700000}%
\pgfsetlinewidth{0.000000pt}%
\definecolor{currentstroke}{rgb}{0.000000,0.000000,0.000000}%
\pgfsetstrokecolor{currentstroke}%
\pgfsetstrokeopacity{0.700000}%
\pgfsetdash{}{0pt}%
\pgfpathmoveto{\pgfqpoint{9.942156in}{1.614417in}}%
\pgfpathcurveto{\pgfqpoint{9.947199in}{1.614417in}}{\pgfqpoint{9.952037in}{1.616421in}}{\pgfqpoint{9.955604in}{1.619987in}}%
\pgfpathcurveto{\pgfqpoint{9.959170in}{1.623554in}}{\pgfqpoint{9.961174in}{1.628391in}}{\pgfqpoint{9.961174in}{1.633435in}}%
\pgfpathcurveto{\pgfqpoint{9.961174in}{1.638479in}}{\pgfqpoint{9.959170in}{1.643317in}}{\pgfqpoint{9.955604in}{1.646883in}}%
\pgfpathcurveto{\pgfqpoint{9.952037in}{1.650449in}}{\pgfqpoint{9.947199in}{1.652453in}}{\pgfqpoint{9.942156in}{1.652453in}}%
\pgfpathcurveto{\pgfqpoint{9.937112in}{1.652453in}}{\pgfqpoint{9.932274in}{1.650449in}}{\pgfqpoint{9.928708in}{1.646883in}}%
\pgfpathcurveto{\pgfqpoint{9.925141in}{1.643317in}}{\pgfqpoint{9.923138in}{1.638479in}}{\pgfqpoint{9.923138in}{1.633435in}}%
\pgfpathcurveto{\pgfqpoint{9.923138in}{1.628391in}}{\pgfqpoint{9.925141in}{1.623554in}}{\pgfqpoint{9.928708in}{1.619987in}}%
\pgfpathcurveto{\pgfqpoint{9.932274in}{1.616421in}}{\pgfqpoint{9.937112in}{1.614417in}}{\pgfqpoint{9.942156in}{1.614417in}}%
\pgfpathclose%
\pgfusepath{fill}%
\end{pgfscope}%
\begin{pgfscope}%
\pgfpathrectangle{\pgfqpoint{6.572727in}{0.473000in}}{\pgfqpoint{4.227273in}{3.311000in}}%
\pgfusepath{clip}%
\pgfsetbuttcap%
\pgfsetroundjoin%
\definecolor{currentfill}{rgb}{0.127568,0.566949,0.550556}%
\pgfsetfillcolor{currentfill}%
\pgfsetfillopacity{0.700000}%
\pgfsetlinewidth{0.000000pt}%
\definecolor{currentstroke}{rgb}{0.000000,0.000000,0.000000}%
\pgfsetstrokecolor{currentstroke}%
\pgfsetstrokeopacity{0.700000}%
\pgfsetdash{}{0pt}%
\pgfpathmoveto{\pgfqpoint{7.526435in}{1.780288in}}%
\pgfpathcurveto{\pgfqpoint{7.531478in}{1.780288in}}{\pgfqpoint{7.536316in}{1.782291in}}{\pgfqpoint{7.539883in}{1.785858in}}%
\pgfpathcurveto{\pgfqpoint{7.543449in}{1.789424in}}{\pgfqpoint{7.545453in}{1.794262in}}{\pgfqpoint{7.545453in}{1.799306in}}%
\pgfpathcurveto{\pgfqpoint{7.545453in}{1.804349in}}{\pgfqpoint{7.543449in}{1.809187in}}{\pgfqpoint{7.539883in}{1.812754in}}%
\pgfpathcurveto{\pgfqpoint{7.536316in}{1.816320in}}{\pgfqpoint{7.531478in}{1.818324in}}{\pgfqpoint{7.526435in}{1.818324in}}%
\pgfpathcurveto{\pgfqpoint{7.521391in}{1.818324in}}{\pgfqpoint{7.516553in}{1.816320in}}{\pgfqpoint{7.512987in}{1.812754in}}%
\pgfpathcurveto{\pgfqpoint{7.509420in}{1.809187in}}{\pgfqpoint{7.507417in}{1.804349in}}{\pgfqpoint{7.507417in}{1.799306in}}%
\pgfpathcurveto{\pgfqpoint{7.507417in}{1.794262in}}{\pgfqpoint{7.509420in}{1.789424in}}{\pgfqpoint{7.512987in}{1.785858in}}%
\pgfpathcurveto{\pgfqpoint{7.516553in}{1.782291in}}{\pgfqpoint{7.521391in}{1.780288in}}{\pgfqpoint{7.526435in}{1.780288in}}%
\pgfpathclose%
\pgfusepath{fill}%
\end{pgfscope}%
\begin{pgfscope}%
\pgfpathrectangle{\pgfqpoint{6.572727in}{0.473000in}}{\pgfqpoint{4.227273in}{3.311000in}}%
\pgfusepath{clip}%
\pgfsetbuttcap%
\pgfsetroundjoin%
\definecolor{currentfill}{rgb}{0.127568,0.566949,0.550556}%
\pgfsetfillcolor{currentfill}%
\pgfsetfillopacity{0.700000}%
\pgfsetlinewidth{0.000000pt}%
\definecolor{currentstroke}{rgb}{0.000000,0.000000,0.000000}%
\pgfsetstrokecolor{currentstroke}%
\pgfsetstrokeopacity{0.700000}%
\pgfsetdash{}{0pt}%
\pgfpathmoveto{\pgfqpoint{7.469826in}{1.252052in}}%
\pgfpathcurveto{\pgfqpoint{7.474869in}{1.252052in}}{\pgfqpoint{7.479707in}{1.254056in}}{\pgfqpoint{7.483273in}{1.257622in}}%
\pgfpathcurveto{\pgfqpoint{7.486840in}{1.261189in}}{\pgfqpoint{7.488844in}{1.266026in}}{\pgfqpoint{7.488844in}{1.271070in}}%
\pgfpathcurveto{\pgfqpoint{7.488844in}{1.276114in}}{\pgfqpoint{7.486840in}{1.280952in}}{\pgfqpoint{7.483273in}{1.284518in}}%
\pgfpathcurveto{\pgfqpoint{7.479707in}{1.288084in}}{\pgfqpoint{7.474869in}{1.290088in}}{\pgfqpoint{7.469826in}{1.290088in}}%
\pgfpathcurveto{\pgfqpoint{7.464782in}{1.290088in}}{\pgfqpoint{7.459944in}{1.288084in}}{\pgfqpoint{7.456378in}{1.284518in}}%
\pgfpathcurveto{\pgfqpoint{7.452811in}{1.280952in}}{\pgfqpoint{7.450807in}{1.276114in}}{\pgfqpoint{7.450807in}{1.271070in}}%
\pgfpathcurveto{\pgfqpoint{7.450807in}{1.266026in}}{\pgfqpoint{7.452811in}{1.261189in}}{\pgfqpoint{7.456378in}{1.257622in}}%
\pgfpathcurveto{\pgfqpoint{7.459944in}{1.254056in}}{\pgfqpoint{7.464782in}{1.252052in}}{\pgfqpoint{7.469826in}{1.252052in}}%
\pgfpathclose%
\pgfusepath{fill}%
\end{pgfscope}%
\begin{pgfscope}%
\pgfpathrectangle{\pgfqpoint{6.572727in}{0.473000in}}{\pgfqpoint{4.227273in}{3.311000in}}%
\pgfusepath{clip}%
\pgfsetbuttcap%
\pgfsetroundjoin%
\definecolor{currentfill}{rgb}{0.993248,0.906157,0.143936}%
\pgfsetfillcolor{currentfill}%
\pgfsetfillopacity{0.700000}%
\pgfsetlinewidth{0.000000pt}%
\definecolor{currentstroke}{rgb}{0.000000,0.000000,0.000000}%
\pgfsetstrokecolor{currentstroke}%
\pgfsetstrokeopacity{0.700000}%
\pgfsetdash{}{0pt}%
\pgfpathmoveto{\pgfqpoint{9.474662in}{1.870427in}}%
\pgfpathcurveto{\pgfqpoint{9.479706in}{1.870427in}}{\pgfqpoint{9.484544in}{1.872431in}}{\pgfqpoint{9.488110in}{1.875997in}}%
\pgfpathcurveto{\pgfqpoint{9.491677in}{1.879564in}}{\pgfqpoint{9.493681in}{1.884402in}}{\pgfqpoint{9.493681in}{1.889445in}}%
\pgfpathcurveto{\pgfqpoint{9.493681in}{1.894489in}}{\pgfqpoint{9.491677in}{1.899327in}}{\pgfqpoint{9.488110in}{1.902893in}}%
\pgfpathcurveto{\pgfqpoint{9.484544in}{1.906460in}}{\pgfqpoint{9.479706in}{1.908463in}}{\pgfqpoint{9.474662in}{1.908463in}}%
\pgfpathcurveto{\pgfqpoint{9.469619in}{1.908463in}}{\pgfqpoint{9.464781in}{1.906460in}}{\pgfqpoint{9.461215in}{1.902893in}}%
\pgfpathcurveto{\pgfqpoint{9.457648in}{1.899327in}}{\pgfqpoint{9.455644in}{1.894489in}}{\pgfqpoint{9.455644in}{1.889445in}}%
\pgfpathcurveto{\pgfqpoint{9.455644in}{1.884402in}}{\pgfqpoint{9.457648in}{1.879564in}}{\pgfqpoint{9.461215in}{1.875997in}}%
\pgfpathcurveto{\pgfqpoint{9.464781in}{1.872431in}}{\pgfqpoint{9.469619in}{1.870427in}}{\pgfqpoint{9.474662in}{1.870427in}}%
\pgfpathclose%
\pgfusepath{fill}%
\end{pgfscope}%
\begin{pgfscope}%
\pgfpathrectangle{\pgfqpoint{6.572727in}{0.473000in}}{\pgfqpoint{4.227273in}{3.311000in}}%
\pgfusepath{clip}%
\pgfsetbuttcap%
\pgfsetroundjoin%
\definecolor{currentfill}{rgb}{0.127568,0.566949,0.550556}%
\pgfsetfillcolor{currentfill}%
\pgfsetfillopacity{0.700000}%
\pgfsetlinewidth{0.000000pt}%
\definecolor{currentstroke}{rgb}{0.000000,0.000000,0.000000}%
\pgfsetstrokecolor{currentstroke}%
\pgfsetstrokeopacity{0.700000}%
\pgfsetdash{}{0pt}%
\pgfpathmoveto{\pgfqpoint{7.870710in}{2.781335in}}%
\pgfpathcurveto{\pgfqpoint{7.875753in}{2.781335in}}{\pgfqpoint{7.880591in}{2.783339in}}{\pgfqpoint{7.884158in}{2.786905in}}%
\pgfpathcurveto{\pgfqpoint{7.887724in}{2.790472in}}{\pgfqpoint{7.889728in}{2.795310in}}{\pgfqpoint{7.889728in}{2.800353in}}%
\pgfpathcurveto{\pgfqpoint{7.889728in}{2.805397in}}{\pgfqpoint{7.887724in}{2.810235in}}{\pgfqpoint{7.884158in}{2.813801in}}%
\pgfpathcurveto{\pgfqpoint{7.880591in}{2.817367in}}{\pgfqpoint{7.875753in}{2.819371in}}{\pgfqpoint{7.870710in}{2.819371in}}%
\pgfpathcurveto{\pgfqpoint{7.865666in}{2.819371in}}{\pgfqpoint{7.860828in}{2.817367in}}{\pgfqpoint{7.857262in}{2.813801in}}%
\pgfpathcurveto{\pgfqpoint{7.853695in}{2.810235in}}{\pgfqpoint{7.851692in}{2.805397in}}{\pgfqpoint{7.851692in}{2.800353in}}%
\pgfpathcurveto{\pgfqpoint{7.851692in}{2.795310in}}{\pgfqpoint{7.853695in}{2.790472in}}{\pgfqpoint{7.857262in}{2.786905in}}%
\pgfpathcurveto{\pgfqpoint{7.860828in}{2.783339in}}{\pgfqpoint{7.865666in}{2.781335in}}{\pgfqpoint{7.870710in}{2.781335in}}%
\pgfpathclose%
\pgfusepath{fill}%
\end{pgfscope}%
\begin{pgfscope}%
\pgfpathrectangle{\pgfqpoint{6.572727in}{0.473000in}}{\pgfqpoint{4.227273in}{3.311000in}}%
\pgfusepath{clip}%
\pgfsetbuttcap%
\pgfsetroundjoin%
\definecolor{currentfill}{rgb}{0.127568,0.566949,0.550556}%
\pgfsetfillcolor{currentfill}%
\pgfsetfillopacity{0.700000}%
\pgfsetlinewidth{0.000000pt}%
\definecolor{currentstroke}{rgb}{0.000000,0.000000,0.000000}%
\pgfsetstrokecolor{currentstroke}%
\pgfsetstrokeopacity{0.700000}%
\pgfsetdash{}{0pt}%
\pgfpathmoveto{\pgfqpoint{8.498297in}{3.242922in}}%
\pgfpathcurveto{\pgfqpoint{8.503341in}{3.242922in}}{\pgfqpoint{8.508178in}{3.244926in}}{\pgfqpoint{8.511745in}{3.248492in}}%
\pgfpathcurveto{\pgfqpoint{8.515311in}{3.252059in}}{\pgfqpoint{8.517315in}{3.256896in}}{\pgfqpoint{8.517315in}{3.261940in}}%
\pgfpathcurveto{\pgfqpoint{8.517315in}{3.266984in}}{\pgfqpoint{8.515311in}{3.271821in}}{\pgfqpoint{8.511745in}{3.275388in}}%
\pgfpathcurveto{\pgfqpoint{8.508178in}{3.278954in}}{\pgfqpoint{8.503341in}{3.280958in}}{\pgfqpoint{8.498297in}{3.280958in}}%
\pgfpathcurveto{\pgfqpoint{8.493253in}{3.280958in}}{\pgfqpoint{8.488416in}{3.278954in}}{\pgfqpoint{8.484849in}{3.275388in}}%
\pgfpathcurveto{\pgfqpoint{8.481283in}{3.271821in}}{\pgfqpoint{8.479279in}{3.266984in}}{\pgfqpoint{8.479279in}{3.261940in}}%
\pgfpathcurveto{\pgfqpoint{8.479279in}{3.256896in}}{\pgfqpoint{8.481283in}{3.252059in}}{\pgfqpoint{8.484849in}{3.248492in}}%
\pgfpathcurveto{\pgfqpoint{8.488416in}{3.244926in}}{\pgfqpoint{8.493253in}{3.242922in}}{\pgfqpoint{8.498297in}{3.242922in}}%
\pgfpathclose%
\pgfusepath{fill}%
\end{pgfscope}%
\begin{pgfscope}%
\pgfpathrectangle{\pgfqpoint{6.572727in}{0.473000in}}{\pgfqpoint{4.227273in}{3.311000in}}%
\pgfusepath{clip}%
\pgfsetbuttcap%
\pgfsetroundjoin%
\definecolor{currentfill}{rgb}{0.127568,0.566949,0.550556}%
\pgfsetfillcolor{currentfill}%
\pgfsetfillopacity{0.700000}%
\pgfsetlinewidth{0.000000pt}%
\definecolor{currentstroke}{rgb}{0.000000,0.000000,0.000000}%
\pgfsetstrokecolor{currentstroke}%
\pgfsetstrokeopacity{0.700000}%
\pgfsetdash{}{0pt}%
\pgfpathmoveto{\pgfqpoint{8.251208in}{1.719908in}}%
\pgfpathcurveto{\pgfqpoint{8.256252in}{1.719908in}}{\pgfqpoint{8.261089in}{1.721912in}}{\pgfqpoint{8.264656in}{1.725478in}}%
\pgfpathcurveto{\pgfqpoint{8.268222in}{1.729045in}}{\pgfqpoint{8.270226in}{1.733883in}}{\pgfqpoint{8.270226in}{1.738926in}}%
\pgfpathcurveto{\pgfqpoint{8.270226in}{1.743970in}}{\pgfqpoint{8.268222in}{1.748808in}}{\pgfqpoint{8.264656in}{1.752374in}}%
\pgfpathcurveto{\pgfqpoint{8.261089in}{1.755940in}}{\pgfqpoint{8.256252in}{1.757944in}}{\pgfqpoint{8.251208in}{1.757944in}}%
\pgfpathcurveto{\pgfqpoint{8.246164in}{1.757944in}}{\pgfqpoint{8.241326in}{1.755940in}}{\pgfqpoint{8.237760in}{1.752374in}}%
\pgfpathcurveto{\pgfqpoint{8.234194in}{1.748808in}}{\pgfqpoint{8.232190in}{1.743970in}}{\pgfqpoint{8.232190in}{1.738926in}}%
\pgfpathcurveto{\pgfqpoint{8.232190in}{1.733883in}}{\pgfqpoint{8.234194in}{1.729045in}}{\pgfqpoint{8.237760in}{1.725478in}}%
\pgfpathcurveto{\pgfqpoint{8.241326in}{1.721912in}}{\pgfqpoint{8.246164in}{1.719908in}}{\pgfqpoint{8.251208in}{1.719908in}}%
\pgfpathclose%
\pgfusepath{fill}%
\end{pgfscope}%
\begin{pgfscope}%
\pgfpathrectangle{\pgfqpoint{6.572727in}{0.473000in}}{\pgfqpoint{4.227273in}{3.311000in}}%
\pgfusepath{clip}%
\pgfsetbuttcap%
\pgfsetroundjoin%
\definecolor{currentfill}{rgb}{0.127568,0.566949,0.550556}%
\pgfsetfillcolor{currentfill}%
\pgfsetfillopacity{0.700000}%
\pgfsetlinewidth{0.000000pt}%
\definecolor{currentstroke}{rgb}{0.000000,0.000000,0.000000}%
\pgfsetstrokecolor{currentstroke}%
\pgfsetstrokeopacity{0.700000}%
\pgfsetdash{}{0pt}%
\pgfpathmoveto{\pgfqpoint{8.206880in}{2.944750in}}%
\pgfpathcurveto{\pgfqpoint{8.211923in}{2.944750in}}{\pgfqpoint{8.216761in}{2.946754in}}{\pgfqpoint{8.220328in}{2.950320in}}%
\pgfpathcurveto{\pgfqpoint{8.223894in}{2.953887in}}{\pgfqpoint{8.225898in}{2.958724in}}{\pgfqpoint{8.225898in}{2.963768in}}%
\pgfpathcurveto{\pgfqpoint{8.225898in}{2.968812in}}{\pgfqpoint{8.223894in}{2.973649in}}{\pgfqpoint{8.220328in}{2.977216in}}%
\pgfpathcurveto{\pgfqpoint{8.216761in}{2.980782in}}{\pgfqpoint{8.211923in}{2.982786in}}{\pgfqpoint{8.206880in}{2.982786in}}%
\pgfpathcurveto{\pgfqpoint{8.201836in}{2.982786in}}{\pgfqpoint{8.196998in}{2.980782in}}{\pgfqpoint{8.193432in}{2.977216in}}%
\pgfpathcurveto{\pgfqpoint{8.189865in}{2.973649in}}{\pgfqpoint{8.187862in}{2.968812in}}{\pgfqpoint{8.187862in}{2.963768in}}%
\pgfpathcurveto{\pgfqpoint{8.187862in}{2.958724in}}{\pgfqpoint{8.189865in}{2.953887in}}{\pgfqpoint{8.193432in}{2.950320in}}%
\pgfpathcurveto{\pgfqpoint{8.196998in}{2.946754in}}{\pgfqpoint{8.201836in}{2.944750in}}{\pgfqpoint{8.206880in}{2.944750in}}%
\pgfpathclose%
\pgfusepath{fill}%
\end{pgfscope}%
\begin{pgfscope}%
\pgfpathrectangle{\pgfqpoint{6.572727in}{0.473000in}}{\pgfqpoint{4.227273in}{3.311000in}}%
\pgfusepath{clip}%
\pgfsetbuttcap%
\pgfsetroundjoin%
\definecolor{currentfill}{rgb}{0.127568,0.566949,0.550556}%
\pgfsetfillcolor{currentfill}%
\pgfsetfillopacity{0.700000}%
\pgfsetlinewidth{0.000000pt}%
\definecolor{currentstroke}{rgb}{0.000000,0.000000,0.000000}%
\pgfsetstrokecolor{currentstroke}%
\pgfsetstrokeopacity{0.700000}%
\pgfsetdash{}{0pt}%
\pgfpathmoveto{\pgfqpoint{8.291446in}{2.941870in}}%
\pgfpathcurveto{\pgfqpoint{8.296490in}{2.941870in}}{\pgfqpoint{8.301327in}{2.943874in}}{\pgfqpoint{8.304894in}{2.947440in}}%
\pgfpathcurveto{\pgfqpoint{8.308460in}{2.951006in}}{\pgfqpoint{8.310464in}{2.955844in}}{\pgfqpoint{8.310464in}{2.960888in}}%
\pgfpathcurveto{\pgfqpoint{8.310464in}{2.965931in}}{\pgfqpoint{8.308460in}{2.970769in}}{\pgfqpoint{8.304894in}{2.974336in}}%
\pgfpathcurveto{\pgfqpoint{8.301327in}{2.977902in}}{\pgfqpoint{8.296490in}{2.979906in}}{\pgfqpoint{8.291446in}{2.979906in}}%
\pgfpathcurveto{\pgfqpoint{8.286402in}{2.979906in}}{\pgfqpoint{8.281565in}{2.977902in}}{\pgfqpoint{8.277998in}{2.974336in}}%
\pgfpathcurveto{\pgfqpoint{8.274432in}{2.970769in}}{\pgfqpoint{8.272428in}{2.965931in}}{\pgfqpoint{8.272428in}{2.960888in}}%
\pgfpathcurveto{\pgfqpoint{8.272428in}{2.955844in}}{\pgfqpoint{8.274432in}{2.951006in}}{\pgfqpoint{8.277998in}{2.947440in}}%
\pgfpathcurveto{\pgfqpoint{8.281565in}{2.943874in}}{\pgfqpoint{8.286402in}{2.941870in}}{\pgfqpoint{8.291446in}{2.941870in}}%
\pgfpathclose%
\pgfusepath{fill}%
\end{pgfscope}%
\begin{pgfscope}%
\pgfpathrectangle{\pgfqpoint{6.572727in}{0.473000in}}{\pgfqpoint{4.227273in}{3.311000in}}%
\pgfusepath{clip}%
\pgfsetbuttcap%
\pgfsetroundjoin%
\definecolor{currentfill}{rgb}{0.127568,0.566949,0.550556}%
\pgfsetfillcolor{currentfill}%
\pgfsetfillopacity{0.700000}%
\pgfsetlinewidth{0.000000pt}%
\definecolor{currentstroke}{rgb}{0.000000,0.000000,0.000000}%
\pgfsetstrokecolor{currentstroke}%
\pgfsetstrokeopacity{0.700000}%
\pgfsetdash{}{0pt}%
\pgfpathmoveto{\pgfqpoint{8.238555in}{2.934229in}}%
\pgfpathcurveto{\pgfqpoint{8.243599in}{2.934229in}}{\pgfqpoint{8.248437in}{2.936233in}}{\pgfqpoint{8.252003in}{2.939799in}}%
\pgfpathcurveto{\pgfqpoint{8.255569in}{2.943366in}}{\pgfqpoint{8.257573in}{2.948204in}}{\pgfqpoint{8.257573in}{2.953247in}}%
\pgfpathcurveto{\pgfqpoint{8.257573in}{2.958291in}}{\pgfqpoint{8.255569in}{2.963129in}}{\pgfqpoint{8.252003in}{2.966695in}}%
\pgfpathcurveto{\pgfqpoint{8.248437in}{2.970262in}}{\pgfqpoint{8.243599in}{2.972265in}}{\pgfqpoint{8.238555in}{2.972265in}}%
\pgfpathcurveto{\pgfqpoint{8.233511in}{2.972265in}}{\pgfqpoint{8.228674in}{2.970262in}}{\pgfqpoint{8.225107in}{2.966695in}}%
\pgfpathcurveto{\pgfqpoint{8.221541in}{2.963129in}}{\pgfqpoint{8.219537in}{2.958291in}}{\pgfqpoint{8.219537in}{2.953247in}}%
\pgfpathcurveto{\pgfqpoint{8.219537in}{2.948204in}}{\pgfqpoint{8.221541in}{2.943366in}}{\pgfqpoint{8.225107in}{2.939799in}}%
\pgfpathcurveto{\pgfqpoint{8.228674in}{2.936233in}}{\pgfqpoint{8.233511in}{2.934229in}}{\pgfqpoint{8.238555in}{2.934229in}}%
\pgfpathclose%
\pgfusepath{fill}%
\end{pgfscope}%
\begin{pgfscope}%
\pgfpathrectangle{\pgfqpoint{6.572727in}{0.473000in}}{\pgfqpoint{4.227273in}{3.311000in}}%
\pgfusepath{clip}%
\pgfsetbuttcap%
\pgfsetroundjoin%
\definecolor{currentfill}{rgb}{0.993248,0.906157,0.143936}%
\pgfsetfillcolor{currentfill}%
\pgfsetfillopacity{0.700000}%
\pgfsetlinewidth{0.000000pt}%
\definecolor{currentstroke}{rgb}{0.000000,0.000000,0.000000}%
\pgfsetstrokecolor{currentstroke}%
\pgfsetstrokeopacity{0.700000}%
\pgfsetdash{}{0pt}%
\pgfpathmoveto{\pgfqpoint{9.926121in}{1.452992in}}%
\pgfpathcurveto{\pgfqpoint{9.931165in}{1.452992in}}{\pgfqpoint{9.936003in}{1.454996in}}{\pgfqpoint{9.939569in}{1.458563in}}%
\pgfpathcurveto{\pgfqpoint{9.943136in}{1.462129in}}{\pgfqpoint{9.945140in}{1.466967in}}{\pgfqpoint{9.945140in}{1.472011in}}%
\pgfpathcurveto{\pgfqpoint{9.945140in}{1.477054in}}{\pgfqpoint{9.943136in}{1.481892in}}{\pgfqpoint{9.939569in}{1.485458in}}%
\pgfpathcurveto{\pgfqpoint{9.936003in}{1.489025in}}{\pgfqpoint{9.931165in}{1.491029in}}{\pgfqpoint{9.926121in}{1.491029in}}%
\pgfpathcurveto{\pgfqpoint{9.921078in}{1.491029in}}{\pgfqpoint{9.916240in}{1.489025in}}{\pgfqpoint{9.912674in}{1.485458in}}%
\pgfpathcurveto{\pgfqpoint{9.909107in}{1.481892in}}{\pgfqpoint{9.907103in}{1.477054in}}{\pgfqpoint{9.907103in}{1.472011in}}%
\pgfpathcurveto{\pgfqpoint{9.907103in}{1.466967in}}{\pgfqpoint{9.909107in}{1.462129in}}{\pgfqpoint{9.912674in}{1.458563in}}%
\pgfpathcurveto{\pgfqpoint{9.916240in}{1.454996in}}{\pgfqpoint{9.921078in}{1.452992in}}{\pgfqpoint{9.926121in}{1.452992in}}%
\pgfpathclose%
\pgfusepath{fill}%
\end{pgfscope}%
\begin{pgfscope}%
\pgfpathrectangle{\pgfqpoint{6.572727in}{0.473000in}}{\pgfqpoint{4.227273in}{3.311000in}}%
\pgfusepath{clip}%
\pgfsetbuttcap%
\pgfsetroundjoin%
\definecolor{currentfill}{rgb}{0.127568,0.566949,0.550556}%
\pgfsetfillcolor{currentfill}%
\pgfsetfillopacity{0.700000}%
\pgfsetlinewidth{0.000000pt}%
\definecolor{currentstroke}{rgb}{0.000000,0.000000,0.000000}%
\pgfsetstrokecolor{currentstroke}%
\pgfsetstrokeopacity{0.700000}%
\pgfsetdash{}{0pt}%
\pgfpathmoveto{\pgfqpoint{8.287123in}{2.680536in}}%
\pgfpathcurveto{\pgfqpoint{8.292167in}{2.680536in}}{\pgfqpoint{8.297004in}{2.682540in}}{\pgfqpoint{8.300571in}{2.686106in}}%
\pgfpathcurveto{\pgfqpoint{8.304137in}{2.689673in}}{\pgfqpoint{8.306141in}{2.694510in}}{\pgfqpoint{8.306141in}{2.699554in}}%
\pgfpathcurveto{\pgfqpoint{8.306141in}{2.704598in}}{\pgfqpoint{8.304137in}{2.709435in}}{\pgfqpoint{8.300571in}{2.713002in}}%
\pgfpathcurveto{\pgfqpoint{8.297004in}{2.716568in}}{\pgfqpoint{8.292167in}{2.718572in}}{\pgfqpoint{8.287123in}{2.718572in}}%
\pgfpathcurveto{\pgfqpoint{8.282079in}{2.718572in}}{\pgfqpoint{8.277242in}{2.716568in}}{\pgfqpoint{8.273675in}{2.713002in}}%
\pgfpathcurveto{\pgfqpoint{8.270109in}{2.709435in}}{\pgfqpoint{8.268105in}{2.704598in}}{\pgfqpoint{8.268105in}{2.699554in}}%
\pgfpathcurveto{\pgfqpoint{8.268105in}{2.694510in}}{\pgfqpoint{8.270109in}{2.689673in}}{\pgfqpoint{8.273675in}{2.686106in}}%
\pgfpathcurveto{\pgfqpoint{8.277242in}{2.682540in}}{\pgfqpoint{8.282079in}{2.680536in}}{\pgfqpoint{8.287123in}{2.680536in}}%
\pgfpathclose%
\pgfusepath{fill}%
\end{pgfscope}%
\begin{pgfscope}%
\pgfpathrectangle{\pgfqpoint{6.572727in}{0.473000in}}{\pgfqpoint{4.227273in}{3.311000in}}%
\pgfusepath{clip}%
\pgfsetbuttcap%
\pgfsetroundjoin%
\definecolor{currentfill}{rgb}{0.993248,0.906157,0.143936}%
\pgfsetfillcolor{currentfill}%
\pgfsetfillopacity{0.700000}%
\pgfsetlinewidth{0.000000pt}%
\definecolor{currentstroke}{rgb}{0.000000,0.000000,0.000000}%
\pgfsetstrokecolor{currentstroke}%
\pgfsetstrokeopacity{0.700000}%
\pgfsetdash{}{0pt}%
\pgfpathmoveto{\pgfqpoint{9.393143in}{1.422190in}}%
\pgfpathcurveto{\pgfqpoint{9.398187in}{1.422190in}}{\pgfqpoint{9.403025in}{1.424194in}}{\pgfqpoint{9.406591in}{1.427760in}}%
\pgfpathcurveto{\pgfqpoint{9.410157in}{1.431327in}}{\pgfqpoint{9.412161in}{1.436164in}}{\pgfqpoint{9.412161in}{1.441208in}}%
\pgfpathcurveto{\pgfqpoint{9.412161in}{1.446252in}}{\pgfqpoint{9.410157in}{1.451090in}}{\pgfqpoint{9.406591in}{1.454656in}}%
\pgfpathcurveto{\pgfqpoint{9.403025in}{1.458222in}}{\pgfqpoint{9.398187in}{1.460226in}}{\pgfqpoint{9.393143in}{1.460226in}}%
\pgfpathcurveto{\pgfqpoint{9.388099in}{1.460226in}}{\pgfqpoint{9.383262in}{1.458222in}}{\pgfqpoint{9.379695in}{1.454656in}}%
\pgfpathcurveto{\pgfqpoint{9.376129in}{1.451090in}}{\pgfqpoint{9.374125in}{1.446252in}}{\pgfqpoint{9.374125in}{1.441208in}}%
\pgfpathcurveto{\pgfqpoint{9.374125in}{1.436164in}}{\pgfqpoint{9.376129in}{1.431327in}}{\pgfqpoint{9.379695in}{1.427760in}}%
\pgfpathcurveto{\pgfqpoint{9.383262in}{1.424194in}}{\pgfqpoint{9.388099in}{1.422190in}}{\pgfqpoint{9.393143in}{1.422190in}}%
\pgfpathclose%
\pgfusepath{fill}%
\end{pgfscope}%
\begin{pgfscope}%
\pgfpathrectangle{\pgfqpoint{6.572727in}{0.473000in}}{\pgfqpoint{4.227273in}{3.311000in}}%
\pgfusepath{clip}%
\pgfsetbuttcap%
\pgfsetroundjoin%
\definecolor{currentfill}{rgb}{0.127568,0.566949,0.550556}%
\pgfsetfillcolor{currentfill}%
\pgfsetfillopacity{0.700000}%
\pgfsetlinewidth{0.000000pt}%
\definecolor{currentstroke}{rgb}{0.000000,0.000000,0.000000}%
\pgfsetstrokecolor{currentstroke}%
\pgfsetstrokeopacity{0.700000}%
\pgfsetdash{}{0pt}%
\pgfpathmoveto{\pgfqpoint{8.808060in}{3.298964in}}%
\pgfpathcurveto{\pgfqpoint{8.813104in}{3.298964in}}{\pgfqpoint{8.817941in}{3.300968in}}{\pgfqpoint{8.821508in}{3.304534in}}%
\pgfpathcurveto{\pgfqpoint{8.825074in}{3.308101in}}{\pgfqpoint{8.827078in}{3.312939in}}{\pgfqpoint{8.827078in}{3.317982in}}%
\pgfpathcurveto{\pgfqpoint{8.827078in}{3.323026in}}{\pgfqpoint{8.825074in}{3.327864in}}{\pgfqpoint{8.821508in}{3.331430in}}%
\pgfpathcurveto{\pgfqpoint{8.817941in}{3.334997in}}{\pgfqpoint{8.813104in}{3.337000in}}{\pgfqpoint{8.808060in}{3.337000in}}%
\pgfpathcurveto{\pgfqpoint{8.803016in}{3.337000in}}{\pgfqpoint{8.798178in}{3.334997in}}{\pgfqpoint{8.794612in}{3.331430in}}%
\pgfpathcurveto{\pgfqpoint{8.791046in}{3.327864in}}{\pgfqpoint{8.789042in}{3.323026in}}{\pgfqpoint{8.789042in}{3.317982in}}%
\pgfpathcurveto{\pgfqpoint{8.789042in}{3.312939in}}{\pgfqpoint{8.791046in}{3.308101in}}{\pgfqpoint{8.794612in}{3.304534in}}%
\pgfpathcurveto{\pgfqpoint{8.798178in}{3.300968in}}{\pgfqpoint{8.803016in}{3.298964in}}{\pgfqpoint{8.808060in}{3.298964in}}%
\pgfpathclose%
\pgfusepath{fill}%
\end{pgfscope}%
\begin{pgfscope}%
\pgfpathrectangle{\pgfqpoint{6.572727in}{0.473000in}}{\pgfqpoint{4.227273in}{3.311000in}}%
\pgfusepath{clip}%
\pgfsetbuttcap%
\pgfsetroundjoin%
\definecolor{currentfill}{rgb}{0.267004,0.004874,0.329415}%
\pgfsetfillcolor{currentfill}%
\pgfsetfillopacity{0.700000}%
\pgfsetlinewidth{0.000000pt}%
\definecolor{currentstroke}{rgb}{0.000000,0.000000,0.000000}%
\pgfsetstrokecolor{currentstroke}%
\pgfsetstrokeopacity{0.700000}%
\pgfsetdash{}{0pt}%
\pgfpathmoveto{\pgfqpoint{8.843353in}{0.937602in}}%
\pgfpathcurveto{\pgfqpoint{8.848397in}{0.937602in}}{\pgfqpoint{8.853235in}{0.939605in}}{\pgfqpoint{8.856801in}{0.943172in}}%
\pgfpathcurveto{\pgfqpoint{8.860368in}{0.946738in}}{\pgfqpoint{8.862371in}{0.951576in}}{\pgfqpoint{8.862371in}{0.956620in}}%
\pgfpathcurveto{\pgfqpoint{8.862371in}{0.961663in}}{\pgfqpoint{8.860368in}{0.966501in}}{\pgfqpoint{8.856801in}{0.970068in}}%
\pgfpathcurveto{\pgfqpoint{8.853235in}{0.973634in}}{\pgfqpoint{8.848397in}{0.975638in}}{\pgfqpoint{8.843353in}{0.975638in}}%
\pgfpathcurveto{\pgfqpoint{8.838310in}{0.975638in}}{\pgfqpoint{8.833472in}{0.973634in}}{\pgfqpoint{8.829905in}{0.970068in}}%
\pgfpathcurveto{\pgfqpoint{8.826339in}{0.966501in}}{\pgfqpoint{8.824335in}{0.961663in}}{\pgfqpoint{8.824335in}{0.956620in}}%
\pgfpathcurveto{\pgfqpoint{8.824335in}{0.951576in}}{\pgfqpoint{8.826339in}{0.946738in}}{\pgfqpoint{8.829905in}{0.943172in}}%
\pgfpathcurveto{\pgfqpoint{8.833472in}{0.939605in}}{\pgfqpoint{8.838310in}{0.937602in}}{\pgfqpoint{8.843353in}{0.937602in}}%
\pgfpathclose%
\pgfusepath{fill}%
\end{pgfscope}%
\begin{pgfscope}%
\pgfpathrectangle{\pgfqpoint{6.572727in}{0.473000in}}{\pgfqpoint{4.227273in}{3.311000in}}%
\pgfusepath{clip}%
\pgfsetbuttcap%
\pgfsetroundjoin%
\definecolor{currentfill}{rgb}{0.127568,0.566949,0.550556}%
\pgfsetfillcolor{currentfill}%
\pgfsetfillopacity{0.700000}%
\pgfsetlinewidth{0.000000pt}%
\definecolor{currentstroke}{rgb}{0.000000,0.000000,0.000000}%
\pgfsetstrokecolor{currentstroke}%
\pgfsetstrokeopacity{0.700000}%
\pgfsetdash{}{0pt}%
\pgfpathmoveto{\pgfqpoint{7.805746in}{1.652823in}}%
\pgfpathcurveto{\pgfqpoint{7.810790in}{1.652823in}}{\pgfqpoint{7.815627in}{1.654827in}}{\pgfqpoint{7.819194in}{1.658393in}}%
\pgfpathcurveto{\pgfqpoint{7.822760in}{1.661959in}}{\pgfqpoint{7.824764in}{1.666797in}}{\pgfqpoint{7.824764in}{1.671841in}}%
\pgfpathcurveto{\pgfqpoint{7.824764in}{1.676885in}}{\pgfqpoint{7.822760in}{1.681722in}}{\pgfqpoint{7.819194in}{1.685289in}}%
\pgfpathcurveto{\pgfqpoint{7.815627in}{1.688855in}}{\pgfqpoint{7.810790in}{1.690859in}}{\pgfqpoint{7.805746in}{1.690859in}}%
\pgfpathcurveto{\pgfqpoint{7.800702in}{1.690859in}}{\pgfqpoint{7.795864in}{1.688855in}}{\pgfqpoint{7.792298in}{1.685289in}}%
\pgfpathcurveto{\pgfqpoint{7.788732in}{1.681722in}}{\pgfqpoint{7.786728in}{1.676885in}}{\pgfqpoint{7.786728in}{1.671841in}}%
\pgfpathcurveto{\pgfqpoint{7.786728in}{1.666797in}}{\pgfqpoint{7.788732in}{1.661959in}}{\pgfqpoint{7.792298in}{1.658393in}}%
\pgfpathcurveto{\pgfqpoint{7.795864in}{1.654827in}}{\pgfqpoint{7.800702in}{1.652823in}}{\pgfqpoint{7.805746in}{1.652823in}}%
\pgfpathclose%
\pgfusepath{fill}%
\end{pgfscope}%
\begin{pgfscope}%
\pgfpathrectangle{\pgfqpoint{6.572727in}{0.473000in}}{\pgfqpoint{4.227273in}{3.311000in}}%
\pgfusepath{clip}%
\pgfsetbuttcap%
\pgfsetroundjoin%
\definecolor{currentfill}{rgb}{0.993248,0.906157,0.143936}%
\pgfsetfillcolor{currentfill}%
\pgfsetfillopacity{0.700000}%
\pgfsetlinewidth{0.000000pt}%
\definecolor{currentstroke}{rgb}{0.000000,0.000000,0.000000}%
\pgfsetstrokecolor{currentstroke}%
\pgfsetstrokeopacity{0.700000}%
\pgfsetdash{}{0pt}%
\pgfpathmoveto{\pgfqpoint{10.089422in}{1.315312in}}%
\pgfpathcurveto{\pgfqpoint{10.094466in}{1.315312in}}{\pgfqpoint{10.099304in}{1.317316in}}{\pgfqpoint{10.102870in}{1.320883in}}%
\pgfpathcurveto{\pgfqpoint{10.106437in}{1.324449in}}{\pgfqpoint{10.108441in}{1.329287in}}{\pgfqpoint{10.108441in}{1.334331in}}%
\pgfpathcurveto{\pgfqpoint{10.108441in}{1.339374in}}{\pgfqpoint{10.106437in}{1.344212in}}{\pgfqpoint{10.102870in}{1.347778in}}%
\pgfpathcurveto{\pgfqpoint{10.099304in}{1.351345in}}{\pgfqpoint{10.094466in}{1.353349in}}{\pgfqpoint{10.089422in}{1.353349in}}%
\pgfpathcurveto{\pgfqpoint{10.084379in}{1.353349in}}{\pgfqpoint{10.079541in}{1.351345in}}{\pgfqpoint{10.075975in}{1.347778in}}%
\pgfpathcurveto{\pgfqpoint{10.072408in}{1.344212in}}{\pgfqpoint{10.070404in}{1.339374in}}{\pgfqpoint{10.070404in}{1.334331in}}%
\pgfpathcurveto{\pgfqpoint{10.070404in}{1.329287in}}{\pgfqpoint{10.072408in}{1.324449in}}{\pgfqpoint{10.075975in}{1.320883in}}%
\pgfpathcurveto{\pgfqpoint{10.079541in}{1.317316in}}{\pgfqpoint{10.084379in}{1.315312in}}{\pgfqpoint{10.089422in}{1.315312in}}%
\pgfpathclose%
\pgfusepath{fill}%
\end{pgfscope}%
\begin{pgfscope}%
\pgfpathrectangle{\pgfqpoint{6.572727in}{0.473000in}}{\pgfqpoint{4.227273in}{3.311000in}}%
\pgfusepath{clip}%
\pgfsetbuttcap%
\pgfsetroundjoin%
\definecolor{currentfill}{rgb}{0.127568,0.566949,0.550556}%
\pgfsetfillcolor{currentfill}%
\pgfsetfillopacity{0.700000}%
\pgfsetlinewidth{0.000000pt}%
\definecolor{currentstroke}{rgb}{0.000000,0.000000,0.000000}%
\pgfsetstrokecolor{currentstroke}%
\pgfsetstrokeopacity{0.700000}%
\pgfsetdash{}{0pt}%
\pgfpathmoveto{\pgfqpoint{8.485526in}{1.547854in}}%
\pgfpathcurveto{\pgfqpoint{8.490569in}{1.547854in}}{\pgfqpoint{8.495407in}{1.549858in}}{\pgfqpoint{8.498974in}{1.553425in}}%
\pgfpathcurveto{\pgfqpoint{8.502540in}{1.556991in}}{\pgfqpoint{8.504544in}{1.561829in}}{\pgfqpoint{8.504544in}{1.566872in}}%
\pgfpathcurveto{\pgfqpoint{8.504544in}{1.571916in}}{\pgfqpoint{8.502540in}{1.576754in}}{\pgfqpoint{8.498974in}{1.580320in}}%
\pgfpathcurveto{\pgfqpoint{8.495407in}{1.583887in}}{\pgfqpoint{8.490569in}{1.585891in}}{\pgfqpoint{8.485526in}{1.585891in}}%
\pgfpathcurveto{\pgfqpoint{8.480482in}{1.585891in}}{\pgfqpoint{8.475644in}{1.583887in}}{\pgfqpoint{8.472078in}{1.580320in}}%
\pgfpathcurveto{\pgfqpoint{8.468512in}{1.576754in}}{\pgfqpoint{8.466508in}{1.571916in}}{\pgfqpoint{8.466508in}{1.566872in}}%
\pgfpathcurveto{\pgfqpoint{8.466508in}{1.561829in}}{\pgfqpoint{8.468512in}{1.556991in}}{\pgfqpoint{8.472078in}{1.553425in}}%
\pgfpathcurveto{\pgfqpoint{8.475644in}{1.549858in}}{\pgfqpoint{8.480482in}{1.547854in}}{\pgfqpoint{8.485526in}{1.547854in}}%
\pgfpathclose%
\pgfusepath{fill}%
\end{pgfscope}%
\begin{pgfscope}%
\pgfpathrectangle{\pgfqpoint{6.572727in}{0.473000in}}{\pgfqpoint{4.227273in}{3.311000in}}%
\pgfusepath{clip}%
\pgfsetbuttcap%
\pgfsetroundjoin%
\definecolor{currentfill}{rgb}{0.127568,0.566949,0.550556}%
\pgfsetfillcolor{currentfill}%
\pgfsetfillopacity{0.700000}%
\pgfsetlinewidth{0.000000pt}%
\definecolor{currentstroke}{rgb}{0.000000,0.000000,0.000000}%
\pgfsetstrokecolor{currentstroke}%
\pgfsetstrokeopacity{0.700000}%
\pgfsetdash{}{0pt}%
\pgfpathmoveto{\pgfqpoint{7.997013in}{2.818031in}}%
\pgfpathcurveto{\pgfqpoint{8.002056in}{2.818031in}}{\pgfqpoint{8.006894in}{2.820035in}}{\pgfqpoint{8.010460in}{2.823601in}}%
\pgfpathcurveto{\pgfqpoint{8.014027in}{2.827168in}}{\pgfqpoint{8.016031in}{2.832005in}}{\pgfqpoint{8.016031in}{2.837049in}}%
\pgfpathcurveto{\pgfqpoint{8.016031in}{2.842093in}}{\pgfqpoint{8.014027in}{2.846931in}}{\pgfqpoint{8.010460in}{2.850497in}}%
\pgfpathcurveto{\pgfqpoint{8.006894in}{2.854063in}}{\pgfqpoint{8.002056in}{2.856067in}}{\pgfqpoint{7.997013in}{2.856067in}}%
\pgfpathcurveto{\pgfqpoint{7.991969in}{2.856067in}}{\pgfqpoint{7.987131in}{2.854063in}}{\pgfqpoint{7.983565in}{2.850497in}}%
\pgfpathcurveto{\pgfqpoint{7.979998in}{2.846931in}}{\pgfqpoint{7.977994in}{2.842093in}}{\pgfqpoint{7.977994in}{2.837049in}}%
\pgfpathcurveto{\pgfqpoint{7.977994in}{2.832005in}}{\pgfqpoint{7.979998in}{2.827168in}}{\pgfqpoint{7.983565in}{2.823601in}}%
\pgfpathcurveto{\pgfqpoint{7.987131in}{2.820035in}}{\pgfqpoint{7.991969in}{2.818031in}}{\pgfqpoint{7.997013in}{2.818031in}}%
\pgfpathclose%
\pgfusepath{fill}%
\end{pgfscope}%
\begin{pgfscope}%
\pgfpathrectangle{\pgfqpoint{6.572727in}{0.473000in}}{\pgfqpoint{4.227273in}{3.311000in}}%
\pgfusepath{clip}%
\pgfsetbuttcap%
\pgfsetroundjoin%
\definecolor{currentfill}{rgb}{0.127568,0.566949,0.550556}%
\pgfsetfillcolor{currentfill}%
\pgfsetfillopacity{0.700000}%
\pgfsetlinewidth{0.000000pt}%
\definecolor{currentstroke}{rgb}{0.000000,0.000000,0.000000}%
\pgfsetstrokecolor{currentstroke}%
\pgfsetstrokeopacity{0.700000}%
\pgfsetdash{}{0pt}%
\pgfpathmoveto{\pgfqpoint{7.491708in}{1.633355in}}%
\pgfpathcurveto{\pgfqpoint{7.496752in}{1.633355in}}{\pgfqpoint{7.501590in}{1.635359in}}{\pgfqpoint{7.505156in}{1.638925in}}%
\pgfpathcurveto{\pgfqpoint{7.508723in}{1.642492in}}{\pgfqpoint{7.510727in}{1.647329in}}{\pgfqpoint{7.510727in}{1.652373in}}%
\pgfpathcurveto{\pgfqpoint{7.510727in}{1.657417in}}{\pgfqpoint{7.508723in}{1.662255in}}{\pgfqpoint{7.505156in}{1.665821in}}%
\pgfpathcurveto{\pgfqpoint{7.501590in}{1.669387in}}{\pgfqpoint{7.496752in}{1.671391in}}{\pgfqpoint{7.491708in}{1.671391in}}%
\pgfpathcurveto{\pgfqpoint{7.486665in}{1.671391in}}{\pgfqpoint{7.481827in}{1.669387in}}{\pgfqpoint{7.478261in}{1.665821in}}%
\pgfpathcurveto{\pgfqpoint{7.474694in}{1.662255in}}{\pgfqpoint{7.472690in}{1.657417in}}{\pgfqpoint{7.472690in}{1.652373in}}%
\pgfpathcurveto{\pgfqpoint{7.472690in}{1.647329in}}{\pgfqpoint{7.474694in}{1.642492in}}{\pgfqpoint{7.478261in}{1.638925in}}%
\pgfpathcurveto{\pgfqpoint{7.481827in}{1.635359in}}{\pgfqpoint{7.486665in}{1.633355in}}{\pgfqpoint{7.491708in}{1.633355in}}%
\pgfpathclose%
\pgfusepath{fill}%
\end{pgfscope}%
\begin{pgfscope}%
\pgfpathrectangle{\pgfqpoint{6.572727in}{0.473000in}}{\pgfqpoint{4.227273in}{3.311000in}}%
\pgfusepath{clip}%
\pgfsetbuttcap%
\pgfsetroundjoin%
\definecolor{currentfill}{rgb}{0.127568,0.566949,0.550556}%
\pgfsetfillcolor{currentfill}%
\pgfsetfillopacity{0.700000}%
\pgfsetlinewidth{0.000000pt}%
\definecolor{currentstroke}{rgb}{0.000000,0.000000,0.000000}%
\pgfsetstrokecolor{currentstroke}%
\pgfsetstrokeopacity{0.700000}%
\pgfsetdash{}{0pt}%
\pgfpathmoveto{\pgfqpoint{7.995655in}{3.241583in}}%
\pgfpathcurveto{\pgfqpoint{8.000699in}{3.241583in}}{\pgfqpoint{8.005537in}{3.243587in}}{\pgfqpoint{8.009103in}{3.247154in}}%
\pgfpathcurveto{\pgfqpoint{8.012670in}{3.250720in}}{\pgfqpoint{8.014674in}{3.255558in}}{\pgfqpoint{8.014674in}{3.260602in}}%
\pgfpathcurveto{\pgfqpoint{8.014674in}{3.265645in}}{\pgfqpoint{8.012670in}{3.270483in}}{\pgfqpoint{8.009103in}{3.274049in}}%
\pgfpathcurveto{\pgfqpoint{8.005537in}{3.277616in}}{\pgfqpoint{8.000699in}{3.279620in}}{\pgfqpoint{7.995655in}{3.279620in}}%
\pgfpathcurveto{\pgfqpoint{7.990612in}{3.279620in}}{\pgfqpoint{7.985774in}{3.277616in}}{\pgfqpoint{7.982208in}{3.274049in}}%
\pgfpathcurveto{\pgfqpoint{7.978641in}{3.270483in}}{\pgfqpoint{7.976637in}{3.265645in}}{\pgfqpoint{7.976637in}{3.260602in}}%
\pgfpathcurveto{\pgfqpoint{7.976637in}{3.255558in}}{\pgfqpoint{7.978641in}{3.250720in}}{\pgfqpoint{7.982208in}{3.247154in}}%
\pgfpathcurveto{\pgfqpoint{7.985774in}{3.243587in}}{\pgfqpoint{7.990612in}{3.241583in}}{\pgfqpoint{7.995655in}{3.241583in}}%
\pgfpathclose%
\pgfusepath{fill}%
\end{pgfscope}%
\begin{pgfscope}%
\pgfpathrectangle{\pgfqpoint{6.572727in}{0.473000in}}{\pgfqpoint{4.227273in}{3.311000in}}%
\pgfusepath{clip}%
\pgfsetbuttcap%
\pgfsetroundjoin%
\definecolor{currentfill}{rgb}{0.127568,0.566949,0.550556}%
\pgfsetfillcolor{currentfill}%
\pgfsetfillopacity{0.700000}%
\pgfsetlinewidth{0.000000pt}%
\definecolor{currentstroke}{rgb}{0.000000,0.000000,0.000000}%
\pgfsetstrokecolor{currentstroke}%
\pgfsetstrokeopacity{0.700000}%
\pgfsetdash{}{0pt}%
\pgfpathmoveto{\pgfqpoint{7.800085in}{1.028551in}}%
\pgfpathcurveto{\pgfqpoint{7.805129in}{1.028551in}}{\pgfqpoint{7.809967in}{1.030555in}}{\pgfqpoint{7.813533in}{1.034121in}}%
\pgfpathcurveto{\pgfqpoint{7.817099in}{1.037688in}}{\pgfqpoint{7.819103in}{1.042526in}}{\pgfqpoint{7.819103in}{1.047569in}}%
\pgfpathcurveto{\pgfqpoint{7.819103in}{1.052613in}}{\pgfqpoint{7.817099in}{1.057451in}}{\pgfqpoint{7.813533in}{1.061017in}}%
\pgfpathcurveto{\pgfqpoint{7.809967in}{1.064584in}}{\pgfqpoint{7.805129in}{1.066587in}}{\pgfqpoint{7.800085in}{1.066587in}}%
\pgfpathcurveto{\pgfqpoint{7.795042in}{1.066587in}}{\pgfqpoint{7.790204in}{1.064584in}}{\pgfqpoint{7.786637in}{1.061017in}}%
\pgfpathcurveto{\pgfqpoint{7.783071in}{1.057451in}}{\pgfqpoint{7.781067in}{1.052613in}}{\pgfqpoint{7.781067in}{1.047569in}}%
\pgfpathcurveto{\pgfqpoint{7.781067in}{1.042526in}}{\pgfqpoint{7.783071in}{1.037688in}}{\pgfqpoint{7.786637in}{1.034121in}}%
\pgfpathcurveto{\pgfqpoint{7.790204in}{1.030555in}}{\pgfqpoint{7.795042in}{1.028551in}}{\pgfqpoint{7.800085in}{1.028551in}}%
\pgfpathclose%
\pgfusepath{fill}%
\end{pgfscope}%
\begin{pgfscope}%
\pgfpathrectangle{\pgfqpoint{6.572727in}{0.473000in}}{\pgfqpoint{4.227273in}{3.311000in}}%
\pgfusepath{clip}%
\pgfsetbuttcap%
\pgfsetroundjoin%
\definecolor{currentfill}{rgb}{0.127568,0.566949,0.550556}%
\pgfsetfillcolor{currentfill}%
\pgfsetfillopacity{0.700000}%
\pgfsetlinewidth{0.000000pt}%
\definecolor{currentstroke}{rgb}{0.000000,0.000000,0.000000}%
\pgfsetstrokecolor{currentstroke}%
\pgfsetstrokeopacity{0.700000}%
\pgfsetdash{}{0pt}%
\pgfpathmoveto{\pgfqpoint{7.781376in}{1.556645in}}%
\pgfpathcurveto{\pgfqpoint{7.786420in}{1.556645in}}{\pgfqpoint{7.791258in}{1.558649in}}{\pgfqpoint{7.794824in}{1.562215in}}%
\pgfpathcurveto{\pgfqpoint{7.798391in}{1.565782in}}{\pgfqpoint{7.800395in}{1.570620in}}{\pgfqpoint{7.800395in}{1.575663in}}%
\pgfpathcurveto{\pgfqpoint{7.800395in}{1.580707in}}{\pgfqpoint{7.798391in}{1.585545in}}{\pgfqpoint{7.794824in}{1.589111in}}%
\pgfpathcurveto{\pgfqpoint{7.791258in}{1.592678in}}{\pgfqpoint{7.786420in}{1.594681in}}{\pgfqpoint{7.781376in}{1.594681in}}%
\pgfpathcurveto{\pgfqpoint{7.776333in}{1.594681in}}{\pgfqpoint{7.771495in}{1.592678in}}{\pgfqpoint{7.767929in}{1.589111in}}%
\pgfpathcurveto{\pgfqpoint{7.764362in}{1.585545in}}{\pgfqpoint{7.762358in}{1.580707in}}{\pgfqpoint{7.762358in}{1.575663in}}%
\pgfpathcurveto{\pgfqpoint{7.762358in}{1.570620in}}{\pgfqpoint{7.764362in}{1.565782in}}{\pgfqpoint{7.767929in}{1.562215in}}%
\pgfpathcurveto{\pgfqpoint{7.771495in}{1.558649in}}{\pgfqpoint{7.776333in}{1.556645in}}{\pgfqpoint{7.781376in}{1.556645in}}%
\pgfpathclose%
\pgfusepath{fill}%
\end{pgfscope}%
\begin{pgfscope}%
\pgfpathrectangle{\pgfqpoint{6.572727in}{0.473000in}}{\pgfqpoint{4.227273in}{3.311000in}}%
\pgfusepath{clip}%
\pgfsetbuttcap%
\pgfsetroundjoin%
\definecolor{currentfill}{rgb}{0.127568,0.566949,0.550556}%
\pgfsetfillcolor{currentfill}%
\pgfsetfillopacity{0.700000}%
\pgfsetlinewidth{0.000000pt}%
\definecolor{currentstroke}{rgb}{0.000000,0.000000,0.000000}%
\pgfsetstrokecolor{currentstroke}%
\pgfsetstrokeopacity{0.700000}%
\pgfsetdash{}{0pt}%
\pgfpathmoveto{\pgfqpoint{8.579273in}{2.913509in}}%
\pgfpathcurveto{\pgfqpoint{8.584317in}{2.913509in}}{\pgfqpoint{8.589154in}{2.915513in}}{\pgfqpoint{8.592721in}{2.919079in}}%
\pgfpathcurveto{\pgfqpoint{8.596287in}{2.922646in}}{\pgfqpoint{8.598291in}{2.927483in}}{\pgfqpoint{8.598291in}{2.932527in}}%
\pgfpathcurveto{\pgfqpoint{8.598291in}{2.937571in}}{\pgfqpoint{8.596287in}{2.942409in}}{\pgfqpoint{8.592721in}{2.945975in}}%
\pgfpathcurveto{\pgfqpoint{8.589154in}{2.949541in}}{\pgfqpoint{8.584317in}{2.951545in}}{\pgfqpoint{8.579273in}{2.951545in}}%
\pgfpathcurveto{\pgfqpoint{8.574229in}{2.951545in}}{\pgfqpoint{8.569392in}{2.949541in}}{\pgfqpoint{8.565825in}{2.945975in}}%
\pgfpathcurveto{\pgfqpoint{8.562259in}{2.942409in}}{\pgfqpoint{8.560255in}{2.937571in}}{\pgfqpoint{8.560255in}{2.932527in}}%
\pgfpathcurveto{\pgfqpoint{8.560255in}{2.927483in}}{\pgfqpoint{8.562259in}{2.922646in}}{\pgfqpoint{8.565825in}{2.919079in}}%
\pgfpathcurveto{\pgfqpoint{8.569392in}{2.915513in}}{\pgfqpoint{8.574229in}{2.913509in}}{\pgfqpoint{8.579273in}{2.913509in}}%
\pgfpathclose%
\pgfusepath{fill}%
\end{pgfscope}%
\begin{pgfscope}%
\pgfpathrectangle{\pgfqpoint{6.572727in}{0.473000in}}{\pgfqpoint{4.227273in}{3.311000in}}%
\pgfusepath{clip}%
\pgfsetbuttcap%
\pgfsetroundjoin%
\definecolor{currentfill}{rgb}{0.993248,0.906157,0.143936}%
\pgfsetfillcolor{currentfill}%
\pgfsetfillopacity{0.700000}%
\pgfsetlinewidth{0.000000pt}%
\definecolor{currentstroke}{rgb}{0.000000,0.000000,0.000000}%
\pgfsetstrokecolor{currentstroke}%
\pgfsetstrokeopacity{0.700000}%
\pgfsetdash{}{0pt}%
\pgfpathmoveto{\pgfqpoint{9.428551in}{1.516665in}}%
\pgfpathcurveto{\pgfqpoint{9.433595in}{1.516665in}}{\pgfqpoint{9.438433in}{1.518669in}}{\pgfqpoint{9.441999in}{1.522235in}}%
\pgfpathcurveto{\pgfqpoint{9.445566in}{1.525802in}}{\pgfqpoint{9.447570in}{1.530639in}}{\pgfqpoint{9.447570in}{1.535683in}}%
\pgfpathcurveto{\pgfqpoint{9.447570in}{1.540727in}}{\pgfqpoint{9.445566in}{1.545564in}}{\pgfqpoint{9.441999in}{1.549131in}}%
\pgfpathcurveto{\pgfqpoint{9.438433in}{1.552697in}}{\pgfqpoint{9.433595in}{1.554701in}}{\pgfqpoint{9.428551in}{1.554701in}}%
\pgfpathcurveto{\pgfqpoint{9.423508in}{1.554701in}}{\pgfqpoint{9.418670in}{1.552697in}}{\pgfqpoint{9.415104in}{1.549131in}}%
\pgfpathcurveto{\pgfqpoint{9.411537in}{1.545564in}}{\pgfqpoint{9.409533in}{1.540727in}}{\pgfqpoint{9.409533in}{1.535683in}}%
\pgfpathcurveto{\pgfqpoint{9.409533in}{1.530639in}}{\pgfqpoint{9.411537in}{1.525802in}}{\pgfqpoint{9.415104in}{1.522235in}}%
\pgfpathcurveto{\pgfqpoint{9.418670in}{1.518669in}}{\pgfqpoint{9.423508in}{1.516665in}}{\pgfqpoint{9.428551in}{1.516665in}}%
\pgfpathclose%
\pgfusepath{fill}%
\end{pgfscope}%
\begin{pgfscope}%
\pgfpathrectangle{\pgfqpoint{6.572727in}{0.473000in}}{\pgfqpoint{4.227273in}{3.311000in}}%
\pgfusepath{clip}%
\pgfsetbuttcap%
\pgfsetroundjoin%
\definecolor{currentfill}{rgb}{0.127568,0.566949,0.550556}%
\pgfsetfillcolor{currentfill}%
\pgfsetfillopacity{0.700000}%
\pgfsetlinewidth{0.000000pt}%
\definecolor{currentstroke}{rgb}{0.000000,0.000000,0.000000}%
\pgfsetstrokecolor{currentstroke}%
\pgfsetstrokeopacity{0.700000}%
\pgfsetdash{}{0pt}%
\pgfpathmoveto{\pgfqpoint{7.940758in}{1.735961in}}%
\pgfpathcurveto{\pgfqpoint{7.945801in}{1.735961in}}{\pgfqpoint{7.950639in}{1.737965in}}{\pgfqpoint{7.954206in}{1.741531in}}%
\pgfpathcurveto{\pgfqpoint{7.957772in}{1.745098in}}{\pgfqpoint{7.959776in}{1.749935in}}{\pgfqpoint{7.959776in}{1.754979in}}%
\pgfpathcurveto{\pgfqpoint{7.959776in}{1.760023in}}{\pgfqpoint{7.957772in}{1.764860in}}{\pgfqpoint{7.954206in}{1.768427in}}%
\pgfpathcurveto{\pgfqpoint{7.950639in}{1.771993in}}{\pgfqpoint{7.945801in}{1.773997in}}{\pgfqpoint{7.940758in}{1.773997in}}%
\pgfpathcurveto{\pgfqpoint{7.935714in}{1.773997in}}{\pgfqpoint{7.930876in}{1.771993in}}{\pgfqpoint{7.927310in}{1.768427in}}%
\pgfpathcurveto{\pgfqpoint{7.923743in}{1.764860in}}{\pgfqpoint{7.921740in}{1.760023in}}{\pgfqpoint{7.921740in}{1.754979in}}%
\pgfpathcurveto{\pgfqpoint{7.921740in}{1.749935in}}{\pgfqpoint{7.923743in}{1.745098in}}{\pgfqpoint{7.927310in}{1.741531in}}%
\pgfpathcurveto{\pgfqpoint{7.930876in}{1.737965in}}{\pgfqpoint{7.935714in}{1.735961in}}{\pgfqpoint{7.940758in}{1.735961in}}%
\pgfpathclose%
\pgfusepath{fill}%
\end{pgfscope}%
\begin{pgfscope}%
\pgfpathrectangle{\pgfqpoint{6.572727in}{0.473000in}}{\pgfqpoint{4.227273in}{3.311000in}}%
\pgfusepath{clip}%
\pgfsetbuttcap%
\pgfsetroundjoin%
\definecolor{currentfill}{rgb}{0.993248,0.906157,0.143936}%
\pgfsetfillcolor{currentfill}%
\pgfsetfillopacity{0.700000}%
\pgfsetlinewidth{0.000000pt}%
\definecolor{currentstroke}{rgb}{0.000000,0.000000,0.000000}%
\pgfsetstrokecolor{currentstroke}%
\pgfsetstrokeopacity{0.700000}%
\pgfsetdash{}{0pt}%
\pgfpathmoveto{\pgfqpoint{9.011288in}{1.510454in}}%
\pgfpathcurveto{\pgfqpoint{9.016332in}{1.510454in}}{\pgfqpoint{9.021170in}{1.512457in}}{\pgfqpoint{9.024736in}{1.516024in}}%
\pgfpathcurveto{\pgfqpoint{9.028303in}{1.519590in}}{\pgfqpoint{9.030307in}{1.524428in}}{\pgfqpoint{9.030307in}{1.529472in}}%
\pgfpathcurveto{\pgfqpoint{9.030307in}{1.534515in}}{\pgfqpoint{9.028303in}{1.539353in}}{\pgfqpoint{9.024736in}{1.542920in}}%
\pgfpathcurveto{\pgfqpoint{9.021170in}{1.546486in}}{\pgfqpoint{9.016332in}{1.548490in}}{\pgfqpoint{9.011288in}{1.548490in}}%
\pgfpathcurveto{\pgfqpoint{9.006245in}{1.548490in}}{\pgfqpoint{9.001407in}{1.546486in}}{\pgfqpoint{8.997841in}{1.542920in}}%
\pgfpathcurveto{\pgfqpoint{8.994274in}{1.539353in}}{\pgfqpoint{8.992270in}{1.534515in}}{\pgfqpoint{8.992270in}{1.529472in}}%
\pgfpathcurveto{\pgfqpoint{8.992270in}{1.524428in}}{\pgfqpoint{8.994274in}{1.519590in}}{\pgfqpoint{8.997841in}{1.516024in}}%
\pgfpathcurveto{\pgfqpoint{9.001407in}{1.512457in}}{\pgfqpoint{9.006245in}{1.510454in}}{\pgfqpoint{9.011288in}{1.510454in}}%
\pgfpathclose%
\pgfusepath{fill}%
\end{pgfscope}%
\begin{pgfscope}%
\pgfpathrectangle{\pgfqpoint{6.572727in}{0.473000in}}{\pgfqpoint{4.227273in}{3.311000in}}%
\pgfusepath{clip}%
\pgfsetbuttcap%
\pgfsetroundjoin%
\definecolor{currentfill}{rgb}{0.993248,0.906157,0.143936}%
\pgfsetfillcolor{currentfill}%
\pgfsetfillopacity{0.700000}%
\pgfsetlinewidth{0.000000pt}%
\definecolor{currentstroke}{rgb}{0.000000,0.000000,0.000000}%
\pgfsetstrokecolor{currentstroke}%
\pgfsetstrokeopacity{0.700000}%
\pgfsetdash{}{0pt}%
\pgfpathmoveto{\pgfqpoint{10.085005in}{0.854703in}}%
\pgfpathcurveto{\pgfqpoint{10.090049in}{0.854703in}}{\pgfqpoint{10.094887in}{0.856706in}}{\pgfqpoint{10.098453in}{0.860273in}}%
\pgfpathcurveto{\pgfqpoint{10.102019in}{0.863839in}}{\pgfqpoint{10.104023in}{0.868677in}}{\pgfqpoint{10.104023in}{0.873721in}}%
\pgfpathcurveto{\pgfqpoint{10.104023in}{0.878764in}}{\pgfqpoint{10.102019in}{0.883602in}}{\pgfqpoint{10.098453in}{0.887169in}}%
\pgfpathcurveto{\pgfqpoint{10.094887in}{0.890735in}}{\pgfqpoint{10.090049in}{0.892739in}}{\pgfqpoint{10.085005in}{0.892739in}}%
\pgfpathcurveto{\pgfqpoint{10.079962in}{0.892739in}}{\pgfqpoint{10.075124in}{0.890735in}}{\pgfqpoint{10.071557in}{0.887169in}}%
\pgfpathcurveto{\pgfqpoint{10.067991in}{0.883602in}}{\pgfqpoint{10.065987in}{0.878764in}}{\pgfqpoint{10.065987in}{0.873721in}}%
\pgfpathcurveto{\pgfqpoint{10.065987in}{0.868677in}}{\pgfqpoint{10.067991in}{0.863839in}}{\pgfqpoint{10.071557in}{0.860273in}}%
\pgfpathcurveto{\pgfqpoint{10.075124in}{0.856706in}}{\pgfqpoint{10.079962in}{0.854703in}}{\pgfqpoint{10.085005in}{0.854703in}}%
\pgfpathclose%
\pgfusepath{fill}%
\end{pgfscope}%
\begin{pgfscope}%
\pgfpathrectangle{\pgfqpoint{6.572727in}{0.473000in}}{\pgfqpoint{4.227273in}{3.311000in}}%
\pgfusepath{clip}%
\pgfsetbuttcap%
\pgfsetroundjoin%
\definecolor{currentfill}{rgb}{0.127568,0.566949,0.550556}%
\pgfsetfillcolor{currentfill}%
\pgfsetfillopacity{0.700000}%
\pgfsetlinewidth{0.000000pt}%
\definecolor{currentstroke}{rgb}{0.000000,0.000000,0.000000}%
\pgfsetstrokecolor{currentstroke}%
\pgfsetstrokeopacity{0.700000}%
\pgfsetdash{}{0pt}%
\pgfpathmoveto{\pgfqpoint{8.170251in}{2.770620in}}%
\pgfpathcurveto{\pgfqpoint{8.175295in}{2.770620in}}{\pgfqpoint{8.180133in}{2.772624in}}{\pgfqpoint{8.183699in}{2.776190in}}%
\pgfpathcurveto{\pgfqpoint{8.187265in}{2.779757in}}{\pgfqpoint{8.189269in}{2.784594in}}{\pgfqpoint{8.189269in}{2.789638in}}%
\pgfpathcurveto{\pgfqpoint{8.189269in}{2.794682in}}{\pgfqpoint{8.187265in}{2.799520in}}{\pgfqpoint{8.183699in}{2.803086in}}%
\pgfpathcurveto{\pgfqpoint{8.180133in}{2.806652in}}{\pgfqpoint{8.175295in}{2.808656in}}{\pgfqpoint{8.170251in}{2.808656in}}%
\pgfpathcurveto{\pgfqpoint{8.165207in}{2.808656in}}{\pgfqpoint{8.160370in}{2.806652in}}{\pgfqpoint{8.156803in}{2.803086in}}%
\pgfpathcurveto{\pgfqpoint{8.153237in}{2.799520in}}{\pgfqpoint{8.151233in}{2.794682in}}{\pgfqpoint{8.151233in}{2.789638in}}%
\pgfpathcurveto{\pgfqpoint{8.151233in}{2.784594in}}{\pgfqpoint{8.153237in}{2.779757in}}{\pgfqpoint{8.156803in}{2.776190in}}%
\pgfpathcurveto{\pgfqpoint{8.160370in}{2.772624in}}{\pgfqpoint{8.165207in}{2.770620in}}{\pgfqpoint{8.170251in}{2.770620in}}%
\pgfpathclose%
\pgfusepath{fill}%
\end{pgfscope}%
\begin{pgfscope}%
\pgfpathrectangle{\pgfqpoint{6.572727in}{0.473000in}}{\pgfqpoint{4.227273in}{3.311000in}}%
\pgfusepath{clip}%
\pgfsetbuttcap%
\pgfsetroundjoin%
\definecolor{currentfill}{rgb}{0.127568,0.566949,0.550556}%
\pgfsetfillcolor{currentfill}%
\pgfsetfillopacity{0.700000}%
\pgfsetlinewidth{0.000000pt}%
\definecolor{currentstroke}{rgb}{0.000000,0.000000,0.000000}%
\pgfsetstrokecolor{currentstroke}%
\pgfsetstrokeopacity{0.700000}%
\pgfsetdash{}{0pt}%
\pgfpathmoveto{\pgfqpoint{7.906933in}{1.504196in}}%
\pgfpathcurveto{\pgfqpoint{7.911977in}{1.504196in}}{\pgfqpoint{7.916814in}{1.506200in}}{\pgfqpoint{7.920381in}{1.509766in}}%
\pgfpathcurveto{\pgfqpoint{7.923947in}{1.513332in}}{\pgfqpoint{7.925951in}{1.518170in}}{\pgfqpoint{7.925951in}{1.523214in}}%
\pgfpathcurveto{\pgfqpoint{7.925951in}{1.528258in}}{\pgfqpoint{7.923947in}{1.533095in}}{\pgfqpoint{7.920381in}{1.536662in}}%
\pgfpathcurveto{\pgfqpoint{7.916814in}{1.540228in}}{\pgfqpoint{7.911977in}{1.542232in}}{\pgfqpoint{7.906933in}{1.542232in}}%
\pgfpathcurveto{\pgfqpoint{7.901889in}{1.542232in}}{\pgfqpoint{7.897052in}{1.540228in}}{\pgfqpoint{7.893485in}{1.536662in}}%
\pgfpathcurveto{\pgfqpoint{7.889919in}{1.533095in}}{\pgfqpoint{7.887915in}{1.528258in}}{\pgfqpoint{7.887915in}{1.523214in}}%
\pgfpathcurveto{\pgfqpoint{7.887915in}{1.518170in}}{\pgfqpoint{7.889919in}{1.513332in}}{\pgfqpoint{7.893485in}{1.509766in}}%
\pgfpathcurveto{\pgfqpoint{7.897052in}{1.506200in}}{\pgfqpoint{7.901889in}{1.504196in}}{\pgfqpoint{7.906933in}{1.504196in}}%
\pgfpathclose%
\pgfusepath{fill}%
\end{pgfscope}%
\begin{pgfscope}%
\pgfpathrectangle{\pgfqpoint{6.572727in}{0.473000in}}{\pgfqpoint{4.227273in}{3.311000in}}%
\pgfusepath{clip}%
\pgfsetbuttcap%
\pgfsetroundjoin%
\definecolor{currentfill}{rgb}{0.127568,0.566949,0.550556}%
\pgfsetfillcolor{currentfill}%
\pgfsetfillopacity{0.700000}%
\pgfsetlinewidth{0.000000pt}%
\definecolor{currentstroke}{rgb}{0.000000,0.000000,0.000000}%
\pgfsetstrokecolor{currentstroke}%
\pgfsetstrokeopacity{0.700000}%
\pgfsetdash{}{0pt}%
\pgfpathmoveto{\pgfqpoint{7.456083in}{0.896155in}}%
\pgfpathcurveto{\pgfqpoint{7.461127in}{0.896155in}}{\pgfqpoint{7.465965in}{0.898159in}}{\pgfqpoint{7.469531in}{0.901725in}}%
\pgfpathcurveto{\pgfqpoint{7.473097in}{0.905292in}}{\pgfqpoint{7.475101in}{0.910129in}}{\pgfqpoint{7.475101in}{0.915173in}}%
\pgfpathcurveto{\pgfqpoint{7.475101in}{0.920217in}}{\pgfqpoint{7.473097in}{0.925055in}}{\pgfqpoint{7.469531in}{0.928621in}}%
\pgfpathcurveto{\pgfqpoint{7.465965in}{0.932187in}}{\pgfqpoint{7.461127in}{0.934191in}}{\pgfqpoint{7.456083in}{0.934191in}}%
\pgfpathcurveto{\pgfqpoint{7.451039in}{0.934191in}}{\pgfqpoint{7.446202in}{0.932187in}}{\pgfqpoint{7.442635in}{0.928621in}}%
\pgfpathcurveto{\pgfqpoint{7.439069in}{0.925055in}}{\pgfqpoint{7.437065in}{0.920217in}}{\pgfqpoint{7.437065in}{0.915173in}}%
\pgfpathcurveto{\pgfqpoint{7.437065in}{0.910129in}}{\pgfqpoint{7.439069in}{0.905292in}}{\pgfqpoint{7.442635in}{0.901725in}}%
\pgfpathcurveto{\pgfqpoint{7.446202in}{0.898159in}}{\pgfqpoint{7.451039in}{0.896155in}}{\pgfqpoint{7.456083in}{0.896155in}}%
\pgfpathclose%
\pgfusepath{fill}%
\end{pgfscope}%
\begin{pgfscope}%
\pgfpathrectangle{\pgfqpoint{6.572727in}{0.473000in}}{\pgfqpoint{4.227273in}{3.311000in}}%
\pgfusepath{clip}%
\pgfsetbuttcap%
\pgfsetroundjoin%
\definecolor{currentfill}{rgb}{0.127568,0.566949,0.550556}%
\pgfsetfillcolor{currentfill}%
\pgfsetfillopacity{0.700000}%
\pgfsetlinewidth{0.000000pt}%
\definecolor{currentstroke}{rgb}{0.000000,0.000000,0.000000}%
\pgfsetstrokecolor{currentstroke}%
\pgfsetstrokeopacity{0.700000}%
\pgfsetdash{}{0pt}%
\pgfpathmoveto{\pgfqpoint{7.844510in}{1.234085in}}%
\pgfpathcurveto{\pgfqpoint{7.849554in}{1.234085in}}{\pgfqpoint{7.854392in}{1.236089in}}{\pgfqpoint{7.857958in}{1.239656in}}%
\pgfpathcurveto{\pgfqpoint{7.861525in}{1.243222in}}{\pgfqpoint{7.863528in}{1.248060in}}{\pgfqpoint{7.863528in}{1.253103in}}%
\pgfpathcurveto{\pgfqpoint{7.863528in}{1.258147in}}{\pgfqpoint{7.861525in}{1.262985in}}{\pgfqpoint{7.857958in}{1.266551in}}%
\pgfpathcurveto{\pgfqpoint{7.854392in}{1.270118in}}{\pgfqpoint{7.849554in}{1.272122in}}{\pgfqpoint{7.844510in}{1.272122in}}%
\pgfpathcurveto{\pgfqpoint{7.839467in}{1.272122in}}{\pgfqpoint{7.834629in}{1.270118in}}{\pgfqpoint{7.831062in}{1.266551in}}%
\pgfpathcurveto{\pgfqpoint{7.827496in}{1.262985in}}{\pgfqpoint{7.825492in}{1.258147in}}{\pgfqpoint{7.825492in}{1.253103in}}%
\pgfpathcurveto{\pgfqpoint{7.825492in}{1.248060in}}{\pgfqpoint{7.827496in}{1.243222in}}{\pgfqpoint{7.831062in}{1.239656in}}%
\pgfpathcurveto{\pgfqpoint{7.834629in}{1.236089in}}{\pgfqpoint{7.839467in}{1.234085in}}{\pgfqpoint{7.844510in}{1.234085in}}%
\pgfpathclose%
\pgfusepath{fill}%
\end{pgfscope}%
\begin{pgfscope}%
\pgfpathrectangle{\pgfqpoint{6.572727in}{0.473000in}}{\pgfqpoint{4.227273in}{3.311000in}}%
\pgfusepath{clip}%
\pgfsetbuttcap%
\pgfsetroundjoin%
\definecolor{currentfill}{rgb}{0.993248,0.906157,0.143936}%
\pgfsetfillcolor{currentfill}%
\pgfsetfillopacity{0.700000}%
\pgfsetlinewidth{0.000000pt}%
\definecolor{currentstroke}{rgb}{0.000000,0.000000,0.000000}%
\pgfsetstrokecolor{currentstroke}%
\pgfsetstrokeopacity{0.700000}%
\pgfsetdash{}{0pt}%
\pgfpathmoveto{\pgfqpoint{9.944993in}{0.941595in}}%
\pgfpathcurveto{\pgfqpoint{9.950036in}{0.941595in}}{\pgfqpoint{9.954874in}{0.943598in}}{\pgfqpoint{9.958441in}{0.947165in}}%
\pgfpathcurveto{\pgfqpoint{9.962007in}{0.950731in}}{\pgfqpoint{9.964011in}{0.955569in}}{\pgfqpoint{9.964011in}{0.960613in}}%
\pgfpathcurveto{\pgfqpoint{9.964011in}{0.965656in}}{\pgfqpoint{9.962007in}{0.970494in}}{\pgfqpoint{9.958441in}{0.974061in}}%
\pgfpathcurveto{\pgfqpoint{9.954874in}{0.977627in}}{\pgfqpoint{9.950036in}{0.979631in}}{\pgfqpoint{9.944993in}{0.979631in}}%
\pgfpathcurveto{\pgfqpoint{9.939949in}{0.979631in}}{\pgfqpoint{9.935111in}{0.977627in}}{\pgfqpoint{9.931545in}{0.974061in}}%
\pgfpathcurveto{\pgfqpoint{9.927978in}{0.970494in}}{\pgfqpoint{9.925975in}{0.965656in}}{\pgfqpoint{9.925975in}{0.960613in}}%
\pgfpathcurveto{\pgfqpoint{9.925975in}{0.955569in}}{\pgfqpoint{9.927978in}{0.950731in}}{\pgfqpoint{9.931545in}{0.947165in}}%
\pgfpathcurveto{\pgfqpoint{9.935111in}{0.943598in}}{\pgfqpoint{9.939949in}{0.941595in}}{\pgfqpoint{9.944993in}{0.941595in}}%
\pgfpathclose%
\pgfusepath{fill}%
\end{pgfscope}%
\begin{pgfscope}%
\pgfpathrectangle{\pgfqpoint{6.572727in}{0.473000in}}{\pgfqpoint{4.227273in}{3.311000in}}%
\pgfusepath{clip}%
\pgfsetbuttcap%
\pgfsetroundjoin%
\definecolor{currentfill}{rgb}{0.127568,0.566949,0.550556}%
\pgfsetfillcolor{currentfill}%
\pgfsetfillopacity{0.700000}%
\pgfsetlinewidth{0.000000pt}%
\definecolor{currentstroke}{rgb}{0.000000,0.000000,0.000000}%
\pgfsetstrokecolor{currentstroke}%
\pgfsetstrokeopacity{0.700000}%
\pgfsetdash{}{0pt}%
\pgfpathmoveto{\pgfqpoint{7.891715in}{1.690926in}}%
\pgfpathcurveto{\pgfqpoint{7.896759in}{1.690926in}}{\pgfqpoint{7.901597in}{1.692930in}}{\pgfqpoint{7.905163in}{1.696497in}}%
\pgfpathcurveto{\pgfqpoint{7.908729in}{1.700063in}}{\pgfqpoint{7.910733in}{1.704901in}}{\pgfqpoint{7.910733in}{1.709945in}}%
\pgfpathcurveto{\pgfqpoint{7.910733in}{1.714988in}}{\pgfqpoint{7.908729in}{1.719826in}}{\pgfqpoint{7.905163in}{1.723392in}}%
\pgfpathcurveto{\pgfqpoint{7.901597in}{1.726959in}}{\pgfqpoint{7.896759in}{1.728963in}}{\pgfqpoint{7.891715in}{1.728963in}}%
\pgfpathcurveto{\pgfqpoint{7.886672in}{1.728963in}}{\pgfqpoint{7.881834in}{1.726959in}}{\pgfqpoint{7.878267in}{1.723392in}}%
\pgfpathcurveto{\pgfqpoint{7.874701in}{1.719826in}}{\pgfqpoint{7.872697in}{1.714988in}}{\pgfqpoint{7.872697in}{1.709945in}}%
\pgfpathcurveto{\pgfqpoint{7.872697in}{1.704901in}}{\pgfqpoint{7.874701in}{1.700063in}}{\pgfqpoint{7.878267in}{1.696497in}}%
\pgfpathcurveto{\pgfqpoint{7.881834in}{1.692930in}}{\pgfqpoint{7.886672in}{1.690926in}}{\pgfqpoint{7.891715in}{1.690926in}}%
\pgfpathclose%
\pgfusepath{fill}%
\end{pgfscope}%
\begin{pgfscope}%
\pgfpathrectangle{\pgfqpoint{6.572727in}{0.473000in}}{\pgfqpoint{4.227273in}{3.311000in}}%
\pgfusepath{clip}%
\pgfsetbuttcap%
\pgfsetroundjoin%
\definecolor{currentfill}{rgb}{0.127568,0.566949,0.550556}%
\pgfsetfillcolor{currentfill}%
\pgfsetfillopacity{0.700000}%
\pgfsetlinewidth{0.000000pt}%
\definecolor{currentstroke}{rgb}{0.000000,0.000000,0.000000}%
\pgfsetstrokecolor{currentstroke}%
\pgfsetstrokeopacity{0.700000}%
\pgfsetdash{}{0pt}%
\pgfpathmoveto{\pgfqpoint{8.383135in}{2.439571in}}%
\pgfpathcurveto{\pgfqpoint{8.388178in}{2.439571in}}{\pgfqpoint{8.393016in}{2.441575in}}{\pgfqpoint{8.396583in}{2.445141in}}%
\pgfpathcurveto{\pgfqpoint{8.400149in}{2.448708in}}{\pgfqpoint{8.402153in}{2.453546in}}{\pgfqpoint{8.402153in}{2.458589in}}%
\pgfpathcurveto{\pgfqpoint{8.402153in}{2.463633in}}{\pgfqpoint{8.400149in}{2.468471in}}{\pgfqpoint{8.396583in}{2.472037in}}%
\pgfpathcurveto{\pgfqpoint{8.393016in}{2.475603in}}{\pgfqpoint{8.388178in}{2.477607in}}{\pgfqpoint{8.383135in}{2.477607in}}%
\pgfpathcurveto{\pgfqpoint{8.378091in}{2.477607in}}{\pgfqpoint{8.373253in}{2.475603in}}{\pgfqpoint{8.369687in}{2.472037in}}%
\pgfpathcurveto{\pgfqpoint{8.366120in}{2.468471in}}{\pgfqpoint{8.364117in}{2.463633in}}{\pgfqpoint{8.364117in}{2.458589in}}%
\pgfpathcurveto{\pgfqpoint{8.364117in}{2.453546in}}{\pgfqpoint{8.366120in}{2.448708in}}{\pgfqpoint{8.369687in}{2.445141in}}%
\pgfpathcurveto{\pgfqpoint{8.373253in}{2.441575in}}{\pgfqpoint{8.378091in}{2.439571in}}{\pgfqpoint{8.383135in}{2.439571in}}%
\pgfpathclose%
\pgfusepath{fill}%
\end{pgfscope}%
\begin{pgfscope}%
\pgfpathrectangle{\pgfqpoint{6.572727in}{0.473000in}}{\pgfqpoint{4.227273in}{3.311000in}}%
\pgfusepath{clip}%
\pgfsetbuttcap%
\pgfsetroundjoin%
\definecolor{currentfill}{rgb}{0.127568,0.566949,0.550556}%
\pgfsetfillcolor{currentfill}%
\pgfsetfillopacity{0.700000}%
\pgfsetlinewidth{0.000000pt}%
\definecolor{currentstroke}{rgb}{0.000000,0.000000,0.000000}%
\pgfsetstrokecolor{currentstroke}%
\pgfsetstrokeopacity{0.700000}%
\pgfsetdash{}{0pt}%
\pgfpathmoveto{\pgfqpoint{8.008951in}{1.925815in}}%
\pgfpathcurveto{\pgfqpoint{8.013995in}{1.925815in}}{\pgfqpoint{8.018833in}{1.927819in}}{\pgfqpoint{8.022399in}{1.931385in}}%
\pgfpathcurveto{\pgfqpoint{8.025966in}{1.934952in}}{\pgfqpoint{8.027969in}{1.939790in}}{\pgfqpoint{8.027969in}{1.944833in}}%
\pgfpathcurveto{\pgfqpoint{8.027969in}{1.949877in}}{\pgfqpoint{8.025966in}{1.954715in}}{\pgfqpoint{8.022399in}{1.958281in}}%
\pgfpathcurveto{\pgfqpoint{8.018833in}{1.961847in}}{\pgfqpoint{8.013995in}{1.963851in}}{\pgfqpoint{8.008951in}{1.963851in}}%
\pgfpathcurveto{\pgfqpoint{8.003908in}{1.963851in}}{\pgfqpoint{7.999070in}{1.961847in}}{\pgfqpoint{7.995503in}{1.958281in}}%
\pgfpathcurveto{\pgfqpoint{7.991937in}{1.954715in}}{\pgfqpoint{7.989933in}{1.949877in}}{\pgfqpoint{7.989933in}{1.944833in}}%
\pgfpathcurveto{\pgfqpoint{7.989933in}{1.939790in}}{\pgfqpoint{7.991937in}{1.934952in}}{\pgfqpoint{7.995503in}{1.931385in}}%
\pgfpathcurveto{\pgfqpoint{7.999070in}{1.927819in}}{\pgfqpoint{8.003908in}{1.925815in}}{\pgfqpoint{8.008951in}{1.925815in}}%
\pgfpathclose%
\pgfusepath{fill}%
\end{pgfscope}%
\begin{pgfscope}%
\pgfpathrectangle{\pgfqpoint{6.572727in}{0.473000in}}{\pgfqpoint{4.227273in}{3.311000in}}%
\pgfusepath{clip}%
\pgfsetbuttcap%
\pgfsetroundjoin%
\definecolor{currentfill}{rgb}{0.127568,0.566949,0.550556}%
\pgfsetfillcolor{currentfill}%
\pgfsetfillopacity{0.700000}%
\pgfsetlinewidth{0.000000pt}%
\definecolor{currentstroke}{rgb}{0.000000,0.000000,0.000000}%
\pgfsetstrokecolor{currentstroke}%
\pgfsetstrokeopacity{0.700000}%
\pgfsetdash{}{0pt}%
\pgfpathmoveto{\pgfqpoint{8.280721in}{3.002412in}}%
\pgfpathcurveto{\pgfqpoint{8.285764in}{3.002412in}}{\pgfqpoint{8.290602in}{3.004416in}}{\pgfqpoint{8.294168in}{3.007983in}}%
\pgfpathcurveto{\pgfqpoint{8.297735in}{3.011549in}}{\pgfqpoint{8.299739in}{3.016387in}}{\pgfqpoint{8.299739in}{3.021430in}}%
\pgfpathcurveto{\pgfqpoint{8.299739in}{3.026474in}}{\pgfqpoint{8.297735in}{3.031312in}}{\pgfqpoint{8.294168in}{3.034878in}}%
\pgfpathcurveto{\pgfqpoint{8.290602in}{3.038445in}}{\pgfqpoint{8.285764in}{3.040449in}}{\pgfqpoint{8.280721in}{3.040449in}}%
\pgfpathcurveto{\pgfqpoint{8.275677in}{3.040449in}}{\pgfqpoint{8.270839in}{3.038445in}}{\pgfqpoint{8.267273in}{3.034878in}}%
\pgfpathcurveto{\pgfqpoint{8.263706in}{3.031312in}}{\pgfqpoint{8.261702in}{3.026474in}}{\pgfqpoint{8.261702in}{3.021430in}}%
\pgfpathcurveto{\pgfqpoint{8.261702in}{3.016387in}}{\pgfqpoint{8.263706in}{3.011549in}}{\pgfqpoint{8.267273in}{3.007983in}}%
\pgfpathcurveto{\pgfqpoint{8.270839in}{3.004416in}}{\pgfqpoint{8.275677in}{3.002412in}}{\pgfqpoint{8.280721in}{3.002412in}}%
\pgfpathclose%
\pgfusepath{fill}%
\end{pgfscope}%
\begin{pgfscope}%
\pgfpathrectangle{\pgfqpoint{6.572727in}{0.473000in}}{\pgfqpoint{4.227273in}{3.311000in}}%
\pgfusepath{clip}%
\pgfsetbuttcap%
\pgfsetroundjoin%
\definecolor{currentfill}{rgb}{0.127568,0.566949,0.550556}%
\pgfsetfillcolor{currentfill}%
\pgfsetfillopacity{0.700000}%
\pgfsetlinewidth{0.000000pt}%
\definecolor{currentstroke}{rgb}{0.000000,0.000000,0.000000}%
\pgfsetstrokecolor{currentstroke}%
\pgfsetstrokeopacity{0.700000}%
\pgfsetdash{}{0pt}%
\pgfpathmoveto{\pgfqpoint{8.272375in}{1.621493in}}%
\pgfpathcurveto{\pgfqpoint{8.277419in}{1.621493in}}{\pgfqpoint{8.282256in}{1.623497in}}{\pgfqpoint{8.285823in}{1.627064in}}%
\pgfpathcurveto{\pgfqpoint{8.289389in}{1.630630in}}{\pgfqpoint{8.291393in}{1.635468in}}{\pgfqpoint{8.291393in}{1.640511in}}%
\pgfpathcurveto{\pgfqpoint{8.291393in}{1.645555in}}{\pgfqpoint{8.289389in}{1.650393in}}{\pgfqpoint{8.285823in}{1.653959in}}%
\pgfpathcurveto{\pgfqpoint{8.282256in}{1.657526in}}{\pgfqpoint{8.277419in}{1.659530in}}{\pgfqpoint{8.272375in}{1.659530in}}%
\pgfpathcurveto{\pgfqpoint{8.267331in}{1.659530in}}{\pgfqpoint{8.262494in}{1.657526in}}{\pgfqpoint{8.258927in}{1.653959in}}%
\pgfpathcurveto{\pgfqpoint{8.255361in}{1.650393in}}{\pgfqpoint{8.253357in}{1.645555in}}{\pgfqpoint{8.253357in}{1.640511in}}%
\pgfpathcurveto{\pgfqpoint{8.253357in}{1.635468in}}{\pgfqpoint{8.255361in}{1.630630in}}{\pgfqpoint{8.258927in}{1.627064in}}%
\pgfpathcurveto{\pgfqpoint{8.262494in}{1.623497in}}{\pgfqpoint{8.267331in}{1.621493in}}{\pgfqpoint{8.272375in}{1.621493in}}%
\pgfpathclose%
\pgfusepath{fill}%
\end{pgfscope}%
\begin{pgfscope}%
\pgfpathrectangle{\pgfqpoint{6.572727in}{0.473000in}}{\pgfqpoint{4.227273in}{3.311000in}}%
\pgfusepath{clip}%
\pgfsetbuttcap%
\pgfsetroundjoin%
\definecolor{currentfill}{rgb}{0.127568,0.566949,0.550556}%
\pgfsetfillcolor{currentfill}%
\pgfsetfillopacity{0.700000}%
\pgfsetlinewidth{0.000000pt}%
\definecolor{currentstroke}{rgb}{0.000000,0.000000,0.000000}%
\pgfsetstrokecolor{currentstroke}%
\pgfsetstrokeopacity{0.700000}%
\pgfsetdash{}{0pt}%
\pgfpathmoveto{\pgfqpoint{8.306057in}{3.019835in}}%
\pgfpathcurveto{\pgfqpoint{8.311101in}{3.019835in}}{\pgfqpoint{8.315939in}{3.021839in}}{\pgfqpoint{8.319505in}{3.025405in}}%
\pgfpathcurveto{\pgfqpoint{8.323071in}{3.028972in}}{\pgfqpoint{8.325075in}{3.033810in}}{\pgfqpoint{8.325075in}{3.038853in}}%
\pgfpathcurveto{\pgfqpoint{8.325075in}{3.043897in}}{\pgfqpoint{8.323071in}{3.048735in}}{\pgfqpoint{8.319505in}{3.052301in}}%
\pgfpathcurveto{\pgfqpoint{8.315939in}{3.055868in}}{\pgfqpoint{8.311101in}{3.057871in}}{\pgfqpoint{8.306057in}{3.057871in}}%
\pgfpathcurveto{\pgfqpoint{8.301013in}{3.057871in}}{\pgfqpoint{8.296176in}{3.055868in}}{\pgfqpoint{8.292609in}{3.052301in}}%
\pgfpathcurveto{\pgfqpoint{8.289043in}{3.048735in}}{\pgfqpoint{8.287039in}{3.043897in}}{\pgfqpoint{8.287039in}{3.038853in}}%
\pgfpathcurveto{\pgfqpoint{8.287039in}{3.033810in}}{\pgfqpoint{8.289043in}{3.028972in}}{\pgfqpoint{8.292609in}{3.025405in}}%
\pgfpathcurveto{\pgfqpoint{8.296176in}{3.021839in}}{\pgfqpoint{8.301013in}{3.019835in}}{\pgfqpoint{8.306057in}{3.019835in}}%
\pgfpathclose%
\pgfusepath{fill}%
\end{pgfscope}%
\begin{pgfscope}%
\pgfpathrectangle{\pgfqpoint{6.572727in}{0.473000in}}{\pgfqpoint{4.227273in}{3.311000in}}%
\pgfusepath{clip}%
\pgfsetbuttcap%
\pgfsetroundjoin%
\definecolor{currentfill}{rgb}{0.993248,0.906157,0.143936}%
\pgfsetfillcolor{currentfill}%
\pgfsetfillopacity{0.700000}%
\pgfsetlinewidth{0.000000pt}%
\definecolor{currentstroke}{rgb}{0.000000,0.000000,0.000000}%
\pgfsetstrokecolor{currentstroke}%
\pgfsetstrokeopacity{0.700000}%
\pgfsetdash{}{0pt}%
\pgfpathmoveto{\pgfqpoint{9.770135in}{1.549567in}}%
\pgfpathcurveto{\pgfqpoint{9.775179in}{1.549567in}}{\pgfqpoint{9.780017in}{1.551571in}}{\pgfqpoint{9.783583in}{1.555138in}}%
\pgfpathcurveto{\pgfqpoint{9.787150in}{1.558704in}}{\pgfqpoint{9.789154in}{1.563542in}}{\pgfqpoint{9.789154in}{1.568585in}}%
\pgfpathcurveto{\pgfqpoint{9.789154in}{1.573629in}}{\pgfqpoint{9.787150in}{1.578467in}}{\pgfqpoint{9.783583in}{1.582033in}}%
\pgfpathcurveto{\pgfqpoint{9.780017in}{1.585600in}}{\pgfqpoint{9.775179in}{1.587604in}}{\pgfqpoint{9.770135in}{1.587604in}}%
\pgfpathcurveto{\pgfqpoint{9.765092in}{1.587604in}}{\pgfqpoint{9.760254in}{1.585600in}}{\pgfqpoint{9.756688in}{1.582033in}}%
\pgfpathcurveto{\pgfqpoint{9.753121in}{1.578467in}}{\pgfqpoint{9.751117in}{1.573629in}}{\pgfqpoint{9.751117in}{1.568585in}}%
\pgfpathcurveto{\pgfqpoint{9.751117in}{1.563542in}}{\pgfqpoint{9.753121in}{1.558704in}}{\pgfqpoint{9.756688in}{1.555138in}}%
\pgfpathcurveto{\pgfqpoint{9.760254in}{1.551571in}}{\pgfqpoint{9.765092in}{1.549567in}}{\pgfqpoint{9.770135in}{1.549567in}}%
\pgfpathclose%
\pgfusepath{fill}%
\end{pgfscope}%
\begin{pgfscope}%
\pgfpathrectangle{\pgfqpoint{6.572727in}{0.473000in}}{\pgfqpoint{4.227273in}{3.311000in}}%
\pgfusepath{clip}%
\pgfsetbuttcap%
\pgfsetroundjoin%
\definecolor{currentfill}{rgb}{0.127568,0.566949,0.550556}%
\pgfsetfillcolor{currentfill}%
\pgfsetfillopacity{0.700000}%
\pgfsetlinewidth{0.000000pt}%
\definecolor{currentstroke}{rgb}{0.000000,0.000000,0.000000}%
\pgfsetstrokecolor{currentstroke}%
\pgfsetstrokeopacity{0.700000}%
\pgfsetdash{}{0pt}%
\pgfpathmoveto{\pgfqpoint{8.649131in}{3.241642in}}%
\pgfpathcurveto{\pgfqpoint{8.654175in}{3.241642in}}{\pgfqpoint{8.659012in}{3.243646in}}{\pgfqpoint{8.662579in}{3.247213in}}%
\pgfpathcurveto{\pgfqpoint{8.666145in}{3.250779in}}{\pgfqpoint{8.668149in}{3.255617in}}{\pgfqpoint{8.668149in}{3.260661in}}%
\pgfpathcurveto{\pgfqpoint{8.668149in}{3.265704in}}{\pgfqpoint{8.666145in}{3.270542in}}{\pgfqpoint{8.662579in}{3.274108in}}%
\pgfpathcurveto{\pgfqpoint{8.659012in}{3.277675in}}{\pgfqpoint{8.654175in}{3.279679in}}{\pgfqpoint{8.649131in}{3.279679in}}%
\pgfpathcurveto{\pgfqpoint{8.644087in}{3.279679in}}{\pgfqpoint{8.639250in}{3.277675in}}{\pgfqpoint{8.635683in}{3.274108in}}%
\pgfpathcurveto{\pgfqpoint{8.632117in}{3.270542in}}{\pgfqpoint{8.630113in}{3.265704in}}{\pgfqpoint{8.630113in}{3.260661in}}%
\pgfpathcurveto{\pgfqpoint{8.630113in}{3.255617in}}{\pgfqpoint{8.632117in}{3.250779in}}{\pgfqpoint{8.635683in}{3.247213in}}%
\pgfpathcurveto{\pgfqpoint{8.639250in}{3.243646in}}{\pgfqpoint{8.644087in}{3.241642in}}{\pgfqpoint{8.649131in}{3.241642in}}%
\pgfpathclose%
\pgfusepath{fill}%
\end{pgfscope}%
\begin{pgfscope}%
\pgfpathrectangle{\pgfqpoint{6.572727in}{0.473000in}}{\pgfqpoint{4.227273in}{3.311000in}}%
\pgfusepath{clip}%
\pgfsetbuttcap%
\pgfsetroundjoin%
\definecolor{currentfill}{rgb}{0.993248,0.906157,0.143936}%
\pgfsetfillcolor{currentfill}%
\pgfsetfillopacity{0.700000}%
\pgfsetlinewidth{0.000000pt}%
\definecolor{currentstroke}{rgb}{0.000000,0.000000,0.000000}%
\pgfsetstrokecolor{currentstroke}%
\pgfsetstrokeopacity{0.700000}%
\pgfsetdash{}{0pt}%
\pgfpathmoveto{\pgfqpoint{9.342677in}{1.068716in}}%
\pgfpathcurveto{\pgfqpoint{9.347721in}{1.068716in}}{\pgfqpoint{9.352558in}{1.070720in}}{\pgfqpoint{9.356125in}{1.074286in}}%
\pgfpathcurveto{\pgfqpoint{9.359691in}{1.077853in}}{\pgfqpoint{9.361695in}{1.082691in}}{\pgfqpoint{9.361695in}{1.087734in}}%
\pgfpathcurveto{\pgfqpoint{9.361695in}{1.092778in}}{\pgfqpoint{9.359691in}{1.097616in}}{\pgfqpoint{9.356125in}{1.101182in}}%
\pgfpathcurveto{\pgfqpoint{9.352558in}{1.104748in}}{\pgfqpoint{9.347721in}{1.106752in}}{\pgfqpoint{9.342677in}{1.106752in}}%
\pgfpathcurveto{\pgfqpoint{9.337633in}{1.106752in}}{\pgfqpoint{9.332796in}{1.104748in}}{\pgfqpoint{9.329229in}{1.101182in}}%
\pgfpathcurveto{\pgfqpoint{9.325663in}{1.097616in}}{\pgfqpoint{9.323659in}{1.092778in}}{\pgfqpoint{9.323659in}{1.087734in}}%
\pgfpathcurveto{\pgfqpoint{9.323659in}{1.082691in}}{\pgfqpoint{9.325663in}{1.077853in}}{\pgfqpoint{9.329229in}{1.074286in}}%
\pgfpathcurveto{\pgfqpoint{9.332796in}{1.070720in}}{\pgfqpoint{9.337633in}{1.068716in}}{\pgfqpoint{9.342677in}{1.068716in}}%
\pgfpathclose%
\pgfusepath{fill}%
\end{pgfscope}%
\begin{pgfscope}%
\pgfpathrectangle{\pgfqpoint{6.572727in}{0.473000in}}{\pgfqpoint{4.227273in}{3.311000in}}%
\pgfusepath{clip}%
\pgfsetbuttcap%
\pgfsetroundjoin%
\definecolor{currentfill}{rgb}{0.993248,0.906157,0.143936}%
\pgfsetfillcolor{currentfill}%
\pgfsetfillopacity{0.700000}%
\pgfsetlinewidth{0.000000pt}%
\definecolor{currentstroke}{rgb}{0.000000,0.000000,0.000000}%
\pgfsetstrokecolor{currentstroke}%
\pgfsetstrokeopacity{0.700000}%
\pgfsetdash{}{0pt}%
\pgfpathmoveto{\pgfqpoint{9.502821in}{1.550051in}}%
\pgfpathcurveto{\pgfqpoint{9.507864in}{1.550051in}}{\pgfqpoint{9.512702in}{1.552055in}}{\pgfqpoint{9.516268in}{1.555622in}}%
\pgfpathcurveto{\pgfqpoint{9.519835in}{1.559188in}}{\pgfqpoint{9.521839in}{1.564026in}}{\pgfqpoint{9.521839in}{1.569070in}}%
\pgfpathcurveto{\pgfqpoint{9.521839in}{1.574113in}}{\pgfqpoint{9.519835in}{1.578951in}}{\pgfqpoint{9.516268in}{1.582517in}}%
\pgfpathcurveto{\pgfqpoint{9.512702in}{1.586084in}}{\pgfqpoint{9.507864in}{1.588088in}}{\pgfqpoint{9.502821in}{1.588088in}}%
\pgfpathcurveto{\pgfqpoint{9.497777in}{1.588088in}}{\pgfqpoint{9.492939in}{1.586084in}}{\pgfqpoint{9.489373in}{1.582517in}}%
\pgfpathcurveto{\pgfqpoint{9.485806in}{1.578951in}}{\pgfqpoint{9.483802in}{1.574113in}}{\pgfqpoint{9.483802in}{1.569070in}}%
\pgfpathcurveto{\pgfqpoint{9.483802in}{1.564026in}}{\pgfqpoint{9.485806in}{1.559188in}}{\pgfqpoint{9.489373in}{1.555622in}}%
\pgfpathcurveto{\pgfqpoint{9.492939in}{1.552055in}}{\pgfqpoint{9.497777in}{1.550051in}}{\pgfqpoint{9.502821in}{1.550051in}}%
\pgfpathclose%
\pgfusepath{fill}%
\end{pgfscope}%
\begin{pgfscope}%
\pgfpathrectangle{\pgfqpoint{6.572727in}{0.473000in}}{\pgfqpoint{4.227273in}{3.311000in}}%
\pgfusepath{clip}%
\pgfsetbuttcap%
\pgfsetroundjoin%
\definecolor{currentfill}{rgb}{0.127568,0.566949,0.550556}%
\pgfsetfillcolor{currentfill}%
\pgfsetfillopacity{0.700000}%
\pgfsetlinewidth{0.000000pt}%
\definecolor{currentstroke}{rgb}{0.000000,0.000000,0.000000}%
\pgfsetstrokecolor{currentstroke}%
\pgfsetstrokeopacity{0.700000}%
\pgfsetdash{}{0pt}%
\pgfpathmoveto{\pgfqpoint{7.253956in}{1.401462in}}%
\pgfpathcurveto{\pgfqpoint{7.258999in}{1.401462in}}{\pgfqpoint{7.263837in}{1.403466in}}{\pgfqpoint{7.267404in}{1.407032in}}%
\pgfpathcurveto{\pgfqpoint{7.270970in}{1.410598in}}{\pgfqpoint{7.272974in}{1.415436in}}{\pgfqpoint{7.272974in}{1.420480in}}%
\pgfpathcurveto{\pgfqpoint{7.272974in}{1.425524in}}{\pgfqpoint{7.270970in}{1.430361in}}{\pgfqpoint{7.267404in}{1.433928in}}%
\pgfpathcurveto{\pgfqpoint{7.263837in}{1.437494in}}{\pgfqpoint{7.258999in}{1.439498in}}{\pgfqpoint{7.253956in}{1.439498in}}%
\pgfpathcurveto{\pgfqpoint{7.248912in}{1.439498in}}{\pgfqpoint{7.244074in}{1.437494in}}{\pgfqpoint{7.240508in}{1.433928in}}%
\pgfpathcurveto{\pgfqpoint{7.236942in}{1.430361in}}{\pgfqpoint{7.234938in}{1.425524in}}{\pgfqpoint{7.234938in}{1.420480in}}%
\pgfpathcurveto{\pgfqpoint{7.234938in}{1.415436in}}{\pgfqpoint{7.236942in}{1.410598in}}{\pgfqpoint{7.240508in}{1.407032in}}%
\pgfpathcurveto{\pgfqpoint{7.244074in}{1.403466in}}{\pgfqpoint{7.248912in}{1.401462in}}{\pgfqpoint{7.253956in}{1.401462in}}%
\pgfpathclose%
\pgfusepath{fill}%
\end{pgfscope}%
\begin{pgfscope}%
\pgfpathrectangle{\pgfqpoint{6.572727in}{0.473000in}}{\pgfqpoint{4.227273in}{3.311000in}}%
\pgfusepath{clip}%
\pgfsetbuttcap%
\pgfsetroundjoin%
\definecolor{currentfill}{rgb}{0.127568,0.566949,0.550556}%
\pgfsetfillcolor{currentfill}%
\pgfsetfillopacity{0.700000}%
\pgfsetlinewidth{0.000000pt}%
\definecolor{currentstroke}{rgb}{0.000000,0.000000,0.000000}%
\pgfsetstrokecolor{currentstroke}%
\pgfsetstrokeopacity{0.700000}%
\pgfsetdash{}{0pt}%
\pgfpathmoveto{\pgfqpoint{8.312186in}{1.692776in}}%
\pgfpathcurveto{\pgfqpoint{8.317230in}{1.692776in}}{\pgfqpoint{8.322068in}{1.694779in}}{\pgfqpoint{8.325634in}{1.698346in}}%
\pgfpathcurveto{\pgfqpoint{8.329200in}{1.701912in}}{\pgfqpoint{8.331204in}{1.706750in}}{\pgfqpoint{8.331204in}{1.711794in}}%
\pgfpathcurveto{\pgfqpoint{8.331204in}{1.716837in}}{\pgfqpoint{8.329200in}{1.721675in}}{\pgfqpoint{8.325634in}{1.725242in}}%
\pgfpathcurveto{\pgfqpoint{8.322068in}{1.728808in}}{\pgfqpoint{8.317230in}{1.730812in}}{\pgfqpoint{8.312186in}{1.730812in}}%
\pgfpathcurveto{\pgfqpoint{8.307142in}{1.730812in}}{\pgfqpoint{8.302305in}{1.728808in}}{\pgfqpoint{8.298738in}{1.725242in}}%
\pgfpathcurveto{\pgfqpoint{8.295172in}{1.721675in}}{\pgfqpoint{8.293168in}{1.716837in}}{\pgfqpoint{8.293168in}{1.711794in}}%
\pgfpathcurveto{\pgfqpoint{8.293168in}{1.706750in}}{\pgfqpoint{8.295172in}{1.701912in}}{\pgfqpoint{8.298738in}{1.698346in}}%
\pgfpathcurveto{\pgfqpoint{8.302305in}{1.694779in}}{\pgfqpoint{8.307142in}{1.692776in}}{\pgfqpoint{8.312186in}{1.692776in}}%
\pgfpathclose%
\pgfusepath{fill}%
\end{pgfscope}%
\begin{pgfscope}%
\pgfpathrectangle{\pgfqpoint{6.572727in}{0.473000in}}{\pgfqpoint{4.227273in}{3.311000in}}%
\pgfusepath{clip}%
\pgfsetbuttcap%
\pgfsetroundjoin%
\definecolor{currentfill}{rgb}{0.127568,0.566949,0.550556}%
\pgfsetfillcolor{currentfill}%
\pgfsetfillopacity{0.700000}%
\pgfsetlinewidth{0.000000pt}%
\definecolor{currentstroke}{rgb}{0.000000,0.000000,0.000000}%
\pgfsetstrokecolor{currentstroke}%
\pgfsetstrokeopacity{0.700000}%
\pgfsetdash{}{0pt}%
\pgfpathmoveto{\pgfqpoint{8.155875in}{3.315325in}}%
\pgfpathcurveto{\pgfqpoint{8.160918in}{3.315325in}}{\pgfqpoint{8.165756in}{3.317329in}}{\pgfqpoint{8.169322in}{3.320895in}}%
\pgfpathcurveto{\pgfqpoint{8.172889in}{3.324461in}}{\pgfqpoint{8.174893in}{3.329299in}}{\pgfqpoint{8.174893in}{3.334343in}}%
\pgfpathcurveto{\pgfqpoint{8.174893in}{3.339387in}}{\pgfqpoint{8.172889in}{3.344224in}}{\pgfqpoint{8.169322in}{3.347791in}}%
\pgfpathcurveto{\pgfqpoint{8.165756in}{3.351357in}}{\pgfqpoint{8.160918in}{3.353361in}}{\pgfqpoint{8.155875in}{3.353361in}}%
\pgfpathcurveto{\pgfqpoint{8.150831in}{3.353361in}}{\pgfqpoint{8.145993in}{3.351357in}}{\pgfqpoint{8.142427in}{3.347791in}}%
\pgfpathcurveto{\pgfqpoint{8.138860in}{3.344224in}}{\pgfqpoint{8.136856in}{3.339387in}}{\pgfqpoint{8.136856in}{3.334343in}}%
\pgfpathcurveto{\pgfqpoint{8.136856in}{3.329299in}}{\pgfqpoint{8.138860in}{3.324461in}}{\pgfqpoint{8.142427in}{3.320895in}}%
\pgfpathcurveto{\pgfqpoint{8.145993in}{3.317329in}}{\pgfqpoint{8.150831in}{3.315325in}}{\pgfqpoint{8.155875in}{3.315325in}}%
\pgfpathclose%
\pgfusepath{fill}%
\end{pgfscope}%
\begin{pgfscope}%
\pgfpathrectangle{\pgfqpoint{6.572727in}{0.473000in}}{\pgfqpoint{4.227273in}{3.311000in}}%
\pgfusepath{clip}%
\pgfsetbuttcap%
\pgfsetroundjoin%
\definecolor{currentfill}{rgb}{0.127568,0.566949,0.550556}%
\pgfsetfillcolor{currentfill}%
\pgfsetfillopacity{0.700000}%
\pgfsetlinewidth{0.000000pt}%
\definecolor{currentstroke}{rgb}{0.000000,0.000000,0.000000}%
\pgfsetstrokecolor{currentstroke}%
\pgfsetstrokeopacity{0.700000}%
\pgfsetdash{}{0pt}%
\pgfpathmoveto{\pgfqpoint{8.160949in}{1.367942in}}%
\pgfpathcurveto{\pgfqpoint{8.165993in}{1.367942in}}{\pgfqpoint{8.170831in}{1.369946in}}{\pgfqpoint{8.174397in}{1.373513in}}%
\pgfpathcurveto{\pgfqpoint{8.177964in}{1.377079in}}{\pgfqpoint{8.179968in}{1.381917in}}{\pgfqpoint{8.179968in}{1.386961in}}%
\pgfpathcurveto{\pgfqpoint{8.179968in}{1.392004in}}{\pgfqpoint{8.177964in}{1.396842in}}{\pgfqpoint{8.174397in}{1.400408in}}%
\pgfpathcurveto{\pgfqpoint{8.170831in}{1.403975in}}{\pgfqpoint{8.165993in}{1.405979in}}{\pgfqpoint{8.160949in}{1.405979in}}%
\pgfpathcurveto{\pgfqpoint{8.155906in}{1.405979in}}{\pgfqpoint{8.151068in}{1.403975in}}{\pgfqpoint{8.147502in}{1.400408in}}%
\pgfpathcurveto{\pgfqpoint{8.143935in}{1.396842in}}{\pgfqpoint{8.141931in}{1.392004in}}{\pgfqpoint{8.141931in}{1.386961in}}%
\pgfpathcurveto{\pgfqpoint{8.141931in}{1.381917in}}{\pgfqpoint{8.143935in}{1.377079in}}{\pgfqpoint{8.147502in}{1.373513in}}%
\pgfpathcurveto{\pgfqpoint{8.151068in}{1.369946in}}{\pgfqpoint{8.155906in}{1.367942in}}{\pgfqpoint{8.160949in}{1.367942in}}%
\pgfpathclose%
\pgfusepath{fill}%
\end{pgfscope}%
\begin{pgfscope}%
\pgfpathrectangle{\pgfqpoint{6.572727in}{0.473000in}}{\pgfqpoint{4.227273in}{3.311000in}}%
\pgfusepath{clip}%
\pgfsetbuttcap%
\pgfsetroundjoin%
\definecolor{currentfill}{rgb}{0.127568,0.566949,0.550556}%
\pgfsetfillcolor{currentfill}%
\pgfsetfillopacity{0.700000}%
\pgfsetlinewidth{0.000000pt}%
\definecolor{currentstroke}{rgb}{0.000000,0.000000,0.000000}%
\pgfsetstrokecolor{currentstroke}%
\pgfsetstrokeopacity{0.700000}%
\pgfsetdash{}{0pt}%
\pgfpathmoveto{\pgfqpoint{8.148663in}{2.746411in}}%
\pgfpathcurveto{\pgfqpoint{8.153706in}{2.746411in}}{\pgfqpoint{8.158544in}{2.748415in}}{\pgfqpoint{8.162111in}{2.751981in}}%
\pgfpathcurveto{\pgfqpoint{8.165677in}{2.755548in}}{\pgfqpoint{8.167681in}{2.760385in}}{\pgfqpoint{8.167681in}{2.765429in}}%
\pgfpathcurveto{\pgfqpoint{8.167681in}{2.770473in}}{\pgfqpoint{8.165677in}{2.775310in}}{\pgfqpoint{8.162111in}{2.778877in}}%
\pgfpathcurveto{\pgfqpoint{8.158544in}{2.782443in}}{\pgfqpoint{8.153706in}{2.784447in}}{\pgfqpoint{8.148663in}{2.784447in}}%
\pgfpathcurveto{\pgfqpoint{8.143619in}{2.784447in}}{\pgfqpoint{8.138781in}{2.782443in}}{\pgfqpoint{8.135215in}{2.778877in}}%
\pgfpathcurveto{\pgfqpoint{8.131648in}{2.775310in}}{\pgfqpoint{8.129645in}{2.770473in}}{\pgfqpoint{8.129645in}{2.765429in}}%
\pgfpathcurveto{\pgfqpoint{8.129645in}{2.760385in}}{\pgfqpoint{8.131648in}{2.755548in}}{\pgfqpoint{8.135215in}{2.751981in}}%
\pgfpathcurveto{\pgfqpoint{8.138781in}{2.748415in}}{\pgfqpoint{8.143619in}{2.746411in}}{\pgfqpoint{8.148663in}{2.746411in}}%
\pgfpathclose%
\pgfusepath{fill}%
\end{pgfscope}%
\begin{pgfscope}%
\pgfpathrectangle{\pgfqpoint{6.572727in}{0.473000in}}{\pgfqpoint{4.227273in}{3.311000in}}%
\pgfusepath{clip}%
\pgfsetbuttcap%
\pgfsetroundjoin%
\definecolor{currentfill}{rgb}{0.993248,0.906157,0.143936}%
\pgfsetfillcolor{currentfill}%
\pgfsetfillopacity{0.700000}%
\pgfsetlinewidth{0.000000pt}%
\definecolor{currentstroke}{rgb}{0.000000,0.000000,0.000000}%
\pgfsetstrokecolor{currentstroke}%
\pgfsetstrokeopacity{0.700000}%
\pgfsetdash{}{0pt}%
\pgfpathmoveto{\pgfqpoint{9.442863in}{1.459358in}}%
\pgfpathcurveto{\pgfqpoint{9.447906in}{1.459358in}}{\pgfqpoint{9.452744in}{1.461362in}}{\pgfqpoint{9.456311in}{1.464928in}}%
\pgfpathcurveto{\pgfqpoint{9.459877in}{1.468495in}}{\pgfqpoint{9.461881in}{1.473332in}}{\pgfqpoint{9.461881in}{1.478376in}}%
\pgfpathcurveto{\pgfqpoint{9.461881in}{1.483420in}}{\pgfqpoint{9.459877in}{1.488257in}}{\pgfqpoint{9.456311in}{1.491824in}}%
\pgfpathcurveto{\pgfqpoint{9.452744in}{1.495390in}}{\pgfqpoint{9.447906in}{1.497394in}}{\pgfqpoint{9.442863in}{1.497394in}}%
\pgfpathcurveto{\pgfqpoint{9.437819in}{1.497394in}}{\pgfqpoint{9.432981in}{1.495390in}}{\pgfqpoint{9.429415in}{1.491824in}}%
\pgfpathcurveto{\pgfqpoint{9.425848in}{1.488257in}}{\pgfqpoint{9.423845in}{1.483420in}}{\pgfqpoint{9.423845in}{1.478376in}}%
\pgfpathcurveto{\pgfqpoint{9.423845in}{1.473332in}}{\pgfqpoint{9.425848in}{1.468495in}}{\pgfqpoint{9.429415in}{1.464928in}}%
\pgfpathcurveto{\pgfqpoint{9.432981in}{1.461362in}}{\pgfqpoint{9.437819in}{1.459358in}}{\pgfqpoint{9.442863in}{1.459358in}}%
\pgfpathclose%
\pgfusepath{fill}%
\end{pgfscope}%
\begin{pgfscope}%
\pgfpathrectangle{\pgfqpoint{6.572727in}{0.473000in}}{\pgfqpoint{4.227273in}{3.311000in}}%
\pgfusepath{clip}%
\pgfsetbuttcap%
\pgfsetroundjoin%
\definecolor{currentfill}{rgb}{0.127568,0.566949,0.550556}%
\pgfsetfillcolor{currentfill}%
\pgfsetfillopacity{0.700000}%
\pgfsetlinewidth{0.000000pt}%
\definecolor{currentstroke}{rgb}{0.000000,0.000000,0.000000}%
\pgfsetstrokecolor{currentstroke}%
\pgfsetstrokeopacity{0.700000}%
\pgfsetdash{}{0pt}%
\pgfpathmoveto{\pgfqpoint{8.100618in}{1.788277in}}%
\pgfpathcurveto{\pgfqpoint{8.105662in}{1.788277in}}{\pgfqpoint{8.110500in}{1.790281in}}{\pgfqpoint{8.114066in}{1.793848in}}%
\pgfpathcurveto{\pgfqpoint{8.117633in}{1.797414in}}{\pgfqpoint{8.119637in}{1.802252in}}{\pgfqpoint{8.119637in}{1.807296in}}%
\pgfpathcurveto{\pgfqpoint{8.119637in}{1.812339in}}{\pgfqpoint{8.117633in}{1.817177in}}{\pgfqpoint{8.114066in}{1.820743in}}%
\pgfpathcurveto{\pgfqpoint{8.110500in}{1.824310in}}{\pgfqpoint{8.105662in}{1.826314in}}{\pgfqpoint{8.100618in}{1.826314in}}%
\pgfpathcurveto{\pgfqpoint{8.095575in}{1.826314in}}{\pgfqpoint{8.090737in}{1.824310in}}{\pgfqpoint{8.087171in}{1.820743in}}%
\pgfpathcurveto{\pgfqpoint{8.083604in}{1.817177in}}{\pgfqpoint{8.081600in}{1.812339in}}{\pgfqpoint{8.081600in}{1.807296in}}%
\pgfpathcurveto{\pgfqpoint{8.081600in}{1.802252in}}{\pgfqpoint{8.083604in}{1.797414in}}{\pgfqpoint{8.087171in}{1.793848in}}%
\pgfpathcurveto{\pgfqpoint{8.090737in}{1.790281in}}{\pgfqpoint{8.095575in}{1.788277in}}{\pgfqpoint{8.100618in}{1.788277in}}%
\pgfpathclose%
\pgfusepath{fill}%
\end{pgfscope}%
\begin{pgfscope}%
\pgfpathrectangle{\pgfqpoint{6.572727in}{0.473000in}}{\pgfqpoint{4.227273in}{3.311000in}}%
\pgfusepath{clip}%
\pgfsetbuttcap%
\pgfsetroundjoin%
\definecolor{currentfill}{rgb}{0.127568,0.566949,0.550556}%
\pgfsetfillcolor{currentfill}%
\pgfsetfillopacity{0.700000}%
\pgfsetlinewidth{0.000000pt}%
\definecolor{currentstroke}{rgb}{0.000000,0.000000,0.000000}%
\pgfsetstrokecolor{currentstroke}%
\pgfsetstrokeopacity{0.700000}%
\pgfsetdash{}{0pt}%
\pgfpathmoveto{\pgfqpoint{8.936811in}{2.356481in}}%
\pgfpathcurveto{\pgfqpoint{8.941855in}{2.356481in}}{\pgfqpoint{8.946692in}{2.358485in}}{\pgfqpoint{8.950259in}{2.362052in}}%
\pgfpathcurveto{\pgfqpoint{8.953825in}{2.365618in}}{\pgfqpoint{8.955829in}{2.370456in}}{\pgfqpoint{8.955829in}{2.375500in}}%
\pgfpathcurveto{\pgfqpoint{8.955829in}{2.380543in}}{\pgfqpoint{8.953825in}{2.385381in}}{\pgfqpoint{8.950259in}{2.388947in}}%
\pgfpathcurveto{\pgfqpoint{8.946692in}{2.392514in}}{\pgfqpoint{8.941855in}{2.394518in}}{\pgfqpoint{8.936811in}{2.394518in}}%
\pgfpathcurveto{\pgfqpoint{8.931767in}{2.394518in}}{\pgfqpoint{8.926929in}{2.392514in}}{\pgfqpoint{8.923363in}{2.388947in}}%
\pgfpathcurveto{\pgfqpoint{8.919797in}{2.385381in}}{\pgfqpoint{8.917793in}{2.380543in}}{\pgfqpoint{8.917793in}{2.375500in}}%
\pgfpathcurveto{\pgfqpoint{8.917793in}{2.370456in}}{\pgfqpoint{8.919797in}{2.365618in}}{\pgfqpoint{8.923363in}{2.362052in}}%
\pgfpathcurveto{\pgfqpoint{8.926929in}{2.358485in}}{\pgfqpoint{8.931767in}{2.356481in}}{\pgfqpoint{8.936811in}{2.356481in}}%
\pgfpathclose%
\pgfusepath{fill}%
\end{pgfscope}%
\begin{pgfscope}%
\pgfpathrectangle{\pgfqpoint{6.572727in}{0.473000in}}{\pgfqpoint{4.227273in}{3.311000in}}%
\pgfusepath{clip}%
\pgfsetbuttcap%
\pgfsetroundjoin%
\definecolor{currentfill}{rgb}{0.993248,0.906157,0.143936}%
\pgfsetfillcolor{currentfill}%
\pgfsetfillopacity{0.700000}%
\pgfsetlinewidth{0.000000pt}%
\definecolor{currentstroke}{rgb}{0.000000,0.000000,0.000000}%
\pgfsetstrokecolor{currentstroke}%
\pgfsetstrokeopacity{0.700000}%
\pgfsetdash{}{0pt}%
\pgfpathmoveto{\pgfqpoint{8.822130in}{1.342145in}}%
\pgfpathcurveto{\pgfqpoint{8.827174in}{1.342145in}}{\pgfqpoint{8.832012in}{1.344149in}}{\pgfqpoint{8.835578in}{1.347715in}}%
\pgfpathcurveto{\pgfqpoint{8.839144in}{1.351282in}}{\pgfqpoint{8.841148in}{1.356119in}}{\pgfqpoint{8.841148in}{1.361163in}}%
\pgfpathcurveto{\pgfqpoint{8.841148in}{1.366207in}}{\pgfqpoint{8.839144in}{1.371045in}}{\pgfqpoint{8.835578in}{1.374611in}}%
\pgfpathcurveto{\pgfqpoint{8.832012in}{1.378177in}}{\pgfqpoint{8.827174in}{1.380181in}}{\pgfqpoint{8.822130in}{1.380181in}}%
\pgfpathcurveto{\pgfqpoint{8.817086in}{1.380181in}}{\pgfqpoint{8.812249in}{1.378177in}}{\pgfqpoint{8.808682in}{1.374611in}}%
\pgfpathcurveto{\pgfqpoint{8.805116in}{1.371045in}}{\pgfqpoint{8.803112in}{1.366207in}}{\pgfqpoint{8.803112in}{1.361163in}}%
\pgfpathcurveto{\pgfqpoint{8.803112in}{1.356119in}}{\pgfqpoint{8.805116in}{1.351282in}}{\pgfqpoint{8.808682in}{1.347715in}}%
\pgfpathcurveto{\pgfqpoint{8.812249in}{1.344149in}}{\pgfqpoint{8.817086in}{1.342145in}}{\pgfqpoint{8.822130in}{1.342145in}}%
\pgfpathclose%
\pgfusepath{fill}%
\end{pgfscope}%
\begin{pgfscope}%
\pgfpathrectangle{\pgfqpoint{6.572727in}{0.473000in}}{\pgfqpoint{4.227273in}{3.311000in}}%
\pgfusepath{clip}%
\pgfsetbuttcap%
\pgfsetroundjoin%
\definecolor{currentfill}{rgb}{0.127568,0.566949,0.550556}%
\pgfsetfillcolor{currentfill}%
\pgfsetfillopacity{0.700000}%
\pgfsetlinewidth{0.000000pt}%
\definecolor{currentstroke}{rgb}{0.000000,0.000000,0.000000}%
\pgfsetstrokecolor{currentstroke}%
\pgfsetstrokeopacity{0.700000}%
\pgfsetdash{}{0pt}%
\pgfpathmoveto{\pgfqpoint{8.157294in}{1.332677in}}%
\pgfpathcurveto{\pgfqpoint{8.162337in}{1.332677in}}{\pgfqpoint{8.167175in}{1.334681in}}{\pgfqpoint{8.170742in}{1.338247in}}%
\pgfpathcurveto{\pgfqpoint{8.174308in}{1.341814in}}{\pgfqpoint{8.176312in}{1.346651in}}{\pgfqpoint{8.176312in}{1.351695in}}%
\pgfpathcurveto{\pgfqpoint{8.176312in}{1.356739in}}{\pgfqpoint{8.174308in}{1.361576in}}{\pgfqpoint{8.170742in}{1.365143in}}%
\pgfpathcurveto{\pgfqpoint{8.167175in}{1.368709in}}{\pgfqpoint{8.162337in}{1.370713in}}{\pgfqpoint{8.157294in}{1.370713in}}%
\pgfpathcurveto{\pgfqpoint{8.152250in}{1.370713in}}{\pgfqpoint{8.147412in}{1.368709in}}{\pgfqpoint{8.143846in}{1.365143in}}%
\pgfpathcurveto{\pgfqpoint{8.140280in}{1.361576in}}{\pgfqpoint{8.138276in}{1.356739in}}{\pgfqpoint{8.138276in}{1.351695in}}%
\pgfpathcurveto{\pgfqpoint{8.138276in}{1.346651in}}{\pgfqpoint{8.140280in}{1.341814in}}{\pgfqpoint{8.143846in}{1.338247in}}%
\pgfpathcurveto{\pgfqpoint{8.147412in}{1.334681in}}{\pgfqpoint{8.152250in}{1.332677in}}{\pgfqpoint{8.157294in}{1.332677in}}%
\pgfpathclose%
\pgfusepath{fill}%
\end{pgfscope}%
\begin{pgfscope}%
\pgfpathrectangle{\pgfqpoint{6.572727in}{0.473000in}}{\pgfqpoint{4.227273in}{3.311000in}}%
\pgfusepath{clip}%
\pgfsetbuttcap%
\pgfsetroundjoin%
\definecolor{currentfill}{rgb}{0.993248,0.906157,0.143936}%
\pgfsetfillcolor{currentfill}%
\pgfsetfillopacity{0.700000}%
\pgfsetlinewidth{0.000000pt}%
\definecolor{currentstroke}{rgb}{0.000000,0.000000,0.000000}%
\pgfsetstrokecolor{currentstroke}%
\pgfsetstrokeopacity{0.700000}%
\pgfsetdash{}{0pt}%
\pgfpathmoveto{\pgfqpoint{9.732150in}{1.567526in}}%
\pgfpathcurveto{\pgfqpoint{9.737193in}{1.567526in}}{\pgfqpoint{9.742031in}{1.569530in}}{\pgfqpoint{9.745597in}{1.573096in}}%
\pgfpathcurveto{\pgfqpoint{9.749164in}{1.576663in}}{\pgfqpoint{9.751168in}{1.581500in}}{\pgfqpoint{9.751168in}{1.586544in}}%
\pgfpathcurveto{\pgfqpoint{9.751168in}{1.591588in}}{\pgfqpoint{9.749164in}{1.596426in}}{\pgfqpoint{9.745597in}{1.599992in}}%
\pgfpathcurveto{\pgfqpoint{9.742031in}{1.603558in}}{\pgfqpoint{9.737193in}{1.605562in}}{\pgfqpoint{9.732150in}{1.605562in}}%
\pgfpathcurveto{\pgfqpoint{9.727106in}{1.605562in}}{\pgfqpoint{9.722268in}{1.603558in}}{\pgfqpoint{9.718702in}{1.599992in}}%
\pgfpathcurveto{\pgfqpoint{9.715135in}{1.596426in}}{\pgfqpoint{9.713131in}{1.591588in}}{\pgfqpoint{9.713131in}{1.586544in}}%
\pgfpathcurveto{\pgfqpoint{9.713131in}{1.581500in}}{\pgfqpoint{9.715135in}{1.576663in}}{\pgfqpoint{9.718702in}{1.573096in}}%
\pgfpathcurveto{\pgfqpoint{9.722268in}{1.569530in}}{\pgfqpoint{9.727106in}{1.567526in}}{\pgfqpoint{9.732150in}{1.567526in}}%
\pgfpathclose%
\pgfusepath{fill}%
\end{pgfscope}%
\begin{pgfscope}%
\pgfpathrectangle{\pgfqpoint{6.572727in}{0.473000in}}{\pgfqpoint{4.227273in}{3.311000in}}%
\pgfusepath{clip}%
\pgfsetbuttcap%
\pgfsetroundjoin%
\definecolor{currentfill}{rgb}{0.993248,0.906157,0.143936}%
\pgfsetfillcolor{currentfill}%
\pgfsetfillopacity{0.700000}%
\pgfsetlinewidth{0.000000pt}%
\definecolor{currentstroke}{rgb}{0.000000,0.000000,0.000000}%
\pgfsetstrokecolor{currentstroke}%
\pgfsetstrokeopacity{0.700000}%
\pgfsetdash{}{0pt}%
\pgfpathmoveto{\pgfqpoint{10.044806in}{1.248068in}}%
\pgfpathcurveto{\pgfqpoint{10.049850in}{1.248068in}}{\pgfqpoint{10.054688in}{1.250072in}}{\pgfqpoint{10.058254in}{1.253638in}}%
\pgfpathcurveto{\pgfqpoint{10.061821in}{1.257205in}}{\pgfqpoint{10.063824in}{1.262042in}}{\pgfqpoint{10.063824in}{1.267086in}}%
\pgfpathcurveto{\pgfqpoint{10.063824in}{1.272130in}}{\pgfqpoint{10.061821in}{1.276968in}}{\pgfqpoint{10.058254in}{1.280534in}}%
\pgfpathcurveto{\pgfqpoint{10.054688in}{1.284100in}}{\pgfqpoint{10.049850in}{1.286104in}}{\pgfqpoint{10.044806in}{1.286104in}}%
\pgfpathcurveto{\pgfqpoint{10.039763in}{1.286104in}}{\pgfqpoint{10.034925in}{1.284100in}}{\pgfqpoint{10.031358in}{1.280534in}}%
\pgfpathcurveto{\pgfqpoint{10.027792in}{1.276968in}}{\pgfqpoint{10.025788in}{1.272130in}}{\pgfqpoint{10.025788in}{1.267086in}}%
\pgfpathcurveto{\pgfqpoint{10.025788in}{1.262042in}}{\pgfqpoint{10.027792in}{1.257205in}}{\pgfqpoint{10.031358in}{1.253638in}}%
\pgfpathcurveto{\pgfqpoint{10.034925in}{1.250072in}}{\pgfqpoint{10.039763in}{1.248068in}}{\pgfqpoint{10.044806in}{1.248068in}}%
\pgfpathclose%
\pgfusepath{fill}%
\end{pgfscope}%
\begin{pgfscope}%
\pgfpathrectangle{\pgfqpoint{6.572727in}{0.473000in}}{\pgfqpoint{4.227273in}{3.311000in}}%
\pgfusepath{clip}%
\pgfsetbuttcap%
\pgfsetroundjoin%
\definecolor{currentfill}{rgb}{0.127568,0.566949,0.550556}%
\pgfsetfillcolor{currentfill}%
\pgfsetfillopacity{0.700000}%
\pgfsetlinewidth{0.000000pt}%
\definecolor{currentstroke}{rgb}{0.000000,0.000000,0.000000}%
\pgfsetstrokecolor{currentstroke}%
\pgfsetstrokeopacity{0.700000}%
\pgfsetdash{}{0pt}%
\pgfpathmoveto{\pgfqpoint{8.453272in}{2.732885in}}%
\pgfpathcurveto{\pgfqpoint{8.458315in}{2.732885in}}{\pgfqpoint{8.463153in}{2.734888in}}{\pgfqpoint{8.466720in}{2.738455in}}%
\pgfpathcurveto{\pgfqpoint{8.470286in}{2.742021in}}{\pgfqpoint{8.472290in}{2.746859in}}{\pgfqpoint{8.472290in}{2.751903in}}%
\pgfpathcurveto{\pgfqpoint{8.472290in}{2.756946in}}{\pgfqpoint{8.470286in}{2.761784in}}{\pgfqpoint{8.466720in}{2.765351in}}%
\pgfpathcurveto{\pgfqpoint{8.463153in}{2.768917in}}{\pgfqpoint{8.458315in}{2.770921in}}{\pgfqpoint{8.453272in}{2.770921in}}%
\pgfpathcurveto{\pgfqpoint{8.448228in}{2.770921in}}{\pgfqpoint{8.443390in}{2.768917in}}{\pgfqpoint{8.439824in}{2.765351in}}%
\pgfpathcurveto{\pgfqpoint{8.436257in}{2.761784in}}{\pgfqpoint{8.434254in}{2.756946in}}{\pgfqpoint{8.434254in}{2.751903in}}%
\pgfpathcurveto{\pgfqpoint{8.434254in}{2.746859in}}{\pgfqpoint{8.436257in}{2.742021in}}{\pgfqpoint{8.439824in}{2.738455in}}%
\pgfpathcurveto{\pgfqpoint{8.443390in}{2.734888in}}{\pgfqpoint{8.448228in}{2.732885in}}{\pgfqpoint{8.453272in}{2.732885in}}%
\pgfpathclose%
\pgfusepath{fill}%
\end{pgfscope}%
\begin{pgfscope}%
\pgfpathrectangle{\pgfqpoint{6.572727in}{0.473000in}}{\pgfqpoint{4.227273in}{3.311000in}}%
\pgfusepath{clip}%
\pgfsetbuttcap%
\pgfsetroundjoin%
\definecolor{currentfill}{rgb}{0.127568,0.566949,0.550556}%
\pgfsetfillcolor{currentfill}%
\pgfsetfillopacity{0.700000}%
\pgfsetlinewidth{0.000000pt}%
\definecolor{currentstroke}{rgb}{0.000000,0.000000,0.000000}%
\pgfsetstrokecolor{currentstroke}%
\pgfsetstrokeopacity{0.700000}%
\pgfsetdash{}{0pt}%
\pgfpathmoveto{\pgfqpoint{8.931081in}{2.480428in}}%
\pgfpathcurveto{\pgfqpoint{8.936125in}{2.480428in}}{\pgfqpoint{8.940962in}{2.482432in}}{\pgfqpoint{8.944529in}{2.485998in}}%
\pgfpathcurveto{\pgfqpoint{8.948095in}{2.489565in}}{\pgfqpoint{8.950099in}{2.494402in}}{\pgfqpoint{8.950099in}{2.499446in}}%
\pgfpathcurveto{\pgfqpoint{8.950099in}{2.504490in}}{\pgfqpoint{8.948095in}{2.509328in}}{\pgfqpoint{8.944529in}{2.512894in}}%
\pgfpathcurveto{\pgfqpoint{8.940962in}{2.516460in}}{\pgfqpoint{8.936125in}{2.518464in}}{\pgfqpoint{8.931081in}{2.518464in}}%
\pgfpathcurveto{\pgfqpoint{8.926037in}{2.518464in}}{\pgfqpoint{8.921200in}{2.516460in}}{\pgfqpoint{8.917633in}{2.512894in}}%
\pgfpathcurveto{\pgfqpoint{8.914067in}{2.509328in}}{\pgfqpoint{8.912063in}{2.504490in}}{\pgfqpoint{8.912063in}{2.499446in}}%
\pgfpathcurveto{\pgfqpoint{8.912063in}{2.494402in}}{\pgfqpoint{8.914067in}{2.489565in}}{\pgfqpoint{8.917633in}{2.485998in}}%
\pgfpathcurveto{\pgfqpoint{8.921200in}{2.482432in}}{\pgfqpoint{8.926037in}{2.480428in}}{\pgfqpoint{8.931081in}{2.480428in}}%
\pgfpathclose%
\pgfusepath{fill}%
\end{pgfscope}%
\begin{pgfscope}%
\pgfpathrectangle{\pgfqpoint{6.572727in}{0.473000in}}{\pgfqpoint{4.227273in}{3.311000in}}%
\pgfusepath{clip}%
\pgfsetbuttcap%
\pgfsetroundjoin%
\definecolor{currentfill}{rgb}{0.127568,0.566949,0.550556}%
\pgfsetfillcolor{currentfill}%
\pgfsetfillopacity{0.700000}%
\pgfsetlinewidth{0.000000pt}%
\definecolor{currentstroke}{rgb}{0.000000,0.000000,0.000000}%
\pgfsetstrokecolor{currentstroke}%
\pgfsetstrokeopacity{0.700000}%
\pgfsetdash{}{0pt}%
\pgfpathmoveto{\pgfqpoint{8.350032in}{3.011074in}}%
\pgfpathcurveto{\pgfqpoint{8.355076in}{3.011074in}}{\pgfqpoint{8.359914in}{3.013078in}}{\pgfqpoint{8.363480in}{3.016644in}}%
\pgfpathcurveto{\pgfqpoint{8.367047in}{3.020211in}}{\pgfqpoint{8.369050in}{3.025048in}}{\pgfqpoint{8.369050in}{3.030092in}}%
\pgfpathcurveto{\pgfqpoint{8.369050in}{3.035136in}}{\pgfqpoint{8.367047in}{3.039974in}}{\pgfqpoint{8.363480in}{3.043540in}}%
\pgfpathcurveto{\pgfqpoint{8.359914in}{3.047106in}}{\pgfqpoint{8.355076in}{3.049110in}}{\pgfqpoint{8.350032in}{3.049110in}}%
\pgfpathcurveto{\pgfqpoint{8.344989in}{3.049110in}}{\pgfqpoint{8.340151in}{3.047106in}}{\pgfqpoint{8.336584in}{3.043540in}}%
\pgfpathcurveto{\pgfqpoint{8.333018in}{3.039974in}}{\pgfqpoint{8.331014in}{3.035136in}}{\pgfqpoint{8.331014in}{3.030092in}}%
\pgfpathcurveto{\pgfqpoint{8.331014in}{3.025048in}}{\pgfqpoint{8.333018in}{3.020211in}}{\pgfqpoint{8.336584in}{3.016644in}}%
\pgfpathcurveto{\pgfqpoint{8.340151in}{3.013078in}}{\pgfqpoint{8.344989in}{3.011074in}}{\pgfqpoint{8.350032in}{3.011074in}}%
\pgfpathclose%
\pgfusepath{fill}%
\end{pgfscope}%
\begin{pgfscope}%
\pgfpathrectangle{\pgfqpoint{6.572727in}{0.473000in}}{\pgfqpoint{4.227273in}{3.311000in}}%
\pgfusepath{clip}%
\pgfsetbuttcap%
\pgfsetroundjoin%
\definecolor{currentfill}{rgb}{0.127568,0.566949,0.550556}%
\pgfsetfillcolor{currentfill}%
\pgfsetfillopacity{0.700000}%
\pgfsetlinewidth{0.000000pt}%
\definecolor{currentstroke}{rgb}{0.000000,0.000000,0.000000}%
\pgfsetstrokecolor{currentstroke}%
\pgfsetstrokeopacity{0.700000}%
\pgfsetdash{}{0pt}%
\pgfpathmoveto{\pgfqpoint{8.659534in}{2.973322in}}%
\pgfpathcurveto{\pgfqpoint{8.664578in}{2.973322in}}{\pgfqpoint{8.669415in}{2.975326in}}{\pgfqpoint{8.672982in}{2.978892in}}%
\pgfpathcurveto{\pgfqpoint{8.676548in}{2.982458in}}{\pgfqpoint{8.678552in}{2.987296in}}{\pgfqpoint{8.678552in}{2.992340in}}%
\pgfpathcurveto{\pgfqpoint{8.678552in}{2.997384in}}{\pgfqpoint{8.676548in}{3.002221in}}{\pgfqpoint{8.672982in}{3.005788in}}%
\pgfpathcurveto{\pgfqpoint{8.669415in}{3.009354in}}{\pgfqpoint{8.664578in}{3.011358in}}{\pgfqpoint{8.659534in}{3.011358in}}%
\pgfpathcurveto{\pgfqpoint{8.654490in}{3.011358in}}{\pgfqpoint{8.649653in}{3.009354in}}{\pgfqpoint{8.646086in}{3.005788in}}%
\pgfpathcurveto{\pgfqpoint{8.642520in}{3.002221in}}{\pgfqpoint{8.640516in}{2.997384in}}{\pgfqpoint{8.640516in}{2.992340in}}%
\pgfpathcurveto{\pgfqpoint{8.640516in}{2.987296in}}{\pgfqpoint{8.642520in}{2.982458in}}{\pgfqpoint{8.646086in}{2.978892in}}%
\pgfpathcurveto{\pgfqpoint{8.649653in}{2.975326in}}{\pgfqpoint{8.654490in}{2.973322in}}{\pgfqpoint{8.659534in}{2.973322in}}%
\pgfpathclose%
\pgfusepath{fill}%
\end{pgfscope}%
\begin{pgfscope}%
\pgfpathrectangle{\pgfqpoint{6.572727in}{0.473000in}}{\pgfqpoint{4.227273in}{3.311000in}}%
\pgfusepath{clip}%
\pgfsetbuttcap%
\pgfsetroundjoin%
\definecolor{currentfill}{rgb}{0.993248,0.906157,0.143936}%
\pgfsetfillcolor{currentfill}%
\pgfsetfillopacity{0.700000}%
\pgfsetlinewidth{0.000000pt}%
\definecolor{currentstroke}{rgb}{0.000000,0.000000,0.000000}%
\pgfsetstrokecolor{currentstroke}%
\pgfsetstrokeopacity{0.700000}%
\pgfsetdash{}{0pt}%
\pgfpathmoveto{\pgfqpoint{9.611851in}{1.513263in}}%
\pgfpathcurveto{\pgfqpoint{9.616894in}{1.513263in}}{\pgfqpoint{9.621732in}{1.515267in}}{\pgfqpoint{9.625299in}{1.518834in}}%
\pgfpathcurveto{\pgfqpoint{9.628865in}{1.522400in}}{\pgfqpoint{9.630869in}{1.527238in}}{\pgfqpoint{9.630869in}{1.532282in}}%
\pgfpathcurveto{\pgfqpoint{9.630869in}{1.537325in}}{\pgfqpoint{9.628865in}{1.542163in}}{\pgfqpoint{9.625299in}{1.545729in}}%
\pgfpathcurveto{\pgfqpoint{9.621732in}{1.549296in}}{\pgfqpoint{9.616894in}{1.551300in}}{\pgfqpoint{9.611851in}{1.551300in}}%
\pgfpathcurveto{\pgfqpoint{9.606807in}{1.551300in}}{\pgfqpoint{9.601969in}{1.549296in}}{\pgfqpoint{9.598403in}{1.545729in}}%
\pgfpathcurveto{\pgfqpoint{9.594836in}{1.542163in}}{\pgfqpoint{9.592833in}{1.537325in}}{\pgfqpoint{9.592833in}{1.532282in}}%
\pgfpathcurveto{\pgfqpoint{9.592833in}{1.527238in}}{\pgfqpoint{9.594836in}{1.522400in}}{\pgfqpoint{9.598403in}{1.518834in}}%
\pgfpathcurveto{\pgfqpoint{9.601969in}{1.515267in}}{\pgfqpoint{9.606807in}{1.513263in}}{\pgfqpoint{9.611851in}{1.513263in}}%
\pgfpathclose%
\pgfusepath{fill}%
\end{pgfscope}%
\begin{pgfscope}%
\pgfpathrectangle{\pgfqpoint{6.572727in}{0.473000in}}{\pgfqpoint{4.227273in}{3.311000in}}%
\pgfusepath{clip}%
\pgfsetbuttcap%
\pgfsetroundjoin%
\definecolor{currentfill}{rgb}{0.993248,0.906157,0.143936}%
\pgfsetfillcolor{currentfill}%
\pgfsetfillopacity{0.700000}%
\pgfsetlinewidth{0.000000pt}%
\definecolor{currentstroke}{rgb}{0.000000,0.000000,0.000000}%
\pgfsetstrokecolor{currentstroke}%
\pgfsetstrokeopacity{0.700000}%
\pgfsetdash{}{0pt}%
\pgfpathmoveto{\pgfqpoint{9.500164in}{1.546064in}}%
\pgfpathcurveto{\pgfqpoint{9.505207in}{1.546064in}}{\pgfqpoint{9.510045in}{1.548068in}}{\pgfqpoint{9.513612in}{1.551634in}}%
\pgfpathcurveto{\pgfqpoint{9.517178in}{1.555200in}}{\pgfqpoint{9.519182in}{1.560038in}}{\pgfqpoint{9.519182in}{1.565082in}}%
\pgfpathcurveto{\pgfqpoint{9.519182in}{1.570126in}}{\pgfqpoint{9.517178in}{1.574963in}}{\pgfqpoint{9.513612in}{1.578530in}}%
\pgfpathcurveto{\pgfqpoint{9.510045in}{1.582096in}}{\pgfqpoint{9.505207in}{1.584100in}}{\pgfqpoint{9.500164in}{1.584100in}}%
\pgfpathcurveto{\pgfqpoint{9.495120in}{1.584100in}}{\pgfqpoint{9.490282in}{1.582096in}}{\pgfqpoint{9.486716in}{1.578530in}}%
\pgfpathcurveto{\pgfqpoint{9.483150in}{1.574963in}}{\pgfqpoint{9.481146in}{1.570126in}}{\pgfqpoint{9.481146in}{1.565082in}}%
\pgfpathcurveto{\pgfqpoint{9.481146in}{1.560038in}}{\pgfqpoint{9.483150in}{1.555200in}}{\pgfqpoint{9.486716in}{1.551634in}}%
\pgfpathcurveto{\pgfqpoint{9.490282in}{1.548068in}}{\pgfqpoint{9.495120in}{1.546064in}}{\pgfqpoint{9.500164in}{1.546064in}}%
\pgfpathclose%
\pgfusepath{fill}%
\end{pgfscope}%
\begin{pgfscope}%
\pgfpathrectangle{\pgfqpoint{6.572727in}{0.473000in}}{\pgfqpoint{4.227273in}{3.311000in}}%
\pgfusepath{clip}%
\pgfsetbuttcap%
\pgfsetroundjoin%
\definecolor{currentfill}{rgb}{0.993248,0.906157,0.143936}%
\pgfsetfillcolor{currentfill}%
\pgfsetfillopacity{0.700000}%
\pgfsetlinewidth{0.000000pt}%
\definecolor{currentstroke}{rgb}{0.000000,0.000000,0.000000}%
\pgfsetstrokecolor{currentstroke}%
\pgfsetstrokeopacity{0.700000}%
\pgfsetdash{}{0pt}%
\pgfpathmoveto{\pgfqpoint{10.086043in}{1.519211in}}%
\pgfpathcurveto{\pgfqpoint{10.091087in}{1.519211in}}{\pgfqpoint{10.095925in}{1.521215in}}{\pgfqpoint{10.099491in}{1.524781in}}%
\pgfpathcurveto{\pgfqpoint{10.103057in}{1.528348in}}{\pgfqpoint{10.105061in}{1.533185in}}{\pgfqpoint{10.105061in}{1.538229in}}%
\pgfpathcurveto{\pgfqpoint{10.105061in}{1.543273in}}{\pgfqpoint{10.103057in}{1.548110in}}{\pgfqpoint{10.099491in}{1.551677in}}%
\pgfpathcurveto{\pgfqpoint{10.095925in}{1.555243in}}{\pgfqpoint{10.091087in}{1.557247in}}{\pgfqpoint{10.086043in}{1.557247in}}%
\pgfpathcurveto{\pgfqpoint{10.080999in}{1.557247in}}{\pgfqpoint{10.076162in}{1.555243in}}{\pgfqpoint{10.072595in}{1.551677in}}%
\pgfpathcurveto{\pgfqpoint{10.069029in}{1.548110in}}{\pgfqpoint{10.067025in}{1.543273in}}{\pgfqpoint{10.067025in}{1.538229in}}%
\pgfpathcurveto{\pgfqpoint{10.067025in}{1.533185in}}{\pgfqpoint{10.069029in}{1.528348in}}{\pgfqpoint{10.072595in}{1.524781in}}%
\pgfpathcurveto{\pgfqpoint{10.076162in}{1.521215in}}{\pgfqpoint{10.080999in}{1.519211in}}{\pgfqpoint{10.086043in}{1.519211in}}%
\pgfpathclose%
\pgfusepath{fill}%
\end{pgfscope}%
\begin{pgfscope}%
\pgfpathrectangle{\pgfqpoint{6.572727in}{0.473000in}}{\pgfqpoint{4.227273in}{3.311000in}}%
\pgfusepath{clip}%
\pgfsetbuttcap%
\pgfsetroundjoin%
\definecolor{currentfill}{rgb}{0.993248,0.906157,0.143936}%
\pgfsetfillcolor{currentfill}%
\pgfsetfillopacity{0.700000}%
\pgfsetlinewidth{0.000000pt}%
\definecolor{currentstroke}{rgb}{0.000000,0.000000,0.000000}%
\pgfsetstrokecolor{currentstroke}%
\pgfsetstrokeopacity{0.700000}%
\pgfsetdash{}{0pt}%
\pgfpathmoveto{\pgfqpoint{9.325643in}{1.183407in}}%
\pgfpathcurveto{\pgfqpoint{9.330686in}{1.183407in}}{\pgfqpoint{9.335524in}{1.185411in}}{\pgfqpoint{9.339090in}{1.188977in}}%
\pgfpathcurveto{\pgfqpoint{9.342657in}{1.192544in}}{\pgfqpoint{9.344661in}{1.197382in}}{\pgfqpoint{9.344661in}{1.202425in}}%
\pgfpathcurveto{\pgfqpoint{9.344661in}{1.207469in}}{\pgfqpoint{9.342657in}{1.212307in}}{\pgfqpoint{9.339090in}{1.215873in}}%
\pgfpathcurveto{\pgfqpoint{9.335524in}{1.219440in}}{\pgfqpoint{9.330686in}{1.221443in}}{\pgfqpoint{9.325643in}{1.221443in}}%
\pgfpathcurveto{\pgfqpoint{9.320599in}{1.221443in}}{\pgfqpoint{9.315761in}{1.219440in}}{\pgfqpoint{9.312195in}{1.215873in}}%
\pgfpathcurveto{\pgfqpoint{9.308628in}{1.212307in}}{\pgfqpoint{9.306624in}{1.207469in}}{\pgfqpoint{9.306624in}{1.202425in}}%
\pgfpathcurveto{\pgfqpoint{9.306624in}{1.197382in}}{\pgfqpoint{9.308628in}{1.192544in}}{\pgfqpoint{9.312195in}{1.188977in}}%
\pgfpathcurveto{\pgfqpoint{9.315761in}{1.185411in}}{\pgfqpoint{9.320599in}{1.183407in}}{\pgfqpoint{9.325643in}{1.183407in}}%
\pgfpathclose%
\pgfusepath{fill}%
\end{pgfscope}%
\begin{pgfscope}%
\pgfpathrectangle{\pgfqpoint{6.572727in}{0.473000in}}{\pgfqpoint{4.227273in}{3.311000in}}%
\pgfusepath{clip}%
\pgfsetbuttcap%
\pgfsetroundjoin%
\definecolor{currentfill}{rgb}{0.993248,0.906157,0.143936}%
\pgfsetfillcolor{currentfill}%
\pgfsetfillopacity{0.700000}%
\pgfsetlinewidth{0.000000pt}%
\definecolor{currentstroke}{rgb}{0.000000,0.000000,0.000000}%
\pgfsetstrokecolor{currentstroke}%
\pgfsetstrokeopacity{0.700000}%
\pgfsetdash{}{0pt}%
\pgfpathmoveto{\pgfqpoint{9.240433in}{1.447211in}}%
\pgfpathcurveto{\pgfqpoint{9.245477in}{1.447211in}}{\pgfqpoint{9.250314in}{1.449215in}}{\pgfqpoint{9.253881in}{1.452782in}}%
\pgfpathcurveto{\pgfqpoint{9.257447in}{1.456348in}}{\pgfqpoint{9.259451in}{1.461186in}}{\pgfqpoint{9.259451in}{1.466229in}}%
\pgfpathcurveto{\pgfqpoint{9.259451in}{1.471273in}}{\pgfqpoint{9.257447in}{1.476111in}}{\pgfqpoint{9.253881in}{1.479677in}}%
\pgfpathcurveto{\pgfqpoint{9.250314in}{1.483244in}}{\pgfqpoint{9.245477in}{1.485248in}}{\pgfqpoint{9.240433in}{1.485248in}}%
\pgfpathcurveto{\pgfqpoint{9.235389in}{1.485248in}}{\pgfqpoint{9.230552in}{1.483244in}}{\pgfqpoint{9.226985in}{1.479677in}}%
\pgfpathcurveto{\pgfqpoint{9.223419in}{1.476111in}}{\pgfqpoint{9.221415in}{1.471273in}}{\pgfqpoint{9.221415in}{1.466229in}}%
\pgfpathcurveto{\pgfqpoint{9.221415in}{1.461186in}}{\pgfqpoint{9.223419in}{1.456348in}}{\pgfqpoint{9.226985in}{1.452782in}}%
\pgfpathcurveto{\pgfqpoint{9.230552in}{1.449215in}}{\pgfqpoint{9.235389in}{1.447211in}}{\pgfqpoint{9.240433in}{1.447211in}}%
\pgfpathclose%
\pgfusepath{fill}%
\end{pgfscope}%
\begin{pgfscope}%
\pgfpathrectangle{\pgfqpoint{6.572727in}{0.473000in}}{\pgfqpoint{4.227273in}{3.311000in}}%
\pgfusepath{clip}%
\pgfsetbuttcap%
\pgfsetroundjoin%
\definecolor{currentfill}{rgb}{0.993248,0.906157,0.143936}%
\pgfsetfillcolor{currentfill}%
\pgfsetfillopacity{0.700000}%
\pgfsetlinewidth{0.000000pt}%
\definecolor{currentstroke}{rgb}{0.000000,0.000000,0.000000}%
\pgfsetstrokecolor{currentstroke}%
\pgfsetstrokeopacity{0.700000}%
\pgfsetdash{}{0pt}%
\pgfpathmoveto{\pgfqpoint{9.772256in}{1.954833in}}%
\pgfpathcurveto{\pgfqpoint{9.777300in}{1.954833in}}{\pgfqpoint{9.782138in}{1.956837in}}{\pgfqpoint{9.785704in}{1.960403in}}%
\pgfpathcurveto{\pgfqpoint{9.789270in}{1.963970in}}{\pgfqpoint{9.791274in}{1.968807in}}{\pgfqpoint{9.791274in}{1.973851in}}%
\pgfpathcurveto{\pgfqpoint{9.791274in}{1.978895in}}{\pgfqpoint{9.789270in}{1.983733in}}{\pgfqpoint{9.785704in}{1.987299in}}%
\pgfpathcurveto{\pgfqpoint{9.782138in}{1.990865in}}{\pgfqpoint{9.777300in}{1.992869in}}{\pgfqpoint{9.772256in}{1.992869in}}%
\pgfpathcurveto{\pgfqpoint{9.767213in}{1.992869in}}{\pgfqpoint{9.762375in}{1.990865in}}{\pgfqpoint{9.758808in}{1.987299in}}%
\pgfpathcurveto{\pgfqpoint{9.755242in}{1.983733in}}{\pgfqpoint{9.753238in}{1.978895in}}{\pgfqpoint{9.753238in}{1.973851in}}%
\pgfpathcurveto{\pgfqpoint{9.753238in}{1.968807in}}{\pgfqpoint{9.755242in}{1.963970in}}{\pgfqpoint{9.758808in}{1.960403in}}%
\pgfpathcurveto{\pgfqpoint{9.762375in}{1.956837in}}{\pgfqpoint{9.767213in}{1.954833in}}{\pgfqpoint{9.772256in}{1.954833in}}%
\pgfpathclose%
\pgfusepath{fill}%
\end{pgfscope}%
\begin{pgfscope}%
\pgfpathrectangle{\pgfqpoint{6.572727in}{0.473000in}}{\pgfqpoint{4.227273in}{3.311000in}}%
\pgfusepath{clip}%
\pgfsetbuttcap%
\pgfsetroundjoin%
\definecolor{currentfill}{rgb}{0.127568,0.566949,0.550556}%
\pgfsetfillcolor{currentfill}%
\pgfsetfillopacity{0.700000}%
\pgfsetlinewidth{0.000000pt}%
\definecolor{currentstroke}{rgb}{0.000000,0.000000,0.000000}%
\pgfsetstrokecolor{currentstroke}%
\pgfsetstrokeopacity{0.700000}%
\pgfsetdash{}{0pt}%
\pgfpathmoveto{\pgfqpoint{8.436285in}{2.614749in}}%
\pgfpathcurveto{\pgfqpoint{8.441329in}{2.614749in}}{\pgfqpoint{8.446167in}{2.616753in}}{\pgfqpoint{8.449733in}{2.620319in}}%
\pgfpathcurveto{\pgfqpoint{8.453300in}{2.623885in}}{\pgfqpoint{8.455303in}{2.628723in}}{\pgfqpoint{8.455303in}{2.633767in}}%
\pgfpathcurveto{\pgfqpoint{8.455303in}{2.638811in}}{\pgfqpoint{8.453300in}{2.643648in}}{\pgfqpoint{8.449733in}{2.647215in}}%
\pgfpathcurveto{\pgfqpoint{8.446167in}{2.650781in}}{\pgfqpoint{8.441329in}{2.652785in}}{\pgfqpoint{8.436285in}{2.652785in}}%
\pgfpathcurveto{\pgfqpoint{8.431242in}{2.652785in}}{\pgfqpoint{8.426404in}{2.650781in}}{\pgfqpoint{8.422837in}{2.647215in}}%
\pgfpathcurveto{\pgfqpoint{8.419271in}{2.643648in}}{\pgfqpoint{8.417267in}{2.638811in}}{\pgfqpoint{8.417267in}{2.633767in}}%
\pgfpathcurveto{\pgfqpoint{8.417267in}{2.628723in}}{\pgfqpoint{8.419271in}{2.623885in}}{\pgfqpoint{8.422837in}{2.620319in}}%
\pgfpathcurveto{\pgfqpoint{8.426404in}{2.616753in}}{\pgfqpoint{8.431242in}{2.614749in}}{\pgfqpoint{8.436285in}{2.614749in}}%
\pgfpathclose%
\pgfusepath{fill}%
\end{pgfscope}%
\begin{pgfscope}%
\pgfpathrectangle{\pgfqpoint{6.572727in}{0.473000in}}{\pgfqpoint{4.227273in}{3.311000in}}%
\pgfusepath{clip}%
\pgfsetbuttcap%
\pgfsetroundjoin%
\definecolor{currentfill}{rgb}{0.993248,0.906157,0.143936}%
\pgfsetfillcolor{currentfill}%
\pgfsetfillopacity{0.700000}%
\pgfsetlinewidth{0.000000pt}%
\definecolor{currentstroke}{rgb}{0.000000,0.000000,0.000000}%
\pgfsetstrokecolor{currentstroke}%
\pgfsetstrokeopacity{0.700000}%
\pgfsetdash{}{0pt}%
\pgfpathmoveto{\pgfqpoint{9.664250in}{1.772691in}}%
\pgfpathcurveto{\pgfqpoint{9.669293in}{1.772691in}}{\pgfqpoint{9.674131in}{1.774695in}}{\pgfqpoint{9.677698in}{1.778261in}}%
\pgfpathcurveto{\pgfqpoint{9.681264in}{1.781827in}}{\pgfqpoint{9.683268in}{1.786665in}}{\pgfqpoint{9.683268in}{1.791709in}}%
\pgfpathcurveto{\pgfqpoint{9.683268in}{1.796753in}}{\pgfqpoint{9.681264in}{1.801590in}}{\pgfqpoint{9.677698in}{1.805157in}}%
\pgfpathcurveto{\pgfqpoint{9.674131in}{1.808723in}}{\pgfqpoint{9.669293in}{1.810727in}}{\pgfqpoint{9.664250in}{1.810727in}}%
\pgfpathcurveto{\pgfqpoint{9.659206in}{1.810727in}}{\pgfqpoint{9.654368in}{1.808723in}}{\pgfqpoint{9.650802in}{1.805157in}}%
\pgfpathcurveto{\pgfqpoint{9.647235in}{1.801590in}}{\pgfqpoint{9.645232in}{1.796753in}}{\pgfqpoint{9.645232in}{1.791709in}}%
\pgfpathcurveto{\pgfqpoint{9.645232in}{1.786665in}}{\pgfqpoint{9.647235in}{1.781827in}}{\pgfqpoint{9.650802in}{1.778261in}}%
\pgfpathcurveto{\pgfqpoint{9.654368in}{1.774695in}}{\pgfqpoint{9.659206in}{1.772691in}}{\pgfqpoint{9.664250in}{1.772691in}}%
\pgfpathclose%
\pgfusepath{fill}%
\end{pgfscope}%
\begin{pgfscope}%
\pgfpathrectangle{\pgfqpoint{6.572727in}{0.473000in}}{\pgfqpoint{4.227273in}{3.311000in}}%
\pgfusepath{clip}%
\pgfsetbuttcap%
\pgfsetroundjoin%
\definecolor{currentfill}{rgb}{0.127568,0.566949,0.550556}%
\pgfsetfillcolor{currentfill}%
\pgfsetfillopacity{0.700000}%
\pgfsetlinewidth{0.000000pt}%
\definecolor{currentstroke}{rgb}{0.000000,0.000000,0.000000}%
\pgfsetstrokecolor{currentstroke}%
\pgfsetstrokeopacity{0.700000}%
\pgfsetdash{}{0pt}%
\pgfpathmoveto{\pgfqpoint{7.774562in}{2.611523in}}%
\pgfpathcurveto{\pgfqpoint{7.779606in}{2.611523in}}{\pgfqpoint{7.784444in}{2.613526in}}{\pgfqpoint{7.788010in}{2.617093in}}%
\pgfpathcurveto{\pgfqpoint{7.791577in}{2.620659in}}{\pgfqpoint{7.793580in}{2.625497in}}{\pgfqpoint{7.793580in}{2.630541in}}%
\pgfpathcurveto{\pgfqpoint{7.793580in}{2.635584in}}{\pgfqpoint{7.791577in}{2.640422in}}{\pgfqpoint{7.788010in}{2.643989in}}%
\pgfpathcurveto{\pgfqpoint{7.784444in}{2.647555in}}{\pgfqpoint{7.779606in}{2.649559in}}{\pgfqpoint{7.774562in}{2.649559in}}%
\pgfpathcurveto{\pgfqpoint{7.769519in}{2.649559in}}{\pgfqpoint{7.764681in}{2.647555in}}{\pgfqpoint{7.761114in}{2.643989in}}%
\pgfpathcurveto{\pgfqpoint{7.757548in}{2.640422in}}{\pgfqpoint{7.755544in}{2.635584in}}{\pgfqpoint{7.755544in}{2.630541in}}%
\pgfpathcurveto{\pgfqpoint{7.755544in}{2.625497in}}{\pgfqpoint{7.757548in}{2.620659in}}{\pgfqpoint{7.761114in}{2.617093in}}%
\pgfpathcurveto{\pgfqpoint{7.764681in}{2.613526in}}{\pgfqpoint{7.769519in}{2.611523in}}{\pgfqpoint{7.774562in}{2.611523in}}%
\pgfpathclose%
\pgfusepath{fill}%
\end{pgfscope}%
\begin{pgfscope}%
\pgfpathrectangle{\pgfqpoint{6.572727in}{0.473000in}}{\pgfqpoint{4.227273in}{3.311000in}}%
\pgfusepath{clip}%
\pgfsetbuttcap%
\pgfsetroundjoin%
\definecolor{currentfill}{rgb}{0.127568,0.566949,0.550556}%
\pgfsetfillcolor{currentfill}%
\pgfsetfillopacity{0.700000}%
\pgfsetlinewidth{0.000000pt}%
\definecolor{currentstroke}{rgb}{0.000000,0.000000,0.000000}%
\pgfsetstrokecolor{currentstroke}%
\pgfsetstrokeopacity{0.700000}%
\pgfsetdash{}{0pt}%
\pgfpathmoveto{\pgfqpoint{7.723091in}{1.507923in}}%
\pgfpathcurveto{\pgfqpoint{7.728135in}{1.507923in}}{\pgfqpoint{7.732972in}{1.509927in}}{\pgfqpoint{7.736539in}{1.513494in}}%
\pgfpathcurveto{\pgfqpoint{7.740105in}{1.517060in}}{\pgfqpoint{7.742109in}{1.521898in}}{\pgfqpoint{7.742109in}{1.526941in}}%
\pgfpathcurveto{\pgfqpoint{7.742109in}{1.531985in}}{\pgfqpoint{7.740105in}{1.536823in}}{\pgfqpoint{7.736539in}{1.540389in}}%
\pgfpathcurveto{\pgfqpoint{7.732972in}{1.543956in}}{\pgfqpoint{7.728135in}{1.545960in}}{\pgfqpoint{7.723091in}{1.545960in}}%
\pgfpathcurveto{\pgfqpoint{7.718047in}{1.545960in}}{\pgfqpoint{7.713210in}{1.543956in}}{\pgfqpoint{7.709643in}{1.540389in}}%
\pgfpathcurveto{\pgfqpoint{7.706077in}{1.536823in}}{\pgfqpoint{7.704073in}{1.531985in}}{\pgfqpoint{7.704073in}{1.526941in}}%
\pgfpathcurveto{\pgfqpoint{7.704073in}{1.521898in}}{\pgfqpoint{7.706077in}{1.517060in}}{\pgfqpoint{7.709643in}{1.513494in}}%
\pgfpathcurveto{\pgfqpoint{7.713210in}{1.509927in}}{\pgfqpoint{7.718047in}{1.507923in}}{\pgfqpoint{7.723091in}{1.507923in}}%
\pgfpathclose%
\pgfusepath{fill}%
\end{pgfscope}%
\begin{pgfscope}%
\pgfpathrectangle{\pgfqpoint{6.572727in}{0.473000in}}{\pgfqpoint{4.227273in}{3.311000in}}%
\pgfusepath{clip}%
\pgfsetbuttcap%
\pgfsetroundjoin%
\definecolor{currentfill}{rgb}{0.993248,0.906157,0.143936}%
\pgfsetfillcolor{currentfill}%
\pgfsetfillopacity{0.700000}%
\pgfsetlinewidth{0.000000pt}%
\definecolor{currentstroke}{rgb}{0.000000,0.000000,0.000000}%
\pgfsetstrokecolor{currentstroke}%
\pgfsetstrokeopacity{0.700000}%
\pgfsetdash{}{0pt}%
\pgfpathmoveto{\pgfqpoint{9.771242in}{2.200736in}}%
\pgfpathcurveto{\pgfqpoint{9.776285in}{2.200736in}}{\pgfqpoint{9.781123in}{2.202740in}}{\pgfqpoint{9.784689in}{2.206307in}}%
\pgfpathcurveto{\pgfqpoint{9.788256in}{2.209873in}}{\pgfqpoint{9.790260in}{2.214711in}}{\pgfqpoint{9.790260in}{2.219754in}}%
\pgfpathcurveto{\pgfqpoint{9.790260in}{2.224798in}}{\pgfqpoint{9.788256in}{2.229636in}}{\pgfqpoint{9.784689in}{2.233202in}}%
\pgfpathcurveto{\pgfqpoint{9.781123in}{2.236769in}}{\pgfqpoint{9.776285in}{2.238773in}}{\pgfqpoint{9.771242in}{2.238773in}}%
\pgfpathcurveto{\pgfqpoint{9.766198in}{2.238773in}}{\pgfqpoint{9.761360in}{2.236769in}}{\pgfqpoint{9.757794in}{2.233202in}}%
\pgfpathcurveto{\pgfqpoint{9.754227in}{2.229636in}}{\pgfqpoint{9.752223in}{2.224798in}}{\pgfqpoint{9.752223in}{2.219754in}}%
\pgfpathcurveto{\pgfqpoint{9.752223in}{2.214711in}}{\pgfqpoint{9.754227in}{2.209873in}}{\pgfqpoint{9.757794in}{2.206307in}}%
\pgfpathcurveto{\pgfqpoint{9.761360in}{2.202740in}}{\pgfqpoint{9.766198in}{2.200736in}}{\pgfqpoint{9.771242in}{2.200736in}}%
\pgfpathclose%
\pgfusepath{fill}%
\end{pgfscope}%
\begin{pgfscope}%
\pgfpathrectangle{\pgfqpoint{6.572727in}{0.473000in}}{\pgfqpoint{4.227273in}{3.311000in}}%
\pgfusepath{clip}%
\pgfsetbuttcap%
\pgfsetroundjoin%
\definecolor{currentfill}{rgb}{0.993248,0.906157,0.143936}%
\pgfsetfillcolor{currentfill}%
\pgfsetfillopacity{0.700000}%
\pgfsetlinewidth{0.000000pt}%
\definecolor{currentstroke}{rgb}{0.000000,0.000000,0.000000}%
\pgfsetstrokecolor{currentstroke}%
\pgfsetstrokeopacity{0.700000}%
\pgfsetdash{}{0pt}%
\pgfpathmoveto{\pgfqpoint{9.498366in}{2.233539in}}%
\pgfpathcurveto{\pgfqpoint{9.503410in}{2.233539in}}{\pgfqpoint{9.508248in}{2.235543in}}{\pgfqpoint{9.511814in}{2.239110in}}%
\pgfpathcurveto{\pgfqpoint{9.515381in}{2.242676in}}{\pgfqpoint{9.517385in}{2.247514in}}{\pgfqpoint{9.517385in}{2.252558in}}%
\pgfpathcurveto{\pgfqpoint{9.517385in}{2.257601in}}{\pgfqpoint{9.515381in}{2.262439in}}{\pgfqpoint{9.511814in}{2.266005in}}%
\pgfpathcurveto{\pgfqpoint{9.508248in}{2.269572in}}{\pgfqpoint{9.503410in}{2.271576in}}{\pgfqpoint{9.498366in}{2.271576in}}%
\pgfpathcurveto{\pgfqpoint{9.493323in}{2.271576in}}{\pgfqpoint{9.488485in}{2.269572in}}{\pgfqpoint{9.484919in}{2.266005in}}%
\pgfpathcurveto{\pgfqpoint{9.481352in}{2.262439in}}{\pgfqpoint{9.479348in}{2.257601in}}{\pgfqpoint{9.479348in}{2.252558in}}%
\pgfpathcurveto{\pgfqpoint{9.479348in}{2.247514in}}{\pgfqpoint{9.481352in}{2.242676in}}{\pgfqpoint{9.484919in}{2.239110in}}%
\pgfpathcurveto{\pgfqpoint{9.488485in}{2.235543in}}{\pgfqpoint{9.493323in}{2.233539in}}{\pgfqpoint{9.498366in}{2.233539in}}%
\pgfpathclose%
\pgfusepath{fill}%
\end{pgfscope}%
\begin{pgfscope}%
\pgfpathrectangle{\pgfqpoint{6.572727in}{0.473000in}}{\pgfqpoint{4.227273in}{3.311000in}}%
\pgfusepath{clip}%
\pgfsetbuttcap%
\pgfsetroundjoin%
\definecolor{currentfill}{rgb}{0.127568,0.566949,0.550556}%
\pgfsetfillcolor{currentfill}%
\pgfsetfillopacity{0.700000}%
\pgfsetlinewidth{0.000000pt}%
\definecolor{currentstroke}{rgb}{0.000000,0.000000,0.000000}%
\pgfsetstrokecolor{currentstroke}%
\pgfsetstrokeopacity{0.700000}%
\pgfsetdash{}{0pt}%
\pgfpathmoveto{\pgfqpoint{8.061790in}{1.741150in}}%
\pgfpathcurveto{\pgfqpoint{8.066834in}{1.741150in}}{\pgfqpoint{8.071671in}{1.743153in}}{\pgfqpoint{8.075238in}{1.746720in}}%
\pgfpathcurveto{\pgfqpoint{8.078804in}{1.750286in}}{\pgfqpoint{8.080808in}{1.755124in}}{\pgfqpoint{8.080808in}{1.760168in}}%
\pgfpathcurveto{\pgfqpoint{8.080808in}{1.765211in}}{\pgfqpoint{8.078804in}{1.770049in}}{\pgfqpoint{8.075238in}{1.773616in}}%
\pgfpathcurveto{\pgfqpoint{8.071671in}{1.777182in}}{\pgfqpoint{8.066834in}{1.779186in}}{\pgfqpoint{8.061790in}{1.779186in}}%
\pgfpathcurveto{\pgfqpoint{8.056746in}{1.779186in}}{\pgfqpoint{8.051908in}{1.777182in}}{\pgfqpoint{8.048342in}{1.773616in}}%
\pgfpathcurveto{\pgfqpoint{8.044776in}{1.770049in}}{\pgfqpoint{8.042772in}{1.765211in}}{\pgfqpoint{8.042772in}{1.760168in}}%
\pgfpathcurveto{\pgfqpoint{8.042772in}{1.755124in}}{\pgfqpoint{8.044776in}{1.750286in}}{\pgfqpoint{8.048342in}{1.746720in}}%
\pgfpathcurveto{\pgfqpoint{8.051908in}{1.743153in}}{\pgfqpoint{8.056746in}{1.741150in}}{\pgfqpoint{8.061790in}{1.741150in}}%
\pgfpathclose%
\pgfusepath{fill}%
\end{pgfscope}%
\begin{pgfscope}%
\pgfpathrectangle{\pgfqpoint{6.572727in}{0.473000in}}{\pgfqpoint{4.227273in}{3.311000in}}%
\pgfusepath{clip}%
\pgfsetbuttcap%
\pgfsetroundjoin%
\definecolor{currentfill}{rgb}{0.127568,0.566949,0.550556}%
\pgfsetfillcolor{currentfill}%
\pgfsetfillopacity{0.700000}%
\pgfsetlinewidth{0.000000pt}%
\definecolor{currentstroke}{rgb}{0.000000,0.000000,0.000000}%
\pgfsetstrokecolor{currentstroke}%
\pgfsetstrokeopacity{0.700000}%
\pgfsetdash{}{0pt}%
\pgfpathmoveto{\pgfqpoint{7.891853in}{2.542040in}}%
\pgfpathcurveto{\pgfqpoint{7.896897in}{2.542040in}}{\pgfqpoint{7.901735in}{2.544044in}}{\pgfqpoint{7.905301in}{2.547610in}}%
\pgfpathcurveto{\pgfqpoint{7.908867in}{2.551177in}}{\pgfqpoint{7.910871in}{2.556014in}}{\pgfqpoint{7.910871in}{2.561058in}}%
\pgfpathcurveto{\pgfqpoint{7.910871in}{2.566102in}}{\pgfqpoint{7.908867in}{2.570940in}}{\pgfqpoint{7.905301in}{2.574506in}}%
\pgfpathcurveto{\pgfqpoint{7.901735in}{2.578072in}}{\pgfqpoint{7.896897in}{2.580076in}}{\pgfqpoint{7.891853in}{2.580076in}}%
\pgfpathcurveto{\pgfqpoint{7.886809in}{2.580076in}}{\pgfqpoint{7.881972in}{2.578072in}}{\pgfqpoint{7.878405in}{2.574506in}}%
\pgfpathcurveto{\pgfqpoint{7.874839in}{2.570940in}}{\pgfqpoint{7.872835in}{2.566102in}}{\pgfqpoint{7.872835in}{2.561058in}}%
\pgfpathcurveto{\pgfqpoint{7.872835in}{2.556014in}}{\pgfqpoint{7.874839in}{2.551177in}}{\pgfqpoint{7.878405in}{2.547610in}}%
\pgfpathcurveto{\pgfqpoint{7.881972in}{2.544044in}}{\pgfqpoint{7.886809in}{2.542040in}}{\pgfqpoint{7.891853in}{2.542040in}}%
\pgfpathclose%
\pgfusepath{fill}%
\end{pgfscope}%
\begin{pgfscope}%
\pgfpathrectangle{\pgfqpoint{6.572727in}{0.473000in}}{\pgfqpoint{4.227273in}{3.311000in}}%
\pgfusepath{clip}%
\pgfsetbuttcap%
\pgfsetroundjoin%
\definecolor{currentfill}{rgb}{0.993248,0.906157,0.143936}%
\pgfsetfillcolor{currentfill}%
\pgfsetfillopacity{0.700000}%
\pgfsetlinewidth{0.000000pt}%
\definecolor{currentstroke}{rgb}{0.000000,0.000000,0.000000}%
\pgfsetstrokecolor{currentstroke}%
\pgfsetstrokeopacity{0.700000}%
\pgfsetdash{}{0pt}%
\pgfpathmoveto{\pgfqpoint{9.984279in}{1.524155in}}%
\pgfpathcurveto{\pgfqpoint{9.989322in}{1.524155in}}{\pgfqpoint{9.994160in}{1.526159in}}{\pgfqpoint{9.997727in}{1.529725in}}%
\pgfpathcurveto{\pgfqpoint{10.001293in}{1.533292in}}{\pgfqpoint{10.003297in}{1.538130in}}{\pgfqpoint{10.003297in}{1.543173in}}%
\pgfpathcurveto{\pgfqpoint{10.003297in}{1.548217in}}{\pgfqpoint{10.001293in}{1.553055in}}{\pgfqpoint{9.997727in}{1.556621in}}%
\pgfpathcurveto{\pgfqpoint{9.994160in}{1.560188in}}{\pgfqpoint{9.989322in}{1.562191in}}{\pgfqpoint{9.984279in}{1.562191in}}%
\pgfpathcurveto{\pgfqpoint{9.979235in}{1.562191in}}{\pgfqpoint{9.974397in}{1.560188in}}{\pgfqpoint{9.970831in}{1.556621in}}%
\pgfpathcurveto{\pgfqpoint{9.967264in}{1.553055in}}{\pgfqpoint{9.965261in}{1.548217in}}{\pgfqpoint{9.965261in}{1.543173in}}%
\pgfpathcurveto{\pgfqpoint{9.965261in}{1.538130in}}{\pgfqpoint{9.967264in}{1.533292in}}{\pgfqpoint{9.970831in}{1.529725in}}%
\pgfpathcurveto{\pgfqpoint{9.974397in}{1.526159in}}{\pgfqpoint{9.979235in}{1.524155in}}{\pgfqpoint{9.984279in}{1.524155in}}%
\pgfpathclose%
\pgfusepath{fill}%
\end{pgfscope}%
\begin{pgfscope}%
\pgfpathrectangle{\pgfqpoint{6.572727in}{0.473000in}}{\pgfqpoint{4.227273in}{3.311000in}}%
\pgfusepath{clip}%
\pgfsetbuttcap%
\pgfsetroundjoin%
\definecolor{currentfill}{rgb}{0.127568,0.566949,0.550556}%
\pgfsetfillcolor{currentfill}%
\pgfsetfillopacity{0.700000}%
\pgfsetlinewidth{0.000000pt}%
\definecolor{currentstroke}{rgb}{0.000000,0.000000,0.000000}%
\pgfsetstrokecolor{currentstroke}%
\pgfsetstrokeopacity{0.700000}%
\pgfsetdash{}{0pt}%
\pgfpathmoveto{\pgfqpoint{7.950170in}{2.120265in}}%
\pgfpathcurveto{\pgfqpoint{7.955214in}{2.120265in}}{\pgfqpoint{7.960052in}{2.122269in}}{\pgfqpoint{7.963618in}{2.125836in}}%
\pgfpathcurveto{\pgfqpoint{7.967185in}{2.129402in}}{\pgfqpoint{7.969189in}{2.134240in}}{\pgfqpoint{7.969189in}{2.139283in}}%
\pgfpathcurveto{\pgfqpoint{7.969189in}{2.144327in}}{\pgfqpoint{7.967185in}{2.149165in}}{\pgfqpoint{7.963618in}{2.152731in}}%
\pgfpathcurveto{\pgfqpoint{7.960052in}{2.156298in}}{\pgfqpoint{7.955214in}{2.158302in}}{\pgfqpoint{7.950170in}{2.158302in}}%
\pgfpathcurveto{\pgfqpoint{7.945127in}{2.158302in}}{\pgfqpoint{7.940289in}{2.156298in}}{\pgfqpoint{7.936723in}{2.152731in}}%
\pgfpathcurveto{\pgfqpoint{7.933156in}{2.149165in}}{\pgfqpoint{7.931152in}{2.144327in}}{\pgfqpoint{7.931152in}{2.139283in}}%
\pgfpathcurveto{\pgfqpoint{7.931152in}{2.134240in}}{\pgfqpoint{7.933156in}{2.129402in}}{\pgfqpoint{7.936723in}{2.125836in}}%
\pgfpathcurveto{\pgfqpoint{7.940289in}{2.122269in}}{\pgfqpoint{7.945127in}{2.120265in}}{\pgfqpoint{7.950170in}{2.120265in}}%
\pgfpathclose%
\pgfusepath{fill}%
\end{pgfscope}%
\begin{pgfscope}%
\pgfpathrectangle{\pgfqpoint{6.572727in}{0.473000in}}{\pgfqpoint{4.227273in}{3.311000in}}%
\pgfusepath{clip}%
\pgfsetbuttcap%
\pgfsetroundjoin%
\definecolor{currentfill}{rgb}{0.993248,0.906157,0.143936}%
\pgfsetfillcolor{currentfill}%
\pgfsetfillopacity{0.700000}%
\pgfsetlinewidth{0.000000pt}%
\definecolor{currentstroke}{rgb}{0.000000,0.000000,0.000000}%
\pgfsetstrokecolor{currentstroke}%
\pgfsetstrokeopacity{0.700000}%
\pgfsetdash{}{0pt}%
\pgfpathmoveto{\pgfqpoint{10.138325in}{1.665157in}}%
\pgfpathcurveto{\pgfqpoint{10.143369in}{1.665157in}}{\pgfqpoint{10.148207in}{1.667161in}}{\pgfqpoint{10.151773in}{1.670727in}}%
\pgfpathcurveto{\pgfqpoint{10.155340in}{1.674294in}}{\pgfqpoint{10.157343in}{1.679131in}}{\pgfqpoint{10.157343in}{1.684175in}}%
\pgfpathcurveto{\pgfqpoint{10.157343in}{1.689219in}}{\pgfqpoint{10.155340in}{1.694056in}}{\pgfqpoint{10.151773in}{1.697623in}}%
\pgfpathcurveto{\pgfqpoint{10.148207in}{1.701189in}}{\pgfqpoint{10.143369in}{1.703193in}}{\pgfqpoint{10.138325in}{1.703193in}}%
\pgfpathcurveto{\pgfqpoint{10.133282in}{1.703193in}}{\pgfqpoint{10.128444in}{1.701189in}}{\pgfqpoint{10.124877in}{1.697623in}}%
\pgfpathcurveto{\pgfqpoint{10.121311in}{1.694056in}}{\pgfqpoint{10.119307in}{1.689219in}}{\pgfqpoint{10.119307in}{1.684175in}}%
\pgfpathcurveto{\pgfqpoint{10.119307in}{1.679131in}}{\pgfqpoint{10.121311in}{1.674294in}}{\pgfqpoint{10.124877in}{1.670727in}}%
\pgfpathcurveto{\pgfqpoint{10.128444in}{1.667161in}}{\pgfqpoint{10.133282in}{1.665157in}}{\pgfqpoint{10.138325in}{1.665157in}}%
\pgfpathclose%
\pgfusepath{fill}%
\end{pgfscope}%
\begin{pgfscope}%
\pgfpathrectangle{\pgfqpoint{6.572727in}{0.473000in}}{\pgfqpoint{4.227273in}{3.311000in}}%
\pgfusepath{clip}%
\pgfsetbuttcap%
\pgfsetroundjoin%
\definecolor{currentfill}{rgb}{0.127568,0.566949,0.550556}%
\pgfsetfillcolor{currentfill}%
\pgfsetfillopacity{0.700000}%
\pgfsetlinewidth{0.000000pt}%
\definecolor{currentstroke}{rgb}{0.000000,0.000000,0.000000}%
\pgfsetstrokecolor{currentstroke}%
\pgfsetstrokeopacity{0.700000}%
\pgfsetdash{}{0pt}%
\pgfpathmoveto{\pgfqpoint{8.195980in}{2.707754in}}%
\pgfpathcurveto{\pgfqpoint{8.201024in}{2.707754in}}{\pgfqpoint{8.205861in}{2.709758in}}{\pgfqpoint{8.209428in}{2.713324in}}%
\pgfpathcurveto{\pgfqpoint{8.212994in}{2.716891in}}{\pgfqpoint{8.214998in}{2.721729in}}{\pgfqpoint{8.214998in}{2.726772in}}%
\pgfpathcurveto{\pgfqpoint{8.214998in}{2.731816in}}{\pgfqpoint{8.212994in}{2.736654in}}{\pgfqpoint{8.209428in}{2.740220in}}%
\pgfpathcurveto{\pgfqpoint{8.205861in}{2.743787in}}{\pgfqpoint{8.201024in}{2.745790in}}{\pgfqpoint{8.195980in}{2.745790in}}%
\pgfpathcurveto{\pgfqpoint{8.190936in}{2.745790in}}{\pgfqpoint{8.186099in}{2.743787in}}{\pgfqpoint{8.182532in}{2.740220in}}%
\pgfpathcurveto{\pgfqpoint{8.178966in}{2.736654in}}{\pgfqpoint{8.176962in}{2.731816in}}{\pgfqpoint{8.176962in}{2.726772in}}%
\pgfpathcurveto{\pgfqpoint{8.176962in}{2.721729in}}{\pgfqpoint{8.178966in}{2.716891in}}{\pgfqpoint{8.182532in}{2.713324in}}%
\pgfpathcurveto{\pgfqpoint{8.186099in}{2.709758in}}{\pgfqpoint{8.190936in}{2.707754in}}{\pgfqpoint{8.195980in}{2.707754in}}%
\pgfpathclose%
\pgfusepath{fill}%
\end{pgfscope}%
\begin{pgfscope}%
\pgfpathrectangle{\pgfqpoint{6.572727in}{0.473000in}}{\pgfqpoint{4.227273in}{3.311000in}}%
\pgfusepath{clip}%
\pgfsetbuttcap%
\pgfsetroundjoin%
\definecolor{currentfill}{rgb}{0.993248,0.906157,0.143936}%
\pgfsetfillcolor{currentfill}%
\pgfsetfillopacity{0.700000}%
\pgfsetlinewidth{0.000000pt}%
\definecolor{currentstroke}{rgb}{0.000000,0.000000,0.000000}%
\pgfsetstrokecolor{currentstroke}%
\pgfsetstrokeopacity{0.700000}%
\pgfsetdash{}{0pt}%
\pgfpathmoveto{\pgfqpoint{9.428478in}{1.610914in}}%
\pgfpathcurveto{\pgfqpoint{9.433522in}{1.610914in}}{\pgfqpoint{9.438360in}{1.612918in}}{\pgfqpoint{9.441926in}{1.616485in}}%
\pgfpathcurveto{\pgfqpoint{9.445492in}{1.620051in}}{\pgfqpoint{9.447496in}{1.624889in}}{\pgfqpoint{9.447496in}{1.629933in}}%
\pgfpathcurveto{\pgfqpoint{9.447496in}{1.634976in}}{\pgfqpoint{9.445492in}{1.639814in}}{\pgfqpoint{9.441926in}{1.643380in}}%
\pgfpathcurveto{\pgfqpoint{9.438360in}{1.646947in}}{\pgfqpoint{9.433522in}{1.648951in}}{\pgfqpoint{9.428478in}{1.648951in}}%
\pgfpathcurveto{\pgfqpoint{9.423434in}{1.648951in}}{\pgfqpoint{9.418597in}{1.646947in}}{\pgfqpoint{9.415030in}{1.643380in}}%
\pgfpathcurveto{\pgfqpoint{9.411464in}{1.639814in}}{\pgfqpoint{9.409460in}{1.634976in}}{\pgfqpoint{9.409460in}{1.629933in}}%
\pgfpathcurveto{\pgfqpoint{9.409460in}{1.624889in}}{\pgfqpoint{9.411464in}{1.620051in}}{\pgfqpoint{9.415030in}{1.616485in}}%
\pgfpathcurveto{\pgfqpoint{9.418597in}{1.612918in}}{\pgfqpoint{9.423434in}{1.610914in}}{\pgfqpoint{9.428478in}{1.610914in}}%
\pgfpathclose%
\pgfusepath{fill}%
\end{pgfscope}%
\begin{pgfscope}%
\pgfpathrectangle{\pgfqpoint{6.572727in}{0.473000in}}{\pgfqpoint{4.227273in}{3.311000in}}%
\pgfusepath{clip}%
\pgfsetbuttcap%
\pgfsetroundjoin%
\definecolor{currentfill}{rgb}{0.993248,0.906157,0.143936}%
\pgfsetfillcolor{currentfill}%
\pgfsetfillopacity{0.700000}%
\pgfsetlinewidth{0.000000pt}%
\definecolor{currentstroke}{rgb}{0.000000,0.000000,0.000000}%
\pgfsetstrokecolor{currentstroke}%
\pgfsetstrokeopacity{0.700000}%
\pgfsetdash{}{0pt}%
\pgfpathmoveto{\pgfqpoint{10.116195in}{1.305248in}}%
\pgfpathcurveto{\pgfqpoint{10.121239in}{1.305248in}}{\pgfqpoint{10.126077in}{1.307252in}}{\pgfqpoint{10.129643in}{1.310819in}}%
\pgfpathcurveto{\pgfqpoint{10.133209in}{1.314385in}}{\pgfqpoint{10.135213in}{1.319223in}}{\pgfqpoint{10.135213in}{1.324267in}}%
\pgfpathcurveto{\pgfqpoint{10.135213in}{1.329310in}}{\pgfqpoint{10.133209in}{1.334148in}}{\pgfqpoint{10.129643in}{1.337714in}}%
\pgfpathcurveto{\pgfqpoint{10.126077in}{1.341281in}}{\pgfqpoint{10.121239in}{1.343285in}}{\pgfqpoint{10.116195in}{1.343285in}}%
\pgfpathcurveto{\pgfqpoint{10.111152in}{1.343285in}}{\pgfqpoint{10.106314in}{1.341281in}}{\pgfqpoint{10.102747in}{1.337714in}}%
\pgfpathcurveto{\pgfqpoint{10.099181in}{1.334148in}}{\pgfqpoint{10.097177in}{1.329310in}}{\pgfqpoint{10.097177in}{1.324267in}}%
\pgfpathcurveto{\pgfqpoint{10.097177in}{1.319223in}}{\pgfqpoint{10.099181in}{1.314385in}}{\pgfqpoint{10.102747in}{1.310819in}}%
\pgfpathcurveto{\pgfqpoint{10.106314in}{1.307252in}}{\pgfqpoint{10.111152in}{1.305248in}}{\pgfqpoint{10.116195in}{1.305248in}}%
\pgfpathclose%
\pgfusepath{fill}%
\end{pgfscope}%
\begin{pgfscope}%
\pgfpathrectangle{\pgfqpoint{6.572727in}{0.473000in}}{\pgfqpoint{4.227273in}{3.311000in}}%
\pgfusepath{clip}%
\pgfsetbuttcap%
\pgfsetroundjoin%
\definecolor{currentfill}{rgb}{0.127568,0.566949,0.550556}%
\pgfsetfillcolor{currentfill}%
\pgfsetfillopacity{0.700000}%
\pgfsetlinewidth{0.000000pt}%
\definecolor{currentstroke}{rgb}{0.000000,0.000000,0.000000}%
\pgfsetstrokecolor{currentstroke}%
\pgfsetstrokeopacity{0.700000}%
\pgfsetdash{}{0pt}%
\pgfpathmoveto{\pgfqpoint{8.103464in}{2.658894in}}%
\pgfpathcurveto{\pgfqpoint{8.108508in}{2.658894in}}{\pgfqpoint{8.113345in}{2.660898in}}{\pgfqpoint{8.116912in}{2.664464in}}%
\pgfpathcurveto{\pgfqpoint{8.120478in}{2.668031in}}{\pgfqpoint{8.122482in}{2.672868in}}{\pgfqpoint{8.122482in}{2.677912in}}%
\pgfpathcurveto{\pgfqpoint{8.122482in}{2.682956in}}{\pgfqpoint{8.120478in}{2.687794in}}{\pgfqpoint{8.116912in}{2.691360in}}%
\pgfpathcurveto{\pgfqpoint{8.113345in}{2.694926in}}{\pgfqpoint{8.108508in}{2.696930in}}{\pgfqpoint{8.103464in}{2.696930in}}%
\pgfpathcurveto{\pgfqpoint{8.098420in}{2.696930in}}{\pgfqpoint{8.093582in}{2.694926in}}{\pgfqpoint{8.090016in}{2.691360in}}%
\pgfpathcurveto{\pgfqpoint{8.086450in}{2.687794in}}{\pgfqpoint{8.084446in}{2.682956in}}{\pgfqpoint{8.084446in}{2.677912in}}%
\pgfpathcurveto{\pgfqpoint{8.084446in}{2.672868in}}{\pgfqpoint{8.086450in}{2.668031in}}{\pgfqpoint{8.090016in}{2.664464in}}%
\pgfpathcurveto{\pgfqpoint{8.093582in}{2.660898in}}{\pgfqpoint{8.098420in}{2.658894in}}{\pgfqpoint{8.103464in}{2.658894in}}%
\pgfpathclose%
\pgfusepath{fill}%
\end{pgfscope}%
\begin{pgfscope}%
\pgfpathrectangle{\pgfqpoint{6.572727in}{0.473000in}}{\pgfqpoint{4.227273in}{3.311000in}}%
\pgfusepath{clip}%
\pgfsetbuttcap%
\pgfsetroundjoin%
\definecolor{currentfill}{rgb}{0.993248,0.906157,0.143936}%
\pgfsetfillcolor{currentfill}%
\pgfsetfillopacity{0.700000}%
\pgfsetlinewidth{0.000000pt}%
\definecolor{currentstroke}{rgb}{0.000000,0.000000,0.000000}%
\pgfsetstrokecolor{currentstroke}%
\pgfsetstrokeopacity{0.700000}%
\pgfsetdash{}{0pt}%
\pgfpathmoveto{\pgfqpoint{9.523161in}{1.112130in}}%
\pgfpathcurveto{\pgfqpoint{9.528205in}{1.112130in}}{\pgfqpoint{9.533043in}{1.114134in}}{\pgfqpoint{9.536609in}{1.117700in}}%
\pgfpathcurveto{\pgfqpoint{9.540176in}{1.121267in}}{\pgfqpoint{9.542180in}{1.126104in}}{\pgfqpoint{9.542180in}{1.131148in}}%
\pgfpathcurveto{\pgfqpoint{9.542180in}{1.136192in}}{\pgfqpoint{9.540176in}{1.141029in}}{\pgfqpoint{9.536609in}{1.144596in}}%
\pgfpathcurveto{\pgfqpoint{9.533043in}{1.148162in}}{\pgfqpoint{9.528205in}{1.150166in}}{\pgfqpoint{9.523161in}{1.150166in}}%
\pgfpathcurveto{\pgfqpoint{9.518118in}{1.150166in}}{\pgfqpoint{9.513280in}{1.148162in}}{\pgfqpoint{9.509714in}{1.144596in}}%
\pgfpathcurveto{\pgfqpoint{9.506147in}{1.141029in}}{\pgfqpoint{9.504143in}{1.136192in}}{\pgfqpoint{9.504143in}{1.131148in}}%
\pgfpathcurveto{\pgfqpoint{9.504143in}{1.126104in}}{\pgfqpoint{9.506147in}{1.121267in}}{\pgfqpoint{9.509714in}{1.117700in}}%
\pgfpathcurveto{\pgfqpoint{9.513280in}{1.114134in}}{\pgfqpoint{9.518118in}{1.112130in}}{\pgfqpoint{9.523161in}{1.112130in}}%
\pgfpathclose%
\pgfusepath{fill}%
\end{pgfscope}%
\begin{pgfscope}%
\pgfpathrectangle{\pgfqpoint{6.572727in}{0.473000in}}{\pgfqpoint{4.227273in}{3.311000in}}%
\pgfusepath{clip}%
\pgfsetbuttcap%
\pgfsetroundjoin%
\definecolor{currentfill}{rgb}{0.993248,0.906157,0.143936}%
\pgfsetfillcolor{currentfill}%
\pgfsetfillopacity{0.700000}%
\pgfsetlinewidth{0.000000pt}%
\definecolor{currentstroke}{rgb}{0.000000,0.000000,0.000000}%
\pgfsetstrokecolor{currentstroke}%
\pgfsetstrokeopacity{0.700000}%
\pgfsetdash{}{0pt}%
\pgfpathmoveto{\pgfqpoint{9.822215in}{1.792320in}}%
\pgfpathcurveto{\pgfqpoint{9.827259in}{1.792320in}}{\pgfqpoint{9.832097in}{1.794323in}}{\pgfqpoint{9.835663in}{1.797890in}}%
\pgfpathcurveto{\pgfqpoint{9.839229in}{1.801456in}}{\pgfqpoint{9.841233in}{1.806294in}}{\pgfqpoint{9.841233in}{1.811338in}}%
\pgfpathcurveto{\pgfqpoint{9.841233in}{1.816381in}}{\pgfqpoint{9.839229in}{1.821219in}}{\pgfqpoint{9.835663in}{1.824786in}}%
\pgfpathcurveto{\pgfqpoint{9.832097in}{1.828352in}}{\pgfqpoint{9.827259in}{1.830356in}}{\pgfqpoint{9.822215in}{1.830356in}}%
\pgfpathcurveto{\pgfqpoint{9.817171in}{1.830356in}}{\pgfqpoint{9.812334in}{1.828352in}}{\pgfqpoint{9.808767in}{1.824786in}}%
\pgfpathcurveto{\pgfqpoint{9.805201in}{1.821219in}}{\pgfqpoint{9.803197in}{1.816381in}}{\pgfqpoint{9.803197in}{1.811338in}}%
\pgfpathcurveto{\pgfqpoint{9.803197in}{1.806294in}}{\pgfqpoint{9.805201in}{1.801456in}}{\pgfqpoint{9.808767in}{1.797890in}}%
\pgfpathcurveto{\pgfqpoint{9.812334in}{1.794323in}}{\pgfqpoint{9.817171in}{1.792320in}}{\pgfqpoint{9.822215in}{1.792320in}}%
\pgfpathclose%
\pgfusepath{fill}%
\end{pgfscope}%
\begin{pgfscope}%
\pgfpathrectangle{\pgfqpoint{6.572727in}{0.473000in}}{\pgfqpoint{4.227273in}{3.311000in}}%
\pgfusepath{clip}%
\pgfsetbuttcap%
\pgfsetroundjoin%
\definecolor{currentfill}{rgb}{0.993248,0.906157,0.143936}%
\pgfsetfillcolor{currentfill}%
\pgfsetfillopacity{0.700000}%
\pgfsetlinewidth{0.000000pt}%
\definecolor{currentstroke}{rgb}{0.000000,0.000000,0.000000}%
\pgfsetstrokecolor{currentstroke}%
\pgfsetstrokeopacity{0.700000}%
\pgfsetdash{}{0pt}%
\pgfpathmoveto{\pgfqpoint{10.196545in}{1.238225in}}%
\pgfpathcurveto{\pgfqpoint{10.201589in}{1.238225in}}{\pgfqpoint{10.206427in}{1.240229in}}{\pgfqpoint{10.209993in}{1.243796in}}%
\pgfpathcurveto{\pgfqpoint{10.213560in}{1.247362in}}{\pgfqpoint{10.215563in}{1.252200in}}{\pgfqpoint{10.215563in}{1.257244in}}%
\pgfpathcurveto{\pgfqpoint{10.215563in}{1.262287in}}{\pgfqpoint{10.213560in}{1.267125in}}{\pgfqpoint{10.209993in}{1.270691in}}%
\pgfpathcurveto{\pgfqpoint{10.206427in}{1.274258in}}{\pgfqpoint{10.201589in}{1.276262in}}{\pgfqpoint{10.196545in}{1.276262in}}%
\pgfpathcurveto{\pgfqpoint{10.191502in}{1.276262in}}{\pgfqpoint{10.186664in}{1.274258in}}{\pgfqpoint{10.183097in}{1.270691in}}%
\pgfpathcurveto{\pgfqpoint{10.179531in}{1.267125in}}{\pgfqpoint{10.177527in}{1.262287in}}{\pgfqpoint{10.177527in}{1.257244in}}%
\pgfpathcurveto{\pgfqpoint{10.177527in}{1.252200in}}{\pgfqpoint{10.179531in}{1.247362in}}{\pgfqpoint{10.183097in}{1.243796in}}%
\pgfpathcurveto{\pgfqpoint{10.186664in}{1.240229in}}{\pgfqpoint{10.191502in}{1.238225in}}{\pgfqpoint{10.196545in}{1.238225in}}%
\pgfpathclose%
\pgfusepath{fill}%
\end{pgfscope}%
\begin{pgfscope}%
\pgfpathrectangle{\pgfqpoint{6.572727in}{0.473000in}}{\pgfqpoint{4.227273in}{3.311000in}}%
\pgfusepath{clip}%
\pgfsetbuttcap%
\pgfsetroundjoin%
\definecolor{currentfill}{rgb}{0.993248,0.906157,0.143936}%
\pgfsetfillcolor{currentfill}%
\pgfsetfillopacity{0.700000}%
\pgfsetlinewidth{0.000000pt}%
\definecolor{currentstroke}{rgb}{0.000000,0.000000,0.000000}%
\pgfsetstrokecolor{currentstroke}%
\pgfsetstrokeopacity{0.700000}%
\pgfsetdash{}{0pt}%
\pgfpathmoveto{\pgfqpoint{9.040576in}{1.835315in}}%
\pgfpathcurveto{\pgfqpoint{9.045619in}{1.835315in}}{\pgfqpoint{9.050457in}{1.837319in}}{\pgfqpoint{9.054023in}{1.840885in}}%
\pgfpathcurveto{\pgfqpoint{9.057590in}{1.844452in}}{\pgfqpoint{9.059594in}{1.849290in}}{\pgfqpoint{9.059594in}{1.854333in}}%
\pgfpathcurveto{\pgfqpoint{9.059594in}{1.859377in}}{\pgfqpoint{9.057590in}{1.864215in}}{\pgfqpoint{9.054023in}{1.867781in}}%
\pgfpathcurveto{\pgfqpoint{9.050457in}{1.871347in}}{\pgfqpoint{9.045619in}{1.873351in}}{\pgfqpoint{9.040576in}{1.873351in}}%
\pgfpathcurveto{\pgfqpoint{9.035532in}{1.873351in}}{\pgfqpoint{9.030694in}{1.871347in}}{\pgfqpoint{9.027128in}{1.867781in}}%
\pgfpathcurveto{\pgfqpoint{9.023561in}{1.864215in}}{\pgfqpoint{9.021557in}{1.859377in}}{\pgfqpoint{9.021557in}{1.854333in}}%
\pgfpathcurveto{\pgfqpoint{9.021557in}{1.849290in}}{\pgfqpoint{9.023561in}{1.844452in}}{\pgfqpoint{9.027128in}{1.840885in}}%
\pgfpathcurveto{\pgfqpoint{9.030694in}{1.837319in}}{\pgfqpoint{9.035532in}{1.835315in}}{\pgfqpoint{9.040576in}{1.835315in}}%
\pgfpathclose%
\pgfusepath{fill}%
\end{pgfscope}%
\begin{pgfscope}%
\pgfpathrectangle{\pgfqpoint{6.572727in}{0.473000in}}{\pgfqpoint{4.227273in}{3.311000in}}%
\pgfusepath{clip}%
\pgfsetbuttcap%
\pgfsetroundjoin%
\definecolor{currentfill}{rgb}{0.993248,0.906157,0.143936}%
\pgfsetfillcolor{currentfill}%
\pgfsetfillopacity{0.700000}%
\pgfsetlinewidth{0.000000pt}%
\definecolor{currentstroke}{rgb}{0.000000,0.000000,0.000000}%
\pgfsetstrokecolor{currentstroke}%
\pgfsetstrokeopacity{0.700000}%
\pgfsetdash{}{0pt}%
\pgfpathmoveto{\pgfqpoint{9.724121in}{1.315804in}}%
\pgfpathcurveto{\pgfqpoint{9.729165in}{1.315804in}}{\pgfqpoint{9.734003in}{1.317808in}}{\pgfqpoint{9.737569in}{1.321374in}}%
\pgfpathcurveto{\pgfqpoint{9.741135in}{1.324941in}}{\pgfqpoint{9.743139in}{1.329778in}}{\pgfqpoint{9.743139in}{1.334822in}}%
\pgfpathcurveto{\pgfqpoint{9.743139in}{1.339866in}}{\pgfqpoint{9.741135in}{1.344703in}}{\pgfqpoint{9.737569in}{1.348270in}}%
\pgfpathcurveto{\pgfqpoint{9.734003in}{1.351836in}}{\pgfqpoint{9.729165in}{1.353840in}}{\pgfqpoint{9.724121in}{1.353840in}}%
\pgfpathcurveto{\pgfqpoint{9.719077in}{1.353840in}}{\pgfqpoint{9.714240in}{1.351836in}}{\pgfqpoint{9.710673in}{1.348270in}}%
\pgfpathcurveto{\pgfqpoint{9.707107in}{1.344703in}}{\pgfqpoint{9.705103in}{1.339866in}}{\pgfqpoint{9.705103in}{1.334822in}}%
\pgfpathcurveto{\pgfqpoint{9.705103in}{1.329778in}}{\pgfqpoint{9.707107in}{1.324941in}}{\pgfqpoint{9.710673in}{1.321374in}}%
\pgfpathcurveto{\pgfqpoint{9.714240in}{1.317808in}}{\pgfqpoint{9.719077in}{1.315804in}}{\pgfqpoint{9.724121in}{1.315804in}}%
\pgfpathclose%
\pgfusepath{fill}%
\end{pgfscope}%
\begin{pgfscope}%
\pgfpathrectangle{\pgfqpoint{6.572727in}{0.473000in}}{\pgfqpoint{4.227273in}{3.311000in}}%
\pgfusepath{clip}%
\pgfsetbuttcap%
\pgfsetroundjoin%
\definecolor{currentfill}{rgb}{0.993248,0.906157,0.143936}%
\pgfsetfillcolor{currentfill}%
\pgfsetfillopacity{0.700000}%
\pgfsetlinewidth{0.000000pt}%
\definecolor{currentstroke}{rgb}{0.000000,0.000000,0.000000}%
\pgfsetstrokecolor{currentstroke}%
\pgfsetstrokeopacity{0.700000}%
\pgfsetdash{}{0pt}%
\pgfpathmoveto{\pgfqpoint{10.414029in}{1.709472in}}%
\pgfpathcurveto{\pgfqpoint{10.419072in}{1.709472in}}{\pgfqpoint{10.423910in}{1.711476in}}{\pgfqpoint{10.427477in}{1.715042in}}%
\pgfpathcurveto{\pgfqpoint{10.431043in}{1.718609in}}{\pgfqpoint{10.433047in}{1.723447in}}{\pgfqpoint{10.433047in}{1.728490in}}%
\pgfpathcurveto{\pgfqpoint{10.433047in}{1.733534in}}{\pgfqpoint{10.431043in}{1.738372in}}{\pgfqpoint{10.427477in}{1.741938in}}%
\pgfpathcurveto{\pgfqpoint{10.423910in}{1.745505in}}{\pgfqpoint{10.419072in}{1.747508in}}{\pgfqpoint{10.414029in}{1.747508in}}%
\pgfpathcurveto{\pgfqpoint{10.408985in}{1.747508in}}{\pgfqpoint{10.404147in}{1.745505in}}{\pgfqpoint{10.400581in}{1.741938in}}%
\pgfpathcurveto{\pgfqpoint{10.397015in}{1.738372in}}{\pgfqpoint{10.395011in}{1.733534in}}{\pgfqpoint{10.395011in}{1.728490in}}%
\pgfpathcurveto{\pgfqpoint{10.395011in}{1.723447in}}{\pgfqpoint{10.397015in}{1.718609in}}{\pgfqpoint{10.400581in}{1.715042in}}%
\pgfpathcurveto{\pgfqpoint{10.404147in}{1.711476in}}{\pgfqpoint{10.408985in}{1.709472in}}{\pgfqpoint{10.414029in}{1.709472in}}%
\pgfpathclose%
\pgfusepath{fill}%
\end{pgfscope}%
\begin{pgfscope}%
\pgfpathrectangle{\pgfqpoint{6.572727in}{0.473000in}}{\pgfqpoint{4.227273in}{3.311000in}}%
\pgfusepath{clip}%
\pgfsetbuttcap%
\pgfsetroundjoin%
\definecolor{currentfill}{rgb}{0.267004,0.004874,0.329415}%
\pgfsetfillcolor{currentfill}%
\pgfsetfillopacity{0.700000}%
\pgfsetlinewidth{0.000000pt}%
\definecolor{currentstroke}{rgb}{0.000000,0.000000,0.000000}%
\pgfsetstrokecolor{currentstroke}%
\pgfsetstrokeopacity{0.700000}%
\pgfsetdash{}{0pt}%
\pgfpathmoveto{\pgfqpoint{8.703118in}{0.801613in}}%
\pgfpathcurveto{\pgfqpoint{8.708162in}{0.801613in}}{\pgfqpoint{8.713000in}{0.803617in}}{\pgfqpoint{8.716566in}{0.807183in}}%
\pgfpathcurveto{\pgfqpoint{8.720133in}{0.810750in}}{\pgfqpoint{8.722136in}{0.815588in}}{\pgfqpoint{8.722136in}{0.820631in}}%
\pgfpathcurveto{\pgfqpoint{8.722136in}{0.825675in}}{\pgfqpoint{8.720133in}{0.830513in}}{\pgfqpoint{8.716566in}{0.834079in}}%
\pgfpathcurveto{\pgfqpoint{8.713000in}{0.837645in}}{\pgfqpoint{8.708162in}{0.839649in}}{\pgfqpoint{8.703118in}{0.839649in}}%
\pgfpathcurveto{\pgfqpoint{8.698075in}{0.839649in}}{\pgfqpoint{8.693237in}{0.837645in}}{\pgfqpoint{8.689670in}{0.834079in}}%
\pgfpathcurveto{\pgfqpoint{8.686104in}{0.830513in}}{\pgfqpoint{8.684100in}{0.825675in}}{\pgfqpoint{8.684100in}{0.820631in}}%
\pgfpathcurveto{\pgfqpoint{8.684100in}{0.815588in}}{\pgfqpoint{8.686104in}{0.810750in}}{\pgfqpoint{8.689670in}{0.807183in}}%
\pgfpathcurveto{\pgfqpoint{8.693237in}{0.803617in}}{\pgfqpoint{8.698075in}{0.801613in}}{\pgfqpoint{8.703118in}{0.801613in}}%
\pgfpathclose%
\pgfusepath{fill}%
\end{pgfscope}%
\begin{pgfscope}%
\pgfpathrectangle{\pgfqpoint{6.572727in}{0.473000in}}{\pgfqpoint{4.227273in}{3.311000in}}%
\pgfusepath{clip}%
\pgfsetbuttcap%
\pgfsetroundjoin%
\definecolor{currentfill}{rgb}{0.993248,0.906157,0.143936}%
\pgfsetfillcolor{currentfill}%
\pgfsetfillopacity{0.700000}%
\pgfsetlinewidth{0.000000pt}%
\definecolor{currentstroke}{rgb}{0.000000,0.000000,0.000000}%
\pgfsetstrokecolor{currentstroke}%
\pgfsetstrokeopacity{0.700000}%
\pgfsetdash{}{0pt}%
\pgfpathmoveto{\pgfqpoint{9.323233in}{2.155404in}}%
\pgfpathcurveto{\pgfqpoint{9.328276in}{2.155404in}}{\pgfqpoint{9.333114in}{2.157408in}}{\pgfqpoint{9.336680in}{2.160975in}}%
\pgfpathcurveto{\pgfqpoint{9.340247in}{2.164541in}}{\pgfqpoint{9.342251in}{2.169379in}}{\pgfqpoint{9.342251in}{2.174423in}}%
\pgfpathcurveto{\pgfqpoint{9.342251in}{2.179466in}}{\pgfqpoint{9.340247in}{2.184304in}}{\pgfqpoint{9.336680in}{2.187870in}}%
\pgfpathcurveto{\pgfqpoint{9.333114in}{2.191437in}}{\pgfqpoint{9.328276in}{2.193441in}}{\pgfqpoint{9.323233in}{2.193441in}}%
\pgfpathcurveto{\pgfqpoint{9.318189in}{2.193441in}}{\pgfqpoint{9.313351in}{2.191437in}}{\pgfqpoint{9.309785in}{2.187870in}}%
\pgfpathcurveto{\pgfqpoint{9.306218in}{2.184304in}}{\pgfqpoint{9.304214in}{2.179466in}}{\pgfqpoint{9.304214in}{2.174423in}}%
\pgfpathcurveto{\pgfqpoint{9.304214in}{2.169379in}}{\pgfqpoint{9.306218in}{2.164541in}}{\pgfqpoint{9.309785in}{2.160975in}}%
\pgfpathcurveto{\pgfqpoint{9.313351in}{2.157408in}}{\pgfqpoint{9.318189in}{2.155404in}}{\pgfqpoint{9.323233in}{2.155404in}}%
\pgfpathclose%
\pgfusepath{fill}%
\end{pgfscope}%
\begin{pgfscope}%
\pgfpathrectangle{\pgfqpoint{6.572727in}{0.473000in}}{\pgfqpoint{4.227273in}{3.311000in}}%
\pgfusepath{clip}%
\pgfsetbuttcap%
\pgfsetroundjoin%
\definecolor{currentfill}{rgb}{0.127568,0.566949,0.550556}%
\pgfsetfillcolor{currentfill}%
\pgfsetfillopacity{0.700000}%
\pgfsetlinewidth{0.000000pt}%
\definecolor{currentstroke}{rgb}{0.000000,0.000000,0.000000}%
\pgfsetstrokecolor{currentstroke}%
\pgfsetstrokeopacity{0.700000}%
\pgfsetdash{}{0pt}%
\pgfpathmoveto{\pgfqpoint{7.856344in}{1.165711in}}%
\pgfpathcurveto{\pgfqpoint{7.861387in}{1.165711in}}{\pgfqpoint{7.866225in}{1.167715in}}{\pgfqpoint{7.869792in}{1.171282in}}%
\pgfpathcurveto{\pgfqpoint{7.873358in}{1.174848in}}{\pgfqpoint{7.875362in}{1.179686in}}{\pgfqpoint{7.875362in}{1.184730in}}%
\pgfpathcurveto{\pgfqpoint{7.875362in}{1.189773in}}{\pgfqpoint{7.873358in}{1.194611in}}{\pgfqpoint{7.869792in}{1.198177in}}%
\pgfpathcurveto{\pgfqpoint{7.866225in}{1.201744in}}{\pgfqpoint{7.861387in}{1.203748in}}{\pgfqpoint{7.856344in}{1.203748in}}%
\pgfpathcurveto{\pgfqpoint{7.851300in}{1.203748in}}{\pgfqpoint{7.846462in}{1.201744in}}{\pgfqpoint{7.842896in}{1.198177in}}%
\pgfpathcurveto{\pgfqpoint{7.839330in}{1.194611in}}{\pgfqpoint{7.837326in}{1.189773in}}{\pgfqpoint{7.837326in}{1.184730in}}%
\pgfpathcurveto{\pgfqpoint{7.837326in}{1.179686in}}{\pgfqpoint{7.839330in}{1.174848in}}{\pgfqpoint{7.842896in}{1.171282in}}%
\pgfpathcurveto{\pgfqpoint{7.846462in}{1.167715in}}{\pgfqpoint{7.851300in}{1.165711in}}{\pgfqpoint{7.856344in}{1.165711in}}%
\pgfpathclose%
\pgfusepath{fill}%
\end{pgfscope}%
\begin{pgfscope}%
\pgfpathrectangle{\pgfqpoint{6.572727in}{0.473000in}}{\pgfqpoint{4.227273in}{3.311000in}}%
\pgfusepath{clip}%
\pgfsetbuttcap%
\pgfsetroundjoin%
\definecolor{currentfill}{rgb}{0.127568,0.566949,0.550556}%
\pgfsetfillcolor{currentfill}%
\pgfsetfillopacity{0.700000}%
\pgfsetlinewidth{0.000000pt}%
\definecolor{currentstroke}{rgb}{0.000000,0.000000,0.000000}%
\pgfsetstrokecolor{currentstroke}%
\pgfsetstrokeopacity{0.700000}%
\pgfsetdash{}{0pt}%
\pgfpathmoveto{\pgfqpoint{7.767347in}{1.632768in}}%
\pgfpathcurveto{\pgfqpoint{7.772391in}{1.632768in}}{\pgfqpoint{7.777228in}{1.634772in}}{\pgfqpoint{7.780795in}{1.638338in}}%
\pgfpathcurveto{\pgfqpoint{7.784361in}{1.641905in}}{\pgfqpoint{7.786365in}{1.646743in}}{\pgfqpoint{7.786365in}{1.651786in}}%
\pgfpathcurveto{\pgfqpoint{7.786365in}{1.656830in}}{\pgfqpoint{7.784361in}{1.661668in}}{\pgfqpoint{7.780795in}{1.665234in}}%
\pgfpathcurveto{\pgfqpoint{7.777228in}{1.668801in}}{\pgfqpoint{7.772391in}{1.670804in}}{\pgfqpoint{7.767347in}{1.670804in}}%
\pgfpathcurveto{\pgfqpoint{7.762303in}{1.670804in}}{\pgfqpoint{7.757466in}{1.668801in}}{\pgfqpoint{7.753899in}{1.665234in}}%
\pgfpathcurveto{\pgfqpoint{7.750333in}{1.661668in}}{\pgfqpoint{7.748329in}{1.656830in}}{\pgfqpoint{7.748329in}{1.651786in}}%
\pgfpathcurveto{\pgfqpoint{7.748329in}{1.646743in}}{\pgfqpoint{7.750333in}{1.641905in}}{\pgfqpoint{7.753899in}{1.638338in}}%
\pgfpathcurveto{\pgfqpoint{7.757466in}{1.634772in}}{\pgfqpoint{7.762303in}{1.632768in}}{\pgfqpoint{7.767347in}{1.632768in}}%
\pgfpathclose%
\pgfusepath{fill}%
\end{pgfscope}%
\begin{pgfscope}%
\pgfpathrectangle{\pgfqpoint{6.572727in}{0.473000in}}{\pgfqpoint{4.227273in}{3.311000in}}%
\pgfusepath{clip}%
\pgfsetbuttcap%
\pgfsetroundjoin%
\definecolor{currentfill}{rgb}{0.993248,0.906157,0.143936}%
\pgfsetfillcolor{currentfill}%
\pgfsetfillopacity{0.700000}%
\pgfsetlinewidth{0.000000pt}%
\definecolor{currentstroke}{rgb}{0.000000,0.000000,0.000000}%
\pgfsetstrokecolor{currentstroke}%
\pgfsetstrokeopacity{0.700000}%
\pgfsetdash{}{0pt}%
\pgfpathmoveto{\pgfqpoint{9.706365in}{2.092854in}}%
\pgfpathcurveto{\pgfqpoint{9.711408in}{2.092854in}}{\pgfqpoint{9.716246in}{2.094858in}}{\pgfqpoint{9.719813in}{2.098425in}}%
\pgfpathcurveto{\pgfqpoint{9.723379in}{2.101991in}}{\pgfqpoint{9.725383in}{2.106829in}}{\pgfqpoint{9.725383in}{2.111873in}}%
\pgfpathcurveto{\pgfqpoint{9.725383in}{2.116916in}}{\pgfqpoint{9.723379in}{2.121754in}}{\pgfqpoint{9.719813in}{2.125320in}}%
\pgfpathcurveto{\pgfqpoint{9.716246in}{2.128887in}}{\pgfqpoint{9.711408in}{2.130891in}}{\pgfqpoint{9.706365in}{2.130891in}}%
\pgfpathcurveto{\pgfqpoint{9.701321in}{2.130891in}}{\pgfqpoint{9.696483in}{2.128887in}}{\pgfqpoint{9.692917in}{2.125320in}}%
\pgfpathcurveto{\pgfqpoint{9.689350in}{2.121754in}}{\pgfqpoint{9.687347in}{2.116916in}}{\pgfqpoint{9.687347in}{2.111873in}}%
\pgfpathcurveto{\pgfqpoint{9.687347in}{2.106829in}}{\pgfqpoint{9.689350in}{2.101991in}}{\pgfqpoint{9.692917in}{2.098425in}}%
\pgfpathcurveto{\pgfqpoint{9.696483in}{2.094858in}}{\pgfqpoint{9.701321in}{2.092854in}}{\pgfqpoint{9.706365in}{2.092854in}}%
\pgfpathclose%
\pgfusepath{fill}%
\end{pgfscope}%
\begin{pgfscope}%
\pgfpathrectangle{\pgfqpoint{6.572727in}{0.473000in}}{\pgfqpoint{4.227273in}{3.311000in}}%
\pgfusepath{clip}%
\pgfsetbuttcap%
\pgfsetroundjoin%
\definecolor{currentfill}{rgb}{0.127568,0.566949,0.550556}%
\pgfsetfillcolor{currentfill}%
\pgfsetfillopacity{0.700000}%
\pgfsetlinewidth{0.000000pt}%
\definecolor{currentstroke}{rgb}{0.000000,0.000000,0.000000}%
\pgfsetstrokecolor{currentstroke}%
\pgfsetstrokeopacity{0.700000}%
\pgfsetdash{}{0pt}%
\pgfpathmoveto{\pgfqpoint{7.866135in}{1.684300in}}%
\pgfpathcurveto{\pgfqpoint{7.871179in}{1.684300in}}{\pgfqpoint{7.876017in}{1.686303in}}{\pgfqpoint{7.879583in}{1.689870in}}%
\pgfpathcurveto{\pgfqpoint{7.883150in}{1.693436in}}{\pgfqpoint{7.885153in}{1.698274in}}{\pgfqpoint{7.885153in}{1.703318in}}%
\pgfpathcurveto{\pgfqpoint{7.885153in}{1.708361in}}{\pgfqpoint{7.883150in}{1.713199in}}{\pgfqpoint{7.879583in}{1.716766in}}%
\pgfpathcurveto{\pgfqpoint{7.876017in}{1.720332in}}{\pgfqpoint{7.871179in}{1.722336in}}{\pgfqpoint{7.866135in}{1.722336in}}%
\pgfpathcurveto{\pgfqpoint{7.861092in}{1.722336in}}{\pgfqpoint{7.856254in}{1.720332in}}{\pgfqpoint{7.852687in}{1.716766in}}%
\pgfpathcurveto{\pgfqpoint{7.849121in}{1.713199in}}{\pgfqpoint{7.847117in}{1.708361in}}{\pgfqpoint{7.847117in}{1.703318in}}%
\pgfpathcurveto{\pgfqpoint{7.847117in}{1.698274in}}{\pgfqpoint{7.849121in}{1.693436in}}{\pgfqpoint{7.852687in}{1.689870in}}%
\pgfpathcurveto{\pgfqpoint{7.856254in}{1.686303in}}{\pgfqpoint{7.861092in}{1.684300in}}{\pgfqpoint{7.866135in}{1.684300in}}%
\pgfpathclose%
\pgfusepath{fill}%
\end{pgfscope}%
\begin{pgfscope}%
\pgfpathrectangle{\pgfqpoint{6.572727in}{0.473000in}}{\pgfqpoint{4.227273in}{3.311000in}}%
\pgfusepath{clip}%
\pgfsetbuttcap%
\pgfsetroundjoin%
\definecolor{currentfill}{rgb}{0.127568,0.566949,0.550556}%
\pgfsetfillcolor{currentfill}%
\pgfsetfillopacity{0.700000}%
\pgfsetlinewidth{0.000000pt}%
\definecolor{currentstroke}{rgb}{0.000000,0.000000,0.000000}%
\pgfsetstrokecolor{currentstroke}%
\pgfsetstrokeopacity{0.700000}%
\pgfsetdash{}{0pt}%
\pgfpathmoveto{\pgfqpoint{8.416112in}{2.683515in}}%
\pgfpathcurveto{\pgfqpoint{8.421155in}{2.683515in}}{\pgfqpoint{8.425993in}{2.685519in}}{\pgfqpoint{8.429560in}{2.689085in}}%
\pgfpathcurveto{\pgfqpoint{8.433126in}{2.692652in}}{\pgfqpoint{8.435130in}{2.697490in}}{\pgfqpoint{8.435130in}{2.702533in}}%
\pgfpathcurveto{\pgfqpoint{8.435130in}{2.707577in}}{\pgfqpoint{8.433126in}{2.712415in}}{\pgfqpoint{8.429560in}{2.715981in}}%
\pgfpathcurveto{\pgfqpoint{8.425993in}{2.719548in}}{\pgfqpoint{8.421155in}{2.721551in}}{\pgfqpoint{8.416112in}{2.721551in}}%
\pgfpathcurveto{\pgfqpoint{8.411068in}{2.721551in}}{\pgfqpoint{8.406230in}{2.719548in}}{\pgfqpoint{8.402664in}{2.715981in}}%
\pgfpathcurveto{\pgfqpoint{8.399097in}{2.712415in}}{\pgfqpoint{8.397094in}{2.707577in}}{\pgfqpoint{8.397094in}{2.702533in}}%
\pgfpathcurveto{\pgfqpoint{8.397094in}{2.697490in}}{\pgfqpoint{8.399097in}{2.692652in}}{\pgfqpoint{8.402664in}{2.689085in}}%
\pgfpathcurveto{\pgfqpoint{8.406230in}{2.685519in}}{\pgfqpoint{8.411068in}{2.683515in}}{\pgfqpoint{8.416112in}{2.683515in}}%
\pgfpathclose%
\pgfusepath{fill}%
\end{pgfscope}%
\begin{pgfscope}%
\pgfpathrectangle{\pgfqpoint{6.572727in}{0.473000in}}{\pgfqpoint{4.227273in}{3.311000in}}%
\pgfusepath{clip}%
\pgfsetbuttcap%
\pgfsetroundjoin%
\definecolor{currentfill}{rgb}{0.993248,0.906157,0.143936}%
\pgfsetfillcolor{currentfill}%
\pgfsetfillopacity{0.700000}%
\pgfsetlinewidth{0.000000pt}%
\definecolor{currentstroke}{rgb}{0.000000,0.000000,0.000000}%
\pgfsetstrokecolor{currentstroke}%
\pgfsetstrokeopacity{0.700000}%
\pgfsetdash{}{0pt}%
\pgfpathmoveto{\pgfqpoint{9.629258in}{1.661069in}}%
\pgfpathcurveto{\pgfqpoint{9.634302in}{1.661069in}}{\pgfqpoint{9.639140in}{1.663073in}}{\pgfqpoint{9.642706in}{1.666639in}}%
\pgfpathcurveto{\pgfqpoint{9.646273in}{1.670206in}}{\pgfqpoint{9.648277in}{1.675043in}}{\pgfqpoint{9.648277in}{1.680087in}}%
\pgfpathcurveto{\pgfqpoint{9.648277in}{1.685131in}}{\pgfqpoint{9.646273in}{1.689969in}}{\pgfqpoint{9.642706in}{1.693535in}}%
\pgfpathcurveto{\pgfqpoint{9.639140in}{1.697101in}}{\pgfqpoint{9.634302in}{1.699105in}}{\pgfqpoint{9.629258in}{1.699105in}}%
\pgfpathcurveto{\pgfqpoint{9.624215in}{1.699105in}}{\pgfqpoint{9.619377in}{1.697101in}}{\pgfqpoint{9.615811in}{1.693535in}}%
\pgfpathcurveto{\pgfqpoint{9.612244in}{1.689969in}}{\pgfqpoint{9.610240in}{1.685131in}}{\pgfqpoint{9.610240in}{1.680087in}}%
\pgfpathcurveto{\pgfqpoint{9.610240in}{1.675043in}}{\pgfqpoint{9.612244in}{1.670206in}}{\pgfqpoint{9.615811in}{1.666639in}}%
\pgfpathcurveto{\pgfqpoint{9.619377in}{1.663073in}}{\pgfqpoint{9.624215in}{1.661069in}}{\pgfqpoint{9.629258in}{1.661069in}}%
\pgfpathclose%
\pgfusepath{fill}%
\end{pgfscope}%
\begin{pgfscope}%
\pgfpathrectangle{\pgfqpoint{6.572727in}{0.473000in}}{\pgfqpoint{4.227273in}{3.311000in}}%
\pgfusepath{clip}%
\pgfsetbuttcap%
\pgfsetroundjoin%
\definecolor{currentfill}{rgb}{0.993248,0.906157,0.143936}%
\pgfsetfillcolor{currentfill}%
\pgfsetfillopacity{0.700000}%
\pgfsetlinewidth{0.000000pt}%
\definecolor{currentstroke}{rgb}{0.000000,0.000000,0.000000}%
\pgfsetstrokecolor{currentstroke}%
\pgfsetstrokeopacity{0.700000}%
\pgfsetdash{}{0pt}%
\pgfpathmoveto{\pgfqpoint{8.962418in}{1.284400in}}%
\pgfpathcurveto{\pgfqpoint{8.967462in}{1.284400in}}{\pgfqpoint{8.972299in}{1.286404in}}{\pgfqpoint{8.975866in}{1.289970in}}%
\pgfpathcurveto{\pgfqpoint{8.979432in}{1.293536in}}{\pgfqpoint{8.981436in}{1.298374in}}{\pgfqpoint{8.981436in}{1.303418in}}%
\pgfpathcurveto{\pgfqpoint{8.981436in}{1.308462in}}{\pgfqpoint{8.979432in}{1.313299in}}{\pgfqpoint{8.975866in}{1.316866in}}%
\pgfpathcurveto{\pgfqpoint{8.972299in}{1.320432in}}{\pgfqpoint{8.967462in}{1.322436in}}{\pgfqpoint{8.962418in}{1.322436in}}%
\pgfpathcurveto{\pgfqpoint{8.957374in}{1.322436in}}{\pgfqpoint{8.952536in}{1.320432in}}{\pgfqpoint{8.948970in}{1.316866in}}%
\pgfpathcurveto{\pgfqpoint{8.945404in}{1.313299in}}{\pgfqpoint{8.943400in}{1.308462in}}{\pgfqpoint{8.943400in}{1.303418in}}%
\pgfpathcurveto{\pgfqpoint{8.943400in}{1.298374in}}{\pgfqpoint{8.945404in}{1.293536in}}{\pgfqpoint{8.948970in}{1.289970in}}%
\pgfpathcurveto{\pgfqpoint{8.952536in}{1.286404in}}{\pgfqpoint{8.957374in}{1.284400in}}{\pgfqpoint{8.962418in}{1.284400in}}%
\pgfpathclose%
\pgfusepath{fill}%
\end{pgfscope}%
\begin{pgfscope}%
\pgfpathrectangle{\pgfqpoint{6.572727in}{0.473000in}}{\pgfqpoint{4.227273in}{3.311000in}}%
\pgfusepath{clip}%
\pgfsetbuttcap%
\pgfsetroundjoin%
\definecolor{currentfill}{rgb}{0.993248,0.906157,0.143936}%
\pgfsetfillcolor{currentfill}%
\pgfsetfillopacity{0.700000}%
\pgfsetlinewidth{0.000000pt}%
\definecolor{currentstroke}{rgb}{0.000000,0.000000,0.000000}%
\pgfsetstrokecolor{currentstroke}%
\pgfsetstrokeopacity{0.700000}%
\pgfsetdash{}{0pt}%
\pgfpathmoveto{\pgfqpoint{9.790870in}{1.874755in}}%
\pgfpathcurveto{\pgfqpoint{9.795913in}{1.874755in}}{\pgfqpoint{9.800751in}{1.876759in}}{\pgfqpoint{9.804317in}{1.880325in}}%
\pgfpathcurveto{\pgfqpoint{9.807884in}{1.883892in}}{\pgfqpoint{9.809888in}{1.888729in}}{\pgfqpoint{9.809888in}{1.893773in}}%
\pgfpathcurveto{\pgfqpoint{9.809888in}{1.898817in}}{\pgfqpoint{9.807884in}{1.903655in}}{\pgfqpoint{9.804317in}{1.907221in}}%
\pgfpathcurveto{\pgfqpoint{9.800751in}{1.910787in}}{\pgfqpoint{9.795913in}{1.912791in}}{\pgfqpoint{9.790870in}{1.912791in}}%
\pgfpathcurveto{\pgfqpoint{9.785826in}{1.912791in}}{\pgfqpoint{9.780988in}{1.910787in}}{\pgfqpoint{9.777422in}{1.907221in}}%
\pgfpathcurveto{\pgfqpoint{9.773855in}{1.903655in}}{\pgfqpoint{9.771851in}{1.898817in}}{\pgfqpoint{9.771851in}{1.893773in}}%
\pgfpathcurveto{\pgfqpoint{9.771851in}{1.888729in}}{\pgfqpoint{9.773855in}{1.883892in}}{\pgfqpoint{9.777422in}{1.880325in}}%
\pgfpathcurveto{\pgfqpoint{9.780988in}{1.876759in}}{\pgfqpoint{9.785826in}{1.874755in}}{\pgfqpoint{9.790870in}{1.874755in}}%
\pgfpathclose%
\pgfusepath{fill}%
\end{pgfscope}%
\begin{pgfscope}%
\pgfpathrectangle{\pgfqpoint{6.572727in}{0.473000in}}{\pgfqpoint{4.227273in}{3.311000in}}%
\pgfusepath{clip}%
\pgfsetbuttcap%
\pgfsetroundjoin%
\definecolor{currentfill}{rgb}{0.127568,0.566949,0.550556}%
\pgfsetfillcolor{currentfill}%
\pgfsetfillopacity{0.700000}%
\pgfsetlinewidth{0.000000pt}%
\definecolor{currentstroke}{rgb}{0.000000,0.000000,0.000000}%
\pgfsetstrokecolor{currentstroke}%
\pgfsetstrokeopacity{0.700000}%
\pgfsetdash{}{0pt}%
\pgfpathmoveto{\pgfqpoint{7.402153in}{1.266058in}}%
\pgfpathcurveto{\pgfqpoint{7.407197in}{1.266058in}}{\pgfqpoint{7.412035in}{1.268062in}}{\pgfqpoint{7.415601in}{1.271628in}}%
\pgfpathcurveto{\pgfqpoint{7.419168in}{1.275194in}}{\pgfqpoint{7.421171in}{1.280032in}}{\pgfqpoint{7.421171in}{1.285076in}}%
\pgfpathcurveto{\pgfqpoint{7.421171in}{1.290120in}}{\pgfqpoint{7.419168in}{1.294957in}}{\pgfqpoint{7.415601in}{1.298524in}}%
\pgfpathcurveto{\pgfqpoint{7.412035in}{1.302090in}}{\pgfqpoint{7.407197in}{1.304094in}}{\pgfqpoint{7.402153in}{1.304094in}}%
\pgfpathcurveto{\pgfqpoint{7.397110in}{1.304094in}}{\pgfqpoint{7.392272in}{1.302090in}}{\pgfqpoint{7.388705in}{1.298524in}}%
\pgfpathcurveto{\pgfqpoint{7.385139in}{1.294957in}}{\pgfqpoint{7.383135in}{1.290120in}}{\pgfqpoint{7.383135in}{1.285076in}}%
\pgfpathcurveto{\pgfqpoint{7.383135in}{1.280032in}}{\pgfqpoint{7.385139in}{1.275194in}}{\pgfqpoint{7.388705in}{1.271628in}}%
\pgfpathcurveto{\pgfqpoint{7.392272in}{1.268062in}}{\pgfqpoint{7.397110in}{1.266058in}}{\pgfqpoint{7.402153in}{1.266058in}}%
\pgfpathclose%
\pgfusepath{fill}%
\end{pgfscope}%
\begin{pgfscope}%
\pgfpathrectangle{\pgfqpoint{6.572727in}{0.473000in}}{\pgfqpoint{4.227273in}{3.311000in}}%
\pgfusepath{clip}%
\pgfsetbuttcap%
\pgfsetroundjoin%
\definecolor{currentfill}{rgb}{0.993248,0.906157,0.143936}%
\pgfsetfillcolor{currentfill}%
\pgfsetfillopacity{0.700000}%
\pgfsetlinewidth{0.000000pt}%
\definecolor{currentstroke}{rgb}{0.000000,0.000000,0.000000}%
\pgfsetstrokecolor{currentstroke}%
\pgfsetstrokeopacity{0.700000}%
\pgfsetdash{}{0pt}%
\pgfpathmoveto{\pgfqpoint{9.794170in}{2.126505in}}%
\pgfpathcurveto{\pgfqpoint{9.799213in}{2.126505in}}{\pgfqpoint{9.804051in}{2.128509in}}{\pgfqpoint{9.807617in}{2.132075in}}%
\pgfpathcurveto{\pgfqpoint{9.811184in}{2.135641in}}{\pgfqpoint{9.813188in}{2.140479in}}{\pgfqpoint{9.813188in}{2.145523in}}%
\pgfpathcurveto{\pgfqpoint{9.813188in}{2.150567in}}{\pgfqpoint{9.811184in}{2.155404in}}{\pgfqpoint{9.807617in}{2.158971in}}%
\pgfpathcurveto{\pgfqpoint{9.804051in}{2.162537in}}{\pgfqpoint{9.799213in}{2.164541in}}{\pgfqpoint{9.794170in}{2.164541in}}%
\pgfpathcurveto{\pgfqpoint{9.789126in}{2.164541in}}{\pgfqpoint{9.784288in}{2.162537in}}{\pgfqpoint{9.780722in}{2.158971in}}%
\pgfpathcurveto{\pgfqpoint{9.777155in}{2.155404in}}{\pgfqpoint{9.775151in}{2.150567in}}{\pgfqpoint{9.775151in}{2.145523in}}%
\pgfpathcurveto{\pgfqpoint{9.775151in}{2.140479in}}{\pgfqpoint{9.777155in}{2.135641in}}{\pgfqpoint{9.780722in}{2.132075in}}%
\pgfpathcurveto{\pgfqpoint{9.784288in}{2.128509in}}{\pgfqpoint{9.789126in}{2.126505in}}{\pgfqpoint{9.794170in}{2.126505in}}%
\pgfpathclose%
\pgfusepath{fill}%
\end{pgfscope}%
\begin{pgfscope}%
\pgfpathrectangle{\pgfqpoint{6.572727in}{0.473000in}}{\pgfqpoint{4.227273in}{3.311000in}}%
\pgfusepath{clip}%
\pgfsetbuttcap%
\pgfsetroundjoin%
\definecolor{currentfill}{rgb}{0.127568,0.566949,0.550556}%
\pgfsetfillcolor{currentfill}%
\pgfsetfillopacity{0.700000}%
\pgfsetlinewidth{0.000000pt}%
\definecolor{currentstroke}{rgb}{0.000000,0.000000,0.000000}%
\pgfsetstrokecolor{currentstroke}%
\pgfsetstrokeopacity{0.700000}%
\pgfsetdash{}{0pt}%
\pgfpathmoveto{\pgfqpoint{7.970063in}{1.264322in}}%
\pgfpathcurveto{\pgfqpoint{7.975106in}{1.264322in}}{\pgfqpoint{7.979944in}{1.266326in}}{\pgfqpoint{7.983511in}{1.269892in}}%
\pgfpathcurveto{\pgfqpoint{7.987077in}{1.273459in}}{\pgfqpoint{7.989081in}{1.278296in}}{\pgfqpoint{7.989081in}{1.283340in}}%
\pgfpathcurveto{\pgfqpoint{7.989081in}{1.288384in}}{\pgfqpoint{7.987077in}{1.293221in}}{\pgfqpoint{7.983511in}{1.296788in}}%
\pgfpathcurveto{\pgfqpoint{7.979944in}{1.300354in}}{\pgfqpoint{7.975106in}{1.302358in}}{\pgfqpoint{7.970063in}{1.302358in}}%
\pgfpathcurveto{\pgfqpoint{7.965019in}{1.302358in}}{\pgfqpoint{7.960181in}{1.300354in}}{\pgfqpoint{7.956615in}{1.296788in}}%
\pgfpathcurveto{\pgfqpoint{7.953048in}{1.293221in}}{\pgfqpoint{7.951045in}{1.288384in}}{\pgfqpoint{7.951045in}{1.283340in}}%
\pgfpathcurveto{\pgfqpoint{7.951045in}{1.278296in}}{\pgfqpoint{7.953048in}{1.273459in}}{\pgfqpoint{7.956615in}{1.269892in}}%
\pgfpathcurveto{\pgfqpoint{7.960181in}{1.266326in}}{\pgfqpoint{7.965019in}{1.264322in}}{\pgfqpoint{7.970063in}{1.264322in}}%
\pgfpathclose%
\pgfusepath{fill}%
\end{pgfscope}%
\begin{pgfscope}%
\pgfpathrectangle{\pgfqpoint{6.572727in}{0.473000in}}{\pgfqpoint{4.227273in}{3.311000in}}%
\pgfusepath{clip}%
\pgfsetbuttcap%
\pgfsetroundjoin%
\definecolor{currentfill}{rgb}{0.127568,0.566949,0.550556}%
\pgfsetfillcolor{currentfill}%
\pgfsetfillopacity{0.700000}%
\pgfsetlinewidth{0.000000pt}%
\definecolor{currentstroke}{rgb}{0.000000,0.000000,0.000000}%
\pgfsetstrokecolor{currentstroke}%
\pgfsetstrokeopacity{0.700000}%
\pgfsetdash{}{0pt}%
\pgfpathmoveto{\pgfqpoint{7.979843in}{1.635148in}}%
\pgfpathcurveto{\pgfqpoint{7.984887in}{1.635148in}}{\pgfqpoint{7.989724in}{1.637152in}}{\pgfqpoint{7.993291in}{1.640719in}}%
\pgfpathcurveto{\pgfqpoint{7.996857in}{1.644285in}}{\pgfqpoint{7.998861in}{1.649123in}}{\pgfqpoint{7.998861in}{1.654166in}}%
\pgfpathcurveto{\pgfqpoint{7.998861in}{1.659210in}}{\pgfqpoint{7.996857in}{1.664048in}}{\pgfqpoint{7.993291in}{1.667614in}}%
\pgfpathcurveto{\pgfqpoint{7.989724in}{1.671181in}}{\pgfqpoint{7.984887in}{1.673185in}}{\pgfqpoint{7.979843in}{1.673185in}}%
\pgfpathcurveto{\pgfqpoint{7.974799in}{1.673185in}}{\pgfqpoint{7.969962in}{1.671181in}}{\pgfqpoint{7.966395in}{1.667614in}}%
\pgfpathcurveto{\pgfqpoint{7.962829in}{1.664048in}}{\pgfqpoint{7.960825in}{1.659210in}}{\pgfqpoint{7.960825in}{1.654166in}}%
\pgfpathcurveto{\pgfqpoint{7.960825in}{1.649123in}}{\pgfqpoint{7.962829in}{1.644285in}}{\pgfqpoint{7.966395in}{1.640719in}}%
\pgfpathcurveto{\pgfqpoint{7.969962in}{1.637152in}}{\pgfqpoint{7.974799in}{1.635148in}}{\pgfqpoint{7.979843in}{1.635148in}}%
\pgfpathclose%
\pgfusepath{fill}%
\end{pgfscope}%
\begin{pgfscope}%
\pgfpathrectangle{\pgfqpoint{6.572727in}{0.473000in}}{\pgfqpoint{4.227273in}{3.311000in}}%
\pgfusepath{clip}%
\pgfsetbuttcap%
\pgfsetroundjoin%
\definecolor{currentfill}{rgb}{0.127568,0.566949,0.550556}%
\pgfsetfillcolor{currentfill}%
\pgfsetfillopacity{0.700000}%
\pgfsetlinewidth{0.000000pt}%
\definecolor{currentstroke}{rgb}{0.000000,0.000000,0.000000}%
\pgfsetstrokecolor{currentstroke}%
\pgfsetstrokeopacity{0.700000}%
\pgfsetdash{}{0pt}%
\pgfpathmoveto{\pgfqpoint{7.864756in}{3.035413in}}%
\pgfpathcurveto{\pgfqpoint{7.869800in}{3.035413in}}{\pgfqpoint{7.874638in}{3.037417in}}{\pgfqpoint{7.878204in}{3.040983in}}%
\pgfpathcurveto{\pgfqpoint{7.881771in}{3.044550in}}{\pgfqpoint{7.883775in}{3.049388in}}{\pgfqpoint{7.883775in}{3.054431in}}%
\pgfpathcurveto{\pgfqpoint{7.883775in}{3.059475in}}{\pgfqpoint{7.881771in}{3.064313in}}{\pgfqpoint{7.878204in}{3.067879in}}%
\pgfpathcurveto{\pgfqpoint{7.874638in}{3.071446in}}{\pgfqpoint{7.869800in}{3.073449in}}{\pgfqpoint{7.864756in}{3.073449in}}%
\pgfpathcurveto{\pgfqpoint{7.859713in}{3.073449in}}{\pgfqpoint{7.854875in}{3.071446in}}{\pgfqpoint{7.851309in}{3.067879in}}%
\pgfpathcurveto{\pgfqpoint{7.847742in}{3.064313in}}{\pgfqpoint{7.845738in}{3.059475in}}{\pgfqpoint{7.845738in}{3.054431in}}%
\pgfpathcurveto{\pgfqpoint{7.845738in}{3.049388in}}{\pgfqpoint{7.847742in}{3.044550in}}{\pgfqpoint{7.851309in}{3.040983in}}%
\pgfpathcurveto{\pgfqpoint{7.854875in}{3.037417in}}{\pgfqpoint{7.859713in}{3.035413in}}{\pgfqpoint{7.864756in}{3.035413in}}%
\pgfpathclose%
\pgfusepath{fill}%
\end{pgfscope}%
\begin{pgfscope}%
\pgfpathrectangle{\pgfqpoint{6.572727in}{0.473000in}}{\pgfqpoint{4.227273in}{3.311000in}}%
\pgfusepath{clip}%
\pgfsetbuttcap%
\pgfsetroundjoin%
\definecolor{currentfill}{rgb}{0.993248,0.906157,0.143936}%
\pgfsetfillcolor{currentfill}%
\pgfsetfillopacity{0.700000}%
\pgfsetlinewidth{0.000000pt}%
\definecolor{currentstroke}{rgb}{0.000000,0.000000,0.000000}%
\pgfsetstrokecolor{currentstroke}%
\pgfsetstrokeopacity{0.700000}%
\pgfsetdash{}{0pt}%
\pgfpathmoveto{\pgfqpoint{9.403020in}{1.044260in}}%
\pgfpathcurveto{\pgfqpoint{9.408063in}{1.044260in}}{\pgfqpoint{9.412901in}{1.046264in}}{\pgfqpoint{9.416467in}{1.049831in}}%
\pgfpathcurveto{\pgfqpoint{9.420034in}{1.053397in}}{\pgfqpoint{9.422038in}{1.058235in}}{\pgfqpoint{9.422038in}{1.063279in}}%
\pgfpathcurveto{\pgfqpoint{9.422038in}{1.068322in}}{\pgfqpoint{9.420034in}{1.073160in}}{\pgfqpoint{9.416467in}{1.076726in}}%
\pgfpathcurveto{\pgfqpoint{9.412901in}{1.080293in}}{\pgfqpoint{9.408063in}{1.082297in}}{\pgfqpoint{9.403020in}{1.082297in}}%
\pgfpathcurveto{\pgfqpoint{9.397976in}{1.082297in}}{\pgfqpoint{9.393138in}{1.080293in}}{\pgfqpoint{9.389572in}{1.076726in}}%
\pgfpathcurveto{\pgfqpoint{9.386005in}{1.073160in}}{\pgfqpoint{9.384001in}{1.068322in}}{\pgfqpoint{9.384001in}{1.063279in}}%
\pgfpathcurveto{\pgfqpoint{9.384001in}{1.058235in}}{\pgfqpoint{9.386005in}{1.053397in}}{\pgfqpoint{9.389572in}{1.049831in}}%
\pgfpathcurveto{\pgfqpoint{9.393138in}{1.046264in}}{\pgfqpoint{9.397976in}{1.044260in}}{\pgfqpoint{9.403020in}{1.044260in}}%
\pgfpathclose%
\pgfusepath{fill}%
\end{pgfscope}%
\begin{pgfscope}%
\pgfpathrectangle{\pgfqpoint{6.572727in}{0.473000in}}{\pgfqpoint{4.227273in}{3.311000in}}%
\pgfusepath{clip}%
\pgfsetbuttcap%
\pgfsetroundjoin%
\definecolor{currentfill}{rgb}{0.127568,0.566949,0.550556}%
\pgfsetfillcolor{currentfill}%
\pgfsetfillopacity{0.700000}%
\pgfsetlinewidth{0.000000pt}%
\definecolor{currentstroke}{rgb}{0.000000,0.000000,0.000000}%
\pgfsetstrokecolor{currentstroke}%
\pgfsetstrokeopacity{0.700000}%
\pgfsetdash{}{0pt}%
\pgfpathmoveto{\pgfqpoint{7.918575in}{2.144695in}}%
\pgfpathcurveto{\pgfqpoint{7.923619in}{2.144695in}}{\pgfqpoint{7.928456in}{2.146699in}}{\pgfqpoint{7.932023in}{2.150265in}}%
\pgfpathcurveto{\pgfqpoint{7.935589in}{2.153832in}}{\pgfqpoint{7.937593in}{2.158670in}}{\pgfqpoint{7.937593in}{2.163713in}}%
\pgfpathcurveto{\pgfqpoint{7.937593in}{2.168757in}}{\pgfqpoint{7.935589in}{2.173595in}}{\pgfqpoint{7.932023in}{2.177161in}}%
\pgfpathcurveto{\pgfqpoint{7.928456in}{2.180728in}}{\pgfqpoint{7.923619in}{2.182731in}}{\pgfqpoint{7.918575in}{2.182731in}}%
\pgfpathcurveto{\pgfqpoint{7.913531in}{2.182731in}}{\pgfqpoint{7.908694in}{2.180728in}}{\pgfqpoint{7.905127in}{2.177161in}}%
\pgfpathcurveto{\pgfqpoint{7.901561in}{2.173595in}}{\pgfqpoint{7.899557in}{2.168757in}}{\pgfqpoint{7.899557in}{2.163713in}}%
\pgfpathcurveto{\pgfqpoint{7.899557in}{2.158670in}}{\pgfqpoint{7.901561in}{2.153832in}}{\pgfqpoint{7.905127in}{2.150265in}}%
\pgfpathcurveto{\pgfqpoint{7.908694in}{2.146699in}}{\pgfqpoint{7.913531in}{2.144695in}}{\pgfqpoint{7.918575in}{2.144695in}}%
\pgfpathclose%
\pgfusepath{fill}%
\end{pgfscope}%
\begin{pgfscope}%
\pgfpathrectangle{\pgfqpoint{6.572727in}{0.473000in}}{\pgfqpoint{4.227273in}{3.311000in}}%
\pgfusepath{clip}%
\pgfsetbuttcap%
\pgfsetroundjoin%
\definecolor{currentfill}{rgb}{0.127568,0.566949,0.550556}%
\pgfsetfillcolor{currentfill}%
\pgfsetfillopacity{0.700000}%
\pgfsetlinewidth{0.000000pt}%
\definecolor{currentstroke}{rgb}{0.000000,0.000000,0.000000}%
\pgfsetstrokecolor{currentstroke}%
\pgfsetstrokeopacity{0.700000}%
\pgfsetdash{}{0pt}%
\pgfpathmoveto{\pgfqpoint{8.004817in}{1.222496in}}%
\pgfpathcurveto{\pgfqpoint{8.009861in}{1.222496in}}{\pgfqpoint{8.014699in}{1.224500in}}{\pgfqpoint{8.018265in}{1.228066in}}%
\pgfpathcurveto{\pgfqpoint{8.021832in}{1.231632in}}{\pgfqpoint{8.023835in}{1.236470in}}{\pgfqpoint{8.023835in}{1.241514in}}%
\pgfpathcurveto{\pgfqpoint{8.023835in}{1.246558in}}{\pgfqpoint{8.021832in}{1.251395in}}{\pgfqpoint{8.018265in}{1.254962in}}%
\pgfpathcurveto{\pgfqpoint{8.014699in}{1.258528in}}{\pgfqpoint{8.009861in}{1.260532in}}{\pgfqpoint{8.004817in}{1.260532in}}%
\pgfpathcurveto{\pgfqpoint{7.999774in}{1.260532in}}{\pgfqpoint{7.994936in}{1.258528in}}{\pgfqpoint{7.991369in}{1.254962in}}%
\pgfpathcurveto{\pgfqpoint{7.987803in}{1.251395in}}{\pgfqpoint{7.985799in}{1.246558in}}{\pgfqpoint{7.985799in}{1.241514in}}%
\pgfpathcurveto{\pgfqpoint{7.985799in}{1.236470in}}{\pgfqpoint{7.987803in}{1.231632in}}{\pgfqpoint{7.991369in}{1.228066in}}%
\pgfpathcurveto{\pgfqpoint{7.994936in}{1.224500in}}{\pgfqpoint{7.999774in}{1.222496in}}{\pgfqpoint{8.004817in}{1.222496in}}%
\pgfpathclose%
\pgfusepath{fill}%
\end{pgfscope}%
\begin{pgfscope}%
\pgfpathrectangle{\pgfqpoint{6.572727in}{0.473000in}}{\pgfqpoint{4.227273in}{3.311000in}}%
\pgfusepath{clip}%
\pgfsetbuttcap%
\pgfsetroundjoin%
\definecolor{currentfill}{rgb}{0.127568,0.566949,0.550556}%
\pgfsetfillcolor{currentfill}%
\pgfsetfillopacity{0.700000}%
\pgfsetlinewidth{0.000000pt}%
\definecolor{currentstroke}{rgb}{0.000000,0.000000,0.000000}%
\pgfsetstrokecolor{currentstroke}%
\pgfsetstrokeopacity{0.700000}%
\pgfsetdash{}{0pt}%
\pgfpathmoveto{\pgfqpoint{8.196048in}{1.379197in}}%
\pgfpathcurveto{\pgfqpoint{8.201092in}{1.379197in}}{\pgfqpoint{8.205930in}{1.381201in}}{\pgfqpoint{8.209496in}{1.384767in}}%
\pgfpathcurveto{\pgfqpoint{8.213063in}{1.388334in}}{\pgfqpoint{8.215066in}{1.393172in}}{\pgfqpoint{8.215066in}{1.398215in}}%
\pgfpathcurveto{\pgfqpoint{8.215066in}{1.403259in}}{\pgfqpoint{8.213063in}{1.408097in}}{\pgfqpoint{8.209496in}{1.411663in}}%
\pgfpathcurveto{\pgfqpoint{8.205930in}{1.415230in}}{\pgfqpoint{8.201092in}{1.417233in}}{\pgfqpoint{8.196048in}{1.417233in}}%
\pgfpathcurveto{\pgfqpoint{8.191005in}{1.417233in}}{\pgfqpoint{8.186167in}{1.415230in}}{\pgfqpoint{8.182600in}{1.411663in}}%
\pgfpathcurveto{\pgfqpoint{8.179034in}{1.408097in}}{\pgfqpoint{8.177030in}{1.403259in}}{\pgfqpoint{8.177030in}{1.398215in}}%
\pgfpathcurveto{\pgfqpoint{8.177030in}{1.393172in}}{\pgfqpoint{8.179034in}{1.388334in}}{\pgfqpoint{8.182600in}{1.384767in}}%
\pgfpathcurveto{\pgfqpoint{8.186167in}{1.381201in}}{\pgfqpoint{8.191005in}{1.379197in}}{\pgfqpoint{8.196048in}{1.379197in}}%
\pgfpathclose%
\pgfusepath{fill}%
\end{pgfscope}%
\begin{pgfscope}%
\pgfpathrectangle{\pgfqpoint{6.572727in}{0.473000in}}{\pgfqpoint{4.227273in}{3.311000in}}%
\pgfusepath{clip}%
\pgfsetbuttcap%
\pgfsetroundjoin%
\definecolor{currentfill}{rgb}{0.127568,0.566949,0.550556}%
\pgfsetfillcolor{currentfill}%
\pgfsetfillopacity{0.700000}%
\pgfsetlinewidth{0.000000pt}%
\definecolor{currentstroke}{rgb}{0.000000,0.000000,0.000000}%
\pgfsetstrokecolor{currentstroke}%
\pgfsetstrokeopacity{0.700000}%
\pgfsetdash{}{0pt}%
\pgfpathmoveto{\pgfqpoint{8.345686in}{2.509321in}}%
\pgfpathcurveto{\pgfqpoint{8.350730in}{2.509321in}}{\pgfqpoint{8.355568in}{2.511325in}}{\pgfqpoint{8.359134in}{2.514891in}}%
\pgfpathcurveto{\pgfqpoint{8.362700in}{2.518458in}}{\pgfqpoint{8.364704in}{2.523296in}}{\pgfqpoint{8.364704in}{2.528339in}}%
\pgfpathcurveto{\pgfqpoint{8.364704in}{2.533383in}}{\pgfqpoint{8.362700in}{2.538221in}}{\pgfqpoint{8.359134in}{2.541787in}}%
\pgfpathcurveto{\pgfqpoint{8.355568in}{2.545354in}}{\pgfqpoint{8.350730in}{2.547357in}}{\pgfqpoint{8.345686in}{2.547357in}}%
\pgfpathcurveto{\pgfqpoint{8.340642in}{2.547357in}}{\pgfqpoint{8.335805in}{2.545354in}}{\pgfqpoint{8.332238in}{2.541787in}}%
\pgfpathcurveto{\pgfqpoint{8.328672in}{2.538221in}}{\pgfqpoint{8.326668in}{2.533383in}}{\pgfqpoint{8.326668in}{2.528339in}}%
\pgfpathcurveto{\pgfqpoint{8.326668in}{2.523296in}}{\pgfqpoint{8.328672in}{2.518458in}}{\pgfqpoint{8.332238in}{2.514891in}}%
\pgfpathcurveto{\pgfqpoint{8.335805in}{2.511325in}}{\pgfqpoint{8.340642in}{2.509321in}}{\pgfqpoint{8.345686in}{2.509321in}}%
\pgfpathclose%
\pgfusepath{fill}%
\end{pgfscope}%
\begin{pgfscope}%
\pgfpathrectangle{\pgfqpoint{6.572727in}{0.473000in}}{\pgfqpoint{4.227273in}{3.311000in}}%
\pgfusepath{clip}%
\pgfsetbuttcap%
\pgfsetroundjoin%
\definecolor{currentfill}{rgb}{0.993248,0.906157,0.143936}%
\pgfsetfillcolor{currentfill}%
\pgfsetfillopacity{0.700000}%
\pgfsetlinewidth{0.000000pt}%
\definecolor{currentstroke}{rgb}{0.000000,0.000000,0.000000}%
\pgfsetstrokecolor{currentstroke}%
\pgfsetstrokeopacity{0.700000}%
\pgfsetdash{}{0pt}%
\pgfpathmoveto{\pgfqpoint{9.600133in}{1.403307in}}%
\pgfpathcurveto{\pgfqpoint{9.605177in}{1.403307in}}{\pgfqpoint{9.610015in}{1.405310in}}{\pgfqpoint{9.613581in}{1.408877in}}%
\pgfpathcurveto{\pgfqpoint{9.617147in}{1.412443in}}{\pgfqpoint{9.619151in}{1.417281in}}{\pgfqpoint{9.619151in}{1.422325in}}%
\pgfpathcurveto{\pgfqpoint{9.619151in}{1.427368in}}{\pgfqpoint{9.617147in}{1.432206in}}{\pgfqpoint{9.613581in}{1.435773in}}%
\pgfpathcurveto{\pgfqpoint{9.610015in}{1.439339in}}{\pgfqpoint{9.605177in}{1.441343in}}{\pgfqpoint{9.600133in}{1.441343in}}%
\pgfpathcurveto{\pgfqpoint{9.595089in}{1.441343in}}{\pgfqpoint{9.590252in}{1.439339in}}{\pgfqpoint{9.586685in}{1.435773in}}%
\pgfpathcurveto{\pgfqpoint{9.583119in}{1.432206in}}{\pgfqpoint{9.581115in}{1.427368in}}{\pgfqpoint{9.581115in}{1.422325in}}%
\pgfpathcurveto{\pgfqpoint{9.581115in}{1.417281in}}{\pgfqpoint{9.583119in}{1.412443in}}{\pgfqpoint{9.586685in}{1.408877in}}%
\pgfpathcurveto{\pgfqpoint{9.590252in}{1.405310in}}{\pgfqpoint{9.595089in}{1.403307in}}{\pgfqpoint{9.600133in}{1.403307in}}%
\pgfpathclose%
\pgfusepath{fill}%
\end{pgfscope}%
\begin{pgfscope}%
\pgfpathrectangle{\pgfqpoint{6.572727in}{0.473000in}}{\pgfqpoint{4.227273in}{3.311000in}}%
\pgfusepath{clip}%
\pgfsetbuttcap%
\pgfsetroundjoin%
\definecolor{currentfill}{rgb}{0.127568,0.566949,0.550556}%
\pgfsetfillcolor{currentfill}%
\pgfsetfillopacity{0.700000}%
\pgfsetlinewidth{0.000000pt}%
\definecolor{currentstroke}{rgb}{0.000000,0.000000,0.000000}%
\pgfsetstrokecolor{currentstroke}%
\pgfsetstrokeopacity{0.700000}%
\pgfsetdash{}{0pt}%
\pgfpathmoveto{\pgfqpoint{8.391248in}{1.034232in}}%
\pgfpathcurveto{\pgfqpoint{8.396292in}{1.034232in}}{\pgfqpoint{8.401129in}{1.036236in}}{\pgfqpoint{8.404696in}{1.039802in}}%
\pgfpathcurveto{\pgfqpoint{8.408262in}{1.043369in}}{\pgfqpoint{8.410266in}{1.048207in}}{\pgfqpoint{8.410266in}{1.053250in}}%
\pgfpathcurveto{\pgfqpoint{8.410266in}{1.058294in}}{\pgfqpoint{8.408262in}{1.063132in}}{\pgfqpoint{8.404696in}{1.066698in}}%
\pgfpathcurveto{\pgfqpoint{8.401129in}{1.070265in}}{\pgfqpoint{8.396292in}{1.072268in}}{\pgfqpoint{8.391248in}{1.072268in}}%
\pgfpathcurveto{\pgfqpoint{8.386204in}{1.072268in}}{\pgfqpoint{8.381366in}{1.070265in}}{\pgfqpoint{8.377800in}{1.066698in}}%
\pgfpathcurveto{\pgfqpoint{8.374234in}{1.063132in}}{\pgfqpoint{8.372230in}{1.058294in}}{\pgfqpoint{8.372230in}{1.053250in}}%
\pgfpathcurveto{\pgfqpoint{8.372230in}{1.048207in}}{\pgfqpoint{8.374234in}{1.043369in}}{\pgfqpoint{8.377800in}{1.039802in}}%
\pgfpathcurveto{\pgfqpoint{8.381366in}{1.036236in}}{\pgfqpoint{8.386204in}{1.034232in}}{\pgfqpoint{8.391248in}{1.034232in}}%
\pgfpathclose%
\pgfusepath{fill}%
\end{pgfscope}%
\begin{pgfscope}%
\pgfpathrectangle{\pgfqpoint{6.572727in}{0.473000in}}{\pgfqpoint{4.227273in}{3.311000in}}%
\pgfusepath{clip}%
\pgfsetbuttcap%
\pgfsetroundjoin%
\definecolor{currentfill}{rgb}{0.993248,0.906157,0.143936}%
\pgfsetfillcolor{currentfill}%
\pgfsetfillopacity{0.700000}%
\pgfsetlinewidth{0.000000pt}%
\definecolor{currentstroke}{rgb}{0.000000,0.000000,0.000000}%
\pgfsetstrokecolor{currentstroke}%
\pgfsetstrokeopacity{0.700000}%
\pgfsetdash{}{0pt}%
\pgfpathmoveto{\pgfqpoint{9.446347in}{1.377742in}}%
\pgfpathcurveto{\pgfqpoint{9.451391in}{1.377742in}}{\pgfqpoint{9.456229in}{1.379745in}}{\pgfqpoint{9.459795in}{1.383312in}}%
\pgfpathcurveto{\pgfqpoint{9.463362in}{1.386878in}}{\pgfqpoint{9.465366in}{1.391716in}}{\pgfqpoint{9.465366in}{1.396760in}}%
\pgfpathcurveto{\pgfqpoint{9.465366in}{1.401803in}}{\pgfqpoint{9.463362in}{1.406641in}}{\pgfqpoint{9.459795in}{1.410208in}}%
\pgfpathcurveto{\pgfqpoint{9.456229in}{1.413774in}}{\pgfqpoint{9.451391in}{1.415778in}}{\pgfqpoint{9.446347in}{1.415778in}}%
\pgfpathcurveto{\pgfqpoint{9.441304in}{1.415778in}}{\pgfqpoint{9.436466in}{1.413774in}}{\pgfqpoint{9.432900in}{1.410208in}}%
\pgfpathcurveto{\pgfqpoint{9.429333in}{1.406641in}}{\pgfqpoint{9.427329in}{1.401803in}}{\pgfqpoint{9.427329in}{1.396760in}}%
\pgfpathcurveto{\pgfqpoint{9.427329in}{1.391716in}}{\pgfqpoint{9.429333in}{1.386878in}}{\pgfqpoint{9.432900in}{1.383312in}}%
\pgfpathcurveto{\pgfqpoint{9.436466in}{1.379745in}}{\pgfqpoint{9.441304in}{1.377742in}}{\pgfqpoint{9.446347in}{1.377742in}}%
\pgfpathclose%
\pgfusepath{fill}%
\end{pgfscope}%
\begin{pgfscope}%
\pgfpathrectangle{\pgfqpoint{6.572727in}{0.473000in}}{\pgfqpoint{4.227273in}{3.311000in}}%
\pgfusepath{clip}%
\pgfsetbuttcap%
\pgfsetroundjoin%
\definecolor{currentfill}{rgb}{0.127568,0.566949,0.550556}%
\pgfsetfillcolor{currentfill}%
\pgfsetfillopacity{0.700000}%
\pgfsetlinewidth{0.000000pt}%
\definecolor{currentstroke}{rgb}{0.000000,0.000000,0.000000}%
\pgfsetstrokecolor{currentstroke}%
\pgfsetstrokeopacity{0.700000}%
\pgfsetdash{}{0pt}%
\pgfpathmoveto{\pgfqpoint{8.137568in}{2.552620in}}%
\pgfpathcurveto{\pgfqpoint{8.142612in}{2.552620in}}{\pgfqpoint{8.147450in}{2.554624in}}{\pgfqpoint{8.151016in}{2.558191in}}%
\pgfpathcurveto{\pgfqpoint{8.154582in}{2.561757in}}{\pgfqpoint{8.156586in}{2.566595in}}{\pgfqpoint{8.156586in}{2.571638in}}%
\pgfpathcurveto{\pgfqpoint{8.156586in}{2.576682in}}{\pgfqpoint{8.154582in}{2.581520in}}{\pgfqpoint{8.151016in}{2.585086in}}%
\pgfpathcurveto{\pgfqpoint{8.147450in}{2.588653in}}{\pgfqpoint{8.142612in}{2.590657in}}{\pgfqpoint{8.137568in}{2.590657in}}%
\pgfpathcurveto{\pgfqpoint{8.132524in}{2.590657in}}{\pgfqpoint{8.127687in}{2.588653in}}{\pgfqpoint{8.124120in}{2.585086in}}%
\pgfpathcurveto{\pgfqpoint{8.120554in}{2.581520in}}{\pgfqpoint{8.118550in}{2.576682in}}{\pgfqpoint{8.118550in}{2.571638in}}%
\pgfpathcurveto{\pgfqpoint{8.118550in}{2.566595in}}{\pgfqpoint{8.120554in}{2.561757in}}{\pgfqpoint{8.124120in}{2.558191in}}%
\pgfpathcurveto{\pgfqpoint{8.127687in}{2.554624in}}{\pgfqpoint{8.132524in}{2.552620in}}{\pgfqpoint{8.137568in}{2.552620in}}%
\pgfpathclose%
\pgfusepath{fill}%
\end{pgfscope}%
\begin{pgfscope}%
\pgfpathrectangle{\pgfqpoint{6.572727in}{0.473000in}}{\pgfqpoint{4.227273in}{3.311000in}}%
\pgfusepath{clip}%
\pgfsetbuttcap%
\pgfsetroundjoin%
\definecolor{currentfill}{rgb}{0.127568,0.566949,0.550556}%
\pgfsetfillcolor{currentfill}%
\pgfsetfillopacity{0.700000}%
\pgfsetlinewidth{0.000000pt}%
\definecolor{currentstroke}{rgb}{0.000000,0.000000,0.000000}%
\pgfsetstrokecolor{currentstroke}%
\pgfsetstrokeopacity{0.700000}%
\pgfsetdash{}{0pt}%
\pgfpathmoveto{\pgfqpoint{8.005069in}{2.676586in}}%
\pgfpathcurveto{\pgfqpoint{8.010113in}{2.676586in}}{\pgfqpoint{8.014950in}{2.678590in}}{\pgfqpoint{8.018517in}{2.682156in}}%
\pgfpathcurveto{\pgfqpoint{8.022083in}{2.685723in}}{\pgfqpoint{8.024087in}{2.690561in}}{\pgfqpoint{8.024087in}{2.695604in}}%
\pgfpathcurveto{\pgfqpoint{8.024087in}{2.700648in}}{\pgfqpoint{8.022083in}{2.705486in}}{\pgfqpoint{8.018517in}{2.709052in}}%
\pgfpathcurveto{\pgfqpoint{8.014950in}{2.712619in}}{\pgfqpoint{8.010113in}{2.714622in}}{\pgfqpoint{8.005069in}{2.714622in}}%
\pgfpathcurveto{\pgfqpoint{8.000025in}{2.714622in}}{\pgfqpoint{7.995188in}{2.712619in}}{\pgfqpoint{7.991621in}{2.709052in}}%
\pgfpathcurveto{\pgfqpoint{7.988055in}{2.705486in}}{\pgfqpoint{7.986051in}{2.700648in}}{\pgfqpoint{7.986051in}{2.695604in}}%
\pgfpathcurveto{\pgfqpoint{7.986051in}{2.690561in}}{\pgfqpoint{7.988055in}{2.685723in}}{\pgfqpoint{7.991621in}{2.682156in}}%
\pgfpathcurveto{\pgfqpoint{7.995188in}{2.678590in}}{\pgfqpoint{8.000025in}{2.676586in}}{\pgfqpoint{8.005069in}{2.676586in}}%
\pgfpathclose%
\pgfusepath{fill}%
\end{pgfscope}%
\begin{pgfscope}%
\pgfpathrectangle{\pgfqpoint{6.572727in}{0.473000in}}{\pgfqpoint{4.227273in}{3.311000in}}%
\pgfusepath{clip}%
\pgfsetbuttcap%
\pgfsetroundjoin%
\definecolor{currentfill}{rgb}{0.127568,0.566949,0.550556}%
\pgfsetfillcolor{currentfill}%
\pgfsetfillopacity{0.700000}%
\pgfsetlinewidth{0.000000pt}%
\definecolor{currentstroke}{rgb}{0.000000,0.000000,0.000000}%
\pgfsetstrokecolor{currentstroke}%
\pgfsetstrokeopacity{0.700000}%
\pgfsetdash{}{0pt}%
\pgfpathmoveto{\pgfqpoint{8.138261in}{1.413456in}}%
\pgfpathcurveto{\pgfqpoint{8.143304in}{1.413456in}}{\pgfqpoint{8.148142in}{1.415460in}}{\pgfqpoint{8.151709in}{1.419026in}}%
\pgfpathcurveto{\pgfqpoint{8.155275in}{1.422593in}}{\pgfqpoint{8.157279in}{1.427430in}}{\pgfqpoint{8.157279in}{1.432474in}}%
\pgfpathcurveto{\pgfqpoint{8.157279in}{1.437518in}}{\pgfqpoint{8.155275in}{1.442356in}}{\pgfqpoint{8.151709in}{1.445922in}}%
\pgfpathcurveto{\pgfqpoint{8.148142in}{1.449488in}}{\pgfqpoint{8.143304in}{1.451492in}}{\pgfqpoint{8.138261in}{1.451492in}}%
\pgfpathcurveto{\pgfqpoint{8.133217in}{1.451492in}}{\pgfqpoint{8.128379in}{1.449488in}}{\pgfqpoint{8.124813in}{1.445922in}}%
\pgfpathcurveto{\pgfqpoint{8.121247in}{1.442356in}}{\pgfqpoint{8.119243in}{1.437518in}}{\pgfqpoint{8.119243in}{1.432474in}}%
\pgfpathcurveto{\pgfqpoint{8.119243in}{1.427430in}}{\pgfqpoint{8.121247in}{1.422593in}}{\pgfqpoint{8.124813in}{1.419026in}}%
\pgfpathcurveto{\pgfqpoint{8.128379in}{1.415460in}}{\pgfqpoint{8.133217in}{1.413456in}}{\pgfqpoint{8.138261in}{1.413456in}}%
\pgfpathclose%
\pgfusepath{fill}%
\end{pgfscope}%
\begin{pgfscope}%
\pgfpathrectangle{\pgfqpoint{6.572727in}{0.473000in}}{\pgfqpoint{4.227273in}{3.311000in}}%
\pgfusepath{clip}%
\pgfsetbuttcap%
\pgfsetroundjoin%
\definecolor{currentfill}{rgb}{0.127568,0.566949,0.550556}%
\pgfsetfillcolor{currentfill}%
\pgfsetfillopacity{0.700000}%
\pgfsetlinewidth{0.000000pt}%
\definecolor{currentstroke}{rgb}{0.000000,0.000000,0.000000}%
\pgfsetstrokecolor{currentstroke}%
\pgfsetstrokeopacity{0.700000}%
\pgfsetdash{}{0pt}%
\pgfpathmoveto{\pgfqpoint{8.041117in}{2.649927in}}%
\pgfpathcurveto{\pgfqpoint{8.046161in}{2.649927in}}{\pgfqpoint{8.050999in}{2.651931in}}{\pgfqpoint{8.054565in}{2.655498in}}%
\pgfpathcurveto{\pgfqpoint{8.058132in}{2.659064in}}{\pgfqpoint{8.060136in}{2.663902in}}{\pgfqpoint{8.060136in}{2.668945in}}%
\pgfpathcurveto{\pgfqpoint{8.060136in}{2.673989in}}{\pgfqpoint{8.058132in}{2.678827in}}{\pgfqpoint{8.054565in}{2.682393in}}%
\pgfpathcurveto{\pgfqpoint{8.050999in}{2.685960in}}{\pgfqpoint{8.046161in}{2.687964in}}{\pgfqpoint{8.041117in}{2.687964in}}%
\pgfpathcurveto{\pgfqpoint{8.036074in}{2.687964in}}{\pgfqpoint{8.031236in}{2.685960in}}{\pgfqpoint{8.027670in}{2.682393in}}%
\pgfpathcurveto{\pgfqpoint{8.024103in}{2.678827in}}{\pgfqpoint{8.022099in}{2.673989in}}{\pgfqpoint{8.022099in}{2.668945in}}%
\pgfpathcurveto{\pgfqpoint{8.022099in}{2.663902in}}{\pgfqpoint{8.024103in}{2.659064in}}{\pgfqpoint{8.027670in}{2.655498in}}%
\pgfpathcurveto{\pgfqpoint{8.031236in}{2.651931in}}{\pgfqpoint{8.036074in}{2.649927in}}{\pgfqpoint{8.041117in}{2.649927in}}%
\pgfpathclose%
\pgfusepath{fill}%
\end{pgfscope}%
\begin{pgfscope}%
\pgfpathrectangle{\pgfqpoint{6.572727in}{0.473000in}}{\pgfqpoint{4.227273in}{3.311000in}}%
\pgfusepath{clip}%
\pgfsetbuttcap%
\pgfsetroundjoin%
\definecolor{currentfill}{rgb}{0.127568,0.566949,0.550556}%
\pgfsetfillcolor{currentfill}%
\pgfsetfillopacity{0.700000}%
\pgfsetlinewidth{0.000000pt}%
\definecolor{currentstroke}{rgb}{0.000000,0.000000,0.000000}%
\pgfsetstrokecolor{currentstroke}%
\pgfsetstrokeopacity{0.700000}%
\pgfsetdash{}{0pt}%
\pgfpathmoveto{\pgfqpoint{8.234294in}{1.488282in}}%
\pgfpathcurveto{\pgfqpoint{8.239338in}{1.488282in}}{\pgfqpoint{8.244176in}{1.490286in}}{\pgfqpoint{8.247742in}{1.493852in}}%
\pgfpathcurveto{\pgfqpoint{8.251309in}{1.497418in}}{\pgfqpoint{8.253313in}{1.502256in}}{\pgfqpoint{8.253313in}{1.507300in}}%
\pgfpathcurveto{\pgfqpoint{8.253313in}{1.512344in}}{\pgfqpoint{8.251309in}{1.517181in}}{\pgfqpoint{8.247742in}{1.520748in}}%
\pgfpathcurveto{\pgfqpoint{8.244176in}{1.524314in}}{\pgfqpoint{8.239338in}{1.526318in}}{\pgfqpoint{8.234294in}{1.526318in}}%
\pgfpathcurveto{\pgfqpoint{8.229251in}{1.526318in}}{\pgfqpoint{8.224413in}{1.524314in}}{\pgfqpoint{8.220847in}{1.520748in}}%
\pgfpathcurveto{\pgfqpoint{8.217280in}{1.517181in}}{\pgfqpoint{8.215276in}{1.512344in}}{\pgfqpoint{8.215276in}{1.507300in}}%
\pgfpathcurveto{\pgfqpoint{8.215276in}{1.502256in}}{\pgfqpoint{8.217280in}{1.497418in}}{\pgfqpoint{8.220847in}{1.493852in}}%
\pgfpathcurveto{\pgfqpoint{8.224413in}{1.490286in}}{\pgfqpoint{8.229251in}{1.488282in}}{\pgfqpoint{8.234294in}{1.488282in}}%
\pgfpathclose%
\pgfusepath{fill}%
\end{pgfscope}%
\begin{pgfscope}%
\pgfpathrectangle{\pgfqpoint{6.572727in}{0.473000in}}{\pgfqpoint{4.227273in}{3.311000in}}%
\pgfusepath{clip}%
\pgfsetbuttcap%
\pgfsetroundjoin%
\definecolor{currentfill}{rgb}{0.127568,0.566949,0.550556}%
\pgfsetfillcolor{currentfill}%
\pgfsetfillopacity{0.700000}%
\pgfsetlinewidth{0.000000pt}%
\definecolor{currentstroke}{rgb}{0.000000,0.000000,0.000000}%
\pgfsetstrokecolor{currentstroke}%
\pgfsetstrokeopacity{0.700000}%
\pgfsetdash{}{0pt}%
\pgfpathmoveto{\pgfqpoint{7.417881in}{2.112839in}}%
\pgfpathcurveto{\pgfqpoint{7.422924in}{2.112839in}}{\pgfqpoint{7.427762in}{2.114843in}}{\pgfqpoint{7.431329in}{2.118410in}}%
\pgfpathcurveto{\pgfqpoint{7.434895in}{2.121976in}}{\pgfqpoint{7.436899in}{2.126814in}}{\pgfqpoint{7.436899in}{2.131858in}}%
\pgfpathcurveto{\pgfqpoint{7.436899in}{2.136901in}}{\pgfqpoint{7.434895in}{2.141739in}}{\pgfqpoint{7.431329in}{2.145305in}}%
\pgfpathcurveto{\pgfqpoint{7.427762in}{2.148872in}}{\pgfqpoint{7.422924in}{2.150876in}}{\pgfqpoint{7.417881in}{2.150876in}}%
\pgfpathcurveto{\pgfqpoint{7.412837in}{2.150876in}}{\pgfqpoint{7.407999in}{2.148872in}}{\pgfqpoint{7.404433in}{2.145305in}}%
\pgfpathcurveto{\pgfqpoint{7.400866in}{2.141739in}}{\pgfqpoint{7.398863in}{2.136901in}}{\pgfqpoint{7.398863in}{2.131858in}}%
\pgfpathcurveto{\pgfqpoint{7.398863in}{2.126814in}}{\pgfqpoint{7.400866in}{2.121976in}}{\pgfqpoint{7.404433in}{2.118410in}}%
\pgfpathcurveto{\pgfqpoint{7.407999in}{2.114843in}}{\pgfqpoint{7.412837in}{2.112839in}}{\pgfqpoint{7.417881in}{2.112839in}}%
\pgfpathclose%
\pgfusepath{fill}%
\end{pgfscope}%
\begin{pgfscope}%
\pgfpathrectangle{\pgfqpoint{6.572727in}{0.473000in}}{\pgfqpoint{4.227273in}{3.311000in}}%
\pgfusepath{clip}%
\pgfsetbuttcap%
\pgfsetroundjoin%
\definecolor{currentfill}{rgb}{0.127568,0.566949,0.550556}%
\pgfsetfillcolor{currentfill}%
\pgfsetfillopacity{0.700000}%
\pgfsetlinewidth{0.000000pt}%
\definecolor{currentstroke}{rgb}{0.000000,0.000000,0.000000}%
\pgfsetstrokecolor{currentstroke}%
\pgfsetstrokeopacity{0.700000}%
\pgfsetdash{}{0pt}%
\pgfpathmoveto{\pgfqpoint{7.862585in}{1.137740in}}%
\pgfpathcurveto{\pgfqpoint{7.867629in}{1.137740in}}{\pgfqpoint{7.872467in}{1.139744in}}{\pgfqpoint{7.876033in}{1.143310in}}%
\pgfpathcurveto{\pgfqpoint{7.879600in}{1.146877in}}{\pgfqpoint{7.881604in}{1.151715in}}{\pgfqpoint{7.881604in}{1.156758in}}%
\pgfpathcurveto{\pgfqpoint{7.881604in}{1.161802in}}{\pgfqpoint{7.879600in}{1.166640in}}{\pgfqpoint{7.876033in}{1.170206in}}%
\pgfpathcurveto{\pgfqpoint{7.872467in}{1.173773in}}{\pgfqpoint{7.867629in}{1.175776in}}{\pgfqpoint{7.862585in}{1.175776in}}%
\pgfpathcurveto{\pgfqpoint{7.857542in}{1.175776in}}{\pgfqpoint{7.852704in}{1.173773in}}{\pgfqpoint{7.849138in}{1.170206in}}%
\pgfpathcurveto{\pgfqpoint{7.845571in}{1.166640in}}{\pgfqpoint{7.843567in}{1.161802in}}{\pgfqpoint{7.843567in}{1.156758in}}%
\pgfpathcurveto{\pgfqpoint{7.843567in}{1.151715in}}{\pgfqpoint{7.845571in}{1.146877in}}{\pgfqpoint{7.849138in}{1.143310in}}%
\pgfpathcurveto{\pgfqpoint{7.852704in}{1.139744in}}{\pgfqpoint{7.857542in}{1.137740in}}{\pgfqpoint{7.862585in}{1.137740in}}%
\pgfpathclose%
\pgfusepath{fill}%
\end{pgfscope}%
\begin{pgfscope}%
\pgfpathrectangle{\pgfqpoint{6.572727in}{0.473000in}}{\pgfqpoint{4.227273in}{3.311000in}}%
\pgfusepath{clip}%
\pgfsetbuttcap%
\pgfsetroundjoin%
\definecolor{currentfill}{rgb}{0.127568,0.566949,0.550556}%
\pgfsetfillcolor{currentfill}%
\pgfsetfillopacity{0.700000}%
\pgfsetlinewidth{0.000000pt}%
\definecolor{currentstroke}{rgb}{0.000000,0.000000,0.000000}%
\pgfsetstrokecolor{currentstroke}%
\pgfsetstrokeopacity{0.700000}%
\pgfsetdash{}{0pt}%
\pgfpathmoveto{\pgfqpoint{8.169487in}{1.753210in}}%
\pgfpathcurveto{\pgfqpoint{8.174530in}{1.753210in}}{\pgfqpoint{8.179368in}{1.755214in}}{\pgfqpoint{8.182935in}{1.758780in}}%
\pgfpathcurveto{\pgfqpoint{8.186501in}{1.762347in}}{\pgfqpoint{8.188505in}{1.767185in}}{\pgfqpoint{8.188505in}{1.772228in}}%
\pgfpathcurveto{\pgfqpoint{8.188505in}{1.777272in}}{\pgfqpoint{8.186501in}{1.782110in}}{\pgfqpoint{8.182935in}{1.785676in}}%
\pgfpathcurveto{\pgfqpoint{8.179368in}{1.789243in}}{\pgfqpoint{8.174530in}{1.791246in}}{\pgfqpoint{8.169487in}{1.791246in}}%
\pgfpathcurveto{\pgfqpoint{8.164443in}{1.791246in}}{\pgfqpoint{8.159605in}{1.789243in}}{\pgfqpoint{8.156039in}{1.785676in}}%
\pgfpathcurveto{\pgfqpoint{8.152473in}{1.782110in}}{\pgfqpoint{8.150469in}{1.777272in}}{\pgfqpoint{8.150469in}{1.772228in}}%
\pgfpathcurveto{\pgfqpoint{8.150469in}{1.767185in}}{\pgfqpoint{8.152473in}{1.762347in}}{\pgfqpoint{8.156039in}{1.758780in}}%
\pgfpathcurveto{\pgfqpoint{8.159605in}{1.755214in}}{\pgfqpoint{8.164443in}{1.753210in}}{\pgfqpoint{8.169487in}{1.753210in}}%
\pgfpathclose%
\pgfusepath{fill}%
\end{pgfscope}%
\begin{pgfscope}%
\pgfpathrectangle{\pgfqpoint{6.572727in}{0.473000in}}{\pgfqpoint{4.227273in}{3.311000in}}%
\pgfusepath{clip}%
\pgfsetbuttcap%
\pgfsetroundjoin%
\definecolor{currentfill}{rgb}{0.127568,0.566949,0.550556}%
\pgfsetfillcolor{currentfill}%
\pgfsetfillopacity{0.700000}%
\pgfsetlinewidth{0.000000pt}%
\definecolor{currentstroke}{rgb}{0.000000,0.000000,0.000000}%
\pgfsetstrokecolor{currentstroke}%
\pgfsetstrokeopacity{0.700000}%
\pgfsetdash{}{0pt}%
\pgfpathmoveto{\pgfqpoint{7.683748in}{1.928807in}}%
\pgfpathcurveto{\pgfqpoint{7.688792in}{1.928807in}}{\pgfqpoint{7.693629in}{1.930811in}}{\pgfqpoint{7.697196in}{1.934377in}}%
\pgfpathcurveto{\pgfqpoint{7.700762in}{1.937944in}}{\pgfqpoint{7.702766in}{1.942781in}}{\pgfqpoint{7.702766in}{1.947825in}}%
\pgfpathcurveto{\pgfqpoint{7.702766in}{1.952869in}}{\pgfqpoint{7.700762in}{1.957707in}}{\pgfqpoint{7.697196in}{1.961273in}}%
\pgfpathcurveto{\pgfqpoint{7.693629in}{1.964839in}}{\pgfqpoint{7.688792in}{1.966843in}}{\pgfqpoint{7.683748in}{1.966843in}}%
\pgfpathcurveto{\pgfqpoint{7.678704in}{1.966843in}}{\pgfqpoint{7.673867in}{1.964839in}}{\pgfqpoint{7.670300in}{1.961273in}}%
\pgfpathcurveto{\pgfqpoint{7.666734in}{1.957707in}}{\pgfqpoint{7.664730in}{1.952869in}}{\pgfqpoint{7.664730in}{1.947825in}}%
\pgfpathcurveto{\pgfqpoint{7.664730in}{1.942781in}}{\pgfqpoint{7.666734in}{1.937944in}}{\pgfqpoint{7.670300in}{1.934377in}}%
\pgfpathcurveto{\pgfqpoint{7.673867in}{1.930811in}}{\pgfqpoint{7.678704in}{1.928807in}}{\pgfqpoint{7.683748in}{1.928807in}}%
\pgfpathclose%
\pgfusepath{fill}%
\end{pgfscope}%
\begin{pgfscope}%
\pgfpathrectangle{\pgfqpoint{6.572727in}{0.473000in}}{\pgfqpoint{4.227273in}{3.311000in}}%
\pgfusepath{clip}%
\pgfsetbuttcap%
\pgfsetroundjoin%
\definecolor{currentfill}{rgb}{0.993248,0.906157,0.143936}%
\pgfsetfillcolor{currentfill}%
\pgfsetfillopacity{0.700000}%
\pgfsetlinewidth{0.000000pt}%
\definecolor{currentstroke}{rgb}{0.000000,0.000000,0.000000}%
\pgfsetstrokecolor{currentstroke}%
\pgfsetstrokeopacity{0.700000}%
\pgfsetdash{}{0pt}%
\pgfpathmoveto{\pgfqpoint{9.038013in}{1.553876in}}%
\pgfpathcurveto{\pgfqpoint{9.043056in}{1.553876in}}{\pgfqpoint{9.047894in}{1.555880in}}{\pgfqpoint{9.051461in}{1.559446in}}%
\pgfpathcurveto{\pgfqpoint{9.055027in}{1.563013in}}{\pgfqpoint{9.057031in}{1.567851in}}{\pgfqpoint{9.057031in}{1.572894in}}%
\pgfpathcurveto{\pgfqpoint{9.057031in}{1.577938in}}{\pgfqpoint{9.055027in}{1.582776in}}{\pgfqpoint{9.051461in}{1.586342in}}%
\pgfpathcurveto{\pgfqpoint{9.047894in}{1.589909in}}{\pgfqpoint{9.043056in}{1.591912in}}{\pgfqpoint{9.038013in}{1.591912in}}%
\pgfpathcurveto{\pgfqpoint{9.032969in}{1.591912in}}{\pgfqpoint{9.028131in}{1.589909in}}{\pgfqpoint{9.024565in}{1.586342in}}%
\pgfpathcurveto{\pgfqpoint{9.020998in}{1.582776in}}{\pgfqpoint{9.018995in}{1.577938in}}{\pgfqpoint{9.018995in}{1.572894in}}%
\pgfpathcurveto{\pgfqpoint{9.018995in}{1.567851in}}{\pgfqpoint{9.020998in}{1.563013in}}{\pgfqpoint{9.024565in}{1.559446in}}%
\pgfpathcurveto{\pgfqpoint{9.028131in}{1.555880in}}{\pgfqpoint{9.032969in}{1.553876in}}{\pgfqpoint{9.038013in}{1.553876in}}%
\pgfpathclose%
\pgfusepath{fill}%
\end{pgfscope}%
\begin{pgfscope}%
\pgfpathrectangle{\pgfqpoint{6.572727in}{0.473000in}}{\pgfqpoint{4.227273in}{3.311000in}}%
\pgfusepath{clip}%
\pgfsetbuttcap%
\pgfsetroundjoin%
\definecolor{currentfill}{rgb}{0.127568,0.566949,0.550556}%
\pgfsetfillcolor{currentfill}%
\pgfsetfillopacity{0.700000}%
\pgfsetlinewidth{0.000000pt}%
\definecolor{currentstroke}{rgb}{0.000000,0.000000,0.000000}%
\pgfsetstrokecolor{currentstroke}%
\pgfsetstrokeopacity{0.700000}%
\pgfsetdash{}{0pt}%
\pgfpathmoveto{\pgfqpoint{8.481208in}{2.281921in}}%
\pgfpathcurveto{\pgfqpoint{8.486252in}{2.281921in}}{\pgfqpoint{8.491090in}{2.283925in}}{\pgfqpoint{8.494656in}{2.287491in}}%
\pgfpathcurveto{\pgfqpoint{8.498223in}{2.291058in}}{\pgfqpoint{8.500226in}{2.295895in}}{\pgfqpoint{8.500226in}{2.300939in}}%
\pgfpathcurveto{\pgfqpoint{8.500226in}{2.305983in}}{\pgfqpoint{8.498223in}{2.310820in}}{\pgfqpoint{8.494656in}{2.314387in}}%
\pgfpathcurveto{\pgfqpoint{8.491090in}{2.317953in}}{\pgfqpoint{8.486252in}{2.319957in}}{\pgfqpoint{8.481208in}{2.319957in}}%
\pgfpathcurveto{\pgfqpoint{8.476165in}{2.319957in}}{\pgfqpoint{8.471327in}{2.317953in}}{\pgfqpoint{8.467760in}{2.314387in}}%
\pgfpathcurveto{\pgfqpoint{8.464194in}{2.310820in}}{\pgfqpoint{8.462190in}{2.305983in}}{\pgfqpoint{8.462190in}{2.300939in}}%
\pgfpathcurveto{\pgfqpoint{8.462190in}{2.295895in}}{\pgfqpoint{8.464194in}{2.291058in}}{\pgfqpoint{8.467760in}{2.287491in}}%
\pgfpathcurveto{\pgfqpoint{8.471327in}{2.283925in}}{\pgfqpoint{8.476165in}{2.281921in}}{\pgfqpoint{8.481208in}{2.281921in}}%
\pgfpathclose%
\pgfusepath{fill}%
\end{pgfscope}%
\begin{pgfscope}%
\pgfpathrectangle{\pgfqpoint{6.572727in}{0.473000in}}{\pgfqpoint{4.227273in}{3.311000in}}%
\pgfusepath{clip}%
\pgfsetbuttcap%
\pgfsetroundjoin%
\definecolor{currentfill}{rgb}{0.993248,0.906157,0.143936}%
\pgfsetfillcolor{currentfill}%
\pgfsetfillopacity{0.700000}%
\pgfsetlinewidth{0.000000pt}%
\definecolor{currentstroke}{rgb}{0.000000,0.000000,0.000000}%
\pgfsetstrokecolor{currentstroke}%
\pgfsetstrokeopacity{0.700000}%
\pgfsetdash{}{0pt}%
\pgfpathmoveto{\pgfqpoint{9.314467in}{1.884049in}}%
\pgfpathcurveto{\pgfqpoint{9.319511in}{1.884049in}}{\pgfqpoint{9.324349in}{1.886053in}}{\pgfqpoint{9.327915in}{1.889619in}}%
\pgfpathcurveto{\pgfqpoint{9.331481in}{1.893186in}}{\pgfqpoint{9.333485in}{1.898023in}}{\pgfqpoint{9.333485in}{1.903067in}}%
\pgfpathcurveto{\pgfqpoint{9.333485in}{1.908111in}}{\pgfqpoint{9.331481in}{1.912948in}}{\pgfqpoint{9.327915in}{1.916515in}}%
\pgfpathcurveto{\pgfqpoint{9.324349in}{1.920081in}}{\pgfqpoint{9.319511in}{1.922085in}}{\pgfqpoint{9.314467in}{1.922085in}}%
\pgfpathcurveto{\pgfqpoint{9.309423in}{1.922085in}}{\pgfqpoint{9.304586in}{1.920081in}}{\pgfqpoint{9.301019in}{1.916515in}}%
\pgfpathcurveto{\pgfqpoint{9.297453in}{1.912948in}}{\pgfqpoint{9.295449in}{1.908111in}}{\pgfqpoint{9.295449in}{1.903067in}}%
\pgfpathcurveto{\pgfqpoint{9.295449in}{1.898023in}}{\pgfqpoint{9.297453in}{1.893186in}}{\pgfqpoint{9.301019in}{1.889619in}}%
\pgfpathcurveto{\pgfqpoint{9.304586in}{1.886053in}}{\pgfqpoint{9.309423in}{1.884049in}}{\pgfqpoint{9.314467in}{1.884049in}}%
\pgfpathclose%
\pgfusepath{fill}%
\end{pgfscope}%
\begin{pgfscope}%
\pgfpathrectangle{\pgfqpoint{6.572727in}{0.473000in}}{\pgfqpoint{4.227273in}{3.311000in}}%
\pgfusepath{clip}%
\pgfsetbuttcap%
\pgfsetroundjoin%
\definecolor{currentfill}{rgb}{0.127568,0.566949,0.550556}%
\pgfsetfillcolor{currentfill}%
\pgfsetfillopacity{0.700000}%
\pgfsetlinewidth{0.000000pt}%
\definecolor{currentstroke}{rgb}{0.000000,0.000000,0.000000}%
\pgfsetstrokecolor{currentstroke}%
\pgfsetstrokeopacity{0.700000}%
\pgfsetdash{}{0pt}%
\pgfpathmoveto{\pgfqpoint{7.843992in}{0.897554in}}%
\pgfpathcurveto{\pgfqpoint{7.849036in}{0.897554in}}{\pgfqpoint{7.853873in}{0.899558in}}{\pgfqpoint{7.857440in}{0.903125in}}%
\pgfpathcurveto{\pgfqpoint{7.861006in}{0.906691in}}{\pgfqpoint{7.863010in}{0.911529in}}{\pgfqpoint{7.863010in}{0.916573in}}%
\pgfpathcurveto{\pgfqpoint{7.863010in}{0.921616in}}{\pgfqpoint{7.861006in}{0.926454in}}{\pgfqpoint{7.857440in}{0.930020in}}%
\pgfpathcurveto{\pgfqpoint{7.853873in}{0.933587in}}{\pgfqpoint{7.849036in}{0.935591in}}{\pgfqpoint{7.843992in}{0.935591in}}%
\pgfpathcurveto{\pgfqpoint{7.838948in}{0.935591in}}{\pgfqpoint{7.834110in}{0.933587in}}{\pgfqpoint{7.830544in}{0.930020in}}%
\pgfpathcurveto{\pgfqpoint{7.826978in}{0.926454in}}{\pgfqpoint{7.824974in}{0.921616in}}{\pgfqpoint{7.824974in}{0.916573in}}%
\pgfpathcurveto{\pgfqpoint{7.824974in}{0.911529in}}{\pgfqpoint{7.826978in}{0.906691in}}{\pgfqpoint{7.830544in}{0.903125in}}%
\pgfpathcurveto{\pgfqpoint{7.834110in}{0.899558in}}{\pgfqpoint{7.838948in}{0.897554in}}{\pgfqpoint{7.843992in}{0.897554in}}%
\pgfpathclose%
\pgfusepath{fill}%
\end{pgfscope}%
\begin{pgfscope}%
\pgfpathrectangle{\pgfqpoint{6.572727in}{0.473000in}}{\pgfqpoint{4.227273in}{3.311000in}}%
\pgfusepath{clip}%
\pgfsetbuttcap%
\pgfsetroundjoin%
\definecolor{currentfill}{rgb}{0.127568,0.566949,0.550556}%
\pgfsetfillcolor{currentfill}%
\pgfsetfillopacity{0.700000}%
\pgfsetlinewidth{0.000000pt}%
\definecolor{currentstroke}{rgb}{0.000000,0.000000,0.000000}%
\pgfsetstrokecolor{currentstroke}%
\pgfsetstrokeopacity{0.700000}%
\pgfsetdash{}{0pt}%
\pgfpathmoveto{\pgfqpoint{8.501472in}{2.808643in}}%
\pgfpathcurveto{\pgfqpoint{8.506515in}{2.808643in}}{\pgfqpoint{8.511353in}{2.810647in}}{\pgfqpoint{8.514919in}{2.814214in}}%
\pgfpathcurveto{\pgfqpoint{8.518486in}{2.817780in}}{\pgfqpoint{8.520490in}{2.822618in}}{\pgfqpoint{8.520490in}{2.827662in}}%
\pgfpathcurveto{\pgfqpoint{8.520490in}{2.832705in}}{\pgfqpoint{8.518486in}{2.837543in}}{\pgfqpoint{8.514919in}{2.841109in}}%
\pgfpathcurveto{\pgfqpoint{8.511353in}{2.844676in}}{\pgfqpoint{8.506515in}{2.846680in}}{\pgfqpoint{8.501472in}{2.846680in}}%
\pgfpathcurveto{\pgfqpoint{8.496428in}{2.846680in}}{\pgfqpoint{8.491590in}{2.844676in}}{\pgfqpoint{8.488024in}{2.841109in}}%
\pgfpathcurveto{\pgfqpoint{8.484457in}{2.837543in}}{\pgfqpoint{8.482453in}{2.832705in}}{\pgfqpoint{8.482453in}{2.827662in}}%
\pgfpathcurveto{\pgfqpoint{8.482453in}{2.822618in}}{\pgfqpoint{8.484457in}{2.817780in}}{\pgfqpoint{8.488024in}{2.814214in}}%
\pgfpathcurveto{\pgfqpoint{8.491590in}{2.810647in}}{\pgfqpoint{8.496428in}{2.808643in}}{\pgfqpoint{8.501472in}{2.808643in}}%
\pgfpathclose%
\pgfusepath{fill}%
\end{pgfscope}%
\begin{pgfscope}%
\pgfpathrectangle{\pgfqpoint{6.572727in}{0.473000in}}{\pgfqpoint{4.227273in}{3.311000in}}%
\pgfusepath{clip}%
\pgfsetbuttcap%
\pgfsetroundjoin%
\definecolor{currentfill}{rgb}{0.127568,0.566949,0.550556}%
\pgfsetfillcolor{currentfill}%
\pgfsetfillopacity{0.700000}%
\pgfsetlinewidth{0.000000pt}%
\definecolor{currentstroke}{rgb}{0.000000,0.000000,0.000000}%
\pgfsetstrokecolor{currentstroke}%
\pgfsetstrokeopacity{0.700000}%
\pgfsetdash{}{0pt}%
\pgfpathmoveto{\pgfqpoint{8.110796in}{2.859808in}}%
\pgfpathcurveto{\pgfqpoint{8.115839in}{2.859808in}}{\pgfqpoint{8.120677in}{2.861812in}}{\pgfqpoint{8.124244in}{2.865378in}}%
\pgfpathcurveto{\pgfqpoint{8.127810in}{2.868945in}}{\pgfqpoint{8.129814in}{2.873782in}}{\pgfqpoint{8.129814in}{2.878826in}}%
\pgfpathcurveto{\pgfqpoint{8.129814in}{2.883870in}}{\pgfqpoint{8.127810in}{2.888708in}}{\pgfqpoint{8.124244in}{2.892274in}}%
\pgfpathcurveto{\pgfqpoint{8.120677in}{2.895840in}}{\pgfqpoint{8.115839in}{2.897844in}}{\pgfqpoint{8.110796in}{2.897844in}}%
\pgfpathcurveto{\pgfqpoint{8.105752in}{2.897844in}}{\pgfqpoint{8.100914in}{2.895840in}}{\pgfqpoint{8.097348in}{2.892274in}}%
\pgfpathcurveto{\pgfqpoint{8.093781in}{2.888708in}}{\pgfqpoint{8.091778in}{2.883870in}}{\pgfqpoint{8.091778in}{2.878826in}}%
\pgfpathcurveto{\pgfqpoint{8.091778in}{2.873782in}}{\pgfqpoint{8.093781in}{2.868945in}}{\pgfqpoint{8.097348in}{2.865378in}}%
\pgfpathcurveto{\pgfqpoint{8.100914in}{2.861812in}}{\pgfqpoint{8.105752in}{2.859808in}}{\pgfqpoint{8.110796in}{2.859808in}}%
\pgfpathclose%
\pgfusepath{fill}%
\end{pgfscope}%
\begin{pgfscope}%
\pgfpathrectangle{\pgfqpoint{6.572727in}{0.473000in}}{\pgfqpoint{4.227273in}{3.311000in}}%
\pgfusepath{clip}%
\pgfsetbuttcap%
\pgfsetroundjoin%
\definecolor{currentfill}{rgb}{0.127568,0.566949,0.550556}%
\pgfsetfillcolor{currentfill}%
\pgfsetfillopacity{0.700000}%
\pgfsetlinewidth{0.000000pt}%
\definecolor{currentstroke}{rgb}{0.000000,0.000000,0.000000}%
\pgfsetstrokecolor{currentstroke}%
\pgfsetstrokeopacity{0.700000}%
\pgfsetdash{}{0pt}%
\pgfpathmoveto{\pgfqpoint{8.297181in}{1.899928in}}%
\pgfpathcurveto{\pgfqpoint{8.302224in}{1.899928in}}{\pgfqpoint{8.307062in}{1.901932in}}{\pgfqpoint{8.310629in}{1.905499in}}%
\pgfpathcurveto{\pgfqpoint{8.314195in}{1.909065in}}{\pgfqpoint{8.316199in}{1.913903in}}{\pgfqpoint{8.316199in}{1.918946in}}%
\pgfpathcurveto{\pgfqpoint{8.316199in}{1.923990in}}{\pgfqpoint{8.314195in}{1.928828in}}{\pgfqpoint{8.310629in}{1.932394in}}%
\pgfpathcurveto{\pgfqpoint{8.307062in}{1.935961in}}{\pgfqpoint{8.302224in}{1.937965in}}{\pgfqpoint{8.297181in}{1.937965in}}%
\pgfpathcurveto{\pgfqpoint{8.292137in}{1.937965in}}{\pgfqpoint{8.287299in}{1.935961in}}{\pgfqpoint{8.283733in}{1.932394in}}%
\pgfpathcurveto{\pgfqpoint{8.280167in}{1.928828in}}{\pgfqpoint{8.278163in}{1.923990in}}{\pgfqpoint{8.278163in}{1.918946in}}%
\pgfpathcurveto{\pgfqpoint{8.278163in}{1.913903in}}{\pgfqpoint{8.280167in}{1.909065in}}{\pgfqpoint{8.283733in}{1.905499in}}%
\pgfpathcurveto{\pgfqpoint{8.287299in}{1.901932in}}{\pgfqpoint{8.292137in}{1.899928in}}{\pgfqpoint{8.297181in}{1.899928in}}%
\pgfpathclose%
\pgfusepath{fill}%
\end{pgfscope}%
\begin{pgfscope}%
\pgfpathrectangle{\pgfqpoint{6.572727in}{0.473000in}}{\pgfqpoint{4.227273in}{3.311000in}}%
\pgfusepath{clip}%
\pgfsetbuttcap%
\pgfsetroundjoin%
\definecolor{currentfill}{rgb}{0.127568,0.566949,0.550556}%
\pgfsetfillcolor{currentfill}%
\pgfsetfillopacity{0.700000}%
\pgfsetlinewidth{0.000000pt}%
\definecolor{currentstroke}{rgb}{0.000000,0.000000,0.000000}%
\pgfsetstrokecolor{currentstroke}%
\pgfsetstrokeopacity{0.700000}%
\pgfsetdash{}{0pt}%
\pgfpathmoveto{\pgfqpoint{8.738845in}{1.417627in}}%
\pgfpathcurveto{\pgfqpoint{8.743889in}{1.417627in}}{\pgfqpoint{8.748727in}{1.419631in}}{\pgfqpoint{8.752293in}{1.423197in}}%
\pgfpathcurveto{\pgfqpoint{8.755860in}{1.426764in}}{\pgfqpoint{8.757863in}{1.431601in}}{\pgfqpoint{8.757863in}{1.436645in}}%
\pgfpathcurveto{\pgfqpoint{8.757863in}{1.441689in}}{\pgfqpoint{8.755860in}{1.446526in}}{\pgfqpoint{8.752293in}{1.450093in}}%
\pgfpathcurveto{\pgfqpoint{8.748727in}{1.453659in}}{\pgfqpoint{8.743889in}{1.455663in}}{\pgfqpoint{8.738845in}{1.455663in}}%
\pgfpathcurveto{\pgfqpoint{8.733802in}{1.455663in}}{\pgfqpoint{8.728964in}{1.453659in}}{\pgfqpoint{8.725397in}{1.450093in}}%
\pgfpathcurveto{\pgfqpoint{8.721831in}{1.446526in}}{\pgfqpoint{8.719827in}{1.441689in}}{\pgfqpoint{8.719827in}{1.436645in}}%
\pgfpathcurveto{\pgfqpoint{8.719827in}{1.431601in}}{\pgfqpoint{8.721831in}{1.426764in}}{\pgfqpoint{8.725397in}{1.423197in}}%
\pgfpathcurveto{\pgfqpoint{8.728964in}{1.419631in}}{\pgfqpoint{8.733802in}{1.417627in}}{\pgfqpoint{8.738845in}{1.417627in}}%
\pgfpathclose%
\pgfusepath{fill}%
\end{pgfscope}%
\begin{pgfscope}%
\pgfpathrectangle{\pgfqpoint{6.572727in}{0.473000in}}{\pgfqpoint{4.227273in}{3.311000in}}%
\pgfusepath{clip}%
\pgfsetbuttcap%
\pgfsetroundjoin%
\definecolor{currentfill}{rgb}{0.993248,0.906157,0.143936}%
\pgfsetfillcolor{currentfill}%
\pgfsetfillopacity{0.700000}%
\pgfsetlinewidth{0.000000pt}%
\definecolor{currentstroke}{rgb}{0.000000,0.000000,0.000000}%
\pgfsetstrokecolor{currentstroke}%
\pgfsetstrokeopacity{0.700000}%
\pgfsetdash{}{0pt}%
\pgfpathmoveto{\pgfqpoint{9.144984in}{1.283865in}}%
\pgfpathcurveto{\pgfqpoint{9.150027in}{1.283865in}}{\pgfqpoint{9.154865in}{1.285869in}}{\pgfqpoint{9.158431in}{1.289436in}}%
\pgfpathcurveto{\pgfqpoint{9.161998in}{1.293002in}}{\pgfqpoint{9.164002in}{1.297840in}}{\pgfqpoint{9.164002in}{1.302884in}}%
\pgfpathcurveto{\pgfqpoint{9.164002in}{1.307927in}}{\pgfqpoint{9.161998in}{1.312765in}}{\pgfqpoint{9.158431in}{1.316331in}}%
\pgfpathcurveto{\pgfqpoint{9.154865in}{1.319898in}}{\pgfqpoint{9.150027in}{1.321902in}}{\pgfqpoint{9.144984in}{1.321902in}}%
\pgfpathcurveto{\pgfqpoint{9.139940in}{1.321902in}}{\pgfqpoint{9.135102in}{1.319898in}}{\pgfqpoint{9.131536in}{1.316331in}}%
\pgfpathcurveto{\pgfqpoint{9.127969in}{1.312765in}}{\pgfqpoint{9.125965in}{1.307927in}}{\pgfqpoint{9.125965in}{1.302884in}}%
\pgfpathcurveto{\pgfqpoint{9.125965in}{1.297840in}}{\pgfqpoint{9.127969in}{1.293002in}}{\pgfqpoint{9.131536in}{1.289436in}}%
\pgfpathcurveto{\pgfqpoint{9.135102in}{1.285869in}}{\pgfqpoint{9.139940in}{1.283865in}}{\pgfqpoint{9.144984in}{1.283865in}}%
\pgfpathclose%
\pgfusepath{fill}%
\end{pgfscope}%
\begin{pgfscope}%
\pgfpathrectangle{\pgfqpoint{6.572727in}{0.473000in}}{\pgfqpoint{4.227273in}{3.311000in}}%
\pgfusepath{clip}%
\pgfsetbuttcap%
\pgfsetroundjoin%
\definecolor{currentfill}{rgb}{0.993248,0.906157,0.143936}%
\pgfsetfillcolor{currentfill}%
\pgfsetfillopacity{0.700000}%
\pgfsetlinewidth{0.000000pt}%
\definecolor{currentstroke}{rgb}{0.000000,0.000000,0.000000}%
\pgfsetstrokecolor{currentstroke}%
\pgfsetstrokeopacity{0.700000}%
\pgfsetdash{}{0pt}%
\pgfpathmoveto{\pgfqpoint{9.338292in}{1.462672in}}%
\pgfpathcurveto{\pgfqpoint{9.343335in}{1.462672in}}{\pgfqpoint{9.348173in}{1.464676in}}{\pgfqpoint{9.351740in}{1.468243in}}%
\pgfpathcurveto{\pgfqpoint{9.355306in}{1.471809in}}{\pgfqpoint{9.357310in}{1.476647in}}{\pgfqpoint{9.357310in}{1.481690in}}%
\pgfpathcurveto{\pgfqpoint{9.357310in}{1.486734in}}{\pgfqpoint{9.355306in}{1.491572in}}{\pgfqpoint{9.351740in}{1.495138in}}%
\pgfpathcurveto{\pgfqpoint{9.348173in}{1.498705in}}{\pgfqpoint{9.343335in}{1.500709in}}{\pgfqpoint{9.338292in}{1.500709in}}%
\pgfpathcurveto{\pgfqpoint{9.333248in}{1.500709in}}{\pgfqpoint{9.328410in}{1.498705in}}{\pgfqpoint{9.324844in}{1.495138in}}%
\pgfpathcurveto{\pgfqpoint{9.321277in}{1.491572in}}{\pgfqpoint{9.319274in}{1.486734in}}{\pgfqpoint{9.319274in}{1.481690in}}%
\pgfpathcurveto{\pgfqpoint{9.319274in}{1.476647in}}{\pgfqpoint{9.321277in}{1.471809in}}{\pgfqpoint{9.324844in}{1.468243in}}%
\pgfpathcurveto{\pgfqpoint{9.328410in}{1.464676in}}{\pgfqpoint{9.333248in}{1.462672in}}{\pgfqpoint{9.338292in}{1.462672in}}%
\pgfpathclose%
\pgfusepath{fill}%
\end{pgfscope}%
\begin{pgfscope}%
\pgfpathrectangle{\pgfqpoint{6.572727in}{0.473000in}}{\pgfqpoint{4.227273in}{3.311000in}}%
\pgfusepath{clip}%
\pgfsetbuttcap%
\pgfsetroundjoin%
\definecolor{currentfill}{rgb}{0.127568,0.566949,0.550556}%
\pgfsetfillcolor{currentfill}%
\pgfsetfillopacity{0.700000}%
\pgfsetlinewidth{0.000000pt}%
\definecolor{currentstroke}{rgb}{0.000000,0.000000,0.000000}%
\pgfsetstrokecolor{currentstroke}%
\pgfsetstrokeopacity{0.700000}%
\pgfsetdash{}{0pt}%
\pgfpathmoveto{\pgfqpoint{8.082784in}{2.713587in}}%
\pgfpathcurveto{\pgfqpoint{8.087827in}{2.713587in}}{\pgfqpoint{8.092665in}{2.715591in}}{\pgfqpoint{8.096232in}{2.719158in}}%
\pgfpathcurveto{\pgfqpoint{8.099798in}{2.722724in}}{\pgfqpoint{8.101802in}{2.727562in}}{\pgfqpoint{8.101802in}{2.732605in}}%
\pgfpathcurveto{\pgfqpoint{8.101802in}{2.737649in}}{\pgfqpoint{8.099798in}{2.742487in}}{\pgfqpoint{8.096232in}{2.746053in}}%
\pgfpathcurveto{\pgfqpoint{8.092665in}{2.749620in}}{\pgfqpoint{8.087827in}{2.751624in}}{\pgfqpoint{8.082784in}{2.751624in}}%
\pgfpathcurveto{\pgfqpoint{8.077740in}{2.751624in}}{\pgfqpoint{8.072902in}{2.749620in}}{\pgfqpoint{8.069336in}{2.746053in}}%
\pgfpathcurveto{\pgfqpoint{8.065769in}{2.742487in}}{\pgfqpoint{8.063766in}{2.737649in}}{\pgfqpoint{8.063766in}{2.732605in}}%
\pgfpathcurveto{\pgfqpoint{8.063766in}{2.727562in}}{\pgfqpoint{8.065769in}{2.722724in}}{\pgfqpoint{8.069336in}{2.719158in}}%
\pgfpathcurveto{\pgfqpoint{8.072902in}{2.715591in}}{\pgfqpoint{8.077740in}{2.713587in}}{\pgfqpoint{8.082784in}{2.713587in}}%
\pgfpathclose%
\pgfusepath{fill}%
\end{pgfscope}%
\begin{pgfscope}%
\pgfpathrectangle{\pgfqpoint{6.572727in}{0.473000in}}{\pgfqpoint{4.227273in}{3.311000in}}%
\pgfusepath{clip}%
\pgfsetbuttcap%
\pgfsetroundjoin%
\definecolor{currentfill}{rgb}{0.993248,0.906157,0.143936}%
\pgfsetfillcolor{currentfill}%
\pgfsetfillopacity{0.700000}%
\pgfsetlinewidth{0.000000pt}%
\definecolor{currentstroke}{rgb}{0.000000,0.000000,0.000000}%
\pgfsetstrokecolor{currentstroke}%
\pgfsetstrokeopacity{0.700000}%
\pgfsetdash{}{0pt}%
\pgfpathmoveto{\pgfqpoint{10.523575in}{1.482138in}}%
\pgfpathcurveto{\pgfqpoint{10.528619in}{1.482138in}}{\pgfqpoint{10.533457in}{1.484142in}}{\pgfqpoint{10.537023in}{1.487708in}}%
\pgfpathcurveto{\pgfqpoint{10.540590in}{1.491275in}}{\pgfqpoint{10.542593in}{1.496112in}}{\pgfqpoint{10.542593in}{1.501156in}}%
\pgfpathcurveto{\pgfqpoint{10.542593in}{1.506200in}}{\pgfqpoint{10.540590in}{1.511038in}}{\pgfqpoint{10.537023in}{1.514604in}}%
\pgfpathcurveto{\pgfqpoint{10.533457in}{1.518170in}}{\pgfqpoint{10.528619in}{1.520174in}}{\pgfqpoint{10.523575in}{1.520174in}}%
\pgfpathcurveto{\pgfqpoint{10.518532in}{1.520174in}}{\pgfqpoint{10.513694in}{1.518170in}}{\pgfqpoint{10.510127in}{1.514604in}}%
\pgfpathcurveto{\pgfqpoint{10.506561in}{1.511038in}}{\pgfqpoint{10.504557in}{1.506200in}}{\pgfqpoint{10.504557in}{1.501156in}}%
\pgfpathcurveto{\pgfqpoint{10.504557in}{1.496112in}}{\pgfqpoint{10.506561in}{1.491275in}}{\pgfqpoint{10.510127in}{1.487708in}}%
\pgfpathcurveto{\pgfqpoint{10.513694in}{1.484142in}}{\pgfqpoint{10.518532in}{1.482138in}}{\pgfqpoint{10.523575in}{1.482138in}}%
\pgfpathclose%
\pgfusepath{fill}%
\end{pgfscope}%
\begin{pgfscope}%
\pgfpathrectangle{\pgfqpoint{6.572727in}{0.473000in}}{\pgfqpoint{4.227273in}{3.311000in}}%
\pgfusepath{clip}%
\pgfsetbuttcap%
\pgfsetroundjoin%
\definecolor{currentfill}{rgb}{0.127568,0.566949,0.550556}%
\pgfsetfillcolor{currentfill}%
\pgfsetfillopacity{0.700000}%
\pgfsetlinewidth{0.000000pt}%
\definecolor{currentstroke}{rgb}{0.000000,0.000000,0.000000}%
\pgfsetstrokecolor{currentstroke}%
\pgfsetstrokeopacity{0.700000}%
\pgfsetdash{}{0pt}%
\pgfpathmoveto{\pgfqpoint{7.985853in}{3.122955in}}%
\pgfpathcurveto{\pgfqpoint{7.990897in}{3.122955in}}{\pgfqpoint{7.995735in}{3.124959in}}{\pgfqpoint{7.999301in}{3.128525in}}%
\pgfpathcurveto{\pgfqpoint{8.002868in}{3.132092in}}{\pgfqpoint{8.004872in}{3.136930in}}{\pgfqpoint{8.004872in}{3.141973in}}%
\pgfpathcurveto{\pgfqpoint{8.004872in}{3.147017in}}{\pgfqpoint{8.002868in}{3.151855in}}{\pgfqpoint{7.999301in}{3.155421in}}%
\pgfpathcurveto{\pgfqpoint{7.995735in}{3.158987in}}{\pgfqpoint{7.990897in}{3.160991in}}{\pgfqpoint{7.985853in}{3.160991in}}%
\pgfpathcurveto{\pgfqpoint{7.980810in}{3.160991in}}{\pgfqpoint{7.975972in}{3.158987in}}{\pgfqpoint{7.972406in}{3.155421in}}%
\pgfpathcurveto{\pgfqpoint{7.968839in}{3.151855in}}{\pgfqpoint{7.966835in}{3.147017in}}{\pgfqpoint{7.966835in}{3.141973in}}%
\pgfpathcurveto{\pgfqpoint{7.966835in}{3.136930in}}{\pgfqpoint{7.968839in}{3.132092in}}{\pgfqpoint{7.972406in}{3.128525in}}%
\pgfpathcurveto{\pgfqpoint{7.975972in}{3.124959in}}{\pgfqpoint{7.980810in}{3.122955in}}{\pgfqpoint{7.985853in}{3.122955in}}%
\pgfpathclose%
\pgfusepath{fill}%
\end{pgfscope}%
\begin{pgfscope}%
\pgfpathrectangle{\pgfqpoint{6.572727in}{0.473000in}}{\pgfqpoint{4.227273in}{3.311000in}}%
\pgfusepath{clip}%
\pgfsetbuttcap%
\pgfsetroundjoin%
\definecolor{currentfill}{rgb}{0.993248,0.906157,0.143936}%
\pgfsetfillcolor{currentfill}%
\pgfsetfillopacity{0.700000}%
\pgfsetlinewidth{0.000000pt}%
\definecolor{currentstroke}{rgb}{0.000000,0.000000,0.000000}%
\pgfsetstrokecolor{currentstroke}%
\pgfsetstrokeopacity{0.700000}%
\pgfsetdash{}{0pt}%
\pgfpathmoveto{\pgfqpoint{9.121314in}{1.545481in}}%
\pgfpathcurveto{\pgfqpoint{9.126358in}{1.545481in}}{\pgfqpoint{9.131196in}{1.547485in}}{\pgfqpoint{9.134762in}{1.551052in}}%
\pgfpathcurveto{\pgfqpoint{9.138329in}{1.554618in}}{\pgfqpoint{9.140332in}{1.559456in}}{\pgfqpoint{9.140332in}{1.564500in}}%
\pgfpathcurveto{\pgfqpoint{9.140332in}{1.569543in}}{\pgfqpoint{9.138329in}{1.574381in}}{\pgfqpoint{9.134762in}{1.577947in}}%
\pgfpathcurveto{\pgfqpoint{9.131196in}{1.581514in}}{\pgfqpoint{9.126358in}{1.583518in}}{\pgfqpoint{9.121314in}{1.583518in}}%
\pgfpathcurveto{\pgfqpoint{9.116271in}{1.583518in}}{\pgfqpoint{9.111433in}{1.581514in}}{\pgfqpoint{9.107866in}{1.577947in}}%
\pgfpathcurveto{\pgfqpoint{9.104300in}{1.574381in}}{\pgfqpoint{9.102296in}{1.569543in}}{\pgfqpoint{9.102296in}{1.564500in}}%
\pgfpathcurveto{\pgfqpoint{9.102296in}{1.559456in}}{\pgfqpoint{9.104300in}{1.554618in}}{\pgfqpoint{9.107866in}{1.551052in}}%
\pgfpathcurveto{\pgfqpoint{9.111433in}{1.547485in}}{\pgfqpoint{9.116271in}{1.545481in}}{\pgfqpoint{9.121314in}{1.545481in}}%
\pgfpathclose%
\pgfusepath{fill}%
\end{pgfscope}%
\begin{pgfscope}%
\pgfpathrectangle{\pgfqpoint{6.572727in}{0.473000in}}{\pgfqpoint{4.227273in}{3.311000in}}%
\pgfusepath{clip}%
\pgfsetbuttcap%
\pgfsetroundjoin%
\definecolor{currentfill}{rgb}{0.993248,0.906157,0.143936}%
\pgfsetfillcolor{currentfill}%
\pgfsetfillopacity{0.700000}%
\pgfsetlinewidth{0.000000pt}%
\definecolor{currentstroke}{rgb}{0.000000,0.000000,0.000000}%
\pgfsetstrokecolor{currentstroke}%
\pgfsetstrokeopacity{0.700000}%
\pgfsetdash{}{0pt}%
\pgfpathmoveto{\pgfqpoint{9.732590in}{1.857460in}}%
\pgfpathcurveto{\pgfqpoint{9.737634in}{1.857460in}}{\pgfqpoint{9.742472in}{1.859464in}}{\pgfqpoint{9.746038in}{1.863030in}}%
\pgfpathcurveto{\pgfqpoint{9.749605in}{1.866596in}}{\pgfqpoint{9.751608in}{1.871434in}}{\pgfqpoint{9.751608in}{1.876478in}}%
\pgfpathcurveto{\pgfqpoint{9.751608in}{1.881522in}}{\pgfqpoint{9.749605in}{1.886359in}}{\pgfqpoint{9.746038in}{1.889926in}}%
\pgfpathcurveto{\pgfqpoint{9.742472in}{1.893492in}}{\pgfqpoint{9.737634in}{1.895496in}}{\pgfqpoint{9.732590in}{1.895496in}}%
\pgfpathcurveto{\pgfqpoint{9.727547in}{1.895496in}}{\pgfqpoint{9.722709in}{1.893492in}}{\pgfqpoint{9.719142in}{1.889926in}}%
\pgfpathcurveto{\pgfqpoint{9.715576in}{1.886359in}}{\pgfqpoint{9.713572in}{1.881522in}}{\pgfqpoint{9.713572in}{1.876478in}}%
\pgfpathcurveto{\pgfqpoint{9.713572in}{1.871434in}}{\pgfqpoint{9.715576in}{1.866596in}}{\pgfqpoint{9.719142in}{1.863030in}}%
\pgfpathcurveto{\pgfqpoint{9.722709in}{1.859464in}}{\pgfqpoint{9.727547in}{1.857460in}}{\pgfqpoint{9.732590in}{1.857460in}}%
\pgfpathclose%
\pgfusepath{fill}%
\end{pgfscope}%
\begin{pgfscope}%
\pgfpathrectangle{\pgfqpoint{6.572727in}{0.473000in}}{\pgfqpoint{4.227273in}{3.311000in}}%
\pgfusepath{clip}%
\pgfsetbuttcap%
\pgfsetroundjoin%
\definecolor{currentfill}{rgb}{0.127568,0.566949,0.550556}%
\pgfsetfillcolor{currentfill}%
\pgfsetfillopacity{0.700000}%
\pgfsetlinewidth{0.000000pt}%
\definecolor{currentstroke}{rgb}{0.000000,0.000000,0.000000}%
\pgfsetstrokecolor{currentstroke}%
\pgfsetstrokeopacity{0.700000}%
\pgfsetdash{}{0pt}%
\pgfpathmoveto{\pgfqpoint{7.764354in}{1.421914in}}%
\pgfpathcurveto{\pgfqpoint{7.769397in}{1.421914in}}{\pgfqpoint{7.774235in}{1.423918in}}{\pgfqpoint{7.777802in}{1.427485in}}%
\pgfpathcurveto{\pgfqpoint{7.781368in}{1.431051in}}{\pgfqpoint{7.783372in}{1.435889in}}{\pgfqpoint{7.783372in}{1.440932in}}%
\pgfpathcurveto{\pgfqpoint{7.783372in}{1.445976in}}{\pgfqpoint{7.781368in}{1.450814in}}{\pgfqpoint{7.777802in}{1.454380in}}%
\pgfpathcurveto{\pgfqpoint{7.774235in}{1.457947in}}{\pgfqpoint{7.769397in}{1.459951in}}{\pgfqpoint{7.764354in}{1.459951in}}%
\pgfpathcurveto{\pgfqpoint{7.759310in}{1.459951in}}{\pgfqpoint{7.754472in}{1.457947in}}{\pgfqpoint{7.750906in}{1.454380in}}%
\pgfpathcurveto{\pgfqpoint{7.747340in}{1.450814in}}{\pgfqpoint{7.745336in}{1.445976in}}{\pgfqpoint{7.745336in}{1.440932in}}%
\pgfpathcurveto{\pgfqpoint{7.745336in}{1.435889in}}{\pgfqpoint{7.747340in}{1.431051in}}{\pgfqpoint{7.750906in}{1.427485in}}%
\pgfpathcurveto{\pgfqpoint{7.754472in}{1.423918in}}{\pgfqpoint{7.759310in}{1.421914in}}{\pgfqpoint{7.764354in}{1.421914in}}%
\pgfpathclose%
\pgfusepath{fill}%
\end{pgfscope}%
\begin{pgfscope}%
\pgfpathrectangle{\pgfqpoint{6.572727in}{0.473000in}}{\pgfqpoint{4.227273in}{3.311000in}}%
\pgfusepath{clip}%
\pgfsetbuttcap%
\pgfsetroundjoin%
\definecolor{currentfill}{rgb}{0.993248,0.906157,0.143936}%
\pgfsetfillcolor{currentfill}%
\pgfsetfillopacity{0.700000}%
\pgfsetlinewidth{0.000000pt}%
\definecolor{currentstroke}{rgb}{0.000000,0.000000,0.000000}%
\pgfsetstrokecolor{currentstroke}%
\pgfsetstrokeopacity{0.700000}%
\pgfsetdash{}{0pt}%
\pgfpathmoveto{\pgfqpoint{10.185394in}{1.583727in}}%
\pgfpathcurveto{\pgfqpoint{10.190437in}{1.583727in}}{\pgfqpoint{10.195275in}{1.585731in}}{\pgfqpoint{10.198841in}{1.589297in}}%
\pgfpathcurveto{\pgfqpoint{10.202408in}{1.592864in}}{\pgfqpoint{10.204412in}{1.597701in}}{\pgfqpoint{10.204412in}{1.602745in}}%
\pgfpathcurveto{\pgfqpoint{10.204412in}{1.607789in}}{\pgfqpoint{10.202408in}{1.612627in}}{\pgfqpoint{10.198841in}{1.616193in}}%
\pgfpathcurveto{\pgfqpoint{10.195275in}{1.619759in}}{\pgfqpoint{10.190437in}{1.621763in}}{\pgfqpoint{10.185394in}{1.621763in}}%
\pgfpathcurveto{\pgfqpoint{10.180350in}{1.621763in}}{\pgfqpoint{10.175512in}{1.619759in}}{\pgfqpoint{10.171946in}{1.616193in}}%
\pgfpathcurveto{\pgfqpoint{10.168379in}{1.612627in}}{\pgfqpoint{10.166375in}{1.607789in}}{\pgfqpoint{10.166375in}{1.602745in}}%
\pgfpathcurveto{\pgfqpoint{10.166375in}{1.597701in}}{\pgfqpoint{10.168379in}{1.592864in}}{\pgfqpoint{10.171946in}{1.589297in}}%
\pgfpathcurveto{\pgfqpoint{10.175512in}{1.585731in}}{\pgfqpoint{10.180350in}{1.583727in}}{\pgfqpoint{10.185394in}{1.583727in}}%
\pgfpathclose%
\pgfusepath{fill}%
\end{pgfscope}%
\begin{pgfscope}%
\pgfpathrectangle{\pgfqpoint{6.572727in}{0.473000in}}{\pgfqpoint{4.227273in}{3.311000in}}%
\pgfusepath{clip}%
\pgfsetbuttcap%
\pgfsetroundjoin%
\definecolor{currentfill}{rgb}{0.993248,0.906157,0.143936}%
\pgfsetfillcolor{currentfill}%
\pgfsetfillopacity{0.700000}%
\pgfsetlinewidth{0.000000pt}%
\definecolor{currentstroke}{rgb}{0.000000,0.000000,0.000000}%
\pgfsetstrokecolor{currentstroke}%
\pgfsetstrokeopacity{0.700000}%
\pgfsetdash{}{0pt}%
\pgfpathmoveto{\pgfqpoint{9.875598in}{1.569478in}}%
\pgfpathcurveto{\pgfqpoint{9.880641in}{1.569478in}}{\pgfqpoint{9.885479in}{1.571482in}}{\pgfqpoint{9.889046in}{1.575048in}}%
\pgfpathcurveto{\pgfqpoint{9.892612in}{1.578615in}}{\pgfqpoint{9.894616in}{1.583453in}}{\pgfqpoint{9.894616in}{1.588496in}}%
\pgfpathcurveto{\pgfqpoint{9.894616in}{1.593540in}}{\pgfqpoint{9.892612in}{1.598378in}}{\pgfqpoint{9.889046in}{1.601944in}}%
\pgfpathcurveto{\pgfqpoint{9.885479in}{1.605511in}}{\pgfqpoint{9.880641in}{1.607514in}}{\pgfqpoint{9.875598in}{1.607514in}}%
\pgfpathcurveto{\pgfqpoint{9.870554in}{1.607514in}}{\pgfqpoint{9.865716in}{1.605511in}}{\pgfqpoint{9.862150in}{1.601944in}}%
\pgfpathcurveto{\pgfqpoint{9.858583in}{1.598378in}}{\pgfqpoint{9.856580in}{1.593540in}}{\pgfqpoint{9.856580in}{1.588496in}}%
\pgfpathcurveto{\pgfqpoint{9.856580in}{1.583453in}}{\pgfqpoint{9.858583in}{1.578615in}}{\pgfqpoint{9.862150in}{1.575048in}}%
\pgfpathcurveto{\pgfqpoint{9.865716in}{1.571482in}}{\pgfqpoint{9.870554in}{1.569478in}}{\pgfqpoint{9.875598in}{1.569478in}}%
\pgfpathclose%
\pgfusepath{fill}%
\end{pgfscope}%
\begin{pgfscope}%
\pgfpathrectangle{\pgfqpoint{6.572727in}{0.473000in}}{\pgfqpoint{4.227273in}{3.311000in}}%
\pgfusepath{clip}%
\pgfsetbuttcap%
\pgfsetroundjoin%
\definecolor{currentfill}{rgb}{0.993248,0.906157,0.143936}%
\pgfsetfillcolor{currentfill}%
\pgfsetfillopacity{0.700000}%
\pgfsetlinewidth{0.000000pt}%
\definecolor{currentstroke}{rgb}{0.000000,0.000000,0.000000}%
\pgfsetstrokecolor{currentstroke}%
\pgfsetstrokeopacity{0.700000}%
\pgfsetdash{}{0pt}%
\pgfpathmoveto{\pgfqpoint{9.118270in}{1.195579in}}%
\pgfpathcurveto{\pgfqpoint{9.123314in}{1.195579in}}{\pgfqpoint{9.128151in}{1.197583in}}{\pgfqpoint{9.131718in}{1.201149in}}%
\pgfpathcurveto{\pgfqpoint{9.135284in}{1.204716in}}{\pgfqpoint{9.137288in}{1.209553in}}{\pgfqpoint{9.137288in}{1.214597in}}%
\pgfpathcurveto{\pgfqpoint{9.137288in}{1.219641in}}{\pgfqpoint{9.135284in}{1.224479in}}{\pgfqpoint{9.131718in}{1.228045in}}%
\pgfpathcurveto{\pgfqpoint{9.128151in}{1.231611in}}{\pgfqpoint{9.123314in}{1.233615in}}{\pgfqpoint{9.118270in}{1.233615in}}%
\pgfpathcurveto{\pgfqpoint{9.113226in}{1.233615in}}{\pgfqpoint{9.108389in}{1.231611in}}{\pgfqpoint{9.104822in}{1.228045in}}%
\pgfpathcurveto{\pgfqpoint{9.101256in}{1.224479in}}{\pgfqpoint{9.099252in}{1.219641in}}{\pgfqpoint{9.099252in}{1.214597in}}%
\pgfpathcurveto{\pgfqpoint{9.099252in}{1.209553in}}{\pgfqpoint{9.101256in}{1.204716in}}{\pgfqpoint{9.104822in}{1.201149in}}%
\pgfpathcurveto{\pgfqpoint{9.108389in}{1.197583in}}{\pgfqpoint{9.113226in}{1.195579in}}{\pgfqpoint{9.118270in}{1.195579in}}%
\pgfpathclose%
\pgfusepath{fill}%
\end{pgfscope}%
\begin{pgfscope}%
\pgfpathrectangle{\pgfqpoint{6.572727in}{0.473000in}}{\pgfqpoint{4.227273in}{3.311000in}}%
\pgfusepath{clip}%
\pgfsetbuttcap%
\pgfsetroundjoin%
\definecolor{currentfill}{rgb}{0.127568,0.566949,0.550556}%
\pgfsetfillcolor{currentfill}%
\pgfsetfillopacity{0.700000}%
\pgfsetlinewidth{0.000000pt}%
\definecolor{currentstroke}{rgb}{0.000000,0.000000,0.000000}%
\pgfsetstrokecolor{currentstroke}%
\pgfsetstrokeopacity{0.700000}%
\pgfsetdash{}{0pt}%
\pgfpathmoveto{\pgfqpoint{8.034419in}{1.570135in}}%
\pgfpathcurveto{\pgfqpoint{8.039463in}{1.570135in}}{\pgfqpoint{8.044301in}{1.572139in}}{\pgfqpoint{8.047867in}{1.575705in}}%
\pgfpathcurveto{\pgfqpoint{8.051433in}{1.579272in}}{\pgfqpoint{8.053437in}{1.584109in}}{\pgfqpoint{8.053437in}{1.589153in}}%
\pgfpathcurveto{\pgfqpoint{8.053437in}{1.594197in}}{\pgfqpoint{8.051433in}{1.599034in}}{\pgfqpoint{8.047867in}{1.602601in}}%
\pgfpathcurveto{\pgfqpoint{8.044301in}{1.606167in}}{\pgfqpoint{8.039463in}{1.608171in}}{\pgfqpoint{8.034419in}{1.608171in}}%
\pgfpathcurveto{\pgfqpoint{8.029375in}{1.608171in}}{\pgfqpoint{8.024538in}{1.606167in}}{\pgfqpoint{8.020971in}{1.602601in}}%
\pgfpathcurveto{\pgfqpoint{8.017405in}{1.599034in}}{\pgfqpoint{8.015401in}{1.594197in}}{\pgfqpoint{8.015401in}{1.589153in}}%
\pgfpathcurveto{\pgfqpoint{8.015401in}{1.584109in}}{\pgfqpoint{8.017405in}{1.579272in}}{\pgfqpoint{8.020971in}{1.575705in}}%
\pgfpathcurveto{\pgfqpoint{8.024538in}{1.572139in}}{\pgfqpoint{8.029375in}{1.570135in}}{\pgfqpoint{8.034419in}{1.570135in}}%
\pgfpathclose%
\pgfusepath{fill}%
\end{pgfscope}%
\begin{pgfscope}%
\pgfpathrectangle{\pgfqpoint{6.572727in}{0.473000in}}{\pgfqpoint{4.227273in}{3.311000in}}%
\pgfusepath{clip}%
\pgfsetbuttcap%
\pgfsetroundjoin%
\definecolor{currentfill}{rgb}{0.127568,0.566949,0.550556}%
\pgfsetfillcolor{currentfill}%
\pgfsetfillopacity{0.700000}%
\pgfsetlinewidth{0.000000pt}%
\definecolor{currentstroke}{rgb}{0.000000,0.000000,0.000000}%
\pgfsetstrokecolor{currentstroke}%
\pgfsetstrokeopacity{0.700000}%
\pgfsetdash{}{0pt}%
\pgfpathmoveto{\pgfqpoint{7.957217in}{2.770450in}}%
\pgfpathcurveto{\pgfqpoint{7.962261in}{2.770450in}}{\pgfqpoint{7.967098in}{2.772454in}}{\pgfqpoint{7.970665in}{2.776021in}}%
\pgfpathcurveto{\pgfqpoint{7.974231in}{2.779587in}}{\pgfqpoint{7.976235in}{2.784425in}}{\pgfqpoint{7.976235in}{2.789469in}}%
\pgfpathcurveto{\pgfqpoint{7.976235in}{2.794512in}}{\pgfqpoint{7.974231in}{2.799350in}}{\pgfqpoint{7.970665in}{2.802916in}}%
\pgfpathcurveto{\pgfqpoint{7.967098in}{2.806483in}}{\pgfqpoint{7.962261in}{2.808487in}}{\pgfqpoint{7.957217in}{2.808487in}}%
\pgfpathcurveto{\pgfqpoint{7.952173in}{2.808487in}}{\pgfqpoint{7.947335in}{2.806483in}}{\pgfqpoint{7.943769in}{2.802916in}}%
\pgfpathcurveto{\pgfqpoint{7.940203in}{2.799350in}}{\pgfqpoint{7.938199in}{2.794512in}}{\pgfqpoint{7.938199in}{2.789469in}}%
\pgfpathcurveto{\pgfqpoint{7.938199in}{2.784425in}}{\pgfqpoint{7.940203in}{2.779587in}}{\pgfqpoint{7.943769in}{2.776021in}}%
\pgfpathcurveto{\pgfqpoint{7.947335in}{2.772454in}}{\pgfqpoint{7.952173in}{2.770450in}}{\pgfqpoint{7.957217in}{2.770450in}}%
\pgfpathclose%
\pgfusepath{fill}%
\end{pgfscope}%
\begin{pgfscope}%
\pgfpathrectangle{\pgfqpoint{6.572727in}{0.473000in}}{\pgfqpoint{4.227273in}{3.311000in}}%
\pgfusepath{clip}%
\pgfsetbuttcap%
\pgfsetroundjoin%
\definecolor{currentfill}{rgb}{0.127568,0.566949,0.550556}%
\pgfsetfillcolor{currentfill}%
\pgfsetfillopacity{0.700000}%
\pgfsetlinewidth{0.000000pt}%
\definecolor{currentstroke}{rgb}{0.000000,0.000000,0.000000}%
\pgfsetstrokecolor{currentstroke}%
\pgfsetstrokeopacity{0.700000}%
\pgfsetdash{}{0pt}%
\pgfpathmoveto{\pgfqpoint{7.447726in}{1.617285in}}%
\pgfpathcurveto{\pgfqpoint{7.452769in}{1.617285in}}{\pgfqpoint{7.457607in}{1.619289in}}{\pgfqpoint{7.461173in}{1.622855in}}%
\pgfpathcurveto{\pgfqpoint{7.464740in}{1.626422in}}{\pgfqpoint{7.466744in}{1.631259in}}{\pgfqpoint{7.466744in}{1.636303in}}%
\pgfpathcurveto{\pgfqpoint{7.466744in}{1.641347in}}{\pgfqpoint{7.464740in}{1.646185in}}{\pgfqpoint{7.461173in}{1.649751in}}%
\pgfpathcurveto{\pgfqpoint{7.457607in}{1.653317in}}{\pgfqpoint{7.452769in}{1.655321in}}{\pgfqpoint{7.447726in}{1.655321in}}%
\pgfpathcurveto{\pgfqpoint{7.442682in}{1.655321in}}{\pgfqpoint{7.437844in}{1.653317in}}{\pgfqpoint{7.434278in}{1.649751in}}%
\pgfpathcurveto{\pgfqpoint{7.430711in}{1.646185in}}{\pgfqpoint{7.428707in}{1.641347in}}{\pgfqpoint{7.428707in}{1.636303in}}%
\pgfpathcurveto{\pgfqpoint{7.428707in}{1.631259in}}{\pgfqpoint{7.430711in}{1.626422in}}{\pgfqpoint{7.434278in}{1.622855in}}%
\pgfpathcurveto{\pgfqpoint{7.437844in}{1.619289in}}{\pgfqpoint{7.442682in}{1.617285in}}{\pgfqpoint{7.447726in}{1.617285in}}%
\pgfpathclose%
\pgfusepath{fill}%
\end{pgfscope}%
\begin{pgfscope}%
\pgfpathrectangle{\pgfqpoint{6.572727in}{0.473000in}}{\pgfqpoint{4.227273in}{3.311000in}}%
\pgfusepath{clip}%
\pgfsetbuttcap%
\pgfsetroundjoin%
\definecolor{currentfill}{rgb}{0.127568,0.566949,0.550556}%
\pgfsetfillcolor{currentfill}%
\pgfsetfillopacity{0.700000}%
\pgfsetlinewidth{0.000000pt}%
\definecolor{currentstroke}{rgb}{0.000000,0.000000,0.000000}%
\pgfsetstrokecolor{currentstroke}%
\pgfsetstrokeopacity{0.700000}%
\pgfsetdash{}{0pt}%
\pgfpathmoveto{\pgfqpoint{8.375764in}{3.043770in}}%
\pgfpathcurveto{\pgfqpoint{8.380807in}{3.043770in}}{\pgfqpoint{8.385645in}{3.045774in}}{\pgfqpoint{8.389211in}{3.049340in}}%
\pgfpathcurveto{\pgfqpoint{8.392778in}{3.052907in}}{\pgfqpoint{8.394782in}{3.057744in}}{\pgfqpoint{8.394782in}{3.062788in}}%
\pgfpathcurveto{\pgfqpoint{8.394782in}{3.067832in}}{\pgfqpoint{8.392778in}{3.072669in}}{\pgfqpoint{8.389211in}{3.076236in}}%
\pgfpathcurveto{\pgfqpoint{8.385645in}{3.079802in}}{\pgfqpoint{8.380807in}{3.081806in}}{\pgfqpoint{8.375764in}{3.081806in}}%
\pgfpathcurveto{\pgfqpoint{8.370720in}{3.081806in}}{\pgfqpoint{8.365882in}{3.079802in}}{\pgfqpoint{8.362316in}{3.076236in}}%
\pgfpathcurveto{\pgfqpoint{8.358749in}{3.072669in}}{\pgfqpoint{8.356745in}{3.067832in}}{\pgfqpoint{8.356745in}{3.062788in}}%
\pgfpathcurveto{\pgfqpoint{8.356745in}{3.057744in}}{\pgfqpoint{8.358749in}{3.052907in}}{\pgfqpoint{8.362316in}{3.049340in}}%
\pgfpathcurveto{\pgfqpoint{8.365882in}{3.045774in}}{\pgfqpoint{8.370720in}{3.043770in}}{\pgfqpoint{8.375764in}{3.043770in}}%
\pgfpathclose%
\pgfusepath{fill}%
\end{pgfscope}%
\begin{pgfscope}%
\pgfpathrectangle{\pgfqpoint{6.572727in}{0.473000in}}{\pgfqpoint{4.227273in}{3.311000in}}%
\pgfusepath{clip}%
\pgfsetbuttcap%
\pgfsetroundjoin%
\definecolor{currentfill}{rgb}{0.127568,0.566949,0.550556}%
\pgfsetfillcolor{currentfill}%
\pgfsetfillopacity{0.700000}%
\pgfsetlinewidth{0.000000pt}%
\definecolor{currentstroke}{rgb}{0.000000,0.000000,0.000000}%
\pgfsetstrokecolor{currentstroke}%
\pgfsetstrokeopacity{0.700000}%
\pgfsetdash{}{0pt}%
\pgfpathmoveto{\pgfqpoint{7.971135in}{2.530892in}}%
\pgfpathcurveto{\pgfqpoint{7.976178in}{2.530892in}}{\pgfqpoint{7.981016in}{2.532896in}}{\pgfqpoint{7.984582in}{2.536462in}}%
\pgfpathcurveto{\pgfqpoint{7.988149in}{2.540029in}}{\pgfqpoint{7.990153in}{2.544867in}}{\pgfqpoint{7.990153in}{2.549910in}}%
\pgfpathcurveto{\pgfqpoint{7.990153in}{2.554954in}}{\pgfqpoint{7.988149in}{2.559792in}}{\pgfqpoint{7.984582in}{2.563358in}}%
\pgfpathcurveto{\pgfqpoint{7.981016in}{2.566925in}}{\pgfqpoint{7.976178in}{2.568928in}}{\pgfqpoint{7.971135in}{2.568928in}}%
\pgfpathcurveto{\pgfqpoint{7.966091in}{2.568928in}}{\pgfqpoint{7.961253in}{2.566925in}}{\pgfqpoint{7.957687in}{2.563358in}}%
\pgfpathcurveto{\pgfqpoint{7.954120in}{2.559792in}}{\pgfqpoint{7.952116in}{2.554954in}}{\pgfqpoint{7.952116in}{2.549910in}}%
\pgfpathcurveto{\pgfqpoint{7.952116in}{2.544867in}}{\pgfqpoint{7.954120in}{2.540029in}}{\pgfqpoint{7.957687in}{2.536462in}}%
\pgfpathcurveto{\pgfqpoint{7.961253in}{2.532896in}}{\pgfqpoint{7.966091in}{2.530892in}}{\pgfqpoint{7.971135in}{2.530892in}}%
\pgfpathclose%
\pgfusepath{fill}%
\end{pgfscope}%
\begin{pgfscope}%
\pgfpathrectangle{\pgfqpoint{6.572727in}{0.473000in}}{\pgfqpoint{4.227273in}{3.311000in}}%
\pgfusepath{clip}%
\pgfsetbuttcap%
\pgfsetroundjoin%
\definecolor{currentfill}{rgb}{0.127568,0.566949,0.550556}%
\pgfsetfillcolor{currentfill}%
\pgfsetfillopacity{0.700000}%
\pgfsetlinewidth{0.000000pt}%
\definecolor{currentstroke}{rgb}{0.000000,0.000000,0.000000}%
\pgfsetstrokecolor{currentstroke}%
\pgfsetstrokeopacity{0.700000}%
\pgfsetdash{}{0pt}%
\pgfpathmoveto{\pgfqpoint{7.306662in}{3.026054in}}%
\pgfpathcurveto{\pgfqpoint{7.311706in}{3.026054in}}{\pgfqpoint{7.316544in}{3.028057in}}{\pgfqpoint{7.320110in}{3.031624in}}%
\pgfpathcurveto{\pgfqpoint{7.323677in}{3.035190in}}{\pgfqpoint{7.325681in}{3.040028in}}{\pgfqpoint{7.325681in}{3.045072in}}%
\pgfpathcurveto{\pgfqpoint{7.325681in}{3.050115in}}{\pgfqpoint{7.323677in}{3.054953in}}{\pgfqpoint{7.320110in}{3.058520in}}%
\pgfpathcurveto{\pgfqpoint{7.316544in}{3.062086in}}{\pgfqpoint{7.311706in}{3.064090in}}{\pgfqpoint{7.306662in}{3.064090in}}%
\pgfpathcurveto{\pgfqpoint{7.301619in}{3.064090in}}{\pgfqpoint{7.296781in}{3.062086in}}{\pgfqpoint{7.293215in}{3.058520in}}%
\pgfpathcurveto{\pgfqpoint{7.289648in}{3.054953in}}{\pgfqpoint{7.287644in}{3.050115in}}{\pgfqpoint{7.287644in}{3.045072in}}%
\pgfpathcurveto{\pgfqpoint{7.287644in}{3.040028in}}{\pgfqpoint{7.289648in}{3.035190in}}{\pgfqpoint{7.293215in}{3.031624in}}%
\pgfpathcurveto{\pgfqpoint{7.296781in}{3.028057in}}{\pgfqpoint{7.301619in}{3.026054in}}{\pgfqpoint{7.306662in}{3.026054in}}%
\pgfpathclose%
\pgfusepath{fill}%
\end{pgfscope}%
\begin{pgfscope}%
\pgfpathrectangle{\pgfqpoint{6.572727in}{0.473000in}}{\pgfqpoint{4.227273in}{3.311000in}}%
\pgfusepath{clip}%
\pgfsetbuttcap%
\pgfsetroundjoin%
\definecolor{currentfill}{rgb}{0.127568,0.566949,0.550556}%
\pgfsetfillcolor{currentfill}%
\pgfsetfillopacity{0.700000}%
\pgfsetlinewidth{0.000000pt}%
\definecolor{currentstroke}{rgb}{0.000000,0.000000,0.000000}%
\pgfsetstrokecolor{currentstroke}%
\pgfsetstrokeopacity{0.700000}%
\pgfsetdash{}{0pt}%
\pgfpathmoveto{\pgfqpoint{7.989772in}{1.851380in}}%
\pgfpathcurveto{\pgfqpoint{7.994815in}{1.851380in}}{\pgfqpoint{7.999653in}{1.853384in}}{\pgfqpoint{8.003220in}{1.856951in}}%
\pgfpathcurveto{\pgfqpoint{8.006786in}{1.860517in}}{\pgfqpoint{8.008790in}{1.865355in}}{\pgfqpoint{8.008790in}{1.870398in}}%
\pgfpathcurveto{\pgfqpoint{8.008790in}{1.875442in}}{\pgfqpoint{8.006786in}{1.880280in}}{\pgfqpoint{8.003220in}{1.883846in}}%
\pgfpathcurveto{\pgfqpoint{7.999653in}{1.887413in}}{\pgfqpoint{7.994815in}{1.889417in}}{\pgfqpoint{7.989772in}{1.889417in}}%
\pgfpathcurveto{\pgfqpoint{7.984728in}{1.889417in}}{\pgfqpoint{7.979890in}{1.887413in}}{\pgfqpoint{7.976324in}{1.883846in}}%
\pgfpathcurveto{\pgfqpoint{7.972757in}{1.880280in}}{\pgfqpoint{7.970754in}{1.875442in}}{\pgfqpoint{7.970754in}{1.870398in}}%
\pgfpathcurveto{\pgfqpoint{7.970754in}{1.865355in}}{\pgfqpoint{7.972757in}{1.860517in}}{\pgfqpoint{7.976324in}{1.856951in}}%
\pgfpathcurveto{\pgfqpoint{7.979890in}{1.853384in}}{\pgfqpoint{7.984728in}{1.851380in}}{\pgfqpoint{7.989772in}{1.851380in}}%
\pgfpathclose%
\pgfusepath{fill}%
\end{pgfscope}%
\begin{pgfscope}%
\pgfpathrectangle{\pgfqpoint{6.572727in}{0.473000in}}{\pgfqpoint{4.227273in}{3.311000in}}%
\pgfusepath{clip}%
\pgfsetbuttcap%
\pgfsetroundjoin%
\definecolor{currentfill}{rgb}{0.127568,0.566949,0.550556}%
\pgfsetfillcolor{currentfill}%
\pgfsetfillopacity{0.700000}%
\pgfsetlinewidth{0.000000pt}%
\definecolor{currentstroke}{rgb}{0.000000,0.000000,0.000000}%
\pgfsetstrokecolor{currentstroke}%
\pgfsetstrokeopacity{0.700000}%
\pgfsetdash{}{0pt}%
\pgfpathmoveto{\pgfqpoint{8.164127in}{0.867854in}}%
\pgfpathcurveto{\pgfqpoint{8.169171in}{0.867854in}}{\pgfqpoint{8.174009in}{0.869858in}}{\pgfqpoint{8.177575in}{0.873425in}}%
\pgfpathcurveto{\pgfqpoint{8.181142in}{0.876991in}}{\pgfqpoint{8.183145in}{0.881829in}}{\pgfqpoint{8.183145in}{0.886873in}}%
\pgfpathcurveto{\pgfqpoint{8.183145in}{0.891916in}}{\pgfqpoint{8.181142in}{0.896754in}}{\pgfqpoint{8.177575in}{0.900320in}}%
\pgfpathcurveto{\pgfqpoint{8.174009in}{0.903887in}}{\pgfqpoint{8.169171in}{0.905891in}}{\pgfqpoint{8.164127in}{0.905891in}}%
\pgfpathcurveto{\pgfqpoint{8.159084in}{0.905891in}}{\pgfqpoint{8.154246in}{0.903887in}}{\pgfqpoint{8.150679in}{0.900320in}}%
\pgfpathcurveto{\pgfqpoint{8.147113in}{0.896754in}}{\pgfqpoint{8.145109in}{0.891916in}}{\pgfqpoint{8.145109in}{0.886873in}}%
\pgfpathcurveto{\pgfqpoint{8.145109in}{0.881829in}}{\pgfqpoint{8.147113in}{0.876991in}}{\pgfqpoint{8.150679in}{0.873425in}}%
\pgfpathcurveto{\pgfqpoint{8.154246in}{0.869858in}}{\pgfqpoint{8.159084in}{0.867854in}}{\pgfqpoint{8.164127in}{0.867854in}}%
\pgfpathclose%
\pgfusepath{fill}%
\end{pgfscope}%
\begin{pgfscope}%
\pgfpathrectangle{\pgfqpoint{6.572727in}{0.473000in}}{\pgfqpoint{4.227273in}{3.311000in}}%
\pgfusepath{clip}%
\pgfsetbuttcap%
\pgfsetroundjoin%
\definecolor{currentfill}{rgb}{0.127568,0.566949,0.550556}%
\pgfsetfillcolor{currentfill}%
\pgfsetfillopacity{0.700000}%
\pgfsetlinewidth{0.000000pt}%
\definecolor{currentstroke}{rgb}{0.000000,0.000000,0.000000}%
\pgfsetstrokecolor{currentstroke}%
\pgfsetstrokeopacity{0.700000}%
\pgfsetdash{}{0pt}%
\pgfpathmoveto{\pgfqpoint{8.302321in}{2.885059in}}%
\pgfpathcurveto{\pgfqpoint{8.307365in}{2.885059in}}{\pgfqpoint{8.312202in}{2.887063in}}{\pgfqpoint{8.315769in}{2.890630in}}%
\pgfpathcurveto{\pgfqpoint{8.319335in}{2.894196in}}{\pgfqpoint{8.321339in}{2.899034in}}{\pgfqpoint{8.321339in}{2.904077in}}%
\pgfpathcurveto{\pgfqpoint{8.321339in}{2.909121in}}{\pgfqpoint{8.319335in}{2.913959in}}{\pgfqpoint{8.315769in}{2.917525in}}%
\pgfpathcurveto{\pgfqpoint{8.312202in}{2.921092in}}{\pgfqpoint{8.307365in}{2.923096in}}{\pgfqpoint{8.302321in}{2.923096in}}%
\pgfpathcurveto{\pgfqpoint{8.297277in}{2.923096in}}{\pgfqpoint{8.292440in}{2.921092in}}{\pgfqpoint{8.288873in}{2.917525in}}%
\pgfpathcurveto{\pgfqpoint{8.285307in}{2.913959in}}{\pgfqpoint{8.283303in}{2.909121in}}{\pgfqpoint{8.283303in}{2.904077in}}%
\pgfpathcurveto{\pgfqpoint{8.283303in}{2.899034in}}{\pgfqpoint{8.285307in}{2.894196in}}{\pgfqpoint{8.288873in}{2.890630in}}%
\pgfpathcurveto{\pgfqpoint{8.292440in}{2.887063in}}{\pgfqpoint{8.297277in}{2.885059in}}{\pgfqpoint{8.302321in}{2.885059in}}%
\pgfpathclose%
\pgfusepath{fill}%
\end{pgfscope}%
\begin{pgfscope}%
\pgfpathrectangle{\pgfqpoint{6.572727in}{0.473000in}}{\pgfqpoint{4.227273in}{3.311000in}}%
\pgfusepath{clip}%
\pgfsetbuttcap%
\pgfsetroundjoin%
\definecolor{currentfill}{rgb}{0.127568,0.566949,0.550556}%
\pgfsetfillcolor{currentfill}%
\pgfsetfillopacity{0.700000}%
\pgfsetlinewidth{0.000000pt}%
\definecolor{currentstroke}{rgb}{0.000000,0.000000,0.000000}%
\pgfsetstrokecolor{currentstroke}%
\pgfsetstrokeopacity{0.700000}%
\pgfsetdash{}{0pt}%
\pgfpathmoveto{\pgfqpoint{7.920534in}{1.616468in}}%
\pgfpathcurveto{\pgfqpoint{7.925577in}{1.616468in}}{\pgfqpoint{7.930415in}{1.618472in}}{\pgfqpoint{7.933982in}{1.622038in}}%
\pgfpathcurveto{\pgfqpoint{7.937548in}{1.625605in}}{\pgfqpoint{7.939552in}{1.630442in}}{\pgfqpoint{7.939552in}{1.635486in}}%
\pgfpathcurveto{\pgfqpoint{7.939552in}{1.640530in}}{\pgfqpoint{7.937548in}{1.645368in}}{\pgfqpoint{7.933982in}{1.648934in}}%
\pgfpathcurveto{\pgfqpoint{7.930415in}{1.652500in}}{\pgfqpoint{7.925577in}{1.654504in}}{\pgfqpoint{7.920534in}{1.654504in}}%
\pgfpathcurveto{\pgfqpoint{7.915490in}{1.654504in}}{\pgfqpoint{7.910652in}{1.652500in}}{\pgfqpoint{7.907086in}{1.648934in}}%
\pgfpathcurveto{\pgfqpoint{7.903520in}{1.645368in}}{\pgfqpoint{7.901516in}{1.640530in}}{\pgfqpoint{7.901516in}{1.635486in}}%
\pgfpathcurveto{\pgfqpoint{7.901516in}{1.630442in}}{\pgfqpoint{7.903520in}{1.625605in}}{\pgfqpoint{7.907086in}{1.622038in}}%
\pgfpathcurveto{\pgfqpoint{7.910652in}{1.618472in}}{\pgfqpoint{7.915490in}{1.616468in}}{\pgfqpoint{7.920534in}{1.616468in}}%
\pgfpathclose%
\pgfusepath{fill}%
\end{pgfscope}%
\begin{pgfscope}%
\pgfpathrectangle{\pgfqpoint{6.572727in}{0.473000in}}{\pgfqpoint{4.227273in}{3.311000in}}%
\pgfusepath{clip}%
\pgfsetbuttcap%
\pgfsetroundjoin%
\definecolor{currentfill}{rgb}{0.127568,0.566949,0.550556}%
\pgfsetfillcolor{currentfill}%
\pgfsetfillopacity{0.700000}%
\pgfsetlinewidth{0.000000pt}%
\definecolor{currentstroke}{rgb}{0.000000,0.000000,0.000000}%
\pgfsetstrokecolor{currentstroke}%
\pgfsetstrokeopacity{0.700000}%
\pgfsetdash{}{0pt}%
\pgfpathmoveto{\pgfqpoint{7.973924in}{1.443626in}}%
\pgfpathcurveto{\pgfqpoint{7.978968in}{1.443626in}}{\pgfqpoint{7.983806in}{1.445630in}}{\pgfqpoint{7.987372in}{1.449197in}}%
\pgfpathcurveto{\pgfqpoint{7.990938in}{1.452763in}}{\pgfqpoint{7.992942in}{1.457601in}}{\pgfqpoint{7.992942in}{1.462645in}}%
\pgfpathcurveto{\pgfqpoint{7.992942in}{1.467688in}}{\pgfqpoint{7.990938in}{1.472526in}}{\pgfqpoint{7.987372in}{1.476092in}}%
\pgfpathcurveto{\pgfqpoint{7.983806in}{1.479659in}}{\pgfqpoint{7.978968in}{1.481663in}}{\pgfqpoint{7.973924in}{1.481663in}}%
\pgfpathcurveto{\pgfqpoint{7.968880in}{1.481663in}}{\pgfqpoint{7.964043in}{1.479659in}}{\pgfqpoint{7.960476in}{1.476092in}}%
\pgfpathcurveto{\pgfqpoint{7.956910in}{1.472526in}}{\pgfqpoint{7.954906in}{1.467688in}}{\pgfqpoint{7.954906in}{1.462645in}}%
\pgfpathcurveto{\pgfqpoint{7.954906in}{1.457601in}}{\pgfqpoint{7.956910in}{1.452763in}}{\pgfqpoint{7.960476in}{1.449197in}}%
\pgfpathcurveto{\pgfqpoint{7.964043in}{1.445630in}}{\pgfqpoint{7.968880in}{1.443626in}}{\pgfqpoint{7.973924in}{1.443626in}}%
\pgfpathclose%
\pgfusepath{fill}%
\end{pgfscope}%
\begin{pgfscope}%
\pgfpathrectangle{\pgfqpoint{6.572727in}{0.473000in}}{\pgfqpoint{4.227273in}{3.311000in}}%
\pgfusepath{clip}%
\pgfsetbuttcap%
\pgfsetroundjoin%
\definecolor{currentfill}{rgb}{0.993248,0.906157,0.143936}%
\pgfsetfillcolor{currentfill}%
\pgfsetfillopacity{0.700000}%
\pgfsetlinewidth{0.000000pt}%
\definecolor{currentstroke}{rgb}{0.000000,0.000000,0.000000}%
\pgfsetstrokecolor{currentstroke}%
\pgfsetstrokeopacity{0.700000}%
\pgfsetdash{}{0pt}%
\pgfpathmoveto{\pgfqpoint{9.695945in}{2.218140in}}%
\pgfpathcurveto{\pgfqpoint{9.700988in}{2.218140in}}{\pgfqpoint{9.705826in}{2.220143in}}{\pgfqpoint{9.709392in}{2.223710in}}%
\pgfpathcurveto{\pgfqpoint{9.712959in}{2.227276in}}{\pgfqpoint{9.714963in}{2.232114in}}{\pgfqpoint{9.714963in}{2.237158in}}%
\pgfpathcurveto{\pgfqpoint{9.714963in}{2.242201in}}{\pgfqpoint{9.712959in}{2.247039in}}{\pgfqpoint{9.709392in}{2.250606in}}%
\pgfpathcurveto{\pgfqpoint{9.705826in}{2.254172in}}{\pgfqpoint{9.700988in}{2.256176in}}{\pgfqpoint{9.695945in}{2.256176in}}%
\pgfpathcurveto{\pgfqpoint{9.690901in}{2.256176in}}{\pgfqpoint{9.686063in}{2.254172in}}{\pgfqpoint{9.682497in}{2.250606in}}%
\pgfpathcurveto{\pgfqpoint{9.678930in}{2.247039in}}{\pgfqpoint{9.676926in}{2.242201in}}{\pgfqpoint{9.676926in}{2.237158in}}%
\pgfpathcurveto{\pgfqpoint{9.676926in}{2.232114in}}{\pgfqpoint{9.678930in}{2.227276in}}{\pgfqpoint{9.682497in}{2.223710in}}%
\pgfpathcurveto{\pgfqpoint{9.686063in}{2.220143in}}{\pgfqpoint{9.690901in}{2.218140in}}{\pgfqpoint{9.695945in}{2.218140in}}%
\pgfpathclose%
\pgfusepath{fill}%
\end{pgfscope}%
\begin{pgfscope}%
\pgfpathrectangle{\pgfqpoint{6.572727in}{0.473000in}}{\pgfqpoint{4.227273in}{3.311000in}}%
\pgfusepath{clip}%
\pgfsetbuttcap%
\pgfsetroundjoin%
\definecolor{currentfill}{rgb}{0.127568,0.566949,0.550556}%
\pgfsetfillcolor{currentfill}%
\pgfsetfillopacity{0.700000}%
\pgfsetlinewidth{0.000000pt}%
\definecolor{currentstroke}{rgb}{0.000000,0.000000,0.000000}%
\pgfsetstrokecolor{currentstroke}%
\pgfsetstrokeopacity{0.700000}%
\pgfsetdash{}{0pt}%
\pgfpathmoveto{\pgfqpoint{8.414693in}{1.287792in}}%
\pgfpathcurveto{\pgfqpoint{8.419737in}{1.287792in}}{\pgfqpoint{8.424574in}{1.289795in}}{\pgfqpoint{8.428141in}{1.293362in}}%
\pgfpathcurveto{\pgfqpoint{8.431707in}{1.296928in}}{\pgfqpoint{8.433711in}{1.301766in}}{\pgfqpoint{8.433711in}{1.306810in}}%
\pgfpathcurveto{\pgfqpoint{8.433711in}{1.311853in}}{\pgfqpoint{8.431707in}{1.316691in}}{\pgfqpoint{8.428141in}{1.320258in}}%
\pgfpathcurveto{\pgfqpoint{8.424574in}{1.323824in}}{\pgfqpoint{8.419737in}{1.325828in}}{\pgfqpoint{8.414693in}{1.325828in}}%
\pgfpathcurveto{\pgfqpoint{8.409649in}{1.325828in}}{\pgfqpoint{8.404811in}{1.323824in}}{\pgfqpoint{8.401245in}{1.320258in}}%
\pgfpathcurveto{\pgfqpoint{8.397679in}{1.316691in}}{\pgfqpoint{8.395675in}{1.311853in}}{\pgfqpoint{8.395675in}{1.306810in}}%
\pgfpathcurveto{\pgfqpoint{8.395675in}{1.301766in}}{\pgfqpoint{8.397679in}{1.296928in}}{\pgfqpoint{8.401245in}{1.293362in}}%
\pgfpathcurveto{\pgfqpoint{8.404811in}{1.289795in}}{\pgfqpoint{8.409649in}{1.287792in}}{\pgfqpoint{8.414693in}{1.287792in}}%
\pgfpathclose%
\pgfusepath{fill}%
\end{pgfscope}%
\begin{pgfscope}%
\pgfpathrectangle{\pgfqpoint{6.572727in}{0.473000in}}{\pgfqpoint{4.227273in}{3.311000in}}%
\pgfusepath{clip}%
\pgfsetbuttcap%
\pgfsetroundjoin%
\definecolor{currentfill}{rgb}{0.127568,0.566949,0.550556}%
\pgfsetfillcolor{currentfill}%
\pgfsetfillopacity{0.700000}%
\pgfsetlinewidth{0.000000pt}%
\definecolor{currentstroke}{rgb}{0.000000,0.000000,0.000000}%
\pgfsetstrokecolor{currentstroke}%
\pgfsetstrokeopacity{0.700000}%
\pgfsetdash{}{0pt}%
\pgfpathmoveto{\pgfqpoint{8.411561in}{2.195232in}}%
\pgfpathcurveto{\pgfqpoint{8.416604in}{2.195232in}}{\pgfqpoint{8.421442in}{2.197236in}}{\pgfqpoint{8.425009in}{2.200802in}}%
\pgfpathcurveto{\pgfqpoint{8.428575in}{2.204369in}}{\pgfqpoint{8.430579in}{2.209206in}}{\pgfqpoint{8.430579in}{2.214250in}}%
\pgfpathcurveto{\pgfqpoint{8.430579in}{2.219294in}}{\pgfqpoint{8.428575in}{2.224131in}}{\pgfqpoint{8.425009in}{2.227698in}}%
\pgfpathcurveto{\pgfqpoint{8.421442in}{2.231264in}}{\pgfqpoint{8.416604in}{2.233268in}}{\pgfqpoint{8.411561in}{2.233268in}}%
\pgfpathcurveto{\pgfqpoint{8.406517in}{2.233268in}}{\pgfqpoint{8.401679in}{2.231264in}}{\pgfqpoint{8.398113in}{2.227698in}}%
\pgfpathcurveto{\pgfqpoint{8.394547in}{2.224131in}}{\pgfqpoint{8.392543in}{2.219294in}}{\pgfqpoint{8.392543in}{2.214250in}}%
\pgfpathcurveto{\pgfqpoint{8.392543in}{2.209206in}}{\pgfqpoint{8.394547in}{2.204369in}}{\pgfqpoint{8.398113in}{2.200802in}}%
\pgfpathcurveto{\pgfqpoint{8.401679in}{2.197236in}}{\pgfqpoint{8.406517in}{2.195232in}}{\pgfqpoint{8.411561in}{2.195232in}}%
\pgfpathclose%
\pgfusepath{fill}%
\end{pgfscope}%
\begin{pgfscope}%
\pgfpathrectangle{\pgfqpoint{6.572727in}{0.473000in}}{\pgfqpoint{4.227273in}{3.311000in}}%
\pgfusepath{clip}%
\pgfsetbuttcap%
\pgfsetroundjoin%
\definecolor{currentfill}{rgb}{0.127568,0.566949,0.550556}%
\pgfsetfillcolor{currentfill}%
\pgfsetfillopacity{0.700000}%
\pgfsetlinewidth{0.000000pt}%
\definecolor{currentstroke}{rgb}{0.000000,0.000000,0.000000}%
\pgfsetstrokecolor{currentstroke}%
\pgfsetstrokeopacity{0.700000}%
\pgfsetdash{}{0pt}%
\pgfpathmoveto{\pgfqpoint{7.618611in}{1.581554in}}%
\pgfpathcurveto{\pgfqpoint{7.623655in}{1.581554in}}{\pgfqpoint{7.628493in}{1.583558in}}{\pgfqpoint{7.632059in}{1.587124in}}%
\pgfpathcurveto{\pgfqpoint{7.635625in}{1.590691in}}{\pgfqpoint{7.637629in}{1.595528in}}{\pgfqpoint{7.637629in}{1.600572in}}%
\pgfpathcurveto{\pgfqpoint{7.637629in}{1.605616in}}{\pgfqpoint{7.635625in}{1.610453in}}{\pgfqpoint{7.632059in}{1.614020in}}%
\pgfpathcurveto{\pgfqpoint{7.628493in}{1.617586in}}{\pgfqpoint{7.623655in}{1.619590in}}{\pgfqpoint{7.618611in}{1.619590in}}%
\pgfpathcurveto{\pgfqpoint{7.613568in}{1.619590in}}{\pgfqpoint{7.608730in}{1.617586in}}{\pgfqpoint{7.605163in}{1.614020in}}%
\pgfpathcurveto{\pgfqpoint{7.601597in}{1.610453in}}{\pgfqpoint{7.599593in}{1.605616in}}{\pgfqpoint{7.599593in}{1.600572in}}%
\pgfpathcurveto{\pgfqpoint{7.599593in}{1.595528in}}{\pgfqpoint{7.601597in}{1.590691in}}{\pgfqpoint{7.605163in}{1.587124in}}%
\pgfpathcurveto{\pgfqpoint{7.608730in}{1.583558in}}{\pgfqpoint{7.613568in}{1.581554in}}{\pgfqpoint{7.618611in}{1.581554in}}%
\pgfpathclose%
\pgfusepath{fill}%
\end{pgfscope}%
\begin{pgfscope}%
\pgfpathrectangle{\pgfqpoint{6.572727in}{0.473000in}}{\pgfqpoint{4.227273in}{3.311000in}}%
\pgfusepath{clip}%
\pgfsetbuttcap%
\pgfsetroundjoin%
\definecolor{currentfill}{rgb}{0.127568,0.566949,0.550556}%
\pgfsetfillcolor{currentfill}%
\pgfsetfillopacity{0.700000}%
\pgfsetlinewidth{0.000000pt}%
\definecolor{currentstroke}{rgb}{0.000000,0.000000,0.000000}%
\pgfsetstrokecolor{currentstroke}%
\pgfsetstrokeopacity{0.700000}%
\pgfsetdash{}{0pt}%
\pgfpathmoveto{\pgfqpoint{8.589034in}{3.279884in}}%
\pgfpathcurveto{\pgfqpoint{8.594078in}{3.279884in}}{\pgfqpoint{8.598916in}{3.281888in}}{\pgfqpoint{8.602482in}{3.285454in}}%
\pgfpathcurveto{\pgfqpoint{8.606049in}{3.289021in}}{\pgfqpoint{8.608052in}{3.293859in}}{\pgfqpoint{8.608052in}{3.298902in}}%
\pgfpathcurveto{\pgfqpoint{8.608052in}{3.303946in}}{\pgfqpoint{8.606049in}{3.308784in}}{\pgfqpoint{8.602482in}{3.312350in}}%
\pgfpathcurveto{\pgfqpoint{8.598916in}{3.315916in}}{\pgfqpoint{8.594078in}{3.317920in}}{\pgfqpoint{8.589034in}{3.317920in}}%
\pgfpathcurveto{\pgfqpoint{8.583991in}{3.317920in}}{\pgfqpoint{8.579153in}{3.315916in}}{\pgfqpoint{8.575586in}{3.312350in}}%
\pgfpathcurveto{\pgfqpoint{8.572020in}{3.308784in}}{\pgfqpoint{8.570016in}{3.303946in}}{\pgfqpoint{8.570016in}{3.298902in}}%
\pgfpathcurveto{\pgfqpoint{8.570016in}{3.293859in}}{\pgfqpoint{8.572020in}{3.289021in}}{\pgfqpoint{8.575586in}{3.285454in}}%
\pgfpathcurveto{\pgfqpoint{8.579153in}{3.281888in}}{\pgfqpoint{8.583991in}{3.279884in}}{\pgfqpoint{8.589034in}{3.279884in}}%
\pgfpathclose%
\pgfusepath{fill}%
\end{pgfscope}%
\begin{pgfscope}%
\pgfpathrectangle{\pgfqpoint{6.572727in}{0.473000in}}{\pgfqpoint{4.227273in}{3.311000in}}%
\pgfusepath{clip}%
\pgfsetbuttcap%
\pgfsetroundjoin%
\definecolor{currentfill}{rgb}{0.127568,0.566949,0.550556}%
\pgfsetfillcolor{currentfill}%
\pgfsetfillopacity{0.700000}%
\pgfsetlinewidth{0.000000pt}%
\definecolor{currentstroke}{rgb}{0.000000,0.000000,0.000000}%
\pgfsetstrokecolor{currentstroke}%
\pgfsetstrokeopacity{0.700000}%
\pgfsetdash{}{0pt}%
\pgfpathmoveto{\pgfqpoint{7.887958in}{2.912706in}}%
\pgfpathcurveto{\pgfqpoint{7.893001in}{2.912706in}}{\pgfqpoint{7.897839in}{2.914710in}}{\pgfqpoint{7.901406in}{2.918276in}}%
\pgfpathcurveto{\pgfqpoint{7.904972in}{2.921842in}}{\pgfqpoint{7.906976in}{2.926680in}}{\pgfqpoint{7.906976in}{2.931724in}}%
\pgfpathcurveto{\pgfqpoint{7.906976in}{2.936768in}}{\pgfqpoint{7.904972in}{2.941605in}}{\pgfqpoint{7.901406in}{2.945172in}}%
\pgfpathcurveto{\pgfqpoint{7.897839in}{2.948738in}}{\pgfqpoint{7.893001in}{2.950742in}}{\pgfqpoint{7.887958in}{2.950742in}}%
\pgfpathcurveto{\pgfqpoint{7.882914in}{2.950742in}}{\pgfqpoint{7.878076in}{2.948738in}}{\pgfqpoint{7.874510in}{2.945172in}}%
\pgfpathcurveto{\pgfqpoint{7.870944in}{2.941605in}}{\pgfqpoint{7.868940in}{2.936768in}}{\pgfqpoint{7.868940in}{2.931724in}}%
\pgfpathcurveto{\pgfqpoint{7.868940in}{2.926680in}}{\pgfqpoint{7.870944in}{2.921842in}}{\pgfqpoint{7.874510in}{2.918276in}}%
\pgfpathcurveto{\pgfqpoint{7.878076in}{2.914710in}}{\pgfqpoint{7.882914in}{2.912706in}}{\pgfqpoint{7.887958in}{2.912706in}}%
\pgfpathclose%
\pgfusepath{fill}%
\end{pgfscope}%
\begin{pgfscope}%
\pgfpathrectangle{\pgfqpoint{6.572727in}{0.473000in}}{\pgfqpoint{4.227273in}{3.311000in}}%
\pgfusepath{clip}%
\pgfsetbuttcap%
\pgfsetroundjoin%
\definecolor{currentfill}{rgb}{0.127568,0.566949,0.550556}%
\pgfsetfillcolor{currentfill}%
\pgfsetfillopacity{0.700000}%
\pgfsetlinewidth{0.000000pt}%
\definecolor{currentstroke}{rgb}{0.000000,0.000000,0.000000}%
\pgfsetstrokecolor{currentstroke}%
\pgfsetstrokeopacity{0.700000}%
\pgfsetdash{}{0pt}%
\pgfpathmoveto{\pgfqpoint{8.083421in}{2.868050in}}%
\pgfpathcurveto{\pgfqpoint{8.088464in}{2.868050in}}{\pgfqpoint{8.093302in}{2.870053in}}{\pgfqpoint{8.096868in}{2.873620in}}%
\pgfpathcurveto{\pgfqpoint{8.100435in}{2.877186in}}{\pgfqpoint{8.102439in}{2.882024in}}{\pgfqpoint{8.102439in}{2.887068in}}%
\pgfpathcurveto{\pgfqpoint{8.102439in}{2.892111in}}{\pgfqpoint{8.100435in}{2.896949in}}{\pgfqpoint{8.096868in}{2.900516in}}%
\pgfpathcurveto{\pgfqpoint{8.093302in}{2.904082in}}{\pgfqpoint{8.088464in}{2.906086in}}{\pgfqpoint{8.083421in}{2.906086in}}%
\pgfpathcurveto{\pgfqpoint{8.078377in}{2.906086in}}{\pgfqpoint{8.073539in}{2.904082in}}{\pgfqpoint{8.069973in}{2.900516in}}%
\pgfpathcurveto{\pgfqpoint{8.066406in}{2.896949in}}{\pgfqpoint{8.064402in}{2.892111in}}{\pgfqpoint{8.064402in}{2.887068in}}%
\pgfpathcurveto{\pgfqpoint{8.064402in}{2.882024in}}{\pgfqpoint{8.066406in}{2.877186in}}{\pgfqpoint{8.069973in}{2.873620in}}%
\pgfpathcurveto{\pgfqpoint{8.073539in}{2.870053in}}{\pgfqpoint{8.078377in}{2.868050in}}{\pgfqpoint{8.083421in}{2.868050in}}%
\pgfpathclose%
\pgfusepath{fill}%
\end{pgfscope}%
\begin{pgfscope}%
\pgfpathrectangle{\pgfqpoint{6.572727in}{0.473000in}}{\pgfqpoint{4.227273in}{3.311000in}}%
\pgfusepath{clip}%
\pgfsetbuttcap%
\pgfsetroundjoin%
\definecolor{currentfill}{rgb}{0.993248,0.906157,0.143936}%
\pgfsetfillcolor{currentfill}%
\pgfsetfillopacity{0.700000}%
\pgfsetlinewidth{0.000000pt}%
\definecolor{currentstroke}{rgb}{0.000000,0.000000,0.000000}%
\pgfsetstrokecolor{currentstroke}%
\pgfsetstrokeopacity{0.700000}%
\pgfsetdash{}{0pt}%
\pgfpathmoveto{\pgfqpoint{9.490124in}{1.756530in}}%
\pgfpathcurveto{\pgfqpoint{9.495167in}{1.756530in}}{\pgfqpoint{9.500005in}{1.758534in}}{\pgfqpoint{9.503572in}{1.762100in}}%
\pgfpathcurveto{\pgfqpoint{9.507138in}{1.765667in}}{\pgfqpoint{9.509142in}{1.770504in}}{\pgfqpoint{9.509142in}{1.775548in}}%
\pgfpathcurveto{\pgfqpoint{9.509142in}{1.780592in}}{\pgfqpoint{9.507138in}{1.785429in}}{\pgfqpoint{9.503572in}{1.788996in}}%
\pgfpathcurveto{\pgfqpoint{9.500005in}{1.792562in}}{\pgfqpoint{9.495167in}{1.794566in}}{\pgfqpoint{9.490124in}{1.794566in}}%
\pgfpathcurveto{\pgfqpoint{9.485080in}{1.794566in}}{\pgfqpoint{9.480242in}{1.792562in}}{\pgfqpoint{9.476676in}{1.788996in}}%
\pgfpathcurveto{\pgfqpoint{9.473109in}{1.785429in}}{\pgfqpoint{9.471106in}{1.780592in}}{\pgfqpoint{9.471106in}{1.775548in}}%
\pgfpathcurveto{\pgfqpoint{9.471106in}{1.770504in}}{\pgfqpoint{9.473109in}{1.765667in}}{\pgfqpoint{9.476676in}{1.762100in}}%
\pgfpathcurveto{\pgfqpoint{9.480242in}{1.758534in}}{\pgfqpoint{9.485080in}{1.756530in}}{\pgfqpoint{9.490124in}{1.756530in}}%
\pgfpathclose%
\pgfusepath{fill}%
\end{pgfscope}%
\begin{pgfscope}%
\pgfpathrectangle{\pgfqpoint{6.572727in}{0.473000in}}{\pgfqpoint{4.227273in}{3.311000in}}%
\pgfusepath{clip}%
\pgfsetbuttcap%
\pgfsetroundjoin%
\definecolor{currentfill}{rgb}{0.127568,0.566949,0.550556}%
\pgfsetfillcolor{currentfill}%
\pgfsetfillopacity{0.700000}%
\pgfsetlinewidth{0.000000pt}%
\definecolor{currentstroke}{rgb}{0.000000,0.000000,0.000000}%
\pgfsetstrokecolor{currentstroke}%
\pgfsetstrokeopacity{0.700000}%
\pgfsetdash{}{0pt}%
\pgfpathmoveto{\pgfqpoint{7.472172in}{1.560479in}}%
\pgfpathcurveto{\pgfqpoint{7.477216in}{1.560479in}}{\pgfqpoint{7.482054in}{1.562483in}}{\pgfqpoint{7.485620in}{1.566050in}}%
\pgfpathcurveto{\pgfqpoint{7.489187in}{1.569616in}}{\pgfqpoint{7.491191in}{1.574454in}}{\pgfqpoint{7.491191in}{1.579498in}}%
\pgfpathcurveto{\pgfqpoint{7.491191in}{1.584541in}}{\pgfqpoint{7.489187in}{1.589379in}}{\pgfqpoint{7.485620in}{1.592945in}}%
\pgfpathcurveto{\pgfqpoint{7.482054in}{1.596512in}}{\pgfqpoint{7.477216in}{1.598516in}}{\pgfqpoint{7.472172in}{1.598516in}}%
\pgfpathcurveto{\pgfqpoint{7.467129in}{1.598516in}}{\pgfqpoint{7.462291in}{1.596512in}}{\pgfqpoint{7.458725in}{1.592945in}}%
\pgfpathcurveto{\pgfqpoint{7.455158in}{1.589379in}}{\pgfqpoint{7.453154in}{1.584541in}}{\pgfqpoint{7.453154in}{1.579498in}}%
\pgfpathcurveto{\pgfqpoint{7.453154in}{1.574454in}}{\pgfqpoint{7.455158in}{1.569616in}}{\pgfqpoint{7.458725in}{1.566050in}}%
\pgfpathcurveto{\pgfqpoint{7.462291in}{1.562483in}}{\pgfqpoint{7.467129in}{1.560479in}}{\pgfqpoint{7.472172in}{1.560479in}}%
\pgfpathclose%
\pgfusepath{fill}%
\end{pgfscope}%
\begin{pgfscope}%
\pgfpathrectangle{\pgfqpoint{6.572727in}{0.473000in}}{\pgfqpoint{4.227273in}{3.311000in}}%
\pgfusepath{clip}%
\pgfsetbuttcap%
\pgfsetroundjoin%
\definecolor{currentfill}{rgb}{0.993248,0.906157,0.143936}%
\pgfsetfillcolor{currentfill}%
\pgfsetfillopacity{0.700000}%
\pgfsetlinewidth{0.000000pt}%
\definecolor{currentstroke}{rgb}{0.000000,0.000000,0.000000}%
\pgfsetstrokecolor{currentstroke}%
\pgfsetstrokeopacity{0.700000}%
\pgfsetdash{}{0pt}%
\pgfpathmoveto{\pgfqpoint{8.958383in}{1.690412in}}%
\pgfpathcurveto{\pgfqpoint{8.963427in}{1.690412in}}{\pgfqpoint{8.968265in}{1.692416in}}{\pgfqpoint{8.971831in}{1.695983in}}%
\pgfpathcurveto{\pgfqpoint{8.975398in}{1.699549in}}{\pgfqpoint{8.977402in}{1.704387in}}{\pgfqpoint{8.977402in}{1.709430in}}%
\pgfpathcurveto{\pgfqpoint{8.977402in}{1.714474in}}{\pgfqpoint{8.975398in}{1.719312in}}{\pgfqpoint{8.971831in}{1.722878in}}%
\pgfpathcurveto{\pgfqpoint{8.968265in}{1.726445in}}{\pgfqpoint{8.963427in}{1.728449in}}{\pgfqpoint{8.958383in}{1.728449in}}%
\pgfpathcurveto{\pgfqpoint{8.953340in}{1.728449in}}{\pgfqpoint{8.948502in}{1.726445in}}{\pgfqpoint{8.944936in}{1.722878in}}%
\pgfpathcurveto{\pgfqpoint{8.941369in}{1.719312in}}{\pgfqpoint{8.939365in}{1.714474in}}{\pgfqpoint{8.939365in}{1.709430in}}%
\pgfpathcurveto{\pgfqpoint{8.939365in}{1.704387in}}{\pgfqpoint{8.941369in}{1.699549in}}{\pgfqpoint{8.944936in}{1.695983in}}%
\pgfpathcurveto{\pgfqpoint{8.948502in}{1.692416in}}{\pgfqpoint{8.953340in}{1.690412in}}{\pgfqpoint{8.958383in}{1.690412in}}%
\pgfpathclose%
\pgfusepath{fill}%
\end{pgfscope}%
\begin{pgfscope}%
\pgfpathrectangle{\pgfqpoint{6.572727in}{0.473000in}}{\pgfqpoint{4.227273in}{3.311000in}}%
\pgfusepath{clip}%
\pgfsetbuttcap%
\pgfsetroundjoin%
\definecolor{currentfill}{rgb}{0.993248,0.906157,0.143936}%
\pgfsetfillcolor{currentfill}%
\pgfsetfillopacity{0.700000}%
\pgfsetlinewidth{0.000000pt}%
\definecolor{currentstroke}{rgb}{0.000000,0.000000,0.000000}%
\pgfsetstrokecolor{currentstroke}%
\pgfsetstrokeopacity{0.700000}%
\pgfsetdash{}{0pt}%
\pgfpathmoveto{\pgfqpoint{9.053885in}{1.730773in}}%
\pgfpathcurveto{\pgfqpoint{9.058929in}{1.730773in}}{\pgfqpoint{9.063766in}{1.732777in}}{\pgfqpoint{9.067333in}{1.736344in}}%
\pgfpathcurveto{\pgfqpoint{9.070899in}{1.739910in}}{\pgfqpoint{9.072903in}{1.744748in}}{\pgfqpoint{9.072903in}{1.749791in}}%
\pgfpathcurveto{\pgfqpoint{9.072903in}{1.754835in}}{\pgfqpoint{9.070899in}{1.759673in}}{\pgfqpoint{9.067333in}{1.763239in}}%
\pgfpathcurveto{\pgfqpoint{9.063766in}{1.766806in}}{\pgfqpoint{9.058929in}{1.768810in}}{\pgfqpoint{9.053885in}{1.768810in}}%
\pgfpathcurveto{\pgfqpoint{9.048841in}{1.768810in}}{\pgfqpoint{9.044004in}{1.766806in}}{\pgfqpoint{9.040437in}{1.763239in}}%
\pgfpathcurveto{\pgfqpoint{9.036871in}{1.759673in}}{\pgfqpoint{9.034867in}{1.754835in}}{\pgfqpoint{9.034867in}{1.749791in}}%
\pgfpathcurveto{\pgfqpoint{9.034867in}{1.744748in}}{\pgfqpoint{9.036871in}{1.739910in}}{\pgfqpoint{9.040437in}{1.736344in}}%
\pgfpathcurveto{\pgfqpoint{9.044004in}{1.732777in}}{\pgfqpoint{9.048841in}{1.730773in}}{\pgfqpoint{9.053885in}{1.730773in}}%
\pgfpathclose%
\pgfusepath{fill}%
\end{pgfscope}%
\begin{pgfscope}%
\pgfpathrectangle{\pgfqpoint{6.572727in}{0.473000in}}{\pgfqpoint{4.227273in}{3.311000in}}%
\pgfusepath{clip}%
\pgfsetbuttcap%
\pgfsetroundjoin%
\definecolor{currentfill}{rgb}{0.993248,0.906157,0.143936}%
\pgfsetfillcolor{currentfill}%
\pgfsetfillopacity{0.700000}%
\pgfsetlinewidth{0.000000pt}%
\definecolor{currentstroke}{rgb}{0.000000,0.000000,0.000000}%
\pgfsetstrokecolor{currentstroke}%
\pgfsetstrokeopacity{0.700000}%
\pgfsetdash{}{0pt}%
\pgfpathmoveto{\pgfqpoint{9.016784in}{2.020695in}}%
\pgfpathcurveto{\pgfqpoint{9.021828in}{2.020695in}}{\pgfqpoint{9.026666in}{2.022699in}}{\pgfqpoint{9.030232in}{2.026265in}}%
\pgfpathcurveto{\pgfqpoint{9.033799in}{2.029832in}}{\pgfqpoint{9.035802in}{2.034669in}}{\pgfqpoint{9.035802in}{2.039713in}}%
\pgfpathcurveto{\pgfqpoint{9.035802in}{2.044757in}}{\pgfqpoint{9.033799in}{2.049595in}}{\pgfqpoint{9.030232in}{2.053161in}}%
\pgfpathcurveto{\pgfqpoint{9.026666in}{2.056727in}}{\pgfqpoint{9.021828in}{2.058731in}}{\pgfqpoint{9.016784in}{2.058731in}}%
\pgfpathcurveto{\pgfqpoint{9.011741in}{2.058731in}}{\pgfqpoint{9.006903in}{2.056727in}}{\pgfqpoint{9.003336in}{2.053161in}}%
\pgfpathcurveto{\pgfqpoint{8.999770in}{2.049595in}}{\pgfqpoint{8.997766in}{2.044757in}}{\pgfqpoint{8.997766in}{2.039713in}}%
\pgfpathcurveto{\pgfqpoint{8.997766in}{2.034669in}}{\pgfqpoint{8.999770in}{2.029832in}}{\pgfqpoint{9.003336in}{2.026265in}}%
\pgfpathcurveto{\pgfqpoint{9.006903in}{2.022699in}}{\pgfqpoint{9.011741in}{2.020695in}}{\pgfqpoint{9.016784in}{2.020695in}}%
\pgfpathclose%
\pgfusepath{fill}%
\end{pgfscope}%
\begin{pgfscope}%
\pgfpathrectangle{\pgfqpoint{6.572727in}{0.473000in}}{\pgfqpoint{4.227273in}{3.311000in}}%
\pgfusepath{clip}%
\pgfsetbuttcap%
\pgfsetroundjoin%
\definecolor{currentfill}{rgb}{0.127568,0.566949,0.550556}%
\pgfsetfillcolor{currentfill}%
\pgfsetfillopacity{0.700000}%
\pgfsetlinewidth{0.000000pt}%
\definecolor{currentstroke}{rgb}{0.000000,0.000000,0.000000}%
\pgfsetstrokecolor{currentstroke}%
\pgfsetstrokeopacity{0.700000}%
\pgfsetdash{}{0pt}%
\pgfpathmoveto{\pgfqpoint{7.885119in}{1.284653in}}%
\pgfpathcurveto{\pgfqpoint{7.890163in}{1.284653in}}{\pgfqpoint{7.895000in}{1.286657in}}{\pgfqpoint{7.898567in}{1.290223in}}%
\pgfpathcurveto{\pgfqpoint{7.902133in}{1.293789in}}{\pgfqpoint{7.904137in}{1.298627in}}{\pgfqpoint{7.904137in}{1.303671in}}%
\pgfpathcurveto{\pgfqpoint{7.904137in}{1.308715in}}{\pgfqpoint{7.902133in}{1.313552in}}{\pgfqpoint{7.898567in}{1.317119in}}%
\pgfpathcurveto{\pgfqpoint{7.895000in}{1.320685in}}{\pgfqpoint{7.890163in}{1.322689in}}{\pgfqpoint{7.885119in}{1.322689in}}%
\pgfpathcurveto{\pgfqpoint{7.880075in}{1.322689in}}{\pgfqpoint{7.875238in}{1.320685in}}{\pgfqpoint{7.871671in}{1.317119in}}%
\pgfpathcurveto{\pgfqpoint{7.868105in}{1.313552in}}{\pgfqpoint{7.866101in}{1.308715in}}{\pgfqpoint{7.866101in}{1.303671in}}%
\pgfpathcurveto{\pgfqpoint{7.866101in}{1.298627in}}{\pgfqpoint{7.868105in}{1.293789in}}{\pgfqpoint{7.871671in}{1.290223in}}%
\pgfpathcurveto{\pgfqpoint{7.875238in}{1.286657in}}{\pgfqpoint{7.880075in}{1.284653in}}{\pgfqpoint{7.885119in}{1.284653in}}%
\pgfpathclose%
\pgfusepath{fill}%
\end{pgfscope}%
\begin{pgfscope}%
\pgfpathrectangle{\pgfqpoint{6.572727in}{0.473000in}}{\pgfqpoint{4.227273in}{3.311000in}}%
\pgfusepath{clip}%
\pgfsetbuttcap%
\pgfsetroundjoin%
\definecolor{currentfill}{rgb}{0.127568,0.566949,0.550556}%
\pgfsetfillcolor{currentfill}%
\pgfsetfillopacity{0.700000}%
\pgfsetlinewidth{0.000000pt}%
\definecolor{currentstroke}{rgb}{0.000000,0.000000,0.000000}%
\pgfsetstrokecolor{currentstroke}%
\pgfsetstrokeopacity{0.700000}%
\pgfsetdash{}{0pt}%
\pgfpathmoveto{\pgfqpoint{7.830735in}{2.562929in}}%
\pgfpathcurveto{\pgfqpoint{7.835779in}{2.562929in}}{\pgfqpoint{7.840616in}{2.564933in}}{\pgfqpoint{7.844183in}{2.568500in}}%
\pgfpathcurveto{\pgfqpoint{7.847749in}{2.572066in}}{\pgfqpoint{7.849753in}{2.576904in}}{\pgfqpoint{7.849753in}{2.581948in}}%
\pgfpathcurveto{\pgfqpoint{7.849753in}{2.586991in}}{\pgfqpoint{7.847749in}{2.591829in}}{\pgfqpoint{7.844183in}{2.595395in}}%
\pgfpathcurveto{\pgfqpoint{7.840616in}{2.598962in}}{\pgfqpoint{7.835779in}{2.600966in}}{\pgfqpoint{7.830735in}{2.600966in}}%
\pgfpathcurveto{\pgfqpoint{7.825691in}{2.600966in}}{\pgfqpoint{7.820853in}{2.598962in}}{\pgfqpoint{7.817287in}{2.595395in}}%
\pgfpathcurveto{\pgfqpoint{7.813721in}{2.591829in}}{\pgfqpoint{7.811717in}{2.586991in}}{\pgfqpoint{7.811717in}{2.581948in}}%
\pgfpathcurveto{\pgfqpoint{7.811717in}{2.576904in}}{\pgfqpoint{7.813721in}{2.572066in}}{\pgfqpoint{7.817287in}{2.568500in}}%
\pgfpathcurveto{\pgfqpoint{7.820853in}{2.564933in}}{\pgfqpoint{7.825691in}{2.562929in}}{\pgfqpoint{7.830735in}{2.562929in}}%
\pgfpathclose%
\pgfusepath{fill}%
\end{pgfscope}%
\begin{pgfscope}%
\pgfpathrectangle{\pgfqpoint{6.572727in}{0.473000in}}{\pgfqpoint{4.227273in}{3.311000in}}%
\pgfusepath{clip}%
\pgfsetbuttcap%
\pgfsetroundjoin%
\definecolor{currentfill}{rgb}{0.127568,0.566949,0.550556}%
\pgfsetfillcolor{currentfill}%
\pgfsetfillopacity{0.700000}%
\pgfsetlinewidth{0.000000pt}%
\definecolor{currentstroke}{rgb}{0.000000,0.000000,0.000000}%
\pgfsetstrokecolor{currentstroke}%
\pgfsetstrokeopacity{0.700000}%
\pgfsetdash{}{0pt}%
\pgfpathmoveto{\pgfqpoint{7.642837in}{1.532749in}}%
\pgfpathcurveto{\pgfqpoint{7.647881in}{1.532749in}}{\pgfqpoint{7.652719in}{1.534753in}}{\pgfqpoint{7.656285in}{1.538319in}}%
\pgfpathcurveto{\pgfqpoint{7.659851in}{1.541886in}}{\pgfqpoint{7.661855in}{1.546723in}}{\pgfqpoint{7.661855in}{1.551767in}}%
\pgfpathcurveto{\pgfqpoint{7.661855in}{1.556811in}}{\pgfqpoint{7.659851in}{1.561648in}}{\pgfqpoint{7.656285in}{1.565215in}}%
\pgfpathcurveto{\pgfqpoint{7.652719in}{1.568781in}}{\pgfqpoint{7.647881in}{1.570785in}}{\pgfqpoint{7.642837in}{1.570785in}}%
\pgfpathcurveto{\pgfqpoint{7.637793in}{1.570785in}}{\pgfqpoint{7.632956in}{1.568781in}}{\pgfqpoint{7.629389in}{1.565215in}}%
\pgfpathcurveto{\pgfqpoint{7.625823in}{1.561648in}}{\pgfqpoint{7.623819in}{1.556811in}}{\pgfqpoint{7.623819in}{1.551767in}}%
\pgfpathcurveto{\pgfqpoint{7.623819in}{1.546723in}}{\pgfqpoint{7.625823in}{1.541886in}}{\pgfqpoint{7.629389in}{1.538319in}}%
\pgfpathcurveto{\pgfqpoint{7.632956in}{1.534753in}}{\pgfqpoint{7.637793in}{1.532749in}}{\pgfqpoint{7.642837in}{1.532749in}}%
\pgfpathclose%
\pgfusepath{fill}%
\end{pgfscope}%
\begin{pgfscope}%
\pgfpathrectangle{\pgfqpoint{6.572727in}{0.473000in}}{\pgfqpoint{4.227273in}{3.311000in}}%
\pgfusepath{clip}%
\pgfsetbuttcap%
\pgfsetroundjoin%
\definecolor{currentfill}{rgb}{0.127568,0.566949,0.550556}%
\pgfsetfillcolor{currentfill}%
\pgfsetfillopacity{0.700000}%
\pgfsetlinewidth{0.000000pt}%
\definecolor{currentstroke}{rgb}{0.000000,0.000000,0.000000}%
\pgfsetstrokecolor{currentstroke}%
\pgfsetstrokeopacity{0.700000}%
\pgfsetdash{}{0pt}%
\pgfpathmoveto{\pgfqpoint{8.592896in}{2.755522in}}%
\pgfpathcurveto{\pgfqpoint{8.597940in}{2.755522in}}{\pgfqpoint{8.602778in}{2.757526in}}{\pgfqpoint{8.606344in}{2.761093in}}%
\pgfpathcurveto{\pgfqpoint{8.609910in}{2.764659in}}{\pgfqpoint{8.611914in}{2.769497in}}{\pgfqpoint{8.611914in}{2.774540in}}%
\pgfpathcurveto{\pgfqpoint{8.611914in}{2.779584in}}{\pgfqpoint{8.609910in}{2.784422in}}{\pgfqpoint{8.606344in}{2.787988in}}%
\pgfpathcurveto{\pgfqpoint{8.602778in}{2.791555in}}{\pgfqpoint{8.597940in}{2.793559in}}{\pgfqpoint{8.592896in}{2.793559in}}%
\pgfpathcurveto{\pgfqpoint{8.587853in}{2.793559in}}{\pgfqpoint{8.583015in}{2.791555in}}{\pgfqpoint{8.579448in}{2.787988in}}%
\pgfpathcurveto{\pgfqpoint{8.575882in}{2.784422in}}{\pgfqpoint{8.573878in}{2.779584in}}{\pgfqpoint{8.573878in}{2.774540in}}%
\pgfpathcurveto{\pgfqpoint{8.573878in}{2.769497in}}{\pgfqpoint{8.575882in}{2.764659in}}{\pgfqpoint{8.579448in}{2.761093in}}%
\pgfpathcurveto{\pgfqpoint{8.583015in}{2.757526in}}{\pgfqpoint{8.587853in}{2.755522in}}{\pgfqpoint{8.592896in}{2.755522in}}%
\pgfpathclose%
\pgfusepath{fill}%
\end{pgfscope}%
\begin{pgfscope}%
\pgfpathrectangle{\pgfqpoint{6.572727in}{0.473000in}}{\pgfqpoint{4.227273in}{3.311000in}}%
\pgfusepath{clip}%
\pgfsetbuttcap%
\pgfsetroundjoin%
\definecolor{currentfill}{rgb}{0.127568,0.566949,0.550556}%
\pgfsetfillcolor{currentfill}%
\pgfsetfillopacity{0.700000}%
\pgfsetlinewidth{0.000000pt}%
\definecolor{currentstroke}{rgb}{0.000000,0.000000,0.000000}%
\pgfsetstrokecolor{currentstroke}%
\pgfsetstrokeopacity{0.700000}%
\pgfsetdash{}{0pt}%
\pgfpathmoveto{\pgfqpoint{8.318046in}{2.790452in}}%
\pgfpathcurveto{\pgfqpoint{8.323090in}{2.790452in}}{\pgfqpoint{8.327928in}{2.792456in}}{\pgfqpoint{8.331494in}{2.796022in}}%
\pgfpathcurveto{\pgfqpoint{8.335061in}{2.799589in}}{\pgfqpoint{8.337065in}{2.804427in}}{\pgfqpoint{8.337065in}{2.809470in}}%
\pgfpathcurveto{\pgfqpoint{8.337065in}{2.814514in}}{\pgfqpoint{8.335061in}{2.819352in}}{\pgfqpoint{8.331494in}{2.822918in}}%
\pgfpathcurveto{\pgfqpoint{8.327928in}{2.826485in}}{\pgfqpoint{8.323090in}{2.828488in}}{\pgfqpoint{8.318046in}{2.828488in}}%
\pgfpathcurveto{\pgfqpoint{8.313003in}{2.828488in}}{\pgfqpoint{8.308165in}{2.826485in}}{\pgfqpoint{8.304599in}{2.822918in}}%
\pgfpathcurveto{\pgfqpoint{8.301032in}{2.819352in}}{\pgfqpoint{8.299028in}{2.814514in}}{\pgfqpoint{8.299028in}{2.809470in}}%
\pgfpathcurveto{\pgfqpoint{8.299028in}{2.804427in}}{\pgfqpoint{8.301032in}{2.799589in}}{\pgfqpoint{8.304599in}{2.796022in}}%
\pgfpathcurveto{\pgfqpoint{8.308165in}{2.792456in}}{\pgfqpoint{8.313003in}{2.790452in}}{\pgfqpoint{8.318046in}{2.790452in}}%
\pgfpathclose%
\pgfusepath{fill}%
\end{pgfscope}%
\begin{pgfscope}%
\pgfpathrectangle{\pgfqpoint{6.572727in}{0.473000in}}{\pgfqpoint{4.227273in}{3.311000in}}%
\pgfusepath{clip}%
\pgfsetbuttcap%
\pgfsetroundjoin%
\definecolor{currentfill}{rgb}{0.127568,0.566949,0.550556}%
\pgfsetfillcolor{currentfill}%
\pgfsetfillopacity{0.700000}%
\pgfsetlinewidth{0.000000pt}%
\definecolor{currentstroke}{rgb}{0.000000,0.000000,0.000000}%
\pgfsetstrokecolor{currentstroke}%
\pgfsetstrokeopacity{0.700000}%
\pgfsetdash{}{0pt}%
\pgfpathmoveto{\pgfqpoint{8.800912in}{2.992435in}}%
\pgfpathcurveto{\pgfqpoint{8.805956in}{2.992435in}}{\pgfqpoint{8.810794in}{2.994439in}}{\pgfqpoint{8.814360in}{2.998006in}}%
\pgfpathcurveto{\pgfqpoint{8.817926in}{3.001572in}}{\pgfqpoint{8.819930in}{3.006410in}}{\pgfqpoint{8.819930in}{3.011453in}}%
\pgfpathcurveto{\pgfqpoint{8.819930in}{3.016497in}}{\pgfqpoint{8.817926in}{3.021335in}}{\pgfqpoint{8.814360in}{3.024901in}}%
\pgfpathcurveto{\pgfqpoint{8.810794in}{3.028468in}}{\pgfqpoint{8.805956in}{3.030472in}}{\pgfqpoint{8.800912in}{3.030472in}}%
\pgfpathcurveto{\pgfqpoint{8.795869in}{3.030472in}}{\pgfqpoint{8.791031in}{3.028468in}}{\pgfqpoint{8.787464in}{3.024901in}}%
\pgfpathcurveto{\pgfqpoint{8.783898in}{3.021335in}}{\pgfqpoint{8.781894in}{3.016497in}}{\pgfqpoint{8.781894in}{3.011453in}}%
\pgfpathcurveto{\pgfqpoint{8.781894in}{3.006410in}}{\pgfqpoint{8.783898in}{3.001572in}}{\pgfqpoint{8.787464in}{2.998006in}}%
\pgfpathcurveto{\pgfqpoint{8.791031in}{2.994439in}}{\pgfqpoint{8.795869in}{2.992435in}}{\pgfqpoint{8.800912in}{2.992435in}}%
\pgfpathclose%
\pgfusepath{fill}%
\end{pgfscope}%
\begin{pgfscope}%
\pgfpathrectangle{\pgfqpoint{6.572727in}{0.473000in}}{\pgfqpoint{4.227273in}{3.311000in}}%
\pgfusepath{clip}%
\pgfsetbuttcap%
\pgfsetroundjoin%
\definecolor{currentfill}{rgb}{0.127568,0.566949,0.550556}%
\pgfsetfillcolor{currentfill}%
\pgfsetfillopacity{0.700000}%
\pgfsetlinewidth{0.000000pt}%
\definecolor{currentstroke}{rgb}{0.000000,0.000000,0.000000}%
\pgfsetstrokecolor{currentstroke}%
\pgfsetstrokeopacity{0.700000}%
\pgfsetdash{}{0pt}%
\pgfpathmoveto{\pgfqpoint{8.195678in}{3.167947in}}%
\pgfpathcurveto{\pgfqpoint{8.200722in}{3.167947in}}{\pgfqpoint{8.205560in}{3.169951in}}{\pgfqpoint{8.209126in}{3.173517in}}%
\pgfpathcurveto{\pgfqpoint{8.212693in}{3.177084in}}{\pgfqpoint{8.214696in}{3.181922in}}{\pgfqpoint{8.214696in}{3.186965in}}%
\pgfpathcurveto{\pgfqpoint{8.214696in}{3.192009in}}{\pgfqpoint{8.212693in}{3.196847in}}{\pgfqpoint{8.209126in}{3.200413in}}%
\pgfpathcurveto{\pgfqpoint{8.205560in}{3.203980in}}{\pgfqpoint{8.200722in}{3.205983in}}{\pgfqpoint{8.195678in}{3.205983in}}%
\pgfpathcurveto{\pgfqpoint{8.190635in}{3.205983in}}{\pgfqpoint{8.185797in}{3.203980in}}{\pgfqpoint{8.182230in}{3.200413in}}%
\pgfpathcurveto{\pgfqpoint{8.178664in}{3.196847in}}{\pgfqpoint{8.176660in}{3.192009in}}{\pgfqpoint{8.176660in}{3.186965in}}%
\pgfpathcurveto{\pgfqpoint{8.176660in}{3.181922in}}{\pgfqpoint{8.178664in}{3.177084in}}{\pgfqpoint{8.182230in}{3.173517in}}%
\pgfpathcurveto{\pgfqpoint{8.185797in}{3.169951in}}{\pgfqpoint{8.190635in}{3.167947in}}{\pgfqpoint{8.195678in}{3.167947in}}%
\pgfpathclose%
\pgfusepath{fill}%
\end{pgfscope}%
\begin{pgfscope}%
\pgfpathrectangle{\pgfqpoint{6.572727in}{0.473000in}}{\pgfqpoint{4.227273in}{3.311000in}}%
\pgfusepath{clip}%
\pgfsetbuttcap%
\pgfsetroundjoin%
\definecolor{currentfill}{rgb}{0.127568,0.566949,0.550556}%
\pgfsetfillcolor{currentfill}%
\pgfsetfillopacity{0.700000}%
\pgfsetlinewidth{0.000000pt}%
\definecolor{currentstroke}{rgb}{0.000000,0.000000,0.000000}%
\pgfsetstrokecolor{currentstroke}%
\pgfsetstrokeopacity{0.700000}%
\pgfsetdash{}{0pt}%
\pgfpathmoveto{\pgfqpoint{8.146326in}{3.075848in}}%
\pgfpathcurveto{\pgfqpoint{8.151370in}{3.075848in}}{\pgfqpoint{8.156208in}{3.077852in}}{\pgfqpoint{8.159774in}{3.081418in}}%
\pgfpathcurveto{\pgfqpoint{8.163341in}{3.084985in}}{\pgfqpoint{8.165345in}{3.089823in}}{\pgfqpoint{8.165345in}{3.094866in}}%
\pgfpathcurveto{\pgfqpoint{8.165345in}{3.099910in}}{\pgfqpoint{8.163341in}{3.104748in}}{\pgfqpoint{8.159774in}{3.108314in}}%
\pgfpathcurveto{\pgfqpoint{8.156208in}{3.111881in}}{\pgfqpoint{8.151370in}{3.113884in}}{\pgfqpoint{8.146326in}{3.113884in}}%
\pgfpathcurveto{\pgfqpoint{8.141283in}{3.113884in}}{\pgfqpoint{8.136445in}{3.111881in}}{\pgfqpoint{8.132879in}{3.108314in}}%
\pgfpathcurveto{\pgfqpoint{8.129312in}{3.104748in}}{\pgfqpoint{8.127308in}{3.099910in}}{\pgfqpoint{8.127308in}{3.094866in}}%
\pgfpathcurveto{\pgfqpoint{8.127308in}{3.089823in}}{\pgfqpoint{8.129312in}{3.084985in}}{\pgfqpoint{8.132879in}{3.081418in}}%
\pgfpathcurveto{\pgfqpoint{8.136445in}{3.077852in}}{\pgfqpoint{8.141283in}{3.075848in}}{\pgfqpoint{8.146326in}{3.075848in}}%
\pgfpathclose%
\pgfusepath{fill}%
\end{pgfscope}%
\begin{pgfscope}%
\pgfpathrectangle{\pgfqpoint{6.572727in}{0.473000in}}{\pgfqpoint{4.227273in}{3.311000in}}%
\pgfusepath{clip}%
\pgfsetbuttcap%
\pgfsetroundjoin%
\definecolor{currentfill}{rgb}{0.127568,0.566949,0.550556}%
\pgfsetfillcolor{currentfill}%
\pgfsetfillopacity{0.700000}%
\pgfsetlinewidth{0.000000pt}%
\definecolor{currentstroke}{rgb}{0.000000,0.000000,0.000000}%
\pgfsetstrokecolor{currentstroke}%
\pgfsetstrokeopacity{0.700000}%
\pgfsetdash{}{0pt}%
\pgfpathmoveto{\pgfqpoint{7.797254in}{2.705355in}}%
\pgfpathcurveto{\pgfqpoint{7.802297in}{2.705355in}}{\pgfqpoint{7.807135in}{2.707358in}}{\pgfqpoint{7.810702in}{2.710925in}}%
\pgfpathcurveto{\pgfqpoint{7.814268in}{2.714491in}}{\pgfqpoint{7.816272in}{2.719329in}}{\pgfqpoint{7.816272in}{2.724373in}}%
\pgfpathcurveto{\pgfqpoint{7.816272in}{2.729416in}}{\pgfqpoint{7.814268in}{2.734254in}}{\pgfqpoint{7.810702in}{2.737821in}}%
\pgfpathcurveto{\pgfqpoint{7.807135in}{2.741387in}}{\pgfqpoint{7.802297in}{2.743391in}}{\pgfqpoint{7.797254in}{2.743391in}}%
\pgfpathcurveto{\pgfqpoint{7.792210in}{2.743391in}}{\pgfqpoint{7.787372in}{2.741387in}}{\pgfqpoint{7.783806in}{2.737821in}}%
\pgfpathcurveto{\pgfqpoint{7.780239in}{2.734254in}}{\pgfqpoint{7.778236in}{2.729416in}}{\pgfqpoint{7.778236in}{2.724373in}}%
\pgfpathcurveto{\pgfqpoint{7.778236in}{2.719329in}}{\pgfqpoint{7.780239in}{2.714491in}}{\pgfqpoint{7.783806in}{2.710925in}}%
\pgfpathcurveto{\pgfqpoint{7.787372in}{2.707358in}}{\pgfqpoint{7.792210in}{2.705355in}}{\pgfqpoint{7.797254in}{2.705355in}}%
\pgfpathclose%
\pgfusepath{fill}%
\end{pgfscope}%
\begin{pgfscope}%
\pgfpathrectangle{\pgfqpoint{6.572727in}{0.473000in}}{\pgfqpoint{4.227273in}{3.311000in}}%
\pgfusepath{clip}%
\pgfsetbuttcap%
\pgfsetroundjoin%
\definecolor{currentfill}{rgb}{0.127568,0.566949,0.550556}%
\pgfsetfillcolor{currentfill}%
\pgfsetfillopacity{0.700000}%
\pgfsetlinewidth{0.000000pt}%
\definecolor{currentstroke}{rgb}{0.000000,0.000000,0.000000}%
\pgfsetstrokecolor{currentstroke}%
\pgfsetstrokeopacity{0.700000}%
\pgfsetdash{}{0pt}%
\pgfpathmoveto{\pgfqpoint{7.873351in}{2.020472in}}%
\pgfpathcurveto{\pgfqpoint{7.878395in}{2.020472in}}{\pgfqpoint{7.883232in}{2.022476in}}{\pgfqpoint{7.886799in}{2.026042in}}%
\pgfpathcurveto{\pgfqpoint{7.890365in}{2.029609in}}{\pgfqpoint{7.892369in}{2.034447in}}{\pgfqpoint{7.892369in}{2.039490in}}%
\pgfpathcurveto{\pgfqpoint{7.892369in}{2.044534in}}{\pgfqpoint{7.890365in}{2.049372in}}{\pgfqpoint{7.886799in}{2.052938in}}%
\pgfpathcurveto{\pgfqpoint{7.883232in}{2.056505in}}{\pgfqpoint{7.878395in}{2.058508in}}{\pgfqpoint{7.873351in}{2.058508in}}%
\pgfpathcurveto{\pgfqpoint{7.868307in}{2.058508in}}{\pgfqpoint{7.863469in}{2.056505in}}{\pgfqpoint{7.859903in}{2.052938in}}%
\pgfpathcurveto{\pgfqpoint{7.856337in}{2.049372in}}{\pgfqpoint{7.854333in}{2.044534in}}{\pgfqpoint{7.854333in}{2.039490in}}%
\pgfpathcurveto{\pgfqpoint{7.854333in}{2.034447in}}{\pgfqpoint{7.856337in}{2.029609in}}{\pgfqpoint{7.859903in}{2.026042in}}%
\pgfpathcurveto{\pgfqpoint{7.863469in}{2.022476in}}{\pgfqpoint{7.868307in}{2.020472in}}{\pgfqpoint{7.873351in}{2.020472in}}%
\pgfpathclose%
\pgfusepath{fill}%
\end{pgfscope}%
\begin{pgfscope}%
\pgfpathrectangle{\pgfqpoint{6.572727in}{0.473000in}}{\pgfqpoint{4.227273in}{3.311000in}}%
\pgfusepath{clip}%
\pgfsetbuttcap%
\pgfsetroundjoin%
\definecolor{currentfill}{rgb}{0.127568,0.566949,0.550556}%
\pgfsetfillcolor{currentfill}%
\pgfsetfillopacity{0.700000}%
\pgfsetlinewidth{0.000000pt}%
\definecolor{currentstroke}{rgb}{0.000000,0.000000,0.000000}%
\pgfsetstrokecolor{currentstroke}%
\pgfsetstrokeopacity{0.700000}%
\pgfsetdash{}{0pt}%
\pgfpathmoveto{\pgfqpoint{8.143509in}{2.569938in}}%
\pgfpathcurveto{\pgfqpoint{8.148553in}{2.569938in}}{\pgfqpoint{8.153390in}{2.571942in}}{\pgfqpoint{8.156957in}{2.575509in}}%
\pgfpathcurveto{\pgfqpoint{8.160523in}{2.579075in}}{\pgfqpoint{8.162527in}{2.583913in}}{\pgfqpoint{8.162527in}{2.588956in}}%
\pgfpathcurveto{\pgfqpoint{8.162527in}{2.594000in}}{\pgfqpoint{8.160523in}{2.598838in}}{\pgfqpoint{8.156957in}{2.602404in}}%
\pgfpathcurveto{\pgfqpoint{8.153390in}{2.605971in}}{\pgfqpoint{8.148553in}{2.607975in}}{\pgfqpoint{8.143509in}{2.607975in}}%
\pgfpathcurveto{\pgfqpoint{8.138465in}{2.607975in}}{\pgfqpoint{8.133627in}{2.605971in}}{\pgfqpoint{8.130061in}{2.602404in}}%
\pgfpathcurveto{\pgfqpoint{8.126495in}{2.598838in}}{\pgfqpoint{8.124491in}{2.594000in}}{\pgfqpoint{8.124491in}{2.588956in}}%
\pgfpathcurveto{\pgfqpoint{8.124491in}{2.583913in}}{\pgfqpoint{8.126495in}{2.579075in}}{\pgfqpoint{8.130061in}{2.575509in}}%
\pgfpathcurveto{\pgfqpoint{8.133627in}{2.571942in}}{\pgfqpoint{8.138465in}{2.569938in}}{\pgfqpoint{8.143509in}{2.569938in}}%
\pgfpathclose%
\pgfusepath{fill}%
\end{pgfscope}%
\begin{pgfscope}%
\pgfpathrectangle{\pgfqpoint{6.572727in}{0.473000in}}{\pgfqpoint{4.227273in}{3.311000in}}%
\pgfusepath{clip}%
\pgfsetbuttcap%
\pgfsetroundjoin%
\definecolor{currentfill}{rgb}{0.993248,0.906157,0.143936}%
\pgfsetfillcolor{currentfill}%
\pgfsetfillopacity{0.700000}%
\pgfsetlinewidth{0.000000pt}%
\definecolor{currentstroke}{rgb}{0.000000,0.000000,0.000000}%
\pgfsetstrokecolor{currentstroke}%
\pgfsetstrokeopacity{0.700000}%
\pgfsetdash{}{0pt}%
\pgfpathmoveto{\pgfqpoint{10.118185in}{1.856474in}}%
\pgfpathcurveto{\pgfqpoint{10.123228in}{1.856474in}}{\pgfqpoint{10.128066in}{1.858478in}}{\pgfqpoint{10.131632in}{1.862044in}}%
\pgfpathcurveto{\pgfqpoint{10.135199in}{1.865611in}}{\pgfqpoint{10.137203in}{1.870448in}}{\pgfqpoint{10.137203in}{1.875492in}}%
\pgfpathcurveto{\pgfqpoint{10.137203in}{1.880536in}}{\pgfqpoint{10.135199in}{1.885373in}}{\pgfqpoint{10.131632in}{1.888940in}}%
\pgfpathcurveto{\pgfqpoint{10.128066in}{1.892506in}}{\pgfqpoint{10.123228in}{1.894510in}}{\pgfqpoint{10.118185in}{1.894510in}}%
\pgfpathcurveto{\pgfqpoint{10.113141in}{1.894510in}}{\pgfqpoint{10.108303in}{1.892506in}}{\pgfqpoint{10.104737in}{1.888940in}}%
\pgfpathcurveto{\pgfqpoint{10.101170in}{1.885373in}}{\pgfqpoint{10.099166in}{1.880536in}}{\pgfqpoint{10.099166in}{1.875492in}}%
\pgfpathcurveto{\pgfqpoint{10.099166in}{1.870448in}}{\pgfqpoint{10.101170in}{1.865611in}}{\pgfqpoint{10.104737in}{1.862044in}}%
\pgfpathcurveto{\pgfqpoint{10.108303in}{1.858478in}}{\pgfqpoint{10.113141in}{1.856474in}}{\pgfqpoint{10.118185in}{1.856474in}}%
\pgfpathclose%
\pgfusepath{fill}%
\end{pgfscope}%
\begin{pgfscope}%
\pgfpathrectangle{\pgfqpoint{6.572727in}{0.473000in}}{\pgfqpoint{4.227273in}{3.311000in}}%
\pgfusepath{clip}%
\pgfsetbuttcap%
\pgfsetroundjoin%
\definecolor{currentfill}{rgb}{0.127568,0.566949,0.550556}%
\pgfsetfillcolor{currentfill}%
\pgfsetfillopacity{0.700000}%
\pgfsetlinewidth{0.000000pt}%
\definecolor{currentstroke}{rgb}{0.000000,0.000000,0.000000}%
\pgfsetstrokecolor{currentstroke}%
\pgfsetstrokeopacity{0.700000}%
\pgfsetdash{}{0pt}%
\pgfpathmoveto{\pgfqpoint{8.077415in}{2.813230in}}%
\pgfpathcurveto{\pgfqpoint{8.082459in}{2.813230in}}{\pgfqpoint{8.087297in}{2.815233in}}{\pgfqpoint{8.090863in}{2.818800in}}%
\pgfpathcurveto{\pgfqpoint{8.094429in}{2.822366in}}{\pgfqpoint{8.096433in}{2.827204in}}{\pgfqpoint{8.096433in}{2.832248in}}%
\pgfpathcurveto{\pgfqpoint{8.096433in}{2.837291in}}{\pgfqpoint{8.094429in}{2.842129in}}{\pgfqpoint{8.090863in}{2.845696in}}%
\pgfpathcurveto{\pgfqpoint{8.087297in}{2.849262in}}{\pgfqpoint{8.082459in}{2.851266in}}{\pgfqpoint{8.077415in}{2.851266in}}%
\pgfpathcurveto{\pgfqpoint{8.072371in}{2.851266in}}{\pgfqpoint{8.067534in}{2.849262in}}{\pgfqpoint{8.063967in}{2.845696in}}%
\pgfpathcurveto{\pgfqpoint{8.060401in}{2.842129in}}{\pgfqpoint{8.058397in}{2.837291in}}{\pgfqpoint{8.058397in}{2.832248in}}%
\pgfpathcurveto{\pgfqpoint{8.058397in}{2.827204in}}{\pgfqpoint{8.060401in}{2.822366in}}{\pgfqpoint{8.063967in}{2.818800in}}%
\pgfpathcurveto{\pgfqpoint{8.067534in}{2.815233in}}{\pgfqpoint{8.072371in}{2.813230in}}{\pgfqpoint{8.077415in}{2.813230in}}%
\pgfpathclose%
\pgfusepath{fill}%
\end{pgfscope}%
\begin{pgfscope}%
\pgfpathrectangle{\pgfqpoint{6.572727in}{0.473000in}}{\pgfqpoint{4.227273in}{3.311000in}}%
\pgfusepath{clip}%
\pgfsetbuttcap%
\pgfsetroundjoin%
\definecolor{currentfill}{rgb}{0.127568,0.566949,0.550556}%
\pgfsetfillcolor{currentfill}%
\pgfsetfillopacity{0.700000}%
\pgfsetlinewidth{0.000000pt}%
\definecolor{currentstroke}{rgb}{0.000000,0.000000,0.000000}%
\pgfsetstrokecolor{currentstroke}%
\pgfsetstrokeopacity{0.700000}%
\pgfsetdash{}{0pt}%
\pgfpathmoveto{\pgfqpoint{8.196651in}{3.186460in}}%
\pgfpathcurveto{\pgfqpoint{8.201695in}{3.186460in}}{\pgfqpoint{8.206533in}{3.188464in}}{\pgfqpoint{8.210099in}{3.192030in}}%
\pgfpathcurveto{\pgfqpoint{8.213665in}{3.195597in}}{\pgfqpoint{8.215669in}{3.200435in}}{\pgfqpoint{8.215669in}{3.205478in}}%
\pgfpathcurveto{\pgfqpoint{8.215669in}{3.210522in}}{\pgfqpoint{8.213665in}{3.215360in}}{\pgfqpoint{8.210099in}{3.218926in}}%
\pgfpathcurveto{\pgfqpoint{8.206533in}{3.222493in}}{\pgfqpoint{8.201695in}{3.224496in}}{\pgfqpoint{8.196651in}{3.224496in}}%
\pgfpathcurveto{\pgfqpoint{8.191607in}{3.224496in}}{\pgfqpoint{8.186770in}{3.222493in}}{\pgfqpoint{8.183203in}{3.218926in}}%
\pgfpathcurveto{\pgfqpoint{8.179637in}{3.215360in}}{\pgfqpoint{8.177633in}{3.210522in}}{\pgfqpoint{8.177633in}{3.205478in}}%
\pgfpathcurveto{\pgfqpoint{8.177633in}{3.200435in}}{\pgfqpoint{8.179637in}{3.195597in}}{\pgfqpoint{8.183203in}{3.192030in}}%
\pgfpathcurveto{\pgfqpoint{8.186770in}{3.188464in}}{\pgfqpoint{8.191607in}{3.186460in}}{\pgfqpoint{8.196651in}{3.186460in}}%
\pgfpathclose%
\pgfusepath{fill}%
\end{pgfscope}%
\begin{pgfscope}%
\pgfpathrectangle{\pgfqpoint{6.572727in}{0.473000in}}{\pgfqpoint{4.227273in}{3.311000in}}%
\pgfusepath{clip}%
\pgfsetbuttcap%
\pgfsetroundjoin%
\definecolor{currentfill}{rgb}{0.127568,0.566949,0.550556}%
\pgfsetfillcolor{currentfill}%
\pgfsetfillopacity{0.700000}%
\pgfsetlinewidth{0.000000pt}%
\definecolor{currentstroke}{rgb}{0.000000,0.000000,0.000000}%
\pgfsetstrokecolor{currentstroke}%
\pgfsetstrokeopacity{0.700000}%
\pgfsetdash{}{0pt}%
\pgfpathmoveto{\pgfqpoint{8.566652in}{2.768872in}}%
\pgfpathcurveto{\pgfqpoint{8.571695in}{2.768872in}}{\pgfqpoint{8.576533in}{2.770876in}}{\pgfqpoint{8.580100in}{2.774442in}}%
\pgfpathcurveto{\pgfqpoint{8.583666in}{2.778009in}}{\pgfqpoint{8.585670in}{2.782846in}}{\pgfqpoint{8.585670in}{2.787890in}}%
\pgfpathcurveto{\pgfqpoint{8.585670in}{2.792934in}}{\pgfqpoint{8.583666in}{2.797772in}}{\pgfqpoint{8.580100in}{2.801338in}}%
\pgfpathcurveto{\pgfqpoint{8.576533in}{2.804904in}}{\pgfqpoint{8.571695in}{2.806908in}}{\pgfqpoint{8.566652in}{2.806908in}}%
\pgfpathcurveto{\pgfqpoint{8.561608in}{2.806908in}}{\pgfqpoint{8.556770in}{2.804904in}}{\pgfqpoint{8.553204in}{2.801338in}}%
\pgfpathcurveto{\pgfqpoint{8.549638in}{2.797772in}}{\pgfqpoint{8.547634in}{2.792934in}}{\pgfqpoint{8.547634in}{2.787890in}}%
\pgfpathcurveto{\pgfqpoint{8.547634in}{2.782846in}}{\pgfqpoint{8.549638in}{2.778009in}}{\pgfqpoint{8.553204in}{2.774442in}}%
\pgfpathcurveto{\pgfqpoint{8.556770in}{2.770876in}}{\pgfqpoint{8.561608in}{2.768872in}}{\pgfqpoint{8.566652in}{2.768872in}}%
\pgfpathclose%
\pgfusepath{fill}%
\end{pgfscope}%
\begin{pgfscope}%
\pgfpathrectangle{\pgfqpoint{6.572727in}{0.473000in}}{\pgfqpoint{4.227273in}{3.311000in}}%
\pgfusepath{clip}%
\pgfsetbuttcap%
\pgfsetroundjoin%
\definecolor{currentfill}{rgb}{0.993248,0.906157,0.143936}%
\pgfsetfillcolor{currentfill}%
\pgfsetfillopacity{0.700000}%
\pgfsetlinewidth{0.000000pt}%
\definecolor{currentstroke}{rgb}{0.000000,0.000000,0.000000}%
\pgfsetstrokecolor{currentstroke}%
\pgfsetstrokeopacity{0.700000}%
\pgfsetdash{}{0pt}%
\pgfpathmoveto{\pgfqpoint{9.713458in}{1.756205in}}%
\pgfpathcurveto{\pgfqpoint{9.718502in}{1.756205in}}{\pgfqpoint{9.723339in}{1.758209in}}{\pgfqpoint{9.726906in}{1.761776in}}%
\pgfpathcurveto{\pgfqpoint{9.730472in}{1.765342in}}{\pgfqpoint{9.732476in}{1.770180in}}{\pgfqpoint{9.732476in}{1.775224in}}%
\pgfpathcurveto{\pgfqpoint{9.732476in}{1.780267in}}{\pgfqpoint{9.730472in}{1.785105in}}{\pgfqpoint{9.726906in}{1.788671in}}%
\pgfpathcurveto{\pgfqpoint{9.723339in}{1.792238in}}{\pgfqpoint{9.718502in}{1.794242in}}{\pgfqpoint{9.713458in}{1.794242in}}%
\pgfpathcurveto{\pgfqpoint{9.708414in}{1.794242in}}{\pgfqpoint{9.703577in}{1.792238in}}{\pgfqpoint{9.700010in}{1.788671in}}%
\pgfpathcurveto{\pgfqpoint{9.696444in}{1.785105in}}{\pgfqpoint{9.694440in}{1.780267in}}{\pgfqpoint{9.694440in}{1.775224in}}%
\pgfpathcurveto{\pgfqpoint{9.694440in}{1.770180in}}{\pgfqpoint{9.696444in}{1.765342in}}{\pgfqpoint{9.700010in}{1.761776in}}%
\pgfpathcurveto{\pgfqpoint{9.703577in}{1.758209in}}{\pgfqpoint{9.708414in}{1.756205in}}{\pgfqpoint{9.713458in}{1.756205in}}%
\pgfpathclose%
\pgfusepath{fill}%
\end{pgfscope}%
\begin{pgfscope}%
\pgfpathrectangle{\pgfqpoint{6.572727in}{0.473000in}}{\pgfqpoint{4.227273in}{3.311000in}}%
\pgfusepath{clip}%
\pgfsetbuttcap%
\pgfsetroundjoin%
\definecolor{currentfill}{rgb}{0.993248,0.906157,0.143936}%
\pgfsetfillcolor{currentfill}%
\pgfsetfillopacity{0.700000}%
\pgfsetlinewidth{0.000000pt}%
\definecolor{currentstroke}{rgb}{0.000000,0.000000,0.000000}%
\pgfsetstrokecolor{currentstroke}%
\pgfsetstrokeopacity{0.700000}%
\pgfsetdash{}{0pt}%
\pgfpathmoveto{\pgfqpoint{9.549198in}{0.979354in}}%
\pgfpathcurveto{\pgfqpoint{9.554242in}{0.979354in}}{\pgfqpoint{9.559080in}{0.981358in}}{\pgfqpoint{9.562646in}{0.984925in}}%
\pgfpathcurveto{\pgfqpoint{9.566212in}{0.988491in}}{\pgfqpoint{9.568216in}{0.993329in}}{\pgfqpoint{9.568216in}{0.998372in}}%
\pgfpathcurveto{\pgfqpoint{9.568216in}{1.003416in}}{\pgfqpoint{9.566212in}{1.008254in}}{\pgfqpoint{9.562646in}{1.011820in}}%
\pgfpathcurveto{\pgfqpoint{9.559080in}{1.015387in}}{\pgfqpoint{9.554242in}{1.017391in}}{\pgfqpoint{9.549198in}{1.017391in}}%
\pgfpathcurveto{\pgfqpoint{9.544154in}{1.017391in}}{\pgfqpoint{9.539317in}{1.015387in}}{\pgfqpoint{9.535750in}{1.011820in}}%
\pgfpathcurveto{\pgfqpoint{9.532184in}{1.008254in}}{\pgfqpoint{9.530180in}{1.003416in}}{\pgfqpoint{9.530180in}{0.998372in}}%
\pgfpathcurveto{\pgfqpoint{9.530180in}{0.993329in}}{\pgfqpoint{9.532184in}{0.988491in}}{\pgfqpoint{9.535750in}{0.984925in}}%
\pgfpathcurveto{\pgfqpoint{9.539317in}{0.981358in}}{\pgfqpoint{9.544154in}{0.979354in}}{\pgfqpoint{9.549198in}{0.979354in}}%
\pgfpathclose%
\pgfusepath{fill}%
\end{pgfscope}%
\begin{pgfscope}%
\pgfpathrectangle{\pgfqpoint{6.572727in}{0.473000in}}{\pgfqpoint{4.227273in}{3.311000in}}%
\pgfusepath{clip}%
\pgfsetbuttcap%
\pgfsetroundjoin%
\definecolor{currentfill}{rgb}{0.993248,0.906157,0.143936}%
\pgfsetfillcolor{currentfill}%
\pgfsetfillopacity{0.700000}%
\pgfsetlinewidth{0.000000pt}%
\definecolor{currentstroke}{rgb}{0.000000,0.000000,0.000000}%
\pgfsetstrokecolor{currentstroke}%
\pgfsetstrokeopacity{0.700000}%
\pgfsetdash{}{0pt}%
\pgfpathmoveto{\pgfqpoint{9.938517in}{1.281933in}}%
\pgfpathcurveto{\pgfqpoint{9.943561in}{1.281933in}}{\pgfqpoint{9.948398in}{1.283937in}}{\pgfqpoint{9.951965in}{1.287503in}}%
\pgfpathcurveto{\pgfqpoint{9.955531in}{1.291070in}}{\pgfqpoint{9.957535in}{1.295907in}}{\pgfqpoint{9.957535in}{1.300951in}}%
\pgfpathcurveto{\pgfqpoint{9.957535in}{1.305995in}}{\pgfqpoint{9.955531in}{1.310833in}}{\pgfqpoint{9.951965in}{1.314399in}}%
\pgfpathcurveto{\pgfqpoint{9.948398in}{1.317965in}}{\pgfqpoint{9.943561in}{1.319969in}}{\pgfqpoint{9.938517in}{1.319969in}}%
\pgfpathcurveto{\pgfqpoint{9.933473in}{1.319969in}}{\pgfqpoint{9.928636in}{1.317965in}}{\pgfqpoint{9.925069in}{1.314399in}}%
\pgfpathcurveto{\pgfqpoint{9.921503in}{1.310833in}}{\pgfqpoint{9.919499in}{1.305995in}}{\pgfqpoint{9.919499in}{1.300951in}}%
\pgfpathcurveto{\pgfqpoint{9.919499in}{1.295907in}}{\pgfqpoint{9.921503in}{1.291070in}}{\pgfqpoint{9.925069in}{1.287503in}}%
\pgfpathcurveto{\pgfqpoint{9.928636in}{1.283937in}}{\pgfqpoint{9.933473in}{1.281933in}}{\pgfqpoint{9.938517in}{1.281933in}}%
\pgfpathclose%
\pgfusepath{fill}%
\end{pgfscope}%
\begin{pgfscope}%
\pgfpathrectangle{\pgfqpoint{6.572727in}{0.473000in}}{\pgfqpoint{4.227273in}{3.311000in}}%
\pgfusepath{clip}%
\pgfsetbuttcap%
\pgfsetroundjoin%
\definecolor{currentfill}{rgb}{0.993248,0.906157,0.143936}%
\pgfsetfillcolor{currentfill}%
\pgfsetfillopacity{0.700000}%
\pgfsetlinewidth{0.000000pt}%
\definecolor{currentstroke}{rgb}{0.000000,0.000000,0.000000}%
\pgfsetstrokecolor{currentstroke}%
\pgfsetstrokeopacity{0.700000}%
\pgfsetdash{}{0pt}%
\pgfpathmoveto{\pgfqpoint{9.637918in}{1.751345in}}%
\pgfpathcurveto{\pgfqpoint{9.642962in}{1.751345in}}{\pgfqpoint{9.647800in}{1.753349in}}{\pgfqpoint{9.651366in}{1.756916in}}%
\pgfpathcurveto{\pgfqpoint{9.654933in}{1.760482in}}{\pgfqpoint{9.656937in}{1.765320in}}{\pgfqpoint{9.656937in}{1.770364in}}%
\pgfpathcurveto{\pgfqpoint{9.656937in}{1.775407in}}{\pgfqpoint{9.654933in}{1.780245in}}{\pgfqpoint{9.651366in}{1.783811in}}%
\pgfpathcurveto{\pgfqpoint{9.647800in}{1.787378in}}{\pgfqpoint{9.642962in}{1.789382in}}{\pgfqpoint{9.637918in}{1.789382in}}%
\pgfpathcurveto{\pgfqpoint{9.632875in}{1.789382in}}{\pgfqpoint{9.628037in}{1.787378in}}{\pgfqpoint{9.624471in}{1.783811in}}%
\pgfpathcurveto{\pgfqpoint{9.620904in}{1.780245in}}{\pgfqpoint{9.618900in}{1.775407in}}{\pgfqpoint{9.618900in}{1.770364in}}%
\pgfpathcurveto{\pgfqpoint{9.618900in}{1.765320in}}{\pgfqpoint{9.620904in}{1.760482in}}{\pgfqpoint{9.624471in}{1.756916in}}%
\pgfpathcurveto{\pgfqpoint{9.628037in}{1.753349in}}{\pgfqpoint{9.632875in}{1.751345in}}{\pgfqpoint{9.637918in}{1.751345in}}%
\pgfpathclose%
\pgfusepath{fill}%
\end{pgfscope}%
\begin{pgfscope}%
\pgfpathrectangle{\pgfqpoint{6.572727in}{0.473000in}}{\pgfqpoint{4.227273in}{3.311000in}}%
\pgfusepath{clip}%
\pgfsetbuttcap%
\pgfsetroundjoin%
\definecolor{currentfill}{rgb}{0.993248,0.906157,0.143936}%
\pgfsetfillcolor{currentfill}%
\pgfsetfillopacity{0.700000}%
\pgfsetlinewidth{0.000000pt}%
\definecolor{currentstroke}{rgb}{0.000000,0.000000,0.000000}%
\pgfsetstrokecolor{currentstroke}%
\pgfsetstrokeopacity{0.700000}%
\pgfsetdash{}{0pt}%
\pgfpathmoveto{\pgfqpoint{9.935366in}{1.272064in}}%
\pgfpathcurveto{\pgfqpoint{9.940410in}{1.272064in}}{\pgfqpoint{9.945248in}{1.274068in}}{\pgfqpoint{9.948814in}{1.277635in}}%
\pgfpathcurveto{\pgfqpoint{9.952381in}{1.281201in}}{\pgfqpoint{9.954385in}{1.286039in}}{\pgfqpoint{9.954385in}{1.291083in}}%
\pgfpathcurveto{\pgfqpoint{9.954385in}{1.296126in}}{\pgfqpoint{9.952381in}{1.300964in}}{\pgfqpoint{9.948814in}{1.304530in}}%
\pgfpathcurveto{\pgfqpoint{9.945248in}{1.308097in}}{\pgfqpoint{9.940410in}{1.310101in}}{\pgfqpoint{9.935366in}{1.310101in}}%
\pgfpathcurveto{\pgfqpoint{9.930323in}{1.310101in}}{\pgfqpoint{9.925485in}{1.308097in}}{\pgfqpoint{9.921919in}{1.304530in}}%
\pgfpathcurveto{\pgfqpoint{9.918352in}{1.300964in}}{\pgfqpoint{9.916348in}{1.296126in}}{\pgfqpoint{9.916348in}{1.291083in}}%
\pgfpathcurveto{\pgfqpoint{9.916348in}{1.286039in}}{\pgfqpoint{9.918352in}{1.281201in}}{\pgfqpoint{9.921919in}{1.277635in}}%
\pgfpathcurveto{\pgfqpoint{9.925485in}{1.274068in}}{\pgfqpoint{9.930323in}{1.272064in}}{\pgfqpoint{9.935366in}{1.272064in}}%
\pgfpathclose%
\pgfusepath{fill}%
\end{pgfscope}%
\begin{pgfscope}%
\pgfpathrectangle{\pgfqpoint{6.572727in}{0.473000in}}{\pgfqpoint{4.227273in}{3.311000in}}%
\pgfusepath{clip}%
\pgfsetbuttcap%
\pgfsetroundjoin%
\definecolor{currentfill}{rgb}{0.993248,0.906157,0.143936}%
\pgfsetfillcolor{currentfill}%
\pgfsetfillopacity{0.700000}%
\pgfsetlinewidth{0.000000pt}%
\definecolor{currentstroke}{rgb}{0.000000,0.000000,0.000000}%
\pgfsetstrokecolor{currentstroke}%
\pgfsetstrokeopacity{0.700000}%
\pgfsetdash{}{0pt}%
\pgfpathmoveto{\pgfqpoint{9.632648in}{1.614172in}}%
\pgfpathcurveto{\pgfqpoint{9.637691in}{1.614172in}}{\pgfqpoint{9.642529in}{1.616176in}}{\pgfqpoint{9.646095in}{1.619743in}}%
\pgfpathcurveto{\pgfqpoint{9.649662in}{1.623309in}}{\pgfqpoint{9.651666in}{1.628147in}}{\pgfqpoint{9.651666in}{1.633191in}}%
\pgfpathcurveto{\pgfqpoint{9.651666in}{1.638234in}}{\pgfqpoint{9.649662in}{1.643072in}}{\pgfqpoint{9.646095in}{1.646638in}}%
\pgfpathcurveto{\pgfqpoint{9.642529in}{1.650205in}}{\pgfqpoint{9.637691in}{1.652209in}}{\pgfqpoint{9.632648in}{1.652209in}}%
\pgfpathcurveto{\pgfqpoint{9.627604in}{1.652209in}}{\pgfqpoint{9.622766in}{1.650205in}}{\pgfqpoint{9.619200in}{1.646638in}}%
\pgfpathcurveto{\pgfqpoint{9.615633in}{1.643072in}}{\pgfqpoint{9.613629in}{1.638234in}}{\pgfqpoint{9.613629in}{1.633191in}}%
\pgfpathcurveto{\pgfqpoint{9.613629in}{1.628147in}}{\pgfqpoint{9.615633in}{1.623309in}}{\pgfqpoint{9.619200in}{1.619743in}}%
\pgfpathcurveto{\pgfqpoint{9.622766in}{1.616176in}}{\pgfqpoint{9.627604in}{1.614172in}}{\pgfqpoint{9.632648in}{1.614172in}}%
\pgfpathclose%
\pgfusepath{fill}%
\end{pgfscope}%
\begin{pgfscope}%
\pgfpathrectangle{\pgfqpoint{6.572727in}{0.473000in}}{\pgfqpoint{4.227273in}{3.311000in}}%
\pgfusepath{clip}%
\pgfsetbuttcap%
\pgfsetroundjoin%
\definecolor{currentfill}{rgb}{0.127568,0.566949,0.550556}%
\pgfsetfillcolor{currentfill}%
\pgfsetfillopacity{0.700000}%
\pgfsetlinewidth{0.000000pt}%
\definecolor{currentstroke}{rgb}{0.000000,0.000000,0.000000}%
\pgfsetstrokecolor{currentstroke}%
\pgfsetstrokeopacity{0.700000}%
\pgfsetdash{}{0pt}%
\pgfpathmoveto{\pgfqpoint{8.044212in}{1.052410in}}%
\pgfpathcurveto{\pgfqpoint{8.049255in}{1.052410in}}{\pgfqpoint{8.054093in}{1.054414in}}{\pgfqpoint{8.057660in}{1.057980in}}%
\pgfpathcurveto{\pgfqpoint{8.061226in}{1.061547in}}{\pgfqpoint{8.063230in}{1.066385in}}{\pgfqpoint{8.063230in}{1.071428in}}%
\pgfpathcurveto{\pgfqpoint{8.063230in}{1.076472in}}{\pgfqpoint{8.061226in}{1.081310in}}{\pgfqpoint{8.057660in}{1.084876in}}%
\pgfpathcurveto{\pgfqpoint{8.054093in}{1.088442in}}{\pgfqpoint{8.049255in}{1.090446in}}{\pgfqpoint{8.044212in}{1.090446in}}%
\pgfpathcurveto{\pgfqpoint{8.039168in}{1.090446in}}{\pgfqpoint{8.034330in}{1.088442in}}{\pgfqpoint{8.030764in}{1.084876in}}%
\pgfpathcurveto{\pgfqpoint{8.027197in}{1.081310in}}{\pgfqpoint{8.025194in}{1.076472in}}{\pgfqpoint{8.025194in}{1.071428in}}%
\pgfpathcurveto{\pgfqpoint{8.025194in}{1.066385in}}{\pgfqpoint{8.027197in}{1.061547in}}{\pgfqpoint{8.030764in}{1.057980in}}%
\pgfpathcurveto{\pgfqpoint{8.034330in}{1.054414in}}{\pgfqpoint{8.039168in}{1.052410in}}{\pgfqpoint{8.044212in}{1.052410in}}%
\pgfpathclose%
\pgfusepath{fill}%
\end{pgfscope}%
\begin{pgfscope}%
\pgfpathrectangle{\pgfqpoint{6.572727in}{0.473000in}}{\pgfqpoint{4.227273in}{3.311000in}}%
\pgfusepath{clip}%
\pgfsetbuttcap%
\pgfsetroundjoin%
\definecolor{currentfill}{rgb}{0.127568,0.566949,0.550556}%
\pgfsetfillcolor{currentfill}%
\pgfsetfillopacity{0.700000}%
\pgfsetlinewidth{0.000000pt}%
\definecolor{currentstroke}{rgb}{0.000000,0.000000,0.000000}%
\pgfsetstrokecolor{currentstroke}%
\pgfsetstrokeopacity{0.700000}%
\pgfsetdash{}{0pt}%
\pgfpathmoveto{\pgfqpoint{7.670830in}{1.271674in}}%
\pgfpathcurveto{\pgfqpoint{7.675874in}{1.271674in}}{\pgfqpoint{7.680712in}{1.273678in}}{\pgfqpoint{7.684278in}{1.277244in}}%
\pgfpathcurveto{\pgfqpoint{7.687845in}{1.280811in}}{\pgfqpoint{7.689848in}{1.285649in}}{\pgfqpoint{7.689848in}{1.290692in}}%
\pgfpathcurveto{\pgfqpoint{7.689848in}{1.295736in}}{\pgfqpoint{7.687845in}{1.300574in}}{\pgfqpoint{7.684278in}{1.304140in}}%
\pgfpathcurveto{\pgfqpoint{7.680712in}{1.307707in}}{\pgfqpoint{7.675874in}{1.309710in}}{\pgfqpoint{7.670830in}{1.309710in}}%
\pgfpathcurveto{\pgfqpoint{7.665787in}{1.309710in}}{\pgfqpoint{7.660949in}{1.307707in}}{\pgfqpoint{7.657382in}{1.304140in}}%
\pgfpathcurveto{\pgfqpoint{7.653816in}{1.300574in}}{\pgfqpoint{7.651812in}{1.295736in}}{\pgfqpoint{7.651812in}{1.290692in}}%
\pgfpathcurveto{\pgfqpoint{7.651812in}{1.285649in}}{\pgfqpoint{7.653816in}{1.280811in}}{\pgfqpoint{7.657382in}{1.277244in}}%
\pgfpathcurveto{\pgfqpoint{7.660949in}{1.273678in}}{\pgfqpoint{7.665787in}{1.271674in}}{\pgfqpoint{7.670830in}{1.271674in}}%
\pgfpathclose%
\pgfusepath{fill}%
\end{pgfscope}%
\begin{pgfscope}%
\pgfpathrectangle{\pgfqpoint{6.572727in}{0.473000in}}{\pgfqpoint{4.227273in}{3.311000in}}%
\pgfusepath{clip}%
\pgfsetbuttcap%
\pgfsetroundjoin%
\definecolor{currentfill}{rgb}{0.993248,0.906157,0.143936}%
\pgfsetfillcolor{currentfill}%
\pgfsetfillopacity{0.700000}%
\pgfsetlinewidth{0.000000pt}%
\definecolor{currentstroke}{rgb}{0.000000,0.000000,0.000000}%
\pgfsetstrokecolor{currentstroke}%
\pgfsetstrokeopacity{0.700000}%
\pgfsetdash{}{0pt}%
\pgfpathmoveto{\pgfqpoint{9.727923in}{1.880183in}}%
\pgfpathcurveto{\pgfqpoint{9.732967in}{1.880183in}}{\pgfqpoint{9.737804in}{1.882187in}}{\pgfqpoint{9.741371in}{1.885754in}}%
\pgfpathcurveto{\pgfqpoint{9.744937in}{1.889320in}}{\pgfqpoint{9.746941in}{1.894158in}}{\pgfqpoint{9.746941in}{1.899201in}}%
\pgfpathcurveto{\pgfqpoint{9.746941in}{1.904245in}}{\pgfqpoint{9.744937in}{1.909083in}}{\pgfqpoint{9.741371in}{1.912649in}}%
\pgfpathcurveto{\pgfqpoint{9.737804in}{1.916216in}}{\pgfqpoint{9.732967in}{1.918220in}}{\pgfqpoint{9.727923in}{1.918220in}}%
\pgfpathcurveto{\pgfqpoint{9.722879in}{1.918220in}}{\pgfqpoint{9.718042in}{1.916216in}}{\pgfqpoint{9.714475in}{1.912649in}}%
\pgfpathcurveto{\pgfqpoint{9.710909in}{1.909083in}}{\pgfqpoint{9.708905in}{1.904245in}}{\pgfqpoint{9.708905in}{1.899201in}}%
\pgfpathcurveto{\pgfqpoint{9.708905in}{1.894158in}}{\pgfqpoint{9.710909in}{1.889320in}}{\pgfqpoint{9.714475in}{1.885754in}}%
\pgfpathcurveto{\pgfqpoint{9.718042in}{1.882187in}}{\pgfqpoint{9.722879in}{1.880183in}}{\pgfqpoint{9.727923in}{1.880183in}}%
\pgfpathclose%
\pgfusepath{fill}%
\end{pgfscope}%
\begin{pgfscope}%
\pgfpathrectangle{\pgfqpoint{6.572727in}{0.473000in}}{\pgfqpoint{4.227273in}{3.311000in}}%
\pgfusepath{clip}%
\pgfsetbuttcap%
\pgfsetroundjoin%
\definecolor{currentfill}{rgb}{0.127568,0.566949,0.550556}%
\pgfsetfillcolor{currentfill}%
\pgfsetfillopacity{0.700000}%
\pgfsetlinewidth{0.000000pt}%
\definecolor{currentstroke}{rgb}{0.000000,0.000000,0.000000}%
\pgfsetstrokecolor{currentstroke}%
\pgfsetstrokeopacity{0.700000}%
\pgfsetdash{}{0pt}%
\pgfpathmoveto{\pgfqpoint{7.486500in}{1.378280in}}%
\pgfpathcurveto{\pgfqpoint{7.491544in}{1.378280in}}{\pgfqpoint{7.496381in}{1.380284in}}{\pgfqpoint{7.499948in}{1.383850in}}%
\pgfpathcurveto{\pgfqpoint{7.503514in}{1.387417in}}{\pgfqpoint{7.505518in}{1.392254in}}{\pgfqpoint{7.505518in}{1.397298in}}%
\pgfpathcurveto{\pgfqpoint{7.505518in}{1.402342in}}{\pgfqpoint{7.503514in}{1.407180in}}{\pgfqpoint{7.499948in}{1.410746in}}%
\pgfpathcurveto{\pgfqpoint{7.496381in}{1.414312in}}{\pgfqpoint{7.491544in}{1.416316in}}{\pgfqpoint{7.486500in}{1.416316in}}%
\pgfpathcurveto{\pgfqpoint{7.481456in}{1.416316in}}{\pgfqpoint{7.476619in}{1.414312in}}{\pgfqpoint{7.473052in}{1.410746in}}%
\pgfpathcurveto{\pgfqpoint{7.469486in}{1.407180in}}{\pgfqpoint{7.467482in}{1.402342in}}{\pgfqpoint{7.467482in}{1.397298in}}%
\pgfpathcurveto{\pgfqpoint{7.467482in}{1.392254in}}{\pgfqpoint{7.469486in}{1.387417in}}{\pgfqpoint{7.473052in}{1.383850in}}%
\pgfpathcurveto{\pgfqpoint{7.476619in}{1.380284in}}{\pgfqpoint{7.481456in}{1.378280in}}{\pgfqpoint{7.486500in}{1.378280in}}%
\pgfpathclose%
\pgfusepath{fill}%
\end{pgfscope}%
\begin{pgfscope}%
\pgfpathrectangle{\pgfqpoint{6.572727in}{0.473000in}}{\pgfqpoint{4.227273in}{3.311000in}}%
\pgfusepath{clip}%
\pgfsetbuttcap%
\pgfsetroundjoin%
\definecolor{currentfill}{rgb}{0.127568,0.566949,0.550556}%
\pgfsetfillcolor{currentfill}%
\pgfsetfillopacity{0.700000}%
\pgfsetlinewidth{0.000000pt}%
\definecolor{currentstroke}{rgb}{0.000000,0.000000,0.000000}%
\pgfsetstrokecolor{currentstroke}%
\pgfsetstrokeopacity{0.700000}%
\pgfsetdash{}{0pt}%
\pgfpathmoveto{\pgfqpoint{7.183260in}{1.307464in}}%
\pgfpathcurveto{\pgfqpoint{7.188303in}{1.307464in}}{\pgfqpoint{7.193141in}{1.309468in}}{\pgfqpoint{7.196708in}{1.313034in}}%
\pgfpathcurveto{\pgfqpoint{7.200274in}{1.316601in}}{\pgfqpoint{7.202278in}{1.321439in}}{\pgfqpoint{7.202278in}{1.326482in}}%
\pgfpathcurveto{\pgfqpoint{7.202278in}{1.331526in}}{\pgfqpoint{7.200274in}{1.336364in}}{\pgfqpoint{7.196708in}{1.339930in}}%
\pgfpathcurveto{\pgfqpoint{7.193141in}{1.343497in}}{\pgfqpoint{7.188303in}{1.345500in}}{\pgfqpoint{7.183260in}{1.345500in}}%
\pgfpathcurveto{\pgfqpoint{7.178216in}{1.345500in}}{\pgfqpoint{7.173378in}{1.343497in}}{\pgfqpoint{7.169812in}{1.339930in}}%
\pgfpathcurveto{\pgfqpoint{7.166245in}{1.336364in}}{\pgfqpoint{7.164241in}{1.331526in}}{\pgfqpoint{7.164241in}{1.326482in}}%
\pgfpathcurveto{\pgfqpoint{7.164241in}{1.321439in}}{\pgfqpoint{7.166245in}{1.316601in}}{\pgfqpoint{7.169812in}{1.313034in}}%
\pgfpathcurveto{\pgfqpoint{7.173378in}{1.309468in}}{\pgfqpoint{7.178216in}{1.307464in}}{\pgfqpoint{7.183260in}{1.307464in}}%
\pgfpathclose%
\pgfusepath{fill}%
\end{pgfscope}%
\begin{pgfscope}%
\pgfpathrectangle{\pgfqpoint{6.572727in}{0.473000in}}{\pgfqpoint{4.227273in}{3.311000in}}%
\pgfusepath{clip}%
\pgfsetbuttcap%
\pgfsetroundjoin%
\definecolor{currentfill}{rgb}{0.127568,0.566949,0.550556}%
\pgfsetfillcolor{currentfill}%
\pgfsetfillopacity{0.700000}%
\pgfsetlinewidth{0.000000pt}%
\definecolor{currentstroke}{rgb}{0.000000,0.000000,0.000000}%
\pgfsetstrokecolor{currentstroke}%
\pgfsetstrokeopacity{0.700000}%
\pgfsetdash{}{0pt}%
\pgfpathmoveto{\pgfqpoint{8.591138in}{2.717856in}}%
\pgfpathcurveto{\pgfqpoint{8.596182in}{2.717856in}}{\pgfqpoint{8.601020in}{2.719860in}}{\pgfqpoint{8.604586in}{2.723427in}}%
\pgfpathcurveto{\pgfqpoint{8.608153in}{2.726993in}}{\pgfqpoint{8.610156in}{2.731831in}}{\pgfqpoint{8.610156in}{2.736875in}}%
\pgfpathcurveto{\pgfqpoint{8.610156in}{2.741918in}}{\pgfqpoint{8.608153in}{2.746756in}}{\pgfqpoint{8.604586in}{2.750322in}}%
\pgfpathcurveto{\pgfqpoint{8.601020in}{2.753889in}}{\pgfqpoint{8.596182in}{2.755893in}}{\pgfqpoint{8.591138in}{2.755893in}}%
\pgfpathcurveto{\pgfqpoint{8.586095in}{2.755893in}}{\pgfqpoint{8.581257in}{2.753889in}}{\pgfqpoint{8.577690in}{2.750322in}}%
\pgfpathcurveto{\pgfqpoint{8.574124in}{2.746756in}}{\pgfqpoint{8.572120in}{2.741918in}}{\pgfqpoint{8.572120in}{2.736875in}}%
\pgfpathcurveto{\pgfqpoint{8.572120in}{2.731831in}}{\pgfqpoint{8.574124in}{2.726993in}}{\pgfqpoint{8.577690in}{2.723427in}}%
\pgfpathcurveto{\pgfqpoint{8.581257in}{2.719860in}}{\pgfqpoint{8.586095in}{2.717856in}}{\pgfqpoint{8.591138in}{2.717856in}}%
\pgfpathclose%
\pgfusepath{fill}%
\end{pgfscope}%
\begin{pgfscope}%
\pgfpathrectangle{\pgfqpoint{6.572727in}{0.473000in}}{\pgfqpoint{4.227273in}{3.311000in}}%
\pgfusepath{clip}%
\pgfsetbuttcap%
\pgfsetroundjoin%
\definecolor{currentfill}{rgb}{0.267004,0.004874,0.329415}%
\pgfsetfillcolor{currentfill}%
\pgfsetfillopacity{0.700000}%
\pgfsetlinewidth{0.000000pt}%
\definecolor{currentstroke}{rgb}{0.000000,0.000000,0.000000}%
\pgfsetstrokecolor{currentstroke}%
\pgfsetstrokeopacity{0.700000}%
\pgfsetdash{}{0pt}%
\pgfpathmoveto{\pgfqpoint{7.203953in}{2.303408in}}%
\pgfpathcurveto{\pgfqpoint{7.208997in}{2.303408in}}{\pgfqpoint{7.213834in}{2.305412in}}{\pgfqpoint{7.217401in}{2.308978in}}%
\pgfpathcurveto{\pgfqpoint{7.220967in}{2.312545in}}{\pgfqpoint{7.222971in}{2.317382in}}{\pgfqpoint{7.222971in}{2.322426in}}%
\pgfpathcurveto{\pgfqpoint{7.222971in}{2.327470in}}{\pgfqpoint{7.220967in}{2.332307in}}{\pgfqpoint{7.217401in}{2.335874in}}%
\pgfpathcurveto{\pgfqpoint{7.213834in}{2.339440in}}{\pgfqpoint{7.208997in}{2.341444in}}{\pgfqpoint{7.203953in}{2.341444in}}%
\pgfpathcurveto{\pgfqpoint{7.198909in}{2.341444in}}{\pgfqpoint{7.194071in}{2.339440in}}{\pgfqpoint{7.190505in}{2.335874in}}%
\pgfpathcurveto{\pgfqpoint{7.186939in}{2.332307in}}{\pgfqpoint{7.184935in}{2.327470in}}{\pgfqpoint{7.184935in}{2.322426in}}%
\pgfpathcurveto{\pgfqpoint{7.184935in}{2.317382in}}{\pgfqpoint{7.186939in}{2.312545in}}{\pgfqpoint{7.190505in}{2.308978in}}%
\pgfpathcurveto{\pgfqpoint{7.194071in}{2.305412in}}{\pgfqpoint{7.198909in}{2.303408in}}{\pgfqpoint{7.203953in}{2.303408in}}%
\pgfpathclose%
\pgfusepath{fill}%
\end{pgfscope}%
\begin{pgfscope}%
\pgfpathrectangle{\pgfqpoint{6.572727in}{0.473000in}}{\pgfqpoint{4.227273in}{3.311000in}}%
\pgfusepath{clip}%
\pgfsetbuttcap%
\pgfsetroundjoin%
\definecolor{currentfill}{rgb}{0.127568,0.566949,0.550556}%
\pgfsetfillcolor{currentfill}%
\pgfsetfillopacity{0.700000}%
\pgfsetlinewidth{0.000000pt}%
\definecolor{currentstroke}{rgb}{0.000000,0.000000,0.000000}%
\pgfsetstrokecolor{currentstroke}%
\pgfsetstrokeopacity{0.700000}%
\pgfsetdash{}{0pt}%
\pgfpathmoveto{\pgfqpoint{8.146426in}{2.429999in}}%
\pgfpathcurveto{\pgfqpoint{8.151469in}{2.429999in}}{\pgfqpoint{8.156307in}{2.432003in}}{\pgfqpoint{8.159873in}{2.435570in}}%
\pgfpathcurveto{\pgfqpoint{8.163440in}{2.439136in}}{\pgfqpoint{8.165444in}{2.443974in}}{\pgfqpoint{8.165444in}{2.449017in}}%
\pgfpathcurveto{\pgfqpoint{8.165444in}{2.454061in}}{\pgfqpoint{8.163440in}{2.458899in}}{\pgfqpoint{8.159873in}{2.462465in}}%
\pgfpathcurveto{\pgfqpoint{8.156307in}{2.466032in}}{\pgfqpoint{8.151469in}{2.468036in}}{\pgfqpoint{8.146426in}{2.468036in}}%
\pgfpathcurveto{\pgfqpoint{8.141382in}{2.468036in}}{\pgfqpoint{8.136544in}{2.466032in}}{\pgfqpoint{8.132978in}{2.462465in}}%
\pgfpathcurveto{\pgfqpoint{8.129411in}{2.458899in}}{\pgfqpoint{8.127407in}{2.454061in}}{\pgfqpoint{8.127407in}{2.449017in}}%
\pgfpathcurveto{\pgfqpoint{8.127407in}{2.443974in}}{\pgfqpoint{8.129411in}{2.439136in}}{\pgfqpoint{8.132978in}{2.435570in}}%
\pgfpathcurveto{\pgfqpoint{8.136544in}{2.432003in}}{\pgfqpoint{8.141382in}{2.429999in}}{\pgfqpoint{8.146426in}{2.429999in}}%
\pgfpathclose%
\pgfusepath{fill}%
\end{pgfscope}%
\begin{pgfscope}%
\pgfpathrectangle{\pgfqpoint{6.572727in}{0.473000in}}{\pgfqpoint{4.227273in}{3.311000in}}%
\pgfusepath{clip}%
\pgfsetbuttcap%
\pgfsetroundjoin%
\definecolor{currentfill}{rgb}{0.993248,0.906157,0.143936}%
\pgfsetfillcolor{currentfill}%
\pgfsetfillopacity{0.700000}%
\pgfsetlinewidth{0.000000pt}%
\definecolor{currentstroke}{rgb}{0.000000,0.000000,0.000000}%
\pgfsetstrokecolor{currentstroke}%
\pgfsetstrokeopacity{0.700000}%
\pgfsetdash{}{0pt}%
\pgfpathmoveto{\pgfqpoint{9.958396in}{2.292602in}}%
\pgfpathcurveto{\pgfqpoint{9.963440in}{2.292602in}}{\pgfqpoint{9.968278in}{2.294606in}}{\pgfqpoint{9.971844in}{2.298173in}}%
\pgfpathcurveto{\pgfqpoint{9.975411in}{2.301739in}}{\pgfqpoint{9.977414in}{2.306577in}}{\pgfqpoint{9.977414in}{2.311621in}}%
\pgfpathcurveto{\pgfqpoint{9.977414in}{2.316664in}}{\pgfqpoint{9.975411in}{2.321502in}}{\pgfqpoint{9.971844in}{2.325068in}}%
\pgfpathcurveto{\pgfqpoint{9.968278in}{2.328635in}}{\pgfqpoint{9.963440in}{2.330639in}}{\pgfqpoint{9.958396in}{2.330639in}}%
\pgfpathcurveto{\pgfqpoint{9.953353in}{2.330639in}}{\pgfqpoint{9.948515in}{2.328635in}}{\pgfqpoint{9.944948in}{2.325068in}}%
\pgfpathcurveto{\pgfqpoint{9.941382in}{2.321502in}}{\pgfqpoint{9.939378in}{2.316664in}}{\pgfqpoint{9.939378in}{2.311621in}}%
\pgfpathcurveto{\pgfqpoint{9.939378in}{2.306577in}}{\pgfqpoint{9.941382in}{2.301739in}}{\pgfqpoint{9.944948in}{2.298173in}}%
\pgfpathcurveto{\pgfqpoint{9.948515in}{2.294606in}}{\pgfqpoint{9.953353in}{2.292602in}}{\pgfqpoint{9.958396in}{2.292602in}}%
\pgfpathclose%
\pgfusepath{fill}%
\end{pgfscope}%
\begin{pgfscope}%
\pgfpathrectangle{\pgfqpoint{6.572727in}{0.473000in}}{\pgfqpoint{4.227273in}{3.311000in}}%
\pgfusepath{clip}%
\pgfsetbuttcap%
\pgfsetroundjoin%
\definecolor{currentfill}{rgb}{0.127568,0.566949,0.550556}%
\pgfsetfillcolor{currentfill}%
\pgfsetfillopacity{0.700000}%
\pgfsetlinewidth{0.000000pt}%
\definecolor{currentstroke}{rgb}{0.000000,0.000000,0.000000}%
\pgfsetstrokecolor{currentstroke}%
\pgfsetstrokeopacity{0.700000}%
\pgfsetdash{}{0pt}%
\pgfpathmoveto{\pgfqpoint{8.131257in}{2.246665in}}%
\pgfpathcurveto{\pgfqpoint{8.136300in}{2.246665in}}{\pgfqpoint{8.141138in}{2.248669in}}{\pgfqpoint{8.144704in}{2.252236in}}%
\pgfpathcurveto{\pgfqpoint{8.148271in}{2.255802in}}{\pgfqpoint{8.150275in}{2.260640in}}{\pgfqpoint{8.150275in}{2.265683in}}%
\pgfpathcurveto{\pgfqpoint{8.150275in}{2.270727in}}{\pgfqpoint{8.148271in}{2.275565in}}{\pgfqpoint{8.144704in}{2.279131in}}%
\pgfpathcurveto{\pgfqpoint{8.141138in}{2.282698in}}{\pgfqpoint{8.136300in}{2.284702in}}{\pgfqpoint{8.131257in}{2.284702in}}%
\pgfpathcurveto{\pgfqpoint{8.126213in}{2.284702in}}{\pgfqpoint{8.121375in}{2.282698in}}{\pgfqpoint{8.117809in}{2.279131in}}%
\pgfpathcurveto{\pgfqpoint{8.114242in}{2.275565in}}{\pgfqpoint{8.112238in}{2.270727in}}{\pgfqpoint{8.112238in}{2.265683in}}%
\pgfpathcurveto{\pgfqpoint{8.112238in}{2.260640in}}{\pgfqpoint{8.114242in}{2.255802in}}{\pgfqpoint{8.117809in}{2.252236in}}%
\pgfpathcurveto{\pgfqpoint{8.121375in}{2.248669in}}{\pgfqpoint{8.126213in}{2.246665in}}{\pgfqpoint{8.131257in}{2.246665in}}%
\pgfpathclose%
\pgfusepath{fill}%
\end{pgfscope}%
\begin{pgfscope}%
\pgfpathrectangle{\pgfqpoint{6.572727in}{0.473000in}}{\pgfqpoint{4.227273in}{3.311000in}}%
\pgfusepath{clip}%
\pgfsetbuttcap%
\pgfsetroundjoin%
\definecolor{currentfill}{rgb}{0.127568,0.566949,0.550556}%
\pgfsetfillcolor{currentfill}%
\pgfsetfillopacity{0.700000}%
\pgfsetlinewidth{0.000000pt}%
\definecolor{currentstroke}{rgb}{0.000000,0.000000,0.000000}%
\pgfsetstrokecolor{currentstroke}%
\pgfsetstrokeopacity{0.700000}%
\pgfsetdash{}{0pt}%
\pgfpathmoveto{\pgfqpoint{8.039184in}{1.819329in}}%
\pgfpathcurveto{\pgfqpoint{8.044228in}{1.819329in}}{\pgfqpoint{8.049065in}{1.821333in}}{\pgfqpoint{8.052632in}{1.824900in}}%
\pgfpathcurveto{\pgfqpoint{8.056198in}{1.828466in}}{\pgfqpoint{8.058202in}{1.833304in}}{\pgfqpoint{8.058202in}{1.838347in}}%
\pgfpathcurveto{\pgfqpoint{8.058202in}{1.843391in}}{\pgfqpoint{8.056198in}{1.848229in}}{\pgfqpoint{8.052632in}{1.851795in}}%
\pgfpathcurveto{\pgfqpoint{8.049065in}{1.855362in}}{\pgfqpoint{8.044228in}{1.857366in}}{\pgfqpoint{8.039184in}{1.857366in}}%
\pgfpathcurveto{\pgfqpoint{8.034140in}{1.857366in}}{\pgfqpoint{8.029303in}{1.855362in}}{\pgfqpoint{8.025736in}{1.851795in}}%
\pgfpathcurveto{\pgfqpoint{8.022170in}{1.848229in}}{\pgfqpoint{8.020166in}{1.843391in}}{\pgfqpoint{8.020166in}{1.838347in}}%
\pgfpathcurveto{\pgfqpoint{8.020166in}{1.833304in}}{\pgfqpoint{8.022170in}{1.828466in}}{\pgfqpoint{8.025736in}{1.824900in}}%
\pgfpathcurveto{\pgfqpoint{8.029303in}{1.821333in}}{\pgfqpoint{8.034140in}{1.819329in}}{\pgfqpoint{8.039184in}{1.819329in}}%
\pgfpathclose%
\pgfusepath{fill}%
\end{pgfscope}%
\begin{pgfscope}%
\pgfpathrectangle{\pgfqpoint{6.572727in}{0.473000in}}{\pgfqpoint{4.227273in}{3.311000in}}%
\pgfusepath{clip}%
\pgfsetbuttcap%
\pgfsetroundjoin%
\definecolor{currentfill}{rgb}{0.127568,0.566949,0.550556}%
\pgfsetfillcolor{currentfill}%
\pgfsetfillopacity{0.700000}%
\pgfsetlinewidth{0.000000pt}%
\definecolor{currentstroke}{rgb}{0.000000,0.000000,0.000000}%
\pgfsetstrokecolor{currentstroke}%
\pgfsetstrokeopacity{0.700000}%
\pgfsetdash{}{0pt}%
\pgfpathmoveto{\pgfqpoint{8.147547in}{1.809375in}}%
\pgfpathcurveto{\pgfqpoint{8.152590in}{1.809375in}}{\pgfqpoint{8.157428in}{1.811379in}}{\pgfqpoint{8.160995in}{1.814946in}}%
\pgfpathcurveto{\pgfqpoint{8.164561in}{1.818512in}}{\pgfqpoint{8.166565in}{1.823350in}}{\pgfqpoint{8.166565in}{1.828394in}}%
\pgfpathcurveto{\pgfqpoint{8.166565in}{1.833437in}}{\pgfqpoint{8.164561in}{1.838275in}}{\pgfqpoint{8.160995in}{1.841841in}}%
\pgfpathcurveto{\pgfqpoint{8.157428in}{1.845408in}}{\pgfqpoint{8.152590in}{1.847412in}}{\pgfqpoint{8.147547in}{1.847412in}}%
\pgfpathcurveto{\pgfqpoint{8.142503in}{1.847412in}}{\pgfqpoint{8.137665in}{1.845408in}}{\pgfqpoint{8.134099in}{1.841841in}}%
\pgfpathcurveto{\pgfqpoint{8.130532in}{1.838275in}}{\pgfqpoint{8.128529in}{1.833437in}}{\pgfqpoint{8.128529in}{1.828394in}}%
\pgfpathcurveto{\pgfqpoint{8.128529in}{1.823350in}}{\pgfqpoint{8.130532in}{1.818512in}}{\pgfqpoint{8.134099in}{1.814946in}}%
\pgfpathcurveto{\pgfqpoint{8.137665in}{1.811379in}}{\pgfqpoint{8.142503in}{1.809375in}}{\pgfqpoint{8.147547in}{1.809375in}}%
\pgfpathclose%
\pgfusepath{fill}%
\end{pgfscope}%
\begin{pgfscope}%
\pgfpathrectangle{\pgfqpoint{6.572727in}{0.473000in}}{\pgfqpoint{4.227273in}{3.311000in}}%
\pgfusepath{clip}%
\pgfsetbuttcap%
\pgfsetroundjoin%
\definecolor{currentfill}{rgb}{0.127568,0.566949,0.550556}%
\pgfsetfillcolor{currentfill}%
\pgfsetfillopacity{0.700000}%
\pgfsetlinewidth{0.000000pt}%
\definecolor{currentstroke}{rgb}{0.000000,0.000000,0.000000}%
\pgfsetstrokecolor{currentstroke}%
\pgfsetstrokeopacity{0.700000}%
\pgfsetdash{}{0pt}%
\pgfpathmoveto{\pgfqpoint{7.346350in}{1.385383in}}%
\pgfpathcurveto{\pgfqpoint{7.351394in}{1.385383in}}{\pgfqpoint{7.356232in}{1.387387in}}{\pgfqpoint{7.359798in}{1.390954in}}%
\pgfpathcurveto{\pgfqpoint{7.363365in}{1.394520in}}{\pgfqpoint{7.365368in}{1.399358in}}{\pgfqpoint{7.365368in}{1.404401in}}%
\pgfpathcurveto{\pgfqpoint{7.365368in}{1.409445in}}{\pgfqpoint{7.363365in}{1.414283in}}{\pgfqpoint{7.359798in}{1.417849in}}%
\pgfpathcurveto{\pgfqpoint{7.356232in}{1.421416in}}{\pgfqpoint{7.351394in}{1.423420in}}{\pgfqpoint{7.346350in}{1.423420in}}%
\pgfpathcurveto{\pgfqpoint{7.341307in}{1.423420in}}{\pgfqpoint{7.336469in}{1.421416in}}{\pgfqpoint{7.332902in}{1.417849in}}%
\pgfpathcurveto{\pgfqpoint{7.329336in}{1.414283in}}{\pgfqpoint{7.327332in}{1.409445in}}{\pgfqpoint{7.327332in}{1.404401in}}%
\pgfpathcurveto{\pgfqpoint{7.327332in}{1.399358in}}{\pgfqpoint{7.329336in}{1.394520in}}{\pgfqpoint{7.332902in}{1.390954in}}%
\pgfpathcurveto{\pgfqpoint{7.336469in}{1.387387in}}{\pgfqpoint{7.341307in}{1.385383in}}{\pgfqpoint{7.346350in}{1.385383in}}%
\pgfpathclose%
\pgfusepath{fill}%
\end{pgfscope}%
\begin{pgfscope}%
\pgfpathrectangle{\pgfqpoint{6.572727in}{0.473000in}}{\pgfqpoint{4.227273in}{3.311000in}}%
\pgfusepath{clip}%
\pgfsetbuttcap%
\pgfsetroundjoin%
\definecolor{currentfill}{rgb}{0.993248,0.906157,0.143936}%
\pgfsetfillcolor{currentfill}%
\pgfsetfillopacity{0.700000}%
\pgfsetlinewidth{0.000000pt}%
\definecolor{currentstroke}{rgb}{0.000000,0.000000,0.000000}%
\pgfsetstrokecolor{currentstroke}%
\pgfsetstrokeopacity{0.700000}%
\pgfsetdash{}{0pt}%
\pgfpathmoveto{\pgfqpoint{9.787159in}{1.346671in}}%
\pgfpathcurveto{\pgfqpoint{9.792203in}{1.346671in}}{\pgfqpoint{9.797040in}{1.348675in}}{\pgfqpoint{9.800607in}{1.352242in}}%
\pgfpathcurveto{\pgfqpoint{9.804173in}{1.355808in}}{\pgfqpoint{9.806177in}{1.360646in}}{\pgfqpoint{9.806177in}{1.365689in}}%
\pgfpathcurveto{\pgfqpoint{9.806177in}{1.370733in}}{\pgfqpoint{9.804173in}{1.375571in}}{\pgfqpoint{9.800607in}{1.379137in}}%
\pgfpathcurveto{\pgfqpoint{9.797040in}{1.382704in}}{\pgfqpoint{9.792203in}{1.384708in}}{\pgfqpoint{9.787159in}{1.384708in}}%
\pgfpathcurveto{\pgfqpoint{9.782115in}{1.384708in}}{\pgfqpoint{9.777278in}{1.382704in}}{\pgfqpoint{9.773711in}{1.379137in}}%
\pgfpathcurveto{\pgfqpoint{9.770145in}{1.375571in}}{\pgfqpoint{9.768141in}{1.370733in}}{\pgfqpoint{9.768141in}{1.365689in}}%
\pgfpathcurveto{\pgfqpoint{9.768141in}{1.360646in}}{\pgfqpoint{9.770145in}{1.355808in}}{\pgfqpoint{9.773711in}{1.352242in}}%
\pgfpathcurveto{\pgfqpoint{9.777278in}{1.348675in}}{\pgfqpoint{9.782115in}{1.346671in}}{\pgfqpoint{9.787159in}{1.346671in}}%
\pgfpathclose%
\pgfusepath{fill}%
\end{pgfscope}%
\begin{pgfscope}%
\pgfpathrectangle{\pgfqpoint{6.572727in}{0.473000in}}{\pgfqpoint{4.227273in}{3.311000in}}%
\pgfusepath{clip}%
\pgfsetbuttcap%
\pgfsetroundjoin%
\definecolor{currentfill}{rgb}{0.127568,0.566949,0.550556}%
\pgfsetfillcolor{currentfill}%
\pgfsetfillopacity{0.700000}%
\pgfsetlinewidth{0.000000pt}%
\definecolor{currentstroke}{rgb}{0.000000,0.000000,0.000000}%
\pgfsetstrokecolor{currentstroke}%
\pgfsetstrokeopacity{0.700000}%
\pgfsetdash{}{0pt}%
\pgfpathmoveto{\pgfqpoint{8.359853in}{2.521639in}}%
\pgfpathcurveto{\pgfqpoint{8.364896in}{2.521639in}}{\pgfqpoint{8.369734in}{2.523643in}}{\pgfqpoint{8.373300in}{2.527210in}}%
\pgfpathcurveto{\pgfqpoint{8.376867in}{2.530776in}}{\pgfqpoint{8.378871in}{2.535614in}}{\pgfqpoint{8.378871in}{2.540657in}}%
\pgfpathcurveto{\pgfqpoint{8.378871in}{2.545701in}}{\pgfqpoint{8.376867in}{2.550539in}}{\pgfqpoint{8.373300in}{2.554105in}}%
\pgfpathcurveto{\pgfqpoint{8.369734in}{2.557672in}}{\pgfqpoint{8.364896in}{2.559676in}}{\pgfqpoint{8.359853in}{2.559676in}}%
\pgfpathcurveto{\pgfqpoint{8.354809in}{2.559676in}}{\pgfqpoint{8.349971in}{2.557672in}}{\pgfqpoint{8.346405in}{2.554105in}}%
\pgfpathcurveto{\pgfqpoint{8.342838in}{2.550539in}}{\pgfqpoint{8.340834in}{2.545701in}}{\pgfqpoint{8.340834in}{2.540657in}}%
\pgfpathcurveto{\pgfqpoint{8.340834in}{2.535614in}}{\pgfqpoint{8.342838in}{2.530776in}}{\pgfqpoint{8.346405in}{2.527210in}}%
\pgfpathcurveto{\pgfqpoint{8.349971in}{2.523643in}}{\pgfqpoint{8.354809in}{2.521639in}}{\pgfqpoint{8.359853in}{2.521639in}}%
\pgfpathclose%
\pgfusepath{fill}%
\end{pgfscope}%
\begin{pgfscope}%
\pgfpathrectangle{\pgfqpoint{6.572727in}{0.473000in}}{\pgfqpoint{4.227273in}{3.311000in}}%
\pgfusepath{clip}%
\pgfsetbuttcap%
\pgfsetroundjoin%
\definecolor{currentfill}{rgb}{0.993248,0.906157,0.143936}%
\pgfsetfillcolor{currentfill}%
\pgfsetfillopacity{0.700000}%
\pgfsetlinewidth{0.000000pt}%
\definecolor{currentstroke}{rgb}{0.000000,0.000000,0.000000}%
\pgfsetstrokecolor{currentstroke}%
\pgfsetstrokeopacity{0.700000}%
\pgfsetdash{}{0pt}%
\pgfpathmoveto{\pgfqpoint{9.451051in}{1.265029in}}%
\pgfpathcurveto{\pgfqpoint{9.456095in}{1.265029in}}{\pgfqpoint{9.460933in}{1.267033in}}{\pgfqpoint{9.464499in}{1.270599in}}%
\pgfpathcurveto{\pgfqpoint{9.468066in}{1.274166in}}{\pgfqpoint{9.470070in}{1.279003in}}{\pgfqpoint{9.470070in}{1.284047in}}%
\pgfpathcurveto{\pgfqpoint{9.470070in}{1.289091in}}{\pgfqpoint{9.468066in}{1.293929in}}{\pgfqpoint{9.464499in}{1.297495in}}%
\pgfpathcurveto{\pgfqpoint{9.460933in}{1.301061in}}{\pgfqpoint{9.456095in}{1.303065in}}{\pgfqpoint{9.451051in}{1.303065in}}%
\pgfpathcurveto{\pgfqpoint{9.446008in}{1.303065in}}{\pgfqpoint{9.441170in}{1.301061in}}{\pgfqpoint{9.437604in}{1.297495in}}%
\pgfpathcurveto{\pgfqpoint{9.434037in}{1.293929in}}{\pgfqpoint{9.432033in}{1.289091in}}{\pgfqpoint{9.432033in}{1.284047in}}%
\pgfpathcurveto{\pgfqpoint{9.432033in}{1.279003in}}{\pgfqpoint{9.434037in}{1.274166in}}{\pgfqpoint{9.437604in}{1.270599in}}%
\pgfpathcurveto{\pgfqpoint{9.441170in}{1.267033in}}{\pgfqpoint{9.446008in}{1.265029in}}{\pgfqpoint{9.451051in}{1.265029in}}%
\pgfpathclose%
\pgfusepath{fill}%
\end{pgfscope}%
\begin{pgfscope}%
\pgfpathrectangle{\pgfqpoint{6.572727in}{0.473000in}}{\pgfqpoint{4.227273in}{3.311000in}}%
\pgfusepath{clip}%
\pgfsetbuttcap%
\pgfsetroundjoin%
\definecolor{currentfill}{rgb}{0.127568,0.566949,0.550556}%
\pgfsetfillcolor{currentfill}%
\pgfsetfillopacity{0.700000}%
\pgfsetlinewidth{0.000000pt}%
\definecolor{currentstroke}{rgb}{0.000000,0.000000,0.000000}%
\pgfsetstrokecolor{currentstroke}%
\pgfsetstrokeopacity{0.700000}%
\pgfsetdash{}{0pt}%
\pgfpathmoveto{\pgfqpoint{8.622912in}{2.783198in}}%
\pgfpathcurveto{\pgfqpoint{8.627956in}{2.783198in}}{\pgfqpoint{8.632794in}{2.785202in}}{\pgfqpoint{8.636360in}{2.788768in}}%
\pgfpathcurveto{\pgfqpoint{8.639927in}{2.792334in}}{\pgfqpoint{8.641931in}{2.797172in}}{\pgfqpoint{8.641931in}{2.802216in}}%
\pgfpathcurveto{\pgfqpoint{8.641931in}{2.807260in}}{\pgfqpoint{8.639927in}{2.812097in}}{\pgfqpoint{8.636360in}{2.815664in}}%
\pgfpathcurveto{\pgfqpoint{8.632794in}{2.819230in}}{\pgfqpoint{8.627956in}{2.821234in}}{\pgfqpoint{8.622912in}{2.821234in}}%
\pgfpathcurveto{\pgfqpoint{8.617869in}{2.821234in}}{\pgfqpoint{8.613031in}{2.819230in}}{\pgfqpoint{8.609465in}{2.815664in}}%
\pgfpathcurveto{\pgfqpoint{8.605898in}{2.812097in}}{\pgfqpoint{8.603894in}{2.807260in}}{\pgfqpoint{8.603894in}{2.802216in}}%
\pgfpathcurveto{\pgfqpoint{8.603894in}{2.797172in}}{\pgfqpoint{8.605898in}{2.792334in}}{\pgfqpoint{8.609465in}{2.788768in}}%
\pgfpathcurveto{\pgfqpoint{8.613031in}{2.785202in}}{\pgfqpoint{8.617869in}{2.783198in}}{\pgfqpoint{8.622912in}{2.783198in}}%
\pgfpathclose%
\pgfusepath{fill}%
\end{pgfscope}%
\begin{pgfscope}%
\pgfpathrectangle{\pgfqpoint{6.572727in}{0.473000in}}{\pgfqpoint{4.227273in}{3.311000in}}%
\pgfusepath{clip}%
\pgfsetbuttcap%
\pgfsetroundjoin%
\definecolor{currentfill}{rgb}{0.993248,0.906157,0.143936}%
\pgfsetfillcolor{currentfill}%
\pgfsetfillopacity{0.700000}%
\pgfsetlinewidth{0.000000pt}%
\definecolor{currentstroke}{rgb}{0.000000,0.000000,0.000000}%
\pgfsetstrokecolor{currentstroke}%
\pgfsetstrokeopacity{0.700000}%
\pgfsetdash{}{0pt}%
\pgfpathmoveto{\pgfqpoint{9.626565in}{2.288251in}}%
\pgfpathcurveto{\pgfqpoint{9.631608in}{2.288251in}}{\pgfqpoint{9.636446in}{2.290254in}}{\pgfqpoint{9.640013in}{2.293821in}}%
\pgfpathcurveto{\pgfqpoint{9.643579in}{2.297387in}}{\pgfqpoint{9.645583in}{2.302225in}}{\pgfqpoint{9.645583in}{2.307269in}}%
\pgfpathcurveto{\pgfqpoint{9.645583in}{2.312312in}}{\pgfqpoint{9.643579in}{2.317150in}}{\pgfqpoint{9.640013in}{2.320717in}}%
\pgfpathcurveto{\pgfqpoint{9.636446in}{2.324283in}}{\pgfqpoint{9.631608in}{2.326287in}}{\pgfqpoint{9.626565in}{2.326287in}}%
\pgfpathcurveto{\pgfqpoint{9.621521in}{2.326287in}}{\pgfqpoint{9.616683in}{2.324283in}}{\pgfqpoint{9.613117in}{2.320717in}}%
\pgfpathcurveto{\pgfqpoint{9.609551in}{2.317150in}}{\pgfqpoint{9.607547in}{2.312312in}}{\pgfqpoint{9.607547in}{2.307269in}}%
\pgfpathcurveto{\pgfqpoint{9.607547in}{2.302225in}}{\pgfqpoint{9.609551in}{2.297387in}}{\pgfqpoint{9.613117in}{2.293821in}}%
\pgfpathcurveto{\pgfqpoint{9.616683in}{2.290254in}}{\pgfqpoint{9.621521in}{2.288251in}}{\pgfqpoint{9.626565in}{2.288251in}}%
\pgfpathclose%
\pgfusepath{fill}%
\end{pgfscope}%
\begin{pgfscope}%
\pgfpathrectangle{\pgfqpoint{6.572727in}{0.473000in}}{\pgfqpoint{4.227273in}{3.311000in}}%
\pgfusepath{clip}%
\pgfsetbuttcap%
\pgfsetroundjoin%
\definecolor{currentfill}{rgb}{0.993248,0.906157,0.143936}%
\pgfsetfillcolor{currentfill}%
\pgfsetfillopacity{0.700000}%
\pgfsetlinewidth{0.000000pt}%
\definecolor{currentstroke}{rgb}{0.000000,0.000000,0.000000}%
\pgfsetstrokecolor{currentstroke}%
\pgfsetstrokeopacity{0.700000}%
\pgfsetdash{}{0pt}%
\pgfpathmoveto{\pgfqpoint{10.067406in}{1.082378in}}%
\pgfpathcurveto{\pgfqpoint{10.072449in}{1.082378in}}{\pgfqpoint{10.077287in}{1.084382in}}{\pgfqpoint{10.080853in}{1.087948in}}%
\pgfpathcurveto{\pgfqpoint{10.084420in}{1.091514in}}{\pgfqpoint{10.086424in}{1.096352in}}{\pgfqpoint{10.086424in}{1.101396in}}%
\pgfpathcurveto{\pgfqpoint{10.086424in}{1.106440in}}{\pgfqpoint{10.084420in}{1.111277in}}{\pgfqpoint{10.080853in}{1.114844in}}%
\pgfpathcurveto{\pgfqpoint{10.077287in}{1.118410in}}{\pgfqpoint{10.072449in}{1.120414in}}{\pgfqpoint{10.067406in}{1.120414in}}%
\pgfpathcurveto{\pgfqpoint{10.062362in}{1.120414in}}{\pgfqpoint{10.057524in}{1.118410in}}{\pgfqpoint{10.053958in}{1.114844in}}%
\pgfpathcurveto{\pgfqpoint{10.050391in}{1.111277in}}{\pgfqpoint{10.048387in}{1.106440in}}{\pgfqpoint{10.048387in}{1.101396in}}%
\pgfpathcurveto{\pgfqpoint{10.048387in}{1.096352in}}{\pgfqpoint{10.050391in}{1.091514in}}{\pgfqpoint{10.053958in}{1.087948in}}%
\pgfpathcurveto{\pgfqpoint{10.057524in}{1.084382in}}{\pgfqpoint{10.062362in}{1.082378in}}{\pgfqpoint{10.067406in}{1.082378in}}%
\pgfpathclose%
\pgfusepath{fill}%
\end{pgfscope}%
\begin{pgfscope}%
\pgfpathrectangle{\pgfqpoint{6.572727in}{0.473000in}}{\pgfqpoint{4.227273in}{3.311000in}}%
\pgfusepath{clip}%
\pgfsetbuttcap%
\pgfsetroundjoin%
\definecolor{currentfill}{rgb}{0.127568,0.566949,0.550556}%
\pgfsetfillcolor{currentfill}%
\pgfsetfillopacity{0.700000}%
\pgfsetlinewidth{0.000000pt}%
\definecolor{currentstroke}{rgb}{0.000000,0.000000,0.000000}%
\pgfsetstrokecolor{currentstroke}%
\pgfsetstrokeopacity{0.700000}%
\pgfsetdash{}{0pt}%
\pgfpathmoveto{\pgfqpoint{7.406433in}{1.074386in}}%
\pgfpathcurveto{\pgfqpoint{7.411477in}{1.074386in}}{\pgfqpoint{7.416315in}{1.076390in}}{\pgfqpoint{7.419881in}{1.079957in}}%
\pgfpathcurveto{\pgfqpoint{7.423448in}{1.083523in}}{\pgfqpoint{7.425451in}{1.088361in}}{\pgfqpoint{7.425451in}{1.093405in}}%
\pgfpathcurveto{\pgfqpoint{7.425451in}{1.098448in}}{\pgfqpoint{7.423448in}{1.103286in}}{\pgfqpoint{7.419881in}{1.106852in}}%
\pgfpathcurveto{\pgfqpoint{7.416315in}{1.110419in}}{\pgfqpoint{7.411477in}{1.112423in}}{\pgfqpoint{7.406433in}{1.112423in}}%
\pgfpathcurveto{\pgfqpoint{7.401390in}{1.112423in}}{\pgfqpoint{7.396552in}{1.110419in}}{\pgfqpoint{7.392985in}{1.106852in}}%
\pgfpathcurveto{\pgfqpoint{7.389419in}{1.103286in}}{\pgfqpoint{7.387415in}{1.098448in}}{\pgfqpoint{7.387415in}{1.093405in}}%
\pgfpathcurveto{\pgfqpoint{7.387415in}{1.088361in}}{\pgfqpoint{7.389419in}{1.083523in}}{\pgfqpoint{7.392985in}{1.079957in}}%
\pgfpathcurveto{\pgfqpoint{7.396552in}{1.076390in}}{\pgfqpoint{7.401390in}{1.074386in}}{\pgfqpoint{7.406433in}{1.074386in}}%
\pgfpathclose%
\pgfusepath{fill}%
\end{pgfscope}%
\begin{pgfscope}%
\pgfpathrectangle{\pgfqpoint{6.572727in}{0.473000in}}{\pgfqpoint{4.227273in}{3.311000in}}%
\pgfusepath{clip}%
\pgfsetbuttcap%
\pgfsetroundjoin%
\definecolor{currentfill}{rgb}{0.127568,0.566949,0.550556}%
\pgfsetfillcolor{currentfill}%
\pgfsetfillopacity{0.700000}%
\pgfsetlinewidth{0.000000pt}%
\definecolor{currentstroke}{rgb}{0.000000,0.000000,0.000000}%
\pgfsetstrokecolor{currentstroke}%
\pgfsetstrokeopacity{0.700000}%
\pgfsetdash{}{0pt}%
\pgfpathmoveto{\pgfqpoint{7.584939in}{0.922299in}}%
\pgfpathcurveto{\pgfqpoint{7.589983in}{0.922299in}}{\pgfqpoint{7.594821in}{0.924303in}}{\pgfqpoint{7.598387in}{0.927870in}}%
\pgfpathcurveto{\pgfqpoint{7.601954in}{0.931436in}}{\pgfqpoint{7.603957in}{0.936274in}}{\pgfqpoint{7.603957in}{0.941317in}}%
\pgfpathcurveto{\pgfqpoint{7.603957in}{0.946361in}}{\pgfqpoint{7.601954in}{0.951199in}}{\pgfqpoint{7.598387in}{0.954765in}}%
\pgfpathcurveto{\pgfqpoint{7.594821in}{0.958332in}}{\pgfqpoint{7.589983in}{0.960336in}}{\pgfqpoint{7.584939in}{0.960336in}}%
\pgfpathcurveto{\pgfqpoint{7.579896in}{0.960336in}}{\pgfqpoint{7.575058in}{0.958332in}}{\pgfqpoint{7.571491in}{0.954765in}}%
\pgfpathcurveto{\pgfqpoint{7.567925in}{0.951199in}}{\pgfqpoint{7.565921in}{0.946361in}}{\pgfqpoint{7.565921in}{0.941317in}}%
\pgfpathcurveto{\pgfqpoint{7.565921in}{0.936274in}}{\pgfqpoint{7.567925in}{0.931436in}}{\pgfqpoint{7.571491in}{0.927870in}}%
\pgfpathcurveto{\pgfqpoint{7.575058in}{0.924303in}}{\pgfqpoint{7.579896in}{0.922299in}}{\pgfqpoint{7.584939in}{0.922299in}}%
\pgfpathclose%
\pgfusepath{fill}%
\end{pgfscope}%
\begin{pgfscope}%
\pgfpathrectangle{\pgfqpoint{6.572727in}{0.473000in}}{\pgfqpoint{4.227273in}{3.311000in}}%
\pgfusepath{clip}%
\pgfsetbuttcap%
\pgfsetroundjoin%
\definecolor{currentfill}{rgb}{0.127568,0.566949,0.550556}%
\pgfsetfillcolor{currentfill}%
\pgfsetfillopacity{0.700000}%
\pgfsetlinewidth{0.000000pt}%
\definecolor{currentstroke}{rgb}{0.000000,0.000000,0.000000}%
\pgfsetstrokecolor{currentstroke}%
\pgfsetstrokeopacity{0.700000}%
\pgfsetdash{}{0pt}%
\pgfpathmoveto{\pgfqpoint{7.637912in}{1.396067in}}%
\pgfpathcurveto{\pgfqpoint{7.642955in}{1.396067in}}{\pgfqpoint{7.647793in}{1.398071in}}{\pgfqpoint{7.651360in}{1.401637in}}%
\pgfpathcurveto{\pgfqpoint{7.654926in}{1.405203in}}{\pgfqpoint{7.656930in}{1.410041in}}{\pgfqpoint{7.656930in}{1.415085in}}%
\pgfpathcurveto{\pgfqpoint{7.656930in}{1.420129in}}{\pgfqpoint{7.654926in}{1.424966in}}{\pgfqpoint{7.651360in}{1.428533in}}%
\pgfpathcurveto{\pgfqpoint{7.647793in}{1.432099in}}{\pgfqpoint{7.642955in}{1.434103in}}{\pgfqpoint{7.637912in}{1.434103in}}%
\pgfpathcurveto{\pgfqpoint{7.632868in}{1.434103in}}{\pgfqpoint{7.628030in}{1.432099in}}{\pgfqpoint{7.624464in}{1.428533in}}%
\pgfpathcurveto{\pgfqpoint{7.620898in}{1.424966in}}{\pgfqpoint{7.618894in}{1.420129in}}{\pgfqpoint{7.618894in}{1.415085in}}%
\pgfpathcurveto{\pgfqpoint{7.618894in}{1.410041in}}{\pgfqpoint{7.620898in}{1.405203in}}{\pgfqpoint{7.624464in}{1.401637in}}%
\pgfpathcurveto{\pgfqpoint{7.628030in}{1.398071in}}{\pgfqpoint{7.632868in}{1.396067in}}{\pgfqpoint{7.637912in}{1.396067in}}%
\pgfpathclose%
\pgfusepath{fill}%
\end{pgfscope}%
\begin{pgfscope}%
\pgfpathrectangle{\pgfqpoint{6.572727in}{0.473000in}}{\pgfqpoint{4.227273in}{3.311000in}}%
\pgfusepath{clip}%
\pgfsetbuttcap%
\pgfsetroundjoin%
\definecolor{currentfill}{rgb}{0.127568,0.566949,0.550556}%
\pgfsetfillcolor{currentfill}%
\pgfsetfillopacity{0.700000}%
\pgfsetlinewidth{0.000000pt}%
\definecolor{currentstroke}{rgb}{0.000000,0.000000,0.000000}%
\pgfsetstrokecolor{currentstroke}%
\pgfsetstrokeopacity{0.700000}%
\pgfsetdash{}{0pt}%
\pgfpathmoveto{\pgfqpoint{7.696788in}{1.258995in}}%
\pgfpathcurveto{\pgfqpoint{7.701831in}{1.258995in}}{\pgfqpoint{7.706669in}{1.260998in}}{\pgfqpoint{7.710236in}{1.264565in}}%
\pgfpathcurveto{\pgfqpoint{7.713802in}{1.268131in}}{\pgfqpoint{7.715806in}{1.272969in}}{\pgfqpoint{7.715806in}{1.278013in}}%
\pgfpathcurveto{\pgfqpoint{7.715806in}{1.283056in}}{\pgfqpoint{7.713802in}{1.287894in}}{\pgfqpoint{7.710236in}{1.291461in}}%
\pgfpathcurveto{\pgfqpoint{7.706669in}{1.295027in}}{\pgfqpoint{7.701831in}{1.297031in}}{\pgfqpoint{7.696788in}{1.297031in}}%
\pgfpathcurveto{\pgfqpoint{7.691744in}{1.297031in}}{\pgfqpoint{7.686906in}{1.295027in}}{\pgfqpoint{7.683340in}{1.291461in}}%
\pgfpathcurveto{\pgfqpoint{7.679773in}{1.287894in}}{\pgfqpoint{7.677770in}{1.283056in}}{\pgfqpoint{7.677770in}{1.278013in}}%
\pgfpathcurveto{\pgfqpoint{7.677770in}{1.272969in}}{\pgfqpoint{7.679773in}{1.268131in}}{\pgfqpoint{7.683340in}{1.264565in}}%
\pgfpathcurveto{\pgfqpoint{7.686906in}{1.260998in}}{\pgfqpoint{7.691744in}{1.258995in}}{\pgfqpoint{7.696788in}{1.258995in}}%
\pgfpathclose%
\pgfusepath{fill}%
\end{pgfscope}%
\begin{pgfscope}%
\pgfpathrectangle{\pgfqpoint{6.572727in}{0.473000in}}{\pgfqpoint{4.227273in}{3.311000in}}%
\pgfusepath{clip}%
\pgfsetbuttcap%
\pgfsetroundjoin%
\definecolor{currentfill}{rgb}{0.127568,0.566949,0.550556}%
\pgfsetfillcolor{currentfill}%
\pgfsetfillopacity{0.700000}%
\pgfsetlinewidth{0.000000pt}%
\definecolor{currentstroke}{rgb}{0.000000,0.000000,0.000000}%
\pgfsetstrokecolor{currentstroke}%
\pgfsetstrokeopacity{0.700000}%
\pgfsetdash{}{0pt}%
\pgfpathmoveto{\pgfqpoint{7.882664in}{2.314971in}}%
\pgfpathcurveto{\pgfqpoint{7.887708in}{2.314971in}}{\pgfqpoint{7.892545in}{2.316975in}}{\pgfqpoint{7.896112in}{2.320541in}}%
\pgfpathcurveto{\pgfqpoint{7.899678in}{2.324108in}}{\pgfqpoint{7.901682in}{2.328946in}}{\pgfqpoint{7.901682in}{2.333989in}}%
\pgfpathcurveto{\pgfqpoint{7.901682in}{2.339033in}}{\pgfqpoint{7.899678in}{2.343871in}}{\pgfqpoint{7.896112in}{2.347437in}}%
\pgfpathcurveto{\pgfqpoint{7.892545in}{2.351004in}}{\pgfqpoint{7.887708in}{2.353007in}}{\pgfqpoint{7.882664in}{2.353007in}}%
\pgfpathcurveto{\pgfqpoint{7.877620in}{2.353007in}}{\pgfqpoint{7.872783in}{2.351004in}}{\pgfqpoint{7.869216in}{2.347437in}}%
\pgfpathcurveto{\pgfqpoint{7.865650in}{2.343871in}}{\pgfqpoint{7.863646in}{2.339033in}}{\pgfqpoint{7.863646in}{2.333989in}}%
\pgfpathcurveto{\pgfqpoint{7.863646in}{2.328946in}}{\pgfqpoint{7.865650in}{2.324108in}}{\pgfqpoint{7.869216in}{2.320541in}}%
\pgfpathcurveto{\pgfqpoint{7.872783in}{2.316975in}}{\pgfqpoint{7.877620in}{2.314971in}}{\pgfqpoint{7.882664in}{2.314971in}}%
\pgfpathclose%
\pgfusepath{fill}%
\end{pgfscope}%
\begin{pgfscope}%
\pgfpathrectangle{\pgfqpoint{6.572727in}{0.473000in}}{\pgfqpoint{4.227273in}{3.311000in}}%
\pgfusepath{clip}%
\pgfsetbuttcap%
\pgfsetroundjoin%
\definecolor{currentfill}{rgb}{0.993248,0.906157,0.143936}%
\pgfsetfillcolor{currentfill}%
\pgfsetfillopacity{0.700000}%
\pgfsetlinewidth{0.000000pt}%
\definecolor{currentstroke}{rgb}{0.000000,0.000000,0.000000}%
\pgfsetstrokecolor{currentstroke}%
\pgfsetstrokeopacity{0.700000}%
\pgfsetdash{}{0pt}%
\pgfpathmoveto{\pgfqpoint{9.492020in}{2.402647in}}%
\pgfpathcurveto{\pgfqpoint{9.497064in}{2.402647in}}{\pgfqpoint{9.501902in}{2.404651in}}{\pgfqpoint{9.505468in}{2.408217in}}%
\pgfpathcurveto{\pgfqpoint{9.509035in}{2.411784in}}{\pgfqpoint{9.511039in}{2.416621in}}{\pgfqpoint{9.511039in}{2.421665in}}%
\pgfpathcurveto{\pgfqpoint{9.511039in}{2.426709in}}{\pgfqpoint{9.509035in}{2.431547in}}{\pgfqpoint{9.505468in}{2.435113in}}%
\pgfpathcurveto{\pgfqpoint{9.501902in}{2.438679in}}{\pgfqpoint{9.497064in}{2.440683in}}{\pgfqpoint{9.492020in}{2.440683in}}%
\pgfpathcurveto{\pgfqpoint{9.486977in}{2.440683in}}{\pgfqpoint{9.482139in}{2.438679in}}{\pgfqpoint{9.478573in}{2.435113in}}%
\pgfpathcurveto{\pgfqpoint{9.475006in}{2.431547in}}{\pgfqpoint{9.473002in}{2.426709in}}{\pgfqpoint{9.473002in}{2.421665in}}%
\pgfpathcurveto{\pgfqpoint{9.473002in}{2.416621in}}{\pgfqpoint{9.475006in}{2.411784in}}{\pgfqpoint{9.478573in}{2.408217in}}%
\pgfpathcurveto{\pgfqpoint{9.482139in}{2.404651in}}{\pgfqpoint{9.486977in}{2.402647in}}{\pgfqpoint{9.492020in}{2.402647in}}%
\pgfpathclose%
\pgfusepath{fill}%
\end{pgfscope}%
\begin{pgfscope}%
\pgfpathrectangle{\pgfqpoint{6.572727in}{0.473000in}}{\pgfqpoint{4.227273in}{3.311000in}}%
\pgfusepath{clip}%
\pgfsetbuttcap%
\pgfsetroundjoin%
\definecolor{currentfill}{rgb}{0.127568,0.566949,0.550556}%
\pgfsetfillcolor{currentfill}%
\pgfsetfillopacity{0.700000}%
\pgfsetlinewidth{0.000000pt}%
\definecolor{currentstroke}{rgb}{0.000000,0.000000,0.000000}%
\pgfsetstrokecolor{currentstroke}%
\pgfsetstrokeopacity{0.700000}%
\pgfsetdash{}{0pt}%
\pgfpathmoveto{\pgfqpoint{8.117132in}{2.612915in}}%
\pgfpathcurveto{\pgfqpoint{8.122176in}{2.612915in}}{\pgfqpoint{8.127013in}{2.614919in}}{\pgfqpoint{8.130580in}{2.618485in}}%
\pgfpathcurveto{\pgfqpoint{8.134146in}{2.622052in}}{\pgfqpoint{8.136150in}{2.626890in}}{\pgfqpoint{8.136150in}{2.631933in}}%
\pgfpathcurveto{\pgfqpoint{8.136150in}{2.636977in}}{\pgfqpoint{8.134146in}{2.641815in}}{\pgfqpoint{8.130580in}{2.645381in}}%
\pgfpathcurveto{\pgfqpoint{8.127013in}{2.648948in}}{\pgfqpoint{8.122176in}{2.650951in}}{\pgfqpoint{8.117132in}{2.650951in}}%
\pgfpathcurveto{\pgfqpoint{8.112088in}{2.650951in}}{\pgfqpoint{8.107250in}{2.648948in}}{\pgfqpoint{8.103684in}{2.645381in}}%
\pgfpathcurveto{\pgfqpoint{8.100118in}{2.641815in}}{\pgfqpoint{8.098114in}{2.636977in}}{\pgfqpoint{8.098114in}{2.631933in}}%
\pgfpathcurveto{\pgfqpoint{8.098114in}{2.626890in}}{\pgfqpoint{8.100118in}{2.622052in}}{\pgfqpoint{8.103684in}{2.618485in}}%
\pgfpathcurveto{\pgfqpoint{8.107250in}{2.614919in}}{\pgfqpoint{8.112088in}{2.612915in}}{\pgfqpoint{8.117132in}{2.612915in}}%
\pgfpathclose%
\pgfusepath{fill}%
\end{pgfscope}%
\begin{pgfscope}%
\pgfpathrectangle{\pgfqpoint{6.572727in}{0.473000in}}{\pgfqpoint{4.227273in}{3.311000in}}%
\pgfusepath{clip}%
\pgfsetbuttcap%
\pgfsetroundjoin%
\definecolor{currentfill}{rgb}{0.127568,0.566949,0.550556}%
\pgfsetfillcolor{currentfill}%
\pgfsetfillopacity{0.700000}%
\pgfsetlinewidth{0.000000pt}%
\definecolor{currentstroke}{rgb}{0.000000,0.000000,0.000000}%
\pgfsetstrokecolor{currentstroke}%
\pgfsetstrokeopacity{0.700000}%
\pgfsetdash{}{0pt}%
\pgfpathmoveto{\pgfqpoint{7.589798in}{1.768605in}}%
\pgfpathcurveto{\pgfqpoint{7.594842in}{1.768605in}}{\pgfqpoint{7.599680in}{1.770609in}}{\pgfqpoint{7.603246in}{1.774175in}}%
\pgfpathcurveto{\pgfqpoint{7.606812in}{1.777742in}}{\pgfqpoint{7.608816in}{1.782579in}}{\pgfqpoint{7.608816in}{1.787623in}}%
\pgfpathcurveto{\pgfqpoint{7.608816in}{1.792667in}}{\pgfqpoint{7.606812in}{1.797505in}}{\pgfqpoint{7.603246in}{1.801071in}}%
\pgfpathcurveto{\pgfqpoint{7.599680in}{1.804637in}}{\pgfqpoint{7.594842in}{1.806641in}}{\pgfqpoint{7.589798in}{1.806641in}}%
\pgfpathcurveto{\pgfqpoint{7.584754in}{1.806641in}}{\pgfqpoint{7.579917in}{1.804637in}}{\pgfqpoint{7.576350in}{1.801071in}}%
\pgfpathcurveto{\pgfqpoint{7.572784in}{1.797505in}}{\pgfqpoint{7.570780in}{1.792667in}}{\pgfqpoint{7.570780in}{1.787623in}}%
\pgfpathcurveto{\pgfqpoint{7.570780in}{1.782579in}}{\pgfqpoint{7.572784in}{1.777742in}}{\pgfqpoint{7.576350in}{1.774175in}}%
\pgfpathcurveto{\pgfqpoint{7.579917in}{1.770609in}}{\pgfqpoint{7.584754in}{1.768605in}}{\pgfqpoint{7.589798in}{1.768605in}}%
\pgfpathclose%
\pgfusepath{fill}%
\end{pgfscope}%
\begin{pgfscope}%
\pgfpathrectangle{\pgfqpoint{6.572727in}{0.473000in}}{\pgfqpoint{4.227273in}{3.311000in}}%
\pgfusepath{clip}%
\pgfsetbuttcap%
\pgfsetroundjoin%
\definecolor{currentfill}{rgb}{0.993248,0.906157,0.143936}%
\pgfsetfillcolor{currentfill}%
\pgfsetfillopacity{0.700000}%
\pgfsetlinewidth{0.000000pt}%
\definecolor{currentstroke}{rgb}{0.000000,0.000000,0.000000}%
\pgfsetstrokecolor{currentstroke}%
\pgfsetstrokeopacity{0.700000}%
\pgfsetdash{}{0pt}%
\pgfpathmoveto{\pgfqpoint{9.617024in}{1.388287in}}%
\pgfpathcurveto{\pgfqpoint{9.622067in}{1.388287in}}{\pgfqpoint{9.626905in}{1.390291in}}{\pgfqpoint{9.630471in}{1.393857in}}%
\pgfpathcurveto{\pgfqpoint{9.634038in}{1.397424in}}{\pgfqpoint{9.636042in}{1.402261in}}{\pgfqpoint{9.636042in}{1.407305in}}%
\pgfpathcurveto{\pgfqpoint{9.636042in}{1.412349in}}{\pgfqpoint{9.634038in}{1.417187in}}{\pgfqpoint{9.630471in}{1.420753in}}%
\pgfpathcurveto{\pgfqpoint{9.626905in}{1.424319in}}{\pgfqpoint{9.622067in}{1.426323in}}{\pgfqpoint{9.617024in}{1.426323in}}%
\pgfpathcurveto{\pgfqpoint{9.611980in}{1.426323in}}{\pgfqpoint{9.607142in}{1.424319in}}{\pgfqpoint{9.603576in}{1.420753in}}%
\pgfpathcurveto{\pgfqpoint{9.600009in}{1.417187in}}{\pgfqpoint{9.598005in}{1.412349in}}{\pgfqpoint{9.598005in}{1.407305in}}%
\pgfpathcurveto{\pgfqpoint{9.598005in}{1.402261in}}{\pgfqpoint{9.600009in}{1.397424in}}{\pgfqpoint{9.603576in}{1.393857in}}%
\pgfpathcurveto{\pgfqpoint{9.607142in}{1.390291in}}{\pgfqpoint{9.611980in}{1.388287in}}{\pgfqpoint{9.617024in}{1.388287in}}%
\pgfpathclose%
\pgfusepath{fill}%
\end{pgfscope}%
\begin{pgfscope}%
\pgfpathrectangle{\pgfqpoint{6.572727in}{0.473000in}}{\pgfqpoint{4.227273in}{3.311000in}}%
\pgfusepath{clip}%
\pgfsetbuttcap%
\pgfsetroundjoin%
\definecolor{currentfill}{rgb}{0.127568,0.566949,0.550556}%
\pgfsetfillcolor{currentfill}%
\pgfsetfillopacity{0.700000}%
\pgfsetlinewidth{0.000000pt}%
\definecolor{currentstroke}{rgb}{0.000000,0.000000,0.000000}%
\pgfsetstrokecolor{currentstroke}%
\pgfsetstrokeopacity{0.700000}%
\pgfsetdash{}{0pt}%
\pgfpathmoveto{\pgfqpoint{7.555599in}{1.884723in}}%
\pgfpathcurveto{\pgfqpoint{7.560642in}{1.884723in}}{\pgfqpoint{7.565480in}{1.886727in}}{\pgfqpoint{7.569047in}{1.890293in}}%
\pgfpathcurveto{\pgfqpoint{7.572613in}{1.893860in}}{\pgfqpoint{7.574617in}{1.898698in}}{\pgfqpoint{7.574617in}{1.903741in}}%
\pgfpathcurveto{\pgfqpoint{7.574617in}{1.908785in}}{\pgfqpoint{7.572613in}{1.913623in}}{\pgfqpoint{7.569047in}{1.917189in}}%
\pgfpathcurveto{\pgfqpoint{7.565480in}{1.920756in}}{\pgfqpoint{7.560642in}{1.922759in}}{\pgfqpoint{7.555599in}{1.922759in}}%
\pgfpathcurveto{\pgfqpoint{7.550555in}{1.922759in}}{\pgfqpoint{7.545717in}{1.920756in}}{\pgfqpoint{7.542151in}{1.917189in}}%
\pgfpathcurveto{\pgfqpoint{7.538584in}{1.913623in}}{\pgfqpoint{7.536581in}{1.908785in}}{\pgfqpoint{7.536581in}{1.903741in}}%
\pgfpathcurveto{\pgfqpoint{7.536581in}{1.898698in}}{\pgfqpoint{7.538584in}{1.893860in}}{\pgfqpoint{7.542151in}{1.890293in}}%
\pgfpathcurveto{\pgfqpoint{7.545717in}{1.886727in}}{\pgfqpoint{7.550555in}{1.884723in}}{\pgfqpoint{7.555599in}{1.884723in}}%
\pgfpathclose%
\pgfusepath{fill}%
\end{pgfscope}%
\begin{pgfscope}%
\pgfpathrectangle{\pgfqpoint{6.572727in}{0.473000in}}{\pgfqpoint{4.227273in}{3.311000in}}%
\pgfusepath{clip}%
\pgfsetbuttcap%
\pgfsetroundjoin%
\definecolor{currentfill}{rgb}{0.127568,0.566949,0.550556}%
\pgfsetfillcolor{currentfill}%
\pgfsetfillopacity{0.700000}%
\pgfsetlinewidth{0.000000pt}%
\definecolor{currentstroke}{rgb}{0.000000,0.000000,0.000000}%
\pgfsetstrokecolor{currentstroke}%
\pgfsetstrokeopacity{0.700000}%
\pgfsetdash{}{0pt}%
\pgfpathmoveto{\pgfqpoint{7.765132in}{1.464496in}}%
\pgfpathcurveto{\pgfqpoint{7.770175in}{1.464496in}}{\pgfqpoint{7.775013in}{1.466500in}}{\pgfqpoint{7.778580in}{1.470067in}}%
\pgfpathcurveto{\pgfqpoint{7.782146in}{1.473633in}}{\pgfqpoint{7.784150in}{1.478471in}}{\pgfqpoint{7.784150in}{1.483515in}}%
\pgfpathcurveto{\pgfqpoint{7.784150in}{1.488558in}}{\pgfqpoint{7.782146in}{1.493396in}}{\pgfqpoint{7.778580in}{1.496962in}}%
\pgfpathcurveto{\pgfqpoint{7.775013in}{1.500529in}}{\pgfqpoint{7.770175in}{1.502533in}}{\pgfqpoint{7.765132in}{1.502533in}}%
\pgfpathcurveto{\pgfqpoint{7.760088in}{1.502533in}}{\pgfqpoint{7.755250in}{1.500529in}}{\pgfqpoint{7.751684in}{1.496962in}}%
\pgfpathcurveto{\pgfqpoint{7.748117in}{1.493396in}}{\pgfqpoint{7.746114in}{1.488558in}}{\pgfqpoint{7.746114in}{1.483515in}}%
\pgfpathcurveto{\pgfqpoint{7.746114in}{1.478471in}}{\pgfqpoint{7.748117in}{1.473633in}}{\pgfqpoint{7.751684in}{1.470067in}}%
\pgfpathcurveto{\pgfqpoint{7.755250in}{1.466500in}}{\pgfqpoint{7.760088in}{1.464496in}}{\pgfqpoint{7.765132in}{1.464496in}}%
\pgfpathclose%
\pgfusepath{fill}%
\end{pgfscope}%
\begin{pgfscope}%
\pgfpathrectangle{\pgfqpoint{6.572727in}{0.473000in}}{\pgfqpoint{4.227273in}{3.311000in}}%
\pgfusepath{clip}%
\pgfsetbuttcap%
\pgfsetroundjoin%
\definecolor{currentfill}{rgb}{0.127568,0.566949,0.550556}%
\pgfsetfillcolor{currentfill}%
\pgfsetfillopacity{0.700000}%
\pgfsetlinewidth{0.000000pt}%
\definecolor{currentstroke}{rgb}{0.000000,0.000000,0.000000}%
\pgfsetstrokecolor{currentstroke}%
\pgfsetstrokeopacity{0.700000}%
\pgfsetdash{}{0pt}%
\pgfpathmoveto{\pgfqpoint{8.390330in}{2.441692in}}%
\pgfpathcurveto{\pgfqpoint{8.395374in}{2.441692in}}{\pgfqpoint{8.400212in}{2.443696in}}{\pgfqpoint{8.403778in}{2.447262in}}%
\pgfpathcurveto{\pgfqpoint{8.407344in}{2.450829in}}{\pgfqpoint{8.409348in}{2.455666in}}{\pgfqpoint{8.409348in}{2.460710in}}%
\pgfpathcurveto{\pgfqpoint{8.409348in}{2.465754in}}{\pgfqpoint{8.407344in}{2.470592in}}{\pgfqpoint{8.403778in}{2.474158in}}%
\pgfpathcurveto{\pgfqpoint{8.400212in}{2.477724in}}{\pgfqpoint{8.395374in}{2.479728in}}{\pgfqpoint{8.390330in}{2.479728in}}%
\pgfpathcurveto{\pgfqpoint{8.385286in}{2.479728in}}{\pgfqpoint{8.380449in}{2.477724in}}{\pgfqpoint{8.376882in}{2.474158in}}%
\pgfpathcurveto{\pgfqpoint{8.373316in}{2.470592in}}{\pgfqpoint{8.371312in}{2.465754in}}{\pgfqpoint{8.371312in}{2.460710in}}%
\pgfpathcurveto{\pgfqpoint{8.371312in}{2.455666in}}{\pgfqpoint{8.373316in}{2.450829in}}{\pgfqpoint{8.376882in}{2.447262in}}%
\pgfpathcurveto{\pgfqpoint{8.380449in}{2.443696in}}{\pgfqpoint{8.385286in}{2.441692in}}{\pgfqpoint{8.390330in}{2.441692in}}%
\pgfpathclose%
\pgfusepath{fill}%
\end{pgfscope}%
\begin{pgfscope}%
\pgfpathrectangle{\pgfqpoint{6.572727in}{0.473000in}}{\pgfqpoint{4.227273in}{3.311000in}}%
\pgfusepath{clip}%
\pgfsetbuttcap%
\pgfsetroundjoin%
\definecolor{currentfill}{rgb}{0.127568,0.566949,0.550556}%
\pgfsetfillcolor{currentfill}%
\pgfsetfillopacity{0.700000}%
\pgfsetlinewidth{0.000000pt}%
\definecolor{currentstroke}{rgb}{0.000000,0.000000,0.000000}%
\pgfsetstrokecolor{currentstroke}%
\pgfsetstrokeopacity{0.700000}%
\pgfsetdash{}{0pt}%
\pgfpathmoveto{\pgfqpoint{7.505418in}{1.255786in}}%
\pgfpathcurveto{\pgfqpoint{7.510462in}{1.255786in}}{\pgfqpoint{7.515299in}{1.257790in}}{\pgfqpoint{7.518866in}{1.261357in}}%
\pgfpathcurveto{\pgfqpoint{7.522432in}{1.264923in}}{\pgfqpoint{7.524436in}{1.269761in}}{\pgfqpoint{7.524436in}{1.274804in}}%
\pgfpathcurveto{\pgfqpoint{7.524436in}{1.279848in}}{\pgfqpoint{7.522432in}{1.284686in}}{\pgfqpoint{7.518866in}{1.288252in}}%
\pgfpathcurveto{\pgfqpoint{7.515299in}{1.291819in}}{\pgfqpoint{7.510462in}{1.293823in}}{\pgfqpoint{7.505418in}{1.293823in}}%
\pgfpathcurveto{\pgfqpoint{7.500374in}{1.293823in}}{\pgfqpoint{7.495536in}{1.291819in}}{\pgfqpoint{7.491970in}{1.288252in}}%
\pgfpathcurveto{\pgfqpoint{7.488404in}{1.284686in}}{\pgfqpoint{7.486400in}{1.279848in}}{\pgfqpoint{7.486400in}{1.274804in}}%
\pgfpathcurveto{\pgfqpoint{7.486400in}{1.269761in}}{\pgfqpoint{7.488404in}{1.264923in}}{\pgfqpoint{7.491970in}{1.261357in}}%
\pgfpathcurveto{\pgfqpoint{7.495536in}{1.257790in}}{\pgfqpoint{7.500374in}{1.255786in}}{\pgfqpoint{7.505418in}{1.255786in}}%
\pgfpathclose%
\pgfusepath{fill}%
\end{pgfscope}%
\begin{pgfscope}%
\pgfpathrectangle{\pgfqpoint{6.572727in}{0.473000in}}{\pgfqpoint{4.227273in}{3.311000in}}%
\pgfusepath{clip}%
\pgfsetbuttcap%
\pgfsetroundjoin%
\definecolor{currentfill}{rgb}{0.993248,0.906157,0.143936}%
\pgfsetfillcolor{currentfill}%
\pgfsetfillopacity{0.700000}%
\pgfsetlinewidth{0.000000pt}%
\definecolor{currentstroke}{rgb}{0.000000,0.000000,0.000000}%
\pgfsetstrokecolor{currentstroke}%
\pgfsetstrokeopacity{0.700000}%
\pgfsetdash{}{0pt}%
\pgfpathmoveto{\pgfqpoint{9.308547in}{1.973408in}}%
\pgfpathcurveto{\pgfqpoint{9.313590in}{1.973408in}}{\pgfqpoint{9.318428in}{1.975412in}}{\pgfqpoint{9.321995in}{1.978978in}}%
\pgfpathcurveto{\pgfqpoint{9.325561in}{1.982545in}}{\pgfqpoint{9.327565in}{1.987383in}}{\pgfqpoint{9.327565in}{1.992426in}}%
\pgfpathcurveto{\pgfqpoint{9.327565in}{1.997470in}}{\pgfqpoint{9.325561in}{2.002308in}}{\pgfqpoint{9.321995in}{2.005874in}}%
\pgfpathcurveto{\pgfqpoint{9.318428in}{2.009440in}}{\pgfqpoint{9.313590in}{2.011444in}}{\pgfqpoint{9.308547in}{2.011444in}}%
\pgfpathcurveto{\pgfqpoint{9.303503in}{2.011444in}}{\pgfqpoint{9.298665in}{2.009440in}}{\pgfqpoint{9.295099in}{2.005874in}}%
\pgfpathcurveto{\pgfqpoint{9.291532in}{2.002308in}}{\pgfqpoint{9.289529in}{1.997470in}}{\pgfqpoint{9.289529in}{1.992426in}}%
\pgfpathcurveto{\pgfqpoint{9.289529in}{1.987383in}}{\pgfqpoint{9.291532in}{1.982545in}}{\pgfqpoint{9.295099in}{1.978978in}}%
\pgfpathcurveto{\pgfqpoint{9.298665in}{1.975412in}}{\pgfqpoint{9.303503in}{1.973408in}}{\pgfqpoint{9.308547in}{1.973408in}}%
\pgfpathclose%
\pgfusepath{fill}%
\end{pgfscope}%
\begin{pgfscope}%
\pgfpathrectangle{\pgfqpoint{6.572727in}{0.473000in}}{\pgfqpoint{4.227273in}{3.311000in}}%
\pgfusepath{clip}%
\pgfsetbuttcap%
\pgfsetroundjoin%
\definecolor{currentfill}{rgb}{0.127568,0.566949,0.550556}%
\pgfsetfillcolor{currentfill}%
\pgfsetfillopacity{0.700000}%
\pgfsetlinewidth{0.000000pt}%
\definecolor{currentstroke}{rgb}{0.000000,0.000000,0.000000}%
\pgfsetstrokecolor{currentstroke}%
\pgfsetstrokeopacity{0.700000}%
\pgfsetdash{}{0pt}%
\pgfpathmoveto{\pgfqpoint{7.873616in}{1.605979in}}%
\pgfpathcurveto{\pgfqpoint{7.878660in}{1.605979in}}{\pgfqpoint{7.883498in}{1.607983in}}{\pgfqpoint{7.887064in}{1.611549in}}%
\pgfpathcurveto{\pgfqpoint{7.890630in}{1.615115in}}{\pgfqpoint{7.892634in}{1.619953in}}{\pgfqpoint{7.892634in}{1.624997in}}%
\pgfpathcurveto{\pgfqpoint{7.892634in}{1.630040in}}{\pgfqpoint{7.890630in}{1.634878in}}{\pgfqpoint{7.887064in}{1.638445in}}%
\pgfpathcurveto{\pgfqpoint{7.883498in}{1.642011in}}{\pgfqpoint{7.878660in}{1.644015in}}{\pgfqpoint{7.873616in}{1.644015in}}%
\pgfpathcurveto{\pgfqpoint{7.868572in}{1.644015in}}{\pgfqpoint{7.863735in}{1.642011in}}{\pgfqpoint{7.860168in}{1.638445in}}%
\pgfpathcurveto{\pgfqpoint{7.856602in}{1.634878in}}{\pgfqpoint{7.854598in}{1.630040in}}{\pgfqpoint{7.854598in}{1.624997in}}%
\pgfpathcurveto{\pgfqpoint{7.854598in}{1.619953in}}{\pgfqpoint{7.856602in}{1.615115in}}{\pgfqpoint{7.860168in}{1.611549in}}%
\pgfpathcurveto{\pgfqpoint{7.863735in}{1.607983in}}{\pgfqpoint{7.868572in}{1.605979in}}{\pgfqpoint{7.873616in}{1.605979in}}%
\pgfpathclose%
\pgfusepath{fill}%
\end{pgfscope}%
\begin{pgfscope}%
\pgfpathrectangle{\pgfqpoint{6.572727in}{0.473000in}}{\pgfqpoint{4.227273in}{3.311000in}}%
\pgfusepath{clip}%
\pgfsetbuttcap%
\pgfsetroundjoin%
\definecolor{currentfill}{rgb}{0.993248,0.906157,0.143936}%
\pgfsetfillcolor{currentfill}%
\pgfsetfillopacity{0.700000}%
\pgfsetlinewidth{0.000000pt}%
\definecolor{currentstroke}{rgb}{0.000000,0.000000,0.000000}%
\pgfsetstrokecolor{currentstroke}%
\pgfsetstrokeopacity{0.700000}%
\pgfsetdash{}{0pt}%
\pgfpathmoveto{\pgfqpoint{9.908448in}{1.455429in}}%
\pgfpathcurveto{\pgfqpoint{9.913492in}{1.455429in}}{\pgfqpoint{9.918329in}{1.457432in}}{\pgfqpoint{9.921896in}{1.460999in}}%
\pgfpathcurveto{\pgfqpoint{9.925462in}{1.464565in}}{\pgfqpoint{9.927466in}{1.469403in}}{\pgfqpoint{9.927466in}{1.474447in}}%
\pgfpathcurveto{\pgfqpoint{9.927466in}{1.479490in}}{\pgfqpoint{9.925462in}{1.484328in}}{\pgfqpoint{9.921896in}{1.487895in}}%
\pgfpathcurveto{\pgfqpoint{9.918329in}{1.491461in}}{\pgfqpoint{9.913492in}{1.493465in}}{\pgfqpoint{9.908448in}{1.493465in}}%
\pgfpathcurveto{\pgfqpoint{9.903404in}{1.493465in}}{\pgfqpoint{9.898566in}{1.491461in}}{\pgfqpoint{9.895000in}{1.487895in}}%
\pgfpathcurveto{\pgfqpoint{9.891434in}{1.484328in}}{\pgfqpoint{9.889430in}{1.479490in}}{\pgfqpoint{9.889430in}{1.474447in}}%
\pgfpathcurveto{\pgfqpoint{9.889430in}{1.469403in}}{\pgfqpoint{9.891434in}{1.464565in}}{\pgfqpoint{9.895000in}{1.460999in}}%
\pgfpathcurveto{\pgfqpoint{9.898566in}{1.457432in}}{\pgfqpoint{9.903404in}{1.455429in}}{\pgfqpoint{9.908448in}{1.455429in}}%
\pgfpathclose%
\pgfusepath{fill}%
\end{pgfscope}%
\begin{pgfscope}%
\pgfpathrectangle{\pgfqpoint{6.572727in}{0.473000in}}{\pgfqpoint{4.227273in}{3.311000in}}%
\pgfusepath{clip}%
\pgfsetbuttcap%
\pgfsetroundjoin%
\definecolor{currentfill}{rgb}{0.127568,0.566949,0.550556}%
\pgfsetfillcolor{currentfill}%
\pgfsetfillopacity{0.700000}%
\pgfsetlinewidth{0.000000pt}%
\definecolor{currentstroke}{rgb}{0.000000,0.000000,0.000000}%
\pgfsetstrokecolor{currentstroke}%
\pgfsetstrokeopacity{0.700000}%
\pgfsetdash{}{0pt}%
\pgfpathmoveto{\pgfqpoint{7.982655in}{1.688191in}}%
\pgfpathcurveto{\pgfqpoint{7.987699in}{1.688191in}}{\pgfqpoint{7.992537in}{1.690195in}}{\pgfqpoint{7.996103in}{1.693762in}}%
\pgfpathcurveto{\pgfqpoint{7.999669in}{1.697328in}}{\pgfqpoint{8.001673in}{1.702166in}}{\pgfqpoint{8.001673in}{1.707209in}}%
\pgfpathcurveto{\pgfqpoint{8.001673in}{1.712253in}}{\pgfqpoint{7.999669in}{1.717091in}}{\pgfqpoint{7.996103in}{1.720657in}}%
\pgfpathcurveto{\pgfqpoint{7.992537in}{1.724224in}}{\pgfqpoint{7.987699in}{1.726228in}}{\pgfqpoint{7.982655in}{1.726228in}}%
\pgfpathcurveto{\pgfqpoint{7.977612in}{1.726228in}}{\pgfqpoint{7.972774in}{1.724224in}}{\pgfqpoint{7.969207in}{1.720657in}}%
\pgfpathcurveto{\pgfqpoint{7.965641in}{1.717091in}}{\pgfqpoint{7.963637in}{1.712253in}}{\pgfqpoint{7.963637in}{1.707209in}}%
\pgfpathcurveto{\pgfqpoint{7.963637in}{1.702166in}}{\pgfqpoint{7.965641in}{1.697328in}}{\pgfqpoint{7.969207in}{1.693762in}}%
\pgfpathcurveto{\pgfqpoint{7.972774in}{1.690195in}}{\pgfqpoint{7.977612in}{1.688191in}}{\pgfqpoint{7.982655in}{1.688191in}}%
\pgfpathclose%
\pgfusepath{fill}%
\end{pgfscope}%
\begin{pgfscope}%
\pgfpathrectangle{\pgfqpoint{6.572727in}{0.473000in}}{\pgfqpoint{4.227273in}{3.311000in}}%
\pgfusepath{clip}%
\pgfsetbuttcap%
\pgfsetroundjoin%
\definecolor{currentfill}{rgb}{0.127568,0.566949,0.550556}%
\pgfsetfillcolor{currentfill}%
\pgfsetfillopacity{0.700000}%
\pgfsetlinewidth{0.000000pt}%
\definecolor{currentstroke}{rgb}{0.000000,0.000000,0.000000}%
\pgfsetstrokecolor{currentstroke}%
\pgfsetstrokeopacity{0.700000}%
\pgfsetdash{}{0pt}%
\pgfpathmoveto{\pgfqpoint{7.948766in}{2.433244in}}%
\pgfpathcurveto{\pgfqpoint{7.953810in}{2.433244in}}{\pgfqpoint{7.958647in}{2.435248in}}{\pgfqpoint{7.962214in}{2.438815in}}%
\pgfpathcurveto{\pgfqpoint{7.965780in}{2.442381in}}{\pgfqpoint{7.967784in}{2.447219in}}{\pgfqpoint{7.967784in}{2.452262in}}%
\pgfpathcurveto{\pgfqpoint{7.967784in}{2.457306in}}{\pgfqpoint{7.965780in}{2.462144in}}{\pgfqpoint{7.962214in}{2.465710in}}%
\pgfpathcurveto{\pgfqpoint{7.958647in}{2.469277in}}{\pgfqpoint{7.953810in}{2.471281in}}{\pgfqpoint{7.948766in}{2.471281in}}%
\pgfpathcurveto{\pgfqpoint{7.943722in}{2.471281in}}{\pgfqpoint{7.938885in}{2.469277in}}{\pgfqpoint{7.935318in}{2.465710in}}%
\pgfpathcurveto{\pgfqpoint{7.931752in}{2.462144in}}{\pgfqpoint{7.929748in}{2.457306in}}{\pgfqpoint{7.929748in}{2.452262in}}%
\pgfpathcurveto{\pgfqpoint{7.929748in}{2.447219in}}{\pgfqpoint{7.931752in}{2.442381in}}{\pgfqpoint{7.935318in}{2.438815in}}%
\pgfpathcurveto{\pgfqpoint{7.938885in}{2.435248in}}{\pgfqpoint{7.943722in}{2.433244in}}{\pgfqpoint{7.948766in}{2.433244in}}%
\pgfpathclose%
\pgfusepath{fill}%
\end{pgfscope}%
\begin{pgfscope}%
\pgfpathrectangle{\pgfqpoint{6.572727in}{0.473000in}}{\pgfqpoint{4.227273in}{3.311000in}}%
\pgfusepath{clip}%
\pgfsetbuttcap%
\pgfsetroundjoin%
\definecolor{currentfill}{rgb}{0.127568,0.566949,0.550556}%
\pgfsetfillcolor{currentfill}%
\pgfsetfillopacity{0.700000}%
\pgfsetlinewidth{0.000000pt}%
\definecolor{currentstroke}{rgb}{0.000000,0.000000,0.000000}%
\pgfsetstrokecolor{currentstroke}%
\pgfsetstrokeopacity{0.700000}%
\pgfsetdash{}{0pt}%
\pgfpathmoveto{\pgfqpoint{7.267644in}{3.010229in}}%
\pgfpathcurveto{\pgfqpoint{7.272687in}{3.010229in}}{\pgfqpoint{7.277525in}{3.012233in}}{\pgfqpoint{7.281091in}{3.015800in}}%
\pgfpathcurveto{\pgfqpoint{7.284658in}{3.019366in}}{\pgfqpoint{7.286662in}{3.024204in}}{\pgfqpoint{7.286662in}{3.029248in}}%
\pgfpathcurveto{\pgfqpoint{7.286662in}{3.034291in}}{\pgfqpoint{7.284658in}{3.039129in}}{\pgfqpoint{7.281091in}{3.042695in}}%
\pgfpathcurveto{\pgfqpoint{7.277525in}{3.046262in}}{\pgfqpoint{7.272687in}{3.048266in}}{\pgfqpoint{7.267644in}{3.048266in}}%
\pgfpathcurveto{\pgfqpoint{7.262600in}{3.048266in}}{\pgfqpoint{7.257762in}{3.046262in}}{\pgfqpoint{7.254196in}{3.042695in}}%
\pgfpathcurveto{\pgfqpoint{7.250629in}{3.039129in}}{\pgfqpoint{7.248625in}{3.034291in}}{\pgfqpoint{7.248625in}{3.029248in}}%
\pgfpathcurveto{\pgfqpoint{7.248625in}{3.024204in}}{\pgfqpoint{7.250629in}{3.019366in}}{\pgfqpoint{7.254196in}{3.015800in}}%
\pgfpathcurveto{\pgfqpoint{7.257762in}{3.012233in}}{\pgfqpoint{7.262600in}{3.010229in}}{\pgfqpoint{7.267644in}{3.010229in}}%
\pgfpathclose%
\pgfusepath{fill}%
\end{pgfscope}%
\begin{pgfscope}%
\pgfpathrectangle{\pgfqpoint{6.572727in}{0.473000in}}{\pgfqpoint{4.227273in}{3.311000in}}%
\pgfusepath{clip}%
\pgfsetbuttcap%
\pgfsetroundjoin%
\definecolor{currentfill}{rgb}{0.993248,0.906157,0.143936}%
\pgfsetfillcolor{currentfill}%
\pgfsetfillopacity{0.700000}%
\pgfsetlinewidth{0.000000pt}%
\definecolor{currentstroke}{rgb}{0.000000,0.000000,0.000000}%
\pgfsetstrokecolor{currentstroke}%
\pgfsetstrokeopacity{0.700000}%
\pgfsetdash{}{0pt}%
\pgfpathmoveto{\pgfqpoint{9.764049in}{1.890603in}}%
\pgfpathcurveto{\pgfqpoint{9.769093in}{1.890603in}}{\pgfqpoint{9.773931in}{1.892607in}}{\pgfqpoint{9.777497in}{1.896173in}}%
\pgfpathcurveto{\pgfqpoint{9.781064in}{1.899740in}}{\pgfqpoint{9.783067in}{1.904577in}}{\pgfqpoint{9.783067in}{1.909621in}}%
\pgfpathcurveto{\pgfqpoint{9.783067in}{1.914665in}}{\pgfqpoint{9.781064in}{1.919503in}}{\pgfqpoint{9.777497in}{1.923069in}}%
\pgfpathcurveto{\pgfqpoint{9.773931in}{1.926635in}}{\pgfqpoint{9.769093in}{1.928639in}}{\pgfqpoint{9.764049in}{1.928639in}}%
\pgfpathcurveto{\pgfqpoint{9.759006in}{1.928639in}}{\pgfqpoint{9.754168in}{1.926635in}}{\pgfqpoint{9.750601in}{1.923069in}}%
\pgfpathcurveto{\pgfqpoint{9.747035in}{1.919503in}}{\pgfqpoint{9.745031in}{1.914665in}}{\pgfqpoint{9.745031in}{1.909621in}}%
\pgfpathcurveto{\pgfqpoint{9.745031in}{1.904577in}}{\pgfqpoint{9.747035in}{1.899740in}}{\pgfqpoint{9.750601in}{1.896173in}}%
\pgfpathcurveto{\pgfqpoint{9.754168in}{1.892607in}}{\pgfqpoint{9.759006in}{1.890603in}}{\pgfqpoint{9.764049in}{1.890603in}}%
\pgfpathclose%
\pgfusepath{fill}%
\end{pgfscope}%
\begin{pgfscope}%
\pgfpathrectangle{\pgfqpoint{6.572727in}{0.473000in}}{\pgfqpoint{4.227273in}{3.311000in}}%
\pgfusepath{clip}%
\pgfsetbuttcap%
\pgfsetroundjoin%
\definecolor{currentfill}{rgb}{0.127568,0.566949,0.550556}%
\pgfsetfillcolor{currentfill}%
\pgfsetfillopacity{0.700000}%
\pgfsetlinewidth{0.000000pt}%
\definecolor{currentstroke}{rgb}{0.000000,0.000000,0.000000}%
\pgfsetstrokecolor{currentstroke}%
\pgfsetstrokeopacity{0.700000}%
\pgfsetdash{}{0pt}%
\pgfpathmoveto{\pgfqpoint{7.877260in}{2.290570in}}%
\pgfpathcurveto{\pgfqpoint{7.882304in}{2.290570in}}{\pgfqpoint{7.887142in}{2.292574in}}{\pgfqpoint{7.890708in}{2.296140in}}%
\pgfpathcurveto{\pgfqpoint{7.894275in}{2.299707in}}{\pgfqpoint{7.896279in}{2.304544in}}{\pgfqpoint{7.896279in}{2.309588in}}%
\pgfpathcurveto{\pgfqpoint{7.896279in}{2.314632in}}{\pgfqpoint{7.894275in}{2.319470in}}{\pgfqpoint{7.890708in}{2.323036in}}%
\pgfpathcurveto{\pgfqpoint{7.887142in}{2.326602in}}{\pgfqpoint{7.882304in}{2.328606in}}{\pgfqpoint{7.877260in}{2.328606in}}%
\pgfpathcurveto{\pgfqpoint{7.872217in}{2.328606in}}{\pgfqpoint{7.867379in}{2.326602in}}{\pgfqpoint{7.863813in}{2.323036in}}%
\pgfpathcurveto{\pgfqpoint{7.860246in}{2.319470in}}{\pgfqpoint{7.858242in}{2.314632in}}{\pgfqpoint{7.858242in}{2.309588in}}%
\pgfpathcurveto{\pgfqpoint{7.858242in}{2.304544in}}{\pgfqpoint{7.860246in}{2.299707in}}{\pgfqpoint{7.863813in}{2.296140in}}%
\pgfpathcurveto{\pgfqpoint{7.867379in}{2.292574in}}{\pgfqpoint{7.872217in}{2.290570in}}{\pgfqpoint{7.877260in}{2.290570in}}%
\pgfpathclose%
\pgfusepath{fill}%
\end{pgfscope}%
\begin{pgfscope}%
\pgfpathrectangle{\pgfqpoint{6.572727in}{0.473000in}}{\pgfqpoint{4.227273in}{3.311000in}}%
\pgfusepath{clip}%
\pgfsetbuttcap%
\pgfsetroundjoin%
\definecolor{currentfill}{rgb}{0.127568,0.566949,0.550556}%
\pgfsetfillcolor{currentfill}%
\pgfsetfillopacity{0.700000}%
\pgfsetlinewidth{0.000000pt}%
\definecolor{currentstroke}{rgb}{0.000000,0.000000,0.000000}%
\pgfsetstrokecolor{currentstroke}%
\pgfsetstrokeopacity{0.700000}%
\pgfsetdash{}{0pt}%
\pgfpathmoveto{\pgfqpoint{8.057528in}{2.660822in}}%
\pgfpathcurveto{\pgfqpoint{8.062571in}{2.660822in}}{\pgfqpoint{8.067409in}{2.662825in}}{\pgfqpoint{8.070976in}{2.666392in}}%
\pgfpathcurveto{\pgfqpoint{8.074542in}{2.669958in}}{\pgfqpoint{8.076546in}{2.674796in}}{\pgfqpoint{8.076546in}{2.679840in}}%
\pgfpathcurveto{\pgfqpoint{8.076546in}{2.684883in}}{\pgfqpoint{8.074542in}{2.689721in}}{\pgfqpoint{8.070976in}{2.693288in}}%
\pgfpathcurveto{\pgfqpoint{8.067409in}{2.696854in}}{\pgfqpoint{8.062571in}{2.698858in}}{\pgfqpoint{8.057528in}{2.698858in}}%
\pgfpathcurveto{\pgfqpoint{8.052484in}{2.698858in}}{\pgfqpoint{8.047646in}{2.696854in}}{\pgfqpoint{8.044080in}{2.693288in}}%
\pgfpathcurveto{\pgfqpoint{8.040513in}{2.689721in}}{\pgfqpoint{8.038510in}{2.684883in}}{\pgfqpoint{8.038510in}{2.679840in}}%
\pgfpathcurveto{\pgfqpoint{8.038510in}{2.674796in}}{\pgfqpoint{8.040513in}{2.669958in}}{\pgfqpoint{8.044080in}{2.666392in}}%
\pgfpathcurveto{\pgfqpoint{8.047646in}{2.662825in}}{\pgfqpoint{8.052484in}{2.660822in}}{\pgfqpoint{8.057528in}{2.660822in}}%
\pgfpathclose%
\pgfusepath{fill}%
\end{pgfscope}%
\begin{pgfscope}%
\pgfpathrectangle{\pgfqpoint{6.572727in}{0.473000in}}{\pgfqpoint{4.227273in}{3.311000in}}%
\pgfusepath{clip}%
\pgfsetbuttcap%
\pgfsetroundjoin%
\definecolor{currentfill}{rgb}{0.993248,0.906157,0.143936}%
\pgfsetfillcolor{currentfill}%
\pgfsetfillopacity{0.700000}%
\pgfsetlinewidth{0.000000pt}%
\definecolor{currentstroke}{rgb}{0.000000,0.000000,0.000000}%
\pgfsetstrokecolor{currentstroke}%
\pgfsetstrokeopacity{0.700000}%
\pgfsetdash{}{0pt}%
\pgfpathmoveto{\pgfqpoint{9.430775in}{1.901732in}}%
\pgfpathcurveto{\pgfqpoint{9.435819in}{1.901732in}}{\pgfqpoint{9.440656in}{1.903736in}}{\pgfqpoint{9.444223in}{1.907302in}}%
\pgfpathcurveto{\pgfqpoint{9.447789in}{1.910869in}}{\pgfqpoint{9.449793in}{1.915707in}}{\pgfqpoint{9.449793in}{1.920750in}}%
\pgfpathcurveto{\pgfqpoint{9.449793in}{1.925794in}}{\pgfqpoint{9.447789in}{1.930632in}}{\pgfqpoint{9.444223in}{1.934198in}}%
\pgfpathcurveto{\pgfqpoint{9.440656in}{1.937765in}}{\pgfqpoint{9.435819in}{1.939768in}}{\pgfqpoint{9.430775in}{1.939768in}}%
\pgfpathcurveto{\pgfqpoint{9.425731in}{1.939768in}}{\pgfqpoint{9.420894in}{1.937765in}}{\pgfqpoint{9.417327in}{1.934198in}}%
\pgfpathcurveto{\pgfqpoint{9.413761in}{1.930632in}}{\pgfqpoint{9.411757in}{1.925794in}}{\pgfqpoint{9.411757in}{1.920750in}}%
\pgfpathcurveto{\pgfqpoint{9.411757in}{1.915707in}}{\pgfqpoint{9.413761in}{1.910869in}}{\pgfqpoint{9.417327in}{1.907302in}}%
\pgfpathcurveto{\pgfqpoint{9.420894in}{1.903736in}}{\pgfqpoint{9.425731in}{1.901732in}}{\pgfqpoint{9.430775in}{1.901732in}}%
\pgfpathclose%
\pgfusepath{fill}%
\end{pgfscope}%
\begin{pgfscope}%
\pgfpathrectangle{\pgfqpoint{6.572727in}{0.473000in}}{\pgfqpoint{4.227273in}{3.311000in}}%
\pgfusepath{clip}%
\pgfsetbuttcap%
\pgfsetroundjoin%
\definecolor{currentfill}{rgb}{0.127568,0.566949,0.550556}%
\pgfsetfillcolor{currentfill}%
\pgfsetfillopacity{0.700000}%
\pgfsetlinewidth{0.000000pt}%
\definecolor{currentstroke}{rgb}{0.000000,0.000000,0.000000}%
\pgfsetstrokecolor{currentstroke}%
\pgfsetstrokeopacity{0.700000}%
\pgfsetdash{}{0pt}%
\pgfpathmoveto{\pgfqpoint{7.415492in}{2.115379in}}%
\pgfpathcurveto{\pgfqpoint{7.420535in}{2.115379in}}{\pgfqpoint{7.425373in}{2.117383in}}{\pgfqpoint{7.428940in}{2.120949in}}%
\pgfpathcurveto{\pgfqpoint{7.432506in}{2.124516in}}{\pgfqpoint{7.434510in}{2.129354in}}{\pgfqpoint{7.434510in}{2.134397in}}%
\pgfpathcurveto{\pgfqpoint{7.434510in}{2.139441in}}{\pgfqpoint{7.432506in}{2.144279in}}{\pgfqpoint{7.428940in}{2.147845in}}%
\pgfpathcurveto{\pgfqpoint{7.425373in}{2.151412in}}{\pgfqpoint{7.420535in}{2.153415in}}{\pgfqpoint{7.415492in}{2.153415in}}%
\pgfpathcurveto{\pgfqpoint{7.410448in}{2.153415in}}{\pgfqpoint{7.405610in}{2.151412in}}{\pgfqpoint{7.402044in}{2.147845in}}%
\pgfpathcurveto{\pgfqpoint{7.398477in}{2.144279in}}{\pgfqpoint{7.396474in}{2.139441in}}{\pgfqpoint{7.396474in}{2.134397in}}%
\pgfpathcurveto{\pgfqpoint{7.396474in}{2.129354in}}{\pgfqpoint{7.398477in}{2.124516in}}{\pgfqpoint{7.402044in}{2.120949in}}%
\pgfpathcurveto{\pgfqpoint{7.405610in}{2.117383in}}{\pgfqpoint{7.410448in}{2.115379in}}{\pgfqpoint{7.415492in}{2.115379in}}%
\pgfpathclose%
\pgfusepath{fill}%
\end{pgfscope}%
\begin{pgfscope}%
\pgfpathrectangle{\pgfqpoint{6.572727in}{0.473000in}}{\pgfqpoint{4.227273in}{3.311000in}}%
\pgfusepath{clip}%
\pgfsetbuttcap%
\pgfsetroundjoin%
\definecolor{currentfill}{rgb}{0.127568,0.566949,0.550556}%
\pgfsetfillcolor{currentfill}%
\pgfsetfillopacity{0.700000}%
\pgfsetlinewidth{0.000000pt}%
\definecolor{currentstroke}{rgb}{0.000000,0.000000,0.000000}%
\pgfsetstrokecolor{currentstroke}%
\pgfsetstrokeopacity{0.700000}%
\pgfsetdash{}{0pt}%
\pgfpathmoveto{\pgfqpoint{8.174750in}{2.675283in}}%
\pgfpathcurveto{\pgfqpoint{8.179794in}{2.675283in}}{\pgfqpoint{8.184632in}{2.677287in}}{\pgfqpoint{8.188198in}{2.680853in}}%
\pgfpathcurveto{\pgfqpoint{8.191764in}{2.684419in}}{\pgfqpoint{8.193768in}{2.689257in}}{\pgfqpoint{8.193768in}{2.694301in}}%
\pgfpathcurveto{\pgfqpoint{8.193768in}{2.699345in}}{\pgfqpoint{8.191764in}{2.704182in}}{\pgfqpoint{8.188198in}{2.707749in}}%
\pgfpathcurveto{\pgfqpoint{8.184632in}{2.711315in}}{\pgfqpoint{8.179794in}{2.713319in}}{\pgfqpoint{8.174750in}{2.713319in}}%
\pgfpathcurveto{\pgfqpoint{8.169706in}{2.713319in}}{\pgfqpoint{8.164869in}{2.711315in}}{\pgfqpoint{8.161302in}{2.707749in}}%
\pgfpathcurveto{\pgfqpoint{8.157736in}{2.704182in}}{\pgfqpoint{8.155732in}{2.699345in}}{\pgfqpoint{8.155732in}{2.694301in}}%
\pgfpathcurveto{\pgfqpoint{8.155732in}{2.689257in}}{\pgfqpoint{8.157736in}{2.684419in}}{\pgfqpoint{8.161302in}{2.680853in}}%
\pgfpathcurveto{\pgfqpoint{8.164869in}{2.677287in}}{\pgfqpoint{8.169706in}{2.675283in}}{\pgfqpoint{8.174750in}{2.675283in}}%
\pgfpathclose%
\pgfusepath{fill}%
\end{pgfscope}%
\begin{pgfscope}%
\pgfpathrectangle{\pgfqpoint{6.572727in}{0.473000in}}{\pgfqpoint{4.227273in}{3.311000in}}%
\pgfusepath{clip}%
\pgfsetbuttcap%
\pgfsetroundjoin%
\definecolor{currentfill}{rgb}{0.127568,0.566949,0.550556}%
\pgfsetfillcolor{currentfill}%
\pgfsetfillopacity{0.700000}%
\pgfsetlinewidth{0.000000pt}%
\definecolor{currentstroke}{rgb}{0.000000,0.000000,0.000000}%
\pgfsetstrokecolor{currentstroke}%
\pgfsetstrokeopacity{0.700000}%
\pgfsetdash{}{0pt}%
\pgfpathmoveto{\pgfqpoint{7.873512in}{1.429079in}}%
\pgfpathcurveto{\pgfqpoint{7.878555in}{1.429079in}}{\pgfqpoint{7.883393in}{1.431083in}}{\pgfqpoint{7.886959in}{1.434649in}}%
\pgfpathcurveto{\pgfqpoint{7.890526in}{1.438216in}}{\pgfqpoint{7.892530in}{1.443054in}}{\pgfqpoint{7.892530in}{1.448097in}}%
\pgfpathcurveto{\pgfqpoint{7.892530in}{1.453141in}}{\pgfqpoint{7.890526in}{1.457979in}}{\pgfqpoint{7.886959in}{1.461545in}}%
\pgfpathcurveto{\pgfqpoint{7.883393in}{1.465111in}}{\pgfqpoint{7.878555in}{1.467115in}}{\pgfqpoint{7.873512in}{1.467115in}}%
\pgfpathcurveto{\pgfqpoint{7.868468in}{1.467115in}}{\pgfqpoint{7.863630in}{1.465111in}}{\pgfqpoint{7.860064in}{1.461545in}}%
\pgfpathcurveto{\pgfqpoint{7.856497in}{1.457979in}}{\pgfqpoint{7.854493in}{1.453141in}}{\pgfqpoint{7.854493in}{1.448097in}}%
\pgfpathcurveto{\pgfqpoint{7.854493in}{1.443054in}}{\pgfqpoint{7.856497in}{1.438216in}}{\pgfqpoint{7.860064in}{1.434649in}}%
\pgfpathcurveto{\pgfqpoint{7.863630in}{1.431083in}}{\pgfqpoint{7.868468in}{1.429079in}}{\pgfqpoint{7.873512in}{1.429079in}}%
\pgfpathclose%
\pgfusepath{fill}%
\end{pgfscope}%
\begin{pgfscope}%
\pgfpathrectangle{\pgfqpoint{6.572727in}{0.473000in}}{\pgfqpoint{4.227273in}{3.311000in}}%
\pgfusepath{clip}%
\pgfsetbuttcap%
\pgfsetroundjoin%
\definecolor{currentfill}{rgb}{0.127568,0.566949,0.550556}%
\pgfsetfillcolor{currentfill}%
\pgfsetfillopacity{0.700000}%
\pgfsetlinewidth{0.000000pt}%
\definecolor{currentstroke}{rgb}{0.000000,0.000000,0.000000}%
\pgfsetstrokecolor{currentstroke}%
\pgfsetstrokeopacity{0.700000}%
\pgfsetdash{}{0pt}%
\pgfpathmoveto{\pgfqpoint{8.491858in}{2.943700in}}%
\pgfpathcurveto{\pgfqpoint{8.496901in}{2.943700in}}{\pgfqpoint{8.501739in}{2.945704in}}{\pgfqpoint{8.505306in}{2.949271in}}%
\pgfpathcurveto{\pgfqpoint{8.508872in}{2.952837in}}{\pgfqpoint{8.510876in}{2.957675in}}{\pgfqpoint{8.510876in}{2.962719in}}%
\pgfpathcurveto{\pgfqpoint{8.510876in}{2.967762in}}{\pgfqpoint{8.508872in}{2.972600in}}{\pgfqpoint{8.505306in}{2.976166in}}%
\pgfpathcurveto{\pgfqpoint{8.501739in}{2.979733in}}{\pgfqpoint{8.496901in}{2.981737in}}{\pgfqpoint{8.491858in}{2.981737in}}%
\pgfpathcurveto{\pgfqpoint{8.486814in}{2.981737in}}{\pgfqpoint{8.481976in}{2.979733in}}{\pgfqpoint{8.478410in}{2.976166in}}%
\pgfpathcurveto{\pgfqpoint{8.474843in}{2.972600in}}{\pgfqpoint{8.472840in}{2.967762in}}{\pgfqpoint{8.472840in}{2.962719in}}%
\pgfpathcurveto{\pgfqpoint{8.472840in}{2.957675in}}{\pgfqpoint{8.474843in}{2.952837in}}{\pgfqpoint{8.478410in}{2.949271in}}%
\pgfpathcurveto{\pgfqpoint{8.481976in}{2.945704in}}{\pgfqpoint{8.486814in}{2.943700in}}{\pgfqpoint{8.491858in}{2.943700in}}%
\pgfpathclose%
\pgfusepath{fill}%
\end{pgfscope}%
\begin{pgfscope}%
\pgfpathrectangle{\pgfqpoint{6.572727in}{0.473000in}}{\pgfqpoint{4.227273in}{3.311000in}}%
\pgfusepath{clip}%
\pgfsetbuttcap%
\pgfsetroundjoin%
\definecolor{currentfill}{rgb}{0.127568,0.566949,0.550556}%
\pgfsetfillcolor{currentfill}%
\pgfsetfillopacity{0.700000}%
\pgfsetlinewidth{0.000000pt}%
\definecolor{currentstroke}{rgb}{0.000000,0.000000,0.000000}%
\pgfsetstrokecolor{currentstroke}%
\pgfsetstrokeopacity{0.700000}%
\pgfsetdash{}{0pt}%
\pgfpathmoveto{\pgfqpoint{7.709426in}{1.325937in}}%
\pgfpathcurveto{\pgfqpoint{7.714470in}{1.325937in}}{\pgfqpoint{7.719308in}{1.327941in}}{\pgfqpoint{7.722874in}{1.331508in}}%
\pgfpathcurveto{\pgfqpoint{7.726440in}{1.335074in}}{\pgfqpoint{7.728444in}{1.339912in}}{\pgfqpoint{7.728444in}{1.344956in}}%
\pgfpathcurveto{\pgfqpoint{7.728444in}{1.349999in}}{\pgfqpoint{7.726440in}{1.354837in}}{\pgfqpoint{7.722874in}{1.358403in}}%
\pgfpathcurveto{\pgfqpoint{7.719308in}{1.361970in}}{\pgfqpoint{7.714470in}{1.363974in}}{\pgfqpoint{7.709426in}{1.363974in}}%
\pgfpathcurveto{\pgfqpoint{7.704382in}{1.363974in}}{\pgfqpoint{7.699545in}{1.361970in}}{\pgfqpoint{7.695978in}{1.358403in}}%
\pgfpathcurveto{\pgfqpoint{7.692412in}{1.354837in}}{\pgfqpoint{7.690408in}{1.349999in}}{\pgfqpoint{7.690408in}{1.344956in}}%
\pgfpathcurveto{\pgfqpoint{7.690408in}{1.339912in}}{\pgfqpoint{7.692412in}{1.335074in}}{\pgfqpoint{7.695978in}{1.331508in}}%
\pgfpathcurveto{\pgfqpoint{7.699545in}{1.327941in}}{\pgfqpoint{7.704382in}{1.325937in}}{\pgfqpoint{7.709426in}{1.325937in}}%
\pgfpathclose%
\pgfusepath{fill}%
\end{pgfscope}%
\begin{pgfscope}%
\pgfpathrectangle{\pgfqpoint{6.572727in}{0.473000in}}{\pgfqpoint{4.227273in}{3.311000in}}%
\pgfusepath{clip}%
\pgfsetbuttcap%
\pgfsetroundjoin%
\definecolor{currentfill}{rgb}{0.127568,0.566949,0.550556}%
\pgfsetfillcolor{currentfill}%
\pgfsetfillopacity{0.700000}%
\pgfsetlinewidth{0.000000pt}%
\definecolor{currentstroke}{rgb}{0.000000,0.000000,0.000000}%
\pgfsetstrokecolor{currentstroke}%
\pgfsetstrokeopacity{0.700000}%
\pgfsetdash{}{0pt}%
\pgfpathmoveto{\pgfqpoint{8.343571in}{1.160980in}}%
\pgfpathcurveto{\pgfqpoint{8.348614in}{1.160980in}}{\pgfqpoint{8.353452in}{1.162984in}}{\pgfqpoint{8.357018in}{1.166550in}}%
\pgfpathcurveto{\pgfqpoint{8.360585in}{1.170116in}}{\pgfqpoint{8.362589in}{1.174954in}}{\pgfqpoint{8.362589in}{1.179998in}}%
\pgfpathcurveto{\pgfqpoint{8.362589in}{1.185042in}}{\pgfqpoint{8.360585in}{1.189879in}}{\pgfqpoint{8.357018in}{1.193446in}}%
\pgfpathcurveto{\pgfqpoint{8.353452in}{1.197012in}}{\pgfqpoint{8.348614in}{1.199016in}}{\pgfqpoint{8.343571in}{1.199016in}}%
\pgfpathcurveto{\pgfqpoint{8.338527in}{1.199016in}}{\pgfqpoint{8.333689in}{1.197012in}}{\pgfqpoint{8.330123in}{1.193446in}}%
\pgfpathcurveto{\pgfqpoint{8.326556in}{1.189879in}}{\pgfqpoint{8.324552in}{1.185042in}}{\pgfqpoint{8.324552in}{1.179998in}}%
\pgfpathcurveto{\pgfqpoint{8.324552in}{1.174954in}}{\pgfqpoint{8.326556in}{1.170116in}}{\pgfqpoint{8.330123in}{1.166550in}}%
\pgfpathcurveto{\pgfqpoint{8.333689in}{1.162984in}}{\pgfqpoint{8.338527in}{1.160980in}}{\pgfqpoint{8.343571in}{1.160980in}}%
\pgfpathclose%
\pgfusepath{fill}%
\end{pgfscope}%
\begin{pgfscope}%
\pgfpathrectangle{\pgfqpoint{6.572727in}{0.473000in}}{\pgfqpoint{4.227273in}{3.311000in}}%
\pgfusepath{clip}%
\pgfsetbuttcap%
\pgfsetroundjoin%
\definecolor{currentfill}{rgb}{0.127568,0.566949,0.550556}%
\pgfsetfillcolor{currentfill}%
\pgfsetfillopacity{0.700000}%
\pgfsetlinewidth{0.000000pt}%
\definecolor{currentstroke}{rgb}{0.000000,0.000000,0.000000}%
\pgfsetstrokecolor{currentstroke}%
\pgfsetstrokeopacity{0.700000}%
\pgfsetdash{}{0pt}%
\pgfpathmoveto{\pgfqpoint{7.792378in}{1.354147in}}%
\pgfpathcurveto{\pgfqpoint{7.797421in}{1.354147in}}{\pgfqpoint{7.802259in}{1.356150in}}{\pgfqpoint{7.805825in}{1.359717in}}%
\pgfpathcurveto{\pgfqpoint{7.809392in}{1.363283in}}{\pgfqpoint{7.811396in}{1.368121in}}{\pgfqpoint{7.811396in}{1.373165in}}%
\pgfpathcurveto{\pgfqpoint{7.811396in}{1.378208in}}{\pgfqpoint{7.809392in}{1.383046in}}{\pgfqpoint{7.805825in}{1.386613in}}%
\pgfpathcurveto{\pgfqpoint{7.802259in}{1.390179in}}{\pgfqpoint{7.797421in}{1.392183in}}{\pgfqpoint{7.792378in}{1.392183in}}%
\pgfpathcurveto{\pgfqpoint{7.787334in}{1.392183in}}{\pgfqpoint{7.782496in}{1.390179in}}{\pgfqpoint{7.778930in}{1.386613in}}%
\pgfpathcurveto{\pgfqpoint{7.775363in}{1.383046in}}{\pgfqpoint{7.773359in}{1.378208in}}{\pgfqpoint{7.773359in}{1.373165in}}%
\pgfpathcurveto{\pgfqpoint{7.773359in}{1.368121in}}{\pgfqpoint{7.775363in}{1.363283in}}{\pgfqpoint{7.778930in}{1.359717in}}%
\pgfpathcurveto{\pgfqpoint{7.782496in}{1.356150in}}{\pgfqpoint{7.787334in}{1.354147in}}{\pgfqpoint{7.792378in}{1.354147in}}%
\pgfpathclose%
\pgfusepath{fill}%
\end{pgfscope}%
\begin{pgfscope}%
\pgfpathrectangle{\pgfqpoint{6.572727in}{0.473000in}}{\pgfqpoint{4.227273in}{3.311000in}}%
\pgfusepath{clip}%
\pgfsetbuttcap%
\pgfsetroundjoin%
\definecolor{currentfill}{rgb}{0.127568,0.566949,0.550556}%
\pgfsetfillcolor{currentfill}%
\pgfsetfillopacity{0.700000}%
\pgfsetlinewidth{0.000000pt}%
\definecolor{currentstroke}{rgb}{0.000000,0.000000,0.000000}%
\pgfsetstrokecolor{currentstroke}%
\pgfsetstrokeopacity{0.700000}%
\pgfsetdash{}{0pt}%
\pgfpathmoveto{\pgfqpoint{7.940147in}{1.635300in}}%
\pgfpathcurveto{\pgfqpoint{7.945191in}{1.635300in}}{\pgfqpoint{7.950028in}{1.637304in}}{\pgfqpoint{7.953595in}{1.640871in}}%
\pgfpathcurveto{\pgfqpoint{7.957161in}{1.644437in}}{\pgfqpoint{7.959165in}{1.649275in}}{\pgfqpoint{7.959165in}{1.654318in}}%
\pgfpathcurveto{\pgfqpoint{7.959165in}{1.659362in}}{\pgfqpoint{7.957161in}{1.664200in}}{\pgfqpoint{7.953595in}{1.667766in}}%
\pgfpathcurveto{\pgfqpoint{7.950028in}{1.671333in}}{\pgfqpoint{7.945191in}{1.673337in}}{\pgfqpoint{7.940147in}{1.673337in}}%
\pgfpathcurveto{\pgfqpoint{7.935103in}{1.673337in}}{\pgfqpoint{7.930265in}{1.671333in}}{\pgfqpoint{7.926699in}{1.667766in}}%
\pgfpathcurveto{\pgfqpoint{7.923133in}{1.664200in}}{\pgfqpoint{7.921129in}{1.659362in}}{\pgfqpoint{7.921129in}{1.654318in}}%
\pgfpathcurveto{\pgfqpoint{7.921129in}{1.649275in}}{\pgfqpoint{7.923133in}{1.644437in}}{\pgfqpoint{7.926699in}{1.640871in}}%
\pgfpathcurveto{\pgfqpoint{7.930265in}{1.637304in}}{\pgfqpoint{7.935103in}{1.635300in}}{\pgfqpoint{7.940147in}{1.635300in}}%
\pgfpathclose%
\pgfusepath{fill}%
\end{pgfscope}%
\begin{pgfscope}%
\pgfpathrectangle{\pgfqpoint{6.572727in}{0.473000in}}{\pgfqpoint{4.227273in}{3.311000in}}%
\pgfusepath{clip}%
\pgfsetbuttcap%
\pgfsetroundjoin%
\definecolor{currentfill}{rgb}{0.127568,0.566949,0.550556}%
\pgfsetfillcolor{currentfill}%
\pgfsetfillopacity{0.700000}%
\pgfsetlinewidth{0.000000pt}%
\definecolor{currentstroke}{rgb}{0.000000,0.000000,0.000000}%
\pgfsetstrokecolor{currentstroke}%
\pgfsetstrokeopacity{0.700000}%
\pgfsetdash{}{0pt}%
\pgfpathmoveto{\pgfqpoint{8.380058in}{1.276987in}}%
\pgfpathcurveto{\pgfqpoint{8.385102in}{1.276987in}}{\pgfqpoint{8.389940in}{1.278991in}}{\pgfqpoint{8.393506in}{1.282558in}}%
\pgfpathcurveto{\pgfqpoint{8.397073in}{1.286124in}}{\pgfqpoint{8.399077in}{1.290962in}}{\pgfqpoint{8.399077in}{1.296006in}}%
\pgfpathcurveto{\pgfqpoint{8.399077in}{1.301049in}}{\pgfqpoint{8.397073in}{1.305887in}}{\pgfqpoint{8.393506in}{1.309453in}}%
\pgfpathcurveto{\pgfqpoint{8.389940in}{1.313020in}}{\pgfqpoint{8.385102in}{1.315024in}}{\pgfqpoint{8.380058in}{1.315024in}}%
\pgfpathcurveto{\pgfqpoint{8.375015in}{1.315024in}}{\pgfqpoint{8.370177in}{1.313020in}}{\pgfqpoint{8.366611in}{1.309453in}}%
\pgfpathcurveto{\pgfqpoint{8.363044in}{1.305887in}}{\pgfqpoint{8.361040in}{1.301049in}}{\pgfqpoint{8.361040in}{1.296006in}}%
\pgfpathcurveto{\pgfqpoint{8.361040in}{1.290962in}}{\pgfqpoint{8.363044in}{1.286124in}}{\pgfqpoint{8.366611in}{1.282558in}}%
\pgfpathcurveto{\pgfqpoint{8.370177in}{1.278991in}}{\pgfqpoint{8.375015in}{1.276987in}}{\pgfqpoint{8.380058in}{1.276987in}}%
\pgfpathclose%
\pgfusepath{fill}%
\end{pgfscope}%
\begin{pgfscope}%
\pgfpathrectangle{\pgfqpoint{6.572727in}{0.473000in}}{\pgfqpoint{4.227273in}{3.311000in}}%
\pgfusepath{clip}%
\pgfsetbuttcap%
\pgfsetroundjoin%
\definecolor{currentfill}{rgb}{0.993248,0.906157,0.143936}%
\pgfsetfillcolor{currentfill}%
\pgfsetfillopacity{0.700000}%
\pgfsetlinewidth{0.000000pt}%
\definecolor{currentstroke}{rgb}{0.000000,0.000000,0.000000}%
\pgfsetstrokecolor{currentstroke}%
\pgfsetstrokeopacity{0.700000}%
\pgfsetdash{}{0pt}%
\pgfpathmoveto{\pgfqpoint{9.698185in}{1.941396in}}%
\pgfpathcurveto{\pgfqpoint{9.703229in}{1.941396in}}{\pgfqpoint{9.708067in}{1.943400in}}{\pgfqpoint{9.711633in}{1.946966in}}%
\pgfpathcurveto{\pgfqpoint{9.715199in}{1.950533in}}{\pgfqpoint{9.717203in}{1.955370in}}{\pgfqpoint{9.717203in}{1.960414in}}%
\pgfpathcurveto{\pgfqpoint{9.717203in}{1.965458in}}{\pgfqpoint{9.715199in}{1.970295in}}{\pgfqpoint{9.711633in}{1.973862in}}%
\pgfpathcurveto{\pgfqpoint{9.708067in}{1.977428in}}{\pgfqpoint{9.703229in}{1.979432in}}{\pgfqpoint{9.698185in}{1.979432in}}%
\pgfpathcurveto{\pgfqpoint{9.693141in}{1.979432in}}{\pgfqpoint{9.688304in}{1.977428in}}{\pgfqpoint{9.684737in}{1.973862in}}%
\pgfpathcurveto{\pgfqpoint{9.681171in}{1.970295in}}{\pgfqpoint{9.679167in}{1.965458in}}{\pgfqpoint{9.679167in}{1.960414in}}%
\pgfpathcurveto{\pgfqpoint{9.679167in}{1.955370in}}{\pgfqpoint{9.681171in}{1.950533in}}{\pgfqpoint{9.684737in}{1.946966in}}%
\pgfpathcurveto{\pgfqpoint{9.688304in}{1.943400in}}{\pgfqpoint{9.693141in}{1.941396in}}{\pgfqpoint{9.698185in}{1.941396in}}%
\pgfpathclose%
\pgfusepath{fill}%
\end{pgfscope}%
\begin{pgfscope}%
\pgfpathrectangle{\pgfqpoint{6.572727in}{0.473000in}}{\pgfqpoint{4.227273in}{3.311000in}}%
\pgfusepath{clip}%
\pgfsetbuttcap%
\pgfsetroundjoin%
\definecolor{currentfill}{rgb}{0.993248,0.906157,0.143936}%
\pgfsetfillcolor{currentfill}%
\pgfsetfillopacity{0.700000}%
\pgfsetlinewidth{0.000000pt}%
\definecolor{currentstroke}{rgb}{0.000000,0.000000,0.000000}%
\pgfsetstrokecolor{currentstroke}%
\pgfsetstrokeopacity{0.700000}%
\pgfsetdash{}{0pt}%
\pgfpathmoveto{\pgfqpoint{9.397700in}{1.277516in}}%
\pgfpathcurveto{\pgfqpoint{9.402744in}{1.277516in}}{\pgfqpoint{9.407582in}{1.279520in}}{\pgfqpoint{9.411148in}{1.283087in}}%
\pgfpathcurveto{\pgfqpoint{9.414715in}{1.286653in}}{\pgfqpoint{9.416719in}{1.291491in}}{\pgfqpoint{9.416719in}{1.296534in}}%
\pgfpathcurveto{\pgfqpoint{9.416719in}{1.301578in}}{\pgfqpoint{9.414715in}{1.306416in}}{\pgfqpoint{9.411148in}{1.309982in}}%
\pgfpathcurveto{\pgfqpoint{9.407582in}{1.313549in}}{\pgfqpoint{9.402744in}{1.315553in}}{\pgfqpoint{9.397700in}{1.315553in}}%
\pgfpathcurveto{\pgfqpoint{9.392657in}{1.315553in}}{\pgfqpoint{9.387819in}{1.313549in}}{\pgfqpoint{9.384253in}{1.309982in}}%
\pgfpathcurveto{\pgfqpoint{9.380686in}{1.306416in}}{\pgfqpoint{9.378682in}{1.301578in}}{\pgfqpoint{9.378682in}{1.296534in}}%
\pgfpathcurveto{\pgfqpoint{9.378682in}{1.291491in}}{\pgfqpoint{9.380686in}{1.286653in}}{\pgfqpoint{9.384253in}{1.283087in}}%
\pgfpathcurveto{\pgfqpoint{9.387819in}{1.279520in}}{\pgfqpoint{9.392657in}{1.277516in}}{\pgfqpoint{9.397700in}{1.277516in}}%
\pgfpathclose%
\pgfusepath{fill}%
\end{pgfscope}%
\begin{pgfscope}%
\pgfpathrectangle{\pgfqpoint{6.572727in}{0.473000in}}{\pgfqpoint{4.227273in}{3.311000in}}%
\pgfusepath{clip}%
\pgfsetbuttcap%
\pgfsetroundjoin%
\definecolor{currentfill}{rgb}{0.127568,0.566949,0.550556}%
\pgfsetfillcolor{currentfill}%
\pgfsetfillopacity{0.700000}%
\pgfsetlinewidth{0.000000pt}%
\definecolor{currentstroke}{rgb}{0.000000,0.000000,0.000000}%
\pgfsetstrokecolor{currentstroke}%
\pgfsetstrokeopacity{0.700000}%
\pgfsetdash{}{0pt}%
\pgfpathmoveto{\pgfqpoint{8.575376in}{1.522607in}}%
\pgfpathcurveto{\pgfqpoint{8.580420in}{1.522607in}}{\pgfqpoint{8.585258in}{1.524610in}}{\pgfqpoint{8.588824in}{1.528177in}}%
\pgfpathcurveto{\pgfqpoint{8.592390in}{1.531743in}}{\pgfqpoint{8.594394in}{1.536581in}}{\pgfqpoint{8.594394in}{1.541625in}}%
\pgfpathcurveto{\pgfqpoint{8.594394in}{1.546668in}}{\pgfqpoint{8.592390in}{1.551506in}}{\pgfqpoint{8.588824in}{1.555073in}}%
\pgfpathcurveto{\pgfqpoint{8.585258in}{1.558639in}}{\pgfqpoint{8.580420in}{1.560643in}}{\pgfqpoint{8.575376in}{1.560643in}}%
\pgfpathcurveto{\pgfqpoint{8.570333in}{1.560643in}}{\pgfqpoint{8.565495in}{1.558639in}}{\pgfqpoint{8.561928in}{1.555073in}}%
\pgfpathcurveto{\pgfqpoint{8.558362in}{1.551506in}}{\pgfqpoint{8.556358in}{1.546668in}}{\pgfqpoint{8.556358in}{1.541625in}}%
\pgfpathcurveto{\pgfqpoint{8.556358in}{1.536581in}}{\pgfqpoint{8.558362in}{1.531743in}}{\pgfqpoint{8.561928in}{1.528177in}}%
\pgfpathcurveto{\pgfqpoint{8.565495in}{1.524610in}}{\pgfqpoint{8.570333in}{1.522607in}}{\pgfqpoint{8.575376in}{1.522607in}}%
\pgfpathclose%
\pgfusepath{fill}%
\end{pgfscope}%
\begin{pgfscope}%
\pgfpathrectangle{\pgfqpoint{6.572727in}{0.473000in}}{\pgfqpoint{4.227273in}{3.311000in}}%
\pgfusepath{clip}%
\pgfsetbuttcap%
\pgfsetroundjoin%
\definecolor{currentfill}{rgb}{0.127568,0.566949,0.550556}%
\pgfsetfillcolor{currentfill}%
\pgfsetfillopacity{0.700000}%
\pgfsetlinewidth{0.000000pt}%
\definecolor{currentstroke}{rgb}{0.000000,0.000000,0.000000}%
\pgfsetstrokecolor{currentstroke}%
\pgfsetstrokeopacity{0.700000}%
\pgfsetdash{}{0pt}%
\pgfpathmoveto{\pgfqpoint{7.961889in}{2.987791in}}%
\pgfpathcurveto{\pgfqpoint{7.966932in}{2.987791in}}{\pgfqpoint{7.971770in}{2.989795in}}{\pgfqpoint{7.975336in}{2.993361in}}%
\pgfpathcurveto{\pgfqpoint{7.978903in}{2.996927in}}{\pgfqpoint{7.980907in}{3.001765in}}{\pgfqpoint{7.980907in}{3.006809in}}%
\pgfpathcurveto{\pgfqpoint{7.980907in}{3.011853in}}{\pgfqpoint{7.978903in}{3.016690in}}{\pgfqpoint{7.975336in}{3.020257in}}%
\pgfpathcurveto{\pgfqpoint{7.971770in}{3.023823in}}{\pgfqpoint{7.966932in}{3.025827in}}{\pgfqpoint{7.961889in}{3.025827in}}%
\pgfpathcurveto{\pgfqpoint{7.956845in}{3.025827in}}{\pgfqpoint{7.952007in}{3.023823in}}{\pgfqpoint{7.948441in}{3.020257in}}%
\pgfpathcurveto{\pgfqpoint{7.944874in}{3.016690in}}{\pgfqpoint{7.942870in}{3.011853in}}{\pgfqpoint{7.942870in}{3.006809in}}%
\pgfpathcurveto{\pgfqpoint{7.942870in}{3.001765in}}{\pgfqpoint{7.944874in}{2.996927in}}{\pgfqpoint{7.948441in}{2.993361in}}%
\pgfpathcurveto{\pgfqpoint{7.952007in}{2.989795in}}{\pgfqpoint{7.956845in}{2.987791in}}{\pgfqpoint{7.961889in}{2.987791in}}%
\pgfpathclose%
\pgfusepath{fill}%
\end{pgfscope}%
\begin{pgfscope}%
\pgfpathrectangle{\pgfqpoint{6.572727in}{0.473000in}}{\pgfqpoint{4.227273in}{3.311000in}}%
\pgfusepath{clip}%
\pgfsetbuttcap%
\pgfsetroundjoin%
\definecolor{currentfill}{rgb}{0.993248,0.906157,0.143936}%
\pgfsetfillcolor{currentfill}%
\pgfsetfillopacity{0.700000}%
\pgfsetlinewidth{0.000000pt}%
\definecolor{currentstroke}{rgb}{0.000000,0.000000,0.000000}%
\pgfsetstrokecolor{currentstroke}%
\pgfsetstrokeopacity{0.700000}%
\pgfsetdash{}{0pt}%
\pgfpathmoveto{\pgfqpoint{9.692532in}{2.122403in}}%
\pgfpathcurveto{\pgfqpoint{9.697576in}{2.122403in}}{\pgfqpoint{9.702413in}{2.124406in}}{\pgfqpoint{9.705980in}{2.127973in}}%
\pgfpathcurveto{\pgfqpoint{9.709546in}{2.131539in}}{\pgfqpoint{9.711550in}{2.136377in}}{\pgfqpoint{9.711550in}{2.141421in}}%
\pgfpathcurveto{\pgfqpoint{9.711550in}{2.146464in}}{\pgfqpoint{9.709546in}{2.151302in}}{\pgfqpoint{9.705980in}{2.154869in}}%
\pgfpathcurveto{\pgfqpoint{9.702413in}{2.158435in}}{\pgfqpoint{9.697576in}{2.160439in}}{\pgfqpoint{9.692532in}{2.160439in}}%
\pgfpathcurveto{\pgfqpoint{9.687488in}{2.160439in}}{\pgfqpoint{9.682651in}{2.158435in}}{\pgfqpoint{9.679084in}{2.154869in}}%
\pgfpathcurveto{\pgfqpoint{9.675518in}{2.151302in}}{\pgfqpoint{9.673514in}{2.146464in}}{\pgfqpoint{9.673514in}{2.141421in}}%
\pgfpathcurveto{\pgfqpoint{9.673514in}{2.136377in}}{\pgfqpoint{9.675518in}{2.131539in}}{\pgfqpoint{9.679084in}{2.127973in}}%
\pgfpathcurveto{\pgfqpoint{9.682651in}{2.124406in}}{\pgfqpoint{9.687488in}{2.122403in}}{\pgfqpoint{9.692532in}{2.122403in}}%
\pgfpathclose%
\pgfusepath{fill}%
\end{pgfscope}%
\begin{pgfscope}%
\pgfpathrectangle{\pgfqpoint{6.572727in}{0.473000in}}{\pgfqpoint{4.227273in}{3.311000in}}%
\pgfusepath{clip}%
\pgfsetbuttcap%
\pgfsetroundjoin%
\definecolor{currentfill}{rgb}{0.127568,0.566949,0.550556}%
\pgfsetfillcolor{currentfill}%
\pgfsetfillopacity{0.700000}%
\pgfsetlinewidth{0.000000pt}%
\definecolor{currentstroke}{rgb}{0.000000,0.000000,0.000000}%
\pgfsetstrokecolor{currentstroke}%
\pgfsetstrokeopacity{0.700000}%
\pgfsetdash{}{0pt}%
\pgfpathmoveto{\pgfqpoint{8.102790in}{1.283563in}}%
\pgfpathcurveto{\pgfqpoint{8.107833in}{1.283563in}}{\pgfqpoint{8.112671in}{1.285567in}}{\pgfqpoint{8.116238in}{1.289133in}}%
\pgfpathcurveto{\pgfqpoint{8.119804in}{1.292700in}}{\pgfqpoint{8.121808in}{1.297538in}}{\pgfqpoint{8.121808in}{1.302581in}}%
\pgfpathcurveto{\pgfqpoint{8.121808in}{1.307625in}}{\pgfqpoint{8.119804in}{1.312463in}}{\pgfqpoint{8.116238in}{1.316029in}}%
\pgfpathcurveto{\pgfqpoint{8.112671in}{1.319596in}}{\pgfqpoint{8.107833in}{1.321599in}}{\pgfqpoint{8.102790in}{1.321599in}}%
\pgfpathcurveto{\pgfqpoint{8.097746in}{1.321599in}}{\pgfqpoint{8.092908in}{1.319596in}}{\pgfqpoint{8.089342in}{1.316029in}}%
\pgfpathcurveto{\pgfqpoint{8.085775in}{1.312463in}}{\pgfqpoint{8.083772in}{1.307625in}}{\pgfqpoint{8.083772in}{1.302581in}}%
\pgfpathcurveto{\pgfqpoint{8.083772in}{1.297538in}}{\pgfqpoint{8.085775in}{1.292700in}}{\pgfqpoint{8.089342in}{1.289133in}}%
\pgfpathcurveto{\pgfqpoint{8.092908in}{1.285567in}}{\pgfqpoint{8.097746in}{1.283563in}}{\pgfqpoint{8.102790in}{1.283563in}}%
\pgfpathclose%
\pgfusepath{fill}%
\end{pgfscope}%
\begin{pgfscope}%
\pgfpathrectangle{\pgfqpoint{6.572727in}{0.473000in}}{\pgfqpoint{4.227273in}{3.311000in}}%
\pgfusepath{clip}%
\pgfsetbuttcap%
\pgfsetroundjoin%
\definecolor{currentfill}{rgb}{0.127568,0.566949,0.550556}%
\pgfsetfillcolor{currentfill}%
\pgfsetfillopacity{0.700000}%
\pgfsetlinewidth{0.000000pt}%
\definecolor{currentstroke}{rgb}{0.000000,0.000000,0.000000}%
\pgfsetstrokecolor{currentstroke}%
\pgfsetstrokeopacity{0.700000}%
\pgfsetdash{}{0pt}%
\pgfpathmoveto{\pgfqpoint{7.747129in}{1.222256in}}%
\pgfpathcurveto{\pgfqpoint{7.752173in}{1.222256in}}{\pgfqpoint{7.757010in}{1.224260in}}{\pgfqpoint{7.760577in}{1.227827in}}%
\pgfpathcurveto{\pgfqpoint{7.764143in}{1.231393in}}{\pgfqpoint{7.766147in}{1.236231in}}{\pgfqpoint{7.766147in}{1.241275in}}%
\pgfpathcurveto{\pgfqpoint{7.766147in}{1.246318in}}{\pgfqpoint{7.764143in}{1.251156in}}{\pgfqpoint{7.760577in}{1.254722in}}%
\pgfpathcurveto{\pgfqpoint{7.757010in}{1.258289in}}{\pgfqpoint{7.752173in}{1.260293in}}{\pgfqpoint{7.747129in}{1.260293in}}%
\pgfpathcurveto{\pgfqpoint{7.742085in}{1.260293in}}{\pgfqpoint{7.737248in}{1.258289in}}{\pgfqpoint{7.733681in}{1.254722in}}%
\pgfpathcurveto{\pgfqpoint{7.730115in}{1.251156in}}{\pgfqpoint{7.728111in}{1.246318in}}{\pgfqpoint{7.728111in}{1.241275in}}%
\pgfpathcurveto{\pgfqpoint{7.728111in}{1.236231in}}{\pgfqpoint{7.730115in}{1.231393in}}{\pgfqpoint{7.733681in}{1.227827in}}%
\pgfpathcurveto{\pgfqpoint{7.737248in}{1.224260in}}{\pgfqpoint{7.742085in}{1.222256in}}{\pgfqpoint{7.747129in}{1.222256in}}%
\pgfpathclose%
\pgfusepath{fill}%
\end{pgfscope}%
\begin{pgfscope}%
\pgfpathrectangle{\pgfqpoint{6.572727in}{0.473000in}}{\pgfqpoint{4.227273in}{3.311000in}}%
\pgfusepath{clip}%
\pgfsetbuttcap%
\pgfsetroundjoin%
\definecolor{currentfill}{rgb}{0.127568,0.566949,0.550556}%
\pgfsetfillcolor{currentfill}%
\pgfsetfillopacity{0.700000}%
\pgfsetlinewidth{0.000000pt}%
\definecolor{currentstroke}{rgb}{0.000000,0.000000,0.000000}%
\pgfsetstrokecolor{currentstroke}%
\pgfsetstrokeopacity{0.700000}%
\pgfsetdash{}{0pt}%
\pgfpathmoveto{\pgfqpoint{8.573791in}{3.267384in}}%
\pgfpathcurveto{\pgfqpoint{8.578835in}{3.267384in}}{\pgfqpoint{8.583672in}{3.269388in}}{\pgfqpoint{8.587239in}{3.272954in}}%
\pgfpathcurveto{\pgfqpoint{8.590805in}{3.276521in}}{\pgfqpoint{8.592809in}{3.281358in}}{\pgfqpoint{8.592809in}{3.286402in}}%
\pgfpathcurveto{\pgfqpoint{8.592809in}{3.291446in}}{\pgfqpoint{8.590805in}{3.296283in}}{\pgfqpoint{8.587239in}{3.299850in}}%
\pgfpathcurveto{\pgfqpoint{8.583672in}{3.303416in}}{\pgfqpoint{8.578835in}{3.305420in}}{\pgfqpoint{8.573791in}{3.305420in}}%
\pgfpathcurveto{\pgfqpoint{8.568747in}{3.305420in}}{\pgfqpoint{8.563909in}{3.303416in}}{\pgfqpoint{8.560343in}{3.299850in}}%
\pgfpathcurveto{\pgfqpoint{8.556777in}{3.296283in}}{\pgfqpoint{8.554773in}{3.291446in}}{\pgfqpoint{8.554773in}{3.286402in}}%
\pgfpathcurveto{\pgfqpoint{8.554773in}{3.281358in}}{\pgfqpoint{8.556777in}{3.276521in}}{\pgfqpoint{8.560343in}{3.272954in}}%
\pgfpathcurveto{\pgfqpoint{8.563909in}{3.269388in}}{\pgfqpoint{8.568747in}{3.267384in}}{\pgfqpoint{8.573791in}{3.267384in}}%
\pgfpathclose%
\pgfusepath{fill}%
\end{pgfscope}%
\begin{pgfscope}%
\pgfpathrectangle{\pgfqpoint{6.572727in}{0.473000in}}{\pgfqpoint{4.227273in}{3.311000in}}%
\pgfusepath{clip}%
\pgfsetbuttcap%
\pgfsetroundjoin%
\definecolor{currentfill}{rgb}{0.127568,0.566949,0.550556}%
\pgfsetfillcolor{currentfill}%
\pgfsetfillopacity{0.700000}%
\pgfsetlinewidth{0.000000pt}%
\definecolor{currentstroke}{rgb}{0.000000,0.000000,0.000000}%
\pgfsetstrokecolor{currentstroke}%
\pgfsetstrokeopacity{0.700000}%
\pgfsetdash{}{0pt}%
\pgfpathmoveto{\pgfqpoint{7.415957in}{2.465296in}}%
\pgfpathcurveto{\pgfqpoint{7.421001in}{2.465296in}}{\pgfqpoint{7.425839in}{2.467300in}}{\pgfqpoint{7.429405in}{2.470867in}}%
\pgfpathcurveto{\pgfqpoint{7.432972in}{2.474433in}}{\pgfqpoint{7.434976in}{2.479271in}}{\pgfqpoint{7.434976in}{2.484315in}}%
\pgfpathcurveto{\pgfqpoint{7.434976in}{2.489358in}}{\pgfqpoint{7.432972in}{2.494196in}}{\pgfqpoint{7.429405in}{2.497762in}}%
\pgfpathcurveto{\pgfqpoint{7.425839in}{2.501329in}}{\pgfqpoint{7.421001in}{2.503333in}}{\pgfqpoint{7.415957in}{2.503333in}}%
\pgfpathcurveto{\pgfqpoint{7.410914in}{2.503333in}}{\pgfqpoint{7.406076in}{2.501329in}}{\pgfqpoint{7.402510in}{2.497762in}}%
\pgfpathcurveto{\pgfqpoint{7.398943in}{2.494196in}}{\pgfqpoint{7.396939in}{2.489358in}}{\pgfqpoint{7.396939in}{2.484315in}}%
\pgfpathcurveto{\pgfqpoint{7.396939in}{2.479271in}}{\pgfqpoint{7.398943in}{2.474433in}}{\pgfqpoint{7.402510in}{2.470867in}}%
\pgfpathcurveto{\pgfqpoint{7.406076in}{2.467300in}}{\pgfqpoint{7.410914in}{2.465296in}}{\pgfqpoint{7.415957in}{2.465296in}}%
\pgfpathclose%
\pgfusepath{fill}%
\end{pgfscope}%
\begin{pgfscope}%
\pgfpathrectangle{\pgfqpoint{6.572727in}{0.473000in}}{\pgfqpoint{4.227273in}{3.311000in}}%
\pgfusepath{clip}%
\pgfsetbuttcap%
\pgfsetroundjoin%
\definecolor{currentfill}{rgb}{0.127568,0.566949,0.550556}%
\pgfsetfillcolor{currentfill}%
\pgfsetfillopacity{0.700000}%
\pgfsetlinewidth{0.000000pt}%
\definecolor{currentstroke}{rgb}{0.000000,0.000000,0.000000}%
\pgfsetstrokecolor{currentstroke}%
\pgfsetstrokeopacity{0.700000}%
\pgfsetdash{}{0pt}%
\pgfpathmoveto{\pgfqpoint{8.074452in}{2.968464in}}%
\pgfpathcurveto{\pgfqpoint{8.079496in}{2.968464in}}{\pgfqpoint{8.084333in}{2.970468in}}{\pgfqpoint{8.087900in}{2.974034in}}%
\pgfpathcurveto{\pgfqpoint{8.091466in}{2.977601in}}{\pgfqpoint{8.093470in}{2.982439in}}{\pgfqpoint{8.093470in}{2.987482in}}%
\pgfpathcurveto{\pgfqpoint{8.093470in}{2.992526in}}{\pgfqpoint{8.091466in}{2.997364in}}{\pgfqpoint{8.087900in}{3.000930in}}%
\pgfpathcurveto{\pgfqpoint{8.084333in}{3.004497in}}{\pgfqpoint{8.079496in}{3.006500in}}{\pgfqpoint{8.074452in}{3.006500in}}%
\pgfpathcurveto{\pgfqpoint{8.069408in}{3.006500in}}{\pgfqpoint{8.064571in}{3.004497in}}{\pgfqpoint{8.061004in}{3.000930in}}%
\pgfpathcurveto{\pgfqpoint{8.057438in}{2.997364in}}{\pgfqpoint{8.055434in}{2.992526in}}{\pgfqpoint{8.055434in}{2.987482in}}%
\pgfpathcurveto{\pgfqpoint{8.055434in}{2.982439in}}{\pgfqpoint{8.057438in}{2.977601in}}{\pgfqpoint{8.061004in}{2.974034in}}%
\pgfpathcurveto{\pgfqpoint{8.064571in}{2.970468in}}{\pgfqpoint{8.069408in}{2.968464in}}{\pgfqpoint{8.074452in}{2.968464in}}%
\pgfpathclose%
\pgfusepath{fill}%
\end{pgfscope}%
\begin{pgfscope}%
\pgfpathrectangle{\pgfqpoint{6.572727in}{0.473000in}}{\pgfqpoint{4.227273in}{3.311000in}}%
\pgfusepath{clip}%
\pgfsetbuttcap%
\pgfsetroundjoin%
\definecolor{currentfill}{rgb}{0.127568,0.566949,0.550556}%
\pgfsetfillcolor{currentfill}%
\pgfsetfillopacity{0.700000}%
\pgfsetlinewidth{0.000000pt}%
\definecolor{currentstroke}{rgb}{0.000000,0.000000,0.000000}%
\pgfsetstrokecolor{currentstroke}%
\pgfsetstrokeopacity{0.700000}%
\pgfsetdash{}{0pt}%
\pgfpathmoveto{\pgfqpoint{8.042473in}{1.197760in}}%
\pgfpathcurveto{\pgfqpoint{8.047517in}{1.197760in}}{\pgfqpoint{8.052355in}{1.199764in}}{\pgfqpoint{8.055921in}{1.203330in}}%
\pgfpathcurveto{\pgfqpoint{8.059488in}{1.206897in}}{\pgfqpoint{8.061492in}{1.211735in}}{\pgfqpoint{8.061492in}{1.216778in}}%
\pgfpathcurveto{\pgfqpoint{8.061492in}{1.221822in}}{\pgfqpoint{8.059488in}{1.226660in}}{\pgfqpoint{8.055921in}{1.230226in}}%
\pgfpathcurveto{\pgfqpoint{8.052355in}{1.233793in}}{\pgfqpoint{8.047517in}{1.235796in}}{\pgfqpoint{8.042473in}{1.235796in}}%
\pgfpathcurveto{\pgfqpoint{8.037430in}{1.235796in}}{\pgfqpoint{8.032592in}{1.233793in}}{\pgfqpoint{8.029026in}{1.230226in}}%
\pgfpathcurveto{\pgfqpoint{8.025459in}{1.226660in}}{\pgfqpoint{8.023455in}{1.221822in}}{\pgfqpoint{8.023455in}{1.216778in}}%
\pgfpathcurveto{\pgfqpoint{8.023455in}{1.211735in}}{\pgfqpoint{8.025459in}{1.206897in}}{\pgfqpoint{8.029026in}{1.203330in}}%
\pgfpathcurveto{\pgfqpoint{8.032592in}{1.199764in}}{\pgfqpoint{8.037430in}{1.197760in}}{\pgfqpoint{8.042473in}{1.197760in}}%
\pgfpathclose%
\pgfusepath{fill}%
\end{pgfscope}%
\begin{pgfscope}%
\pgfpathrectangle{\pgfqpoint{6.572727in}{0.473000in}}{\pgfqpoint{4.227273in}{3.311000in}}%
\pgfusepath{clip}%
\pgfsetbuttcap%
\pgfsetroundjoin%
\definecolor{currentfill}{rgb}{0.127568,0.566949,0.550556}%
\pgfsetfillcolor{currentfill}%
\pgfsetfillopacity{0.700000}%
\pgfsetlinewidth{0.000000pt}%
\definecolor{currentstroke}{rgb}{0.000000,0.000000,0.000000}%
\pgfsetstrokecolor{currentstroke}%
\pgfsetstrokeopacity{0.700000}%
\pgfsetdash{}{0pt}%
\pgfpathmoveto{\pgfqpoint{8.057104in}{2.412395in}}%
\pgfpathcurveto{\pgfqpoint{8.062148in}{2.412395in}}{\pgfqpoint{8.066985in}{2.414399in}}{\pgfqpoint{8.070552in}{2.417965in}}%
\pgfpathcurveto{\pgfqpoint{8.074118in}{2.421531in}}{\pgfqpoint{8.076122in}{2.426369in}}{\pgfqpoint{8.076122in}{2.431413in}}%
\pgfpathcurveto{\pgfqpoint{8.076122in}{2.436456in}}{\pgfqpoint{8.074118in}{2.441294in}}{\pgfqpoint{8.070552in}{2.444861in}}%
\pgfpathcurveto{\pgfqpoint{8.066985in}{2.448427in}}{\pgfqpoint{8.062148in}{2.450431in}}{\pgfqpoint{8.057104in}{2.450431in}}%
\pgfpathcurveto{\pgfqpoint{8.052060in}{2.450431in}}{\pgfqpoint{8.047223in}{2.448427in}}{\pgfqpoint{8.043656in}{2.444861in}}%
\pgfpathcurveto{\pgfqpoint{8.040090in}{2.441294in}}{\pgfqpoint{8.038086in}{2.436456in}}{\pgfqpoint{8.038086in}{2.431413in}}%
\pgfpathcurveto{\pgfqpoint{8.038086in}{2.426369in}}{\pgfqpoint{8.040090in}{2.421531in}}{\pgfqpoint{8.043656in}{2.417965in}}%
\pgfpathcurveto{\pgfqpoint{8.047223in}{2.414399in}}{\pgfqpoint{8.052060in}{2.412395in}}{\pgfqpoint{8.057104in}{2.412395in}}%
\pgfpathclose%
\pgfusepath{fill}%
\end{pgfscope}%
\begin{pgfscope}%
\pgfpathrectangle{\pgfqpoint{6.572727in}{0.473000in}}{\pgfqpoint{4.227273in}{3.311000in}}%
\pgfusepath{clip}%
\pgfsetbuttcap%
\pgfsetroundjoin%
\definecolor{currentfill}{rgb}{0.993248,0.906157,0.143936}%
\pgfsetfillcolor{currentfill}%
\pgfsetfillopacity{0.700000}%
\pgfsetlinewidth{0.000000pt}%
\definecolor{currentstroke}{rgb}{0.000000,0.000000,0.000000}%
\pgfsetstrokecolor{currentstroke}%
\pgfsetstrokeopacity{0.700000}%
\pgfsetdash{}{0pt}%
\pgfpathmoveto{\pgfqpoint{9.446345in}{1.865978in}}%
\pgfpathcurveto{\pgfqpoint{9.451389in}{1.865978in}}{\pgfqpoint{9.456226in}{1.867982in}}{\pgfqpoint{9.459793in}{1.871548in}}%
\pgfpathcurveto{\pgfqpoint{9.463359in}{1.875115in}}{\pgfqpoint{9.465363in}{1.879953in}}{\pgfqpoint{9.465363in}{1.884996in}}%
\pgfpathcurveto{\pgfqpoint{9.465363in}{1.890040in}}{\pgfqpoint{9.463359in}{1.894878in}}{\pgfqpoint{9.459793in}{1.898444in}}%
\pgfpathcurveto{\pgfqpoint{9.456226in}{1.902011in}}{\pgfqpoint{9.451389in}{1.904014in}}{\pgfqpoint{9.446345in}{1.904014in}}%
\pgfpathcurveto{\pgfqpoint{9.441301in}{1.904014in}}{\pgfqpoint{9.436463in}{1.902011in}}{\pgfqpoint{9.432897in}{1.898444in}}%
\pgfpathcurveto{\pgfqpoint{9.429331in}{1.894878in}}{\pgfqpoint{9.427327in}{1.890040in}}{\pgfqpoint{9.427327in}{1.884996in}}%
\pgfpathcurveto{\pgfqpoint{9.427327in}{1.879953in}}{\pgfqpoint{9.429331in}{1.875115in}}{\pgfqpoint{9.432897in}{1.871548in}}%
\pgfpathcurveto{\pgfqpoint{9.436463in}{1.867982in}}{\pgfqpoint{9.441301in}{1.865978in}}{\pgfqpoint{9.446345in}{1.865978in}}%
\pgfpathclose%
\pgfusepath{fill}%
\end{pgfscope}%
\begin{pgfscope}%
\pgfpathrectangle{\pgfqpoint{6.572727in}{0.473000in}}{\pgfqpoint{4.227273in}{3.311000in}}%
\pgfusepath{clip}%
\pgfsetbuttcap%
\pgfsetroundjoin%
\definecolor{currentfill}{rgb}{0.993248,0.906157,0.143936}%
\pgfsetfillcolor{currentfill}%
\pgfsetfillopacity{0.700000}%
\pgfsetlinewidth{0.000000pt}%
\definecolor{currentstroke}{rgb}{0.000000,0.000000,0.000000}%
\pgfsetstrokecolor{currentstroke}%
\pgfsetstrokeopacity{0.700000}%
\pgfsetdash{}{0pt}%
\pgfpathmoveto{\pgfqpoint{9.648030in}{1.461707in}}%
\pgfpathcurveto{\pgfqpoint{9.653074in}{1.461707in}}{\pgfqpoint{9.657912in}{1.463710in}}{\pgfqpoint{9.661478in}{1.467277in}}%
\pgfpathcurveto{\pgfqpoint{9.665044in}{1.470843in}}{\pgfqpoint{9.667048in}{1.475681in}}{\pgfqpoint{9.667048in}{1.480725in}}%
\pgfpathcurveto{\pgfqpoint{9.667048in}{1.485768in}}{\pgfqpoint{9.665044in}{1.490606in}}{\pgfqpoint{9.661478in}{1.494173in}}%
\pgfpathcurveto{\pgfqpoint{9.657912in}{1.497739in}}{\pgfqpoint{9.653074in}{1.499743in}}{\pgfqpoint{9.648030in}{1.499743in}}%
\pgfpathcurveto{\pgfqpoint{9.642986in}{1.499743in}}{\pgfqpoint{9.638149in}{1.497739in}}{\pgfqpoint{9.634582in}{1.494173in}}%
\pgfpathcurveto{\pgfqpoint{9.631016in}{1.490606in}}{\pgfqpoint{9.629012in}{1.485768in}}{\pgfqpoint{9.629012in}{1.480725in}}%
\pgfpathcurveto{\pgfqpoint{9.629012in}{1.475681in}}{\pgfqpoint{9.631016in}{1.470843in}}{\pgfqpoint{9.634582in}{1.467277in}}%
\pgfpathcurveto{\pgfqpoint{9.638149in}{1.463710in}}{\pgfqpoint{9.642986in}{1.461707in}}{\pgfqpoint{9.648030in}{1.461707in}}%
\pgfpathclose%
\pgfusepath{fill}%
\end{pgfscope}%
\begin{pgfscope}%
\pgfpathrectangle{\pgfqpoint{6.572727in}{0.473000in}}{\pgfqpoint{4.227273in}{3.311000in}}%
\pgfusepath{clip}%
\pgfsetbuttcap%
\pgfsetroundjoin%
\definecolor{currentfill}{rgb}{0.127568,0.566949,0.550556}%
\pgfsetfillcolor{currentfill}%
\pgfsetfillopacity{0.700000}%
\pgfsetlinewidth{0.000000pt}%
\definecolor{currentstroke}{rgb}{0.000000,0.000000,0.000000}%
\pgfsetstrokecolor{currentstroke}%
\pgfsetstrokeopacity{0.700000}%
\pgfsetdash{}{0pt}%
\pgfpathmoveto{\pgfqpoint{7.824022in}{1.597599in}}%
\pgfpathcurveto{\pgfqpoint{7.829066in}{1.597599in}}{\pgfqpoint{7.833904in}{1.599603in}}{\pgfqpoint{7.837470in}{1.603170in}}%
\pgfpathcurveto{\pgfqpoint{7.841037in}{1.606736in}}{\pgfqpoint{7.843040in}{1.611574in}}{\pgfqpoint{7.843040in}{1.616618in}}%
\pgfpathcurveto{\pgfqpoint{7.843040in}{1.621661in}}{\pgfqpoint{7.841037in}{1.626499in}}{\pgfqpoint{7.837470in}{1.630065in}}%
\pgfpathcurveto{\pgfqpoint{7.833904in}{1.633632in}}{\pgfqpoint{7.829066in}{1.635636in}}{\pgfqpoint{7.824022in}{1.635636in}}%
\pgfpathcurveto{\pgfqpoint{7.818979in}{1.635636in}}{\pgfqpoint{7.814141in}{1.633632in}}{\pgfqpoint{7.810574in}{1.630065in}}%
\pgfpathcurveto{\pgfqpoint{7.807008in}{1.626499in}}{\pgfqpoint{7.805004in}{1.621661in}}{\pgfqpoint{7.805004in}{1.616618in}}%
\pgfpathcurveto{\pgfqpoint{7.805004in}{1.611574in}}{\pgfqpoint{7.807008in}{1.606736in}}{\pgfqpoint{7.810574in}{1.603170in}}%
\pgfpathcurveto{\pgfqpoint{7.814141in}{1.599603in}}{\pgfqpoint{7.818979in}{1.597599in}}{\pgfqpoint{7.824022in}{1.597599in}}%
\pgfpathclose%
\pgfusepath{fill}%
\end{pgfscope}%
\begin{pgfscope}%
\pgfpathrectangle{\pgfqpoint{6.572727in}{0.473000in}}{\pgfqpoint{4.227273in}{3.311000in}}%
\pgfusepath{clip}%
\pgfsetbuttcap%
\pgfsetroundjoin%
\definecolor{currentfill}{rgb}{0.127568,0.566949,0.550556}%
\pgfsetfillcolor{currentfill}%
\pgfsetfillopacity{0.700000}%
\pgfsetlinewidth{0.000000pt}%
\definecolor{currentstroke}{rgb}{0.000000,0.000000,0.000000}%
\pgfsetstrokecolor{currentstroke}%
\pgfsetstrokeopacity{0.700000}%
\pgfsetdash{}{0pt}%
\pgfpathmoveto{\pgfqpoint{8.428063in}{3.157987in}}%
\pgfpathcurveto{\pgfqpoint{8.433107in}{3.157987in}}{\pgfqpoint{8.437945in}{3.159991in}}{\pgfqpoint{8.441511in}{3.163557in}}%
\pgfpathcurveto{\pgfqpoint{8.445078in}{3.167124in}}{\pgfqpoint{8.447081in}{3.171961in}}{\pgfqpoint{8.447081in}{3.177005in}}%
\pgfpathcurveto{\pgfqpoint{8.447081in}{3.182049in}}{\pgfqpoint{8.445078in}{3.186886in}}{\pgfqpoint{8.441511in}{3.190453in}}%
\pgfpathcurveto{\pgfqpoint{8.437945in}{3.194019in}}{\pgfqpoint{8.433107in}{3.196023in}}{\pgfqpoint{8.428063in}{3.196023in}}%
\pgfpathcurveto{\pgfqpoint{8.423020in}{3.196023in}}{\pgfqpoint{8.418182in}{3.194019in}}{\pgfqpoint{8.414615in}{3.190453in}}%
\pgfpathcurveto{\pgfqpoint{8.411049in}{3.186886in}}{\pgfqpoint{8.409045in}{3.182049in}}{\pgfqpoint{8.409045in}{3.177005in}}%
\pgfpathcurveto{\pgfqpoint{8.409045in}{3.171961in}}{\pgfqpoint{8.411049in}{3.167124in}}{\pgfqpoint{8.414615in}{3.163557in}}%
\pgfpathcurveto{\pgfqpoint{8.418182in}{3.159991in}}{\pgfqpoint{8.423020in}{3.157987in}}{\pgfqpoint{8.428063in}{3.157987in}}%
\pgfpathclose%
\pgfusepath{fill}%
\end{pgfscope}%
\begin{pgfscope}%
\pgfpathrectangle{\pgfqpoint{6.572727in}{0.473000in}}{\pgfqpoint{4.227273in}{3.311000in}}%
\pgfusepath{clip}%
\pgfsetbuttcap%
\pgfsetroundjoin%
\definecolor{currentfill}{rgb}{0.993248,0.906157,0.143936}%
\pgfsetfillcolor{currentfill}%
\pgfsetfillopacity{0.700000}%
\pgfsetlinewidth{0.000000pt}%
\definecolor{currentstroke}{rgb}{0.000000,0.000000,0.000000}%
\pgfsetstrokecolor{currentstroke}%
\pgfsetstrokeopacity{0.700000}%
\pgfsetdash{}{0pt}%
\pgfpathmoveto{\pgfqpoint{9.258780in}{1.217350in}}%
\pgfpathcurveto{\pgfqpoint{9.263824in}{1.217350in}}{\pgfqpoint{9.268662in}{1.219354in}}{\pgfqpoint{9.272228in}{1.222920in}}%
\pgfpathcurveto{\pgfqpoint{9.275794in}{1.226486in}}{\pgfqpoint{9.277798in}{1.231324in}}{\pgfqpoint{9.277798in}{1.236368in}}%
\pgfpathcurveto{\pgfqpoint{9.277798in}{1.241411in}}{\pgfqpoint{9.275794in}{1.246249in}}{\pgfqpoint{9.272228in}{1.249816in}}%
\pgfpathcurveto{\pgfqpoint{9.268662in}{1.253382in}}{\pgfqpoint{9.263824in}{1.255386in}}{\pgfqpoint{9.258780in}{1.255386in}}%
\pgfpathcurveto{\pgfqpoint{9.253736in}{1.255386in}}{\pgfqpoint{9.248899in}{1.253382in}}{\pgfqpoint{9.245332in}{1.249816in}}%
\pgfpathcurveto{\pgfqpoint{9.241766in}{1.246249in}}{\pgfqpoint{9.239762in}{1.241411in}}{\pgfqpoint{9.239762in}{1.236368in}}%
\pgfpathcurveto{\pgfqpoint{9.239762in}{1.231324in}}{\pgfqpoint{9.241766in}{1.226486in}}{\pgfqpoint{9.245332in}{1.222920in}}%
\pgfpathcurveto{\pgfqpoint{9.248899in}{1.219354in}}{\pgfqpoint{9.253736in}{1.217350in}}{\pgfqpoint{9.258780in}{1.217350in}}%
\pgfpathclose%
\pgfusepath{fill}%
\end{pgfscope}%
\begin{pgfscope}%
\pgfpathrectangle{\pgfqpoint{6.572727in}{0.473000in}}{\pgfqpoint{4.227273in}{3.311000in}}%
\pgfusepath{clip}%
\pgfsetbuttcap%
\pgfsetroundjoin%
\definecolor{currentfill}{rgb}{0.127568,0.566949,0.550556}%
\pgfsetfillcolor{currentfill}%
\pgfsetfillopacity{0.700000}%
\pgfsetlinewidth{0.000000pt}%
\definecolor{currentstroke}{rgb}{0.000000,0.000000,0.000000}%
\pgfsetstrokecolor{currentstroke}%
\pgfsetstrokeopacity{0.700000}%
\pgfsetdash{}{0pt}%
\pgfpathmoveto{\pgfqpoint{7.915753in}{2.936822in}}%
\pgfpathcurveto{\pgfqpoint{7.920796in}{2.936822in}}{\pgfqpoint{7.925634in}{2.938826in}}{\pgfqpoint{7.929200in}{2.942393in}}%
\pgfpathcurveto{\pgfqpoint{7.932767in}{2.945959in}}{\pgfqpoint{7.934771in}{2.950797in}}{\pgfqpoint{7.934771in}{2.955840in}}%
\pgfpathcurveto{\pgfqpoint{7.934771in}{2.960884in}}{\pgfqpoint{7.932767in}{2.965722in}}{\pgfqpoint{7.929200in}{2.969288in}}%
\pgfpathcurveto{\pgfqpoint{7.925634in}{2.972855in}}{\pgfqpoint{7.920796in}{2.974859in}}{\pgfqpoint{7.915753in}{2.974859in}}%
\pgfpathcurveto{\pgfqpoint{7.910709in}{2.974859in}}{\pgfqpoint{7.905871in}{2.972855in}}{\pgfqpoint{7.902305in}{2.969288in}}%
\pgfpathcurveto{\pgfqpoint{7.898738in}{2.965722in}}{\pgfqpoint{7.896734in}{2.960884in}}{\pgfqpoint{7.896734in}{2.955840in}}%
\pgfpathcurveto{\pgfqpoint{7.896734in}{2.950797in}}{\pgfqpoint{7.898738in}{2.945959in}}{\pgfqpoint{7.902305in}{2.942393in}}%
\pgfpathcurveto{\pgfqpoint{7.905871in}{2.938826in}}{\pgfqpoint{7.910709in}{2.936822in}}{\pgfqpoint{7.915753in}{2.936822in}}%
\pgfpathclose%
\pgfusepath{fill}%
\end{pgfscope}%
\begin{pgfscope}%
\pgfpathrectangle{\pgfqpoint{6.572727in}{0.473000in}}{\pgfqpoint{4.227273in}{3.311000in}}%
\pgfusepath{clip}%
\pgfsetbuttcap%
\pgfsetroundjoin%
\definecolor{currentfill}{rgb}{0.993248,0.906157,0.143936}%
\pgfsetfillcolor{currentfill}%
\pgfsetfillopacity{0.700000}%
\pgfsetlinewidth{0.000000pt}%
\definecolor{currentstroke}{rgb}{0.000000,0.000000,0.000000}%
\pgfsetstrokecolor{currentstroke}%
\pgfsetstrokeopacity{0.700000}%
\pgfsetdash{}{0pt}%
\pgfpathmoveto{\pgfqpoint{9.496340in}{1.468252in}}%
\pgfpathcurveto{\pgfqpoint{9.501384in}{1.468252in}}{\pgfqpoint{9.506222in}{1.470256in}}{\pgfqpoint{9.509788in}{1.473823in}}%
\pgfpathcurveto{\pgfqpoint{9.513354in}{1.477389in}}{\pgfqpoint{9.515358in}{1.482227in}}{\pgfqpoint{9.515358in}{1.487270in}}%
\pgfpathcurveto{\pgfqpoint{9.515358in}{1.492314in}}{\pgfqpoint{9.513354in}{1.497152in}}{\pgfqpoint{9.509788in}{1.500718in}}%
\pgfpathcurveto{\pgfqpoint{9.506222in}{1.504285in}}{\pgfqpoint{9.501384in}{1.506289in}}{\pgfqpoint{9.496340in}{1.506289in}}%
\pgfpathcurveto{\pgfqpoint{9.491296in}{1.506289in}}{\pgfqpoint{9.486459in}{1.504285in}}{\pgfqpoint{9.482892in}{1.500718in}}%
\pgfpathcurveto{\pgfqpoint{9.479326in}{1.497152in}}{\pgfqpoint{9.477322in}{1.492314in}}{\pgfqpoint{9.477322in}{1.487270in}}%
\pgfpathcurveto{\pgfqpoint{9.477322in}{1.482227in}}{\pgfqpoint{9.479326in}{1.477389in}}{\pgfqpoint{9.482892in}{1.473823in}}%
\pgfpathcurveto{\pgfqpoint{9.486459in}{1.470256in}}{\pgfqpoint{9.491296in}{1.468252in}}{\pgfqpoint{9.496340in}{1.468252in}}%
\pgfpathclose%
\pgfusepath{fill}%
\end{pgfscope}%
\begin{pgfscope}%
\pgfpathrectangle{\pgfqpoint{6.572727in}{0.473000in}}{\pgfqpoint{4.227273in}{3.311000in}}%
\pgfusepath{clip}%
\pgfsetbuttcap%
\pgfsetroundjoin%
\definecolor{currentfill}{rgb}{0.993248,0.906157,0.143936}%
\pgfsetfillcolor{currentfill}%
\pgfsetfillopacity{0.700000}%
\pgfsetlinewidth{0.000000pt}%
\definecolor{currentstroke}{rgb}{0.000000,0.000000,0.000000}%
\pgfsetstrokecolor{currentstroke}%
\pgfsetstrokeopacity{0.700000}%
\pgfsetdash{}{0pt}%
\pgfpathmoveto{\pgfqpoint{10.133801in}{1.003676in}}%
\pgfpathcurveto{\pgfqpoint{10.138845in}{1.003676in}}{\pgfqpoint{10.143683in}{1.005680in}}{\pgfqpoint{10.147249in}{1.009246in}}%
\pgfpathcurveto{\pgfqpoint{10.150816in}{1.012812in}}{\pgfqpoint{10.152820in}{1.017650in}}{\pgfqpoint{10.152820in}{1.022694in}}%
\pgfpathcurveto{\pgfqpoint{10.152820in}{1.027737in}}{\pgfqpoint{10.150816in}{1.032575in}}{\pgfqpoint{10.147249in}{1.036142in}}%
\pgfpathcurveto{\pgfqpoint{10.143683in}{1.039708in}}{\pgfqpoint{10.138845in}{1.041712in}}{\pgfqpoint{10.133801in}{1.041712in}}%
\pgfpathcurveto{\pgfqpoint{10.128758in}{1.041712in}}{\pgfqpoint{10.123920in}{1.039708in}}{\pgfqpoint{10.120354in}{1.036142in}}%
\pgfpathcurveto{\pgfqpoint{10.116787in}{1.032575in}}{\pgfqpoint{10.114783in}{1.027737in}}{\pgfqpoint{10.114783in}{1.022694in}}%
\pgfpathcurveto{\pgfqpoint{10.114783in}{1.017650in}}{\pgfqpoint{10.116787in}{1.012812in}}{\pgfqpoint{10.120354in}{1.009246in}}%
\pgfpathcurveto{\pgfqpoint{10.123920in}{1.005680in}}{\pgfqpoint{10.128758in}{1.003676in}}{\pgfqpoint{10.133801in}{1.003676in}}%
\pgfpathclose%
\pgfusepath{fill}%
\end{pgfscope}%
\begin{pgfscope}%
\pgfpathrectangle{\pgfqpoint{6.572727in}{0.473000in}}{\pgfqpoint{4.227273in}{3.311000in}}%
\pgfusepath{clip}%
\pgfsetbuttcap%
\pgfsetroundjoin%
\definecolor{currentfill}{rgb}{0.993248,0.906157,0.143936}%
\pgfsetfillcolor{currentfill}%
\pgfsetfillopacity{0.700000}%
\pgfsetlinewidth{0.000000pt}%
\definecolor{currentstroke}{rgb}{0.000000,0.000000,0.000000}%
\pgfsetstrokecolor{currentstroke}%
\pgfsetstrokeopacity{0.700000}%
\pgfsetdash{}{0pt}%
\pgfpathmoveto{\pgfqpoint{9.791706in}{1.646493in}}%
\pgfpathcurveto{\pgfqpoint{9.796749in}{1.646493in}}{\pgfqpoint{9.801587in}{1.648497in}}{\pgfqpoint{9.805153in}{1.652063in}}%
\pgfpathcurveto{\pgfqpoint{9.808720in}{1.655630in}}{\pgfqpoint{9.810724in}{1.660468in}}{\pgfqpoint{9.810724in}{1.665511in}}%
\pgfpathcurveto{\pgfqpoint{9.810724in}{1.670555in}}{\pgfqpoint{9.808720in}{1.675393in}}{\pgfqpoint{9.805153in}{1.678959in}}%
\pgfpathcurveto{\pgfqpoint{9.801587in}{1.682525in}}{\pgfqpoint{9.796749in}{1.684529in}}{\pgfqpoint{9.791706in}{1.684529in}}%
\pgfpathcurveto{\pgfqpoint{9.786662in}{1.684529in}}{\pgfqpoint{9.781824in}{1.682525in}}{\pgfqpoint{9.778258in}{1.678959in}}%
\pgfpathcurveto{\pgfqpoint{9.774691in}{1.675393in}}{\pgfqpoint{9.772687in}{1.670555in}}{\pgfqpoint{9.772687in}{1.665511in}}%
\pgfpathcurveto{\pgfqpoint{9.772687in}{1.660468in}}{\pgfqpoint{9.774691in}{1.655630in}}{\pgfqpoint{9.778258in}{1.652063in}}%
\pgfpathcurveto{\pgfqpoint{9.781824in}{1.648497in}}{\pgfqpoint{9.786662in}{1.646493in}}{\pgfqpoint{9.791706in}{1.646493in}}%
\pgfpathclose%
\pgfusepath{fill}%
\end{pgfscope}%
\begin{pgfscope}%
\pgfpathrectangle{\pgfqpoint{6.572727in}{0.473000in}}{\pgfqpoint{4.227273in}{3.311000in}}%
\pgfusepath{clip}%
\pgfsetbuttcap%
\pgfsetroundjoin%
\definecolor{currentfill}{rgb}{0.127568,0.566949,0.550556}%
\pgfsetfillcolor{currentfill}%
\pgfsetfillopacity{0.700000}%
\pgfsetlinewidth{0.000000pt}%
\definecolor{currentstroke}{rgb}{0.000000,0.000000,0.000000}%
\pgfsetstrokecolor{currentstroke}%
\pgfsetstrokeopacity{0.700000}%
\pgfsetdash{}{0pt}%
\pgfpathmoveto{\pgfqpoint{8.327714in}{2.957836in}}%
\pgfpathcurveto{\pgfqpoint{8.332758in}{2.957836in}}{\pgfqpoint{8.337595in}{2.959839in}}{\pgfqpoint{8.341162in}{2.963406in}}%
\pgfpathcurveto{\pgfqpoint{8.344728in}{2.966972in}}{\pgfqpoint{8.346732in}{2.971810in}}{\pgfqpoint{8.346732in}{2.976854in}}%
\pgfpathcurveto{\pgfqpoint{8.346732in}{2.981897in}}{\pgfqpoint{8.344728in}{2.986735in}}{\pgfqpoint{8.341162in}{2.990302in}}%
\pgfpathcurveto{\pgfqpoint{8.337595in}{2.993868in}}{\pgfqpoint{8.332758in}{2.995872in}}{\pgfqpoint{8.327714in}{2.995872in}}%
\pgfpathcurveto{\pgfqpoint{8.322670in}{2.995872in}}{\pgfqpoint{8.317833in}{2.993868in}}{\pgfqpoint{8.314266in}{2.990302in}}%
\pgfpathcurveto{\pgfqpoint{8.310700in}{2.986735in}}{\pgfqpoint{8.308696in}{2.981897in}}{\pgfqpoint{8.308696in}{2.976854in}}%
\pgfpathcurveto{\pgfqpoint{8.308696in}{2.971810in}}{\pgfqpoint{8.310700in}{2.966972in}}{\pgfqpoint{8.314266in}{2.963406in}}%
\pgfpathcurveto{\pgfqpoint{8.317833in}{2.959839in}}{\pgfqpoint{8.322670in}{2.957836in}}{\pgfqpoint{8.327714in}{2.957836in}}%
\pgfpathclose%
\pgfusepath{fill}%
\end{pgfscope}%
\begin{pgfscope}%
\pgfpathrectangle{\pgfqpoint{6.572727in}{0.473000in}}{\pgfqpoint{4.227273in}{3.311000in}}%
\pgfusepath{clip}%
\pgfsetbuttcap%
\pgfsetroundjoin%
\definecolor{currentfill}{rgb}{0.993248,0.906157,0.143936}%
\pgfsetfillcolor{currentfill}%
\pgfsetfillopacity{0.700000}%
\pgfsetlinewidth{0.000000pt}%
\definecolor{currentstroke}{rgb}{0.000000,0.000000,0.000000}%
\pgfsetstrokecolor{currentstroke}%
\pgfsetstrokeopacity{0.700000}%
\pgfsetdash{}{0pt}%
\pgfpathmoveto{\pgfqpoint{9.386556in}{1.603020in}}%
\pgfpathcurveto{\pgfqpoint{9.391600in}{1.603020in}}{\pgfqpoint{9.396438in}{1.605024in}}{\pgfqpoint{9.400004in}{1.608590in}}%
\pgfpathcurveto{\pgfqpoint{9.403570in}{1.612157in}}{\pgfqpoint{9.405574in}{1.616994in}}{\pgfqpoint{9.405574in}{1.622038in}}%
\pgfpathcurveto{\pgfqpoint{9.405574in}{1.627082in}}{\pgfqpoint{9.403570in}{1.631920in}}{\pgfqpoint{9.400004in}{1.635486in}}%
\pgfpathcurveto{\pgfqpoint{9.396438in}{1.639052in}}{\pgfqpoint{9.391600in}{1.641056in}}{\pgfqpoint{9.386556in}{1.641056in}}%
\pgfpathcurveto{\pgfqpoint{9.381513in}{1.641056in}}{\pgfqpoint{9.376675in}{1.639052in}}{\pgfqpoint{9.373108in}{1.635486in}}%
\pgfpathcurveto{\pgfqpoint{9.369542in}{1.631920in}}{\pgfqpoint{9.367538in}{1.627082in}}{\pgfqpoint{9.367538in}{1.622038in}}%
\pgfpathcurveto{\pgfqpoint{9.367538in}{1.616994in}}{\pgfqpoint{9.369542in}{1.612157in}}{\pgfqpoint{9.373108in}{1.608590in}}%
\pgfpathcurveto{\pgfqpoint{9.376675in}{1.605024in}}{\pgfqpoint{9.381513in}{1.603020in}}{\pgfqpoint{9.386556in}{1.603020in}}%
\pgfpathclose%
\pgfusepath{fill}%
\end{pgfscope}%
\begin{pgfscope}%
\pgfpathrectangle{\pgfqpoint{6.572727in}{0.473000in}}{\pgfqpoint{4.227273in}{3.311000in}}%
\pgfusepath{clip}%
\pgfsetbuttcap%
\pgfsetroundjoin%
\definecolor{currentfill}{rgb}{0.127568,0.566949,0.550556}%
\pgfsetfillcolor{currentfill}%
\pgfsetfillopacity{0.700000}%
\pgfsetlinewidth{0.000000pt}%
\definecolor{currentstroke}{rgb}{0.000000,0.000000,0.000000}%
\pgfsetstrokecolor{currentstroke}%
\pgfsetstrokeopacity{0.700000}%
\pgfsetdash{}{0pt}%
\pgfpathmoveto{\pgfqpoint{7.638030in}{1.285126in}}%
\pgfpathcurveto{\pgfqpoint{7.643074in}{1.285126in}}{\pgfqpoint{7.647912in}{1.287130in}}{\pgfqpoint{7.651478in}{1.290696in}}%
\pgfpathcurveto{\pgfqpoint{7.655045in}{1.294263in}}{\pgfqpoint{7.657049in}{1.299100in}}{\pgfqpoint{7.657049in}{1.304144in}}%
\pgfpathcurveto{\pgfqpoint{7.657049in}{1.309188in}}{\pgfqpoint{7.655045in}{1.314026in}}{\pgfqpoint{7.651478in}{1.317592in}}%
\pgfpathcurveto{\pgfqpoint{7.647912in}{1.321158in}}{\pgfqpoint{7.643074in}{1.323162in}}{\pgfqpoint{7.638030in}{1.323162in}}%
\pgfpathcurveto{\pgfqpoint{7.632987in}{1.323162in}}{\pgfqpoint{7.628149in}{1.321158in}}{\pgfqpoint{7.624583in}{1.317592in}}%
\pgfpathcurveto{\pgfqpoint{7.621016in}{1.314026in}}{\pgfqpoint{7.619012in}{1.309188in}}{\pgfqpoint{7.619012in}{1.304144in}}%
\pgfpathcurveto{\pgfqpoint{7.619012in}{1.299100in}}{\pgfqpoint{7.621016in}{1.294263in}}{\pgfqpoint{7.624583in}{1.290696in}}%
\pgfpathcurveto{\pgfqpoint{7.628149in}{1.287130in}}{\pgfqpoint{7.632987in}{1.285126in}}{\pgfqpoint{7.638030in}{1.285126in}}%
\pgfpathclose%
\pgfusepath{fill}%
\end{pgfscope}%
\begin{pgfscope}%
\pgfpathrectangle{\pgfqpoint{6.572727in}{0.473000in}}{\pgfqpoint{4.227273in}{3.311000in}}%
\pgfusepath{clip}%
\pgfsetbuttcap%
\pgfsetroundjoin%
\definecolor{currentfill}{rgb}{0.127568,0.566949,0.550556}%
\pgfsetfillcolor{currentfill}%
\pgfsetfillopacity{0.700000}%
\pgfsetlinewidth{0.000000pt}%
\definecolor{currentstroke}{rgb}{0.000000,0.000000,0.000000}%
\pgfsetstrokecolor{currentstroke}%
\pgfsetstrokeopacity{0.700000}%
\pgfsetdash{}{0pt}%
\pgfpathmoveto{\pgfqpoint{7.604546in}{2.371375in}}%
\pgfpathcurveto{\pgfqpoint{7.609590in}{2.371375in}}{\pgfqpoint{7.614427in}{2.373378in}}{\pgfqpoint{7.617994in}{2.376945in}}%
\pgfpathcurveto{\pgfqpoint{7.621560in}{2.380511in}}{\pgfqpoint{7.623564in}{2.385349in}}{\pgfqpoint{7.623564in}{2.390393in}}%
\pgfpathcurveto{\pgfqpoint{7.623564in}{2.395436in}}{\pgfqpoint{7.621560in}{2.400274in}}{\pgfqpoint{7.617994in}{2.403841in}}%
\pgfpathcurveto{\pgfqpoint{7.614427in}{2.407407in}}{\pgfqpoint{7.609590in}{2.409411in}}{\pgfqpoint{7.604546in}{2.409411in}}%
\pgfpathcurveto{\pgfqpoint{7.599502in}{2.409411in}}{\pgfqpoint{7.594665in}{2.407407in}}{\pgfqpoint{7.591098in}{2.403841in}}%
\pgfpathcurveto{\pgfqpoint{7.587532in}{2.400274in}}{\pgfqpoint{7.585528in}{2.395436in}}{\pgfqpoint{7.585528in}{2.390393in}}%
\pgfpathcurveto{\pgfqpoint{7.585528in}{2.385349in}}{\pgfqpoint{7.587532in}{2.380511in}}{\pgfqpoint{7.591098in}{2.376945in}}%
\pgfpathcurveto{\pgfqpoint{7.594665in}{2.373378in}}{\pgfqpoint{7.599502in}{2.371375in}}{\pgfqpoint{7.604546in}{2.371375in}}%
\pgfpathclose%
\pgfusepath{fill}%
\end{pgfscope}%
\begin{pgfscope}%
\pgfpathrectangle{\pgfqpoint{6.572727in}{0.473000in}}{\pgfqpoint{4.227273in}{3.311000in}}%
\pgfusepath{clip}%
\pgfsetbuttcap%
\pgfsetroundjoin%
\definecolor{currentfill}{rgb}{0.127568,0.566949,0.550556}%
\pgfsetfillcolor{currentfill}%
\pgfsetfillopacity{0.700000}%
\pgfsetlinewidth{0.000000pt}%
\definecolor{currentstroke}{rgb}{0.000000,0.000000,0.000000}%
\pgfsetstrokecolor{currentstroke}%
\pgfsetstrokeopacity{0.700000}%
\pgfsetdash{}{0pt}%
\pgfpathmoveto{\pgfqpoint{8.385452in}{2.795926in}}%
\pgfpathcurveto{\pgfqpoint{8.390496in}{2.795926in}}{\pgfqpoint{8.395334in}{2.797930in}}{\pgfqpoint{8.398900in}{2.801496in}}%
\pgfpathcurveto{\pgfqpoint{8.402467in}{2.805063in}}{\pgfqpoint{8.404470in}{2.809900in}}{\pgfqpoint{8.404470in}{2.814944in}}%
\pgfpathcurveto{\pgfqpoint{8.404470in}{2.819988in}}{\pgfqpoint{8.402467in}{2.824825in}}{\pgfqpoint{8.398900in}{2.828392in}}%
\pgfpathcurveto{\pgfqpoint{8.395334in}{2.831958in}}{\pgfqpoint{8.390496in}{2.833962in}}{\pgfqpoint{8.385452in}{2.833962in}}%
\pgfpathcurveto{\pgfqpoint{8.380409in}{2.833962in}}{\pgfqpoint{8.375571in}{2.831958in}}{\pgfqpoint{8.372004in}{2.828392in}}%
\pgfpathcurveto{\pgfqpoint{8.368438in}{2.824825in}}{\pgfqpoint{8.366434in}{2.819988in}}{\pgfqpoint{8.366434in}{2.814944in}}%
\pgfpathcurveto{\pgfqpoint{8.366434in}{2.809900in}}{\pgfqpoint{8.368438in}{2.805063in}}{\pgfqpoint{8.372004in}{2.801496in}}%
\pgfpathcurveto{\pgfqpoint{8.375571in}{2.797930in}}{\pgfqpoint{8.380409in}{2.795926in}}{\pgfqpoint{8.385452in}{2.795926in}}%
\pgfpathclose%
\pgfusepath{fill}%
\end{pgfscope}%
\begin{pgfscope}%
\pgfpathrectangle{\pgfqpoint{6.572727in}{0.473000in}}{\pgfqpoint{4.227273in}{3.311000in}}%
\pgfusepath{clip}%
\pgfsetbuttcap%
\pgfsetroundjoin%
\definecolor{currentfill}{rgb}{0.127568,0.566949,0.550556}%
\pgfsetfillcolor{currentfill}%
\pgfsetfillopacity{0.700000}%
\pgfsetlinewidth{0.000000pt}%
\definecolor{currentstroke}{rgb}{0.000000,0.000000,0.000000}%
\pgfsetstrokecolor{currentstroke}%
\pgfsetstrokeopacity{0.700000}%
\pgfsetdash{}{0pt}%
\pgfpathmoveto{\pgfqpoint{8.414996in}{2.900755in}}%
\pgfpathcurveto{\pgfqpoint{8.420040in}{2.900755in}}{\pgfqpoint{8.424878in}{2.902759in}}{\pgfqpoint{8.428444in}{2.906325in}}%
\pgfpathcurveto{\pgfqpoint{8.432010in}{2.909892in}}{\pgfqpoint{8.434014in}{2.914729in}}{\pgfqpoint{8.434014in}{2.919773in}}%
\pgfpathcurveto{\pgfqpoint{8.434014in}{2.924817in}}{\pgfqpoint{8.432010in}{2.929654in}}{\pgfqpoint{8.428444in}{2.933221in}}%
\pgfpathcurveto{\pgfqpoint{8.424878in}{2.936787in}}{\pgfqpoint{8.420040in}{2.938791in}}{\pgfqpoint{8.414996in}{2.938791in}}%
\pgfpathcurveto{\pgfqpoint{8.409952in}{2.938791in}}{\pgfqpoint{8.405115in}{2.936787in}}{\pgfqpoint{8.401548in}{2.933221in}}%
\pgfpathcurveto{\pgfqpoint{8.397982in}{2.929654in}}{\pgfqpoint{8.395978in}{2.924817in}}{\pgfqpoint{8.395978in}{2.919773in}}%
\pgfpathcurveto{\pgfqpoint{8.395978in}{2.914729in}}{\pgfqpoint{8.397982in}{2.909892in}}{\pgfqpoint{8.401548in}{2.906325in}}%
\pgfpathcurveto{\pgfqpoint{8.405115in}{2.902759in}}{\pgfqpoint{8.409952in}{2.900755in}}{\pgfqpoint{8.414996in}{2.900755in}}%
\pgfpathclose%
\pgfusepath{fill}%
\end{pgfscope}%
\begin{pgfscope}%
\pgfpathrectangle{\pgfqpoint{6.572727in}{0.473000in}}{\pgfqpoint{4.227273in}{3.311000in}}%
\pgfusepath{clip}%
\pgfsetbuttcap%
\pgfsetroundjoin%
\definecolor{currentfill}{rgb}{0.993248,0.906157,0.143936}%
\pgfsetfillcolor{currentfill}%
\pgfsetfillopacity{0.700000}%
\pgfsetlinewidth{0.000000pt}%
\definecolor{currentstroke}{rgb}{0.000000,0.000000,0.000000}%
\pgfsetstrokecolor{currentstroke}%
\pgfsetstrokeopacity{0.700000}%
\pgfsetdash{}{0pt}%
\pgfpathmoveto{\pgfqpoint{9.486798in}{1.231874in}}%
\pgfpathcurveto{\pgfqpoint{9.491841in}{1.231874in}}{\pgfqpoint{9.496679in}{1.233878in}}{\pgfqpoint{9.500246in}{1.237444in}}%
\pgfpathcurveto{\pgfqpoint{9.503812in}{1.241010in}}{\pgfqpoint{9.505816in}{1.245848in}}{\pgfqpoint{9.505816in}{1.250892in}}%
\pgfpathcurveto{\pgfqpoint{9.505816in}{1.255936in}}{\pgfqpoint{9.503812in}{1.260773in}}{\pgfqpoint{9.500246in}{1.264340in}}%
\pgfpathcurveto{\pgfqpoint{9.496679in}{1.267906in}}{\pgfqpoint{9.491841in}{1.269910in}}{\pgfqpoint{9.486798in}{1.269910in}}%
\pgfpathcurveto{\pgfqpoint{9.481754in}{1.269910in}}{\pgfqpoint{9.476916in}{1.267906in}}{\pgfqpoint{9.473350in}{1.264340in}}%
\pgfpathcurveto{\pgfqpoint{9.469783in}{1.260773in}}{\pgfqpoint{9.467780in}{1.255936in}}{\pgfqpoint{9.467780in}{1.250892in}}%
\pgfpathcurveto{\pgfqpoint{9.467780in}{1.245848in}}{\pgfqpoint{9.469783in}{1.241010in}}{\pgfqpoint{9.473350in}{1.237444in}}%
\pgfpathcurveto{\pgfqpoint{9.476916in}{1.233878in}}{\pgfqpoint{9.481754in}{1.231874in}}{\pgfqpoint{9.486798in}{1.231874in}}%
\pgfpathclose%
\pgfusepath{fill}%
\end{pgfscope}%
\begin{pgfscope}%
\pgfpathrectangle{\pgfqpoint{6.572727in}{0.473000in}}{\pgfqpoint{4.227273in}{3.311000in}}%
\pgfusepath{clip}%
\pgfsetbuttcap%
\pgfsetroundjoin%
\definecolor{currentfill}{rgb}{0.127568,0.566949,0.550556}%
\pgfsetfillcolor{currentfill}%
\pgfsetfillopacity{0.700000}%
\pgfsetlinewidth{0.000000pt}%
\definecolor{currentstroke}{rgb}{0.000000,0.000000,0.000000}%
\pgfsetstrokecolor{currentstroke}%
\pgfsetstrokeopacity{0.700000}%
\pgfsetdash{}{0pt}%
\pgfpathmoveto{\pgfqpoint{7.913414in}{1.774695in}}%
\pgfpathcurveto{\pgfqpoint{7.918458in}{1.774695in}}{\pgfqpoint{7.923295in}{1.776699in}}{\pgfqpoint{7.926862in}{1.780265in}}%
\pgfpathcurveto{\pgfqpoint{7.930428in}{1.783832in}}{\pgfqpoint{7.932432in}{1.788669in}}{\pgfqpoint{7.932432in}{1.793713in}}%
\pgfpathcurveto{\pgfqpoint{7.932432in}{1.798757in}}{\pgfqpoint{7.930428in}{1.803594in}}{\pgfqpoint{7.926862in}{1.807161in}}%
\pgfpathcurveto{\pgfqpoint{7.923295in}{1.810727in}}{\pgfqpoint{7.918458in}{1.812731in}}{\pgfqpoint{7.913414in}{1.812731in}}%
\pgfpathcurveto{\pgfqpoint{7.908370in}{1.812731in}}{\pgfqpoint{7.903532in}{1.810727in}}{\pgfqpoint{7.899966in}{1.807161in}}%
\pgfpathcurveto{\pgfqpoint{7.896400in}{1.803594in}}{\pgfqpoint{7.894396in}{1.798757in}}{\pgfqpoint{7.894396in}{1.793713in}}%
\pgfpathcurveto{\pgfqpoint{7.894396in}{1.788669in}}{\pgfqpoint{7.896400in}{1.783832in}}{\pgfqpoint{7.899966in}{1.780265in}}%
\pgfpathcurveto{\pgfqpoint{7.903532in}{1.776699in}}{\pgfqpoint{7.908370in}{1.774695in}}{\pgfqpoint{7.913414in}{1.774695in}}%
\pgfpathclose%
\pgfusepath{fill}%
\end{pgfscope}%
\begin{pgfscope}%
\pgfpathrectangle{\pgfqpoint{6.572727in}{0.473000in}}{\pgfqpoint{4.227273in}{3.311000in}}%
\pgfusepath{clip}%
\pgfsetbuttcap%
\pgfsetroundjoin%
\definecolor{currentfill}{rgb}{0.127568,0.566949,0.550556}%
\pgfsetfillcolor{currentfill}%
\pgfsetfillopacity{0.700000}%
\pgfsetlinewidth{0.000000pt}%
\definecolor{currentstroke}{rgb}{0.000000,0.000000,0.000000}%
\pgfsetstrokecolor{currentstroke}%
\pgfsetstrokeopacity{0.700000}%
\pgfsetdash{}{0pt}%
\pgfpathmoveto{\pgfqpoint{7.821927in}{2.080158in}}%
\pgfpathcurveto{\pgfqpoint{7.826971in}{2.080158in}}{\pgfqpoint{7.831808in}{2.082162in}}{\pgfqpoint{7.835375in}{2.085728in}}%
\pgfpathcurveto{\pgfqpoint{7.838941in}{2.089295in}}{\pgfqpoint{7.840945in}{2.094132in}}{\pgfqpoint{7.840945in}{2.099176in}}%
\pgfpathcurveto{\pgfqpoint{7.840945in}{2.104220in}}{\pgfqpoint{7.838941in}{2.109057in}}{\pgfqpoint{7.835375in}{2.112624in}}%
\pgfpathcurveto{\pgfqpoint{7.831808in}{2.116190in}}{\pgfqpoint{7.826971in}{2.118194in}}{\pgfqpoint{7.821927in}{2.118194in}}%
\pgfpathcurveto{\pgfqpoint{7.816883in}{2.118194in}}{\pgfqpoint{7.812045in}{2.116190in}}{\pgfqpoint{7.808479in}{2.112624in}}%
\pgfpathcurveto{\pgfqpoint{7.804913in}{2.109057in}}{\pgfqpoint{7.802909in}{2.104220in}}{\pgfqpoint{7.802909in}{2.099176in}}%
\pgfpathcurveto{\pgfqpoint{7.802909in}{2.094132in}}{\pgfqpoint{7.804913in}{2.089295in}}{\pgfqpoint{7.808479in}{2.085728in}}%
\pgfpathcurveto{\pgfqpoint{7.812045in}{2.082162in}}{\pgfqpoint{7.816883in}{2.080158in}}{\pgfqpoint{7.821927in}{2.080158in}}%
\pgfpathclose%
\pgfusepath{fill}%
\end{pgfscope}%
\begin{pgfscope}%
\pgfpathrectangle{\pgfqpoint{6.572727in}{0.473000in}}{\pgfqpoint{4.227273in}{3.311000in}}%
\pgfusepath{clip}%
\pgfsetbuttcap%
\pgfsetroundjoin%
\definecolor{currentfill}{rgb}{0.127568,0.566949,0.550556}%
\pgfsetfillcolor{currentfill}%
\pgfsetfillopacity{0.700000}%
\pgfsetlinewidth{0.000000pt}%
\definecolor{currentstroke}{rgb}{0.000000,0.000000,0.000000}%
\pgfsetstrokecolor{currentstroke}%
\pgfsetstrokeopacity{0.700000}%
\pgfsetdash{}{0pt}%
\pgfpathmoveto{\pgfqpoint{8.164135in}{2.444711in}}%
\pgfpathcurveto{\pgfqpoint{8.169179in}{2.444711in}}{\pgfqpoint{8.174016in}{2.446715in}}{\pgfqpoint{8.177583in}{2.450282in}}%
\pgfpathcurveto{\pgfqpoint{8.181149in}{2.453848in}}{\pgfqpoint{8.183153in}{2.458686in}}{\pgfqpoint{8.183153in}{2.463729in}}%
\pgfpathcurveto{\pgfqpoint{8.183153in}{2.468773in}}{\pgfqpoint{8.181149in}{2.473611in}}{\pgfqpoint{8.177583in}{2.477177in}}%
\pgfpathcurveto{\pgfqpoint{8.174016in}{2.480744in}}{\pgfqpoint{8.169179in}{2.482748in}}{\pgfqpoint{8.164135in}{2.482748in}}%
\pgfpathcurveto{\pgfqpoint{8.159091in}{2.482748in}}{\pgfqpoint{8.154253in}{2.480744in}}{\pgfqpoint{8.150687in}{2.477177in}}%
\pgfpathcurveto{\pgfqpoint{8.147121in}{2.473611in}}{\pgfqpoint{8.145117in}{2.468773in}}{\pgfqpoint{8.145117in}{2.463729in}}%
\pgfpathcurveto{\pgfqpoint{8.145117in}{2.458686in}}{\pgfqpoint{8.147121in}{2.453848in}}{\pgfqpoint{8.150687in}{2.450282in}}%
\pgfpathcurveto{\pgfqpoint{8.154253in}{2.446715in}}{\pgfqpoint{8.159091in}{2.444711in}}{\pgfqpoint{8.164135in}{2.444711in}}%
\pgfpathclose%
\pgfusepath{fill}%
\end{pgfscope}%
\begin{pgfscope}%
\pgfpathrectangle{\pgfqpoint{6.572727in}{0.473000in}}{\pgfqpoint{4.227273in}{3.311000in}}%
\pgfusepath{clip}%
\pgfsetbuttcap%
\pgfsetroundjoin%
\definecolor{currentfill}{rgb}{0.127568,0.566949,0.550556}%
\pgfsetfillcolor{currentfill}%
\pgfsetfillopacity{0.700000}%
\pgfsetlinewidth{0.000000pt}%
\definecolor{currentstroke}{rgb}{0.000000,0.000000,0.000000}%
\pgfsetstrokecolor{currentstroke}%
\pgfsetstrokeopacity{0.700000}%
\pgfsetdash{}{0pt}%
\pgfpathmoveto{\pgfqpoint{7.910866in}{2.665424in}}%
\pgfpathcurveto{\pgfqpoint{7.915910in}{2.665424in}}{\pgfqpoint{7.920748in}{2.667428in}}{\pgfqpoint{7.924314in}{2.670994in}}%
\pgfpathcurveto{\pgfqpoint{7.927881in}{2.674561in}}{\pgfqpoint{7.929885in}{2.679398in}}{\pgfqpoint{7.929885in}{2.684442in}}%
\pgfpathcurveto{\pgfqpoint{7.929885in}{2.689486in}}{\pgfqpoint{7.927881in}{2.694324in}}{\pgfqpoint{7.924314in}{2.697890in}}%
\pgfpathcurveto{\pgfqpoint{7.920748in}{2.701456in}}{\pgfqpoint{7.915910in}{2.703460in}}{\pgfqpoint{7.910866in}{2.703460in}}%
\pgfpathcurveto{\pgfqpoint{7.905823in}{2.703460in}}{\pgfqpoint{7.900985in}{2.701456in}}{\pgfqpoint{7.897419in}{2.697890in}}%
\pgfpathcurveto{\pgfqpoint{7.893852in}{2.694324in}}{\pgfqpoint{7.891848in}{2.689486in}}{\pgfqpoint{7.891848in}{2.684442in}}%
\pgfpathcurveto{\pgfqpoint{7.891848in}{2.679398in}}{\pgfqpoint{7.893852in}{2.674561in}}{\pgfqpoint{7.897419in}{2.670994in}}%
\pgfpathcurveto{\pgfqpoint{7.900985in}{2.667428in}}{\pgfqpoint{7.905823in}{2.665424in}}{\pgfqpoint{7.910866in}{2.665424in}}%
\pgfpathclose%
\pgfusepath{fill}%
\end{pgfscope}%
\begin{pgfscope}%
\pgfpathrectangle{\pgfqpoint{6.572727in}{0.473000in}}{\pgfqpoint{4.227273in}{3.311000in}}%
\pgfusepath{clip}%
\pgfsetbuttcap%
\pgfsetroundjoin%
\definecolor{currentfill}{rgb}{0.127568,0.566949,0.550556}%
\pgfsetfillcolor{currentfill}%
\pgfsetfillopacity{0.700000}%
\pgfsetlinewidth{0.000000pt}%
\definecolor{currentstroke}{rgb}{0.000000,0.000000,0.000000}%
\pgfsetstrokecolor{currentstroke}%
\pgfsetstrokeopacity{0.700000}%
\pgfsetdash{}{0pt}%
\pgfpathmoveto{\pgfqpoint{7.957590in}{2.133546in}}%
\pgfpathcurveto{\pgfqpoint{7.962633in}{2.133546in}}{\pgfqpoint{7.967471in}{2.135550in}}{\pgfqpoint{7.971037in}{2.139117in}}%
\pgfpathcurveto{\pgfqpoint{7.974604in}{2.142683in}}{\pgfqpoint{7.976608in}{2.147521in}}{\pgfqpoint{7.976608in}{2.152564in}}%
\pgfpathcurveto{\pgfqpoint{7.976608in}{2.157608in}}{\pgfqpoint{7.974604in}{2.162446in}}{\pgfqpoint{7.971037in}{2.166012in}}%
\pgfpathcurveto{\pgfqpoint{7.967471in}{2.169579in}}{\pgfqpoint{7.962633in}{2.171583in}}{\pgfqpoint{7.957590in}{2.171583in}}%
\pgfpathcurveto{\pgfqpoint{7.952546in}{2.171583in}}{\pgfqpoint{7.947708in}{2.169579in}}{\pgfqpoint{7.944142in}{2.166012in}}%
\pgfpathcurveto{\pgfqpoint{7.940575in}{2.162446in}}{\pgfqpoint{7.938571in}{2.157608in}}{\pgfqpoint{7.938571in}{2.152564in}}%
\pgfpathcurveto{\pgfqpoint{7.938571in}{2.147521in}}{\pgfqpoint{7.940575in}{2.142683in}}{\pgfqpoint{7.944142in}{2.139117in}}%
\pgfpathcurveto{\pgfqpoint{7.947708in}{2.135550in}}{\pgfqpoint{7.952546in}{2.133546in}}{\pgfqpoint{7.957590in}{2.133546in}}%
\pgfpathclose%
\pgfusepath{fill}%
\end{pgfscope}%
\begin{pgfscope}%
\pgfpathrectangle{\pgfqpoint{6.572727in}{0.473000in}}{\pgfqpoint{4.227273in}{3.311000in}}%
\pgfusepath{clip}%
\pgfsetbuttcap%
\pgfsetroundjoin%
\definecolor{currentfill}{rgb}{0.993248,0.906157,0.143936}%
\pgfsetfillcolor{currentfill}%
\pgfsetfillopacity{0.700000}%
\pgfsetlinewidth{0.000000pt}%
\definecolor{currentstroke}{rgb}{0.000000,0.000000,0.000000}%
\pgfsetstrokecolor{currentstroke}%
\pgfsetstrokeopacity{0.700000}%
\pgfsetdash{}{0pt}%
\pgfpathmoveto{\pgfqpoint{10.137840in}{1.605803in}}%
\pgfpathcurveto{\pgfqpoint{10.142883in}{1.605803in}}{\pgfqpoint{10.147721in}{1.607807in}}{\pgfqpoint{10.151288in}{1.611373in}}%
\pgfpathcurveto{\pgfqpoint{10.154854in}{1.614940in}}{\pgfqpoint{10.156858in}{1.619778in}}{\pgfqpoint{10.156858in}{1.624821in}}%
\pgfpathcurveto{\pgfqpoint{10.156858in}{1.629865in}}{\pgfqpoint{10.154854in}{1.634703in}}{\pgfqpoint{10.151288in}{1.638269in}}%
\pgfpathcurveto{\pgfqpoint{10.147721in}{1.641835in}}{\pgfqpoint{10.142883in}{1.643839in}}{\pgfqpoint{10.137840in}{1.643839in}}%
\pgfpathcurveto{\pgfqpoint{10.132796in}{1.643839in}}{\pgfqpoint{10.127958in}{1.641835in}}{\pgfqpoint{10.124392in}{1.638269in}}%
\pgfpathcurveto{\pgfqpoint{10.120825in}{1.634703in}}{\pgfqpoint{10.118822in}{1.629865in}}{\pgfqpoint{10.118822in}{1.624821in}}%
\pgfpathcurveto{\pgfqpoint{10.118822in}{1.619778in}}{\pgfqpoint{10.120825in}{1.614940in}}{\pgfqpoint{10.124392in}{1.611373in}}%
\pgfpathcurveto{\pgfqpoint{10.127958in}{1.607807in}}{\pgfqpoint{10.132796in}{1.605803in}}{\pgfqpoint{10.137840in}{1.605803in}}%
\pgfpathclose%
\pgfusepath{fill}%
\end{pgfscope}%
\begin{pgfscope}%
\pgfpathrectangle{\pgfqpoint{6.572727in}{0.473000in}}{\pgfqpoint{4.227273in}{3.311000in}}%
\pgfusepath{clip}%
\pgfsetbuttcap%
\pgfsetroundjoin%
\definecolor{currentfill}{rgb}{0.127568,0.566949,0.550556}%
\pgfsetfillcolor{currentfill}%
\pgfsetfillopacity{0.700000}%
\pgfsetlinewidth{0.000000pt}%
\definecolor{currentstroke}{rgb}{0.000000,0.000000,0.000000}%
\pgfsetstrokecolor{currentstroke}%
\pgfsetstrokeopacity{0.700000}%
\pgfsetdash{}{0pt}%
\pgfpathmoveto{\pgfqpoint{7.864466in}{1.585323in}}%
\pgfpathcurveto{\pgfqpoint{7.869509in}{1.585323in}}{\pgfqpoint{7.874347in}{1.587327in}}{\pgfqpoint{7.877913in}{1.590894in}}%
\pgfpathcurveto{\pgfqpoint{7.881480in}{1.594460in}}{\pgfqpoint{7.883484in}{1.599298in}}{\pgfqpoint{7.883484in}{1.604341in}}%
\pgfpathcurveto{\pgfqpoint{7.883484in}{1.609385in}}{\pgfqpoint{7.881480in}{1.614223in}}{\pgfqpoint{7.877913in}{1.617789in}}%
\pgfpathcurveto{\pgfqpoint{7.874347in}{1.621356in}}{\pgfqpoint{7.869509in}{1.623360in}}{\pgfqpoint{7.864466in}{1.623360in}}%
\pgfpathcurveto{\pgfqpoint{7.859422in}{1.623360in}}{\pgfqpoint{7.854584in}{1.621356in}}{\pgfqpoint{7.851018in}{1.617789in}}%
\pgfpathcurveto{\pgfqpoint{7.847451in}{1.614223in}}{\pgfqpoint{7.845447in}{1.609385in}}{\pgfqpoint{7.845447in}{1.604341in}}%
\pgfpathcurveto{\pgfqpoint{7.845447in}{1.599298in}}{\pgfqpoint{7.847451in}{1.594460in}}{\pgfqpoint{7.851018in}{1.590894in}}%
\pgfpathcurveto{\pgfqpoint{7.854584in}{1.587327in}}{\pgfqpoint{7.859422in}{1.585323in}}{\pgfqpoint{7.864466in}{1.585323in}}%
\pgfpathclose%
\pgfusepath{fill}%
\end{pgfscope}%
\begin{pgfscope}%
\pgfpathrectangle{\pgfqpoint{6.572727in}{0.473000in}}{\pgfqpoint{4.227273in}{3.311000in}}%
\pgfusepath{clip}%
\pgfsetbuttcap%
\pgfsetroundjoin%
\definecolor{currentfill}{rgb}{0.993248,0.906157,0.143936}%
\pgfsetfillcolor{currentfill}%
\pgfsetfillopacity{0.700000}%
\pgfsetlinewidth{0.000000pt}%
\definecolor{currentstroke}{rgb}{0.000000,0.000000,0.000000}%
\pgfsetstrokecolor{currentstroke}%
\pgfsetstrokeopacity{0.700000}%
\pgfsetdash{}{0pt}%
\pgfpathmoveto{\pgfqpoint{9.629775in}{1.525353in}}%
\pgfpathcurveto{\pgfqpoint{9.634819in}{1.525353in}}{\pgfqpoint{9.639657in}{1.527357in}}{\pgfqpoint{9.643223in}{1.530923in}}%
\pgfpathcurveto{\pgfqpoint{9.646790in}{1.534490in}}{\pgfqpoint{9.648793in}{1.539328in}}{\pgfqpoint{9.648793in}{1.544371in}}%
\pgfpathcurveto{\pgfqpoint{9.648793in}{1.549415in}}{\pgfqpoint{9.646790in}{1.554253in}}{\pgfqpoint{9.643223in}{1.557819in}}%
\pgfpathcurveto{\pgfqpoint{9.639657in}{1.561386in}}{\pgfqpoint{9.634819in}{1.563389in}}{\pgfqpoint{9.629775in}{1.563389in}}%
\pgfpathcurveto{\pgfqpoint{9.624732in}{1.563389in}}{\pgfqpoint{9.619894in}{1.561386in}}{\pgfqpoint{9.616327in}{1.557819in}}%
\pgfpathcurveto{\pgfqpoint{9.612761in}{1.554253in}}{\pgfqpoint{9.610757in}{1.549415in}}{\pgfqpoint{9.610757in}{1.544371in}}%
\pgfpathcurveto{\pgfqpoint{9.610757in}{1.539328in}}{\pgfqpoint{9.612761in}{1.534490in}}{\pgfqpoint{9.616327in}{1.530923in}}%
\pgfpathcurveto{\pgfqpoint{9.619894in}{1.527357in}}{\pgfqpoint{9.624732in}{1.525353in}}{\pgfqpoint{9.629775in}{1.525353in}}%
\pgfpathclose%
\pgfusepath{fill}%
\end{pgfscope}%
\begin{pgfscope}%
\pgfpathrectangle{\pgfqpoint{6.572727in}{0.473000in}}{\pgfqpoint{4.227273in}{3.311000in}}%
\pgfusepath{clip}%
\pgfsetbuttcap%
\pgfsetroundjoin%
\definecolor{currentfill}{rgb}{0.993248,0.906157,0.143936}%
\pgfsetfillcolor{currentfill}%
\pgfsetfillopacity{0.700000}%
\pgfsetlinewidth{0.000000pt}%
\definecolor{currentstroke}{rgb}{0.000000,0.000000,0.000000}%
\pgfsetstrokecolor{currentstroke}%
\pgfsetstrokeopacity{0.700000}%
\pgfsetdash{}{0pt}%
\pgfpathmoveto{\pgfqpoint{9.785555in}{1.625864in}}%
\pgfpathcurveto{\pgfqpoint{9.790598in}{1.625864in}}{\pgfqpoint{9.795436in}{1.627868in}}{\pgfqpoint{9.799002in}{1.631434in}}%
\pgfpathcurveto{\pgfqpoint{9.802569in}{1.635001in}}{\pgfqpoint{9.804573in}{1.639839in}}{\pgfqpoint{9.804573in}{1.644882in}}%
\pgfpathcurveto{\pgfqpoint{9.804573in}{1.649926in}}{\pgfqpoint{9.802569in}{1.654764in}}{\pgfqpoint{9.799002in}{1.658330in}}%
\pgfpathcurveto{\pgfqpoint{9.795436in}{1.661896in}}{\pgfqpoint{9.790598in}{1.663900in}}{\pgfqpoint{9.785555in}{1.663900in}}%
\pgfpathcurveto{\pgfqpoint{9.780511in}{1.663900in}}{\pgfqpoint{9.775673in}{1.661896in}}{\pgfqpoint{9.772107in}{1.658330in}}%
\pgfpathcurveto{\pgfqpoint{9.768540in}{1.654764in}}{\pgfqpoint{9.766536in}{1.649926in}}{\pgfqpoint{9.766536in}{1.644882in}}%
\pgfpathcurveto{\pgfqpoint{9.766536in}{1.639839in}}{\pgfqpoint{9.768540in}{1.635001in}}{\pgfqpoint{9.772107in}{1.631434in}}%
\pgfpathcurveto{\pgfqpoint{9.775673in}{1.627868in}}{\pgfqpoint{9.780511in}{1.625864in}}{\pgfqpoint{9.785555in}{1.625864in}}%
\pgfpathclose%
\pgfusepath{fill}%
\end{pgfscope}%
\begin{pgfscope}%
\pgfpathrectangle{\pgfqpoint{6.572727in}{0.473000in}}{\pgfqpoint{4.227273in}{3.311000in}}%
\pgfusepath{clip}%
\pgfsetbuttcap%
\pgfsetroundjoin%
\definecolor{currentfill}{rgb}{0.993248,0.906157,0.143936}%
\pgfsetfillcolor{currentfill}%
\pgfsetfillopacity{0.700000}%
\pgfsetlinewidth{0.000000pt}%
\definecolor{currentstroke}{rgb}{0.000000,0.000000,0.000000}%
\pgfsetstrokecolor{currentstroke}%
\pgfsetstrokeopacity{0.700000}%
\pgfsetdash{}{0pt}%
\pgfpathmoveto{\pgfqpoint{9.315914in}{1.236061in}}%
\pgfpathcurveto{\pgfqpoint{9.320957in}{1.236061in}}{\pgfqpoint{9.325795in}{1.238065in}}{\pgfqpoint{9.329362in}{1.241631in}}%
\pgfpathcurveto{\pgfqpoint{9.332928in}{1.245197in}}{\pgfqpoint{9.334932in}{1.250035in}}{\pgfqpoint{9.334932in}{1.255079in}}%
\pgfpathcurveto{\pgfqpoint{9.334932in}{1.260123in}}{\pgfqpoint{9.332928in}{1.264960in}}{\pgfqpoint{9.329362in}{1.268527in}}%
\pgfpathcurveto{\pgfqpoint{9.325795in}{1.272093in}}{\pgfqpoint{9.320957in}{1.274097in}}{\pgfqpoint{9.315914in}{1.274097in}}%
\pgfpathcurveto{\pgfqpoint{9.310870in}{1.274097in}}{\pgfqpoint{9.306032in}{1.272093in}}{\pgfqpoint{9.302466in}{1.268527in}}%
\pgfpathcurveto{\pgfqpoint{9.298899in}{1.264960in}}{\pgfqpoint{9.296896in}{1.260123in}}{\pgfqpoint{9.296896in}{1.255079in}}%
\pgfpathcurveto{\pgfqpoint{9.296896in}{1.250035in}}{\pgfqpoint{9.298899in}{1.245197in}}{\pgfqpoint{9.302466in}{1.241631in}}%
\pgfpathcurveto{\pgfqpoint{9.306032in}{1.238065in}}{\pgfqpoint{9.310870in}{1.236061in}}{\pgfqpoint{9.315914in}{1.236061in}}%
\pgfpathclose%
\pgfusepath{fill}%
\end{pgfscope}%
\begin{pgfscope}%
\pgfpathrectangle{\pgfqpoint{6.572727in}{0.473000in}}{\pgfqpoint{4.227273in}{3.311000in}}%
\pgfusepath{clip}%
\pgfsetbuttcap%
\pgfsetroundjoin%
\definecolor{currentfill}{rgb}{0.127568,0.566949,0.550556}%
\pgfsetfillcolor{currentfill}%
\pgfsetfillopacity{0.700000}%
\pgfsetlinewidth{0.000000pt}%
\definecolor{currentstroke}{rgb}{0.000000,0.000000,0.000000}%
\pgfsetstrokecolor{currentstroke}%
\pgfsetstrokeopacity{0.700000}%
\pgfsetdash{}{0pt}%
\pgfpathmoveto{\pgfqpoint{8.526535in}{2.926582in}}%
\pgfpathcurveto{\pgfqpoint{8.531579in}{2.926582in}}{\pgfqpoint{8.536417in}{2.928585in}}{\pgfqpoint{8.539983in}{2.932152in}}%
\pgfpathcurveto{\pgfqpoint{8.543550in}{2.935718in}}{\pgfqpoint{8.545553in}{2.940556in}}{\pgfqpoint{8.545553in}{2.945600in}}%
\pgfpathcurveto{\pgfqpoint{8.545553in}{2.950643in}}{\pgfqpoint{8.543550in}{2.955481in}}{\pgfqpoint{8.539983in}{2.959048in}}%
\pgfpathcurveto{\pgfqpoint{8.536417in}{2.962614in}}{\pgfqpoint{8.531579in}{2.964618in}}{\pgfqpoint{8.526535in}{2.964618in}}%
\pgfpathcurveto{\pgfqpoint{8.521492in}{2.964618in}}{\pgfqpoint{8.516654in}{2.962614in}}{\pgfqpoint{8.513087in}{2.959048in}}%
\pgfpathcurveto{\pgfqpoint{8.509521in}{2.955481in}}{\pgfqpoint{8.507517in}{2.950643in}}{\pgfqpoint{8.507517in}{2.945600in}}%
\pgfpathcurveto{\pgfqpoint{8.507517in}{2.940556in}}{\pgfqpoint{8.509521in}{2.935718in}}{\pgfqpoint{8.513087in}{2.932152in}}%
\pgfpathcurveto{\pgfqpoint{8.516654in}{2.928585in}}{\pgfqpoint{8.521492in}{2.926582in}}{\pgfqpoint{8.526535in}{2.926582in}}%
\pgfpathclose%
\pgfusepath{fill}%
\end{pgfscope}%
\begin{pgfscope}%
\pgfpathrectangle{\pgfqpoint{6.572727in}{0.473000in}}{\pgfqpoint{4.227273in}{3.311000in}}%
\pgfusepath{clip}%
\pgfsetbuttcap%
\pgfsetroundjoin%
\definecolor{currentfill}{rgb}{0.127568,0.566949,0.550556}%
\pgfsetfillcolor{currentfill}%
\pgfsetfillopacity{0.700000}%
\pgfsetlinewidth{0.000000pt}%
\definecolor{currentstroke}{rgb}{0.000000,0.000000,0.000000}%
\pgfsetstrokecolor{currentstroke}%
\pgfsetstrokeopacity{0.700000}%
\pgfsetdash{}{0pt}%
\pgfpathmoveto{\pgfqpoint{7.632062in}{1.185624in}}%
\pgfpathcurveto{\pgfqpoint{7.637105in}{1.185624in}}{\pgfqpoint{7.641943in}{1.187628in}}{\pgfqpoint{7.645510in}{1.191195in}}%
\pgfpathcurveto{\pgfqpoint{7.649076in}{1.194761in}}{\pgfqpoint{7.651080in}{1.199599in}}{\pgfqpoint{7.651080in}{1.204643in}}%
\pgfpathcurveto{\pgfqpoint{7.651080in}{1.209686in}}{\pgfqpoint{7.649076in}{1.214524in}}{\pgfqpoint{7.645510in}{1.218090in}}%
\pgfpathcurveto{\pgfqpoint{7.641943in}{1.221657in}}{\pgfqpoint{7.637105in}{1.223661in}}{\pgfqpoint{7.632062in}{1.223661in}}%
\pgfpathcurveto{\pgfqpoint{7.627018in}{1.223661in}}{\pgfqpoint{7.622180in}{1.221657in}}{\pgfqpoint{7.618614in}{1.218090in}}%
\pgfpathcurveto{\pgfqpoint{7.615047in}{1.214524in}}{\pgfqpoint{7.613044in}{1.209686in}}{\pgfqpoint{7.613044in}{1.204643in}}%
\pgfpathcurveto{\pgfqpoint{7.613044in}{1.199599in}}{\pgfqpoint{7.615047in}{1.194761in}}{\pgfqpoint{7.618614in}{1.191195in}}%
\pgfpathcurveto{\pgfqpoint{7.622180in}{1.187628in}}{\pgfqpoint{7.627018in}{1.185624in}}{\pgfqpoint{7.632062in}{1.185624in}}%
\pgfpathclose%
\pgfusepath{fill}%
\end{pgfscope}%
\begin{pgfscope}%
\pgfpathrectangle{\pgfqpoint{6.572727in}{0.473000in}}{\pgfqpoint{4.227273in}{3.311000in}}%
\pgfusepath{clip}%
\pgfsetbuttcap%
\pgfsetroundjoin%
\definecolor{currentfill}{rgb}{0.993248,0.906157,0.143936}%
\pgfsetfillcolor{currentfill}%
\pgfsetfillopacity{0.700000}%
\pgfsetlinewidth{0.000000pt}%
\definecolor{currentstroke}{rgb}{0.000000,0.000000,0.000000}%
\pgfsetstrokecolor{currentstroke}%
\pgfsetstrokeopacity{0.700000}%
\pgfsetdash{}{0pt}%
\pgfpathmoveto{\pgfqpoint{9.806089in}{1.305919in}}%
\pgfpathcurveto{\pgfqpoint{9.811133in}{1.305919in}}{\pgfqpoint{9.815971in}{1.307923in}}{\pgfqpoint{9.819537in}{1.311489in}}%
\pgfpathcurveto{\pgfqpoint{9.823103in}{1.315056in}}{\pgfqpoint{9.825107in}{1.319893in}}{\pgfqpoint{9.825107in}{1.324937in}}%
\pgfpathcurveto{\pgfqpoint{9.825107in}{1.329981in}}{\pgfqpoint{9.823103in}{1.334818in}}{\pgfqpoint{9.819537in}{1.338385in}}%
\pgfpathcurveto{\pgfqpoint{9.815971in}{1.341951in}}{\pgfqpoint{9.811133in}{1.343955in}}{\pgfqpoint{9.806089in}{1.343955in}}%
\pgfpathcurveto{\pgfqpoint{9.801045in}{1.343955in}}{\pgfqpoint{9.796208in}{1.341951in}}{\pgfqpoint{9.792641in}{1.338385in}}%
\pgfpathcurveto{\pgfqpoint{9.789075in}{1.334818in}}{\pgfqpoint{9.787071in}{1.329981in}}{\pgfqpoint{9.787071in}{1.324937in}}%
\pgfpathcurveto{\pgfqpoint{9.787071in}{1.319893in}}{\pgfqpoint{9.789075in}{1.315056in}}{\pgfqpoint{9.792641in}{1.311489in}}%
\pgfpathcurveto{\pgfqpoint{9.796208in}{1.307923in}}{\pgfqpoint{9.801045in}{1.305919in}}{\pgfqpoint{9.806089in}{1.305919in}}%
\pgfpathclose%
\pgfusepath{fill}%
\end{pgfscope}%
\begin{pgfscope}%
\pgfpathrectangle{\pgfqpoint{6.572727in}{0.473000in}}{\pgfqpoint{4.227273in}{3.311000in}}%
\pgfusepath{clip}%
\pgfsetbuttcap%
\pgfsetroundjoin%
\definecolor{currentfill}{rgb}{0.127568,0.566949,0.550556}%
\pgfsetfillcolor{currentfill}%
\pgfsetfillopacity{0.700000}%
\pgfsetlinewidth{0.000000pt}%
\definecolor{currentstroke}{rgb}{0.000000,0.000000,0.000000}%
\pgfsetstrokecolor{currentstroke}%
\pgfsetstrokeopacity{0.700000}%
\pgfsetdash{}{0pt}%
\pgfpathmoveto{\pgfqpoint{7.583548in}{1.185265in}}%
\pgfpathcurveto{\pgfqpoint{7.588592in}{1.185265in}}{\pgfqpoint{7.593429in}{1.187269in}}{\pgfqpoint{7.596996in}{1.190835in}}%
\pgfpathcurveto{\pgfqpoint{7.600562in}{1.194402in}}{\pgfqpoint{7.602566in}{1.199240in}}{\pgfqpoint{7.602566in}{1.204283in}}%
\pgfpathcurveto{\pgfqpoint{7.602566in}{1.209327in}}{\pgfqpoint{7.600562in}{1.214165in}}{\pgfqpoint{7.596996in}{1.217731in}}%
\pgfpathcurveto{\pgfqpoint{7.593429in}{1.221297in}}{\pgfqpoint{7.588592in}{1.223301in}}{\pgfqpoint{7.583548in}{1.223301in}}%
\pgfpathcurveto{\pgfqpoint{7.578504in}{1.223301in}}{\pgfqpoint{7.573666in}{1.221297in}}{\pgfqpoint{7.570100in}{1.217731in}}%
\pgfpathcurveto{\pgfqpoint{7.566534in}{1.214165in}}{\pgfqpoint{7.564530in}{1.209327in}}{\pgfqpoint{7.564530in}{1.204283in}}%
\pgfpathcurveto{\pgfqpoint{7.564530in}{1.199240in}}{\pgfqpoint{7.566534in}{1.194402in}}{\pgfqpoint{7.570100in}{1.190835in}}%
\pgfpathcurveto{\pgfqpoint{7.573666in}{1.187269in}}{\pgfqpoint{7.578504in}{1.185265in}}{\pgfqpoint{7.583548in}{1.185265in}}%
\pgfpathclose%
\pgfusepath{fill}%
\end{pgfscope}%
\begin{pgfscope}%
\pgfpathrectangle{\pgfqpoint{6.572727in}{0.473000in}}{\pgfqpoint{4.227273in}{3.311000in}}%
\pgfusepath{clip}%
\pgfsetbuttcap%
\pgfsetroundjoin%
\definecolor{currentfill}{rgb}{0.127568,0.566949,0.550556}%
\pgfsetfillcolor{currentfill}%
\pgfsetfillopacity{0.700000}%
\pgfsetlinewidth{0.000000pt}%
\definecolor{currentstroke}{rgb}{0.000000,0.000000,0.000000}%
\pgfsetstrokecolor{currentstroke}%
\pgfsetstrokeopacity{0.700000}%
\pgfsetdash{}{0pt}%
\pgfpathmoveto{\pgfqpoint{7.070485in}{1.247421in}}%
\pgfpathcurveto{\pgfqpoint{7.075528in}{1.247421in}}{\pgfqpoint{7.080366in}{1.249425in}}{\pgfqpoint{7.083933in}{1.252992in}}%
\pgfpathcurveto{\pgfqpoint{7.087499in}{1.256558in}}{\pgfqpoint{7.089503in}{1.261396in}}{\pgfqpoint{7.089503in}{1.266439in}}%
\pgfpathcurveto{\pgfqpoint{7.089503in}{1.271483in}}{\pgfqpoint{7.087499in}{1.276321in}}{\pgfqpoint{7.083933in}{1.279887in}}%
\pgfpathcurveto{\pgfqpoint{7.080366in}{1.283454in}}{\pgfqpoint{7.075528in}{1.285458in}}{\pgfqpoint{7.070485in}{1.285458in}}%
\pgfpathcurveto{\pgfqpoint{7.065441in}{1.285458in}}{\pgfqpoint{7.060603in}{1.283454in}}{\pgfqpoint{7.057037in}{1.279887in}}%
\pgfpathcurveto{\pgfqpoint{7.053470in}{1.276321in}}{\pgfqpoint{7.051467in}{1.271483in}}{\pgfqpoint{7.051467in}{1.266439in}}%
\pgfpathcurveto{\pgfqpoint{7.051467in}{1.261396in}}{\pgfqpoint{7.053470in}{1.256558in}}{\pgfqpoint{7.057037in}{1.252992in}}%
\pgfpathcurveto{\pgfqpoint{7.060603in}{1.249425in}}{\pgfqpoint{7.065441in}{1.247421in}}{\pgfqpoint{7.070485in}{1.247421in}}%
\pgfpathclose%
\pgfusepath{fill}%
\end{pgfscope}%
\begin{pgfscope}%
\pgfpathrectangle{\pgfqpoint{6.572727in}{0.473000in}}{\pgfqpoint{4.227273in}{3.311000in}}%
\pgfusepath{clip}%
\pgfsetbuttcap%
\pgfsetroundjoin%
\definecolor{currentfill}{rgb}{0.127568,0.566949,0.550556}%
\pgfsetfillcolor{currentfill}%
\pgfsetfillopacity{0.700000}%
\pgfsetlinewidth{0.000000pt}%
\definecolor{currentstroke}{rgb}{0.000000,0.000000,0.000000}%
\pgfsetstrokecolor{currentstroke}%
\pgfsetstrokeopacity{0.700000}%
\pgfsetdash{}{0pt}%
\pgfpathmoveto{\pgfqpoint{7.793212in}{1.586071in}}%
\pgfpathcurveto{\pgfqpoint{7.798255in}{1.586071in}}{\pgfqpoint{7.803093in}{1.588075in}}{\pgfqpoint{7.806660in}{1.591641in}}%
\pgfpathcurveto{\pgfqpoint{7.810226in}{1.595208in}}{\pgfqpoint{7.812230in}{1.600046in}}{\pgfqpoint{7.812230in}{1.605089in}}%
\pgfpathcurveto{\pgfqpoint{7.812230in}{1.610133in}}{\pgfqpoint{7.810226in}{1.614971in}}{\pgfqpoint{7.806660in}{1.618537in}}%
\pgfpathcurveto{\pgfqpoint{7.803093in}{1.622104in}}{\pgfqpoint{7.798255in}{1.624107in}}{\pgfqpoint{7.793212in}{1.624107in}}%
\pgfpathcurveto{\pgfqpoint{7.788168in}{1.624107in}}{\pgfqpoint{7.783330in}{1.622104in}}{\pgfqpoint{7.779764in}{1.618537in}}%
\pgfpathcurveto{\pgfqpoint{7.776197in}{1.614971in}}{\pgfqpoint{7.774194in}{1.610133in}}{\pgfqpoint{7.774194in}{1.605089in}}%
\pgfpathcurveto{\pgfqpoint{7.774194in}{1.600046in}}{\pgfqpoint{7.776197in}{1.595208in}}{\pgfqpoint{7.779764in}{1.591641in}}%
\pgfpathcurveto{\pgfqpoint{7.783330in}{1.588075in}}{\pgfqpoint{7.788168in}{1.586071in}}{\pgfqpoint{7.793212in}{1.586071in}}%
\pgfpathclose%
\pgfusepath{fill}%
\end{pgfscope}%
\begin{pgfscope}%
\pgfpathrectangle{\pgfqpoint{6.572727in}{0.473000in}}{\pgfqpoint{4.227273in}{3.311000in}}%
\pgfusepath{clip}%
\pgfsetbuttcap%
\pgfsetroundjoin%
\definecolor{currentfill}{rgb}{0.127568,0.566949,0.550556}%
\pgfsetfillcolor{currentfill}%
\pgfsetfillopacity{0.700000}%
\pgfsetlinewidth{0.000000pt}%
\definecolor{currentstroke}{rgb}{0.000000,0.000000,0.000000}%
\pgfsetstrokecolor{currentstroke}%
\pgfsetstrokeopacity{0.700000}%
\pgfsetdash{}{0pt}%
\pgfpathmoveto{\pgfqpoint{8.054637in}{2.479909in}}%
\pgfpathcurveto{\pgfqpoint{8.059681in}{2.479909in}}{\pgfqpoint{8.064518in}{2.481913in}}{\pgfqpoint{8.068085in}{2.485479in}}%
\pgfpathcurveto{\pgfqpoint{8.071651in}{2.489045in}}{\pgfqpoint{8.073655in}{2.493883in}}{\pgfqpoint{8.073655in}{2.498927in}}%
\pgfpathcurveto{\pgfqpoint{8.073655in}{2.503971in}}{\pgfqpoint{8.071651in}{2.508808in}}{\pgfqpoint{8.068085in}{2.512375in}}%
\pgfpathcurveto{\pgfqpoint{8.064518in}{2.515941in}}{\pgfqpoint{8.059681in}{2.517945in}}{\pgfqpoint{8.054637in}{2.517945in}}%
\pgfpathcurveto{\pgfqpoint{8.049593in}{2.517945in}}{\pgfqpoint{8.044755in}{2.515941in}}{\pgfqpoint{8.041189in}{2.512375in}}%
\pgfpathcurveto{\pgfqpoint{8.037623in}{2.508808in}}{\pgfqpoint{8.035619in}{2.503971in}}{\pgfqpoint{8.035619in}{2.498927in}}%
\pgfpathcurveto{\pgfqpoint{8.035619in}{2.493883in}}{\pgfqpoint{8.037623in}{2.489045in}}{\pgfqpoint{8.041189in}{2.485479in}}%
\pgfpathcurveto{\pgfqpoint{8.044755in}{2.481913in}}{\pgfqpoint{8.049593in}{2.479909in}}{\pgfqpoint{8.054637in}{2.479909in}}%
\pgfpathclose%
\pgfusepath{fill}%
\end{pgfscope}%
\begin{pgfscope}%
\pgfpathrectangle{\pgfqpoint{6.572727in}{0.473000in}}{\pgfqpoint{4.227273in}{3.311000in}}%
\pgfusepath{clip}%
\pgfsetbuttcap%
\pgfsetroundjoin%
\definecolor{currentfill}{rgb}{0.127568,0.566949,0.550556}%
\pgfsetfillcolor{currentfill}%
\pgfsetfillopacity{0.700000}%
\pgfsetlinewidth{0.000000pt}%
\definecolor{currentstroke}{rgb}{0.000000,0.000000,0.000000}%
\pgfsetstrokecolor{currentstroke}%
\pgfsetstrokeopacity{0.700000}%
\pgfsetdash{}{0pt}%
\pgfpathmoveto{\pgfqpoint{7.599768in}{1.279870in}}%
\pgfpathcurveto{\pgfqpoint{7.604812in}{1.279870in}}{\pgfqpoint{7.609650in}{1.281874in}}{\pgfqpoint{7.613216in}{1.285441in}}%
\pgfpathcurveto{\pgfqpoint{7.616782in}{1.289007in}}{\pgfqpoint{7.618786in}{1.293845in}}{\pgfqpoint{7.618786in}{1.298889in}}%
\pgfpathcurveto{\pgfqpoint{7.618786in}{1.303932in}}{\pgfqpoint{7.616782in}{1.308770in}}{\pgfqpoint{7.613216in}{1.312336in}}%
\pgfpathcurveto{\pgfqpoint{7.609650in}{1.315903in}}{\pgfqpoint{7.604812in}{1.317907in}}{\pgfqpoint{7.599768in}{1.317907in}}%
\pgfpathcurveto{\pgfqpoint{7.594724in}{1.317907in}}{\pgfqpoint{7.589887in}{1.315903in}}{\pgfqpoint{7.586320in}{1.312336in}}%
\pgfpathcurveto{\pgfqpoint{7.582754in}{1.308770in}}{\pgfqpoint{7.580750in}{1.303932in}}{\pgfqpoint{7.580750in}{1.298889in}}%
\pgfpathcurveto{\pgfqpoint{7.580750in}{1.293845in}}{\pgfqpoint{7.582754in}{1.289007in}}{\pgfqpoint{7.586320in}{1.285441in}}%
\pgfpathcurveto{\pgfqpoint{7.589887in}{1.281874in}}{\pgfqpoint{7.594724in}{1.279870in}}{\pgfqpoint{7.599768in}{1.279870in}}%
\pgfpathclose%
\pgfusepath{fill}%
\end{pgfscope}%
\begin{pgfscope}%
\pgfpathrectangle{\pgfqpoint{6.572727in}{0.473000in}}{\pgfqpoint{4.227273in}{3.311000in}}%
\pgfusepath{clip}%
\pgfsetbuttcap%
\pgfsetroundjoin%
\definecolor{currentfill}{rgb}{0.127568,0.566949,0.550556}%
\pgfsetfillcolor{currentfill}%
\pgfsetfillopacity{0.700000}%
\pgfsetlinewidth{0.000000pt}%
\definecolor{currentstroke}{rgb}{0.000000,0.000000,0.000000}%
\pgfsetstrokecolor{currentstroke}%
\pgfsetstrokeopacity{0.700000}%
\pgfsetdash{}{0pt}%
\pgfpathmoveto{\pgfqpoint{8.375172in}{3.013765in}}%
\pgfpathcurveto{\pgfqpoint{8.380216in}{3.013765in}}{\pgfqpoint{8.385054in}{3.015769in}}{\pgfqpoint{8.388620in}{3.019335in}}%
\pgfpathcurveto{\pgfqpoint{8.392187in}{3.022901in}}{\pgfqpoint{8.394191in}{3.027739in}}{\pgfqpoint{8.394191in}{3.032783in}}%
\pgfpathcurveto{\pgfqpoint{8.394191in}{3.037827in}}{\pgfqpoint{8.392187in}{3.042664in}}{\pgfqpoint{8.388620in}{3.046231in}}%
\pgfpathcurveto{\pgfqpoint{8.385054in}{3.049797in}}{\pgfqpoint{8.380216in}{3.051801in}}{\pgfqpoint{8.375172in}{3.051801in}}%
\pgfpathcurveto{\pgfqpoint{8.370129in}{3.051801in}}{\pgfqpoint{8.365291in}{3.049797in}}{\pgfqpoint{8.361725in}{3.046231in}}%
\pgfpathcurveto{\pgfqpoint{8.358158in}{3.042664in}}{\pgfqpoint{8.356154in}{3.037827in}}{\pgfqpoint{8.356154in}{3.032783in}}%
\pgfpathcurveto{\pgfqpoint{8.356154in}{3.027739in}}{\pgfqpoint{8.358158in}{3.022901in}}{\pgfqpoint{8.361725in}{3.019335in}}%
\pgfpathcurveto{\pgfqpoint{8.365291in}{3.015769in}}{\pgfqpoint{8.370129in}{3.013765in}}{\pgfqpoint{8.375172in}{3.013765in}}%
\pgfpathclose%
\pgfusepath{fill}%
\end{pgfscope}%
\begin{pgfscope}%
\pgfpathrectangle{\pgfqpoint{6.572727in}{0.473000in}}{\pgfqpoint{4.227273in}{3.311000in}}%
\pgfusepath{clip}%
\pgfsetbuttcap%
\pgfsetroundjoin%
\definecolor{currentfill}{rgb}{0.127568,0.566949,0.550556}%
\pgfsetfillcolor{currentfill}%
\pgfsetfillopacity{0.700000}%
\pgfsetlinewidth{0.000000pt}%
\definecolor{currentstroke}{rgb}{0.000000,0.000000,0.000000}%
\pgfsetstrokecolor{currentstroke}%
\pgfsetstrokeopacity{0.700000}%
\pgfsetdash{}{0pt}%
\pgfpathmoveto{\pgfqpoint{7.989577in}{2.757321in}}%
\pgfpathcurveto{\pgfqpoint{7.994620in}{2.757321in}}{\pgfqpoint{7.999458in}{2.759325in}}{\pgfqpoint{8.003025in}{2.762891in}}%
\pgfpathcurveto{\pgfqpoint{8.006591in}{2.766457in}}{\pgfqpoint{8.008595in}{2.771295in}}{\pgfqpoint{8.008595in}{2.776339in}}%
\pgfpathcurveto{\pgfqpoint{8.008595in}{2.781383in}}{\pgfqpoint{8.006591in}{2.786220in}}{\pgfqpoint{8.003025in}{2.789787in}}%
\pgfpathcurveto{\pgfqpoint{7.999458in}{2.793353in}}{\pgfqpoint{7.994620in}{2.795357in}}{\pgfqpoint{7.989577in}{2.795357in}}%
\pgfpathcurveto{\pgfqpoint{7.984533in}{2.795357in}}{\pgfqpoint{7.979695in}{2.793353in}}{\pgfqpoint{7.976129in}{2.789787in}}%
\pgfpathcurveto{\pgfqpoint{7.972562in}{2.786220in}}{\pgfqpoint{7.970559in}{2.781383in}}{\pgfqpoint{7.970559in}{2.776339in}}%
\pgfpathcurveto{\pgfqpoint{7.970559in}{2.771295in}}{\pgfqpoint{7.972562in}{2.766457in}}{\pgfqpoint{7.976129in}{2.762891in}}%
\pgfpathcurveto{\pgfqpoint{7.979695in}{2.759325in}}{\pgfqpoint{7.984533in}{2.757321in}}{\pgfqpoint{7.989577in}{2.757321in}}%
\pgfpathclose%
\pgfusepath{fill}%
\end{pgfscope}%
\begin{pgfscope}%
\pgfpathrectangle{\pgfqpoint{6.572727in}{0.473000in}}{\pgfqpoint{4.227273in}{3.311000in}}%
\pgfusepath{clip}%
\pgfsetbuttcap%
\pgfsetroundjoin%
\definecolor{currentfill}{rgb}{0.127568,0.566949,0.550556}%
\pgfsetfillcolor{currentfill}%
\pgfsetfillopacity{0.700000}%
\pgfsetlinewidth{0.000000pt}%
\definecolor{currentstroke}{rgb}{0.000000,0.000000,0.000000}%
\pgfsetstrokecolor{currentstroke}%
\pgfsetstrokeopacity{0.700000}%
\pgfsetdash{}{0pt}%
\pgfpathmoveto{\pgfqpoint{8.127369in}{1.494145in}}%
\pgfpathcurveto{\pgfqpoint{8.132413in}{1.494145in}}{\pgfqpoint{8.137251in}{1.496149in}}{\pgfqpoint{8.140817in}{1.499715in}}%
\pgfpathcurveto{\pgfqpoint{8.144384in}{1.503282in}}{\pgfqpoint{8.146387in}{1.508119in}}{\pgfqpoint{8.146387in}{1.513163in}}%
\pgfpathcurveto{\pgfqpoint{8.146387in}{1.518207in}}{\pgfqpoint{8.144384in}{1.523045in}}{\pgfqpoint{8.140817in}{1.526611in}}%
\pgfpathcurveto{\pgfqpoint{8.137251in}{1.530177in}}{\pgfqpoint{8.132413in}{1.532181in}}{\pgfqpoint{8.127369in}{1.532181in}}%
\pgfpathcurveto{\pgfqpoint{8.122326in}{1.532181in}}{\pgfqpoint{8.117488in}{1.530177in}}{\pgfqpoint{8.113921in}{1.526611in}}%
\pgfpathcurveto{\pgfqpoint{8.110355in}{1.523045in}}{\pgfqpoint{8.108351in}{1.518207in}}{\pgfqpoint{8.108351in}{1.513163in}}%
\pgfpathcurveto{\pgfqpoint{8.108351in}{1.508119in}}{\pgfqpoint{8.110355in}{1.503282in}}{\pgfqpoint{8.113921in}{1.499715in}}%
\pgfpathcurveto{\pgfqpoint{8.117488in}{1.496149in}}{\pgfqpoint{8.122326in}{1.494145in}}{\pgfqpoint{8.127369in}{1.494145in}}%
\pgfpathclose%
\pgfusepath{fill}%
\end{pgfscope}%
\begin{pgfscope}%
\pgfpathrectangle{\pgfqpoint{6.572727in}{0.473000in}}{\pgfqpoint{4.227273in}{3.311000in}}%
\pgfusepath{clip}%
\pgfsetbuttcap%
\pgfsetroundjoin%
\definecolor{currentfill}{rgb}{0.993248,0.906157,0.143936}%
\pgfsetfillcolor{currentfill}%
\pgfsetfillopacity{0.700000}%
\pgfsetlinewidth{0.000000pt}%
\definecolor{currentstroke}{rgb}{0.000000,0.000000,0.000000}%
\pgfsetstrokecolor{currentstroke}%
\pgfsetstrokeopacity{0.700000}%
\pgfsetdash{}{0pt}%
\pgfpathmoveto{\pgfqpoint{9.781117in}{1.502651in}}%
\pgfpathcurveto{\pgfqpoint{9.786161in}{1.502651in}}{\pgfqpoint{9.790999in}{1.504655in}}{\pgfqpoint{9.794565in}{1.508221in}}%
\pgfpathcurveto{\pgfqpoint{9.798132in}{1.511788in}}{\pgfqpoint{9.800136in}{1.516625in}}{\pgfqpoint{9.800136in}{1.521669in}}%
\pgfpathcurveto{\pgfqpoint{9.800136in}{1.526713in}}{\pgfqpoint{9.798132in}{1.531551in}}{\pgfqpoint{9.794565in}{1.535117in}}%
\pgfpathcurveto{\pgfqpoint{9.790999in}{1.538683in}}{\pgfqpoint{9.786161in}{1.540687in}}{\pgfqpoint{9.781117in}{1.540687in}}%
\pgfpathcurveto{\pgfqpoint{9.776074in}{1.540687in}}{\pgfqpoint{9.771236in}{1.538683in}}{\pgfqpoint{9.767670in}{1.535117in}}%
\pgfpathcurveto{\pgfqpoint{9.764103in}{1.531551in}}{\pgfqpoint{9.762099in}{1.526713in}}{\pgfqpoint{9.762099in}{1.521669in}}%
\pgfpathcurveto{\pgfqpoint{9.762099in}{1.516625in}}{\pgfqpoint{9.764103in}{1.511788in}}{\pgfqpoint{9.767670in}{1.508221in}}%
\pgfpathcurveto{\pgfqpoint{9.771236in}{1.504655in}}{\pgfqpoint{9.776074in}{1.502651in}}{\pgfqpoint{9.781117in}{1.502651in}}%
\pgfpathclose%
\pgfusepath{fill}%
\end{pgfscope}%
\begin{pgfscope}%
\pgfpathrectangle{\pgfqpoint{6.572727in}{0.473000in}}{\pgfqpoint{4.227273in}{3.311000in}}%
\pgfusepath{clip}%
\pgfsetbuttcap%
\pgfsetroundjoin%
\definecolor{currentfill}{rgb}{0.993248,0.906157,0.143936}%
\pgfsetfillcolor{currentfill}%
\pgfsetfillopacity{0.700000}%
\pgfsetlinewidth{0.000000pt}%
\definecolor{currentstroke}{rgb}{0.000000,0.000000,0.000000}%
\pgfsetstrokecolor{currentstroke}%
\pgfsetstrokeopacity{0.700000}%
\pgfsetdash{}{0pt}%
\pgfpathmoveto{\pgfqpoint{9.497266in}{1.560925in}}%
\pgfpathcurveto{\pgfqpoint{9.502309in}{1.560925in}}{\pgfqpoint{9.507147in}{1.562929in}}{\pgfqpoint{9.510713in}{1.566495in}}%
\pgfpathcurveto{\pgfqpoint{9.514280in}{1.570061in}}{\pgfqpoint{9.516284in}{1.574899in}}{\pgfqpoint{9.516284in}{1.579943in}}%
\pgfpathcurveto{\pgfqpoint{9.516284in}{1.584987in}}{\pgfqpoint{9.514280in}{1.589824in}}{\pgfqpoint{9.510713in}{1.593391in}}%
\pgfpathcurveto{\pgfqpoint{9.507147in}{1.596957in}}{\pgfqpoint{9.502309in}{1.598961in}}{\pgfqpoint{9.497266in}{1.598961in}}%
\pgfpathcurveto{\pgfqpoint{9.492222in}{1.598961in}}{\pgfqpoint{9.487384in}{1.596957in}}{\pgfqpoint{9.483818in}{1.593391in}}%
\pgfpathcurveto{\pgfqpoint{9.480251in}{1.589824in}}{\pgfqpoint{9.478247in}{1.584987in}}{\pgfqpoint{9.478247in}{1.579943in}}%
\pgfpathcurveto{\pgfqpoint{9.478247in}{1.574899in}}{\pgfqpoint{9.480251in}{1.570061in}}{\pgfqpoint{9.483818in}{1.566495in}}%
\pgfpathcurveto{\pgfqpoint{9.487384in}{1.562929in}}{\pgfqpoint{9.492222in}{1.560925in}}{\pgfqpoint{9.497266in}{1.560925in}}%
\pgfpathclose%
\pgfusepath{fill}%
\end{pgfscope}%
\begin{pgfscope}%
\pgfpathrectangle{\pgfqpoint{6.572727in}{0.473000in}}{\pgfqpoint{4.227273in}{3.311000in}}%
\pgfusepath{clip}%
\pgfsetbuttcap%
\pgfsetroundjoin%
\definecolor{currentfill}{rgb}{0.127568,0.566949,0.550556}%
\pgfsetfillcolor{currentfill}%
\pgfsetfillopacity{0.700000}%
\pgfsetlinewidth{0.000000pt}%
\definecolor{currentstroke}{rgb}{0.000000,0.000000,0.000000}%
\pgfsetstrokecolor{currentstroke}%
\pgfsetstrokeopacity{0.700000}%
\pgfsetdash{}{0pt}%
\pgfpathmoveto{\pgfqpoint{7.090086in}{1.493310in}}%
\pgfpathcurveto{\pgfqpoint{7.095130in}{1.493310in}}{\pgfqpoint{7.099967in}{1.495314in}}{\pgfqpoint{7.103534in}{1.498880in}}%
\pgfpathcurveto{\pgfqpoint{7.107100in}{1.502446in}}{\pgfqpoint{7.109104in}{1.507284in}}{\pgfqpoint{7.109104in}{1.512328in}}%
\pgfpathcurveto{\pgfqpoint{7.109104in}{1.517372in}}{\pgfqpoint{7.107100in}{1.522209in}}{\pgfqpoint{7.103534in}{1.525776in}}%
\pgfpathcurveto{\pgfqpoint{7.099967in}{1.529342in}}{\pgfqpoint{7.095130in}{1.531346in}}{\pgfqpoint{7.090086in}{1.531346in}}%
\pgfpathcurveto{\pgfqpoint{7.085042in}{1.531346in}}{\pgfqpoint{7.080205in}{1.529342in}}{\pgfqpoint{7.076638in}{1.525776in}}%
\pgfpathcurveto{\pgfqpoint{7.073072in}{1.522209in}}{\pgfqpoint{7.071068in}{1.517372in}}{\pgfqpoint{7.071068in}{1.512328in}}%
\pgfpathcurveto{\pgfqpoint{7.071068in}{1.507284in}}{\pgfqpoint{7.073072in}{1.502446in}}{\pgfqpoint{7.076638in}{1.498880in}}%
\pgfpathcurveto{\pgfqpoint{7.080205in}{1.495314in}}{\pgfqpoint{7.085042in}{1.493310in}}{\pgfqpoint{7.090086in}{1.493310in}}%
\pgfpathclose%
\pgfusepath{fill}%
\end{pgfscope}%
\begin{pgfscope}%
\pgfpathrectangle{\pgfqpoint{6.572727in}{0.473000in}}{\pgfqpoint{4.227273in}{3.311000in}}%
\pgfusepath{clip}%
\pgfsetbuttcap%
\pgfsetroundjoin%
\definecolor{currentfill}{rgb}{0.127568,0.566949,0.550556}%
\pgfsetfillcolor{currentfill}%
\pgfsetfillopacity{0.700000}%
\pgfsetlinewidth{0.000000pt}%
\definecolor{currentstroke}{rgb}{0.000000,0.000000,0.000000}%
\pgfsetstrokecolor{currentstroke}%
\pgfsetstrokeopacity{0.700000}%
\pgfsetdash{}{0pt}%
\pgfpathmoveto{\pgfqpoint{8.236202in}{3.168433in}}%
\pgfpathcurveto{\pgfqpoint{8.241246in}{3.168433in}}{\pgfqpoint{8.246084in}{3.170436in}}{\pgfqpoint{8.249650in}{3.174003in}}%
\pgfpathcurveto{\pgfqpoint{8.253217in}{3.177569in}}{\pgfqpoint{8.255220in}{3.182407in}}{\pgfqpoint{8.255220in}{3.187451in}}%
\pgfpathcurveto{\pgfqpoint{8.255220in}{3.192494in}}{\pgfqpoint{8.253217in}{3.197332in}}{\pgfqpoint{8.249650in}{3.200899in}}%
\pgfpathcurveto{\pgfqpoint{8.246084in}{3.204465in}}{\pgfqpoint{8.241246in}{3.206469in}}{\pgfqpoint{8.236202in}{3.206469in}}%
\pgfpathcurveto{\pgfqpoint{8.231159in}{3.206469in}}{\pgfqpoint{8.226321in}{3.204465in}}{\pgfqpoint{8.222754in}{3.200899in}}%
\pgfpathcurveto{\pgfqpoint{8.219188in}{3.197332in}}{\pgfqpoint{8.217184in}{3.192494in}}{\pgfqpoint{8.217184in}{3.187451in}}%
\pgfpathcurveto{\pgfqpoint{8.217184in}{3.182407in}}{\pgfqpoint{8.219188in}{3.177569in}}{\pgfqpoint{8.222754in}{3.174003in}}%
\pgfpathcurveto{\pgfqpoint{8.226321in}{3.170436in}}{\pgfqpoint{8.231159in}{3.168433in}}{\pgfqpoint{8.236202in}{3.168433in}}%
\pgfpathclose%
\pgfusepath{fill}%
\end{pgfscope}%
\begin{pgfscope}%
\pgfpathrectangle{\pgfqpoint{6.572727in}{0.473000in}}{\pgfqpoint{4.227273in}{3.311000in}}%
\pgfusepath{clip}%
\pgfsetbuttcap%
\pgfsetroundjoin%
\definecolor{currentfill}{rgb}{0.127568,0.566949,0.550556}%
\pgfsetfillcolor{currentfill}%
\pgfsetfillopacity{0.700000}%
\pgfsetlinewidth{0.000000pt}%
\definecolor{currentstroke}{rgb}{0.000000,0.000000,0.000000}%
\pgfsetstrokecolor{currentstroke}%
\pgfsetstrokeopacity{0.700000}%
\pgfsetdash{}{0pt}%
\pgfpathmoveto{\pgfqpoint{7.645761in}{2.962208in}}%
\pgfpathcurveto{\pgfqpoint{7.650804in}{2.962208in}}{\pgfqpoint{7.655642in}{2.964212in}}{\pgfqpoint{7.659208in}{2.967779in}}%
\pgfpathcurveto{\pgfqpoint{7.662775in}{2.971345in}}{\pgfqpoint{7.664779in}{2.976183in}}{\pgfqpoint{7.664779in}{2.981226in}}%
\pgfpathcurveto{\pgfqpoint{7.664779in}{2.986270in}}{\pgfqpoint{7.662775in}{2.991108in}}{\pgfqpoint{7.659208in}{2.994674in}}%
\pgfpathcurveto{\pgfqpoint{7.655642in}{2.998241in}}{\pgfqpoint{7.650804in}{3.000245in}}{\pgfqpoint{7.645761in}{3.000245in}}%
\pgfpathcurveto{\pgfqpoint{7.640717in}{3.000245in}}{\pgfqpoint{7.635879in}{2.998241in}}{\pgfqpoint{7.632313in}{2.994674in}}%
\pgfpathcurveto{\pgfqpoint{7.628746in}{2.991108in}}{\pgfqpoint{7.626742in}{2.986270in}}{\pgfqpoint{7.626742in}{2.981226in}}%
\pgfpathcurveto{\pgfqpoint{7.626742in}{2.976183in}}{\pgfqpoint{7.628746in}{2.971345in}}{\pgfqpoint{7.632313in}{2.967779in}}%
\pgfpathcurveto{\pgfqpoint{7.635879in}{2.964212in}}{\pgfqpoint{7.640717in}{2.962208in}}{\pgfqpoint{7.645761in}{2.962208in}}%
\pgfpathclose%
\pgfusepath{fill}%
\end{pgfscope}%
\begin{pgfscope}%
\pgfpathrectangle{\pgfqpoint{6.572727in}{0.473000in}}{\pgfqpoint{4.227273in}{3.311000in}}%
\pgfusepath{clip}%
\pgfsetbuttcap%
\pgfsetroundjoin%
\definecolor{currentfill}{rgb}{0.127568,0.566949,0.550556}%
\pgfsetfillcolor{currentfill}%
\pgfsetfillopacity{0.700000}%
\pgfsetlinewidth{0.000000pt}%
\definecolor{currentstroke}{rgb}{0.000000,0.000000,0.000000}%
\pgfsetstrokecolor{currentstroke}%
\pgfsetstrokeopacity{0.700000}%
\pgfsetdash{}{0pt}%
\pgfpathmoveto{\pgfqpoint{8.310008in}{2.672964in}}%
\pgfpathcurveto{\pgfqpoint{8.315052in}{2.672964in}}{\pgfqpoint{8.319890in}{2.674968in}}{\pgfqpoint{8.323456in}{2.678534in}}%
\pgfpathcurveto{\pgfqpoint{8.327023in}{2.682100in}}{\pgfqpoint{8.329026in}{2.686938in}}{\pgfqpoint{8.329026in}{2.691982in}}%
\pgfpathcurveto{\pgfqpoint{8.329026in}{2.697026in}}{\pgfqpoint{8.327023in}{2.701863in}}{\pgfqpoint{8.323456in}{2.705430in}}%
\pgfpathcurveto{\pgfqpoint{8.319890in}{2.708996in}}{\pgfqpoint{8.315052in}{2.711000in}}{\pgfqpoint{8.310008in}{2.711000in}}%
\pgfpathcurveto{\pgfqpoint{8.304965in}{2.711000in}}{\pgfqpoint{8.300127in}{2.708996in}}{\pgfqpoint{8.296560in}{2.705430in}}%
\pgfpathcurveto{\pgfqpoint{8.292994in}{2.701863in}}{\pgfqpoint{8.290990in}{2.697026in}}{\pgfqpoint{8.290990in}{2.691982in}}%
\pgfpathcurveto{\pgfqpoint{8.290990in}{2.686938in}}{\pgfqpoint{8.292994in}{2.682100in}}{\pgfqpoint{8.296560in}{2.678534in}}%
\pgfpathcurveto{\pgfqpoint{8.300127in}{2.674968in}}{\pgfqpoint{8.304965in}{2.672964in}}{\pgfqpoint{8.310008in}{2.672964in}}%
\pgfpathclose%
\pgfusepath{fill}%
\end{pgfscope}%
\begin{pgfscope}%
\pgfpathrectangle{\pgfqpoint{6.572727in}{0.473000in}}{\pgfqpoint{4.227273in}{3.311000in}}%
\pgfusepath{clip}%
\pgfsetbuttcap%
\pgfsetroundjoin%
\definecolor{currentfill}{rgb}{0.993248,0.906157,0.143936}%
\pgfsetfillcolor{currentfill}%
\pgfsetfillopacity{0.700000}%
\pgfsetlinewidth{0.000000pt}%
\definecolor{currentstroke}{rgb}{0.000000,0.000000,0.000000}%
\pgfsetstrokecolor{currentstroke}%
\pgfsetstrokeopacity{0.700000}%
\pgfsetdash{}{0pt}%
\pgfpathmoveto{\pgfqpoint{9.612338in}{1.444158in}}%
\pgfpathcurveto{\pgfqpoint{9.617382in}{1.444158in}}{\pgfqpoint{9.622220in}{1.446162in}}{\pgfqpoint{9.625786in}{1.449728in}}%
\pgfpathcurveto{\pgfqpoint{9.629353in}{1.453294in}}{\pgfqpoint{9.631357in}{1.458132in}}{\pgfqpoint{9.631357in}{1.463176in}}%
\pgfpathcurveto{\pgfqpoint{9.631357in}{1.468220in}}{\pgfqpoint{9.629353in}{1.473057in}}{\pgfqpoint{9.625786in}{1.476624in}}%
\pgfpathcurveto{\pgfqpoint{9.622220in}{1.480190in}}{\pgfqpoint{9.617382in}{1.482194in}}{\pgfqpoint{9.612338in}{1.482194in}}%
\pgfpathcurveto{\pgfqpoint{9.607295in}{1.482194in}}{\pgfqpoint{9.602457in}{1.480190in}}{\pgfqpoint{9.598891in}{1.476624in}}%
\pgfpathcurveto{\pgfqpoint{9.595324in}{1.473057in}}{\pgfqpoint{9.593320in}{1.468220in}}{\pgfqpoint{9.593320in}{1.463176in}}%
\pgfpathcurveto{\pgfqpoint{9.593320in}{1.458132in}}{\pgfqpoint{9.595324in}{1.453294in}}{\pgfqpoint{9.598891in}{1.449728in}}%
\pgfpathcurveto{\pgfqpoint{9.602457in}{1.446162in}}{\pgfqpoint{9.607295in}{1.444158in}}{\pgfqpoint{9.612338in}{1.444158in}}%
\pgfpathclose%
\pgfusepath{fill}%
\end{pgfscope}%
\begin{pgfscope}%
\pgfpathrectangle{\pgfqpoint{6.572727in}{0.473000in}}{\pgfqpoint{4.227273in}{3.311000in}}%
\pgfusepath{clip}%
\pgfsetbuttcap%
\pgfsetroundjoin%
\definecolor{currentfill}{rgb}{0.993248,0.906157,0.143936}%
\pgfsetfillcolor{currentfill}%
\pgfsetfillopacity{0.700000}%
\pgfsetlinewidth{0.000000pt}%
\definecolor{currentstroke}{rgb}{0.000000,0.000000,0.000000}%
\pgfsetstrokecolor{currentstroke}%
\pgfsetstrokeopacity{0.700000}%
\pgfsetdash{}{0pt}%
\pgfpathmoveto{\pgfqpoint{9.350576in}{1.349291in}}%
\pgfpathcurveto{\pgfqpoint{9.355619in}{1.349291in}}{\pgfqpoint{9.360457in}{1.351295in}}{\pgfqpoint{9.364024in}{1.354861in}}%
\pgfpathcurveto{\pgfqpoint{9.367590in}{1.358428in}}{\pgfqpoint{9.369594in}{1.363265in}}{\pgfqpoint{9.369594in}{1.368309in}}%
\pgfpathcurveto{\pgfqpoint{9.369594in}{1.373353in}}{\pgfqpoint{9.367590in}{1.378190in}}{\pgfqpoint{9.364024in}{1.381757in}}%
\pgfpathcurveto{\pgfqpoint{9.360457in}{1.385323in}}{\pgfqpoint{9.355619in}{1.387327in}}{\pgfqpoint{9.350576in}{1.387327in}}%
\pgfpathcurveto{\pgfqpoint{9.345532in}{1.387327in}}{\pgfqpoint{9.340694in}{1.385323in}}{\pgfqpoint{9.337128in}{1.381757in}}%
\pgfpathcurveto{\pgfqpoint{9.333561in}{1.378190in}}{\pgfqpoint{9.331558in}{1.373353in}}{\pgfqpoint{9.331558in}{1.368309in}}%
\pgfpathcurveto{\pgfqpoint{9.331558in}{1.363265in}}{\pgfqpoint{9.333561in}{1.358428in}}{\pgfqpoint{9.337128in}{1.354861in}}%
\pgfpathcurveto{\pgfqpoint{9.340694in}{1.351295in}}{\pgfqpoint{9.345532in}{1.349291in}}{\pgfqpoint{9.350576in}{1.349291in}}%
\pgfpathclose%
\pgfusepath{fill}%
\end{pgfscope}%
\begin{pgfscope}%
\pgfpathrectangle{\pgfqpoint{6.572727in}{0.473000in}}{\pgfqpoint{4.227273in}{3.311000in}}%
\pgfusepath{clip}%
\pgfsetbuttcap%
\pgfsetroundjoin%
\definecolor{currentfill}{rgb}{0.127568,0.566949,0.550556}%
\pgfsetfillcolor{currentfill}%
\pgfsetfillopacity{0.700000}%
\pgfsetlinewidth{0.000000pt}%
\definecolor{currentstroke}{rgb}{0.000000,0.000000,0.000000}%
\pgfsetstrokecolor{currentstroke}%
\pgfsetstrokeopacity{0.700000}%
\pgfsetdash{}{0pt}%
\pgfpathmoveto{\pgfqpoint{8.407916in}{2.919014in}}%
\pgfpathcurveto{\pgfqpoint{8.412960in}{2.919014in}}{\pgfqpoint{8.417797in}{2.921018in}}{\pgfqpoint{8.421364in}{2.924585in}}%
\pgfpathcurveto{\pgfqpoint{8.424930in}{2.928151in}}{\pgfqpoint{8.426934in}{2.932989in}}{\pgfqpoint{8.426934in}{2.938033in}}%
\pgfpathcurveto{\pgfqpoint{8.426934in}{2.943076in}}{\pgfqpoint{8.424930in}{2.947914in}}{\pgfqpoint{8.421364in}{2.951480in}}%
\pgfpathcurveto{\pgfqpoint{8.417797in}{2.955047in}}{\pgfqpoint{8.412960in}{2.957051in}}{\pgfqpoint{8.407916in}{2.957051in}}%
\pgfpathcurveto{\pgfqpoint{8.402872in}{2.957051in}}{\pgfqpoint{8.398035in}{2.955047in}}{\pgfqpoint{8.394468in}{2.951480in}}%
\pgfpathcurveto{\pgfqpoint{8.390902in}{2.947914in}}{\pgfqpoint{8.388898in}{2.943076in}}{\pgfqpoint{8.388898in}{2.938033in}}%
\pgfpathcurveto{\pgfqpoint{8.388898in}{2.932989in}}{\pgfqpoint{8.390902in}{2.928151in}}{\pgfqpoint{8.394468in}{2.924585in}}%
\pgfpathcurveto{\pgfqpoint{8.398035in}{2.921018in}}{\pgfqpoint{8.402872in}{2.919014in}}{\pgfqpoint{8.407916in}{2.919014in}}%
\pgfpathclose%
\pgfusepath{fill}%
\end{pgfscope}%
\begin{pgfscope}%
\pgfpathrectangle{\pgfqpoint{6.572727in}{0.473000in}}{\pgfqpoint{4.227273in}{3.311000in}}%
\pgfusepath{clip}%
\pgfsetbuttcap%
\pgfsetroundjoin%
\definecolor{currentfill}{rgb}{0.993248,0.906157,0.143936}%
\pgfsetfillcolor{currentfill}%
\pgfsetfillopacity{0.700000}%
\pgfsetlinewidth{0.000000pt}%
\definecolor{currentstroke}{rgb}{0.000000,0.000000,0.000000}%
\pgfsetstrokecolor{currentstroke}%
\pgfsetstrokeopacity{0.700000}%
\pgfsetdash{}{0pt}%
\pgfpathmoveto{\pgfqpoint{9.464407in}{1.744415in}}%
\pgfpathcurveto{\pgfqpoint{9.469451in}{1.744415in}}{\pgfqpoint{9.474288in}{1.746419in}}{\pgfqpoint{9.477855in}{1.749986in}}%
\pgfpathcurveto{\pgfqpoint{9.481421in}{1.753552in}}{\pgfqpoint{9.483425in}{1.758390in}}{\pgfqpoint{9.483425in}{1.763433in}}%
\pgfpathcurveto{\pgfqpoint{9.483425in}{1.768477in}}{\pgfqpoint{9.481421in}{1.773315in}}{\pgfqpoint{9.477855in}{1.776881in}}%
\pgfpathcurveto{\pgfqpoint{9.474288in}{1.780448in}}{\pgfqpoint{9.469451in}{1.782452in}}{\pgfqpoint{9.464407in}{1.782452in}}%
\pgfpathcurveto{\pgfqpoint{9.459363in}{1.782452in}}{\pgfqpoint{9.454526in}{1.780448in}}{\pgfqpoint{9.450959in}{1.776881in}}%
\pgfpathcurveto{\pgfqpoint{9.447393in}{1.773315in}}{\pgfqpoint{9.445389in}{1.768477in}}{\pgfqpoint{9.445389in}{1.763433in}}%
\pgfpathcurveto{\pgfqpoint{9.445389in}{1.758390in}}{\pgfqpoint{9.447393in}{1.753552in}}{\pgfqpoint{9.450959in}{1.749986in}}%
\pgfpathcurveto{\pgfqpoint{9.454526in}{1.746419in}}{\pgfqpoint{9.459363in}{1.744415in}}{\pgfqpoint{9.464407in}{1.744415in}}%
\pgfpathclose%
\pgfusepath{fill}%
\end{pgfscope}%
\begin{pgfscope}%
\pgfpathrectangle{\pgfqpoint{6.572727in}{0.473000in}}{\pgfqpoint{4.227273in}{3.311000in}}%
\pgfusepath{clip}%
\pgfsetbuttcap%
\pgfsetroundjoin%
\definecolor{currentfill}{rgb}{0.127568,0.566949,0.550556}%
\pgfsetfillcolor{currentfill}%
\pgfsetfillopacity{0.700000}%
\pgfsetlinewidth{0.000000pt}%
\definecolor{currentstroke}{rgb}{0.000000,0.000000,0.000000}%
\pgfsetstrokecolor{currentstroke}%
\pgfsetstrokeopacity{0.700000}%
\pgfsetdash{}{0pt}%
\pgfpathmoveto{\pgfqpoint{8.176808in}{2.537178in}}%
\pgfpathcurveto{\pgfqpoint{8.181852in}{2.537178in}}{\pgfqpoint{8.186689in}{2.539182in}}{\pgfqpoint{8.190256in}{2.542749in}}%
\pgfpathcurveto{\pgfqpoint{8.193822in}{2.546315in}}{\pgfqpoint{8.195826in}{2.551153in}}{\pgfqpoint{8.195826in}{2.556197in}}%
\pgfpathcurveto{\pgfqpoint{8.195826in}{2.561240in}}{\pgfqpoint{8.193822in}{2.566078in}}{\pgfqpoint{8.190256in}{2.569644in}}%
\pgfpathcurveto{\pgfqpoint{8.186689in}{2.573211in}}{\pgfqpoint{8.181852in}{2.575215in}}{\pgfqpoint{8.176808in}{2.575215in}}%
\pgfpathcurveto{\pgfqpoint{8.171764in}{2.575215in}}{\pgfqpoint{8.166926in}{2.573211in}}{\pgfqpoint{8.163360in}{2.569644in}}%
\pgfpathcurveto{\pgfqpoint{8.159794in}{2.566078in}}{\pgfqpoint{8.157790in}{2.561240in}}{\pgfqpoint{8.157790in}{2.556197in}}%
\pgfpathcurveto{\pgfqpoint{8.157790in}{2.551153in}}{\pgfqpoint{8.159794in}{2.546315in}}{\pgfqpoint{8.163360in}{2.542749in}}%
\pgfpathcurveto{\pgfqpoint{8.166926in}{2.539182in}}{\pgfqpoint{8.171764in}{2.537178in}}{\pgfqpoint{8.176808in}{2.537178in}}%
\pgfpathclose%
\pgfusepath{fill}%
\end{pgfscope}%
\begin{pgfscope}%
\pgfpathrectangle{\pgfqpoint{6.572727in}{0.473000in}}{\pgfqpoint{4.227273in}{3.311000in}}%
\pgfusepath{clip}%
\pgfsetbuttcap%
\pgfsetroundjoin%
\definecolor{currentfill}{rgb}{0.993248,0.906157,0.143936}%
\pgfsetfillcolor{currentfill}%
\pgfsetfillopacity{0.700000}%
\pgfsetlinewidth{0.000000pt}%
\definecolor{currentstroke}{rgb}{0.000000,0.000000,0.000000}%
\pgfsetstrokecolor{currentstroke}%
\pgfsetstrokeopacity{0.700000}%
\pgfsetdash{}{0pt}%
\pgfpathmoveto{\pgfqpoint{9.450364in}{1.845767in}}%
\pgfpathcurveto{\pgfqpoint{9.455408in}{1.845767in}}{\pgfqpoint{9.460246in}{1.847771in}}{\pgfqpoint{9.463812in}{1.851337in}}%
\pgfpathcurveto{\pgfqpoint{9.467379in}{1.854904in}}{\pgfqpoint{9.469382in}{1.859742in}}{\pgfqpoint{9.469382in}{1.864785in}}%
\pgfpathcurveto{\pgfqpoint{9.469382in}{1.869829in}}{\pgfqpoint{9.467379in}{1.874667in}}{\pgfqpoint{9.463812in}{1.878233in}}%
\pgfpathcurveto{\pgfqpoint{9.460246in}{1.881799in}}{\pgfqpoint{9.455408in}{1.883803in}}{\pgfqpoint{9.450364in}{1.883803in}}%
\pgfpathcurveto{\pgfqpoint{9.445321in}{1.883803in}}{\pgfqpoint{9.440483in}{1.881799in}}{\pgfqpoint{9.436916in}{1.878233in}}%
\pgfpathcurveto{\pgfqpoint{9.433350in}{1.874667in}}{\pgfqpoint{9.431346in}{1.869829in}}{\pgfqpoint{9.431346in}{1.864785in}}%
\pgfpathcurveto{\pgfqpoint{9.431346in}{1.859742in}}{\pgfqpoint{9.433350in}{1.854904in}}{\pgfqpoint{9.436916in}{1.851337in}}%
\pgfpathcurveto{\pgfqpoint{9.440483in}{1.847771in}}{\pgfqpoint{9.445321in}{1.845767in}}{\pgfqpoint{9.450364in}{1.845767in}}%
\pgfpathclose%
\pgfusepath{fill}%
\end{pgfscope}%
\begin{pgfscope}%
\pgfpathrectangle{\pgfqpoint{6.572727in}{0.473000in}}{\pgfqpoint{4.227273in}{3.311000in}}%
\pgfusepath{clip}%
\pgfsetbuttcap%
\pgfsetroundjoin%
\definecolor{currentfill}{rgb}{0.127568,0.566949,0.550556}%
\pgfsetfillcolor{currentfill}%
\pgfsetfillopacity{0.700000}%
\pgfsetlinewidth{0.000000pt}%
\definecolor{currentstroke}{rgb}{0.000000,0.000000,0.000000}%
\pgfsetstrokecolor{currentstroke}%
\pgfsetstrokeopacity{0.700000}%
\pgfsetdash{}{0pt}%
\pgfpathmoveto{\pgfqpoint{9.039734in}{3.254525in}}%
\pgfpathcurveto{\pgfqpoint{9.044777in}{3.254525in}}{\pgfqpoint{9.049615in}{3.256529in}}{\pgfqpoint{9.053182in}{3.260095in}}%
\pgfpathcurveto{\pgfqpoint{9.056748in}{3.263662in}}{\pgfqpoint{9.058752in}{3.268499in}}{\pgfqpoint{9.058752in}{3.273543in}}%
\pgfpathcurveto{\pgfqpoint{9.058752in}{3.278587in}}{\pgfqpoint{9.056748in}{3.283424in}}{\pgfqpoint{9.053182in}{3.286991in}}%
\pgfpathcurveto{\pgfqpoint{9.049615in}{3.290557in}}{\pgfqpoint{9.044777in}{3.292561in}}{\pgfqpoint{9.039734in}{3.292561in}}%
\pgfpathcurveto{\pgfqpoint{9.034690in}{3.292561in}}{\pgfqpoint{9.029852in}{3.290557in}}{\pgfqpoint{9.026286in}{3.286991in}}%
\pgfpathcurveto{\pgfqpoint{9.022720in}{3.283424in}}{\pgfqpoint{9.020716in}{3.278587in}}{\pgfqpoint{9.020716in}{3.273543in}}%
\pgfpathcurveto{\pgfqpoint{9.020716in}{3.268499in}}{\pgfqpoint{9.022720in}{3.263662in}}{\pgfqpoint{9.026286in}{3.260095in}}%
\pgfpathcurveto{\pgfqpoint{9.029852in}{3.256529in}}{\pgfqpoint{9.034690in}{3.254525in}}{\pgfqpoint{9.039734in}{3.254525in}}%
\pgfpathclose%
\pgfusepath{fill}%
\end{pgfscope}%
\begin{pgfscope}%
\pgfpathrectangle{\pgfqpoint{6.572727in}{0.473000in}}{\pgfqpoint{4.227273in}{3.311000in}}%
\pgfusepath{clip}%
\pgfsetbuttcap%
\pgfsetroundjoin%
\definecolor{currentfill}{rgb}{0.127568,0.566949,0.550556}%
\pgfsetfillcolor{currentfill}%
\pgfsetfillopacity{0.700000}%
\pgfsetlinewidth{0.000000pt}%
\definecolor{currentstroke}{rgb}{0.000000,0.000000,0.000000}%
\pgfsetstrokecolor{currentstroke}%
\pgfsetstrokeopacity{0.700000}%
\pgfsetdash{}{0pt}%
\pgfpathmoveto{\pgfqpoint{8.006949in}{1.353855in}}%
\pgfpathcurveto{\pgfqpoint{8.011993in}{1.353855in}}{\pgfqpoint{8.016831in}{1.355859in}}{\pgfqpoint{8.020397in}{1.359425in}}%
\pgfpathcurveto{\pgfqpoint{8.023963in}{1.362991in}}{\pgfqpoint{8.025967in}{1.367829in}}{\pgfqpoint{8.025967in}{1.372873in}}%
\pgfpathcurveto{\pgfqpoint{8.025967in}{1.377916in}}{\pgfqpoint{8.023963in}{1.382754in}}{\pgfqpoint{8.020397in}{1.386321in}}%
\pgfpathcurveto{\pgfqpoint{8.016831in}{1.389887in}}{\pgfqpoint{8.011993in}{1.391891in}}{\pgfqpoint{8.006949in}{1.391891in}}%
\pgfpathcurveto{\pgfqpoint{8.001905in}{1.391891in}}{\pgfqpoint{7.997068in}{1.389887in}}{\pgfqpoint{7.993501in}{1.386321in}}%
\pgfpathcurveto{\pgfqpoint{7.989935in}{1.382754in}}{\pgfqpoint{7.987931in}{1.377916in}}{\pgfqpoint{7.987931in}{1.372873in}}%
\pgfpathcurveto{\pgfqpoint{7.987931in}{1.367829in}}{\pgfqpoint{7.989935in}{1.362991in}}{\pgfqpoint{7.993501in}{1.359425in}}%
\pgfpathcurveto{\pgfqpoint{7.997068in}{1.355859in}}{\pgfqpoint{8.001905in}{1.353855in}}{\pgfqpoint{8.006949in}{1.353855in}}%
\pgfpathclose%
\pgfusepath{fill}%
\end{pgfscope}%
\begin{pgfscope}%
\pgfpathrectangle{\pgfqpoint{6.572727in}{0.473000in}}{\pgfqpoint{4.227273in}{3.311000in}}%
\pgfusepath{clip}%
\pgfsetbuttcap%
\pgfsetroundjoin%
\definecolor{currentfill}{rgb}{0.127568,0.566949,0.550556}%
\pgfsetfillcolor{currentfill}%
\pgfsetfillopacity{0.700000}%
\pgfsetlinewidth{0.000000pt}%
\definecolor{currentstroke}{rgb}{0.000000,0.000000,0.000000}%
\pgfsetstrokecolor{currentstroke}%
\pgfsetstrokeopacity{0.700000}%
\pgfsetdash{}{0pt}%
\pgfpathmoveto{\pgfqpoint{7.678302in}{0.970329in}}%
\pgfpathcurveto{\pgfqpoint{7.683346in}{0.970329in}}{\pgfqpoint{7.688184in}{0.972333in}}{\pgfqpoint{7.691750in}{0.975899in}}%
\pgfpathcurveto{\pgfqpoint{7.695317in}{0.979466in}}{\pgfqpoint{7.697321in}{0.984304in}}{\pgfqpoint{7.697321in}{0.989347in}}%
\pgfpathcurveto{\pgfqpoint{7.697321in}{0.994391in}}{\pgfqpoint{7.695317in}{0.999229in}}{\pgfqpoint{7.691750in}{1.002795in}}%
\pgfpathcurveto{\pgfqpoint{7.688184in}{1.006361in}}{\pgfqpoint{7.683346in}{1.008365in}}{\pgfqpoint{7.678302in}{1.008365in}}%
\pgfpathcurveto{\pgfqpoint{7.673259in}{1.008365in}}{\pgfqpoint{7.668421in}{1.006361in}}{\pgfqpoint{7.664855in}{1.002795in}}%
\pgfpathcurveto{\pgfqpoint{7.661288in}{0.999229in}}{\pgfqpoint{7.659284in}{0.994391in}}{\pgfqpoint{7.659284in}{0.989347in}}%
\pgfpathcurveto{\pgfqpoint{7.659284in}{0.984304in}}{\pgfqpoint{7.661288in}{0.979466in}}{\pgfqpoint{7.664855in}{0.975899in}}%
\pgfpathcurveto{\pgfqpoint{7.668421in}{0.972333in}}{\pgfqpoint{7.673259in}{0.970329in}}{\pgfqpoint{7.678302in}{0.970329in}}%
\pgfpathclose%
\pgfusepath{fill}%
\end{pgfscope}%
\begin{pgfscope}%
\pgfpathrectangle{\pgfqpoint{6.572727in}{0.473000in}}{\pgfqpoint{4.227273in}{3.311000in}}%
\pgfusepath{clip}%
\pgfsetbuttcap%
\pgfsetroundjoin%
\definecolor{currentfill}{rgb}{0.993248,0.906157,0.143936}%
\pgfsetfillcolor{currentfill}%
\pgfsetfillopacity{0.700000}%
\pgfsetlinewidth{0.000000pt}%
\definecolor{currentstroke}{rgb}{0.000000,0.000000,0.000000}%
\pgfsetstrokecolor{currentstroke}%
\pgfsetstrokeopacity{0.700000}%
\pgfsetdash{}{0pt}%
\pgfpathmoveto{\pgfqpoint{9.677000in}{1.225782in}}%
\pgfpathcurveto{\pgfqpoint{9.682043in}{1.225782in}}{\pgfqpoint{9.686881in}{1.227786in}}{\pgfqpoint{9.690448in}{1.231352in}}%
\pgfpathcurveto{\pgfqpoint{9.694014in}{1.234918in}}{\pgfqpoint{9.696018in}{1.239756in}}{\pgfqpoint{9.696018in}{1.244800in}}%
\pgfpathcurveto{\pgfqpoint{9.696018in}{1.249844in}}{\pgfqpoint{9.694014in}{1.254681in}}{\pgfqpoint{9.690448in}{1.258248in}}%
\pgfpathcurveto{\pgfqpoint{9.686881in}{1.261814in}}{\pgfqpoint{9.682043in}{1.263818in}}{\pgfqpoint{9.677000in}{1.263818in}}%
\pgfpathcurveto{\pgfqpoint{9.671956in}{1.263818in}}{\pgfqpoint{9.667118in}{1.261814in}}{\pgfqpoint{9.663552in}{1.258248in}}%
\pgfpathcurveto{\pgfqpoint{9.659985in}{1.254681in}}{\pgfqpoint{9.657982in}{1.249844in}}{\pgfqpoint{9.657982in}{1.244800in}}%
\pgfpathcurveto{\pgfqpoint{9.657982in}{1.239756in}}{\pgfqpoint{9.659985in}{1.234918in}}{\pgfqpoint{9.663552in}{1.231352in}}%
\pgfpathcurveto{\pgfqpoint{9.667118in}{1.227786in}}{\pgfqpoint{9.671956in}{1.225782in}}{\pgfqpoint{9.677000in}{1.225782in}}%
\pgfpathclose%
\pgfusepath{fill}%
\end{pgfscope}%
\begin{pgfscope}%
\pgfpathrectangle{\pgfqpoint{6.572727in}{0.473000in}}{\pgfqpoint{4.227273in}{3.311000in}}%
\pgfusepath{clip}%
\pgfsetbuttcap%
\pgfsetroundjoin%
\definecolor{currentfill}{rgb}{0.127568,0.566949,0.550556}%
\pgfsetfillcolor{currentfill}%
\pgfsetfillopacity{0.700000}%
\pgfsetlinewidth{0.000000pt}%
\definecolor{currentstroke}{rgb}{0.000000,0.000000,0.000000}%
\pgfsetstrokecolor{currentstroke}%
\pgfsetstrokeopacity{0.700000}%
\pgfsetdash{}{0pt}%
\pgfpathmoveto{\pgfqpoint{7.666068in}{1.314827in}}%
\pgfpathcurveto{\pgfqpoint{7.671112in}{1.314827in}}{\pgfqpoint{7.675950in}{1.316830in}}{\pgfqpoint{7.679516in}{1.320397in}}%
\pgfpathcurveto{\pgfqpoint{7.683083in}{1.323963in}}{\pgfqpoint{7.685086in}{1.328801in}}{\pgfqpoint{7.685086in}{1.333845in}}%
\pgfpathcurveto{\pgfqpoint{7.685086in}{1.338888in}}{\pgfqpoint{7.683083in}{1.343726in}}{\pgfqpoint{7.679516in}{1.347293in}}%
\pgfpathcurveto{\pgfqpoint{7.675950in}{1.350859in}}{\pgfqpoint{7.671112in}{1.352863in}}{\pgfqpoint{7.666068in}{1.352863in}}%
\pgfpathcurveto{\pgfqpoint{7.661025in}{1.352863in}}{\pgfqpoint{7.656187in}{1.350859in}}{\pgfqpoint{7.652620in}{1.347293in}}%
\pgfpathcurveto{\pgfqpoint{7.649054in}{1.343726in}}{\pgfqpoint{7.647050in}{1.338888in}}{\pgfqpoint{7.647050in}{1.333845in}}%
\pgfpathcurveto{\pgfqpoint{7.647050in}{1.328801in}}{\pgfqpoint{7.649054in}{1.323963in}}{\pgfqpoint{7.652620in}{1.320397in}}%
\pgfpathcurveto{\pgfqpoint{7.656187in}{1.316830in}}{\pgfqpoint{7.661025in}{1.314827in}}{\pgfqpoint{7.666068in}{1.314827in}}%
\pgfpathclose%
\pgfusepath{fill}%
\end{pgfscope}%
\begin{pgfscope}%
\pgfpathrectangle{\pgfqpoint{6.572727in}{0.473000in}}{\pgfqpoint{4.227273in}{3.311000in}}%
\pgfusepath{clip}%
\pgfsetbuttcap%
\pgfsetroundjoin%
\definecolor{currentfill}{rgb}{0.993248,0.906157,0.143936}%
\pgfsetfillcolor{currentfill}%
\pgfsetfillopacity{0.700000}%
\pgfsetlinewidth{0.000000pt}%
\definecolor{currentstroke}{rgb}{0.000000,0.000000,0.000000}%
\pgfsetstrokecolor{currentstroke}%
\pgfsetstrokeopacity{0.700000}%
\pgfsetdash{}{0pt}%
\pgfpathmoveto{\pgfqpoint{9.636833in}{1.724331in}}%
\pgfpathcurveto{\pgfqpoint{9.641876in}{1.724331in}}{\pgfqpoint{9.646714in}{1.726335in}}{\pgfqpoint{9.650280in}{1.729901in}}%
\pgfpathcurveto{\pgfqpoint{9.653847in}{1.733467in}}{\pgfqpoint{9.655851in}{1.738305in}}{\pgfqpoint{9.655851in}{1.743349in}}%
\pgfpathcurveto{\pgfqpoint{9.655851in}{1.748392in}}{\pgfqpoint{9.653847in}{1.753230in}}{\pgfqpoint{9.650280in}{1.756797in}}%
\pgfpathcurveto{\pgfqpoint{9.646714in}{1.760363in}}{\pgfqpoint{9.641876in}{1.762367in}}{\pgfqpoint{9.636833in}{1.762367in}}%
\pgfpathcurveto{\pgfqpoint{9.631789in}{1.762367in}}{\pgfqpoint{9.626951in}{1.760363in}}{\pgfqpoint{9.623385in}{1.756797in}}%
\pgfpathcurveto{\pgfqpoint{9.619818in}{1.753230in}}{\pgfqpoint{9.617814in}{1.748392in}}{\pgfqpoint{9.617814in}{1.743349in}}%
\pgfpathcurveto{\pgfqpoint{9.617814in}{1.738305in}}{\pgfqpoint{9.619818in}{1.733467in}}{\pgfqpoint{9.623385in}{1.729901in}}%
\pgfpathcurveto{\pgfqpoint{9.626951in}{1.726335in}}{\pgfqpoint{9.631789in}{1.724331in}}{\pgfqpoint{9.636833in}{1.724331in}}%
\pgfpathclose%
\pgfusepath{fill}%
\end{pgfscope}%
\begin{pgfscope}%
\pgfpathrectangle{\pgfqpoint{6.572727in}{0.473000in}}{\pgfqpoint{4.227273in}{3.311000in}}%
\pgfusepath{clip}%
\pgfsetbuttcap%
\pgfsetroundjoin%
\definecolor{currentfill}{rgb}{0.993248,0.906157,0.143936}%
\pgfsetfillcolor{currentfill}%
\pgfsetfillopacity{0.700000}%
\pgfsetlinewidth{0.000000pt}%
\definecolor{currentstroke}{rgb}{0.000000,0.000000,0.000000}%
\pgfsetstrokecolor{currentstroke}%
\pgfsetstrokeopacity{0.700000}%
\pgfsetdash{}{0pt}%
\pgfpathmoveto{\pgfqpoint{9.946611in}{1.898390in}}%
\pgfpathcurveto{\pgfqpoint{9.951655in}{1.898390in}}{\pgfqpoint{9.956493in}{1.900394in}}{\pgfqpoint{9.960059in}{1.903961in}}%
\pgfpathcurveto{\pgfqpoint{9.963625in}{1.907527in}}{\pgfqpoint{9.965629in}{1.912365in}}{\pgfqpoint{9.965629in}{1.917409in}}%
\pgfpathcurveto{\pgfqpoint{9.965629in}{1.922452in}}{\pgfqpoint{9.963625in}{1.927290in}}{\pgfqpoint{9.960059in}{1.930856in}}%
\pgfpathcurveto{\pgfqpoint{9.956493in}{1.934423in}}{\pgfqpoint{9.951655in}{1.936427in}}{\pgfqpoint{9.946611in}{1.936427in}}%
\pgfpathcurveto{\pgfqpoint{9.941568in}{1.936427in}}{\pgfqpoint{9.936730in}{1.934423in}}{\pgfqpoint{9.933163in}{1.930856in}}%
\pgfpathcurveto{\pgfqpoint{9.929597in}{1.927290in}}{\pgfqpoint{9.927593in}{1.922452in}}{\pgfqpoint{9.927593in}{1.917409in}}%
\pgfpathcurveto{\pgfqpoint{9.927593in}{1.912365in}}{\pgfqpoint{9.929597in}{1.907527in}}{\pgfqpoint{9.933163in}{1.903961in}}%
\pgfpathcurveto{\pgfqpoint{9.936730in}{1.900394in}}{\pgfqpoint{9.941568in}{1.898390in}}{\pgfqpoint{9.946611in}{1.898390in}}%
\pgfpathclose%
\pgfusepath{fill}%
\end{pgfscope}%
\begin{pgfscope}%
\pgfpathrectangle{\pgfqpoint{6.572727in}{0.473000in}}{\pgfqpoint{4.227273in}{3.311000in}}%
\pgfusepath{clip}%
\pgfsetbuttcap%
\pgfsetroundjoin%
\definecolor{currentfill}{rgb}{0.127568,0.566949,0.550556}%
\pgfsetfillcolor{currentfill}%
\pgfsetfillopacity{0.700000}%
\pgfsetlinewidth{0.000000pt}%
\definecolor{currentstroke}{rgb}{0.000000,0.000000,0.000000}%
\pgfsetstrokecolor{currentstroke}%
\pgfsetstrokeopacity{0.700000}%
\pgfsetdash{}{0pt}%
\pgfpathmoveto{\pgfqpoint{8.294514in}{1.451395in}}%
\pgfpathcurveto{\pgfqpoint{8.299558in}{1.451395in}}{\pgfqpoint{8.304396in}{1.453399in}}{\pgfqpoint{8.307962in}{1.456965in}}%
\pgfpathcurveto{\pgfqpoint{8.311528in}{1.460532in}}{\pgfqpoint{8.313532in}{1.465369in}}{\pgfqpoint{8.313532in}{1.470413in}}%
\pgfpathcurveto{\pgfqpoint{8.313532in}{1.475457in}}{\pgfqpoint{8.311528in}{1.480294in}}{\pgfqpoint{8.307962in}{1.483861in}}%
\pgfpathcurveto{\pgfqpoint{8.304396in}{1.487427in}}{\pgfqpoint{8.299558in}{1.489431in}}{\pgfqpoint{8.294514in}{1.489431in}}%
\pgfpathcurveto{\pgfqpoint{8.289470in}{1.489431in}}{\pgfqpoint{8.284633in}{1.487427in}}{\pgfqpoint{8.281066in}{1.483861in}}%
\pgfpathcurveto{\pgfqpoint{8.277500in}{1.480294in}}{\pgfqpoint{8.275496in}{1.475457in}}{\pgfqpoint{8.275496in}{1.470413in}}%
\pgfpathcurveto{\pgfqpoint{8.275496in}{1.465369in}}{\pgfqpoint{8.277500in}{1.460532in}}{\pgfqpoint{8.281066in}{1.456965in}}%
\pgfpathcurveto{\pgfqpoint{8.284633in}{1.453399in}}{\pgfqpoint{8.289470in}{1.451395in}}{\pgfqpoint{8.294514in}{1.451395in}}%
\pgfpathclose%
\pgfusepath{fill}%
\end{pgfscope}%
\begin{pgfscope}%
\pgfpathrectangle{\pgfqpoint{6.572727in}{0.473000in}}{\pgfqpoint{4.227273in}{3.311000in}}%
\pgfusepath{clip}%
\pgfsetbuttcap%
\pgfsetroundjoin%
\definecolor{currentfill}{rgb}{0.127568,0.566949,0.550556}%
\pgfsetfillcolor{currentfill}%
\pgfsetfillopacity{0.700000}%
\pgfsetlinewidth{0.000000pt}%
\definecolor{currentstroke}{rgb}{0.000000,0.000000,0.000000}%
\pgfsetstrokecolor{currentstroke}%
\pgfsetstrokeopacity{0.700000}%
\pgfsetdash{}{0pt}%
\pgfpathmoveto{\pgfqpoint{7.598975in}{1.200886in}}%
\pgfpathcurveto{\pgfqpoint{7.604018in}{1.200886in}}{\pgfqpoint{7.608856in}{1.202890in}}{\pgfqpoint{7.612422in}{1.206456in}}%
\pgfpathcurveto{\pgfqpoint{7.615989in}{1.210022in}}{\pgfqpoint{7.617993in}{1.214860in}}{\pgfqpoint{7.617993in}{1.219904in}}%
\pgfpathcurveto{\pgfqpoint{7.617993in}{1.224947in}}{\pgfqpoint{7.615989in}{1.229785in}}{\pgfqpoint{7.612422in}{1.233352in}}%
\pgfpathcurveto{\pgfqpoint{7.608856in}{1.236918in}}{\pgfqpoint{7.604018in}{1.238922in}}{\pgfqpoint{7.598975in}{1.238922in}}%
\pgfpathcurveto{\pgfqpoint{7.593931in}{1.238922in}}{\pgfqpoint{7.589093in}{1.236918in}}{\pgfqpoint{7.585527in}{1.233352in}}%
\pgfpathcurveto{\pgfqpoint{7.581960in}{1.229785in}}{\pgfqpoint{7.579956in}{1.224947in}}{\pgfqpoint{7.579956in}{1.219904in}}%
\pgfpathcurveto{\pgfqpoint{7.579956in}{1.214860in}}{\pgfqpoint{7.581960in}{1.210022in}}{\pgfqpoint{7.585527in}{1.206456in}}%
\pgfpathcurveto{\pgfqpoint{7.589093in}{1.202890in}}{\pgfqpoint{7.593931in}{1.200886in}}{\pgfqpoint{7.598975in}{1.200886in}}%
\pgfpathclose%
\pgfusepath{fill}%
\end{pgfscope}%
\begin{pgfscope}%
\pgfpathrectangle{\pgfqpoint{6.572727in}{0.473000in}}{\pgfqpoint{4.227273in}{3.311000in}}%
\pgfusepath{clip}%
\pgfsetbuttcap%
\pgfsetroundjoin%
\definecolor{currentfill}{rgb}{0.127568,0.566949,0.550556}%
\pgfsetfillcolor{currentfill}%
\pgfsetfillopacity{0.700000}%
\pgfsetlinewidth{0.000000pt}%
\definecolor{currentstroke}{rgb}{0.000000,0.000000,0.000000}%
\pgfsetstrokecolor{currentstroke}%
\pgfsetstrokeopacity{0.700000}%
\pgfsetdash{}{0pt}%
\pgfpathmoveto{\pgfqpoint{7.973042in}{3.323606in}}%
\pgfpathcurveto{\pgfqpoint{7.978086in}{3.323606in}}{\pgfqpoint{7.982924in}{3.325610in}}{\pgfqpoint{7.986490in}{3.329176in}}%
\pgfpathcurveto{\pgfqpoint{7.990057in}{3.332743in}}{\pgfqpoint{7.992061in}{3.337581in}}{\pgfqpoint{7.992061in}{3.342624in}}%
\pgfpathcurveto{\pgfqpoint{7.992061in}{3.347668in}}{\pgfqpoint{7.990057in}{3.352506in}}{\pgfqpoint{7.986490in}{3.356072in}}%
\pgfpathcurveto{\pgfqpoint{7.982924in}{3.359639in}}{\pgfqpoint{7.978086in}{3.361642in}}{\pgfqpoint{7.973042in}{3.361642in}}%
\pgfpathcurveto{\pgfqpoint{7.967999in}{3.361642in}}{\pgfqpoint{7.963161in}{3.359639in}}{\pgfqpoint{7.959595in}{3.356072in}}%
\pgfpathcurveto{\pgfqpoint{7.956028in}{3.352506in}}{\pgfqpoint{7.954024in}{3.347668in}}{\pgfqpoint{7.954024in}{3.342624in}}%
\pgfpathcurveto{\pgfqpoint{7.954024in}{3.337581in}}{\pgfqpoint{7.956028in}{3.332743in}}{\pgfqpoint{7.959595in}{3.329176in}}%
\pgfpathcurveto{\pgfqpoint{7.963161in}{3.325610in}}{\pgfqpoint{7.967999in}{3.323606in}}{\pgfqpoint{7.973042in}{3.323606in}}%
\pgfpathclose%
\pgfusepath{fill}%
\end{pgfscope}%
\begin{pgfscope}%
\pgfpathrectangle{\pgfqpoint{6.572727in}{0.473000in}}{\pgfqpoint{4.227273in}{3.311000in}}%
\pgfusepath{clip}%
\pgfsetbuttcap%
\pgfsetroundjoin%
\definecolor{currentfill}{rgb}{0.127568,0.566949,0.550556}%
\pgfsetfillcolor{currentfill}%
\pgfsetfillopacity{0.700000}%
\pgfsetlinewidth{0.000000pt}%
\definecolor{currentstroke}{rgb}{0.000000,0.000000,0.000000}%
\pgfsetstrokecolor{currentstroke}%
\pgfsetstrokeopacity{0.700000}%
\pgfsetdash{}{0pt}%
\pgfpathmoveto{\pgfqpoint{7.046167in}{1.750087in}}%
\pgfpathcurveto{\pgfqpoint{7.051210in}{1.750087in}}{\pgfqpoint{7.056048in}{1.752091in}}{\pgfqpoint{7.059614in}{1.755657in}}%
\pgfpathcurveto{\pgfqpoint{7.063181in}{1.759223in}}{\pgfqpoint{7.065185in}{1.764061in}}{\pgfqpoint{7.065185in}{1.769105in}}%
\pgfpathcurveto{\pgfqpoint{7.065185in}{1.774149in}}{\pgfqpoint{7.063181in}{1.778986in}}{\pgfqpoint{7.059614in}{1.782553in}}%
\pgfpathcurveto{\pgfqpoint{7.056048in}{1.786119in}}{\pgfqpoint{7.051210in}{1.788123in}}{\pgfqpoint{7.046167in}{1.788123in}}%
\pgfpathcurveto{\pgfqpoint{7.041123in}{1.788123in}}{\pgfqpoint{7.036285in}{1.786119in}}{\pgfqpoint{7.032719in}{1.782553in}}%
\pgfpathcurveto{\pgfqpoint{7.029152in}{1.778986in}}{\pgfqpoint{7.027148in}{1.774149in}}{\pgfqpoint{7.027148in}{1.769105in}}%
\pgfpathcurveto{\pgfqpoint{7.027148in}{1.764061in}}{\pgfqpoint{7.029152in}{1.759223in}}{\pgfqpoint{7.032719in}{1.755657in}}%
\pgfpathcurveto{\pgfqpoint{7.036285in}{1.752091in}}{\pgfqpoint{7.041123in}{1.750087in}}{\pgfqpoint{7.046167in}{1.750087in}}%
\pgfpathclose%
\pgfusepath{fill}%
\end{pgfscope}%
\begin{pgfscope}%
\pgfpathrectangle{\pgfqpoint{6.572727in}{0.473000in}}{\pgfqpoint{4.227273in}{3.311000in}}%
\pgfusepath{clip}%
\pgfsetbuttcap%
\pgfsetroundjoin%
\definecolor{currentfill}{rgb}{0.127568,0.566949,0.550556}%
\pgfsetfillcolor{currentfill}%
\pgfsetfillopacity{0.700000}%
\pgfsetlinewidth{0.000000pt}%
\definecolor{currentstroke}{rgb}{0.000000,0.000000,0.000000}%
\pgfsetstrokecolor{currentstroke}%
\pgfsetstrokeopacity{0.700000}%
\pgfsetdash{}{0pt}%
\pgfpathmoveto{\pgfqpoint{8.564128in}{2.900011in}}%
\pgfpathcurveto{\pgfqpoint{8.569171in}{2.900011in}}{\pgfqpoint{8.574009in}{2.902015in}}{\pgfqpoint{8.577575in}{2.905581in}}%
\pgfpathcurveto{\pgfqpoint{8.581142in}{2.909148in}}{\pgfqpoint{8.583146in}{2.913985in}}{\pgfqpoint{8.583146in}{2.919029in}}%
\pgfpathcurveto{\pgfqpoint{8.583146in}{2.924073in}}{\pgfqpoint{8.581142in}{2.928910in}}{\pgfqpoint{8.577575in}{2.932477in}}%
\pgfpathcurveto{\pgfqpoint{8.574009in}{2.936043in}}{\pgfqpoint{8.569171in}{2.938047in}}{\pgfqpoint{8.564128in}{2.938047in}}%
\pgfpathcurveto{\pgfqpoint{8.559084in}{2.938047in}}{\pgfqpoint{8.554246in}{2.936043in}}{\pgfqpoint{8.550680in}{2.932477in}}%
\pgfpathcurveto{\pgfqpoint{8.547113in}{2.928910in}}{\pgfqpoint{8.545109in}{2.924073in}}{\pgfqpoint{8.545109in}{2.919029in}}%
\pgfpathcurveto{\pgfqpoint{8.545109in}{2.913985in}}{\pgfqpoint{8.547113in}{2.909148in}}{\pgfqpoint{8.550680in}{2.905581in}}%
\pgfpathcurveto{\pgfqpoint{8.554246in}{2.902015in}}{\pgfqpoint{8.559084in}{2.900011in}}{\pgfqpoint{8.564128in}{2.900011in}}%
\pgfpathclose%
\pgfusepath{fill}%
\end{pgfscope}%
\begin{pgfscope}%
\pgfpathrectangle{\pgfqpoint{6.572727in}{0.473000in}}{\pgfqpoint{4.227273in}{3.311000in}}%
\pgfusepath{clip}%
\pgfsetbuttcap%
\pgfsetroundjoin%
\definecolor{currentfill}{rgb}{0.993248,0.906157,0.143936}%
\pgfsetfillcolor{currentfill}%
\pgfsetfillopacity{0.700000}%
\pgfsetlinewidth{0.000000pt}%
\definecolor{currentstroke}{rgb}{0.000000,0.000000,0.000000}%
\pgfsetstrokecolor{currentstroke}%
\pgfsetstrokeopacity{0.700000}%
\pgfsetdash{}{0pt}%
\pgfpathmoveto{\pgfqpoint{10.011983in}{1.210693in}}%
\pgfpathcurveto{\pgfqpoint{10.017026in}{1.210693in}}{\pgfqpoint{10.021864in}{1.212697in}}{\pgfqpoint{10.025431in}{1.216264in}}%
\pgfpathcurveto{\pgfqpoint{10.028997in}{1.219830in}}{\pgfqpoint{10.031001in}{1.224668in}}{\pgfqpoint{10.031001in}{1.229712in}}%
\pgfpathcurveto{\pgfqpoint{10.031001in}{1.234755in}}{\pgfqpoint{10.028997in}{1.239593in}}{\pgfqpoint{10.025431in}{1.243159in}}%
\pgfpathcurveto{\pgfqpoint{10.021864in}{1.246726in}}{\pgfqpoint{10.017026in}{1.248730in}}{\pgfqpoint{10.011983in}{1.248730in}}%
\pgfpathcurveto{\pgfqpoint{10.006939in}{1.248730in}}{\pgfqpoint{10.002101in}{1.246726in}}{\pgfqpoint{9.998535in}{1.243159in}}%
\pgfpathcurveto{\pgfqpoint{9.994968in}{1.239593in}}{\pgfqpoint{9.992965in}{1.234755in}}{\pgfqpoint{9.992965in}{1.229712in}}%
\pgfpathcurveto{\pgfqpoint{9.992965in}{1.224668in}}{\pgfqpoint{9.994968in}{1.219830in}}{\pgfqpoint{9.998535in}{1.216264in}}%
\pgfpathcurveto{\pgfqpoint{10.002101in}{1.212697in}}{\pgfqpoint{10.006939in}{1.210693in}}{\pgfqpoint{10.011983in}{1.210693in}}%
\pgfpathclose%
\pgfusepath{fill}%
\end{pgfscope}%
\begin{pgfscope}%
\pgfpathrectangle{\pgfqpoint{6.572727in}{0.473000in}}{\pgfqpoint{4.227273in}{3.311000in}}%
\pgfusepath{clip}%
\pgfsetbuttcap%
\pgfsetroundjoin%
\definecolor{currentfill}{rgb}{0.993248,0.906157,0.143936}%
\pgfsetfillcolor{currentfill}%
\pgfsetfillopacity{0.700000}%
\pgfsetlinewidth{0.000000pt}%
\definecolor{currentstroke}{rgb}{0.000000,0.000000,0.000000}%
\pgfsetstrokecolor{currentstroke}%
\pgfsetstrokeopacity{0.700000}%
\pgfsetdash{}{0pt}%
\pgfpathmoveto{\pgfqpoint{9.948581in}{1.461755in}}%
\pgfpathcurveto{\pgfqpoint{9.953625in}{1.461755in}}{\pgfqpoint{9.958462in}{1.463759in}}{\pgfqpoint{9.962029in}{1.467325in}}%
\pgfpathcurveto{\pgfqpoint{9.965595in}{1.470892in}}{\pgfqpoint{9.967599in}{1.475729in}}{\pgfqpoint{9.967599in}{1.480773in}}%
\pgfpathcurveto{\pgfqpoint{9.967599in}{1.485817in}}{\pgfqpoint{9.965595in}{1.490654in}}{\pgfqpoint{9.962029in}{1.494221in}}%
\pgfpathcurveto{\pgfqpoint{9.958462in}{1.497787in}}{\pgfqpoint{9.953625in}{1.499791in}}{\pgfqpoint{9.948581in}{1.499791in}}%
\pgfpathcurveto{\pgfqpoint{9.943537in}{1.499791in}}{\pgfqpoint{9.938700in}{1.497787in}}{\pgfqpoint{9.935133in}{1.494221in}}%
\pgfpathcurveto{\pgfqpoint{9.931567in}{1.490654in}}{\pgfqpoint{9.929563in}{1.485817in}}{\pgfqpoint{9.929563in}{1.480773in}}%
\pgfpathcurveto{\pgfqpoint{9.929563in}{1.475729in}}{\pgfqpoint{9.931567in}{1.470892in}}{\pgfqpoint{9.935133in}{1.467325in}}%
\pgfpathcurveto{\pgfqpoint{9.938700in}{1.463759in}}{\pgfqpoint{9.943537in}{1.461755in}}{\pgfqpoint{9.948581in}{1.461755in}}%
\pgfpathclose%
\pgfusepath{fill}%
\end{pgfscope}%
\begin{pgfscope}%
\pgfpathrectangle{\pgfqpoint{6.572727in}{0.473000in}}{\pgfqpoint{4.227273in}{3.311000in}}%
\pgfusepath{clip}%
\pgfsetbuttcap%
\pgfsetroundjoin%
\definecolor{currentfill}{rgb}{0.127568,0.566949,0.550556}%
\pgfsetfillcolor{currentfill}%
\pgfsetfillopacity{0.700000}%
\pgfsetlinewidth{0.000000pt}%
\definecolor{currentstroke}{rgb}{0.000000,0.000000,0.000000}%
\pgfsetstrokecolor{currentstroke}%
\pgfsetstrokeopacity{0.700000}%
\pgfsetdash{}{0pt}%
\pgfpathmoveto{\pgfqpoint{8.444460in}{2.954266in}}%
\pgfpathcurveto{\pgfqpoint{8.449504in}{2.954266in}}{\pgfqpoint{8.454342in}{2.956270in}}{\pgfqpoint{8.457908in}{2.959836in}}%
\pgfpathcurveto{\pgfqpoint{8.461475in}{2.963403in}}{\pgfqpoint{8.463479in}{2.968241in}}{\pgfqpoint{8.463479in}{2.973284in}}%
\pgfpathcurveto{\pgfqpoint{8.463479in}{2.978328in}}{\pgfqpoint{8.461475in}{2.983166in}}{\pgfqpoint{8.457908in}{2.986732in}}%
\pgfpathcurveto{\pgfqpoint{8.454342in}{2.990299in}}{\pgfqpoint{8.449504in}{2.992302in}}{\pgfqpoint{8.444460in}{2.992302in}}%
\pgfpathcurveto{\pgfqpoint{8.439417in}{2.992302in}}{\pgfqpoint{8.434579in}{2.990299in}}{\pgfqpoint{8.431013in}{2.986732in}}%
\pgfpathcurveto{\pgfqpoint{8.427446in}{2.983166in}}{\pgfqpoint{8.425442in}{2.978328in}}{\pgfqpoint{8.425442in}{2.973284in}}%
\pgfpathcurveto{\pgfqpoint{8.425442in}{2.968241in}}{\pgfqpoint{8.427446in}{2.963403in}}{\pgfqpoint{8.431013in}{2.959836in}}%
\pgfpathcurveto{\pgfqpoint{8.434579in}{2.956270in}}{\pgfqpoint{8.439417in}{2.954266in}}{\pgfqpoint{8.444460in}{2.954266in}}%
\pgfpathclose%
\pgfusepath{fill}%
\end{pgfscope}%
\begin{pgfscope}%
\pgfpathrectangle{\pgfqpoint{6.572727in}{0.473000in}}{\pgfqpoint{4.227273in}{3.311000in}}%
\pgfusepath{clip}%
\pgfsetbuttcap%
\pgfsetroundjoin%
\definecolor{currentfill}{rgb}{0.127568,0.566949,0.550556}%
\pgfsetfillcolor{currentfill}%
\pgfsetfillopacity{0.700000}%
\pgfsetlinewidth{0.000000pt}%
\definecolor{currentstroke}{rgb}{0.000000,0.000000,0.000000}%
\pgfsetstrokecolor{currentstroke}%
\pgfsetstrokeopacity{0.700000}%
\pgfsetdash{}{0pt}%
\pgfpathmoveto{\pgfqpoint{7.196845in}{1.255987in}}%
\pgfpathcurveto{\pgfqpoint{7.201889in}{1.255987in}}{\pgfqpoint{7.206727in}{1.257991in}}{\pgfqpoint{7.210293in}{1.261557in}}%
\pgfpathcurveto{\pgfqpoint{7.213860in}{1.265123in}}{\pgfqpoint{7.215864in}{1.269961in}}{\pgfqpoint{7.215864in}{1.275005in}}%
\pgfpathcurveto{\pgfqpoint{7.215864in}{1.280049in}}{\pgfqpoint{7.213860in}{1.284886in}}{\pgfqpoint{7.210293in}{1.288453in}}%
\pgfpathcurveto{\pgfqpoint{7.206727in}{1.292019in}}{\pgfqpoint{7.201889in}{1.294023in}}{\pgfqpoint{7.196845in}{1.294023in}}%
\pgfpathcurveto{\pgfqpoint{7.191802in}{1.294023in}}{\pgfqpoint{7.186964in}{1.292019in}}{\pgfqpoint{7.183398in}{1.288453in}}%
\pgfpathcurveto{\pgfqpoint{7.179831in}{1.284886in}}{\pgfqpoint{7.177827in}{1.280049in}}{\pgfqpoint{7.177827in}{1.275005in}}%
\pgfpathcurveto{\pgfqpoint{7.177827in}{1.269961in}}{\pgfqpoint{7.179831in}{1.265123in}}{\pgfqpoint{7.183398in}{1.261557in}}%
\pgfpathcurveto{\pgfqpoint{7.186964in}{1.257991in}}{\pgfqpoint{7.191802in}{1.255987in}}{\pgfqpoint{7.196845in}{1.255987in}}%
\pgfpathclose%
\pgfusepath{fill}%
\end{pgfscope}%
\begin{pgfscope}%
\pgfpathrectangle{\pgfqpoint{6.572727in}{0.473000in}}{\pgfqpoint{4.227273in}{3.311000in}}%
\pgfusepath{clip}%
\pgfsetbuttcap%
\pgfsetroundjoin%
\definecolor{currentfill}{rgb}{0.127568,0.566949,0.550556}%
\pgfsetfillcolor{currentfill}%
\pgfsetfillopacity{0.700000}%
\pgfsetlinewidth{0.000000pt}%
\definecolor{currentstroke}{rgb}{0.000000,0.000000,0.000000}%
\pgfsetstrokecolor{currentstroke}%
\pgfsetstrokeopacity{0.700000}%
\pgfsetdash{}{0pt}%
\pgfpathmoveto{\pgfqpoint{8.261624in}{2.744758in}}%
\pgfpathcurveto{\pgfqpoint{8.266668in}{2.744758in}}{\pgfqpoint{8.271506in}{2.746762in}}{\pgfqpoint{8.275072in}{2.750329in}}%
\pgfpathcurveto{\pgfqpoint{8.278639in}{2.753895in}}{\pgfqpoint{8.280643in}{2.758733in}}{\pgfqpoint{8.280643in}{2.763777in}}%
\pgfpathcurveto{\pgfqpoint{8.280643in}{2.768820in}}{\pgfqpoint{8.278639in}{2.773658in}}{\pgfqpoint{8.275072in}{2.777224in}}%
\pgfpathcurveto{\pgfqpoint{8.271506in}{2.780791in}}{\pgfqpoint{8.266668in}{2.782795in}}{\pgfqpoint{8.261624in}{2.782795in}}%
\pgfpathcurveto{\pgfqpoint{8.256581in}{2.782795in}}{\pgfqpoint{8.251743in}{2.780791in}}{\pgfqpoint{8.248177in}{2.777224in}}%
\pgfpathcurveto{\pgfqpoint{8.244610in}{2.773658in}}{\pgfqpoint{8.242606in}{2.768820in}}{\pgfqpoint{8.242606in}{2.763777in}}%
\pgfpathcurveto{\pgfqpoint{8.242606in}{2.758733in}}{\pgfqpoint{8.244610in}{2.753895in}}{\pgfqpoint{8.248177in}{2.750329in}}%
\pgfpathcurveto{\pgfqpoint{8.251743in}{2.746762in}}{\pgfqpoint{8.256581in}{2.744758in}}{\pgfqpoint{8.261624in}{2.744758in}}%
\pgfpathclose%
\pgfusepath{fill}%
\end{pgfscope}%
\begin{pgfscope}%
\pgfpathrectangle{\pgfqpoint{6.572727in}{0.473000in}}{\pgfqpoint{4.227273in}{3.311000in}}%
\pgfusepath{clip}%
\pgfsetbuttcap%
\pgfsetroundjoin%
\definecolor{currentfill}{rgb}{0.993248,0.906157,0.143936}%
\pgfsetfillcolor{currentfill}%
\pgfsetfillopacity{0.700000}%
\pgfsetlinewidth{0.000000pt}%
\definecolor{currentstroke}{rgb}{0.000000,0.000000,0.000000}%
\pgfsetstrokecolor{currentstroke}%
\pgfsetstrokeopacity{0.700000}%
\pgfsetdash{}{0pt}%
\pgfpathmoveto{\pgfqpoint{9.804221in}{1.715732in}}%
\pgfpathcurveto{\pgfqpoint{9.809265in}{1.715732in}}{\pgfqpoint{9.814103in}{1.717736in}}{\pgfqpoint{9.817669in}{1.721302in}}%
\pgfpathcurveto{\pgfqpoint{9.821236in}{1.724869in}}{\pgfqpoint{9.823239in}{1.729706in}}{\pgfqpoint{9.823239in}{1.734750in}}%
\pgfpathcurveto{\pgfqpoint{9.823239in}{1.739794in}}{\pgfqpoint{9.821236in}{1.744631in}}{\pgfqpoint{9.817669in}{1.748198in}}%
\pgfpathcurveto{\pgfqpoint{9.814103in}{1.751764in}}{\pgfqpoint{9.809265in}{1.753768in}}{\pgfqpoint{9.804221in}{1.753768in}}%
\pgfpathcurveto{\pgfqpoint{9.799178in}{1.753768in}}{\pgfqpoint{9.794340in}{1.751764in}}{\pgfqpoint{9.790773in}{1.748198in}}%
\pgfpathcurveto{\pgfqpoint{9.787207in}{1.744631in}}{\pgfqpoint{9.785203in}{1.739794in}}{\pgfqpoint{9.785203in}{1.734750in}}%
\pgfpathcurveto{\pgfqpoint{9.785203in}{1.729706in}}{\pgfqpoint{9.787207in}{1.724869in}}{\pgfqpoint{9.790773in}{1.721302in}}%
\pgfpathcurveto{\pgfqpoint{9.794340in}{1.717736in}}{\pgfqpoint{9.799178in}{1.715732in}}{\pgfqpoint{9.804221in}{1.715732in}}%
\pgfpathclose%
\pgfusepath{fill}%
\end{pgfscope}%
\begin{pgfscope}%
\pgfpathrectangle{\pgfqpoint{6.572727in}{0.473000in}}{\pgfqpoint{4.227273in}{3.311000in}}%
\pgfusepath{clip}%
\pgfsetbuttcap%
\pgfsetroundjoin%
\definecolor{currentfill}{rgb}{0.993248,0.906157,0.143936}%
\pgfsetfillcolor{currentfill}%
\pgfsetfillopacity{0.700000}%
\pgfsetlinewidth{0.000000pt}%
\definecolor{currentstroke}{rgb}{0.000000,0.000000,0.000000}%
\pgfsetstrokecolor{currentstroke}%
\pgfsetstrokeopacity{0.700000}%
\pgfsetdash{}{0pt}%
\pgfpathmoveto{\pgfqpoint{9.471324in}{1.417819in}}%
\pgfpathcurveto{\pgfqpoint{9.476368in}{1.417819in}}{\pgfqpoint{9.481206in}{1.419822in}}{\pgfqpoint{9.484772in}{1.423389in}}%
\pgfpathcurveto{\pgfqpoint{9.488339in}{1.426955in}}{\pgfqpoint{9.490343in}{1.431793in}}{\pgfqpoint{9.490343in}{1.436837in}}%
\pgfpathcurveto{\pgfqpoint{9.490343in}{1.441880in}}{\pgfqpoint{9.488339in}{1.446718in}}{\pgfqpoint{9.484772in}{1.450285in}}%
\pgfpathcurveto{\pgfqpoint{9.481206in}{1.453851in}}{\pgfqpoint{9.476368in}{1.455855in}}{\pgfqpoint{9.471324in}{1.455855in}}%
\pgfpathcurveto{\pgfqpoint{9.466281in}{1.455855in}}{\pgfqpoint{9.461443in}{1.453851in}}{\pgfqpoint{9.457877in}{1.450285in}}%
\pgfpathcurveto{\pgfqpoint{9.454310in}{1.446718in}}{\pgfqpoint{9.452306in}{1.441880in}}{\pgfqpoint{9.452306in}{1.436837in}}%
\pgfpathcurveto{\pgfqpoint{9.452306in}{1.431793in}}{\pgfqpoint{9.454310in}{1.426955in}}{\pgfqpoint{9.457877in}{1.423389in}}%
\pgfpathcurveto{\pgfqpoint{9.461443in}{1.419822in}}{\pgfqpoint{9.466281in}{1.417819in}}{\pgfqpoint{9.471324in}{1.417819in}}%
\pgfpathclose%
\pgfusepath{fill}%
\end{pgfscope}%
\begin{pgfscope}%
\pgfpathrectangle{\pgfqpoint{6.572727in}{0.473000in}}{\pgfqpoint{4.227273in}{3.311000in}}%
\pgfusepath{clip}%
\pgfsetbuttcap%
\pgfsetroundjoin%
\definecolor{currentfill}{rgb}{0.993248,0.906157,0.143936}%
\pgfsetfillcolor{currentfill}%
\pgfsetfillopacity{0.700000}%
\pgfsetlinewidth{0.000000pt}%
\definecolor{currentstroke}{rgb}{0.000000,0.000000,0.000000}%
\pgfsetstrokecolor{currentstroke}%
\pgfsetstrokeopacity{0.700000}%
\pgfsetdash{}{0pt}%
\pgfpathmoveto{\pgfqpoint{9.078876in}{2.320816in}}%
\pgfpathcurveto{\pgfqpoint{9.083920in}{2.320816in}}{\pgfqpoint{9.088758in}{2.322819in}}{\pgfqpoint{9.092324in}{2.326386in}}%
\pgfpathcurveto{\pgfqpoint{9.095891in}{2.329952in}}{\pgfqpoint{9.097894in}{2.334790in}}{\pgfqpoint{9.097894in}{2.339834in}}%
\pgfpathcurveto{\pgfqpoint{9.097894in}{2.344877in}}{\pgfqpoint{9.095891in}{2.349715in}}{\pgfqpoint{9.092324in}{2.353282in}}%
\pgfpathcurveto{\pgfqpoint{9.088758in}{2.356848in}}{\pgfqpoint{9.083920in}{2.358852in}}{\pgfqpoint{9.078876in}{2.358852in}}%
\pgfpathcurveto{\pgfqpoint{9.073833in}{2.358852in}}{\pgfqpoint{9.068995in}{2.356848in}}{\pgfqpoint{9.065428in}{2.353282in}}%
\pgfpathcurveto{\pgfqpoint{9.061862in}{2.349715in}}{\pgfqpoint{9.059858in}{2.344877in}}{\pgfqpoint{9.059858in}{2.339834in}}%
\pgfpathcurveto{\pgfqpoint{9.059858in}{2.334790in}}{\pgfqpoint{9.061862in}{2.329952in}}{\pgfqpoint{9.065428in}{2.326386in}}%
\pgfpathcurveto{\pgfqpoint{9.068995in}{2.322819in}}{\pgfqpoint{9.073833in}{2.320816in}}{\pgfqpoint{9.078876in}{2.320816in}}%
\pgfpathclose%
\pgfusepath{fill}%
\end{pgfscope}%
\begin{pgfscope}%
\pgfpathrectangle{\pgfqpoint{6.572727in}{0.473000in}}{\pgfqpoint{4.227273in}{3.311000in}}%
\pgfusepath{clip}%
\pgfsetbuttcap%
\pgfsetroundjoin%
\definecolor{currentfill}{rgb}{0.127568,0.566949,0.550556}%
\pgfsetfillcolor{currentfill}%
\pgfsetfillopacity{0.700000}%
\pgfsetlinewidth{0.000000pt}%
\definecolor{currentstroke}{rgb}{0.000000,0.000000,0.000000}%
\pgfsetstrokecolor{currentstroke}%
\pgfsetstrokeopacity{0.700000}%
\pgfsetdash{}{0pt}%
\pgfpathmoveto{\pgfqpoint{8.182845in}{1.341568in}}%
\pgfpathcurveto{\pgfqpoint{8.187889in}{1.341568in}}{\pgfqpoint{8.192727in}{1.343572in}}{\pgfqpoint{8.196293in}{1.347138in}}%
\pgfpathcurveto{\pgfqpoint{8.199860in}{1.350704in}}{\pgfqpoint{8.201863in}{1.355542in}}{\pgfqpoint{8.201863in}{1.360586in}}%
\pgfpathcurveto{\pgfqpoint{8.201863in}{1.365630in}}{\pgfqpoint{8.199860in}{1.370467in}}{\pgfqpoint{8.196293in}{1.374034in}}%
\pgfpathcurveto{\pgfqpoint{8.192727in}{1.377600in}}{\pgfqpoint{8.187889in}{1.379604in}}{\pgfqpoint{8.182845in}{1.379604in}}%
\pgfpathcurveto{\pgfqpoint{8.177802in}{1.379604in}}{\pgfqpoint{8.172964in}{1.377600in}}{\pgfqpoint{8.169397in}{1.374034in}}%
\pgfpathcurveto{\pgfqpoint{8.165831in}{1.370467in}}{\pgfqpoint{8.163827in}{1.365630in}}{\pgfqpoint{8.163827in}{1.360586in}}%
\pgfpathcurveto{\pgfqpoint{8.163827in}{1.355542in}}{\pgfqpoint{8.165831in}{1.350704in}}{\pgfqpoint{8.169397in}{1.347138in}}%
\pgfpathcurveto{\pgfqpoint{8.172964in}{1.343572in}}{\pgfqpoint{8.177802in}{1.341568in}}{\pgfqpoint{8.182845in}{1.341568in}}%
\pgfpathclose%
\pgfusepath{fill}%
\end{pgfscope}%
\begin{pgfscope}%
\pgfpathrectangle{\pgfqpoint{6.572727in}{0.473000in}}{\pgfqpoint{4.227273in}{3.311000in}}%
\pgfusepath{clip}%
\pgfsetbuttcap%
\pgfsetroundjoin%
\definecolor{currentfill}{rgb}{0.127568,0.566949,0.550556}%
\pgfsetfillcolor{currentfill}%
\pgfsetfillopacity{0.700000}%
\pgfsetlinewidth{0.000000pt}%
\definecolor{currentstroke}{rgb}{0.000000,0.000000,0.000000}%
\pgfsetstrokecolor{currentstroke}%
\pgfsetstrokeopacity{0.700000}%
\pgfsetdash{}{0pt}%
\pgfpathmoveto{\pgfqpoint{7.817637in}{1.866861in}}%
\pgfpathcurveto{\pgfqpoint{7.822681in}{1.866861in}}{\pgfqpoint{7.827519in}{1.868865in}}{\pgfqpoint{7.831085in}{1.872431in}}%
\pgfpathcurveto{\pgfqpoint{7.834652in}{1.875998in}}{\pgfqpoint{7.836655in}{1.880836in}}{\pgfqpoint{7.836655in}{1.885879in}}%
\pgfpathcurveto{\pgfqpoint{7.836655in}{1.890923in}}{\pgfqpoint{7.834652in}{1.895761in}}{\pgfqpoint{7.831085in}{1.899327in}}%
\pgfpathcurveto{\pgfqpoint{7.827519in}{1.902894in}}{\pgfqpoint{7.822681in}{1.904897in}}{\pgfqpoint{7.817637in}{1.904897in}}%
\pgfpathcurveto{\pgfqpoint{7.812594in}{1.904897in}}{\pgfqpoint{7.807756in}{1.902894in}}{\pgfqpoint{7.804189in}{1.899327in}}%
\pgfpathcurveto{\pgfqpoint{7.800623in}{1.895761in}}{\pgfqpoint{7.798619in}{1.890923in}}{\pgfqpoint{7.798619in}{1.885879in}}%
\pgfpathcurveto{\pgfqpoint{7.798619in}{1.880836in}}{\pgfqpoint{7.800623in}{1.875998in}}{\pgfqpoint{7.804189in}{1.872431in}}%
\pgfpathcurveto{\pgfqpoint{7.807756in}{1.868865in}}{\pgfqpoint{7.812594in}{1.866861in}}{\pgfqpoint{7.817637in}{1.866861in}}%
\pgfpathclose%
\pgfusepath{fill}%
\end{pgfscope}%
\begin{pgfscope}%
\pgfpathrectangle{\pgfqpoint{6.572727in}{0.473000in}}{\pgfqpoint{4.227273in}{3.311000in}}%
\pgfusepath{clip}%
\pgfsetbuttcap%
\pgfsetroundjoin%
\definecolor{currentfill}{rgb}{0.127568,0.566949,0.550556}%
\pgfsetfillcolor{currentfill}%
\pgfsetfillopacity{0.700000}%
\pgfsetlinewidth{0.000000pt}%
\definecolor{currentstroke}{rgb}{0.000000,0.000000,0.000000}%
\pgfsetstrokecolor{currentstroke}%
\pgfsetstrokeopacity{0.700000}%
\pgfsetdash{}{0pt}%
\pgfpathmoveto{\pgfqpoint{7.984190in}{0.944550in}}%
\pgfpathcurveto{\pgfqpoint{7.989234in}{0.944550in}}{\pgfqpoint{7.994072in}{0.946554in}}{\pgfqpoint{7.997638in}{0.950120in}}%
\pgfpathcurveto{\pgfqpoint{8.001204in}{0.953687in}}{\pgfqpoint{8.003208in}{0.958524in}}{\pgfqpoint{8.003208in}{0.963568in}}%
\pgfpathcurveto{\pgfqpoint{8.003208in}{0.968612in}}{\pgfqpoint{8.001204in}{0.973450in}}{\pgfqpoint{7.997638in}{0.977016in}}%
\pgfpathcurveto{\pgfqpoint{7.994072in}{0.980582in}}{\pgfqpoint{7.989234in}{0.982586in}}{\pgfqpoint{7.984190in}{0.982586in}}%
\pgfpathcurveto{\pgfqpoint{7.979146in}{0.982586in}}{\pgfqpoint{7.974309in}{0.980582in}}{\pgfqpoint{7.970742in}{0.977016in}}%
\pgfpathcurveto{\pgfqpoint{7.967176in}{0.973450in}}{\pgfqpoint{7.965172in}{0.968612in}}{\pgfqpoint{7.965172in}{0.963568in}}%
\pgfpathcurveto{\pgfqpoint{7.965172in}{0.958524in}}{\pgfqpoint{7.967176in}{0.953687in}}{\pgfqpoint{7.970742in}{0.950120in}}%
\pgfpathcurveto{\pgfqpoint{7.974309in}{0.946554in}}{\pgfqpoint{7.979146in}{0.944550in}}{\pgfqpoint{7.984190in}{0.944550in}}%
\pgfpathclose%
\pgfusepath{fill}%
\end{pgfscope}%
\begin{pgfscope}%
\pgfpathrectangle{\pgfqpoint{6.572727in}{0.473000in}}{\pgfqpoint{4.227273in}{3.311000in}}%
\pgfusepath{clip}%
\pgfsetbuttcap%
\pgfsetroundjoin%
\definecolor{currentfill}{rgb}{0.127568,0.566949,0.550556}%
\pgfsetfillcolor{currentfill}%
\pgfsetfillopacity{0.700000}%
\pgfsetlinewidth{0.000000pt}%
\definecolor{currentstroke}{rgb}{0.000000,0.000000,0.000000}%
\pgfsetstrokecolor{currentstroke}%
\pgfsetstrokeopacity{0.700000}%
\pgfsetdash{}{0pt}%
\pgfpathmoveto{\pgfqpoint{8.648901in}{3.026604in}}%
\pgfpathcurveto{\pgfqpoint{8.653945in}{3.026604in}}{\pgfqpoint{8.658783in}{3.028608in}}{\pgfqpoint{8.662349in}{3.032174in}}%
\pgfpathcurveto{\pgfqpoint{8.665916in}{3.035741in}}{\pgfqpoint{8.667920in}{3.040578in}}{\pgfqpoint{8.667920in}{3.045622in}}%
\pgfpathcurveto{\pgfqpoint{8.667920in}{3.050666in}}{\pgfqpoint{8.665916in}{3.055504in}}{\pgfqpoint{8.662349in}{3.059070in}}%
\pgfpathcurveto{\pgfqpoint{8.658783in}{3.062636in}}{\pgfqpoint{8.653945in}{3.064640in}}{\pgfqpoint{8.648901in}{3.064640in}}%
\pgfpathcurveto{\pgfqpoint{8.643858in}{3.064640in}}{\pgfqpoint{8.639020in}{3.062636in}}{\pgfqpoint{8.635454in}{3.059070in}}%
\pgfpathcurveto{\pgfqpoint{8.631887in}{3.055504in}}{\pgfqpoint{8.629883in}{3.050666in}}{\pgfqpoint{8.629883in}{3.045622in}}%
\pgfpathcurveto{\pgfqpoint{8.629883in}{3.040578in}}{\pgfqpoint{8.631887in}{3.035741in}}{\pgfqpoint{8.635454in}{3.032174in}}%
\pgfpathcurveto{\pgfqpoint{8.639020in}{3.028608in}}{\pgfqpoint{8.643858in}{3.026604in}}{\pgfqpoint{8.648901in}{3.026604in}}%
\pgfpathclose%
\pgfusepath{fill}%
\end{pgfscope}%
\begin{pgfscope}%
\pgfpathrectangle{\pgfqpoint{6.572727in}{0.473000in}}{\pgfqpoint{4.227273in}{3.311000in}}%
\pgfusepath{clip}%
\pgfsetbuttcap%
\pgfsetroundjoin%
\definecolor{currentfill}{rgb}{0.127568,0.566949,0.550556}%
\pgfsetfillcolor{currentfill}%
\pgfsetfillopacity{0.700000}%
\pgfsetlinewidth{0.000000pt}%
\definecolor{currentstroke}{rgb}{0.000000,0.000000,0.000000}%
\pgfsetstrokecolor{currentstroke}%
\pgfsetstrokeopacity{0.700000}%
\pgfsetdash{}{0pt}%
\pgfpathmoveto{\pgfqpoint{7.866339in}{2.690306in}}%
\pgfpathcurveto{\pgfqpoint{7.871383in}{2.690306in}}{\pgfqpoint{7.876221in}{2.692310in}}{\pgfqpoint{7.879787in}{2.695877in}}%
\pgfpathcurveto{\pgfqpoint{7.883354in}{2.699443in}}{\pgfqpoint{7.885357in}{2.704281in}}{\pgfqpoint{7.885357in}{2.709324in}}%
\pgfpathcurveto{\pgfqpoint{7.885357in}{2.714368in}}{\pgfqpoint{7.883354in}{2.719206in}}{\pgfqpoint{7.879787in}{2.722772in}}%
\pgfpathcurveto{\pgfqpoint{7.876221in}{2.726339in}}{\pgfqpoint{7.871383in}{2.728343in}}{\pgfqpoint{7.866339in}{2.728343in}}%
\pgfpathcurveto{\pgfqpoint{7.861296in}{2.728343in}}{\pgfqpoint{7.856458in}{2.726339in}}{\pgfqpoint{7.852891in}{2.722772in}}%
\pgfpathcurveto{\pgfqpoint{7.849325in}{2.719206in}}{\pgfqpoint{7.847321in}{2.714368in}}{\pgfqpoint{7.847321in}{2.709324in}}%
\pgfpathcurveto{\pgfqpoint{7.847321in}{2.704281in}}{\pgfqpoint{7.849325in}{2.699443in}}{\pgfqpoint{7.852891in}{2.695877in}}%
\pgfpathcurveto{\pgfqpoint{7.856458in}{2.692310in}}{\pgfqpoint{7.861296in}{2.690306in}}{\pgfqpoint{7.866339in}{2.690306in}}%
\pgfpathclose%
\pgfusepath{fill}%
\end{pgfscope}%
\begin{pgfscope}%
\pgfpathrectangle{\pgfqpoint{6.572727in}{0.473000in}}{\pgfqpoint{4.227273in}{3.311000in}}%
\pgfusepath{clip}%
\pgfsetbuttcap%
\pgfsetroundjoin%
\definecolor{currentfill}{rgb}{0.127568,0.566949,0.550556}%
\pgfsetfillcolor{currentfill}%
\pgfsetfillopacity{0.700000}%
\pgfsetlinewidth{0.000000pt}%
\definecolor{currentstroke}{rgb}{0.000000,0.000000,0.000000}%
\pgfsetstrokecolor{currentstroke}%
\pgfsetstrokeopacity{0.700000}%
\pgfsetdash{}{0pt}%
\pgfpathmoveto{\pgfqpoint{8.411086in}{2.840134in}}%
\pgfpathcurveto{\pgfqpoint{8.416129in}{2.840134in}}{\pgfqpoint{8.420967in}{2.842137in}}{\pgfqpoint{8.424534in}{2.845704in}}%
\pgfpathcurveto{\pgfqpoint{8.428100in}{2.849270in}}{\pgfqpoint{8.430104in}{2.854108in}}{\pgfqpoint{8.430104in}{2.859152in}}%
\pgfpathcurveto{\pgfqpoint{8.430104in}{2.864195in}}{\pgfqpoint{8.428100in}{2.869033in}}{\pgfqpoint{8.424534in}{2.872600in}}%
\pgfpathcurveto{\pgfqpoint{8.420967in}{2.876166in}}{\pgfqpoint{8.416129in}{2.878170in}}{\pgfqpoint{8.411086in}{2.878170in}}%
\pgfpathcurveto{\pgfqpoint{8.406042in}{2.878170in}}{\pgfqpoint{8.401204in}{2.876166in}}{\pgfqpoint{8.397638in}{2.872600in}}%
\pgfpathcurveto{\pgfqpoint{8.394071in}{2.869033in}}{\pgfqpoint{8.392068in}{2.864195in}}{\pgfqpoint{8.392068in}{2.859152in}}%
\pgfpathcurveto{\pgfqpoint{8.392068in}{2.854108in}}{\pgfqpoint{8.394071in}{2.849270in}}{\pgfqpoint{8.397638in}{2.845704in}}%
\pgfpathcurveto{\pgfqpoint{8.401204in}{2.842137in}}{\pgfqpoint{8.406042in}{2.840134in}}{\pgfqpoint{8.411086in}{2.840134in}}%
\pgfpathclose%
\pgfusepath{fill}%
\end{pgfscope}%
\begin{pgfscope}%
\pgfpathrectangle{\pgfqpoint{6.572727in}{0.473000in}}{\pgfqpoint{4.227273in}{3.311000in}}%
\pgfusepath{clip}%
\pgfsetbuttcap%
\pgfsetroundjoin%
\definecolor{currentfill}{rgb}{0.127568,0.566949,0.550556}%
\pgfsetfillcolor{currentfill}%
\pgfsetfillopacity{0.700000}%
\pgfsetlinewidth{0.000000pt}%
\definecolor{currentstroke}{rgb}{0.000000,0.000000,0.000000}%
\pgfsetstrokecolor{currentstroke}%
\pgfsetstrokeopacity{0.700000}%
\pgfsetdash{}{0pt}%
\pgfpathmoveto{\pgfqpoint{7.301251in}{1.733795in}}%
\pgfpathcurveto{\pgfqpoint{7.306294in}{1.733795in}}{\pgfqpoint{7.311132in}{1.735799in}}{\pgfqpoint{7.314698in}{1.739366in}}%
\pgfpathcurveto{\pgfqpoint{7.318265in}{1.742932in}}{\pgfqpoint{7.320269in}{1.747770in}}{\pgfqpoint{7.320269in}{1.752814in}}%
\pgfpathcurveto{\pgfqpoint{7.320269in}{1.757857in}}{\pgfqpoint{7.318265in}{1.762695in}}{\pgfqpoint{7.314698in}{1.766261in}}%
\pgfpathcurveto{\pgfqpoint{7.311132in}{1.769828in}}{\pgfqpoint{7.306294in}{1.771832in}}{\pgfqpoint{7.301251in}{1.771832in}}%
\pgfpathcurveto{\pgfqpoint{7.296207in}{1.771832in}}{\pgfqpoint{7.291369in}{1.769828in}}{\pgfqpoint{7.287803in}{1.766261in}}%
\pgfpathcurveto{\pgfqpoint{7.284236in}{1.762695in}}{\pgfqpoint{7.282232in}{1.757857in}}{\pgfqpoint{7.282232in}{1.752814in}}%
\pgfpathcurveto{\pgfqpoint{7.282232in}{1.747770in}}{\pgfqpoint{7.284236in}{1.742932in}}{\pgfqpoint{7.287803in}{1.739366in}}%
\pgfpathcurveto{\pgfqpoint{7.291369in}{1.735799in}}{\pgfqpoint{7.296207in}{1.733795in}}{\pgfqpoint{7.301251in}{1.733795in}}%
\pgfpathclose%
\pgfusepath{fill}%
\end{pgfscope}%
\begin{pgfscope}%
\pgfpathrectangle{\pgfqpoint{6.572727in}{0.473000in}}{\pgfqpoint{4.227273in}{3.311000in}}%
\pgfusepath{clip}%
\pgfsetbuttcap%
\pgfsetroundjoin%
\definecolor{currentfill}{rgb}{0.993248,0.906157,0.143936}%
\pgfsetfillcolor{currentfill}%
\pgfsetfillopacity{0.700000}%
\pgfsetlinewidth{0.000000pt}%
\definecolor{currentstroke}{rgb}{0.000000,0.000000,0.000000}%
\pgfsetstrokecolor{currentstroke}%
\pgfsetstrokeopacity{0.700000}%
\pgfsetdash{}{0pt}%
\pgfpathmoveto{\pgfqpoint{9.701916in}{1.216113in}}%
\pgfpathcurveto{\pgfqpoint{9.706960in}{1.216113in}}{\pgfqpoint{9.711798in}{1.218117in}}{\pgfqpoint{9.715364in}{1.221683in}}%
\pgfpathcurveto{\pgfqpoint{9.718931in}{1.225250in}}{\pgfqpoint{9.720935in}{1.230087in}}{\pgfqpoint{9.720935in}{1.235131in}}%
\pgfpathcurveto{\pgfqpoint{9.720935in}{1.240175in}}{\pgfqpoint{9.718931in}{1.245012in}}{\pgfqpoint{9.715364in}{1.248579in}}%
\pgfpathcurveto{\pgfqpoint{9.711798in}{1.252145in}}{\pgfqpoint{9.706960in}{1.254149in}}{\pgfqpoint{9.701916in}{1.254149in}}%
\pgfpathcurveto{\pgfqpoint{9.696873in}{1.254149in}}{\pgfqpoint{9.692035in}{1.252145in}}{\pgfqpoint{9.688469in}{1.248579in}}%
\pgfpathcurveto{\pgfqpoint{9.684902in}{1.245012in}}{\pgfqpoint{9.682898in}{1.240175in}}{\pgfqpoint{9.682898in}{1.235131in}}%
\pgfpathcurveto{\pgfqpoint{9.682898in}{1.230087in}}{\pgfqpoint{9.684902in}{1.225250in}}{\pgfqpoint{9.688469in}{1.221683in}}%
\pgfpathcurveto{\pgfqpoint{9.692035in}{1.218117in}}{\pgfqpoint{9.696873in}{1.216113in}}{\pgfqpoint{9.701916in}{1.216113in}}%
\pgfpathclose%
\pgfusepath{fill}%
\end{pgfscope}%
\begin{pgfscope}%
\pgfpathrectangle{\pgfqpoint{6.572727in}{0.473000in}}{\pgfqpoint{4.227273in}{3.311000in}}%
\pgfusepath{clip}%
\pgfsetbuttcap%
\pgfsetroundjoin%
\definecolor{currentfill}{rgb}{0.993248,0.906157,0.143936}%
\pgfsetfillcolor{currentfill}%
\pgfsetfillopacity{0.700000}%
\pgfsetlinewidth{0.000000pt}%
\definecolor{currentstroke}{rgb}{0.000000,0.000000,0.000000}%
\pgfsetstrokecolor{currentstroke}%
\pgfsetstrokeopacity{0.700000}%
\pgfsetdash{}{0pt}%
\pgfpathmoveto{\pgfqpoint{9.674932in}{2.166184in}}%
\pgfpathcurveto{\pgfqpoint{9.679975in}{2.166184in}}{\pgfqpoint{9.684813in}{2.168188in}}{\pgfqpoint{9.688379in}{2.171755in}}%
\pgfpathcurveto{\pgfqpoint{9.691946in}{2.175321in}}{\pgfqpoint{9.693950in}{2.180159in}}{\pgfqpoint{9.693950in}{2.185202in}}%
\pgfpathcurveto{\pgfqpoint{9.693950in}{2.190246in}}{\pgfqpoint{9.691946in}{2.195084in}}{\pgfqpoint{9.688379in}{2.198650in}}%
\pgfpathcurveto{\pgfqpoint{9.684813in}{2.202217in}}{\pgfqpoint{9.679975in}{2.204221in}}{\pgfqpoint{9.674932in}{2.204221in}}%
\pgfpathcurveto{\pgfqpoint{9.669888in}{2.204221in}}{\pgfqpoint{9.665050in}{2.202217in}}{\pgfqpoint{9.661484in}{2.198650in}}%
\pgfpathcurveto{\pgfqpoint{9.657917in}{2.195084in}}{\pgfqpoint{9.655913in}{2.190246in}}{\pgfqpoint{9.655913in}{2.185202in}}%
\pgfpathcurveto{\pgfqpoint{9.655913in}{2.180159in}}{\pgfqpoint{9.657917in}{2.175321in}}{\pgfqpoint{9.661484in}{2.171755in}}%
\pgfpathcurveto{\pgfqpoint{9.665050in}{2.168188in}}{\pgfqpoint{9.669888in}{2.166184in}}{\pgfqpoint{9.674932in}{2.166184in}}%
\pgfpathclose%
\pgfusepath{fill}%
\end{pgfscope}%
\begin{pgfscope}%
\pgfpathrectangle{\pgfqpoint{6.572727in}{0.473000in}}{\pgfqpoint{4.227273in}{3.311000in}}%
\pgfusepath{clip}%
\pgfsetbuttcap%
\pgfsetroundjoin%
\definecolor{currentfill}{rgb}{0.127568,0.566949,0.550556}%
\pgfsetfillcolor{currentfill}%
\pgfsetfillopacity{0.700000}%
\pgfsetlinewidth{0.000000pt}%
\definecolor{currentstroke}{rgb}{0.000000,0.000000,0.000000}%
\pgfsetstrokecolor{currentstroke}%
\pgfsetstrokeopacity{0.700000}%
\pgfsetdash{}{0pt}%
\pgfpathmoveto{\pgfqpoint{8.138498in}{1.997444in}}%
\pgfpathcurveto{\pgfqpoint{8.143542in}{1.997444in}}{\pgfqpoint{8.148380in}{1.999447in}}{\pgfqpoint{8.151946in}{2.003014in}}%
\pgfpathcurveto{\pgfqpoint{8.155512in}{2.006580in}}{\pgfqpoint{8.157516in}{2.011418in}}{\pgfqpoint{8.157516in}{2.016462in}}%
\pgfpathcurveto{\pgfqpoint{8.157516in}{2.021505in}}{\pgfqpoint{8.155512in}{2.026343in}}{\pgfqpoint{8.151946in}{2.029910in}}%
\pgfpathcurveto{\pgfqpoint{8.148380in}{2.033476in}}{\pgfqpoint{8.143542in}{2.035480in}}{\pgfqpoint{8.138498in}{2.035480in}}%
\pgfpathcurveto{\pgfqpoint{8.133454in}{2.035480in}}{\pgfqpoint{8.128617in}{2.033476in}}{\pgfqpoint{8.125050in}{2.029910in}}%
\pgfpathcurveto{\pgfqpoint{8.121484in}{2.026343in}}{\pgfqpoint{8.119480in}{2.021505in}}{\pgfqpoint{8.119480in}{2.016462in}}%
\pgfpathcurveto{\pgfqpoint{8.119480in}{2.011418in}}{\pgfqpoint{8.121484in}{2.006580in}}{\pgfqpoint{8.125050in}{2.003014in}}%
\pgfpathcurveto{\pgfqpoint{8.128617in}{1.999447in}}{\pgfqpoint{8.133454in}{1.997444in}}{\pgfqpoint{8.138498in}{1.997444in}}%
\pgfpathclose%
\pgfusepath{fill}%
\end{pgfscope}%
\begin{pgfscope}%
\pgfpathrectangle{\pgfqpoint{6.572727in}{0.473000in}}{\pgfqpoint{4.227273in}{3.311000in}}%
\pgfusepath{clip}%
\pgfsetbuttcap%
\pgfsetroundjoin%
\definecolor{currentfill}{rgb}{0.993248,0.906157,0.143936}%
\pgfsetfillcolor{currentfill}%
\pgfsetfillopacity{0.700000}%
\pgfsetlinewidth{0.000000pt}%
\definecolor{currentstroke}{rgb}{0.000000,0.000000,0.000000}%
\pgfsetstrokecolor{currentstroke}%
\pgfsetstrokeopacity{0.700000}%
\pgfsetdash{}{0pt}%
\pgfpathmoveto{\pgfqpoint{9.293637in}{1.736244in}}%
\pgfpathcurveto{\pgfqpoint{9.298680in}{1.736244in}}{\pgfqpoint{9.303518in}{1.738248in}}{\pgfqpoint{9.307084in}{1.741815in}}%
\pgfpathcurveto{\pgfqpoint{9.310651in}{1.745381in}}{\pgfqpoint{9.312655in}{1.750219in}}{\pgfqpoint{9.312655in}{1.755263in}}%
\pgfpathcurveto{\pgfqpoint{9.312655in}{1.760306in}}{\pgfqpoint{9.310651in}{1.765144in}}{\pgfqpoint{9.307084in}{1.768710in}}%
\pgfpathcurveto{\pgfqpoint{9.303518in}{1.772277in}}{\pgfqpoint{9.298680in}{1.774281in}}{\pgfqpoint{9.293637in}{1.774281in}}%
\pgfpathcurveto{\pgfqpoint{9.288593in}{1.774281in}}{\pgfqpoint{9.283755in}{1.772277in}}{\pgfqpoint{9.280189in}{1.768710in}}%
\pgfpathcurveto{\pgfqpoint{9.276622in}{1.765144in}}{\pgfqpoint{9.274618in}{1.760306in}}{\pgfqpoint{9.274618in}{1.755263in}}%
\pgfpathcurveto{\pgfqpoint{9.274618in}{1.750219in}}{\pgfqpoint{9.276622in}{1.745381in}}{\pgfqpoint{9.280189in}{1.741815in}}%
\pgfpathcurveto{\pgfqpoint{9.283755in}{1.738248in}}{\pgfqpoint{9.288593in}{1.736244in}}{\pgfqpoint{9.293637in}{1.736244in}}%
\pgfpathclose%
\pgfusepath{fill}%
\end{pgfscope}%
\begin{pgfscope}%
\pgfpathrectangle{\pgfqpoint{6.572727in}{0.473000in}}{\pgfqpoint{4.227273in}{3.311000in}}%
\pgfusepath{clip}%
\pgfsetbuttcap%
\pgfsetroundjoin%
\definecolor{currentfill}{rgb}{0.993248,0.906157,0.143936}%
\pgfsetfillcolor{currentfill}%
\pgfsetfillopacity{0.700000}%
\pgfsetlinewidth{0.000000pt}%
\definecolor{currentstroke}{rgb}{0.000000,0.000000,0.000000}%
\pgfsetstrokecolor{currentstroke}%
\pgfsetstrokeopacity{0.700000}%
\pgfsetdash{}{0pt}%
\pgfpathmoveto{\pgfqpoint{9.159888in}{1.562373in}}%
\pgfpathcurveto{\pgfqpoint{9.164932in}{1.562373in}}{\pgfqpoint{9.169769in}{1.564377in}}{\pgfqpoint{9.173336in}{1.567944in}}%
\pgfpathcurveto{\pgfqpoint{9.176902in}{1.571510in}}{\pgfqpoint{9.178906in}{1.576348in}}{\pgfqpoint{9.178906in}{1.581391in}}%
\pgfpathcurveto{\pgfqpoint{9.178906in}{1.586435in}}{\pgfqpoint{9.176902in}{1.591273in}}{\pgfqpoint{9.173336in}{1.594839in}}%
\pgfpathcurveto{\pgfqpoint{9.169769in}{1.598406in}}{\pgfqpoint{9.164932in}{1.600410in}}{\pgfqpoint{9.159888in}{1.600410in}}%
\pgfpathcurveto{\pgfqpoint{9.154844in}{1.600410in}}{\pgfqpoint{9.150006in}{1.598406in}}{\pgfqpoint{9.146440in}{1.594839in}}%
\pgfpathcurveto{\pgfqpoint{9.142874in}{1.591273in}}{\pgfqpoint{9.140870in}{1.586435in}}{\pgfqpoint{9.140870in}{1.581391in}}%
\pgfpathcurveto{\pgfqpoint{9.140870in}{1.576348in}}{\pgfqpoint{9.142874in}{1.571510in}}{\pgfqpoint{9.146440in}{1.567944in}}%
\pgfpathcurveto{\pgfqpoint{9.150006in}{1.564377in}}{\pgfqpoint{9.154844in}{1.562373in}}{\pgfqpoint{9.159888in}{1.562373in}}%
\pgfpathclose%
\pgfusepath{fill}%
\end{pgfscope}%
\begin{pgfscope}%
\pgfpathrectangle{\pgfqpoint{6.572727in}{0.473000in}}{\pgfqpoint{4.227273in}{3.311000in}}%
\pgfusepath{clip}%
\pgfsetbuttcap%
\pgfsetroundjoin%
\definecolor{currentfill}{rgb}{0.993248,0.906157,0.143936}%
\pgfsetfillcolor{currentfill}%
\pgfsetfillopacity{0.700000}%
\pgfsetlinewidth{0.000000pt}%
\definecolor{currentstroke}{rgb}{0.000000,0.000000,0.000000}%
\pgfsetstrokecolor{currentstroke}%
\pgfsetstrokeopacity{0.700000}%
\pgfsetdash{}{0pt}%
\pgfpathmoveto{\pgfqpoint{9.820642in}{1.505142in}}%
\pgfpathcurveto{\pgfqpoint{9.825686in}{1.505142in}}{\pgfqpoint{9.830523in}{1.507146in}}{\pgfqpoint{9.834090in}{1.510712in}}%
\pgfpathcurveto{\pgfqpoint{9.837656in}{1.514279in}}{\pgfqpoint{9.839660in}{1.519117in}}{\pgfqpoint{9.839660in}{1.524160in}}%
\pgfpathcurveto{\pgfqpoint{9.839660in}{1.529204in}}{\pgfqpoint{9.837656in}{1.534042in}}{\pgfqpoint{9.834090in}{1.537608in}}%
\pgfpathcurveto{\pgfqpoint{9.830523in}{1.541175in}}{\pgfqpoint{9.825686in}{1.543178in}}{\pgfqpoint{9.820642in}{1.543178in}}%
\pgfpathcurveto{\pgfqpoint{9.815598in}{1.543178in}}{\pgfqpoint{9.810760in}{1.541175in}}{\pgfqpoint{9.807194in}{1.537608in}}%
\pgfpathcurveto{\pgfqpoint{9.803628in}{1.534042in}}{\pgfqpoint{9.801624in}{1.529204in}}{\pgfqpoint{9.801624in}{1.524160in}}%
\pgfpathcurveto{\pgfqpoint{9.801624in}{1.519117in}}{\pgfqpoint{9.803628in}{1.514279in}}{\pgfqpoint{9.807194in}{1.510712in}}%
\pgfpathcurveto{\pgfqpoint{9.810760in}{1.507146in}}{\pgfqpoint{9.815598in}{1.505142in}}{\pgfqpoint{9.820642in}{1.505142in}}%
\pgfpathclose%
\pgfusepath{fill}%
\end{pgfscope}%
\begin{pgfscope}%
\pgfpathrectangle{\pgfqpoint{6.572727in}{0.473000in}}{\pgfqpoint{4.227273in}{3.311000in}}%
\pgfusepath{clip}%
\pgfsetbuttcap%
\pgfsetroundjoin%
\definecolor{currentfill}{rgb}{0.127568,0.566949,0.550556}%
\pgfsetfillcolor{currentfill}%
\pgfsetfillopacity{0.700000}%
\pgfsetlinewidth{0.000000pt}%
\definecolor{currentstroke}{rgb}{0.000000,0.000000,0.000000}%
\pgfsetstrokecolor{currentstroke}%
\pgfsetstrokeopacity{0.700000}%
\pgfsetdash{}{0pt}%
\pgfpathmoveto{\pgfqpoint{7.495660in}{1.919016in}}%
\pgfpathcurveto{\pgfqpoint{7.500704in}{1.919016in}}{\pgfqpoint{7.505541in}{1.921019in}}{\pgfqpoint{7.509108in}{1.924586in}}%
\pgfpathcurveto{\pgfqpoint{7.512674in}{1.928152in}}{\pgfqpoint{7.514678in}{1.932990in}}{\pgfqpoint{7.514678in}{1.938034in}}%
\pgfpathcurveto{\pgfqpoint{7.514678in}{1.943077in}}{\pgfqpoint{7.512674in}{1.947915in}}{\pgfqpoint{7.509108in}{1.951482in}}%
\pgfpathcurveto{\pgfqpoint{7.505541in}{1.955048in}}{\pgfqpoint{7.500704in}{1.957052in}}{\pgfqpoint{7.495660in}{1.957052in}}%
\pgfpathcurveto{\pgfqpoint{7.490616in}{1.957052in}}{\pgfqpoint{7.485778in}{1.955048in}}{\pgfqpoint{7.482212in}{1.951482in}}%
\pgfpathcurveto{\pgfqpoint{7.478646in}{1.947915in}}{\pgfqpoint{7.476642in}{1.943077in}}{\pgfqpoint{7.476642in}{1.938034in}}%
\pgfpathcurveto{\pgfqpoint{7.476642in}{1.932990in}}{\pgfqpoint{7.478646in}{1.928152in}}{\pgfqpoint{7.482212in}{1.924586in}}%
\pgfpathcurveto{\pgfqpoint{7.485778in}{1.921019in}}{\pgfqpoint{7.490616in}{1.919016in}}{\pgfqpoint{7.495660in}{1.919016in}}%
\pgfpathclose%
\pgfusepath{fill}%
\end{pgfscope}%
\begin{pgfscope}%
\pgfpathrectangle{\pgfqpoint{6.572727in}{0.473000in}}{\pgfqpoint{4.227273in}{3.311000in}}%
\pgfusepath{clip}%
\pgfsetbuttcap%
\pgfsetroundjoin%
\definecolor{currentfill}{rgb}{0.127568,0.566949,0.550556}%
\pgfsetfillcolor{currentfill}%
\pgfsetfillopacity{0.700000}%
\pgfsetlinewidth{0.000000pt}%
\definecolor{currentstroke}{rgb}{0.000000,0.000000,0.000000}%
\pgfsetstrokecolor{currentstroke}%
\pgfsetstrokeopacity{0.700000}%
\pgfsetdash{}{0pt}%
\pgfpathmoveto{\pgfqpoint{8.061798in}{3.452918in}}%
\pgfpathcurveto{\pgfqpoint{8.066841in}{3.452918in}}{\pgfqpoint{8.071679in}{3.454922in}}{\pgfqpoint{8.075245in}{3.458488in}}%
\pgfpathcurveto{\pgfqpoint{8.078812in}{3.462054in}}{\pgfqpoint{8.080816in}{3.466892in}}{\pgfqpoint{8.080816in}{3.471936in}}%
\pgfpathcurveto{\pgfqpoint{8.080816in}{3.476980in}}{\pgfqpoint{8.078812in}{3.481817in}}{\pgfqpoint{8.075245in}{3.485384in}}%
\pgfpathcurveto{\pgfqpoint{8.071679in}{3.488950in}}{\pgfqpoint{8.066841in}{3.490954in}}{\pgfqpoint{8.061798in}{3.490954in}}%
\pgfpathcurveto{\pgfqpoint{8.056754in}{3.490954in}}{\pgfqpoint{8.051916in}{3.488950in}}{\pgfqpoint{8.048350in}{3.485384in}}%
\pgfpathcurveto{\pgfqpoint{8.044783in}{3.481817in}}{\pgfqpoint{8.042779in}{3.476980in}}{\pgfqpoint{8.042779in}{3.471936in}}%
\pgfpathcurveto{\pgfqpoint{8.042779in}{3.466892in}}{\pgfqpoint{8.044783in}{3.462054in}}{\pgfqpoint{8.048350in}{3.458488in}}%
\pgfpathcurveto{\pgfqpoint{8.051916in}{3.454922in}}{\pgfqpoint{8.056754in}{3.452918in}}{\pgfqpoint{8.061798in}{3.452918in}}%
\pgfpathclose%
\pgfusepath{fill}%
\end{pgfscope}%
\begin{pgfscope}%
\pgfpathrectangle{\pgfqpoint{6.572727in}{0.473000in}}{\pgfqpoint{4.227273in}{3.311000in}}%
\pgfusepath{clip}%
\pgfsetbuttcap%
\pgfsetroundjoin%
\definecolor{currentfill}{rgb}{0.993248,0.906157,0.143936}%
\pgfsetfillcolor{currentfill}%
\pgfsetfillopacity{0.700000}%
\pgfsetlinewidth{0.000000pt}%
\definecolor{currentstroke}{rgb}{0.000000,0.000000,0.000000}%
\pgfsetstrokecolor{currentstroke}%
\pgfsetstrokeopacity{0.700000}%
\pgfsetdash{}{0pt}%
\pgfpathmoveto{\pgfqpoint{9.826447in}{1.328604in}}%
\pgfpathcurveto{\pgfqpoint{9.831491in}{1.328604in}}{\pgfqpoint{9.836329in}{1.330608in}}{\pgfqpoint{9.839895in}{1.334174in}}%
\pgfpathcurveto{\pgfqpoint{9.843461in}{1.337740in}}{\pgfqpoint{9.845465in}{1.342578in}}{\pgfqpoint{9.845465in}{1.347622in}}%
\pgfpathcurveto{\pgfqpoint{9.845465in}{1.352666in}}{\pgfqpoint{9.843461in}{1.357503in}}{\pgfqpoint{9.839895in}{1.361070in}}%
\pgfpathcurveto{\pgfqpoint{9.836329in}{1.364636in}}{\pgfqpoint{9.831491in}{1.366640in}}{\pgfqpoint{9.826447in}{1.366640in}}%
\pgfpathcurveto{\pgfqpoint{9.821404in}{1.366640in}}{\pgfqpoint{9.816566in}{1.364636in}}{\pgfqpoint{9.812999in}{1.361070in}}%
\pgfpathcurveto{\pgfqpoint{9.809433in}{1.357503in}}{\pgfqpoint{9.807429in}{1.352666in}}{\pgfqpoint{9.807429in}{1.347622in}}%
\pgfpathcurveto{\pgfqpoint{9.807429in}{1.342578in}}{\pgfqpoint{9.809433in}{1.337740in}}{\pgfqpoint{9.812999in}{1.334174in}}%
\pgfpathcurveto{\pgfqpoint{9.816566in}{1.330608in}}{\pgfqpoint{9.821404in}{1.328604in}}{\pgfqpoint{9.826447in}{1.328604in}}%
\pgfpathclose%
\pgfusepath{fill}%
\end{pgfscope}%
\begin{pgfscope}%
\pgfpathrectangle{\pgfqpoint{6.572727in}{0.473000in}}{\pgfqpoint{4.227273in}{3.311000in}}%
\pgfusepath{clip}%
\pgfsetbuttcap%
\pgfsetroundjoin%
\definecolor{currentfill}{rgb}{0.993248,0.906157,0.143936}%
\pgfsetfillcolor{currentfill}%
\pgfsetfillopacity{0.700000}%
\pgfsetlinewidth{0.000000pt}%
\definecolor{currentstroke}{rgb}{0.000000,0.000000,0.000000}%
\pgfsetstrokecolor{currentstroke}%
\pgfsetstrokeopacity{0.700000}%
\pgfsetdash{}{0pt}%
\pgfpathmoveto{\pgfqpoint{10.093114in}{1.256393in}}%
\pgfpathcurveto{\pgfqpoint{10.098158in}{1.256393in}}{\pgfqpoint{10.102996in}{1.258397in}}{\pgfqpoint{10.106562in}{1.261963in}}%
\pgfpathcurveto{\pgfqpoint{10.110129in}{1.265530in}}{\pgfqpoint{10.112133in}{1.270367in}}{\pgfqpoint{10.112133in}{1.275411in}}%
\pgfpathcurveto{\pgfqpoint{10.112133in}{1.280455in}}{\pgfqpoint{10.110129in}{1.285293in}}{\pgfqpoint{10.106562in}{1.288859in}}%
\pgfpathcurveto{\pgfqpoint{10.102996in}{1.292425in}}{\pgfqpoint{10.098158in}{1.294429in}}{\pgfqpoint{10.093114in}{1.294429in}}%
\pgfpathcurveto{\pgfqpoint{10.088071in}{1.294429in}}{\pgfqpoint{10.083233in}{1.292425in}}{\pgfqpoint{10.079667in}{1.288859in}}%
\pgfpathcurveto{\pgfqpoint{10.076100in}{1.285293in}}{\pgfqpoint{10.074096in}{1.280455in}}{\pgfqpoint{10.074096in}{1.275411in}}%
\pgfpathcurveto{\pgfqpoint{10.074096in}{1.270367in}}{\pgfqpoint{10.076100in}{1.265530in}}{\pgfqpoint{10.079667in}{1.261963in}}%
\pgfpathcurveto{\pgfqpoint{10.083233in}{1.258397in}}{\pgfqpoint{10.088071in}{1.256393in}}{\pgfqpoint{10.093114in}{1.256393in}}%
\pgfpathclose%
\pgfusepath{fill}%
\end{pgfscope}%
\begin{pgfscope}%
\pgfpathrectangle{\pgfqpoint{6.572727in}{0.473000in}}{\pgfqpoint{4.227273in}{3.311000in}}%
\pgfusepath{clip}%
\pgfsetbuttcap%
\pgfsetroundjoin%
\definecolor{currentfill}{rgb}{0.993248,0.906157,0.143936}%
\pgfsetfillcolor{currentfill}%
\pgfsetfillopacity{0.700000}%
\pgfsetlinewidth{0.000000pt}%
\definecolor{currentstroke}{rgb}{0.000000,0.000000,0.000000}%
\pgfsetstrokecolor{currentstroke}%
\pgfsetstrokeopacity{0.700000}%
\pgfsetdash{}{0pt}%
\pgfpathmoveto{\pgfqpoint{9.257791in}{2.235985in}}%
\pgfpathcurveto{\pgfqpoint{9.262835in}{2.235985in}}{\pgfqpoint{9.267673in}{2.237989in}}{\pgfqpoint{9.271239in}{2.241555in}}%
\pgfpathcurveto{\pgfqpoint{9.274806in}{2.245122in}}{\pgfqpoint{9.276810in}{2.249959in}}{\pgfqpoint{9.276810in}{2.255003in}}%
\pgfpathcurveto{\pgfqpoint{9.276810in}{2.260047in}}{\pgfqpoint{9.274806in}{2.264884in}}{\pgfqpoint{9.271239in}{2.268451in}}%
\pgfpathcurveto{\pgfqpoint{9.267673in}{2.272017in}}{\pgfqpoint{9.262835in}{2.274021in}}{\pgfqpoint{9.257791in}{2.274021in}}%
\pgfpathcurveto{\pgfqpoint{9.252748in}{2.274021in}}{\pgfqpoint{9.247910in}{2.272017in}}{\pgfqpoint{9.244344in}{2.268451in}}%
\pgfpathcurveto{\pgfqpoint{9.240777in}{2.264884in}}{\pgfqpoint{9.238773in}{2.260047in}}{\pgfqpoint{9.238773in}{2.255003in}}%
\pgfpathcurveto{\pgfqpoint{9.238773in}{2.249959in}}{\pgfqpoint{9.240777in}{2.245122in}}{\pgfqpoint{9.244344in}{2.241555in}}%
\pgfpathcurveto{\pgfqpoint{9.247910in}{2.237989in}}{\pgfqpoint{9.252748in}{2.235985in}}{\pgfqpoint{9.257791in}{2.235985in}}%
\pgfpathclose%
\pgfusepath{fill}%
\end{pgfscope}%
\begin{pgfscope}%
\pgfpathrectangle{\pgfqpoint{6.572727in}{0.473000in}}{\pgfqpoint{4.227273in}{3.311000in}}%
\pgfusepath{clip}%
\pgfsetbuttcap%
\pgfsetroundjoin%
\definecolor{currentfill}{rgb}{0.127568,0.566949,0.550556}%
\pgfsetfillcolor{currentfill}%
\pgfsetfillopacity{0.700000}%
\pgfsetlinewidth{0.000000pt}%
\definecolor{currentstroke}{rgb}{0.000000,0.000000,0.000000}%
\pgfsetstrokecolor{currentstroke}%
\pgfsetstrokeopacity{0.700000}%
\pgfsetdash{}{0pt}%
\pgfpathmoveto{\pgfqpoint{8.069953in}{1.293698in}}%
\pgfpathcurveto{\pgfqpoint{8.074997in}{1.293698in}}{\pgfqpoint{8.079834in}{1.295702in}}{\pgfqpoint{8.083401in}{1.299269in}}%
\pgfpathcurveto{\pgfqpoint{8.086967in}{1.302835in}}{\pgfqpoint{8.088971in}{1.307673in}}{\pgfqpoint{8.088971in}{1.312717in}}%
\pgfpathcurveto{\pgfqpoint{8.088971in}{1.317760in}}{\pgfqpoint{8.086967in}{1.322598in}}{\pgfqpoint{8.083401in}{1.326164in}}%
\pgfpathcurveto{\pgfqpoint{8.079834in}{1.329731in}}{\pgfqpoint{8.074997in}{1.331735in}}{\pgfqpoint{8.069953in}{1.331735in}}%
\pgfpathcurveto{\pgfqpoint{8.064909in}{1.331735in}}{\pgfqpoint{8.060071in}{1.329731in}}{\pgfqpoint{8.056505in}{1.326164in}}%
\pgfpathcurveto{\pgfqpoint{8.052939in}{1.322598in}}{\pgfqpoint{8.050935in}{1.317760in}}{\pgfqpoint{8.050935in}{1.312717in}}%
\pgfpathcurveto{\pgfqpoint{8.050935in}{1.307673in}}{\pgfqpoint{8.052939in}{1.302835in}}{\pgfqpoint{8.056505in}{1.299269in}}%
\pgfpathcurveto{\pgfqpoint{8.060071in}{1.295702in}}{\pgfqpoint{8.064909in}{1.293698in}}{\pgfqpoint{8.069953in}{1.293698in}}%
\pgfpathclose%
\pgfusepath{fill}%
\end{pgfscope}%
\begin{pgfscope}%
\pgfpathrectangle{\pgfqpoint{6.572727in}{0.473000in}}{\pgfqpoint{4.227273in}{3.311000in}}%
\pgfusepath{clip}%
\pgfsetbuttcap%
\pgfsetroundjoin%
\definecolor{currentfill}{rgb}{0.993248,0.906157,0.143936}%
\pgfsetfillcolor{currentfill}%
\pgfsetfillopacity{0.700000}%
\pgfsetlinewidth{0.000000pt}%
\definecolor{currentstroke}{rgb}{0.000000,0.000000,0.000000}%
\pgfsetstrokecolor{currentstroke}%
\pgfsetstrokeopacity{0.700000}%
\pgfsetdash{}{0pt}%
\pgfpathmoveto{\pgfqpoint{10.028396in}{1.686798in}}%
\pgfpathcurveto{\pgfqpoint{10.033440in}{1.686798in}}{\pgfqpoint{10.038278in}{1.688802in}}{\pgfqpoint{10.041844in}{1.692368in}}%
\pgfpathcurveto{\pgfqpoint{10.045411in}{1.695935in}}{\pgfqpoint{10.047414in}{1.700772in}}{\pgfqpoint{10.047414in}{1.705816in}}%
\pgfpathcurveto{\pgfqpoint{10.047414in}{1.710860in}}{\pgfqpoint{10.045411in}{1.715698in}}{\pgfqpoint{10.041844in}{1.719264in}}%
\pgfpathcurveto{\pgfqpoint{10.038278in}{1.722830in}}{\pgfqpoint{10.033440in}{1.724834in}}{\pgfqpoint{10.028396in}{1.724834in}}%
\pgfpathcurveto{\pgfqpoint{10.023353in}{1.724834in}}{\pgfqpoint{10.018515in}{1.722830in}}{\pgfqpoint{10.014948in}{1.719264in}}%
\pgfpathcurveto{\pgfqpoint{10.011382in}{1.715698in}}{\pgfqpoint{10.009378in}{1.710860in}}{\pgfqpoint{10.009378in}{1.705816in}}%
\pgfpathcurveto{\pgfqpoint{10.009378in}{1.700772in}}{\pgfqpoint{10.011382in}{1.695935in}}{\pgfqpoint{10.014948in}{1.692368in}}%
\pgfpathcurveto{\pgfqpoint{10.018515in}{1.688802in}}{\pgfqpoint{10.023353in}{1.686798in}}{\pgfqpoint{10.028396in}{1.686798in}}%
\pgfpathclose%
\pgfusepath{fill}%
\end{pgfscope}%
\begin{pgfscope}%
\pgfpathrectangle{\pgfqpoint{6.572727in}{0.473000in}}{\pgfqpoint{4.227273in}{3.311000in}}%
\pgfusepath{clip}%
\pgfsetbuttcap%
\pgfsetroundjoin%
\definecolor{currentfill}{rgb}{0.127568,0.566949,0.550556}%
\pgfsetfillcolor{currentfill}%
\pgfsetfillopacity{0.700000}%
\pgfsetlinewidth{0.000000pt}%
\definecolor{currentstroke}{rgb}{0.000000,0.000000,0.000000}%
\pgfsetstrokecolor{currentstroke}%
\pgfsetstrokeopacity{0.700000}%
\pgfsetdash{}{0pt}%
\pgfpathmoveto{\pgfqpoint{8.604466in}{3.557085in}}%
\pgfpathcurveto{\pgfqpoint{8.609510in}{3.557085in}}{\pgfqpoint{8.614348in}{3.559089in}}{\pgfqpoint{8.617914in}{3.562656in}}%
\pgfpathcurveto{\pgfqpoint{8.621481in}{3.566222in}}{\pgfqpoint{8.623485in}{3.571060in}}{\pgfqpoint{8.623485in}{3.576103in}}%
\pgfpathcurveto{\pgfqpoint{8.623485in}{3.581147in}}{\pgfqpoint{8.621481in}{3.585985in}}{\pgfqpoint{8.617914in}{3.589551in}}%
\pgfpathcurveto{\pgfqpoint{8.614348in}{3.593118in}}{\pgfqpoint{8.609510in}{3.595122in}}{\pgfqpoint{8.604466in}{3.595122in}}%
\pgfpathcurveto{\pgfqpoint{8.599423in}{3.595122in}}{\pgfqpoint{8.594585in}{3.593118in}}{\pgfqpoint{8.591019in}{3.589551in}}%
\pgfpathcurveto{\pgfqpoint{8.587452in}{3.585985in}}{\pgfqpoint{8.585448in}{3.581147in}}{\pgfqpoint{8.585448in}{3.576103in}}%
\pgfpathcurveto{\pgfqpoint{8.585448in}{3.571060in}}{\pgfqpoint{8.587452in}{3.566222in}}{\pgfqpoint{8.591019in}{3.562656in}}%
\pgfpathcurveto{\pgfqpoint{8.594585in}{3.559089in}}{\pgfqpoint{8.599423in}{3.557085in}}{\pgfqpoint{8.604466in}{3.557085in}}%
\pgfpathclose%
\pgfusepath{fill}%
\end{pgfscope}%
\begin{pgfscope}%
\pgfpathrectangle{\pgfqpoint{6.572727in}{0.473000in}}{\pgfqpoint{4.227273in}{3.311000in}}%
\pgfusepath{clip}%
\pgfsetbuttcap%
\pgfsetroundjoin%
\definecolor{currentfill}{rgb}{0.127568,0.566949,0.550556}%
\pgfsetfillcolor{currentfill}%
\pgfsetfillopacity{0.700000}%
\pgfsetlinewidth{0.000000pt}%
\definecolor{currentstroke}{rgb}{0.000000,0.000000,0.000000}%
\pgfsetstrokecolor{currentstroke}%
\pgfsetstrokeopacity{0.700000}%
\pgfsetdash{}{0pt}%
\pgfpathmoveto{\pgfqpoint{7.838982in}{2.824834in}}%
\pgfpathcurveto{\pgfqpoint{7.844026in}{2.824834in}}{\pgfqpoint{7.848863in}{2.826838in}}{\pgfqpoint{7.852430in}{2.830404in}}%
\pgfpathcurveto{\pgfqpoint{7.855996in}{2.833971in}}{\pgfqpoint{7.858000in}{2.838808in}}{\pgfqpoint{7.858000in}{2.843852in}}%
\pgfpathcurveto{\pgfqpoint{7.858000in}{2.848896in}}{\pgfqpoint{7.855996in}{2.853734in}}{\pgfqpoint{7.852430in}{2.857300in}}%
\pgfpathcurveto{\pgfqpoint{7.848863in}{2.860866in}}{\pgfqpoint{7.844026in}{2.862870in}}{\pgfqpoint{7.838982in}{2.862870in}}%
\pgfpathcurveto{\pgfqpoint{7.833938in}{2.862870in}}{\pgfqpoint{7.829100in}{2.860866in}}{\pgfqpoint{7.825534in}{2.857300in}}%
\pgfpathcurveto{\pgfqpoint{7.821968in}{2.853734in}}{\pgfqpoint{7.819964in}{2.848896in}}{\pgfqpoint{7.819964in}{2.843852in}}%
\pgfpathcurveto{\pgfqpoint{7.819964in}{2.838808in}}{\pgfqpoint{7.821968in}{2.833971in}}{\pgfqpoint{7.825534in}{2.830404in}}%
\pgfpathcurveto{\pgfqpoint{7.829100in}{2.826838in}}{\pgfqpoint{7.833938in}{2.824834in}}{\pgfqpoint{7.838982in}{2.824834in}}%
\pgfpathclose%
\pgfusepath{fill}%
\end{pgfscope}%
\begin{pgfscope}%
\pgfpathrectangle{\pgfqpoint{6.572727in}{0.473000in}}{\pgfqpoint{4.227273in}{3.311000in}}%
\pgfusepath{clip}%
\pgfsetbuttcap%
\pgfsetroundjoin%
\definecolor{currentfill}{rgb}{0.127568,0.566949,0.550556}%
\pgfsetfillcolor{currentfill}%
\pgfsetfillopacity{0.700000}%
\pgfsetlinewidth{0.000000pt}%
\definecolor{currentstroke}{rgb}{0.000000,0.000000,0.000000}%
\pgfsetstrokecolor{currentstroke}%
\pgfsetstrokeopacity{0.700000}%
\pgfsetdash{}{0pt}%
\pgfpathmoveto{\pgfqpoint{7.485068in}{1.083969in}}%
\pgfpathcurveto{\pgfqpoint{7.490112in}{1.083969in}}{\pgfqpoint{7.494949in}{1.085973in}}{\pgfqpoint{7.498516in}{1.089540in}}%
\pgfpathcurveto{\pgfqpoint{7.502082in}{1.093106in}}{\pgfqpoint{7.504086in}{1.097944in}}{\pgfqpoint{7.504086in}{1.102988in}}%
\pgfpathcurveto{\pgfqpoint{7.504086in}{1.108031in}}{\pgfqpoint{7.502082in}{1.112869in}}{\pgfqpoint{7.498516in}{1.116435in}}%
\pgfpathcurveto{\pgfqpoint{7.494949in}{1.120002in}}{\pgfqpoint{7.490112in}{1.122006in}}{\pgfqpoint{7.485068in}{1.122006in}}%
\pgfpathcurveto{\pgfqpoint{7.480024in}{1.122006in}}{\pgfqpoint{7.475187in}{1.120002in}}{\pgfqpoint{7.471620in}{1.116435in}}%
\pgfpathcurveto{\pgfqpoint{7.468054in}{1.112869in}}{\pgfqpoint{7.466050in}{1.108031in}}{\pgfqpoint{7.466050in}{1.102988in}}%
\pgfpathcurveto{\pgfqpoint{7.466050in}{1.097944in}}{\pgfqpoint{7.468054in}{1.093106in}}{\pgfqpoint{7.471620in}{1.089540in}}%
\pgfpathcurveto{\pgfqpoint{7.475187in}{1.085973in}}{\pgfqpoint{7.480024in}{1.083969in}}{\pgfqpoint{7.485068in}{1.083969in}}%
\pgfpathclose%
\pgfusepath{fill}%
\end{pgfscope}%
\begin{pgfscope}%
\pgfpathrectangle{\pgfqpoint{6.572727in}{0.473000in}}{\pgfqpoint{4.227273in}{3.311000in}}%
\pgfusepath{clip}%
\pgfsetbuttcap%
\pgfsetroundjoin%
\definecolor{currentfill}{rgb}{0.127568,0.566949,0.550556}%
\pgfsetfillcolor{currentfill}%
\pgfsetfillopacity{0.700000}%
\pgfsetlinewidth{0.000000pt}%
\definecolor{currentstroke}{rgb}{0.000000,0.000000,0.000000}%
\pgfsetstrokecolor{currentstroke}%
\pgfsetstrokeopacity{0.700000}%
\pgfsetdash{}{0pt}%
\pgfpathmoveto{\pgfqpoint{7.363760in}{1.432989in}}%
\pgfpathcurveto{\pgfqpoint{7.368803in}{1.432989in}}{\pgfqpoint{7.373641in}{1.434993in}}{\pgfqpoint{7.377207in}{1.438559in}}%
\pgfpathcurveto{\pgfqpoint{7.380774in}{1.442126in}}{\pgfqpoint{7.382778in}{1.446963in}}{\pgfqpoint{7.382778in}{1.452007in}}%
\pgfpathcurveto{\pgfqpoint{7.382778in}{1.457051in}}{\pgfqpoint{7.380774in}{1.461889in}}{\pgfqpoint{7.377207in}{1.465455in}}%
\pgfpathcurveto{\pgfqpoint{7.373641in}{1.469021in}}{\pgfqpoint{7.368803in}{1.471025in}}{\pgfqpoint{7.363760in}{1.471025in}}%
\pgfpathcurveto{\pgfqpoint{7.358716in}{1.471025in}}{\pgfqpoint{7.353878in}{1.469021in}}{\pgfqpoint{7.350312in}{1.465455in}}%
\pgfpathcurveto{\pgfqpoint{7.346745in}{1.461889in}}{\pgfqpoint{7.344741in}{1.457051in}}{\pgfqpoint{7.344741in}{1.452007in}}%
\pgfpathcurveto{\pgfqpoint{7.344741in}{1.446963in}}{\pgfqpoint{7.346745in}{1.442126in}}{\pgfqpoint{7.350312in}{1.438559in}}%
\pgfpathcurveto{\pgfqpoint{7.353878in}{1.434993in}}{\pgfqpoint{7.358716in}{1.432989in}}{\pgfqpoint{7.363760in}{1.432989in}}%
\pgfpathclose%
\pgfusepath{fill}%
\end{pgfscope}%
\begin{pgfscope}%
\pgfpathrectangle{\pgfqpoint{6.572727in}{0.473000in}}{\pgfqpoint{4.227273in}{3.311000in}}%
\pgfusepath{clip}%
\pgfsetbuttcap%
\pgfsetroundjoin%
\definecolor{currentfill}{rgb}{0.127568,0.566949,0.550556}%
\pgfsetfillcolor{currentfill}%
\pgfsetfillopacity{0.700000}%
\pgfsetlinewidth{0.000000pt}%
\definecolor{currentstroke}{rgb}{0.000000,0.000000,0.000000}%
\pgfsetstrokecolor{currentstroke}%
\pgfsetstrokeopacity{0.700000}%
\pgfsetdash{}{0pt}%
\pgfpathmoveto{\pgfqpoint{7.886747in}{2.032902in}}%
\pgfpathcurveto{\pgfqpoint{7.891791in}{2.032902in}}{\pgfqpoint{7.896629in}{2.034906in}}{\pgfqpoint{7.900195in}{2.038472in}}%
\pgfpathcurveto{\pgfqpoint{7.903761in}{2.042039in}}{\pgfqpoint{7.905765in}{2.046876in}}{\pgfqpoint{7.905765in}{2.051920in}}%
\pgfpathcurveto{\pgfqpoint{7.905765in}{2.056964in}}{\pgfqpoint{7.903761in}{2.061802in}}{\pgfqpoint{7.900195in}{2.065368in}}%
\pgfpathcurveto{\pgfqpoint{7.896629in}{2.068934in}}{\pgfqpoint{7.891791in}{2.070938in}}{\pgfqpoint{7.886747in}{2.070938in}}%
\pgfpathcurveto{\pgfqpoint{7.881703in}{2.070938in}}{\pgfqpoint{7.876866in}{2.068934in}}{\pgfqpoint{7.873299in}{2.065368in}}%
\pgfpathcurveto{\pgfqpoint{7.869733in}{2.061802in}}{\pgfqpoint{7.867729in}{2.056964in}}{\pgfqpoint{7.867729in}{2.051920in}}%
\pgfpathcurveto{\pgfqpoint{7.867729in}{2.046876in}}{\pgfqpoint{7.869733in}{2.042039in}}{\pgfqpoint{7.873299in}{2.038472in}}%
\pgfpathcurveto{\pgfqpoint{7.876866in}{2.034906in}}{\pgfqpoint{7.881703in}{2.032902in}}{\pgfqpoint{7.886747in}{2.032902in}}%
\pgfpathclose%
\pgfusepath{fill}%
\end{pgfscope}%
\begin{pgfscope}%
\pgfpathrectangle{\pgfqpoint{6.572727in}{0.473000in}}{\pgfqpoint{4.227273in}{3.311000in}}%
\pgfusepath{clip}%
\pgfsetbuttcap%
\pgfsetroundjoin%
\definecolor{currentfill}{rgb}{0.267004,0.004874,0.329415}%
\pgfsetfillcolor{currentfill}%
\pgfsetfillopacity{0.700000}%
\pgfsetlinewidth{0.000000pt}%
\definecolor{currentstroke}{rgb}{0.000000,0.000000,0.000000}%
\pgfsetstrokecolor{currentstroke}%
\pgfsetstrokeopacity{0.700000}%
\pgfsetdash{}{0pt}%
\pgfpathmoveto{\pgfqpoint{7.008253in}{2.022971in}}%
\pgfpathcurveto{\pgfqpoint{7.013296in}{2.022971in}}{\pgfqpoint{7.018134in}{2.024975in}}{\pgfqpoint{7.021700in}{2.028541in}}%
\pgfpathcurveto{\pgfqpoint{7.025267in}{2.032108in}}{\pgfqpoint{7.027271in}{2.036946in}}{\pgfqpoint{7.027271in}{2.041989in}}%
\pgfpathcurveto{\pgfqpoint{7.027271in}{2.047033in}}{\pgfqpoint{7.025267in}{2.051871in}}{\pgfqpoint{7.021700in}{2.055437in}}%
\pgfpathcurveto{\pgfqpoint{7.018134in}{2.059003in}}{\pgfqpoint{7.013296in}{2.061007in}}{\pgfqpoint{7.008253in}{2.061007in}}%
\pgfpathcurveto{\pgfqpoint{7.003209in}{2.061007in}}{\pgfqpoint{6.998371in}{2.059003in}}{\pgfqpoint{6.994805in}{2.055437in}}%
\pgfpathcurveto{\pgfqpoint{6.991238in}{2.051871in}}{\pgfqpoint{6.989234in}{2.047033in}}{\pgfqpoint{6.989234in}{2.041989in}}%
\pgfpathcurveto{\pgfqpoint{6.989234in}{2.036946in}}{\pgfqpoint{6.991238in}{2.032108in}}{\pgfqpoint{6.994805in}{2.028541in}}%
\pgfpathcurveto{\pgfqpoint{6.998371in}{2.024975in}}{\pgfqpoint{7.003209in}{2.022971in}}{\pgfqpoint{7.008253in}{2.022971in}}%
\pgfpathclose%
\pgfusepath{fill}%
\end{pgfscope}%
\begin{pgfscope}%
\pgfpathrectangle{\pgfqpoint{6.572727in}{0.473000in}}{\pgfqpoint{4.227273in}{3.311000in}}%
\pgfusepath{clip}%
\pgfsetbuttcap%
\pgfsetroundjoin%
\definecolor{currentfill}{rgb}{0.127568,0.566949,0.550556}%
\pgfsetfillcolor{currentfill}%
\pgfsetfillopacity{0.700000}%
\pgfsetlinewidth{0.000000pt}%
\definecolor{currentstroke}{rgb}{0.000000,0.000000,0.000000}%
\pgfsetstrokecolor{currentstroke}%
\pgfsetstrokeopacity{0.700000}%
\pgfsetdash{}{0pt}%
\pgfpathmoveto{\pgfqpoint{7.762669in}{1.974060in}}%
\pgfpathcurveto{\pgfqpoint{7.767713in}{1.974060in}}{\pgfqpoint{7.772551in}{1.976063in}}{\pgfqpoint{7.776117in}{1.979630in}}%
\pgfpathcurveto{\pgfqpoint{7.779684in}{1.983196in}}{\pgfqpoint{7.781688in}{1.988034in}}{\pgfqpoint{7.781688in}{1.993078in}}%
\pgfpathcurveto{\pgfqpoint{7.781688in}{1.998121in}}{\pgfqpoint{7.779684in}{2.002959in}}{\pgfqpoint{7.776117in}{2.006526in}}%
\pgfpathcurveto{\pgfqpoint{7.772551in}{2.010092in}}{\pgfqpoint{7.767713in}{2.012096in}}{\pgfqpoint{7.762669in}{2.012096in}}%
\pgfpathcurveto{\pgfqpoint{7.757626in}{2.012096in}}{\pgfqpoint{7.752788in}{2.010092in}}{\pgfqpoint{7.749222in}{2.006526in}}%
\pgfpathcurveto{\pgfqpoint{7.745655in}{2.002959in}}{\pgfqpoint{7.743651in}{1.998121in}}{\pgfqpoint{7.743651in}{1.993078in}}%
\pgfpathcurveto{\pgfqpoint{7.743651in}{1.988034in}}{\pgfqpoint{7.745655in}{1.983196in}}{\pgfqpoint{7.749222in}{1.979630in}}%
\pgfpathcurveto{\pgfqpoint{7.752788in}{1.976063in}}{\pgfqpoint{7.757626in}{1.974060in}}{\pgfqpoint{7.762669in}{1.974060in}}%
\pgfpathclose%
\pgfusepath{fill}%
\end{pgfscope}%
\begin{pgfscope}%
\pgfpathrectangle{\pgfqpoint{6.572727in}{0.473000in}}{\pgfqpoint{4.227273in}{3.311000in}}%
\pgfusepath{clip}%
\pgfsetbuttcap%
\pgfsetroundjoin%
\definecolor{currentfill}{rgb}{0.127568,0.566949,0.550556}%
\pgfsetfillcolor{currentfill}%
\pgfsetfillopacity{0.700000}%
\pgfsetlinewidth{0.000000pt}%
\definecolor{currentstroke}{rgb}{0.000000,0.000000,0.000000}%
\pgfsetstrokecolor{currentstroke}%
\pgfsetstrokeopacity{0.700000}%
\pgfsetdash{}{0pt}%
\pgfpathmoveto{\pgfqpoint{8.051190in}{1.561032in}}%
\pgfpathcurveto{\pgfqpoint{8.056233in}{1.561032in}}{\pgfqpoint{8.061071in}{1.563036in}}{\pgfqpoint{8.064638in}{1.566603in}}%
\pgfpathcurveto{\pgfqpoint{8.068204in}{1.570169in}}{\pgfqpoint{8.070208in}{1.575007in}}{\pgfqpoint{8.070208in}{1.580050in}}%
\pgfpathcurveto{\pgfqpoint{8.070208in}{1.585094in}}{\pgfqpoint{8.068204in}{1.589932in}}{\pgfqpoint{8.064638in}{1.593498in}}%
\pgfpathcurveto{\pgfqpoint{8.061071in}{1.597065in}}{\pgfqpoint{8.056233in}{1.599069in}}{\pgfqpoint{8.051190in}{1.599069in}}%
\pgfpathcurveto{\pgfqpoint{8.046146in}{1.599069in}}{\pgfqpoint{8.041308in}{1.597065in}}{\pgfqpoint{8.037742in}{1.593498in}}%
\pgfpathcurveto{\pgfqpoint{8.034176in}{1.589932in}}{\pgfqpoint{8.032172in}{1.585094in}}{\pgfqpoint{8.032172in}{1.580050in}}%
\pgfpathcurveto{\pgfqpoint{8.032172in}{1.575007in}}{\pgfqpoint{8.034176in}{1.570169in}}{\pgfqpoint{8.037742in}{1.566603in}}%
\pgfpathcurveto{\pgfqpoint{8.041308in}{1.563036in}}{\pgfqpoint{8.046146in}{1.561032in}}{\pgfqpoint{8.051190in}{1.561032in}}%
\pgfpathclose%
\pgfusepath{fill}%
\end{pgfscope}%
\begin{pgfscope}%
\pgfpathrectangle{\pgfqpoint{6.572727in}{0.473000in}}{\pgfqpoint{4.227273in}{3.311000in}}%
\pgfusepath{clip}%
\pgfsetbuttcap%
\pgfsetroundjoin%
\definecolor{currentfill}{rgb}{0.127568,0.566949,0.550556}%
\pgfsetfillcolor{currentfill}%
\pgfsetfillopacity{0.700000}%
\pgfsetlinewidth{0.000000pt}%
\definecolor{currentstroke}{rgb}{0.000000,0.000000,0.000000}%
\pgfsetstrokecolor{currentstroke}%
\pgfsetstrokeopacity{0.700000}%
\pgfsetdash{}{0pt}%
\pgfpathmoveto{\pgfqpoint{7.897754in}{2.670197in}}%
\pgfpathcurveto{\pgfqpoint{7.902798in}{2.670197in}}{\pgfqpoint{7.907635in}{2.672201in}}{\pgfqpoint{7.911202in}{2.675768in}}%
\pgfpathcurveto{\pgfqpoint{7.914768in}{2.679334in}}{\pgfqpoint{7.916772in}{2.684172in}}{\pgfqpoint{7.916772in}{2.689216in}}%
\pgfpathcurveto{\pgfqpoint{7.916772in}{2.694259in}}{\pgfqpoint{7.914768in}{2.699097in}}{\pgfqpoint{7.911202in}{2.702663in}}%
\pgfpathcurveto{\pgfqpoint{7.907635in}{2.706230in}}{\pgfqpoint{7.902798in}{2.708234in}}{\pgfqpoint{7.897754in}{2.708234in}}%
\pgfpathcurveto{\pgfqpoint{7.892710in}{2.708234in}}{\pgfqpoint{7.887872in}{2.706230in}}{\pgfqpoint{7.884306in}{2.702663in}}%
\pgfpathcurveto{\pgfqpoint{7.880740in}{2.699097in}}{\pgfqpoint{7.878736in}{2.694259in}}{\pgfqpoint{7.878736in}{2.689216in}}%
\pgfpathcurveto{\pgfqpoint{7.878736in}{2.684172in}}{\pgfqpoint{7.880740in}{2.679334in}}{\pgfqpoint{7.884306in}{2.675768in}}%
\pgfpathcurveto{\pgfqpoint{7.887872in}{2.672201in}}{\pgfqpoint{7.892710in}{2.670197in}}{\pgfqpoint{7.897754in}{2.670197in}}%
\pgfpathclose%
\pgfusepath{fill}%
\end{pgfscope}%
\begin{pgfscope}%
\pgfpathrectangle{\pgfqpoint{6.572727in}{0.473000in}}{\pgfqpoint{4.227273in}{3.311000in}}%
\pgfusepath{clip}%
\pgfsetbuttcap%
\pgfsetroundjoin%
\definecolor{currentfill}{rgb}{0.127568,0.566949,0.550556}%
\pgfsetfillcolor{currentfill}%
\pgfsetfillopacity{0.700000}%
\pgfsetlinewidth{0.000000pt}%
\definecolor{currentstroke}{rgb}{0.000000,0.000000,0.000000}%
\pgfsetstrokecolor{currentstroke}%
\pgfsetstrokeopacity{0.700000}%
\pgfsetdash{}{0pt}%
\pgfpathmoveto{\pgfqpoint{8.591351in}{1.890807in}}%
\pgfpathcurveto{\pgfqpoint{8.596394in}{1.890807in}}{\pgfqpoint{8.601232in}{1.892811in}}{\pgfqpoint{8.604799in}{1.896378in}}%
\pgfpathcurveto{\pgfqpoint{8.608365in}{1.899944in}}{\pgfqpoint{8.610369in}{1.904782in}}{\pgfqpoint{8.610369in}{1.909825in}}%
\pgfpathcurveto{\pgfqpoint{8.610369in}{1.914869in}}{\pgfqpoint{8.608365in}{1.919707in}}{\pgfqpoint{8.604799in}{1.923273in}}%
\pgfpathcurveto{\pgfqpoint{8.601232in}{1.926840in}}{\pgfqpoint{8.596394in}{1.928844in}}{\pgfqpoint{8.591351in}{1.928844in}}%
\pgfpathcurveto{\pgfqpoint{8.586307in}{1.928844in}}{\pgfqpoint{8.581469in}{1.926840in}}{\pgfqpoint{8.577903in}{1.923273in}}%
\pgfpathcurveto{\pgfqpoint{8.574336in}{1.919707in}}{\pgfqpoint{8.572333in}{1.914869in}}{\pgfqpoint{8.572333in}{1.909825in}}%
\pgfpathcurveto{\pgfqpoint{8.572333in}{1.904782in}}{\pgfqpoint{8.574336in}{1.899944in}}{\pgfqpoint{8.577903in}{1.896378in}}%
\pgfpathcurveto{\pgfqpoint{8.581469in}{1.892811in}}{\pgfqpoint{8.586307in}{1.890807in}}{\pgfqpoint{8.591351in}{1.890807in}}%
\pgfpathclose%
\pgfusepath{fill}%
\end{pgfscope}%
\begin{pgfscope}%
\pgfpathrectangle{\pgfqpoint{6.572727in}{0.473000in}}{\pgfqpoint{4.227273in}{3.311000in}}%
\pgfusepath{clip}%
\pgfsetbuttcap%
\pgfsetroundjoin%
\definecolor{currentfill}{rgb}{0.127568,0.566949,0.550556}%
\pgfsetfillcolor{currentfill}%
\pgfsetfillopacity{0.700000}%
\pgfsetlinewidth{0.000000pt}%
\definecolor{currentstroke}{rgb}{0.000000,0.000000,0.000000}%
\pgfsetstrokecolor{currentstroke}%
\pgfsetstrokeopacity{0.700000}%
\pgfsetdash{}{0pt}%
\pgfpathmoveto{\pgfqpoint{8.008299in}{2.063530in}}%
\pgfpathcurveto{\pgfqpoint{8.013343in}{2.063530in}}{\pgfqpoint{8.018181in}{2.065534in}}{\pgfqpoint{8.021747in}{2.069100in}}%
\pgfpathcurveto{\pgfqpoint{8.025313in}{2.072666in}}{\pgfqpoint{8.027317in}{2.077504in}}{\pgfqpoint{8.027317in}{2.082548in}}%
\pgfpathcurveto{\pgfqpoint{8.027317in}{2.087592in}}{\pgfqpoint{8.025313in}{2.092429in}}{\pgfqpoint{8.021747in}{2.095996in}}%
\pgfpathcurveto{\pgfqpoint{8.018181in}{2.099562in}}{\pgfqpoint{8.013343in}{2.101566in}}{\pgfqpoint{8.008299in}{2.101566in}}%
\pgfpathcurveto{\pgfqpoint{8.003255in}{2.101566in}}{\pgfqpoint{7.998418in}{2.099562in}}{\pgfqpoint{7.994851in}{2.095996in}}%
\pgfpathcurveto{\pgfqpoint{7.991285in}{2.092429in}}{\pgfqpoint{7.989281in}{2.087592in}}{\pgfqpoint{7.989281in}{2.082548in}}%
\pgfpathcurveto{\pgfqpoint{7.989281in}{2.077504in}}{\pgfqpoint{7.991285in}{2.072666in}}{\pgfqpoint{7.994851in}{2.069100in}}%
\pgfpathcurveto{\pgfqpoint{7.998418in}{2.065534in}}{\pgfqpoint{8.003255in}{2.063530in}}{\pgfqpoint{8.008299in}{2.063530in}}%
\pgfpathclose%
\pgfusepath{fill}%
\end{pgfscope}%
\begin{pgfscope}%
\pgfpathrectangle{\pgfqpoint{6.572727in}{0.473000in}}{\pgfqpoint{4.227273in}{3.311000in}}%
\pgfusepath{clip}%
\pgfsetbuttcap%
\pgfsetroundjoin%
\definecolor{currentfill}{rgb}{0.127568,0.566949,0.550556}%
\pgfsetfillcolor{currentfill}%
\pgfsetfillopacity{0.700000}%
\pgfsetlinewidth{0.000000pt}%
\definecolor{currentstroke}{rgb}{0.000000,0.000000,0.000000}%
\pgfsetstrokecolor{currentstroke}%
\pgfsetstrokeopacity{0.700000}%
\pgfsetdash{}{0pt}%
\pgfpathmoveto{\pgfqpoint{8.025682in}{2.789795in}}%
\pgfpathcurveto{\pgfqpoint{8.030726in}{2.789795in}}{\pgfqpoint{8.035564in}{2.791799in}}{\pgfqpoint{8.039130in}{2.795365in}}%
\pgfpathcurveto{\pgfqpoint{8.042696in}{2.798932in}}{\pgfqpoint{8.044700in}{2.803770in}}{\pgfqpoint{8.044700in}{2.808813in}}%
\pgfpathcurveto{\pgfqpoint{8.044700in}{2.813857in}}{\pgfqpoint{8.042696in}{2.818695in}}{\pgfqpoint{8.039130in}{2.822261in}}%
\pgfpathcurveto{\pgfqpoint{8.035564in}{2.825828in}}{\pgfqpoint{8.030726in}{2.827831in}}{\pgfqpoint{8.025682in}{2.827831in}}%
\pgfpathcurveto{\pgfqpoint{8.020638in}{2.827831in}}{\pgfqpoint{8.015801in}{2.825828in}}{\pgfqpoint{8.012234in}{2.822261in}}%
\pgfpathcurveto{\pgfqpoint{8.008668in}{2.818695in}}{\pgfqpoint{8.006664in}{2.813857in}}{\pgfqpoint{8.006664in}{2.808813in}}%
\pgfpathcurveto{\pgfqpoint{8.006664in}{2.803770in}}{\pgfqpoint{8.008668in}{2.798932in}}{\pgfqpoint{8.012234in}{2.795365in}}%
\pgfpathcurveto{\pgfqpoint{8.015801in}{2.791799in}}{\pgfqpoint{8.020638in}{2.789795in}}{\pgfqpoint{8.025682in}{2.789795in}}%
\pgfpathclose%
\pgfusepath{fill}%
\end{pgfscope}%
\begin{pgfscope}%
\pgfpathrectangle{\pgfqpoint{6.572727in}{0.473000in}}{\pgfqpoint{4.227273in}{3.311000in}}%
\pgfusepath{clip}%
\pgfsetbuttcap%
\pgfsetroundjoin%
\definecolor{currentfill}{rgb}{0.127568,0.566949,0.550556}%
\pgfsetfillcolor{currentfill}%
\pgfsetfillopacity{0.700000}%
\pgfsetlinewidth{0.000000pt}%
\definecolor{currentstroke}{rgb}{0.000000,0.000000,0.000000}%
\pgfsetstrokecolor{currentstroke}%
\pgfsetstrokeopacity{0.700000}%
\pgfsetdash{}{0pt}%
\pgfpathmoveto{\pgfqpoint{7.696817in}{2.954383in}}%
\pgfpathcurveto{\pgfqpoint{7.701861in}{2.954383in}}{\pgfqpoint{7.706699in}{2.956387in}}{\pgfqpoint{7.710265in}{2.959953in}}%
\pgfpathcurveto{\pgfqpoint{7.713831in}{2.963520in}}{\pgfqpoint{7.715835in}{2.968357in}}{\pgfqpoint{7.715835in}{2.973401in}}%
\pgfpathcurveto{\pgfqpoint{7.715835in}{2.978445in}}{\pgfqpoint{7.713831in}{2.983282in}}{\pgfqpoint{7.710265in}{2.986849in}}%
\pgfpathcurveto{\pgfqpoint{7.706699in}{2.990415in}}{\pgfqpoint{7.701861in}{2.992419in}}{\pgfqpoint{7.696817in}{2.992419in}}%
\pgfpathcurveto{\pgfqpoint{7.691774in}{2.992419in}}{\pgfqpoint{7.686936in}{2.990415in}}{\pgfqpoint{7.683369in}{2.986849in}}%
\pgfpathcurveto{\pgfqpoint{7.679803in}{2.983282in}}{\pgfqpoint{7.677799in}{2.978445in}}{\pgfqpoint{7.677799in}{2.973401in}}%
\pgfpathcurveto{\pgfqpoint{7.677799in}{2.968357in}}{\pgfqpoint{7.679803in}{2.963520in}}{\pgfqpoint{7.683369in}{2.959953in}}%
\pgfpathcurveto{\pgfqpoint{7.686936in}{2.956387in}}{\pgfqpoint{7.691774in}{2.954383in}}{\pgfqpoint{7.696817in}{2.954383in}}%
\pgfpathclose%
\pgfusepath{fill}%
\end{pgfscope}%
\begin{pgfscope}%
\pgfpathrectangle{\pgfqpoint{6.572727in}{0.473000in}}{\pgfqpoint{4.227273in}{3.311000in}}%
\pgfusepath{clip}%
\pgfsetbuttcap%
\pgfsetroundjoin%
\definecolor{currentfill}{rgb}{0.993248,0.906157,0.143936}%
\pgfsetfillcolor{currentfill}%
\pgfsetfillopacity{0.700000}%
\pgfsetlinewidth{0.000000pt}%
\definecolor{currentstroke}{rgb}{0.000000,0.000000,0.000000}%
\pgfsetstrokecolor{currentstroke}%
\pgfsetstrokeopacity{0.700000}%
\pgfsetdash{}{0pt}%
\pgfpathmoveto{\pgfqpoint{9.263842in}{1.355551in}}%
\pgfpathcurveto{\pgfqpoint{9.268886in}{1.355551in}}{\pgfqpoint{9.273724in}{1.357555in}}{\pgfqpoint{9.277290in}{1.361121in}}%
\pgfpathcurveto{\pgfqpoint{9.280857in}{1.364688in}}{\pgfqpoint{9.282860in}{1.369525in}}{\pgfqpoint{9.282860in}{1.374569in}}%
\pgfpathcurveto{\pgfqpoint{9.282860in}{1.379613in}}{\pgfqpoint{9.280857in}{1.384450in}}{\pgfqpoint{9.277290in}{1.388017in}}%
\pgfpathcurveto{\pgfqpoint{9.273724in}{1.391583in}}{\pgfqpoint{9.268886in}{1.393587in}}{\pgfqpoint{9.263842in}{1.393587in}}%
\pgfpathcurveto{\pgfqpoint{9.258799in}{1.393587in}}{\pgfqpoint{9.253961in}{1.391583in}}{\pgfqpoint{9.250394in}{1.388017in}}%
\pgfpathcurveto{\pgfqpoint{9.246828in}{1.384450in}}{\pgfqpoint{9.244824in}{1.379613in}}{\pgfqpoint{9.244824in}{1.374569in}}%
\pgfpathcurveto{\pgfqpoint{9.244824in}{1.369525in}}{\pgfqpoint{9.246828in}{1.364688in}}{\pgfqpoint{9.250394in}{1.361121in}}%
\pgfpathcurveto{\pgfqpoint{9.253961in}{1.357555in}}{\pgfqpoint{9.258799in}{1.355551in}}{\pgfqpoint{9.263842in}{1.355551in}}%
\pgfpathclose%
\pgfusepath{fill}%
\end{pgfscope}%
\begin{pgfscope}%
\pgfpathrectangle{\pgfqpoint{6.572727in}{0.473000in}}{\pgfqpoint{4.227273in}{3.311000in}}%
\pgfusepath{clip}%
\pgfsetbuttcap%
\pgfsetroundjoin%
\definecolor{currentfill}{rgb}{0.993248,0.906157,0.143936}%
\pgfsetfillcolor{currentfill}%
\pgfsetfillopacity{0.700000}%
\pgfsetlinewidth{0.000000pt}%
\definecolor{currentstroke}{rgb}{0.000000,0.000000,0.000000}%
\pgfsetstrokecolor{currentstroke}%
\pgfsetstrokeopacity{0.700000}%
\pgfsetdash{}{0pt}%
\pgfpathmoveto{\pgfqpoint{9.568155in}{1.584917in}}%
\pgfpathcurveto{\pgfqpoint{9.573199in}{1.584917in}}{\pgfqpoint{9.578036in}{1.586921in}}{\pgfqpoint{9.581603in}{1.590487in}}%
\pgfpathcurveto{\pgfqpoint{9.585169in}{1.594054in}}{\pgfqpoint{9.587173in}{1.598891in}}{\pgfqpoint{9.587173in}{1.603935in}}%
\pgfpathcurveto{\pgfqpoint{9.587173in}{1.608979in}}{\pgfqpoint{9.585169in}{1.613816in}}{\pgfqpoint{9.581603in}{1.617383in}}%
\pgfpathcurveto{\pgfqpoint{9.578036in}{1.620949in}}{\pgfqpoint{9.573199in}{1.622953in}}{\pgfqpoint{9.568155in}{1.622953in}}%
\pgfpathcurveto{\pgfqpoint{9.563111in}{1.622953in}}{\pgfqpoint{9.558273in}{1.620949in}}{\pgfqpoint{9.554707in}{1.617383in}}%
\pgfpathcurveto{\pgfqpoint{9.551141in}{1.613816in}}{\pgfqpoint{9.549137in}{1.608979in}}{\pgfqpoint{9.549137in}{1.603935in}}%
\pgfpathcurveto{\pgfqpoint{9.549137in}{1.598891in}}{\pgfqpoint{9.551141in}{1.594054in}}{\pgfqpoint{9.554707in}{1.590487in}}%
\pgfpathcurveto{\pgfqpoint{9.558273in}{1.586921in}}{\pgfqpoint{9.563111in}{1.584917in}}{\pgfqpoint{9.568155in}{1.584917in}}%
\pgfpathclose%
\pgfusepath{fill}%
\end{pgfscope}%
\begin{pgfscope}%
\pgfpathrectangle{\pgfqpoint{6.572727in}{0.473000in}}{\pgfqpoint{4.227273in}{3.311000in}}%
\pgfusepath{clip}%
\pgfsetbuttcap%
\pgfsetroundjoin%
\definecolor{currentfill}{rgb}{0.993248,0.906157,0.143936}%
\pgfsetfillcolor{currentfill}%
\pgfsetfillopacity{0.700000}%
\pgfsetlinewidth{0.000000pt}%
\definecolor{currentstroke}{rgb}{0.000000,0.000000,0.000000}%
\pgfsetstrokecolor{currentstroke}%
\pgfsetstrokeopacity{0.700000}%
\pgfsetdash{}{0pt}%
\pgfpathmoveto{\pgfqpoint{9.920171in}{1.151969in}}%
\pgfpathcurveto{\pgfqpoint{9.925215in}{1.151969in}}{\pgfqpoint{9.930053in}{1.153973in}}{\pgfqpoint{9.933619in}{1.157540in}}%
\pgfpathcurveto{\pgfqpoint{9.937186in}{1.161106in}}{\pgfqpoint{9.939190in}{1.165944in}}{\pgfqpoint{9.939190in}{1.170987in}}%
\pgfpathcurveto{\pgfqpoint{9.939190in}{1.176031in}}{\pgfqpoint{9.937186in}{1.180869in}}{\pgfqpoint{9.933619in}{1.184435in}}%
\pgfpathcurveto{\pgfqpoint{9.930053in}{1.188002in}}{\pgfqpoint{9.925215in}{1.190006in}}{\pgfqpoint{9.920171in}{1.190006in}}%
\pgfpathcurveto{\pgfqpoint{9.915128in}{1.190006in}}{\pgfqpoint{9.910290in}{1.188002in}}{\pgfqpoint{9.906724in}{1.184435in}}%
\pgfpathcurveto{\pgfqpoint{9.903157in}{1.180869in}}{\pgfqpoint{9.901153in}{1.176031in}}{\pgfqpoint{9.901153in}{1.170987in}}%
\pgfpathcurveto{\pgfqpoint{9.901153in}{1.165944in}}{\pgfqpoint{9.903157in}{1.161106in}}{\pgfqpoint{9.906724in}{1.157540in}}%
\pgfpathcurveto{\pgfqpoint{9.910290in}{1.153973in}}{\pgfqpoint{9.915128in}{1.151969in}}{\pgfqpoint{9.920171in}{1.151969in}}%
\pgfpathclose%
\pgfusepath{fill}%
\end{pgfscope}%
\begin{pgfscope}%
\pgfpathrectangle{\pgfqpoint{6.572727in}{0.473000in}}{\pgfqpoint{4.227273in}{3.311000in}}%
\pgfusepath{clip}%
\pgfsetbuttcap%
\pgfsetroundjoin%
\definecolor{currentfill}{rgb}{0.127568,0.566949,0.550556}%
\pgfsetfillcolor{currentfill}%
\pgfsetfillopacity{0.700000}%
\pgfsetlinewidth{0.000000pt}%
\definecolor{currentstroke}{rgb}{0.000000,0.000000,0.000000}%
\pgfsetstrokecolor{currentstroke}%
\pgfsetstrokeopacity{0.700000}%
\pgfsetdash{}{0pt}%
\pgfpathmoveto{\pgfqpoint{7.993456in}{3.362594in}}%
\pgfpathcurveto{\pgfqpoint{7.998500in}{3.362594in}}{\pgfqpoint{8.003338in}{3.364598in}}{\pgfqpoint{8.006904in}{3.368164in}}%
\pgfpathcurveto{\pgfqpoint{8.010470in}{3.371731in}}{\pgfqpoint{8.012474in}{3.376568in}}{\pgfqpoint{8.012474in}{3.381612in}}%
\pgfpathcurveto{\pgfqpoint{8.012474in}{3.386656in}}{\pgfqpoint{8.010470in}{3.391494in}}{\pgfqpoint{8.006904in}{3.395060in}}%
\pgfpathcurveto{\pgfqpoint{8.003338in}{3.398626in}}{\pgfqpoint{7.998500in}{3.400630in}}{\pgfqpoint{7.993456in}{3.400630in}}%
\pgfpathcurveto{\pgfqpoint{7.988412in}{3.400630in}}{\pgfqpoint{7.983575in}{3.398626in}}{\pgfqpoint{7.980008in}{3.395060in}}%
\pgfpathcurveto{\pgfqpoint{7.976442in}{3.391494in}}{\pgfqpoint{7.974438in}{3.386656in}}{\pgfqpoint{7.974438in}{3.381612in}}%
\pgfpathcurveto{\pgfqpoint{7.974438in}{3.376568in}}{\pgfqpoint{7.976442in}{3.371731in}}{\pgfqpoint{7.980008in}{3.368164in}}%
\pgfpathcurveto{\pgfqpoint{7.983575in}{3.364598in}}{\pgfqpoint{7.988412in}{3.362594in}}{\pgfqpoint{7.993456in}{3.362594in}}%
\pgfpathclose%
\pgfusepath{fill}%
\end{pgfscope}%
\begin{pgfscope}%
\pgfpathrectangle{\pgfqpoint{6.572727in}{0.473000in}}{\pgfqpoint{4.227273in}{3.311000in}}%
\pgfusepath{clip}%
\pgfsetbuttcap%
\pgfsetroundjoin%
\definecolor{currentfill}{rgb}{0.127568,0.566949,0.550556}%
\pgfsetfillcolor{currentfill}%
\pgfsetfillopacity{0.700000}%
\pgfsetlinewidth{0.000000pt}%
\definecolor{currentstroke}{rgb}{0.000000,0.000000,0.000000}%
\pgfsetstrokecolor{currentstroke}%
\pgfsetstrokeopacity{0.700000}%
\pgfsetdash{}{0pt}%
\pgfpathmoveto{\pgfqpoint{7.379559in}{1.542448in}}%
\pgfpathcurveto{\pgfqpoint{7.384602in}{1.542448in}}{\pgfqpoint{7.389440in}{1.544452in}}{\pgfqpoint{7.393007in}{1.548018in}}%
\pgfpathcurveto{\pgfqpoint{7.396573in}{1.551585in}}{\pgfqpoint{7.398577in}{1.556423in}}{\pgfqpoint{7.398577in}{1.561466in}}%
\pgfpathcurveto{\pgfqpoint{7.398577in}{1.566510in}}{\pgfqpoint{7.396573in}{1.571348in}}{\pgfqpoint{7.393007in}{1.574914in}}%
\pgfpathcurveto{\pgfqpoint{7.389440in}{1.578481in}}{\pgfqpoint{7.384602in}{1.580484in}}{\pgfqpoint{7.379559in}{1.580484in}}%
\pgfpathcurveto{\pgfqpoint{7.374515in}{1.580484in}}{\pgfqpoint{7.369677in}{1.578481in}}{\pgfqpoint{7.366111in}{1.574914in}}%
\pgfpathcurveto{\pgfqpoint{7.362544in}{1.571348in}}{\pgfqpoint{7.360541in}{1.566510in}}{\pgfqpoint{7.360541in}{1.561466in}}%
\pgfpathcurveto{\pgfqpoint{7.360541in}{1.556423in}}{\pgfqpoint{7.362544in}{1.551585in}}{\pgfqpoint{7.366111in}{1.548018in}}%
\pgfpathcurveto{\pgfqpoint{7.369677in}{1.544452in}}{\pgfqpoint{7.374515in}{1.542448in}}{\pgfqpoint{7.379559in}{1.542448in}}%
\pgfpathclose%
\pgfusepath{fill}%
\end{pgfscope}%
\begin{pgfscope}%
\pgfpathrectangle{\pgfqpoint{6.572727in}{0.473000in}}{\pgfqpoint{4.227273in}{3.311000in}}%
\pgfusepath{clip}%
\pgfsetbuttcap%
\pgfsetroundjoin%
\definecolor{currentfill}{rgb}{0.127568,0.566949,0.550556}%
\pgfsetfillcolor{currentfill}%
\pgfsetfillopacity{0.700000}%
\pgfsetlinewidth{0.000000pt}%
\definecolor{currentstroke}{rgb}{0.000000,0.000000,0.000000}%
\pgfsetstrokecolor{currentstroke}%
\pgfsetstrokeopacity{0.700000}%
\pgfsetdash{}{0pt}%
\pgfpathmoveto{\pgfqpoint{8.224182in}{2.806918in}}%
\pgfpathcurveto{\pgfqpoint{8.229226in}{2.806918in}}{\pgfqpoint{8.234064in}{2.808922in}}{\pgfqpoint{8.237630in}{2.812489in}}%
\pgfpathcurveto{\pgfqpoint{8.241197in}{2.816055in}}{\pgfqpoint{8.243201in}{2.820893in}}{\pgfqpoint{8.243201in}{2.825937in}}%
\pgfpathcurveto{\pgfqpoint{8.243201in}{2.830980in}}{\pgfqpoint{8.241197in}{2.835818in}}{\pgfqpoint{8.237630in}{2.839384in}}%
\pgfpathcurveto{\pgfqpoint{8.234064in}{2.842951in}}{\pgfqpoint{8.229226in}{2.844955in}}{\pgfqpoint{8.224182in}{2.844955in}}%
\pgfpathcurveto{\pgfqpoint{8.219139in}{2.844955in}}{\pgfqpoint{8.214301in}{2.842951in}}{\pgfqpoint{8.210735in}{2.839384in}}%
\pgfpathcurveto{\pgfqpoint{8.207168in}{2.835818in}}{\pgfqpoint{8.205164in}{2.830980in}}{\pgfqpoint{8.205164in}{2.825937in}}%
\pgfpathcurveto{\pgfqpoint{8.205164in}{2.820893in}}{\pgfqpoint{8.207168in}{2.816055in}}{\pgfqpoint{8.210735in}{2.812489in}}%
\pgfpathcurveto{\pgfqpoint{8.214301in}{2.808922in}}{\pgfqpoint{8.219139in}{2.806918in}}{\pgfqpoint{8.224182in}{2.806918in}}%
\pgfpathclose%
\pgfusepath{fill}%
\end{pgfscope}%
\begin{pgfscope}%
\pgfpathrectangle{\pgfqpoint{6.572727in}{0.473000in}}{\pgfqpoint{4.227273in}{3.311000in}}%
\pgfusepath{clip}%
\pgfsetbuttcap%
\pgfsetroundjoin%
\definecolor{currentfill}{rgb}{0.993248,0.906157,0.143936}%
\pgfsetfillcolor{currentfill}%
\pgfsetfillopacity{0.700000}%
\pgfsetlinewidth{0.000000pt}%
\definecolor{currentstroke}{rgb}{0.000000,0.000000,0.000000}%
\pgfsetstrokecolor{currentstroke}%
\pgfsetstrokeopacity{0.700000}%
\pgfsetdash{}{0pt}%
\pgfpathmoveto{\pgfqpoint{9.570389in}{1.142085in}}%
\pgfpathcurveto{\pgfqpoint{9.575432in}{1.142085in}}{\pgfqpoint{9.580270in}{1.144089in}}{\pgfqpoint{9.583836in}{1.147656in}}%
\pgfpathcurveto{\pgfqpoint{9.587403in}{1.151222in}}{\pgfqpoint{9.589407in}{1.156060in}}{\pgfqpoint{9.589407in}{1.161104in}}%
\pgfpathcurveto{\pgfqpoint{9.589407in}{1.166147in}}{\pgfqpoint{9.587403in}{1.170985in}}{\pgfqpoint{9.583836in}{1.174551in}}%
\pgfpathcurveto{\pgfqpoint{9.580270in}{1.178118in}}{\pgfqpoint{9.575432in}{1.180122in}}{\pgfqpoint{9.570389in}{1.180122in}}%
\pgfpathcurveto{\pgfqpoint{9.565345in}{1.180122in}}{\pgfqpoint{9.560507in}{1.178118in}}{\pgfqpoint{9.556941in}{1.174551in}}%
\pgfpathcurveto{\pgfqpoint{9.553374in}{1.170985in}}{\pgfqpoint{9.551370in}{1.166147in}}{\pgfqpoint{9.551370in}{1.161104in}}%
\pgfpathcurveto{\pgfqpoint{9.551370in}{1.156060in}}{\pgfqpoint{9.553374in}{1.151222in}}{\pgfqpoint{9.556941in}{1.147656in}}%
\pgfpathcurveto{\pgfqpoint{9.560507in}{1.144089in}}{\pgfqpoint{9.565345in}{1.142085in}}{\pgfqpoint{9.570389in}{1.142085in}}%
\pgfpathclose%
\pgfusepath{fill}%
\end{pgfscope}%
\begin{pgfscope}%
\pgfpathrectangle{\pgfqpoint{6.572727in}{0.473000in}}{\pgfqpoint{4.227273in}{3.311000in}}%
\pgfusepath{clip}%
\pgfsetbuttcap%
\pgfsetroundjoin%
\definecolor{currentfill}{rgb}{0.993248,0.906157,0.143936}%
\pgfsetfillcolor{currentfill}%
\pgfsetfillopacity{0.700000}%
\pgfsetlinewidth{0.000000pt}%
\definecolor{currentstroke}{rgb}{0.000000,0.000000,0.000000}%
\pgfsetstrokecolor{currentstroke}%
\pgfsetstrokeopacity{0.700000}%
\pgfsetdash{}{0pt}%
\pgfpathmoveto{\pgfqpoint{9.253134in}{1.277944in}}%
\pgfpathcurveto{\pgfqpoint{9.258177in}{1.277944in}}{\pgfqpoint{9.263015in}{1.279948in}}{\pgfqpoint{9.266581in}{1.283514in}}%
\pgfpathcurveto{\pgfqpoint{9.270148in}{1.287081in}}{\pgfqpoint{9.272152in}{1.291918in}}{\pgfqpoint{9.272152in}{1.296962in}}%
\pgfpathcurveto{\pgfqpoint{9.272152in}{1.302006in}}{\pgfqpoint{9.270148in}{1.306844in}}{\pgfqpoint{9.266581in}{1.310410in}}%
\pgfpathcurveto{\pgfqpoint{9.263015in}{1.313976in}}{\pgfqpoint{9.258177in}{1.315980in}}{\pgfqpoint{9.253134in}{1.315980in}}%
\pgfpathcurveto{\pgfqpoint{9.248090in}{1.315980in}}{\pgfqpoint{9.243252in}{1.313976in}}{\pgfqpoint{9.239686in}{1.310410in}}%
\pgfpathcurveto{\pgfqpoint{9.236119in}{1.306844in}}{\pgfqpoint{9.234115in}{1.302006in}}{\pgfqpoint{9.234115in}{1.296962in}}%
\pgfpathcurveto{\pgfqpoint{9.234115in}{1.291918in}}{\pgfqpoint{9.236119in}{1.287081in}}{\pgfqpoint{9.239686in}{1.283514in}}%
\pgfpathcurveto{\pgfqpoint{9.243252in}{1.279948in}}{\pgfqpoint{9.248090in}{1.277944in}}{\pgfqpoint{9.253134in}{1.277944in}}%
\pgfpathclose%
\pgfusepath{fill}%
\end{pgfscope}%
\begin{pgfscope}%
\pgfpathrectangle{\pgfqpoint{6.572727in}{0.473000in}}{\pgfqpoint{4.227273in}{3.311000in}}%
\pgfusepath{clip}%
\pgfsetbuttcap%
\pgfsetroundjoin%
\definecolor{currentfill}{rgb}{0.127568,0.566949,0.550556}%
\pgfsetfillcolor{currentfill}%
\pgfsetfillopacity{0.700000}%
\pgfsetlinewidth{0.000000pt}%
\definecolor{currentstroke}{rgb}{0.000000,0.000000,0.000000}%
\pgfsetstrokecolor{currentstroke}%
\pgfsetstrokeopacity{0.700000}%
\pgfsetdash{}{0pt}%
\pgfpathmoveto{\pgfqpoint{8.658814in}{2.446085in}}%
\pgfpathcurveto{\pgfqpoint{8.663858in}{2.446085in}}{\pgfqpoint{8.668696in}{2.448089in}}{\pgfqpoint{8.672262in}{2.451655in}}%
\pgfpathcurveto{\pgfqpoint{8.675829in}{2.455221in}}{\pgfqpoint{8.677833in}{2.460059in}}{\pgfqpoint{8.677833in}{2.465103in}}%
\pgfpathcurveto{\pgfqpoint{8.677833in}{2.470146in}}{\pgfqpoint{8.675829in}{2.474984in}}{\pgfqpoint{8.672262in}{2.478551in}}%
\pgfpathcurveto{\pgfqpoint{8.668696in}{2.482117in}}{\pgfqpoint{8.663858in}{2.484121in}}{\pgfqpoint{8.658814in}{2.484121in}}%
\pgfpathcurveto{\pgfqpoint{8.653771in}{2.484121in}}{\pgfqpoint{8.648933in}{2.482117in}}{\pgfqpoint{8.645367in}{2.478551in}}%
\pgfpathcurveto{\pgfqpoint{8.641800in}{2.474984in}}{\pgfqpoint{8.639796in}{2.470146in}}{\pgfqpoint{8.639796in}{2.465103in}}%
\pgfpathcurveto{\pgfqpoint{8.639796in}{2.460059in}}{\pgfqpoint{8.641800in}{2.455221in}}{\pgfqpoint{8.645367in}{2.451655in}}%
\pgfpathcurveto{\pgfqpoint{8.648933in}{2.448089in}}{\pgfqpoint{8.653771in}{2.446085in}}{\pgfqpoint{8.658814in}{2.446085in}}%
\pgfpathclose%
\pgfusepath{fill}%
\end{pgfscope}%
\begin{pgfscope}%
\pgfpathrectangle{\pgfqpoint{6.572727in}{0.473000in}}{\pgfqpoint{4.227273in}{3.311000in}}%
\pgfusepath{clip}%
\pgfsetbuttcap%
\pgfsetroundjoin%
\definecolor{currentfill}{rgb}{0.127568,0.566949,0.550556}%
\pgfsetfillcolor{currentfill}%
\pgfsetfillopacity{0.700000}%
\pgfsetlinewidth{0.000000pt}%
\definecolor{currentstroke}{rgb}{0.000000,0.000000,0.000000}%
\pgfsetstrokecolor{currentstroke}%
\pgfsetstrokeopacity{0.700000}%
\pgfsetdash{}{0pt}%
\pgfpathmoveto{\pgfqpoint{7.867878in}{0.991183in}}%
\pgfpathcurveto{\pgfqpoint{7.872922in}{0.991183in}}{\pgfqpoint{7.877760in}{0.993187in}}{\pgfqpoint{7.881326in}{0.996754in}}%
\pgfpathcurveto{\pgfqpoint{7.884892in}{1.000320in}}{\pgfqpoint{7.886896in}{1.005158in}}{\pgfqpoint{7.886896in}{1.010202in}}%
\pgfpathcurveto{\pgfqpoint{7.886896in}{1.015245in}}{\pgfqpoint{7.884892in}{1.020083in}}{\pgfqpoint{7.881326in}{1.023649in}}%
\pgfpathcurveto{\pgfqpoint{7.877760in}{1.027216in}}{\pgfqpoint{7.872922in}{1.029220in}}{\pgfqpoint{7.867878in}{1.029220in}}%
\pgfpathcurveto{\pgfqpoint{7.862834in}{1.029220in}}{\pgfqpoint{7.857997in}{1.027216in}}{\pgfqpoint{7.854430in}{1.023649in}}%
\pgfpathcurveto{\pgfqpoint{7.850864in}{1.020083in}}{\pgfqpoint{7.848860in}{1.015245in}}{\pgfqpoint{7.848860in}{1.010202in}}%
\pgfpathcurveto{\pgfqpoint{7.848860in}{1.005158in}}{\pgfqpoint{7.850864in}{1.000320in}}{\pgfqpoint{7.854430in}{0.996754in}}%
\pgfpathcurveto{\pgfqpoint{7.857997in}{0.993187in}}{\pgfqpoint{7.862834in}{0.991183in}}{\pgfqpoint{7.867878in}{0.991183in}}%
\pgfpathclose%
\pgfusepath{fill}%
\end{pgfscope}%
\begin{pgfscope}%
\pgfpathrectangle{\pgfqpoint{6.572727in}{0.473000in}}{\pgfqpoint{4.227273in}{3.311000in}}%
\pgfusepath{clip}%
\pgfsetbuttcap%
\pgfsetroundjoin%
\definecolor{currentfill}{rgb}{0.993248,0.906157,0.143936}%
\pgfsetfillcolor{currentfill}%
\pgfsetfillopacity{0.700000}%
\pgfsetlinewidth{0.000000pt}%
\definecolor{currentstroke}{rgb}{0.000000,0.000000,0.000000}%
\pgfsetstrokecolor{currentstroke}%
\pgfsetstrokeopacity{0.700000}%
\pgfsetdash{}{0pt}%
\pgfpathmoveto{\pgfqpoint{9.600641in}{1.314915in}}%
\pgfpathcurveto{\pgfqpoint{9.605684in}{1.314915in}}{\pgfqpoint{9.610522in}{1.316919in}}{\pgfqpoint{9.614088in}{1.320486in}}%
\pgfpathcurveto{\pgfqpoint{9.617655in}{1.324052in}}{\pgfqpoint{9.619659in}{1.328890in}}{\pgfqpoint{9.619659in}{1.333934in}}%
\pgfpathcurveto{\pgfqpoint{9.619659in}{1.338977in}}{\pgfqpoint{9.617655in}{1.343815in}}{\pgfqpoint{9.614088in}{1.347381in}}%
\pgfpathcurveto{\pgfqpoint{9.610522in}{1.350948in}}{\pgfqpoint{9.605684in}{1.352952in}}{\pgfqpoint{9.600641in}{1.352952in}}%
\pgfpathcurveto{\pgfqpoint{9.595597in}{1.352952in}}{\pgfqpoint{9.590759in}{1.350948in}}{\pgfqpoint{9.587193in}{1.347381in}}%
\pgfpathcurveto{\pgfqpoint{9.583626in}{1.343815in}}{\pgfqpoint{9.581622in}{1.338977in}}{\pgfqpoint{9.581622in}{1.333934in}}%
\pgfpathcurveto{\pgfqpoint{9.581622in}{1.328890in}}{\pgfqpoint{9.583626in}{1.324052in}}{\pgfqpoint{9.587193in}{1.320486in}}%
\pgfpathcurveto{\pgfqpoint{9.590759in}{1.316919in}}{\pgfqpoint{9.595597in}{1.314915in}}{\pgfqpoint{9.600641in}{1.314915in}}%
\pgfpathclose%
\pgfusepath{fill}%
\end{pgfscope}%
\begin{pgfscope}%
\pgfpathrectangle{\pgfqpoint{6.572727in}{0.473000in}}{\pgfqpoint{4.227273in}{3.311000in}}%
\pgfusepath{clip}%
\pgfsetbuttcap%
\pgfsetroundjoin%
\definecolor{currentfill}{rgb}{0.127568,0.566949,0.550556}%
\pgfsetfillcolor{currentfill}%
\pgfsetfillopacity{0.700000}%
\pgfsetlinewidth{0.000000pt}%
\definecolor{currentstroke}{rgb}{0.000000,0.000000,0.000000}%
\pgfsetstrokecolor{currentstroke}%
\pgfsetstrokeopacity{0.700000}%
\pgfsetdash{}{0pt}%
\pgfpathmoveto{\pgfqpoint{7.764700in}{3.362119in}}%
\pgfpathcurveto{\pgfqpoint{7.769744in}{3.362119in}}{\pgfqpoint{7.774582in}{3.364123in}}{\pgfqpoint{7.778148in}{3.367689in}}%
\pgfpathcurveto{\pgfqpoint{7.781714in}{3.371256in}}{\pgfqpoint{7.783718in}{3.376093in}}{\pgfqpoint{7.783718in}{3.381137in}}%
\pgfpathcurveto{\pgfqpoint{7.783718in}{3.386181in}}{\pgfqpoint{7.781714in}{3.391018in}}{\pgfqpoint{7.778148in}{3.394585in}}%
\pgfpathcurveto{\pgfqpoint{7.774582in}{3.398151in}}{\pgfqpoint{7.769744in}{3.400155in}}{\pgfqpoint{7.764700in}{3.400155in}}%
\pgfpathcurveto{\pgfqpoint{7.759657in}{3.400155in}}{\pgfqpoint{7.754819in}{3.398151in}}{\pgfqpoint{7.751252in}{3.394585in}}%
\pgfpathcurveto{\pgfqpoint{7.747686in}{3.391018in}}{\pgfqpoint{7.745682in}{3.386181in}}{\pgfqpoint{7.745682in}{3.381137in}}%
\pgfpathcurveto{\pgfqpoint{7.745682in}{3.376093in}}{\pgfqpoint{7.747686in}{3.371256in}}{\pgfqpoint{7.751252in}{3.367689in}}%
\pgfpathcurveto{\pgfqpoint{7.754819in}{3.364123in}}{\pgfqpoint{7.759657in}{3.362119in}}{\pgfqpoint{7.764700in}{3.362119in}}%
\pgfpathclose%
\pgfusepath{fill}%
\end{pgfscope}%
\begin{pgfscope}%
\pgfpathrectangle{\pgfqpoint{6.572727in}{0.473000in}}{\pgfqpoint{4.227273in}{3.311000in}}%
\pgfusepath{clip}%
\pgfsetbuttcap%
\pgfsetroundjoin%
\definecolor{currentfill}{rgb}{0.993248,0.906157,0.143936}%
\pgfsetfillcolor{currentfill}%
\pgfsetfillopacity{0.700000}%
\pgfsetlinewidth{0.000000pt}%
\definecolor{currentstroke}{rgb}{0.000000,0.000000,0.000000}%
\pgfsetstrokecolor{currentstroke}%
\pgfsetstrokeopacity{0.700000}%
\pgfsetdash{}{0pt}%
\pgfpathmoveto{\pgfqpoint{10.262451in}{1.454611in}}%
\pgfpathcurveto{\pgfqpoint{10.267495in}{1.454611in}}{\pgfqpoint{10.272332in}{1.456615in}}{\pgfqpoint{10.275899in}{1.460181in}}%
\pgfpathcurveto{\pgfqpoint{10.279465in}{1.463748in}}{\pgfqpoint{10.281469in}{1.468585in}}{\pgfqpoint{10.281469in}{1.473629in}}%
\pgfpathcurveto{\pgfqpoint{10.281469in}{1.478673in}}{\pgfqpoint{10.279465in}{1.483511in}}{\pgfqpoint{10.275899in}{1.487077in}}%
\pgfpathcurveto{\pgfqpoint{10.272332in}{1.490643in}}{\pgfqpoint{10.267495in}{1.492647in}}{\pgfqpoint{10.262451in}{1.492647in}}%
\pgfpathcurveto{\pgfqpoint{10.257407in}{1.492647in}}{\pgfqpoint{10.252570in}{1.490643in}}{\pgfqpoint{10.249003in}{1.487077in}}%
\pgfpathcurveto{\pgfqpoint{10.245437in}{1.483511in}}{\pgfqpoint{10.243433in}{1.478673in}}{\pgfqpoint{10.243433in}{1.473629in}}%
\pgfpathcurveto{\pgfqpoint{10.243433in}{1.468585in}}{\pgfqpoint{10.245437in}{1.463748in}}{\pgfqpoint{10.249003in}{1.460181in}}%
\pgfpathcurveto{\pgfqpoint{10.252570in}{1.456615in}}{\pgfqpoint{10.257407in}{1.454611in}}{\pgfqpoint{10.262451in}{1.454611in}}%
\pgfpathclose%
\pgfusepath{fill}%
\end{pgfscope}%
\begin{pgfscope}%
\pgfpathrectangle{\pgfqpoint{6.572727in}{0.473000in}}{\pgfqpoint{4.227273in}{3.311000in}}%
\pgfusepath{clip}%
\pgfsetbuttcap%
\pgfsetroundjoin%
\definecolor{currentfill}{rgb}{0.993248,0.906157,0.143936}%
\pgfsetfillcolor{currentfill}%
\pgfsetfillopacity{0.700000}%
\pgfsetlinewidth{0.000000pt}%
\definecolor{currentstroke}{rgb}{0.000000,0.000000,0.000000}%
\pgfsetstrokecolor{currentstroke}%
\pgfsetstrokeopacity{0.700000}%
\pgfsetdash{}{0pt}%
\pgfpathmoveto{\pgfqpoint{9.255775in}{1.305181in}}%
\pgfpathcurveto{\pgfqpoint{9.260819in}{1.305181in}}{\pgfqpoint{9.265656in}{1.307185in}}{\pgfqpoint{9.269223in}{1.310751in}}%
\pgfpathcurveto{\pgfqpoint{9.272789in}{1.314318in}}{\pgfqpoint{9.274793in}{1.319156in}}{\pgfqpoint{9.274793in}{1.324199in}}%
\pgfpathcurveto{\pgfqpoint{9.274793in}{1.329243in}}{\pgfqpoint{9.272789in}{1.334081in}}{\pgfqpoint{9.269223in}{1.337647in}}%
\pgfpathcurveto{\pgfqpoint{9.265656in}{1.341214in}}{\pgfqpoint{9.260819in}{1.343217in}}{\pgfqpoint{9.255775in}{1.343217in}}%
\pgfpathcurveto{\pgfqpoint{9.250731in}{1.343217in}}{\pgfqpoint{9.245894in}{1.341214in}}{\pgfqpoint{9.242327in}{1.337647in}}%
\pgfpathcurveto{\pgfqpoint{9.238761in}{1.334081in}}{\pgfqpoint{9.236757in}{1.329243in}}{\pgfqpoint{9.236757in}{1.324199in}}%
\pgfpathcurveto{\pgfqpoint{9.236757in}{1.319156in}}{\pgfqpoint{9.238761in}{1.314318in}}{\pgfqpoint{9.242327in}{1.310751in}}%
\pgfpathcurveto{\pgfqpoint{9.245894in}{1.307185in}}{\pgfqpoint{9.250731in}{1.305181in}}{\pgfqpoint{9.255775in}{1.305181in}}%
\pgfpathclose%
\pgfusepath{fill}%
\end{pgfscope}%
\begin{pgfscope}%
\pgfpathrectangle{\pgfqpoint{6.572727in}{0.473000in}}{\pgfqpoint{4.227273in}{3.311000in}}%
\pgfusepath{clip}%
\pgfsetbuttcap%
\pgfsetroundjoin%
\definecolor{currentfill}{rgb}{0.127568,0.566949,0.550556}%
\pgfsetfillcolor{currentfill}%
\pgfsetfillopacity{0.700000}%
\pgfsetlinewidth{0.000000pt}%
\definecolor{currentstroke}{rgb}{0.000000,0.000000,0.000000}%
\pgfsetstrokecolor{currentstroke}%
\pgfsetstrokeopacity{0.700000}%
\pgfsetdash{}{0pt}%
\pgfpathmoveto{\pgfqpoint{7.691415in}{1.436357in}}%
\pgfpathcurveto{\pgfqpoint{7.696459in}{1.436357in}}{\pgfqpoint{7.701297in}{1.438361in}}{\pgfqpoint{7.704863in}{1.441928in}}%
\pgfpathcurveto{\pgfqpoint{7.708430in}{1.445494in}}{\pgfqpoint{7.710433in}{1.450332in}}{\pgfqpoint{7.710433in}{1.455375in}}%
\pgfpathcurveto{\pgfqpoint{7.710433in}{1.460419in}}{\pgfqpoint{7.708430in}{1.465257in}}{\pgfqpoint{7.704863in}{1.468823in}}%
\pgfpathcurveto{\pgfqpoint{7.701297in}{1.472390in}}{\pgfqpoint{7.696459in}{1.474394in}}{\pgfqpoint{7.691415in}{1.474394in}}%
\pgfpathcurveto{\pgfqpoint{7.686372in}{1.474394in}}{\pgfqpoint{7.681534in}{1.472390in}}{\pgfqpoint{7.677967in}{1.468823in}}%
\pgfpathcurveto{\pgfqpoint{7.674401in}{1.465257in}}{\pgfqpoint{7.672397in}{1.460419in}}{\pgfqpoint{7.672397in}{1.455375in}}%
\pgfpathcurveto{\pgfqpoint{7.672397in}{1.450332in}}{\pgfqpoint{7.674401in}{1.445494in}}{\pgfqpoint{7.677967in}{1.441928in}}%
\pgfpathcurveto{\pgfqpoint{7.681534in}{1.438361in}}{\pgfqpoint{7.686372in}{1.436357in}}{\pgfqpoint{7.691415in}{1.436357in}}%
\pgfpathclose%
\pgfusepath{fill}%
\end{pgfscope}%
\begin{pgfscope}%
\pgfpathrectangle{\pgfqpoint{6.572727in}{0.473000in}}{\pgfqpoint{4.227273in}{3.311000in}}%
\pgfusepath{clip}%
\pgfsetbuttcap%
\pgfsetroundjoin%
\definecolor{currentfill}{rgb}{0.127568,0.566949,0.550556}%
\pgfsetfillcolor{currentfill}%
\pgfsetfillopacity{0.700000}%
\pgfsetlinewidth{0.000000pt}%
\definecolor{currentstroke}{rgb}{0.000000,0.000000,0.000000}%
\pgfsetstrokecolor{currentstroke}%
\pgfsetstrokeopacity{0.700000}%
\pgfsetdash{}{0pt}%
\pgfpathmoveto{\pgfqpoint{7.949581in}{2.576689in}}%
\pgfpathcurveto{\pgfqpoint{7.954624in}{2.576689in}}{\pgfqpoint{7.959462in}{2.578693in}}{\pgfqpoint{7.963029in}{2.582259in}}%
\pgfpathcurveto{\pgfqpoint{7.966595in}{2.585826in}}{\pgfqpoint{7.968599in}{2.590664in}}{\pgfqpoint{7.968599in}{2.595707in}}%
\pgfpathcurveto{\pgfqpoint{7.968599in}{2.600751in}}{\pgfqpoint{7.966595in}{2.605589in}}{\pgfqpoint{7.963029in}{2.609155in}}%
\pgfpathcurveto{\pgfqpoint{7.959462in}{2.612721in}}{\pgfqpoint{7.954624in}{2.614725in}}{\pgfqpoint{7.949581in}{2.614725in}}%
\pgfpathcurveto{\pgfqpoint{7.944537in}{2.614725in}}{\pgfqpoint{7.939699in}{2.612721in}}{\pgfqpoint{7.936133in}{2.609155in}}%
\pgfpathcurveto{\pgfqpoint{7.932566in}{2.605589in}}{\pgfqpoint{7.930563in}{2.600751in}}{\pgfqpoint{7.930563in}{2.595707in}}%
\pgfpathcurveto{\pgfqpoint{7.930563in}{2.590664in}}{\pgfqpoint{7.932566in}{2.585826in}}{\pgfqpoint{7.936133in}{2.582259in}}%
\pgfpathcurveto{\pgfqpoint{7.939699in}{2.578693in}}{\pgfqpoint{7.944537in}{2.576689in}}{\pgfqpoint{7.949581in}{2.576689in}}%
\pgfpathclose%
\pgfusepath{fill}%
\end{pgfscope}%
\begin{pgfscope}%
\pgfpathrectangle{\pgfqpoint{6.572727in}{0.473000in}}{\pgfqpoint{4.227273in}{3.311000in}}%
\pgfusepath{clip}%
\pgfsetbuttcap%
\pgfsetroundjoin%
\definecolor{currentfill}{rgb}{0.127568,0.566949,0.550556}%
\pgfsetfillcolor{currentfill}%
\pgfsetfillopacity{0.700000}%
\pgfsetlinewidth{0.000000pt}%
\definecolor{currentstroke}{rgb}{0.000000,0.000000,0.000000}%
\pgfsetstrokecolor{currentstroke}%
\pgfsetstrokeopacity{0.700000}%
\pgfsetdash{}{0pt}%
\pgfpathmoveto{\pgfqpoint{7.651143in}{1.866642in}}%
\pgfpathcurveto{\pgfqpoint{7.656187in}{1.866642in}}{\pgfqpoint{7.661025in}{1.868646in}}{\pgfqpoint{7.664591in}{1.872213in}}%
\pgfpathcurveto{\pgfqpoint{7.668157in}{1.875779in}}{\pgfqpoint{7.670161in}{1.880617in}}{\pgfqpoint{7.670161in}{1.885660in}}%
\pgfpathcurveto{\pgfqpoint{7.670161in}{1.890704in}}{\pgfqpoint{7.668157in}{1.895542in}}{\pgfqpoint{7.664591in}{1.899108in}}%
\pgfpathcurveto{\pgfqpoint{7.661025in}{1.902675in}}{\pgfqpoint{7.656187in}{1.904679in}}{\pgfqpoint{7.651143in}{1.904679in}}%
\pgfpathcurveto{\pgfqpoint{7.646099in}{1.904679in}}{\pgfqpoint{7.641262in}{1.902675in}}{\pgfqpoint{7.637695in}{1.899108in}}%
\pgfpathcurveto{\pgfqpoint{7.634129in}{1.895542in}}{\pgfqpoint{7.632125in}{1.890704in}}{\pgfqpoint{7.632125in}{1.885660in}}%
\pgfpathcurveto{\pgfqpoint{7.632125in}{1.880617in}}{\pgfqpoint{7.634129in}{1.875779in}}{\pgfqpoint{7.637695in}{1.872213in}}%
\pgfpathcurveto{\pgfqpoint{7.641262in}{1.868646in}}{\pgfqpoint{7.646099in}{1.866642in}}{\pgfqpoint{7.651143in}{1.866642in}}%
\pgfpathclose%
\pgfusepath{fill}%
\end{pgfscope}%
\begin{pgfscope}%
\pgfpathrectangle{\pgfqpoint{6.572727in}{0.473000in}}{\pgfqpoint{4.227273in}{3.311000in}}%
\pgfusepath{clip}%
\pgfsetbuttcap%
\pgfsetroundjoin%
\definecolor{currentfill}{rgb}{0.127568,0.566949,0.550556}%
\pgfsetfillcolor{currentfill}%
\pgfsetfillopacity{0.700000}%
\pgfsetlinewidth{0.000000pt}%
\definecolor{currentstroke}{rgb}{0.000000,0.000000,0.000000}%
\pgfsetstrokecolor{currentstroke}%
\pgfsetstrokeopacity{0.700000}%
\pgfsetdash{}{0pt}%
\pgfpathmoveto{\pgfqpoint{7.946003in}{2.229844in}}%
\pgfpathcurveto{\pgfqpoint{7.951046in}{2.229844in}}{\pgfqpoint{7.955884in}{2.231848in}}{\pgfqpoint{7.959451in}{2.235414in}}%
\pgfpathcurveto{\pgfqpoint{7.963017in}{2.238981in}}{\pgfqpoint{7.965021in}{2.243819in}}{\pgfqpoint{7.965021in}{2.248862in}}%
\pgfpathcurveto{\pgfqpoint{7.965021in}{2.253906in}}{\pgfqpoint{7.963017in}{2.258744in}}{\pgfqpoint{7.959451in}{2.262310in}}%
\pgfpathcurveto{\pgfqpoint{7.955884in}{2.265877in}}{\pgfqpoint{7.951046in}{2.267881in}}{\pgfqpoint{7.946003in}{2.267881in}}%
\pgfpathcurveto{\pgfqpoint{7.940959in}{2.267881in}}{\pgfqpoint{7.936121in}{2.265877in}}{\pgfqpoint{7.932555in}{2.262310in}}%
\pgfpathcurveto{\pgfqpoint{7.928988in}{2.258744in}}{\pgfqpoint{7.926985in}{2.253906in}}{\pgfqpoint{7.926985in}{2.248862in}}%
\pgfpathcurveto{\pgfqpoint{7.926985in}{2.243819in}}{\pgfqpoint{7.928988in}{2.238981in}}{\pgfqpoint{7.932555in}{2.235414in}}%
\pgfpathcurveto{\pgfqpoint{7.936121in}{2.231848in}}{\pgfqpoint{7.940959in}{2.229844in}}{\pgfqpoint{7.946003in}{2.229844in}}%
\pgfpathclose%
\pgfusepath{fill}%
\end{pgfscope}%
\begin{pgfscope}%
\pgfpathrectangle{\pgfqpoint{6.572727in}{0.473000in}}{\pgfqpoint{4.227273in}{3.311000in}}%
\pgfusepath{clip}%
\pgfsetbuttcap%
\pgfsetroundjoin%
\definecolor{currentfill}{rgb}{0.127568,0.566949,0.550556}%
\pgfsetfillcolor{currentfill}%
\pgfsetfillopacity{0.700000}%
\pgfsetlinewidth{0.000000pt}%
\definecolor{currentstroke}{rgb}{0.000000,0.000000,0.000000}%
\pgfsetstrokecolor{currentstroke}%
\pgfsetstrokeopacity{0.700000}%
\pgfsetdash{}{0pt}%
\pgfpathmoveto{\pgfqpoint{7.762822in}{1.073468in}}%
\pgfpathcurveto{\pgfqpoint{7.767866in}{1.073468in}}{\pgfqpoint{7.772703in}{1.075472in}}{\pgfqpoint{7.776270in}{1.079039in}}%
\pgfpathcurveto{\pgfqpoint{7.779836in}{1.082605in}}{\pgfqpoint{7.781840in}{1.087443in}}{\pgfqpoint{7.781840in}{1.092486in}}%
\pgfpathcurveto{\pgfqpoint{7.781840in}{1.097530in}}{\pgfqpoint{7.779836in}{1.102368in}}{\pgfqpoint{7.776270in}{1.105934in}}%
\pgfpathcurveto{\pgfqpoint{7.772703in}{1.109501in}}{\pgfqpoint{7.767866in}{1.111505in}}{\pgfqpoint{7.762822in}{1.111505in}}%
\pgfpathcurveto{\pgfqpoint{7.757778in}{1.111505in}}{\pgfqpoint{7.752941in}{1.109501in}}{\pgfqpoint{7.749374in}{1.105934in}}%
\pgfpathcurveto{\pgfqpoint{7.745808in}{1.102368in}}{\pgfqpoint{7.743804in}{1.097530in}}{\pgfqpoint{7.743804in}{1.092486in}}%
\pgfpathcurveto{\pgfqpoint{7.743804in}{1.087443in}}{\pgfqpoint{7.745808in}{1.082605in}}{\pgfqpoint{7.749374in}{1.079039in}}%
\pgfpathcurveto{\pgfqpoint{7.752941in}{1.075472in}}{\pgfqpoint{7.757778in}{1.073468in}}{\pgfqpoint{7.762822in}{1.073468in}}%
\pgfpathclose%
\pgfusepath{fill}%
\end{pgfscope}%
\begin{pgfscope}%
\pgfpathrectangle{\pgfqpoint{6.572727in}{0.473000in}}{\pgfqpoint{4.227273in}{3.311000in}}%
\pgfusepath{clip}%
\pgfsetbuttcap%
\pgfsetroundjoin%
\definecolor{currentfill}{rgb}{0.127568,0.566949,0.550556}%
\pgfsetfillcolor{currentfill}%
\pgfsetfillopacity{0.700000}%
\pgfsetlinewidth{0.000000pt}%
\definecolor{currentstroke}{rgb}{0.000000,0.000000,0.000000}%
\pgfsetstrokecolor{currentstroke}%
\pgfsetstrokeopacity{0.700000}%
\pgfsetdash{}{0pt}%
\pgfpathmoveto{\pgfqpoint{8.124940in}{1.508447in}}%
\pgfpathcurveto{\pgfqpoint{8.129983in}{1.508447in}}{\pgfqpoint{8.134821in}{1.510451in}}{\pgfqpoint{8.138387in}{1.514017in}}%
\pgfpathcurveto{\pgfqpoint{8.141954in}{1.517583in}}{\pgfqpoint{8.143958in}{1.522421in}}{\pgfqpoint{8.143958in}{1.527465in}}%
\pgfpathcurveto{\pgfqpoint{8.143958in}{1.532509in}}{\pgfqpoint{8.141954in}{1.537346in}}{\pgfqpoint{8.138387in}{1.540913in}}%
\pgfpathcurveto{\pgfqpoint{8.134821in}{1.544479in}}{\pgfqpoint{8.129983in}{1.546483in}}{\pgfqpoint{8.124940in}{1.546483in}}%
\pgfpathcurveto{\pgfqpoint{8.119896in}{1.546483in}}{\pgfqpoint{8.115058in}{1.544479in}}{\pgfqpoint{8.111492in}{1.540913in}}%
\pgfpathcurveto{\pgfqpoint{8.107925in}{1.537346in}}{\pgfqpoint{8.105921in}{1.532509in}}{\pgfqpoint{8.105921in}{1.527465in}}%
\pgfpathcurveto{\pgfqpoint{8.105921in}{1.522421in}}{\pgfqpoint{8.107925in}{1.517583in}}{\pgfqpoint{8.111492in}{1.514017in}}%
\pgfpathcurveto{\pgfqpoint{8.115058in}{1.510451in}}{\pgfqpoint{8.119896in}{1.508447in}}{\pgfqpoint{8.124940in}{1.508447in}}%
\pgfpathclose%
\pgfusepath{fill}%
\end{pgfscope}%
\begin{pgfscope}%
\pgfpathrectangle{\pgfqpoint{6.572727in}{0.473000in}}{\pgfqpoint{4.227273in}{3.311000in}}%
\pgfusepath{clip}%
\pgfsetbuttcap%
\pgfsetroundjoin%
\definecolor{currentfill}{rgb}{0.993248,0.906157,0.143936}%
\pgfsetfillcolor{currentfill}%
\pgfsetfillopacity{0.700000}%
\pgfsetlinewidth{0.000000pt}%
\definecolor{currentstroke}{rgb}{0.000000,0.000000,0.000000}%
\pgfsetstrokecolor{currentstroke}%
\pgfsetstrokeopacity{0.700000}%
\pgfsetdash{}{0pt}%
\pgfpathmoveto{\pgfqpoint{10.010201in}{1.870905in}}%
\pgfpathcurveto{\pgfqpoint{10.015244in}{1.870905in}}{\pgfqpoint{10.020082in}{1.872909in}}{\pgfqpoint{10.023649in}{1.876475in}}%
\pgfpathcurveto{\pgfqpoint{10.027215in}{1.880042in}}{\pgfqpoint{10.029219in}{1.884879in}}{\pgfqpoint{10.029219in}{1.889923in}}%
\pgfpathcurveto{\pgfqpoint{10.029219in}{1.894967in}}{\pgfqpoint{10.027215in}{1.899804in}}{\pgfqpoint{10.023649in}{1.903371in}}%
\pgfpathcurveto{\pgfqpoint{10.020082in}{1.906937in}}{\pgfqpoint{10.015244in}{1.908941in}}{\pgfqpoint{10.010201in}{1.908941in}}%
\pgfpathcurveto{\pgfqpoint{10.005157in}{1.908941in}}{\pgfqpoint{10.000319in}{1.906937in}}{\pgfqpoint{9.996753in}{1.903371in}}%
\pgfpathcurveto{\pgfqpoint{9.993186in}{1.899804in}}{\pgfqpoint{9.991183in}{1.894967in}}{\pgfqpoint{9.991183in}{1.889923in}}%
\pgfpathcurveto{\pgfqpoint{9.991183in}{1.884879in}}{\pgfqpoint{9.993186in}{1.880042in}}{\pgfqpoint{9.996753in}{1.876475in}}%
\pgfpathcurveto{\pgfqpoint{10.000319in}{1.872909in}}{\pgfqpoint{10.005157in}{1.870905in}}{\pgfqpoint{10.010201in}{1.870905in}}%
\pgfpathclose%
\pgfusepath{fill}%
\end{pgfscope}%
\begin{pgfscope}%
\pgfpathrectangle{\pgfqpoint{6.572727in}{0.473000in}}{\pgfqpoint{4.227273in}{3.311000in}}%
\pgfusepath{clip}%
\pgfsetbuttcap%
\pgfsetroundjoin%
\definecolor{currentfill}{rgb}{0.127568,0.566949,0.550556}%
\pgfsetfillcolor{currentfill}%
\pgfsetfillopacity{0.700000}%
\pgfsetlinewidth{0.000000pt}%
\definecolor{currentstroke}{rgb}{0.000000,0.000000,0.000000}%
\pgfsetstrokecolor{currentstroke}%
\pgfsetstrokeopacity{0.700000}%
\pgfsetdash{}{0pt}%
\pgfpathmoveto{\pgfqpoint{8.010571in}{2.674472in}}%
\pgfpathcurveto{\pgfqpoint{8.015614in}{2.674472in}}{\pgfqpoint{8.020452in}{2.676476in}}{\pgfqpoint{8.024018in}{2.680043in}}%
\pgfpathcurveto{\pgfqpoint{8.027585in}{2.683609in}}{\pgfqpoint{8.029589in}{2.688447in}}{\pgfqpoint{8.029589in}{2.693490in}}%
\pgfpathcurveto{\pgfqpoint{8.029589in}{2.698534in}}{\pgfqpoint{8.027585in}{2.703372in}}{\pgfqpoint{8.024018in}{2.706938in}}%
\pgfpathcurveto{\pgfqpoint{8.020452in}{2.710505in}}{\pgfqpoint{8.015614in}{2.712509in}}{\pgfqpoint{8.010571in}{2.712509in}}%
\pgfpathcurveto{\pgfqpoint{8.005527in}{2.712509in}}{\pgfqpoint{8.000689in}{2.710505in}}{\pgfqpoint{7.997123in}{2.706938in}}%
\pgfpathcurveto{\pgfqpoint{7.993556in}{2.703372in}}{\pgfqpoint{7.991552in}{2.698534in}}{\pgfqpoint{7.991552in}{2.693490in}}%
\pgfpathcurveto{\pgfqpoint{7.991552in}{2.688447in}}{\pgfqpoint{7.993556in}{2.683609in}}{\pgfqpoint{7.997123in}{2.680043in}}%
\pgfpathcurveto{\pgfqpoint{8.000689in}{2.676476in}}{\pgfqpoint{8.005527in}{2.674472in}}{\pgfqpoint{8.010571in}{2.674472in}}%
\pgfpathclose%
\pgfusepath{fill}%
\end{pgfscope}%
\begin{pgfscope}%
\pgfpathrectangle{\pgfqpoint{6.572727in}{0.473000in}}{\pgfqpoint{4.227273in}{3.311000in}}%
\pgfusepath{clip}%
\pgfsetbuttcap%
\pgfsetroundjoin%
\definecolor{currentfill}{rgb}{0.993248,0.906157,0.143936}%
\pgfsetfillcolor{currentfill}%
\pgfsetfillopacity{0.700000}%
\pgfsetlinewidth{0.000000pt}%
\definecolor{currentstroke}{rgb}{0.000000,0.000000,0.000000}%
\pgfsetstrokecolor{currentstroke}%
\pgfsetstrokeopacity{0.700000}%
\pgfsetdash{}{0pt}%
\pgfpathmoveto{\pgfqpoint{10.233045in}{1.078865in}}%
\pgfpathcurveto{\pgfqpoint{10.238089in}{1.078865in}}{\pgfqpoint{10.242927in}{1.080869in}}{\pgfqpoint{10.246493in}{1.084436in}}%
\pgfpathcurveto{\pgfqpoint{10.250060in}{1.088002in}}{\pgfqpoint{10.252064in}{1.092840in}}{\pgfqpoint{10.252064in}{1.097884in}}%
\pgfpathcurveto{\pgfqpoint{10.252064in}{1.102927in}}{\pgfqpoint{10.250060in}{1.107765in}}{\pgfqpoint{10.246493in}{1.111331in}}%
\pgfpathcurveto{\pgfqpoint{10.242927in}{1.114898in}}{\pgfqpoint{10.238089in}{1.116902in}}{\pgfqpoint{10.233045in}{1.116902in}}%
\pgfpathcurveto{\pgfqpoint{10.228002in}{1.116902in}}{\pgfqpoint{10.223164in}{1.114898in}}{\pgfqpoint{10.219598in}{1.111331in}}%
\pgfpathcurveto{\pgfqpoint{10.216031in}{1.107765in}}{\pgfqpoint{10.214027in}{1.102927in}}{\pgfqpoint{10.214027in}{1.097884in}}%
\pgfpathcurveto{\pgfqpoint{10.214027in}{1.092840in}}{\pgfqpoint{10.216031in}{1.088002in}}{\pgfqpoint{10.219598in}{1.084436in}}%
\pgfpathcurveto{\pgfqpoint{10.223164in}{1.080869in}}{\pgfqpoint{10.228002in}{1.078865in}}{\pgfqpoint{10.233045in}{1.078865in}}%
\pgfpathclose%
\pgfusepath{fill}%
\end{pgfscope}%
\begin{pgfscope}%
\pgfpathrectangle{\pgfqpoint{6.572727in}{0.473000in}}{\pgfqpoint{4.227273in}{3.311000in}}%
\pgfusepath{clip}%
\pgfsetbuttcap%
\pgfsetroundjoin%
\definecolor{currentfill}{rgb}{0.127568,0.566949,0.550556}%
\pgfsetfillcolor{currentfill}%
\pgfsetfillopacity{0.700000}%
\pgfsetlinewidth{0.000000pt}%
\definecolor{currentstroke}{rgb}{0.000000,0.000000,0.000000}%
\pgfsetstrokecolor{currentstroke}%
\pgfsetstrokeopacity{0.700000}%
\pgfsetdash{}{0pt}%
\pgfpathmoveto{\pgfqpoint{7.491528in}{1.479171in}}%
\pgfpathcurveto{\pgfqpoint{7.496572in}{1.479171in}}{\pgfqpoint{7.501409in}{1.481174in}}{\pgfqpoint{7.504976in}{1.484741in}}%
\pgfpathcurveto{\pgfqpoint{7.508542in}{1.488307in}}{\pgfqpoint{7.510546in}{1.493145in}}{\pgfqpoint{7.510546in}{1.498189in}}%
\pgfpathcurveto{\pgfqpoint{7.510546in}{1.503232in}}{\pgfqpoint{7.508542in}{1.508070in}}{\pgfqpoint{7.504976in}{1.511637in}}%
\pgfpathcurveto{\pgfqpoint{7.501409in}{1.515203in}}{\pgfqpoint{7.496572in}{1.517207in}}{\pgfqpoint{7.491528in}{1.517207in}}%
\pgfpathcurveto{\pgfqpoint{7.486484in}{1.517207in}}{\pgfqpoint{7.481646in}{1.515203in}}{\pgfqpoint{7.478080in}{1.511637in}}%
\pgfpathcurveto{\pgfqpoint{7.474514in}{1.508070in}}{\pgfqpoint{7.472510in}{1.503232in}}{\pgfqpoint{7.472510in}{1.498189in}}%
\pgfpathcurveto{\pgfqpoint{7.472510in}{1.493145in}}{\pgfqpoint{7.474514in}{1.488307in}}{\pgfqpoint{7.478080in}{1.484741in}}%
\pgfpathcurveto{\pgfqpoint{7.481646in}{1.481174in}}{\pgfqpoint{7.486484in}{1.479171in}}{\pgfqpoint{7.491528in}{1.479171in}}%
\pgfpathclose%
\pgfusepath{fill}%
\end{pgfscope}%
\begin{pgfscope}%
\pgfpathrectangle{\pgfqpoint{6.572727in}{0.473000in}}{\pgfqpoint{4.227273in}{3.311000in}}%
\pgfusepath{clip}%
\pgfsetbuttcap%
\pgfsetroundjoin%
\definecolor{currentfill}{rgb}{0.127568,0.566949,0.550556}%
\pgfsetfillcolor{currentfill}%
\pgfsetfillopacity{0.700000}%
\pgfsetlinewidth{0.000000pt}%
\definecolor{currentstroke}{rgb}{0.000000,0.000000,0.000000}%
\pgfsetstrokecolor{currentstroke}%
\pgfsetstrokeopacity{0.700000}%
\pgfsetdash{}{0pt}%
\pgfpathmoveto{\pgfqpoint{8.031887in}{1.316098in}}%
\pgfpathcurveto{\pgfqpoint{8.036930in}{1.316098in}}{\pgfqpoint{8.041768in}{1.318102in}}{\pgfqpoint{8.045334in}{1.321668in}}%
\pgfpathcurveto{\pgfqpoint{8.048901in}{1.325235in}}{\pgfqpoint{8.050905in}{1.330073in}}{\pgfqpoint{8.050905in}{1.335116in}}%
\pgfpathcurveto{\pgfqpoint{8.050905in}{1.340160in}}{\pgfqpoint{8.048901in}{1.344998in}}{\pgfqpoint{8.045334in}{1.348564in}}%
\pgfpathcurveto{\pgfqpoint{8.041768in}{1.352130in}}{\pgfqpoint{8.036930in}{1.354134in}}{\pgfqpoint{8.031887in}{1.354134in}}%
\pgfpathcurveto{\pgfqpoint{8.026843in}{1.354134in}}{\pgfqpoint{8.022005in}{1.352130in}}{\pgfqpoint{8.018439in}{1.348564in}}%
\pgfpathcurveto{\pgfqpoint{8.014872in}{1.344998in}}{\pgfqpoint{8.012868in}{1.340160in}}{\pgfqpoint{8.012868in}{1.335116in}}%
\pgfpathcurveto{\pgfqpoint{8.012868in}{1.330073in}}{\pgfqpoint{8.014872in}{1.325235in}}{\pgfqpoint{8.018439in}{1.321668in}}%
\pgfpathcurveto{\pgfqpoint{8.022005in}{1.318102in}}{\pgfqpoint{8.026843in}{1.316098in}}{\pgfqpoint{8.031887in}{1.316098in}}%
\pgfpathclose%
\pgfusepath{fill}%
\end{pgfscope}%
\begin{pgfscope}%
\pgfpathrectangle{\pgfqpoint{6.572727in}{0.473000in}}{\pgfqpoint{4.227273in}{3.311000in}}%
\pgfusepath{clip}%
\pgfsetbuttcap%
\pgfsetroundjoin%
\definecolor{currentfill}{rgb}{0.127568,0.566949,0.550556}%
\pgfsetfillcolor{currentfill}%
\pgfsetfillopacity{0.700000}%
\pgfsetlinewidth{0.000000pt}%
\definecolor{currentstroke}{rgb}{0.000000,0.000000,0.000000}%
\pgfsetstrokecolor{currentstroke}%
\pgfsetstrokeopacity{0.700000}%
\pgfsetdash{}{0pt}%
\pgfpathmoveto{\pgfqpoint{8.073258in}{2.333348in}}%
\pgfpathcurveto{\pgfqpoint{8.078302in}{2.333348in}}{\pgfqpoint{8.083139in}{2.335352in}}{\pgfqpoint{8.086706in}{2.338918in}}%
\pgfpathcurveto{\pgfqpoint{8.090272in}{2.342485in}}{\pgfqpoint{8.092276in}{2.347322in}}{\pgfqpoint{8.092276in}{2.352366in}}%
\pgfpathcurveto{\pgfqpoint{8.092276in}{2.357410in}}{\pgfqpoint{8.090272in}{2.362247in}}{\pgfqpoint{8.086706in}{2.365814in}}%
\pgfpathcurveto{\pgfqpoint{8.083139in}{2.369380in}}{\pgfqpoint{8.078302in}{2.371384in}}{\pgfqpoint{8.073258in}{2.371384in}}%
\pgfpathcurveto{\pgfqpoint{8.068214in}{2.371384in}}{\pgfqpoint{8.063376in}{2.369380in}}{\pgfqpoint{8.059810in}{2.365814in}}%
\pgfpathcurveto{\pgfqpoint{8.056244in}{2.362247in}}{\pgfqpoint{8.054240in}{2.357410in}}{\pgfqpoint{8.054240in}{2.352366in}}%
\pgfpathcurveto{\pgfqpoint{8.054240in}{2.347322in}}{\pgfqpoint{8.056244in}{2.342485in}}{\pgfqpoint{8.059810in}{2.338918in}}%
\pgfpathcurveto{\pgfqpoint{8.063376in}{2.335352in}}{\pgfqpoint{8.068214in}{2.333348in}}{\pgfqpoint{8.073258in}{2.333348in}}%
\pgfpathclose%
\pgfusepath{fill}%
\end{pgfscope}%
\begin{pgfscope}%
\pgfpathrectangle{\pgfqpoint{6.572727in}{0.473000in}}{\pgfqpoint{4.227273in}{3.311000in}}%
\pgfusepath{clip}%
\pgfsetbuttcap%
\pgfsetroundjoin%
\definecolor{currentfill}{rgb}{0.127568,0.566949,0.550556}%
\pgfsetfillcolor{currentfill}%
\pgfsetfillopacity{0.700000}%
\pgfsetlinewidth{0.000000pt}%
\definecolor{currentstroke}{rgb}{0.000000,0.000000,0.000000}%
\pgfsetstrokecolor{currentstroke}%
\pgfsetstrokeopacity{0.700000}%
\pgfsetdash{}{0pt}%
\pgfpathmoveto{\pgfqpoint{7.636168in}{1.263727in}}%
\pgfpathcurveto{\pgfqpoint{7.641212in}{1.263727in}}{\pgfqpoint{7.646049in}{1.265731in}}{\pgfqpoint{7.649616in}{1.269297in}}%
\pgfpathcurveto{\pgfqpoint{7.653182in}{1.272864in}}{\pgfqpoint{7.655186in}{1.277701in}}{\pgfqpoint{7.655186in}{1.282745in}}%
\pgfpathcurveto{\pgfqpoint{7.655186in}{1.287789in}}{\pgfqpoint{7.653182in}{1.292627in}}{\pgfqpoint{7.649616in}{1.296193in}}%
\pgfpathcurveto{\pgfqpoint{7.646049in}{1.299759in}}{\pgfqpoint{7.641212in}{1.301763in}}{\pgfqpoint{7.636168in}{1.301763in}}%
\pgfpathcurveto{\pgfqpoint{7.631124in}{1.301763in}}{\pgfqpoint{7.626286in}{1.299759in}}{\pgfqpoint{7.622720in}{1.296193in}}%
\pgfpathcurveto{\pgfqpoint{7.619154in}{1.292627in}}{\pgfqpoint{7.617150in}{1.287789in}}{\pgfqpoint{7.617150in}{1.282745in}}%
\pgfpathcurveto{\pgfqpoint{7.617150in}{1.277701in}}{\pgfqpoint{7.619154in}{1.272864in}}{\pgfqpoint{7.622720in}{1.269297in}}%
\pgfpathcurveto{\pgfqpoint{7.626286in}{1.265731in}}{\pgfqpoint{7.631124in}{1.263727in}}{\pgfqpoint{7.636168in}{1.263727in}}%
\pgfpathclose%
\pgfusepath{fill}%
\end{pgfscope}%
\begin{pgfscope}%
\pgfpathrectangle{\pgfqpoint{6.572727in}{0.473000in}}{\pgfqpoint{4.227273in}{3.311000in}}%
\pgfusepath{clip}%
\pgfsetbuttcap%
\pgfsetroundjoin%
\definecolor{currentfill}{rgb}{0.993248,0.906157,0.143936}%
\pgfsetfillcolor{currentfill}%
\pgfsetfillopacity{0.700000}%
\pgfsetlinewidth{0.000000pt}%
\definecolor{currentstroke}{rgb}{0.000000,0.000000,0.000000}%
\pgfsetstrokecolor{currentstroke}%
\pgfsetstrokeopacity{0.700000}%
\pgfsetdash{}{0pt}%
\pgfpathmoveto{\pgfqpoint{9.859125in}{1.749592in}}%
\pgfpathcurveto{\pgfqpoint{9.864169in}{1.749592in}}{\pgfqpoint{9.869007in}{1.751596in}}{\pgfqpoint{9.872573in}{1.755162in}}%
\pgfpathcurveto{\pgfqpoint{9.876139in}{1.758729in}}{\pgfqpoint{9.878143in}{1.763567in}}{\pgfqpoint{9.878143in}{1.768610in}}%
\pgfpathcurveto{\pgfqpoint{9.878143in}{1.773654in}}{\pgfqpoint{9.876139in}{1.778492in}}{\pgfqpoint{9.872573in}{1.782058in}}%
\pgfpathcurveto{\pgfqpoint{9.869007in}{1.785625in}}{\pgfqpoint{9.864169in}{1.787628in}}{\pgfqpoint{9.859125in}{1.787628in}}%
\pgfpathcurveto{\pgfqpoint{9.854082in}{1.787628in}}{\pgfqpoint{9.849244in}{1.785625in}}{\pgfqpoint{9.845677in}{1.782058in}}%
\pgfpathcurveto{\pgfqpoint{9.842111in}{1.778492in}}{\pgfqpoint{9.840107in}{1.773654in}}{\pgfqpoint{9.840107in}{1.768610in}}%
\pgfpathcurveto{\pgfqpoint{9.840107in}{1.763567in}}{\pgfqpoint{9.842111in}{1.758729in}}{\pgfqpoint{9.845677in}{1.755162in}}%
\pgfpathcurveto{\pgfqpoint{9.849244in}{1.751596in}}{\pgfqpoint{9.854082in}{1.749592in}}{\pgfqpoint{9.859125in}{1.749592in}}%
\pgfpathclose%
\pgfusepath{fill}%
\end{pgfscope}%
\begin{pgfscope}%
\pgfpathrectangle{\pgfqpoint{6.572727in}{0.473000in}}{\pgfqpoint{4.227273in}{3.311000in}}%
\pgfusepath{clip}%
\pgfsetbuttcap%
\pgfsetroundjoin%
\definecolor{currentfill}{rgb}{0.127568,0.566949,0.550556}%
\pgfsetfillcolor{currentfill}%
\pgfsetfillopacity{0.700000}%
\pgfsetlinewidth{0.000000pt}%
\definecolor{currentstroke}{rgb}{0.000000,0.000000,0.000000}%
\pgfsetstrokecolor{currentstroke}%
\pgfsetstrokeopacity{0.700000}%
\pgfsetdash{}{0pt}%
\pgfpathmoveto{\pgfqpoint{7.248329in}{1.324149in}}%
\pgfpathcurveto{\pgfqpoint{7.253373in}{1.324149in}}{\pgfqpoint{7.258211in}{1.326153in}}{\pgfqpoint{7.261777in}{1.329720in}}%
\pgfpathcurveto{\pgfqpoint{7.265344in}{1.333286in}}{\pgfqpoint{7.267347in}{1.338124in}}{\pgfqpoint{7.267347in}{1.343168in}}%
\pgfpathcurveto{\pgfqpoint{7.267347in}{1.348211in}}{\pgfqpoint{7.265344in}{1.353049in}}{\pgfqpoint{7.261777in}{1.356615in}}%
\pgfpathcurveto{\pgfqpoint{7.258211in}{1.360182in}}{\pgfqpoint{7.253373in}{1.362186in}}{\pgfqpoint{7.248329in}{1.362186in}}%
\pgfpathcurveto{\pgfqpoint{7.243286in}{1.362186in}}{\pgfqpoint{7.238448in}{1.360182in}}{\pgfqpoint{7.234881in}{1.356615in}}%
\pgfpathcurveto{\pgfqpoint{7.231315in}{1.353049in}}{\pgfqpoint{7.229311in}{1.348211in}}{\pgfqpoint{7.229311in}{1.343168in}}%
\pgfpathcurveto{\pgfqpoint{7.229311in}{1.338124in}}{\pgfqpoint{7.231315in}{1.333286in}}{\pgfqpoint{7.234881in}{1.329720in}}%
\pgfpathcurveto{\pgfqpoint{7.238448in}{1.326153in}}{\pgfqpoint{7.243286in}{1.324149in}}{\pgfqpoint{7.248329in}{1.324149in}}%
\pgfpathclose%
\pgfusepath{fill}%
\end{pgfscope}%
\begin{pgfscope}%
\pgfpathrectangle{\pgfqpoint{6.572727in}{0.473000in}}{\pgfqpoint{4.227273in}{3.311000in}}%
\pgfusepath{clip}%
\pgfsetbuttcap%
\pgfsetroundjoin%
\definecolor{currentfill}{rgb}{0.993248,0.906157,0.143936}%
\pgfsetfillcolor{currentfill}%
\pgfsetfillopacity{0.700000}%
\pgfsetlinewidth{0.000000pt}%
\definecolor{currentstroke}{rgb}{0.000000,0.000000,0.000000}%
\pgfsetstrokecolor{currentstroke}%
\pgfsetstrokeopacity{0.700000}%
\pgfsetdash{}{0pt}%
\pgfpathmoveto{\pgfqpoint{9.765802in}{1.944577in}}%
\pgfpathcurveto{\pgfqpoint{9.770846in}{1.944577in}}{\pgfqpoint{9.775684in}{1.946580in}}{\pgfqpoint{9.779250in}{1.950147in}}%
\pgfpathcurveto{\pgfqpoint{9.782817in}{1.953713in}}{\pgfqpoint{9.784820in}{1.958551in}}{\pgfqpoint{9.784820in}{1.963595in}}%
\pgfpathcurveto{\pgfqpoint{9.784820in}{1.968638in}}{\pgfqpoint{9.782817in}{1.973476in}}{\pgfqpoint{9.779250in}{1.977043in}}%
\pgfpathcurveto{\pgfqpoint{9.775684in}{1.980609in}}{\pgfqpoint{9.770846in}{1.982613in}}{\pgfqpoint{9.765802in}{1.982613in}}%
\pgfpathcurveto{\pgfqpoint{9.760759in}{1.982613in}}{\pgfqpoint{9.755921in}{1.980609in}}{\pgfqpoint{9.752354in}{1.977043in}}%
\pgfpathcurveto{\pgfqpoint{9.748788in}{1.973476in}}{\pgfqpoint{9.746784in}{1.968638in}}{\pgfqpoint{9.746784in}{1.963595in}}%
\pgfpathcurveto{\pgfqpoint{9.746784in}{1.958551in}}{\pgfqpoint{9.748788in}{1.953713in}}{\pgfqpoint{9.752354in}{1.950147in}}%
\pgfpathcurveto{\pgfqpoint{9.755921in}{1.946580in}}{\pgfqpoint{9.760759in}{1.944577in}}{\pgfqpoint{9.765802in}{1.944577in}}%
\pgfpathclose%
\pgfusepath{fill}%
\end{pgfscope}%
\begin{pgfscope}%
\pgfpathrectangle{\pgfqpoint{6.572727in}{0.473000in}}{\pgfqpoint{4.227273in}{3.311000in}}%
\pgfusepath{clip}%
\pgfsetbuttcap%
\pgfsetroundjoin%
\definecolor{currentfill}{rgb}{0.127568,0.566949,0.550556}%
\pgfsetfillcolor{currentfill}%
\pgfsetfillopacity{0.700000}%
\pgfsetlinewidth{0.000000pt}%
\definecolor{currentstroke}{rgb}{0.000000,0.000000,0.000000}%
\pgfsetstrokecolor{currentstroke}%
\pgfsetstrokeopacity{0.700000}%
\pgfsetdash{}{0pt}%
\pgfpathmoveto{\pgfqpoint{8.297741in}{3.293917in}}%
\pgfpathcurveto{\pgfqpoint{8.302785in}{3.293917in}}{\pgfqpoint{8.307623in}{3.295921in}}{\pgfqpoint{8.311189in}{3.299488in}}%
\pgfpathcurveto{\pgfqpoint{8.314756in}{3.303054in}}{\pgfqpoint{8.316759in}{3.307892in}}{\pgfqpoint{8.316759in}{3.312935in}}%
\pgfpathcurveto{\pgfqpoint{8.316759in}{3.317979in}}{\pgfqpoint{8.314756in}{3.322817in}}{\pgfqpoint{8.311189in}{3.326383in}}%
\pgfpathcurveto{\pgfqpoint{8.307623in}{3.329950in}}{\pgfqpoint{8.302785in}{3.331954in}}{\pgfqpoint{8.297741in}{3.331954in}}%
\pgfpathcurveto{\pgfqpoint{8.292698in}{3.331954in}}{\pgfqpoint{8.287860in}{3.329950in}}{\pgfqpoint{8.284293in}{3.326383in}}%
\pgfpathcurveto{\pgfqpoint{8.280727in}{3.322817in}}{\pgfqpoint{8.278723in}{3.317979in}}{\pgfqpoint{8.278723in}{3.312935in}}%
\pgfpathcurveto{\pgfqpoint{8.278723in}{3.307892in}}{\pgfqpoint{8.280727in}{3.303054in}}{\pgfqpoint{8.284293in}{3.299488in}}%
\pgfpathcurveto{\pgfqpoint{8.287860in}{3.295921in}}{\pgfqpoint{8.292698in}{3.293917in}}{\pgfqpoint{8.297741in}{3.293917in}}%
\pgfpathclose%
\pgfusepath{fill}%
\end{pgfscope}%
\begin{pgfscope}%
\pgfpathrectangle{\pgfqpoint{6.572727in}{0.473000in}}{\pgfqpoint{4.227273in}{3.311000in}}%
\pgfusepath{clip}%
\pgfsetbuttcap%
\pgfsetroundjoin%
\definecolor{currentfill}{rgb}{0.127568,0.566949,0.550556}%
\pgfsetfillcolor{currentfill}%
\pgfsetfillopacity{0.700000}%
\pgfsetlinewidth{0.000000pt}%
\definecolor{currentstroke}{rgb}{0.000000,0.000000,0.000000}%
\pgfsetstrokecolor{currentstroke}%
\pgfsetstrokeopacity{0.700000}%
\pgfsetdash{}{0pt}%
\pgfpathmoveto{\pgfqpoint{8.332766in}{1.727203in}}%
\pgfpathcurveto{\pgfqpoint{8.337809in}{1.727203in}}{\pgfqpoint{8.342647in}{1.729207in}}{\pgfqpoint{8.346214in}{1.732774in}}%
\pgfpathcurveto{\pgfqpoint{8.349780in}{1.736340in}}{\pgfqpoint{8.351784in}{1.741178in}}{\pgfqpoint{8.351784in}{1.746221in}}%
\pgfpathcurveto{\pgfqpoint{8.351784in}{1.751265in}}{\pgfqpoint{8.349780in}{1.756103in}}{\pgfqpoint{8.346214in}{1.759669in}}%
\pgfpathcurveto{\pgfqpoint{8.342647in}{1.763236in}}{\pgfqpoint{8.337809in}{1.765240in}}{\pgfqpoint{8.332766in}{1.765240in}}%
\pgfpathcurveto{\pgfqpoint{8.327722in}{1.765240in}}{\pgfqpoint{8.322884in}{1.763236in}}{\pgfqpoint{8.319318in}{1.759669in}}%
\pgfpathcurveto{\pgfqpoint{8.315751in}{1.756103in}}{\pgfqpoint{8.313748in}{1.751265in}}{\pgfqpoint{8.313748in}{1.746221in}}%
\pgfpathcurveto{\pgfqpoint{8.313748in}{1.741178in}}{\pgfqpoint{8.315751in}{1.736340in}}{\pgfqpoint{8.319318in}{1.732774in}}%
\pgfpathcurveto{\pgfqpoint{8.322884in}{1.729207in}}{\pgfqpoint{8.327722in}{1.727203in}}{\pgfqpoint{8.332766in}{1.727203in}}%
\pgfpathclose%
\pgfusepath{fill}%
\end{pgfscope}%
\begin{pgfscope}%
\pgfpathrectangle{\pgfqpoint{6.572727in}{0.473000in}}{\pgfqpoint{4.227273in}{3.311000in}}%
\pgfusepath{clip}%
\pgfsetbuttcap%
\pgfsetroundjoin%
\definecolor{currentfill}{rgb}{0.993248,0.906157,0.143936}%
\pgfsetfillcolor{currentfill}%
\pgfsetfillopacity{0.700000}%
\pgfsetlinewidth{0.000000pt}%
\definecolor{currentstroke}{rgb}{0.000000,0.000000,0.000000}%
\pgfsetstrokecolor{currentstroke}%
\pgfsetstrokeopacity{0.700000}%
\pgfsetdash{}{0pt}%
\pgfpathmoveto{\pgfqpoint{9.636829in}{1.359118in}}%
\pgfpathcurveto{\pgfqpoint{9.641873in}{1.359118in}}{\pgfqpoint{9.646711in}{1.361122in}}{\pgfqpoint{9.650277in}{1.364688in}}%
\pgfpathcurveto{\pgfqpoint{9.653843in}{1.368254in}}{\pgfqpoint{9.655847in}{1.373092in}}{\pgfqpoint{9.655847in}{1.378136in}}%
\pgfpathcurveto{\pgfqpoint{9.655847in}{1.383180in}}{\pgfqpoint{9.653843in}{1.388017in}}{\pgfqpoint{9.650277in}{1.391584in}}%
\pgfpathcurveto{\pgfqpoint{9.646711in}{1.395150in}}{\pgfqpoint{9.641873in}{1.397154in}}{\pgfqpoint{9.636829in}{1.397154in}}%
\pgfpathcurveto{\pgfqpoint{9.631786in}{1.397154in}}{\pgfqpoint{9.626948in}{1.395150in}}{\pgfqpoint{9.623381in}{1.391584in}}%
\pgfpathcurveto{\pgfqpoint{9.619815in}{1.388017in}}{\pgfqpoint{9.617811in}{1.383180in}}{\pgfqpoint{9.617811in}{1.378136in}}%
\pgfpathcurveto{\pgfqpoint{9.617811in}{1.373092in}}{\pgfqpoint{9.619815in}{1.368254in}}{\pgfqpoint{9.623381in}{1.364688in}}%
\pgfpathcurveto{\pgfqpoint{9.626948in}{1.361122in}}{\pgfqpoint{9.631786in}{1.359118in}}{\pgfqpoint{9.636829in}{1.359118in}}%
\pgfpathclose%
\pgfusepath{fill}%
\end{pgfscope}%
\begin{pgfscope}%
\pgfpathrectangle{\pgfqpoint{6.572727in}{0.473000in}}{\pgfqpoint{4.227273in}{3.311000in}}%
\pgfusepath{clip}%
\pgfsetbuttcap%
\pgfsetroundjoin%
\definecolor{currentfill}{rgb}{0.127568,0.566949,0.550556}%
\pgfsetfillcolor{currentfill}%
\pgfsetfillopacity{0.700000}%
\pgfsetlinewidth{0.000000pt}%
\definecolor{currentstroke}{rgb}{0.000000,0.000000,0.000000}%
\pgfsetstrokecolor{currentstroke}%
\pgfsetstrokeopacity{0.700000}%
\pgfsetdash{}{0pt}%
\pgfpathmoveto{\pgfqpoint{7.857265in}{3.081800in}}%
\pgfpathcurveto{\pgfqpoint{7.862309in}{3.081800in}}{\pgfqpoint{7.867147in}{3.083804in}}{\pgfqpoint{7.870713in}{3.087371in}}%
\pgfpathcurveto{\pgfqpoint{7.874280in}{3.090937in}}{\pgfqpoint{7.876283in}{3.095775in}}{\pgfqpoint{7.876283in}{3.100819in}}%
\pgfpathcurveto{\pgfqpoint{7.876283in}{3.105862in}}{\pgfqpoint{7.874280in}{3.110700in}}{\pgfqpoint{7.870713in}{3.114266in}}%
\pgfpathcurveto{\pgfqpoint{7.867147in}{3.117833in}}{\pgfqpoint{7.862309in}{3.119837in}}{\pgfqpoint{7.857265in}{3.119837in}}%
\pgfpathcurveto{\pgfqpoint{7.852222in}{3.119837in}}{\pgfqpoint{7.847384in}{3.117833in}}{\pgfqpoint{7.843817in}{3.114266in}}%
\pgfpathcurveto{\pgfqpoint{7.840251in}{3.110700in}}{\pgfqpoint{7.838247in}{3.105862in}}{\pgfqpoint{7.838247in}{3.100819in}}%
\pgfpathcurveto{\pgfqpoint{7.838247in}{3.095775in}}{\pgfqpoint{7.840251in}{3.090937in}}{\pgfqpoint{7.843817in}{3.087371in}}%
\pgfpathcurveto{\pgfqpoint{7.847384in}{3.083804in}}{\pgfqpoint{7.852222in}{3.081800in}}{\pgfqpoint{7.857265in}{3.081800in}}%
\pgfpathclose%
\pgfusepath{fill}%
\end{pgfscope}%
\begin{pgfscope}%
\pgfpathrectangle{\pgfqpoint{6.572727in}{0.473000in}}{\pgfqpoint{4.227273in}{3.311000in}}%
\pgfusepath{clip}%
\pgfsetbuttcap%
\pgfsetroundjoin%
\definecolor{currentfill}{rgb}{0.127568,0.566949,0.550556}%
\pgfsetfillcolor{currentfill}%
\pgfsetfillopacity{0.700000}%
\pgfsetlinewidth{0.000000pt}%
\definecolor{currentstroke}{rgb}{0.000000,0.000000,0.000000}%
\pgfsetstrokecolor{currentstroke}%
\pgfsetstrokeopacity{0.700000}%
\pgfsetdash{}{0pt}%
\pgfpathmoveto{\pgfqpoint{7.903005in}{2.747634in}}%
\pgfpathcurveto{\pgfqpoint{7.908049in}{2.747634in}}{\pgfqpoint{7.912886in}{2.749638in}}{\pgfqpoint{7.916453in}{2.753204in}}%
\pgfpathcurveto{\pgfqpoint{7.920019in}{2.756771in}}{\pgfqpoint{7.922023in}{2.761608in}}{\pgfqpoint{7.922023in}{2.766652in}}%
\pgfpathcurveto{\pgfqpoint{7.922023in}{2.771696in}}{\pgfqpoint{7.920019in}{2.776533in}}{\pgfqpoint{7.916453in}{2.780100in}}%
\pgfpathcurveto{\pgfqpoint{7.912886in}{2.783666in}}{\pgfqpoint{7.908049in}{2.785670in}}{\pgfqpoint{7.903005in}{2.785670in}}%
\pgfpathcurveto{\pgfqpoint{7.897961in}{2.785670in}}{\pgfqpoint{7.893124in}{2.783666in}}{\pgfqpoint{7.889557in}{2.780100in}}%
\pgfpathcurveto{\pgfqpoint{7.885991in}{2.776533in}}{\pgfqpoint{7.883987in}{2.771696in}}{\pgfqpoint{7.883987in}{2.766652in}}%
\pgfpathcurveto{\pgfqpoint{7.883987in}{2.761608in}}{\pgfqpoint{7.885991in}{2.756771in}}{\pgfqpoint{7.889557in}{2.753204in}}%
\pgfpathcurveto{\pgfqpoint{7.893124in}{2.749638in}}{\pgfqpoint{7.897961in}{2.747634in}}{\pgfqpoint{7.903005in}{2.747634in}}%
\pgfpathclose%
\pgfusepath{fill}%
\end{pgfscope}%
\begin{pgfscope}%
\pgfpathrectangle{\pgfqpoint{6.572727in}{0.473000in}}{\pgfqpoint{4.227273in}{3.311000in}}%
\pgfusepath{clip}%
\pgfsetbuttcap%
\pgfsetroundjoin%
\definecolor{currentfill}{rgb}{0.127568,0.566949,0.550556}%
\pgfsetfillcolor{currentfill}%
\pgfsetfillopacity{0.700000}%
\pgfsetlinewidth{0.000000pt}%
\definecolor{currentstroke}{rgb}{0.000000,0.000000,0.000000}%
\pgfsetstrokecolor{currentstroke}%
\pgfsetstrokeopacity{0.700000}%
\pgfsetdash{}{0pt}%
\pgfpathmoveto{\pgfqpoint{8.694913in}{2.829080in}}%
\pgfpathcurveto{\pgfqpoint{8.699956in}{2.829080in}}{\pgfqpoint{8.704794in}{2.831084in}}{\pgfqpoint{8.708361in}{2.834650in}}%
\pgfpathcurveto{\pgfqpoint{8.711927in}{2.838217in}}{\pgfqpoint{8.713931in}{2.843054in}}{\pgfqpoint{8.713931in}{2.848098in}}%
\pgfpathcurveto{\pgfqpoint{8.713931in}{2.853142in}}{\pgfqpoint{8.711927in}{2.857979in}}{\pgfqpoint{8.708361in}{2.861546in}}%
\pgfpathcurveto{\pgfqpoint{8.704794in}{2.865112in}}{\pgfqpoint{8.699956in}{2.867116in}}{\pgfqpoint{8.694913in}{2.867116in}}%
\pgfpathcurveto{\pgfqpoint{8.689869in}{2.867116in}}{\pgfqpoint{8.685031in}{2.865112in}}{\pgfqpoint{8.681465in}{2.861546in}}%
\pgfpathcurveto{\pgfqpoint{8.677898in}{2.857979in}}{\pgfqpoint{8.675895in}{2.853142in}}{\pgfqpoint{8.675895in}{2.848098in}}%
\pgfpathcurveto{\pgfqpoint{8.675895in}{2.843054in}}{\pgfqpoint{8.677898in}{2.838217in}}{\pgfqpoint{8.681465in}{2.834650in}}%
\pgfpathcurveto{\pgfqpoint{8.685031in}{2.831084in}}{\pgfqpoint{8.689869in}{2.829080in}}{\pgfqpoint{8.694913in}{2.829080in}}%
\pgfpathclose%
\pgfusepath{fill}%
\end{pgfscope}%
\begin{pgfscope}%
\pgfpathrectangle{\pgfqpoint{6.572727in}{0.473000in}}{\pgfqpoint{4.227273in}{3.311000in}}%
\pgfusepath{clip}%
\pgfsetbuttcap%
\pgfsetroundjoin%
\definecolor{currentfill}{rgb}{0.127568,0.566949,0.550556}%
\pgfsetfillcolor{currentfill}%
\pgfsetfillopacity{0.700000}%
\pgfsetlinewidth{0.000000pt}%
\definecolor{currentstroke}{rgb}{0.000000,0.000000,0.000000}%
\pgfsetstrokecolor{currentstroke}%
\pgfsetstrokeopacity{0.700000}%
\pgfsetdash{}{0pt}%
\pgfpathmoveto{\pgfqpoint{7.486427in}{1.067988in}}%
\pgfpathcurveto{\pgfqpoint{7.491471in}{1.067988in}}{\pgfqpoint{7.496309in}{1.069991in}}{\pgfqpoint{7.499875in}{1.073558in}}%
\pgfpathcurveto{\pgfqpoint{7.503441in}{1.077124in}}{\pgfqpoint{7.505445in}{1.081962in}}{\pgfqpoint{7.505445in}{1.087006in}}%
\pgfpathcurveto{\pgfqpoint{7.505445in}{1.092049in}}{\pgfqpoint{7.503441in}{1.096887in}}{\pgfqpoint{7.499875in}{1.100454in}}%
\pgfpathcurveto{\pgfqpoint{7.496309in}{1.104020in}}{\pgfqpoint{7.491471in}{1.106024in}}{\pgfqpoint{7.486427in}{1.106024in}}%
\pgfpathcurveto{\pgfqpoint{7.481383in}{1.106024in}}{\pgfqpoint{7.476546in}{1.104020in}}{\pgfqpoint{7.472979in}{1.100454in}}%
\pgfpathcurveto{\pgfqpoint{7.469413in}{1.096887in}}{\pgfqpoint{7.467409in}{1.092049in}}{\pgfqpoint{7.467409in}{1.087006in}}%
\pgfpathcurveto{\pgfqpoint{7.467409in}{1.081962in}}{\pgfqpoint{7.469413in}{1.077124in}}{\pgfqpoint{7.472979in}{1.073558in}}%
\pgfpathcurveto{\pgfqpoint{7.476546in}{1.069991in}}{\pgfqpoint{7.481383in}{1.067988in}}{\pgfqpoint{7.486427in}{1.067988in}}%
\pgfpathclose%
\pgfusepath{fill}%
\end{pgfscope}%
\begin{pgfscope}%
\pgfpathrectangle{\pgfqpoint{6.572727in}{0.473000in}}{\pgfqpoint{4.227273in}{3.311000in}}%
\pgfusepath{clip}%
\pgfsetbuttcap%
\pgfsetroundjoin%
\definecolor{currentfill}{rgb}{0.127568,0.566949,0.550556}%
\pgfsetfillcolor{currentfill}%
\pgfsetfillopacity{0.700000}%
\pgfsetlinewidth{0.000000pt}%
\definecolor{currentstroke}{rgb}{0.000000,0.000000,0.000000}%
\pgfsetstrokecolor{currentstroke}%
\pgfsetstrokeopacity{0.700000}%
\pgfsetdash{}{0pt}%
\pgfpathmoveto{\pgfqpoint{7.674943in}{1.331788in}}%
\pgfpathcurveto{\pgfqpoint{7.679987in}{1.331788in}}{\pgfqpoint{7.684825in}{1.333792in}}{\pgfqpoint{7.688391in}{1.337358in}}%
\pgfpathcurveto{\pgfqpoint{7.691957in}{1.340924in}}{\pgfqpoint{7.693961in}{1.345762in}}{\pgfqpoint{7.693961in}{1.350806in}}%
\pgfpathcurveto{\pgfqpoint{7.693961in}{1.355850in}}{\pgfqpoint{7.691957in}{1.360687in}}{\pgfqpoint{7.688391in}{1.364254in}}%
\pgfpathcurveto{\pgfqpoint{7.684825in}{1.367820in}}{\pgfqpoint{7.679987in}{1.369824in}}{\pgfqpoint{7.674943in}{1.369824in}}%
\pgfpathcurveto{\pgfqpoint{7.669899in}{1.369824in}}{\pgfqpoint{7.665062in}{1.367820in}}{\pgfqpoint{7.661495in}{1.364254in}}%
\pgfpathcurveto{\pgfqpoint{7.657929in}{1.360687in}}{\pgfqpoint{7.655925in}{1.355850in}}{\pgfqpoint{7.655925in}{1.350806in}}%
\pgfpathcurveto{\pgfqpoint{7.655925in}{1.345762in}}{\pgfqpoint{7.657929in}{1.340924in}}{\pgfqpoint{7.661495in}{1.337358in}}%
\pgfpathcurveto{\pgfqpoint{7.665062in}{1.333792in}}{\pgfqpoint{7.669899in}{1.331788in}}{\pgfqpoint{7.674943in}{1.331788in}}%
\pgfpathclose%
\pgfusepath{fill}%
\end{pgfscope}%
\begin{pgfscope}%
\pgfpathrectangle{\pgfqpoint{6.572727in}{0.473000in}}{\pgfqpoint{4.227273in}{3.311000in}}%
\pgfusepath{clip}%
\pgfsetbuttcap%
\pgfsetroundjoin%
\definecolor{currentfill}{rgb}{0.127568,0.566949,0.550556}%
\pgfsetfillcolor{currentfill}%
\pgfsetfillopacity{0.700000}%
\pgfsetlinewidth{0.000000pt}%
\definecolor{currentstroke}{rgb}{0.000000,0.000000,0.000000}%
\pgfsetstrokecolor{currentstroke}%
\pgfsetstrokeopacity{0.700000}%
\pgfsetdash{}{0pt}%
\pgfpathmoveto{\pgfqpoint{8.347863in}{2.475127in}}%
\pgfpathcurveto{\pgfqpoint{8.352907in}{2.475127in}}{\pgfqpoint{8.357745in}{2.477131in}}{\pgfqpoint{8.361311in}{2.480697in}}%
\pgfpathcurveto{\pgfqpoint{8.364878in}{2.484263in}}{\pgfqpoint{8.366882in}{2.489101in}}{\pgfqpoint{8.366882in}{2.494145in}}%
\pgfpathcurveto{\pgfqpoint{8.366882in}{2.499189in}}{\pgfqpoint{8.364878in}{2.504026in}}{\pgfqpoint{8.361311in}{2.507593in}}%
\pgfpathcurveto{\pgfqpoint{8.357745in}{2.511159in}}{\pgfqpoint{8.352907in}{2.513163in}}{\pgfqpoint{8.347863in}{2.513163in}}%
\pgfpathcurveto{\pgfqpoint{8.342820in}{2.513163in}}{\pgfqpoint{8.337982in}{2.511159in}}{\pgfqpoint{8.334416in}{2.507593in}}%
\pgfpathcurveto{\pgfqpoint{8.330849in}{2.504026in}}{\pgfqpoint{8.328845in}{2.499189in}}{\pgfqpoint{8.328845in}{2.494145in}}%
\pgfpathcurveto{\pgfqpoint{8.328845in}{2.489101in}}{\pgfqpoint{8.330849in}{2.484263in}}{\pgfqpoint{8.334416in}{2.480697in}}%
\pgfpathcurveto{\pgfqpoint{8.337982in}{2.477131in}}{\pgfqpoint{8.342820in}{2.475127in}}{\pgfqpoint{8.347863in}{2.475127in}}%
\pgfpathclose%
\pgfusepath{fill}%
\end{pgfscope}%
\begin{pgfscope}%
\pgfpathrectangle{\pgfqpoint{6.572727in}{0.473000in}}{\pgfqpoint{4.227273in}{3.311000in}}%
\pgfusepath{clip}%
\pgfsetbuttcap%
\pgfsetroundjoin%
\definecolor{currentfill}{rgb}{0.993248,0.906157,0.143936}%
\pgfsetfillcolor{currentfill}%
\pgfsetfillopacity{0.700000}%
\pgfsetlinewidth{0.000000pt}%
\definecolor{currentstroke}{rgb}{0.000000,0.000000,0.000000}%
\pgfsetstrokecolor{currentstroke}%
\pgfsetstrokeopacity{0.700000}%
\pgfsetdash{}{0pt}%
\pgfpathmoveto{\pgfqpoint{9.203207in}{1.114206in}}%
\pgfpathcurveto{\pgfqpoint{9.208251in}{1.114206in}}{\pgfqpoint{9.213089in}{1.116209in}}{\pgfqpoint{9.216655in}{1.119776in}}%
\pgfpathcurveto{\pgfqpoint{9.220222in}{1.123342in}}{\pgfqpoint{9.222226in}{1.128180in}}{\pgfqpoint{9.222226in}{1.133224in}}%
\pgfpathcurveto{\pgfqpoint{9.222226in}{1.138267in}}{\pgfqpoint{9.220222in}{1.143105in}}{\pgfqpoint{9.216655in}{1.146672in}}%
\pgfpathcurveto{\pgfqpoint{9.213089in}{1.150238in}}{\pgfqpoint{9.208251in}{1.152242in}}{\pgfqpoint{9.203207in}{1.152242in}}%
\pgfpathcurveto{\pgfqpoint{9.198164in}{1.152242in}}{\pgfqpoint{9.193326in}{1.150238in}}{\pgfqpoint{9.189760in}{1.146672in}}%
\pgfpathcurveto{\pgfqpoint{9.186193in}{1.143105in}}{\pgfqpoint{9.184189in}{1.138267in}}{\pgfqpoint{9.184189in}{1.133224in}}%
\pgfpathcurveto{\pgfqpoint{9.184189in}{1.128180in}}{\pgfqpoint{9.186193in}{1.123342in}}{\pgfqpoint{9.189760in}{1.119776in}}%
\pgfpathcurveto{\pgfqpoint{9.193326in}{1.116209in}}{\pgfqpoint{9.198164in}{1.114206in}}{\pgfqpoint{9.203207in}{1.114206in}}%
\pgfpathclose%
\pgfusepath{fill}%
\end{pgfscope}%
\begin{pgfscope}%
\pgfpathrectangle{\pgfqpoint{6.572727in}{0.473000in}}{\pgfqpoint{4.227273in}{3.311000in}}%
\pgfusepath{clip}%
\pgfsetbuttcap%
\pgfsetroundjoin%
\definecolor{currentfill}{rgb}{0.993248,0.906157,0.143936}%
\pgfsetfillcolor{currentfill}%
\pgfsetfillopacity{0.700000}%
\pgfsetlinewidth{0.000000pt}%
\definecolor{currentstroke}{rgb}{0.000000,0.000000,0.000000}%
\pgfsetstrokecolor{currentstroke}%
\pgfsetstrokeopacity{0.700000}%
\pgfsetdash{}{0pt}%
\pgfpathmoveto{\pgfqpoint{9.020750in}{0.958497in}}%
\pgfpathcurveto{\pgfqpoint{9.025794in}{0.958497in}}{\pgfqpoint{9.030632in}{0.960501in}}{\pgfqpoint{9.034198in}{0.964068in}}%
\pgfpathcurveto{\pgfqpoint{9.037765in}{0.967634in}}{\pgfqpoint{9.039769in}{0.972472in}}{\pgfqpoint{9.039769in}{0.977515in}}%
\pgfpathcurveto{\pgfqpoint{9.039769in}{0.982559in}}{\pgfqpoint{9.037765in}{0.987397in}}{\pgfqpoint{9.034198in}{0.990963in}}%
\pgfpathcurveto{\pgfqpoint{9.030632in}{0.994530in}}{\pgfqpoint{9.025794in}{0.996534in}}{\pgfqpoint{9.020750in}{0.996534in}}%
\pgfpathcurveto{\pgfqpoint{9.015707in}{0.996534in}}{\pgfqpoint{9.010869in}{0.994530in}}{\pgfqpoint{9.007303in}{0.990963in}}%
\pgfpathcurveto{\pgfqpoint{9.003736in}{0.987397in}}{\pgfqpoint{9.001732in}{0.982559in}}{\pgfqpoint{9.001732in}{0.977515in}}%
\pgfpathcurveto{\pgfqpoint{9.001732in}{0.972472in}}{\pgfqpoint{9.003736in}{0.967634in}}{\pgfqpoint{9.007303in}{0.964068in}}%
\pgfpathcurveto{\pgfqpoint{9.010869in}{0.960501in}}{\pgfqpoint{9.015707in}{0.958497in}}{\pgfqpoint{9.020750in}{0.958497in}}%
\pgfpathclose%
\pgfusepath{fill}%
\end{pgfscope}%
\begin{pgfscope}%
\pgfpathrectangle{\pgfqpoint{6.572727in}{0.473000in}}{\pgfqpoint{4.227273in}{3.311000in}}%
\pgfusepath{clip}%
\pgfsetbuttcap%
\pgfsetroundjoin%
\definecolor{currentfill}{rgb}{0.127568,0.566949,0.550556}%
\pgfsetfillcolor{currentfill}%
\pgfsetfillopacity{0.700000}%
\pgfsetlinewidth{0.000000pt}%
\definecolor{currentstroke}{rgb}{0.000000,0.000000,0.000000}%
\pgfsetstrokecolor{currentstroke}%
\pgfsetstrokeopacity{0.700000}%
\pgfsetdash{}{0pt}%
\pgfpathmoveto{\pgfqpoint{8.228015in}{2.881154in}}%
\pgfpathcurveto{\pgfqpoint{8.233058in}{2.881154in}}{\pgfqpoint{8.237896in}{2.883158in}}{\pgfqpoint{8.241463in}{2.886724in}}%
\pgfpathcurveto{\pgfqpoint{8.245029in}{2.890291in}}{\pgfqpoint{8.247033in}{2.895128in}}{\pgfqpoint{8.247033in}{2.900172in}}%
\pgfpathcurveto{\pgfqpoint{8.247033in}{2.905216in}}{\pgfqpoint{8.245029in}{2.910053in}}{\pgfqpoint{8.241463in}{2.913620in}}%
\pgfpathcurveto{\pgfqpoint{8.237896in}{2.917186in}}{\pgfqpoint{8.233058in}{2.919190in}}{\pgfqpoint{8.228015in}{2.919190in}}%
\pgfpathcurveto{\pgfqpoint{8.222971in}{2.919190in}}{\pgfqpoint{8.218133in}{2.917186in}}{\pgfqpoint{8.214567in}{2.913620in}}%
\pgfpathcurveto{\pgfqpoint{8.211000in}{2.910053in}}{\pgfqpoint{8.208997in}{2.905216in}}{\pgfqpoint{8.208997in}{2.900172in}}%
\pgfpathcurveto{\pgfqpoint{8.208997in}{2.895128in}}{\pgfqpoint{8.211000in}{2.890291in}}{\pgfqpoint{8.214567in}{2.886724in}}%
\pgfpathcurveto{\pgfqpoint{8.218133in}{2.883158in}}{\pgfqpoint{8.222971in}{2.881154in}}{\pgfqpoint{8.228015in}{2.881154in}}%
\pgfpathclose%
\pgfusepath{fill}%
\end{pgfscope}%
\begin{pgfscope}%
\pgfpathrectangle{\pgfqpoint{6.572727in}{0.473000in}}{\pgfqpoint{4.227273in}{3.311000in}}%
\pgfusepath{clip}%
\pgfsetbuttcap%
\pgfsetroundjoin%
\definecolor{currentfill}{rgb}{0.993248,0.906157,0.143936}%
\pgfsetfillcolor{currentfill}%
\pgfsetfillopacity{0.700000}%
\pgfsetlinewidth{0.000000pt}%
\definecolor{currentstroke}{rgb}{0.000000,0.000000,0.000000}%
\pgfsetstrokecolor{currentstroke}%
\pgfsetstrokeopacity{0.700000}%
\pgfsetdash{}{0pt}%
\pgfpathmoveto{\pgfqpoint{9.623363in}{1.296293in}}%
\pgfpathcurveto{\pgfqpoint{9.628406in}{1.296293in}}{\pgfqpoint{9.633244in}{1.298297in}}{\pgfqpoint{9.636810in}{1.301863in}}%
\pgfpathcurveto{\pgfqpoint{9.640377in}{1.305429in}}{\pgfqpoint{9.642381in}{1.310267in}}{\pgfqpoint{9.642381in}{1.315311in}}%
\pgfpathcurveto{\pgfqpoint{9.642381in}{1.320354in}}{\pgfqpoint{9.640377in}{1.325192in}}{\pgfqpoint{9.636810in}{1.328759in}}%
\pgfpathcurveto{\pgfqpoint{9.633244in}{1.332325in}}{\pgfqpoint{9.628406in}{1.334329in}}{\pgfqpoint{9.623363in}{1.334329in}}%
\pgfpathcurveto{\pgfqpoint{9.618319in}{1.334329in}}{\pgfqpoint{9.613481in}{1.332325in}}{\pgfqpoint{9.609915in}{1.328759in}}%
\pgfpathcurveto{\pgfqpoint{9.606348in}{1.325192in}}{\pgfqpoint{9.604344in}{1.320354in}}{\pgfqpoint{9.604344in}{1.315311in}}%
\pgfpathcurveto{\pgfqpoint{9.604344in}{1.310267in}}{\pgfqpoint{9.606348in}{1.305429in}}{\pgfqpoint{9.609915in}{1.301863in}}%
\pgfpathcurveto{\pgfqpoint{9.613481in}{1.298297in}}{\pgfqpoint{9.618319in}{1.296293in}}{\pgfqpoint{9.623363in}{1.296293in}}%
\pgfpathclose%
\pgfusepath{fill}%
\end{pgfscope}%
\begin{pgfscope}%
\pgfpathrectangle{\pgfqpoint{6.572727in}{0.473000in}}{\pgfqpoint{4.227273in}{3.311000in}}%
\pgfusepath{clip}%
\pgfsetbuttcap%
\pgfsetroundjoin%
\definecolor{currentfill}{rgb}{0.993248,0.906157,0.143936}%
\pgfsetfillcolor{currentfill}%
\pgfsetfillopacity{0.700000}%
\pgfsetlinewidth{0.000000pt}%
\definecolor{currentstroke}{rgb}{0.000000,0.000000,0.000000}%
\pgfsetstrokecolor{currentstroke}%
\pgfsetstrokeopacity{0.700000}%
\pgfsetdash{}{0pt}%
\pgfpathmoveto{\pgfqpoint{9.234232in}{1.366790in}}%
\pgfpathcurveto{\pgfqpoint{9.239276in}{1.366790in}}{\pgfqpoint{9.244114in}{1.368794in}}{\pgfqpoint{9.247680in}{1.372360in}}%
\pgfpathcurveto{\pgfqpoint{9.251246in}{1.375927in}}{\pgfqpoint{9.253250in}{1.380765in}}{\pgfqpoint{9.253250in}{1.385808in}}%
\pgfpathcurveto{\pgfqpoint{9.253250in}{1.390852in}}{\pgfqpoint{9.251246in}{1.395690in}}{\pgfqpoint{9.247680in}{1.399256in}}%
\pgfpathcurveto{\pgfqpoint{9.244114in}{1.402823in}}{\pgfqpoint{9.239276in}{1.404826in}}{\pgfqpoint{9.234232in}{1.404826in}}%
\pgfpathcurveto{\pgfqpoint{9.229188in}{1.404826in}}{\pgfqpoint{9.224351in}{1.402823in}}{\pgfqpoint{9.220784in}{1.399256in}}%
\pgfpathcurveto{\pgfqpoint{9.217218in}{1.395690in}}{\pgfqpoint{9.215214in}{1.390852in}}{\pgfqpoint{9.215214in}{1.385808in}}%
\pgfpathcurveto{\pgfqpoint{9.215214in}{1.380765in}}{\pgfqpoint{9.217218in}{1.375927in}}{\pgfqpoint{9.220784in}{1.372360in}}%
\pgfpathcurveto{\pgfqpoint{9.224351in}{1.368794in}}{\pgfqpoint{9.229188in}{1.366790in}}{\pgfqpoint{9.234232in}{1.366790in}}%
\pgfpathclose%
\pgfusepath{fill}%
\end{pgfscope}%
\begin{pgfscope}%
\pgfpathrectangle{\pgfqpoint{6.572727in}{0.473000in}}{\pgfqpoint{4.227273in}{3.311000in}}%
\pgfusepath{clip}%
\pgfsetbuttcap%
\pgfsetroundjoin%
\definecolor{currentfill}{rgb}{0.127568,0.566949,0.550556}%
\pgfsetfillcolor{currentfill}%
\pgfsetfillopacity{0.700000}%
\pgfsetlinewidth{0.000000pt}%
\definecolor{currentstroke}{rgb}{0.000000,0.000000,0.000000}%
\pgfsetstrokecolor{currentstroke}%
\pgfsetstrokeopacity{0.700000}%
\pgfsetdash{}{0pt}%
\pgfpathmoveto{\pgfqpoint{7.543704in}{1.876943in}}%
\pgfpathcurveto{\pgfqpoint{7.548748in}{1.876943in}}{\pgfqpoint{7.553585in}{1.878947in}}{\pgfqpoint{7.557152in}{1.882514in}}%
\pgfpathcurveto{\pgfqpoint{7.560718in}{1.886080in}}{\pgfqpoint{7.562722in}{1.890918in}}{\pgfqpoint{7.562722in}{1.895961in}}%
\pgfpathcurveto{\pgfqpoint{7.562722in}{1.901005in}}{\pgfqpoint{7.560718in}{1.905843in}}{\pgfqpoint{7.557152in}{1.909409in}}%
\pgfpathcurveto{\pgfqpoint{7.553585in}{1.912976in}}{\pgfqpoint{7.548748in}{1.914980in}}{\pgfqpoint{7.543704in}{1.914980in}}%
\pgfpathcurveto{\pgfqpoint{7.538660in}{1.914980in}}{\pgfqpoint{7.533823in}{1.912976in}}{\pgfqpoint{7.530256in}{1.909409in}}%
\pgfpathcurveto{\pgfqpoint{7.526690in}{1.905843in}}{\pgfqpoint{7.524686in}{1.901005in}}{\pgfqpoint{7.524686in}{1.895961in}}%
\pgfpathcurveto{\pgfqpoint{7.524686in}{1.890918in}}{\pgfqpoint{7.526690in}{1.886080in}}{\pgfqpoint{7.530256in}{1.882514in}}%
\pgfpathcurveto{\pgfqpoint{7.533823in}{1.878947in}}{\pgfqpoint{7.538660in}{1.876943in}}{\pgfqpoint{7.543704in}{1.876943in}}%
\pgfpathclose%
\pgfusepath{fill}%
\end{pgfscope}%
\begin{pgfscope}%
\pgfpathrectangle{\pgfqpoint{6.572727in}{0.473000in}}{\pgfqpoint{4.227273in}{3.311000in}}%
\pgfusepath{clip}%
\pgfsetbuttcap%
\pgfsetroundjoin%
\definecolor{currentfill}{rgb}{0.127568,0.566949,0.550556}%
\pgfsetfillcolor{currentfill}%
\pgfsetfillopacity{0.700000}%
\pgfsetlinewidth{0.000000pt}%
\definecolor{currentstroke}{rgb}{0.000000,0.000000,0.000000}%
\pgfsetstrokecolor{currentstroke}%
\pgfsetstrokeopacity{0.700000}%
\pgfsetdash{}{0pt}%
\pgfpathmoveto{\pgfqpoint{7.281759in}{1.842407in}}%
\pgfpathcurveto{\pgfqpoint{7.286803in}{1.842407in}}{\pgfqpoint{7.291640in}{1.844411in}}{\pgfqpoint{7.295207in}{1.847977in}}%
\pgfpathcurveto{\pgfqpoint{7.298773in}{1.851544in}}{\pgfqpoint{7.300777in}{1.856382in}}{\pgfqpoint{7.300777in}{1.861425in}}%
\pgfpathcurveto{\pgfqpoint{7.300777in}{1.866469in}}{\pgfqpoint{7.298773in}{1.871307in}}{\pgfqpoint{7.295207in}{1.874873in}}%
\pgfpathcurveto{\pgfqpoint{7.291640in}{1.878440in}}{\pgfqpoint{7.286803in}{1.880443in}}{\pgfqpoint{7.281759in}{1.880443in}}%
\pgfpathcurveto{\pgfqpoint{7.276715in}{1.880443in}}{\pgfqpoint{7.271878in}{1.878440in}}{\pgfqpoint{7.268311in}{1.874873in}}%
\pgfpathcurveto{\pgfqpoint{7.264745in}{1.871307in}}{\pgfqpoint{7.262741in}{1.866469in}}{\pgfqpoint{7.262741in}{1.861425in}}%
\pgfpathcurveto{\pgfqpoint{7.262741in}{1.856382in}}{\pgfqpoint{7.264745in}{1.851544in}}{\pgfqpoint{7.268311in}{1.847977in}}%
\pgfpathcurveto{\pgfqpoint{7.271878in}{1.844411in}}{\pgfqpoint{7.276715in}{1.842407in}}{\pgfqpoint{7.281759in}{1.842407in}}%
\pgfpathclose%
\pgfusepath{fill}%
\end{pgfscope}%
\begin{pgfscope}%
\pgfpathrectangle{\pgfqpoint{6.572727in}{0.473000in}}{\pgfqpoint{4.227273in}{3.311000in}}%
\pgfusepath{clip}%
\pgfsetbuttcap%
\pgfsetroundjoin%
\definecolor{currentfill}{rgb}{0.127568,0.566949,0.550556}%
\pgfsetfillcolor{currentfill}%
\pgfsetfillopacity{0.700000}%
\pgfsetlinewidth{0.000000pt}%
\definecolor{currentstroke}{rgb}{0.000000,0.000000,0.000000}%
\pgfsetstrokecolor{currentstroke}%
\pgfsetstrokeopacity{0.700000}%
\pgfsetdash{}{0pt}%
\pgfpathmoveto{\pgfqpoint{7.755262in}{2.148767in}}%
\pgfpathcurveto{\pgfqpoint{7.760305in}{2.148767in}}{\pgfqpoint{7.765143in}{2.150771in}}{\pgfqpoint{7.768710in}{2.154338in}}%
\pgfpathcurveto{\pgfqpoint{7.772276in}{2.157904in}}{\pgfqpoint{7.774280in}{2.162742in}}{\pgfqpoint{7.774280in}{2.167785in}}%
\pgfpathcurveto{\pgfqpoint{7.774280in}{2.172829in}}{\pgfqpoint{7.772276in}{2.177667in}}{\pgfqpoint{7.768710in}{2.181233in}}%
\pgfpathcurveto{\pgfqpoint{7.765143in}{2.184800in}}{\pgfqpoint{7.760305in}{2.186804in}}{\pgfqpoint{7.755262in}{2.186804in}}%
\pgfpathcurveto{\pgfqpoint{7.750218in}{2.186804in}}{\pgfqpoint{7.745380in}{2.184800in}}{\pgfqpoint{7.741814in}{2.181233in}}%
\pgfpathcurveto{\pgfqpoint{7.738248in}{2.177667in}}{\pgfqpoint{7.736244in}{2.172829in}}{\pgfqpoint{7.736244in}{2.167785in}}%
\pgfpathcurveto{\pgfqpoint{7.736244in}{2.162742in}}{\pgfqpoint{7.738248in}{2.157904in}}{\pgfqpoint{7.741814in}{2.154338in}}%
\pgfpathcurveto{\pgfqpoint{7.745380in}{2.150771in}}{\pgfqpoint{7.750218in}{2.148767in}}{\pgfqpoint{7.755262in}{2.148767in}}%
\pgfpathclose%
\pgfusepath{fill}%
\end{pgfscope}%
\begin{pgfscope}%
\pgfpathrectangle{\pgfqpoint{6.572727in}{0.473000in}}{\pgfqpoint{4.227273in}{3.311000in}}%
\pgfusepath{clip}%
\pgfsetbuttcap%
\pgfsetroundjoin%
\definecolor{currentfill}{rgb}{0.127568,0.566949,0.550556}%
\pgfsetfillcolor{currentfill}%
\pgfsetfillopacity{0.700000}%
\pgfsetlinewidth{0.000000pt}%
\definecolor{currentstroke}{rgb}{0.000000,0.000000,0.000000}%
\pgfsetstrokecolor{currentstroke}%
\pgfsetstrokeopacity{0.700000}%
\pgfsetdash{}{0pt}%
\pgfpathmoveto{\pgfqpoint{8.323257in}{2.403047in}}%
\pgfpathcurveto{\pgfqpoint{8.328301in}{2.403047in}}{\pgfqpoint{8.333139in}{2.405051in}}{\pgfqpoint{8.336705in}{2.408617in}}%
\pgfpathcurveto{\pgfqpoint{8.340271in}{2.412183in}}{\pgfqpoint{8.342275in}{2.417021in}}{\pgfqpoint{8.342275in}{2.422065in}}%
\pgfpathcurveto{\pgfqpoint{8.342275in}{2.427108in}}{\pgfqpoint{8.340271in}{2.431946in}}{\pgfqpoint{8.336705in}{2.435513in}}%
\pgfpathcurveto{\pgfqpoint{8.333139in}{2.439079in}}{\pgfqpoint{8.328301in}{2.441083in}}{\pgfqpoint{8.323257in}{2.441083in}}%
\pgfpathcurveto{\pgfqpoint{8.318213in}{2.441083in}}{\pgfqpoint{8.313376in}{2.439079in}}{\pgfqpoint{8.309809in}{2.435513in}}%
\pgfpathcurveto{\pgfqpoint{8.306243in}{2.431946in}}{\pgfqpoint{8.304239in}{2.427108in}}{\pgfqpoint{8.304239in}{2.422065in}}%
\pgfpathcurveto{\pgfqpoint{8.304239in}{2.417021in}}{\pgfqpoint{8.306243in}{2.412183in}}{\pgfqpoint{8.309809in}{2.408617in}}%
\pgfpathcurveto{\pgfqpoint{8.313376in}{2.405051in}}{\pgfqpoint{8.318213in}{2.403047in}}{\pgfqpoint{8.323257in}{2.403047in}}%
\pgfpathclose%
\pgfusepath{fill}%
\end{pgfscope}%
\begin{pgfscope}%
\pgfpathrectangle{\pgfqpoint{6.572727in}{0.473000in}}{\pgfqpoint{4.227273in}{3.311000in}}%
\pgfusepath{clip}%
\pgfsetbuttcap%
\pgfsetroundjoin%
\definecolor{currentfill}{rgb}{0.127568,0.566949,0.550556}%
\pgfsetfillcolor{currentfill}%
\pgfsetfillopacity{0.700000}%
\pgfsetlinewidth{0.000000pt}%
\definecolor{currentstroke}{rgb}{0.000000,0.000000,0.000000}%
\pgfsetstrokecolor{currentstroke}%
\pgfsetstrokeopacity{0.700000}%
\pgfsetdash{}{0pt}%
\pgfpathmoveto{\pgfqpoint{8.300424in}{1.415503in}}%
\pgfpathcurveto{\pgfqpoint{8.305468in}{1.415503in}}{\pgfqpoint{8.310306in}{1.417506in}}{\pgfqpoint{8.313872in}{1.421073in}}%
\pgfpathcurveto{\pgfqpoint{8.317439in}{1.424639in}}{\pgfqpoint{8.319442in}{1.429477in}}{\pgfqpoint{8.319442in}{1.434521in}}%
\pgfpathcurveto{\pgfqpoint{8.319442in}{1.439564in}}{\pgfqpoint{8.317439in}{1.444402in}}{\pgfqpoint{8.313872in}{1.447969in}}%
\pgfpathcurveto{\pgfqpoint{8.310306in}{1.451535in}}{\pgfqpoint{8.305468in}{1.453539in}}{\pgfqpoint{8.300424in}{1.453539in}}%
\pgfpathcurveto{\pgfqpoint{8.295381in}{1.453539in}}{\pgfqpoint{8.290543in}{1.451535in}}{\pgfqpoint{8.286976in}{1.447969in}}%
\pgfpathcurveto{\pgfqpoint{8.283410in}{1.444402in}}{\pgfqpoint{8.281406in}{1.439564in}}{\pgfqpoint{8.281406in}{1.434521in}}%
\pgfpathcurveto{\pgfqpoint{8.281406in}{1.429477in}}{\pgfqpoint{8.283410in}{1.424639in}}{\pgfqpoint{8.286976in}{1.421073in}}%
\pgfpathcurveto{\pgfqpoint{8.290543in}{1.417506in}}{\pgfqpoint{8.295381in}{1.415503in}}{\pgfqpoint{8.300424in}{1.415503in}}%
\pgfpathclose%
\pgfusepath{fill}%
\end{pgfscope}%
\begin{pgfscope}%
\pgfpathrectangle{\pgfqpoint{6.572727in}{0.473000in}}{\pgfqpoint{4.227273in}{3.311000in}}%
\pgfusepath{clip}%
\pgfsetbuttcap%
\pgfsetroundjoin%
\definecolor{currentfill}{rgb}{0.993248,0.906157,0.143936}%
\pgfsetfillcolor{currentfill}%
\pgfsetfillopacity{0.700000}%
\pgfsetlinewidth{0.000000pt}%
\definecolor{currentstroke}{rgb}{0.000000,0.000000,0.000000}%
\pgfsetstrokecolor{currentstroke}%
\pgfsetstrokeopacity{0.700000}%
\pgfsetdash{}{0pt}%
\pgfpathmoveto{\pgfqpoint{9.664183in}{1.333482in}}%
\pgfpathcurveto{\pgfqpoint{9.669227in}{1.333482in}}{\pgfqpoint{9.674065in}{1.335486in}}{\pgfqpoint{9.677631in}{1.339052in}}%
\pgfpathcurveto{\pgfqpoint{9.681198in}{1.342619in}}{\pgfqpoint{9.683202in}{1.347456in}}{\pgfqpoint{9.683202in}{1.352500in}}%
\pgfpathcurveto{\pgfqpoint{9.683202in}{1.357544in}}{\pgfqpoint{9.681198in}{1.362381in}}{\pgfqpoint{9.677631in}{1.365948in}}%
\pgfpathcurveto{\pgfqpoint{9.674065in}{1.369514in}}{\pgfqpoint{9.669227in}{1.371518in}}{\pgfqpoint{9.664183in}{1.371518in}}%
\pgfpathcurveto{\pgfqpoint{9.659140in}{1.371518in}}{\pgfqpoint{9.654302in}{1.369514in}}{\pgfqpoint{9.650736in}{1.365948in}}%
\pgfpathcurveto{\pgfqpoint{9.647169in}{1.362381in}}{\pgfqpoint{9.645165in}{1.357544in}}{\pgfqpoint{9.645165in}{1.352500in}}%
\pgfpathcurveto{\pgfqpoint{9.645165in}{1.347456in}}{\pgfqpoint{9.647169in}{1.342619in}}{\pgfqpoint{9.650736in}{1.339052in}}%
\pgfpathcurveto{\pgfqpoint{9.654302in}{1.335486in}}{\pgfqpoint{9.659140in}{1.333482in}}{\pgfqpoint{9.664183in}{1.333482in}}%
\pgfpathclose%
\pgfusepath{fill}%
\end{pgfscope}%
\begin{pgfscope}%
\pgfpathrectangle{\pgfqpoint{6.572727in}{0.473000in}}{\pgfqpoint{4.227273in}{3.311000in}}%
\pgfusepath{clip}%
\pgfsetbuttcap%
\pgfsetroundjoin%
\definecolor{currentfill}{rgb}{0.127568,0.566949,0.550556}%
\pgfsetfillcolor{currentfill}%
\pgfsetfillopacity{0.700000}%
\pgfsetlinewidth{0.000000pt}%
\definecolor{currentstroke}{rgb}{0.000000,0.000000,0.000000}%
\pgfsetstrokecolor{currentstroke}%
\pgfsetstrokeopacity{0.700000}%
\pgfsetdash{}{0pt}%
\pgfpathmoveto{\pgfqpoint{7.833822in}{1.082560in}}%
\pgfpathcurveto{\pgfqpoint{7.838865in}{1.082560in}}{\pgfqpoint{7.843703in}{1.084564in}}{\pgfqpoint{7.847269in}{1.088130in}}%
\pgfpathcurveto{\pgfqpoint{7.850836in}{1.091697in}}{\pgfqpoint{7.852840in}{1.096535in}}{\pgfqpoint{7.852840in}{1.101578in}}%
\pgfpathcurveto{\pgfqpoint{7.852840in}{1.106622in}}{\pgfqpoint{7.850836in}{1.111460in}}{\pgfqpoint{7.847269in}{1.115026in}}%
\pgfpathcurveto{\pgfqpoint{7.843703in}{1.118592in}}{\pgfqpoint{7.838865in}{1.120596in}}{\pgfqpoint{7.833822in}{1.120596in}}%
\pgfpathcurveto{\pgfqpoint{7.828778in}{1.120596in}}{\pgfqpoint{7.823940in}{1.118592in}}{\pgfqpoint{7.820374in}{1.115026in}}%
\pgfpathcurveto{\pgfqpoint{7.816807in}{1.111460in}}{\pgfqpoint{7.814803in}{1.106622in}}{\pgfqpoint{7.814803in}{1.101578in}}%
\pgfpathcurveto{\pgfqpoint{7.814803in}{1.096535in}}{\pgfqpoint{7.816807in}{1.091697in}}{\pgfqpoint{7.820374in}{1.088130in}}%
\pgfpathcurveto{\pgfqpoint{7.823940in}{1.084564in}}{\pgfqpoint{7.828778in}{1.082560in}}{\pgfqpoint{7.833822in}{1.082560in}}%
\pgfpathclose%
\pgfusepath{fill}%
\end{pgfscope}%
\begin{pgfscope}%
\pgfpathrectangle{\pgfqpoint{6.572727in}{0.473000in}}{\pgfqpoint{4.227273in}{3.311000in}}%
\pgfusepath{clip}%
\pgfsetbuttcap%
\pgfsetroundjoin%
\definecolor{currentfill}{rgb}{0.127568,0.566949,0.550556}%
\pgfsetfillcolor{currentfill}%
\pgfsetfillopacity{0.700000}%
\pgfsetlinewidth{0.000000pt}%
\definecolor{currentstroke}{rgb}{0.000000,0.000000,0.000000}%
\pgfsetstrokecolor{currentstroke}%
\pgfsetstrokeopacity{0.700000}%
\pgfsetdash{}{0pt}%
\pgfpathmoveto{\pgfqpoint{8.321344in}{3.134122in}}%
\pgfpathcurveto{\pgfqpoint{8.326388in}{3.134122in}}{\pgfqpoint{8.331226in}{3.136126in}}{\pgfqpoint{8.334792in}{3.139692in}}%
\pgfpathcurveto{\pgfqpoint{8.338359in}{3.143259in}}{\pgfqpoint{8.340363in}{3.148096in}}{\pgfqpoint{8.340363in}{3.153140in}}%
\pgfpathcurveto{\pgfqpoint{8.340363in}{3.158184in}}{\pgfqpoint{8.338359in}{3.163022in}}{\pgfqpoint{8.334792in}{3.166588in}}%
\pgfpathcurveto{\pgfqpoint{8.331226in}{3.170154in}}{\pgfqpoint{8.326388in}{3.172158in}}{\pgfqpoint{8.321344in}{3.172158in}}%
\pgfpathcurveto{\pgfqpoint{8.316301in}{3.172158in}}{\pgfqpoint{8.311463in}{3.170154in}}{\pgfqpoint{8.307897in}{3.166588in}}%
\pgfpathcurveto{\pgfqpoint{8.304330in}{3.163022in}}{\pgfqpoint{8.302326in}{3.158184in}}{\pgfqpoint{8.302326in}{3.153140in}}%
\pgfpathcurveto{\pgfqpoint{8.302326in}{3.148096in}}{\pgfqpoint{8.304330in}{3.143259in}}{\pgfqpoint{8.307897in}{3.139692in}}%
\pgfpathcurveto{\pgfqpoint{8.311463in}{3.136126in}}{\pgfqpoint{8.316301in}{3.134122in}}{\pgfqpoint{8.321344in}{3.134122in}}%
\pgfpathclose%
\pgfusepath{fill}%
\end{pgfscope}%
\begin{pgfscope}%
\pgfpathrectangle{\pgfqpoint{6.572727in}{0.473000in}}{\pgfqpoint{4.227273in}{3.311000in}}%
\pgfusepath{clip}%
\pgfsetbuttcap%
\pgfsetroundjoin%
\definecolor{currentfill}{rgb}{0.127568,0.566949,0.550556}%
\pgfsetfillcolor{currentfill}%
\pgfsetfillopacity{0.700000}%
\pgfsetlinewidth{0.000000pt}%
\definecolor{currentstroke}{rgb}{0.000000,0.000000,0.000000}%
\pgfsetstrokecolor{currentstroke}%
\pgfsetstrokeopacity{0.700000}%
\pgfsetdash{}{0pt}%
\pgfpathmoveto{\pgfqpoint{8.114934in}{2.266581in}}%
\pgfpathcurveto{\pgfqpoint{8.119978in}{2.266581in}}{\pgfqpoint{8.124815in}{2.268585in}}{\pgfqpoint{8.128382in}{2.272152in}}%
\pgfpathcurveto{\pgfqpoint{8.131948in}{2.275718in}}{\pgfqpoint{8.133952in}{2.280556in}}{\pgfqpoint{8.133952in}{2.285599in}}%
\pgfpathcurveto{\pgfqpoint{8.133952in}{2.290643in}}{\pgfqpoint{8.131948in}{2.295481in}}{\pgfqpoint{8.128382in}{2.299047in}}%
\pgfpathcurveto{\pgfqpoint{8.124815in}{2.302614in}}{\pgfqpoint{8.119978in}{2.304618in}}{\pgfqpoint{8.114934in}{2.304618in}}%
\pgfpathcurveto{\pgfqpoint{8.109890in}{2.304618in}}{\pgfqpoint{8.105053in}{2.302614in}}{\pgfqpoint{8.101486in}{2.299047in}}%
\pgfpathcurveto{\pgfqpoint{8.097920in}{2.295481in}}{\pgfqpoint{8.095916in}{2.290643in}}{\pgfqpoint{8.095916in}{2.285599in}}%
\pgfpathcurveto{\pgfqpoint{8.095916in}{2.280556in}}{\pgfqpoint{8.097920in}{2.275718in}}{\pgfqpoint{8.101486in}{2.272152in}}%
\pgfpathcurveto{\pgfqpoint{8.105053in}{2.268585in}}{\pgfqpoint{8.109890in}{2.266581in}}{\pgfqpoint{8.114934in}{2.266581in}}%
\pgfpathclose%
\pgfusepath{fill}%
\end{pgfscope}%
\begin{pgfscope}%
\pgfpathrectangle{\pgfqpoint{6.572727in}{0.473000in}}{\pgfqpoint{4.227273in}{3.311000in}}%
\pgfusepath{clip}%
\pgfsetbuttcap%
\pgfsetroundjoin%
\definecolor{currentfill}{rgb}{0.993248,0.906157,0.143936}%
\pgfsetfillcolor{currentfill}%
\pgfsetfillopacity{0.700000}%
\pgfsetlinewidth{0.000000pt}%
\definecolor{currentstroke}{rgb}{0.000000,0.000000,0.000000}%
\pgfsetstrokecolor{currentstroke}%
\pgfsetstrokeopacity{0.700000}%
\pgfsetdash{}{0pt}%
\pgfpathmoveto{\pgfqpoint{9.606244in}{1.579269in}}%
\pgfpathcurveto{\pgfqpoint{9.611288in}{1.579269in}}{\pgfqpoint{9.616126in}{1.581273in}}{\pgfqpoint{9.619692in}{1.584839in}}%
\pgfpathcurveto{\pgfqpoint{9.623258in}{1.588405in}}{\pgfqpoint{9.625262in}{1.593243in}}{\pgfqpoint{9.625262in}{1.598287in}}%
\pgfpathcurveto{\pgfqpoint{9.625262in}{1.603331in}}{\pgfqpoint{9.623258in}{1.608168in}}{\pgfqpoint{9.619692in}{1.611735in}}%
\pgfpathcurveto{\pgfqpoint{9.616126in}{1.615301in}}{\pgfqpoint{9.611288in}{1.617305in}}{\pgfqpoint{9.606244in}{1.617305in}}%
\pgfpathcurveto{\pgfqpoint{9.601201in}{1.617305in}}{\pgfqpoint{9.596363in}{1.615301in}}{\pgfqpoint{9.592796in}{1.611735in}}%
\pgfpathcurveto{\pgfqpoint{9.589230in}{1.608168in}}{\pgfqpoint{9.587226in}{1.603331in}}{\pgfqpoint{9.587226in}{1.598287in}}%
\pgfpathcurveto{\pgfqpoint{9.587226in}{1.593243in}}{\pgfqpoint{9.589230in}{1.588405in}}{\pgfqpoint{9.592796in}{1.584839in}}%
\pgfpathcurveto{\pgfqpoint{9.596363in}{1.581273in}}{\pgfqpoint{9.601201in}{1.579269in}}{\pgfqpoint{9.606244in}{1.579269in}}%
\pgfpathclose%
\pgfusepath{fill}%
\end{pgfscope}%
\begin{pgfscope}%
\pgfpathrectangle{\pgfqpoint{6.572727in}{0.473000in}}{\pgfqpoint{4.227273in}{3.311000in}}%
\pgfusepath{clip}%
\pgfsetbuttcap%
\pgfsetroundjoin%
\definecolor{currentfill}{rgb}{0.127568,0.566949,0.550556}%
\pgfsetfillcolor{currentfill}%
\pgfsetfillopacity{0.700000}%
\pgfsetlinewidth{0.000000pt}%
\definecolor{currentstroke}{rgb}{0.000000,0.000000,0.000000}%
\pgfsetstrokecolor{currentstroke}%
\pgfsetstrokeopacity{0.700000}%
\pgfsetdash{}{0pt}%
\pgfpathmoveto{\pgfqpoint{7.802058in}{2.335533in}}%
\pgfpathcurveto{\pgfqpoint{7.807101in}{2.335533in}}{\pgfqpoint{7.811939in}{2.337537in}}{\pgfqpoint{7.815505in}{2.341103in}}%
\pgfpathcurveto{\pgfqpoint{7.819072in}{2.344670in}}{\pgfqpoint{7.821076in}{2.349507in}}{\pgfqpoint{7.821076in}{2.354551in}}%
\pgfpathcurveto{\pgfqpoint{7.821076in}{2.359595in}}{\pgfqpoint{7.819072in}{2.364432in}}{\pgfqpoint{7.815505in}{2.367999in}}%
\pgfpathcurveto{\pgfqpoint{7.811939in}{2.371565in}}{\pgfqpoint{7.807101in}{2.373569in}}{\pgfqpoint{7.802058in}{2.373569in}}%
\pgfpathcurveto{\pgfqpoint{7.797014in}{2.373569in}}{\pgfqpoint{7.792176in}{2.371565in}}{\pgfqpoint{7.788610in}{2.367999in}}%
\pgfpathcurveto{\pgfqpoint{7.785043in}{2.364432in}}{\pgfqpoint{7.783039in}{2.359595in}}{\pgfqpoint{7.783039in}{2.354551in}}%
\pgfpathcurveto{\pgfqpoint{7.783039in}{2.349507in}}{\pgfqpoint{7.785043in}{2.344670in}}{\pgfqpoint{7.788610in}{2.341103in}}%
\pgfpathcurveto{\pgfqpoint{7.792176in}{2.337537in}}{\pgfqpoint{7.797014in}{2.335533in}}{\pgfqpoint{7.802058in}{2.335533in}}%
\pgfpathclose%
\pgfusepath{fill}%
\end{pgfscope}%
\begin{pgfscope}%
\pgfpathrectangle{\pgfqpoint{6.572727in}{0.473000in}}{\pgfqpoint{4.227273in}{3.311000in}}%
\pgfusepath{clip}%
\pgfsetbuttcap%
\pgfsetroundjoin%
\definecolor{currentfill}{rgb}{0.993248,0.906157,0.143936}%
\pgfsetfillcolor{currentfill}%
\pgfsetfillopacity{0.700000}%
\pgfsetlinewidth{0.000000pt}%
\definecolor{currentstroke}{rgb}{0.000000,0.000000,0.000000}%
\pgfsetstrokecolor{currentstroke}%
\pgfsetstrokeopacity{0.700000}%
\pgfsetdash{}{0pt}%
\pgfpathmoveto{\pgfqpoint{9.261556in}{1.252259in}}%
\pgfpathcurveto{\pgfqpoint{9.266599in}{1.252259in}}{\pgfqpoint{9.271437in}{1.254263in}}{\pgfqpoint{9.275003in}{1.257830in}}%
\pgfpathcurveto{\pgfqpoint{9.278570in}{1.261396in}}{\pgfqpoint{9.280574in}{1.266234in}}{\pgfqpoint{9.280574in}{1.271278in}}%
\pgfpathcurveto{\pgfqpoint{9.280574in}{1.276321in}}{\pgfqpoint{9.278570in}{1.281159in}}{\pgfqpoint{9.275003in}{1.284725in}}%
\pgfpathcurveto{\pgfqpoint{9.271437in}{1.288292in}}{\pgfqpoint{9.266599in}{1.290296in}}{\pgfqpoint{9.261556in}{1.290296in}}%
\pgfpathcurveto{\pgfqpoint{9.256512in}{1.290296in}}{\pgfqpoint{9.251674in}{1.288292in}}{\pgfqpoint{9.248108in}{1.284725in}}%
\pgfpathcurveto{\pgfqpoint{9.244541in}{1.281159in}}{\pgfqpoint{9.242537in}{1.276321in}}{\pgfqpoint{9.242537in}{1.271278in}}%
\pgfpathcurveto{\pgfqpoint{9.242537in}{1.266234in}}{\pgfqpoint{9.244541in}{1.261396in}}{\pgfqpoint{9.248108in}{1.257830in}}%
\pgfpathcurveto{\pgfqpoint{9.251674in}{1.254263in}}{\pgfqpoint{9.256512in}{1.252259in}}{\pgfqpoint{9.261556in}{1.252259in}}%
\pgfpathclose%
\pgfusepath{fill}%
\end{pgfscope}%
\begin{pgfscope}%
\pgfpathrectangle{\pgfqpoint{6.572727in}{0.473000in}}{\pgfqpoint{4.227273in}{3.311000in}}%
\pgfusepath{clip}%
\pgfsetbuttcap%
\pgfsetroundjoin%
\definecolor{currentfill}{rgb}{0.127568,0.566949,0.550556}%
\pgfsetfillcolor{currentfill}%
\pgfsetfillopacity{0.700000}%
\pgfsetlinewidth{0.000000pt}%
\definecolor{currentstroke}{rgb}{0.000000,0.000000,0.000000}%
\pgfsetstrokecolor{currentstroke}%
\pgfsetstrokeopacity{0.700000}%
\pgfsetdash{}{0pt}%
\pgfpathmoveto{\pgfqpoint{8.208631in}{2.585411in}}%
\pgfpathcurveto{\pgfqpoint{8.213675in}{2.585411in}}{\pgfqpoint{8.218513in}{2.587415in}}{\pgfqpoint{8.222079in}{2.590982in}}%
\pgfpathcurveto{\pgfqpoint{8.225645in}{2.594548in}}{\pgfqpoint{8.227649in}{2.599386in}}{\pgfqpoint{8.227649in}{2.604430in}}%
\pgfpathcurveto{\pgfqpoint{8.227649in}{2.609473in}}{\pgfqpoint{8.225645in}{2.614311in}}{\pgfqpoint{8.222079in}{2.617877in}}%
\pgfpathcurveto{\pgfqpoint{8.218513in}{2.621444in}}{\pgfqpoint{8.213675in}{2.623448in}}{\pgfqpoint{8.208631in}{2.623448in}}%
\pgfpathcurveto{\pgfqpoint{8.203588in}{2.623448in}}{\pgfqpoint{8.198750in}{2.621444in}}{\pgfqpoint{8.195183in}{2.617877in}}%
\pgfpathcurveto{\pgfqpoint{8.191617in}{2.614311in}}{\pgfqpoint{8.189613in}{2.609473in}}{\pgfqpoint{8.189613in}{2.604430in}}%
\pgfpathcurveto{\pgfqpoint{8.189613in}{2.599386in}}{\pgfqpoint{8.191617in}{2.594548in}}{\pgfqpoint{8.195183in}{2.590982in}}%
\pgfpathcurveto{\pgfqpoint{8.198750in}{2.587415in}}{\pgfqpoint{8.203588in}{2.585411in}}{\pgfqpoint{8.208631in}{2.585411in}}%
\pgfpathclose%
\pgfusepath{fill}%
\end{pgfscope}%
\begin{pgfscope}%
\pgfpathrectangle{\pgfqpoint{6.572727in}{0.473000in}}{\pgfqpoint{4.227273in}{3.311000in}}%
\pgfusepath{clip}%
\pgfsetbuttcap%
\pgfsetroundjoin%
\definecolor{currentfill}{rgb}{0.127568,0.566949,0.550556}%
\pgfsetfillcolor{currentfill}%
\pgfsetfillopacity{0.700000}%
\pgfsetlinewidth{0.000000pt}%
\definecolor{currentstroke}{rgb}{0.000000,0.000000,0.000000}%
\pgfsetstrokecolor{currentstroke}%
\pgfsetstrokeopacity{0.700000}%
\pgfsetdash{}{0pt}%
\pgfpathmoveto{\pgfqpoint{7.589720in}{1.917715in}}%
\pgfpathcurveto{\pgfqpoint{7.594764in}{1.917715in}}{\pgfqpoint{7.599601in}{1.919718in}}{\pgfqpoint{7.603168in}{1.923285in}}%
\pgfpathcurveto{\pgfqpoint{7.606734in}{1.926851in}}{\pgfqpoint{7.608738in}{1.931689in}}{\pgfqpoint{7.608738in}{1.936733in}}%
\pgfpathcurveto{\pgfqpoint{7.608738in}{1.941776in}}{\pgfqpoint{7.606734in}{1.946614in}}{\pgfqpoint{7.603168in}{1.950181in}}%
\pgfpathcurveto{\pgfqpoint{7.599601in}{1.953747in}}{\pgfqpoint{7.594764in}{1.955751in}}{\pgfqpoint{7.589720in}{1.955751in}}%
\pgfpathcurveto{\pgfqpoint{7.584676in}{1.955751in}}{\pgfqpoint{7.579838in}{1.953747in}}{\pgfqpoint{7.576272in}{1.950181in}}%
\pgfpathcurveto{\pgfqpoint{7.572706in}{1.946614in}}{\pgfqpoint{7.570702in}{1.941776in}}{\pgfqpoint{7.570702in}{1.936733in}}%
\pgfpathcurveto{\pgfqpoint{7.570702in}{1.931689in}}{\pgfqpoint{7.572706in}{1.926851in}}{\pgfqpoint{7.576272in}{1.923285in}}%
\pgfpathcurveto{\pgfqpoint{7.579838in}{1.919718in}}{\pgfqpoint{7.584676in}{1.917715in}}{\pgfqpoint{7.589720in}{1.917715in}}%
\pgfpathclose%
\pgfusepath{fill}%
\end{pgfscope}%
\begin{pgfscope}%
\pgfpathrectangle{\pgfqpoint{6.572727in}{0.473000in}}{\pgfqpoint{4.227273in}{3.311000in}}%
\pgfusepath{clip}%
\pgfsetbuttcap%
\pgfsetroundjoin%
\definecolor{currentfill}{rgb}{0.127568,0.566949,0.550556}%
\pgfsetfillcolor{currentfill}%
\pgfsetfillopacity{0.700000}%
\pgfsetlinewidth{0.000000pt}%
\definecolor{currentstroke}{rgb}{0.000000,0.000000,0.000000}%
\pgfsetstrokecolor{currentstroke}%
\pgfsetstrokeopacity{0.700000}%
\pgfsetdash{}{0pt}%
\pgfpathmoveto{\pgfqpoint{7.919911in}{1.986993in}}%
\pgfpathcurveto{\pgfqpoint{7.924954in}{1.986993in}}{\pgfqpoint{7.929792in}{1.988997in}}{\pgfqpoint{7.933358in}{1.992563in}}%
\pgfpathcurveto{\pgfqpoint{7.936925in}{1.996130in}}{\pgfqpoint{7.938929in}{2.000967in}}{\pgfqpoint{7.938929in}{2.006011in}}%
\pgfpathcurveto{\pgfqpoint{7.938929in}{2.011055in}}{\pgfqpoint{7.936925in}{2.015892in}}{\pgfqpoint{7.933358in}{2.019459in}}%
\pgfpathcurveto{\pgfqpoint{7.929792in}{2.023025in}}{\pgfqpoint{7.924954in}{2.025029in}}{\pgfqpoint{7.919911in}{2.025029in}}%
\pgfpathcurveto{\pgfqpoint{7.914867in}{2.025029in}}{\pgfqpoint{7.910029in}{2.023025in}}{\pgfqpoint{7.906463in}{2.019459in}}%
\pgfpathcurveto{\pgfqpoint{7.902896in}{2.015892in}}{\pgfqpoint{7.900892in}{2.011055in}}{\pgfqpoint{7.900892in}{2.006011in}}%
\pgfpathcurveto{\pgfqpoint{7.900892in}{2.000967in}}{\pgfqpoint{7.902896in}{1.996130in}}{\pgfqpoint{7.906463in}{1.992563in}}%
\pgfpathcurveto{\pgfqpoint{7.910029in}{1.988997in}}{\pgfqpoint{7.914867in}{1.986993in}}{\pgfqpoint{7.919911in}{1.986993in}}%
\pgfpathclose%
\pgfusepath{fill}%
\end{pgfscope}%
\begin{pgfscope}%
\pgfpathrectangle{\pgfqpoint{6.572727in}{0.473000in}}{\pgfqpoint{4.227273in}{3.311000in}}%
\pgfusepath{clip}%
\pgfsetbuttcap%
\pgfsetroundjoin%
\definecolor{currentfill}{rgb}{0.993248,0.906157,0.143936}%
\pgfsetfillcolor{currentfill}%
\pgfsetfillopacity{0.700000}%
\pgfsetlinewidth{0.000000pt}%
\definecolor{currentstroke}{rgb}{0.000000,0.000000,0.000000}%
\pgfsetstrokecolor{currentstroke}%
\pgfsetstrokeopacity{0.700000}%
\pgfsetdash{}{0pt}%
\pgfpathmoveto{\pgfqpoint{9.923362in}{1.632489in}}%
\pgfpathcurveto{\pgfqpoint{9.928405in}{1.632489in}}{\pgfqpoint{9.933243in}{1.634493in}}{\pgfqpoint{9.936809in}{1.638060in}}%
\pgfpathcurveto{\pgfqpoint{9.940376in}{1.641626in}}{\pgfqpoint{9.942380in}{1.646464in}}{\pgfqpoint{9.942380in}{1.651507in}}%
\pgfpathcurveto{\pgfqpoint{9.942380in}{1.656551in}}{\pgfqpoint{9.940376in}{1.661389in}}{\pgfqpoint{9.936809in}{1.664955in}}%
\pgfpathcurveto{\pgfqpoint{9.933243in}{1.668522in}}{\pgfqpoint{9.928405in}{1.670526in}}{\pgfqpoint{9.923362in}{1.670526in}}%
\pgfpathcurveto{\pgfqpoint{9.918318in}{1.670526in}}{\pgfqpoint{9.913480in}{1.668522in}}{\pgfqpoint{9.909914in}{1.664955in}}%
\pgfpathcurveto{\pgfqpoint{9.906347in}{1.661389in}}{\pgfqpoint{9.904343in}{1.656551in}}{\pgfqpoint{9.904343in}{1.651507in}}%
\pgfpathcurveto{\pgfqpoint{9.904343in}{1.646464in}}{\pgfqpoint{9.906347in}{1.641626in}}{\pgfqpoint{9.909914in}{1.638060in}}%
\pgfpathcurveto{\pgfqpoint{9.913480in}{1.634493in}}{\pgfqpoint{9.918318in}{1.632489in}}{\pgfqpoint{9.923362in}{1.632489in}}%
\pgfpathclose%
\pgfusepath{fill}%
\end{pgfscope}%
\begin{pgfscope}%
\pgfpathrectangle{\pgfqpoint{6.572727in}{0.473000in}}{\pgfqpoint{4.227273in}{3.311000in}}%
\pgfusepath{clip}%
\pgfsetbuttcap%
\pgfsetroundjoin%
\definecolor{currentfill}{rgb}{0.127568,0.566949,0.550556}%
\pgfsetfillcolor{currentfill}%
\pgfsetfillopacity{0.700000}%
\pgfsetlinewidth{0.000000pt}%
\definecolor{currentstroke}{rgb}{0.000000,0.000000,0.000000}%
\pgfsetstrokecolor{currentstroke}%
\pgfsetstrokeopacity{0.700000}%
\pgfsetdash{}{0pt}%
\pgfpathmoveto{\pgfqpoint{8.561788in}{1.918929in}}%
\pgfpathcurveto{\pgfqpoint{8.566831in}{1.918929in}}{\pgfqpoint{8.571669in}{1.920933in}}{\pgfqpoint{8.575236in}{1.924500in}}%
\pgfpathcurveto{\pgfqpoint{8.578802in}{1.928066in}}{\pgfqpoint{8.580806in}{1.932904in}}{\pgfqpoint{8.580806in}{1.937948in}}%
\pgfpathcurveto{\pgfqpoint{8.580806in}{1.942991in}}{\pgfqpoint{8.578802in}{1.947829in}}{\pgfqpoint{8.575236in}{1.951395in}}%
\pgfpathcurveto{\pgfqpoint{8.571669in}{1.954962in}}{\pgfqpoint{8.566831in}{1.956966in}}{\pgfqpoint{8.561788in}{1.956966in}}%
\pgfpathcurveto{\pgfqpoint{8.556744in}{1.956966in}}{\pgfqpoint{8.551906in}{1.954962in}}{\pgfqpoint{8.548340in}{1.951395in}}%
\pgfpathcurveto{\pgfqpoint{8.544773in}{1.947829in}}{\pgfqpoint{8.542770in}{1.942991in}}{\pgfqpoint{8.542770in}{1.937948in}}%
\pgfpathcurveto{\pgfqpoint{8.542770in}{1.932904in}}{\pgfqpoint{8.544773in}{1.928066in}}{\pgfqpoint{8.548340in}{1.924500in}}%
\pgfpathcurveto{\pgfqpoint{8.551906in}{1.920933in}}{\pgfqpoint{8.556744in}{1.918929in}}{\pgfqpoint{8.561788in}{1.918929in}}%
\pgfpathclose%
\pgfusepath{fill}%
\end{pgfscope}%
\begin{pgfscope}%
\pgfpathrectangle{\pgfqpoint{6.572727in}{0.473000in}}{\pgfqpoint{4.227273in}{3.311000in}}%
\pgfusepath{clip}%
\pgfsetbuttcap%
\pgfsetroundjoin%
\definecolor{currentfill}{rgb}{0.127568,0.566949,0.550556}%
\pgfsetfillcolor{currentfill}%
\pgfsetfillopacity{0.700000}%
\pgfsetlinewidth{0.000000pt}%
\definecolor{currentstroke}{rgb}{0.000000,0.000000,0.000000}%
\pgfsetstrokecolor{currentstroke}%
\pgfsetstrokeopacity{0.700000}%
\pgfsetdash{}{0pt}%
\pgfpathmoveto{\pgfqpoint{8.647684in}{2.729024in}}%
\pgfpathcurveto{\pgfqpoint{8.652728in}{2.729024in}}{\pgfqpoint{8.657566in}{2.731027in}}{\pgfqpoint{8.661132in}{2.734594in}}%
\pgfpathcurveto{\pgfqpoint{8.664698in}{2.738160in}}{\pgfqpoint{8.666702in}{2.742998in}}{\pgfqpoint{8.666702in}{2.748042in}}%
\pgfpathcurveto{\pgfqpoint{8.666702in}{2.753085in}}{\pgfqpoint{8.664698in}{2.757923in}}{\pgfqpoint{8.661132in}{2.761490in}}%
\pgfpathcurveto{\pgfqpoint{8.657566in}{2.765056in}}{\pgfqpoint{8.652728in}{2.767060in}}{\pgfqpoint{8.647684in}{2.767060in}}%
\pgfpathcurveto{\pgfqpoint{8.642640in}{2.767060in}}{\pgfqpoint{8.637803in}{2.765056in}}{\pgfqpoint{8.634236in}{2.761490in}}%
\pgfpathcurveto{\pgfqpoint{8.630670in}{2.757923in}}{\pgfqpoint{8.628666in}{2.753085in}}{\pgfqpoint{8.628666in}{2.748042in}}%
\pgfpathcurveto{\pgfqpoint{8.628666in}{2.742998in}}{\pgfqpoint{8.630670in}{2.738160in}}{\pgfqpoint{8.634236in}{2.734594in}}%
\pgfpathcurveto{\pgfqpoint{8.637803in}{2.731027in}}{\pgfqpoint{8.642640in}{2.729024in}}{\pgfqpoint{8.647684in}{2.729024in}}%
\pgfpathclose%
\pgfusepath{fill}%
\end{pgfscope}%
\begin{pgfscope}%
\pgfpathrectangle{\pgfqpoint{6.572727in}{0.473000in}}{\pgfqpoint{4.227273in}{3.311000in}}%
\pgfusepath{clip}%
\pgfsetbuttcap%
\pgfsetroundjoin%
\definecolor{currentfill}{rgb}{0.127568,0.566949,0.550556}%
\pgfsetfillcolor{currentfill}%
\pgfsetfillopacity{0.700000}%
\pgfsetlinewidth{0.000000pt}%
\definecolor{currentstroke}{rgb}{0.000000,0.000000,0.000000}%
\pgfsetstrokecolor{currentstroke}%
\pgfsetstrokeopacity{0.700000}%
\pgfsetdash{}{0pt}%
\pgfpathmoveto{\pgfqpoint{8.316492in}{2.211350in}}%
\pgfpathcurveto{\pgfqpoint{8.321535in}{2.211350in}}{\pgfqpoint{8.326373in}{2.213354in}}{\pgfqpoint{8.329940in}{2.216920in}}%
\pgfpathcurveto{\pgfqpoint{8.333506in}{2.220487in}}{\pgfqpoint{8.335510in}{2.225324in}}{\pgfqpoint{8.335510in}{2.230368in}}%
\pgfpathcurveto{\pgfqpoint{8.335510in}{2.235412in}}{\pgfqpoint{8.333506in}{2.240249in}}{\pgfqpoint{8.329940in}{2.243816in}}%
\pgfpathcurveto{\pgfqpoint{8.326373in}{2.247382in}}{\pgfqpoint{8.321535in}{2.249386in}}{\pgfqpoint{8.316492in}{2.249386in}}%
\pgfpathcurveto{\pgfqpoint{8.311448in}{2.249386in}}{\pgfqpoint{8.306610in}{2.247382in}}{\pgfqpoint{8.303044in}{2.243816in}}%
\pgfpathcurveto{\pgfqpoint{8.299478in}{2.240249in}}{\pgfqpoint{8.297474in}{2.235412in}}{\pgfqpoint{8.297474in}{2.230368in}}%
\pgfpathcurveto{\pgfqpoint{8.297474in}{2.225324in}}{\pgfqpoint{8.299478in}{2.220487in}}{\pgfqpoint{8.303044in}{2.216920in}}%
\pgfpathcurveto{\pgfqpoint{8.306610in}{2.213354in}}{\pgfqpoint{8.311448in}{2.211350in}}{\pgfqpoint{8.316492in}{2.211350in}}%
\pgfpathclose%
\pgfusepath{fill}%
\end{pgfscope}%
\begin{pgfscope}%
\pgfpathrectangle{\pgfqpoint{6.572727in}{0.473000in}}{\pgfqpoint{4.227273in}{3.311000in}}%
\pgfusepath{clip}%
\pgfsetbuttcap%
\pgfsetroundjoin%
\definecolor{currentfill}{rgb}{0.127568,0.566949,0.550556}%
\pgfsetfillcolor{currentfill}%
\pgfsetfillopacity{0.700000}%
\pgfsetlinewidth{0.000000pt}%
\definecolor{currentstroke}{rgb}{0.000000,0.000000,0.000000}%
\pgfsetstrokecolor{currentstroke}%
\pgfsetstrokeopacity{0.700000}%
\pgfsetdash{}{0pt}%
\pgfpathmoveto{\pgfqpoint{8.881533in}{3.036386in}}%
\pgfpathcurveto{\pgfqpoint{8.886576in}{3.036386in}}{\pgfqpoint{8.891414in}{3.038389in}}{\pgfqpoint{8.894980in}{3.041956in}}%
\pgfpathcurveto{\pgfqpoint{8.898547in}{3.045522in}}{\pgfqpoint{8.900551in}{3.050360in}}{\pgfqpoint{8.900551in}{3.055404in}}%
\pgfpathcurveto{\pgfqpoint{8.900551in}{3.060447in}}{\pgfqpoint{8.898547in}{3.065285in}}{\pgfqpoint{8.894980in}{3.068852in}}%
\pgfpathcurveto{\pgfqpoint{8.891414in}{3.072418in}}{\pgfqpoint{8.886576in}{3.074422in}}{\pgfqpoint{8.881533in}{3.074422in}}%
\pgfpathcurveto{\pgfqpoint{8.876489in}{3.074422in}}{\pgfqpoint{8.871651in}{3.072418in}}{\pgfqpoint{8.868085in}{3.068852in}}%
\pgfpathcurveto{\pgfqpoint{8.864518in}{3.065285in}}{\pgfqpoint{8.862514in}{3.060447in}}{\pgfqpoint{8.862514in}{3.055404in}}%
\pgfpathcurveto{\pgfqpoint{8.862514in}{3.050360in}}{\pgfqpoint{8.864518in}{3.045522in}}{\pgfqpoint{8.868085in}{3.041956in}}%
\pgfpathcurveto{\pgfqpoint{8.871651in}{3.038389in}}{\pgfqpoint{8.876489in}{3.036386in}}{\pgfqpoint{8.881533in}{3.036386in}}%
\pgfpathclose%
\pgfusepath{fill}%
\end{pgfscope}%
\begin{pgfscope}%
\pgfpathrectangle{\pgfqpoint{6.572727in}{0.473000in}}{\pgfqpoint{4.227273in}{3.311000in}}%
\pgfusepath{clip}%
\pgfsetbuttcap%
\pgfsetroundjoin%
\definecolor{currentfill}{rgb}{0.127568,0.566949,0.550556}%
\pgfsetfillcolor{currentfill}%
\pgfsetfillopacity{0.700000}%
\pgfsetlinewidth{0.000000pt}%
\definecolor{currentstroke}{rgb}{0.000000,0.000000,0.000000}%
\pgfsetstrokecolor{currentstroke}%
\pgfsetstrokeopacity{0.700000}%
\pgfsetdash{}{0pt}%
\pgfpathmoveto{\pgfqpoint{7.455704in}{1.991892in}}%
\pgfpathcurveto{\pgfqpoint{7.460748in}{1.991892in}}{\pgfqpoint{7.465586in}{1.993896in}}{\pgfqpoint{7.469152in}{1.997463in}}%
\pgfpathcurveto{\pgfqpoint{7.472718in}{2.001029in}}{\pgfqpoint{7.474722in}{2.005867in}}{\pgfqpoint{7.474722in}{2.010911in}}%
\pgfpathcurveto{\pgfqpoint{7.474722in}{2.015954in}}{\pgfqpoint{7.472718in}{2.020792in}}{\pgfqpoint{7.469152in}{2.024358in}}%
\pgfpathcurveto{\pgfqpoint{7.465586in}{2.027925in}}{\pgfqpoint{7.460748in}{2.029929in}}{\pgfqpoint{7.455704in}{2.029929in}}%
\pgfpathcurveto{\pgfqpoint{7.450661in}{2.029929in}}{\pgfqpoint{7.445823in}{2.027925in}}{\pgfqpoint{7.442256in}{2.024358in}}%
\pgfpathcurveto{\pgfqpoint{7.438690in}{2.020792in}}{\pgfqpoint{7.436686in}{2.015954in}}{\pgfqpoint{7.436686in}{2.010911in}}%
\pgfpathcurveto{\pgfqpoint{7.436686in}{2.005867in}}{\pgfqpoint{7.438690in}{2.001029in}}{\pgfqpoint{7.442256in}{1.997463in}}%
\pgfpathcurveto{\pgfqpoint{7.445823in}{1.993896in}}{\pgfqpoint{7.450661in}{1.991892in}}{\pgfqpoint{7.455704in}{1.991892in}}%
\pgfpathclose%
\pgfusepath{fill}%
\end{pgfscope}%
\begin{pgfscope}%
\pgfpathrectangle{\pgfqpoint{6.572727in}{0.473000in}}{\pgfqpoint{4.227273in}{3.311000in}}%
\pgfusepath{clip}%
\pgfsetbuttcap%
\pgfsetroundjoin%
\definecolor{currentfill}{rgb}{0.127568,0.566949,0.550556}%
\pgfsetfillcolor{currentfill}%
\pgfsetfillopacity{0.700000}%
\pgfsetlinewidth{0.000000pt}%
\definecolor{currentstroke}{rgb}{0.000000,0.000000,0.000000}%
\pgfsetstrokecolor{currentstroke}%
\pgfsetstrokeopacity{0.700000}%
\pgfsetdash{}{0pt}%
\pgfpathmoveto{\pgfqpoint{8.261768in}{1.759038in}}%
\pgfpathcurveto{\pgfqpoint{8.266812in}{1.759038in}}{\pgfqpoint{8.271650in}{1.761042in}}{\pgfqpoint{8.275216in}{1.764608in}}%
\pgfpathcurveto{\pgfqpoint{8.278783in}{1.768175in}}{\pgfqpoint{8.280786in}{1.773013in}}{\pgfqpoint{8.280786in}{1.778056in}}%
\pgfpathcurveto{\pgfqpoint{8.280786in}{1.783100in}}{\pgfqpoint{8.278783in}{1.787938in}}{\pgfqpoint{8.275216in}{1.791504in}}%
\pgfpathcurveto{\pgfqpoint{8.271650in}{1.795070in}}{\pgfqpoint{8.266812in}{1.797074in}}{\pgfqpoint{8.261768in}{1.797074in}}%
\pgfpathcurveto{\pgfqpoint{8.256725in}{1.797074in}}{\pgfqpoint{8.251887in}{1.795070in}}{\pgfqpoint{8.248320in}{1.791504in}}%
\pgfpathcurveto{\pgfqpoint{8.244754in}{1.787938in}}{\pgfqpoint{8.242750in}{1.783100in}}{\pgfqpoint{8.242750in}{1.778056in}}%
\pgfpathcurveto{\pgfqpoint{8.242750in}{1.773013in}}{\pgfqpoint{8.244754in}{1.768175in}}{\pgfqpoint{8.248320in}{1.764608in}}%
\pgfpathcurveto{\pgfqpoint{8.251887in}{1.761042in}}{\pgfqpoint{8.256725in}{1.759038in}}{\pgfqpoint{8.261768in}{1.759038in}}%
\pgfpathclose%
\pgfusepath{fill}%
\end{pgfscope}%
\begin{pgfscope}%
\pgfpathrectangle{\pgfqpoint{6.572727in}{0.473000in}}{\pgfqpoint{4.227273in}{3.311000in}}%
\pgfusepath{clip}%
\pgfsetbuttcap%
\pgfsetroundjoin%
\definecolor{currentfill}{rgb}{0.993248,0.906157,0.143936}%
\pgfsetfillcolor{currentfill}%
\pgfsetfillopacity{0.700000}%
\pgfsetlinewidth{0.000000pt}%
\definecolor{currentstroke}{rgb}{0.000000,0.000000,0.000000}%
\pgfsetstrokecolor{currentstroke}%
\pgfsetstrokeopacity{0.700000}%
\pgfsetdash{}{0pt}%
\pgfpathmoveto{\pgfqpoint{9.843072in}{1.467046in}}%
\pgfpathcurveto{\pgfqpoint{9.848116in}{1.467046in}}{\pgfqpoint{9.852954in}{1.469050in}}{\pgfqpoint{9.856520in}{1.472616in}}%
\pgfpathcurveto{\pgfqpoint{9.860086in}{1.476183in}}{\pgfqpoint{9.862090in}{1.481020in}}{\pgfqpoint{9.862090in}{1.486064in}}%
\pgfpathcurveto{\pgfqpoint{9.862090in}{1.491108in}}{\pgfqpoint{9.860086in}{1.495945in}}{\pgfqpoint{9.856520in}{1.499512in}}%
\pgfpathcurveto{\pgfqpoint{9.852954in}{1.503078in}}{\pgfqpoint{9.848116in}{1.505082in}}{\pgfqpoint{9.843072in}{1.505082in}}%
\pgfpathcurveto{\pgfqpoint{9.838029in}{1.505082in}}{\pgfqpoint{9.833191in}{1.503078in}}{\pgfqpoint{9.829624in}{1.499512in}}%
\pgfpathcurveto{\pgfqpoint{9.826058in}{1.495945in}}{\pgfqpoint{9.824054in}{1.491108in}}{\pgfqpoint{9.824054in}{1.486064in}}%
\pgfpathcurveto{\pgfqpoint{9.824054in}{1.481020in}}{\pgfqpoint{9.826058in}{1.476183in}}{\pgfqpoint{9.829624in}{1.472616in}}%
\pgfpathcurveto{\pgfqpoint{9.833191in}{1.469050in}}{\pgfqpoint{9.838029in}{1.467046in}}{\pgfqpoint{9.843072in}{1.467046in}}%
\pgfpathclose%
\pgfusepath{fill}%
\end{pgfscope}%
\begin{pgfscope}%
\pgfpathrectangle{\pgfqpoint{6.572727in}{0.473000in}}{\pgfqpoint{4.227273in}{3.311000in}}%
\pgfusepath{clip}%
\pgfsetbuttcap%
\pgfsetroundjoin%
\definecolor{currentfill}{rgb}{0.127568,0.566949,0.550556}%
\pgfsetfillcolor{currentfill}%
\pgfsetfillopacity{0.700000}%
\pgfsetlinewidth{0.000000pt}%
\definecolor{currentstroke}{rgb}{0.000000,0.000000,0.000000}%
\pgfsetstrokecolor{currentstroke}%
\pgfsetstrokeopacity{0.700000}%
\pgfsetdash{}{0pt}%
\pgfpathmoveto{\pgfqpoint{8.493389in}{2.679579in}}%
\pgfpathcurveto{\pgfqpoint{8.498432in}{2.679579in}}{\pgfqpoint{8.503270in}{2.681582in}}{\pgfqpoint{8.506836in}{2.685149in}}%
\pgfpathcurveto{\pgfqpoint{8.510403in}{2.688715in}}{\pgfqpoint{8.512407in}{2.693553in}}{\pgfqpoint{8.512407in}{2.698597in}}%
\pgfpathcurveto{\pgfqpoint{8.512407in}{2.703640in}}{\pgfqpoint{8.510403in}{2.708478in}}{\pgfqpoint{8.506836in}{2.712045in}}%
\pgfpathcurveto{\pgfqpoint{8.503270in}{2.715611in}}{\pgfqpoint{8.498432in}{2.717615in}}{\pgfqpoint{8.493389in}{2.717615in}}%
\pgfpathcurveto{\pgfqpoint{8.488345in}{2.717615in}}{\pgfqpoint{8.483507in}{2.715611in}}{\pgfqpoint{8.479941in}{2.712045in}}%
\pgfpathcurveto{\pgfqpoint{8.476374in}{2.708478in}}{\pgfqpoint{8.474370in}{2.703640in}}{\pgfqpoint{8.474370in}{2.698597in}}%
\pgfpathcurveto{\pgfqpoint{8.474370in}{2.693553in}}{\pgfqpoint{8.476374in}{2.688715in}}{\pgfqpoint{8.479941in}{2.685149in}}%
\pgfpathcurveto{\pgfqpoint{8.483507in}{2.681582in}}{\pgfqpoint{8.488345in}{2.679579in}}{\pgfqpoint{8.493389in}{2.679579in}}%
\pgfpathclose%
\pgfusepath{fill}%
\end{pgfscope}%
\begin{pgfscope}%
\pgfpathrectangle{\pgfqpoint{6.572727in}{0.473000in}}{\pgfqpoint{4.227273in}{3.311000in}}%
\pgfusepath{clip}%
\pgfsetbuttcap%
\pgfsetroundjoin%
\definecolor{currentfill}{rgb}{0.127568,0.566949,0.550556}%
\pgfsetfillcolor{currentfill}%
\pgfsetfillopacity{0.700000}%
\pgfsetlinewidth{0.000000pt}%
\definecolor{currentstroke}{rgb}{0.000000,0.000000,0.000000}%
\pgfsetstrokecolor{currentstroke}%
\pgfsetstrokeopacity{0.700000}%
\pgfsetdash{}{0pt}%
\pgfpathmoveto{\pgfqpoint{7.884330in}{1.728412in}}%
\pgfpathcurveto{\pgfqpoint{7.889374in}{1.728412in}}{\pgfqpoint{7.894212in}{1.730416in}}{\pgfqpoint{7.897778in}{1.733983in}}%
\pgfpathcurveto{\pgfqpoint{7.901345in}{1.737549in}}{\pgfqpoint{7.903349in}{1.742387in}}{\pgfqpoint{7.903349in}{1.747430in}}%
\pgfpathcurveto{\pgfqpoint{7.903349in}{1.752474in}}{\pgfqpoint{7.901345in}{1.757312in}}{\pgfqpoint{7.897778in}{1.760878in}}%
\pgfpathcurveto{\pgfqpoint{7.894212in}{1.764445in}}{\pgfqpoint{7.889374in}{1.766449in}}{\pgfqpoint{7.884330in}{1.766449in}}%
\pgfpathcurveto{\pgfqpoint{7.879287in}{1.766449in}}{\pgfqpoint{7.874449in}{1.764445in}}{\pgfqpoint{7.870883in}{1.760878in}}%
\pgfpathcurveto{\pgfqpoint{7.867316in}{1.757312in}}{\pgfqpoint{7.865312in}{1.752474in}}{\pgfqpoint{7.865312in}{1.747430in}}%
\pgfpathcurveto{\pgfqpoint{7.865312in}{1.742387in}}{\pgfqpoint{7.867316in}{1.737549in}}{\pgfqpoint{7.870883in}{1.733983in}}%
\pgfpathcurveto{\pgfqpoint{7.874449in}{1.730416in}}{\pgfqpoint{7.879287in}{1.728412in}}{\pgfqpoint{7.884330in}{1.728412in}}%
\pgfpathclose%
\pgfusepath{fill}%
\end{pgfscope}%
\begin{pgfscope}%
\pgfpathrectangle{\pgfqpoint{6.572727in}{0.473000in}}{\pgfqpoint{4.227273in}{3.311000in}}%
\pgfusepath{clip}%
\pgfsetbuttcap%
\pgfsetroundjoin%
\definecolor{currentfill}{rgb}{0.127568,0.566949,0.550556}%
\pgfsetfillcolor{currentfill}%
\pgfsetfillopacity{0.700000}%
\pgfsetlinewidth{0.000000pt}%
\definecolor{currentstroke}{rgb}{0.000000,0.000000,0.000000}%
\pgfsetstrokecolor{currentstroke}%
\pgfsetstrokeopacity{0.700000}%
\pgfsetdash{}{0pt}%
\pgfpathmoveto{\pgfqpoint{7.794499in}{3.089868in}}%
\pgfpathcurveto{\pgfqpoint{7.799542in}{3.089868in}}{\pgfqpoint{7.804380in}{3.091872in}}{\pgfqpoint{7.807947in}{3.095439in}}%
\pgfpathcurveto{\pgfqpoint{7.811513in}{3.099005in}}{\pgfqpoint{7.813517in}{3.103843in}}{\pgfqpoint{7.813517in}{3.108886in}}%
\pgfpathcurveto{\pgfqpoint{7.813517in}{3.113930in}}{\pgfqpoint{7.811513in}{3.118768in}}{\pgfqpoint{7.807947in}{3.122334in}}%
\pgfpathcurveto{\pgfqpoint{7.804380in}{3.125901in}}{\pgfqpoint{7.799542in}{3.127905in}}{\pgfqpoint{7.794499in}{3.127905in}}%
\pgfpathcurveto{\pgfqpoint{7.789455in}{3.127905in}}{\pgfqpoint{7.784617in}{3.125901in}}{\pgfqpoint{7.781051in}{3.122334in}}%
\pgfpathcurveto{\pgfqpoint{7.777485in}{3.118768in}}{\pgfqpoint{7.775481in}{3.113930in}}{\pgfqpoint{7.775481in}{3.108886in}}%
\pgfpathcurveto{\pgfqpoint{7.775481in}{3.103843in}}{\pgfqpoint{7.777485in}{3.099005in}}{\pgfqpoint{7.781051in}{3.095439in}}%
\pgfpathcurveto{\pgfqpoint{7.784617in}{3.091872in}}{\pgfqpoint{7.789455in}{3.089868in}}{\pgfqpoint{7.794499in}{3.089868in}}%
\pgfpathclose%
\pgfusepath{fill}%
\end{pgfscope}%
\begin{pgfscope}%
\pgfpathrectangle{\pgfqpoint{6.572727in}{0.473000in}}{\pgfqpoint{4.227273in}{3.311000in}}%
\pgfusepath{clip}%
\pgfsetbuttcap%
\pgfsetroundjoin%
\definecolor{currentfill}{rgb}{0.993248,0.906157,0.143936}%
\pgfsetfillcolor{currentfill}%
\pgfsetfillopacity{0.700000}%
\pgfsetlinewidth{0.000000pt}%
\definecolor{currentstroke}{rgb}{0.000000,0.000000,0.000000}%
\pgfsetstrokecolor{currentstroke}%
\pgfsetstrokeopacity{0.700000}%
\pgfsetdash{}{0pt}%
\pgfpathmoveto{\pgfqpoint{9.464640in}{1.937329in}}%
\pgfpathcurveto{\pgfqpoint{9.469684in}{1.937329in}}{\pgfqpoint{9.474522in}{1.939333in}}{\pgfqpoint{9.478088in}{1.942900in}}%
\pgfpathcurveto{\pgfqpoint{9.481654in}{1.946466in}}{\pgfqpoint{9.483658in}{1.951304in}}{\pgfqpoint{9.483658in}{1.956348in}}%
\pgfpathcurveto{\pgfqpoint{9.483658in}{1.961391in}}{\pgfqpoint{9.481654in}{1.966229in}}{\pgfqpoint{9.478088in}{1.969795in}}%
\pgfpathcurveto{\pgfqpoint{9.474522in}{1.973362in}}{\pgfqpoint{9.469684in}{1.975366in}}{\pgfqpoint{9.464640in}{1.975366in}}%
\pgfpathcurveto{\pgfqpoint{9.459596in}{1.975366in}}{\pgfqpoint{9.454759in}{1.973362in}}{\pgfqpoint{9.451192in}{1.969795in}}%
\pgfpathcurveto{\pgfqpoint{9.447626in}{1.966229in}}{\pgfqpoint{9.445622in}{1.961391in}}{\pgfqpoint{9.445622in}{1.956348in}}%
\pgfpathcurveto{\pgfqpoint{9.445622in}{1.951304in}}{\pgfqpoint{9.447626in}{1.946466in}}{\pgfqpoint{9.451192in}{1.942900in}}%
\pgfpathcurveto{\pgfqpoint{9.454759in}{1.939333in}}{\pgfqpoint{9.459596in}{1.937329in}}{\pgfqpoint{9.464640in}{1.937329in}}%
\pgfpathclose%
\pgfusepath{fill}%
\end{pgfscope}%
\begin{pgfscope}%
\pgfpathrectangle{\pgfqpoint{6.572727in}{0.473000in}}{\pgfqpoint{4.227273in}{3.311000in}}%
\pgfusepath{clip}%
\pgfsetbuttcap%
\pgfsetroundjoin%
\definecolor{currentfill}{rgb}{0.127568,0.566949,0.550556}%
\pgfsetfillcolor{currentfill}%
\pgfsetfillopacity{0.700000}%
\pgfsetlinewidth{0.000000pt}%
\definecolor{currentstroke}{rgb}{0.000000,0.000000,0.000000}%
\pgfsetstrokecolor{currentstroke}%
\pgfsetstrokeopacity{0.700000}%
\pgfsetdash{}{0pt}%
\pgfpathmoveto{\pgfqpoint{8.272907in}{1.230084in}}%
\pgfpathcurveto{\pgfqpoint{8.277951in}{1.230084in}}{\pgfqpoint{8.282788in}{1.232087in}}{\pgfqpoint{8.286355in}{1.235654in}}%
\pgfpathcurveto{\pgfqpoint{8.289921in}{1.239220in}}{\pgfqpoint{8.291925in}{1.244058in}}{\pgfqpoint{8.291925in}{1.249102in}}%
\pgfpathcurveto{\pgfqpoint{8.291925in}{1.254145in}}{\pgfqpoint{8.289921in}{1.258983in}}{\pgfqpoint{8.286355in}{1.262550in}}%
\pgfpathcurveto{\pgfqpoint{8.282788in}{1.266116in}}{\pgfqpoint{8.277951in}{1.268120in}}{\pgfqpoint{8.272907in}{1.268120in}}%
\pgfpathcurveto{\pgfqpoint{8.267863in}{1.268120in}}{\pgfqpoint{8.263025in}{1.266116in}}{\pgfqpoint{8.259459in}{1.262550in}}%
\pgfpathcurveto{\pgfqpoint{8.255893in}{1.258983in}}{\pgfqpoint{8.253889in}{1.254145in}}{\pgfqpoint{8.253889in}{1.249102in}}%
\pgfpathcurveto{\pgfqpoint{8.253889in}{1.244058in}}{\pgfqpoint{8.255893in}{1.239220in}}{\pgfqpoint{8.259459in}{1.235654in}}%
\pgfpathcurveto{\pgfqpoint{8.263025in}{1.232087in}}{\pgfqpoint{8.267863in}{1.230084in}}{\pgfqpoint{8.272907in}{1.230084in}}%
\pgfpathclose%
\pgfusepath{fill}%
\end{pgfscope}%
\begin{pgfscope}%
\pgfpathrectangle{\pgfqpoint{6.572727in}{0.473000in}}{\pgfqpoint{4.227273in}{3.311000in}}%
\pgfusepath{clip}%
\pgfsetbuttcap%
\pgfsetroundjoin%
\definecolor{currentfill}{rgb}{0.127568,0.566949,0.550556}%
\pgfsetfillcolor{currentfill}%
\pgfsetfillopacity{0.700000}%
\pgfsetlinewidth{0.000000pt}%
\definecolor{currentstroke}{rgb}{0.000000,0.000000,0.000000}%
\pgfsetstrokecolor{currentstroke}%
\pgfsetstrokeopacity{0.700000}%
\pgfsetdash{}{0pt}%
\pgfpathmoveto{\pgfqpoint{8.250350in}{1.623556in}}%
\pgfpathcurveto{\pgfqpoint{8.255394in}{1.623556in}}{\pgfqpoint{8.260232in}{1.625559in}}{\pgfqpoint{8.263798in}{1.629126in}}%
\pgfpathcurveto{\pgfqpoint{8.267364in}{1.632692in}}{\pgfqpoint{8.269368in}{1.637530in}}{\pgfqpoint{8.269368in}{1.642574in}}%
\pgfpathcurveto{\pgfqpoint{8.269368in}{1.647617in}}{\pgfqpoint{8.267364in}{1.652455in}}{\pgfqpoint{8.263798in}{1.656022in}}%
\pgfpathcurveto{\pgfqpoint{8.260232in}{1.659588in}}{\pgfqpoint{8.255394in}{1.661592in}}{\pgfqpoint{8.250350in}{1.661592in}}%
\pgfpathcurveto{\pgfqpoint{8.245307in}{1.661592in}}{\pgfqpoint{8.240469in}{1.659588in}}{\pgfqpoint{8.236902in}{1.656022in}}%
\pgfpathcurveto{\pgfqpoint{8.233336in}{1.652455in}}{\pgfqpoint{8.231332in}{1.647617in}}{\pgfqpoint{8.231332in}{1.642574in}}%
\pgfpathcurveto{\pgfqpoint{8.231332in}{1.637530in}}{\pgfqpoint{8.233336in}{1.632692in}}{\pgfqpoint{8.236902in}{1.629126in}}%
\pgfpathcurveto{\pgfqpoint{8.240469in}{1.625559in}}{\pgfqpoint{8.245307in}{1.623556in}}{\pgfqpoint{8.250350in}{1.623556in}}%
\pgfpathclose%
\pgfusepath{fill}%
\end{pgfscope}%
\begin{pgfscope}%
\pgfpathrectangle{\pgfqpoint{6.572727in}{0.473000in}}{\pgfqpoint{4.227273in}{3.311000in}}%
\pgfusepath{clip}%
\pgfsetbuttcap%
\pgfsetroundjoin%
\definecolor{currentfill}{rgb}{0.993248,0.906157,0.143936}%
\pgfsetfillcolor{currentfill}%
\pgfsetfillopacity{0.700000}%
\pgfsetlinewidth{0.000000pt}%
\definecolor{currentstroke}{rgb}{0.000000,0.000000,0.000000}%
\pgfsetstrokecolor{currentstroke}%
\pgfsetstrokeopacity{0.700000}%
\pgfsetdash{}{0pt}%
\pgfpathmoveto{\pgfqpoint{9.945971in}{1.765383in}}%
\pgfpathcurveto{\pgfqpoint{9.951015in}{1.765383in}}{\pgfqpoint{9.955853in}{1.767387in}}{\pgfqpoint{9.959419in}{1.770953in}}%
\pgfpathcurveto{\pgfqpoint{9.962986in}{1.774520in}}{\pgfqpoint{9.964989in}{1.779357in}}{\pgfqpoint{9.964989in}{1.784401in}}%
\pgfpathcurveto{\pgfqpoint{9.964989in}{1.789445in}}{\pgfqpoint{9.962986in}{1.794283in}}{\pgfqpoint{9.959419in}{1.797849in}}%
\pgfpathcurveto{\pgfqpoint{9.955853in}{1.801415in}}{\pgfqpoint{9.951015in}{1.803419in}}{\pgfqpoint{9.945971in}{1.803419in}}%
\pgfpathcurveto{\pgfqpoint{9.940928in}{1.803419in}}{\pgfqpoint{9.936090in}{1.801415in}}{\pgfqpoint{9.932523in}{1.797849in}}%
\pgfpathcurveto{\pgfqpoint{9.928957in}{1.794283in}}{\pgfqpoint{9.926953in}{1.789445in}}{\pgfqpoint{9.926953in}{1.784401in}}%
\pgfpathcurveto{\pgfqpoint{9.926953in}{1.779357in}}{\pgfqpoint{9.928957in}{1.774520in}}{\pgfqpoint{9.932523in}{1.770953in}}%
\pgfpathcurveto{\pgfqpoint{9.936090in}{1.767387in}}{\pgfqpoint{9.940928in}{1.765383in}}{\pgfqpoint{9.945971in}{1.765383in}}%
\pgfpathclose%
\pgfusepath{fill}%
\end{pgfscope}%
\begin{pgfscope}%
\pgfpathrectangle{\pgfqpoint{6.572727in}{0.473000in}}{\pgfqpoint{4.227273in}{3.311000in}}%
\pgfusepath{clip}%
\pgfsetbuttcap%
\pgfsetroundjoin%
\definecolor{currentfill}{rgb}{0.127568,0.566949,0.550556}%
\pgfsetfillcolor{currentfill}%
\pgfsetfillopacity{0.700000}%
\pgfsetlinewidth{0.000000pt}%
\definecolor{currentstroke}{rgb}{0.000000,0.000000,0.000000}%
\pgfsetstrokecolor{currentstroke}%
\pgfsetstrokeopacity{0.700000}%
\pgfsetdash{}{0pt}%
\pgfpathmoveto{\pgfqpoint{7.906949in}{1.870374in}}%
\pgfpathcurveto{\pgfqpoint{7.911993in}{1.870374in}}{\pgfqpoint{7.916831in}{1.872378in}}{\pgfqpoint{7.920397in}{1.875944in}}%
\pgfpathcurveto{\pgfqpoint{7.923963in}{1.879511in}}{\pgfqpoint{7.925967in}{1.884348in}}{\pgfqpoint{7.925967in}{1.889392in}}%
\pgfpathcurveto{\pgfqpoint{7.925967in}{1.894436in}}{\pgfqpoint{7.923963in}{1.899273in}}{\pgfqpoint{7.920397in}{1.902840in}}%
\pgfpathcurveto{\pgfqpoint{7.916831in}{1.906406in}}{\pgfqpoint{7.911993in}{1.908410in}}{\pgfqpoint{7.906949in}{1.908410in}}%
\pgfpathcurveto{\pgfqpoint{7.901905in}{1.908410in}}{\pgfqpoint{7.897068in}{1.906406in}}{\pgfqpoint{7.893501in}{1.902840in}}%
\pgfpathcurveto{\pgfqpoint{7.889935in}{1.899273in}}{\pgfqpoint{7.887931in}{1.894436in}}{\pgfqpoint{7.887931in}{1.889392in}}%
\pgfpathcurveto{\pgfqpoint{7.887931in}{1.884348in}}{\pgfqpoint{7.889935in}{1.879511in}}{\pgfqpoint{7.893501in}{1.875944in}}%
\pgfpathcurveto{\pgfqpoint{7.897068in}{1.872378in}}{\pgfqpoint{7.901905in}{1.870374in}}{\pgfqpoint{7.906949in}{1.870374in}}%
\pgfpathclose%
\pgfusepath{fill}%
\end{pgfscope}%
\begin{pgfscope}%
\pgfpathrectangle{\pgfqpoint{6.572727in}{0.473000in}}{\pgfqpoint{4.227273in}{3.311000in}}%
\pgfusepath{clip}%
\pgfsetbuttcap%
\pgfsetroundjoin%
\definecolor{currentfill}{rgb}{0.993248,0.906157,0.143936}%
\pgfsetfillcolor{currentfill}%
\pgfsetfillopacity{0.700000}%
\pgfsetlinewidth{0.000000pt}%
\definecolor{currentstroke}{rgb}{0.000000,0.000000,0.000000}%
\pgfsetstrokecolor{currentstroke}%
\pgfsetstrokeopacity{0.700000}%
\pgfsetdash{}{0pt}%
\pgfpathmoveto{\pgfqpoint{9.251025in}{1.581225in}}%
\pgfpathcurveto{\pgfqpoint{9.256069in}{1.581225in}}{\pgfqpoint{9.260907in}{1.583229in}}{\pgfqpoint{9.264473in}{1.586795in}}%
\pgfpathcurveto{\pgfqpoint{9.268040in}{1.590361in}}{\pgfqpoint{9.270044in}{1.595199in}}{\pgfqpoint{9.270044in}{1.600243in}}%
\pgfpathcurveto{\pgfqpoint{9.270044in}{1.605286in}}{\pgfqpoint{9.268040in}{1.610124in}}{\pgfqpoint{9.264473in}{1.613691in}}%
\pgfpathcurveto{\pgfqpoint{9.260907in}{1.617257in}}{\pgfqpoint{9.256069in}{1.619261in}}{\pgfqpoint{9.251025in}{1.619261in}}%
\pgfpathcurveto{\pgfqpoint{9.245982in}{1.619261in}}{\pgfqpoint{9.241144in}{1.617257in}}{\pgfqpoint{9.237578in}{1.613691in}}%
\pgfpathcurveto{\pgfqpoint{9.234011in}{1.610124in}}{\pgfqpoint{9.232007in}{1.605286in}}{\pgfqpoint{9.232007in}{1.600243in}}%
\pgfpathcurveto{\pgfqpoint{9.232007in}{1.595199in}}{\pgfqpoint{9.234011in}{1.590361in}}{\pgfqpoint{9.237578in}{1.586795in}}%
\pgfpathcurveto{\pgfqpoint{9.241144in}{1.583229in}}{\pgfqpoint{9.245982in}{1.581225in}}{\pgfqpoint{9.251025in}{1.581225in}}%
\pgfpathclose%
\pgfusepath{fill}%
\end{pgfscope}%
\begin{pgfscope}%
\pgfpathrectangle{\pgfqpoint{6.572727in}{0.473000in}}{\pgfqpoint{4.227273in}{3.311000in}}%
\pgfusepath{clip}%
\pgfsetbuttcap%
\pgfsetroundjoin%
\definecolor{currentfill}{rgb}{0.127568,0.566949,0.550556}%
\pgfsetfillcolor{currentfill}%
\pgfsetfillopacity{0.700000}%
\pgfsetlinewidth{0.000000pt}%
\definecolor{currentstroke}{rgb}{0.000000,0.000000,0.000000}%
\pgfsetstrokecolor{currentstroke}%
\pgfsetstrokeopacity{0.700000}%
\pgfsetdash{}{0pt}%
\pgfpathmoveto{\pgfqpoint{7.804424in}{1.842950in}}%
\pgfpathcurveto{\pgfqpoint{7.809468in}{1.842950in}}{\pgfqpoint{7.814306in}{1.844954in}}{\pgfqpoint{7.817872in}{1.848521in}}%
\pgfpathcurveto{\pgfqpoint{7.821439in}{1.852087in}}{\pgfqpoint{7.823442in}{1.856925in}}{\pgfqpoint{7.823442in}{1.861969in}}%
\pgfpathcurveto{\pgfqpoint{7.823442in}{1.867012in}}{\pgfqpoint{7.821439in}{1.871850in}}{\pgfqpoint{7.817872in}{1.875416in}}%
\pgfpathcurveto{\pgfqpoint{7.814306in}{1.878983in}}{\pgfqpoint{7.809468in}{1.880987in}}{\pgfqpoint{7.804424in}{1.880987in}}%
\pgfpathcurveto{\pgfqpoint{7.799381in}{1.880987in}}{\pgfqpoint{7.794543in}{1.878983in}}{\pgfqpoint{7.790976in}{1.875416in}}%
\pgfpathcurveto{\pgfqpoint{7.787410in}{1.871850in}}{\pgfqpoint{7.785406in}{1.867012in}}{\pgfqpoint{7.785406in}{1.861969in}}%
\pgfpathcurveto{\pgfqpoint{7.785406in}{1.856925in}}{\pgfqpoint{7.787410in}{1.852087in}}{\pgfqpoint{7.790976in}{1.848521in}}%
\pgfpathcurveto{\pgfqpoint{7.794543in}{1.844954in}}{\pgfqpoint{7.799381in}{1.842950in}}{\pgfqpoint{7.804424in}{1.842950in}}%
\pgfpathclose%
\pgfusepath{fill}%
\end{pgfscope}%
\begin{pgfscope}%
\pgfpathrectangle{\pgfqpoint{6.572727in}{0.473000in}}{\pgfqpoint{4.227273in}{3.311000in}}%
\pgfusepath{clip}%
\pgfsetbuttcap%
\pgfsetroundjoin%
\definecolor{currentfill}{rgb}{0.127568,0.566949,0.550556}%
\pgfsetfillcolor{currentfill}%
\pgfsetfillopacity{0.700000}%
\pgfsetlinewidth{0.000000pt}%
\definecolor{currentstroke}{rgb}{0.000000,0.000000,0.000000}%
\pgfsetstrokecolor{currentstroke}%
\pgfsetstrokeopacity{0.700000}%
\pgfsetdash{}{0pt}%
\pgfpathmoveto{\pgfqpoint{8.593939in}{2.756744in}}%
\pgfpathcurveto{\pgfqpoint{8.598982in}{2.756744in}}{\pgfqpoint{8.603820in}{2.758748in}}{\pgfqpoint{8.607386in}{2.762315in}}%
\pgfpathcurveto{\pgfqpoint{8.610953in}{2.765881in}}{\pgfqpoint{8.612957in}{2.770719in}}{\pgfqpoint{8.612957in}{2.775763in}}%
\pgfpathcurveto{\pgfqpoint{8.612957in}{2.780806in}}{\pgfqpoint{8.610953in}{2.785644in}}{\pgfqpoint{8.607386in}{2.789210in}}%
\pgfpathcurveto{\pgfqpoint{8.603820in}{2.792777in}}{\pgfqpoint{8.598982in}{2.794781in}}{\pgfqpoint{8.593939in}{2.794781in}}%
\pgfpathcurveto{\pgfqpoint{8.588895in}{2.794781in}}{\pgfqpoint{8.584057in}{2.792777in}}{\pgfqpoint{8.580491in}{2.789210in}}%
\pgfpathcurveto{\pgfqpoint{8.576924in}{2.785644in}}{\pgfqpoint{8.574920in}{2.780806in}}{\pgfqpoint{8.574920in}{2.775763in}}%
\pgfpathcurveto{\pgfqpoint{8.574920in}{2.770719in}}{\pgfqpoint{8.576924in}{2.765881in}}{\pgfqpoint{8.580491in}{2.762315in}}%
\pgfpathcurveto{\pgfqpoint{8.584057in}{2.758748in}}{\pgfqpoint{8.588895in}{2.756744in}}{\pgfqpoint{8.593939in}{2.756744in}}%
\pgfpathclose%
\pgfusepath{fill}%
\end{pgfscope}%
\begin{pgfscope}%
\pgfpathrectangle{\pgfqpoint{6.572727in}{0.473000in}}{\pgfqpoint{4.227273in}{3.311000in}}%
\pgfusepath{clip}%
\pgfsetbuttcap%
\pgfsetroundjoin%
\definecolor{currentfill}{rgb}{0.993248,0.906157,0.143936}%
\pgfsetfillcolor{currentfill}%
\pgfsetfillopacity{0.700000}%
\pgfsetlinewidth{0.000000pt}%
\definecolor{currentstroke}{rgb}{0.000000,0.000000,0.000000}%
\pgfsetstrokecolor{currentstroke}%
\pgfsetstrokeopacity{0.700000}%
\pgfsetdash{}{0pt}%
\pgfpathmoveto{\pgfqpoint{9.787090in}{1.588167in}}%
\pgfpathcurveto{\pgfqpoint{9.792134in}{1.588167in}}{\pgfqpoint{9.796971in}{1.590171in}}{\pgfqpoint{9.800538in}{1.593738in}}%
\pgfpathcurveto{\pgfqpoint{9.804104in}{1.597304in}}{\pgfqpoint{9.806108in}{1.602142in}}{\pgfqpoint{9.806108in}{1.607185in}}%
\pgfpathcurveto{\pgfqpoint{9.806108in}{1.612229in}}{\pgfqpoint{9.804104in}{1.617067in}}{\pgfqpoint{9.800538in}{1.620633in}}%
\pgfpathcurveto{\pgfqpoint{9.796971in}{1.624200in}}{\pgfqpoint{9.792134in}{1.626204in}}{\pgfqpoint{9.787090in}{1.626204in}}%
\pgfpathcurveto{\pgfqpoint{9.782046in}{1.626204in}}{\pgfqpoint{9.777208in}{1.624200in}}{\pgfqpoint{9.773642in}{1.620633in}}%
\pgfpathcurveto{\pgfqpoint{9.770076in}{1.617067in}}{\pgfqpoint{9.768072in}{1.612229in}}{\pgfqpoint{9.768072in}{1.607185in}}%
\pgfpathcurveto{\pgfqpoint{9.768072in}{1.602142in}}{\pgfqpoint{9.770076in}{1.597304in}}{\pgfqpoint{9.773642in}{1.593738in}}%
\pgfpathcurveto{\pgfqpoint{9.777208in}{1.590171in}}{\pgfqpoint{9.782046in}{1.588167in}}{\pgfqpoint{9.787090in}{1.588167in}}%
\pgfpathclose%
\pgfusepath{fill}%
\end{pgfscope}%
\begin{pgfscope}%
\pgfpathrectangle{\pgfqpoint{6.572727in}{0.473000in}}{\pgfqpoint{4.227273in}{3.311000in}}%
\pgfusepath{clip}%
\pgfsetbuttcap%
\pgfsetroundjoin%
\definecolor{currentfill}{rgb}{0.127568,0.566949,0.550556}%
\pgfsetfillcolor{currentfill}%
\pgfsetfillopacity{0.700000}%
\pgfsetlinewidth{0.000000pt}%
\definecolor{currentstroke}{rgb}{0.000000,0.000000,0.000000}%
\pgfsetstrokecolor{currentstroke}%
\pgfsetstrokeopacity{0.700000}%
\pgfsetdash{}{0pt}%
\pgfpathmoveto{\pgfqpoint{8.258408in}{3.064288in}}%
\pgfpathcurveto{\pgfqpoint{8.263452in}{3.064288in}}{\pgfqpoint{8.268290in}{3.066292in}}{\pgfqpoint{8.271856in}{3.069858in}}%
\pgfpathcurveto{\pgfqpoint{8.275422in}{3.073425in}}{\pgfqpoint{8.277426in}{3.078262in}}{\pgfqpoint{8.277426in}{3.083306in}}%
\pgfpathcurveto{\pgfqpoint{8.277426in}{3.088350in}}{\pgfqpoint{8.275422in}{3.093188in}}{\pgfqpoint{8.271856in}{3.096754in}}%
\pgfpathcurveto{\pgfqpoint{8.268290in}{3.100320in}}{\pgfqpoint{8.263452in}{3.102324in}}{\pgfqpoint{8.258408in}{3.102324in}}%
\pgfpathcurveto{\pgfqpoint{8.253364in}{3.102324in}}{\pgfqpoint{8.248527in}{3.100320in}}{\pgfqpoint{8.244960in}{3.096754in}}%
\pgfpathcurveto{\pgfqpoint{8.241394in}{3.093188in}}{\pgfqpoint{8.239390in}{3.088350in}}{\pgfqpoint{8.239390in}{3.083306in}}%
\pgfpathcurveto{\pgfqpoint{8.239390in}{3.078262in}}{\pgfqpoint{8.241394in}{3.073425in}}{\pgfqpoint{8.244960in}{3.069858in}}%
\pgfpathcurveto{\pgfqpoint{8.248527in}{3.066292in}}{\pgfqpoint{8.253364in}{3.064288in}}{\pgfqpoint{8.258408in}{3.064288in}}%
\pgfpathclose%
\pgfusepath{fill}%
\end{pgfscope}%
\begin{pgfscope}%
\pgfpathrectangle{\pgfqpoint{6.572727in}{0.473000in}}{\pgfqpoint{4.227273in}{3.311000in}}%
\pgfusepath{clip}%
\pgfsetbuttcap%
\pgfsetroundjoin%
\definecolor{currentfill}{rgb}{0.127568,0.566949,0.550556}%
\pgfsetfillcolor{currentfill}%
\pgfsetfillopacity{0.700000}%
\pgfsetlinewidth{0.000000pt}%
\definecolor{currentstroke}{rgb}{0.000000,0.000000,0.000000}%
\pgfsetstrokecolor{currentstroke}%
\pgfsetstrokeopacity{0.700000}%
\pgfsetdash{}{0pt}%
\pgfpathmoveto{\pgfqpoint{8.370768in}{1.466291in}}%
\pgfpathcurveto{\pgfqpoint{8.375811in}{1.466291in}}{\pgfqpoint{8.380649in}{1.468295in}}{\pgfqpoint{8.384215in}{1.471861in}}%
\pgfpathcurveto{\pgfqpoint{8.387782in}{1.475428in}}{\pgfqpoint{8.389786in}{1.480266in}}{\pgfqpoint{8.389786in}{1.485309in}}%
\pgfpathcurveto{\pgfqpoint{8.389786in}{1.490353in}}{\pgfqpoint{8.387782in}{1.495191in}}{\pgfqpoint{8.384215in}{1.498757in}}%
\pgfpathcurveto{\pgfqpoint{8.380649in}{1.502324in}}{\pgfqpoint{8.375811in}{1.504327in}}{\pgfqpoint{8.370768in}{1.504327in}}%
\pgfpathcurveto{\pgfqpoint{8.365724in}{1.504327in}}{\pgfqpoint{8.360886in}{1.502324in}}{\pgfqpoint{8.357320in}{1.498757in}}%
\pgfpathcurveto{\pgfqpoint{8.353753in}{1.495191in}}{\pgfqpoint{8.351749in}{1.490353in}}{\pgfqpoint{8.351749in}{1.485309in}}%
\pgfpathcurveto{\pgfqpoint{8.351749in}{1.480266in}}{\pgfqpoint{8.353753in}{1.475428in}}{\pgfqpoint{8.357320in}{1.471861in}}%
\pgfpathcurveto{\pgfqpoint{8.360886in}{1.468295in}}{\pgfqpoint{8.365724in}{1.466291in}}{\pgfqpoint{8.370768in}{1.466291in}}%
\pgfpathclose%
\pgfusepath{fill}%
\end{pgfscope}%
\begin{pgfscope}%
\pgfpathrectangle{\pgfqpoint{6.572727in}{0.473000in}}{\pgfqpoint{4.227273in}{3.311000in}}%
\pgfusepath{clip}%
\pgfsetbuttcap%
\pgfsetroundjoin%
\definecolor{currentfill}{rgb}{0.127568,0.566949,0.550556}%
\pgfsetfillcolor{currentfill}%
\pgfsetfillopacity{0.700000}%
\pgfsetlinewidth{0.000000pt}%
\definecolor{currentstroke}{rgb}{0.000000,0.000000,0.000000}%
\pgfsetstrokecolor{currentstroke}%
\pgfsetstrokeopacity{0.700000}%
\pgfsetdash{}{0pt}%
\pgfpathmoveto{\pgfqpoint{7.401273in}{1.189853in}}%
\pgfpathcurveto{\pgfqpoint{7.406317in}{1.189853in}}{\pgfqpoint{7.411155in}{1.191857in}}{\pgfqpoint{7.414721in}{1.195424in}}%
\pgfpathcurveto{\pgfqpoint{7.418288in}{1.198990in}}{\pgfqpoint{7.420291in}{1.203828in}}{\pgfqpoint{7.420291in}{1.208872in}}%
\pgfpathcurveto{\pgfqpoint{7.420291in}{1.213915in}}{\pgfqpoint{7.418288in}{1.218753in}}{\pgfqpoint{7.414721in}{1.222319in}}%
\pgfpathcurveto{\pgfqpoint{7.411155in}{1.225886in}}{\pgfqpoint{7.406317in}{1.227890in}}{\pgfqpoint{7.401273in}{1.227890in}}%
\pgfpathcurveto{\pgfqpoint{7.396230in}{1.227890in}}{\pgfqpoint{7.391392in}{1.225886in}}{\pgfqpoint{7.387825in}{1.222319in}}%
\pgfpathcurveto{\pgfqpoint{7.384259in}{1.218753in}}{\pgfqpoint{7.382255in}{1.213915in}}{\pgfqpoint{7.382255in}{1.208872in}}%
\pgfpathcurveto{\pgfqpoint{7.382255in}{1.203828in}}{\pgfqpoint{7.384259in}{1.198990in}}{\pgfqpoint{7.387825in}{1.195424in}}%
\pgfpathcurveto{\pgfqpoint{7.391392in}{1.191857in}}{\pgfqpoint{7.396230in}{1.189853in}}{\pgfqpoint{7.401273in}{1.189853in}}%
\pgfpathclose%
\pgfusepath{fill}%
\end{pgfscope}%
\begin{pgfscope}%
\pgfpathrectangle{\pgfqpoint{6.572727in}{0.473000in}}{\pgfqpoint{4.227273in}{3.311000in}}%
\pgfusepath{clip}%
\pgfsetbuttcap%
\pgfsetroundjoin%
\definecolor{currentfill}{rgb}{0.127568,0.566949,0.550556}%
\pgfsetfillcolor{currentfill}%
\pgfsetfillopacity{0.700000}%
\pgfsetlinewidth{0.000000pt}%
\definecolor{currentstroke}{rgb}{0.000000,0.000000,0.000000}%
\pgfsetstrokecolor{currentstroke}%
\pgfsetstrokeopacity{0.700000}%
\pgfsetdash{}{0pt}%
\pgfpathmoveto{\pgfqpoint{8.392658in}{2.578275in}}%
\pgfpathcurveto{\pgfqpoint{8.397702in}{2.578275in}}{\pgfqpoint{8.402539in}{2.580279in}}{\pgfqpoint{8.406106in}{2.583845in}}%
\pgfpathcurveto{\pgfqpoint{8.409672in}{2.587411in}}{\pgfqpoint{8.411676in}{2.592249in}}{\pgfqpoint{8.411676in}{2.597293in}}%
\pgfpathcurveto{\pgfqpoint{8.411676in}{2.602337in}}{\pgfqpoint{8.409672in}{2.607174in}}{\pgfqpoint{8.406106in}{2.610741in}}%
\pgfpathcurveto{\pgfqpoint{8.402539in}{2.614307in}}{\pgfqpoint{8.397702in}{2.616311in}}{\pgfqpoint{8.392658in}{2.616311in}}%
\pgfpathcurveto{\pgfqpoint{8.387614in}{2.616311in}}{\pgfqpoint{8.382777in}{2.614307in}}{\pgfqpoint{8.379210in}{2.610741in}}%
\pgfpathcurveto{\pgfqpoint{8.375644in}{2.607174in}}{\pgfqpoint{8.373640in}{2.602337in}}{\pgfqpoint{8.373640in}{2.597293in}}%
\pgfpathcurveto{\pgfqpoint{8.373640in}{2.592249in}}{\pgfqpoint{8.375644in}{2.587411in}}{\pgfqpoint{8.379210in}{2.583845in}}%
\pgfpathcurveto{\pgfqpoint{8.382777in}{2.580279in}}{\pgfqpoint{8.387614in}{2.578275in}}{\pgfqpoint{8.392658in}{2.578275in}}%
\pgfpathclose%
\pgfusepath{fill}%
\end{pgfscope}%
\begin{pgfscope}%
\pgfpathrectangle{\pgfqpoint{6.572727in}{0.473000in}}{\pgfqpoint{4.227273in}{3.311000in}}%
\pgfusepath{clip}%
\pgfsetbuttcap%
\pgfsetroundjoin%
\definecolor{currentfill}{rgb}{0.127568,0.566949,0.550556}%
\pgfsetfillcolor{currentfill}%
\pgfsetfillopacity{0.700000}%
\pgfsetlinewidth{0.000000pt}%
\definecolor{currentstroke}{rgb}{0.000000,0.000000,0.000000}%
\pgfsetstrokecolor{currentstroke}%
\pgfsetstrokeopacity{0.700000}%
\pgfsetdash{}{0pt}%
\pgfpathmoveto{\pgfqpoint{8.593693in}{2.836673in}}%
\pgfpathcurveto{\pgfqpoint{8.598737in}{2.836673in}}{\pgfqpoint{8.603575in}{2.838676in}}{\pgfqpoint{8.607141in}{2.842243in}}%
\pgfpathcurveto{\pgfqpoint{8.610707in}{2.845809in}}{\pgfqpoint{8.612711in}{2.850647in}}{\pgfqpoint{8.612711in}{2.855691in}}%
\pgfpathcurveto{\pgfqpoint{8.612711in}{2.860734in}}{\pgfqpoint{8.610707in}{2.865572in}}{\pgfqpoint{8.607141in}{2.869139in}}%
\pgfpathcurveto{\pgfqpoint{8.603575in}{2.872705in}}{\pgfqpoint{8.598737in}{2.874709in}}{\pgfqpoint{8.593693in}{2.874709in}}%
\pgfpathcurveto{\pgfqpoint{8.588650in}{2.874709in}}{\pgfqpoint{8.583812in}{2.872705in}}{\pgfqpoint{8.580245in}{2.869139in}}%
\pgfpathcurveto{\pgfqpoint{8.576679in}{2.865572in}}{\pgfqpoint{8.574675in}{2.860734in}}{\pgfqpoint{8.574675in}{2.855691in}}%
\pgfpathcurveto{\pgfqpoint{8.574675in}{2.850647in}}{\pgfqpoint{8.576679in}{2.845809in}}{\pgfqpoint{8.580245in}{2.842243in}}%
\pgfpathcurveto{\pgfqpoint{8.583812in}{2.838676in}}{\pgfqpoint{8.588650in}{2.836673in}}{\pgfqpoint{8.593693in}{2.836673in}}%
\pgfpathclose%
\pgfusepath{fill}%
\end{pgfscope}%
\begin{pgfscope}%
\pgfpathrectangle{\pgfqpoint{6.572727in}{0.473000in}}{\pgfqpoint{4.227273in}{3.311000in}}%
\pgfusepath{clip}%
\pgfsetbuttcap%
\pgfsetroundjoin%
\definecolor{currentfill}{rgb}{0.993248,0.906157,0.143936}%
\pgfsetfillcolor{currentfill}%
\pgfsetfillopacity{0.700000}%
\pgfsetlinewidth{0.000000pt}%
\definecolor{currentstroke}{rgb}{0.000000,0.000000,0.000000}%
\pgfsetstrokecolor{currentstroke}%
\pgfsetstrokeopacity{0.700000}%
\pgfsetdash{}{0pt}%
\pgfpathmoveto{\pgfqpoint{8.962311in}{1.197021in}}%
\pgfpathcurveto{\pgfqpoint{8.967355in}{1.197021in}}{\pgfqpoint{8.972192in}{1.199024in}}{\pgfqpoint{8.975759in}{1.202591in}}%
\pgfpathcurveto{\pgfqpoint{8.979325in}{1.206157in}}{\pgfqpoint{8.981329in}{1.210995in}}{\pgfqpoint{8.981329in}{1.216039in}}%
\pgfpathcurveto{\pgfqpoint{8.981329in}{1.221082in}}{\pgfqpoint{8.979325in}{1.225920in}}{\pgfqpoint{8.975759in}{1.229487in}}%
\pgfpathcurveto{\pgfqpoint{8.972192in}{1.233053in}}{\pgfqpoint{8.967355in}{1.235057in}}{\pgfqpoint{8.962311in}{1.235057in}}%
\pgfpathcurveto{\pgfqpoint{8.957267in}{1.235057in}}{\pgfqpoint{8.952429in}{1.233053in}}{\pgfqpoint{8.948863in}{1.229487in}}%
\pgfpathcurveto{\pgfqpoint{8.945297in}{1.225920in}}{\pgfqpoint{8.943293in}{1.221082in}}{\pgfqpoint{8.943293in}{1.216039in}}%
\pgfpathcurveto{\pgfqpoint{8.943293in}{1.210995in}}{\pgfqpoint{8.945297in}{1.206157in}}{\pgfqpoint{8.948863in}{1.202591in}}%
\pgfpathcurveto{\pgfqpoint{8.952429in}{1.199024in}}{\pgfqpoint{8.957267in}{1.197021in}}{\pgfqpoint{8.962311in}{1.197021in}}%
\pgfpathclose%
\pgfusepath{fill}%
\end{pgfscope}%
\begin{pgfscope}%
\pgfpathrectangle{\pgfqpoint{6.572727in}{0.473000in}}{\pgfqpoint{4.227273in}{3.311000in}}%
\pgfusepath{clip}%
\pgfsetbuttcap%
\pgfsetroundjoin%
\definecolor{currentfill}{rgb}{0.127568,0.566949,0.550556}%
\pgfsetfillcolor{currentfill}%
\pgfsetfillopacity{0.700000}%
\pgfsetlinewidth{0.000000pt}%
\definecolor{currentstroke}{rgb}{0.000000,0.000000,0.000000}%
\pgfsetstrokecolor{currentstroke}%
\pgfsetstrokeopacity{0.700000}%
\pgfsetdash{}{0pt}%
\pgfpathmoveto{\pgfqpoint{8.247980in}{2.983694in}}%
\pgfpathcurveto{\pgfqpoint{8.253024in}{2.983694in}}{\pgfqpoint{8.257862in}{2.985698in}}{\pgfqpoint{8.261428in}{2.989264in}}%
\pgfpathcurveto{\pgfqpoint{8.264994in}{2.992831in}}{\pgfqpoint{8.266998in}{2.997669in}}{\pgfqpoint{8.266998in}{3.002712in}}%
\pgfpathcurveto{\pgfqpoint{8.266998in}{3.007756in}}{\pgfqpoint{8.264994in}{3.012594in}}{\pgfqpoint{8.261428in}{3.016160in}}%
\pgfpathcurveto{\pgfqpoint{8.257862in}{3.019726in}}{\pgfqpoint{8.253024in}{3.021730in}}{\pgfqpoint{8.247980in}{3.021730in}}%
\pgfpathcurveto{\pgfqpoint{8.242936in}{3.021730in}}{\pgfqpoint{8.238099in}{3.019726in}}{\pgfqpoint{8.234532in}{3.016160in}}%
\pgfpathcurveto{\pgfqpoint{8.230966in}{3.012594in}}{\pgfqpoint{8.228962in}{3.007756in}}{\pgfqpoint{8.228962in}{3.002712in}}%
\pgfpathcurveto{\pgfqpoint{8.228962in}{2.997669in}}{\pgfqpoint{8.230966in}{2.992831in}}{\pgfqpoint{8.234532in}{2.989264in}}%
\pgfpathcurveto{\pgfqpoint{8.238099in}{2.985698in}}{\pgfqpoint{8.242936in}{2.983694in}}{\pgfqpoint{8.247980in}{2.983694in}}%
\pgfpathclose%
\pgfusepath{fill}%
\end{pgfscope}%
\begin{pgfscope}%
\pgfpathrectangle{\pgfqpoint{6.572727in}{0.473000in}}{\pgfqpoint{4.227273in}{3.311000in}}%
\pgfusepath{clip}%
\pgfsetbuttcap%
\pgfsetroundjoin%
\definecolor{currentfill}{rgb}{0.127568,0.566949,0.550556}%
\pgfsetfillcolor{currentfill}%
\pgfsetfillopacity{0.700000}%
\pgfsetlinewidth{0.000000pt}%
\definecolor{currentstroke}{rgb}{0.000000,0.000000,0.000000}%
\pgfsetstrokecolor{currentstroke}%
\pgfsetstrokeopacity{0.700000}%
\pgfsetdash{}{0pt}%
\pgfpathmoveto{\pgfqpoint{8.512728in}{2.850534in}}%
\pgfpathcurveto{\pgfqpoint{8.517772in}{2.850534in}}{\pgfqpoint{8.522610in}{2.852538in}}{\pgfqpoint{8.526176in}{2.856104in}}%
\pgfpathcurveto{\pgfqpoint{8.529743in}{2.859670in}}{\pgfqpoint{8.531747in}{2.864508in}}{\pgfqpoint{8.531747in}{2.869552in}}%
\pgfpathcurveto{\pgfqpoint{8.531747in}{2.874596in}}{\pgfqpoint{8.529743in}{2.879433in}}{\pgfqpoint{8.526176in}{2.883000in}}%
\pgfpathcurveto{\pgfqpoint{8.522610in}{2.886566in}}{\pgfqpoint{8.517772in}{2.888570in}}{\pgfqpoint{8.512728in}{2.888570in}}%
\pgfpathcurveto{\pgfqpoint{8.507685in}{2.888570in}}{\pgfqpoint{8.502847in}{2.886566in}}{\pgfqpoint{8.499281in}{2.883000in}}%
\pgfpathcurveto{\pgfqpoint{8.495714in}{2.879433in}}{\pgfqpoint{8.493710in}{2.874596in}}{\pgfqpoint{8.493710in}{2.869552in}}%
\pgfpathcurveto{\pgfqpoint{8.493710in}{2.864508in}}{\pgfqpoint{8.495714in}{2.859670in}}{\pgfqpoint{8.499281in}{2.856104in}}%
\pgfpathcurveto{\pgfqpoint{8.502847in}{2.852538in}}{\pgfqpoint{8.507685in}{2.850534in}}{\pgfqpoint{8.512728in}{2.850534in}}%
\pgfpathclose%
\pgfusepath{fill}%
\end{pgfscope}%
\begin{pgfscope}%
\pgfpathrectangle{\pgfqpoint{6.572727in}{0.473000in}}{\pgfqpoint{4.227273in}{3.311000in}}%
\pgfusepath{clip}%
\pgfsetbuttcap%
\pgfsetroundjoin%
\definecolor{currentfill}{rgb}{0.127568,0.566949,0.550556}%
\pgfsetfillcolor{currentfill}%
\pgfsetfillopacity{0.700000}%
\pgfsetlinewidth{0.000000pt}%
\definecolor{currentstroke}{rgb}{0.000000,0.000000,0.000000}%
\pgfsetstrokecolor{currentstroke}%
\pgfsetstrokeopacity{0.700000}%
\pgfsetdash{}{0pt}%
\pgfpathmoveto{\pgfqpoint{7.567537in}{2.647473in}}%
\pgfpathcurveto{\pgfqpoint{7.572581in}{2.647473in}}{\pgfqpoint{7.577418in}{2.649477in}}{\pgfqpoint{7.580985in}{2.653043in}}%
\pgfpathcurveto{\pgfqpoint{7.584551in}{2.656610in}}{\pgfqpoint{7.586555in}{2.661447in}}{\pgfqpoint{7.586555in}{2.666491in}}%
\pgfpathcurveto{\pgfqpoint{7.586555in}{2.671535in}}{\pgfqpoint{7.584551in}{2.676373in}}{\pgfqpoint{7.580985in}{2.679939in}}%
\pgfpathcurveto{\pgfqpoint{7.577418in}{2.683505in}}{\pgfqpoint{7.572581in}{2.685509in}}{\pgfqpoint{7.567537in}{2.685509in}}%
\pgfpathcurveto{\pgfqpoint{7.562493in}{2.685509in}}{\pgfqpoint{7.557655in}{2.683505in}}{\pgfqpoint{7.554089in}{2.679939in}}%
\pgfpathcurveto{\pgfqpoint{7.550523in}{2.676373in}}{\pgfqpoint{7.548519in}{2.671535in}}{\pgfqpoint{7.548519in}{2.666491in}}%
\pgfpathcurveto{\pgfqpoint{7.548519in}{2.661447in}}{\pgfqpoint{7.550523in}{2.656610in}}{\pgfqpoint{7.554089in}{2.653043in}}%
\pgfpathcurveto{\pgfqpoint{7.557655in}{2.649477in}}{\pgfqpoint{7.562493in}{2.647473in}}{\pgfqpoint{7.567537in}{2.647473in}}%
\pgfpathclose%
\pgfusepath{fill}%
\end{pgfscope}%
\begin{pgfscope}%
\pgfpathrectangle{\pgfqpoint{6.572727in}{0.473000in}}{\pgfqpoint{4.227273in}{3.311000in}}%
\pgfusepath{clip}%
\pgfsetbuttcap%
\pgfsetroundjoin%
\definecolor{currentfill}{rgb}{0.127568,0.566949,0.550556}%
\pgfsetfillcolor{currentfill}%
\pgfsetfillopacity{0.700000}%
\pgfsetlinewidth{0.000000pt}%
\definecolor{currentstroke}{rgb}{0.000000,0.000000,0.000000}%
\pgfsetstrokecolor{currentstroke}%
\pgfsetstrokeopacity{0.700000}%
\pgfsetdash{}{0pt}%
\pgfpathmoveto{\pgfqpoint{8.304035in}{1.815693in}}%
\pgfpathcurveto{\pgfqpoint{8.309079in}{1.815693in}}{\pgfqpoint{8.313917in}{1.817697in}}{\pgfqpoint{8.317483in}{1.821264in}}%
\pgfpathcurveto{\pgfqpoint{8.321050in}{1.824830in}}{\pgfqpoint{8.323054in}{1.829668in}}{\pgfqpoint{8.323054in}{1.834711in}}%
\pgfpathcurveto{\pgfqpoint{8.323054in}{1.839755in}}{\pgfqpoint{8.321050in}{1.844593in}}{\pgfqpoint{8.317483in}{1.848159in}}%
\pgfpathcurveto{\pgfqpoint{8.313917in}{1.851726in}}{\pgfqpoint{8.309079in}{1.853730in}}{\pgfqpoint{8.304035in}{1.853730in}}%
\pgfpathcurveto{\pgfqpoint{8.298992in}{1.853730in}}{\pgfqpoint{8.294154in}{1.851726in}}{\pgfqpoint{8.290588in}{1.848159in}}%
\pgfpathcurveto{\pgfqpoint{8.287021in}{1.844593in}}{\pgfqpoint{8.285017in}{1.839755in}}{\pgfqpoint{8.285017in}{1.834711in}}%
\pgfpathcurveto{\pgfqpoint{8.285017in}{1.829668in}}{\pgfqpoint{8.287021in}{1.824830in}}{\pgfqpoint{8.290588in}{1.821264in}}%
\pgfpathcurveto{\pgfqpoint{8.294154in}{1.817697in}}{\pgfqpoint{8.298992in}{1.815693in}}{\pgfqpoint{8.304035in}{1.815693in}}%
\pgfpathclose%
\pgfusepath{fill}%
\end{pgfscope}%
\begin{pgfscope}%
\pgfpathrectangle{\pgfqpoint{6.572727in}{0.473000in}}{\pgfqpoint{4.227273in}{3.311000in}}%
\pgfusepath{clip}%
\pgfsetbuttcap%
\pgfsetroundjoin%
\definecolor{currentfill}{rgb}{0.127568,0.566949,0.550556}%
\pgfsetfillcolor{currentfill}%
\pgfsetfillopacity{0.700000}%
\pgfsetlinewidth{0.000000pt}%
\definecolor{currentstroke}{rgb}{0.000000,0.000000,0.000000}%
\pgfsetstrokecolor{currentstroke}%
\pgfsetstrokeopacity{0.700000}%
\pgfsetdash{}{0pt}%
\pgfpathmoveto{\pgfqpoint{7.945251in}{1.733490in}}%
\pgfpathcurveto{\pgfqpoint{7.950295in}{1.733490in}}{\pgfqpoint{7.955132in}{1.735494in}}{\pgfqpoint{7.958699in}{1.739060in}}%
\pgfpathcurveto{\pgfqpoint{7.962265in}{1.742626in}}{\pgfqpoint{7.964269in}{1.747464in}}{\pgfqpoint{7.964269in}{1.752508in}}%
\pgfpathcurveto{\pgfqpoint{7.964269in}{1.757552in}}{\pgfqpoint{7.962265in}{1.762389in}}{\pgfqpoint{7.958699in}{1.765956in}}%
\pgfpathcurveto{\pgfqpoint{7.955132in}{1.769522in}}{\pgfqpoint{7.950295in}{1.771526in}}{\pgfqpoint{7.945251in}{1.771526in}}%
\pgfpathcurveto{\pgfqpoint{7.940207in}{1.771526in}}{\pgfqpoint{7.935370in}{1.769522in}}{\pgfqpoint{7.931803in}{1.765956in}}%
\pgfpathcurveto{\pgfqpoint{7.928237in}{1.762389in}}{\pgfqpoint{7.926233in}{1.757552in}}{\pgfqpoint{7.926233in}{1.752508in}}%
\pgfpathcurveto{\pgfqpoint{7.926233in}{1.747464in}}{\pgfqpoint{7.928237in}{1.742626in}}{\pgfqpoint{7.931803in}{1.739060in}}%
\pgfpathcurveto{\pgfqpoint{7.935370in}{1.735494in}}{\pgfqpoint{7.940207in}{1.733490in}}{\pgfqpoint{7.945251in}{1.733490in}}%
\pgfpathclose%
\pgfusepath{fill}%
\end{pgfscope}%
\begin{pgfscope}%
\pgfpathrectangle{\pgfqpoint{6.572727in}{0.473000in}}{\pgfqpoint{4.227273in}{3.311000in}}%
\pgfusepath{clip}%
\pgfsetbuttcap%
\pgfsetroundjoin%
\definecolor{currentfill}{rgb}{0.127568,0.566949,0.550556}%
\pgfsetfillcolor{currentfill}%
\pgfsetfillopacity{0.700000}%
\pgfsetlinewidth{0.000000pt}%
\definecolor{currentstroke}{rgb}{0.000000,0.000000,0.000000}%
\pgfsetstrokecolor{currentstroke}%
\pgfsetstrokeopacity{0.700000}%
\pgfsetdash{}{0pt}%
\pgfpathmoveto{\pgfqpoint{8.374078in}{2.934257in}}%
\pgfpathcurveto{\pgfqpoint{8.379122in}{2.934257in}}{\pgfqpoint{8.383960in}{2.936261in}}{\pgfqpoint{8.387526in}{2.939828in}}%
\pgfpathcurveto{\pgfqpoint{8.391092in}{2.943394in}}{\pgfqpoint{8.393096in}{2.948232in}}{\pgfqpoint{8.393096in}{2.953275in}}%
\pgfpathcurveto{\pgfqpoint{8.393096in}{2.958319in}}{\pgfqpoint{8.391092in}{2.963157in}}{\pgfqpoint{8.387526in}{2.966723in}}%
\pgfpathcurveto{\pgfqpoint{8.383960in}{2.970290in}}{\pgfqpoint{8.379122in}{2.972294in}}{\pgfqpoint{8.374078in}{2.972294in}}%
\pgfpathcurveto{\pgfqpoint{8.369035in}{2.972294in}}{\pgfqpoint{8.364197in}{2.970290in}}{\pgfqpoint{8.360630in}{2.966723in}}%
\pgfpathcurveto{\pgfqpoint{8.357064in}{2.963157in}}{\pgfqpoint{8.355060in}{2.958319in}}{\pgfqpoint{8.355060in}{2.953275in}}%
\pgfpathcurveto{\pgfqpoint{8.355060in}{2.948232in}}{\pgfqpoint{8.357064in}{2.943394in}}{\pgfqpoint{8.360630in}{2.939828in}}%
\pgfpathcurveto{\pgfqpoint{8.364197in}{2.936261in}}{\pgfqpoint{8.369035in}{2.934257in}}{\pgfqpoint{8.374078in}{2.934257in}}%
\pgfpathclose%
\pgfusepath{fill}%
\end{pgfscope}%
\begin{pgfscope}%
\pgfpathrectangle{\pgfqpoint{6.572727in}{0.473000in}}{\pgfqpoint{4.227273in}{3.311000in}}%
\pgfusepath{clip}%
\pgfsetbuttcap%
\pgfsetroundjoin%
\definecolor{currentfill}{rgb}{0.127568,0.566949,0.550556}%
\pgfsetfillcolor{currentfill}%
\pgfsetfillopacity{0.700000}%
\pgfsetlinewidth{0.000000pt}%
\definecolor{currentstroke}{rgb}{0.000000,0.000000,0.000000}%
\pgfsetstrokecolor{currentstroke}%
\pgfsetstrokeopacity{0.700000}%
\pgfsetdash{}{0pt}%
\pgfpathmoveto{\pgfqpoint{8.391447in}{1.879795in}}%
\pgfpathcurveto{\pgfqpoint{8.396491in}{1.879795in}}{\pgfqpoint{8.401329in}{1.881799in}}{\pgfqpoint{8.404895in}{1.885366in}}%
\pgfpathcurveto{\pgfqpoint{8.408461in}{1.888932in}}{\pgfqpoint{8.410465in}{1.893770in}}{\pgfqpoint{8.410465in}{1.898813in}}%
\pgfpathcurveto{\pgfqpoint{8.410465in}{1.903857in}}{\pgfqpoint{8.408461in}{1.908695in}}{\pgfqpoint{8.404895in}{1.912261in}}%
\pgfpathcurveto{\pgfqpoint{8.401329in}{1.915828in}}{\pgfqpoint{8.396491in}{1.917832in}}{\pgfqpoint{8.391447in}{1.917832in}}%
\pgfpathcurveto{\pgfqpoint{8.386403in}{1.917832in}}{\pgfqpoint{8.381566in}{1.915828in}}{\pgfqpoint{8.377999in}{1.912261in}}%
\pgfpathcurveto{\pgfqpoint{8.374433in}{1.908695in}}{\pgfqpoint{8.372429in}{1.903857in}}{\pgfqpoint{8.372429in}{1.898813in}}%
\pgfpathcurveto{\pgfqpoint{8.372429in}{1.893770in}}{\pgfqpoint{8.374433in}{1.888932in}}{\pgfqpoint{8.377999in}{1.885366in}}%
\pgfpathcurveto{\pgfqpoint{8.381566in}{1.881799in}}{\pgfqpoint{8.386403in}{1.879795in}}{\pgfqpoint{8.391447in}{1.879795in}}%
\pgfpathclose%
\pgfusepath{fill}%
\end{pgfscope}%
\begin{pgfscope}%
\pgfpathrectangle{\pgfqpoint{6.572727in}{0.473000in}}{\pgfqpoint{4.227273in}{3.311000in}}%
\pgfusepath{clip}%
\pgfsetbuttcap%
\pgfsetroundjoin%
\definecolor{currentfill}{rgb}{0.993248,0.906157,0.143936}%
\pgfsetfillcolor{currentfill}%
\pgfsetfillopacity{0.700000}%
\pgfsetlinewidth{0.000000pt}%
\definecolor{currentstroke}{rgb}{0.000000,0.000000,0.000000}%
\pgfsetstrokecolor{currentstroke}%
\pgfsetstrokeopacity{0.700000}%
\pgfsetdash{}{0pt}%
\pgfpathmoveto{\pgfqpoint{9.050304in}{1.383167in}}%
\pgfpathcurveto{\pgfqpoint{9.055348in}{1.383167in}}{\pgfqpoint{9.060186in}{1.385171in}}{\pgfqpoint{9.063752in}{1.388737in}}%
\pgfpathcurveto{\pgfqpoint{9.067319in}{1.392303in}}{\pgfqpoint{9.069322in}{1.397141in}}{\pgfqpoint{9.069322in}{1.402185in}}%
\pgfpathcurveto{\pgfqpoint{9.069322in}{1.407229in}}{\pgfqpoint{9.067319in}{1.412066in}}{\pgfqpoint{9.063752in}{1.415633in}}%
\pgfpathcurveto{\pgfqpoint{9.060186in}{1.419199in}}{\pgfqpoint{9.055348in}{1.421203in}}{\pgfqpoint{9.050304in}{1.421203in}}%
\pgfpathcurveto{\pgfqpoint{9.045261in}{1.421203in}}{\pgfqpoint{9.040423in}{1.419199in}}{\pgfqpoint{9.036856in}{1.415633in}}%
\pgfpathcurveto{\pgfqpoint{9.033290in}{1.412066in}}{\pgfqpoint{9.031286in}{1.407229in}}{\pgfqpoint{9.031286in}{1.402185in}}%
\pgfpathcurveto{\pgfqpoint{9.031286in}{1.397141in}}{\pgfqpoint{9.033290in}{1.392303in}}{\pgfqpoint{9.036856in}{1.388737in}}%
\pgfpathcurveto{\pgfqpoint{9.040423in}{1.385171in}}{\pgfqpoint{9.045261in}{1.383167in}}{\pgfqpoint{9.050304in}{1.383167in}}%
\pgfpathclose%
\pgfusepath{fill}%
\end{pgfscope}%
\begin{pgfscope}%
\pgfpathrectangle{\pgfqpoint{6.572727in}{0.473000in}}{\pgfqpoint{4.227273in}{3.311000in}}%
\pgfusepath{clip}%
\pgfsetbuttcap%
\pgfsetroundjoin%
\definecolor{currentfill}{rgb}{0.127568,0.566949,0.550556}%
\pgfsetfillcolor{currentfill}%
\pgfsetfillopacity{0.700000}%
\pgfsetlinewidth{0.000000pt}%
\definecolor{currentstroke}{rgb}{0.000000,0.000000,0.000000}%
\pgfsetstrokecolor{currentstroke}%
\pgfsetstrokeopacity{0.700000}%
\pgfsetdash{}{0pt}%
\pgfpathmoveto{\pgfqpoint{8.129471in}{1.779754in}}%
\pgfpathcurveto{\pgfqpoint{8.134515in}{1.779754in}}{\pgfqpoint{8.139353in}{1.781758in}}{\pgfqpoint{8.142919in}{1.785324in}}%
\pgfpathcurveto{\pgfqpoint{8.146486in}{1.788891in}}{\pgfqpoint{8.148490in}{1.793728in}}{\pgfqpoint{8.148490in}{1.798772in}}%
\pgfpathcurveto{\pgfqpoint{8.148490in}{1.803816in}}{\pgfqpoint{8.146486in}{1.808654in}}{\pgfqpoint{8.142919in}{1.812220in}}%
\pgfpathcurveto{\pgfqpoint{8.139353in}{1.815786in}}{\pgfqpoint{8.134515in}{1.817790in}}{\pgfqpoint{8.129471in}{1.817790in}}%
\pgfpathcurveto{\pgfqpoint{8.124428in}{1.817790in}}{\pgfqpoint{8.119590in}{1.815786in}}{\pgfqpoint{8.116024in}{1.812220in}}%
\pgfpathcurveto{\pgfqpoint{8.112457in}{1.808654in}}{\pgfqpoint{8.110453in}{1.803816in}}{\pgfqpoint{8.110453in}{1.798772in}}%
\pgfpathcurveto{\pgfqpoint{8.110453in}{1.793728in}}{\pgfqpoint{8.112457in}{1.788891in}}{\pgfqpoint{8.116024in}{1.785324in}}%
\pgfpathcurveto{\pgfqpoint{8.119590in}{1.781758in}}{\pgfqpoint{8.124428in}{1.779754in}}{\pgfqpoint{8.129471in}{1.779754in}}%
\pgfpathclose%
\pgfusepath{fill}%
\end{pgfscope}%
\begin{pgfscope}%
\pgfpathrectangle{\pgfqpoint{6.572727in}{0.473000in}}{\pgfqpoint{4.227273in}{3.311000in}}%
\pgfusepath{clip}%
\pgfsetbuttcap%
\pgfsetroundjoin%
\definecolor{currentfill}{rgb}{0.127568,0.566949,0.550556}%
\pgfsetfillcolor{currentfill}%
\pgfsetfillopacity{0.700000}%
\pgfsetlinewidth{0.000000pt}%
\definecolor{currentstroke}{rgb}{0.000000,0.000000,0.000000}%
\pgfsetstrokecolor{currentstroke}%
\pgfsetstrokeopacity{0.700000}%
\pgfsetdash{}{0pt}%
\pgfpathmoveto{\pgfqpoint{7.962267in}{2.475956in}}%
\pgfpathcurveto{\pgfqpoint{7.967311in}{2.475956in}}{\pgfqpoint{7.972149in}{2.477960in}}{\pgfqpoint{7.975715in}{2.481526in}}%
\pgfpathcurveto{\pgfqpoint{7.979282in}{2.485093in}}{\pgfqpoint{7.981286in}{2.489930in}}{\pgfqpoint{7.981286in}{2.494974in}}%
\pgfpathcurveto{\pgfqpoint{7.981286in}{2.500018in}}{\pgfqpoint{7.979282in}{2.504856in}}{\pgfqpoint{7.975715in}{2.508422in}}%
\pgfpathcurveto{\pgfqpoint{7.972149in}{2.511988in}}{\pgfqpoint{7.967311in}{2.513992in}}{\pgfqpoint{7.962267in}{2.513992in}}%
\pgfpathcurveto{\pgfqpoint{7.957224in}{2.513992in}}{\pgfqpoint{7.952386in}{2.511988in}}{\pgfqpoint{7.948820in}{2.508422in}}%
\pgfpathcurveto{\pgfqpoint{7.945253in}{2.504856in}}{\pgfqpoint{7.943249in}{2.500018in}}{\pgfqpoint{7.943249in}{2.494974in}}%
\pgfpathcurveto{\pgfqpoint{7.943249in}{2.489930in}}{\pgfqpoint{7.945253in}{2.485093in}}{\pgfqpoint{7.948820in}{2.481526in}}%
\pgfpathcurveto{\pgfqpoint{7.952386in}{2.477960in}}{\pgfqpoint{7.957224in}{2.475956in}}{\pgfqpoint{7.962267in}{2.475956in}}%
\pgfpathclose%
\pgfusepath{fill}%
\end{pgfscope}%
\begin{pgfscope}%
\pgfpathrectangle{\pgfqpoint{6.572727in}{0.473000in}}{\pgfqpoint{4.227273in}{3.311000in}}%
\pgfusepath{clip}%
\pgfsetbuttcap%
\pgfsetroundjoin%
\definecolor{currentfill}{rgb}{0.993248,0.906157,0.143936}%
\pgfsetfillcolor{currentfill}%
\pgfsetfillopacity{0.700000}%
\pgfsetlinewidth{0.000000pt}%
\definecolor{currentstroke}{rgb}{0.000000,0.000000,0.000000}%
\pgfsetstrokecolor{currentstroke}%
\pgfsetstrokeopacity{0.700000}%
\pgfsetdash{}{0pt}%
\pgfpathmoveto{\pgfqpoint{9.766269in}{2.084250in}}%
\pgfpathcurveto{\pgfqpoint{9.771313in}{2.084250in}}{\pgfqpoint{9.776151in}{2.086254in}}{\pgfqpoint{9.779717in}{2.089820in}}%
\pgfpathcurveto{\pgfqpoint{9.783284in}{2.093387in}}{\pgfqpoint{9.785287in}{2.098225in}}{\pgfqpoint{9.785287in}{2.103268in}}%
\pgfpathcurveto{\pgfqpoint{9.785287in}{2.108312in}}{\pgfqpoint{9.783284in}{2.113150in}}{\pgfqpoint{9.779717in}{2.116716in}}%
\pgfpathcurveto{\pgfqpoint{9.776151in}{2.120282in}}{\pgfqpoint{9.771313in}{2.122286in}}{\pgfqpoint{9.766269in}{2.122286in}}%
\pgfpathcurveto{\pgfqpoint{9.761226in}{2.122286in}}{\pgfqpoint{9.756388in}{2.120282in}}{\pgfqpoint{9.752821in}{2.116716in}}%
\pgfpathcurveto{\pgfqpoint{9.749255in}{2.113150in}}{\pgfqpoint{9.747251in}{2.108312in}}{\pgfqpoint{9.747251in}{2.103268in}}%
\pgfpathcurveto{\pgfqpoint{9.747251in}{2.098225in}}{\pgfqpoint{9.749255in}{2.093387in}}{\pgfqpoint{9.752821in}{2.089820in}}%
\pgfpathcurveto{\pgfqpoint{9.756388in}{2.086254in}}{\pgfqpoint{9.761226in}{2.084250in}}{\pgfqpoint{9.766269in}{2.084250in}}%
\pgfpathclose%
\pgfusepath{fill}%
\end{pgfscope}%
\begin{pgfscope}%
\pgfpathrectangle{\pgfqpoint{6.572727in}{0.473000in}}{\pgfqpoint{4.227273in}{3.311000in}}%
\pgfusepath{clip}%
\pgfsetbuttcap%
\pgfsetroundjoin%
\definecolor{currentfill}{rgb}{0.127568,0.566949,0.550556}%
\pgfsetfillcolor{currentfill}%
\pgfsetfillopacity{0.700000}%
\pgfsetlinewidth{0.000000pt}%
\definecolor{currentstroke}{rgb}{0.000000,0.000000,0.000000}%
\pgfsetstrokecolor{currentstroke}%
\pgfsetstrokeopacity{0.700000}%
\pgfsetdash{}{0pt}%
\pgfpathmoveto{\pgfqpoint{7.789275in}{1.793689in}}%
\pgfpathcurveto{\pgfqpoint{7.794319in}{1.793689in}}{\pgfqpoint{7.799157in}{1.795693in}}{\pgfqpoint{7.802723in}{1.799260in}}%
\pgfpathcurveto{\pgfqpoint{7.806289in}{1.802826in}}{\pgfqpoint{7.808293in}{1.807664in}}{\pgfqpoint{7.808293in}{1.812708in}}%
\pgfpathcurveto{\pgfqpoint{7.808293in}{1.817751in}}{\pgfqpoint{7.806289in}{1.822589in}}{\pgfqpoint{7.802723in}{1.826155in}}%
\pgfpathcurveto{\pgfqpoint{7.799157in}{1.829722in}}{\pgfqpoint{7.794319in}{1.831726in}}{\pgfqpoint{7.789275in}{1.831726in}}%
\pgfpathcurveto{\pgfqpoint{7.784232in}{1.831726in}}{\pgfqpoint{7.779394in}{1.829722in}}{\pgfqpoint{7.775827in}{1.826155in}}%
\pgfpathcurveto{\pgfqpoint{7.772261in}{1.822589in}}{\pgfqpoint{7.770257in}{1.817751in}}{\pgfqpoint{7.770257in}{1.812708in}}%
\pgfpathcurveto{\pgfqpoint{7.770257in}{1.807664in}}{\pgfqpoint{7.772261in}{1.802826in}}{\pgfqpoint{7.775827in}{1.799260in}}%
\pgfpathcurveto{\pgfqpoint{7.779394in}{1.795693in}}{\pgfqpoint{7.784232in}{1.793689in}}{\pgfqpoint{7.789275in}{1.793689in}}%
\pgfpathclose%
\pgfusepath{fill}%
\end{pgfscope}%
\begin{pgfscope}%
\pgfpathrectangle{\pgfqpoint{6.572727in}{0.473000in}}{\pgfqpoint{4.227273in}{3.311000in}}%
\pgfusepath{clip}%
\pgfsetbuttcap%
\pgfsetroundjoin%
\definecolor{currentfill}{rgb}{0.993248,0.906157,0.143936}%
\pgfsetfillcolor{currentfill}%
\pgfsetfillopacity{0.700000}%
\pgfsetlinewidth{0.000000pt}%
\definecolor{currentstroke}{rgb}{0.000000,0.000000,0.000000}%
\pgfsetstrokecolor{currentstroke}%
\pgfsetstrokeopacity{0.700000}%
\pgfsetdash{}{0pt}%
\pgfpathmoveto{\pgfqpoint{10.206277in}{1.337593in}}%
\pgfpathcurveto{\pgfqpoint{10.211320in}{1.337593in}}{\pgfqpoint{10.216158in}{1.339597in}}{\pgfqpoint{10.219724in}{1.343163in}}%
\pgfpathcurveto{\pgfqpoint{10.223291in}{1.346729in}}{\pgfqpoint{10.225295in}{1.351567in}}{\pgfqpoint{10.225295in}{1.356611in}}%
\pgfpathcurveto{\pgfqpoint{10.225295in}{1.361654in}}{\pgfqpoint{10.223291in}{1.366492in}}{\pgfqpoint{10.219724in}{1.370059in}}%
\pgfpathcurveto{\pgfqpoint{10.216158in}{1.373625in}}{\pgfqpoint{10.211320in}{1.375629in}}{\pgfqpoint{10.206277in}{1.375629in}}%
\pgfpathcurveto{\pgfqpoint{10.201233in}{1.375629in}}{\pgfqpoint{10.196395in}{1.373625in}}{\pgfqpoint{10.192829in}{1.370059in}}%
\pgfpathcurveto{\pgfqpoint{10.189262in}{1.366492in}}{\pgfqpoint{10.187258in}{1.361654in}}{\pgfqpoint{10.187258in}{1.356611in}}%
\pgfpathcurveto{\pgfqpoint{10.187258in}{1.351567in}}{\pgfqpoint{10.189262in}{1.346729in}}{\pgfqpoint{10.192829in}{1.343163in}}%
\pgfpathcurveto{\pgfqpoint{10.196395in}{1.339597in}}{\pgfqpoint{10.201233in}{1.337593in}}{\pgfqpoint{10.206277in}{1.337593in}}%
\pgfpathclose%
\pgfusepath{fill}%
\end{pgfscope}%
\begin{pgfscope}%
\pgfpathrectangle{\pgfqpoint{6.572727in}{0.473000in}}{\pgfqpoint{4.227273in}{3.311000in}}%
\pgfusepath{clip}%
\pgfsetbuttcap%
\pgfsetroundjoin%
\definecolor{currentfill}{rgb}{0.127568,0.566949,0.550556}%
\pgfsetfillcolor{currentfill}%
\pgfsetfillopacity{0.700000}%
\pgfsetlinewidth{0.000000pt}%
\definecolor{currentstroke}{rgb}{0.000000,0.000000,0.000000}%
\pgfsetstrokecolor{currentstroke}%
\pgfsetstrokeopacity{0.700000}%
\pgfsetdash{}{0pt}%
\pgfpathmoveto{\pgfqpoint{7.979880in}{1.694191in}}%
\pgfpathcurveto{\pgfqpoint{7.984924in}{1.694191in}}{\pgfqpoint{7.989762in}{1.696195in}}{\pgfqpoint{7.993328in}{1.699761in}}%
\pgfpathcurveto{\pgfqpoint{7.996895in}{1.703328in}}{\pgfqpoint{7.998898in}{1.708165in}}{\pgfqpoint{7.998898in}{1.713209in}}%
\pgfpathcurveto{\pgfqpoint{7.998898in}{1.718253in}}{\pgfqpoint{7.996895in}{1.723090in}}{\pgfqpoint{7.993328in}{1.726657in}}%
\pgfpathcurveto{\pgfqpoint{7.989762in}{1.730223in}}{\pgfqpoint{7.984924in}{1.732227in}}{\pgfqpoint{7.979880in}{1.732227in}}%
\pgfpathcurveto{\pgfqpoint{7.974837in}{1.732227in}}{\pgfqpoint{7.969999in}{1.730223in}}{\pgfqpoint{7.966432in}{1.726657in}}%
\pgfpathcurveto{\pgfqpoint{7.962866in}{1.723090in}}{\pgfqpoint{7.960862in}{1.718253in}}{\pgfqpoint{7.960862in}{1.713209in}}%
\pgfpathcurveto{\pgfqpoint{7.960862in}{1.708165in}}{\pgfqpoint{7.962866in}{1.703328in}}{\pgfqpoint{7.966432in}{1.699761in}}%
\pgfpathcurveto{\pgfqpoint{7.969999in}{1.696195in}}{\pgfqpoint{7.974837in}{1.694191in}}{\pgfqpoint{7.979880in}{1.694191in}}%
\pgfpathclose%
\pgfusepath{fill}%
\end{pgfscope}%
\begin{pgfscope}%
\pgfpathrectangle{\pgfqpoint{6.572727in}{0.473000in}}{\pgfqpoint{4.227273in}{3.311000in}}%
\pgfusepath{clip}%
\pgfsetbuttcap%
\pgfsetroundjoin%
\definecolor{currentfill}{rgb}{0.127568,0.566949,0.550556}%
\pgfsetfillcolor{currentfill}%
\pgfsetfillopacity{0.700000}%
\pgfsetlinewidth{0.000000pt}%
\definecolor{currentstroke}{rgb}{0.000000,0.000000,0.000000}%
\pgfsetstrokecolor{currentstroke}%
\pgfsetstrokeopacity{0.700000}%
\pgfsetdash{}{0pt}%
\pgfpathmoveto{\pgfqpoint{7.675073in}{2.946998in}}%
\pgfpathcurveto{\pgfqpoint{7.680117in}{2.946998in}}{\pgfqpoint{7.684955in}{2.949002in}}{\pgfqpoint{7.688521in}{2.952568in}}%
\pgfpathcurveto{\pgfqpoint{7.692088in}{2.956135in}}{\pgfqpoint{7.694091in}{2.960973in}}{\pgfqpoint{7.694091in}{2.966016in}}%
\pgfpathcurveto{\pgfqpoint{7.694091in}{2.971060in}}{\pgfqpoint{7.692088in}{2.975898in}}{\pgfqpoint{7.688521in}{2.979464in}}%
\pgfpathcurveto{\pgfqpoint{7.684955in}{2.983031in}}{\pgfqpoint{7.680117in}{2.985034in}}{\pgfqpoint{7.675073in}{2.985034in}}%
\pgfpathcurveto{\pgfqpoint{7.670030in}{2.985034in}}{\pgfqpoint{7.665192in}{2.983031in}}{\pgfqpoint{7.661625in}{2.979464in}}%
\pgfpathcurveto{\pgfqpoint{7.658059in}{2.975898in}}{\pgfqpoint{7.656055in}{2.971060in}}{\pgfqpoint{7.656055in}{2.966016in}}%
\pgfpathcurveto{\pgfqpoint{7.656055in}{2.960973in}}{\pgfqpoint{7.658059in}{2.956135in}}{\pgfqpoint{7.661625in}{2.952568in}}%
\pgfpathcurveto{\pgfqpoint{7.665192in}{2.949002in}}{\pgfqpoint{7.670030in}{2.946998in}}{\pgfqpoint{7.675073in}{2.946998in}}%
\pgfpathclose%
\pgfusepath{fill}%
\end{pgfscope}%
\begin{pgfscope}%
\pgfpathrectangle{\pgfqpoint{6.572727in}{0.473000in}}{\pgfqpoint{4.227273in}{3.311000in}}%
\pgfusepath{clip}%
\pgfsetbuttcap%
\pgfsetroundjoin%
\definecolor{currentfill}{rgb}{0.127568,0.566949,0.550556}%
\pgfsetfillcolor{currentfill}%
\pgfsetfillopacity{0.700000}%
\pgfsetlinewidth{0.000000pt}%
\definecolor{currentstroke}{rgb}{0.000000,0.000000,0.000000}%
\pgfsetstrokecolor{currentstroke}%
\pgfsetstrokeopacity{0.700000}%
\pgfsetdash{}{0pt}%
\pgfpathmoveto{\pgfqpoint{8.025314in}{2.727784in}}%
\pgfpathcurveto{\pgfqpoint{8.030358in}{2.727784in}}{\pgfqpoint{8.035195in}{2.729788in}}{\pgfqpoint{8.038762in}{2.733354in}}%
\pgfpathcurveto{\pgfqpoint{8.042328in}{2.736920in}}{\pgfqpoint{8.044332in}{2.741758in}}{\pgfqpoint{8.044332in}{2.746802in}}%
\pgfpathcurveto{\pgfqpoint{8.044332in}{2.751846in}}{\pgfqpoint{8.042328in}{2.756683in}}{\pgfqpoint{8.038762in}{2.760250in}}%
\pgfpathcurveto{\pgfqpoint{8.035195in}{2.763816in}}{\pgfqpoint{8.030358in}{2.765820in}}{\pgfqpoint{8.025314in}{2.765820in}}%
\pgfpathcurveto{\pgfqpoint{8.020270in}{2.765820in}}{\pgfqpoint{8.015432in}{2.763816in}}{\pgfqpoint{8.011866in}{2.760250in}}%
\pgfpathcurveto{\pgfqpoint{8.008300in}{2.756683in}}{\pgfqpoint{8.006296in}{2.751846in}}{\pgfqpoint{8.006296in}{2.746802in}}%
\pgfpathcurveto{\pgfqpoint{8.006296in}{2.741758in}}{\pgfqpoint{8.008300in}{2.736920in}}{\pgfqpoint{8.011866in}{2.733354in}}%
\pgfpathcurveto{\pgfqpoint{8.015432in}{2.729788in}}{\pgfqpoint{8.020270in}{2.727784in}}{\pgfqpoint{8.025314in}{2.727784in}}%
\pgfpathclose%
\pgfusepath{fill}%
\end{pgfscope}%
\begin{pgfscope}%
\pgfpathrectangle{\pgfqpoint{6.572727in}{0.473000in}}{\pgfqpoint{4.227273in}{3.311000in}}%
\pgfusepath{clip}%
\pgfsetbuttcap%
\pgfsetroundjoin%
\definecolor{currentfill}{rgb}{0.127568,0.566949,0.550556}%
\pgfsetfillcolor{currentfill}%
\pgfsetfillopacity{0.700000}%
\pgfsetlinewidth{0.000000pt}%
\definecolor{currentstroke}{rgb}{0.000000,0.000000,0.000000}%
\pgfsetstrokecolor{currentstroke}%
\pgfsetstrokeopacity{0.700000}%
\pgfsetdash{}{0pt}%
\pgfpathmoveto{\pgfqpoint{7.433749in}{1.101675in}}%
\pgfpathcurveto{\pgfqpoint{7.438793in}{1.101675in}}{\pgfqpoint{7.443631in}{1.103679in}}{\pgfqpoint{7.447197in}{1.107245in}}%
\pgfpathcurveto{\pgfqpoint{7.450763in}{1.110812in}}{\pgfqpoint{7.452767in}{1.115650in}}{\pgfqpoint{7.452767in}{1.120693in}}%
\pgfpathcurveto{\pgfqpoint{7.452767in}{1.125737in}}{\pgfqpoint{7.450763in}{1.130575in}}{\pgfqpoint{7.447197in}{1.134141in}}%
\pgfpathcurveto{\pgfqpoint{7.443631in}{1.137708in}}{\pgfqpoint{7.438793in}{1.139711in}}{\pgfqpoint{7.433749in}{1.139711in}}%
\pgfpathcurveto{\pgfqpoint{7.428705in}{1.139711in}}{\pgfqpoint{7.423868in}{1.137708in}}{\pgfqpoint{7.420301in}{1.134141in}}%
\pgfpathcurveto{\pgfqpoint{7.416735in}{1.130575in}}{\pgfqpoint{7.414731in}{1.125737in}}{\pgfqpoint{7.414731in}{1.120693in}}%
\pgfpathcurveto{\pgfqpoint{7.414731in}{1.115650in}}{\pgfqpoint{7.416735in}{1.110812in}}{\pgfqpoint{7.420301in}{1.107245in}}%
\pgfpathcurveto{\pgfqpoint{7.423868in}{1.103679in}}{\pgfqpoint{7.428705in}{1.101675in}}{\pgfqpoint{7.433749in}{1.101675in}}%
\pgfpathclose%
\pgfusepath{fill}%
\end{pgfscope}%
\begin{pgfscope}%
\pgfpathrectangle{\pgfqpoint{6.572727in}{0.473000in}}{\pgfqpoint{4.227273in}{3.311000in}}%
\pgfusepath{clip}%
\pgfsetbuttcap%
\pgfsetroundjoin%
\definecolor{currentfill}{rgb}{0.127568,0.566949,0.550556}%
\pgfsetfillcolor{currentfill}%
\pgfsetfillopacity{0.700000}%
\pgfsetlinewidth{0.000000pt}%
\definecolor{currentstroke}{rgb}{0.000000,0.000000,0.000000}%
\pgfsetstrokecolor{currentstroke}%
\pgfsetstrokeopacity{0.700000}%
\pgfsetdash{}{0pt}%
\pgfpathmoveto{\pgfqpoint{7.968491in}{1.544035in}}%
\pgfpathcurveto{\pgfqpoint{7.973535in}{1.544035in}}{\pgfqpoint{7.978373in}{1.546039in}}{\pgfqpoint{7.981939in}{1.549606in}}%
\pgfpathcurveto{\pgfqpoint{7.985506in}{1.553172in}}{\pgfqpoint{7.987510in}{1.558010in}}{\pgfqpoint{7.987510in}{1.563053in}}%
\pgfpathcurveto{\pgfqpoint{7.987510in}{1.568097in}}{\pgfqpoint{7.985506in}{1.572935in}}{\pgfqpoint{7.981939in}{1.576501in}}%
\pgfpathcurveto{\pgfqpoint{7.978373in}{1.580068in}}{\pgfqpoint{7.973535in}{1.582072in}}{\pgfqpoint{7.968491in}{1.582072in}}%
\pgfpathcurveto{\pgfqpoint{7.963448in}{1.582072in}}{\pgfqpoint{7.958610in}{1.580068in}}{\pgfqpoint{7.955044in}{1.576501in}}%
\pgfpathcurveto{\pgfqpoint{7.951477in}{1.572935in}}{\pgfqpoint{7.949473in}{1.568097in}}{\pgfqpoint{7.949473in}{1.563053in}}%
\pgfpathcurveto{\pgfqpoint{7.949473in}{1.558010in}}{\pgfqpoint{7.951477in}{1.553172in}}{\pgfqpoint{7.955044in}{1.549606in}}%
\pgfpathcurveto{\pgfqpoint{7.958610in}{1.546039in}}{\pgfqpoint{7.963448in}{1.544035in}}{\pgfqpoint{7.968491in}{1.544035in}}%
\pgfpathclose%
\pgfusepath{fill}%
\end{pgfscope}%
\begin{pgfscope}%
\pgfpathrectangle{\pgfqpoint{6.572727in}{0.473000in}}{\pgfqpoint{4.227273in}{3.311000in}}%
\pgfusepath{clip}%
\pgfsetbuttcap%
\pgfsetroundjoin%
\definecolor{currentfill}{rgb}{0.127568,0.566949,0.550556}%
\pgfsetfillcolor{currentfill}%
\pgfsetfillopacity{0.700000}%
\pgfsetlinewidth{0.000000pt}%
\definecolor{currentstroke}{rgb}{0.000000,0.000000,0.000000}%
\pgfsetstrokecolor{currentstroke}%
\pgfsetstrokeopacity{0.700000}%
\pgfsetdash{}{0pt}%
\pgfpathmoveto{\pgfqpoint{7.977755in}{1.372947in}}%
\pgfpathcurveto{\pgfqpoint{7.982799in}{1.372947in}}{\pgfqpoint{7.987637in}{1.374950in}}{\pgfqpoint{7.991203in}{1.378517in}}%
\pgfpathcurveto{\pgfqpoint{7.994769in}{1.382083in}}{\pgfqpoint{7.996773in}{1.386921in}}{\pgfqpoint{7.996773in}{1.391965in}}%
\pgfpathcurveto{\pgfqpoint{7.996773in}{1.397008in}}{\pgfqpoint{7.994769in}{1.401846in}}{\pgfqpoint{7.991203in}{1.405413in}}%
\pgfpathcurveto{\pgfqpoint{7.987637in}{1.408979in}}{\pgfqpoint{7.982799in}{1.410983in}}{\pgfqpoint{7.977755in}{1.410983in}}%
\pgfpathcurveto{\pgfqpoint{7.972711in}{1.410983in}}{\pgfqpoint{7.967874in}{1.408979in}}{\pgfqpoint{7.964307in}{1.405413in}}%
\pgfpathcurveto{\pgfqpoint{7.960741in}{1.401846in}}{\pgfqpoint{7.958737in}{1.397008in}}{\pgfqpoint{7.958737in}{1.391965in}}%
\pgfpathcurveto{\pgfqpoint{7.958737in}{1.386921in}}{\pgfqpoint{7.960741in}{1.382083in}}{\pgfqpoint{7.964307in}{1.378517in}}%
\pgfpathcurveto{\pgfqpoint{7.967874in}{1.374950in}}{\pgfqpoint{7.972711in}{1.372947in}}{\pgfqpoint{7.977755in}{1.372947in}}%
\pgfpathclose%
\pgfusepath{fill}%
\end{pgfscope}%
\begin{pgfscope}%
\pgfpathrectangle{\pgfqpoint{6.572727in}{0.473000in}}{\pgfqpoint{4.227273in}{3.311000in}}%
\pgfusepath{clip}%
\pgfsetbuttcap%
\pgfsetroundjoin%
\definecolor{currentfill}{rgb}{0.127568,0.566949,0.550556}%
\pgfsetfillcolor{currentfill}%
\pgfsetfillopacity{0.700000}%
\pgfsetlinewidth{0.000000pt}%
\definecolor{currentstroke}{rgb}{0.000000,0.000000,0.000000}%
\pgfsetstrokecolor{currentstroke}%
\pgfsetstrokeopacity{0.700000}%
\pgfsetdash{}{0pt}%
\pgfpathmoveto{\pgfqpoint{8.078050in}{2.942296in}}%
\pgfpathcurveto{\pgfqpoint{8.083094in}{2.942296in}}{\pgfqpoint{8.087932in}{2.944300in}}{\pgfqpoint{8.091498in}{2.947867in}}%
\pgfpathcurveto{\pgfqpoint{8.095064in}{2.951433in}}{\pgfqpoint{8.097068in}{2.956271in}}{\pgfqpoint{8.097068in}{2.961314in}}%
\pgfpathcurveto{\pgfqpoint{8.097068in}{2.966358in}}{\pgfqpoint{8.095064in}{2.971196in}}{\pgfqpoint{8.091498in}{2.974762in}}%
\pgfpathcurveto{\pgfqpoint{8.087932in}{2.978329in}}{\pgfqpoint{8.083094in}{2.980333in}}{\pgfqpoint{8.078050in}{2.980333in}}%
\pgfpathcurveto{\pgfqpoint{8.073007in}{2.980333in}}{\pgfqpoint{8.068169in}{2.978329in}}{\pgfqpoint{8.064602in}{2.974762in}}%
\pgfpathcurveto{\pgfqpoint{8.061036in}{2.971196in}}{\pgfqpoint{8.059032in}{2.966358in}}{\pgfqpoint{8.059032in}{2.961314in}}%
\pgfpathcurveto{\pgfqpoint{8.059032in}{2.956271in}}{\pgfqpoint{8.061036in}{2.951433in}}{\pgfqpoint{8.064602in}{2.947867in}}%
\pgfpathcurveto{\pgfqpoint{8.068169in}{2.944300in}}{\pgfqpoint{8.073007in}{2.942296in}}{\pgfqpoint{8.078050in}{2.942296in}}%
\pgfpathclose%
\pgfusepath{fill}%
\end{pgfscope}%
\begin{pgfscope}%
\pgfpathrectangle{\pgfqpoint{6.572727in}{0.473000in}}{\pgfqpoint{4.227273in}{3.311000in}}%
\pgfusepath{clip}%
\pgfsetbuttcap%
\pgfsetroundjoin%
\definecolor{currentfill}{rgb}{0.127568,0.566949,0.550556}%
\pgfsetfillcolor{currentfill}%
\pgfsetfillopacity{0.700000}%
\pgfsetlinewidth{0.000000pt}%
\definecolor{currentstroke}{rgb}{0.000000,0.000000,0.000000}%
\pgfsetstrokecolor{currentstroke}%
\pgfsetstrokeopacity{0.700000}%
\pgfsetdash{}{0pt}%
\pgfpathmoveto{\pgfqpoint{8.105524in}{1.259068in}}%
\pgfpathcurveto{\pgfqpoint{8.110568in}{1.259068in}}{\pgfqpoint{8.115406in}{1.261072in}}{\pgfqpoint{8.118972in}{1.264638in}}%
\pgfpathcurveto{\pgfqpoint{8.122539in}{1.268205in}}{\pgfqpoint{8.124543in}{1.273043in}}{\pgfqpoint{8.124543in}{1.278086in}}%
\pgfpathcurveto{\pgfqpoint{8.124543in}{1.283130in}}{\pgfqpoint{8.122539in}{1.287968in}}{\pgfqpoint{8.118972in}{1.291534in}}%
\pgfpathcurveto{\pgfqpoint{8.115406in}{1.295100in}}{\pgfqpoint{8.110568in}{1.297104in}}{\pgfqpoint{8.105524in}{1.297104in}}%
\pgfpathcurveto{\pgfqpoint{8.100481in}{1.297104in}}{\pgfqpoint{8.095643in}{1.295100in}}{\pgfqpoint{8.092077in}{1.291534in}}%
\pgfpathcurveto{\pgfqpoint{8.088510in}{1.287968in}}{\pgfqpoint{8.086506in}{1.283130in}}{\pgfqpoint{8.086506in}{1.278086in}}%
\pgfpathcurveto{\pgfqpoint{8.086506in}{1.273043in}}{\pgfqpoint{8.088510in}{1.268205in}}{\pgfqpoint{8.092077in}{1.264638in}}%
\pgfpathcurveto{\pgfqpoint{8.095643in}{1.261072in}}{\pgfqpoint{8.100481in}{1.259068in}}{\pgfqpoint{8.105524in}{1.259068in}}%
\pgfpathclose%
\pgfusepath{fill}%
\end{pgfscope}%
\begin{pgfscope}%
\pgfpathrectangle{\pgfqpoint{6.572727in}{0.473000in}}{\pgfqpoint{4.227273in}{3.311000in}}%
\pgfusepath{clip}%
\pgfsetbuttcap%
\pgfsetroundjoin%
\definecolor{currentfill}{rgb}{0.127568,0.566949,0.550556}%
\pgfsetfillcolor{currentfill}%
\pgfsetfillopacity{0.700000}%
\pgfsetlinewidth{0.000000pt}%
\definecolor{currentstroke}{rgb}{0.000000,0.000000,0.000000}%
\pgfsetstrokecolor{currentstroke}%
\pgfsetstrokeopacity{0.700000}%
\pgfsetdash{}{0pt}%
\pgfpathmoveto{\pgfqpoint{7.643309in}{1.382077in}}%
\pgfpathcurveto{\pgfqpoint{7.648352in}{1.382077in}}{\pgfqpoint{7.653190in}{1.384081in}}{\pgfqpoint{7.656757in}{1.387647in}}%
\pgfpathcurveto{\pgfqpoint{7.660323in}{1.391213in}}{\pgfqpoint{7.662327in}{1.396051in}}{\pgfqpoint{7.662327in}{1.401095in}}%
\pgfpathcurveto{\pgfqpoint{7.662327in}{1.406139in}}{\pgfqpoint{7.660323in}{1.410976in}}{\pgfqpoint{7.656757in}{1.414543in}}%
\pgfpathcurveto{\pgfqpoint{7.653190in}{1.418109in}}{\pgfqpoint{7.648352in}{1.420113in}}{\pgfqpoint{7.643309in}{1.420113in}}%
\pgfpathcurveto{\pgfqpoint{7.638265in}{1.420113in}}{\pgfqpoint{7.633427in}{1.418109in}}{\pgfqpoint{7.629861in}{1.414543in}}%
\pgfpathcurveto{\pgfqpoint{7.626294in}{1.410976in}}{\pgfqpoint{7.624291in}{1.406139in}}{\pgfqpoint{7.624291in}{1.401095in}}%
\pgfpathcurveto{\pgfqpoint{7.624291in}{1.396051in}}{\pgfqpoint{7.626294in}{1.391213in}}{\pgfqpoint{7.629861in}{1.387647in}}%
\pgfpathcurveto{\pgfqpoint{7.633427in}{1.384081in}}{\pgfqpoint{7.638265in}{1.382077in}}{\pgfqpoint{7.643309in}{1.382077in}}%
\pgfpathclose%
\pgfusepath{fill}%
\end{pgfscope}%
\begin{pgfscope}%
\pgfpathrectangle{\pgfqpoint{6.572727in}{0.473000in}}{\pgfqpoint{4.227273in}{3.311000in}}%
\pgfusepath{clip}%
\pgfsetbuttcap%
\pgfsetroundjoin%
\definecolor{currentfill}{rgb}{0.127568,0.566949,0.550556}%
\pgfsetfillcolor{currentfill}%
\pgfsetfillopacity{0.700000}%
\pgfsetlinewidth{0.000000pt}%
\definecolor{currentstroke}{rgb}{0.000000,0.000000,0.000000}%
\pgfsetstrokecolor{currentstroke}%
\pgfsetstrokeopacity{0.700000}%
\pgfsetdash{}{0pt}%
\pgfpathmoveto{\pgfqpoint{7.772328in}{2.349290in}}%
\pgfpathcurveto{\pgfqpoint{7.777372in}{2.349290in}}{\pgfqpoint{7.782210in}{2.351294in}}{\pgfqpoint{7.785776in}{2.354860in}}%
\pgfpathcurveto{\pgfqpoint{7.789343in}{2.358427in}}{\pgfqpoint{7.791347in}{2.363264in}}{\pgfqpoint{7.791347in}{2.368308in}}%
\pgfpathcurveto{\pgfqpoint{7.791347in}{2.373352in}}{\pgfqpoint{7.789343in}{2.378190in}}{\pgfqpoint{7.785776in}{2.381756in}}%
\pgfpathcurveto{\pgfqpoint{7.782210in}{2.385322in}}{\pgfqpoint{7.777372in}{2.387326in}}{\pgfqpoint{7.772328in}{2.387326in}}%
\pgfpathcurveto{\pgfqpoint{7.767285in}{2.387326in}}{\pgfqpoint{7.762447in}{2.385322in}}{\pgfqpoint{7.758881in}{2.381756in}}%
\pgfpathcurveto{\pgfqpoint{7.755314in}{2.378190in}}{\pgfqpoint{7.753310in}{2.373352in}}{\pgfqpoint{7.753310in}{2.368308in}}%
\pgfpathcurveto{\pgfqpoint{7.753310in}{2.363264in}}{\pgfqpoint{7.755314in}{2.358427in}}{\pgfqpoint{7.758881in}{2.354860in}}%
\pgfpathcurveto{\pgfqpoint{7.762447in}{2.351294in}}{\pgfqpoint{7.767285in}{2.349290in}}{\pgfqpoint{7.772328in}{2.349290in}}%
\pgfpathclose%
\pgfusepath{fill}%
\end{pgfscope}%
\begin{pgfscope}%
\pgfpathrectangle{\pgfqpoint{6.572727in}{0.473000in}}{\pgfqpoint{4.227273in}{3.311000in}}%
\pgfusepath{clip}%
\pgfsetbuttcap%
\pgfsetroundjoin%
\definecolor{currentfill}{rgb}{0.127568,0.566949,0.550556}%
\pgfsetfillcolor{currentfill}%
\pgfsetfillopacity{0.700000}%
\pgfsetlinewidth{0.000000pt}%
\definecolor{currentstroke}{rgb}{0.000000,0.000000,0.000000}%
\pgfsetstrokecolor{currentstroke}%
\pgfsetstrokeopacity{0.700000}%
\pgfsetdash{}{0pt}%
\pgfpathmoveto{\pgfqpoint{8.983886in}{2.792462in}}%
\pgfpathcurveto{\pgfqpoint{8.988930in}{2.792462in}}{\pgfqpoint{8.993767in}{2.794466in}}{\pgfqpoint{8.997334in}{2.798032in}}%
\pgfpathcurveto{\pgfqpoint{9.000900in}{2.801599in}}{\pgfqpoint{9.002904in}{2.806437in}}{\pgfqpoint{9.002904in}{2.811480in}}%
\pgfpathcurveto{\pgfqpoint{9.002904in}{2.816524in}}{\pgfqpoint{9.000900in}{2.821362in}}{\pgfqpoint{8.997334in}{2.824928in}}%
\pgfpathcurveto{\pgfqpoint{8.993767in}{2.828495in}}{\pgfqpoint{8.988930in}{2.830498in}}{\pgfqpoint{8.983886in}{2.830498in}}%
\pgfpathcurveto{\pgfqpoint{8.978842in}{2.830498in}}{\pgfqpoint{8.974005in}{2.828495in}}{\pgfqpoint{8.970438in}{2.824928in}}%
\pgfpathcurveto{\pgfqpoint{8.966872in}{2.821362in}}{\pgfqpoint{8.964868in}{2.816524in}}{\pgfqpoint{8.964868in}{2.811480in}}%
\pgfpathcurveto{\pgfqpoint{8.964868in}{2.806437in}}{\pgfqpoint{8.966872in}{2.801599in}}{\pgfqpoint{8.970438in}{2.798032in}}%
\pgfpathcurveto{\pgfqpoint{8.974005in}{2.794466in}}{\pgfqpoint{8.978842in}{2.792462in}}{\pgfqpoint{8.983886in}{2.792462in}}%
\pgfpathclose%
\pgfusepath{fill}%
\end{pgfscope}%
\begin{pgfscope}%
\pgfpathrectangle{\pgfqpoint{6.572727in}{0.473000in}}{\pgfqpoint{4.227273in}{3.311000in}}%
\pgfusepath{clip}%
\pgfsetbuttcap%
\pgfsetroundjoin%
\definecolor{currentfill}{rgb}{0.127568,0.566949,0.550556}%
\pgfsetfillcolor{currentfill}%
\pgfsetfillopacity{0.700000}%
\pgfsetlinewidth{0.000000pt}%
\definecolor{currentstroke}{rgb}{0.000000,0.000000,0.000000}%
\pgfsetstrokecolor{currentstroke}%
\pgfsetstrokeopacity{0.700000}%
\pgfsetdash{}{0pt}%
\pgfpathmoveto{\pgfqpoint{8.129588in}{1.541057in}}%
\pgfpathcurveto{\pgfqpoint{8.134631in}{1.541057in}}{\pgfqpoint{8.139469in}{1.543061in}}{\pgfqpoint{8.143036in}{1.546627in}}%
\pgfpathcurveto{\pgfqpoint{8.146602in}{1.550194in}}{\pgfqpoint{8.148606in}{1.555031in}}{\pgfqpoint{8.148606in}{1.560075in}}%
\pgfpathcurveto{\pgfqpoint{8.148606in}{1.565119in}}{\pgfqpoint{8.146602in}{1.569956in}}{\pgfqpoint{8.143036in}{1.573523in}}%
\pgfpathcurveto{\pgfqpoint{8.139469in}{1.577089in}}{\pgfqpoint{8.134631in}{1.579093in}}{\pgfqpoint{8.129588in}{1.579093in}}%
\pgfpathcurveto{\pgfqpoint{8.124544in}{1.579093in}}{\pgfqpoint{8.119706in}{1.577089in}}{\pgfqpoint{8.116140in}{1.573523in}}%
\pgfpathcurveto{\pgfqpoint{8.112574in}{1.569956in}}{\pgfqpoint{8.110570in}{1.565119in}}{\pgfqpoint{8.110570in}{1.560075in}}%
\pgfpathcurveto{\pgfqpoint{8.110570in}{1.555031in}}{\pgfqpoint{8.112574in}{1.550194in}}{\pgfqpoint{8.116140in}{1.546627in}}%
\pgfpathcurveto{\pgfqpoint{8.119706in}{1.543061in}}{\pgfqpoint{8.124544in}{1.541057in}}{\pgfqpoint{8.129588in}{1.541057in}}%
\pgfpathclose%
\pgfusepath{fill}%
\end{pgfscope}%
\begin{pgfscope}%
\pgfpathrectangle{\pgfqpoint{6.572727in}{0.473000in}}{\pgfqpoint{4.227273in}{3.311000in}}%
\pgfusepath{clip}%
\pgfsetbuttcap%
\pgfsetroundjoin%
\definecolor{currentfill}{rgb}{0.127568,0.566949,0.550556}%
\pgfsetfillcolor{currentfill}%
\pgfsetfillopacity{0.700000}%
\pgfsetlinewidth{0.000000pt}%
\definecolor{currentstroke}{rgb}{0.000000,0.000000,0.000000}%
\pgfsetstrokecolor{currentstroke}%
\pgfsetstrokeopacity{0.700000}%
\pgfsetdash{}{0pt}%
\pgfpathmoveto{\pgfqpoint{8.301748in}{2.933898in}}%
\pgfpathcurveto{\pgfqpoint{8.306792in}{2.933898in}}{\pgfqpoint{8.311630in}{2.935902in}}{\pgfqpoint{8.315196in}{2.939469in}}%
\pgfpathcurveto{\pgfqpoint{8.318762in}{2.943035in}}{\pgfqpoint{8.320766in}{2.947873in}}{\pgfqpoint{8.320766in}{2.952916in}}%
\pgfpathcurveto{\pgfqpoint{8.320766in}{2.957960in}}{\pgfqpoint{8.318762in}{2.962798in}}{\pgfqpoint{8.315196in}{2.966364in}}%
\pgfpathcurveto{\pgfqpoint{8.311630in}{2.969931in}}{\pgfqpoint{8.306792in}{2.971935in}}{\pgfqpoint{8.301748in}{2.971935in}}%
\pgfpathcurveto{\pgfqpoint{8.296704in}{2.971935in}}{\pgfqpoint{8.291867in}{2.969931in}}{\pgfqpoint{8.288300in}{2.966364in}}%
\pgfpathcurveto{\pgfqpoint{8.284734in}{2.962798in}}{\pgfqpoint{8.282730in}{2.957960in}}{\pgfqpoint{8.282730in}{2.952916in}}%
\pgfpathcurveto{\pgfqpoint{8.282730in}{2.947873in}}{\pgfqpoint{8.284734in}{2.943035in}}{\pgfqpoint{8.288300in}{2.939469in}}%
\pgfpathcurveto{\pgfqpoint{8.291867in}{2.935902in}}{\pgfqpoint{8.296704in}{2.933898in}}{\pgfqpoint{8.301748in}{2.933898in}}%
\pgfpathclose%
\pgfusepath{fill}%
\end{pgfscope}%
\begin{pgfscope}%
\pgfpathrectangle{\pgfqpoint{6.572727in}{0.473000in}}{\pgfqpoint{4.227273in}{3.311000in}}%
\pgfusepath{clip}%
\pgfsetbuttcap%
\pgfsetroundjoin%
\definecolor{currentfill}{rgb}{0.993248,0.906157,0.143936}%
\pgfsetfillcolor{currentfill}%
\pgfsetfillopacity{0.700000}%
\pgfsetlinewidth{0.000000pt}%
\definecolor{currentstroke}{rgb}{0.000000,0.000000,0.000000}%
\pgfsetstrokecolor{currentstroke}%
\pgfsetstrokeopacity{0.700000}%
\pgfsetdash{}{0pt}%
\pgfpathmoveto{\pgfqpoint{9.998091in}{1.789896in}}%
\pgfpathcurveto{\pgfqpoint{10.003135in}{1.789896in}}{\pgfqpoint{10.007973in}{1.791900in}}{\pgfqpoint{10.011539in}{1.795466in}}%
\pgfpathcurveto{\pgfqpoint{10.015106in}{1.799033in}}{\pgfqpoint{10.017110in}{1.803871in}}{\pgfqpoint{10.017110in}{1.808914in}}%
\pgfpathcurveto{\pgfqpoint{10.017110in}{1.813958in}}{\pgfqpoint{10.015106in}{1.818796in}}{\pgfqpoint{10.011539in}{1.822362in}}%
\pgfpathcurveto{\pgfqpoint{10.007973in}{1.825929in}}{\pgfqpoint{10.003135in}{1.827932in}}{\pgfqpoint{9.998091in}{1.827932in}}%
\pgfpathcurveto{\pgfqpoint{9.993048in}{1.827932in}}{\pgfqpoint{9.988210in}{1.825929in}}{\pgfqpoint{9.984644in}{1.822362in}}%
\pgfpathcurveto{\pgfqpoint{9.981077in}{1.818796in}}{\pgfqpoint{9.979073in}{1.813958in}}{\pgfqpoint{9.979073in}{1.808914in}}%
\pgfpathcurveto{\pgfqpoint{9.979073in}{1.803871in}}{\pgfqpoint{9.981077in}{1.799033in}}{\pgfqpoint{9.984644in}{1.795466in}}%
\pgfpathcurveto{\pgfqpoint{9.988210in}{1.791900in}}{\pgfqpoint{9.993048in}{1.789896in}}{\pgfqpoint{9.998091in}{1.789896in}}%
\pgfpathclose%
\pgfusepath{fill}%
\end{pgfscope}%
\begin{pgfscope}%
\pgfpathrectangle{\pgfqpoint{6.572727in}{0.473000in}}{\pgfqpoint{4.227273in}{3.311000in}}%
\pgfusepath{clip}%
\pgfsetbuttcap%
\pgfsetroundjoin%
\definecolor{currentfill}{rgb}{0.993248,0.906157,0.143936}%
\pgfsetfillcolor{currentfill}%
\pgfsetfillopacity{0.700000}%
\pgfsetlinewidth{0.000000pt}%
\definecolor{currentstroke}{rgb}{0.000000,0.000000,0.000000}%
\pgfsetstrokecolor{currentstroke}%
\pgfsetstrokeopacity{0.700000}%
\pgfsetdash{}{0pt}%
\pgfpathmoveto{\pgfqpoint{9.226196in}{1.514583in}}%
\pgfpathcurveto{\pgfqpoint{9.231239in}{1.514583in}}{\pgfqpoint{9.236077in}{1.516587in}}{\pgfqpoint{9.239644in}{1.520154in}}%
\pgfpathcurveto{\pgfqpoint{9.243210in}{1.523720in}}{\pgfqpoint{9.245214in}{1.528558in}}{\pgfqpoint{9.245214in}{1.533601in}}%
\pgfpathcurveto{\pgfqpoint{9.245214in}{1.538645in}}{\pgfqpoint{9.243210in}{1.543483in}}{\pgfqpoint{9.239644in}{1.547049in}}%
\pgfpathcurveto{\pgfqpoint{9.236077in}{1.550616in}}{\pgfqpoint{9.231239in}{1.552620in}}{\pgfqpoint{9.226196in}{1.552620in}}%
\pgfpathcurveto{\pgfqpoint{9.221152in}{1.552620in}}{\pgfqpoint{9.216314in}{1.550616in}}{\pgfqpoint{9.212748in}{1.547049in}}%
\pgfpathcurveto{\pgfqpoint{9.209181in}{1.543483in}}{\pgfqpoint{9.207178in}{1.538645in}}{\pgfqpoint{9.207178in}{1.533601in}}%
\pgfpathcurveto{\pgfqpoint{9.207178in}{1.528558in}}{\pgfqpoint{9.209181in}{1.523720in}}{\pgfqpoint{9.212748in}{1.520154in}}%
\pgfpathcurveto{\pgfqpoint{9.216314in}{1.516587in}}{\pgfqpoint{9.221152in}{1.514583in}}{\pgfqpoint{9.226196in}{1.514583in}}%
\pgfpathclose%
\pgfusepath{fill}%
\end{pgfscope}%
\begin{pgfscope}%
\pgfpathrectangle{\pgfqpoint{6.572727in}{0.473000in}}{\pgfqpoint{4.227273in}{3.311000in}}%
\pgfusepath{clip}%
\pgfsetbuttcap%
\pgfsetroundjoin%
\definecolor{currentfill}{rgb}{0.127568,0.566949,0.550556}%
\pgfsetfillcolor{currentfill}%
\pgfsetfillopacity{0.700000}%
\pgfsetlinewidth{0.000000pt}%
\definecolor{currentstroke}{rgb}{0.000000,0.000000,0.000000}%
\pgfsetstrokecolor{currentstroke}%
\pgfsetstrokeopacity{0.700000}%
\pgfsetdash{}{0pt}%
\pgfpathmoveto{\pgfqpoint{7.568149in}{1.928979in}}%
\pgfpathcurveto{\pgfqpoint{7.573193in}{1.928979in}}{\pgfqpoint{7.578031in}{1.930983in}}{\pgfqpoint{7.581597in}{1.934549in}}%
\pgfpathcurveto{\pgfqpoint{7.585164in}{1.938116in}}{\pgfqpoint{7.587168in}{1.942953in}}{\pgfqpoint{7.587168in}{1.947997in}}%
\pgfpathcurveto{\pgfqpoint{7.587168in}{1.953041in}}{\pgfqpoint{7.585164in}{1.957878in}}{\pgfqpoint{7.581597in}{1.961445in}}%
\pgfpathcurveto{\pgfqpoint{7.578031in}{1.965011in}}{\pgfqpoint{7.573193in}{1.967015in}}{\pgfqpoint{7.568149in}{1.967015in}}%
\pgfpathcurveto{\pgfqpoint{7.563106in}{1.967015in}}{\pgfqpoint{7.558268in}{1.965011in}}{\pgfqpoint{7.554702in}{1.961445in}}%
\pgfpathcurveto{\pgfqpoint{7.551135in}{1.957878in}}{\pgfqpoint{7.549131in}{1.953041in}}{\pgfqpoint{7.549131in}{1.947997in}}%
\pgfpathcurveto{\pgfqpoint{7.549131in}{1.942953in}}{\pgfqpoint{7.551135in}{1.938116in}}{\pgfqpoint{7.554702in}{1.934549in}}%
\pgfpathcurveto{\pgfqpoint{7.558268in}{1.930983in}}{\pgfqpoint{7.563106in}{1.928979in}}{\pgfqpoint{7.568149in}{1.928979in}}%
\pgfpathclose%
\pgfusepath{fill}%
\end{pgfscope}%
\begin{pgfscope}%
\pgfpathrectangle{\pgfqpoint{6.572727in}{0.473000in}}{\pgfqpoint{4.227273in}{3.311000in}}%
\pgfusepath{clip}%
\pgfsetbuttcap%
\pgfsetroundjoin%
\definecolor{currentfill}{rgb}{0.127568,0.566949,0.550556}%
\pgfsetfillcolor{currentfill}%
\pgfsetfillopacity{0.700000}%
\pgfsetlinewidth{0.000000pt}%
\definecolor{currentstroke}{rgb}{0.000000,0.000000,0.000000}%
\pgfsetstrokecolor{currentstroke}%
\pgfsetstrokeopacity{0.700000}%
\pgfsetdash{}{0pt}%
\pgfpathmoveto{\pgfqpoint{7.540108in}{2.115495in}}%
\pgfpathcurveto{\pgfqpoint{7.545151in}{2.115495in}}{\pgfqpoint{7.549989in}{2.117498in}}{\pgfqpoint{7.553556in}{2.121065in}}%
\pgfpathcurveto{\pgfqpoint{7.557122in}{2.124631in}}{\pgfqpoint{7.559126in}{2.129469in}}{\pgfqpoint{7.559126in}{2.134513in}}%
\pgfpathcurveto{\pgfqpoint{7.559126in}{2.139556in}}{\pgfqpoint{7.557122in}{2.144394in}}{\pgfqpoint{7.553556in}{2.147961in}}%
\pgfpathcurveto{\pgfqpoint{7.549989in}{2.151527in}}{\pgfqpoint{7.545151in}{2.153531in}}{\pgfqpoint{7.540108in}{2.153531in}}%
\pgfpathcurveto{\pgfqpoint{7.535064in}{2.153531in}}{\pgfqpoint{7.530226in}{2.151527in}}{\pgfqpoint{7.526660in}{2.147961in}}%
\pgfpathcurveto{\pgfqpoint{7.523093in}{2.144394in}}{\pgfqpoint{7.521090in}{2.139556in}}{\pgfqpoint{7.521090in}{2.134513in}}%
\pgfpathcurveto{\pgfqpoint{7.521090in}{2.129469in}}{\pgfqpoint{7.523093in}{2.124631in}}{\pgfqpoint{7.526660in}{2.121065in}}%
\pgfpathcurveto{\pgfqpoint{7.530226in}{2.117498in}}{\pgfqpoint{7.535064in}{2.115495in}}{\pgfqpoint{7.540108in}{2.115495in}}%
\pgfpathclose%
\pgfusepath{fill}%
\end{pgfscope}%
\begin{pgfscope}%
\pgfpathrectangle{\pgfqpoint{6.572727in}{0.473000in}}{\pgfqpoint{4.227273in}{3.311000in}}%
\pgfusepath{clip}%
\pgfsetbuttcap%
\pgfsetroundjoin%
\definecolor{currentfill}{rgb}{0.993248,0.906157,0.143936}%
\pgfsetfillcolor{currentfill}%
\pgfsetfillopacity{0.700000}%
\pgfsetlinewidth{0.000000pt}%
\definecolor{currentstroke}{rgb}{0.000000,0.000000,0.000000}%
\pgfsetstrokecolor{currentstroke}%
\pgfsetstrokeopacity{0.700000}%
\pgfsetdash{}{0pt}%
\pgfpathmoveto{\pgfqpoint{9.661492in}{1.522133in}}%
\pgfpathcurveto{\pgfqpoint{9.666535in}{1.522133in}}{\pgfqpoint{9.671373in}{1.524137in}}{\pgfqpoint{9.674939in}{1.527703in}}%
\pgfpathcurveto{\pgfqpoint{9.678506in}{1.531270in}}{\pgfqpoint{9.680510in}{1.536107in}}{\pgfqpoint{9.680510in}{1.541151in}}%
\pgfpathcurveto{\pgfqpoint{9.680510in}{1.546195in}}{\pgfqpoint{9.678506in}{1.551033in}}{\pgfqpoint{9.674939in}{1.554599in}}%
\pgfpathcurveto{\pgfqpoint{9.671373in}{1.558165in}}{\pgfqpoint{9.666535in}{1.560169in}}{\pgfqpoint{9.661492in}{1.560169in}}%
\pgfpathcurveto{\pgfqpoint{9.656448in}{1.560169in}}{\pgfqpoint{9.651610in}{1.558165in}}{\pgfqpoint{9.648044in}{1.554599in}}%
\pgfpathcurveto{\pgfqpoint{9.644477in}{1.551033in}}{\pgfqpoint{9.642473in}{1.546195in}}{\pgfqpoint{9.642473in}{1.541151in}}%
\pgfpathcurveto{\pgfqpoint{9.642473in}{1.536107in}}{\pgfqpoint{9.644477in}{1.531270in}}{\pgfqpoint{9.648044in}{1.527703in}}%
\pgfpathcurveto{\pgfqpoint{9.651610in}{1.524137in}}{\pgfqpoint{9.656448in}{1.522133in}}{\pgfqpoint{9.661492in}{1.522133in}}%
\pgfpathclose%
\pgfusepath{fill}%
\end{pgfscope}%
\begin{pgfscope}%
\pgfpathrectangle{\pgfqpoint{6.572727in}{0.473000in}}{\pgfqpoint{4.227273in}{3.311000in}}%
\pgfusepath{clip}%
\pgfsetbuttcap%
\pgfsetroundjoin%
\definecolor{currentfill}{rgb}{0.993248,0.906157,0.143936}%
\pgfsetfillcolor{currentfill}%
\pgfsetfillopacity{0.700000}%
\pgfsetlinewidth{0.000000pt}%
\definecolor{currentstroke}{rgb}{0.000000,0.000000,0.000000}%
\pgfsetstrokecolor{currentstroke}%
\pgfsetstrokeopacity{0.700000}%
\pgfsetdash{}{0pt}%
\pgfpathmoveto{\pgfqpoint{9.978806in}{1.308453in}}%
\pgfpathcurveto{\pgfqpoint{9.983850in}{1.308453in}}{\pgfqpoint{9.988688in}{1.310457in}}{\pgfqpoint{9.992254in}{1.314023in}}%
\pgfpathcurveto{\pgfqpoint{9.995821in}{1.317590in}}{\pgfqpoint{9.997825in}{1.322427in}}{\pgfqpoint{9.997825in}{1.327471in}}%
\pgfpathcurveto{\pgfqpoint{9.997825in}{1.332515in}}{\pgfqpoint{9.995821in}{1.337353in}}{\pgfqpoint{9.992254in}{1.340919in}}%
\pgfpathcurveto{\pgfqpoint{9.988688in}{1.344485in}}{\pgfqpoint{9.983850in}{1.346489in}}{\pgfqpoint{9.978806in}{1.346489in}}%
\pgfpathcurveto{\pgfqpoint{9.973763in}{1.346489in}}{\pgfqpoint{9.968925in}{1.344485in}}{\pgfqpoint{9.965359in}{1.340919in}}%
\pgfpathcurveto{\pgfqpoint{9.961792in}{1.337353in}}{\pgfqpoint{9.959788in}{1.332515in}}{\pgfqpoint{9.959788in}{1.327471in}}%
\pgfpathcurveto{\pgfqpoint{9.959788in}{1.322427in}}{\pgfqpoint{9.961792in}{1.317590in}}{\pgfqpoint{9.965359in}{1.314023in}}%
\pgfpathcurveto{\pgfqpoint{9.968925in}{1.310457in}}{\pgfqpoint{9.973763in}{1.308453in}}{\pgfqpoint{9.978806in}{1.308453in}}%
\pgfpathclose%
\pgfusepath{fill}%
\end{pgfscope}%
\begin{pgfscope}%
\pgfpathrectangle{\pgfqpoint{6.572727in}{0.473000in}}{\pgfqpoint{4.227273in}{3.311000in}}%
\pgfusepath{clip}%
\pgfsetbuttcap%
\pgfsetroundjoin%
\definecolor{currentfill}{rgb}{0.127568,0.566949,0.550556}%
\pgfsetfillcolor{currentfill}%
\pgfsetfillopacity{0.700000}%
\pgfsetlinewidth{0.000000pt}%
\definecolor{currentstroke}{rgb}{0.000000,0.000000,0.000000}%
\pgfsetstrokecolor{currentstroke}%
\pgfsetstrokeopacity{0.700000}%
\pgfsetdash{}{0pt}%
\pgfpathmoveto{\pgfqpoint{8.092817in}{2.557317in}}%
\pgfpathcurveto{\pgfqpoint{8.097861in}{2.557317in}}{\pgfqpoint{8.102698in}{2.559320in}}{\pgfqpoint{8.106265in}{2.562887in}}%
\pgfpathcurveto{\pgfqpoint{8.109831in}{2.566453in}}{\pgfqpoint{8.111835in}{2.571291in}}{\pgfqpoint{8.111835in}{2.576335in}}%
\pgfpathcurveto{\pgfqpoint{8.111835in}{2.581378in}}{\pgfqpoint{8.109831in}{2.586216in}}{\pgfqpoint{8.106265in}{2.589783in}}%
\pgfpathcurveto{\pgfqpoint{8.102698in}{2.593349in}}{\pgfqpoint{8.097861in}{2.595353in}}{\pgfqpoint{8.092817in}{2.595353in}}%
\pgfpathcurveto{\pgfqpoint{8.087773in}{2.595353in}}{\pgfqpoint{8.082936in}{2.593349in}}{\pgfqpoint{8.079369in}{2.589783in}}%
\pgfpathcurveto{\pgfqpoint{8.075803in}{2.586216in}}{\pgfqpoint{8.073799in}{2.581378in}}{\pgfqpoint{8.073799in}{2.576335in}}%
\pgfpathcurveto{\pgfqpoint{8.073799in}{2.571291in}}{\pgfqpoint{8.075803in}{2.566453in}}{\pgfqpoint{8.079369in}{2.562887in}}%
\pgfpathcurveto{\pgfqpoint{8.082936in}{2.559320in}}{\pgfqpoint{8.087773in}{2.557317in}}{\pgfqpoint{8.092817in}{2.557317in}}%
\pgfpathclose%
\pgfusepath{fill}%
\end{pgfscope}%
\begin{pgfscope}%
\pgfpathrectangle{\pgfqpoint{6.572727in}{0.473000in}}{\pgfqpoint{4.227273in}{3.311000in}}%
\pgfusepath{clip}%
\pgfsetbuttcap%
\pgfsetroundjoin%
\definecolor{currentfill}{rgb}{0.127568,0.566949,0.550556}%
\pgfsetfillcolor{currentfill}%
\pgfsetfillopacity{0.700000}%
\pgfsetlinewidth{0.000000pt}%
\definecolor{currentstroke}{rgb}{0.000000,0.000000,0.000000}%
\pgfsetstrokecolor{currentstroke}%
\pgfsetstrokeopacity{0.700000}%
\pgfsetdash{}{0pt}%
\pgfpathmoveto{\pgfqpoint{7.165098in}{1.515172in}}%
\pgfpathcurveto{\pgfqpoint{7.170142in}{1.515172in}}{\pgfqpoint{7.174980in}{1.517176in}}{\pgfqpoint{7.178546in}{1.520742in}}%
\pgfpathcurveto{\pgfqpoint{7.182113in}{1.524309in}}{\pgfqpoint{7.184116in}{1.529146in}}{\pgfqpoint{7.184116in}{1.534190in}}%
\pgfpathcurveto{\pgfqpoint{7.184116in}{1.539234in}}{\pgfqpoint{7.182113in}{1.544072in}}{\pgfqpoint{7.178546in}{1.547638in}}%
\pgfpathcurveto{\pgfqpoint{7.174980in}{1.551204in}}{\pgfqpoint{7.170142in}{1.553208in}}{\pgfqpoint{7.165098in}{1.553208in}}%
\pgfpathcurveto{\pgfqpoint{7.160055in}{1.553208in}}{\pgfqpoint{7.155217in}{1.551204in}}{\pgfqpoint{7.151650in}{1.547638in}}%
\pgfpathcurveto{\pgfqpoint{7.148084in}{1.544072in}}{\pgfqpoint{7.146080in}{1.539234in}}{\pgfqpoint{7.146080in}{1.534190in}}%
\pgfpathcurveto{\pgfqpoint{7.146080in}{1.529146in}}{\pgfqpoint{7.148084in}{1.524309in}}{\pgfqpoint{7.151650in}{1.520742in}}%
\pgfpathcurveto{\pgfqpoint{7.155217in}{1.517176in}}{\pgfqpoint{7.160055in}{1.515172in}}{\pgfqpoint{7.165098in}{1.515172in}}%
\pgfpathclose%
\pgfusepath{fill}%
\end{pgfscope}%
\begin{pgfscope}%
\pgfpathrectangle{\pgfqpoint{6.572727in}{0.473000in}}{\pgfqpoint{4.227273in}{3.311000in}}%
\pgfusepath{clip}%
\pgfsetbuttcap%
\pgfsetroundjoin%
\definecolor{currentfill}{rgb}{0.127568,0.566949,0.550556}%
\pgfsetfillcolor{currentfill}%
\pgfsetfillopacity{0.700000}%
\pgfsetlinewidth{0.000000pt}%
\definecolor{currentstroke}{rgb}{0.000000,0.000000,0.000000}%
\pgfsetstrokecolor{currentstroke}%
\pgfsetstrokeopacity{0.700000}%
\pgfsetdash{}{0pt}%
\pgfpathmoveto{\pgfqpoint{7.917579in}{1.504432in}}%
\pgfpathcurveto{\pgfqpoint{7.922623in}{1.504432in}}{\pgfqpoint{7.927461in}{1.506436in}}{\pgfqpoint{7.931027in}{1.510003in}}%
\pgfpathcurveto{\pgfqpoint{7.934593in}{1.513569in}}{\pgfqpoint{7.936597in}{1.518407in}}{\pgfqpoint{7.936597in}{1.523451in}}%
\pgfpathcurveto{\pgfqpoint{7.936597in}{1.528494in}}{\pgfqpoint{7.934593in}{1.533332in}}{\pgfqpoint{7.931027in}{1.536898in}}%
\pgfpathcurveto{\pgfqpoint{7.927461in}{1.540465in}}{\pgfqpoint{7.922623in}{1.542469in}}{\pgfqpoint{7.917579in}{1.542469in}}%
\pgfpathcurveto{\pgfqpoint{7.912536in}{1.542469in}}{\pgfqpoint{7.907698in}{1.540465in}}{\pgfqpoint{7.904131in}{1.536898in}}%
\pgfpathcurveto{\pgfqpoint{7.900565in}{1.533332in}}{\pgfqpoint{7.898561in}{1.528494in}}{\pgfqpoint{7.898561in}{1.523451in}}%
\pgfpathcurveto{\pgfqpoint{7.898561in}{1.518407in}}{\pgfqpoint{7.900565in}{1.513569in}}{\pgfqpoint{7.904131in}{1.510003in}}%
\pgfpathcurveto{\pgfqpoint{7.907698in}{1.506436in}}{\pgfqpoint{7.912536in}{1.504432in}}{\pgfqpoint{7.917579in}{1.504432in}}%
\pgfpathclose%
\pgfusepath{fill}%
\end{pgfscope}%
\begin{pgfscope}%
\pgfpathrectangle{\pgfqpoint{6.572727in}{0.473000in}}{\pgfqpoint{4.227273in}{3.311000in}}%
\pgfusepath{clip}%
\pgfsetbuttcap%
\pgfsetroundjoin%
\definecolor{currentfill}{rgb}{0.127568,0.566949,0.550556}%
\pgfsetfillcolor{currentfill}%
\pgfsetfillopacity{0.700000}%
\pgfsetlinewidth{0.000000pt}%
\definecolor{currentstroke}{rgb}{0.000000,0.000000,0.000000}%
\pgfsetstrokecolor{currentstroke}%
\pgfsetstrokeopacity{0.700000}%
\pgfsetdash{}{0pt}%
\pgfpathmoveto{\pgfqpoint{7.869032in}{1.212642in}}%
\pgfpathcurveto{\pgfqpoint{7.874076in}{1.212642in}}{\pgfqpoint{7.878914in}{1.214646in}}{\pgfqpoint{7.882480in}{1.218213in}}%
\pgfpathcurveto{\pgfqpoint{7.886046in}{1.221779in}}{\pgfqpoint{7.888050in}{1.226617in}}{\pgfqpoint{7.888050in}{1.231661in}}%
\pgfpathcurveto{\pgfqpoint{7.888050in}{1.236704in}}{\pgfqpoint{7.886046in}{1.241542in}}{\pgfqpoint{7.882480in}{1.245108in}}%
\pgfpathcurveto{\pgfqpoint{7.878914in}{1.248675in}}{\pgfqpoint{7.874076in}{1.250679in}}{\pgfqpoint{7.869032in}{1.250679in}}%
\pgfpathcurveto{\pgfqpoint{7.863988in}{1.250679in}}{\pgfqpoint{7.859151in}{1.248675in}}{\pgfqpoint{7.855584in}{1.245108in}}%
\pgfpathcurveto{\pgfqpoint{7.852018in}{1.241542in}}{\pgfqpoint{7.850014in}{1.236704in}}{\pgfqpoint{7.850014in}{1.231661in}}%
\pgfpathcurveto{\pgfqpoint{7.850014in}{1.226617in}}{\pgfqpoint{7.852018in}{1.221779in}}{\pgfqpoint{7.855584in}{1.218213in}}%
\pgfpathcurveto{\pgfqpoint{7.859151in}{1.214646in}}{\pgfqpoint{7.863988in}{1.212642in}}{\pgfqpoint{7.869032in}{1.212642in}}%
\pgfpathclose%
\pgfusepath{fill}%
\end{pgfscope}%
\begin{pgfscope}%
\pgfpathrectangle{\pgfqpoint{6.572727in}{0.473000in}}{\pgfqpoint{4.227273in}{3.311000in}}%
\pgfusepath{clip}%
\pgfsetbuttcap%
\pgfsetroundjoin%
\definecolor{currentfill}{rgb}{0.993248,0.906157,0.143936}%
\pgfsetfillcolor{currentfill}%
\pgfsetfillopacity{0.700000}%
\pgfsetlinewidth{0.000000pt}%
\definecolor{currentstroke}{rgb}{0.000000,0.000000,0.000000}%
\pgfsetstrokecolor{currentstroke}%
\pgfsetstrokeopacity{0.700000}%
\pgfsetdash{}{0pt}%
\pgfpathmoveto{\pgfqpoint{9.679658in}{1.901261in}}%
\pgfpathcurveto{\pgfqpoint{9.684702in}{1.901261in}}{\pgfqpoint{9.689540in}{1.903265in}}{\pgfqpoint{9.693106in}{1.906831in}}%
\pgfpathcurveto{\pgfqpoint{9.696672in}{1.910397in}}{\pgfqpoint{9.698676in}{1.915235in}}{\pgfqpoint{9.698676in}{1.920279in}}%
\pgfpathcurveto{\pgfqpoint{9.698676in}{1.925323in}}{\pgfqpoint{9.696672in}{1.930160in}}{\pgfqpoint{9.693106in}{1.933727in}}%
\pgfpathcurveto{\pgfqpoint{9.689540in}{1.937293in}}{\pgfqpoint{9.684702in}{1.939297in}}{\pgfqpoint{9.679658in}{1.939297in}}%
\pgfpathcurveto{\pgfqpoint{9.674614in}{1.939297in}}{\pgfqpoint{9.669777in}{1.937293in}}{\pgfqpoint{9.666210in}{1.933727in}}%
\pgfpathcurveto{\pgfqpoint{9.662644in}{1.930160in}}{\pgfqpoint{9.660640in}{1.925323in}}{\pgfqpoint{9.660640in}{1.920279in}}%
\pgfpathcurveto{\pgfqpoint{9.660640in}{1.915235in}}{\pgfqpoint{9.662644in}{1.910397in}}{\pgfqpoint{9.666210in}{1.906831in}}%
\pgfpathcurveto{\pgfqpoint{9.669777in}{1.903265in}}{\pgfqpoint{9.674614in}{1.901261in}}{\pgfqpoint{9.679658in}{1.901261in}}%
\pgfpathclose%
\pgfusepath{fill}%
\end{pgfscope}%
\begin{pgfscope}%
\pgfpathrectangle{\pgfqpoint{6.572727in}{0.473000in}}{\pgfqpoint{4.227273in}{3.311000in}}%
\pgfusepath{clip}%
\pgfsetbuttcap%
\pgfsetroundjoin%
\definecolor{currentfill}{rgb}{0.127568,0.566949,0.550556}%
\pgfsetfillcolor{currentfill}%
\pgfsetfillopacity{0.700000}%
\pgfsetlinewidth{0.000000pt}%
\definecolor{currentstroke}{rgb}{0.000000,0.000000,0.000000}%
\pgfsetstrokecolor{currentstroke}%
\pgfsetstrokeopacity{0.700000}%
\pgfsetdash{}{0pt}%
\pgfpathmoveto{\pgfqpoint{7.533285in}{2.870539in}}%
\pgfpathcurveto{\pgfqpoint{7.538329in}{2.870539in}}{\pgfqpoint{7.543166in}{2.872543in}}{\pgfqpoint{7.546733in}{2.876109in}}%
\pgfpathcurveto{\pgfqpoint{7.550299in}{2.879675in}}{\pgfqpoint{7.552303in}{2.884513in}}{\pgfqpoint{7.552303in}{2.889557in}}%
\pgfpathcurveto{\pgfqpoint{7.552303in}{2.894601in}}{\pgfqpoint{7.550299in}{2.899438in}}{\pgfqpoint{7.546733in}{2.903005in}}%
\pgfpathcurveto{\pgfqpoint{7.543166in}{2.906571in}}{\pgfqpoint{7.538329in}{2.908575in}}{\pgfqpoint{7.533285in}{2.908575in}}%
\pgfpathcurveto{\pgfqpoint{7.528241in}{2.908575in}}{\pgfqpoint{7.523403in}{2.906571in}}{\pgfqpoint{7.519837in}{2.903005in}}%
\pgfpathcurveto{\pgfqpoint{7.516271in}{2.899438in}}{\pgfqpoint{7.514267in}{2.894601in}}{\pgfqpoint{7.514267in}{2.889557in}}%
\pgfpathcurveto{\pgfqpoint{7.514267in}{2.884513in}}{\pgfqpoint{7.516271in}{2.879675in}}{\pgfqpoint{7.519837in}{2.876109in}}%
\pgfpathcurveto{\pgfqpoint{7.523403in}{2.872543in}}{\pgfqpoint{7.528241in}{2.870539in}}{\pgfqpoint{7.533285in}{2.870539in}}%
\pgfpathclose%
\pgfusepath{fill}%
\end{pgfscope}%
\begin{pgfscope}%
\pgfpathrectangle{\pgfqpoint{6.572727in}{0.473000in}}{\pgfqpoint{4.227273in}{3.311000in}}%
\pgfusepath{clip}%
\pgfsetbuttcap%
\pgfsetroundjoin%
\definecolor{currentfill}{rgb}{0.127568,0.566949,0.550556}%
\pgfsetfillcolor{currentfill}%
\pgfsetfillopacity{0.700000}%
\pgfsetlinewidth{0.000000pt}%
\definecolor{currentstroke}{rgb}{0.000000,0.000000,0.000000}%
\pgfsetstrokecolor{currentstroke}%
\pgfsetstrokeopacity{0.700000}%
\pgfsetdash{}{0pt}%
\pgfpathmoveto{\pgfqpoint{7.858219in}{1.914766in}}%
\pgfpathcurveto{\pgfqpoint{7.863263in}{1.914766in}}{\pgfqpoint{7.868101in}{1.916770in}}{\pgfqpoint{7.871667in}{1.920336in}}%
\pgfpathcurveto{\pgfqpoint{7.875233in}{1.923902in}}{\pgfqpoint{7.877237in}{1.928740in}}{\pgfqpoint{7.877237in}{1.933784in}}%
\pgfpathcurveto{\pgfqpoint{7.877237in}{1.938827in}}{\pgfqpoint{7.875233in}{1.943665in}}{\pgfqpoint{7.871667in}{1.947232in}}%
\pgfpathcurveto{\pgfqpoint{7.868101in}{1.950798in}}{\pgfqpoint{7.863263in}{1.952802in}}{\pgfqpoint{7.858219in}{1.952802in}}%
\pgfpathcurveto{\pgfqpoint{7.853175in}{1.952802in}}{\pgfqpoint{7.848338in}{1.950798in}}{\pgfqpoint{7.844771in}{1.947232in}}%
\pgfpathcurveto{\pgfqpoint{7.841205in}{1.943665in}}{\pgfqpoint{7.839201in}{1.938827in}}{\pgfqpoint{7.839201in}{1.933784in}}%
\pgfpathcurveto{\pgfqpoint{7.839201in}{1.928740in}}{\pgfqpoint{7.841205in}{1.923902in}}{\pgfqpoint{7.844771in}{1.920336in}}%
\pgfpathcurveto{\pgfqpoint{7.848338in}{1.916770in}}{\pgfqpoint{7.853175in}{1.914766in}}{\pgfqpoint{7.858219in}{1.914766in}}%
\pgfpathclose%
\pgfusepath{fill}%
\end{pgfscope}%
\begin{pgfscope}%
\pgfpathrectangle{\pgfqpoint{6.572727in}{0.473000in}}{\pgfqpoint{4.227273in}{3.311000in}}%
\pgfusepath{clip}%
\pgfsetbuttcap%
\pgfsetroundjoin%
\definecolor{currentfill}{rgb}{0.127568,0.566949,0.550556}%
\pgfsetfillcolor{currentfill}%
\pgfsetfillopacity{0.700000}%
\pgfsetlinewidth{0.000000pt}%
\definecolor{currentstroke}{rgb}{0.000000,0.000000,0.000000}%
\pgfsetstrokecolor{currentstroke}%
\pgfsetstrokeopacity{0.700000}%
\pgfsetdash{}{0pt}%
\pgfpathmoveto{\pgfqpoint{8.060040in}{3.614482in}}%
\pgfpathcurveto{\pgfqpoint{8.065083in}{3.614482in}}{\pgfqpoint{8.069921in}{3.616486in}}{\pgfqpoint{8.073487in}{3.620052in}}%
\pgfpathcurveto{\pgfqpoint{8.077054in}{3.623619in}}{\pgfqpoint{8.079058in}{3.628456in}}{\pgfqpoint{8.079058in}{3.633500in}}%
\pgfpathcurveto{\pgfqpoint{8.079058in}{3.638544in}}{\pgfqpoint{8.077054in}{3.643381in}}{\pgfqpoint{8.073487in}{3.646948in}}%
\pgfpathcurveto{\pgfqpoint{8.069921in}{3.650514in}}{\pgfqpoint{8.065083in}{3.652518in}}{\pgfqpoint{8.060040in}{3.652518in}}%
\pgfpathcurveto{\pgfqpoint{8.054996in}{3.652518in}}{\pgfqpoint{8.050158in}{3.650514in}}{\pgfqpoint{8.046592in}{3.646948in}}%
\pgfpathcurveto{\pgfqpoint{8.043025in}{3.643381in}}{\pgfqpoint{8.041021in}{3.638544in}}{\pgfqpoint{8.041021in}{3.633500in}}%
\pgfpathcurveto{\pgfqpoint{8.041021in}{3.628456in}}{\pgfqpoint{8.043025in}{3.623619in}}{\pgfqpoint{8.046592in}{3.620052in}}%
\pgfpathcurveto{\pgfqpoint{8.050158in}{3.616486in}}{\pgfqpoint{8.054996in}{3.614482in}}{\pgfqpoint{8.060040in}{3.614482in}}%
\pgfpathclose%
\pgfusepath{fill}%
\end{pgfscope}%
\begin{pgfscope}%
\pgfpathrectangle{\pgfqpoint{6.572727in}{0.473000in}}{\pgfqpoint{4.227273in}{3.311000in}}%
\pgfusepath{clip}%
\pgfsetbuttcap%
\pgfsetroundjoin%
\definecolor{currentfill}{rgb}{0.127568,0.566949,0.550556}%
\pgfsetfillcolor{currentfill}%
\pgfsetfillopacity{0.700000}%
\pgfsetlinewidth{0.000000pt}%
\definecolor{currentstroke}{rgb}{0.000000,0.000000,0.000000}%
\pgfsetstrokecolor{currentstroke}%
\pgfsetstrokeopacity{0.700000}%
\pgfsetdash{}{0pt}%
\pgfpathmoveto{\pgfqpoint{7.588776in}{1.561298in}}%
\pgfpathcurveto{\pgfqpoint{7.593820in}{1.561298in}}{\pgfqpoint{7.598658in}{1.563301in}}{\pgfqpoint{7.602224in}{1.566868in}}%
\pgfpathcurveto{\pgfqpoint{7.605791in}{1.570434in}}{\pgfqpoint{7.607795in}{1.575272in}}{\pgfqpoint{7.607795in}{1.580316in}}%
\pgfpathcurveto{\pgfqpoint{7.607795in}{1.585359in}}{\pgfqpoint{7.605791in}{1.590197in}}{\pgfqpoint{7.602224in}{1.593764in}}%
\pgfpathcurveto{\pgfqpoint{7.598658in}{1.597330in}}{\pgfqpoint{7.593820in}{1.599334in}}{\pgfqpoint{7.588776in}{1.599334in}}%
\pgfpathcurveto{\pgfqpoint{7.583733in}{1.599334in}}{\pgfqpoint{7.578895in}{1.597330in}}{\pgfqpoint{7.575329in}{1.593764in}}%
\pgfpathcurveto{\pgfqpoint{7.571762in}{1.590197in}}{\pgfqpoint{7.569758in}{1.585359in}}{\pgfqpoint{7.569758in}{1.580316in}}%
\pgfpathcurveto{\pgfqpoint{7.569758in}{1.575272in}}{\pgfqpoint{7.571762in}{1.570434in}}{\pgfqpoint{7.575329in}{1.566868in}}%
\pgfpathcurveto{\pgfqpoint{7.578895in}{1.563301in}}{\pgfqpoint{7.583733in}{1.561298in}}{\pgfqpoint{7.588776in}{1.561298in}}%
\pgfpathclose%
\pgfusepath{fill}%
\end{pgfscope}%
\begin{pgfscope}%
\pgfpathrectangle{\pgfqpoint{6.572727in}{0.473000in}}{\pgfqpoint{4.227273in}{3.311000in}}%
\pgfusepath{clip}%
\pgfsetbuttcap%
\pgfsetroundjoin%
\definecolor{currentfill}{rgb}{0.127568,0.566949,0.550556}%
\pgfsetfillcolor{currentfill}%
\pgfsetfillopacity{0.700000}%
\pgfsetlinewidth{0.000000pt}%
\definecolor{currentstroke}{rgb}{0.000000,0.000000,0.000000}%
\pgfsetstrokecolor{currentstroke}%
\pgfsetstrokeopacity{0.700000}%
\pgfsetdash{}{0pt}%
\pgfpathmoveto{\pgfqpoint{8.196337in}{2.942797in}}%
\pgfpathcurveto{\pgfqpoint{8.201381in}{2.942797in}}{\pgfqpoint{8.206219in}{2.944801in}}{\pgfqpoint{8.209785in}{2.948367in}}%
\pgfpathcurveto{\pgfqpoint{8.213351in}{2.951934in}}{\pgfqpoint{8.215355in}{2.956772in}}{\pgfqpoint{8.215355in}{2.961815in}}%
\pgfpathcurveto{\pgfqpoint{8.215355in}{2.966859in}}{\pgfqpoint{8.213351in}{2.971697in}}{\pgfqpoint{8.209785in}{2.975263in}}%
\pgfpathcurveto{\pgfqpoint{8.206219in}{2.978830in}}{\pgfqpoint{8.201381in}{2.980833in}}{\pgfqpoint{8.196337in}{2.980833in}}%
\pgfpathcurveto{\pgfqpoint{8.191293in}{2.980833in}}{\pgfqpoint{8.186456in}{2.978830in}}{\pgfqpoint{8.182889in}{2.975263in}}%
\pgfpathcurveto{\pgfqpoint{8.179323in}{2.971697in}}{\pgfqpoint{8.177319in}{2.966859in}}{\pgfqpoint{8.177319in}{2.961815in}}%
\pgfpathcurveto{\pgfqpoint{8.177319in}{2.956772in}}{\pgfqpoint{8.179323in}{2.951934in}}{\pgfqpoint{8.182889in}{2.948367in}}%
\pgfpathcurveto{\pgfqpoint{8.186456in}{2.944801in}}{\pgfqpoint{8.191293in}{2.942797in}}{\pgfqpoint{8.196337in}{2.942797in}}%
\pgfpathclose%
\pgfusepath{fill}%
\end{pgfscope}%
\begin{pgfscope}%
\pgfpathrectangle{\pgfqpoint{6.572727in}{0.473000in}}{\pgfqpoint{4.227273in}{3.311000in}}%
\pgfusepath{clip}%
\pgfsetbuttcap%
\pgfsetroundjoin%
\definecolor{currentfill}{rgb}{0.993248,0.906157,0.143936}%
\pgfsetfillcolor{currentfill}%
\pgfsetfillopacity{0.700000}%
\pgfsetlinewidth{0.000000pt}%
\definecolor{currentstroke}{rgb}{0.000000,0.000000,0.000000}%
\pgfsetstrokecolor{currentstroke}%
\pgfsetstrokeopacity{0.700000}%
\pgfsetdash{}{0pt}%
\pgfpathmoveto{\pgfqpoint{9.253400in}{1.586724in}}%
\pgfpathcurveto{\pgfqpoint{9.258444in}{1.586724in}}{\pgfqpoint{9.263281in}{1.588728in}}{\pgfqpoint{9.266848in}{1.592295in}}%
\pgfpathcurveto{\pgfqpoint{9.270414in}{1.595861in}}{\pgfqpoint{9.272418in}{1.600699in}}{\pgfqpoint{9.272418in}{1.605743in}}%
\pgfpathcurveto{\pgfqpoint{9.272418in}{1.610786in}}{\pgfqpoint{9.270414in}{1.615624in}}{\pgfqpoint{9.266848in}{1.619190in}}%
\pgfpathcurveto{\pgfqpoint{9.263281in}{1.622757in}}{\pgfqpoint{9.258444in}{1.624761in}}{\pgfqpoint{9.253400in}{1.624761in}}%
\pgfpathcurveto{\pgfqpoint{9.248356in}{1.624761in}}{\pgfqpoint{9.243519in}{1.622757in}}{\pgfqpoint{9.239952in}{1.619190in}}%
\pgfpathcurveto{\pgfqpoint{9.236386in}{1.615624in}}{\pgfqpoint{9.234382in}{1.610786in}}{\pgfqpoint{9.234382in}{1.605743in}}%
\pgfpathcurveto{\pgfqpoint{9.234382in}{1.600699in}}{\pgfqpoint{9.236386in}{1.595861in}}{\pgfqpoint{9.239952in}{1.592295in}}%
\pgfpathcurveto{\pgfqpoint{9.243519in}{1.588728in}}{\pgfqpoint{9.248356in}{1.586724in}}{\pgfqpoint{9.253400in}{1.586724in}}%
\pgfpathclose%
\pgfusepath{fill}%
\end{pgfscope}%
\begin{pgfscope}%
\pgfpathrectangle{\pgfqpoint{6.572727in}{0.473000in}}{\pgfqpoint{4.227273in}{3.311000in}}%
\pgfusepath{clip}%
\pgfsetbuttcap%
\pgfsetroundjoin%
\definecolor{currentfill}{rgb}{0.993248,0.906157,0.143936}%
\pgfsetfillcolor{currentfill}%
\pgfsetfillopacity{0.700000}%
\pgfsetlinewidth{0.000000pt}%
\definecolor{currentstroke}{rgb}{0.000000,0.000000,0.000000}%
\pgfsetstrokecolor{currentstroke}%
\pgfsetstrokeopacity{0.700000}%
\pgfsetdash{}{0pt}%
\pgfpathmoveto{\pgfqpoint{9.654639in}{2.027618in}}%
\pgfpathcurveto{\pgfqpoint{9.659682in}{2.027618in}}{\pgfqpoint{9.664520in}{2.029622in}}{\pgfqpoint{9.668087in}{2.033188in}}%
\pgfpathcurveto{\pgfqpoint{9.671653in}{2.036755in}}{\pgfqpoint{9.673657in}{2.041592in}}{\pgfqpoint{9.673657in}{2.046636in}}%
\pgfpathcurveto{\pgfqpoint{9.673657in}{2.051680in}}{\pgfqpoint{9.671653in}{2.056518in}}{\pgfqpoint{9.668087in}{2.060084in}}%
\pgfpathcurveto{\pgfqpoint{9.664520in}{2.063650in}}{\pgfqpoint{9.659682in}{2.065654in}}{\pgfqpoint{9.654639in}{2.065654in}}%
\pgfpathcurveto{\pgfqpoint{9.649595in}{2.065654in}}{\pgfqpoint{9.644757in}{2.063650in}}{\pgfqpoint{9.641191in}{2.060084in}}%
\pgfpathcurveto{\pgfqpoint{9.637624in}{2.056518in}}{\pgfqpoint{9.635621in}{2.051680in}}{\pgfqpoint{9.635621in}{2.046636in}}%
\pgfpathcurveto{\pgfqpoint{9.635621in}{2.041592in}}{\pgfqpoint{9.637624in}{2.036755in}}{\pgfqpoint{9.641191in}{2.033188in}}%
\pgfpathcurveto{\pgfqpoint{9.644757in}{2.029622in}}{\pgfqpoint{9.649595in}{2.027618in}}{\pgfqpoint{9.654639in}{2.027618in}}%
\pgfpathclose%
\pgfusepath{fill}%
\end{pgfscope}%
\begin{pgfscope}%
\pgfpathrectangle{\pgfqpoint{6.572727in}{0.473000in}}{\pgfqpoint{4.227273in}{3.311000in}}%
\pgfusepath{clip}%
\pgfsetbuttcap%
\pgfsetroundjoin%
\definecolor{currentfill}{rgb}{0.127568,0.566949,0.550556}%
\pgfsetfillcolor{currentfill}%
\pgfsetfillopacity{0.700000}%
\pgfsetlinewidth{0.000000pt}%
\definecolor{currentstroke}{rgb}{0.000000,0.000000,0.000000}%
\pgfsetstrokecolor{currentstroke}%
\pgfsetstrokeopacity{0.700000}%
\pgfsetdash{}{0pt}%
\pgfpathmoveto{\pgfqpoint{7.898775in}{1.091752in}}%
\pgfpathcurveto{\pgfqpoint{7.903818in}{1.091752in}}{\pgfqpoint{7.908656in}{1.093756in}}{\pgfqpoint{7.912222in}{1.097323in}}%
\pgfpathcurveto{\pgfqpoint{7.915789in}{1.100889in}}{\pgfqpoint{7.917793in}{1.105727in}}{\pgfqpoint{7.917793in}{1.110771in}}%
\pgfpathcurveto{\pgfqpoint{7.917793in}{1.115814in}}{\pgfqpoint{7.915789in}{1.120652in}}{\pgfqpoint{7.912222in}{1.124218in}}%
\pgfpathcurveto{\pgfqpoint{7.908656in}{1.127785in}}{\pgfqpoint{7.903818in}{1.129789in}}{\pgfqpoint{7.898775in}{1.129789in}}%
\pgfpathcurveto{\pgfqpoint{7.893731in}{1.129789in}}{\pgfqpoint{7.888893in}{1.127785in}}{\pgfqpoint{7.885327in}{1.124218in}}%
\pgfpathcurveto{\pgfqpoint{7.881760in}{1.120652in}}{\pgfqpoint{7.879756in}{1.115814in}}{\pgfqpoint{7.879756in}{1.110771in}}%
\pgfpathcurveto{\pgfqpoint{7.879756in}{1.105727in}}{\pgfqpoint{7.881760in}{1.100889in}}{\pgfqpoint{7.885327in}{1.097323in}}%
\pgfpathcurveto{\pgfqpoint{7.888893in}{1.093756in}}{\pgfqpoint{7.893731in}{1.091752in}}{\pgfqpoint{7.898775in}{1.091752in}}%
\pgfpathclose%
\pgfusepath{fill}%
\end{pgfscope}%
\begin{pgfscope}%
\pgfpathrectangle{\pgfqpoint{6.572727in}{0.473000in}}{\pgfqpoint{4.227273in}{3.311000in}}%
\pgfusepath{clip}%
\pgfsetbuttcap%
\pgfsetroundjoin%
\definecolor{currentfill}{rgb}{0.127568,0.566949,0.550556}%
\pgfsetfillcolor{currentfill}%
\pgfsetfillopacity{0.700000}%
\pgfsetlinewidth{0.000000pt}%
\definecolor{currentstroke}{rgb}{0.000000,0.000000,0.000000}%
\pgfsetstrokecolor{currentstroke}%
\pgfsetstrokeopacity{0.700000}%
\pgfsetdash{}{0pt}%
\pgfpathmoveto{\pgfqpoint{7.254500in}{1.147804in}}%
\pgfpathcurveto{\pgfqpoint{7.259544in}{1.147804in}}{\pgfqpoint{7.264381in}{1.149808in}}{\pgfqpoint{7.267948in}{1.153374in}}%
\pgfpathcurveto{\pgfqpoint{7.271514in}{1.156941in}}{\pgfqpoint{7.273518in}{1.161779in}}{\pgfqpoint{7.273518in}{1.166822in}}%
\pgfpathcurveto{\pgfqpoint{7.273518in}{1.171866in}}{\pgfqpoint{7.271514in}{1.176704in}}{\pgfqpoint{7.267948in}{1.180270in}}%
\pgfpathcurveto{\pgfqpoint{7.264381in}{1.183837in}}{\pgfqpoint{7.259544in}{1.185840in}}{\pgfqpoint{7.254500in}{1.185840in}}%
\pgfpathcurveto{\pgfqpoint{7.249456in}{1.185840in}}{\pgfqpoint{7.244619in}{1.183837in}}{\pgfqpoint{7.241052in}{1.180270in}}%
\pgfpathcurveto{\pgfqpoint{7.237486in}{1.176704in}}{\pgfqpoint{7.235482in}{1.171866in}}{\pgfqpoint{7.235482in}{1.166822in}}%
\pgfpathcurveto{\pgfqpoint{7.235482in}{1.161779in}}{\pgfqpoint{7.237486in}{1.156941in}}{\pgfqpoint{7.241052in}{1.153374in}}%
\pgfpathcurveto{\pgfqpoint{7.244619in}{1.149808in}}{\pgfqpoint{7.249456in}{1.147804in}}{\pgfqpoint{7.254500in}{1.147804in}}%
\pgfpathclose%
\pgfusepath{fill}%
\end{pgfscope}%
\begin{pgfscope}%
\pgfpathrectangle{\pgfqpoint{6.572727in}{0.473000in}}{\pgfqpoint{4.227273in}{3.311000in}}%
\pgfusepath{clip}%
\pgfsetbuttcap%
\pgfsetroundjoin%
\definecolor{currentfill}{rgb}{0.127568,0.566949,0.550556}%
\pgfsetfillcolor{currentfill}%
\pgfsetfillopacity{0.700000}%
\pgfsetlinewidth{0.000000pt}%
\definecolor{currentstroke}{rgb}{0.000000,0.000000,0.000000}%
\pgfsetstrokecolor{currentstroke}%
\pgfsetstrokeopacity{0.700000}%
\pgfsetdash{}{0pt}%
\pgfpathmoveto{\pgfqpoint{7.777944in}{3.131699in}}%
\pgfpathcurveto{\pgfqpoint{7.782988in}{3.131699in}}{\pgfqpoint{7.787826in}{3.133703in}}{\pgfqpoint{7.791392in}{3.137269in}}%
\pgfpathcurveto{\pgfqpoint{7.794959in}{3.140836in}}{\pgfqpoint{7.796963in}{3.145673in}}{\pgfqpoint{7.796963in}{3.150717in}}%
\pgfpathcurveto{\pgfqpoint{7.796963in}{3.155761in}}{\pgfqpoint{7.794959in}{3.160599in}}{\pgfqpoint{7.791392in}{3.164165in}}%
\pgfpathcurveto{\pgfqpoint{7.787826in}{3.167731in}}{\pgfqpoint{7.782988in}{3.169735in}}{\pgfqpoint{7.777944in}{3.169735in}}%
\pgfpathcurveto{\pgfqpoint{7.772901in}{3.169735in}}{\pgfqpoint{7.768063in}{3.167731in}}{\pgfqpoint{7.764497in}{3.164165in}}%
\pgfpathcurveto{\pgfqpoint{7.760930in}{3.160599in}}{\pgfqpoint{7.758926in}{3.155761in}}{\pgfqpoint{7.758926in}{3.150717in}}%
\pgfpathcurveto{\pgfqpoint{7.758926in}{3.145673in}}{\pgfqpoint{7.760930in}{3.140836in}}{\pgfqpoint{7.764497in}{3.137269in}}%
\pgfpathcurveto{\pgfqpoint{7.768063in}{3.133703in}}{\pgfqpoint{7.772901in}{3.131699in}}{\pgfqpoint{7.777944in}{3.131699in}}%
\pgfpathclose%
\pgfusepath{fill}%
\end{pgfscope}%
\begin{pgfscope}%
\pgfpathrectangle{\pgfqpoint{6.572727in}{0.473000in}}{\pgfqpoint{4.227273in}{3.311000in}}%
\pgfusepath{clip}%
\pgfsetbuttcap%
\pgfsetroundjoin%
\definecolor{currentfill}{rgb}{0.993248,0.906157,0.143936}%
\pgfsetfillcolor{currentfill}%
\pgfsetfillopacity{0.700000}%
\pgfsetlinewidth{0.000000pt}%
\definecolor{currentstroke}{rgb}{0.000000,0.000000,0.000000}%
\pgfsetstrokecolor{currentstroke}%
\pgfsetstrokeopacity{0.700000}%
\pgfsetdash{}{0pt}%
\pgfpathmoveto{\pgfqpoint{9.963304in}{1.739508in}}%
\pgfpathcurveto{\pgfqpoint{9.968348in}{1.739508in}}{\pgfqpoint{9.973186in}{1.741511in}}{\pgfqpoint{9.976752in}{1.745078in}}%
\pgfpathcurveto{\pgfqpoint{9.980319in}{1.748644in}}{\pgfqpoint{9.982322in}{1.753482in}}{\pgfqpoint{9.982322in}{1.758526in}}%
\pgfpathcurveto{\pgfqpoint{9.982322in}{1.763569in}}{\pgfqpoint{9.980319in}{1.768407in}}{\pgfqpoint{9.976752in}{1.771974in}}%
\pgfpathcurveto{\pgfqpoint{9.973186in}{1.775540in}}{\pgfqpoint{9.968348in}{1.777544in}}{\pgfqpoint{9.963304in}{1.777544in}}%
\pgfpathcurveto{\pgfqpoint{9.958261in}{1.777544in}}{\pgfqpoint{9.953423in}{1.775540in}}{\pgfqpoint{9.949856in}{1.771974in}}%
\pgfpathcurveto{\pgfqpoint{9.946290in}{1.768407in}}{\pgfqpoint{9.944286in}{1.763569in}}{\pgfqpoint{9.944286in}{1.758526in}}%
\pgfpathcurveto{\pgfqpoint{9.944286in}{1.753482in}}{\pgfqpoint{9.946290in}{1.748644in}}{\pgfqpoint{9.949856in}{1.745078in}}%
\pgfpathcurveto{\pgfqpoint{9.953423in}{1.741511in}}{\pgfqpoint{9.958261in}{1.739508in}}{\pgfqpoint{9.963304in}{1.739508in}}%
\pgfpathclose%
\pgfusepath{fill}%
\end{pgfscope}%
\begin{pgfscope}%
\pgfpathrectangle{\pgfqpoint{6.572727in}{0.473000in}}{\pgfqpoint{4.227273in}{3.311000in}}%
\pgfusepath{clip}%
\pgfsetbuttcap%
\pgfsetroundjoin%
\definecolor{currentfill}{rgb}{0.993248,0.906157,0.143936}%
\pgfsetfillcolor{currentfill}%
\pgfsetfillopacity{0.700000}%
\pgfsetlinewidth{0.000000pt}%
\definecolor{currentstroke}{rgb}{0.000000,0.000000,0.000000}%
\pgfsetstrokecolor{currentstroke}%
\pgfsetstrokeopacity{0.700000}%
\pgfsetdash{}{0pt}%
\pgfpathmoveto{\pgfqpoint{9.563522in}{2.153141in}}%
\pgfpathcurveto{\pgfqpoint{9.568566in}{2.153141in}}{\pgfqpoint{9.573403in}{2.155145in}}{\pgfqpoint{9.576970in}{2.158711in}}%
\pgfpathcurveto{\pgfqpoint{9.580536in}{2.162278in}}{\pgfqpoint{9.582540in}{2.167116in}}{\pgfqpoint{9.582540in}{2.172159in}}%
\pgfpathcurveto{\pgfqpoint{9.582540in}{2.177203in}}{\pgfqpoint{9.580536in}{2.182041in}}{\pgfqpoint{9.576970in}{2.185607in}}%
\pgfpathcurveto{\pgfqpoint{9.573403in}{2.189174in}}{\pgfqpoint{9.568566in}{2.191177in}}{\pgfqpoint{9.563522in}{2.191177in}}%
\pgfpathcurveto{\pgfqpoint{9.558478in}{2.191177in}}{\pgfqpoint{9.553641in}{2.189174in}}{\pgfqpoint{9.550074in}{2.185607in}}%
\pgfpathcurveto{\pgfqpoint{9.546508in}{2.182041in}}{\pgfqpoint{9.544504in}{2.177203in}}{\pgfqpoint{9.544504in}{2.172159in}}%
\pgfpathcurveto{\pgfqpoint{9.544504in}{2.167116in}}{\pgfqpoint{9.546508in}{2.162278in}}{\pgfqpoint{9.550074in}{2.158711in}}%
\pgfpathcurveto{\pgfqpoint{9.553641in}{2.155145in}}{\pgfqpoint{9.558478in}{2.153141in}}{\pgfqpoint{9.563522in}{2.153141in}}%
\pgfpathclose%
\pgfusepath{fill}%
\end{pgfscope}%
\begin{pgfscope}%
\pgfpathrectangle{\pgfqpoint{6.572727in}{0.473000in}}{\pgfqpoint{4.227273in}{3.311000in}}%
\pgfusepath{clip}%
\pgfsetbuttcap%
\pgfsetroundjoin%
\definecolor{currentfill}{rgb}{0.127568,0.566949,0.550556}%
\pgfsetfillcolor{currentfill}%
\pgfsetfillopacity{0.700000}%
\pgfsetlinewidth{0.000000pt}%
\definecolor{currentstroke}{rgb}{0.000000,0.000000,0.000000}%
\pgfsetstrokecolor{currentstroke}%
\pgfsetstrokeopacity{0.700000}%
\pgfsetdash{}{0pt}%
\pgfpathmoveto{\pgfqpoint{8.235870in}{2.560469in}}%
\pgfpathcurveto{\pgfqpoint{8.240914in}{2.560469in}}{\pgfqpoint{8.245752in}{2.562473in}}{\pgfqpoint{8.249318in}{2.566040in}}%
\pgfpathcurveto{\pgfqpoint{8.252885in}{2.569606in}}{\pgfqpoint{8.254889in}{2.574444in}}{\pgfqpoint{8.254889in}{2.579488in}}%
\pgfpathcurveto{\pgfqpoint{8.254889in}{2.584531in}}{\pgfqpoint{8.252885in}{2.589369in}}{\pgfqpoint{8.249318in}{2.592935in}}%
\pgfpathcurveto{\pgfqpoint{8.245752in}{2.596502in}}{\pgfqpoint{8.240914in}{2.598506in}}{\pgfqpoint{8.235870in}{2.598506in}}%
\pgfpathcurveto{\pgfqpoint{8.230827in}{2.598506in}}{\pgfqpoint{8.225989in}{2.596502in}}{\pgfqpoint{8.222423in}{2.592935in}}%
\pgfpathcurveto{\pgfqpoint{8.218856in}{2.589369in}}{\pgfqpoint{8.216852in}{2.584531in}}{\pgfqpoint{8.216852in}{2.579488in}}%
\pgfpathcurveto{\pgfqpoint{8.216852in}{2.574444in}}{\pgfqpoint{8.218856in}{2.569606in}}{\pgfqpoint{8.222423in}{2.566040in}}%
\pgfpathcurveto{\pgfqpoint{8.225989in}{2.562473in}}{\pgfqpoint{8.230827in}{2.560469in}}{\pgfqpoint{8.235870in}{2.560469in}}%
\pgfpathclose%
\pgfusepath{fill}%
\end{pgfscope}%
\begin{pgfscope}%
\pgfpathrectangle{\pgfqpoint{6.572727in}{0.473000in}}{\pgfqpoint{4.227273in}{3.311000in}}%
\pgfusepath{clip}%
\pgfsetbuttcap%
\pgfsetroundjoin%
\definecolor{currentfill}{rgb}{0.127568,0.566949,0.550556}%
\pgfsetfillcolor{currentfill}%
\pgfsetfillopacity{0.700000}%
\pgfsetlinewidth{0.000000pt}%
\definecolor{currentstroke}{rgb}{0.000000,0.000000,0.000000}%
\pgfsetstrokecolor{currentstroke}%
\pgfsetstrokeopacity{0.700000}%
\pgfsetdash{}{0pt}%
\pgfpathmoveto{\pgfqpoint{7.442700in}{1.968758in}}%
\pgfpathcurveto{\pgfqpoint{7.447744in}{1.968758in}}{\pgfqpoint{7.452582in}{1.970762in}}{\pgfqpoint{7.456148in}{1.974328in}}%
\pgfpathcurveto{\pgfqpoint{7.459714in}{1.977894in}}{\pgfqpoint{7.461718in}{1.982732in}}{\pgfqpoint{7.461718in}{1.987776in}}%
\pgfpathcurveto{\pgfqpoint{7.461718in}{1.992819in}}{\pgfqpoint{7.459714in}{1.997657in}}{\pgfqpoint{7.456148in}{2.001224in}}%
\pgfpathcurveto{\pgfqpoint{7.452582in}{2.004790in}}{\pgfqpoint{7.447744in}{2.006794in}}{\pgfqpoint{7.442700in}{2.006794in}}%
\pgfpathcurveto{\pgfqpoint{7.437656in}{2.006794in}}{\pgfqpoint{7.432819in}{2.004790in}}{\pgfqpoint{7.429252in}{2.001224in}}%
\pgfpathcurveto{\pgfqpoint{7.425686in}{1.997657in}}{\pgfqpoint{7.423682in}{1.992819in}}{\pgfqpoint{7.423682in}{1.987776in}}%
\pgfpathcurveto{\pgfqpoint{7.423682in}{1.982732in}}{\pgfqpoint{7.425686in}{1.977894in}}{\pgfqpoint{7.429252in}{1.974328in}}%
\pgfpathcurveto{\pgfqpoint{7.432819in}{1.970762in}}{\pgfqpoint{7.437656in}{1.968758in}}{\pgfqpoint{7.442700in}{1.968758in}}%
\pgfpathclose%
\pgfusepath{fill}%
\end{pgfscope}%
\begin{pgfscope}%
\pgfpathrectangle{\pgfqpoint{6.572727in}{0.473000in}}{\pgfqpoint{4.227273in}{3.311000in}}%
\pgfusepath{clip}%
\pgfsetbuttcap%
\pgfsetroundjoin%
\definecolor{currentfill}{rgb}{0.993248,0.906157,0.143936}%
\pgfsetfillcolor{currentfill}%
\pgfsetfillopacity{0.700000}%
\pgfsetlinewidth{0.000000pt}%
\definecolor{currentstroke}{rgb}{0.000000,0.000000,0.000000}%
\pgfsetstrokecolor{currentstroke}%
\pgfsetstrokeopacity{0.700000}%
\pgfsetdash{}{0pt}%
\pgfpathmoveto{\pgfqpoint{9.637547in}{1.443146in}}%
\pgfpathcurveto{\pgfqpoint{9.642591in}{1.443146in}}{\pgfqpoint{9.647428in}{1.445150in}}{\pgfqpoint{9.650995in}{1.448716in}}%
\pgfpathcurveto{\pgfqpoint{9.654561in}{1.452283in}}{\pgfqpoint{9.656565in}{1.457120in}}{\pgfqpoint{9.656565in}{1.462164in}}%
\pgfpathcurveto{\pgfqpoint{9.656565in}{1.467208in}}{\pgfqpoint{9.654561in}{1.472045in}}{\pgfqpoint{9.650995in}{1.475612in}}%
\pgfpathcurveto{\pgfqpoint{9.647428in}{1.479178in}}{\pgfqpoint{9.642591in}{1.481182in}}{\pgfqpoint{9.637547in}{1.481182in}}%
\pgfpathcurveto{\pgfqpoint{9.632503in}{1.481182in}}{\pgfqpoint{9.627666in}{1.479178in}}{\pgfqpoint{9.624099in}{1.475612in}}%
\pgfpathcurveto{\pgfqpoint{9.620533in}{1.472045in}}{\pgfqpoint{9.618529in}{1.467208in}}{\pgfqpoint{9.618529in}{1.462164in}}%
\pgfpathcurveto{\pgfqpoint{9.618529in}{1.457120in}}{\pgfqpoint{9.620533in}{1.452283in}}{\pgfqpoint{9.624099in}{1.448716in}}%
\pgfpathcurveto{\pgfqpoint{9.627666in}{1.445150in}}{\pgfqpoint{9.632503in}{1.443146in}}{\pgfqpoint{9.637547in}{1.443146in}}%
\pgfpathclose%
\pgfusepath{fill}%
\end{pgfscope}%
\begin{pgfscope}%
\pgfpathrectangle{\pgfqpoint{6.572727in}{0.473000in}}{\pgfqpoint{4.227273in}{3.311000in}}%
\pgfusepath{clip}%
\pgfsetbuttcap%
\pgfsetroundjoin%
\definecolor{currentfill}{rgb}{0.993248,0.906157,0.143936}%
\pgfsetfillcolor{currentfill}%
\pgfsetfillopacity{0.700000}%
\pgfsetlinewidth{0.000000pt}%
\definecolor{currentstroke}{rgb}{0.000000,0.000000,0.000000}%
\pgfsetstrokecolor{currentstroke}%
\pgfsetstrokeopacity{0.700000}%
\pgfsetdash{}{0pt}%
\pgfpathmoveto{\pgfqpoint{9.799364in}{1.228886in}}%
\pgfpathcurveto{\pgfqpoint{9.804407in}{1.228886in}}{\pgfqpoint{9.809245in}{1.230890in}}{\pgfqpoint{9.812811in}{1.234456in}}%
\pgfpathcurveto{\pgfqpoint{9.816378in}{1.238023in}}{\pgfqpoint{9.818382in}{1.242860in}}{\pgfqpoint{9.818382in}{1.247904in}}%
\pgfpathcurveto{\pgfqpoint{9.818382in}{1.252948in}}{\pgfqpoint{9.816378in}{1.257786in}}{\pgfqpoint{9.812811in}{1.261352in}}%
\pgfpathcurveto{\pgfqpoint{9.809245in}{1.264918in}}{\pgfqpoint{9.804407in}{1.266922in}}{\pgfqpoint{9.799364in}{1.266922in}}%
\pgfpathcurveto{\pgfqpoint{9.794320in}{1.266922in}}{\pgfqpoint{9.789482in}{1.264918in}}{\pgfqpoint{9.785916in}{1.261352in}}%
\pgfpathcurveto{\pgfqpoint{9.782349in}{1.257786in}}{\pgfqpoint{9.780345in}{1.252948in}}{\pgfqpoint{9.780345in}{1.247904in}}%
\pgfpathcurveto{\pgfqpoint{9.780345in}{1.242860in}}{\pgfqpoint{9.782349in}{1.238023in}}{\pgfqpoint{9.785916in}{1.234456in}}%
\pgfpathcurveto{\pgfqpoint{9.789482in}{1.230890in}}{\pgfqpoint{9.794320in}{1.228886in}}{\pgfqpoint{9.799364in}{1.228886in}}%
\pgfpathclose%
\pgfusepath{fill}%
\end{pgfscope}%
\begin{pgfscope}%
\pgfpathrectangle{\pgfqpoint{6.572727in}{0.473000in}}{\pgfqpoint{4.227273in}{3.311000in}}%
\pgfusepath{clip}%
\pgfsetbuttcap%
\pgfsetroundjoin%
\definecolor{currentfill}{rgb}{0.127568,0.566949,0.550556}%
\pgfsetfillcolor{currentfill}%
\pgfsetfillopacity{0.700000}%
\pgfsetlinewidth{0.000000pt}%
\definecolor{currentstroke}{rgb}{0.000000,0.000000,0.000000}%
\pgfsetstrokecolor{currentstroke}%
\pgfsetstrokeopacity{0.700000}%
\pgfsetdash{}{0pt}%
\pgfpathmoveto{\pgfqpoint{8.002259in}{2.866203in}}%
\pgfpathcurveto{\pgfqpoint{8.007302in}{2.866203in}}{\pgfqpoint{8.012140in}{2.868207in}}{\pgfqpoint{8.015707in}{2.871773in}}%
\pgfpathcurveto{\pgfqpoint{8.019273in}{2.875340in}}{\pgfqpoint{8.021277in}{2.880178in}}{\pgfqpoint{8.021277in}{2.885221in}}%
\pgfpathcurveto{\pgfqpoint{8.021277in}{2.890265in}}{\pgfqpoint{8.019273in}{2.895103in}}{\pgfqpoint{8.015707in}{2.898669in}}%
\pgfpathcurveto{\pgfqpoint{8.012140in}{2.902235in}}{\pgfqpoint{8.007302in}{2.904239in}}{\pgfqpoint{8.002259in}{2.904239in}}%
\pgfpathcurveto{\pgfqpoint{7.997215in}{2.904239in}}{\pgfqpoint{7.992377in}{2.902235in}}{\pgfqpoint{7.988811in}{2.898669in}}%
\pgfpathcurveto{\pgfqpoint{7.985244in}{2.895103in}}{\pgfqpoint{7.983241in}{2.890265in}}{\pgfqpoint{7.983241in}{2.885221in}}%
\pgfpathcurveto{\pgfqpoint{7.983241in}{2.880178in}}{\pgfqpoint{7.985244in}{2.875340in}}{\pgfqpoint{7.988811in}{2.871773in}}%
\pgfpathcurveto{\pgfqpoint{7.992377in}{2.868207in}}{\pgfqpoint{7.997215in}{2.866203in}}{\pgfqpoint{8.002259in}{2.866203in}}%
\pgfpathclose%
\pgfusepath{fill}%
\end{pgfscope}%
\begin{pgfscope}%
\pgfpathrectangle{\pgfqpoint{6.572727in}{0.473000in}}{\pgfqpoint{4.227273in}{3.311000in}}%
\pgfusepath{clip}%
\pgfsetbuttcap%
\pgfsetroundjoin%
\definecolor{currentfill}{rgb}{0.127568,0.566949,0.550556}%
\pgfsetfillcolor{currentfill}%
\pgfsetfillopacity{0.700000}%
\pgfsetlinewidth{0.000000pt}%
\definecolor{currentstroke}{rgb}{0.000000,0.000000,0.000000}%
\pgfsetstrokecolor{currentstroke}%
\pgfsetstrokeopacity{0.700000}%
\pgfsetdash{}{0pt}%
\pgfpathmoveto{\pgfqpoint{7.437299in}{1.634578in}}%
\pgfpathcurveto{\pgfqpoint{7.442342in}{1.634578in}}{\pgfqpoint{7.447180in}{1.636582in}}{\pgfqpoint{7.450746in}{1.640148in}}%
\pgfpathcurveto{\pgfqpoint{7.454313in}{1.643715in}}{\pgfqpoint{7.456317in}{1.648553in}}{\pgfqpoint{7.456317in}{1.653596in}}%
\pgfpathcurveto{\pgfqpoint{7.456317in}{1.658640in}}{\pgfqpoint{7.454313in}{1.663478in}}{\pgfqpoint{7.450746in}{1.667044in}}%
\pgfpathcurveto{\pgfqpoint{7.447180in}{1.670610in}}{\pgfqpoint{7.442342in}{1.672614in}}{\pgfqpoint{7.437299in}{1.672614in}}%
\pgfpathcurveto{\pgfqpoint{7.432255in}{1.672614in}}{\pgfqpoint{7.427417in}{1.670610in}}{\pgfqpoint{7.423851in}{1.667044in}}%
\pgfpathcurveto{\pgfqpoint{7.420284in}{1.663478in}}{\pgfqpoint{7.418280in}{1.658640in}}{\pgfqpoint{7.418280in}{1.653596in}}%
\pgfpathcurveto{\pgfqpoint{7.418280in}{1.648553in}}{\pgfqpoint{7.420284in}{1.643715in}}{\pgfqpoint{7.423851in}{1.640148in}}%
\pgfpathcurveto{\pgfqpoint{7.427417in}{1.636582in}}{\pgfqpoint{7.432255in}{1.634578in}}{\pgfqpoint{7.437299in}{1.634578in}}%
\pgfpathclose%
\pgfusepath{fill}%
\end{pgfscope}%
\begin{pgfscope}%
\pgfpathrectangle{\pgfqpoint{6.572727in}{0.473000in}}{\pgfqpoint{4.227273in}{3.311000in}}%
\pgfusepath{clip}%
\pgfsetbuttcap%
\pgfsetroundjoin%
\definecolor{currentfill}{rgb}{0.993248,0.906157,0.143936}%
\pgfsetfillcolor{currentfill}%
\pgfsetfillopacity{0.700000}%
\pgfsetlinewidth{0.000000pt}%
\definecolor{currentstroke}{rgb}{0.000000,0.000000,0.000000}%
\pgfsetstrokecolor{currentstroke}%
\pgfsetstrokeopacity{0.700000}%
\pgfsetdash{}{0pt}%
\pgfpathmoveto{\pgfqpoint{9.747416in}{1.355533in}}%
\pgfpathcurveto{\pgfqpoint{9.752460in}{1.355533in}}{\pgfqpoint{9.757297in}{1.357537in}}{\pgfqpoint{9.760864in}{1.361104in}}%
\pgfpathcurveto{\pgfqpoint{9.764430in}{1.364670in}}{\pgfqpoint{9.766434in}{1.369508in}}{\pgfqpoint{9.766434in}{1.374552in}}%
\pgfpathcurveto{\pgfqpoint{9.766434in}{1.379595in}}{\pgfqpoint{9.764430in}{1.384433in}}{\pgfqpoint{9.760864in}{1.387999in}}%
\pgfpathcurveto{\pgfqpoint{9.757297in}{1.391566in}}{\pgfqpoint{9.752460in}{1.393570in}}{\pgfqpoint{9.747416in}{1.393570in}}%
\pgfpathcurveto{\pgfqpoint{9.742372in}{1.393570in}}{\pgfqpoint{9.737535in}{1.391566in}}{\pgfqpoint{9.733968in}{1.387999in}}%
\pgfpathcurveto{\pgfqpoint{9.730402in}{1.384433in}}{\pgfqpoint{9.728398in}{1.379595in}}{\pgfqpoint{9.728398in}{1.374552in}}%
\pgfpathcurveto{\pgfqpoint{9.728398in}{1.369508in}}{\pgfqpoint{9.730402in}{1.364670in}}{\pgfqpoint{9.733968in}{1.361104in}}%
\pgfpathcurveto{\pgfqpoint{9.737535in}{1.357537in}}{\pgfqpoint{9.742372in}{1.355533in}}{\pgfqpoint{9.747416in}{1.355533in}}%
\pgfpathclose%
\pgfusepath{fill}%
\end{pgfscope}%
\begin{pgfscope}%
\pgfpathrectangle{\pgfqpoint{6.572727in}{0.473000in}}{\pgfqpoint{4.227273in}{3.311000in}}%
\pgfusepath{clip}%
\pgfsetbuttcap%
\pgfsetroundjoin%
\definecolor{currentfill}{rgb}{0.127568,0.566949,0.550556}%
\pgfsetfillcolor{currentfill}%
\pgfsetfillopacity{0.700000}%
\pgfsetlinewidth{0.000000pt}%
\definecolor{currentstroke}{rgb}{0.000000,0.000000,0.000000}%
\pgfsetstrokecolor{currentstroke}%
\pgfsetstrokeopacity{0.700000}%
\pgfsetdash{}{0pt}%
\pgfpathmoveto{\pgfqpoint{8.350238in}{2.031014in}}%
\pgfpathcurveto{\pgfqpoint{8.355282in}{2.031014in}}{\pgfqpoint{8.360120in}{2.033018in}}{\pgfqpoint{8.363686in}{2.036584in}}%
\pgfpathcurveto{\pgfqpoint{8.367253in}{2.040151in}}{\pgfqpoint{8.369257in}{2.044988in}}{\pgfqpoint{8.369257in}{2.050032in}}%
\pgfpathcurveto{\pgfqpoint{8.369257in}{2.055076in}}{\pgfqpoint{8.367253in}{2.059913in}}{\pgfqpoint{8.363686in}{2.063480in}}%
\pgfpathcurveto{\pgfqpoint{8.360120in}{2.067046in}}{\pgfqpoint{8.355282in}{2.069050in}}{\pgfqpoint{8.350238in}{2.069050in}}%
\pgfpathcurveto{\pgfqpoint{8.345195in}{2.069050in}}{\pgfqpoint{8.340357in}{2.067046in}}{\pgfqpoint{8.336790in}{2.063480in}}%
\pgfpathcurveto{\pgfqpoint{8.333224in}{2.059913in}}{\pgfqpoint{8.331220in}{2.055076in}}{\pgfqpoint{8.331220in}{2.050032in}}%
\pgfpathcurveto{\pgfqpoint{8.331220in}{2.044988in}}{\pgfqpoint{8.333224in}{2.040151in}}{\pgfqpoint{8.336790in}{2.036584in}}%
\pgfpathcurveto{\pgfqpoint{8.340357in}{2.033018in}}{\pgfqpoint{8.345195in}{2.031014in}}{\pgfqpoint{8.350238in}{2.031014in}}%
\pgfpathclose%
\pgfusepath{fill}%
\end{pgfscope}%
\begin{pgfscope}%
\pgfpathrectangle{\pgfqpoint{6.572727in}{0.473000in}}{\pgfqpoint{4.227273in}{3.311000in}}%
\pgfusepath{clip}%
\pgfsetbuttcap%
\pgfsetroundjoin%
\definecolor{currentfill}{rgb}{0.267004,0.004874,0.329415}%
\pgfsetfillcolor{currentfill}%
\pgfsetfillopacity{0.700000}%
\pgfsetlinewidth{0.000000pt}%
\definecolor{currentstroke}{rgb}{0.000000,0.000000,0.000000}%
\pgfsetstrokecolor{currentstroke}%
\pgfsetstrokeopacity{0.700000}%
\pgfsetdash{}{0pt}%
\pgfpathmoveto{\pgfqpoint{10.537250in}{2.120194in}}%
\pgfpathcurveto{\pgfqpoint{10.542293in}{2.120194in}}{\pgfqpoint{10.547131in}{2.122198in}}{\pgfqpoint{10.550697in}{2.125765in}}%
\pgfpathcurveto{\pgfqpoint{10.554264in}{2.129331in}}{\pgfqpoint{10.556268in}{2.134169in}}{\pgfqpoint{10.556268in}{2.139212in}}%
\pgfpathcurveto{\pgfqpoint{10.556268in}{2.144256in}}{\pgfqpoint{10.554264in}{2.149094in}}{\pgfqpoint{10.550697in}{2.152660in}}%
\pgfpathcurveto{\pgfqpoint{10.547131in}{2.156227in}}{\pgfqpoint{10.542293in}{2.158231in}}{\pgfqpoint{10.537250in}{2.158231in}}%
\pgfpathcurveto{\pgfqpoint{10.532206in}{2.158231in}}{\pgfqpoint{10.527368in}{2.156227in}}{\pgfqpoint{10.523802in}{2.152660in}}%
\pgfpathcurveto{\pgfqpoint{10.520235in}{2.149094in}}{\pgfqpoint{10.518231in}{2.144256in}}{\pgfqpoint{10.518231in}{2.139212in}}%
\pgfpathcurveto{\pgfqpoint{10.518231in}{2.134169in}}{\pgfqpoint{10.520235in}{2.129331in}}{\pgfqpoint{10.523802in}{2.125765in}}%
\pgfpathcurveto{\pgfqpoint{10.527368in}{2.122198in}}{\pgfqpoint{10.532206in}{2.120194in}}{\pgfqpoint{10.537250in}{2.120194in}}%
\pgfpathclose%
\pgfusepath{fill}%
\end{pgfscope}%
\begin{pgfscope}%
\pgfpathrectangle{\pgfqpoint{6.572727in}{0.473000in}}{\pgfqpoint{4.227273in}{3.311000in}}%
\pgfusepath{clip}%
\pgfsetbuttcap%
\pgfsetroundjoin%
\definecolor{currentfill}{rgb}{0.993248,0.906157,0.143936}%
\pgfsetfillcolor{currentfill}%
\pgfsetfillopacity{0.700000}%
\pgfsetlinewidth{0.000000pt}%
\definecolor{currentstroke}{rgb}{0.000000,0.000000,0.000000}%
\pgfsetstrokecolor{currentstroke}%
\pgfsetstrokeopacity{0.700000}%
\pgfsetdash{}{0pt}%
\pgfpathmoveto{\pgfqpoint{9.727474in}{1.449294in}}%
\pgfpathcurveto{\pgfqpoint{9.732518in}{1.449294in}}{\pgfqpoint{9.737355in}{1.451298in}}{\pgfqpoint{9.740922in}{1.454864in}}%
\pgfpathcurveto{\pgfqpoint{9.744488in}{1.458431in}}{\pgfqpoint{9.746492in}{1.463269in}}{\pgfqpoint{9.746492in}{1.468312in}}%
\pgfpathcurveto{\pgfqpoint{9.746492in}{1.473356in}}{\pgfqpoint{9.744488in}{1.478194in}}{\pgfqpoint{9.740922in}{1.481760in}}%
\pgfpathcurveto{\pgfqpoint{9.737355in}{1.485326in}}{\pgfqpoint{9.732518in}{1.487330in}}{\pgfqpoint{9.727474in}{1.487330in}}%
\pgfpathcurveto{\pgfqpoint{9.722430in}{1.487330in}}{\pgfqpoint{9.717593in}{1.485326in}}{\pgfqpoint{9.714026in}{1.481760in}}%
\pgfpathcurveto{\pgfqpoint{9.710460in}{1.478194in}}{\pgfqpoint{9.708456in}{1.473356in}}{\pgfqpoint{9.708456in}{1.468312in}}%
\pgfpathcurveto{\pgfqpoint{9.708456in}{1.463269in}}{\pgfqpoint{9.710460in}{1.458431in}}{\pgfqpoint{9.714026in}{1.454864in}}%
\pgfpathcurveto{\pgfqpoint{9.717593in}{1.451298in}}{\pgfqpoint{9.722430in}{1.449294in}}{\pgfqpoint{9.727474in}{1.449294in}}%
\pgfpathclose%
\pgfusepath{fill}%
\end{pgfscope}%
\begin{pgfscope}%
\pgfpathrectangle{\pgfqpoint{6.572727in}{0.473000in}}{\pgfqpoint{4.227273in}{3.311000in}}%
\pgfusepath{clip}%
\pgfsetbuttcap%
\pgfsetroundjoin%
\definecolor{currentfill}{rgb}{0.127568,0.566949,0.550556}%
\pgfsetfillcolor{currentfill}%
\pgfsetfillopacity{0.700000}%
\pgfsetlinewidth{0.000000pt}%
\definecolor{currentstroke}{rgb}{0.000000,0.000000,0.000000}%
\pgfsetstrokecolor{currentstroke}%
\pgfsetstrokeopacity{0.700000}%
\pgfsetdash{}{0pt}%
\pgfpathmoveto{\pgfqpoint{8.011155in}{3.033686in}}%
\pgfpathcurveto{\pgfqpoint{8.016199in}{3.033686in}}{\pgfqpoint{8.021037in}{3.035690in}}{\pgfqpoint{8.024603in}{3.039257in}}%
\pgfpathcurveto{\pgfqpoint{8.028170in}{3.042823in}}{\pgfqpoint{8.030173in}{3.047661in}}{\pgfqpoint{8.030173in}{3.052705in}}%
\pgfpathcurveto{\pgfqpoint{8.030173in}{3.057748in}}{\pgfqpoint{8.028170in}{3.062586in}}{\pgfqpoint{8.024603in}{3.066152in}}%
\pgfpathcurveto{\pgfqpoint{8.021037in}{3.069719in}}{\pgfqpoint{8.016199in}{3.071723in}}{\pgfqpoint{8.011155in}{3.071723in}}%
\pgfpathcurveto{\pgfqpoint{8.006112in}{3.071723in}}{\pgfqpoint{8.001274in}{3.069719in}}{\pgfqpoint{7.997707in}{3.066152in}}%
\pgfpathcurveto{\pgfqpoint{7.994141in}{3.062586in}}{\pgfqpoint{7.992137in}{3.057748in}}{\pgfqpoint{7.992137in}{3.052705in}}%
\pgfpathcurveto{\pgfqpoint{7.992137in}{3.047661in}}{\pgfqpoint{7.994141in}{3.042823in}}{\pgfqpoint{7.997707in}{3.039257in}}%
\pgfpathcurveto{\pgfqpoint{8.001274in}{3.035690in}}{\pgfqpoint{8.006112in}{3.033686in}}{\pgfqpoint{8.011155in}{3.033686in}}%
\pgfpathclose%
\pgfusepath{fill}%
\end{pgfscope}%
\begin{pgfscope}%
\pgfpathrectangle{\pgfqpoint{6.572727in}{0.473000in}}{\pgfqpoint{4.227273in}{3.311000in}}%
\pgfusepath{clip}%
\pgfsetbuttcap%
\pgfsetroundjoin%
\definecolor{currentfill}{rgb}{0.993248,0.906157,0.143936}%
\pgfsetfillcolor{currentfill}%
\pgfsetfillopacity{0.700000}%
\pgfsetlinewidth{0.000000pt}%
\definecolor{currentstroke}{rgb}{0.000000,0.000000,0.000000}%
\pgfsetstrokecolor{currentstroke}%
\pgfsetstrokeopacity{0.700000}%
\pgfsetdash{}{0pt}%
\pgfpathmoveto{\pgfqpoint{10.145355in}{1.705477in}}%
\pgfpathcurveto{\pgfqpoint{10.150399in}{1.705477in}}{\pgfqpoint{10.155237in}{1.707481in}}{\pgfqpoint{10.158803in}{1.711047in}}%
\pgfpathcurveto{\pgfqpoint{10.162370in}{1.714614in}}{\pgfqpoint{10.164373in}{1.719452in}}{\pgfqpoint{10.164373in}{1.724495in}}%
\pgfpathcurveto{\pgfqpoint{10.164373in}{1.729539in}}{\pgfqpoint{10.162370in}{1.734377in}}{\pgfqpoint{10.158803in}{1.737943in}}%
\pgfpathcurveto{\pgfqpoint{10.155237in}{1.741509in}}{\pgfqpoint{10.150399in}{1.743513in}}{\pgfqpoint{10.145355in}{1.743513in}}%
\pgfpathcurveto{\pgfqpoint{10.140312in}{1.743513in}}{\pgfqpoint{10.135474in}{1.741509in}}{\pgfqpoint{10.131907in}{1.737943in}}%
\pgfpathcurveto{\pgfqpoint{10.128341in}{1.734377in}}{\pgfqpoint{10.126337in}{1.729539in}}{\pgfqpoint{10.126337in}{1.724495in}}%
\pgfpathcurveto{\pgfqpoint{10.126337in}{1.719452in}}{\pgfqpoint{10.128341in}{1.714614in}}{\pgfqpoint{10.131907in}{1.711047in}}%
\pgfpathcurveto{\pgfqpoint{10.135474in}{1.707481in}}{\pgfqpoint{10.140312in}{1.705477in}}{\pgfqpoint{10.145355in}{1.705477in}}%
\pgfpathclose%
\pgfusepath{fill}%
\end{pgfscope}%
\begin{pgfscope}%
\pgfpathrectangle{\pgfqpoint{6.572727in}{0.473000in}}{\pgfqpoint{4.227273in}{3.311000in}}%
\pgfusepath{clip}%
\pgfsetbuttcap%
\pgfsetroundjoin%
\definecolor{currentfill}{rgb}{0.267004,0.004874,0.329415}%
\pgfsetfillcolor{currentfill}%
\pgfsetfillopacity{0.700000}%
\pgfsetlinewidth{0.000000pt}%
\definecolor{currentstroke}{rgb}{0.000000,0.000000,0.000000}%
\pgfsetstrokecolor{currentstroke}%
\pgfsetstrokeopacity{0.700000}%
\pgfsetdash{}{0pt}%
\pgfpathmoveto{\pgfqpoint{7.498594in}{3.498070in}}%
\pgfpathcurveto{\pgfqpoint{7.503638in}{3.498070in}}{\pgfqpoint{7.508475in}{3.500073in}}{\pgfqpoint{7.512042in}{3.503640in}}%
\pgfpathcurveto{\pgfqpoint{7.515608in}{3.507206in}}{\pgfqpoint{7.517612in}{3.512044in}}{\pgfqpoint{7.517612in}{3.517088in}}%
\pgfpathcurveto{\pgfqpoint{7.517612in}{3.522131in}}{\pgfqpoint{7.515608in}{3.526969in}}{\pgfqpoint{7.512042in}{3.530536in}}%
\pgfpathcurveto{\pgfqpoint{7.508475in}{3.534102in}}{\pgfqpoint{7.503638in}{3.536106in}}{\pgfqpoint{7.498594in}{3.536106in}}%
\pgfpathcurveto{\pgfqpoint{7.493550in}{3.536106in}}{\pgfqpoint{7.488713in}{3.534102in}}{\pgfqpoint{7.485146in}{3.530536in}}%
\pgfpathcurveto{\pgfqpoint{7.481580in}{3.526969in}}{\pgfqpoint{7.479576in}{3.522131in}}{\pgfqpoint{7.479576in}{3.517088in}}%
\pgfpathcurveto{\pgfqpoint{7.479576in}{3.512044in}}{\pgfqpoint{7.481580in}{3.507206in}}{\pgfqpoint{7.485146in}{3.503640in}}%
\pgfpathcurveto{\pgfqpoint{7.488713in}{3.500073in}}{\pgfqpoint{7.493550in}{3.498070in}}{\pgfqpoint{7.498594in}{3.498070in}}%
\pgfpathclose%
\pgfusepath{fill}%
\end{pgfscope}%
\begin{pgfscope}%
\pgfpathrectangle{\pgfqpoint{6.572727in}{0.473000in}}{\pgfqpoint{4.227273in}{3.311000in}}%
\pgfusepath{clip}%
\pgfsetbuttcap%
\pgfsetroundjoin%
\definecolor{currentfill}{rgb}{0.993248,0.906157,0.143936}%
\pgfsetfillcolor{currentfill}%
\pgfsetfillopacity{0.700000}%
\pgfsetlinewidth{0.000000pt}%
\definecolor{currentstroke}{rgb}{0.000000,0.000000,0.000000}%
\pgfsetstrokecolor{currentstroke}%
\pgfsetstrokeopacity{0.700000}%
\pgfsetdash{}{0pt}%
\pgfpathmoveto{\pgfqpoint{9.475706in}{1.060224in}}%
\pgfpathcurveto{\pgfqpoint{9.480749in}{1.060224in}}{\pgfqpoint{9.485587in}{1.062228in}}{\pgfqpoint{9.489154in}{1.065794in}}%
\pgfpathcurveto{\pgfqpoint{9.492720in}{1.069361in}}{\pgfqpoint{9.494724in}{1.074199in}}{\pgfqpoint{9.494724in}{1.079242in}}%
\pgfpathcurveto{\pgfqpoint{9.494724in}{1.084286in}}{\pgfqpoint{9.492720in}{1.089124in}}{\pgfqpoint{9.489154in}{1.092690in}}%
\pgfpathcurveto{\pgfqpoint{9.485587in}{1.096257in}}{\pgfqpoint{9.480749in}{1.098260in}}{\pgfqpoint{9.475706in}{1.098260in}}%
\pgfpathcurveto{\pgfqpoint{9.470662in}{1.098260in}}{\pgfqpoint{9.465824in}{1.096257in}}{\pgfqpoint{9.462258in}{1.092690in}}%
\pgfpathcurveto{\pgfqpoint{9.458692in}{1.089124in}}{\pgfqpoint{9.456688in}{1.084286in}}{\pgfqpoint{9.456688in}{1.079242in}}%
\pgfpathcurveto{\pgfqpoint{9.456688in}{1.074199in}}{\pgfqpoint{9.458692in}{1.069361in}}{\pgfqpoint{9.462258in}{1.065794in}}%
\pgfpathcurveto{\pgfqpoint{9.465824in}{1.062228in}}{\pgfqpoint{9.470662in}{1.060224in}}{\pgfqpoint{9.475706in}{1.060224in}}%
\pgfpathclose%
\pgfusepath{fill}%
\end{pgfscope}%
\begin{pgfscope}%
\pgfpathrectangle{\pgfqpoint{6.572727in}{0.473000in}}{\pgfqpoint{4.227273in}{3.311000in}}%
\pgfusepath{clip}%
\pgfsetbuttcap%
\pgfsetroundjoin%
\definecolor{currentfill}{rgb}{0.993248,0.906157,0.143936}%
\pgfsetfillcolor{currentfill}%
\pgfsetfillopacity{0.700000}%
\pgfsetlinewidth{0.000000pt}%
\definecolor{currentstroke}{rgb}{0.000000,0.000000,0.000000}%
\pgfsetstrokecolor{currentstroke}%
\pgfsetstrokeopacity{0.700000}%
\pgfsetdash{}{0pt}%
\pgfpathmoveto{\pgfqpoint{8.989587in}{2.154175in}}%
\pgfpathcurveto{\pgfqpoint{8.994631in}{2.154175in}}{\pgfqpoint{8.999469in}{2.156179in}}{\pgfqpoint{9.003035in}{2.159745in}}%
\pgfpathcurveto{\pgfqpoint{9.006602in}{2.163312in}}{\pgfqpoint{9.008605in}{2.168150in}}{\pgfqpoint{9.008605in}{2.173193in}}%
\pgfpathcurveto{\pgfqpoint{9.008605in}{2.178237in}}{\pgfqpoint{9.006602in}{2.183075in}}{\pgfqpoint{9.003035in}{2.186641in}}%
\pgfpathcurveto{\pgfqpoint{8.999469in}{2.190207in}}{\pgfqpoint{8.994631in}{2.192211in}}{\pgfqpoint{8.989587in}{2.192211in}}%
\pgfpathcurveto{\pgfqpoint{8.984544in}{2.192211in}}{\pgfqpoint{8.979706in}{2.190207in}}{\pgfqpoint{8.976139in}{2.186641in}}%
\pgfpathcurveto{\pgfqpoint{8.972573in}{2.183075in}}{\pgfqpoint{8.970569in}{2.178237in}}{\pgfqpoint{8.970569in}{2.173193in}}%
\pgfpathcurveto{\pgfqpoint{8.970569in}{2.168150in}}{\pgfqpoint{8.972573in}{2.163312in}}{\pgfqpoint{8.976139in}{2.159745in}}%
\pgfpathcurveto{\pgfqpoint{8.979706in}{2.156179in}}{\pgfqpoint{8.984544in}{2.154175in}}{\pgfqpoint{8.989587in}{2.154175in}}%
\pgfpathclose%
\pgfusepath{fill}%
\end{pgfscope}%
\begin{pgfscope}%
\pgfpathrectangle{\pgfqpoint{6.572727in}{0.473000in}}{\pgfqpoint{4.227273in}{3.311000in}}%
\pgfusepath{clip}%
\pgfsetbuttcap%
\pgfsetroundjoin%
\definecolor{currentfill}{rgb}{0.267004,0.004874,0.329415}%
\pgfsetfillcolor{currentfill}%
\pgfsetfillopacity{0.700000}%
\pgfsetlinewidth{0.000000pt}%
\definecolor{currentstroke}{rgb}{0.000000,0.000000,0.000000}%
\pgfsetstrokecolor{currentstroke}%
\pgfsetstrokeopacity{0.700000}%
\pgfsetdash{}{0pt}%
\pgfpathmoveto{\pgfqpoint{7.544241in}{0.604482in}}%
\pgfpathcurveto{\pgfqpoint{7.549285in}{0.604482in}}{\pgfqpoint{7.554123in}{0.606486in}}{\pgfqpoint{7.557689in}{0.610052in}}%
\pgfpathcurveto{\pgfqpoint{7.561256in}{0.613619in}}{\pgfqpoint{7.563260in}{0.618456in}}{\pgfqpoint{7.563260in}{0.623500in}}%
\pgfpathcurveto{\pgfqpoint{7.563260in}{0.628544in}}{\pgfqpoint{7.561256in}{0.633381in}}{\pgfqpoint{7.557689in}{0.636948in}}%
\pgfpathcurveto{\pgfqpoint{7.554123in}{0.640514in}}{\pgfqpoint{7.549285in}{0.642518in}}{\pgfqpoint{7.544241in}{0.642518in}}%
\pgfpathcurveto{\pgfqpoint{7.539198in}{0.642518in}}{\pgfqpoint{7.534360in}{0.640514in}}{\pgfqpoint{7.530794in}{0.636948in}}%
\pgfpathcurveto{\pgfqpoint{7.527227in}{0.633381in}}{\pgfqpoint{7.525223in}{0.628544in}}{\pgfqpoint{7.525223in}{0.623500in}}%
\pgfpathcurveto{\pgfqpoint{7.525223in}{0.618456in}}{\pgfqpoint{7.527227in}{0.613619in}}{\pgfqpoint{7.530794in}{0.610052in}}%
\pgfpathcurveto{\pgfqpoint{7.534360in}{0.606486in}}{\pgfqpoint{7.539198in}{0.604482in}}{\pgfqpoint{7.544241in}{0.604482in}}%
\pgfpathclose%
\pgfusepath{fill}%
\end{pgfscope}%
\begin{pgfscope}%
\pgfpathrectangle{\pgfqpoint{6.572727in}{0.473000in}}{\pgfqpoint{4.227273in}{3.311000in}}%
\pgfusepath{clip}%
\pgfsetbuttcap%
\pgfsetroundjoin%
\definecolor{currentfill}{rgb}{0.127568,0.566949,0.550556}%
\pgfsetfillcolor{currentfill}%
\pgfsetfillopacity{0.700000}%
\pgfsetlinewidth{0.000000pt}%
\definecolor{currentstroke}{rgb}{0.000000,0.000000,0.000000}%
\pgfsetstrokecolor{currentstroke}%
\pgfsetstrokeopacity{0.700000}%
\pgfsetdash{}{0pt}%
\pgfpathmoveto{\pgfqpoint{8.273392in}{1.518914in}}%
\pgfpathcurveto{\pgfqpoint{8.278436in}{1.518914in}}{\pgfqpoint{8.283274in}{1.520918in}}{\pgfqpoint{8.286840in}{1.524484in}}%
\pgfpathcurveto{\pgfqpoint{8.290407in}{1.528051in}}{\pgfqpoint{8.292411in}{1.532888in}}{\pgfqpoint{8.292411in}{1.537932in}}%
\pgfpathcurveto{\pgfqpoint{8.292411in}{1.542976in}}{\pgfqpoint{8.290407in}{1.547813in}}{\pgfqpoint{8.286840in}{1.551380in}}%
\pgfpathcurveto{\pgfqpoint{8.283274in}{1.554946in}}{\pgfqpoint{8.278436in}{1.556950in}}{\pgfqpoint{8.273392in}{1.556950in}}%
\pgfpathcurveto{\pgfqpoint{8.268349in}{1.556950in}}{\pgfqpoint{8.263511in}{1.554946in}}{\pgfqpoint{8.259945in}{1.551380in}}%
\pgfpathcurveto{\pgfqpoint{8.256378in}{1.547813in}}{\pgfqpoint{8.254374in}{1.542976in}}{\pgfqpoint{8.254374in}{1.537932in}}%
\pgfpathcurveto{\pgfqpoint{8.254374in}{1.532888in}}{\pgfqpoint{8.256378in}{1.528051in}}{\pgfqpoint{8.259945in}{1.524484in}}%
\pgfpathcurveto{\pgfqpoint{8.263511in}{1.520918in}}{\pgfqpoint{8.268349in}{1.518914in}}{\pgfqpoint{8.273392in}{1.518914in}}%
\pgfpathclose%
\pgfusepath{fill}%
\end{pgfscope}%
\begin{pgfscope}%
\pgfpathrectangle{\pgfqpoint{6.572727in}{0.473000in}}{\pgfqpoint{4.227273in}{3.311000in}}%
\pgfusepath{clip}%
\pgfsetbuttcap%
\pgfsetroundjoin%
\definecolor{currentfill}{rgb}{0.127568,0.566949,0.550556}%
\pgfsetfillcolor{currentfill}%
\pgfsetfillopacity{0.700000}%
\pgfsetlinewidth{0.000000pt}%
\definecolor{currentstroke}{rgb}{0.000000,0.000000,0.000000}%
\pgfsetstrokecolor{currentstroke}%
\pgfsetstrokeopacity{0.700000}%
\pgfsetdash{}{0pt}%
\pgfpathmoveto{\pgfqpoint{7.494979in}{1.249256in}}%
\pgfpathcurveto{\pgfqpoint{7.500022in}{1.249256in}}{\pgfqpoint{7.504860in}{1.251260in}}{\pgfqpoint{7.508427in}{1.254826in}}%
\pgfpathcurveto{\pgfqpoint{7.511993in}{1.258392in}}{\pgfqpoint{7.513997in}{1.263230in}}{\pgfqpoint{7.513997in}{1.268274in}}%
\pgfpathcurveto{\pgfqpoint{7.513997in}{1.273318in}}{\pgfqpoint{7.511993in}{1.278155in}}{\pgfqpoint{7.508427in}{1.281722in}}%
\pgfpathcurveto{\pgfqpoint{7.504860in}{1.285288in}}{\pgfqpoint{7.500022in}{1.287292in}}{\pgfqpoint{7.494979in}{1.287292in}}%
\pgfpathcurveto{\pgfqpoint{7.489935in}{1.287292in}}{\pgfqpoint{7.485097in}{1.285288in}}{\pgfqpoint{7.481531in}{1.281722in}}%
\pgfpathcurveto{\pgfqpoint{7.477964in}{1.278155in}}{\pgfqpoint{7.475960in}{1.273318in}}{\pgfqpoint{7.475960in}{1.268274in}}%
\pgfpathcurveto{\pgfqpoint{7.475960in}{1.263230in}}{\pgfqpoint{7.477964in}{1.258392in}}{\pgfqpoint{7.481531in}{1.254826in}}%
\pgfpathcurveto{\pgfqpoint{7.485097in}{1.251260in}}{\pgfqpoint{7.489935in}{1.249256in}}{\pgfqpoint{7.494979in}{1.249256in}}%
\pgfpathclose%
\pgfusepath{fill}%
\end{pgfscope}%
\begin{pgfscope}%
\pgfpathrectangle{\pgfqpoint{6.572727in}{0.473000in}}{\pgfqpoint{4.227273in}{3.311000in}}%
\pgfusepath{clip}%
\pgfsetbuttcap%
\pgfsetroundjoin%
\definecolor{currentfill}{rgb}{0.127568,0.566949,0.550556}%
\pgfsetfillcolor{currentfill}%
\pgfsetfillopacity{0.700000}%
\pgfsetlinewidth{0.000000pt}%
\definecolor{currentstroke}{rgb}{0.000000,0.000000,0.000000}%
\pgfsetstrokecolor{currentstroke}%
\pgfsetstrokeopacity{0.700000}%
\pgfsetdash{}{0pt}%
\pgfpathmoveto{\pgfqpoint{7.871762in}{2.266122in}}%
\pgfpathcurveto{\pgfqpoint{7.876806in}{2.266122in}}{\pgfqpoint{7.881643in}{2.268126in}}{\pgfqpoint{7.885210in}{2.271692in}}%
\pgfpathcurveto{\pgfqpoint{7.888776in}{2.275259in}}{\pgfqpoint{7.890780in}{2.280097in}}{\pgfqpoint{7.890780in}{2.285140in}}%
\pgfpathcurveto{\pgfqpoint{7.890780in}{2.290184in}}{\pgfqpoint{7.888776in}{2.295022in}}{\pgfqpoint{7.885210in}{2.298588in}}%
\pgfpathcurveto{\pgfqpoint{7.881643in}{2.302155in}}{\pgfqpoint{7.876806in}{2.304158in}}{\pgfqpoint{7.871762in}{2.304158in}}%
\pgfpathcurveto{\pgfqpoint{7.866718in}{2.304158in}}{\pgfqpoint{7.861881in}{2.302155in}}{\pgfqpoint{7.858314in}{2.298588in}}%
\pgfpathcurveto{\pgfqpoint{7.854748in}{2.295022in}}{\pgfqpoint{7.852744in}{2.290184in}}{\pgfqpoint{7.852744in}{2.285140in}}%
\pgfpathcurveto{\pgfqpoint{7.852744in}{2.280097in}}{\pgfqpoint{7.854748in}{2.275259in}}{\pgfqpoint{7.858314in}{2.271692in}}%
\pgfpathcurveto{\pgfqpoint{7.861881in}{2.268126in}}{\pgfqpoint{7.866718in}{2.266122in}}{\pgfqpoint{7.871762in}{2.266122in}}%
\pgfpathclose%
\pgfusepath{fill}%
\end{pgfscope}%
\begin{pgfscope}%
\pgfpathrectangle{\pgfqpoint{6.572727in}{0.473000in}}{\pgfqpoint{4.227273in}{3.311000in}}%
\pgfusepath{clip}%
\pgfsetbuttcap%
\pgfsetroundjoin%
\definecolor{currentfill}{rgb}{0.993248,0.906157,0.143936}%
\pgfsetfillcolor{currentfill}%
\pgfsetfillopacity{0.700000}%
\pgfsetlinewidth{0.000000pt}%
\definecolor{currentstroke}{rgb}{0.000000,0.000000,0.000000}%
\pgfsetstrokecolor{currentstroke}%
\pgfsetstrokeopacity{0.700000}%
\pgfsetdash{}{0pt}%
\pgfpathmoveto{\pgfqpoint{9.449838in}{1.644064in}}%
\pgfpathcurveto{\pgfqpoint{9.454882in}{1.644064in}}{\pgfqpoint{9.459720in}{1.646068in}}{\pgfqpoint{9.463286in}{1.649634in}}%
\pgfpathcurveto{\pgfqpoint{9.466852in}{1.653200in}}{\pgfqpoint{9.468856in}{1.658038in}}{\pgfqpoint{9.468856in}{1.663082in}}%
\pgfpathcurveto{\pgfqpoint{9.468856in}{1.668125in}}{\pgfqpoint{9.466852in}{1.672963in}}{\pgfqpoint{9.463286in}{1.676530in}}%
\pgfpathcurveto{\pgfqpoint{9.459720in}{1.680096in}}{\pgfqpoint{9.454882in}{1.682100in}}{\pgfqpoint{9.449838in}{1.682100in}}%
\pgfpathcurveto{\pgfqpoint{9.444795in}{1.682100in}}{\pgfqpoint{9.439957in}{1.680096in}}{\pgfqpoint{9.436390in}{1.676530in}}%
\pgfpathcurveto{\pgfqpoint{9.432824in}{1.672963in}}{\pgfqpoint{9.430820in}{1.668125in}}{\pgfqpoint{9.430820in}{1.663082in}}%
\pgfpathcurveto{\pgfqpoint{9.430820in}{1.658038in}}{\pgfqpoint{9.432824in}{1.653200in}}{\pgfqpoint{9.436390in}{1.649634in}}%
\pgfpathcurveto{\pgfqpoint{9.439957in}{1.646068in}}{\pgfqpoint{9.444795in}{1.644064in}}{\pgfqpoint{9.449838in}{1.644064in}}%
\pgfpathclose%
\pgfusepath{fill}%
\end{pgfscope}%
\begin{pgfscope}%
\pgfpathrectangle{\pgfqpoint{6.572727in}{0.473000in}}{\pgfqpoint{4.227273in}{3.311000in}}%
\pgfusepath{clip}%
\pgfsetbuttcap%
\pgfsetroundjoin%
\definecolor{currentfill}{rgb}{0.993248,0.906157,0.143936}%
\pgfsetfillcolor{currentfill}%
\pgfsetfillopacity{0.700000}%
\pgfsetlinewidth{0.000000pt}%
\definecolor{currentstroke}{rgb}{0.000000,0.000000,0.000000}%
\pgfsetstrokecolor{currentstroke}%
\pgfsetstrokeopacity{0.700000}%
\pgfsetdash{}{0pt}%
\pgfpathmoveto{\pgfqpoint{9.454653in}{1.677190in}}%
\pgfpathcurveto{\pgfqpoint{9.459696in}{1.677190in}}{\pgfqpoint{9.464534in}{1.679193in}}{\pgfqpoint{9.468101in}{1.682760in}}%
\pgfpathcurveto{\pgfqpoint{9.471667in}{1.686326in}}{\pgfqpoint{9.473671in}{1.691164in}}{\pgfqpoint{9.473671in}{1.696208in}}%
\pgfpathcurveto{\pgfqpoint{9.473671in}{1.701251in}}{\pgfqpoint{9.471667in}{1.706089in}}{\pgfqpoint{9.468101in}{1.709656in}}%
\pgfpathcurveto{\pgfqpoint{9.464534in}{1.713222in}}{\pgfqpoint{9.459696in}{1.715226in}}{\pgfqpoint{9.454653in}{1.715226in}}%
\pgfpathcurveto{\pgfqpoint{9.449609in}{1.715226in}}{\pgfqpoint{9.444771in}{1.713222in}}{\pgfqpoint{9.441205in}{1.709656in}}%
\pgfpathcurveto{\pgfqpoint{9.437639in}{1.706089in}}{\pgfqpoint{9.435635in}{1.701251in}}{\pgfqpoint{9.435635in}{1.696208in}}%
\pgfpathcurveto{\pgfqpoint{9.435635in}{1.691164in}}{\pgfqpoint{9.437639in}{1.686326in}}{\pgfqpoint{9.441205in}{1.682760in}}%
\pgfpathcurveto{\pgfqpoint{9.444771in}{1.679193in}}{\pgfqpoint{9.449609in}{1.677190in}}{\pgfqpoint{9.454653in}{1.677190in}}%
\pgfpathclose%
\pgfusepath{fill}%
\end{pgfscope}%
\begin{pgfscope}%
\pgfpathrectangle{\pgfqpoint{6.572727in}{0.473000in}}{\pgfqpoint{4.227273in}{3.311000in}}%
\pgfusepath{clip}%
\pgfsetbuttcap%
\pgfsetroundjoin%
\definecolor{currentfill}{rgb}{0.127568,0.566949,0.550556}%
\pgfsetfillcolor{currentfill}%
\pgfsetfillopacity{0.700000}%
\pgfsetlinewidth{0.000000pt}%
\definecolor{currentstroke}{rgb}{0.000000,0.000000,0.000000}%
\pgfsetstrokecolor{currentstroke}%
\pgfsetstrokeopacity{0.700000}%
\pgfsetdash{}{0pt}%
\pgfpathmoveto{\pgfqpoint{8.102479in}{2.706245in}}%
\pgfpathcurveto{\pgfqpoint{8.107522in}{2.706245in}}{\pgfqpoint{8.112360in}{2.708249in}}{\pgfqpoint{8.115926in}{2.711815in}}%
\pgfpathcurveto{\pgfqpoint{8.119493in}{2.715381in}}{\pgfqpoint{8.121497in}{2.720219in}}{\pgfqpoint{8.121497in}{2.725263in}}%
\pgfpathcurveto{\pgfqpoint{8.121497in}{2.730307in}}{\pgfqpoint{8.119493in}{2.735144in}}{\pgfqpoint{8.115926in}{2.738711in}}%
\pgfpathcurveto{\pgfqpoint{8.112360in}{2.742277in}}{\pgfqpoint{8.107522in}{2.744281in}}{\pgfqpoint{8.102479in}{2.744281in}}%
\pgfpathcurveto{\pgfqpoint{8.097435in}{2.744281in}}{\pgfqpoint{8.092597in}{2.742277in}}{\pgfqpoint{8.089031in}{2.738711in}}%
\pgfpathcurveto{\pgfqpoint{8.085464in}{2.735144in}}{\pgfqpoint{8.083460in}{2.730307in}}{\pgfqpoint{8.083460in}{2.725263in}}%
\pgfpathcurveto{\pgfqpoint{8.083460in}{2.720219in}}{\pgfqpoint{8.085464in}{2.715381in}}{\pgfqpoint{8.089031in}{2.711815in}}%
\pgfpathcurveto{\pgfqpoint{8.092597in}{2.708249in}}{\pgfqpoint{8.097435in}{2.706245in}}{\pgfqpoint{8.102479in}{2.706245in}}%
\pgfpathclose%
\pgfusepath{fill}%
\end{pgfscope}%
\begin{pgfscope}%
\pgfpathrectangle{\pgfqpoint{6.572727in}{0.473000in}}{\pgfqpoint{4.227273in}{3.311000in}}%
\pgfusepath{clip}%
\pgfsetbuttcap%
\pgfsetroundjoin%
\definecolor{currentfill}{rgb}{0.127568,0.566949,0.550556}%
\pgfsetfillcolor{currentfill}%
\pgfsetfillopacity{0.700000}%
\pgfsetlinewidth{0.000000pt}%
\definecolor{currentstroke}{rgb}{0.000000,0.000000,0.000000}%
\pgfsetstrokecolor{currentstroke}%
\pgfsetstrokeopacity{0.700000}%
\pgfsetdash{}{0pt}%
\pgfpathmoveto{\pgfqpoint{8.361357in}{2.981364in}}%
\pgfpathcurveto{\pgfqpoint{8.366400in}{2.981364in}}{\pgfqpoint{8.371238in}{2.983368in}}{\pgfqpoint{8.374804in}{2.986935in}}%
\pgfpathcurveto{\pgfqpoint{8.378371in}{2.990501in}}{\pgfqpoint{8.380375in}{2.995339in}}{\pgfqpoint{8.380375in}{3.000383in}}%
\pgfpathcurveto{\pgfqpoint{8.380375in}{3.005426in}}{\pgfqpoint{8.378371in}{3.010264in}}{\pgfqpoint{8.374804in}{3.013830in}}%
\pgfpathcurveto{\pgfqpoint{8.371238in}{3.017397in}}{\pgfqpoint{8.366400in}{3.019401in}}{\pgfqpoint{8.361357in}{3.019401in}}%
\pgfpathcurveto{\pgfqpoint{8.356313in}{3.019401in}}{\pgfqpoint{8.351475in}{3.017397in}}{\pgfqpoint{8.347909in}{3.013830in}}%
\pgfpathcurveto{\pgfqpoint{8.344342in}{3.010264in}}{\pgfqpoint{8.342338in}{3.005426in}}{\pgfqpoint{8.342338in}{3.000383in}}%
\pgfpathcurveto{\pgfqpoint{8.342338in}{2.995339in}}{\pgfqpoint{8.344342in}{2.990501in}}{\pgfqpoint{8.347909in}{2.986935in}}%
\pgfpathcurveto{\pgfqpoint{8.351475in}{2.983368in}}{\pgfqpoint{8.356313in}{2.981364in}}{\pgfqpoint{8.361357in}{2.981364in}}%
\pgfpathclose%
\pgfusepath{fill}%
\end{pgfscope}%
\begin{pgfscope}%
\pgfpathrectangle{\pgfqpoint{6.572727in}{0.473000in}}{\pgfqpoint{4.227273in}{3.311000in}}%
\pgfusepath{clip}%
\pgfsetbuttcap%
\pgfsetroundjoin%
\definecolor{currentfill}{rgb}{0.993248,0.906157,0.143936}%
\pgfsetfillcolor{currentfill}%
\pgfsetfillopacity{0.700000}%
\pgfsetlinewidth{0.000000pt}%
\definecolor{currentstroke}{rgb}{0.000000,0.000000,0.000000}%
\pgfsetstrokecolor{currentstroke}%
\pgfsetstrokeopacity{0.700000}%
\pgfsetdash{}{0pt}%
\pgfpathmoveto{\pgfqpoint{9.025197in}{1.507692in}}%
\pgfpathcurveto{\pgfqpoint{9.030241in}{1.507692in}}{\pgfqpoint{9.035079in}{1.509696in}}{\pgfqpoint{9.038645in}{1.513263in}}%
\pgfpathcurveto{\pgfqpoint{9.042212in}{1.516829in}}{\pgfqpoint{9.044216in}{1.521667in}}{\pgfqpoint{9.044216in}{1.526711in}}%
\pgfpathcurveto{\pgfqpoint{9.044216in}{1.531754in}}{\pgfqpoint{9.042212in}{1.536592in}}{\pgfqpoint{9.038645in}{1.540158in}}%
\pgfpathcurveto{\pgfqpoint{9.035079in}{1.543725in}}{\pgfqpoint{9.030241in}{1.545729in}}{\pgfqpoint{9.025197in}{1.545729in}}%
\pgfpathcurveto{\pgfqpoint{9.020154in}{1.545729in}}{\pgfqpoint{9.015316in}{1.543725in}}{\pgfqpoint{9.011750in}{1.540158in}}%
\pgfpathcurveto{\pgfqpoint{9.008183in}{1.536592in}}{\pgfqpoint{9.006179in}{1.531754in}}{\pgfqpoint{9.006179in}{1.526711in}}%
\pgfpathcurveto{\pgfqpoint{9.006179in}{1.521667in}}{\pgfqpoint{9.008183in}{1.516829in}}{\pgfqpoint{9.011750in}{1.513263in}}%
\pgfpathcurveto{\pgfqpoint{9.015316in}{1.509696in}}{\pgfqpoint{9.020154in}{1.507692in}}{\pgfqpoint{9.025197in}{1.507692in}}%
\pgfpathclose%
\pgfusepath{fill}%
\end{pgfscope}%
\begin{pgfscope}%
\pgfpathrectangle{\pgfqpoint{6.572727in}{0.473000in}}{\pgfqpoint{4.227273in}{3.311000in}}%
\pgfusepath{clip}%
\pgfsetbuttcap%
\pgfsetroundjoin%
\definecolor{currentfill}{rgb}{0.127568,0.566949,0.550556}%
\pgfsetfillcolor{currentfill}%
\pgfsetfillopacity{0.700000}%
\pgfsetlinewidth{0.000000pt}%
\definecolor{currentstroke}{rgb}{0.000000,0.000000,0.000000}%
\pgfsetstrokecolor{currentstroke}%
\pgfsetstrokeopacity{0.700000}%
\pgfsetdash{}{0pt}%
\pgfpathmoveto{\pgfqpoint{8.583198in}{2.914014in}}%
\pgfpathcurveto{\pgfqpoint{8.588241in}{2.914014in}}{\pgfqpoint{8.593079in}{2.916018in}}{\pgfqpoint{8.596645in}{2.919584in}}%
\pgfpathcurveto{\pgfqpoint{8.600212in}{2.923151in}}{\pgfqpoint{8.602216in}{2.927988in}}{\pgfqpoint{8.602216in}{2.933032in}}%
\pgfpathcurveto{\pgfqpoint{8.602216in}{2.938076in}}{\pgfqpoint{8.600212in}{2.942914in}}{\pgfqpoint{8.596645in}{2.946480in}}%
\pgfpathcurveto{\pgfqpoint{8.593079in}{2.950046in}}{\pgfqpoint{8.588241in}{2.952050in}}{\pgfqpoint{8.583198in}{2.952050in}}%
\pgfpathcurveto{\pgfqpoint{8.578154in}{2.952050in}}{\pgfqpoint{8.573316in}{2.950046in}}{\pgfqpoint{8.569750in}{2.946480in}}%
\pgfpathcurveto{\pgfqpoint{8.566183in}{2.942914in}}{\pgfqpoint{8.564179in}{2.938076in}}{\pgfqpoint{8.564179in}{2.933032in}}%
\pgfpathcurveto{\pgfqpoint{8.564179in}{2.927988in}}{\pgfqpoint{8.566183in}{2.923151in}}{\pgfqpoint{8.569750in}{2.919584in}}%
\pgfpathcurveto{\pgfqpoint{8.573316in}{2.916018in}}{\pgfqpoint{8.578154in}{2.914014in}}{\pgfqpoint{8.583198in}{2.914014in}}%
\pgfpathclose%
\pgfusepath{fill}%
\end{pgfscope}%
\begin{pgfscope}%
\pgfpathrectangle{\pgfqpoint{6.572727in}{0.473000in}}{\pgfqpoint{4.227273in}{3.311000in}}%
\pgfusepath{clip}%
\pgfsetbuttcap%
\pgfsetroundjoin%
\definecolor{currentfill}{rgb}{0.127568,0.566949,0.550556}%
\pgfsetfillcolor{currentfill}%
\pgfsetfillopacity{0.700000}%
\pgfsetlinewidth{0.000000pt}%
\definecolor{currentstroke}{rgb}{0.000000,0.000000,0.000000}%
\pgfsetstrokecolor{currentstroke}%
\pgfsetstrokeopacity{0.700000}%
\pgfsetdash{}{0pt}%
\pgfpathmoveto{\pgfqpoint{8.106954in}{2.397151in}}%
\pgfpathcurveto{\pgfqpoint{8.111998in}{2.397151in}}{\pgfqpoint{8.116836in}{2.399155in}}{\pgfqpoint{8.120402in}{2.402721in}}%
\pgfpathcurveto{\pgfqpoint{8.123969in}{2.406288in}}{\pgfqpoint{8.125972in}{2.411125in}}{\pgfqpoint{8.125972in}{2.416169in}}%
\pgfpathcurveto{\pgfqpoint{8.125972in}{2.421213in}}{\pgfqpoint{8.123969in}{2.426051in}}{\pgfqpoint{8.120402in}{2.429617in}}%
\pgfpathcurveto{\pgfqpoint{8.116836in}{2.433183in}}{\pgfqpoint{8.111998in}{2.435187in}}{\pgfqpoint{8.106954in}{2.435187in}}%
\pgfpathcurveto{\pgfqpoint{8.101911in}{2.435187in}}{\pgfqpoint{8.097073in}{2.433183in}}{\pgfqpoint{8.093506in}{2.429617in}}%
\pgfpathcurveto{\pgfqpoint{8.089940in}{2.426051in}}{\pgfqpoint{8.087936in}{2.421213in}}{\pgfqpoint{8.087936in}{2.416169in}}%
\pgfpathcurveto{\pgfqpoint{8.087936in}{2.411125in}}{\pgfqpoint{8.089940in}{2.406288in}}{\pgfqpoint{8.093506in}{2.402721in}}%
\pgfpathcurveto{\pgfqpoint{8.097073in}{2.399155in}}{\pgfqpoint{8.101911in}{2.397151in}}{\pgfqpoint{8.106954in}{2.397151in}}%
\pgfpathclose%
\pgfusepath{fill}%
\end{pgfscope}%
\begin{pgfscope}%
\pgfpathrectangle{\pgfqpoint{6.572727in}{0.473000in}}{\pgfqpoint{4.227273in}{3.311000in}}%
\pgfusepath{clip}%
\pgfsetbuttcap%
\pgfsetroundjoin%
\definecolor{currentfill}{rgb}{0.127568,0.566949,0.550556}%
\pgfsetfillcolor{currentfill}%
\pgfsetfillopacity{0.700000}%
\pgfsetlinewidth{0.000000pt}%
\definecolor{currentstroke}{rgb}{0.000000,0.000000,0.000000}%
\pgfsetstrokecolor{currentstroke}%
\pgfsetstrokeopacity{0.700000}%
\pgfsetdash{}{0pt}%
\pgfpathmoveto{\pgfqpoint{8.002811in}{2.635342in}}%
\pgfpathcurveto{\pgfqpoint{8.007855in}{2.635342in}}{\pgfqpoint{8.012693in}{2.637346in}}{\pgfqpoint{8.016259in}{2.640912in}}%
\pgfpathcurveto{\pgfqpoint{8.019825in}{2.644479in}}{\pgfqpoint{8.021829in}{2.649316in}}{\pgfqpoint{8.021829in}{2.654360in}}%
\pgfpathcurveto{\pgfqpoint{8.021829in}{2.659404in}}{\pgfqpoint{8.019825in}{2.664242in}}{\pgfqpoint{8.016259in}{2.667808in}}%
\pgfpathcurveto{\pgfqpoint{8.012693in}{2.671374in}}{\pgfqpoint{8.007855in}{2.673378in}}{\pgfqpoint{8.002811in}{2.673378in}}%
\pgfpathcurveto{\pgfqpoint{7.997767in}{2.673378in}}{\pgfqpoint{7.992930in}{2.671374in}}{\pgfqpoint{7.989363in}{2.667808in}}%
\pgfpathcurveto{\pgfqpoint{7.985797in}{2.664242in}}{\pgfqpoint{7.983793in}{2.659404in}}{\pgfqpoint{7.983793in}{2.654360in}}%
\pgfpathcurveto{\pgfqpoint{7.983793in}{2.649316in}}{\pgfqpoint{7.985797in}{2.644479in}}{\pgfqpoint{7.989363in}{2.640912in}}%
\pgfpathcurveto{\pgfqpoint{7.992930in}{2.637346in}}{\pgfqpoint{7.997767in}{2.635342in}}{\pgfqpoint{8.002811in}{2.635342in}}%
\pgfpathclose%
\pgfusepath{fill}%
\end{pgfscope}%
\begin{pgfscope}%
\pgfpathrectangle{\pgfqpoint{6.572727in}{0.473000in}}{\pgfqpoint{4.227273in}{3.311000in}}%
\pgfusepath{clip}%
\pgfsetbuttcap%
\pgfsetroundjoin%
\definecolor{currentfill}{rgb}{0.127568,0.566949,0.550556}%
\pgfsetfillcolor{currentfill}%
\pgfsetfillopacity{0.700000}%
\pgfsetlinewidth{0.000000pt}%
\definecolor{currentstroke}{rgb}{0.000000,0.000000,0.000000}%
\pgfsetstrokecolor{currentstroke}%
\pgfsetstrokeopacity{0.700000}%
\pgfsetdash{}{0pt}%
\pgfpathmoveto{\pgfqpoint{7.933425in}{0.958665in}}%
\pgfpathcurveto{\pgfqpoint{7.938469in}{0.958665in}}{\pgfqpoint{7.943306in}{0.960669in}}{\pgfqpoint{7.946873in}{0.964235in}}%
\pgfpathcurveto{\pgfqpoint{7.950439in}{0.967801in}}{\pgfqpoint{7.952443in}{0.972639in}}{\pgfqpoint{7.952443in}{0.977683in}}%
\pgfpathcurveto{\pgfqpoint{7.952443in}{0.982727in}}{\pgfqpoint{7.950439in}{0.987564in}}{\pgfqpoint{7.946873in}{0.991131in}}%
\pgfpathcurveto{\pgfqpoint{7.943306in}{0.994697in}}{\pgfqpoint{7.938469in}{0.996701in}}{\pgfqpoint{7.933425in}{0.996701in}}%
\pgfpathcurveto{\pgfqpoint{7.928381in}{0.996701in}}{\pgfqpoint{7.923543in}{0.994697in}}{\pgfqpoint{7.919977in}{0.991131in}}%
\pgfpathcurveto{\pgfqpoint{7.916411in}{0.987564in}}{\pgfqpoint{7.914407in}{0.982727in}}{\pgfqpoint{7.914407in}{0.977683in}}%
\pgfpathcurveto{\pgfqpoint{7.914407in}{0.972639in}}{\pgfqpoint{7.916411in}{0.967801in}}{\pgfqpoint{7.919977in}{0.964235in}}%
\pgfpathcurveto{\pgfqpoint{7.923543in}{0.960669in}}{\pgfqpoint{7.928381in}{0.958665in}}{\pgfqpoint{7.933425in}{0.958665in}}%
\pgfpathclose%
\pgfusepath{fill}%
\end{pgfscope}%
\begin{pgfscope}%
\pgfpathrectangle{\pgfqpoint{6.572727in}{0.473000in}}{\pgfqpoint{4.227273in}{3.311000in}}%
\pgfusepath{clip}%
\pgfsetbuttcap%
\pgfsetroundjoin%
\definecolor{currentfill}{rgb}{0.127568,0.566949,0.550556}%
\pgfsetfillcolor{currentfill}%
\pgfsetfillopacity{0.700000}%
\pgfsetlinewidth{0.000000pt}%
\definecolor{currentstroke}{rgb}{0.000000,0.000000,0.000000}%
\pgfsetstrokecolor{currentstroke}%
\pgfsetstrokeopacity{0.700000}%
\pgfsetdash{}{0pt}%
\pgfpathmoveto{\pgfqpoint{8.204208in}{2.251785in}}%
\pgfpathcurveto{\pgfqpoint{8.209252in}{2.251785in}}{\pgfqpoint{8.214089in}{2.253789in}}{\pgfqpoint{8.217656in}{2.257356in}}%
\pgfpathcurveto{\pgfqpoint{8.221222in}{2.260922in}}{\pgfqpoint{8.223226in}{2.265760in}}{\pgfqpoint{8.223226in}{2.270803in}}%
\pgfpathcurveto{\pgfqpoint{8.223226in}{2.275847in}}{\pgfqpoint{8.221222in}{2.280685in}}{\pgfqpoint{8.217656in}{2.284251in}}%
\pgfpathcurveto{\pgfqpoint{8.214089in}{2.287818in}}{\pgfqpoint{8.209252in}{2.289822in}}{\pgfqpoint{8.204208in}{2.289822in}}%
\pgfpathcurveto{\pgfqpoint{8.199164in}{2.289822in}}{\pgfqpoint{8.194327in}{2.287818in}}{\pgfqpoint{8.190760in}{2.284251in}}%
\pgfpathcurveto{\pgfqpoint{8.187194in}{2.280685in}}{\pgfqpoint{8.185190in}{2.275847in}}{\pgfqpoint{8.185190in}{2.270803in}}%
\pgfpathcurveto{\pgfqpoint{8.185190in}{2.265760in}}{\pgfqpoint{8.187194in}{2.260922in}}{\pgfqpoint{8.190760in}{2.257356in}}%
\pgfpathcurveto{\pgfqpoint{8.194327in}{2.253789in}}{\pgfqpoint{8.199164in}{2.251785in}}{\pgfqpoint{8.204208in}{2.251785in}}%
\pgfpathclose%
\pgfusepath{fill}%
\end{pgfscope}%
\begin{pgfscope}%
\pgfpathrectangle{\pgfqpoint{6.572727in}{0.473000in}}{\pgfqpoint{4.227273in}{3.311000in}}%
\pgfusepath{clip}%
\pgfsetbuttcap%
\pgfsetroundjoin%
\definecolor{currentfill}{rgb}{0.127568,0.566949,0.550556}%
\pgfsetfillcolor{currentfill}%
\pgfsetfillopacity{0.700000}%
\pgfsetlinewidth{0.000000pt}%
\definecolor{currentstroke}{rgb}{0.000000,0.000000,0.000000}%
\pgfsetstrokecolor{currentstroke}%
\pgfsetstrokeopacity{0.700000}%
\pgfsetdash{}{0pt}%
\pgfpathmoveto{\pgfqpoint{7.617970in}{2.029928in}}%
\pgfpathcurveto{\pgfqpoint{7.623014in}{2.029928in}}{\pgfqpoint{7.627851in}{2.031931in}}{\pgfqpoint{7.631418in}{2.035498in}}%
\pgfpathcurveto{\pgfqpoint{7.634984in}{2.039064in}}{\pgfqpoint{7.636988in}{2.043902in}}{\pgfqpoint{7.636988in}{2.048946in}}%
\pgfpathcurveto{\pgfqpoint{7.636988in}{2.053989in}}{\pgfqpoint{7.634984in}{2.058827in}}{\pgfqpoint{7.631418in}{2.062394in}}%
\pgfpathcurveto{\pgfqpoint{7.627851in}{2.065960in}}{\pgfqpoint{7.623014in}{2.067964in}}{\pgfqpoint{7.617970in}{2.067964in}}%
\pgfpathcurveto{\pgfqpoint{7.612926in}{2.067964in}}{\pgfqpoint{7.608089in}{2.065960in}}{\pgfqpoint{7.604522in}{2.062394in}}%
\pgfpathcurveto{\pgfqpoint{7.600956in}{2.058827in}}{\pgfqpoint{7.598952in}{2.053989in}}{\pgfqpoint{7.598952in}{2.048946in}}%
\pgfpathcurveto{\pgfqpoint{7.598952in}{2.043902in}}{\pgfqpoint{7.600956in}{2.039064in}}{\pgfqpoint{7.604522in}{2.035498in}}%
\pgfpathcurveto{\pgfqpoint{7.608089in}{2.031931in}}{\pgfqpoint{7.612926in}{2.029928in}}{\pgfqpoint{7.617970in}{2.029928in}}%
\pgfpathclose%
\pgfusepath{fill}%
\end{pgfscope}%
\begin{pgfscope}%
\pgfpathrectangle{\pgfqpoint{6.572727in}{0.473000in}}{\pgfqpoint{4.227273in}{3.311000in}}%
\pgfusepath{clip}%
\pgfsetbuttcap%
\pgfsetroundjoin%
\definecolor{currentfill}{rgb}{0.993248,0.906157,0.143936}%
\pgfsetfillcolor{currentfill}%
\pgfsetfillopacity{0.700000}%
\pgfsetlinewidth{0.000000pt}%
\definecolor{currentstroke}{rgb}{0.000000,0.000000,0.000000}%
\pgfsetstrokecolor{currentstroke}%
\pgfsetstrokeopacity{0.700000}%
\pgfsetdash{}{0pt}%
\pgfpathmoveto{\pgfqpoint{9.077907in}{1.114727in}}%
\pgfpathcurveto{\pgfqpoint{9.082951in}{1.114727in}}{\pgfqpoint{9.087788in}{1.116731in}}{\pgfqpoint{9.091355in}{1.120297in}}%
\pgfpathcurveto{\pgfqpoint{9.094921in}{1.123864in}}{\pgfqpoint{9.096925in}{1.128701in}}{\pgfqpoint{9.096925in}{1.133745in}}%
\pgfpathcurveto{\pgfqpoint{9.096925in}{1.138789in}}{\pgfqpoint{9.094921in}{1.143626in}}{\pgfqpoint{9.091355in}{1.147193in}}%
\pgfpathcurveto{\pgfqpoint{9.087788in}{1.150759in}}{\pgfqpoint{9.082951in}{1.152763in}}{\pgfqpoint{9.077907in}{1.152763in}}%
\pgfpathcurveto{\pgfqpoint{9.072863in}{1.152763in}}{\pgfqpoint{9.068025in}{1.150759in}}{\pgfqpoint{9.064459in}{1.147193in}}%
\pgfpathcurveto{\pgfqpoint{9.060893in}{1.143626in}}{\pgfqpoint{9.058889in}{1.138789in}}{\pgfqpoint{9.058889in}{1.133745in}}%
\pgfpathcurveto{\pgfqpoint{9.058889in}{1.128701in}}{\pgfqpoint{9.060893in}{1.123864in}}{\pgfqpoint{9.064459in}{1.120297in}}%
\pgfpathcurveto{\pgfqpoint{9.068025in}{1.116731in}}{\pgfqpoint{9.072863in}{1.114727in}}{\pgfqpoint{9.077907in}{1.114727in}}%
\pgfpathclose%
\pgfusepath{fill}%
\end{pgfscope}%
\begin{pgfscope}%
\pgfpathrectangle{\pgfqpoint{6.572727in}{0.473000in}}{\pgfqpoint{4.227273in}{3.311000in}}%
\pgfusepath{clip}%
\pgfsetbuttcap%
\pgfsetroundjoin%
\definecolor{currentfill}{rgb}{0.127568,0.566949,0.550556}%
\pgfsetfillcolor{currentfill}%
\pgfsetfillopacity{0.700000}%
\pgfsetlinewidth{0.000000pt}%
\definecolor{currentstroke}{rgb}{0.000000,0.000000,0.000000}%
\pgfsetstrokecolor{currentstroke}%
\pgfsetstrokeopacity{0.700000}%
\pgfsetdash{}{0pt}%
\pgfpathmoveto{\pgfqpoint{8.162401in}{2.706262in}}%
\pgfpathcurveto{\pgfqpoint{8.167445in}{2.706262in}}{\pgfqpoint{8.172283in}{2.708266in}}{\pgfqpoint{8.175849in}{2.711833in}}%
\pgfpathcurveto{\pgfqpoint{8.179415in}{2.715399in}}{\pgfqpoint{8.181419in}{2.720237in}}{\pgfqpoint{8.181419in}{2.725281in}}%
\pgfpathcurveto{\pgfqpoint{8.181419in}{2.730324in}}{\pgfqpoint{8.179415in}{2.735162in}}{\pgfqpoint{8.175849in}{2.738728in}}%
\pgfpathcurveto{\pgfqpoint{8.172283in}{2.742295in}}{\pgfqpoint{8.167445in}{2.744299in}}{\pgfqpoint{8.162401in}{2.744299in}}%
\pgfpathcurveto{\pgfqpoint{8.157357in}{2.744299in}}{\pgfqpoint{8.152520in}{2.742295in}}{\pgfqpoint{8.148953in}{2.738728in}}%
\pgfpathcurveto{\pgfqpoint{8.145387in}{2.735162in}}{\pgfqpoint{8.143383in}{2.730324in}}{\pgfqpoint{8.143383in}{2.725281in}}%
\pgfpathcurveto{\pgfqpoint{8.143383in}{2.720237in}}{\pgfqpoint{8.145387in}{2.715399in}}{\pgfqpoint{8.148953in}{2.711833in}}%
\pgfpathcurveto{\pgfqpoint{8.152520in}{2.708266in}}{\pgfqpoint{8.157357in}{2.706262in}}{\pgfqpoint{8.162401in}{2.706262in}}%
\pgfpathclose%
\pgfusepath{fill}%
\end{pgfscope}%
\begin{pgfscope}%
\pgfpathrectangle{\pgfqpoint{6.572727in}{0.473000in}}{\pgfqpoint{4.227273in}{3.311000in}}%
\pgfusepath{clip}%
\pgfsetbuttcap%
\pgfsetroundjoin%
\definecolor{currentfill}{rgb}{0.127568,0.566949,0.550556}%
\pgfsetfillcolor{currentfill}%
\pgfsetfillopacity{0.700000}%
\pgfsetlinewidth{0.000000pt}%
\definecolor{currentstroke}{rgb}{0.000000,0.000000,0.000000}%
\pgfsetstrokecolor{currentstroke}%
\pgfsetstrokeopacity{0.700000}%
\pgfsetdash{}{0pt}%
\pgfpathmoveto{\pgfqpoint{7.568948in}{1.402762in}}%
\pgfpathcurveto{\pgfqpoint{7.573992in}{1.402762in}}{\pgfqpoint{7.578830in}{1.404765in}}{\pgfqpoint{7.582396in}{1.408332in}}%
\pgfpathcurveto{\pgfqpoint{7.585962in}{1.411898in}}{\pgfqpoint{7.587966in}{1.416736in}}{\pgfqpoint{7.587966in}{1.421780in}}%
\pgfpathcurveto{\pgfqpoint{7.587966in}{1.426823in}}{\pgfqpoint{7.585962in}{1.431661in}}{\pgfqpoint{7.582396in}{1.435228in}}%
\pgfpathcurveto{\pgfqpoint{7.578830in}{1.438794in}}{\pgfqpoint{7.573992in}{1.440798in}}{\pgfqpoint{7.568948in}{1.440798in}}%
\pgfpathcurveto{\pgfqpoint{7.563904in}{1.440798in}}{\pgfqpoint{7.559067in}{1.438794in}}{\pgfqpoint{7.555500in}{1.435228in}}%
\pgfpathcurveto{\pgfqpoint{7.551934in}{1.431661in}}{\pgfqpoint{7.549930in}{1.426823in}}{\pgfqpoint{7.549930in}{1.421780in}}%
\pgfpathcurveto{\pgfqpoint{7.549930in}{1.416736in}}{\pgfqpoint{7.551934in}{1.411898in}}{\pgfqpoint{7.555500in}{1.408332in}}%
\pgfpathcurveto{\pgfqpoint{7.559067in}{1.404765in}}{\pgfqpoint{7.563904in}{1.402762in}}{\pgfqpoint{7.568948in}{1.402762in}}%
\pgfpathclose%
\pgfusepath{fill}%
\end{pgfscope}%
\begin{pgfscope}%
\pgfpathrectangle{\pgfqpoint{6.572727in}{0.473000in}}{\pgfqpoint{4.227273in}{3.311000in}}%
\pgfusepath{clip}%
\pgfsetbuttcap%
\pgfsetroundjoin%
\definecolor{currentfill}{rgb}{0.993248,0.906157,0.143936}%
\pgfsetfillcolor{currentfill}%
\pgfsetfillopacity{0.700000}%
\pgfsetlinewidth{0.000000pt}%
\definecolor{currentstroke}{rgb}{0.000000,0.000000,0.000000}%
\pgfsetstrokecolor{currentstroke}%
\pgfsetstrokeopacity{0.700000}%
\pgfsetdash{}{0pt}%
\pgfpathmoveto{\pgfqpoint{9.632150in}{1.757008in}}%
\pgfpathcurveto{\pgfqpoint{9.637194in}{1.757008in}}{\pgfqpoint{9.642032in}{1.759012in}}{\pgfqpoint{9.645598in}{1.762578in}}%
\pgfpathcurveto{\pgfqpoint{9.649165in}{1.766145in}}{\pgfqpoint{9.651169in}{1.770983in}}{\pgfqpoint{9.651169in}{1.776026in}}%
\pgfpathcurveto{\pgfqpoint{9.651169in}{1.781070in}}{\pgfqpoint{9.649165in}{1.785908in}}{\pgfqpoint{9.645598in}{1.789474in}}%
\pgfpathcurveto{\pgfqpoint{9.642032in}{1.793041in}}{\pgfqpoint{9.637194in}{1.795044in}}{\pgfqpoint{9.632150in}{1.795044in}}%
\pgfpathcurveto{\pgfqpoint{9.627107in}{1.795044in}}{\pgfqpoint{9.622269in}{1.793041in}}{\pgfqpoint{9.618703in}{1.789474in}}%
\pgfpathcurveto{\pgfqpoint{9.615136in}{1.785908in}}{\pgfqpoint{9.613132in}{1.781070in}}{\pgfqpoint{9.613132in}{1.776026in}}%
\pgfpathcurveto{\pgfqpoint{9.613132in}{1.770983in}}{\pgfqpoint{9.615136in}{1.766145in}}{\pgfqpoint{9.618703in}{1.762578in}}%
\pgfpathcurveto{\pgfqpoint{9.622269in}{1.759012in}}{\pgfqpoint{9.627107in}{1.757008in}}{\pgfqpoint{9.632150in}{1.757008in}}%
\pgfpathclose%
\pgfusepath{fill}%
\end{pgfscope}%
\begin{pgfscope}%
\pgfpathrectangle{\pgfqpoint{6.572727in}{0.473000in}}{\pgfqpoint{4.227273in}{3.311000in}}%
\pgfusepath{clip}%
\pgfsetbuttcap%
\pgfsetroundjoin%
\definecolor{currentfill}{rgb}{0.127568,0.566949,0.550556}%
\pgfsetfillcolor{currentfill}%
\pgfsetfillopacity{0.700000}%
\pgfsetlinewidth{0.000000pt}%
\definecolor{currentstroke}{rgb}{0.000000,0.000000,0.000000}%
\pgfsetstrokecolor{currentstroke}%
\pgfsetstrokeopacity{0.700000}%
\pgfsetdash{}{0pt}%
\pgfpathmoveto{\pgfqpoint{7.644865in}{2.282876in}}%
\pgfpathcurveto{\pgfqpoint{7.649909in}{2.282876in}}{\pgfqpoint{7.654747in}{2.284880in}}{\pgfqpoint{7.658313in}{2.288447in}}%
\pgfpathcurveto{\pgfqpoint{7.661880in}{2.292013in}}{\pgfqpoint{7.663884in}{2.296851in}}{\pgfqpoint{7.663884in}{2.301894in}}%
\pgfpathcurveto{\pgfqpoint{7.663884in}{2.306938in}}{\pgfqpoint{7.661880in}{2.311776in}}{\pgfqpoint{7.658313in}{2.315342in}}%
\pgfpathcurveto{\pgfqpoint{7.654747in}{2.318909in}}{\pgfqpoint{7.649909in}{2.320913in}}{\pgfqpoint{7.644865in}{2.320913in}}%
\pgfpathcurveto{\pgfqpoint{7.639822in}{2.320913in}}{\pgfqpoint{7.634984in}{2.318909in}}{\pgfqpoint{7.631418in}{2.315342in}}%
\pgfpathcurveto{\pgfqpoint{7.627851in}{2.311776in}}{\pgfqpoint{7.625847in}{2.306938in}}{\pgfqpoint{7.625847in}{2.301894in}}%
\pgfpathcurveto{\pgfqpoint{7.625847in}{2.296851in}}{\pgfqpoint{7.627851in}{2.292013in}}{\pgfqpoint{7.631418in}{2.288447in}}%
\pgfpathcurveto{\pgfqpoint{7.634984in}{2.284880in}}{\pgfqpoint{7.639822in}{2.282876in}}{\pgfqpoint{7.644865in}{2.282876in}}%
\pgfpathclose%
\pgfusepath{fill}%
\end{pgfscope}%
\begin{pgfscope}%
\pgfpathrectangle{\pgfqpoint{6.572727in}{0.473000in}}{\pgfqpoint{4.227273in}{3.311000in}}%
\pgfusepath{clip}%
\pgfsetbuttcap%
\pgfsetroundjoin%
\definecolor{currentfill}{rgb}{0.993248,0.906157,0.143936}%
\pgfsetfillcolor{currentfill}%
\pgfsetfillopacity{0.700000}%
\pgfsetlinewidth{0.000000pt}%
\definecolor{currentstroke}{rgb}{0.000000,0.000000,0.000000}%
\pgfsetstrokecolor{currentstroke}%
\pgfsetstrokeopacity{0.700000}%
\pgfsetdash{}{0pt}%
\pgfpathmoveto{\pgfqpoint{9.867545in}{0.704849in}}%
\pgfpathcurveto{\pgfqpoint{9.872589in}{0.704849in}}{\pgfqpoint{9.877426in}{0.706852in}}{\pgfqpoint{9.880993in}{0.710419in}}%
\pgfpathcurveto{\pgfqpoint{9.884559in}{0.713985in}}{\pgfqpoint{9.886563in}{0.718823in}}{\pgfqpoint{9.886563in}{0.723867in}}%
\pgfpathcurveto{\pgfqpoint{9.886563in}{0.728910in}}{\pgfqpoint{9.884559in}{0.733748in}}{\pgfqpoint{9.880993in}{0.737315in}}%
\pgfpathcurveto{\pgfqpoint{9.877426in}{0.740881in}}{\pgfqpoint{9.872589in}{0.742885in}}{\pgfqpoint{9.867545in}{0.742885in}}%
\pgfpathcurveto{\pgfqpoint{9.862501in}{0.742885in}}{\pgfqpoint{9.857663in}{0.740881in}}{\pgfqpoint{9.854097in}{0.737315in}}%
\pgfpathcurveto{\pgfqpoint{9.850531in}{0.733748in}}{\pgfqpoint{9.848527in}{0.728910in}}{\pgfqpoint{9.848527in}{0.723867in}}%
\pgfpathcurveto{\pgfqpoint{9.848527in}{0.718823in}}{\pgfqpoint{9.850531in}{0.713985in}}{\pgfqpoint{9.854097in}{0.710419in}}%
\pgfpathcurveto{\pgfqpoint{9.857663in}{0.706852in}}{\pgfqpoint{9.862501in}{0.704849in}}{\pgfqpoint{9.867545in}{0.704849in}}%
\pgfpathclose%
\pgfusepath{fill}%
\end{pgfscope}%
\begin{pgfscope}%
\pgfpathrectangle{\pgfqpoint{6.572727in}{0.473000in}}{\pgfqpoint{4.227273in}{3.311000in}}%
\pgfusepath{clip}%
\pgfsetbuttcap%
\pgfsetroundjoin%
\definecolor{currentfill}{rgb}{0.127568,0.566949,0.550556}%
\pgfsetfillcolor{currentfill}%
\pgfsetfillopacity{0.700000}%
\pgfsetlinewidth{0.000000pt}%
\definecolor{currentstroke}{rgb}{0.000000,0.000000,0.000000}%
\pgfsetstrokecolor{currentstroke}%
\pgfsetstrokeopacity{0.700000}%
\pgfsetdash{}{0pt}%
\pgfpathmoveto{\pgfqpoint{8.457399in}{2.924361in}}%
\pgfpathcurveto{\pgfqpoint{8.462443in}{2.924361in}}{\pgfqpoint{8.467280in}{2.926365in}}{\pgfqpoint{8.470847in}{2.929931in}}%
\pgfpathcurveto{\pgfqpoint{8.474413in}{2.933497in}}{\pgfqpoint{8.476417in}{2.938335in}}{\pgfqpoint{8.476417in}{2.943379in}}%
\pgfpathcurveto{\pgfqpoint{8.476417in}{2.948423in}}{\pgfqpoint{8.474413in}{2.953260in}}{\pgfqpoint{8.470847in}{2.956827in}}%
\pgfpathcurveto{\pgfqpoint{8.467280in}{2.960393in}}{\pgfqpoint{8.462443in}{2.962397in}}{\pgfqpoint{8.457399in}{2.962397in}}%
\pgfpathcurveto{\pgfqpoint{8.452355in}{2.962397in}}{\pgfqpoint{8.447517in}{2.960393in}}{\pgfqpoint{8.443951in}{2.956827in}}%
\pgfpathcurveto{\pgfqpoint{8.440385in}{2.953260in}}{\pgfqpoint{8.438381in}{2.948423in}}{\pgfqpoint{8.438381in}{2.943379in}}%
\pgfpathcurveto{\pgfqpoint{8.438381in}{2.938335in}}{\pgfqpoint{8.440385in}{2.933497in}}{\pgfqpoint{8.443951in}{2.929931in}}%
\pgfpathcurveto{\pgfqpoint{8.447517in}{2.926365in}}{\pgfqpoint{8.452355in}{2.924361in}}{\pgfqpoint{8.457399in}{2.924361in}}%
\pgfpathclose%
\pgfusepath{fill}%
\end{pgfscope}%
\begin{pgfscope}%
\pgfpathrectangle{\pgfqpoint{6.572727in}{0.473000in}}{\pgfqpoint{4.227273in}{3.311000in}}%
\pgfusepath{clip}%
\pgfsetbuttcap%
\pgfsetroundjoin%
\definecolor{currentfill}{rgb}{0.993248,0.906157,0.143936}%
\pgfsetfillcolor{currentfill}%
\pgfsetfillopacity{0.700000}%
\pgfsetlinewidth{0.000000pt}%
\definecolor{currentstroke}{rgb}{0.000000,0.000000,0.000000}%
\pgfsetstrokecolor{currentstroke}%
\pgfsetstrokeopacity{0.700000}%
\pgfsetdash{}{0pt}%
\pgfpathmoveto{\pgfqpoint{9.993451in}{1.729309in}}%
\pgfpathcurveto{\pgfqpoint{9.998495in}{1.729309in}}{\pgfqpoint{10.003333in}{1.731313in}}{\pgfqpoint{10.006899in}{1.734879in}}%
\pgfpathcurveto{\pgfqpoint{10.010465in}{1.738445in}}{\pgfqpoint{10.012469in}{1.743283in}}{\pgfqpoint{10.012469in}{1.748327in}}%
\pgfpathcurveto{\pgfqpoint{10.012469in}{1.753371in}}{\pgfqpoint{10.010465in}{1.758208in}}{\pgfqpoint{10.006899in}{1.761775in}}%
\pgfpathcurveto{\pgfqpoint{10.003333in}{1.765341in}}{\pgfqpoint{9.998495in}{1.767345in}}{\pgfqpoint{9.993451in}{1.767345in}}%
\pgfpathcurveto{\pgfqpoint{9.988407in}{1.767345in}}{\pgfqpoint{9.983570in}{1.765341in}}{\pgfqpoint{9.980003in}{1.761775in}}%
\pgfpathcurveto{\pgfqpoint{9.976437in}{1.758208in}}{\pgfqpoint{9.974433in}{1.753371in}}{\pgfqpoint{9.974433in}{1.748327in}}%
\pgfpathcurveto{\pgfqpoint{9.974433in}{1.743283in}}{\pgfqpoint{9.976437in}{1.738445in}}{\pgfqpoint{9.980003in}{1.734879in}}%
\pgfpathcurveto{\pgfqpoint{9.983570in}{1.731313in}}{\pgfqpoint{9.988407in}{1.729309in}}{\pgfqpoint{9.993451in}{1.729309in}}%
\pgfpathclose%
\pgfusepath{fill}%
\end{pgfscope}%
\begin{pgfscope}%
\pgfpathrectangle{\pgfqpoint{6.572727in}{0.473000in}}{\pgfqpoint{4.227273in}{3.311000in}}%
\pgfusepath{clip}%
\pgfsetbuttcap%
\pgfsetroundjoin%
\definecolor{currentfill}{rgb}{0.127568,0.566949,0.550556}%
\pgfsetfillcolor{currentfill}%
\pgfsetfillopacity{0.700000}%
\pgfsetlinewidth{0.000000pt}%
\definecolor{currentstroke}{rgb}{0.000000,0.000000,0.000000}%
\pgfsetstrokecolor{currentstroke}%
\pgfsetstrokeopacity{0.700000}%
\pgfsetdash{}{0pt}%
\pgfpathmoveto{\pgfqpoint{8.398872in}{1.653830in}}%
\pgfpathcurveto{\pgfqpoint{8.403915in}{1.653830in}}{\pgfqpoint{8.408753in}{1.655834in}}{\pgfqpoint{8.412320in}{1.659401in}}%
\pgfpathcurveto{\pgfqpoint{8.415886in}{1.662967in}}{\pgfqpoint{8.417890in}{1.667805in}}{\pgfqpoint{8.417890in}{1.672848in}}%
\pgfpathcurveto{\pgfqpoint{8.417890in}{1.677892in}}{\pgfqpoint{8.415886in}{1.682730in}}{\pgfqpoint{8.412320in}{1.686296in}}%
\pgfpathcurveto{\pgfqpoint{8.408753in}{1.689863in}}{\pgfqpoint{8.403915in}{1.691867in}}{\pgfqpoint{8.398872in}{1.691867in}}%
\pgfpathcurveto{\pgfqpoint{8.393828in}{1.691867in}}{\pgfqpoint{8.388990in}{1.689863in}}{\pgfqpoint{8.385424in}{1.686296in}}%
\pgfpathcurveto{\pgfqpoint{8.381857in}{1.682730in}}{\pgfqpoint{8.379854in}{1.677892in}}{\pgfqpoint{8.379854in}{1.672848in}}%
\pgfpathcurveto{\pgfqpoint{8.379854in}{1.667805in}}{\pgfqpoint{8.381857in}{1.662967in}}{\pgfqpoint{8.385424in}{1.659401in}}%
\pgfpathcurveto{\pgfqpoint{8.388990in}{1.655834in}}{\pgfqpoint{8.393828in}{1.653830in}}{\pgfqpoint{8.398872in}{1.653830in}}%
\pgfpathclose%
\pgfusepath{fill}%
\end{pgfscope}%
\begin{pgfscope}%
\pgfpathrectangle{\pgfqpoint{6.572727in}{0.473000in}}{\pgfqpoint{4.227273in}{3.311000in}}%
\pgfusepath{clip}%
\pgfsetbuttcap%
\pgfsetroundjoin%
\definecolor{currentfill}{rgb}{0.993248,0.906157,0.143936}%
\pgfsetfillcolor{currentfill}%
\pgfsetfillopacity{0.700000}%
\pgfsetlinewidth{0.000000pt}%
\definecolor{currentstroke}{rgb}{0.000000,0.000000,0.000000}%
\pgfsetstrokecolor{currentstroke}%
\pgfsetstrokeopacity{0.700000}%
\pgfsetdash{}{0pt}%
\pgfpathmoveto{\pgfqpoint{10.287256in}{1.698878in}}%
\pgfpathcurveto{\pgfqpoint{10.292300in}{1.698878in}}{\pgfqpoint{10.297138in}{1.700882in}}{\pgfqpoint{10.300704in}{1.704448in}}%
\pgfpathcurveto{\pgfqpoint{10.304270in}{1.708015in}}{\pgfqpoint{10.306274in}{1.712852in}}{\pgfqpoint{10.306274in}{1.717896in}}%
\pgfpathcurveto{\pgfqpoint{10.306274in}{1.722940in}}{\pgfqpoint{10.304270in}{1.727778in}}{\pgfqpoint{10.300704in}{1.731344in}}%
\pgfpathcurveto{\pgfqpoint{10.297138in}{1.734910in}}{\pgfqpoint{10.292300in}{1.736914in}}{\pgfqpoint{10.287256in}{1.736914in}}%
\pgfpathcurveto{\pgfqpoint{10.282212in}{1.736914in}}{\pgfqpoint{10.277375in}{1.734910in}}{\pgfqpoint{10.273808in}{1.731344in}}%
\pgfpathcurveto{\pgfqpoint{10.270242in}{1.727778in}}{\pgfqpoint{10.268238in}{1.722940in}}{\pgfqpoint{10.268238in}{1.717896in}}%
\pgfpathcurveto{\pgfqpoint{10.268238in}{1.712852in}}{\pgfqpoint{10.270242in}{1.708015in}}{\pgfqpoint{10.273808in}{1.704448in}}%
\pgfpathcurveto{\pgfqpoint{10.277375in}{1.700882in}}{\pgfqpoint{10.282212in}{1.698878in}}{\pgfqpoint{10.287256in}{1.698878in}}%
\pgfpathclose%
\pgfusepath{fill}%
\end{pgfscope}%
\begin{pgfscope}%
\pgfpathrectangle{\pgfqpoint{6.572727in}{0.473000in}}{\pgfqpoint{4.227273in}{3.311000in}}%
\pgfusepath{clip}%
\pgfsetbuttcap%
\pgfsetroundjoin%
\definecolor{currentfill}{rgb}{0.127568,0.566949,0.550556}%
\pgfsetfillcolor{currentfill}%
\pgfsetfillopacity{0.700000}%
\pgfsetlinewidth{0.000000pt}%
\definecolor{currentstroke}{rgb}{0.000000,0.000000,0.000000}%
\pgfsetstrokecolor{currentstroke}%
\pgfsetstrokeopacity{0.700000}%
\pgfsetdash{}{0pt}%
\pgfpathmoveto{\pgfqpoint{7.906750in}{2.297500in}}%
\pgfpathcurveto{\pgfqpoint{7.911794in}{2.297500in}}{\pgfqpoint{7.916632in}{2.299504in}}{\pgfqpoint{7.920198in}{2.303071in}}%
\pgfpathcurveto{\pgfqpoint{7.923764in}{2.306637in}}{\pgfqpoint{7.925768in}{2.311475in}}{\pgfqpoint{7.925768in}{2.316519in}}%
\pgfpathcurveto{\pgfqpoint{7.925768in}{2.321562in}}{\pgfqpoint{7.923764in}{2.326400in}}{\pgfqpoint{7.920198in}{2.329966in}}%
\pgfpathcurveto{\pgfqpoint{7.916632in}{2.333533in}}{\pgfqpoint{7.911794in}{2.335537in}}{\pgfqpoint{7.906750in}{2.335537in}}%
\pgfpathcurveto{\pgfqpoint{7.901706in}{2.335537in}}{\pgfqpoint{7.896869in}{2.333533in}}{\pgfqpoint{7.893302in}{2.329966in}}%
\pgfpathcurveto{\pgfqpoint{7.889736in}{2.326400in}}{\pgfqpoint{7.887732in}{2.321562in}}{\pgfqpoint{7.887732in}{2.316519in}}%
\pgfpathcurveto{\pgfqpoint{7.887732in}{2.311475in}}{\pgfqpoint{7.889736in}{2.306637in}}{\pgfqpoint{7.893302in}{2.303071in}}%
\pgfpathcurveto{\pgfqpoint{7.896869in}{2.299504in}}{\pgfqpoint{7.901706in}{2.297500in}}{\pgfqpoint{7.906750in}{2.297500in}}%
\pgfpathclose%
\pgfusepath{fill}%
\end{pgfscope}%
\begin{pgfscope}%
\pgfpathrectangle{\pgfqpoint{6.572727in}{0.473000in}}{\pgfqpoint{4.227273in}{3.311000in}}%
\pgfusepath{clip}%
\pgfsetbuttcap%
\pgfsetroundjoin%
\definecolor{currentfill}{rgb}{0.127568,0.566949,0.550556}%
\pgfsetfillcolor{currentfill}%
\pgfsetfillopacity{0.700000}%
\pgfsetlinewidth{0.000000pt}%
\definecolor{currentstroke}{rgb}{0.000000,0.000000,0.000000}%
\pgfsetstrokecolor{currentstroke}%
\pgfsetstrokeopacity{0.700000}%
\pgfsetdash{}{0pt}%
\pgfpathmoveto{\pgfqpoint{7.907470in}{2.891856in}}%
\pgfpathcurveto{\pgfqpoint{7.912514in}{2.891856in}}{\pgfqpoint{7.917352in}{2.893860in}}{\pgfqpoint{7.920918in}{2.897426in}}%
\pgfpathcurveto{\pgfqpoint{7.924485in}{2.900993in}}{\pgfqpoint{7.926489in}{2.905830in}}{\pgfqpoint{7.926489in}{2.910874in}}%
\pgfpathcurveto{\pgfqpoint{7.926489in}{2.915918in}}{\pgfqpoint{7.924485in}{2.920756in}}{\pgfqpoint{7.920918in}{2.924322in}}%
\pgfpathcurveto{\pgfqpoint{7.917352in}{2.927888in}}{\pgfqpoint{7.912514in}{2.929892in}}{\pgfqpoint{7.907470in}{2.929892in}}%
\pgfpathcurveto{\pgfqpoint{7.902427in}{2.929892in}}{\pgfqpoint{7.897589in}{2.927888in}}{\pgfqpoint{7.894023in}{2.924322in}}%
\pgfpathcurveto{\pgfqpoint{7.890456in}{2.920756in}}{\pgfqpoint{7.888452in}{2.915918in}}{\pgfqpoint{7.888452in}{2.910874in}}%
\pgfpathcurveto{\pgfqpoint{7.888452in}{2.905830in}}{\pgfqpoint{7.890456in}{2.900993in}}{\pgfqpoint{7.894023in}{2.897426in}}%
\pgfpathcurveto{\pgfqpoint{7.897589in}{2.893860in}}{\pgfqpoint{7.902427in}{2.891856in}}{\pgfqpoint{7.907470in}{2.891856in}}%
\pgfpathclose%
\pgfusepath{fill}%
\end{pgfscope}%
\begin{pgfscope}%
\pgfpathrectangle{\pgfqpoint{6.572727in}{0.473000in}}{\pgfqpoint{4.227273in}{3.311000in}}%
\pgfusepath{clip}%
\pgfsetbuttcap%
\pgfsetroundjoin%
\definecolor{currentfill}{rgb}{0.127568,0.566949,0.550556}%
\pgfsetfillcolor{currentfill}%
\pgfsetfillopacity{0.700000}%
\pgfsetlinewidth{0.000000pt}%
\definecolor{currentstroke}{rgb}{0.000000,0.000000,0.000000}%
\pgfsetstrokecolor{currentstroke}%
\pgfsetstrokeopacity{0.700000}%
\pgfsetdash{}{0pt}%
\pgfpathmoveto{\pgfqpoint{7.630284in}{1.587831in}}%
\pgfpathcurveto{\pgfqpoint{7.635328in}{1.587831in}}{\pgfqpoint{7.640166in}{1.589835in}}{\pgfqpoint{7.643732in}{1.593401in}}%
\pgfpathcurveto{\pgfqpoint{7.647298in}{1.596967in}}{\pgfqpoint{7.649302in}{1.601805in}}{\pgfqpoint{7.649302in}{1.606849in}}%
\pgfpathcurveto{\pgfqpoint{7.649302in}{1.611893in}}{\pgfqpoint{7.647298in}{1.616730in}}{\pgfqpoint{7.643732in}{1.620297in}}%
\pgfpathcurveto{\pgfqpoint{7.640166in}{1.623863in}}{\pgfqpoint{7.635328in}{1.625867in}}{\pgfqpoint{7.630284in}{1.625867in}}%
\pgfpathcurveto{\pgfqpoint{7.625240in}{1.625867in}}{\pgfqpoint{7.620403in}{1.623863in}}{\pgfqpoint{7.616836in}{1.620297in}}%
\pgfpathcurveto{\pgfqpoint{7.613270in}{1.616730in}}{\pgfqpoint{7.611266in}{1.611893in}}{\pgfqpoint{7.611266in}{1.606849in}}%
\pgfpathcurveto{\pgfqpoint{7.611266in}{1.601805in}}{\pgfqpoint{7.613270in}{1.596967in}}{\pgfqpoint{7.616836in}{1.593401in}}%
\pgfpathcurveto{\pgfqpoint{7.620403in}{1.589835in}}{\pgfqpoint{7.625240in}{1.587831in}}{\pgfqpoint{7.630284in}{1.587831in}}%
\pgfpathclose%
\pgfusepath{fill}%
\end{pgfscope}%
\begin{pgfscope}%
\pgfpathrectangle{\pgfqpoint{6.572727in}{0.473000in}}{\pgfqpoint{4.227273in}{3.311000in}}%
\pgfusepath{clip}%
\pgfsetbuttcap%
\pgfsetroundjoin%
\definecolor{currentfill}{rgb}{0.127568,0.566949,0.550556}%
\pgfsetfillcolor{currentfill}%
\pgfsetfillopacity{0.700000}%
\pgfsetlinewidth{0.000000pt}%
\definecolor{currentstroke}{rgb}{0.000000,0.000000,0.000000}%
\pgfsetstrokecolor{currentstroke}%
\pgfsetstrokeopacity{0.700000}%
\pgfsetdash{}{0pt}%
\pgfpathmoveto{\pgfqpoint{8.717370in}{2.303207in}}%
\pgfpathcurveto{\pgfqpoint{8.722414in}{2.303207in}}{\pgfqpoint{8.727252in}{2.305211in}}{\pgfqpoint{8.730818in}{2.308777in}}%
\pgfpathcurveto{\pgfqpoint{8.734384in}{2.312344in}}{\pgfqpoint{8.736388in}{2.317181in}}{\pgfqpoint{8.736388in}{2.322225in}}%
\pgfpathcurveto{\pgfqpoint{8.736388in}{2.327269in}}{\pgfqpoint{8.734384in}{2.332107in}}{\pgfqpoint{8.730818in}{2.335673in}}%
\pgfpathcurveto{\pgfqpoint{8.727252in}{2.339239in}}{\pgfqpoint{8.722414in}{2.341243in}}{\pgfqpoint{8.717370in}{2.341243in}}%
\pgfpathcurveto{\pgfqpoint{8.712326in}{2.341243in}}{\pgfqpoint{8.707489in}{2.339239in}}{\pgfqpoint{8.703922in}{2.335673in}}%
\pgfpathcurveto{\pgfqpoint{8.700356in}{2.332107in}}{\pgfqpoint{8.698352in}{2.327269in}}{\pgfqpoint{8.698352in}{2.322225in}}%
\pgfpathcurveto{\pgfqpoint{8.698352in}{2.317181in}}{\pgfqpoint{8.700356in}{2.312344in}}{\pgfqpoint{8.703922in}{2.308777in}}%
\pgfpathcurveto{\pgfqpoint{8.707489in}{2.305211in}}{\pgfqpoint{8.712326in}{2.303207in}}{\pgfqpoint{8.717370in}{2.303207in}}%
\pgfpathclose%
\pgfusepath{fill}%
\end{pgfscope}%
\begin{pgfscope}%
\pgfpathrectangle{\pgfqpoint{6.572727in}{0.473000in}}{\pgfqpoint{4.227273in}{3.311000in}}%
\pgfusepath{clip}%
\pgfsetbuttcap%
\pgfsetroundjoin%
\definecolor{currentfill}{rgb}{0.127568,0.566949,0.550556}%
\pgfsetfillcolor{currentfill}%
\pgfsetfillopacity{0.700000}%
\pgfsetlinewidth{0.000000pt}%
\definecolor{currentstroke}{rgb}{0.000000,0.000000,0.000000}%
\pgfsetstrokecolor{currentstroke}%
\pgfsetstrokeopacity{0.700000}%
\pgfsetdash{}{0pt}%
\pgfpathmoveto{\pgfqpoint{8.369347in}{3.483533in}}%
\pgfpathcurveto{\pgfqpoint{8.374390in}{3.483533in}}{\pgfqpoint{8.379228in}{3.485537in}}{\pgfqpoint{8.382795in}{3.489104in}}%
\pgfpathcurveto{\pgfqpoint{8.386361in}{3.492670in}}{\pgfqpoint{8.388365in}{3.497508in}}{\pgfqpoint{8.388365in}{3.502552in}}%
\pgfpathcurveto{\pgfqpoint{8.388365in}{3.507595in}}{\pgfqpoint{8.386361in}{3.512433in}}{\pgfqpoint{8.382795in}{3.515999in}}%
\pgfpathcurveto{\pgfqpoint{8.379228in}{3.519566in}}{\pgfqpoint{8.374390in}{3.521570in}}{\pgfqpoint{8.369347in}{3.521570in}}%
\pgfpathcurveto{\pgfqpoint{8.364303in}{3.521570in}}{\pgfqpoint{8.359465in}{3.519566in}}{\pgfqpoint{8.355899in}{3.515999in}}%
\pgfpathcurveto{\pgfqpoint{8.352332in}{3.512433in}}{\pgfqpoint{8.350329in}{3.507595in}}{\pgfqpoint{8.350329in}{3.502552in}}%
\pgfpathcurveto{\pgfqpoint{8.350329in}{3.497508in}}{\pgfqpoint{8.352332in}{3.492670in}}{\pgfqpoint{8.355899in}{3.489104in}}%
\pgfpathcurveto{\pgfqpoint{8.359465in}{3.485537in}}{\pgfqpoint{8.364303in}{3.483533in}}{\pgfqpoint{8.369347in}{3.483533in}}%
\pgfpathclose%
\pgfusepath{fill}%
\end{pgfscope}%
\begin{pgfscope}%
\pgfpathrectangle{\pgfqpoint{6.572727in}{0.473000in}}{\pgfqpoint{4.227273in}{3.311000in}}%
\pgfusepath{clip}%
\pgfsetbuttcap%
\pgfsetroundjoin%
\definecolor{currentfill}{rgb}{0.127568,0.566949,0.550556}%
\pgfsetfillcolor{currentfill}%
\pgfsetfillopacity{0.700000}%
\pgfsetlinewidth{0.000000pt}%
\definecolor{currentstroke}{rgb}{0.000000,0.000000,0.000000}%
\pgfsetstrokecolor{currentstroke}%
\pgfsetstrokeopacity{0.700000}%
\pgfsetdash{}{0pt}%
\pgfpathmoveto{\pgfqpoint{7.475106in}{2.848774in}}%
\pgfpathcurveto{\pgfqpoint{7.480150in}{2.848774in}}{\pgfqpoint{7.484987in}{2.850778in}}{\pgfqpoint{7.488554in}{2.854344in}}%
\pgfpathcurveto{\pgfqpoint{7.492120in}{2.857910in}}{\pgfqpoint{7.494124in}{2.862748in}}{\pgfqpoint{7.494124in}{2.867792in}}%
\pgfpathcurveto{\pgfqpoint{7.494124in}{2.872836in}}{\pgfqpoint{7.492120in}{2.877673in}}{\pgfqpoint{7.488554in}{2.881240in}}%
\pgfpathcurveto{\pgfqpoint{7.484987in}{2.884806in}}{\pgfqpoint{7.480150in}{2.886810in}}{\pgfqpoint{7.475106in}{2.886810in}}%
\pgfpathcurveto{\pgfqpoint{7.470062in}{2.886810in}}{\pgfqpoint{7.465224in}{2.884806in}}{\pgfqpoint{7.461658in}{2.881240in}}%
\pgfpathcurveto{\pgfqpoint{7.458092in}{2.877673in}}{\pgfqpoint{7.456088in}{2.872836in}}{\pgfqpoint{7.456088in}{2.867792in}}%
\pgfpathcurveto{\pgfqpoint{7.456088in}{2.862748in}}{\pgfqpoint{7.458092in}{2.857910in}}{\pgfqpoint{7.461658in}{2.854344in}}%
\pgfpathcurveto{\pgfqpoint{7.465224in}{2.850778in}}{\pgfqpoint{7.470062in}{2.848774in}}{\pgfqpoint{7.475106in}{2.848774in}}%
\pgfpathclose%
\pgfusepath{fill}%
\end{pgfscope}%
\begin{pgfscope}%
\pgfpathrectangle{\pgfqpoint{6.572727in}{0.473000in}}{\pgfqpoint{4.227273in}{3.311000in}}%
\pgfusepath{clip}%
\pgfsetbuttcap%
\pgfsetroundjoin%
\definecolor{currentfill}{rgb}{0.127568,0.566949,0.550556}%
\pgfsetfillcolor{currentfill}%
\pgfsetfillopacity{0.700000}%
\pgfsetlinewidth{0.000000pt}%
\definecolor{currentstroke}{rgb}{0.000000,0.000000,0.000000}%
\pgfsetstrokecolor{currentstroke}%
\pgfsetstrokeopacity{0.700000}%
\pgfsetdash{}{0pt}%
\pgfpathmoveto{\pgfqpoint{8.010337in}{1.103993in}}%
\pgfpathcurveto{\pgfqpoint{8.015381in}{1.103993in}}{\pgfqpoint{8.020218in}{1.105997in}}{\pgfqpoint{8.023785in}{1.109563in}}%
\pgfpathcurveto{\pgfqpoint{8.027351in}{1.113129in}}{\pgfqpoint{8.029355in}{1.117967in}}{\pgfqpoint{8.029355in}{1.123011in}}%
\pgfpathcurveto{\pgfqpoint{8.029355in}{1.128055in}}{\pgfqpoint{8.027351in}{1.132892in}}{\pgfqpoint{8.023785in}{1.136459in}}%
\pgfpathcurveto{\pgfqpoint{8.020218in}{1.140025in}}{\pgfqpoint{8.015381in}{1.142029in}}{\pgfqpoint{8.010337in}{1.142029in}}%
\pgfpathcurveto{\pgfqpoint{8.005293in}{1.142029in}}{\pgfqpoint{8.000456in}{1.140025in}}{\pgfqpoint{7.996889in}{1.136459in}}%
\pgfpathcurveto{\pgfqpoint{7.993323in}{1.132892in}}{\pgfqpoint{7.991319in}{1.128055in}}{\pgfqpoint{7.991319in}{1.123011in}}%
\pgfpathcurveto{\pgfqpoint{7.991319in}{1.117967in}}{\pgfqpoint{7.993323in}{1.113129in}}{\pgfqpoint{7.996889in}{1.109563in}}%
\pgfpathcurveto{\pgfqpoint{8.000456in}{1.105997in}}{\pgfqpoint{8.005293in}{1.103993in}}{\pgfqpoint{8.010337in}{1.103993in}}%
\pgfpathclose%
\pgfusepath{fill}%
\end{pgfscope}%
\begin{pgfscope}%
\pgfpathrectangle{\pgfqpoint{6.572727in}{0.473000in}}{\pgfqpoint{4.227273in}{3.311000in}}%
\pgfusepath{clip}%
\pgfsetbuttcap%
\pgfsetroundjoin%
\definecolor{currentfill}{rgb}{0.993248,0.906157,0.143936}%
\pgfsetfillcolor{currentfill}%
\pgfsetfillopacity{0.700000}%
\pgfsetlinewidth{0.000000pt}%
\definecolor{currentstroke}{rgb}{0.000000,0.000000,0.000000}%
\pgfsetstrokecolor{currentstroke}%
\pgfsetstrokeopacity{0.700000}%
\pgfsetdash{}{0pt}%
\pgfpathmoveto{\pgfqpoint{9.609688in}{1.665214in}}%
\pgfpathcurveto{\pgfqpoint{9.614731in}{1.665214in}}{\pgfqpoint{9.619569in}{1.667218in}}{\pgfqpoint{9.623135in}{1.670784in}}%
\pgfpathcurveto{\pgfqpoint{9.626702in}{1.674351in}}{\pgfqpoint{9.628706in}{1.679189in}}{\pgfqpoint{9.628706in}{1.684232in}}%
\pgfpathcurveto{\pgfqpoint{9.628706in}{1.689276in}}{\pgfqpoint{9.626702in}{1.694114in}}{\pgfqpoint{9.623135in}{1.697680in}}%
\pgfpathcurveto{\pgfqpoint{9.619569in}{1.701246in}}{\pgfqpoint{9.614731in}{1.703250in}}{\pgfqpoint{9.609688in}{1.703250in}}%
\pgfpathcurveto{\pgfqpoint{9.604644in}{1.703250in}}{\pgfqpoint{9.599806in}{1.701246in}}{\pgfqpoint{9.596240in}{1.697680in}}%
\pgfpathcurveto{\pgfqpoint{9.592673in}{1.694114in}}{\pgfqpoint{9.590669in}{1.689276in}}{\pgfqpoint{9.590669in}{1.684232in}}%
\pgfpathcurveto{\pgfqpoint{9.590669in}{1.679189in}}{\pgfqpoint{9.592673in}{1.674351in}}{\pgfqpoint{9.596240in}{1.670784in}}%
\pgfpathcurveto{\pgfqpoint{9.599806in}{1.667218in}}{\pgfqpoint{9.604644in}{1.665214in}}{\pgfqpoint{9.609688in}{1.665214in}}%
\pgfpathclose%
\pgfusepath{fill}%
\end{pgfscope}%
\begin{pgfscope}%
\pgfpathrectangle{\pgfqpoint{6.572727in}{0.473000in}}{\pgfqpoint{4.227273in}{3.311000in}}%
\pgfusepath{clip}%
\pgfsetbuttcap%
\pgfsetroundjoin%
\definecolor{currentfill}{rgb}{0.993248,0.906157,0.143936}%
\pgfsetfillcolor{currentfill}%
\pgfsetfillopacity{0.700000}%
\pgfsetlinewidth{0.000000pt}%
\definecolor{currentstroke}{rgb}{0.000000,0.000000,0.000000}%
\pgfsetstrokecolor{currentstroke}%
\pgfsetstrokeopacity{0.700000}%
\pgfsetdash{}{0pt}%
\pgfpathmoveto{\pgfqpoint{9.953565in}{2.155748in}}%
\pgfpathcurveto{\pgfqpoint{9.958609in}{2.155748in}}{\pgfqpoint{9.963446in}{2.157752in}}{\pgfqpoint{9.967013in}{2.161319in}}%
\pgfpathcurveto{\pgfqpoint{9.970579in}{2.164885in}}{\pgfqpoint{9.972583in}{2.169723in}}{\pgfqpoint{9.972583in}{2.174767in}}%
\pgfpathcurveto{\pgfqpoint{9.972583in}{2.179810in}}{\pgfqpoint{9.970579in}{2.184648in}}{\pgfqpoint{9.967013in}{2.188214in}}%
\pgfpathcurveto{\pgfqpoint{9.963446in}{2.191781in}}{\pgfqpoint{9.958609in}{2.193785in}}{\pgfqpoint{9.953565in}{2.193785in}}%
\pgfpathcurveto{\pgfqpoint{9.948521in}{2.193785in}}{\pgfqpoint{9.943683in}{2.191781in}}{\pgfqpoint{9.940117in}{2.188214in}}%
\pgfpathcurveto{\pgfqpoint{9.936551in}{2.184648in}}{\pgfqpoint{9.934547in}{2.179810in}}{\pgfqpoint{9.934547in}{2.174767in}}%
\pgfpathcurveto{\pgfqpoint{9.934547in}{2.169723in}}{\pgfqpoint{9.936551in}{2.164885in}}{\pgfqpoint{9.940117in}{2.161319in}}%
\pgfpathcurveto{\pgfqpoint{9.943683in}{2.157752in}}{\pgfqpoint{9.948521in}{2.155748in}}{\pgfqpoint{9.953565in}{2.155748in}}%
\pgfpathclose%
\pgfusepath{fill}%
\end{pgfscope}%
\begin{pgfscope}%
\pgfpathrectangle{\pgfqpoint{6.572727in}{0.473000in}}{\pgfqpoint{4.227273in}{3.311000in}}%
\pgfusepath{clip}%
\pgfsetbuttcap%
\pgfsetroundjoin%
\definecolor{currentfill}{rgb}{0.127568,0.566949,0.550556}%
\pgfsetfillcolor{currentfill}%
\pgfsetfillopacity{0.700000}%
\pgfsetlinewidth{0.000000pt}%
\definecolor{currentstroke}{rgb}{0.000000,0.000000,0.000000}%
\pgfsetstrokecolor{currentstroke}%
\pgfsetstrokeopacity{0.700000}%
\pgfsetdash{}{0pt}%
\pgfpathmoveto{\pgfqpoint{7.889949in}{1.691773in}}%
\pgfpathcurveto{\pgfqpoint{7.894993in}{1.691773in}}{\pgfqpoint{7.899831in}{1.693777in}}{\pgfqpoint{7.903397in}{1.697343in}}%
\pgfpathcurveto{\pgfqpoint{7.906963in}{1.700909in}}{\pgfqpoint{7.908967in}{1.705747in}}{\pgfqpoint{7.908967in}{1.710791in}}%
\pgfpathcurveto{\pgfqpoint{7.908967in}{1.715835in}}{\pgfqpoint{7.906963in}{1.720672in}}{\pgfqpoint{7.903397in}{1.724239in}}%
\pgfpathcurveto{\pgfqpoint{7.899831in}{1.727805in}}{\pgfqpoint{7.894993in}{1.729809in}}{\pgfqpoint{7.889949in}{1.729809in}}%
\pgfpathcurveto{\pgfqpoint{7.884905in}{1.729809in}}{\pgfqpoint{7.880068in}{1.727805in}}{\pgfqpoint{7.876501in}{1.724239in}}%
\pgfpathcurveto{\pgfqpoint{7.872935in}{1.720672in}}{\pgfqpoint{7.870931in}{1.715835in}}{\pgfqpoint{7.870931in}{1.710791in}}%
\pgfpathcurveto{\pgfqpoint{7.870931in}{1.705747in}}{\pgfqpoint{7.872935in}{1.700909in}}{\pgfqpoint{7.876501in}{1.697343in}}%
\pgfpathcurveto{\pgfqpoint{7.880068in}{1.693777in}}{\pgfqpoint{7.884905in}{1.691773in}}{\pgfqpoint{7.889949in}{1.691773in}}%
\pgfpathclose%
\pgfusepath{fill}%
\end{pgfscope}%
\begin{pgfscope}%
\pgfpathrectangle{\pgfqpoint{6.572727in}{0.473000in}}{\pgfqpoint{4.227273in}{3.311000in}}%
\pgfusepath{clip}%
\pgfsetbuttcap%
\pgfsetroundjoin%
\definecolor{currentfill}{rgb}{0.127568,0.566949,0.550556}%
\pgfsetfillcolor{currentfill}%
\pgfsetfillopacity{0.700000}%
\pgfsetlinewidth{0.000000pt}%
\definecolor{currentstroke}{rgb}{0.000000,0.000000,0.000000}%
\pgfsetstrokecolor{currentstroke}%
\pgfsetstrokeopacity{0.700000}%
\pgfsetdash{}{0pt}%
\pgfpathmoveto{\pgfqpoint{7.507862in}{1.150929in}}%
\pgfpathcurveto{\pgfqpoint{7.512906in}{1.150929in}}{\pgfqpoint{7.517744in}{1.152933in}}{\pgfqpoint{7.521310in}{1.156500in}}%
\pgfpathcurveto{\pgfqpoint{7.524877in}{1.160066in}}{\pgfqpoint{7.526880in}{1.164904in}}{\pgfqpoint{7.526880in}{1.169947in}}%
\pgfpathcurveto{\pgfqpoint{7.526880in}{1.174991in}}{\pgfqpoint{7.524877in}{1.179829in}}{\pgfqpoint{7.521310in}{1.183395in}}%
\pgfpathcurveto{\pgfqpoint{7.517744in}{1.186962in}}{\pgfqpoint{7.512906in}{1.188966in}}{\pgfqpoint{7.507862in}{1.188966in}}%
\pgfpathcurveto{\pgfqpoint{7.502819in}{1.188966in}}{\pgfqpoint{7.497981in}{1.186962in}}{\pgfqpoint{7.494414in}{1.183395in}}%
\pgfpathcurveto{\pgfqpoint{7.490848in}{1.179829in}}{\pgfqpoint{7.488844in}{1.174991in}}{\pgfqpoint{7.488844in}{1.169947in}}%
\pgfpathcurveto{\pgfqpoint{7.488844in}{1.164904in}}{\pgfqpoint{7.490848in}{1.160066in}}{\pgfqpoint{7.494414in}{1.156500in}}%
\pgfpathcurveto{\pgfqpoint{7.497981in}{1.152933in}}{\pgfqpoint{7.502819in}{1.150929in}}{\pgfqpoint{7.507862in}{1.150929in}}%
\pgfpathclose%
\pgfusepath{fill}%
\end{pgfscope}%
\begin{pgfscope}%
\pgfpathrectangle{\pgfqpoint{6.572727in}{0.473000in}}{\pgfqpoint{4.227273in}{3.311000in}}%
\pgfusepath{clip}%
\pgfsetbuttcap%
\pgfsetroundjoin%
\definecolor{currentfill}{rgb}{0.127568,0.566949,0.550556}%
\pgfsetfillcolor{currentfill}%
\pgfsetfillopacity{0.700000}%
\pgfsetlinewidth{0.000000pt}%
\definecolor{currentstroke}{rgb}{0.000000,0.000000,0.000000}%
\pgfsetstrokecolor{currentstroke}%
\pgfsetstrokeopacity{0.700000}%
\pgfsetdash{}{0pt}%
\pgfpathmoveto{\pgfqpoint{7.851340in}{3.160155in}}%
\pgfpathcurveto{\pgfqpoint{7.856384in}{3.160155in}}{\pgfqpoint{7.861222in}{3.162158in}}{\pgfqpoint{7.864788in}{3.165725in}}%
\pgfpathcurveto{\pgfqpoint{7.868354in}{3.169291in}}{\pgfqpoint{7.870358in}{3.174129in}}{\pgfqpoint{7.870358in}{3.179173in}}%
\pgfpathcurveto{\pgfqpoint{7.870358in}{3.184216in}}{\pgfqpoint{7.868354in}{3.189054in}}{\pgfqpoint{7.864788in}{3.192621in}}%
\pgfpathcurveto{\pgfqpoint{7.861222in}{3.196187in}}{\pgfqpoint{7.856384in}{3.198191in}}{\pgfqpoint{7.851340in}{3.198191in}}%
\pgfpathcurveto{\pgfqpoint{7.846296in}{3.198191in}}{\pgfqpoint{7.841459in}{3.196187in}}{\pgfqpoint{7.837892in}{3.192621in}}%
\pgfpathcurveto{\pgfqpoint{7.834326in}{3.189054in}}{\pgfqpoint{7.832322in}{3.184216in}}{\pgfqpoint{7.832322in}{3.179173in}}%
\pgfpathcurveto{\pgfqpoint{7.832322in}{3.174129in}}{\pgfqpoint{7.834326in}{3.169291in}}{\pgfqpoint{7.837892in}{3.165725in}}%
\pgfpathcurveto{\pgfqpoint{7.841459in}{3.162158in}}{\pgfqpoint{7.846296in}{3.160155in}}{\pgfqpoint{7.851340in}{3.160155in}}%
\pgfpathclose%
\pgfusepath{fill}%
\end{pgfscope}%
\begin{pgfscope}%
\pgfpathrectangle{\pgfqpoint{6.572727in}{0.473000in}}{\pgfqpoint{4.227273in}{3.311000in}}%
\pgfusepath{clip}%
\pgfsetbuttcap%
\pgfsetroundjoin%
\definecolor{currentfill}{rgb}{0.127568,0.566949,0.550556}%
\pgfsetfillcolor{currentfill}%
\pgfsetfillopacity{0.700000}%
\pgfsetlinewidth{0.000000pt}%
\definecolor{currentstroke}{rgb}{0.000000,0.000000,0.000000}%
\pgfsetstrokecolor{currentstroke}%
\pgfsetstrokeopacity{0.700000}%
\pgfsetdash{}{0pt}%
\pgfpathmoveto{\pgfqpoint{7.773228in}{1.040659in}}%
\pgfpathcurveto{\pgfqpoint{7.778272in}{1.040659in}}{\pgfqpoint{7.783110in}{1.042663in}}{\pgfqpoint{7.786676in}{1.046229in}}%
\pgfpathcurveto{\pgfqpoint{7.790243in}{1.049796in}}{\pgfqpoint{7.792246in}{1.054634in}}{\pgfqpoint{7.792246in}{1.059677in}}%
\pgfpathcurveto{\pgfqpoint{7.792246in}{1.064721in}}{\pgfqpoint{7.790243in}{1.069559in}}{\pgfqpoint{7.786676in}{1.073125in}}%
\pgfpathcurveto{\pgfqpoint{7.783110in}{1.076692in}}{\pgfqpoint{7.778272in}{1.078695in}}{\pgfqpoint{7.773228in}{1.078695in}}%
\pgfpathcurveto{\pgfqpoint{7.768185in}{1.078695in}}{\pgfqpoint{7.763347in}{1.076692in}}{\pgfqpoint{7.759780in}{1.073125in}}%
\pgfpathcurveto{\pgfqpoint{7.756214in}{1.069559in}}{\pgfqpoint{7.754210in}{1.064721in}}{\pgfqpoint{7.754210in}{1.059677in}}%
\pgfpathcurveto{\pgfqpoint{7.754210in}{1.054634in}}{\pgfqpoint{7.756214in}{1.049796in}}{\pgfqpoint{7.759780in}{1.046229in}}%
\pgfpathcurveto{\pgfqpoint{7.763347in}{1.042663in}}{\pgfqpoint{7.768185in}{1.040659in}}{\pgfqpoint{7.773228in}{1.040659in}}%
\pgfpathclose%
\pgfusepath{fill}%
\end{pgfscope}%
\begin{pgfscope}%
\pgfpathrectangle{\pgfqpoint{6.572727in}{0.473000in}}{\pgfqpoint{4.227273in}{3.311000in}}%
\pgfusepath{clip}%
\pgfsetbuttcap%
\pgfsetroundjoin%
\definecolor{currentfill}{rgb}{0.127568,0.566949,0.550556}%
\pgfsetfillcolor{currentfill}%
\pgfsetfillopacity{0.700000}%
\pgfsetlinewidth{0.000000pt}%
\definecolor{currentstroke}{rgb}{0.000000,0.000000,0.000000}%
\pgfsetstrokecolor{currentstroke}%
\pgfsetstrokeopacity{0.700000}%
\pgfsetdash{}{0pt}%
\pgfpathmoveto{\pgfqpoint{7.470716in}{1.037051in}}%
\pgfpathcurveto{\pgfqpoint{7.475760in}{1.037051in}}{\pgfqpoint{7.480597in}{1.039055in}}{\pgfqpoint{7.484164in}{1.042622in}}%
\pgfpathcurveto{\pgfqpoint{7.487730in}{1.046188in}}{\pgfqpoint{7.489734in}{1.051026in}}{\pgfqpoint{7.489734in}{1.056069in}}%
\pgfpathcurveto{\pgfqpoint{7.489734in}{1.061113in}}{\pgfqpoint{7.487730in}{1.065951in}}{\pgfqpoint{7.484164in}{1.069517in}}%
\pgfpathcurveto{\pgfqpoint{7.480597in}{1.073084in}}{\pgfqpoint{7.475760in}{1.075088in}}{\pgfqpoint{7.470716in}{1.075088in}}%
\pgfpathcurveto{\pgfqpoint{7.465672in}{1.075088in}}{\pgfqpoint{7.460835in}{1.073084in}}{\pgfqpoint{7.457268in}{1.069517in}}%
\pgfpathcurveto{\pgfqpoint{7.453702in}{1.065951in}}{\pgfqpoint{7.451698in}{1.061113in}}{\pgfqpoint{7.451698in}{1.056069in}}%
\pgfpathcurveto{\pgfqpoint{7.451698in}{1.051026in}}{\pgfqpoint{7.453702in}{1.046188in}}{\pgfqpoint{7.457268in}{1.042622in}}%
\pgfpathcurveto{\pgfqpoint{7.460835in}{1.039055in}}{\pgfqpoint{7.465672in}{1.037051in}}{\pgfqpoint{7.470716in}{1.037051in}}%
\pgfpathclose%
\pgfusepath{fill}%
\end{pgfscope}%
\begin{pgfscope}%
\pgfpathrectangle{\pgfqpoint{6.572727in}{0.473000in}}{\pgfqpoint{4.227273in}{3.311000in}}%
\pgfusepath{clip}%
\pgfsetbuttcap%
\pgfsetroundjoin%
\definecolor{currentfill}{rgb}{0.127568,0.566949,0.550556}%
\pgfsetfillcolor{currentfill}%
\pgfsetfillopacity{0.700000}%
\pgfsetlinewidth{0.000000pt}%
\definecolor{currentstroke}{rgb}{0.000000,0.000000,0.000000}%
\pgfsetstrokecolor{currentstroke}%
\pgfsetstrokeopacity{0.700000}%
\pgfsetdash{}{0pt}%
\pgfpathmoveto{\pgfqpoint{7.406007in}{1.406681in}}%
\pgfpathcurveto{\pgfqpoint{7.411050in}{1.406681in}}{\pgfqpoint{7.415888in}{1.408685in}}{\pgfqpoint{7.419455in}{1.412252in}}%
\pgfpathcurveto{\pgfqpoint{7.423021in}{1.415818in}}{\pgfqpoint{7.425025in}{1.420656in}}{\pgfqpoint{7.425025in}{1.425700in}}%
\pgfpathcurveto{\pgfqpoint{7.425025in}{1.430743in}}{\pgfqpoint{7.423021in}{1.435581in}}{\pgfqpoint{7.419455in}{1.439147in}}%
\pgfpathcurveto{\pgfqpoint{7.415888in}{1.442714in}}{\pgfqpoint{7.411050in}{1.444718in}}{\pgfqpoint{7.406007in}{1.444718in}}%
\pgfpathcurveto{\pgfqpoint{7.400963in}{1.444718in}}{\pgfqpoint{7.396125in}{1.442714in}}{\pgfqpoint{7.392559in}{1.439147in}}%
\pgfpathcurveto{\pgfqpoint{7.388993in}{1.435581in}}{\pgfqpoint{7.386989in}{1.430743in}}{\pgfqpoint{7.386989in}{1.425700in}}%
\pgfpathcurveto{\pgfqpoint{7.386989in}{1.420656in}}{\pgfqpoint{7.388993in}{1.415818in}}{\pgfqpoint{7.392559in}{1.412252in}}%
\pgfpathcurveto{\pgfqpoint{7.396125in}{1.408685in}}{\pgfqpoint{7.400963in}{1.406681in}}{\pgfqpoint{7.406007in}{1.406681in}}%
\pgfpathclose%
\pgfusepath{fill}%
\end{pgfscope}%
\begin{pgfscope}%
\pgfpathrectangle{\pgfqpoint{6.572727in}{0.473000in}}{\pgfqpoint{4.227273in}{3.311000in}}%
\pgfusepath{clip}%
\pgfsetbuttcap%
\pgfsetroundjoin%
\definecolor{currentfill}{rgb}{0.127568,0.566949,0.550556}%
\pgfsetfillcolor{currentfill}%
\pgfsetfillopacity{0.700000}%
\pgfsetlinewidth{0.000000pt}%
\definecolor{currentstroke}{rgb}{0.000000,0.000000,0.000000}%
\pgfsetstrokecolor{currentstroke}%
\pgfsetstrokeopacity{0.700000}%
\pgfsetdash{}{0pt}%
\pgfpathmoveto{\pgfqpoint{7.992728in}{1.123044in}}%
\pgfpathcurveto{\pgfqpoint{7.997772in}{1.123044in}}{\pgfqpoint{8.002610in}{1.125048in}}{\pgfqpoint{8.006176in}{1.128614in}}%
\pgfpathcurveto{\pgfqpoint{8.009743in}{1.132181in}}{\pgfqpoint{8.011746in}{1.137018in}}{\pgfqpoint{8.011746in}{1.142062in}}%
\pgfpathcurveto{\pgfqpoint{8.011746in}{1.147106in}}{\pgfqpoint{8.009743in}{1.151943in}}{\pgfqpoint{8.006176in}{1.155510in}}%
\pgfpathcurveto{\pgfqpoint{8.002610in}{1.159076in}}{\pgfqpoint{7.997772in}{1.161080in}}{\pgfqpoint{7.992728in}{1.161080in}}%
\pgfpathcurveto{\pgfqpoint{7.987685in}{1.161080in}}{\pgfqpoint{7.982847in}{1.159076in}}{\pgfqpoint{7.979280in}{1.155510in}}%
\pgfpathcurveto{\pgfqpoint{7.975714in}{1.151943in}}{\pgfqpoint{7.973710in}{1.147106in}}{\pgfqpoint{7.973710in}{1.142062in}}%
\pgfpathcurveto{\pgfqpoint{7.973710in}{1.137018in}}{\pgfqpoint{7.975714in}{1.132181in}}{\pgfqpoint{7.979280in}{1.128614in}}%
\pgfpathcurveto{\pgfqpoint{7.982847in}{1.125048in}}{\pgfqpoint{7.987685in}{1.123044in}}{\pgfqpoint{7.992728in}{1.123044in}}%
\pgfpathclose%
\pgfusepath{fill}%
\end{pgfscope}%
\begin{pgfscope}%
\pgfpathrectangle{\pgfqpoint{6.572727in}{0.473000in}}{\pgfqpoint{4.227273in}{3.311000in}}%
\pgfusepath{clip}%
\pgfsetbuttcap%
\pgfsetroundjoin%
\definecolor{currentfill}{rgb}{0.127568,0.566949,0.550556}%
\pgfsetfillcolor{currentfill}%
\pgfsetfillopacity{0.700000}%
\pgfsetlinewidth{0.000000pt}%
\definecolor{currentstroke}{rgb}{0.000000,0.000000,0.000000}%
\pgfsetstrokecolor{currentstroke}%
\pgfsetstrokeopacity{0.700000}%
\pgfsetdash{}{0pt}%
\pgfpathmoveto{\pgfqpoint{8.638296in}{1.591972in}}%
\pgfpathcurveto{\pgfqpoint{8.643340in}{1.591972in}}{\pgfqpoint{8.648178in}{1.593975in}}{\pgfqpoint{8.651744in}{1.597542in}}%
\pgfpathcurveto{\pgfqpoint{8.655311in}{1.601108in}}{\pgfqpoint{8.657315in}{1.605946in}}{\pgfqpoint{8.657315in}{1.610990in}}%
\pgfpathcurveto{\pgfqpoint{8.657315in}{1.616033in}}{\pgfqpoint{8.655311in}{1.620871in}}{\pgfqpoint{8.651744in}{1.624438in}}%
\pgfpathcurveto{\pgfqpoint{8.648178in}{1.628004in}}{\pgfqpoint{8.643340in}{1.630008in}}{\pgfqpoint{8.638296in}{1.630008in}}%
\pgfpathcurveto{\pgfqpoint{8.633253in}{1.630008in}}{\pgfqpoint{8.628415in}{1.628004in}}{\pgfqpoint{8.624848in}{1.624438in}}%
\pgfpathcurveto{\pgfqpoint{8.621282in}{1.620871in}}{\pgfqpoint{8.619278in}{1.616033in}}{\pgfqpoint{8.619278in}{1.610990in}}%
\pgfpathcurveto{\pgfqpoint{8.619278in}{1.605946in}}{\pgfqpoint{8.621282in}{1.601108in}}{\pgfqpoint{8.624848in}{1.597542in}}%
\pgfpathcurveto{\pgfqpoint{8.628415in}{1.593975in}}{\pgfqpoint{8.633253in}{1.591972in}}{\pgfqpoint{8.638296in}{1.591972in}}%
\pgfpathclose%
\pgfusepath{fill}%
\end{pgfscope}%
\begin{pgfscope}%
\pgfpathrectangle{\pgfqpoint{6.572727in}{0.473000in}}{\pgfqpoint{4.227273in}{3.311000in}}%
\pgfusepath{clip}%
\pgfsetbuttcap%
\pgfsetroundjoin%
\definecolor{currentfill}{rgb}{0.127568,0.566949,0.550556}%
\pgfsetfillcolor{currentfill}%
\pgfsetfillopacity{0.700000}%
\pgfsetlinewidth{0.000000pt}%
\definecolor{currentstroke}{rgb}{0.000000,0.000000,0.000000}%
\pgfsetstrokecolor{currentstroke}%
\pgfsetstrokeopacity{0.700000}%
\pgfsetdash{}{0pt}%
\pgfpathmoveto{\pgfqpoint{7.577322in}{1.449858in}}%
\pgfpathcurveto{\pgfqpoint{7.582366in}{1.449858in}}{\pgfqpoint{7.587204in}{1.451862in}}{\pgfqpoint{7.590770in}{1.455429in}}%
\pgfpathcurveto{\pgfqpoint{7.594337in}{1.458995in}}{\pgfqpoint{7.596341in}{1.463833in}}{\pgfqpoint{7.596341in}{1.468876in}}%
\pgfpathcurveto{\pgfqpoint{7.596341in}{1.473920in}}{\pgfqpoint{7.594337in}{1.478758in}}{\pgfqpoint{7.590770in}{1.482324in}}%
\pgfpathcurveto{\pgfqpoint{7.587204in}{1.485891in}}{\pgfqpoint{7.582366in}{1.487895in}}{\pgfqpoint{7.577322in}{1.487895in}}%
\pgfpathcurveto{\pgfqpoint{7.572279in}{1.487895in}}{\pgfqpoint{7.567441in}{1.485891in}}{\pgfqpoint{7.563875in}{1.482324in}}%
\pgfpathcurveto{\pgfqpoint{7.560308in}{1.478758in}}{\pgfqpoint{7.558304in}{1.473920in}}{\pgfqpoint{7.558304in}{1.468876in}}%
\pgfpathcurveto{\pgfqpoint{7.558304in}{1.463833in}}{\pgfqpoint{7.560308in}{1.458995in}}{\pgfqpoint{7.563875in}{1.455429in}}%
\pgfpathcurveto{\pgfqpoint{7.567441in}{1.451862in}}{\pgfqpoint{7.572279in}{1.449858in}}{\pgfqpoint{7.577322in}{1.449858in}}%
\pgfpathclose%
\pgfusepath{fill}%
\end{pgfscope}%
\begin{pgfscope}%
\pgfpathrectangle{\pgfqpoint{6.572727in}{0.473000in}}{\pgfqpoint{4.227273in}{3.311000in}}%
\pgfusepath{clip}%
\pgfsetbuttcap%
\pgfsetroundjoin%
\definecolor{currentfill}{rgb}{0.127568,0.566949,0.550556}%
\pgfsetfillcolor{currentfill}%
\pgfsetfillopacity{0.700000}%
\pgfsetlinewidth{0.000000pt}%
\definecolor{currentstroke}{rgb}{0.000000,0.000000,0.000000}%
\pgfsetstrokecolor{currentstroke}%
\pgfsetstrokeopacity{0.700000}%
\pgfsetdash{}{0pt}%
\pgfpathmoveto{\pgfqpoint{7.253898in}{1.023302in}}%
\pgfpathcurveto{\pgfqpoint{7.258942in}{1.023302in}}{\pgfqpoint{7.263780in}{1.025306in}}{\pgfqpoint{7.267346in}{1.028872in}}%
\pgfpathcurveto{\pgfqpoint{7.270912in}{1.032438in}}{\pgfqpoint{7.272916in}{1.037276in}}{\pgfqpoint{7.272916in}{1.042320in}}%
\pgfpathcurveto{\pgfqpoint{7.272916in}{1.047363in}}{\pgfqpoint{7.270912in}{1.052201in}}{\pgfqpoint{7.267346in}{1.055768in}}%
\pgfpathcurveto{\pgfqpoint{7.263780in}{1.059334in}}{\pgfqpoint{7.258942in}{1.061338in}}{\pgfqpoint{7.253898in}{1.061338in}}%
\pgfpathcurveto{\pgfqpoint{7.248855in}{1.061338in}}{\pgfqpoint{7.244017in}{1.059334in}}{\pgfqpoint{7.240450in}{1.055768in}}%
\pgfpathcurveto{\pgfqpoint{7.236884in}{1.052201in}}{\pgfqpoint{7.234880in}{1.047363in}}{\pgfqpoint{7.234880in}{1.042320in}}%
\pgfpathcurveto{\pgfqpoint{7.234880in}{1.037276in}}{\pgfqpoint{7.236884in}{1.032438in}}{\pgfqpoint{7.240450in}{1.028872in}}%
\pgfpathcurveto{\pgfqpoint{7.244017in}{1.025306in}}{\pgfqpoint{7.248855in}{1.023302in}}{\pgfqpoint{7.253898in}{1.023302in}}%
\pgfpathclose%
\pgfusepath{fill}%
\end{pgfscope}%
\begin{pgfscope}%
\pgfpathrectangle{\pgfqpoint{6.572727in}{0.473000in}}{\pgfqpoint{4.227273in}{3.311000in}}%
\pgfusepath{clip}%
\pgfsetbuttcap%
\pgfsetroundjoin%
\definecolor{currentfill}{rgb}{0.993248,0.906157,0.143936}%
\pgfsetfillcolor{currentfill}%
\pgfsetfillopacity{0.700000}%
\pgfsetlinewidth{0.000000pt}%
\definecolor{currentstroke}{rgb}{0.000000,0.000000,0.000000}%
\pgfsetstrokecolor{currentstroke}%
\pgfsetstrokeopacity{0.700000}%
\pgfsetdash{}{0pt}%
\pgfpathmoveto{\pgfqpoint{10.270034in}{1.805085in}}%
\pgfpathcurveto{\pgfqpoint{10.275078in}{1.805085in}}{\pgfqpoint{10.279916in}{1.807089in}}{\pgfqpoint{10.283482in}{1.810655in}}%
\pgfpathcurveto{\pgfqpoint{10.287049in}{1.814222in}}{\pgfqpoint{10.289053in}{1.819060in}}{\pgfqpoint{10.289053in}{1.824103in}}%
\pgfpathcurveto{\pgfqpoint{10.289053in}{1.829147in}}{\pgfqpoint{10.287049in}{1.833985in}}{\pgfqpoint{10.283482in}{1.837551in}}%
\pgfpathcurveto{\pgfqpoint{10.279916in}{1.841117in}}{\pgfqpoint{10.275078in}{1.843121in}}{\pgfqpoint{10.270034in}{1.843121in}}%
\pgfpathcurveto{\pgfqpoint{10.264991in}{1.843121in}}{\pgfqpoint{10.260153in}{1.841117in}}{\pgfqpoint{10.256587in}{1.837551in}}%
\pgfpathcurveto{\pgfqpoint{10.253020in}{1.833985in}}{\pgfqpoint{10.251016in}{1.829147in}}{\pgfqpoint{10.251016in}{1.824103in}}%
\pgfpathcurveto{\pgfqpoint{10.251016in}{1.819060in}}{\pgfqpoint{10.253020in}{1.814222in}}{\pgfqpoint{10.256587in}{1.810655in}}%
\pgfpathcurveto{\pgfqpoint{10.260153in}{1.807089in}}{\pgfqpoint{10.264991in}{1.805085in}}{\pgfqpoint{10.270034in}{1.805085in}}%
\pgfpathclose%
\pgfusepath{fill}%
\end{pgfscope}%
\begin{pgfscope}%
\pgfpathrectangle{\pgfqpoint{6.572727in}{0.473000in}}{\pgfqpoint{4.227273in}{3.311000in}}%
\pgfusepath{clip}%
\pgfsetbuttcap%
\pgfsetroundjoin%
\definecolor{currentfill}{rgb}{0.127568,0.566949,0.550556}%
\pgfsetfillcolor{currentfill}%
\pgfsetfillopacity{0.700000}%
\pgfsetlinewidth{0.000000pt}%
\definecolor{currentstroke}{rgb}{0.000000,0.000000,0.000000}%
\pgfsetstrokecolor{currentstroke}%
\pgfsetstrokeopacity{0.700000}%
\pgfsetdash{}{0pt}%
\pgfpathmoveto{\pgfqpoint{7.981775in}{1.659148in}}%
\pgfpathcurveto{\pgfqpoint{7.986819in}{1.659148in}}{\pgfqpoint{7.991656in}{1.661152in}}{\pgfqpoint{7.995223in}{1.664719in}}%
\pgfpathcurveto{\pgfqpoint{7.998789in}{1.668285in}}{\pgfqpoint{8.000793in}{1.673123in}}{\pgfqpoint{8.000793in}{1.678167in}}%
\pgfpathcurveto{\pgfqpoint{8.000793in}{1.683210in}}{\pgfqpoint{7.998789in}{1.688048in}}{\pgfqpoint{7.995223in}{1.691614in}}%
\pgfpathcurveto{\pgfqpoint{7.991656in}{1.695181in}}{\pgfqpoint{7.986819in}{1.697185in}}{\pgfqpoint{7.981775in}{1.697185in}}%
\pgfpathcurveto{\pgfqpoint{7.976731in}{1.697185in}}{\pgfqpoint{7.971894in}{1.695181in}}{\pgfqpoint{7.968327in}{1.691614in}}%
\pgfpathcurveto{\pgfqpoint{7.964761in}{1.688048in}}{\pgfqpoint{7.962757in}{1.683210in}}{\pgfqpoint{7.962757in}{1.678167in}}%
\pgfpathcurveto{\pgfqpoint{7.962757in}{1.673123in}}{\pgfqpoint{7.964761in}{1.668285in}}{\pgfqpoint{7.968327in}{1.664719in}}%
\pgfpathcurveto{\pgfqpoint{7.971894in}{1.661152in}}{\pgfqpoint{7.976731in}{1.659148in}}{\pgfqpoint{7.981775in}{1.659148in}}%
\pgfpathclose%
\pgfusepath{fill}%
\end{pgfscope}%
\begin{pgfscope}%
\pgfpathrectangle{\pgfqpoint{6.572727in}{0.473000in}}{\pgfqpoint{4.227273in}{3.311000in}}%
\pgfusepath{clip}%
\pgfsetbuttcap%
\pgfsetroundjoin%
\definecolor{currentfill}{rgb}{0.993248,0.906157,0.143936}%
\pgfsetfillcolor{currentfill}%
\pgfsetfillopacity{0.700000}%
\pgfsetlinewidth{0.000000pt}%
\definecolor{currentstroke}{rgb}{0.000000,0.000000,0.000000}%
\pgfsetstrokecolor{currentstroke}%
\pgfsetstrokeopacity{0.700000}%
\pgfsetdash{}{0pt}%
\pgfpathmoveto{\pgfqpoint{9.923404in}{1.459486in}}%
\pgfpathcurveto{\pgfqpoint{9.928448in}{1.459486in}}{\pgfqpoint{9.933286in}{1.461490in}}{\pgfqpoint{9.936852in}{1.465056in}}%
\pgfpathcurveto{\pgfqpoint{9.940418in}{1.468622in}}{\pgfqpoint{9.942422in}{1.473460in}}{\pgfqpoint{9.942422in}{1.478504in}}%
\pgfpathcurveto{\pgfqpoint{9.942422in}{1.483548in}}{\pgfqpoint{9.940418in}{1.488385in}}{\pgfqpoint{9.936852in}{1.491952in}}%
\pgfpathcurveto{\pgfqpoint{9.933286in}{1.495518in}}{\pgfqpoint{9.928448in}{1.497522in}}{\pgfqpoint{9.923404in}{1.497522in}}%
\pgfpathcurveto{\pgfqpoint{9.918360in}{1.497522in}}{\pgfqpoint{9.913523in}{1.495518in}}{\pgfqpoint{9.909956in}{1.491952in}}%
\pgfpathcurveto{\pgfqpoint{9.906390in}{1.488385in}}{\pgfqpoint{9.904386in}{1.483548in}}{\pgfqpoint{9.904386in}{1.478504in}}%
\pgfpathcurveto{\pgfqpoint{9.904386in}{1.473460in}}{\pgfqpoint{9.906390in}{1.468622in}}{\pgfqpoint{9.909956in}{1.465056in}}%
\pgfpathcurveto{\pgfqpoint{9.913523in}{1.461490in}}{\pgfqpoint{9.918360in}{1.459486in}}{\pgfqpoint{9.923404in}{1.459486in}}%
\pgfpathclose%
\pgfusepath{fill}%
\end{pgfscope}%
\begin{pgfscope}%
\pgfpathrectangle{\pgfqpoint{6.572727in}{0.473000in}}{\pgfqpoint{4.227273in}{3.311000in}}%
\pgfusepath{clip}%
\pgfsetbuttcap%
\pgfsetroundjoin%
\definecolor{currentfill}{rgb}{0.127568,0.566949,0.550556}%
\pgfsetfillcolor{currentfill}%
\pgfsetfillopacity{0.700000}%
\pgfsetlinewidth{0.000000pt}%
\definecolor{currentstroke}{rgb}{0.000000,0.000000,0.000000}%
\pgfsetstrokecolor{currentstroke}%
\pgfsetstrokeopacity{0.700000}%
\pgfsetdash{}{0pt}%
\pgfpathmoveto{\pgfqpoint{7.723658in}{1.679230in}}%
\pgfpathcurveto{\pgfqpoint{7.728702in}{1.679230in}}{\pgfqpoint{7.733540in}{1.681234in}}{\pgfqpoint{7.737106in}{1.684800in}}%
\pgfpathcurveto{\pgfqpoint{7.740673in}{1.688367in}}{\pgfqpoint{7.742677in}{1.693205in}}{\pgfqpoint{7.742677in}{1.698248in}}%
\pgfpathcurveto{\pgfqpoint{7.742677in}{1.703292in}}{\pgfqpoint{7.740673in}{1.708130in}}{\pgfqpoint{7.737106in}{1.711696in}}%
\pgfpathcurveto{\pgfqpoint{7.733540in}{1.715262in}}{\pgfqpoint{7.728702in}{1.717266in}}{\pgfqpoint{7.723658in}{1.717266in}}%
\pgfpathcurveto{\pgfqpoint{7.718615in}{1.717266in}}{\pgfqpoint{7.713777in}{1.715262in}}{\pgfqpoint{7.710211in}{1.711696in}}%
\pgfpathcurveto{\pgfqpoint{7.706644in}{1.708130in}}{\pgfqpoint{7.704640in}{1.703292in}}{\pgfqpoint{7.704640in}{1.698248in}}%
\pgfpathcurveto{\pgfqpoint{7.704640in}{1.693205in}}{\pgfqpoint{7.706644in}{1.688367in}}{\pgfqpoint{7.710211in}{1.684800in}}%
\pgfpathcurveto{\pgfqpoint{7.713777in}{1.681234in}}{\pgfqpoint{7.718615in}{1.679230in}}{\pgfqpoint{7.723658in}{1.679230in}}%
\pgfpathclose%
\pgfusepath{fill}%
\end{pgfscope}%
\begin{pgfscope}%
\pgfpathrectangle{\pgfqpoint{6.572727in}{0.473000in}}{\pgfqpoint{4.227273in}{3.311000in}}%
\pgfusepath{clip}%
\pgfsetbuttcap%
\pgfsetroundjoin%
\definecolor{currentfill}{rgb}{0.993248,0.906157,0.143936}%
\pgfsetfillcolor{currentfill}%
\pgfsetfillopacity{0.700000}%
\pgfsetlinewidth{0.000000pt}%
\definecolor{currentstroke}{rgb}{0.000000,0.000000,0.000000}%
\pgfsetstrokecolor{currentstroke}%
\pgfsetstrokeopacity{0.700000}%
\pgfsetdash{}{0pt}%
\pgfpathmoveto{\pgfqpoint{10.607851in}{1.485616in}}%
\pgfpathcurveto{\pgfqpoint{10.612895in}{1.485616in}}{\pgfqpoint{10.617733in}{1.487620in}}{\pgfqpoint{10.621299in}{1.491187in}}%
\pgfpathcurveto{\pgfqpoint{10.624866in}{1.494753in}}{\pgfqpoint{10.626869in}{1.499591in}}{\pgfqpoint{10.626869in}{1.504635in}}%
\pgfpathcurveto{\pgfqpoint{10.626869in}{1.509678in}}{\pgfqpoint{10.624866in}{1.514516in}}{\pgfqpoint{10.621299in}{1.518082in}}%
\pgfpathcurveto{\pgfqpoint{10.617733in}{1.521649in}}{\pgfqpoint{10.612895in}{1.523653in}}{\pgfqpoint{10.607851in}{1.523653in}}%
\pgfpathcurveto{\pgfqpoint{10.602808in}{1.523653in}}{\pgfqpoint{10.597970in}{1.521649in}}{\pgfqpoint{10.594403in}{1.518082in}}%
\pgfpathcurveto{\pgfqpoint{10.590837in}{1.514516in}}{\pgfqpoint{10.588833in}{1.509678in}}{\pgfqpoint{10.588833in}{1.504635in}}%
\pgfpathcurveto{\pgfqpoint{10.588833in}{1.499591in}}{\pgfqpoint{10.590837in}{1.494753in}}{\pgfqpoint{10.594403in}{1.491187in}}%
\pgfpathcurveto{\pgfqpoint{10.597970in}{1.487620in}}{\pgfqpoint{10.602808in}{1.485616in}}{\pgfqpoint{10.607851in}{1.485616in}}%
\pgfpathclose%
\pgfusepath{fill}%
\end{pgfscope}%
\begin{pgfscope}%
\pgfpathrectangle{\pgfqpoint{6.572727in}{0.473000in}}{\pgfqpoint{4.227273in}{3.311000in}}%
\pgfusepath{clip}%
\pgfsetbuttcap%
\pgfsetroundjoin%
\definecolor{currentfill}{rgb}{0.993248,0.906157,0.143936}%
\pgfsetfillcolor{currentfill}%
\pgfsetfillopacity{0.700000}%
\pgfsetlinewidth{0.000000pt}%
\definecolor{currentstroke}{rgb}{0.000000,0.000000,0.000000}%
\pgfsetstrokecolor{currentstroke}%
\pgfsetstrokeopacity{0.700000}%
\pgfsetdash{}{0pt}%
\pgfpathmoveto{\pgfqpoint{10.345549in}{1.487147in}}%
\pgfpathcurveto{\pgfqpoint{10.350592in}{1.487147in}}{\pgfqpoint{10.355430in}{1.489151in}}{\pgfqpoint{10.358997in}{1.492717in}}%
\pgfpathcurveto{\pgfqpoint{10.362563in}{1.496284in}}{\pgfqpoint{10.364567in}{1.501121in}}{\pgfqpoint{10.364567in}{1.506165in}}%
\pgfpathcurveto{\pgfqpoint{10.364567in}{1.511209in}}{\pgfqpoint{10.362563in}{1.516047in}}{\pgfqpoint{10.358997in}{1.519613in}}%
\pgfpathcurveto{\pgfqpoint{10.355430in}{1.523179in}}{\pgfqpoint{10.350592in}{1.525183in}}{\pgfqpoint{10.345549in}{1.525183in}}%
\pgfpathcurveto{\pgfqpoint{10.340505in}{1.525183in}}{\pgfqpoint{10.335667in}{1.523179in}}{\pgfqpoint{10.332101in}{1.519613in}}%
\pgfpathcurveto{\pgfqpoint{10.328534in}{1.516047in}}{\pgfqpoint{10.326531in}{1.511209in}}{\pgfqpoint{10.326531in}{1.506165in}}%
\pgfpathcurveto{\pgfqpoint{10.326531in}{1.501121in}}{\pgfqpoint{10.328534in}{1.496284in}}{\pgfqpoint{10.332101in}{1.492717in}}%
\pgfpathcurveto{\pgfqpoint{10.335667in}{1.489151in}}{\pgfqpoint{10.340505in}{1.487147in}}{\pgfqpoint{10.345549in}{1.487147in}}%
\pgfpathclose%
\pgfusepath{fill}%
\end{pgfscope}%
\begin{pgfscope}%
\pgfpathrectangle{\pgfqpoint{6.572727in}{0.473000in}}{\pgfqpoint{4.227273in}{3.311000in}}%
\pgfusepath{clip}%
\pgfsetbuttcap%
\pgfsetroundjoin%
\definecolor{currentfill}{rgb}{0.127568,0.566949,0.550556}%
\pgfsetfillcolor{currentfill}%
\pgfsetfillopacity{0.700000}%
\pgfsetlinewidth{0.000000pt}%
\definecolor{currentstroke}{rgb}{0.000000,0.000000,0.000000}%
\pgfsetstrokecolor{currentstroke}%
\pgfsetstrokeopacity{0.700000}%
\pgfsetdash{}{0pt}%
\pgfpathmoveto{\pgfqpoint{7.657607in}{1.100014in}}%
\pgfpathcurveto{\pgfqpoint{7.662650in}{1.100014in}}{\pgfqpoint{7.667488in}{1.102018in}}{\pgfqpoint{7.671054in}{1.105584in}}%
\pgfpathcurveto{\pgfqpoint{7.674621in}{1.109151in}}{\pgfqpoint{7.676625in}{1.113989in}}{\pgfqpoint{7.676625in}{1.119032in}}%
\pgfpathcurveto{\pgfqpoint{7.676625in}{1.124076in}}{\pgfqpoint{7.674621in}{1.128914in}}{\pgfqpoint{7.671054in}{1.132480in}}%
\pgfpathcurveto{\pgfqpoint{7.667488in}{1.136047in}}{\pgfqpoint{7.662650in}{1.138050in}}{\pgfqpoint{7.657607in}{1.138050in}}%
\pgfpathcurveto{\pgfqpoint{7.652563in}{1.138050in}}{\pgfqpoint{7.647725in}{1.136047in}}{\pgfqpoint{7.644159in}{1.132480in}}%
\pgfpathcurveto{\pgfqpoint{7.640592in}{1.128914in}}{\pgfqpoint{7.638588in}{1.124076in}}{\pgfqpoint{7.638588in}{1.119032in}}%
\pgfpathcurveto{\pgfqpoint{7.638588in}{1.113989in}}{\pgfqpoint{7.640592in}{1.109151in}}{\pgfqpoint{7.644159in}{1.105584in}}%
\pgfpathcurveto{\pgfqpoint{7.647725in}{1.102018in}}{\pgfqpoint{7.652563in}{1.100014in}}{\pgfqpoint{7.657607in}{1.100014in}}%
\pgfpathclose%
\pgfusepath{fill}%
\end{pgfscope}%
\begin{pgfscope}%
\pgfpathrectangle{\pgfqpoint{6.572727in}{0.473000in}}{\pgfqpoint{4.227273in}{3.311000in}}%
\pgfusepath{clip}%
\pgfsetbuttcap%
\pgfsetroundjoin%
\definecolor{currentfill}{rgb}{0.127568,0.566949,0.550556}%
\pgfsetfillcolor{currentfill}%
\pgfsetfillopacity{0.700000}%
\pgfsetlinewidth{0.000000pt}%
\definecolor{currentstroke}{rgb}{0.000000,0.000000,0.000000}%
\pgfsetstrokecolor{currentstroke}%
\pgfsetstrokeopacity{0.700000}%
\pgfsetdash{}{0pt}%
\pgfpathmoveto{\pgfqpoint{7.594925in}{2.926310in}}%
\pgfpathcurveto{\pgfqpoint{7.599968in}{2.926310in}}{\pgfqpoint{7.604806in}{2.928314in}}{\pgfqpoint{7.608373in}{2.931881in}}%
\pgfpathcurveto{\pgfqpoint{7.611939in}{2.935447in}}{\pgfqpoint{7.613943in}{2.940285in}}{\pgfqpoint{7.613943in}{2.945328in}}%
\pgfpathcurveto{\pgfqpoint{7.613943in}{2.950372in}}{\pgfqpoint{7.611939in}{2.955210in}}{\pgfqpoint{7.608373in}{2.958776in}}%
\pgfpathcurveto{\pgfqpoint{7.604806in}{2.962343in}}{\pgfqpoint{7.599968in}{2.964347in}}{\pgfqpoint{7.594925in}{2.964347in}}%
\pgfpathcurveto{\pgfqpoint{7.589881in}{2.964347in}}{\pgfqpoint{7.585043in}{2.962343in}}{\pgfqpoint{7.581477in}{2.958776in}}%
\pgfpathcurveto{\pgfqpoint{7.577911in}{2.955210in}}{\pgfqpoint{7.575907in}{2.950372in}}{\pgfqpoint{7.575907in}{2.945328in}}%
\pgfpathcurveto{\pgfqpoint{7.575907in}{2.940285in}}{\pgfqpoint{7.577911in}{2.935447in}}{\pgfqpoint{7.581477in}{2.931881in}}%
\pgfpathcurveto{\pgfqpoint{7.585043in}{2.928314in}}{\pgfqpoint{7.589881in}{2.926310in}}{\pgfqpoint{7.594925in}{2.926310in}}%
\pgfpathclose%
\pgfusepath{fill}%
\end{pgfscope}%
\begin{pgfscope}%
\pgfpathrectangle{\pgfqpoint{6.572727in}{0.473000in}}{\pgfqpoint{4.227273in}{3.311000in}}%
\pgfusepath{clip}%
\pgfsetbuttcap%
\pgfsetroundjoin%
\definecolor{currentfill}{rgb}{0.127568,0.566949,0.550556}%
\pgfsetfillcolor{currentfill}%
\pgfsetfillopacity{0.700000}%
\pgfsetlinewidth{0.000000pt}%
\definecolor{currentstroke}{rgb}{0.000000,0.000000,0.000000}%
\pgfsetstrokecolor{currentstroke}%
\pgfsetstrokeopacity{0.700000}%
\pgfsetdash{}{0pt}%
\pgfpathmoveto{\pgfqpoint{8.084105in}{1.749269in}}%
\pgfpathcurveto{\pgfqpoint{8.089149in}{1.749269in}}{\pgfqpoint{8.093987in}{1.751273in}}{\pgfqpoint{8.097553in}{1.754839in}}%
\pgfpathcurveto{\pgfqpoint{8.101120in}{1.758406in}}{\pgfqpoint{8.103124in}{1.763244in}}{\pgfqpoint{8.103124in}{1.768287in}}%
\pgfpathcurveto{\pgfqpoint{8.103124in}{1.773331in}}{\pgfqpoint{8.101120in}{1.778169in}}{\pgfqpoint{8.097553in}{1.781735in}}%
\pgfpathcurveto{\pgfqpoint{8.093987in}{1.785301in}}{\pgfqpoint{8.089149in}{1.787305in}}{\pgfqpoint{8.084105in}{1.787305in}}%
\pgfpathcurveto{\pgfqpoint{8.079062in}{1.787305in}}{\pgfqpoint{8.074224in}{1.785301in}}{\pgfqpoint{8.070658in}{1.781735in}}%
\pgfpathcurveto{\pgfqpoint{8.067091in}{1.778169in}}{\pgfqpoint{8.065087in}{1.773331in}}{\pgfqpoint{8.065087in}{1.768287in}}%
\pgfpathcurveto{\pgfqpoint{8.065087in}{1.763244in}}{\pgfqpoint{8.067091in}{1.758406in}}{\pgfqpoint{8.070658in}{1.754839in}}%
\pgfpathcurveto{\pgfqpoint{8.074224in}{1.751273in}}{\pgfqpoint{8.079062in}{1.749269in}}{\pgfqpoint{8.084105in}{1.749269in}}%
\pgfpathclose%
\pgfusepath{fill}%
\end{pgfscope}%
\begin{pgfscope}%
\pgfpathrectangle{\pgfqpoint{6.572727in}{0.473000in}}{\pgfqpoint{4.227273in}{3.311000in}}%
\pgfusepath{clip}%
\pgfsetbuttcap%
\pgfsetroundjoin%
\definecolor{currentfill}{rgb}{0.993248,0.906157,0.143936}%
\pgfsetfillcolor{currentfill}%
\pgfsetfillopacity{0.700000}%
\pgfsetlinewidth{0.000000pt}%
\definecolor{currentstroke}{rgb}{0.000000,0.000000,0.000000}%
\pgfsetstrokecolor{currentstroke}%
\pgfsetstrokeopacity{0.700000}%
\pgfsetdash{}{0pt}%
\pgfpathmoveto{\pgfqpoint{9.382759in}{1.579065in}}%
\pgfpathcurveto{\pgfqpoint{9.387803in}{1.579065in}}{\pgfqpoint{9.392641in}{1.581069in}}{\pgfqpoint{9.396207in}{1.584636in}}%
\pgfpathcurveto{\pgfqpoint{9.399774in}{1.588202in}}{\pgfqpoint{9.401778in}{1.593040in}}{\pgfqpoint{9.401778in}{1.598084in}}%
\pgfpathcurveto{\pgfqpoint{9.401778in}{1.603127in}}{\pgfqpoint{9.399774in}{1.607965in}}{\pgfqpoint{9.396207in}{1.611531in}}%
\pgfpathcurveto{\pgfqpoint{9.392641in}{1.615098in}}{\pgfqpoint{9.387803in}{1.617102in}}{\pgfqpoint{9.382759in}{1.617102in}}%
\pgfpathcurveto{\pgfqpoint{9.377716in}{1.617102in}}{\pgfqpoint{9.372878in}{1.615098in}}{\pgfqpoint{9.369312in}{1.611531in}}%
\pgfpathcurveto{\pgfqpoint{9.365745in}{1.607965in}}{\pgfqpoint{9.363741in}{1.603127in}}{\pgfqpoint{9.363741in}{1.598084in}}%
\pgfpathcurveto{\pgfqpoint{9.363741in}{1.593040in}}{\pgfqpoint{9.365745in}{1.588202in}}{\pgfqpoint{9.369312in}{1.584636in}}%
\pgfpathcurveto{\pgfqpoint{9.372878in}{1.581069in}}{\pgfqpoint{9.377716in}{1.579065in}}{\pgfqpoint{9.382759in}{1.579065in}}%
\pgfpathclose%
\pgfusepath{fill}%
\end{pgfscope}%
\begin{pgfscope}%
\pgfpathrectangle{\pgfqpoint{6.572727in}{0.473000in}}{\pgfqpoint{4.227273in}{3.311000in}}%
\pgfusepath{clip}%
\pgfsetbuttcap%
\pgfsetroundjoin%
\definecolor{currentfill}{rgb}{0.127568,0.566949,0.550556}%
\pgfsetfillcolor{currentfill}%
\pgfsetfillopacity{0.700000}%
\pgfsetlinewidth{0.000000pt}%
\definecolor{currentstroke}{rgb}{0.000000,0.000000,0.000000}%
\pgfsetstrokecolor{currentstroke}%
\pgfsetstrokeopacity{0.700000}%
\pgfsetdash{}{0pt}%
\pgfpathmoveto{\pgfqpoint{8.072087in}{2.734873in}}%
\pgfpathcurveto{\pgfqpoint{8.077130in}{2.734873in}}{\pgfqpoint{8.081968in}{2.736877in}}{\pgfqpoint{8.085534in}{2.740443in}}%
\pgfpathcurveto{\pgfqpoint{8.089101in}{2.744010in}}{\pgfqpoint{8.091105in}{2.748848in}}{\pgfqpoint{8.091105in}{2.753891in}}%
\pgfpathcurveto{\pgfqpoint{8.091105in}{2.758935in}}{\pgfqpoint{8.089101in}{2.763773in}}{\pgfqpoint{8.085534in}{2.767339in}}%
\pgfpathcurveto{\pgfqpoint{8.081968in}{2.770906in}}{\pgfqpoint{8.077130in}{2.772909in}}{\pgfqpoint{8.072087in}{2.772909in}}%
\pgfpathcurveto{\pgfqpoint{8.067043in}{2.772909in}}{\pgfqpoint{8.062205in}{2.770906in}}{\pgfqpoint{8.058639in}{2.767339in}}%
\pgfpathcurveto{\pgfqpoint{8.055072in}{2.763773in}}{\pgfqpoint{8.053068in}{2.758935in}}{\pgfqpoint{8.053068in}{2.753891in}}%
\pgfpathcurveto{\pgfqpoint{8.053068in}{2.748848in}}{\pgfqpoint{8.055072in}{2.744010in}}{\pgfqpoint{8.058639in}{2.740443in}}%
\pgfpathcurveto{\pgfqpoint{8.062205in}{2.736877in}}{\pgfqpoint{8.067043in}{2.734873in}}{\pgfqpoint{8.072087in}{2.734873in}}%
\pgfpathclose%
\pgfusepath{fill}%
\end{pgfscope}%
\begin{pgfscope}%
\pgfpathrectangle{\pgfqpoint{6.572727in}{0.473000in}}{\pgfqpoint{4.227273in}{3.311000in}}%
\pgfusepath{clip}%
\pgfsetbuttcap%
\pgfsetroundjoin%
\definecolor{currentfill}{rgb}{0.993248,0.906157,0.143936}%
\pgfsetfillcolor{currentfill}%
\pgfsetfillopacity{0.700000}%
\pgfsetlinewidth{0.000000pt}%
\definecolor{currentstroke}{rgb}{0.000000,0.000000,0.000000}%
\pgfsetstrokecolor{currentstroke}%
\pgfsetstrokeopacity{0.700000}%
\pgfsetdash{}{0pt}%
\pgfpathmoveto{\pgfqpoint{9.928319in}{1.716052in}}%
\pgfpathcurveto{\pgfqpoint{9.933362in}{1.716052in}}{\pgfqpoint{9.938200in}{1.718056in}}{\pgfqpoint{9.941766in}{1.721622in}}%
\pgfpathcurveto{\pgfqpoint{9.945333in}{1.725189in}}{\pgfqpoint{9.947337in}{1.730027in}}{\pgfqpoint{9.947337in}{1.735070in}}%
\pgfpathcurveto{\pgfqpoint{9.947337in}{1.740114in}}{\pgfqpoint{9.945333in}{1.744952in}}{\pgfqpoint{9.941766in}{1.748518in}}%
\pgfpathcurveto{\pgfqpoint{9.938200in}{1.752085in}}{\pgfqpoint{9.933362in}{1.754088in}}{\pgfqpoint{9.928319in}{1.754088in}}%
\pgfpathcurveto{\pgfqpoint{9.923275in}{1.754088in}}{\pgfqpoint{9.918437in}{1.752085in}}{\pgfqpoint{9.914871in}{1.748518in}}%
\pgfpathcurveto{\pgfqpoint{9.911304in}{1.744952in}}{\pgfqpoint{9.909300in}{1.740114in}}{\pgfqpoint{9.909300in}{1.735070in}}%
\pgfpathcurveto{\pgfqpoint{9.909300in}{1.730027in}}{\pgfqpoint{9.911304in}{1.725189in}}{\pgfqpoint{9.914871in}{1.721622in}}%
\pgfpathcurveto{\pgfqpoint{9.918437in}{1.718056in}}{\pgfqpoint{9.923275in}{1.716052in}}{\pgfqpoint{9.928319in}{1.716052in}}%
\pgfpathclose%
\pgfusepath{fill}%
\end{pgfscope}%
\begin{pgfscope}%
\pgfpathrectangle{\pgfqpoint{6.572727in}{0.473000in}}{\pgfqpoint{4.227273in}{3.311000in}}%
\pgfusepath{clip}%
\pgfsetbuttcap%
\pgfsetroundjoin%
\definecolor{currentfill}{rgb}{0.127568,0.566949,0.550556}%
\pgfsetfillcolor{currentfill}%
\pgfsetfillopacity{0.700000}%
\pgfsetlinewidth{0.000000pt}%
\definecolor{currentstroke}{rgb}{0.000000,0.000000,0.000000}%
\pgfsetstrokecolor{currentstroke}%
\pgfsetstrokeopacity{0.700000}%
\pgfsetdash{}{0pt}%
\pgfpathmoveto{\pgfqpoint{7.964010in}{2.326361in}}%
\pgfpathcurveto{\pgfqpoint{7.969053in}{2.326361in}}{\pgfqpoint{7.973891in}{2.328364in}}{\pgfqpoint{7.977457in}{2.331931in}}%
\pgfpathcurveto{\pgfqpoint{7.981024in}{2.335497in}}{\pgfqpoint{7.983028in}{2.340335in}}{\pgfqpoint{7.983028in}{2.345379in}}%
\pgfpathcurveto{\pgfqpoint{7.983028in}{2.350422in}}{\pgfqpoint{7.981024in}{2.355260in}}{\pgfqpoint{7.977457in}{2.358827in}}%
\pgfpathcurveto{\pgfqpoint{7.973891in}{2.362393in}}{\pgfqpoint{7.969053in}{2.364397in}}{\pgfqpoint{7.964010in}{2.364397in}}%
\pgfpathcurveto{\pgfqpoint{7.958966in}{2.364397in}}{\pgfqpoint{7.954128in}{2.362393in}}{\pgfqpoint{7.950562in}{2.358827in}}%
\pgfpathcurveto{\pgfqpoint{7.946995in}{2.355260in}}{\pgfqpoint{7.944991in}{2.350422in}}{\pgfqpoint{7.944991in}{2.345379in}}%
\pgfpathcurveto{\pgfqpoint{7.944991in}{2.340335in}}{\pgfqpoint{7.946995in}{2.335497in}}{\pgfqpoint{7.950562in}{2.331931in}}%
\pgfpathcurveto{\pgfqpoint{7.954128in}{2.328364in}}{\pgfqpoint{7.958966in}{2.326361in}}{\pgfqpoint{7.964010in}{2.326361in}}%
\pgfpathclose%
\pgfusepath{fill}%
\end{pgfscope}%
\begin{pgfscope}%
\pgfpathrectangle{\pgfqpoint{6.572727in}{0.473000in}}{\pgfqpoint{4.227273in}{3.311000in}}%
\pgfusepath{clip}%
\pgfsetbuttcap%
\pgfsetroundjoin%
\definecolor{currentfill}{rgb}{0.993248,0.906157,0.143936}%
\pgfsetfillcolor{currentfill}%
\pgfsetfillopacity{0.700000}%
\pgfsetlinewidth{0.000000pt}%
\definecolor{currentstroke}{rgb}{0.000000,0.000000,0.000000}%
\pgfsetstrokecolor{currentstroke}%
\pgfsetstrokeopacity{0.700000}%
\pgfsetdash{}{0pt}%
\pgfpathmoveto{\pgfqpoint{9.534888in}{1.470232in}}%
\pgfpathcurveto{\pgfqpoint{9.539931in}{1.470232in}}{\pgfqpoint{9.544769in}{1.472236in}}{\pgfqpoint{9.548336in}{1.475802in}}%
\pgfpathcurveto{\pgfqpoint{9.551902in}{1.479369in}}{\pgfqpoint{9.553906in}{1.484206in}}{\pgfqpoint{9.553906in}{1.489250in}}%
\pgfpathcurveto{\pgfqpoint{9.553906in}{1.494294in}}{\pgfqpoint{9.551902in}{1.499132in}}{\pgfqpoint{9.548336in}{1.502698in}}%
\pgfpathcurveto{\pgfqpoint{9.544769in}{1.506264in}}{\pgfqpoint{9.539931in}{1.508268in}}{\pgfqpoint{9.534888in}{1.508268in}}%
\pgfpathcurveto{\pgfqpoint{9.529844in}{1.508268in}}{\pgfqpoint{9.525006in}{1.506264in}}{\pgfqpoint{9.521440in}{1.502698in}}%
\pgfpathcurveto{\pgfqpoint{9.517873in}{1.499132in}}{\pgfqpoint{9.515870in}{1.494294in}}{\pgfqpoint{9.515870in}{1.489250in}}%
\pgfpathcurveto{\pgfqpoint{9.515870in}{1.484206in}}{\pgfqpoint{9.517873in}{1.479369in}}{\pgfqpoint{9.521440in}{1.475802in}}%
\pgfpathcurveto{\pgfqpoint{9.525006in}{1.472236in}}{\pgfqpoint{9.529844in}{1.470232in}}{\pgfqpoint{9.534888in}{1.470232in}}%
\pgfpathclose%
\pgfusepath{fill}%
\end{pgfscope}%
\begin{pgfscope}%
\pgfpathrectangle{\pgfqpoint{6.572727in}{0.473000in}}{\pgfqpoint{4.227273in}{3.311000in}}%
\pgfusepath{clip}%
\pgfsetbuttcap%
\pgfsetroundjoin%
\definecolor{currentfill}{rgb}{0.127568,0.566949,0.550556}%
\pgfsetfillcolor{currentfill}%
\pgfsetfillopacity{0.700000}%
\pgfsetlinewidth{0.000000pt}%
\definecolor{currentstroke}{rgb}{0.000000,0.000000,0.000000}%
\pgfsetstrokecolor{currentstroke}%
\pgfsetstrokeopacity{0.700000}%
\pgfsetdash{}{0pt}%
\pgfpathmoveto{\pgfqpoint{7.980377in}{2.535390in}}%
\pgfpathcurveto{\pgfqpoint{7.985420in}{2.535390in}}{\pgfqpoint{7.990258in}{2.537394in}}{\pgfqpoint{7.993824in}{2.540961in}}%
\pgfpathcurveto{\pgfqpoint{7.997391in}{2.544527in}}{\pgfqpoint{7.999395in}{2.549365in}}{\pgfqpoint{7.999395in}{2.554409in}}%
\pgfpathcurveto{\pgfqpoint{7.999395in}{2.559452in}}{\pgfqpoint{7.997391in}{2.564290in}}{\pgfqpoint{7.993824in}{2.567856in}}%
\pgfpathcurveto{\pgfqpoint{7.990258in}{2.571423in}}{\pgfqpoint{7.985420in}{2.573427in}}{\pgfqpoint{7.980377in}{2.573427in}}%
\pgfpathcurveto{\pgfqpoint{7.975333in}{2.573427in}}{\pgfqpoint{7.970495in}{2.571423in}}{\pgfqpoint{7.966929in}{2.567856in}}%
\pgfpathcurveto{\pgfqpoint{7.963362in}{2.564290in}}{\pgfqpoint{7.961358in}{2.559452in}}{\pgfqpoint{7.961358in}{2.554409in}}%
\pgfpathcurveto{\pgfqpoint{7.961358in}{2.549365in}}{\pgfqpoint{7.963362in}{2.544527in}}{\pgfqpoint{7.966929in}{2.540961in}}%
\pgfpathcurveto{\pgfqpoint{7.970495in}{2.537394in}}{\pgfqpoint{7.975333in}{2.535390in}}{\pgfqpoint{7.980377in}{2.535390in}}%
\pgfpathclose%
\pgfusepath{fill}%
\end{pgfscope}%
\begin{pgfscope}%
\pgfpathrectangle{\pgfqpoint{6.572727in}{0.473000in}}{\pgfqpoint{4.227273in}{3.311000in}}%
\pgfusepath{clip}%
\pgfsetbuttcap%
\pgfsetroundjoin%
\definecolor{currentfill}{rgb}{0.993248,0.906157,0.143936}%
\pgfsetfillcolor{currentfill}%
\pgfsetfillopacity{0.700000}%
\pgfsetlinewidth{0.000000pt}%
\definecolor{currentstroke}{rgb}{0.000000,0.000000,0.000000}%
\pgfsetstrokecolor{currentstroke}%
\pgfsetstrokeopacity{0.700000}%
\pgfsetdash{}{0pt}%
\pgfpathmoveto{\pgfqpoint{9.356800in}{1.178220in}}%
\pgfpathcurveto{\pgfqpoint{9.361844in}{1.178220in}}{\pgfqpoint{9.366681in}{1.180224in}}{\pgfqpoint{9.370248in}{1.183790in}}%
\pgfpathcurveto{\pgfqpoint{9.373814in}{1.187357in}}{\pgfqpoint{9.375818in}{1.192195in}}{\pgfqpoint{9.375818in}{1.197238in}}%
\pgfpathcurveto{\pgfqpoint{9.375818in}{1.202282in}}{\pgfqpoint{9.373814in}{1.207120in}}{\pgfqpoint{9.370248in}{1.210686in}}%
\pgfpathcurveto{\pgfqpoint{9.366681in}{1.214253in}}{\pgfqpoint{9.361844in}{1.216256in}}{\pgfqpoint{9.356800in}{1.216256in}}%
\pgfpathcurveto{\pgfqpoint{9.351756in}{1.216256in}}{\pgfqpoint{9.346918in}{1.214253in}}{\pgfqpoint{9.343352in}{1.210686in}}%
\pgfpathcurveto{\pgfqpoint{9.339786in}{1.207120in}}{\pgfqpoint{9.337782in}{1.202282in}}{\pgfqpoint{9.337782in}{1.197238in}}%
\pgfpathcurveto{\pgfqpoint{9.337782in}{1.192195in}}{\pgfqpoint{9.339786in}{1.187357in}}{\pgfqpoint{9.343352in}{1.183790in}}%
\pgfpathcurveto{\pgfqpoint{9.346918in}{1.180224in}}{\pgfqpoint{9.351756in}{1.178220in}}{\pgfqpoint{9.356800in}{1.178220in}}%
\pgfpathclose%
\pgfusepath{fill}%
\end{pgfscope}%
\begin{pgfscope}%
\pgfpathrectangle{\pgfqpoint{6.572727in}{0.473000in}}{\pgfqpoint{4.227273in}{3.311000in}}%
\pgfusepath{clip}%
\pgfsetbuttcap%
\pgfsetroundjoin%
\definecolor{currentfill}{rgb}{0.127568,0.566949,0.550556}%
\pgfsetfillcolor{currentfill}%
\pgfsetfillopacity{0.700000}%
\pgfsetlinewidth{0.000000pt}%
\definecolor{currentstroke}{rgb}{0.000000,0.000000,0.000000}%
\pgfsetstrokecolor{currentstroke}%
\pgfsetstrokeopacity{0.700000}%
\pgfsetdash{}{0pt}%
\pgfpathmoveto{\pgfqpoint{7.915122in}{1.501687in}}%
\pgfpathcurveto{\pgfqpoint{7.920165in}{1.501687in}}{\pgfqpoint{7.925003in}{1.503690in}}{\pgfqpoint{7.928569in}{1.507257in}}%
\pgfpathcurveto{\pgfqpoint{7.932136in}{1.510823in}}{\pgfqpoint{7.934140in}{1.515661in}}{\pgfqpoint{7.934140in}{1.520705in}}%
\pgfpathcurveto{\pgfqpoint{7.934140in}{1.525748in}}{\pgfqpoint{7.932136in}{1.530586in}}{\pgfqpoint{7.928569in}{1.534153in}}%
\pgfpathcurveto{\pgfqpoint{7.925003in}{1.537719in}}{\pgfqpoint{7.920165in}{1.539723in}}{\pgfqpoint{7.915122in}{1.539723in}}%
\pgfpathcurveto{\pgfqpoint{7.910078in}{1.539723in}}{\pgfqpoint{7.905240in}{1.537719in}}{\pgfqpoint{7.901674in}{1.534153in}}%
\pgfpathcurveto{\pgfqpoint{7.898107in}{1.530586in}}{\pgfqpoint{7.896103in}{1.525748in}}{\pgfqpoint{7.896103in}{1.520705in}}%
\pgfpathcurveto{\pgfqpoint{7.896103in}{1.515661in}}{\pgfqpoint{7.898107in}{1.510823in}}{\pgfqpoint{7.901674in}{1.507257in}}%
\pgfpathcurveto{\pgfqpoint{7.905240in}{1.503690in}}{\pgfqpoint{7.910078in}{1.501687in}}{\pgfqpoint{7.915122in}{1.501687in}}%
\pgfpathclose%
\pgfusepath{fill}%
\end{pgfscope}%
\begin{pgfscope}%
\pgfpathrectangle{\pgfqpoint{6.572727in}{0.473000in}}{\pgfqpoint{4.227273in}{3.311000in}}%
\pgfusepath{clip}%
\pgfsetbuttcap%
\pgfsetroundjoin%
\definecolor{currentfill}{rgb}{0.993248,0.906157,0.143936}%
\pgfsetfillcolor{currentfill}%
\pgfsetfillopacity{0.700000}%
\pgfsetlinewidth{0.000000pt}%
\definecolor{currentstroke}{rgb}{0.000000,0.000000,0.000000}%
\pgfsetstrokecolor{currentstroke}%
\pgfsetstrokeopacity{0.700000}%
\pgfsetdash{}{0pt}%
\pgfpathmoveto{\pgfqpoint{9.906081in}{1.634151in}}%
\pgfpathcurveto{\pgfqpoint{9.911125in}{1.634151in}}{\pgfqpoint{9.915963in}{1.636155in}}{\pgfqpoint{9.919529in}{1.639722in}}%
\pgfpathcurveto{\pgfqpoint{9.923096in}{1.643288in}}{\pgfqpoint{9.925100in}{1.648126in}}{\pgfqpoint{9.925100in}{1.653170in}}%
\pgfpathcurveto{\pgfqpoint{9.925100in}{1.658213in}}{\pgfqpoint{9.923096in}{1.663051in}}{\pgfqpoint{9.919529in}{1.666617in}}%
\pgfpathcurveto{\pgfqpoint{9.915963in}{1.670184in}}{\pgfqpoint{9.911125in}{1.672188in}}{\pgfqpoint{9.906081in}{1.672188in}}%
\pgfpathcurveto{\pgfqpoint{9.901038in}{1.672188in}}{\pgfqpoint{9.896200in}{1.670184in}}{\pgfqpoint{9.892634in}{1.666617in}}%
\pgfpathcurveto{\pgfqpoint{9.889067in}{1.663051in}}{\pgfqpoint{9.887063in}{1.658213in}}{\pgfqpoint{9.887063in}{1.653170in}}%
\pgfpathcurveto{\pgfqpoint{9.887063in}{1.648126in}}{\pgfqpoint{9.889067in}{1.643288in}}{\pgfqpoint{9.892634in}{1.639722in}}%
\pgfpathcurveto{\pgfqpoint{9.896200in}{1.636155in}}{\pgfqpoint{9.901038in}{1.634151in}}{\pgfqpoint{9.906081in}{1.634151in}}%
\pgfpathclose%
\pgfusepath{fill}%
\end{pgfscope}%
\begin{pgfscope}%
\pgfpathrectangle{\pgfqpoint{6.572727in}{0.473000in}}{\pgfqpoint{4.227273in}{3.311000in}}%
\pgfusepath{clip}%
\pgfsetbuttcap%
\pgfsetroundjoin%
\definecolor{currentfill}{rgb}{0.127568,0.566949,0.550556}%
\pgfsetfillcolor{currentfill}%
\pgfsetfillopacity{0.700000}%
\pgfsetlinewidth{0.000000pt}%
\definecolor{currentstroke}{rgb}{0.000000,0.000000,0.000000}%
\pgfsetstrokecolor{currentstroke}%
\pgfsetstrokeopacity{0.700000}%
\pgfsetdash{}{0pt}%
\pgfpathmoveto{\pgfqpoint{7.620807in}{2.936949in}}%
\pgfpathcurveto{\pgfqpoint{7.625850in}{2.936949in}}{\pgfqpoint{7.630688in}{2.938953in}}{\pgfqpoint{7.634255in}{2.942519in}}%
\pgfpathcurveto{\pgfqpoint{7.637821in}{2.946085in}}{\pgfqpoint{7.639825in}{2.950923in}}{\pgfqpoint{7.639825in}{2.955967in}}%
\pgfpathcurveto{\pgfqpoint{7.639825in}{2.961011in}}{\pgfqpoint{7.637821in}{2.965848in}}{\pgfqpoint{7.634255in}{2.969415in}}%
\pgfpathcurveto{\pgfqpoint{7.630688in}{2.972981in}}{\pgfqpoint{7.625850in}{2.974985in}}{\pgfqpoint{7.620807in}{2.974985in}}%
\pgfpathcurveto{\pgfqpoint{7.615763in}{2.974985in}}{\pgfqpoint{7.610925in}{2.972981in}}{\pgfqpoint{7.607359in}{2.969415in}}%
\pgfpathcurveto{\pgfqpoint{7.603792in}{2.965848in}}{\pgfqpoint{7.601789in}{2.961011in}}{\pgfqpoint{7.601789in}{2.955967in}}%
\pgfpathcurveto{\pgfqpoint{7.601789in}{2.950923in}}{\pgfqpoint{7.603792in}{2.946085in}}{\pgfqpoint{7.607359in}{2.942519in}}%
\pgfpathcurveto{\pgfqpoint{7.610925in}{2.938953in}}{\pgfqpoint{7.615763in}{2.936949in}}{\pgfqpoint{7.620807in}{2.936949in}}%
\pgfpathclose%
\pgfusepath{fill}%
\end{pgfscope}%
\begin{pgfscope}%
\pgfpathrectangle{\pgfqpoint{6.572727in}{0.473000in}}{\pgfqpoint{4.227273in}{3.311000in}}%
\pgfusepath{clip}%
\pgfsetbuttcap%
\pgfsetroundjoin%
\definecolor{currentfill}{rgb}{0.993248,0.906157,0.143936}%
\pgfsetfillcolor{currentfill}%
\pgfsetfillopacity{0.700000}%
\pgfsetlinewidth{0.000000pt}%
\definecolor{currentstroke}{rgb}{0.000000,0.000000,0.000000}%
\pgfsetstrokecolor{currentstroke}%
\pgfsetstrokeopacity{0.700000}%
\pgfsetdash{}{0pt}%
\pgfpathmoveto{\pgfqpoint{8.829797in}{1.993927in}}%
\pgfpathcurveto{\pgfqpoint{8.834840in}{1.993927in}}{\pgfqpoint{8.839678in}{1.995931in}}{\pgfqpoint{8.843244in}{1.999497in}}%
\pgfpathcurveto{\pgfqpoint{8.846811in}{2.003064in}}{\pgfqpoint{8.848815in}{2.007901in}}{\pgfqpoint{8.848815in}{2.012945in}}%
\pgfpathcurveto{\pgfqpoint{8.848815in}{2.017989in}}{\pgfqpoint{8.846811in}{2.022826in}}{\pgfqpoint{8.843244in}{2.026393in}}%
\pgfpathcurveto{\pgfqpoint{8.839678in}{2.029959in}}{\pgfqpoint{8.834840in}{2.031963in}}{\pgfqpoint{8.829797in}{2.031963in}}%
\pgfpathcurveto{\pgfqpoint{8.824753in}{2.031963in}}{\pgfqpoint{8.819915in}{2.029959in}}{\pgfqpoint{8.816349in}{2.026393in}}%
\pgfpathcurveto{\pgfqpoint{8.812782in}{2.022826in}}{\pgfqpoint{8.810778in}{2.017989in}}{\pgfqpoint{8.810778in}{2.012945in}}%
\pgfpathcurveto{\pgfqpoint{8.810778in}{2.007901in}}{\pgfqpoint{8.812782in}{2.003064in}}{\pgfqpoint{8.816349in}{1.999497in}}%
\pgfpathcurveto{\pgfqpoint{8.819915in}{1.995931in}}{\pgfqpoint{8.824753in}{1.993927in}}{\pgfqpoint{8.829797in}{1.993927in}}%
\pgfpathclose%
\pgfusepath{fill}%
\end{pgfscope}%
\begin{pgfscope}%
\pgfpathrectangle{\pgfqpoint{6.572727in}{0.473000in}}{\pgfqpoint{4.227273in}{3.311000in}}%
\pgfusepath{clip}%
\pgfsetbuttcap%
\pgfsetroundjoin%
\definecolor{currentfill}{rgb}{0.127568,0.566949,0.550556}%
\pgfsetfillcolor{currentfill}%
\pgfsetfillopacity{0.700000}%
\pgfsetlinewidth{0.000000pt}%
\definecolor{currentstroke}{rgb}{0.000000,0.000000,0.000000}%
\pgfsetstrokecolor{currentstroke}%
\pgfsetstrokeopacity{0.700000}%
\pgfsetdash{}{0pt}%
\pgfpathmoveto{\pgfqpoint{8.213233in}{2.872384in}}%
\pgfpathcurveto{\pgfqpoint{8.218276in}{2.872384in}}{\pgfqpoint{8.223114in}{2.874387in}}{\pgfqpoint{8.226680in}{2.877954in}}%
\pgfpathcurveto{\pgfqpoint{8.230247in}{2.881520in}}{\pgfqpoint{8.232251in}{2.886358in}}{\pgfqpoint{8.232251in}{2.891402in}}%
\pgfpathcurveto{\pgfqpoint{8.232251in}{2.896445in}}{\pgfqpoint{8.230247in}{2.901283in}}{\pgfqpoint{8.226680in}{2.904850in}}%
\pgfpathcurveto{\pgfqpoint{8.223114in}{2.908416in}}{\pgfqpoint{8.218276in}{2.910420in}}{\pgfqpoint{8.213233in}{2.910420in}}%
\pgfpathcurveto{\pgfqpoint{8.208189in}{2.910420in}}{\pgfqpoint{8.203351in}{2.908416in}}{\pgfqpoint{8.199785in}{2.904850in}}%
\pgfpathcurveto{\pgfqpoint{8.196218in}{2.901283in}}{\pgfqpoint{8.194214in}{2.896445in}}{\pgfqpoint{8.194214in}{2.891402in}}%
\pgfpathcurveto{\pgfqpoint{8.194214in}{2.886358in}}{\pgfqpoint{8.196218in}{2.881520in}}{\pgfqpoint{8.199785in}{2.877954in}}%
\pgfpathcurveto{\pgfqpoint{8.203351in}{2.874387in}}{\pgfqpoint{8.208189in}{2.872384in}}{\pgfqpoint{8.213233in}{2.872384in}}%
\pgfpathclose%
\pgfusepath{fill}%
\end{pgfscope}%
\begin{pgfscope}%
\pgfpathrectangle{\pgfqpoint{6.572727in}{0.473000in}}{\pgfqpoint{4.227273in}{3.311000in}}%
\pgfusepath{clip}%
\pgfsetbuttcap%
\pgfsetroundjoin%
\definecolor{currentfill}{rgb}{0.993248,0.906157,0.143936}%
\pgfsetfillcolor{currentfill}%
\pgfsetfillopacity{0.700000}%
\pgfsetlinewidth{0.000000pt}%
\definecolor{currentstroke}{rgb}{0.000000,0.000000,0.000000}%
\pgfsetstrokecolor{currentstroke}%
\pgfsetstrokeopacity{0.700000}%
\pgfsetdash{}{0pt}%
\pgfpathmoveto{\pgfqpoint{9.332851in}{1.553216in}}%
\pgfpathcurveto{\pgfqpoint{9.337894in}{1.553216in}}{\pgfqpoint{9.342732in}{1.555220in}}{\pgfqpoint{9.346298in}{1.558786in}}%
\pgfpathcurveto{\pgfqpoint{9.349865in}{1.562352in}}{\pgfqpoint{9.351869in}{1.567190in}}{\pgfqpoint{9.351869in}{1.572234in}}%
\pgfpathcurveto{\pgfqpoint{9.351869in}{1.577278in}}{\pgfqpoint{9.349865in}{1.582115in}}{\pgfqpoint{9.346298in}{1.585682in}}%
\pgfpathcurveto{\pgfqpoint{9.342732in}{1.589248in}}{\pgfqpoint{9.337894in}{1.591252in}}{\pgfqpoint{9.332851in}{1.591252in}}%
\pgfpathcurveto{\pgfqpoint{9.327807in}{1.591252in}}{\pgfqpoint{9.322969in}{1.589248in}}{\pgfqpoint{9.319403in}{1.585682in}}%
\pgfpathcurveto{\pgfqpoint{9.315836in}{1.582115in}}{\pgfqpoint{9.313832in}{1.577278in}}{\pgfqpoint{9.313832in}{1.572234in}}%
\pgfpathcurveto{\pgfqpoint{9.313832in}{1.567190in}}{\pgfqpoint{9.315836in}{1.562352in}}{\pgfqpoint{9.319403in}{1.558786in}}%
\pgfpathcurveto{\pgfqpoint{9.322969in}{1.555220in}}{\pgfqpoint{9.327807in}{1.553216in}}{\pgfqpoint{9.332851in}{1.553216in}}%
\pgfpathclose%
\pgfusepath{fill}%
\end{pgfscope}%
\begin{pgfscope}%
\pgfpathrectangle{\pgfqpoint{6.572727in}{0.473000in}}{\pgfqpoint{4.227273in}{3.311000in}}%
\pgfusepath{clip}%
\pgfsetbuttcap%
\pgfsetroundjoin%
\definecolor{currentfill}{rgb}{0.993248,0.906157,0.143936}%
\pgfsetfillcolor{currentfill}%
\pgfsetfillopacity{0.700000}%
\pgfsetlinewidth{0.000000pt}%
\definecolor{currentstroke}{rgb}{0.000000,0.000000,0.000000}%
\pgfsetstrokecolor{currentstroke}%
\pgfsetstrokeopacity{0.700000}%
\pgfsetdash{}{0pt}%
\pgfpathmoveto{\pgfqpoint{9.693822in}{1.986831in}}%
\pgfpathcurveto{\pgfqpoint{9.698866in}{1.986831in}}{\pgfqpoint{9.703704in}{1.988835in}}{\pgfqpoint{9.707270in}{1.992402in}}%
\pgfpathcurveto{\pgfqpoint{9.710836in}{1.995968in}}{\pgfqpoint{9.712840in}{2.000806in}}{\pgfqpoint{9.712840in}{2.005849in}}%
\pgfpathcurveto{\pgfqpoint{9.712840in}{2.010893in}}{\pgfqpoint{9.710836in}{2.015731in}}{\pgfqpoint{9.707270in}{2.019297in}}%
\pgfpathcurveto{\pgfqpoint{9.703704in}{2.022864in}}{\pgfqpoint{9.698866in}{2.024868in}}{\pgfqpoint{9.693822in}{2.024868in}}%
\pgfpathcurveto{\pgfqpoint{9.688778in}{2.024868in}}{\pgfqpoint{9.683941in}{2.022864in}}{\pgfqpoint{9.680374in}{2.019297in}}%
\pgfpathcurveto{\pgfqpoint{9.676808in}{2.015731in}}{\pgfqpoint{9.674804in}{2.010893in}}{\pgfqpoint{9.674804in}{2.005849in}}%
\pgfpathcurveto{\pgfqpoint{9.674804in}{2.000806in}}{\pgfqpoint{9.676808in}{1.995968in}}{\pgfqpoint{9.680374in}{1.992402in}}%
\pgfpathcurveto{\pgfqpoint{9.683941in}{1.988835in}}{\pgfqpoint{9.688778in}{1.986831in}}{\pgfqpoint{9.693822in}{1.986831in}}%
\pgfpathclose%
\pgfusepath{fill}%
\end{pgfscope}%
\begin{pgfscope}%
\pgfpathrectangle{\pgfqpoint{6.572727in}{0.473000in}}{\pgfqpoint{4.227273in}{3.311000in}}%
\pgfusepath{clip}%
\pgfsetbuttcap%
\pgfsetroundjoin%
\definecolor{currentfill}{rgb}{0.993248,0.906157,0.143936}%
\pgfsetfillcolor{currentfill}%
\pgfsetfillopacity{0.700000}%
\pgfsetlinewidth{0.000000pt}%
\definecolor{currentstroke}{rgb}{0.000000,0.000000,0.000000}%
\pgfsetstrokecolor{currentstroke}%
\pgfsetstrokeopacity{0.700000}%
\pgfsetdash{}{0pt}%
\pgfpathmoveto{\pgfqpoint{9.584245in}{2.010275in}}%
\pgfpathcurveto{\pgfqpoint{9.589289in}{2.010275in}}{\pgfqpoint{9.594127in}{2.012278in}}{\pgfqpoint{9.597693in}{2.015845in}}%
\pgfpathcurveto{\pgfqpoint{9.601259in}{2.019411in}}{\pgfqpoint{9.603263in}{2.024249in}}{\pgfqpoint{9.603263in}{2.029293in}}%
\pgfpathcurveto{\pgfqpoint{9.603263in}{2.034336in}}{\pgfqpoint{9.601259in}{2.039174in}}{\pgfqpoint{9.597693in}{2.042741in}}%
\pgfpathcurveto{\pgfqpoint{9.594127in}{2.046307in}}{\pgfqpoint{9.589289in}{2.048311in}}{\pgfqpoint{9.584245in}{2.048311in}}%
\pgfpathcurveto{\pgfqpoint{9.579201in}{2.048311in}}{\pgfqpoint{9.574364in}{2.046307in}}{\pgfqpoint{9.570797in}{2.042741in}}%
\pgfpathcurveto{\pgfqpoint{9.567231in}{2.039174in}}{\pgfqpoint{9.565227in}{2.034336in}}{\pgfqpoint{9.565227in}{2.029293in}}%
\pgfpathcurveto{\pgfqpoint{9.565227in}{2.024249in}}{\pgfqpoint{9.567231in}{2.019411in}}{\pgfqpoint{9.570797in}{2.015845in}}%
\pgfpathcurveto{\pgfqpoint{9.574364in}{2.012278in}}{\pgfqpoint{9.579201in}{2.010275in}}{\pgfqpoint{9.584245in}{2.010275in}}%
\pgfpathclose%
\pgfusepath{fill}%
\end{pgfscope}%
\begin{pgfscope}%
\pgfpathrectangle{\pgfqpoint{6.572727in}{0.473000in}}{\pgfqpoint{4.227273in}{3.311000in}}%
\pgfusepath{clip}%
\pgfsetbuttcap%
\pgfsetroundjoin%
\definecolor{currentfill}{rgb}{0.127568,0.566949,0.550556}%
\pgfsetfillcolor{currentfill}%
\pgfsetfillopacity{0.700000}%
\pgfsetlinewidth{0.000000pt}%
\definecolor{currentstroke}{rgb}{0.000000,0.000000,0.000000}%
\pgfsetstrokecolor{currentstroke}%
\pgfsetstrokeopacity{0.700000}%
\pgfsetdash{}{0pt}%
\pgfpathmoveto{\pgfqpoint{7.763943in}{1.446562in}}%
\pgfpathcurveto{\pgfqpoint{7.768987in}{1.446562in}}{\pgfqpoint{7.773825in}{1.448566in}}{\pgfqpoint{7.777391in}{1.452133in}}%
\pgfpathcurveto{\pgfqpoint{7.780958in}{1.455699in}}{\pgfqpoint{7.782961in}{1.460537in}}{\pgfqpoint{7.782961in}{1.465580in}}%
\pgfpathcurveto{\pgfqpoint{7.782961in}{1.470624in}}{\pgfqpoint{7.780958in}{1.475462in}}{\pgfqpoint{7.777391in}{1.479028in}}%
\pgfpathcurveto{\pgfqpoint{7.773825in}{1.482595in}}{\pgfqpoint{7.768987in}{1.484599in}}{\pgfqpoint{7.763943in}{1.484599in}}%
\pgfpathcurveto{\pgfqpoint{7.758900in}{1.484599in}}{\pgfqpoint{7.754062in}{1.482595in}}{\pgfqpoint{7.750495in}{1.479028in}}%
\pgfpathcurveto{\pgfqpoint{7.746929in}{1.475462in}}{\pgfqpoint{7.744925in}{1.470624in}}{\pgfqpoint{7.744925in}{1.465580in}}%
\pgfpathcurveto{\pgfqpoint{7.744925in}{1.460537in}}{\pgfqpoint{7.746929in}{1.455699in}}{\pgfqpoint{7.750495in}{1.452133in}}%
\pgfpathcurveto{\pgfqpoint{7.754062in}{1.448566in}}{\pgfqpoint{7.758900in}{1.446562in}}{\pgfqpoint{7.763943in}{1.446562in}}%
\pgfpathclose%
\pgfusepath{fill}%
\end{pgfscope}%
\begin{pgfscope}%
\pgfpathrectangle{\pgfqpoint{6.572727in}{0.473000in}}{\pgfqpoint{4.227273in}{3.311000in}}%
\pgfusepath{clip}%
\pgfsetbuttcap%
\pgfsetroundjoin%
\definecolor{currentfill}{rgb}{0.993248,0.906157,0.143936}%
\pgfsetfillcolor{currentfill}%
\pgfsetfillopacity{0.700000}%
\pgfsetlinewidth{0.000000pt}%
\definecolor{currentstroke}{rgb}{0.000000,0.000000,0.000000}%
\pgfsetstrokecolor{currentstroke}%
\pgfsetstrokeopacity{0.700000}%
\pgfsetdash{}{0pt}%
\pgfpathmoveto{\pgfqpoint{9.068067in}{1.459995in}}%
\pgfpathcurveto{\pgfqpoint{9.073110in}{1.459995in}}{\pgfqpoint{9.077948in}{1.461999in}}{\pgfqpoint{9.081515in}{1.465566in}}%
\pgfpathcurveto{\pgfqpoint{9.085081in}{1.469132in}}{\pgfqpoint{9.087085in}{1.473970in}}{\pgfqpoint{9.087085in}{1.479013in}}%
\pgfpathcurveto{\pgfqpoint{9.087085in}{1.484057in}}{\pgfqpoint{9.085081in}{1.488895in}}{\pgfqpoint{9.081515in}{1.492461in}}%
\pgfpathcurveto{\pgfqpoint{9.077948in}{1.496028in}}{\pgfqpoint{9.073110in}{1.498032in}}{\pgfqpoint{9.068067in}{1.498032in}}%
\pgfpathcurveto{\pgfqpoint{9.063023in}{1.498032in}}{\pgfqpoint{9.058185in}{1.496028in}}{\pgfqpoint{9.054619in}{1.492461in}}%
\pgfpathcurveto{\pgfqpoint{9.051052in}{1.488895in}}{\pgfqpoint{9.049049in}{1.484057in}}{\pgfqpoint{9.049049in}{1.479013in}}%
\pgfpathcurveto{\pgfqpoint{9.049049in}{1.473970in}}{\pgfqpoint{9.051052in}{1.469132in}}{\pgfqpoint{9.054619in}{1.465566in}}%
\pgfpathcurveto{\pgfqpoint{9.058185in}{1.461999in}}{\pgfqpoint{9.063023in}{1.459995in}}{\pgfqpoint{9.068067in}{1.459995in}}%
\pgfpathclose%
\pgfusepath{fill}%
\end{pgfscope}%
\begin{pgfscope}%
\pgfpathrectangle{\pgfqpoint{6.572727in}{0.473000in}}{\pgfqpoint{4.227273in}{3.311000in}}%
\pgfusepath{clip}%
\pgfsetbuttcap%
\pgfsetroundjoin%
\definecolor{currentfill}{rgb}{0.127568,0.566949,0.550556}%
\pgfsetfillcolor{currentfill}%
\pgfsetfillopacity{0.700000}%
\pgfsetlinewidth{0.000000pt}%
\definecolor{currentstroke}{rgb}{0.000000,0.000000,0.000000}%
\pgfsetstrokecolor{currentstroke}%
\pgfsetstrokeopacity{0.700000}%
\pgfsetdash{}{0pt}%
\pgfpathmoveto{\pgfqpoint{7.934818in}{1.488170in}}%
\pgfpathcurveto{\pgfqpoint{7.939861in}{1.488170in}}{\pgfqpoint{7.944699in}{1.490174in}}{\pgfqpoint{7.948265in}{1.493740in}}%
\pgfpathcurveto{\pgfqpoint{7.951832in}{1.497306in}}{\pgfqpoint{7.953836in}{1.502144in}}{\pgfqpoint{7.953836in}{1.507188in}}%
\pgfpathcurveto{\pgfqpoint{7.953836in}{1.512232in}}{\pgfqpoint{7.951832in}{1.517069in}}{\pgfqpoint{7.948265in}{1.520636in}}%
\pgfpathcurveto{\pgfqpoint{7.944699in}{1.524202in}}{\pgfqpoint{7.939861in}{1.526206in}}{\pgfqpoint{7.934818in}{1.526206in}}%
\pgfpathcurveto{\pgfqpoint{7.929774in}{1.526206in}}{\pgfqpoint{7.924936in}{1.524202in}}{\pgfqpoint{7.921370in}{1.520636in}}%
\pgfpathcurveto{\pgfqpoint{7.917803in}{1.517069in}}{\pgfqpoint{7.915799in}{1.512232in}}{\pgfqpoint{7.915799in}{1.507188in}}%
\pgfpathcurveto{\pgfqpoint{7.915799in}{1.502144in}}{\pgfqpoint{7.917803in}{1.497306in}}{\pgfqpoint{7.921370in}{1.493740in}}%
\pgfpathcurveto{\pgfqpoint{7.924936in}{1.490174in}}{\pgfqpoint{7.929774in}{1.488170in}}{\pgfqpoint{7.934818in}{1.488170in}}%
\pgfpathclose%
\pgfusepath{fill}%
\end{pgfscope}%
\begin{pgfscope}%
\pgfpathrectangle{\pgfqpoint{6.572727in}{0.473000in}}{\pgfqpoint{4.227273in}{3.311000in}}%
\pgfusepath{clip}%
\pgfsetbuttcap%
\pgfsetroundjoin%
\definecolor{currentfill}{rgb}{0.127568,0.566949,0.550556}%
\pgfsetfillcolor{currentfill}%
\pgfsetfillopacity{0.700000}%
\pgfsetlinewidth{0.000000pt}%
\definecolor{currentstroke}{rgb}{0.000000,0.000000,0.000000}%
\pgfsetstrokecolor{currentstroke}%
\pgfsetstrokeopacity{0.700000}%
\pgfsetdash{}{0pt}%
\pgfpathmoveto{\pgfqpoint{8.053854in}{1.922341in}}%
\pgfpathcurveto{\pgfqpoint{8.058898in}{1.922341in}}{\pgfqpoint{8.063736in}{1.924345in}}{\pgfqpoint{8.067302in}{1.927911in}}%
\pgfpathcurveto{\pgfqpoint{8.070869in}{1.931478in}}{\pgfqpoint{8.072873in}{1.936316in}}{\pgfqpoint{8.072873in}{1.941359in}}%
\pgfpathcurveto{\pgfqpoint{8.072873in}{1.946403in}}{\pgfqpoint{8.070869in}{1.951241in}}{\pgfqpoint{8.067302in}{1.954807in}}%
\pgfpathcurveto{\pgfqpoint{8.063736in}{1.958374in}}{\pgfqpoint{8.058898in}{1.960377in}}{\pgfqpoint{8.053854in}{1.960377in}}%
\pgfpathcurveto{\pgfqpoint{8.048811in}{1.960377in}}{\pgfqpoint{8.043973in}{1.958374in}}{\pgfqpoint{8.040407in}{1.954807in}}%
\pgfpathcurveto{\pgfqpoint{8.036840in}{1.951241in}}{\pgfqpoint{8.034836in}{1.946403in}}{\pgfqpoint{8.034836in}{1.941359in}}%
\pgfpathcurveto{\pgfqpoint{8.034836in}{1.936316in}}{\pgfqpoint{8.036840in}{1.931478in}}{\pgfqpoint{8.040407in}{1.927911in}}%
\pgfpathcurveto{\pgfqpoint{8.043973in}{1.924345in}}{\pgfqpoint{8.048811in}{1.922341in}}{\pgfqpoint{8.053854in}{1.922341in}}%
\pgfpathclose%
\pgfusepath{fill}%
\end{pgfscope}%
\begin{pgfscope}%
\pgfpathrectangle{\pgfqpoint{6.572727in}{0.473000in}}{\pgfqpoint{4.227273in}{3.311000in}}%
\pgfusepath{clip}%
\pgfsetbuttcap%
\pgfsetroundjoin%
\definecolor{currentfill}{rgb}{0.127568,0.566949,0.550556}%
\pgfsetfillcolor{currentfill}%
\pgfsetfillopacity{0.700000}%
\pgfsetlinewidth{0.000000pt}%
\definecolor{currentstroke}{rgb}{0.000000,0.000000,0.000000}%
\pgfsetstrokecolor{currentstroke}%
\pgfsetstrokeopacity{0.700000}%
\pgfsetdash{}{0pt}%
\pgfpathmoveto{\pgfqpoint{7.732711in}{1.424989in}}%
\pgfpathcurveto{\pgfqpoint{7.737755in}{1.424989in}}{\pgfqpoint{7.742592in}{1.426993in}}{\pgfqpoint{7.746159in}{1.430560in}}%
\pgfpathcurveto{\pgfqpoint{7.749725in}{1.434126in}}{\pgfqpoint{7.751729in}{1.438964in}}{\pgfqpoint{7.751729in}{1.444007in}}%
\pgfpathcurveto{\pgfqpoint{7.751729in}{1.449051in}}{\pgfqpoint{7.749725in}{1.453889in}}{\pgfqpoint{7.746159in}{1.457455in}}%
\pgfpathcurveto{\pgfqpoint{7.742592in}{1.461022in}}{\pgfqpoint{7.737755in}{1.463026in}}{\pgfqpoint{7.732711in}{1.463026in}}%
\pgfpathcurveto{\pgfqpoint{7.727667in}{1.463026in}}{\pgfqpoint{7.722830in}{1.461022in}}{\pgfqpoint{7.719263in}{1.457455in}}%
\pgfpathcurveto{\pgfqpoint{7.715697in}{1.453889in}}{\pgfqpoint{7.713693in}{1.449051in}}{\pgfqpoint{7.713693in}{1.444007in}}%
\pgfpathcurveto{\pgfqpoint{7.713693in}{1.438964in}}{\pgfqpoint{7.715697in}{1.434126in}}{\pgfqpoint{7.719263in}{1.430560in}}%
\pgfpathcurveto{\pgfqpoint{7.722830in}{1.426993in}}{\pgfqpoint{7.727667in}{1.424989in}}{\pgfqpoint{7.732711in}{1.424989in}}%
\pgfpathclose%
\pgfusepath{fill}%
\end{pgfscope}%
\begin{pgfscope}%
\pgfpathrectangle{\pgfqpoint{6.572727in}{0.473000in}}{\pgfqpoint{4.227273in}{3.311000in}}%
\pgfusepath{clip}%
\pgfsetbuttcap%
\pgfsetroundjoin%
\definecolor{currentfill}{rgb}{0.127568,0.566949,0.550556}%
\pgfsetfillcolor{currentfill}%
\pgfsetfillopacity{0.700000}%
\pgfsetlinewidth{0.000000pt}%
\definecolor{currentstroke}{rgb}{0.000000,0.000000,0.000000}%
\pgfsetstrokecolor{currentstroke}%
\pgfsetstrokeopacity{0.700000}%
\pgfsetdash{}{0pt}%
\pgfpathmoveto{\pgfqpoint{8.593097in}{1.155204in}}%
\pgfpathcurveto{\pgfqpoint{8.598140in}{1.155204in}}{\pgfqpoint{8.602978in}{1.157208in}}{\pgfqpoint{8.606544in}{1.160774in}}%
\pgfpathcurveto{\pgfqpoint{8.610111in}{1.164341in}}{\pgfqpoint{8.612115in}{1.169178in}}{\pgfqpoint{8.612115in}{1.174222in}}%
\pgfpathcurveto{\pgfqpoint{8.612115in}{1.179266in}}{\pgfqpoint{8.610111in}{1.184104in}}{\pgfqpoint{8.606544in}{1.187670in}}%
\pgfpathcurveto{\pgfqpoint{8.602978in}{1.191236in}}{\pgfqpoint{8.598140in}{1.193240in}}{\pgfqpoint{8.593097in}{1.193240in}}%
\pgfpathcurveto{\pgfqpoint{8.588053in}{1.193240in}}{\pgfqpoint{8.583215in}{1.191236in}}{\pgfqpoint{8.579649in}{1.187670in}}%
\pgfpathcurveto{\pgfqpoint{8.576082in}{1.184104in}}{\pgfqpoint{8.574078in}{1.179266in}}{\pgfqpoint{8.574078in}{1.174222in}}%
\pgfpathcurveto{\pgfqpoint{8.574078in}{1.169178in}}{\pgfqpoint{8.576082in}{1.164341in}}{\pgfqpoint{8.579649in}{1.160774in}}%
\pgfpathcurveto{\pgfqpoint{8.583215in}{1.157208in}}{\pgfqpoint{8.588053in}{1.155204in}}{\pgfqpoint{8.593097in}{1.155204in}}%
\pgfpathclose%
\pgfusepath{fill}%
\end{pgfscope}%
\begin{pgfscope}%
\pgfpathrectangle{\pgfqpoint{6.572727in}{0.473000in}}{\pgfqpoint{4.227273in}{3.311000in}}%
\pgfusepath{clip}%
\pgfsetbuttcap%
\pgfsetroundjoin%
\definecolor{currentfill}{rgb}{0.993248,0.906157,0.143936}%
\pgfsetfillcolor{currentfill}%
\pgfsetfillopacity{0.700000}%
\pgfsetlinewidth{0.000000pt}%
\definecolor{currentstroke}{rgb}{0.000000,0.000000,0.000000}%
\pgfsetstrokecolor{currentstroke}%
\pgfsetstrokeopacity{0.700000}%
\pgfsetdash{}{0pt}%
\pgfpathmoveto{\pgfqpoint{9.435072in}{1.811988in}}%
\pgfpathcurveto{\pgfqpoint{9.440116in}{1.811988in}}{\pgfqpoint{9.444953in}{1.813992in}}{\pgfqpoint{9.448520in}{1.817558in}}%
\pgfpathcurveto{\pgfqpoint{9.452086in}{1.821125in}}{\pgfqpoint{9.454090in}{1.825962in}}{\pgfqpoint{9.454090in}{1.831006in}}%
\pgfpathcurveto{\pgfqpoint{9.454090in}{1.836050in}}{\pgfqpoint{9.452086in}{1.840887in}}{\pgfqpoint{9.448520in}{1.844454in}}%
\pgfpathcurveto{\pgfqpoint{9.444953in}{1.848020in}}{\pgfqpoint{9.440116in}{1.850024in}}{\pgfqpoint{9.435072in}{1.850024in}}%
\pgfpathcurveto{\pgfqpoint{9.430028in}{1.850024in}}{\pgfqpoint{9.425190in}{1.848020in}}{\pgfqpoint{9.421624in}{1.844454in}}%
\pgfpathcurveto{\pgfqpoint{9.418058in}{1.840887in}}{\pgfqpoint{9.416054in}{1.836050in}}{\pgfqpoint{9.416054in}{1.831006in}}%
\pgfpathcurveto{\pgfqpoint{9.416054in}{1.825962in}}{\pgfqpoint{9.418058in}{1.821125in}}{\pgfqpoint{9.421624in}{1.817558in}}%
\pgfpathcurveto{\pgfqpoint{9.425190in}{1.813992in}}{\pgfqpoint{9.430028in}{1.811988in}}{\pgfqpoint{9.435072in}{1.811988in}}%
\pgfpathclose%
\pgfusepath{fill}%
\end{pgfscope}%
\begin{pgfscope}%
\pgfpathrectangle{\pgfqpoint{6.572727in}{0.473000in}}{\pgfqpoint{4.227273in}{3.311000in}}%
\pgfusepath{clip}%
\pgfsetbuttcap%
\pgfsetroundjoin%
\definecolor{currentfill}{rgb}{0.127568,0.566949,0.550556}%
\pgfsetfillcolor{currentfill}%
\pgfsetfillopacity{0.700000}%
\pgfsetlinewidth{0.000000pt}%
\definecolor{currentstroke}{rgb}{0.000000,0.000000,0.000000}%
\pgfsetstrokecolor{currentstroke}%
\pgfsetstrokeopacity{0.700000}%
\pgfsetdash{}{0pt}%
\pgfpathmoveto{\pgfqpoint{7.853059in}{1.423830in}}%
\pgfpathcurveto{\pgfqpoint{7.858103in}{1.423830in}}{\pgfqpoint{7.862941in}{1.425834in}}{\pgfqpoint{7.866507in}{1.429401in}}%
\pgfpathcurveto{\pgfqpoint{7.870073in}{1.432967in}}{\pgfqpoint{7.872077in}{1.437805in}}{\pgfqpoint{7.872077in}{1.442848in}}%
\pgfpathcurveto{\pgfqpoint{7.872077in}{1.447892in}}{\pgfqpoint{7.870073in}{1.452730in}}{\pgfqpoint{7.866507in}{1.456296in}}%
\pgfpathcurveto{\pgfqpoint{7.862941in}{1.459863in}}{\pgfqpoint{7.858103in}{1.461867in}}{\pgfqpoint{7.853059in}{1.461867in}}%
\pgfpathcurveto{\pgfqpoint{7.848016in}{1.461867in}}{\pgfqpoint{7.843178in}{1.459863in}}{\pgfqpoint{7.839611in}{1.456296in}}%
\pgfpathcurveto{\pgfqpoint{7.836045in}{1.452730in}}{\pgfqpoint{7.834041in}{1.447892in}}{\pgfqpoint{7.834041in}{1.442848in}}%
\pgfpathcurveto{\pgfqpoint{7.834041in}{1.437805in}}{\pgfqpoint{7.836045in}{1.432967in}}{\pgfqpoint{7.839611in}{1.429401in}}%
\pgfpathcurveto{\pgfqpoint{7.843178in}{1.425834in}}{\pgfqpoint{7.848016in}{1.423830in}}{\pgfqpoint{7.853059in}{1.423830in}}%
\pgfpathclose%
\pgfusepath{fill}%
\end{pgfscope}%
\begin{pgfscope}%
\pgfpathrectangle{\pgfqpoint{6.572727in}{0.473000in}}{\pgfqpoint{4.227273in}{3.311000in}}%
\pgfusepath{clip}%
\pgfsetbuttcap%
\pgfsetroundjoin%
\definecolor{currentfill}{rgb}{0.127568,0.566949,0.550556}%
\pgfsetfillcolor{currentfill}%
\pgfsetfillopacity{0.700000}%
\pgfsetlinewidth{0.000000pt}%
\definecolor{currentstroke}{rgb}{0.000000,0.000000,0.000000}%
\pgfsetstrokecolor{currentstroke}%
\pgfsetstrokeopacity{0.700000}%
\pgfsetdash{}{0pt}%
\pgfpathmoveto{\pgfqpoint{7.529593in}{1.455644in}}%
\pgfpathcurveto{\pgfqpoint{7.534637in}{1.455644in}}{\pgfqpoint{7.539474in}{1.457648in}}{\pgfqpoint{7.543041in}{1.461214in}}%
\pgfpathcurveto{\pgfqpoint{7.546607in}{1.464781in}}{\pgfqpoint{7.548611in}{1.469619in}}{\pgfqpoint{7.548611in}{1.474662in}}%
\pgfpathcurveto{\pgfqpoint{7.548611in}{1.479706in}}{\pgfqpoint{7.546607in}{1.484544in}}{\pgfqpoint{7.543041in}{1.488110in}}%
\pgfpathcurveto{\pgfqpoint{7.539474in}{1.491677in}}{\pgfqpoint{7.534637in}{1.493680in}}{\pgfqpoint{7.529593in}{1.493680in}}%
\pgfpathcurveto{\pgfqpoint{7.524549in}{1.493680in}}{\pgfqpoint{7.519712in}{1.491677in}}{\pgfqpoint{7.516145in}{1.488110in}}%
\pgfpathcurveto{\pgfqpoint{7.512579in}{1.484544in}}{\pgfqpoint{7.510575in}{1.479706in}}{\pgfqpoint{7.510575in}{1.474662in}}%
\pgfpathcurveto{\pgfqpoint{7.510575in}{1.469619in}}{\pgfqpoint{7.512579in}{1.464781in}}{\pgfqpoint{7.516145in}{1.461214in}}%
\pgfpathcurveto{\pgfqpoint{7.519712in}{1.457648in}}{\pgfqpoint{7.524549in}{1.455644in}}{\pgfqpoint{7.529593in}{1.455644in}}%
\pgfpathclose%
\pgfusepath{fill}%
\end{pgfscope}%
\begin{pgfscope}%
\pgfpathrectangle{\pgfqpoint{6.572727in}{0.473000in}}{\pgfqpoint{4.227273in}{3.311000in}}%
\pgfusepath{clip}%
\pgfsetbuttcap%
\pgfsetroundjoin%
\definecolor{currentfill}{rgb}{0.127568,0.566949,0.550556}%
\pgfsetfillcolor{currentfill}%
\pgfsetfillopacity{0.700000}%
\pgfsetlinewidth{0.000000pt}%
\definecolor{currentstroke}{rgb}{0.000000,0.000000,0.000000}%
\pgfsetstrokecolor{currentstroke}%
\pgfsetstrokeopacity{0.700000}%
\pgfsetdash{}{0pt}%
\pgfpathmoveto{\pgfqpoint{8.803362in}{3.382680in}}%
\pgfpathcurveto{\pgfqpoint{8.808406in}{3.382680in}}{\pgfqpoint{8.813243in}{3.384684in}}{\pgfqpoint{8.816810in}{3.388250in}}%
\pgfpathcurveto{\pgfqpoint{8.820376in}{3.391817in}}{\pgfqpoint{8.822380in}{3.396655in}}{\pgfqpoint{8.822380in}{3.401698in}}%
\pgfpathcurveto{\pgfqpoint{8.822380in}{3.406742in}}{\pgfqpoint{8.820376in}{3.411580in}}{\pgfqpoint{8.816810in}{3.415146in}}%
\pgfpathcurveto{\pgfqpoint{8.813243in}{3.418713in}}{\pgfqpoint{8.808406in}{3.420716in}}{\pgfqpoint{8.803362in}{3.420716in}}%
\pgfpathcurveto{\pgfqpoint{8.798318in}{3.420716in}}{\pgfqpoint{8.793481in}{3.418713in}}{\pgfqpoint{8.789914in}{3.415146in}}%
\pgfpathcurveto{\pgfqpoint{8.786348in}{3.411580in}}{\pgfqpoint{8.784344in}{3.406742in}}{\pgfqpoint{8.784344in}{3.401698in}}%
\pgfpathcurveto{\pgfqpoint{8.784344in}{3.396655in}}{\pgfqpoint{8.786348in}{3.391817in}}{\pgfqpoint{8.789914in}{3.388250in}}%
\pgfpathcurveto{\pgfqpoint{8.793481in}{3.384684in}}{\pgfqpoint{8.798318in}{3.382680in}}{\pgfqpoint{8.803362in}{3.382680in}}%
\pgfpathclose%
\pgfusepath{fill}%
\end{pgfscope}%
\begin{pgfscope}%
\pgfpathrectangle{\pgfqpoint{6.572727in}{0.473000in}}{\pgfqpoint{4.227273in}{3.311000in}}%
\pgfusepath{clip}%
\pgfsetbuttcap%
\pgfsetroundjoin%
\definecolor{currentfill}{rgb}{0.127568,0.566949,0.550556}%
\pgfsetfillcolor{currentfill}%
\pgfsetfillopacity{0.700000}%
\pgfsetlinewidth{0.000000pt}%
\definecolor{currentstroke}{rgb}{0.000000,0.000000,0.000000}%
\pgfsetstrokecolor{currentstroke}%
\pgfsetstrokeopacity{0.700000}%
\pgfsetdash{}{0pt}%
\pgfpathmoveto{\pgfqpoint{7.813722in}{3.457631in}}%
\pgfpathcurveto{\pgfqpoint{7.818766in}{3.457631in}}{\pgfqpoint{7.823603in}{3.459635in}}{\pgfqpoint{7.827170in}{3.463202in}}%
\pgfpathcurveto{\pgfqpoint{7.830736in}{3.466768in}}{\pgfqpoint{7.832740in}{3.471606in}}{\pgfqpoint{7.832740in}{3.476649in}}%
\pgfpathcurveto{\pgfqpoint{7.832740in}{3.481693in}}{\pgfqpoint{7.830736in}{3.486531in}}{\pgfqpoint{7.827170in}{3.490097in}}%
\pgfpathcurveto{\pgfqpoint{7.823603in}{3.493664in}}{\pgfqpoint{7.818766in}{3.495668in}}{\pgfqpoint{7.813722in}{3.495668in}}%
\pgfpathcurveto{\pgfqpoint{7.808678in}{3.495668in}}{\pgfqpoint{7.803841in}{3.493664in}}{\pgfqpoint{7.800274in}{3.490097in}}%
\pgfpathcurveto{\pgfqpoint{7.796708in}{3.486531in}}{\pgfqpoint{7.794704in}{3.481693in}}{\pgfqpoint{7.794704in}{3.476649in}}%
\pgfpathcurveto{\pgfqpoint{7.794704in}{3.471606in}}{\pgfqpoint{7.796708in}{3.466768in}}{\pgfqpoint{7.800274in}{3.463202in}}%
\pgfpathcurveto{\pgfqpoint{7.803841in}{3.459635in}}{\pgfqpoint{7.808678in}{3.457631in}}{\pgfqpoint{7.813722in}{3.457631in}}%
\pgfpathclose%
\pgfusepath{fill}%
\end{pgfscope}%
\begin{pgfscope}%
\pgfpathrectangle{\pgfqpoint{6.572727in}{0.473000in}}{\pgfqpoint{4.227273in}{3.311000in}}%
\pgfusepath{clip}%
\pgfsetbuttcap%
\pgfsetroundjoin%
\definecolor{currentfill}{rgb}{0.127568,0.566949,0.550556}%
\pgfsetfillcolor{currentfill}%
\pgfsetfillopacity{0.700000}%
\pgfsetlinewidth{0.000000pt}%
\definecolor{currentstroke}{rgb}{0.000000,0.000000,0.000000}%
\pgfsetstrokecolor{currentstroke}%
\pgfsetstrokeopacity{0.700000}%
\pgfsetdash{}{0pt}%
\pgfpathmoveto{\pgfqpoint{7.181732in}{1.416944in}}%
\pgfpathcurveto{\pgfqpoint{7.186775in}{1.416944in}}{\pgfqpoint{7.191613in}{1.418948in}}{\pgfqpoint{7.195180in}{1.422515in}}%
\pgfpathcurveto{\pgfqpoint{7.198746in}{1.426081in}}{\pgfqpoint{7.200750in}{1.430919in}}{\pgfqpoint{7.200750in}{1.435963in}}%
\pgfpathcurveto{\pgfqpoint{7.200750in}{1.441006in}}{\pgfqpoint{7.198746in}{1.445844in}}{\pgfqpoint{7.195180in}{1.449410in}}%
\pgfpathcurveto{\pgfqpoint{7.191613in}{1.452977in}}{\pgfqpoint{7.186775in}{1.454981in}}{\pgfqpoint{7.181732in}{1.454981in}}%
\pgfpathcurveto{\pgfqpoint{7.176688in}{1.454981in}}{\pgfqpoint{7.171850in}{1.452977in}}{\pgfqpoint{7.168284in}{1.449410in}}%
\pgfpathcurveto{\pgfqpoint{7.164717in}{1.445844in}}{\pgfqpoint{7.162714in}{1.441006in}}{\pgfqpoint{7.162714in}{1.435963in}}%
\pgfpathcurveto{\pgfqpoint{7.162714in}{1.430919in}}{\pgfqpoint{7.164717in}{1.426081in}}{\pgfqpoint{7.168284in}{1.422515in}}%
\pgfpathcurveto{\pgfqpoint{7.171850in}{1.418948in}}{\pgfqpoint{7.176688in}{1.416944in}}{\pgfqpoint{7.181732in}{1.416944in}}%
\pgfpathclose%
\pgfusepath{fill}%
\end{pgfscope}%
\begin{pgfscope}%
\pgfpathrectangle{\pgfqpoint{6.572727in}{0.473000in}}{\pgfqpoint{4.227273in}{3.311000in}}%
\pgfusepath{clip}%
\pgfsetbuttcap%
\pgfsetroundjoin%
\definecolor{currentfill}{rgb}{0.127568,0.566949,0.550556}%
\pgfsetfillcolor{currentfill}%
\pgfsetfillopacity{0.700000}%
\pgfsetlinewidth{0.000000pt}%
\definecolor{currentstroke}{rgb}{0.000000,0.000000,0.000000}%
\pgfsetstrokecolor{currentstroke}%
\pgfsetstrokeopacity{0.700000}%
\pgfsetdash{}{0pt}%
\pgfpathmoveto{\pgfqpoint{8.454971in}{1.527489in}}%
\pgfpathcurveto{\pgfqpoint{8.460015in}{1.527489in}}{\pgfqpoint{8.464853in}{1.529493in}}{\pgfqpoint{8.468419in}{1.533059in}}%
\pgfpathcurveto{\pgfqpoint{8.471985in}{1.536626in}}{\pgfqpoint{8.473989in}{1.541463in}}{\pgfqpoint{8.473989in}{1.546507in}}%
\pgfpathcurveto{\pgfqpoint{8.473989in}{1.551551in}}{\pgfqpoint{8.471985in}{1.556388in}}{\pgfqpoint{8.468419in}{1.559955in}}%
\pgfpathcurveto{\pgfqpoint{8.464853in}{1.563521in}}{\pgfqpoint{8.460015in}{1.565525in}}{\pgfqpoint{8.454971in}{1.565525in}}%
\pgfpathcurveto{\pgfqpoint{8.449928in}{1.565525in}}{\pgfqpoint{8.445090in}{1.563521in}}{\pgfqpoint{8.441523in}{1.559955in}}%
\pgfpathcurveto{\pgfqpoint{8.437957in}{1.556388in}}{\pgfqpoint{8.435953in}{1.551551in}}{\pgfqpoint{8.435953in}{1.546507in}}%
\pgfpathcurveto{\pgfqpoint{8.435953in}{1.541463in}}{\pgfqpoint{8.437957in}{1.536626in}}{\pgfqpoint{8.441523in}{1.533059in}}%
\pgfpathcurveto{\pgfqpoint{8.445090in}{1.529493in}}{\pgfqpoint{8.449928in}{1.527489in}}{\pgfqpoint{8.454971in}{1.527489in}}%
\pgfpathclose%
\pgfusepath{fill}%
\end{pgfscope}%
\begin{pgfscope}%
\pgfpathrectangle{\pgfqpoint{6.572727in}{0.473000in}}{\pgfqpoint{4.227273in}{3.311000in}}%
\pgfusepath{clip}%
\pgfsetbuttcap%
\pgfsetroundjoin%
\definecolor{currentfill}{rgb}{0.993248,0.906157,0.143936}%
\pgfsetfillcolor{currentfill}%
\pgfsetfillopacity{0.700000}%
\pgfsetlinewidth{0.000000pt}%
\definecolor{currentstroke}{rgb}{0.000000,0.000000,0.000000}%
\pgfsetstrokecolor{currentstroke}%
\pgfsetstrokeopacity{0.700000}%
\pgfsetdash{}{0pt}%
\pgfpathmoveto{\pgfqpoint{9.700941in}{2.070366in}}%
\pgfpathcurveto{\pgfqpoint{9.705985in}{2.070366in}}{\pgfqpoint{9.710823in}{2.072370in}}{\pgfqpoint{9.714389in}{2.075937in}}%
\pgfpathcurveto{\pgfqpoint{9.717956in}{2.079503in}}{\pgfqpoint{9.719959in}{2.084341in}}{\pgfqpoint{9.719959in}{2.089384in}}%
\pgfpathcurveto{\pgfqpoint{9.719959in}{2.094428in}}{\pgfqpoint{9.717956in}{2.099266in}}{\pgfqpoint{9.714389in}{2.102832in}}%
\pgfpathcurveto{\pgfqpoint{9.710823in}{2.106399in}}{\pgfqpoint{9.705985in}{2.108403in}}{\pgfqpoint{9.700941in}{2.108403in}}%
\pgfpathcurveto{\pgfqpoint{9.695898in}{2.108403in}}{\pgfqpoint{9.691060in}{2.106399in}}{\pgfqpoint{9.687493in}{2.102832in}}%
\pgfpathcurveto{\pgfqpoint{9.683927in}{2.099266in}}{\pgfqpoint{9.681923in}{2.094428in}}{\pgfqpoint{9.681923in}{2.089384in}}%
\pgfpathcurveto{\pgfqpoint{9.681923in}{2.084341in}}{\pgfqpoint{9.683927in}{2.079503in}}{\pgfqpoint{9.687493in}{2.075937in}}%
\pgfpathcurveto{\pgfqpoint{9.691060in}{2.072370in}}{\pgfqpoint{9.695898in}{2.070366in}}{\pgfqpoint{9.700941in}{2.070366in}}%
\pgfpathclose%
\pgfusepath{fill}%
\end{pgfscope}%
\begin{pgfscope}%
\pgfpathrectangle{\pgfqpoint{6.572727in}{0.473000in}}{\pgfqpoint{4.227273in}{3.311000in}}%
\pgfusepath{clip}%
\pgfsetbuttcap%
\pgfsetroundjoin%
\definecolor{currentfill}{rgb}{0.127568,0.566949,0.550556}%
\pgfsetfillcolor{currentfill}%
\pgfsetfillopacity{0.700000}%
\pgfsetlinewidth{0.000000pt}%
\definecolor{currentstroke}{rgb}{0.000000,0.000000,0.000000}%
\pgfsetstrokecolor{currentstroke}%
\pgfsetstrokeopacity{0.700000}%
\pgfsetdash{}{0pt}%
\pgfpathmoveto{\pgfqpoint{7.978174in}{2.344570in}}%
\pgfpathcurveto{\pgfqpoint{7.983218in}{2.344570in}}{\pgfqpoint{7.988056in}{2.346573in}}{\pgfqpoint{7.991622in}{2.350140in}}%
\pgfpathcurveto{\pgfqpoint{7.995189in}{2.353706in}}{\pgfqpoint{7.997192in}{2.358544in}}{\pgfqpoint{7.997192in}{2.363588in}}%
\pgfpathcurveto{\pgfqpoint{7.997192in}{2.368631in}}{\pgfqpoint{7.995189in}{2.373469in}}{\pgfqpoint{7.991622in}{2.377036in}}%
\pgfpathcurveto{\pgfqpoint{7.988056in}{2.380602in}}{\pgfqpoint{7.983218in}{2.382606in}}{\pgfqpoint{7.978174in}{2.382606in}}%
\pgfpathcurveto{\pgfqpoint{7.973131in}{2.382606in}}{\pgfqpoint{7.968293in}{2.380602in}}{\pgfqpoint{7.964726in}{2.377036in}}%
\pgfpathcurveto{\pgfqpoint{7.961160in}{2.373469in}}{\pgfqpoint{7.959156in}{2.368631in}}{\pgfqpoint{7.959156in}{2.363588in}}%
\pgfpathcurveto{\pgfqpoint{7.959156in}{2.358544in}}{\pgfqpoint{7.961160in}{2.353706in}}{\pgfqpoint{7.964726in}{2.350140in}}%
\pgfpathcurveto{\pgfqpoint{7.968293in}{2.346573in}}{\pgfqpoint{7.973131in}{2.344570in}}{\pgfqpoint{7.978174in}{2.344570in}}%
\pgfpathclose%
\pgfusepath{fill}%
\end{pgfscope}%
\begin{pgfscope}%
\pgfpathrectangle{\pgfqpoint{6.572727in}{0.473000in}}{\pgfqpoint{4.227273in}{3.311000in}}%
\pgfusepath{clip}%
\pgfsetbuttcap%
\pgfsetroundjoin%
\definecolor{currentfill}{rgb}{0.267004,0.004874,0.329415}%
\pgfsetfillcolor{currentfill}%
\pgfsetfillopacity{0.700000}%
\pgfsetlinewidth{0.000000pt}%
\definecolor{currentstroke}{rgb}{0.000000,0.000000,0.000000}%
\pgfsetstrokecolor{currentstroke}%
\pgfsetstrokeopacity{0.700000}%
\pgfsetdash{}{0pt}%
\pgfpathmoveto{\pgfqpoint{9.235591in}{2.782650in}}%
\pgfpathcurveto{\pgfqpoint{9.240635in}{2.782650in}}{\pgfqpoint{9.245472in}{2.784654in}}{\pgfqpoint{9.249039in}{2.788220in}}%
\pgfpathcurveto{\pgfqpoint{9.252605in}{2.791787in}}{\pgfqpoint{9.254609in}{2.796624in}}{\pgfqpoint{9.254609in}{2.801668in}}%
\pgfpathcurveto{\pgfqpoint{9.254609in}{2.806712in}}{\pgfqpoint{9.252605in}{2.811549in}}{\pgfqpoint{9.249039in}{2.815116in}}%
\pgfpathcurveto{\pgfqpoint{9.245472in}{2.818682in}}{\pgfqpoint{9.240635in}{2.820686in}}{\pgfqpoint{9.235591in}{2.820686in}}%
\pgfpathcurveto{\pgfqpoint{9.230547in}{2.820686in}}{\pgfqpoint{9.225709in}{2.818682in}}{\pgfqpoint{9.222143in}{2.815116in}}%
\pgfpathcurveto{\pgfqpoint{9.218577in}{2.811549in}}{\pgfqpoint{9.216573in}{2.806712in}}{\pgfqpoint{9.216573in}{2.801668in}}%
\pgfpathcurveto{\pgfqpoint{9.216573in}{2.796624in}}{\pgfqpoint{9.218577in}{2.791787in}}{\pgfqpoint{9.222143in}{2.788220in}}%
\pgfpathcurveto{\pgfqpoint{9.225709in}{2.784654in}}{\pgfqpoint{9.230547in}{2.782650in}}{\pgfqpoint{9.235591in}{2.782650in}}%
\pgfpathclose%
\pgfusepath{fill}%
\end{pgfscope}%
\begin{pgfscope}%
\pgfpathrectangle{\pgfqpoint{6.572727in}{0.473000in}}{\pgfqpoint{4.227273in}{3.311000in}}%
\pgfusepath{clip}%
\pgfsetbuttcap%
\pgfsetroundjoin%
\definecolor{currentfill}{rgb}{0.127568,0.566949,0.550556}%
\pgfsetfillcolor{currentfill}%
\pgfsetfillopacity{0.700000}%
\pgfsetlinewidth{0.000000pt}%
\definecolor{currentstroke}{rgb}{0.000000,0.000000,0.000000}%
\pgfsetstrokecolor{currentstroke}%
\pgfsetstrokeopacity{0.700000}%
\pgfsetdash{}{0pt}%
\pgfpathmoveto{\pgfqpoint{8.167309in}{1.713082in}}%
\pgfpathcurveto{\pgfqpoint{8.172353in}{1.713082in}}{\pgfqpoint{8.177190in}{1.715086in}}{\pgfqpoint{8.180757in}{1.718652in}}%
\pgfpathcurveto{\pgfqpoint{8.184323in}{1.722219in}}{\pgfqpoint{8.186327in}{1.727057in}}{\pgfqpoint{8.186327in}{1.732100in}}%
\pgfpathcurveto{\pgfqpoint{8.186327in}{1.737144in}}{\pgfqpoint{8.184323in}{1.741982in}}{\pgfqpoint{8.180757in}{1.745548in}}%
\pgfpathcurveto{\pgfqpoint{8.177190in}{1.749115in}}{\pgfqpoint{8.172353in}{1.751118in}}{\pgfqpoint{8.167309in}{1.751118in}}%
\pgfpathcurveto{\pgfqpoint{8.162265in}{1.751118in}}{\pgfqpoint{8.157427in}{1.749115in}}{\pgfqpoint{8.153861in}{1.745548in}}%
\pgfpathcurveto{\pgfqpoint{8.150295in}{1.741982in}}{\pgfqpoint{8.148291in}{1.737144in}}{\pgfqpoint{8.148291in}{1.732100in}}%
\pgfpathcurveto{\pgfqpoint{8.148291in}{1.727057in}}{\pgfqpoint{8.150295in}{1.722219in}}{\pgfqpoint{8.153861in}{1.718652in}}%
\pgfpathcurveto{\pgfqpoint{8.157427in}{1.715086in}}{\pgfqpoint{8.162265in}{1.713082in}}{\pgfqpoint{8.167309in}{1.713082in}}%
\pgfpathclose%
\pgfusepath{fill}%
\end{pgfscope}%
\begin{pgfscope}%
\pgfpathrectangle{\pgfqpoint{6.572727in}{0.473000in}}{\pgfqpoint{4.227273in}{3.311000in}}%
\pgfusepath{clip}%
\pgfsetbuttcap%
\pgfsetroundjoin%
\definecolor{currentfill}{rgb}{0.993248,0.906157,0.143936}%
\pgfsetfillcolor{currentfill}%
\pgfsetfillopacity{0.700000}%
\pgfsetlinewidth{0.000000pt}%
\definecolor{currentstroke}{rgb}{0.000000,0.000000,0.000000}%
\pgfsetstrokecolor{currentstroke}%
\pgfsetstrokeopacity{0.700000}%
\pgfsetdash{}{0pt}%
\pgfpathmoveto{\pgfqpoint{9.014533in}{1.604201in}}%
\pgfpathcurveto{\pgfqpoint{9.019576in}{1.604201in}}{\pgfqpoint{9.024414in}{1.606205in}}{\pgfqpoint{9.027981in}{1.609771in}}%
\pgfpathcurveto{\pgfqpoint{9.031547in}{1.613337in}}{\pgfqpoint{9.033551in}{1.618175in}}{\pgfqpoint{9.033551in}{1.623219in}}%
\pgfpathcurveto{\pgfqpoint{9.033551in}{1.628263in}}{\pgfqpoint{9.031547in}{1.633100in}}{\pgfqpoint{9.027981in}{1.636667in}}%
\pgfpathcurveto{\pgfqpoint{9.024414in}{1.640233in}}{\pgfqpoint{9.019576in}{1.642237in}}{\pgfqpoint{9.014533in}{1.642237in}}%
\pgfpathcurveto{\pgfqpoint{9.009489in}{1.642237in}}{\pgfqpoint{9.004651in}{1.640233in}}{\pgfqpoint{9.001085in}{1.636667in}}%
\pgfpathcurveto{\pgfqpoint{8.997518in}{1.633100in}}{\pgfqpoint{8.995515in}{1.628263in}}{\pgfqpoint{8.995515in}{1.623219in}}%
\pgfpathcurveto{\pgfqpoint{8.995515in}{1.618175in}}{\pgfqpoint{8.997518in}{1.613337in}}{\pgfqpoint{9.001085in}{1.609771in}}%
\pgfpathcurveto{\pgfqpoint{9.004651in}{1.606205in}}{\pgfqpoint{9.009489in}{1.604201in}}{\pgfqpoint{9.014533in}{1.604201in}}%
\pgfpathclose%
\pgfusepath{fill}%
\end{pgfscope}%
\begin{pgfscope}%
\pgfpathrectangle{\pgfqpoint{6.572727in}{0.473000in}}{\pgfqpoint{4.227273in}{3.311000in}}%
\pgfusepath{clip}%
\pgfsetbuttcap%
\pgfsetroundjoin%
\definecolor{currentfill}{rgb}{0.127568,0.566949,0.550556}%
\pgfsetfillcolor{currentfill}%
\pgfsetfillopacity{0.700000}%
\pgfsetlinewidth{0.000000pt}%
\definecolor{currentstroke}{rgb}{0.000000,0.000000,0.000000}%
\pgfsetstrokecolor{currentstroke}%
\pgfsetstrokeopacity{0.700000}%
\pgfsetdash{}{0pt}%
\pgfpathmoveto{\pgfqpoint{7.703718in}{1.265442in}}%
\pgfpathcurveto{\pgfqpoint{7.708762in}{1.265442in}}{\pgfqpoint{7.713600in}{1.267445in}}{\pgfqpoint{7.717166in}{1.271012in}}%
\pgfpathcurveto{\pgfqpoint{7.720733in}{1.274578in}}{\pgfqpoint{7.722737in}{1.279416in}}{\pgfqpoint{7.722737in}{1.284460in}}%
\pgfpathcurveto{\pgfqpoint{7.722737in}{1.289503in}}{\pgfqpoint{7.720733in}{1.294341in}}{\pgfqpoint{7.717166in}{1.297908in}}%
\pgfpathcurveto{\pgfqpoint{7.713600in}{1.301474in}}{\pgfqpoint{7.708762in}{1.303478in}}{\pgfqpoint{7.703718in}{1.303478in}}%
\pgfpathcurveto{\pgfqpoint{7.698675in}{1.303478in}}{\pgfqpoint{7.693837in}{1.301474in}}{\pgfqpoint{7.690271in}{1.297908in}}%
\pgfpathcurveto{\pgfqpoint{7.686704in}{1.294341in}}{\pgfqpoint{7.684700in}{1.289503in}}{\pgfqpoint{7.684700in}{1.284460in}}%
\pgfpathcurveto{\pgfqpoint{7.684700in}{1.279416in}}{\pgfqpoint{7.686704in}{1.274578in}}{\pgfqpoint{7.690271in}{1.271012in}}%
\pgfpathcurveto{\pgfqpoint{7.693837in}{1.267445in}}{\pgfqpoint{7.698675in}{1.265442in}}{\pgfqpoint{7.703718in}{1.265442in}}%
\pgfpathclose%
\pgfusepath{fill}%
\end{pgfscope}%
\begin{pgfscope}%
\pgfpathrectangle{\pgfqpoint{6.572727in}{0.473000in}}{\pgfqpoint{4.227273in}{3.311000in}}%
\pgfusepath{clip}%
\pgfsetbuttcap%
\pgfsetroundjoin%
\definecolor{currentfill}{rgb}{0.127568,0.566949,0.550556}%
\pgfsetfillcolor{currentfill}%
\pgfsetfillopacity{0.700000}%
\pgfsetlinewidth{0.000000pt}%
\definecolor{currentstroke}{rgb}{0.000000,0.000000,0.000000}%
\pgfsetstrokecolor{currentstroke}%
\pgfsetstrokeopacity{0.700000}%
\pgfsetdash{}{0pt}%
\pgfpathmoveto{\pgfqpoint{8.528957in}{2.411308in}}%
\pgfpathcurveto{\pgfqpoint{8.534000in}{2.411308in}}{\pgfqpoint{8.538838in}{2.413312in}}{\pgfqpoint{8.542405in}{2.416878in}}%
\pgfpathcurveto{\pgfqpoint{8.545971in}{2.420445in}}{\pgfqpoint{8.547975in}{2.425282in}}{\pgfqpoint{8.547975in}{2.430326in}}%
\pgfpathcurveto{\pgfqpoint{8.547975in}{2.435370in}}{\pgfqpoint{8.545971in}{2.440207in}}{\pgfqpoint{8.542405in}{2.443774in}}%
\pgfpathcurveto{\pgfqpoint{8.538838in}{2.447340in}}{\pgfqpoint{8.534000in}{2.449344in}}{\pgfqpoint{8.528957in}{2.449344in}}%
\pgfpathcurveto{\pgfqpoint{8.523913in}{2.449344in}}{\pgfqpoint{8.519075in}{2.447340in}}{\pgfqpoint{8.515509in}{2.443774in}}%
\pgfpathcurveto{\pgfqpoint{8.511943in}{2.440207in}}{\pgfqpoint{8.509939in}{2.435370in}}{\pgfqpoint{8.509939in}{2.430326in}}%
\pgfpathcurveto{\pgfqpoint{8.509939in}{2.425282in}}{\pgfqpoint{8.511943in}{2.420445in}}{\pgfqpoint{8.515509in}{2.416878in}}%
\pgfpathcurveto{\pgfqpoint{8.519075in}{2.413312in}}{\pgfqpoint{8.523913in}{2.411308in}}{\pgfqpoint{8.528957in}{2.411308in}}%
\pgfpathclose%
\pgfusepath{fill}%
\end{pgfscope}%
\begin{pgfscope}%
\pgfpathrectangle{\pgfqpoint{6.572727in}{0.473000in}}{\pgfqpoint{4.227273in}{3.311000in}}%
\pgfusepath{clip}%
\pgfsetbuttcap%
\pgfsetroundjoin%
\definecolor{currentfill}{rgb}{0.993248,0.906157,0.143936}%
\pgfsetfillcolor{currentfill}%
\pgfsetfillopacity{0.700000}%
\pgfsetlinewidth{0.000000pt}%
\definecolor{currentstroke}{rgb}{0.000000,0.000000,0.000000}%
\pgfsetstrokecolor{currentstroke}%
\pgfsetstrokeopacity{0.700000}%
\pgfsetdash{}{0pt}%
\pgfpathmoveto{\pgfqpoint{10.242300in}{1.001435in}}%
\pgfpathcurveto{\pgfqpoint{10.247344in}{1.001435in}}{\pgfqpoint{10.252181in}{1.003439in}}{\pgfqpoint{10.255748in}{1.007005in}}%
\pgfpathcurveto{\pgfqpoint{10.259314in}{1.010572in}}{\pgfqpoint{10.261318in}{1.015409in}}{\pgfqpoint{10.261318in}{1.020453in}}%
\pgfpathcurveto{\pgfqpoint{10.261318in}{1.025497in}}{\pgfqpoint{10.259314in}{1.030335in}}{\pgfqpoint{10.255748in}{1.033901in}}%
\pgfpathcurveto{\pgfqpoint{10.252181in}{1.037467in}}{\pgfqpoint{10.247344in}{1.039471in}}{\pgfqpoint{10.242300in}{1.039471in}}%
\pgfpathcurveto{\pgfqpoint{10.237256in}{1.039471in}}{\pgfqpoint{10.232419in}{1.037467in}}{\pgfqpoint{10.228852in}{1.033901in}}%
\pgfpathcurveto{\pgfqpoint{10.225286in}{1.030335in}}{\pgfqpoint{10.223282in}{1.025497in}}{\pgfqpoint{10.223282in}{1.020453in}}%
\pgfpathcurveto{\pgfqpoint{10.223282in}{1.015409in}}{\pgfqpoint{10.225286in}{1.010572in}}{\pgfqpoint{10.228852in}{1.007005in}}%
\pgfpathcurveto{\pgfqpoint{10.232419in}{1.003439in}}{\pgfqpoint{10.237256in}{1.001435in}}{\pgfqpoint{10.242300in}{1.001435in}}%
\pgfpathclose%
\pgfusepath{fill}%
\end{pgfscope}%
\begin{pgfscope}%
\pgfpathrectangle{\pgfqpoint{6.572727in}{0.473000in}}{\pgfqpoint{4.227273in}{3.311000in}}%
\pgfusepath{clip}%
\pgfsetbuttcap%
\pgfsetroundjoin%
\definecolor{currentfill}{rgb}{0.127568,0.566949,0.550556}%
\pgfsetfillcolor{currentfill}%
\pgfsetfillopacity{0.700000}%
\pgfsetlinewidth{0.000000pt}%
\definecolor{currentstroke}{rgb}{0.000000,0.000000,0.000000}%
\pgfsetstrokecolor{currentstroke}%
\pgfsetstrokeopacity{0.700000}%
\pgfsetdash{}{0pt}%
\pgfpathmoveto{\pgfqpoint{8.063184in}{1.372428in}}%
\pgfpathcurveto{\pgfqpoint{8.068228in}{1.372428in}}{\pgfqpoint{8.073065in}{1.374432in}}{\pgfqpoint{8.076632in}{1.377999in}}%
\pgfpathcurveto{\pgfqpoint{8.080198in}{1.381565in}}{\pgfqpoint{8.082202in}{1.386403in}}{\pgfqpoint{8.082202in}{1.391446in}}%
\pgfpathcurveto{\pgfqpoint{8.082202in}{1.396490in}}{\pgfqpoint{8.080198in}{1.401328in}}{\pgfqpoint{8.076632in}{1.404894in}}%
\pgfpathcurveto{\pgfqpoint{8.073065in}{1.408461in}}{\pgfqpoint{8.068228in}{1.410465in}}{\pgfqpoint{8.063184in}{1.410465in}}%
\pgfpathcurveto{\pgfqpoint{8.058140in}{1.410465in}}{\pgfqpoint{8.053303in}{1.408461in}}{\pgfqpoint{8.049736in}{1.404894in}}%
\pgfpathcurveto{\pgfqpoint{8.046170in}{1.401328in}}{\pgfqpoint{8.044166in}{1.396490in}}{\pgfqpoint{8.044166in}{1.391446in}}%
\pgfpathcurveto{\pgfqpoint{8.044166in}{1.386403in}}{\pgfqpoint{8.046170in}{1.381565in}}{\pgfqpoint{8.049736in}{1.377999in}}%
\pgfpathcurveto{\pgfqpoint{8.053303in}{1.374432in}}{\pgfqpoint{8.058140in}{1.372428in}}{\pgfqpoint{8.063184in}{1.372428in}}%
\pgfpathclose%
\pgfusepath{fill}%
\end{pgfscope}%
\begin{pgfscope}%
\pgfpathrectangle{\pgfqpoint{6.572727in}{0.473000in}}{\pgfqpoint{4.227273in}{3.311000in}}%
\pgfusepath{clip}%
\pgfsetbuttcap%
\pgfsetroundjoin%
\definecolor{currentfill}{rgb}{0.993248,0.906157,0.143936}%
\pgfsetfillcolor{currentfill}%
\pgfsetfillopacity{0.700000}%
\pgfsetlinewidth{0.000000pt}%
\definecolor{currentstroke}{rgb}{0.000000,0.000000,0.000000}%
\pgfsetstrokecolor{currentstroke}%
\pgfsetstrokeopacity{0.700000}%
\pgfsetdash{}{0pt}%
\pgfpathmoveto{\pgfqpoint{9.215122in}{1.369623in}}%
\pgfpathcurveto{\pgfqpoint{9.220166in}{1.369623in}}{\pgfqpoint{9.225004in}{1.371627in}}{\pgfqpoint{9.228570in}{1.375193in}}%
\pgfpathcurveto{\pgfqpoint{9.232137in}{1.378760in}}{\pgfqpoint{9.234140in}{1.383597in}}{\pgfqpoint{9.234140in}{1.388641in}}%
\pgfpathcurveto{\pgfqpoint{9.234140in}{1.393685in}}{\pgfqpoint{9.232137in}{1.398522in}}{\pgfqpoint{9.228570in}{1.402089in}}%
\pgfpathcurveto{\pgfqpoint{9.225004in}{1.405655in}}{\pgfqpoint{9.220166in}{1.407659in}}{\pgfqpoint{9.215122in}{1.407659in}}%
\pgfpathcurveto{\pgfqpoint{9.210079in}{1.407659in}}{\pgfqpoint{9.205241in}{1.405655in}}{\pgfqpoint{9.201674in}{1.402089in}}%
\pgfpathcurveto{\pgfqpoint{9.198108in}{1.398522in}}{\pgfqpoint{9.196104in}{1.393685in}}{\pgfqpoint{9.196104in}{1.388641in}}%
\pgfpathcurveto{\pgfqpoint{9.196104in}{1.383597in}}{\pgfqpoint{9.198108in}{1.378760in}}{\pgfqpoint{9.201674in}{1.375193in}}%
\pgfpathcurveto{\pgfqpoint{9.205241in}{1.371627in}}{\pgfqpoint{9.210079in}{1.369623in}}{\pgfqpoint{9.215122in}{1.369623in}}%
\pgfpathclose%
\pgfusepath{fill}%
\end{pgfscope}%
\begin{pgfscope}%
\pgfpathrectangle{\pgfqpoint{6.572727in}{0.473000in}}{\pgfqpoint{4.227273in}{3.311000in}}%
\pgfusepath{clip}%
\pgfsetbuttcap%
\pgfsetroundjoin%
\definecolor{currentfill}{rgb}{0.993248,0.906157,0.143936}%
\pgfsetfillcolor{currentfill}%
\pgfsetfillopacity{0.700000}%
\pgfsetlinewidth{0.000000pt}%
\definecolor{currentstroke}{rgb}{0.000000,0.000000,0.000000}%
\pgfsetstrokecolor{currentstroke}%
\pgfsetstrokeopacity{0.700000}%
\pgfsetdash{}{0pt}%
\pgfpathmoveto{\pgfqpoint{10.168488in}{0.842239in}}%
\pgfpathcurveto{\pgfqpoint{10.173531in}{0.842239in}}{\pgfqpoint{10.178369in}{0.844243in}}{\pgfqpoint{10.181935in}{0.847809in}}%
\pgfpathcurveto{\pgfqpoint{10.185502in}{0.851375in}}{\pgfqpoint{10.187506in}{0.856213in}}{\pgfqpoint{10.187506in}{0.861257in}}%
\pgfpathcurveto{\pgfqpoint{10.187506in}{0.866300in}}{\pgfqpoint{10.185502in}{0.871138in}}{\pgfqpoint{10.181935in}{0.874705in}}%
\pgfpathcurveto{\pgfqpoint{10.178369in}{0.878271in}}{\pgfqpoint{10.173531in}{0.880275in}}{\pgfqpoint{10.168488in}{0.880275in}}%
\pgfpathcurveto{\pgfqpoint{10.163444in}{0.880275in}}{\pgfqpoint{10.158606in}{0.878271in}}{\pgfqpoint{10.155040in}{0.874705in}}%
\pgfpathcurveto{\pgfqpoint{10.151473in}{0.871138in}}{\pgfqpoint{10.149469in}{0.866300in}}{\pgfqpoint{10.149469in}{0.861257in}}%
\pgfpathcurveto{\pgfqpoint{10.149469in}{0.856213in}}{\pgfqpoint{10.151473in}{0.851375in}}{\pgfqpoint{10.155040in}{0.847809in}}%
\pgfpathcurveto{\pgfqpoint{10.158606in}{0.844243in}}{\pgfqpoint{10.163444in}{0.842239in}}{\pgfqpoint{10.168488in}{0.842239in}}%
\pgfpathclose%
\pgfusepath{fill}%
\end{pgfscope}%
\begin{pgfscope}%
\pgfpathrectangle{\pgfqpoint{6.572727in}{0.473000in}}{\pgfqpoint{4.227273in}{3.311000in}}%
\pgfusepath{clip}%
\pgfsetbuttcap%
\pgfsetroundjoin%
\definecolor{currentfill}{rgb}{0.127568,0.566949,0.550556}%
\pgfsetfillcolor{currentfill}%
\pgfsetfillopacity{0.700000}%
\pgfsetlinewidth{0.000000pt}%
\definecolor{currentstroke}{rgb}{0.000000,0.000000,0.000000}%
\pgfsetstrokecolor{currentstroke}%
\pgfsetstrokeopacity{0.700000}%
\pgfsetdash{}{0pt}%
\pgfpathmoveto{\pgfqpoint{8.381077in}{2.854909in}}%
\pgfpathcurveto{\pgfqpoint{8.386120in}{2.854909in}}{\pgfqpoint{8.390958in}{2.856913in}}{\pgfqpoint{8.394524in}{2.860480in}}%
\pgfpathcurveto{\pgfqpoint{8.398091in}{2.864046in}}{\pgfqpoint{8.400095in}{2.868884in}}{\pgfqpoint{8.400095in}{2.873928in}}%
\pgfpathcurveto{\pgfqpoint{8.400095in}{2.878971in}}{\pgfqpoint{8.398091in}{2.883809in}}{\pgfqpoint{8.394524in}{2.887375in}}%
\pgfpathcurveto{\pgfqpoint{8.390958in}{2.890942in}}{\pgfqpoint{8.386120in}{2.892946in}}{\pgfqpoint{8.381077in}{2.892946in}}%
\pgfpathcurveto{\pgfqpoint{8.376033in}{2.892946in}}{\pgfqpoint{8.371195in}{2.890942in}}{\pgfqpoint{8.367629in}{2.887375in}}%
\pgfpathcurveto{\pgfqpoint{8.364062in}{2.883809in}}{\pgfqpoint{8.362058in}{2.878971in}}{\pgfqpoint{8.362058in}{2.873928in}}%
\pgfpathcurveto{\pgfqpoint{8.362058in}{2.868884in}}{\pgfqpoint{8.364062in}{2.864046in}}{\pgfqpoint{8.367629in}{2.860480in}}%
\pgfpathcurveto{\pgfqpoint{8.371195in}{2.856913in}}{\pgfqpoint{8.376033in}{2.854909in}}{\pgfqpoint{8.381077in}{2.854909in}}%
\pgfpathclose%
\pgfusepath{fill}%
\end{pgfscope}%
\begin{pgfscope}%
\pgfpathrectangle{\pgfqpoint{6.572727in}{0.473000in}}{\pgfqpoint{4.227273in}{3.311000in}}%
\pgfusepath{clip}%
\pgfsetbuttcap%
\pgfsetroundjoin%
\definecolor{currentfill}{rgb}{0.127568,0.566949,0.550556}%
\pgfsetfillcolor{currentfill}%
\pgfsetfillopacity{0.700000}%
\pgfsetlinewidth{0.000000pt}%
\definecolor{currentstroke}{rgb}{0.000000,0.000000,0.000000}%
\pgfsetstrokecolor{currentstroke}%
\pgfsetstrokeopacity{0.700000}%
\pgfsetdash{}{0pt}%
\pgfpathmoveto{\pgfqpoint{8.544121in}{2.827096in}}%
\pgfpathcurveto{\pgfqpoint{8.549165in}{2.827096in}}{\pgfqpoint{8.554003in}{2.829100in}}{\pgfqpoint{8.557569in}{2.832666in}}%
\pgfpathcurveto{\pgfqpoint{8.561135in}{2.836232in}}{\pgfqpoint{8.563139in}{2.841070in}}{\pgfqpoint{8.563139in}{2.846114in}}%
\pgfpathcurveto{\pgfqpoint{8.563139in}{2.851158in}}{\pgfqpoint{8.561135in}{2.855995in}}{\pgfqpoint{8.557569in}{2.859562in}}%
\pgfpathcurveto{\pgfqpoint{8.554003in}{2.863128in}}{\pgfqpoint{8.549165in}{2.865132in}}{\pgfqpoint{8.544121in}{2.865132in}}%
\pgfpathcurveto{\pgfqpoint{8.539077in}{2.865132in}}{\pgfqpoint{8.534240in}{2.863128in}}{\pgfqpoint{8.530673in}{2.859562in}}%
\pgfpathcurveto{\pgfqpoint{8.527107in}{2.855995in}}{\pgfqpoint{8.525103in}{2.851158in}}{\pgfqpoint{8.525103in}{2.846114in}}%
\pgfpathcurveto{\pgfqpoint{8.525103in}{2.841070in}}{\pgfqpoint{8.527107in}{2.836232in}}{\pgfqpoint{8.530673in}{2.832666in}}%
\pgfpathcurveto{\pgfqpoint{8.534240in}{2.829100in}}{\pgfqpoint{8.539077in}{2.827096in}}{\pgfqpoint{8.544121in}{2.827096in}}%
\pgfpathclose%
\pgfusepath{fill}%
\end{pgfscope}%
\begin{pgfscope}%
\pgfpathrectangle{\pgfqpoint{6.572727in}{0.473000in}}{\pgfqpoint{4.227273in}{3.311000in}}%
\pgfusepath{clip}%
\pgfsetbuttcap%
\pgfsetroundjoin%
\definecolor{currentfill}{rgb}{0.127568,0.566949,0.550556}%
\pgfsetfillcolor{currentfill}%
\pgfsetfillopacity{0.700000}%
\pgfsetlinewidth{0.000000pt}%
\definecolor{currentstroke}{rgb}{0.000000,0.000000,0.000000}%
\pgfsetstrokecolor{currentstroke}%
\pgfsetstrokeopacity{0.700000}%
\pgfsetdash{}{0pt}%
\pgfpathmoveto{\pgfqpoint{8.456689in}{2.737694in}}%
\pgfpathcurveto{\pgfqpoint{8.461733in}{2.737694in}}{\pgfqpoint{8.466571in}{2.739698in}}{\pgfqpoint{8.470137in}{2.743265in}}%
\pgfpathcurveto{\pgfqpoint{8.473704in}{2.746831in}}{\pgfqpoint{8.475708in}{2.751669in}}{\pgfqpoint{8.475708in}{2.756712in}}%
\pgfpathcurveto{\pgfqpoint{8.475708in}{2.761756in}}{\pgfqpoint{8.473704in}{2.766594in}}{\pgfqpoint{8.470137in}{2.770160in}}%
\pgfpathcurveto{\pgfqpoint{8.466571in}{2.773727in}}{\pgfqpoint{8.461733in}{2.775731in}}{\pgfqpoint{8.456689in}{2.775731in}}%
\pgfpathcurveto{\pgfqpoint{8.451646in}{2.775731in}}{\pgfqpoint{8.446808in}{2.773727in}}{\pgfqpoint{8.443242in}{2.770160in}}%
\pgfpathcurveto{\pgfqpoint{8.439675in}{2.766594in}}{\pgfqpoint{8.437671in}{2.761756in}}{\pgfqpoint{8.437671in}{2.756712in}}%
\pgfpathcurveto{\pgfqpoint{8.437671in}{2.751669in}}{\pgfqpoint{8.439675in}{2.746831in}}{\pgfqpoint{8.443242in}{2.743265in}}%
\pgfpathcurveto{\pgfqpoint{8.446808in}{2.739698in}}{\pgfqpoint{8.451646in}{2.737694in}}{\pgfqpoint{8.456689in}{2.737694in}}%
\pgfpathclose%
\pgfusepath{fill}%
\end{pgfscope}%
\begin{pgfscope}%
\pgfpathrectangle{\pgfqpoint{6.572727in}{0.473000in}}{\pgfqpoint{4.227273in}{3.311000in}}%
\pgfusepath{clip}%
\pgfsetbuttcap%
\pgfsetroundjoin%
\definecolor{currentfill}{rgb}{0.127568,0.566949,0.550556}%
\pgfsetfillcolor{currentfill}%
\pgfsetfillopacity{0.700000}%
\pgfsetlinewidth{0.000000pt}%
\definecolor{currentstroke}{rgb}{0.000000,0.000000,0.000000}%
\pgfsetstrokecolor{currentstroke}%
\pgfsetstrokeopacity{0.700000}%
\pgfsetdash{}{0pt}%
\pgfpathmoveto{\pgfqpoint{8.560071in}{3.189144in}}%
\pgfpathcurveto{\pgfqpoint{8.565115in}{3.189144in}}{\pgfqpoint{8.569952in}{3.191148in}}{\pgfqpoint{8.573519in}{3.194714in}}%
\pgfpathcurveto{\pgfqpoint{8.577085in}{3.198281in}}{\pgfqpoint{8.579089in}{3.203118in}}{\pgfqpoint{8.579089in}{3.208162in}}%
\pgfpathcurveto{\pgfqpoint{8.579089in}{3.213206in}}{\pgfqpoint{8.577085in}{3.218044in}}{\pgfqpoint{8.573519in}{3.221610in}}%
\pgfpathcurveto{\pgfqpoint{8.569952in}{3.225176in}}{\pgfqpoint{8.565115in}{3.227180in}}{\pgfqpoint{8.560071in}{3.227180in}}%
\pgfpathcurveto{\pgfqpoint{8.555027in}{3.227180in}}{\pgfqpoint{8.550189in}{3.225176in}}{\pgfqpoint{8.546623in}{3.221610in}}%
\pgfpathcurveto{\pgfqpoint{8.543057in}{3.218044in}}{\pgfqpoint{8.541053in}{3.213206in}}{\pgfqpoint{8.541053in}{3.208162in}}%
\pgfpathcurveto{\pgfqpoint{8.541053in}{3.203118in}}{\pgfqpoint{8.543057in}{3.198281in}}{\pgfqpoint{8.546623in}{3.194714in}}%
\pgfpathcurveto{\pgfqpoint{8.550189in}{3.191148in}}{\pgfqpoint{8.555027in}{3.189144in}}{\pgfqpoint{8.560071in}{3.189144in}}%
\pgfpathclose%
\pgfusepath{fill}%
\end{pgfscope}%
\begin{pgfscope}%
\pgfpathrectangle{\pgfqpoint{6.572727in}{0.473000in}}{\pgfqpoint{4.227273in}{3.311000in}}%
\pgfusepath{clip}%
\pgfsetbuttcap%
\pgfsetroundjoin%
\definecolor{currentfill}{rgb}{0.127568,0.566949,0.550556}%
\pgfsetfillcolor{currentfill}%
\pgfsetfillopacity{0.700000}%
\pgfsetlinewidth{0.000000pt}%
\definecolor{currentstroke}{rgb}{0.000000,0.000000,0.000000}%
\pgfsetstrokecolor{currentstroke}%
\pgfsetstrokeopacity{0.700000}%
\pgfsetdash{}{0pt}%
\pgfpathmoveto{\pgfqpoint{7.861425in}{1.730461in}}%
\pgfpathcurveto{\pgfqpoint{7.866468in}{1.730461in}}{\pgfqpoint{7.871306in}{1.732465in}}{\pgfqpoint{7.874873in}{1.736032in}}%
\pgfpathcurveto{\pgfqpoint{7.878439in}{1.739598in}}{\pgfqpoint{7.880443in}{1.744436in}}{\pgfqpoint{7.880443in}{1.749480in}}%
\pgfpathcurveto{\pgfqpoint{7.880443in}{1.754523in}}{\pgfqpoint{7.878439in}{1.759361in}}{\pgfqpoint{7.874873in}{1.762927in}}%
\pgfpathcurveto{\pgfqpoint{7.871306in}{1.766494in}}{\pgfqpoint{7.866468in}{1.768498in}}{\pgfqpoint{7.861425in}{1.768498in}}%
\pgfpathcurveto{\pgfqpoint{7.856381in}{1.768498in}}{\pgfqpoint{7.851543in}{1.766494in}}{\pgfqpoint{7.847977in}{1.762927in}}%
\pgfpathcurveto{\pgfqpoint{7.844410in}{1.759361in}}{\pgfqpoint{7.842407in}{1.754523in}}{\pgfqpoint{7.842407in}{1.749480in}}%
\pgfpathcurveto{\pgfqpoint{7.842407in}{1.744436in}}{\pgfqpoint{7.844410in}{1.739598in}}{\pgfqpoint{7.847977in}{1.736032in}}%
\pgfpathcurveto{\pgfqpoint{7.851543in}{1.732465in}}{\pgfqpoint{7.856381in}{1.730461in}}{\pgfqpoint{7.861425in}{1.730461in}}%
\pgfpathclose%
\pgfusepath{fill}%
\end{pgfscope}%
\begin{pgfscope}%
\pgfpathrectangle{\pgfqpoint{6.572727in}{0.473000in}}{\pgfqpoint{4.227273in}{3.311000in}}%
\pgfusepath{clip}%
\pgfsetbuttcap%
\pgfsetroundjoin%
\definecolor{currentfill}{rgb}{0.993248,0.906157,0.143936}%
\pgfsetfillcolor{currentfill}%
\pgfsetfillopacity{0.700000}%
\pgfsetlinewidth{0.000000pt}%
\definecolor{currentstroke}{rgb}{0.000000,0.000000,0.000000}%
\pgfsetstrokecolor{currentstroke}%
\pgfsetstrokeopacity{0.700000}%
\pgfsetdash{}{0pt}%
\pgfpathmoveto{\pgfqpoint{8.986997in}{1.749765in}}%
\pgfpathcurveto{\pgfqpoint{8.992041in}{1.749765in}}{\pgfqpoint{8.996879in}{1.751769in}}{\pgfqpoint{9.000445in}{1.755335in}}%
\pgfpathcurveto{\pgfqpoint{9.004012in}{1.758902in}}{\pgfqpoint{9.006016in}{1.763740in}}{\pgfqpoint{9.006016in}{1.768783in}}%
\pgfpathcurveto{\pgfqpoint{9.006016in}{1.773827in}}{\pgfqpoint{9.004012in}{1.778665in}}{\pgfqpoint{9.000445in}{1.782231in}}%
\pgfpathcurveto{\pgfqpoint{8.996879in}{1.785798in}}{\pgfqpoint{8.992041in}{1.787801in}}{\pgfqpoint{8.986997in}{1.787801in}}%
\pgfpathcurveto{\pgfqpoint{8.981954in}{1.787801in}}{\pgfqpoint{8.977116in}{1.785798in}}{\pgfqpoint{8.973550in}{1.782231in}}%
\pgfpathcurveto{\pgfqpoint{8.969983in}{1.778665in}}{\pgfqpoint{8.967979in}{1.773827in}}{\pgfqpoint{8.967979in}{1.768783in}}%
\pgfpathcurveto{\pgfqpoint{8.967979in}{1.763740in}}{\pgfqpoint{8.969983in}{1.758902in}}{\pgfqpoint{8.973550in}{1.755335in}}%
\pgfpathcurveto{\pgfqpoint{8.977116in}{1.751769in}}{\pgfqpoint{8.981954in}{1.749765in}}{\pgfqpoint{8.986997in}{1.749765in}}%
\pgfpathclose%
\pgfusepath{fill}%
\end{pgfscope}%
\begin{pgfscope}%
\pgfpathrectangle{\pgfqpoint{6.572727in}{0.473000in}}{\pgfqpoint{4.227273in}{3.311000in}}%
\pgfusepath{clip}%
\pgfsetbuttcap%
\pgfsetroundjoin%
\definecolor{currentfill}{rgb}{0.127568,0.566949,0.550556}%
\pgfsetfillcolor{currentfill}%
\pgfsetfillopacity{0.700000}%
\pgfsetlinewidth{0.000000pt}%
\definecolor{currentstroke}{rgb}{0.000000,0.000000,0.000000}%
\pgfsetstrokecolor{currentstroke}%
\pgfsetstrokeopacity{0.700000}%
\pgfsetdash{}{0pt}%
\pgfpathmoveto{\pgfqpoint{7.945923in}{2.316781in}}%
\pgfpathcurveto{\pgfqpoint{7.950966in}{2.316781in}}{\pgfqpoint{7.955804in}{2.318785in}}{\pgfqpoint{7.959370in}{2.322352in}}%
\pgfpathcurveto{\pgfqpoint{7.962937in}{2.325918in}}{\pgfqpoint{7.964941in}{2.330756in}}{\pgfqpoint{7.964941in}{2.335799in}}%
\pgfpathcurveto{\pgfqpoint{7.964941in}{2.340843in}}{\pgfqpoint{7.962937in}{2.345681in}}{\pgfqpoint{7.959370in}{2.349247in}}%
\pgfpathcurveto{\pgfqpoint{7.955804in}{2.352814in}}{\pgfqpoint{7.950966in}{2.354818in}}{\pgfqpoint{7.945923in}{2.354818in}}%
\pgfpathcurveto{\pgfqpoint{7.940879in}{2.354818in}}{\pgfqpoint{7.936041in}{2.352814in}}{\pgfqpoint{7.932475in}{2.349247in}}%
\pgfpathcurveto{\pgfqpoint{7.928908in}{2.345681in}}{\pgfqpoint{7.926904in}{2.340843in}}{\pgfqpoint{7.926904in}{2.335799in}}%
\pgfpathcurveto{\pgfqpoint{7.926904in}{2.330756in}}{\pgfqpoint{7.928908in}{2.325918in}}{\pgfqpoint{7.932475in}{2.322352in}}%
\pgfpathcurveto{\pgfqpoint{7.936041in}{2.318785in}}{\pgfqpoint{7.940879in}{2.316781in}}{\pgfqpoint{7.945923in}{2.316781in}}%
\pgfpathclose%
\pgfusepath{fill}%
\end{pgfscope}%
\begin{pgfscope}%
\pgfpathrectangle{\pgfqpoint{6.572727in}{0.473000in}}{\pgfqpoint{4.227273in}{3.311000in}}%
\pgfusepath{clip}%
\pgfsetbuttcap%
\pgfsetroundjoin%
\definecolor{currentfill}{rgb}{0.127568,0.566949,0.550556}%
\pgfsetfillcolor{currentfill}%
\pgfsetfillopacity{0.700000}%
\pgfsetlinewidth{0.000000pt}%
\definecolor{currentstroke}{rgb}{0.000000,0.000000,0.000000}%
\pgfsetstrokecolor{currentstroke}%
\pgfsetstrokeopacity{0.700000}%
\pgfsetdash{}{0pt}%
\pgfpathmoveto{\pgfqpoint{7.784445in}{3.067496in}}%
\pgfpathcurveto{\pgfqpoint{7.789489in}{3.067496in}}{\pgfqpoint{7.794326in}{3.069499in}}{\pgfqpoint{7.797893in}{3.073066in}}%
\pgfpathcurveto{\pgfqpoint{7.801459in}{3.076632in}}{\pgfqpoint{7.803463in}{3.081470in}}{\pgfqpoint{7.803463in}{3.086514in}}%
\pgfpathcurveto{\pgfqpoint{7.803463in}{3.091557in}}{\pgfqpoint{7.801459in}{3.096395in}}{\pgfqpoint{7.797893in}{3.099962in}}%
\pgfpathcurveto{\pgfqpoint{7.794326in}{3.103528in}}{\pgfqpoint{7.789489in}{3.105532in}}{\pgfqpoint{7.784445in}{3.105532in}}%
\pgfpathcurveto{\pgfqpoint{7.779401in}{3.105532in}}{\pgfqpoint{7.774563in}{3.103528in}}{\pgfqpoint{7.770997in}{3.099962in}}%
\pgfpathcurveto{\pgfqpoint{7.767431in}{3.096395in}}{\pgfqpoint{7.765427in}{3.091557in}}{\pgfqpoint{7.765427in}{3.086514in}}%
\pgfpathcurveto{\pgfqpoint{7.765427in}{3.081470in}}{\pgfqpoint{7.767431in}{3.076632in}}{\pgfqpoint{7.770997in}{3.073066in}}%
\pgfpathcurveto{\pgfqpoint{7.774563in}{3.069499in}}{\pgfqpoint{7.779401in}{3.067496in}}{\pgfqpoint{7.784445in}{3.067496in}}%
\pgfpathclose%
\pgfusepath{fill}%
\end{pgfscope}%
\begin{pgfscope}%
\pgfpathrectangle{\pgfqpoint{6.572727in}{0.473000in}}{\pgfqpoint{4.227273in}{3.311000in}}%
\pgfusepath{clip}%
\pgfsetbuttcap%
\pgfsetroundjoin%
\definecolor{currentfill}{rgb}{0.127568,0.566949,0.550556}%
\pgfsetfillcolor{currentfill}%
\pgfsetfillopacity{0.700000}%
\pgfsetlinewidth{0.000000pt}%
\definecolor{currentstroke}{rgb}{0.000000,0.000000,0.000000}%
\pgfsetstrokecolor{currentstroke}%
\pgfsetstrokeopacity{0.700000}%
\pgfsetdash{}{0pt}%
\pgfpathmoveto{\pgfqpoint{7.712566in}{1.584311in}}%
\pgfpathcurveto{\pgfqpoint{7.717610in}{1.584311in}}{\pgfqpoint{7.722448in}{1.586315in}}{\pgfqpoint{7.726014in}{1.589881in}}%
\pgfpathcurveto{\pgfqpoint{7.729581in}{1.593447in}}{\pgfqpoint{7.731585in}{1.598285in}}{\pgfqpoint{7.731585in}{1.603329in}}%
\pgfpathcurveto{\pgfqpoint{7.731585in}{1.608372in}}{\pgfqpoint{7.729581in}{1.613210in}}{\pgfqpoint{7.726014in}{1.616777in}}%
\pgfpathcurveto{\pgfqpoint{7.722448in}{1.620343in}}{\pgfqpoint{7.717610in}{1.622347in}}{\pgfqpoint{7.712566in}{1.622347in}}%
\pgfpathcurveto{\pgfqpoint{7.707523in}{1.622347in}}{\pgfqpoint{7.702685in}{1.620343in}}{\pgfqpoint{7.699119in}{1.616777in}}%
\pgfpathcurveto{\pgfqpoint{7.695552in}{1.613210in}}{\pgfqpoint{7.693548in}{1.608372in}}{\pgfqpoint{7.693548in}{1.603329in}}%
\pgfpathcurveto{\pgfqpoint{7.693548in}{1.598285in}}{\pgfqpoint{7.695552in}{1.593447in}}{\pgfqpoint{7.699119in}{1.589881in}}%
\pgfpathcurveto{\pgfqpoint{7.702685in}{1.586315in}}{\pgfqpoint{7.707523in}{1.584311in}}{\pgfqpoint{7.712566in}{1.584311in}}%
\pgfpathclose%
\pgfusepath{fill}%
\end{pgfscope}%
\begin{pgfscope}%
\pgfpathrectangle{\pgfqpoint{6.572727in}{0.473000in}}{\pgfqpoint{4.227273in}{3.311000in}}%
\pgfusepath{clip}%
\pgfsetbuttcap%
\pgfsetroundjoin%
\definecolor{currentfill}{rgb}{0.127568,0.566949,0.550556}%
\pgfsetfillcolor{currentfill}%
\pgfsetfillopacity{0.700000}%
\pgfsetlinewidth{0.000000pt}%
\definecolor{currentstroke}{rgb}{0.000000,0.000000,0.000000}%
\pgfsetstrokecolor{currentstroke}%
\pgfsetstrokeopacity{0.700000}%
\pgfsetdash{}{0pt}%
\pgfpathmoveto{\pgfqpoint{8.301600in}{1.601791in}}%
\pgfpathcurveto{\pgfqpoint{8.306644in}{1.601791in}}{\pgfqpoint{8.311482in}{1.603795in}}{\pgfqpoint{8.315048in}{1.607361in}}%
\pgfpathcurveto{\pgfqpoint{8.318615in}{1.610927in}}{\pgfqpoint{8.320619in}{1.615765in}}{\pgfqpoint{8.320619in}{1.620809in}}%
\pgfpathcurveto{\pgfqpoint{8.320619in}{1.625853in}}{\pgfqpoint{8.318615in}{1.630690in}}{\pgfqpoint{8.315048in}{1.634257in}}%
\pgfpathcurveto{\pgfqpoint{8.311482in}{1.637823in}}{\pgfqpoint{8.306644in}{1.639827in}}{\pgfqpoint{8.301600in}{1.639827in}}%
\pgfpathcurveto{\pgfqpoint{8.296557in}{1.639827in}}{\pgfqpoint{8.291719in}{1.637823in}}{\pgfqpoint{8.288153in}{1.634257in}}%
\pgfpathcurveto{\pgfqpoint{8.284586in}{1.630690in}}{\pgfqpoint{8.282582in}{1.625853in}}{\pgfqpoint{8.282582in}{1.620809in}}%
\pgfpathcurveto{\pgfqpoint{8.282582in}{1.615765in}}{\pgfqpoint{8.284586in}{1.610927in}}{\pgfqpoint{8.288153in}{1.607361in}}%
\pgfpathcurveto{\pgfqpoint{8.291719in}{1.603795in}}{\pgfqpoint{8.296557in}{1.601791in}}{\pgfqpoint{8.301600in}{1.601791in}}%
\pgfpathclose%
\pgfusepath{fill}%
\end{pgfscope}%
\begin{pgfscope}%
\pgfpathrectangle{\pgfqpoint{6.572727in}{0.473000in}}{\pgfqpoint{4.227273in}{3.311000in}}%
\pgfusepath{clip}%
\pgfsetbuttcap%
\pgfsetroundjoin%
\definecolor{currentfill}{rgb}{0.127568,0.566949,0.550556}%
\pgfsetfillcolor{currentfill}%
\pgfsetfillopacity{0.700000}%
\pgfsetlinewidth{0.000000pt}%
\definecolor{currentstroke}{rgb}{0.000000,0.000000,0.000000}%
\pgfsetstrokecolor{currentstroke}%
\pgfsetstrokeopacity{0.700000}%
\pgfsetdash{}{0pt}%
\pgfpathmoveto{\pgfqpoint{8.559669in}{2.589578in}}%
\pgfpathcurveto{\pgfqpoint{8.564713in}{2.589578in}}{\pgfqpoint{8.569551in}{2.591581in}}{\pgfqpoint{8.573117in}{2.595148in}}%
\pgfpathcurveto{\pgfqpoint{8.576683in}{2.598714in}}{\pgfqpoint{8.578687in}{2.603552in}}{\pgfqpoint{8.578687in}{2.608596in}}%
\pgfpathcurveto{\pgfqpoint{8.578687in}{2.613639in}}{\pgfqpoint{8.576683in}{2.618477in}}{\pgfqpoint{8.573117in}{2.622044in}}%
\pgfpathcurveto{\pgfqpoint{8.569551in}{2.625610in}}{\pgfqpoint{8.564713in}{2.627614in}}{\pgfqpoint{8.559669in}{2.627614in}}%
\pgfpathcurveto{\pgfqpoint{8.554625in}{2.627614in}}{\pgfqpoint{8.549788in}{2.625610in}}{\pgfqpoint{8.546221in}{2.622044in}}%
\pgfpathcurveto{\pgfqpoint{8.542655in}{2.618477in}}{\pgfqpoint{8.540651in}{2.613639in}}{\pgfqpoint{8.540651in}{2.608596in}}%
\pgfpathcurveto{\pgfqpoint{8.540651in}{2.603552in}}{\pgfqpoint{8.542655in}{2.598714in}}{\pgfqpoint{8.546221in}{2.595148in}}%
\pgfpathcurveto{\pgfqpoint{8.549788in}{2.591581in}}{\pgfqpoint{8.554625in}{2.589578in}}{\pgfqpoint{8.559669in}{2.589578in}}%
\pgfpathclose%
\pgfusepath{fill}%
\end{pgfscope}%
\begin{pgfscope}%
\pgfpathrectangle{\pgfqpoint{6.572727in}{0.473000in}}{\pgfqpoint{4.227273in}{3.311000in}}%
\pgfusepath{clip}%
\pgfsetbuttcap%
\pgfsetroundjoin%
\definecolor{currentfill}{rgb}{0.127568,0.566949,0.550556}%
\pgfsetfillcolor{currentfill}%
\pgfsetfillopacity{0.700000}%
\pgfsetlinewidth{0.000000pt}%
\definecolor{currentstroke}{rgb}{0.000000,0.000000,0.000000}%
\pgfsetstrokecolor{currentstroke}%
\pgfsetstrokeopacity{0.700000}%
\pgfsetdash{}{0pt}%
\pgfpathmoveto{\pgfqpoint{7.904852in}{1.886390in}}%
\pgfpathcurveto{\pgfqpoint{7.909896in}{1.886390in}}{\pgfqpoint{7.914733in}{1.888393in}}{\pgfqpoint{7.918300in}{1.891960in}}%
\pgfpathcurveto{\pgfqpoint{7.921866in}{1.895526in}}{\pgfqpoint{7.923870in}{1.900364in}}{\pgfqpoint{7.923870in}{1.905408in}}%
\pgfpathcurveto{\pgfqpoint{7.923870in}{1.910451in}}{\pgfqpoint{7.921866in}{1.915289in}}{\pgfqpoint{7.918300in}{1.918856in}}%
\pgfpathcurveto{\pgfqpoint{7.914733in}{1.922422in}}{\pgfqpoint{7.909896in}{1.924426in}}{\pgfqpoint{7.904852in}{1.924426in}}%
\pgfpathcurveto{\pgfqpoint{7.899808in}{1.924426in}}{\pgfqpoint{7.894970in}{1.922422in}}{\pgfqpoint{7.891404in}{1.918856in}}%
\pgfpathcurveto{\pgfqpoint{7.887838in}{1.915289in}}{\pgfqpoint{7.885834in}{1.910451in}}{\pgfqpoint{7.885834in}{1.905408in}}%
\pgfpathcurveto{\pgfqpoint{7.885834in}{1.900364in}}{\pgfqpoint{7.887838in}{1.895526in}}{\pgfqpoint{7.891404in}{1.891960in}}%
\pgfpathcurveto{\pgfqpoint{7.894970in}{1.888393in}}{\pgfqpoint{7.899808in}{1.886390in}}{\pgfqpoint{7.904852in}{1.886390in}}%
\pgfpathclose%
\pgfusepath{fill}%
\end{pgfscope}%
\begin{pgfscope}%
\pgfpathrectangle{\pgfqpoint{6.572727in}{0.473000in}}{\pgfqpoint{4.227273in}{3.311000in}}%
\pgfusepath{clip}%
\pgfsetbuttcap%
\pgfsetroundjoin%
\definecolor{currentfill}{rgb}{0.993248,0.906157,0.143936}%
\pgfsetfillcolor{currentfill}%
\pgfsetfillopacity{0.700000}%
\pgfsetlinewidth{0.000000pt}%
\definecolor{currentstroke}{rgb}{0.000000,0.000000,0.000000}%
\pgfsetstrokecolor{currentstroke}%
\pgfsetstrokeopacity{0.700000}%
\pgfsetdash{}{0pt}%
\pgfpathmoveto{\pgfqpoint{10.120906in}{1.419595in}}%
\pgfpathcurveto{\pgfqpoint{10.125950in}{1.419595in}}{\pgfqpoint{10.130788in}{1.421599in}}{\pgfqpoint{10.134354in}{1.425166in}}%
\pgfpathcurveto{\pgfqpoint{10.137921in}{1.428732in}}{\pgfqpoint{10.139925in}{1.433570in}}{\pgfqpoint{10.139925in}{1.438614in}}%
\pgfpathcurveto{\pgfqpoint{10.139925in}{1.443657in}}{\pgfqpoint{10.137921in}{1.448495in}}{\pgfqpoint{10.134354in}{1.452061in}}%
\pgfpathcurveto{\pgfqpoint{10.130788in}{1.455628in}}{\pgfqpoint{10.125950in}{1.457632in}}{\pgfqpoint{10.120906in}{1.457632in}}%
\pgfpathcurveto{\pgfqpoint{10.115863in}{1.457632in}}{\pgfqpoint{10.111025in}{1.455628in}}{\pgfqpoint{10.107459in}{1.452061in}}%
\pgfpathcurveto{\pgfqpoint{10.103892in}{1.448495in}}{\pgfqpoint{10.101888in}{1.443657in}}{\pgfqpoint{10.101888in}{1.438614in}}%
\pgfpathcurveto{\pgfqpoint{10.101888in}{1.433570in}}{\pgfqpoint{10.103892in}{1.428732in}}{\pgfqpoint{10.107459in}{1.425166in}}%
\pgfpathcurveto{\pgfqpoint{10.111025in}{1.421599in}}{\pgfqpoint{10.115863in}{1.419595in}}{\pgfqpoint{10.120906in}{1.419595in}}%
\pgfpathclose%
\pgfusepath{fill}%
\end{pgfscope}%
\begin{pgfscope}%
\pgfpathrectangle{\pgfqpoint{6.572727in}{0.473000in}}{\pgfqpoint{4.227273in}{3.311000in}}%
\pgfusepath{clip}%
\pgfsetbuttcap%
\pgfsetroundjoin%
\definecolor{currentfill}{rgb}{0.993248,0.906157,0.143936}%
\pgfsetfillcolor{currentfill}%
\pgfsetfillopacity{0.700000}%
\pgfsetlinewidth{0.000000pt}%
\definecolor{currentstroke}{rgb}{0.000000,0.000000,0.000000}%
\pgfsetstrokecolor{currentstroke}%
\pgfsetstrokeopacity{0.700000}%
\pgfsetdash{}{0pt}%
\pgfpathmoveto{\pgfqpoint{9.162532in}{1.356641in}}%
\pgfpathcurveto{\pgfqpoint{9.167575in}{1.356641in}}{\pgfqpoint{9.172413in}{1.358645in}}{\pgfqpoint{9.175979in}{1.362212in}}%
\pgfpathcurveto{\pgfqpoint{9.179546in}{1.365778in}}{\pgfqpoint{9.181550in}{1.370616in}}{\pgfqpoint{9.181550in}{1.375660in}}%
\pgfpathcurveto{\pgfqpoint{9.181550in}{1.380703in}}{\pgfqpoint{9.179546in}{1.385541in}}{\pgfqpoint{9.175979in}{1.389107in}}%
\pgfpathcurveto{\pgfqpoint{9.172413in}{1.392674in}}{\pgfqpoint{9.167575in}{1.394678in}}{\pgfqpoint{9.162532in}{1.394678in}}%
\pgfpathcurveto{\pgfqpoint{9.157488in}{1.394678in}}{\pgfqpoint{9.152650in}{1.392674in}}{\pgfqpoint{9.149084in}{1.389107in}}%
\pgfpathcurveto{\pgfqpoint{9.145517in}{1.385541in}}{\pgfqpoint{9.143513in}{1.380703in}}{\pgfqpoint{9.143513in}{1.375660in}}%
\pgfpathcurveto{\pgfqpoint{9.143513in}{1.370616in}}{\pgfqpoint{9.145517in}{1.365778in}}{\pgfqpoint{9.149084in}{1.362212in}}%
\pgfpathcurveto{\pgfqpoint{9.152650in}{1.358645in}}{\pgfqpoint{9.157488in}{1.356641in}}{\pgfqpoint{9.162532in}{1.356641in}}%
\pgfpathclose%
\pgfusepath{fill}%
\end{pgfscope}%
\begin{pgfscope}%
\pgfpathrectangle{\pgfqpoint{6.572727in}{0.473000in}}{\pgfqpoint{4.227273in}{3.311000in}}%
\pgfusepath{clip}%
\pgfsetbuttcap%
\pgfsetroundjoin%
\definecolor{currentfill}{rgb}{0.127568,0.566949,0.550556}%
\pgfsetfillcolor{currentfill}%
\pgfsetfillopacity{0.700000}%
\pgfsetlinewidth{0.000000pt}%
\definecolor{currentstroke}{rgb}{0.000000,0.000000,0.000000}%
\pgfsetstrokecolor{currentstroke}%
\pgfsetstrokeopacity{0.700000}%
\pgfsetdash{}{0pt}%
\pgfpathmoveto{\pgfqpoint{8.793988in}{2.114857in}}%
\pgfpathcurveto{\pgfqpoint{8.799031in}{2.114857in}}{\pgfqpoint{8.803869in}{2.116860in}}{\pgfqpoint{8.807436in}{2.120427in}}%
\pgfpathcurveto{\pgfqpoint{8.811002in}{2.123993in}}{\pgfqpoint{8.813006in}{2.128831in}}{\pgfqpoint{8.813006in}{2.133875in}}%
\pgfpathcurveto{\pgfqpoint{8.813006in}{2.138918in}}{\pgfqpoint{8.811002in}{2.143756in}}{\pgfqpoint{8.807436in}{2.147323in}}%
\pgfpathcurveto{\pgfqpoint{8.803869in}{2.150889in}}{\pgfqpoint{8.799031in}{2.152893in}}{\pgfqpoint{8.793988in}{2.152893in}}%
\pgfpathcurveto{\pgfqpoint{8.788944in}{2.152893in}}{\pgfqpoint{8.784106in}{2.150889in}}{\pgfqpoint{8.780540in}{2.147323in}}%
\pgfpathcurveto{\pgfqpoint{8.776973in}{2.143756in}}{\pgfqpoint{8.774970in}{2.138918in}}{\pgfqpoint{8.774970in}{2.133875in}}%
\pgfpathcurveto{\pgfqpoint{8.774970in}{2.128831in}}{\pgfqpoint{8.776973in}{2.123993in}}{\pgfqpoint{8.780540in}{2.120427in}}%
\pgfpathcurveto{\pgfqpoint{8.784106in}{2.116860in}}{\pgfqpoint{8.788944in}{2.114857in}}{\pgfqpoint{8.793988in}{2.114857in}}%
\pgfpathclose%
\pgfusepath{fill}%
\end{pgfscope}%
\begin{pgfscope}%
\pgfpathrectangle{\pgfqpoint{6.572727in}{0.473000in}}{\pgfqpoint{4.227273in}{3.311000in}}%
\pgfusepath{clip}%
\pgfsetbuttcap%
\pgfsetroundjoin%
\definecolor{currentfill}{rgb}{0.993248,0.906157,0.143936}%
\pgfsetfillcolor{currentfill}%
\pgfsetfillopacity{0.700000}%
\pgfsetlinewidth{0.000000pt}%
\definecolor{currentstroke}{rgb}{0.000000,0.000000,0.000000}%
\pgfsetstrokecolor{currentstroke}%
\pgfsetstrokeopacity{0.700000}%
\pgfsetdash{}{0pt}%
\pgfpathmoveto{\pgfqpoint{9.683772in}{1.997470in}}%
\pgfpathcurveto{\pgfqpoint{9.688815in}{1.997470in}}{\pgfqpoint{9.693653in}{1.999474in}}{\pgfqpoint{9.697220in}{2.003040in}}%
\pgfpathcurveto{\pgfqpoint{9.700786in}{2.006607in}}{\pgfqpoint{9.702790in}{2.011444in}}{\pgfqpoint{9.702790in}{2.016488in}}%
\pgfpathcurveto{\pgfqpoint{9.702790in}{2.021532in}}{\pgfqpoint{9.700786in}{2.026369in}}{\pgfqpoint{9.697220in}{2.029936in}}%
\pgfpathcurveto{\pgfqpoint{9.693653in}{2.033502in}}{\pgfqpoint{9.688815in}{2.035506in}}{\pgfqpoint{9.683772in}{2.035506in}}%
\pgfpathcurveto{\pgfqpoint{9.678728in}{2.035506in}}{\pgfqpoint{9.673890in}{2.033502in}}{\pgfqpoint{9.670324in}{2.029936in}}%
\pgfpathcurveto{\pgfqpoint{9.666758in}{2.026369in}}{\pgfqpoint{9.664754in}{2.021532in}}{\pgfqpoint{9.664754in}{2.016488in}}%
\pgfpathcurveto{\pgfqpoint{9.664754in}{2.011444in}}{\pgfqpoint{9.666758in}{2.006607in}}{\pgfqpoint{9.670324in}{2.003040in}}%
\pgfpathcurveto{\pgfqpoint{9.673890in}{1.999474in}}{\pgfqpoint{9.678728in}{1.997470in}}{\pgfqpoint{9.683772in}{1.997470in}}%
\pgfpathclose%
\pgfusepath{fill}%
\end{pgfscope}%
\begin{pgfscope}%
\pgfpathrectangle{\pgfqpoint{6.572727in}{0.473000in}}{\pgfqpoint{4.227273in}{3.311000in}}%
\pgfusepath{clip}%
\pgfsetbuttcap%
\pgfsetroundjoin%
\definecolor{currentfill}{rgb}{0.127568,0.566949,0.550556}%
\pgfsetfillcolor{currentfill}%
\pgfsetfillopacity{0.700000}%
\pgfsetlinewidth{0.000000pt}%
\definecolor{currentstroke}{rgb}{0.000000,0.000000,0.000000}%
\pgfsetstrokecolor{currentstroke}%
\pgfsetstrokeopacity{0.700000}%
\pgfsetdash{}{0pt}%
\pgfpathmoveto{\pgfqpoint{8.239047in}{2.861221in}}%
\pgfpathcurveto{\pgfqpoint{8.244091in}{2.861221in}}{\pgfqpoint{8.248928in}{2.863225in}}{\pgfqpoint{8.252495in}{2.866792in}}%
\pgfpathcurveto{\pgfqpoint{8.256061in}{2.870358in}}{\pgfqpoint{8.258065in}{2.875196in}}{\pgfqpoint{8.258065in}{2.880239in}}%
\pgfpathcurveto{\pgfqpoint{8.258065in}{2.885283in}}{\pgfqpoint{8.256061in}{2.890121in}}{\pgfqpoint{8.252495in}{2.893687in}}%
\pgfpathcurveto{\pgfqpoint{8.248928in}{2.897254in}}{\pgfqpoint{8.244091in}{2.899258in}}{\pgfqpoint{8.239047in}{2.899258in}}%
\pgfpathcurveto{\pgfqpoint{8.234003in}{2.899258in}}{\pgfqpoint{8.229165in}{2.897254in}}{\pgfqpoint{8.225599in}{2.893687in}}%
\pgfpathcurveto{\pgfqpoint{8.222033in}{2.890121in}}{\pgfqpoint{8.220029in}{2.885283in}}{\pgfqpoint{8.220029in}{2.880239in}}%
\pgfpathcurveto{\pgfqpoint{8.220029in}{2.875196in}}{\pgfqpoint{8.222033in}{2.870358in}}{\pgfqpoint{8.225599in}{2.866792in}}%
\pgfpathcurveto{\pgfqpoint{8.229165in}{2.863225in}}{\pgfqpoint{8.234003in}{2.861221in}}{\pgfqpoint{8.239047in}{2.861221in}}%
\pgfpathclose%
\pgfusepath{fill}%
\end{pgfscope}%
\begin{pgfscope}%
\pgfpathrectangle{\pgfqpoint{6.572727in}{0.473000in}}{\pgfqpoint{4.227273in}{3.311000in}}%
\pgfusepath{clip}%
\pgfsetbuttcap%
\pgfsetroundjoin%
\definecolor{currentfill}{rgb}{0.127568,0.566949,0.550556}%
\pgfsetfillcolor{currentfill}%
\pgfsetfillopacity{0.700000}%
\pgfsetlinewidth{0.000000pt}%
\definecolor{currentstroke}{rgb}{0.000000,0.000000,0.000000}%
\pgfsetstrokecolor{currentstroke}%
\pgfsetstrokeopacity{0.700000}%
\pgfsetdash{}{0pt}%
\pgfpathmoveto{\pgfqpoint{8.050045in}{1.183815in}}%
\pgfpathcurveto{\pgfqpoint{8.055088in}{1.183815in}}{\pgfqpoint{8.059926in}{1.185818in}}{\pgfqpoint{8.063492in}{1.189385in}}%
\pgfpathcurveto{\pgfqpoint{8.067059in}{1.192951in}}{\pgfqpoint{8.069063in}{1.197789in}}{\pgfqpoint{8.069063in}{1.202833in}}%
\pgfpathcurveto{\pgfqpoint{8.069063in}{1.207876in}}{\pgfqpoint{8.067059in}{1.212714in}}{\pgfqpoint{8.063492in}{1.216281in}}%
\pgfpathcurveto{\pgfqpoint{8.059926in}{1.219847in}}{\pgfqpoint{8.055088in}{1.221851in}}{\pgfqpoint{8.050045in}{1.221851in}}%
\pgfpathcurveto{\pgfqpoint{8.045001in}{1.221851in}}{\pgfqpoint{8.040163in}{1.219847in}}{\pgfqpoint{8.036597in}{1.216281in}}%
\pgfpathcurveto{\pgfqpoint{8.033030in}{1.212714in}}{\pgfqpoint{8.031026in}{1.207876in}}{\pgfqpoint{8.031026in}{1.202833in}}%
\pgfpathcurveto{\pgfqpoint{8.031026in}{1.197789in}}{\pgfqpoint{8.033030in}{1.192951in}}{\pgfqpoint{8.036597in}{1.189385in}}%
\pgfpathcurveto{\pgfqpoint{8.040163in}{1.185818in}}{\pgfqpoint{8.045001in}{1.183815in}}{\pgfqpoint{8.050045in}{1.183815in}}%
\pgfpathclose%
\pgfusepath{fill}%
\end{pgfscope}%
\begin{pgfscope}%
\pgfpathrectangle{\pgfqpoint{6.572727in}{0.473000in}}{\pgfqpoint{4.227273in}{3.311000in}}%
\pgfusepath{clip}%
\pgfsetbuttcap%
\pgfsetroundjoin%
\definecolor{currentfill}{rgb}{0.993248,0.906157,0.143936}%
\pgfsetfillcolor{currentfill}%
\pgfsetfillopacity{0.700000}%
\pgfsetlinewidth{0.000000pt}%
\definecolor{currentstroke}{rgb}{0.000000,0.000000,0.000000}%
\pgfsetstrokecolor{currentstroke}%
\pgfsetstrokeopacity{0.700000}%
\pgfsetdash{}{0pt}%
\pgfpathmoveto{\pgfqpoint{9.711457in}{1.930105in}}%
\pgfpathcurveto{\pgfqpoint{9.716501in}{1.930105in}}{\pgfqpoint{9.721339in}{1.932108in}}{\pgfqpoint{9.724905in}{1.935675in}}%
\pgfpathcurveto{\pgfqpoint{9.728471in}{1.939241in}}{\pgfqpoint{9.730475in}{1.944079in}}{\pgfqpoint{9.730475in}{1.949123in}}%
\pgfpathcurveto{\pgfqpoint{9.730475in}{1.954166in}}{\pgfqpoint{9.728471in}{1.959004in}}{\pgfqpoint{9.724905in}{1.962571in}}%
\pgfpathcurveto{\pgfqpoint{9.721339in}{1.966137in}}{\pgfqpoint{9.716501in}{1.968141in}}{\pgfqpoint{9.711457in}{1.968141in}}%
\pgfpathcurveto{\pgfqpoint{9.706413in}{1.968141in}}{\pgfqpoint{9.701576in}{1.966137in}}{\pgfqpoint{9.698009in}{1.962571in}}%
\pgfpathcurveto{\pgfqpoint{9.694443in}{1.959004in}}{\pgfqpoint{9.692439in}{1.954166in}}{\pgfqpoint{9.692439in}{1.949123in}}%
\pgfpathcurveto{\pgfqpoint{9.692439in}{1.944079in}}{\pgfqpoint{9.694443in}{1.939241in}}{\pgfqpoint{9.698009in}{1.935675in}}%
\pgfpathcurveto{\pgfqpoint{9.701576in}{1.932108in}}{\pgfqpoint{9.706413in}{1.930105in}}{\pgfqpoint{9.711457in}{1.930105in}}%
\pgfpathclose%
\pgfusepath{fill}%
\end{pgfscope}%
\begin{pgfscope}%
\pgfpathrectangle{\pgfqpoint{6.572727in}{0.473000in}}{\pgfqpoint{4.227273in}{3.311000in}}%
\pgfusepath{clip}%
\pgfsetbuttcap%
\pgfsetroundjoin%
\definecolor{currentfill}{rgb}{0.127568,0.566949,0.550556}%
\pgfsetfillcolor{currentfill}%
\pgfsetfillopacity{0.700000}%
\pgfsetlinewidth{0.000000pt}%
\definecolor{currentstroke}{rgb}{0.000000,0.000000,0.000000}%
\pgfsetstrokecolor{currentstroke}%
\pgfsetstrokeopacity{0.700000}%
\pgfsetdash{}{0pt}%
\pgfpathmoveto{\pgfqpoint{7.683891in}{1.465955in}}%
\pgfpathcurveto{\pgfqpoint{7.688935in}{1.465955in}}{\pgfqpoint{7.693773in}{1.467959in}}{\pgfqpoint{7.697339in}{1.471525in}}%
\pgfpathcurveto{\pgfqpoint{7.700906in}{1.475092in}}{\pgfqpoint{7.702909in}{1.479930in}}{\pgfqpoint{7.702909in}{1.484973in}}%
\pgfpathcurveto{\pgfqpoint{7.702909in}{1.490017in}}{\pgfqpoint{7.700906in}{1.494855in}}{\pgfqpoint{7.697339in}{1.498421in}}%
\pgfpathcurveto{\pgfqpoint{7.693773in}{1.501988in}}{\pgfqpoint{7.688935in}{1.503991in}}{\pgfqpoint{7.683891in}{1.503991in}}%
\pgfpathcurveto{\pgfqpoint{7.678848in}{1.503991in}}{\pgfqpoint{7.674010in}{1.501988in}}{\pgfqpoint{7.670443in}{1.498421in}}%
\pgfpathcurveto{\pgfqpoint{7.666877in}{1.494855in}}{\pgfqpoint{7.664873in}{1.490017in}}{\pgfqpoint{7.664873in}{1.484973in}}%
\pgfpathcurveto{\pgfqpoint{7.664873in}{1.479930in}}{\pgfqpoint{7.666877in}{1.475092in}}{\pgfqpoint{7.670443in}{1.471525in}}%
\pgfpathcurveto{\pgfqpoint{7.674010in}{1.467959in}}{\pgfqpoint{7.678848in}{1.465955in}}{\pgfqpoint{7.683891in}{1.465955in}}%
\pgfpathclose%
\pgfusepath{fill}%
\end{pgfscope}%
\begin{pgfscope}%
\pgfpathrectangle{\pgfqpoint{6.572727in}{0.473000in}}{\pgfqpoint{4.227273in}{3.311000in}}%
\pgfusepath{clip}%
\pgfsetbuttcap%
\pgfsetroundjoin%
\definecolor{currentfill}{rgb}{0.127568,0.566949,0.550556}%
\pgfsetfillcolor{currentfill}%
\pgfsetfillopacity{0.700000}%
\pgfsetlinewidth{0.000000pt}%
\definecolor{currentstroke}{rgb}{0.000000,0.000000,0.000000}%
\pgfsetstrokecolor{currentstroke}%
\pgfsetstrokeopacity{0.700000}%
\pgfsetdash{}{0pt}%
\pgfpathmoveto{\pgfqpoint{8.415927in}{2.293423in}}%
\pgfpathcurveto{\pgfqpoint{8.420970in}{2.293423in}}{\pgfqpoint{8.425808in}{2.295427in}}{\pgfqpoint{8.429374in}{2.298993in}}%
\pgfpathcurveto{\pgfqpoint{8.432941in}{2.302560in}}{\pgfqpoint{8.434945in}{2.307397in}}{\pgfqpoint{8.434945in}{2.312441in}}%
\pgfpathcurveto{\pgfqpoint{8.434945in}{2.317485in}}{\pgfqpoint{8.432941in}{2.322322in}}{\pgfqpoint{8.429374in}{2.325889in}}%
\pgfpathcurveto{\pgfqpoint{8.425808in}{2.329455in}}{\pgfqpoint{8.420970in}{2.331459in}}{\pgfqpoint{8.415927in}{2.331459in}}%
\pgfpathcurveto{\pgfqpoint{8.410883in}{2.331459in}}{\pgfqpoint{8.406045in}{2.329455in}}{\pgfqpoint{8.402479in}{2.325889in}}%
\pgfpathcurveto{\pgfqpoint{8.398912in}{2.322322in}}{\pgfqpoint{8.396908in}{2.317485in}}{\pgfqpoint{8.396908in}{2.312441in}}%
\pgfpathcurveto{\pgfqpoint{8.396908in}{2.307397in}}{\pgfqpoint{8.398912in}{2.302560in}}{\pgfqpoint{8.402479in}{2.298993in}}%
\pgfpathcurveto{\pgfqpoint{8.406045in}{2.295427in}}{\pgfqpoint{8.410883in}{2.293423in}}{\pgfqpoint{8.415927in}{2.293423in}}%
\pgfpathclose%
\pgfusepath{fill}%
\end{pgfscope}%
\begin{pgfscope}%
\pgfpathrectangle{\pgfqpoint{6.572727in}{0.473000in}}{\pgfqpoint{4.227273in}{3.311000in}}%
\pgfusepath{clip}%
\pgfsetbuttcap%
\pgfsetroundjoin%
\definecolor{currentfill}{rgb}{0.127568,0.566949,0.550556}%
\pgfsetfillcolor{currentfill}%
\pgfsetfillopacity{0.700000}%
\pgfsetlinewidth{0.000000pt}%
\definecolor{currentstroke}{rgb}{0.000000,0.000000,0.000000}%
\pgfsetstrokecolor{currentstroke}%
\pgfsetstrokeopacity{0.700000}%
\pgfsetdash{}{0pt}%
\pgfpathmoveto{\pgfqpoint{7.992050in}{2.401205in}}%
\pgfpathcurveto{\pgfqpoint{7.997094in}{2.401205in}}{\pgfqpoint{8.001931in}{2.403209in}}{\pgfqpoint{8.005498in}{2.406776in}}%
\pgfpathcurveto{\pgfqpoint{8.009064in}{2.410342in}}{\pgfqpoint{8.011068in}{2.415180in}}{\pgfqpoint{8.011068in}{2.420224in}}%
\pgfpathcurveto{\pgfqpoint{8.011068in}{2.425267in}}{\pgfqpoint{8.009064in}{2.430105in}}{\pgfqpoint{8.005498in}{2.433671in}}%
\pgfpathcurveto{\pgfqpoint{8.001931in}{2.437238in}}{\pgfqpoint{7.997094in}{2.439242in}}{\pgfqpoint{7.992050in}{2.439242in}}%
\pgfpathcurveto{\pgfqpoint{7.987006in}{2.439242in}}{\pgfqpoint{7.982168in}{2.437238in}}{\pgfqpoint{7.978602in}{2.433671in}}%
\pgfpathcurveto{\pgfqpoint{7.975036in}{2.430105in}}{\pgfqpoint{7.973032in}{2.425267in}}{\pgfqpoint{7.973032in}{2.420224in}}%
\pgfpathcurveto{\pgfqpoint{7.973032in}{2.415180in}}{\pgfqpoint{7.975036in}{2.410342in}}{\pgfqpoint{7.978602in}{2.406776in}}%
\pgfpathcurveto{\pgfqpoint{7.982168in}{2.403209in}}{\pgfqpoint{7.987006in}{2.401205in}}{\pgfqpoint{7.992050in}{2.401205in}}%
\pgfpathclose%
\pgfusepath{fill}%
\end{pgfscope}%
\begin{pgfscope}%
\pgfpathrectangle{\pgfqpoint{6.572727in}{0.473000in}}{\pgfqpoint{4.227273in}{3.311000in}}%
\pgfusepath{clip}%
\pgfsetbuttcap%
\pgfsetroundjoin%
\definecolor{currentfill}{rgb}{0.993248,0.906157,0.143936}%
\pgfsetfillcolor{currentfill}%
\pgfsetfillopacity{0.700000}%
\pgfsetlinewidth{0.000000pt}%
\definecolor{currentstroke}{rgb}{0.000000,0.000000,0.000000}%
\pgfsetstrokecolor{currentstroke}%
\pgfsetstrokeopacity{0.700000}%
\pgfsetdash{}{0pt}%
\pgfpathmoveto{\pgfqpoint{9.851579in}{1.341629in}}%
\pgfpathcurveto{\pgfqpoint{9.856623in}{1.341629in}}{\pgfqpoint{9.861460in}{1.343633in}}{\pgfqpoint{9.865027in}{1.347200in}}%
\pgfpathcurveto{\pgfqpoint{9.868593in}{1.350766in}}{\pgfqpoint{9.870597in}{1.355604in}}{\pgfqpoint{9.870597in}{1.360647in}}%
\pgfpathcurveto{\pgfqpoint{9.870597in}{1.365691in}}{\pgfqpoint{9.868593in}{1.370529in}}{\pgfqpoint{9.865027in}{1.374095in}}%
\pgfpathcurveto{\pgfqpoint{9.861460in}{1.377662in}}{\pgfqpoint{9.856623in}{1.379666in}}{\pgfqpoint{9.851579in}{1.379666in}}%
\pgfpathcurveto{\pgfqpoint{9.846535in}{1.379666in}}{\pgfqpoint{9.841698in}{1.377662in}}{\pgfqpoint{9.838131in}{1.374095in}}%
\pgfpathcurveto{\pgfqpoint{9.834565in}{1.370529in}}{\pgfqpoint{9.832561in}{1.365691in}}{\pgfqpoint{9.832561in}{1.360647in}}%
\pgfpathcurveto{\pgfqpoint{9.832561in}{1.355604in}}{\pgfqpoint{9.834565in}{1.350766in}}{\pgfqpoint{9.838131in}{1.347200in}}%
\pgfpathcurveto{\pgfqpoint{9.841698in}{1.343633in}}{\pgfqpoint{9.846535in}{1.341629in}}{\pgfqpoint{9.851579in}{1.341629in}}%
\pgfpathclose%
\pgfusepath{fill}%
\end{pgfscope}%
\begin{pgfscope}%
\pgfpathrectangle{\pgfqpoint{6.572727in}{0.473000in}}{\pgfqpoint{4.227273in}{3.311000in}}%
\pgfusepath{clip}%
\pgfsetbuttcap%
\pgfsetroundjoin%
\definecolor{currentfill}{rgb}{0.127568,0.566949,0.550556}%
\pgfsetfillcolor{currentfill}%
\pgfsetfillopacity{0.700000}%
\pgfsetlinewidth{0.000000pt}%
\definecolor{currentstroke}{rgb}{0.000000,0.000000,0.000000}%
\pgfsetstrokecolor{currentstroke}%
\pgfsetstrokeopacity{0.700000}%
\pgfsetdash{}{0pt}%
\pgfpathmoveto{\pgfqpoint{8.376556in}{1.364919in}}%
\pgfpathcurveto{\pgfqpoint{8.381599in}{1.364919in}}{\pgfqpoint{8.386437in}{1.366923in}}{\pgfqpoint{8.390004in}{1.370489in}}%
\pgfpathcurveto{\pgfqpoint{8.393570in}{1.374055in}}{\pgfqpoint{8.395574in}{1.378893in}}{\pgfqpoint{8.395574in}{1.383937in}}%
\pgfpathcurveto{\pgfqpoint{8.395574in}{1.388981in}}{\pgfqpoint{8.393570in}{1.393818in}}{\pgfqpoint{8.390004in}{1.397385in}}%
\pgfpathcurveto{\pgfqpoint{8.386437in}{1.400951in}}{\pgfqpoint{8.381599in}{1.402955in}}{\pgfqpoint{8.376556in}{1.402955in}}%
\pgfpathcurveto{\pgfqpoint{8.371512in}{1.402955in}}{\pgfqpoint{8.366674in}{1.400951in}}{\pgfqpoint{8.363108in}{1.397385in}}%
\pgfpathcurveto{\pgfqpoint{8.359541in}{1.393818in}}{\pgfqpoint{8.357538in}{1.388981in}}{\pgfqpoint{8.357538in}{1.383937in}}%
\pgfpathcurveto{\pgfqpoint{8.357538in}{1.378893in}}{\pgfqpoint{8.359541in}{1.374055in}}{\pgfqpoint{8.363108in}{1.370489in}}%
\pgfpathcurveto{\pgfqpoint{8.366674in}{1.366923in}}{\pgfqpoint{8.371512in}{1.364919in}}{\pgfqpoint{8.376556in}{1.364919in}}%
\pgfpathclose%
\pgfusepath{fill}%
\end{pgfscope}%
\begin{pgfscope}%
\pgfpathrectangle{\pgfqpoint{6.572727in}{0.473000in}}{\pgfqpoint{4.227273in}{3.311000in}}%
\pgfusepath{clip}%
\pgfsetbuttcap%
\pgfsetroundjoin%
\definecolor{currentfill}{rgb}{0.127568,0.566949,0.550556}%
\pgfsetfillcolor{currentfill}%
\pgfsetfillopacity{0.700000}%
\pgfsetlinewidth{0.000000pt}%
\definecolor{currentstroke}{rgb}{0.000000,0.000000,0.000000}%
\pgfsetstrokecolor{currentstroke}%
\pgfsetstrokeopacity{0.700000}%
\pgfsetdash{}{0pt}%
\pgfpathmoveto{\pgfqpoint{8.663505in}{2.394264in}}%
\pgfpathcurveto{\pgfqpoint{8.668549in}{2.394264in}}{\pgfqpoint{8.673387in}{2.396268in}}{\pgfqpoint{8.676953in}{2.399834in}}%
\pgfpathcurveto{\pgfqpoint{8.680519in}{2.403401in}}{\pgfqpoint{8.682523in}{2.408238in}}{\pgfqpoint{8.682523in}{2.413282in}}%
\pgfpathcurveto{\pgfqpoint{8.682523in}{2.418326in}}{\pgfqpoint{8.680519in}{2.423163in}}{\pgfqpoint{8.676953in}{2.426730in}}%
\pgfpathcurveto{\pgfqpoint{8.673387in}{2.430296in}}{\pgfqpoint{8.668549in}{2.432300in}}{\pgfqpoint{8.663505in}{2.432300in}}%
\pgfpathcurveto{\pgfqpoint{8.658461in}{2.432300in}}{\pgfqpoint{8.653624in}{2.430296in}}{\pgfqpoint{8.650057in}{2.426730in}}%
\pgfpathcurveto{\pgfqpoint{8.646491in}{2.423163in}}{\pgfqpoint{8.644487in}{2.418326in}}{\pgfqpoint{8.644487in}{2.413282in}}%
\pgfpathcurveto{\pgfqpoint{8.644487in}{2.408238in}}{\pgfqpoint{8.646491in}{2.403401in}}{\pgfqpoint{8.650057in}{2.399834in}}%
\pgfpathcurveto{\pgfqpoint{8.653624in}{2.396268in}}{\pgfqpoint{8.658461in}{2.394264in}}{\pgfqpoint{8.663505in}{2.394264in}}%
\pgfpathclose%
\pgfusepath{fill}%
\end{pgfscope}%
\begin{pgfscope}%
\pgfpathrectangle{\pgfqpoint{6.572727in}{0.473000in}}{\pgfqpoint{4.227273in}{3.311000in}}%
\pgfusepath{clip}%
\pgfsetbuttcap%
\pgfsetroundjoin%
\definecolor{currentfill}{rgb}{0.993248,0.906157,0.143936}%
\pgfsetfillcolor{currentfill}%
\pgfsetfillopacity{0.700000}%
\pgfsetlinewidth{0.000000pt}%
\definecolor{currentstroke}{rgb}{0.000000,0.000000,0.000000}%
\pgfsetstrokecolor{currentstroke}%
\pgfsetstrokeopacity{0.700000}%
\pgfsetdash{}{0pt}%
\pgfpathmoveto{\pgfqpoint{9.500784in}{1.311439in}}%
\pgfpathcurveto{\pgfqpoint{9.505828in}{1.311439in}}{\pgfqpoint{9.510666in}{1.313443in}}{\pgfqpoint{9.514232in}{1.317009in}}%
\pgfpathcurveto{\pgfqpoint{9.517799in}{1.320576in}}{\pgfqpoint{9.519802in}{1.325414in}}{\pgfqpoint{9.519802in}{1.330457in}}%
\pgfpathcurveto{\pgfqpoint{9.519802in}{1.335501in}}{\pgfqpoint{9.517799in}{1.340339in}}{\pgfqpoint{9.514232in}{1.343905in}}%
\pgfpathcurveto{\pgfqpoint{9.510666in}{1.347472in}}{\pgfqpoint{9.505828in}{1.349475in}}{\pgfqpoint{9.500784in}{1.349475in}}%
\pgfpathcurveto{\pgfqpoint{9.495741in}{1.349475in}}{\pgfqpoint{9.490903in}{1.347472in}}{\pgfqpoint{9.487336in}{1.343905in}}%
\pgfpathcurveto{\pgfqpoint{9.483770in}{1.340339in}}{\pgfqpoint{9.481766in}{1.335501in}}{\pgfqpoint{9.481766in}{1.330457in}}%
\pgfpathcurveto{\pgfqpoint{9.481766in}{1.325414in}}{\pgfqpoint{9.483770in}{1.320576in}}{\pgfqpoint{9.487336in}{1.317009in}}%
\pgfpathcurveto{\pgfqpoint{9.490903in}{1.313443in}}{\pgfqpoint{9.495741in}{1.311439in}}{\pgfqpoint{9.500784in}{1.311439in}}%
\pgfpathclose%
\pgfusepath{fill}%
\end{pgfscope}%
\begin{pgfscope}%
\pgfpathrectangle{\pgfqpoint{6.572727in}{0.473000in}}{\pgfqpoint{4.227273in}{3.311000in}}%
\pgfusepath{clip}%
\pgfsetbuttcap%
\pgfsetroundjoin%
\definecolor{currentfill}{rgb}{0.127568,0.566949,0.550556}%
\pgfsetfillcolor{currentfill}%
\pgfsetfillopacity{0.700000}%
\pgfsetlinewidth{0.000000pt}%
\definecolor{currentstroke}{rgb}{0.000000,0.000000,0.000000}%
\pgfsetstrokecolor{currentstroke}%
\pgfsetstrokeopacity{0.700000}%
\pgfsetdash{}{0pt}%
\pgfpathmoveto{\pgfqpoint{7.629421in}{1.275129in}}%
\pgfpathcurveto{\pgfqpoint{7.634465in}{1.275129in}}{\pgfqpoint{7.639303in}{1.277133in}}{\pgfqpoint{7.642869in}{1.280699in}}%
\pgfpathcurveto{\pgfqpoint{7.646435in}{1.284265in}}{\pgfqpoint{7.648439in}{1.289103in}}{\pgfqpoint{7.648439in}{1.294147in}}%
\pgfpathcurveto{\pgfqpoint{7.648439in}{1.299191in}}{\pgfqpoint{7.646435in}{1.304028in}}{\pgfqpoint{7.642869in}{1.307595in}}%
\pgfpathcurveto{\pgfqpoint{7.639303in}{1.311161in}}{\pgfqpoint{7.634465in}{1.313165in}}{\pgfqpoint{7.629421in}{1.313165in}}%
\pgfpathcurveto{\pgfqpoint{7.624377in}{1.313165in}}{\pgfqpoint{7.619540in}{1.311161in}}{\pgfqpoint{7.615973in}{1.307595in}}%
\pgfpathcurveto{\pgfqpoint{7.612407in}{1.304028in}}{\pgfqpoint{7.610403in}{1.299191in}}{\pgfqpoint{7.610403in}{1.294147in}}%
\pgfpathcurveto{\pgfqpoint{7.610403in}{1.289103in}}{\pgfqpoint{7.612407in}{1.284265in}}{\pgfqpoint{7.615973in}{1.280699in}}%
\pgfpathcurveto{\pgfqpoint{7.619540in}{1.277133in}}{\pgfqpoint{7.624377in}{1.275129in}}{\pgfqpoint{7.629421in}{1.275129in}}%
\pgfpathclose%
\pgfusepath{fill}%
\end{pgfscope}%
\begin{pgfscope}%
\pgfpathrectangle{\pgfqpoint{6.572727in}{0.473000in}}{\pgfqpoint{4.227273in}{3.311000in}}%
\pgfusepath{clip}%
\pgfsetbuttcap%
\pgfsetroundjoin%
\definecolor{currentfill}{rgb}{0.127568,0.566949,0.550556}%
\pgfsetfillcolor{currentfill}%
\pgfsetfillopacity{0.700000}%
\pgfsetlinewidth{0.000000pt}%
\definecolor{currentstroke}{rgb}{0.000000,0.000000,0.000000}%
\pgfsetstrokecolor{currentstroke}%
\pgfsetstrokeopacity{0.700000}%
\pgfsetdash{}{0pt}%
\pgfpathmoveto{\pgfqpoint{8.269253in}{2.908576in}}%
\pgfpathcurveto{\pgfqpoint{8.274296in}{2.908576in}}{\pgfqpoint{8.279134in}{2.910579in}}{\pgfqpoint{8.282701in}{2.914146in}}%
\pgfpathcurveto{\pgfqpoint{8.286267in}{2.917712in}}{\pgfqpoint{8.288271in}{2.922550in}}{\pgfqpoint{8.288271in}{2.927594in}}%
\pgfpathcurveto{\pgfqpoint{8.288271in}{2.932637in}}{\pgfqpoint{8.286267in}{2.937475in}}{\pgfqpoint{8.282701in}{2.941042in}}%
\pgfpathcurveto{\pgfqpoint{8.279134in}{2.944608in}}{\pgfqpoint{8.274296in}{2.946612in}}{\pgfqpoint{8.269253in}{2.946612in}}%
\pgfpathcurveto{\pgfqpoint{8.264209in}{2.946612in}}{\pgfqpoint{8.259371in}{2.944608in}}{\pgfqpoint{8.255805in}{2.941042in}}%
\pgfpathcurveto{\pgfqpoint{8.252238in}{2.937475in}}{\pgfqpoint{8.250235in}{2.932637in}}{\pgfqpoint{8.250235in}{2.927594in}}%
\pgfpathcurveto{\pgfqpoint{8.250235in}{2.922550in}}{\pgfqpoint{8.252238in}{2.917712in}}{\pgfqpoint{8.255805in}{2.914146in}}%
\pgfpathcurveto{\pgfqpoint{8.259371in}{2.910579in}}{\pgfqpoint{8.264209in}{2.908576in}}{\pgfqpoint{8.269253in}{2.908576in}}%
\pgfpathclose%
\pgfusepath{fill}%
\end{pgfscope}%
\begin{pgfscope}%
\pgfpathrectangle{\pgfqpoint{6.572727in}{0.473000in}}{\pgfqpoint{4.227273in}{3.311000in}}%
\pgfusepath{clip}%
\pgfsetbuttcap%
\pgfsetroundjoin%
\definecolor{currentfill}{rgb}{0.127568,0.566949,0.550556}%
\pgfsetfillcolor{currentfill}%
\pgfsetfillopacity{0.700000}%
\pgfsetlinewidth{0.000000pt}%
\definecolor{currentstroke}{rgb}{0.000000,0.000000,0.000000}%
\pgfsetstrokecolor{currentstroke}%
\pgfsetstrokeopacity{0.700000}%
\pgfsetdash{}{0pt}%
\pgfpathmoveto{\pgfqpoint{7.947538in}{1.612614in}}%
\pgfpathcurveto{\pgfqpoint{7.952582in}{1.612614in}}{\pgfqpoint{7.957420in}{1.614618in}}{\pgfqpoint{7.960986in}{1.618185in}}%
\pgfpathcurveto{\pgfqpoint{7.964553in}{1.621751in}}{\pgfqpoint{7.966557in}{1.626589in}}{\pgfqpoint{7.966557in}{1.631632in}}%
\pgfpathcurveto{\pgfqpoint{7.966557in}{1.636676in}}{\pgfqpoint{7.964553in}{1.641514in}}{\pgfqpoint{7.960986in}{1.645080in}}%
\pgfpathcurveto{\pgfqpoint{7.957420in}{1.648647in}}{\pgfqpoint{7.952582in}{1.650651in}}{\pgfqpoint{7.947538in}{1.650651in}}%
\pgfpathcurveto{\pgfqpoint{7.942495in}{1.650651in}}{\pgfqpoint{7.937657in}{1.648647in}}{\pgfqpoint{7.934091in}{1.645080in}}%
\pgfpathcurveto{\pgfqpoint{7.930524in}{1.641514in}}{\pgfqpoint{7.928520in}{1.636676in}}{\pgfqpoint{7.928520in}{1.631632in}}%
\pgfpathcurveto{\pgfqpoint{7.928520in}{1.626589in}}{\pgfqpoint{7.930524in}{1.621751in}}{\pgfqpoint{7.934091in}{1.618185in}}%
\pgfpathcurveto{\pgfqpoint{7.937657in}{1.614618in}}{\pgfqpoint{7.942495in}{1.612614in}}{\pgfqpoint{7.947538in}{1.612614in}}%
\pgfpathclose%
\pgfusepath{fill}%
\end{pgfscope}%
\begin{pgfscope}%
\pgfpathrectangle{\pgfqpoint{6.572727in}{0.473000in}}{\pgfqpoint{4.227273in}{3.311000in}}%
\pgfusepath{clip}%
\pgfsetbuttcap%
\pgfsetroundjoin%
\definecolor{currentfill}{rgb}{0.127568,0.566949,0.550556}%
\pgfsetfillcolor{currentfill}%
\pgfsetfillopacity{0.700000}%
\pgfsetlinewidth{0.000000pt}%
\definecolor{currentstroke}{rgb}{0.000000,0.000000,0.000000}%
\pgfsetstrokecolor{currentstroke}%
\pgfsetstrokeopacity{0.700000}%
\pgfsetdash{}{0pt}%
\pgfpathmoveto{\pgfqpoint{8.072516in}{2.070877in}}%
\pgfpathcurveto{\pgfqpoint{8.077560in}{2.070877in}}{\pgfqpoint{8.082398in}{2.072881in}}{\pgfqpoint{8.085964in}{2.076447in}}%
\pgfpathcurveto{\pgfqpoint{8.089530in}{2.080014in}}{\pgfqpoint{8.091534in}{2.084852in}}{\pgfqpoint{8.091534in}{2.089895in}}%
\pgfpathcurveto{\pgfqpoint{8.091534in}{2.094939in}}{\pgfqpoint{8.089530in}{2.099777in}}{\pgfqpoint{8.085964in}{2.103343in}}%
\pgfpathcurveto{\pgfqpoint{8.082398in}{2.106910in}}{\pgfqpoint{8.077560in}{2.108913in}}{\pgfqpoint{8.072516in}{2.108913in}}%
\pgfpathcurveto{\pgfqpoint{8.067472in}{2.108913in}}{\pgfqpoint{8.062635in}{2.106910in}}{\pgfqpoint{8.059068in}{2.103343in}}%
\pgfpathcurveto{\pgfqpoint{8.055502in}{2.099777in}}{\pgfqpoint{8.053498in}{2.094939in}}{\pgfqpoint{8.053498in}{2.089895in}}%
\pgfpathcurveto{\pgfqpoint{8.053498in}{2.084852in}}{\pgfqpoint{8.055502in}{2.080014in}}{\pgfqpoint{8.059068in}{2.076447in}}%
\pgfpathcurveto{\pgfqpoint{8.062635in}{2.072881in}}{\pgfqpoint{8.067472in}{2.070877in}}{\pgfqpoint{8.072516in}{2.070877in}}%
\pgfpathclose%
\pgfusepath{fill}%
\end{pgfscope}%
\begin{pgfscope}%
\pgfpathrectangle{\pgfqpoint{6.572727in}{0.473000in}}{\pgfqpoint{4.227273in}{3.311000in}}%
\pgfusepath{clip}%
\pgfsetbuttcap%
\pgfsetroundjoin%
\definecolor{currentfill}{rgb}{0.127568,0.566949,0.550556}%
\pgfsetfillcolor{currentfill}%
\pgfsetfillopacity{0.700000}%
\pgfsetlinewidth{0.000000pt}%
\definecolor{currentstroke}{rgb}{0.000000,0.000000,0.000000}%
\pgfsetstrokecolor{currentstroke}%
\pgfsetstrokeopacity{0.700000}%
\pgfsetdash{}{0pt}%
\pgfpathmoveto{\pgfqpoint{7.909375in}{3.181086in}}%
\pgfpathcurveto{\pgfqpoint{7.914418in}{3.181086in}}{\pgfqpoint{7.919256in}{3.183090in}}{\pgfqpoint{7.922823in}{3.186656in}}%
\pgfpathcurveto{\pgfqpoint{7.926389in}{3.190223in}}{\pgfqpoint{7.928393in}{3.195060in}}{\pgfqpoint{7.928393in}{3.200104in}}%
\pgfpathcurveto{\pgfqpoint{7.928393in}{3.205148in}}{\pgfqpoint{7.926389in}{3.209986in}}{\pgfqpoint{7.922823in}{3.213552in}}%
\pgfpathcurveto{\pgfqpoint{7.919256in}{3.217118in}}{\pgfqpoint{7.914418in}{3.219122in}}{\pgfqpoint{7.909375in}{3.219122in}}%
\pgfpathcurveto{\pgfqpoint{7.904331in}{3.219122in}}{\pgfqpoint{7.899493in}{3.217118in}}{\pgfqpoint{7.895927in}{3.213552in}}%
\pgfpathcurveto{\pgfqpoint{7.892361in}{3.209986in}}{\pgfqpoint{7.890357in}{3.205148in}}{\pgfqpoint{7.890357in}{3.200104in}}%
\pgfpathcurveto{\pgfqpoint{7.890357in}{3.195060in}}{\pgfqpoint{7.892361in}{3.190223in}}{\pgfqpoint{7.895927in}{3.186656in}}%
\pgfpathcurveto{\pgfqpoint{7.899493in}{3.183090in}}{\pgfqpoint{7.904331in}{3.181086in}}{\pgfqpoint{7.909375in}{3.181086in}}%
\pgfpathclose%
\pgfusepath{fill}%
\end{pgfscope}%
\begin{pgfscope}%
\pgfpathrectangle{\pgfqpoint{6.572727in}{0.473000in}}{\pgfqpoint{4.227273in}{3.311000in}}%
\pgfusepath{clip}%
\pgfsetbuttcap%
\pgfsetroundjoin%
\definecolor{currentfill}{rgb}{0.127568,0.566949,0.550556}%
\pgfsetfillcolor{currentfill}%
\pgfsetfillopacity{0.700000}%
\pgfsetlinewidth{0.000000pt}%
\definecolor{currentstroke}{rgb}{0.000000,0.000000,0.000000}%
\pgfsetstrokecolor{currentstroke}%
\pgfsetstrokeopacity{0.700000}%
\pgfsetdash{}{0pt}%
\pgfpathmoveto{\pgfqpoint{8.131219in}{2.729274in}}%
\pgfpathcurveto{\pgfqpoint{8.136263in}{2.729274in}}{\pgfqpoint{8.141101in}{2.731278in}}{\pgfqpoint{8.144667in}{2.734845in}}%
\pgfpathcurveto{\pgfqpoint{8.148234in}{2.738411in}}{\pgfqpoint{8.150237in}{2.743249in}}{\pgfqpoint{8.150237in}{2.748292in}}%
\pgfpathcurveto{\pgfqpoint{8.150237in}{2.753336in}}{\pgfqpoint{8.148234in}{2.758174in}}{\pgfqpoint{8.144667in}{2.761740in}}%
\pgfpathcurveto{\pgfqpoint{8.141101in}{2.765307in}}{\pgfqpoint{8.136263in}{2.767311in}}{\pgfqpoint{8.131219in}{2.767311in}}%
\pgfpathcurveto{\pgfqpoint{8.126176in}{2.767311in}}{\pgfqpoint{8.121338in}{2.765307in}}{\pgfqpoint{8.117771in}{2.761740in}}%
\pgfpathcurveto{\pgfqpoint{8.114205in}{2.758174in}}{\pgfqpoint{8.112201in}{2.753336in}}{\pgfqpoint{8.112201in}{2.748292in}}%
\pgfpathcurveto{\pgfqpoint{8.112201in}{2.743249in}}{\pgfqpoint{8.114205in}{2.738411in}}{\pgfqpoint{8.117771in}{2.734845in}}%
\pgfpathcurveto{\pgfqpoint{8.121338in}{2.731278in}}{\pgfqpoint{8.126176in}{2.729274in}}{\pgfqpoint{8.131219in}{2.729274in}}%
\pgfpathclose%
\pgfusepath{fill}%
\end{pgfscope}%
\begin{pgfscope}%
\pgfpathrectangle{\pgfqpoint{6.572727in}{0.473000in}}{\pgfqpoint{4.227273in}{3.311000in}}%
\pgfusepath{clip}%
\pgfsetbuttcap%
\pgfsetroundjoin%
\definecolor{currentfill}{rgb}{0.127568,0.566949,0.550556}%
\pgfsetfillcolor{currentfill}%
\pgfsetfillopacity{0.700000}%
\pgfsetlinewidth{0.000000pt}%
\definecolor{currentstroke}{rgb}{0.000000,0.000000,0.000000}%
\pgfsetstrokecolor{currentstroke}%
\pgfsetstrokeopacity{0.700000}%
\pgfsetdash{}{0pt}%
\pgfpathmoveto{\pgfqpoint{7.992248in}{2.768080in}}%
\pgfpathcurveto{\pgfqpoint{7.997291in}{2.768080in}}{\pgfqpoint{8.002129in}{2.770084in}}{\pgfqpoint{8.005696in}{2.773650in}}%
\pgfpathcurveto{\pgfqpoint{8.009262in}{2.777217in}}{\pgfqpoint{8.011266in}{2.782055in}}{\pgfqpoint{8.011266in}{2.787098in}}%
\pgfpathcurveto{\pgfqpoint{8.011266in}{2.792142in}}{\pgfqpoint{8.009262in}{2.796980in}}{\pgfqpoint{8.005696in}{2.800546in}}%
\pgfpathcurveto{\pgfqpoint{8.002129in}{2.804112in}}{\pgfqpoint{7.997291in}{2.806116in}}{\pgfqpoint{7.992248in}{2.806116in}}%
\pgfpathcurveto{\pgfqpoint{7.987204in}{2.806116in}}{\pgfqpoint{7.982366in}{2.804112in}}{\pgfqpoint{7.978800in}{2.800546in}}%
\pgfpathcurveto{\pgfqpoint{7.975234in}{2.796980in}}{\pgfqpoint{7.973230in}{2.792142in}}{\pgfqpoint{7.973230in}{2.787098in}}%
\pgfpathcurveto{\pgfqpoint{7.973230in}{2.782055in}}{\pgfqpoint{7.975234in}{2.777217in}}{\pgfqpoint{7.978800in}{2.773650in}}%
\pgfpathcurveto{\pgfqpoint{7.982366in}{2.770084in}}{\pgfqpoint{7.987204in}{2.768080in}}{\pgfqpoint{7.992248in}{2.768080in}}%
\pgfpathclose%
\pgfusepath{fill}%
\end{pgfscope}%
\begin{pgfscope}%
\pgfpathrectangle{\pgfqpoint{6.572727in}{0.473000in}}{\pgfqpoint{4.227273in}{3.311000in}}%
\pgfusepath{clip}%
\pgfsetbuttcap%
\pgfsetroundjoin%
\definecolor{currentfill}{rgb}{0.127568,0.566949,0.550556}%
\pgfsetfillcolor{currentfill}%
\pgfsetfillopacity{0.700000}%
\pgfsetlinewidth{0.000000pt}%
\definecolor{currentstroke}{rgb}{0.000000,0.000000,0.000000}%
\pgfsetstrokecolor{currentstroke}%
\pgfsetstrokeopacity{0.700000}%
\pgfsetdash{}{0pt}%
\pgfpathmoveto{\pgfqpoint{8.465141in}{1.481147in}}%
\pgfpathcurveto{\pgfqpoint{8.470185in}{1.481147in}}{\pgfqpoint{8.475023in}{1.483151in}}{\pgfqpoint{8.478589in}{1.486718in}}%
\pgfpathcurveto{\pgfqpoint{8.482156in}{1.490284in}}{\pgfqpoint{8.484159in}{1.495122in}}{\pgfqpoint{8.484159in}{1.500166in}}%
\pgfpathcurveto{\pgfqpoint{8.484159in}{1.505209in}}{\pgfqpoint{8.482156in}{1.510047in}}{\pgfqpoint{8.478589in}{1.513613in}}%
\pgfpathcurveto{\pgfqpoint{8.475023in}{1.517180in}}{\pgfqpoint{8.470185in}{1.519184in}}{\pgfqpoint{8.465141in}{1.519184in}}%
\pgfpathcurveto{\pgfqpoint{8.460098in}{1.519184in}}{\pgfqpoint{8.455260in}{1.517180in}}{\pgfqpoint{8.451693in}{1.513613in}}%
\pgfpathcurveto{\pgfqpoint{8.448127in}{1.510047in}}{\pgfqpoint{8.446123in}{1.505209in}}{\pgfqpoint{8.446123in}{1.500166in}}%
\pgfpathcurveto{\pgfqpoint{8.446123in}{1.495122in}}{\pgfqpoint{8.448127in}{1.490284in}}{\pgfqpoint{8.451693in}{1.486718in}}%
\pgfpathcurveto{\pgfqpoint{8.455260in}{1.483151in}}{\pgfqpoint{8.460098in}{1.481147in}}{\pgfqpoint{8.465141in}{1.481147in}}%
\pgfpathclose%
\pgfusepath{fill}%
\end{pgfscope}%
\begin{pgfscope}%
\pgfpathrectangle{\pgfqpoint{6.572727in}{0.473000in}}{\pgfqpoint{4.227273in}{3.311000in}}%
\pgfusepath{clip}%
\pgfsetbuttcap%
\pgfsetroundjoin%
\definecolor{currentfill}{rgb}{0.127568,0.566949,0.550556}%
\pgfsetfillcolor{currentfill}%
\pgfsetfillopacity{0.700000}%
\pgfsetlinewidth{0.000000pt}%
\definecolor{currentstroke}{rgb}{0.000000,0.000000,0.000000}%
\pgfsetstrokecolor{currentstroke}%
\pgfsetstrokeopacity{0.700000}%
\pgfsetdash{}{0pt}%
\pgfpathmoveto{\pgfqpoint{7.924818in}{2.613724in}}%
\pgfpathcurveto{\pgfqpoint{7.929861in}{2.613724in}}{\pgfqpoint{7.934699in}{2.615728in}}{\pgfqpoint{7.938266in}{2.619295in}}%
\pgfpathcurveto{\pgfqpoint{7.941832in}{2.622861in}}{\pgfqpoint{7.943836in}{2.627699in}}{\pgfqpoint{7.943836in}{2.632743in}}%
\pgfpathcurveto{\pgfqpoint{7.943836in}{2.637786in}}{\pgfqpoint{7.941832in}{2.642624in}}{\pgfqpoint{7.938266in}{2.646190in}}%
\pgfpathcurveto{\pgfqpoint{7.934699in}{2.649757in}}{\pgfqpoint{7.929861in}{2.651761in}}{\pgfqpoint{7.924818in}{2.651761in}}%
\pgfpathcurveto{\pgfqpoint{7.919774in}{2.651761in}}{\pgfqpoint{7.914936in}{2.649757in}}{\pgfqpoint{7.911370in}{2.646190in}}%
\pgfpathcurveto{\pgfqpoint{7.907803in}{2.642624in}}{\pgfqpoint{7.905800in}{2.637786in}}{\pgfqpoint{7.905800in}{2.632743in}}%
\pgfpathcurveto{\pgfqpoint{7.905800in}{2.627699in}}{\pgfqpoint{7.907803in}{2.622861in}}{\pgfqpoint{7.911370in}{2.619295in}}%
\pgfpathcurveto{\pgfqpoint{7.914936in}{2.615728in}}{\pgfqpoint{7.919774in}{2.613724in}}{\pgfqpoint{7.924818in}{2.613724in}}%
\pgfpathclose%
\pgfusepath{fill}%
\end{pgfscope}%
\begin{pgfscope}%
\pgfpathrectangle{\pgfqpoint{6.572727in}{0.473000in}}{\pgfqpoint{4.227273in}{3.311000in}}%
\pgfusepath{clip}%
\pgfsetbuttcap%
\pgfsetroundjoin%
\definecolor{currentfill}{rgb}{0.993248,0.906157,0.143936}%
\pgfsetfillcolor{currentfill}%
\pgfsetfillopacity{0.700000}%
\pgfsetlinewidth{0.000000pt}%
\definecolor{currentstroke}{rgb}{0.000000,0.000000,0.000000}%
\pgfsetstrokecolor{currentstroke}%
\pgfsetstrokeopacity{0.700000}%
\pgfsetdash{}{0pt}%
\pgfpathmoveto{\pgfqpoint{9.627687in}{1.512001in}}%
\pgfpathcurveto{\pgfqpoint{9.632731in}{1.512001in}}{\pgfqpoint{9.637569in}{1.514005in}}{\pgfqpoint{9.641135in}{1.517571in}}%
\pgfpathcurveto{\pgfqpoint{9.644701in}{1.521137in}}{\pgfqpoint{9.646705in}{1.525975in}}{\pgfqpoint{9.646705in}{1.531019in}}%
\pgfpathcurveto{\pgfqpoint{9.646705in}{1.536063in}}{\pgfqpoint{9.644701in}{1.540900in}}{\pgfqpoint{9.641135in}{1.544467in}}%
\pgfpathcurveto{\pgfqpoint{9.637569in}{1.548033in}}{\pgfqpoint{9.632731in}{1.550037in}}{\pgfqpoint{9.627687in}{1.550037in}}%
\pgfpathcurveto{\pgfqpoint{9.622644in}{1.550037in}}{\pgfqpoint{9.617806in}{1.548033in}}{\pgfqpoint{9.614239in}{1.544467in}}%
\pgfpathcurveto{\pgfqpoint{9.610673in}{1.540900in}}{\pgfqpoint{9.608669in}{1.536063in}}{\pgfqpoint{9.608669in}{1.531019in}}%
\pgfpathcurveto{\pgfqpoint{9.608669in}{1.525975in}}{\pgfqpoint{9.610673in}{1.521137in}}{\pgfqpoint{9.614239in}{1.517571in}}%
\pgfpathcurveto{\pgfqpoint{9.617806in}{1.514005in}}{\pgfqpoint{9.622644in}{1.512001in}}{\pgfqpoint{9.627687in}{1.512001in}}%
\pgfpathclose%
\pgfusepath{fill}%
\end{pgfscope}%
\begin{pgfscope}%
\pgfpathrectangle{\pgfqpoint{6.572727in}{0.473000in}}{\pgfqpoint{4.227273in}{3.311000in}}%
\pgfusepath{clip}%
\pgfsetbuttcap%
\pgfsetroundjoin%
\definecolor{currentfill}{rgb}{0.127568,0.566949,0.550556}%
\pgfsetfillcolor{currentfill}%
\pgfsetfillopacity{0.700000}%
\pgfsetlinewidth{0.000000pt}%
\definecolor{currentstroke}{rgb}{0.000000,0.000000,0.000000}%
\pgfsetstrokecolor{currentstroke}%
\pgfsetstrokeopacity{0.700000}%
\pgfsetdash{}{0pt}%
\pgfpathmoveto{\pgfqpoint{7.472487in}{2.162601in}}%
\pgfpathcurveto{\pgfqpoint{7.477531in}{2.162601in}}{\pgfqpoint{7.482369in}{2.164604in}}{\pgfqpoint{7.485935in}{2.168171in}}%
\pgfpathcurveto{\pgfqpoint{7.489501in}{2.171737in}}{\pgfqpoint{7.491505in}{2.176575in}}{\pgfqpoint{7.491505in}{2.181619in}}%
\pgfpathcurveto{\pgfqpoint{7.491505in}{2.186662in}}{\pgfqpoint{7.489501in}{2.191500in}}{\pgfqpoint{7.485935in}{2.195067in}}%
\pgfpathcurveto{\pgfqpoint{7.482369in}{2.198633in}}{\pgfqpoint{7.477531in}{2.200637in}}{\pgfqpoint{7.472487in}{2.200637in}}%
\pgfpathcurveto{\pgfqpoint{7.467444in}{2.200637in}}{\pgfqpoint{7.462606in}{2.198633in}}{\pgfqpoint{7.459039in}{2.195067in}}%
\pgfpathcurveto{\pgfqpoint{7.455473in}{2.191500in}}{\pgfqpoint{7.453469in}{2.186662in}}{\pgfqpoint{7.453469in}{2.181619in}}%
\pgfpathcurveto{\pgfqpoint{7.453469in}{2.176575in}}{\pgfqpoint{7.455473in}{2.171737in}}{\pgfqpoint{7.459039in}{2.168171in}}%
\pgfpathcurveto{\pgfqpoint{7.462606in}{2.164604in}}{\pgfqpoint{7.467444in}{2.162601in}}{\pgfqpoint{7.472487in}{2.162601in}}%
\pgfpathclose%
\pgfusepath{fill}%
\end{pgfscope}%
\begin{pgfscope}%
\pgfpathrectangle{\pgfqpoint{6.572727in}{0.473000in}}{\pgfqpoint{4.227273in}{3.311000in}}%
\pgfusepath{clip}%
\pgfsetbuttcap%
\pgfsetroundjoin%
\definecolor{currentfill}{rgb}{0.127568,0.566949,0.550556}%
\pgfsetfillcolor{currentfill}%
\pgfsetfillopacity{0.700000}%
\pgfsetlinewidth{0.000000pt}%
\definecolor{currentstroke}{rgb}{0.000000,0.000000,0.000000}%
\pgfsetstrokecolor{currentstroke}%
\pgfsetstrokeopacity{0.700000}%
\pgfsetdash{}{0pt}%
\pgfpathmoveto{\pgfqpoint{8.528278in}{3.151786in}}%
\pgfpathcurveto{\pgfqpoint{8.533322in}{3.151786in}}{\pgfqpoint{8.538160in}{3.153790in}}{\pgfqpoint{8.541726in}{3.157357in}}%
\pgfpathcurveto{\pgfqpoint{8.545292in}{3.160923in}}{\pgfqpoint{8.547296in}{3.165761in}}{\pgfqpoint{8.547296in}{3.170805in}}%
\pgfpathcurveto{\pgfqpoint{8.547296in}{3.175848in}}{\pgfqpoint{8.545292in}{3.180686in}}{\pgfqpoint{8.541726in}{3.184252in}}%
\pgfpathcurveto{\pgfqpoint{8.538160in}{3.187819in}}{\pgfqpoint{8.533322in}{3.189823in}}{\pgfqpoint{8.528278in}{3.189823in}}%
\pgfpathcurveto{\pgfqpoint{8.523234in}{3.189823in}}{\pgfqpoint{8.518397in}{3.187819in}}{\pgfqpoint{8.514830in}{3.184252in}}%
\pgfpathcurveto{\pgfqpoint{8.511264in}{3.180686in}}{\pgfqpoint{8.509260in}{3.175848in}}{\pgfqpoint{8.509260in}{3.170805in}}%
\pgfpathcurveto{\pgfqpoint{8.509260in}{3.165761in}}{\pgfqpoint{8.511264in}{3.160923in}}{\pgfqpoint{8.514830in}{3.157357in}}%
\pgfpathcurveto{\pgfqpoint{8.518397in}{3.153790in}}{\pgfqpoint{8.523234in}{3.151786in}}{\pgfqpoint{8.528278in}{3.151786in}}%
\pgfpathclose%
\pgfusepath{fill}%
\end{pgfscope}%
\begin{pgfscope}%
\pgfpathrectangle{\pgfqpoint{6.572727in}{0.473000in}}{\pgfqpoint{4.227273in}{3.311000in}}%
\pgfusepath{clip}%
\pgfsetbuttcap%
\pgfsetroundjoin%
\definecolor{currentfill}{rgb}{0.127568,0.566949,0.550556}%
\pgfsetfillcolor{currentfill}%
\pgfsetfillopacity{0.700000}%
\pgfsetlinewidth{0.000000pt}%
\definecolor{currentstroke}{rgb}{0.000000,0.000000,0.000000}%
\pgfsetstrokecolor{currentstroke}%
\pgfsetstrokeopacity{0.700000}%
\pgfsetdash{}{0pt}%
\pgfpathmoveto{\pgfqpoint{8.062541in}{1.909205in}}%
\pgfpathcurveto{\pgfqpoint{8.067585in}{1.909205in}}{\pgfqpoint{8.072423in}{1.911209in}}{\pgfqpoint{8.075989in}{1.914776in}}%
\pgfpathcurveto{\pgfqpoint{8.079556in}{1.918342in}}{\pgfqpoint{8.081559in}{1.923180in}}{\pgfqpoint{8.081559in}{1.928224in}}%
\pgfpathcurveto{\pgfqpoint{8.081559in}{1.933267in}}{\pgfqpoint{8.079556in}{1.938105in}}{\pgfqpoint{8.075989in}{1.941671in}}%
\pgfpathcurveto{\pgfqpoint{8.072423in}{1.945238in}}{\pgfqpoint{8.067585in}{1.947242in}}{\pgfqpoint{8.062541in}{1.947242in}}%
\pgfpathcurveto{\pgfqpoint{8.057498in}{1.947242in}}{\pgfqpoint{8.052660in}{1.945238in}}{\pgfqpoint{8.049093in}{1.941671in}}%
\pgfpathcurveto{\pgfqpoint{8.045527in}{1.938105in}}{\pgfqpoint{8.043523in}{1.933267in}}{\pgfqpoint{8.043523in}{1.928224in}}%
\pgfpathcurveto{\pgfqpoint{8.043523in}{1.923180in}}{\pgfqpoint{8.045527in}{1.918342in}}{\pgfqpoint{8.049093in}{1.914776in}}%
\pgfpathcurveto{\pgfqpoint{8.052660in}{1.911209in}}{\pgfqpoint{8.057498in}{1.909205in}}{\pgfqpoint{8.062541in}{1.909205in}}%
\pgfpathclose%
\pgfusepath{fill}%
\end{pgfscope}%
\begin{pgfscope}%
\pgfpathrectangle{\pgfqpoint{6.572727in}{0.473000in}}{\pgfqpoint{4.227273in}{3.311000in}}%
\pgfusepath{clip}%
\pgfsetbuttcap%
\pgfsetroundjoin%
\definecolor{currentfill}{rgb}{0.993248,0.906157,0.143936}%
\pgfsetfillcolor{currentfill}%
\pgfsetfillopacity{0.700000}%
\pgfsetlinewidth{0.000000pt}%
\definecolor{currentstroke}{rgb}{0.000000,0.000000,0.000000}%
\pgfsetstrokecolor{currentstroke}%
\pgfsetstrokeopacity{0.700000}%
\pgfsetdash{}{0pt}%
\pgfpathmoveto{\pgfqpoint{9.647632in}{1.812950in}}%
\pgfpathcurveto{\pgfqpoint{9.652676in}{1.812950in}}{\pgfqpoint{9.657514in}{1.814954in}}{\pgfqpoint{9.661080in}{1.818520in}}%
\pgfpathcurveto{\pgfqpoint{9.664647in}{1.822087in}}{\pgfqpoint{9.666651in}{1.826924in}}{\pgfqpoint{9.666651in}{1.831968in}}%
\pgfpathcurveto{\pgfqpoint{9.666651in}{1.837012in}}{\pgfqpoint{9.664647in}{1.841850in}}{\pgfqpoint{9.661080in}{1.845416in}}%
\pgfpathcurveto{\pgfqpoint{9.657514in}{1.848982in}}{\pgfqpoint{9.652676in}{1.850986in}}{\pgfqpoint{9.647632in}{1.850986in}}%
\pgfpathcurveto{\pgfqpoint{9.642589in}{1.850986in}}{\pgfqpoint{9.637751in}{1.848982in}}{\pgfqpoint{9.634185in}{1.845416in}}%
\pgfpathcurveto{\pgfqpoint{9.630618in}{1.841850in}}{\pgfqpoint{9.628614in}{1.837012in}}{\pgfqpoint{9.628614in}{1.831968in}}%
\pgfpathcurveto{\pgfqpoint{9.628614in}{1.826924in}}{\pgfqpoint{9.630618in}{1.822087in}}{\pgfqpoint{9.634185in}{1.818520in}}%
\pgfpathcurveto{\pgfqpoint{9.637751in}{1.814954in}}{\pgfqpoint{9.642589in}{1.812950in}}{\pgfqpoint{9.647632in}{1.812950in}}%
\pgfpathclose%
\pgfusepath{fill}%
\end{pgfscope}%
\begin{pgfscope}%
\pgfpathrectangle{\pgfqpoint{6.572727in}{0.473000in}}{\pgfqpoint{4.227273in}{3.311000in}}%
\pgfusepath{clip}%
\pgfsetbuttcap%
\pgfsetroundjoin%
\definecolor{currentfill}{rgb}{0.127568,0.566949,0.550556}%
\pgfsetfillcolor{currentfill}%
\pgfsetfillopacity{0.700000}%
\pgfsetlinewidth{0.000000pt}%
\definecolor{currentstroke}{rgb}{0.000000,0.000000,0.000000}%
\pgfsetstrokecolor{currentstroke}%
\pgfsetstrokeopacity{0.700000}%
\pgfsetdash{}{0pt}%
\pgfpathmoveto{\pgfqpoint{8.206322in}{1.668720in}}%
\pgfpathcurveto{\pgfqpoint{8.211365in}{1.668720in}}{\pgfqpoint{8.216203in}{1.670724in}}{\pgfqpoint{8.219769in}{1.674290in}}%
\pgfpathcurveto{\pgfqpoint{8.223336in}{1.677856in}}{\pgfqpoint{8.225340in}{1.682694in}}{\pgfqpoint{8.225340in}{1.687738in}}%
\pgfpathcurveto{\pgfqpoint{8.225340in}{1.692782in}}{\pgfqpoint{8.223336in}{1.697619in}}{\pgfqpoint{8.219769in}{1.701186in}}%
\pgfpathcurveto{\pgfqpoint{8.216203in}{1.704752in}}{\pgfqpoint{8.211365in}{1.706756in}}{\pgfqpoint{8.206322in}{1.706756in}}%
\pgfpathcurveto{\pgfqpoint{8.201278in}{1.706756in}}{\pgfqpoint{8.196440in}{1.704752in}}{\pgfqpoint{8.192874in}{1.701186in}}%
\pgfpathcurveto{\pgfqpoint{8.189307in}{1.697619in}}{\pgfqpoint{8.187303in}{1.692782in}}{\pgfqpoint{8.187303in}{1.687738in}}%
\pgfpathcurveto{\pgfqpoint{8.187303in}{1.682694in}}{\pgfqpoint{8.189307in}{1.677856in}}{\pgfqpoint{8.192874in}{1.674290in}}%
\pgfpathcurveto{\pgfqpoint{8.196440in}{1.670724in}}{\pgfqpoint{8.201278in}{1.668720in}}{\pgfqpoint{8.206322in}{1.668720in}}%
\pgfpathclose%
\pgfusepath{fill}%
\end{pgfscope}%
\begin{pgfscope}%
\pgfpathrectangle{\pgfqpoint{6.572727in}{0.473000in}}{\pgfqpoint{4.227273in}{3.311000in}}%
\pgfusepath{clip}%
\pgfsetbuttcap%
\pgfsetroundjoin%
\definecolor{currentfill}{rgb}{0.127568,0.566949,0.550556}%
\pgfsetfillcolor{currentfill}%
\pgfsetfillopacity{0.700000}%
\pgfsetlinewidth{0.000000pt}%
\definecolor{currentstroke}{rgb}{0.000000,0.000000,0.000000}%
\pgfsetstrokecolor{currentstroke}%
\pgfsetstrokeopacity{0.700000}%
\pgfsetdash{}{0pt}%
\pgfpathmoveto{\pgfqpoint{7.334725in}{1.876463in}}%
\pgfpathcurveto{\pgfqpoint{7.339769in}{1.876463in}}{\pgfqpoint{7.344607in}{1.878467in}}{\pgfqpoint{7.348173in}{1.882033in}}%
\pgfpathcurveto{\pgfqpoint{7.351740in}{1.885599in}}{\pgfqpoint{7.353744in}{1.890437in}}{\pgfqpoint{7.353744in}{1.895481in}}%
\pgfpathcurveto{\pgfqpoint{7.353744in}{1.900525in}}{\pgfqpoint{7.351740in}{1.905362in}}{\pgfqpoint{7.348173in}{1.908929in}}%
\pgfpathcurveto{\pgfqpoint{7.344607in}{1.912495in}}{\pgfqpoint{7.339769in}{1.914499in}}{\pgfqpoint{7.334725in}{1.914499in}}%
\pgfpathcurveto{\pgfqpoint{7.329682in}{1.914499in}}{\pgfqpoint{7.324844in}{1.912495in}}{\pgfqpoint{7.321278in}{1.908929in}}%
\pgfpathcurveto{\pgfqpoint{7.317711in}{1.905362in}}{\pgfqpoint{7.315707in}{1.900525in}}{\pgfqpoint{7.315707in}{1.895481in}}%
\pgfpathcurveto{\pgfqpoint{7.315707in}{1.890437in}}{\pgfqpoint{7.317711in}{1.885599in}}{\pgfqpoint{7.321278in}{1.882033in}}%
\pgfpathcurveto{\pgfqpoint{7.324844in}{1.878467in}}{\pgfqpoint{7.329682in}{1.876463in}}{\pgfqpoint{7.334725in}{1.876463in}}%
\pgfpathclose%
\pgfusepath{fill}%
\end{pgfscope}%
\begin{pgfscope}%
\pgfpathrectangle{\pgfqpoint{6.572727in}{0.473000in}}{\pgfqpoint{4.227273in}{3.311000in}}%
\pgfusepath{clip}%
\pgfsetbuttcap%
\pgfsetroundjoin%
\definecolor{currentfill}{rgb}{0.127568,0.566949,0.550556}%
\pgfsetfillcolor{currentfill}%
\pgfsetfillopacity{0.700000}%
\pgfsetlinewidth{0.000000pt}%
\definecolor{currentstroke}{rgb}{0.000000,0.000000,0.000000}%
\pgfsetstrokecolor{currentstroke}%
\pgfsetstrokeopacity{0.700000}%
\pgfsetdash{}{0pt}%
\pgfpathmoveto{\pgfqpoint{7.490342in}{1.397451in}}%
\pgfpathcurveto{\pgfqpoint{7.495385in}{1.397451in}}{\pgfqpoint{7.500223in}{1.399455in}}{\pgfqpoint{7.503790in}{1.403022in}}%
\pgfpathcurveto{\pgfqpoint{7.507356in}{1.406588in}}{\pgfqpoint{7.509360in}{1.411426in}}{\pgfqpoint{7.509360in}{1.416469in}}%
\pgfpathcurveto{\pgfqpoint{7.509360in}{1.421513in}}{\pgfqpoint{7.507356in}{1.426351in}}{\pgfqpoint{7.503790in}{1.429917in}}%
\pgfpathcurveto{\pgfqpoint{7.500223in}{1.433484in}}{\pgfqpoint{7.495385in}{1.435488in}}{\pgfqpoint{7.490342in}{1.435488in}}%
\pgfpathcurveto{\pgfqpoint{7.485298in}{1.435488in}}{\pgfqpoint{7.480460in}{1.433484in}}{\pgfqpoint{7.476894in}{1.429917in}}%
\pgfpathcurveto{\pgfqpoint{7.473327in}{1.426351in}}{\pgfqpoint{7.471324in}{1.421513in}}{\pgfqpoint{7.471324in}{1.416469in}}%
\pgfpathcurveto{\pgfqpoint{7.471324in}{1.411426in}}{\pgfqpoint{7.473327in}{1.406588in}}{\pgfqpoint{7.476894in}{1.403022in}}%
\pgfpathcurveto{\pgfqpoint{7.480460in}{1.399455in}}{\pgfqpoint{7.485298in}{1.397451in}}{\pgfqpoint{7.490342in}{1.397451in}}%
\pgfpathclose%
\pgfusepath{fill}%
\end{pgfscope}%
\begin{pgfscope}%
\pgfpathrectangle{\pgfqpoint{6.572727in}{0.473000in}}{\pgfqpoint{4.227273in}{3.311000in}}%
\pgfusepath{clip}%
\pgfsetbuttcap%
\pgfsetroundjoin%
\definecolor{currentfill}{rgb}{0.127568,0.566949,0.550556}%
\pgfsetfillcolor{currentfill}%
\pgfsetfillopacity{0.700000}%
\pgfsetlinewidth{0.000000pt}%
\definecolor{currentstroke}{rgb}{0.000000,0.000000,0.000000}%
\pgfsetstrokecolor{currentstroke}%
\pgfsetstrokeopacity{0.700000}%
\pgfsetdash{}{0pt}%
\pgfpathmoveto{\pgfqpoint{7.807482in}{1.481697in}}%
\pgfpathcurveto{\pgfqpoint{7.812525in}{1.481697in}}{\pgfqpoint{7.817363in}{1.483701in}}{\pgfqpoint{7.820929in}{1.487268in}}%
\pgfpathcurveto{\pgfqpoint{7.824496in}{1.490834in}}{\pgfqpoint{7.826500in}{1.495672in}}{\pgfqpoint{7.826500in}{1.500715in}}%
\pgfpathcurveto{\pgfqpoint{7.826500in}{1.505759in}}{\pgfqpoint{7.824496in}{1.510597in}}{\pgfqpoint{7.820929in}{1.514163in}}%
\pgfpathcurveto{\pgfqpoint{7.817363in}{1.517730in}}{\pgfqpoint{7.812525in}{1.519734in}}{\pgfqpoint{7.807482in}{1.519734in}}%
\pgfpathcurveto{\pgfqpoint{7.802438in}{1.519734in}}{\pgfqpoint{7.797600in}{1.517730in}}{\pgfqpoint{7.794034in}{1.514163in}}%
\pgfpathcurveto{\pgfqpoint{7.790467in}{1.510597in}}{\pgfqpoint{7.788463in}{1.505759in}}{\pgfqpoint{7.788463in}{1.500715in}}%
\pgfpathcurveto{\pgfqpoint{7.788463in}{1.495672in}}{\pgfqpoint{7.790467in}{1.490834in}}{\pgfqpoint{7.794034in}{1.487268in}}%
\pgfpathcurveto{\pgfqpoint{7.797600in}{1.483701in}}{\pgfqpoint{7.802438in}{1.481697in}}{\pgfqpoint{7.807482in}{1.481697in}}%
\pgfpathclose%
\pgfusepath{fill}%
\end{pgfscope}%
\begin{pgfscope}%
\pgfpathrectangle{\pgfqpoint{6.572727in}{0.473000in}}{\pgfqpoint{4.227273in}{3.311000in}}%
\pgfusepath{clip}%
\pgfsetbuttcap%
\pgfsetroundjoin%
\definecolor{currentfill}{rgb}{0.127568,0.566949,0.550556}%
\pgfsetfillcolor{currentfill}%
\pgfsetfillopacity{0.700000}%
\pgfsetlinewidth{0.000000pt}%
\definecolor{currentstroke}{rgb}{0.000000,0.000000,0.000000}%
\pgfsetstrokecolor{currentstroke}%
\pgfsetstrokeopacity{0.700000}%
\pgfsetdash{}{0pt}%
\pgfpathmoveto{\pgfqpoint{7.499513in}{2.133846in}}%
\pgfpathcurveto{\pgfqpoint{7.504557in}{2.133846in}}{\pgfqpoint{7.509394in}{2.135850in}}{\pgfqpoint{7.512961in}{2.139416in}}%
\pgfpathcurveto{\pgfqpoint{7.516527in}{2.142983in}}{\pgfqpoint{7.518531in}{2.147820in}}{\pgfqpoint{7.518531in}{2.152864in}}%
\pgfpathcurveto{\pgfqpoint{7.518531in}{2.157908in}}{\pgfqpoint{7.516527in}{2.162746in}}{\pgfqpoint{7.512961in}{2.166312in}}%
\pgfpathcurveto{\pgfqpoint{7.509394in}{2.169878in}}{\pgfqpoint{7.504557in}{2.171882in}}{\pgfqpoint{7.499513in}{2.171882in}}%
\pgfpathcurveto{\pgfqpoint{7.494469in}{2.171882in}}{\pgfqpoint{7.489632in}{2.169878in}}{\pgfqpoint{7.486065in}{2.166312in}}%
\pgfpathcurveto{\pgfqpoint{7.482499in}{2.162746in}}{\pgfqpoint{7.480495in}{2.157908in}}{\pgfqpoint{7.480495in}{2.152864in}}%
\pgfpathcurveto{\pgfqpoint{7.480495in}{2.147820in}}{\pgfqpoint{7.482499in}{2.142983in}}{\pgfqpoint{7.486065in}{2.139416in}}%
\pgfpathcurveto{\pgfqpoint{7.489632in}{2.135850in}}{\pgfqpoint{7.494469in}{2.133846in}}{\pgfqpoint{7.499513in}{2.133846in}}%
\pgfpathclose%
\pgfusepath{fill}%
\end{pgfscope}%
\begin{pgfscope}%
\pgfpathrectangle{\pgfqpoint{6.572727in}{0.473000in}}{\pgfqpoint{4.227273in}{3.311000in}}%
\pgfusepath{clip}%
\pgfsetbuttcap%
\pgfsetroundjoin%
\definecolor{currentfill}{rgb}{0.127568,0.566949,0.550556}%
\pgfsetfillcolor{currentfill}%
\pgfsetfillopacity{0.700000}%
\pgfsetlinewidth{0.000000pt}%
\definecolor{currentstroke}{rgb}{0.000000,0.000000,0.000000}%
\pgfsetstrokecolor{currentstroke}%
\pgfsetstrokeopacity{0.700000}%
\pgfsetdash{}{0pt}%
\pgfpathmoveto{\pgfqpoint{7.911405in}{2.824760in}}%
\pgfpathcurveto{\pgfqpoint{7.916448in}{2.824760in}}{\pgfqpoint{7.921286in}{2.826764in}}{\pgfqpoint{7.924853in}{2.830331in}}%
\pgfpathcurveto{\pgfqpoint{7.928419in}{2.833897in}}{\pgfqpoint{7.930423in}{2.838735in}}{\pgfqpoint{7.930423in}{2.843779in}}%
\pgfpathcurveto{\pgfqpoint{7.930423in}{2.848822in}}{\pgfqpoint{7.928419in}{2.853660in}}{\pgfqpoint{7.924853in}{2.857226in}}%
\pgfpathcurveto{\pgfqpoint{7.921286in}{2.860793in}}{\pgfqpoint{7.916448in}{2.862797in}}{\pgfqpoint{7.911405in}{2.862797in}}%
\pgfpathcurveto{\pgfqpoint{7.906361in}{2.862797in}}{\pgfqpoint{7.901523in}{2.860793in}}{\pgfqpoint{7.897957in}{2.857226in}}%
\pgfpathcurveto{\pgfqpoint{7.894390in}{2.853660in}}{\pgfqpoint{7.892387in}{2.848822in}}{\pgfqpoint{7.892387in}{2.843779in}}%
\pgfpathcurveto{\pgfqpoint{7.892387in}{2.838735in}}{\pgfqpoint{7.894390in}{2.833897in}}{\pgfqpoint{7.897957in}{2.830331in}}%
\pgfpathcurveto{\pgfqpoint{7.901523in}{2.826764in}}{\pgfqpoint{7.906361in}{2.824760in}}{\pgfqpoint{7.911405in}{2.824760in}}%
\pgfpathclose%
\pgfusepath{fill}%
\end{pgfscope}%
\begin{pgfscope}%
\pgfpathrectangle{\pgfqpoint{6.572727in}{0.473000in}}{\pgfqpoint{4.227273in}{3.311000in}}%
\pgfusepath{clip}%
\pgfsetbuttcap%
\pgfsetroundjoin%
\definecolor{currentfill}{rgb}{0.127568,0.566949,0.550556}%
\pgfsetfillcolor{currentfill}%
\pgfsetfillopacity{0.700000}%
\pgfsetlinewidth{0.000000pt}%
\definecolor{currentstroke}{rgb}{0.000000,0.000000,0.000000}%
\pgfsetstrokecolor{currentstroke}%
\pgfsetstrokeopacity{0.700000}%
\pgfsetdash{}{0pt}%
\pgfpathmoveto{\pgfqpoint{7.651751in}{2.736208in}}%
\pgfpathcurveto{\pgfqpoint{7.656794in}{2.736208in}}{\pgfqpoint{7.661632in}{2.738212in}}{\pgfqpoint{7.665198in}{2.741779in}}%
\pgfpathcurveto{\pgfqpoint{7.668765in}{2.745345in}}{\pgfqpoint{7.670769in}{2.750183in}}{\pgfqpoint{7.670769in}{2.755227in}}%
\pgfpathcurveto{\pgfqpoint{7.670769in}{2.760270in}}{\pgfqpoint{7.668765in}{2.765108in}}{\pgfqpoint{7.665198in}{2.768674in}}%
\pgfpathcurveto{\pgfqpoint{7.661632in}{2.772241in}}{\pgfqpoint{7.656794in}{2.774245in}}{\pgfqpoint{7.651751in}{2.774245in}}%
\pgfpathcurveto{\pgfqpoint{7.646707in}{2.774245in}}{\pgfqpoint{7.641869in}{2.772241in}}{\pgfqpoint{7.638303in}{2.768674in}}%
\pgfpathcurveto{\pgfqpoint{7.634736in}{2.765108in}}{\pgfqpoint{7.632732in}{2.760270in}}{\pgfqpoint{7.632732in}{2.755227in}}%
\pgfpathcurveto{\pgfqpoint{7.632732in}{2.750183in}}{\pgfqpoint{7.634736in}{2.745345in}}{\pgfqpoint{7.638303in}{2.741779in}}%
\pgfpathcurveto{\pgfqpoint{7.641869in}{2.738212in}}{\pgfqpoint{7.646707in}{2.736208in}}{\pgfqpoint{7.651751in}{2.736208in}}%
\pgfpathclose%
\pgfusepath{fill}%
\end{pgfscope}%
\begin{pgfscope}%
\pgfpathrectangle{\pgfqpoint{6.572727in}{0.473000in}}{\pgfqpoint{4.227273in}{3.311000in}}%
\pgfusepath{clip}%
\pgfsetbuttcap%
\pgfsetroundjoin%
\definecolor{currentfill}{rgb}{0.993248,0.906157,0.143936}%
\pgfsetfillcolor{currentfill}%
\pgfsetfillopacity{0.700000}%
\pgfsetlinewidth{0.000000pt}%
\definecolor{currentstroke}{rgb}{0.000000,0.000000,0.000000}%
\pgfsetstrokecolor{currentstroke}%
\pgfsetstrokeopacity{0.700000}%
\pgfsetdash{}{0pt}%
\pgfpathmoveto{\pgfqpoint{9.465287in}{1.339405in}}%
\pgfpathcurveto{\pgfqpoint{9.470330in}{1.339405in}}{\pgfqpoint{9.475168in}{1.341409in}}{\pgfqpoint{9.478735in}{1.344975in}}%
\pgfpathcurveto{\pgfqpoint{9.482301in}{1.348542in}}{\pgfqpoint{9.484305in}{1.353379in}}{\pgfqpoint{9.484305in}{1.358423in}}%
\pgfpathcurveto{\pgfqpoint{9.484305in}{1.363467in}}{\pgfqpoint{9.482301in}{1.368305in}}{\pgfqpoint{9.478735in}{1.371871in}}%
\pgfpathcurveto{\pgfqpoint{9.475168in}{1.375437in}}{\pgfqpoint{9.470330in}{1.377441in}}{\pgfqpoint{9.465287in}{1.377441in}}%
\pgfpathcurveto{\pgfqpoint{9.460243in}{1.377441in}}{\pgfqpoint{9.455405in}{1.375437in}}{\pgfqpoint{9.451839in}{1.371871in}}%
\pgfpathcurveto{\pgfqpoint{9.448273in}{1.368305in}}{\pgfqpoint{9.446269in}{1.363467in}}{\pgfqpoint{9.446269in}{1.358423in}}%
\pgfpathcurveto{\pgfqpoint{9.446269in}{1.353379in}}{\pgfqpoint{9.448273in}{1.348542in}}{\pgfqpoint{9.451839in}{1.344975in}}%
\pgfpathcurveto{\pgfqpoint{9.455405in}{1.341409in}}{\pgfqpoint{9.460243in}{1.339405in}}{\pgfqpoint{9.465287in}{1.339405in}}%
\pgfpathclose%
\pgfusepath{fill}%
\end{pgfscope}%
\begin{pgfscope}%
\pgfpathrectangle{\pgfqpoint{6.572727in}{0.473000in}}{\pgfqpoint{4.227273in}{3.311000in}}%
\pgfusepath{clip}%
\pgfsetbuttcap%
\pgfsetroundjoin%
\definecolor{currentfill}{rgb}{0.127568,0.566949,0.550556}%
\pgfsetfillcolor{currentfill}%
\pgfsetfillopacity{0.700000}%
\pgfsetlinewidth{0.000000pt}%
\definecolor{currentstroke}{rgb}{0.000000,0.000000,0.000000}%
\pgfsetstrokecolor{currentstroke}%
\pgfsetstrokeopacity{0.700000}%
\pgfsetdash{}{0pt}%
\pgfpathmoveto{\pgfqpoint{8.266369in}{1.544167in}}%
\pgfpathcurveto{\pgfqpoint{8.271412in}{1.544167in}}{\pgfqpoint{8.276250in}{1.546171in}}{\pgfqpoint{8.279816in}{1.549737in}}%
\pgfpathcurveto{\pgfqpoint{8.283383in}{1.553304in}}{\pgfqpoint{8.285387in}{1.558142in}}{\pgfqpoint{8.285387in}{1.563185in}}%
\pgfpathcurveto{\pgfqpoint{8.285387in}{1.568229in}}{\pgfqpoint{8.283383in}{1.573067in}}{\pgfqpoint{8.279816in}{1.576633in}}%
\pgfpathcurveto{\pgfqpoint{8.276250in}{1.580199in}}{\pgfqpoint{8.271412in}{1.582203in}}{\pgfqpoint{8.266369in}{1.582203in}}%
\pgfpathcurveto{\pgfqpoint{8.261325in}{1.582203in}}{\pgfqpoint{8.256487in}{1.580199in}}{\pgfqpoint{8.252921in}{1.576633in}}%
\pgfpathcurveto{\pgfqpoint{8.249354in}{1.573067in}}{\pgfqpoint{8.247350in}{1.568229in}}{\pgfqpoint{8.247350in}{1.563185in}}%
\pgfpathcurveto{\pgfqpoint{8.247350in}{1.558142in}}{\pgfqpoint{8.249354in}{1.553304in}}{\pgfqpoint{8.252921in}{1.549737in}}%
\pgfpathcurveto{\pgfqpoint{8.256487in}{1.546171in}}{\pgfqpoint{8.261325in}{1.544167in}}{\pgfqpoint{8.266369in}{1.544167in}}%
\pgfpathclose%
\pgfusepath{fill}%
\end{pgfscope}%
\begin{pgfscope}%
\pgfpathrectangle{\pgfqpoint{6.572727in}{0.473000in}}{\pgfqpoint{4.227273in}{3.311000in}}%
\pgfusepath{clip}%
\pgfsetbuttcap%
\pgfsetroundjoin%
\definecolor{currentfill}{rgb}{0.127568,0.566949,0.550556}%
\pgfsetfillcolor{currentfill}%
\pgfsetfillopacity{0.700000}%
\pgfsetlinewidth{0.000000pt}%
\definecolor{currentstroke}{rgb}{0.000000,0.000000,0.000000}%
\pgfsetstrokecolor{currentstroke}%
\pgfsetstrokeopacity{0.700000}%
\pgfsetdash{}{0pt}%
\pgfpathmoveto{\pgfqpoint{7.877706in}{1.905673in}}%
\pgfpathcurveto{\pgfqpoint{7.882750in}{1.905673in}}{\pgfqpoint{7.887588in}{1.907677in}}{\pgfqpoint{7.891154in}{1.911244in}}%
\pgfpathcurveto{\pgfqpoint{7.894721in}{1.914810in}}{\pgfqpoint{7.896725in}{1.919648in}}{\pgfqpoint{7.896725in}{1.924691in}}%
\pgfpathcurveto{\pgfqpoint{7.896725in}{1.929735in}}{\pgfqpoint{7.894721in}{1.934573in}}{\pgfqpoint{7.891154in}{1.938139in}}%
\pgfpathcurveto{\pgfqpoint{7.887588in}{1.941706in}}{\pgfqpoint{7.882750in}{1.943710in}}{\pgfqpoint{7.877706in}{1.943710in}}%
\pgfpathcurveto{\pgfqpoint{7.872663in}{1.943710in}}{\pgfqpoint{7.867825in}{1.941706in}}{\pgfqpoint{7.864259in}{1.938139in}}%
\pgfpathcurveto{\pgfqpoint{7.860692in}{1.934573in}}{\pgfqpoint{7.858688in}{1.929735in}}{\pgfqpoint{7.858688in}{1.924691in}}%
\pgfpathcurveto{\pgfqpoint{7.858688in}{1.919648in}}{\pgfqpoint{7.860692in}{1.914810in}}{\pgfqpoint{7.864259in}{1.911244in}}%
\pgfpathcurveto{\pgfqpoint{7.867825in}{1.907677in}}{\pgfqpoint{7.872663in}{1.905673in}}{\pgfqpoint{7.877706in}{1.905673in}}%
\pgfpathclose%
\pgfusepath{fill}%
\end{pgfscope}%
\begin{pgfscope}%
\pgfpathrectangle{\pgfqpoint{6.572727in}{0.473000in}}{\pgfqpoint{4.227273in}{3.311000in}}%
\pgfusepath{clip}%
\pgfsetbuttcap%
\pgfsetroundjoin%
\definecolor{currentfill}{rgb}{0.127568,0.566949,0.550556}%
\pgfsetfillcolor{currentfill}%
\pgfsetfillopacity{0.700000}%
\pgfsetlinewidth{0.000000pt}%
\definecolor{currentstroke}{rgb}{0.000000,0.000000,0.000000}%
\pgfsetstrokecolor{currentstroke}%
\pgfsetstrokeopacity{0.700000}%
\pgfsetdash{}{0pt}%
\pgfpathmoveto{\pgfqpoint{8.063057in}{1.842994in}}%
\pgfpathcurveto{\pgfqpoint{8.068101in}{1.842994in}}{\pgfqpoint{8.072939in}{1.844998in}}{\pgfqpoint{8.076505in}{1.848564in}}%
\pgfpathcurveto{\pgfqpoint{8.080072in}{1.852131in}}{\pgfqpoint{8.082075in}{1.856969in}}{\pgfqpoint{8.082075in}{1.862012in}}%
\pgfpathcurveto{\pgfqpoint{8.082075in}{1.867056in}}{\pgfqpoint{8.080072in}{1.871894in}}{\pgfqpoint{8.076505in}{1.875460in}}%
\pgfpathcurveto{\pgfqpoint{8.072939in}{1.879027in}}{\pgfqpoint{8.068101in}{1.881030in}}{\pgfqpoint{8.063057in}{1.881030in}}%
\pgfpathcurveto{\pgfqpoint{8.058014in}{1.881030in}}{\pgfqpoint{8.053176in}{1.879027in}}{\pgfqpoint{8.049609in}{1.875460in}}%
\pgfpathcurveto{\pgfqpoint{8.046043in}{1.871894in}}{\pgfqpoint{8.044039in}{1.867056in}}{\pgfqpoint{8.044039in}{1.862012in}}%
\pgfpathcurveto{\pgfqpoint{8.044039in}{1.856969in}}{\pgfqpoint{8.046043in}{1.852131in}}{\pgfqpoint{8.049609in}{1.848564in}}%
\pgfpathcurveto{\pgfqpoint{8.053176in}{1.844998in}}{\pgfqpoint{8.058014in}{1.842994in}}{\pgfqpoint{8.063057in}{1.842994in}}%
\pgfpathclose%
\pgfusepath{fill}%
\end{pgfscope}%
\begin{pgfscope}%
\pgfpathrectangle{\pgfqpoint{6.572727in}{0.473000in}}{\pgfqpoint{4.227273in}{3.311000in}}%
\pgfusepath{clip}%
\pgfsetbuttcap%
\pgfsetroundjoin%
\definecolor{currentfill}{rgb}{0.127568,0.566949,0.550556}%
\pgfsetfillcolor{currentfill}%
\pgfsetfillopacity{0.700000}%
\pgfsetlinewidth{0.000000pt}%
\definecolor{currentstroke}{rgb}{0.000000,0.000000,0.000000}%
\pgfsetstrokecolor{currentstroke}%
\pgfsetstrokeopacity{0.700000}%
\pgfsetdash{}{0pt}%
\pgfpathmoveto{\pgfqpoint{7.997701in}{2.687091in}}%
\pgfpathcurveto{\pgfqpoint{8.002744in}{2.687091in}}{\pgfqpoint{8.007582in}{2.689095in}}{\pgfqpoint{8.011148in}{2.692662in}}%
\pgfpathcurveto{\pgfqpoint{8.014715in}{2.696228in}}{\pgfqpoint{8.016719in}{2.701066in}}{\pgfqpoint{8.016719in}{2.706109in}}%
\pgfpathcurveto{\pgfqpoint{8.016719in}{2.711153in}}{\pgfqpoint{8.014715in}{2.715991in}}{\pgfqpoint{8.011148in}{2.719557in}}%
\pgfpathcurveto{\pgfqpoint{8.007582in}{2.723124in}}{\pgfqpoint{8.002744in}{2.725128in}}{\pgfqpoint{7.997701in}{2.725128in}}%
\pgfpathcurveto{\pgfqpoint{7.992657in}{2.725128in}}{\pgfqpoint{7.987819in}{2.723124in}}{\pgfqpoint{7.984253in}{2.719557in}}%
\pgfpathcurveto{\pgfqpoint{7.980686in}{2.715991in}}{\pgfqpoint{7.978682in}{2.711153in}}{\pgfqpoint{7.978682in}{2.706109in}}%
\pgfpathcurveto{\pgfqpoint{7.978682in}{2.701066in}}{\pgfqpoint{7.980686in}{2.696228in}}{\pgfqpoint{7.984253in}{2.692662in}}%
\pgfpathcurveto{\pgfqpoint{7.987819in}{2.689095in}}{\pgfqpoint{7.992657in}{2.687091in}}{\pgfqpoint{7.997701in}{2.687091in}}%
\pgfpathclose%
\pgfusepath{fill}%
\end{pgfscope}%
\begin{pgfscope}%
\pgfpathrectangle{\pgfqpoint{6.572727in}{0.473000in}}{\pgfqpoint{4.227273in}{3.311000in}}%
\pgfusepath{clip}%
\pgfsetbuttcap%
\pgfsetroundjoin%
\definecolor{currentfill}{rgb}{0.993248,0.906157,0.143936}%
\pgfsetfillcolor{currentfill}%
\pgfsetfillopacity{0.700000}%
\pgfsetlinewidth{0.000000pt}%
\definecolor{currentstroke}{rgb}{0.000000,0.000000,0.000000}%
\pgfsetstrokecolor{currentstroke}%
\pgfsetstrokeopacity{0.700000}%
\pgfsetdash{}{0pt}%
\pgfpathmoveto{\pgfqpoint{9.206261in}{1.886270in}}%
\pgfpathcurveto{\pgfqpoint{9.211304in}{1.886270in}}{\pgfqpoint{9.216142in}{1.888274in}}{\pgfqpoint{9.219708in}{1.891840in}}%
\pgfpathcurveto{\pgfqpoint{9.223275in}{1.895406in}}{\pgfqpoint{9.225279in}{1.900244in}}{\pgfqpoint{9.225279in}{1.905288in}}%
\pgfpathcurveto{\pgfqpoint{9.225279in}{1.910331in}}{\pgfqpoint{9.223275in}{1.915169in}}{\pgfqpoint{9.219708in}{1.918736in}}%
\pgfpathcurveto{\pgfqpoint{9.216142in}{1.922302in}}{\pgfqpoint{9.211304in}{1.924306in}}{\pgfqpoint{9.206261in}{1.924306in}}%
\pgfpathcurveto{\pgfqpoint{9.201217in}{1.924306in}}{\pgfqpoint{9.196379in}{1.922302in}}{\pgfqpoint{9.192813in}{1.918736in}}%
\pgfpathcurveto{\pgfqpoint{9.189246in}{1.915169in}}{\pgfqpoint{9.187242in}{1.910331in}}{\pgfqpoint{9.187242in}{1.905288in}}%
\pgfpathcurveto{\pgfqpoint{9.187242in}{1.900244in}}{\pgfqpoint{9.189246in}{1.895406in}}{\pgfqpoint{9.192813in}{1.891840in}}%
\pgfpathcurveto{\pgfqpoint{9.196379in}{1.888274in}}{\pgfqpoint{9.201217in}{1.886270in}}{\pgfqpoint{9.206261in}{1.886270in}}%
\pgfpathclose%
\pgfusepath{fill}%
\end{pgfscope}%
\begin{pgfscope}%
\pgfpathrectangle{\pgfqpoint{6.572727in}{0.473000in}}{\pgfqpoint{4.227273in}{3.311000in}}%
\pgfusepath{clip}%
\pgfsetbuttcap%
\pgfsetroundjoin%
\definecolor{currentfill}{rgb}{0.127568,0.566949,0.550556}%
\pgfsetfillcolor{currentfill}%
\pgfsetfillopacity{0.700000}%
\pgfsetlinewidth{0.000000pt}%
\definecolor{currentstroke}{rgb}{0.000000,0.000000,0.000000}%
\pgfsetstrokecolor{currentstroke}%
\pgfsetstrokeopacity{0.700000}%
\pgfsetdash{}{0pt}%
\pgfpathmoveto{\pgfqpoint{7.603169in}{1.527772in}}%
\pgfpathcurveto{\pgfqpoint{7.608213in}{1.527772in}}{\pgfqpoint{7.613051in}{1.529776in}}{\pgfqpoint{7.616617in}{1.533342in}}%
\pgfpathcurveto{\pgfqpoint{7.620184in}{1.536908in}}{\pgfqpoint{7.622188in}{1.541746in}}{\pgfqpoint{7.622188in}{1.546790in}}%
\pgfpathcurveto{\pgfqpoint{7.622188in}{1.551833in}}{\pgfqpoint{7.620184in}{1.556671in}}{\pgfqpoint{7.616617in}{1.560238in}}%
\pgfpathcurveto{\pgfqpoint{7.613051in}{1.563804in}}{\pgfqpoint{7.608213in}{1.565808in}}{\pgfqpoint{7.603169in}{1.565808in}}%
\pgfpathcurveto{\pgfqpoint{7.598126in}{1.565808in}}{\pgfqpoint{7.593288in}{1.563804in}}{\pgfqpoint{7.589722in}{1.560238in}}%
\pgfpathcurveto{\pgfqpoint{7.586155in}{1.556671in}}{\pgfqpoint{7.584151in}{1.551833in}}{\pgfqpoint{7.584151in}{1.546790in}}%
\pgfpathcurveto{\pgfqpoint{7.584151in}{1.541746in}}{\pgfqpoint{7.586155in}{1.536908in}}{\pgfqpoint{7.589722in}{1.533342in}}%
\pgfpathcurveto{\pgfqpoint{7.593288in}{1.529776in}}{\pgfqpoint{7.598126in}{1.527772in}}{\pgfqpoint{7.603169in}{1.527772in}}%
\pgfpathclose%
\pgfusepath{fill}%
\end{pgfscope}%
\begin{pgfscope}%
\pgfpathrectangle{\pgfqpoint{6.572727in}{0.473000in}}{\pgfqpoint{4.227273in}{3.311000in}}%
\pgfusepath{clip}%
\pgfsetbuttcap%
\pgfsetroundjoin%
\definecolor{currentfill}{rgb}{0.127568,0.566949,0.550556}%
\pgfsetfillcolor{currentfill}%
\pgfsetfillopacity{0.700000}%
\pgfsetlinewidth{0.000000pt}%
\definecolor{currentstroke}{rgb}{0.000000,0.000000,0.000000}%
\pgfsetstrokecolor{currentstroke}%
\pgfsetstrokeopacity{0.700000}%
\pgfsetdash{}{0pt}%
\pgfpathmoveto{\pgfqpoint{8.624924in}{2.474093in}}%
\pgfpathcurveto{\pgfqpoint{8.629968in}{2.474093in}}{\pgfqpoint{8.634806in}{2.476097in}}{\pgfqpoint{8.638372in}{2.479664in}}%
\pgfpathcurveto{\pgfqpoint{8.641938in}{2.483230in}}{\pgfqpoint{8.643942in}{2.488068in}}{\pgfqpoint{8.643942in}{2.493111in}}%
\pgfpathcurveto{\pgfqpoint{8.643942in}{2.498155in}}{\pgfqpoint{8.641938in}{2.502993in}}{\pgfqpoint{8.638372in}{2.506559in}}%
\pgfpathcurveto{\pgfqpoint{8.634806in}{2.510126in}}{\pgfqpoint{8.629968in}{2.512130in}}{\pgfqpoint{8.624924in}{2.512130in}}%
\pgfpathcurveto{\pgfqpoint{8.619880in}{2.512130in}}{\pgfqpoint{8.615043in}{2.510126in}}{\pgfqpoint{8.611476in}{2.506559in}}%
\pgfpathcurveto{\pgfqpoint{8.607910in}{2.502993in}}{\pgfqpoint{8.605906in}{2.498155in}}{\pgfqpoint{8.605906in}{2.493111in}}%
\pgfpathcurveto{\pgfqpoint{8.605906in}{2.488068in}}{\pgfqpoint{8.607910in}{2.483230in}}{\pgfqpoint{8.611476in}{2.479664in}}%
\pgfpathcurveto{\pgfqpoint{8.615043in}{2.476097in}}{\pgfqpoint{8.619880in}{2.474093in}}{\pgfqpoint{8.624924in}{2.474093in}}%
\pgfpathclose%
\pgfusepath{fill}%
\end{pgfscope}%
\begin{pgfscope}%
\pgfpathrectangle{\pgfqpoint{6.572727in}{0.473000in}}{\pgfqpoint{4.227273in}{3.311000in}}%
\pgfusepath{clip}%
\pgfsetbuttcap%
\pgfsetroundjoin%
\definecolor{currentfill}{rgb}{0.993248,0.906157,0.143936}%
\pgfsetfillcolor{currentfill}%
\pgfsetfillopacity{0.700000}%
\pgfsetlinewidth{0.000000pt}%
\definecolor{currentstroke}{rgb}{0.000000,0.000000,0.000000}%
\pgfsetstrokecolor{currentstroke}%
\pgfsetstrokeopacity{0.700000}%
\pgfsetdash{}{0pt}%
\pgfpathmoveto{\pgfqpoint{9.678243in}{1.467329in}}%
\pgfpathcurveto{\pgfqpoint{9.683287in}{1.467329in}}{\pgfqpoint{9.688125in}{1.469333in}}{\pgfqpoint{9.691691in}{1.472899in}}%
\pgfpathcurveto{\pgfqpoint{9.695257in}{1.476466in}}{\pgfqpoint{9.697261in}{1.481303in}}{\pgfqpoint{9.697261in}{1.486347in}}%
\pgfpathcurveto{\pgfqpoint{9.697261in}{1.491391in}}{\pgfqpoint{9.695257in}{1.496228in}}{\pgfqpoint{9.691691in}{1.499795in}}%
\pgfpathcurveto{\pgfqpoint{9.688125in}{1.503361in}}{\pgfqpoint{9.683287in}{1.505365in}}{\pgfqpoint{9.678243in}{1.505365in}}%
\pgfpathcurveto{\pgfqpoint{9.673199in}{1.505365in}}{\pgfqpoint{9.668362in}{1.503361in}}{\pgfqpoint{9.664795in}{1.499795in}}%
\pgfpathcurveto{\pgfqpoint{9.661229in}{1.496228in}}{\pgfqpoint{9.659225in}{1.491391in}}{\pgfqpoint{9.659225in}{1.486347in}}%
\pgfpathcurveto{\pgfqpoint{9.659225in}{1.481303in}}{\pgfqpoint{9.661229in}{1.476466in}}{\pgfqpoint{9.664795in}{1.472899in}}%
\pgfpathcurveto{\pgfqpoint{9.668362in}{1.469333in}}{\pgfqpoint{9.673199in}{1.467329in}}{\pgfqpoint{9.678243in}{1.467329in}}%
\pgfpathclose%
\pgfusepath{fill}%
\end{pgfscope}%
\begin{pgfscope}%
\pgfpathrectangle{\pgfqpoint{6.572727in}{0.473000in}}{\pgfqpoint{4.227273in}{3.311000in}}%
\pgfusepath{clip}%
\pgfsetbuttcap%
\pgfsetroundjoin%
\definecolor{currentfill}{rgb}{0.993248,0.906157,0.143936}%
\pgfsetfillcolor{currentfill}%
\pgfsetfillopacity{0.700000}%
\pgfsetlinewidth{0.000000pt}%
\definecolor{currentstroke}{rgb}{0.000000,0.000000,0.000000}%
\pgfsetstrokecolor{currentstroke}%
\pgfsetstrokeopacity{0.700000}%
\pgfsetdash{}{0pt}%
\pgfpathmoveto{\pgfqpoint{10.026416in}{1.506683in}}%
\pgfpathcurveto{\pgfqpoint{10.031459in}{1.506683in}}{\pgfqpoint{10.036297in}{1.508687in}}{\pgfqpoint{10.039863in}{1.512254in}}%
\pgfpathcurveto{\pgfqpoint{10.043430in}{1.515820in}}{\pgfqpoint{10.045434in}{1.520658in}}{\pgfqpoint{10.045434in}{1.525701in}}%
\pgfpathcurveto{\pgfqpoint{10.045434in}{1.530745in}}{\pgfqpoint{10.043430in}{1.535583in}}{\pgfqpoint{10.039863in}{1.539149in}}%
\pgfpathcurveto{\pgfqpoint{10.036297in}{1.542716in}}{\pgfqpoint{10.031459in}{1.544720in}}{\pgfqpoint{10.026416in}{1.544720in}}%
\pgfpathcurveto{\pgfqpoint{10.021372in}{1.544720in}}{\pgfqpoint{10.016534in}{1.542716in}}{\pgfqpoint{10.012968in}{1.539149in}}%
\pgfpathcurveto{\pgfqpoint{10.009401in}{1.535583in}}{\pgfqpoint{10.007397in}{1.530745in}}{\pgfqpoint{10.007397in}{1.525701in}}%
\pgfpathcurveto{\pgfqpoint{10.007397in}{1.520658in}}{\pgfqpoint{10.009401in}{1.515820in}}{\pgfqpoint{10.012968in}{1.512254in}}%
\pgfpathcurveto{\pgfqpoint{10.016534in}{1.508687in}}{\pgfqpoint{10.021372in}{1.506683in}}{\pgfqpoint{10.026416in}{1.506683in}}%
\pgfpathclose%
\pgfusepath{fill}%
\end{pgfscope}%
\begin{pgfscope}%
\pgfpathrectangle{\pgfqpoint{6.572727in}{0.473000in}}{\pgfqpoint{4.227273in}{3.311000in}}%
\pgfusepath{clip}%
\pgfsetbuttcap%
\pgfsetroundjoin%
\definecolor{currentfill}{rgb}{0.993248,0.906157,0.143936}%
\pgfsetfillcolor{currentfill}%
\pgfsetfillopacity{0.700000}%
\pgfsetlinewidth{0.000000pt}%
\definecolor{currentstroke}{rgb}{0.000000,0.000000,0.000000}%
\pgfsetstrokecolor{currentstroke}%
\pgfsetstrokeopacity{0.700000}%
\pgfsetdash{}{0pt}%
\pgfpathmoveto{\pgfqpoint{9.879733in}{1.452846in}}%
\pgfpathcurveto{\pgfqpoint{9.884777in}{1.452846in}}{\pgfqpoint{9.889615in}{1.454850in}}{\pgfqpoint{9.893181in}{1.458417in}}%
\pgfpathcurveto{\pgfqpoint{9.896748in}{1.461983in}}{\pgfqpoint{9.898752in}{1.466821in}}{\pgfqpoint{9.898752in}{1.471865in}}%
\pgfpathcurveto{\pgfqpoint{9.898752in}{1.476908in}}{\pgfqpoint{9.896748in}{1.481746in}}{\pgfqpoint{9.893181in}{1.485312in}}%
\pgfpathcurveto{\pgfqpoint{9.889615in}{1.488879in}}{\pgfqpoint{9.884777in}{1.490883in}}{\pgfqpoint{9.879733in}{1.490883in}}%
\pgfpathcurveto{\pgfqpoint{9.874690in}{1.490883in}}{\pgfqpoint{9.869852in}{1.488879in}}{\pgfqpoint{9.866286in}{1.485312in}}%
\pgfpathcurveto{\pgfqpoint{9.862719in}{1.481746in}}{\pgfqpoint{9.860715in}{1.476908in}}{\pgfqpoint{9.860715in}{1.471865in}}%
\pgfpathcurveto{\pgfqpoint{9.860715in}{1.466821in}}{\pgfqpoint{9.862719in}{1.461983in}}{\pgfqpoint{9.866286in}{1.458417in}}%
\pgfpathcurveto{\pgfqpoint{9.869852in}{1.454850in}}{\pgfqpoint{9.874690in}{1.452846in}}{\pgfqpoint{9.879733in}{1.452846in}}%
\pgfpathclose%
\pgfusepath{fill}%
\end{pgfscope}%
\begin{pgfscope}%
\pgfpathrectangle{\pgfqpoint{6.572727in}{0.473000in}}{\pgfqpoint{4.227273in}{3.311000in}}%
\pgfusepath{clip}%
\pgfsetbuttcap%
\pgfsetroundjoin%
\definecolor{currentfill}{rgb}{0.127568,0.566949,0.550556}%
\pgfsetfillcolor{currentfill}%
\pgfsetfillopacity{0.700000}%
\pgfsetlinewidth{0.000000pt}%
\definecolor{currentstroke}{rgb}{0.000000,0.000000,0.000000}%
\pgfsetstrokecolor{currentstroke}%
\pgfsetstrokeopacity{0.700000}%
\pgfsetdash{}{0pt}%
\pgfpathmoveto{\pgfqpoint{8.054616in}{1.427265in}}%
\pgfpathcurveto{\pgfqpoint{8.059660in}{1.427265in}}{\pgfqpoint{8.064497in}{1.429269in}}{\pgfqpoint{8.068064in}{1.432836in}}%
\pgfpathcurveto{\pgfqpoint{8.071630in}{1.436402in}}{\pgfqpoint{8.073634in}{1.441240in}}{\pgfqpoint{8.073634in}{1.446283in}}%
\pgfpathcurveto{\pgfqpoint{8.073634in}{1.451327in}}{\pgfqpoint{8.071630in}{1.456165in}}{\pgfqpoint{8.068064in}{1.459731in}}%
\pgfpathcurveto{\pgfqpoint{8.064497in}{1.463298in}}{\pgfqpoint{8.059660in}{1.465302in}}{\pgfqpoint{8.054616in}{1.465302in}}%
\pgfpathcurveto{\pgfqpoint{8.049572in}{1.465302in}}{\pgfqpoint{8.044734in}{1.463298in}}{\pgfqpoint{8.041168in}{1.459731in}}%
\pgfpathcurveto{\pgfqpoint{8.037602in}{1.456165in}}{\pgfqpoint{8.035598in}{1.451327in}}{\pgfqpoint{8.035598in}{1.446283in}}%
\pgfpathcurveto{\pgfqpoint{8.035598in}{1.441240in}}{\pgfqpoint{8.037602in}{1.436402in}}{\pgfqpoint{8.041168in}{1.432836in}}%
\pgfpathcurveto{\pgfqpoint{8.044734in}{1.429269in}}{\pgfqpoint{8.049572in}{1.427265in}}{\pgfqpoint{8.054616in}{1.427265in}}%
\pgfpathclose%
\pgfusepath{fill}%
\end{pgfscope}%
\begin{pgfscope}%
\pgfpathrectangle{\pgfqpoint{6.572727in}{0.473000in}}{\pgfqpoint{4.227273in}{3.311000in}}%
\pgfusepath{clip}%
\pgfsetbuttcap%
\pgfsetroundjoin%
\definecolor{currentfill}{rgb}{0.127568,0.566949,0.550556}%
\pgfsetfillcolor{currentfill}%
\pgfsetfillopacity{0.700000}%
\pgfsetlinewidth{0.000000pt}%
\definecolor{currentstroke}{rgb}{0.000000,0.000000,0.000000}%
\pgfsetstrokecolor{currentstroke}%
\pgfsetstrokeopacity{0.700000}%
\pgfsetdash{}{0pt}%
\pgfpathmoveto{\pgfqpoint{8.951433in}{3.302696in}}%
\pgfpathcurveto{\pgfqpoint{8.956477in}{3.302696in}}{\pgfqpoint{8.961314in}{3.304700in}}{\pgfqpoint{8.964881in}{3.308267in}}%
\pgfpathcurveto{\pgfqpoint{8.968447in}{3.311833in}}{\pgfqpoint{8.970451in}{3.316671in}}{\pgfqpoint{8.970451in}{3.321715in}}%
\pgfpathcurveto{\pgfqpoint{8.970451in}{3.326758in}}{\pgfqpoint{8.968447in}{3.331596in}}{\pgfqpoint{8.964881in}{3.335162in}}%
\pgfpathcurveto{\pgfqpoint{8.961314in}{3.338729in}}{\pgfqpoint{8.956477in}{3.340733in}}{\pgfqpoint{8.951433in}{3.340733in}}%
\pgfpathcurveto{\pgfqpoint{8.946389in}{3.340733in}}{\pgfqpoint{8.941552in}{3.338729in}}{\pgfqpoint{8.937985in}{3.335162in}}%
\pgfpathcurveto{\pgfqpoint{8.934419in}{3.331596in}}{\pgfqpoint{8.932415in}{3.326758in}}{\pgfqpoint{8.932415in}{3.321715in}}%
\pgfpathcurveto{\pgfqpoint{8.932415in}{3.316671in}}{\pgfqpoint{8.934419in}{3.311833in}}{\pgfqpoint{8.937985in}{3.308267in}}%
\pgfpathcurveto{\pgfqpoint{8.941552in}{3.304700in}}{\pgfqpoint{8.946389in}{3.302696in}}{\pgfqpoint{8.951433in}{3.302696in}}%
\pgfpathclose%
\pgfusepath{fill}%
\end{pgfscope}%
\begin{pgfscope}%
\pgfpathrectangle{\pgfqpoint{6.572727in}{0.473000in}}{\pgfqpoint{4.227273in}{3.311000in}}%
\pgfusepath{clip}%
\pgfsetbuttcap%
\pgfsetroundjoin%
\definecolor{currentfill}{rgb}{0.127568,0.566949,0.550556}%
\pgfsetfillcolor{currentfill}%
\pgfsetfillopacity{0.700000}%
\pgfsetlinewidth{0.000000pt}%
\definecolor{currentstroke}{rgb}{0.000000,0.000000,0.000000}%
\pgfsetstrokecolor{currentstroke}%
\pgfsetstrokeopacity{0.700000}%
\pgfsetdash{}{0pt}%
\pgfpathmoveto{\pgfqpoint{8.088472in}{1.730507in}}%
\pgfpathcurveto{\pgfqpoint{8.093516in}{1.730507in}}{\pgfqpoint{8.098354in}{1.732511in}}{\pgfqpoint{8.101920in}{1.736077in}}%
\pgfpathcurveto{\pgfqpoint{8.105486in}{1.739644in}}{\pgfqpoint{8.107490in}{1.744481in}}{\pgfqpoint{8.107490in}{1.749525in}}%
\pgfpathcurveto{\pgfqpoint{8.107490in}{1.754569in}}{\pgfqpoint{8.105486in}{1.759406in}}{\pgfqpoint{8.101920in}{1.762973in}}%
\pgfpathcurveto{\pgfqpoint{8.098354in}{1.766539in}}{\pgfqpoint{8.093516in}{1.768543in}}{\pgfqpoint{8.088472in}{1.768543in}}%
\pgfpathcurveto{\pgfqpoint{8.083428in}{1.768543in}}{\pgfqpoint{8.078591in}{1.766539in}}{\pgfqpoint{8.075024in}{1.762973in}}%
\pgfpathcurveto{\pgfqpoint{8.071458in}{1.759406in}}{\pgfqpoint{8.069454in}{1.754569in}}{\pgfqpoint{8.069454in}{1.749525in}}%
\pgfpathcurveto{\pgfqpoint{8.069454in}{1.744481in}}{\pgfqpoint{8.071458in}{1.739644in}}{\pgfqpoint{8.075024in}{1.736077in}}%
\pgfpathcurveto{\pgfqpoint{8.078591in}{1.732511in}}{\pgfqpoint{8.083428in}{1.730507in}}{\pgfqpoint{8.088472in}{1.730507in}}%
\pgfpathclose%
\pgfusepath{fill}%
\end{pgfscope}%
\begin{pgfscope}%
\pgfpathrectangle{\pgfqpoint{6.572727in}{0.473000in}}{\pgfqpoint{4.227273in}{3.311000in}}%
\pgfusepath{clip}%
\pgfsetbuttcap%
\pgfsetroundjoin%
\definecolor{currentfill}{rgb}{0.993248,0.906157,0.143936}%
\pgfsetfillcolor{currentfill}%
\pgfsetfillopacity{0.700000}%
\pgfsetlinewidth{0.000000pt}%
\definecolor{currentstroke}{rgb}{0.000000,0.000000,0.000000}%
\pgfsetstrokecolor{currentstroke}%
\pgfsetstrokeopacity{0.700000}%
\pgfsetdash{}{0pt}%
\pgfpathmoveto{\pgfqpoint{9.027756in}{1.381827in}}%
\pgfpathcurveto{\pgfqpoint{9.032799in}{1.381827in}}{\pgfqpoint{9.037637in}{1.383830in}}{\pgfqpoint{9.041204in}{1.387397in}}%
\pgfpathcurveto{\pgfqpoint{9.044770in}{1.390963in}}{\pgfqpoint{9.046774in}{1.395801in}}{\pgfqpoint{9.046774in}{1.400845in}}%
\pgfpathcurveto{\pgfqpoint{9.046774in}{1.405888in}}{\pgfqpoint{9.044770in}{1.410726in}}{\pgfqpoint{9.041204in}{1.414293in}}%
\pgfpathcurveto{\pgfqpoint{9.037637in}{1.417859in}}{\pgfqpoint{9.032799in}{1.419863in}}{\pgfqpoint{9.027756in}{1.419863in}}%
\pgfpathcurveto{\pgfqpoint{9.022712in}{1.419863in}}{\pgfqpoint{9.017874in}{1.417859in}}{\pgfqpoint{9.014308in}{1.414293in}}%
\pgfpathcurveto{\pgfqpoint{9.010742in}{1.410726in}}{\pgfqpoint{9.008738in}{1.405888in}}{\pgfqpoint{9.008738in}{1.400845in}}%
\pgfpathcurveto{\pgfqpoint{9.008738in}{1.395801in}}{\pgfqpoint{9.010742in}{1.390963in}}{\pgfqpoint{9.014308in}{1.387397in}}%
\pgfpathcurveto{\pgfqpoint{9.017874in}{1.383830in}}{\pgfqpoint{9.022712in}{1.381827in}}{\pgfqpoint{9.027756in}{1.381827in}}%
\pgfpathclose%
\pgfusepath{fill}%
\end{pgfscope}%
\begin{pgfscope}%
\pgfpathrectangle{\pgfqpoint{6.572727in}{0.473000in}}{\pgfqpoint{4.227273in}{3.311000in}}%
\pgfusepath{clip}%
\pgfsetbuttcap%
\pgfsetroundjoin%
\definecolor{currentfill}{rgb}{0.993248,0.906157,0.143936}%
\pgfsetfillcolor{currentfill}%
\pgfsetfillopacity{0.700000}%
\pgfsetlinewidth{0.000000pt}%
\definecolor{currentstroke}{rgb}{0.000000,0.000000,0.000000}%
\pgfsetstrokecolor{currentstroke}%
\pgfsetstrokeopacity{0.700000}%
\pgfsetdash{}{0pt}%
\pgfpathmoveto{\pgfqpoint{9.303541in}{1.772397in}}%
\pgfpathcurveto{\pgfqpoint{9.308585in}{1.772397in}}{\pgfqpoint{9.313423in}{1.774401in}}{\pgfqpoint{9.316989in}{1.777968in}}%
\pgfpathcurveto{\pgfqpoint{9.320555in}{1.781534in}}{\pgfqpoint{9.322559in}{1.786372in}}{\pgfqpoint{9.322559in}{1.791415in}}%
\pgfpathcurveto{\pgfqpoint{9.322559in}{1.796459in}}{\pgfqpoint{9.320555in}{1.801297in}}{\pgfqpoint{9.316989in}{1.804863in}}%
\pgfpathcurveto{\pgfqpoint{9.313423in}{1.808430in}}{\pgfqpoint{9.308585in}{1.810434in}}{\pgfqpoint{9.303541in}{1.810434in}}%
\pgfpathcurveto{\pgfqpoint{9.298497in}{1.810434in}}{\pgfqpoint{9.293660in}{1.808430in}}{\pgfqpoint{9.290093in}{1.804863in}}%
\pgfpathcurveto{\pgfqpoint{9.286527in}{1.801297in}}{\pgfqpoint{9.284523in}{1.796459in}}{\pgfqpoint{9.284523in}{1.791415in}}%
\pgfpathcurveto{\pgfqpoint{9.284523in}{1.786372in}}{\pgfqpoint{9.286527in}{1.781534in}}{\pgfqpoint{9.290093in}{1.777968in}}%
\pgfpathcurveto{\pgfqpoint{9.293660in}{1.774401in}}{\pgfqpoint{9.298497in}{1.772397in}}{\pgfqpoint{9.303541in}{1.772397in}}%
\pgfpathclose%
\pgfusepath{fill}%
\end{pgfscope}%
\begin{pgfscope}%
\pgfpathrectangle{\pgfqpoint{6.572727in}{0.473000in}}{\pgfqpoint{4.227273in}{3.311000in}}%
\pgfusepath{clip}%
\pgfsetbuttcap%
\pgfsetroundjoin%
\definecolor{currentfill}{rgb}{0.993248,0.906157,0.143936}%
\pgfsetfillcolor{currentfill}%
\pgfsetfillopacity{0.700000}%
\pgfsetlinewidth{0.000000pt}%
\definecolor{currentstroke}{rgb}{0.000000,0.000000,0.000000}%
\pgfsetstrokecolor{currentstroke}%
\pgfsetstrokeopacity{0.700000}%
\pgfsetdash{}{0pt}%
\pgfpathmoveto{\pgfqpoint{9.887278in}{1.469688in}}%
\pgfpathcurveto{\pgfqpoint{9.892322in}{1.469688in}}{\pgfqpoint{9.897160in}{1.471692in}}{\pgfqpoint{9.900726in}{1.475258in}}%
\pgfpathcurveto{\pgfqpoint{9.904293in}{1.478825in}}{\pgfqpoint{9.906297in}{1.483662in}}{\pgfqpoint{9.906297in}{1.488706in}}%
\pgfpathcurveto{\pgfqpoint{9.906297in}{1.493750in}}{\pgfqpoint{9.904293in}{1.498588in}}{\pgfqpoint{9.900726in}{1.502154in}}%
\pgfpathcurveto{\pgfqpoint{9.897160in}{1.505720in}}{\pgfqpoint{9.892322in}{1.507724in}}{\pgfqpoint{9.887278in}{1.507724in}}%
\pgfpathcurveto{\pgfqpoint{9.882235in}{1.507724in}}{\pgfqpoint{9.877397in}{1.505720in}}{\pgfqpoint{9.873831in}{1.502154in}}%
\pgfpathcurveto{\pgfqpoint{9.870264in}{1.498588in}}{\pgfqpoint{9.868260in}{1.493750in}}{\pgfqpoint{9.868260in}{1.488706in}}%
\pgfpathcurveto{\pgfqpoint{9.868260in}{1.483662in}}{\pgfqpoint{9.870264in}{1.478825in}}{\pgfqpoint{9.873831in}{1.475258in}}%
\pgfpathcurveto{\pgfqpoint{9.877397in}{1.471692in}}{\pgfqpoint{9.882235in}{1.469688in}}{\pgfqpoint{9.887278in}{1.469688in}}%
\pgfpathclose%
\pgfusepath{fill}%
\end{pgfscope}%
\begin{pgfscope}%
\pgfpathrectangle{\pgfqpoint{6.572727in}{0.473000in}}{\pgfqpoint{4.227273in}{3.311000in}}%
\pgfusepath{clip}%
\pgfsetbuttcap%
\pgfsetroundjoin%
\definecolor{currentfill}{rgb}{0.993248,0.906157,0.143936}%
\pgfsetfillcolor{currentfill}%
\pgfsetfillopacity{0.700000}%
\pgfsetlinewidth{0.000000pt}%
\definecolor{currentstroke}{rgb}{0.000000,0.000000,0.000000}%
\pgfsetstrokecolor{currentstroke}%
\pgfsetstrokeopacity{0.700000}%
\pgfsetdash{}{0pt}%
\pgfpathmoveto{\pgfqpoint{9.792150in}{1.418646in}}%
\pgfpathcurveto{\pgfqpoint{9.797194in}{1.418646in}}{\pgfqpoint{9.802031in}{1.420650in}}{\pgfqpoint{9.805598in}{1.424216in}}%
\pgfpathcurveto{\pgfqpoint{9.809164in}{1.427783in}}{\pgfqpoint{9.811168in}{1.432620in}}{\pgfqpoint{9.811168in}{1.437664in}}%
\pgfpathcurveto{\pgfqpoint{9.811168in}{1.442708in}}{\pgfqpoint{9.809164in}{1.447546in}}{\pgfqpoint{9.805598in}{1.451112in}}%
\pgfpathcurveto{\pgfqpoint{9.802031in}{1.454678in}}{\pgfqpoint{9.797194in}{1.456682in}}{\pgfqpoint{9.792150in}{1.456682in}}%
\pgfpathcurveto{\pgfqpoint{9.787106in}{1.456682in}}{\pgfqpoint{9.782269in}{1.454678in}}{\pgfqpoint{9.778702in}{1.451112in}}%
\pgfpathcurveto{\pgfqpoint{9.775136in}{1.447546in}}{\pgfqpoint{9.773132in}{1.442708in}}{\pgfqpoint{9.773132in}{1.437664in}}%
\pgfpathcurveto{\pgfqpoint{9.773132in}{1.432620in}}{\pgfqpoint{9.775136in}{1.427783in}}{\pgfqpoint{9.778702in}{1.424216in}}%
\pgfpathcurveto{\pgfqpoint{9.782269in}{1.420650in}}{\pgfqpoint{9.787106in}{1.418646in}}{\pgfqpoint{9.792150in}{1.418646in}}%
\pgfpathclose%
\pgfusepath{fill}%
\end{pgfscope}%
\begin{pgfscope}%
\pgfpathrectangle{\pgfqpoint{6.572727in}{0.473000in}}{\pgfqpoint{4.227273in}{3.311000in}}%
\pgfusepath{clip}%
\pgfsetbuttcap%
\pgfsetroundjoin%
\definecolor{currentfill}{rgb}{0.127568,0.566949,0.550556}%
\pgfsetfillcolor{currentfill}%
\pgfsetfillopacity{0.700000}%
\pgfsetlinewidth{0.000000pt}%
\definecolor{currentstroke}{rgb}{0.000000,0.000000,0.000000}%
\pgfsetstrokecolor{currentstroke}%
\pgfsetstrokeopacity{0.700000}%
\pgfsetdash{}{0pt}%
\pgfpathmoveto{\pgfqpoint{8.204189in}{3.148686in}}%
\pgfpathcurveto{\pgfqpoint{8.209233in}{3.148686in}}{\pgfqpoint{8.214071in}{3.150690in}}{\pgfqpoint{8.217637in}{3.154257in}}%
\pgfpathcurveto{\pgfqpoint{8.221203in}{3.157823in}}{\pgfqpoint{8.223207in}{3.162661in}}{\pgfqpoint{8.223207in}{3.167705in}}%
\pgfpathcurveto{\pgfqpoint{8.223207in}{3.172748in}}{\pgfqpoint{8.221203in}{3.177586in}}{\pgfqpoint{8.217637in}{3.181152in}}%
\pgfpathcurveto{\pgfqpoint{8.214071in}{3.184719in}}{\pgfqpoint{8.209233in}{3.186723in}}{\pgfqpoint{8.204189in}{3.186723in}}%
\pgfpathcurveto{\pgfqpoint{8.199146in}{3.186723in}}{\pgfqpoint{8.194308in}{3.184719in}}{\pgfqpoint{8.190741in}{3.181152in}}%
\pgfpathcurveto{\pgfqpoint{8.187175in}{3.177586in}}{\pgfqpoint{8.185171in}{3.172748in}}{\pgfqpoint{8.185171in}{3.167705in}}%
\pgfpathcurveto{\pgfqpoint{8.185171in}{3.162661in}}{\pgfqpoint{8.187175in}{3.157823in}}{\pgfqpoint{8.190741in}{3.154257in}}%
\pgfpathcurveto{\pgfqpoint{8.194308in}{3.150690in}}{\pgfqpoint{8.199146in}{3.148686in}}{\pgfqpoint{8.204189in}{3.148686in}}%
\pgfpathclose%
\pgfusepath{fill}%
\end{pgfscope}%
\begin{pgfscope}%
\pgfpathrectangle{\pgfqpoint{6.572727in}{0.473000in}}{\pgfqpoint{4.227273in}{3.311000in}}%
\pgfusepath{clip}%
\pgfsetbuttcap%
\pgfsetroundjoin%
\definecolor{currentfill}{rgb}{0.993248,0.906157,0.143936}%
\pgfsetfillcolor{currentfill}%
\pgfsetfillopacity{0.700000}%
\pgfsetlinewidth{0.000000pt}%
\definecolor{currentstroke}{rgb}{0.000000,0.000000,0.000000}%
\pgfsetstrokecolor{currentstroke}%
\pgfsetstrokeopacity{0.700000}%
\pgfsetdash{}{0pt}%
\pgfpathmoveto{\pgfqpoint{9.905322in}{1.411841in}}%
\pgfpathcurveto{\pgfqpoint{9.910366in}{1.411841in}}{\pgfqpoint{9.915204in}{1.413844in}}{\pgfqpoint{9.918770in}{1.417411in}}%
\pgfpathcurveto{\pgfqpoint{9.922336in}{1.420977in}}{\pgfqpoint{9.924340in}{1.425815in}}{\pgfqpoint{9.924340in}{1.430859in}}%
\pgfpathcurveto{\pgfqpoint{9.924340in}{1.435902in}}{\pgfqpoint{9.922336in}{1.440740in}}{\pgfqpoint{9.918770in}{1.444307in}}%
\pgfpathcurveto{\pgfqpoint{9.915204in}{1.447873in}}{\pgfqpoint{9.910366in}{1.449877in}}{\pgfqpoint{9.905322in}{1.449877in}}%
\pgfpathcurveto{\pgfqpoint{9.900278in}{1.449877in}}{\pgfqpoint{9.895441in}{1.447873in}}{\pgfqpoint{9.891874in}{1.444307in}}%
\pgfpathcurveto{\pgfqpoint{9.888308in}{1.440740in}}{\pgfqpoint{9.886304in}{1.435902in}}{\pgfqpoint{9.886304in}{1.430859in}}%
\pgfpathcurveto{\pgfqpoint{9.886304in}{1.425815in}}{\pgfqpoint{9.888308in}{1.420977in}}{\pgfqpoint{9.891874in}{1.417411in}}%
\pgfpathcurveto{\pgfqpoint{9.895441in}{1.413844in}}{\pgfqpoint{9.900278in}{1.411841in}}{\pgfqpoint{9.905322in}{1.411841in}}%
\pgfpathclose%
\pgfusepath{fill}%
\end{pgfscope}%
\begin{pgfscope}%
\pgfpathrectangle{\pgfqpoint{6.572727in}{0.473000in}}{\pgfqpoint{4.227273in}{3.311000in}}%
\pgfusepath{clip}%
\pgfsetbuttcap%
\pgfsetroundjoin%
\definecolor{currentfill}{rgb}{0.127568,0.566949,0.550556}%
\pgfsetfillcolor{currentfill}%
\pgfsetfillopacity{0.700000}%
\pgfsetlinewidth{0.000000pt}%
\definecolor{currentstroke}{rgb}{0.000000,0.000000,0.000000}%
\pgfsetstrokecolor{currentstroke}%
\pgfsetstrokeopacity{0.700000}%
\pgfsetdash{}{0pt}%
\pgfpathmoveto{\pgfqpoint{8.158699in}{1.787335in}}%
\pgfpathcurveto{\pgfqpoint{8.163742in}{1.787335in}}{\pgfqpoint{8.168580in}{1.789339in}}{\pgfqpoint{8.172146in}{1.792906in}}%
\pgfpathcurveto{\pgfqpoint{8.175713in}{1.796472in}}{\pgfqpoint{8.177717in}{1.801310in}}{\pgfqpoint{8.177717in}{1.806354in}}%
\pgfpathcurveto{\pgfqpoint{8.177717in}{1.811397in}}{\pgfqpoint{8.175713in}{1.816235in}}{\pgfqpoint{8.172146in}{1.819801in}}%
\pgfpathcurveto{\pgfqpoint{8.168580in}{1.823368in}}{\pgfqpoint{8.163742in}{1.825372in}}{\pgfqpoint{8.158699in}{1.825372in}}%
\pgfpathcurveto{\pgfqpoint{8.153655in}{1.825372in}}{\pgfqpoint{8.148817in}{1.823368in}}{\pgfqpoint{8.145251in}{1.819801in}}%
\pgfpathcurveto{\pgfqpoint{8.141684in}{1.816235in}}{\pgfqpoint{8.139680in}{1.811397in}}{\pgfqpoint{8.139680in}{1.806354in}}%
\pgfpathcurveto{\pgfqpoint{8.139680in}{1.801310in}}{\pgfqpoint{8.141684in}{1.796472in}}{\pgfqpoint{8.145251in}{1.792906in}}%
\pgfpathcurveto{\pgfqpoint{8.148817in}{1.789339in}}{\pgfqpoint{8.153655in}{1.787335in}}{\pgfqpoint{8.158699in}{1.787335in}}%
\pgfpathclose%
\pgfusepath{fill}%
\end{pgfscope}%
\begin{pgfscope}%
\pgfpathrectangle{\pgfqpoint{6.572727in}{0.473000in}}{\pgfqpoint{4.227273in}{3.311000in}}%
\pgfusepath{clip}%
\pgfsetbuttcap%
\pgfsetroundjoin%
\definecolor{currentfill}{rgb}{0.127568,0.566949,0.550556}%
\pgfsetfillcolor{currentfill}%
\pgfsetfillopacity{0.700000}%
\pgfsetlinewidth{0.000000pt}%
\definecolor{currentstroke}{rgb}{0.000000,0.000000,0.000000}%
\pgfsetstrokecolor{currentstroke}%
\pgfsetstrokeopacity{0.700000}%
\pgfsetdash{}{0pt}%
\pgfpathmoveto{\pgfqpoint{7.893528in}{2.691632in}}%
\pgfpathcurveto{\pgfqpoint{7.898572in}{2.691632in}}{\pgfqpoint{7.903410in}{2.693636in}}{\pgfqpoint{7.906976in}{2.697202in}}%
\pgfpathcurveto{\pgfqpoint{7.910543in}{2.700768in}}{\pgfqpoint{7.912547in}{2.705606in}}{\pgfqpoint{7.912547in}{2.710650in}}%
\pgfpathcurveto{\pgfqpoint{7.912547in}{2.715693in}}{\pgfqpoint{7.910543in}{2.720531in}}{\pgfqpoint{7.906976in}{2.724098in}}%
\pgfpathcurveto{\pgfqpoint{7.903410in}{2.727664in}}{\pgfqpoint{7.898572in}{2.729668in}}{\pgfqpoint{7.893528in}{2.729668in}}%
\pgfpathcurveto{\pgfqpoint{7.888485in}{2.729668in}}{\pgfqpoint{7.883647in}{2.727664in}}{\pgfqpoint{7.880081in}{2.724098in}}%
\pgfpathcurveto{\pgfqpoint{7.876514in}{2.720531in}}{\pgfqpoint{7.874510in}{2.715693in}}{\pgfqpoint{7.874510in}{2.710650in}}%
\pgfpathcurveto{\pgfqpoint{7.874510in}{2.705606in}}{\pgfqpoint{7.876514in}{2.700768in}}{\pgfqpoint{7.880081in}{2.697202in}}%
\pgfpathcurveto{\pgfqpoint{7.883647in}{2.693636in}}{\pgfqpoint{7.888485in}{2.691632in}}{\pgfqpoint{7.893528in}{2.691632in}}%
\pgfpathclose%
\pgfusepath{fill}%
\end{pgfscope}%
\begin{pgfscope}%
\pgfpathrectangle{\pgfqpoint{6.572727in}{0.473000in}}{\pgfqpoint{4.227273in}{3.311000in}}%
\pgfusepath{clip}%
\pgfsetbuttcap%
\pgfsetroundjoin%
\definecolor{currentfill}{rgb}{0.993248,0.906157,0.143936}%
\pgfsetfillcolor{currentfill}%
\pgfsetfillopacity{0.700000}%
\pgfsetlinewidth{0.000000pt}%
\definecolor{currentstroke}{rgb}{0.000000,0.000000,0.000000}%
\pgfsetstrokecolor{currentstroke}%
\pgfsetstrokeopacity{0.700000}%
\pgfsetdash{}{0pt}%
\pgfpathmoveto{\pgfqpoint{9.565851in}{1.574899in}}%
\pgfpathcurveto{\pgfqpoint{9.570895in}{1.574899in}}{\pgfqpoint{9.575732in}{1.576903in}}{\pgfqpoint{9.579299in}{1.580469in}}%
\pgfpathcurveto{\pgfqpoint{9.582865in}{1.584036in}}{\pgfqpoint{9.584869in}{1.588873in}}{\pgfqpoint{9.584869in}{1.593917in}}%
\pgfpathcurveto{\pgfqpoint{9.584869in}{1.598961in}}{\pgfqpoint{9.582865in}{1.603799in}}{\pgfqpoint{9.579299in}{1.607365in}}%
\pgfpathcurveto{\pgfqpoint{9.575732in}{1.610931in}}{\pgfqpoint{9.570895in}{1.612935in}}{\pgfqpoint{9.565851in}{1.612935in}}%
\pgfpathcurveto{\pgfqpoint{9.560807in}{1.612935in}}{\pgfqpoint{9.555970in}{1.610931in}}{\pgfqpoint{9.552403in}{1.607365in}}%
\pgfpathcurveto{\pgfqpoint{9.548837in}{1.603799in}}{\pgfqpoint{9.546833in}{1.598961in}}{\pgfqpoint{9.546833in}{1.593917in}}%
\pgfpathcurveto{\pgfqpoint{9.546833in}{1.588873in}}{\pgfqpoint{9.548837in}{1.584036in}}{\pgfqpoint{9.552403in}{1.580469in}}%
\pgfpathcurveto{\pgfqpoint{9.555970in}{1.576903in}}{\pgfqpoint{9.560807in}{1.574899in}}{\pgfqpoint{9.565851in}{1.574899in}}%
\pgfpathclose%
\pgfusepath{fill}%
\end{pgfscope}%
\begin{pgfscope}%
\pgfpathrectangle{\pgfqpoint{6.572727in}{0.473000in}}{\pgfqpoint{4.227273in}{3.311000in}}%
\pgfusepath{clip}%
\pgfsetbuttcap%
\pgfsetroundjoin%
\definecolor{currentfill}{rgb}{0.127568,0.566949,0.550556}%
\pgfsetfillcolor{currentfill}%
\pgfsetfillopacity{0.700000}%
\pgfsetlinewidth{0.000000pt}%
\definecolor{currentstroke}{rgb}{0.000000,0.000000,0.000000}%
\pgfsetstrokecolor{currentstroke}%
\pgfsetstrokeopacity{0.700000}%
\pgfsetdash{}{0pt}%
\pgfpathmoveto{\pgfqpoint{7.945522in}{1.352324in}}%
\pgfpathcurveto{\pgfqpoint{7.950566in}{1.352324in}}{\pgfqpoint{7.955404in}{1.354328in}}{\pgfqpoint{7.958970in}{1.357895in}}%
\pgfpathcurveto{\pgfqpoint{7.962537in}{1.361461in}}{\pgfqpoint{7.964540in}{1.366299in}}{\pgfqpoint{7.964540in}{1.371342in}}%
\pgfpathcurveto{\pgfqpoint{7.964540in}{1.376386in}}{\pgfqpoint{7.962537in}{1.381224in}}{\pgfqpoint{7.958970in}{1.384790in}}%
\pgfpathcurveto{\pgfqpoint{7.955404in}{1.388357in}}{\pgfqpoint{7.950566in}{1.390361in}}{\pgfqpoint{7.945522in}{1.390361in}}%
\pgfpathcurveto{\pgfqpoint{7.940479in}{1.390361in}}{\pgfqpoint{7.935641in}{1.388357in}}{\pgfqpoint{7.932074in}{1.384790in}}%
\pgfpathcurveto{\pgfqpoint{7.928508in}{1.381224in}}{\pgfqpoint{7.926504in}{1.376386in}}{\pgfqpoint{7.926504in}{1.371342in}}%
\pgfpathcurveto{\pgfqpoint{7.926504in}{1.366299in}}{\pgfqpoint{7.928508in}{1.361461in}}{\pgfqpoint{7.932074in}{1.357895in}}%
\pgfpathcurveto{\pgfqpoint{7.935641in}{1.354328in}}{\pgfqpoint{7.940479in}{1.352324in}}{\pgfqpoint{7.945522in}{1.352324in}}%
\pgfpathclose%
\pgfusepath{fill}%
\end{pgfscope}%
\begin{pgfscope}%
\pgfpathrectangle{\pgfqpoint{6.572727in}{0.473000in}}{\pgfqpoint{4.227273in}{3.311000in}}%
\pgfusepath{clip}%
\pgfsetbuttcap%
\pgfsetroundjoin%
\definecolor{currentfill}{rgb}{0.127568,0.566949,0.550556}%
\pgfsetfillcolor{currentfill}%
\pgfsetfillopacity{0.700000}%
\pgfsetlinewidth{0.000000pt}%
\definecolor{currentstroke}{rgb}{0.000000,0.000000,0.000000}%
\pgfsetstrokecolor{currentstroke}%
\pgfsetstrokeopacity{0.700000}%
\pgfsetdash{}{0pt}%
\pgfpathmoveto{\pgfqpoint{7.508508in}{1.598863in}}%
\pgfpathcurveto{\pgfqpoint{7.513552in}{1.598863in}}{\pgfqpoint{7.518390in}{1.600867in}}{\pgfqpoint{7.521956in}{1.604433in}}%
\pgfpathcurveto{\pgfqpoint{7.525523in}{1.608000in}}{\pgfqpoint{7.527526in}{1.612837in}}{\pgfqpoint{7.527526in}{1.617881in}}%
\pgfpathcurveto{\pgfqpoint{7.527526in}{1.622925in}}{\pgfqpoint{7.525523in}{1.627763in}}{\pgfqpoint{7.521956in}{1.631329in}}%
\pgfpathcurveto{\pgfqpoint{7.518390in}{1.634895in}}{\pgfqpoint{7.513552in}{1.636899in}}{\pgfqpoint{7.508508in}{1.636899in}}%
\pgfpathcurveto{\pgfqpoint{7.503465in}{1.636899in}}{\pgfqpoint{7.498627in}{1.634895in}}{\pgfqpoint{7.495060in}{1.631329in}}%
\pgfpathcurveto{\pgfqpoint{7.491494in}{1.627763in}}{\pgfqpoint{7.489490in}{1.622925in}}{\pgfqpoint{7.489490in}{1.617881in}}%
\pgfpathcurveto{\pgfqpoint{7.489490in}{1.612837in}}{\pgfqpoint{7.491494in}{1.608000in}}{\pgfqpoint{7.495060in}{1.604433in}}%
\pgfpathcurveto{\pgfqpoint{7.498627in}{1.600867in}}{\pgfqpoint{7.503465in}{1.598863in}}{\pgfqpoint{7.508508in}{1.598863in}}%
\pgfpathclose%
\pgfusepath{fill}%
\end{pgfscope}%
\begin{pgfscope}%
\pgfpathrectangle{\pgfqpoint{6.572727in}{0.473000in}}{\pgfqpoint{4.227273in}{3.311000in}}%
\pgfusepath{clip}%
\pgfsetbuttcap%
\pgfsetroundjoin%
\definecolor{currentfill}{rgb}{0.127568,0.566949,0.550556}%
\pgfsetfillcolor{currentfill}%
\pgfsetfillopacity{0.700000}%
\pgfsetlinewidth{0.000000pt}%
\definecolor{currentstroke}{rgb}{0.000000,0.000000,0.000000}%
\pgfsetstrokecolor{currentstroke}%
\pgfsetstrokeopacity{0.700000}%
\pgfsetdash{}{0pt}%
\pgfpathmoveto{\pgfqpoint{8.140730in}{1.722775in}}%
\pgfpathcurveto{\pgfqpoint{8.145774in}{1.722775in}}{\pgfqpoint{8.150612in}{1.724779in}}{\pgfqpoint{8.154178in}{1.728345in}}%
\pgfpathcurveto{\pgfqpoint{8.157745in}{1.731912in}}{\pgfqpoint{8.159748in}{1.736750in}}{\pgfqpoint{8.159748in}{1.741793in}}%
\pgfpathcurveto{\pgfqpoint{8.159748in}{1.746837in}}{\pgfqpoint{8.157745in}{1.751675in}}{\pgfqpoint{8.154178in}{1.755241in}}%
\pgfpathcurveto{\pgfqpoint{8.150612in}{1.758808in}}{\pgfqpoint{8.145774in}{1.760811in}}{\pgfqpoint{8.140730in}{1.760811in}}%
\pgfpathcurveto{\pgfqpoint{8.135687in}{1.760811in}}{\pgfqpoint{8.130849in}{1.758808in}}{\pgfqpoint{8.127282in}{1.755241in}}%
\pgfpathcurveto{\pgfqpoint{8.123716in}{1.751675in}}{\pgfqpoint{8.121712in}{1.746837in}}{\pgfqpoint{8.121712in}{1.741793in}}%
\pgfpathcurveto{\pgfqpoint{8.121712in}{1.736750in}}{\pgfqpoint{8.123716in}{1.731912in}}{\pgfqpoint{8.127282in}{1.728345in}}%
\pgfpathcurveto{\pgfqpoint{8.130849in}{1.724779in}}{\pgfqpoint{8.135687in}{1.722775in}}{\pgfqpoint{8.140730in}{1.722775in}}%
\pgfpathclose%
\pgfusepath{fill}%
\end{pgfscope}%
\begin{pgfscope}%
\pgfpathrectangle{\pgfqpoint{6.572727in}{0.473000in}}{\pgfqpoint{4.227273in}{3.311000in}}%
\pgfusepath{clip}%
\pgfsetbuttcap%
\pgfsetroundjoin%
\definecolor{currentfill}{rgb}{0.127568,0.566949,0.550556}%
\pgfsetfillcolor{currentfill}%
\pgfsetfillopacity{0.700000}%
\pgfsetlinewidth{0.000000pt}%
\definecolor{currentstroke}{rgb}{0.000000,0.000000,0.000000}%
\pgfsetstrokecolor{currentstroke}%
\pgfsetstrokeopacity{0.700000}%
\pgfsetdash{}{0pt}%
\pgfpathmoveto{\pgfqpoint{7.789967in}{2.077803in}}%
\pgfpathcurveto{\pgfqpoint{7.795011in}{2.077803in}}{\pgfqpoint{7.799848in}{2.079807in}}{\pgfqpoint{7.803415in}{2.083373in}}%
\pgfpathcurveto{\pgfqpoint{7.806981in}{2.086939in}}{\pgfqpoint{7.808985in}{2.091777in}}{\pgfqpoint{7.808985in}{2.096821in}}%
\pgfpathcurveto{\pgfqpoint{7.808985in}{2.101865in}}{\pgfqpoint{7.806981in}{2.106702in}}{\pgfqpoint{7.803415in}{2.110269in}}%
\pgfpathcurveto{\pgfqpoint{7.799848in}{2.113835in}}{\pgfqpoint{7.795011in}{2.115839in}}{\pgfqpoint{7.789967in}{2.115839in}}%
\pgfpathcurveto{\pgfqpoint{7.784923in}{2.115839in}}{\pgfqpoint{7.780086in}{2.113835in}}{\pgfqpoint{7.776519in}{2.110269in}}%
\pgfpathcurveto{\pgfqpoint{7.772953in}{2.106702in}}{\pgfqpoint{7.770949in}{2.101865in}}{\pgfqpoint{7.770949in}{2.096821in}}%
\pgfpathcurveto{\pgfqpoint{7.770949in}{2.091777in}}{\pgfqpoint{7.772953in}{2.086939in}}{\pgfqpoint{7.776519in}{2.083373in}}%
\pgfpathcurveto{\pgfqpoint{7.780086in}{2.079807in}}{\pgfqpoint{7.784923in}{2.077803in}}{\pgfqpoint{7.789967in}{2.077803in}}%
\pgfpathclose%
\pgfusepath{fill}%
\end{pgfscope}%
\begin{pgfscope}%
\pgfpathrectangle{\pgfqpoint{6.572727in}{0.473000in}}{\pgfqpoint{4.227273in}{3.311000in}}%
\pgfusepath{clip}%
\pgfsetbuttcap%
\pgfsetroundjoin%
\definecolor{currentfill}{rgb}{0.127568,0.566949,0.550556}%
\pgfsetfillcolor{currentfill}%
\pgfsetfillopacity{0.700000}%
\pgfsetlinewidth{0.000000pt}%
\definecolor{currentstroke}{rgb}{0.000000,0.000000,0.000000}%
\pgfsetstrokecolor{currentstroke}%
\pgfsetstrokeopacity{0.700000}%
\pgfsetdash{}{0pt}%
\pgfpathmoveto{\pgfqpoint{7.241159in}{1.127057in}}%
\pgfpathcurveto{\pgfqpoint{7.246202in}{1.127057in}}{\pgfqpoint{7.251040in}{1.129060in}}{\pgfqpoint{7.254607in}{1.132627in}}%
\pgfpathcurveto{\pgfqpoint{7.258173in}{1.136193in}}{\pgfqpoint{7.260177in}{1.141031in}}{\pgfqpoint{7.260177in}{1.146075in}}%
\pgfpathcurveto{\pgfqpoint{7.260177in}{1.151118in}}{\pgfqpoint{7.258173in}{1.155956in}}{\pgfqpoint{7.254607in}{1.159523in}}%
\pgfpathcurveto{\pgfqpoint{7.251040in}{1.163089in}}{\pgfqpoint{7.246202in}{1.165093in}}{\pgfqpoint{7.241159in}{1.165093in}}%
\pgfpathcurveto{\pgfqpoint{7.236115in}{1.165093in}}{\pgfqpoint{7.231277in}{1.163089in}}{\pgfqpoint{7.227711in}{1.159523in}}%
\pgfpathcurveto{\pgfqpoint{7.224144in}{1.155956in}}{\pgfqpoint{7.222141in}{1.151118in}}{\pgfqpoint{7.222141in}{1.146075in}}%
\pgfpathcurveto{\pgfqpoint{7.222141in}{1.141031in}}{\pgfqpoint{7.224144in}{1.136193in}}{\pgfqpoint{7.227711in}{1.132627in}}%
\pgfpathcurveto{\pgfqpoint{7.231277in}{1.129060in}}{\pgfqpoint{7.236115in}{1.127057in}}{\pgfqpoint{7.241159in}{1.127057in}}%
\pgfpathclose%
\pgfusepath{fill}%
\end{pgfscope}%
\begin{pgfscope}%
\pgfpathrectangle{\pgfqpoint{6.572727in}{0.473000in}}{\pgfqpoint{4.227273in}{3.311000in}}%
\pgfusepath{clip}%
\pgfsetbuttcap%
\pgfsetroundjoin%
\definecolor{currentfill}{rgb}{0.127568,0.566949,0.550556}%
\pgfsetfillcolor{currentfill}%
\pgfsetfillopacity{0.700000}%
\pgfsetlinewidth{0.000000pt}%
\definecolor{currentstroke}{rgb}{0.000000,0.000000,0.000000}%
\pgfsetstrokecolor{currentstroke}%
\pgfsetstrokeopacity{0.700000}%
\pgfsetdash{}{0pt}%
\pgfpathmoveto{\pgfqpoint{7.509983in}{1.563774in}}%
\pgfpathcurveto{\pgfqpoint{7.515027in}{1.563774in}}{\pgfqpoint{7.519865in}{1.565778in}}{\pgfqpoint{7.523431in}{1.569345in}}%
\pgfpathcurveto{\pgfqpoint{7.526998in}{1.572911in}}{\pgfqpoint{7.529002in}{1.577749in}}{\pgfqpoint{7.529002in}{1.582793in}}%
\pgfpathcurveto{\pgfqpoint{7.529002in}{1.587836in}}{\pgfqpoint{7.526998in}{1.592674in}}{\pgfqpoint{7.523431in}{1.596240in}}%
\pgfpathcurveto{\pgfqpoint{7.519865in}{1.599807in}}{\pgfqpoint{7.515027in}{1.601811in}}{\pgfqpoint{7.509983in}{1.601811in}}%
\pgfpathcurveto{\pgfqpoint{7.504940in}{1.601811in}}{\pgfqpoint{7.500102in}{1.599807in}}{\pgfqpoint{7.496536in}{1.596240in}}%
\pgfpathcurveto{\pgfqpoint{7.492969in}{1.592674in}}{\pgfqpoint{7.490965in}{1.587836in}}{\pgfqpoint{7.490965in}{1.582793in}}%
\pgfpathcurveto{\pgfqpoint{7.490965in}{1.577749in}}{\pgfqpoint{7.492969in}{1.572911in}}{\pgfqpoint{7.496536in}{1.569345in}}%
\pgfpathcurveto{\pgfqpoint{7.500102in}{1.565778in}}{\pgfqpoint{7.504940in}{1.563774in}}{\pgfqpoint{7.509983in}{1.563774in}}%
\pgfpathclose%
\pgfusepath{fill}%
\end{pgfscope}%
\begin{pgfscope}%
\pgfpathrectangle{\pgfqpoint{6.572727in}{0.473000in}}{\pgfqpoint{4.227273in}{3.311000in}}%
\pgfusepath{clip}%
\pgfsetbuttcap%
\pgfsetroundjoin%
\definecolor{currentfill}{rgb}{0.127568,0.566949,0.550556}%
\pgfsetfillcolor{currentfill}%
\pgfsetfillopacity{0.700000}%
\pgfsetlinewidth{0.000000pt}%
\definecolor{currentstroke}{rgb}{0.000000,0.000000,0.000000}%
\pgfsetstrokecolor{currentstroke}%
\pgfsetstrokeopacity{0.700000}%
\pgfsetdash{}{0pt}%
\pgfpathmoveto{\pgfqpoint{8.349274in}{3.390402in}}%
\pgfpathcurveto{\pgfqpoint{8.354317in}{3.390402in}}{\pgfqpoint{8.359155in}{3.392405in}}{\pgfqpoint{8.362722in}{3.395972in}}%
\pgfpathcurveto{\pgfqpoint{8.366288in}{3.399538in}}{\pgfqpoint{8.368292in}{3.404376in}}{\pgfqpoint{8.368292in}{3.409420in}}%
\pgfpathcurveto{\pgfqpoint{8.368292in}{3.414463in}}{\pgfqpoint{8.366288in}{3.419301in}}{\pgfqpoint{8.362722in}{3.422868in}}%
\pgfpathcurveto{\pgfqpoint{8.359155in}{3.426434in}}{\pgfqpoint{8.354317in}{3.428438in}}{\pgfqpoint{8.349274in}{3.428438in}}%
\pgfpathcurveto{\pgfqpoint{8.344230in}{3.428438in}}{\pgfqpoint{8.339392in}{3.426434in}}{\pgfqpoint{8.335826in}{3.422868in}}%
\pgfpathcurveto{\pgfqpoint{8.332260in}{3.419301in}}{\pgfqpoint{8.330256in}{3.414463in}}{\pgfqpoint{8.330256in}{3.409420in}}%
\pgfpathcurveto{\pgfqpoint{8.330256in}{3.404376in}}{\pgfqpoint{8.332260in}{3.399538in}}{\pgfqpoint{8.335826in}{3.395972in}}%
\pgfpathcurveto{\pgfqpoint{8.339392in}{3.392405in}}{\pgfqpoint{8.344230in}{3.390402in}}{\pgfqpoint{8.349274in}{3.390402in}}%
\pgfpathclose%
\pgfusepath{fill}%
\end{pgfscope}%
\begin{pgfscope}%
\pgfpathrectangle{\pgfqpoint{6.572727in}{0.473000in}}{\pgfqpoint{4.227273in}{3.311000in}}%
\pgfusepath{clip}%
\pgfsetbuttcap%
\pgfsetroundjoin%
\definecolor{currentfill}{rgb}{0.127568,0.566949,0.550556}%
\pgfsetfillcolor{currentfill}%
\pgfsetfillopacity{0.700000}%
\pgfsetlinewidth{0.000000pt}%
\definecolor{currentstroke}{rgb}{0.000000,0.000000,0.000000}%
\pgfsetstrokecolor{currentstroke}%
\pgfsetstrokeopacity{0.700000}%
\pgfsetdash{}{0pt}%
\pgfpathmoveto{\pgfqpoint{7.844840in}{2.075844in}}%
\pgfpathcurveto{\pgfqpoint{7.849884in}{2.075844in}}{\pgfqpoint{7.854722in}{2.077848in}}{\pgfqpoint{7.858288in}{2.081414in}}%
\pgfpathcurveto{\pgfqpoint{7.861855in}{2.084981in}}{\pgfqpoint{7.863858in}{2.089818in}}{\pgfqpoint{7.863858in}{2.094862in}}%
\pgfpathcurveto{\pgfqpoint{7.863858in}{2.099906in}}{\pgfqpoint{7.861855in}{2.104744in}}{\pgfqpoint{7.858288in}{2.108310in}}%
\pgfpathcurveto{\pgfqpoint{7.854722in}{2.111876in}}{\pgfqpoint{7.849884in}{2.113880in}}{\pgfqpoint{7.844840in}{2.113880in}}%
\pgfpathcurveto{\pgfqpoint{7.839797in}{2.113880in}}{\pgfqpoint{7.834959in}{2.111876in}}{\pgfqpoint{7.831392in}{2.108310in}}%
\pgfpathcurveto{\pgfqpoint{7.827826in}{2.104744in}}{\pgfqpoint{7.825822in}{2.099906in}}{\pgfqpoint{7.825822in}{2.094862in}}%
\pgfpathcurveto{\pgfqpoint{7.825822in}{2.089818in}}{\pgfqpoint{7.827826in}{2.084981in}}{\pgfqpoint{7.831392in}{2.081414in}}%
\pgfpathcurveto{\pgfqpoint{7.834959in}{2.077848in}}{\pgfqpoint{7.839797in}{2.075844in}}{\pgfqpoint{7.844840in}{2.075844in}}%
\pgfpathclose%
\pgfusepath{fill}%
\end{pgfscope}%
\begin{pgfscope}%
\pgfpathrectangle{\pgfqpoint{6.572727in}{0.473000in}}{\pgfqpoint{4.227273in}{3.311000in}}%
\pgfusepath{clip}%
\pgfsetbuttcap%
\pgfsetroundjoin%
\definecolor{currentfill}{rgb}{0.127568,0.566949,0.550556}%
\pgfsetfillcolor{currentfill}%
\pgfsetfillopacity{0.700000}%
\pgfsetlinewidth{0.000000pt}%
\definecolor{currentstroke}{rgb}{0.000000,0.000000,0.000000}%
\pgfsetstrokecolor{currentstroke}%
\pgfsetstrokeopacity{0.700000}%
\pgfsetdash{}{0pt}%
\pgfpathmoveto{\pgfqpoint{8.142850in}{2.266696in}}%
\pgfpathcurveto{\pgfqpoint{8.147893in}{2.266696in}}{\pgfqpoint{8.152731in}{2.268700in}}{\pgfqpoint{8.156298in}{2.272266in}}%
\pgfpathcurveto{\pgfqpoint{8.159864in}{2.275833in}}{\pgfqpoint{8.161868in}{2.280670in}}{\pgfqpoint{8.161868in}{2.285714in}}%
\pgfpathcurveto{\pgfqpoint{8.161868in}{2.290758in}}{\pgfqpoint{8.159864in}{2.295595in}}{\pgfqpoint{8.156298in}{2.299162in}}%
\pgfpathcurveto{\pgfqpoint{8.152731in}{2.302728in}}{\pgfqpoint{8.147893in}{2.304732in}}{\pgfqpoint{8.142850in}{2.304732in}}%
\pgfpathcurveto{\pgfqpoint{8.137806in}{2.304732in}}{\pgfqpoint{8.132968in}{2.302728in}}{\pgfqpoint{8.129402in}{2.299162in}}%
\pgfpathcurveto{\pgfqpoint{8.125836in}{2.295595in}}{\pgfqpoint{8.123832in}{2.290758in}}{\pgfqpoint{8.123832in}{2.285714in}}%
\pgfpathcurveto{\pgfqpoint{8.123832in}{2.280670in}}{\pgfqpoint{8.125836in}{2.275833in}}{\pgfqpoint{8.129402in}{2.272266in}}%
\pgfpathcurveto{\pgfqpoint{8.132968in}{2.268700in}}{\pgfqpoint{8.137806in}{2.266696in}}{\pgfqpoint{8.142850in}{2.266696in}}%
\pgfpathclose%
\pgfusepath{fill}%
\end{pgfscope}%
\begin{pgfscope}%
\pgfpathrectangle{\pgfqpoint{6.572727in}{0.473000in}}{\pgfqpoint{4.227273in}{3.311000in}}%
\pgfusepath{clip}%
\pgfsetbuttcap%
\pgfsetroundjoin%
\definecolor{currentfill}{rgb}{0.127568,0.566949,0.550556}%
\pgfsetfillcolor{currentfill}%
\pgfsetfillopacity{0.700000}%
\pgfsetlinewidth{0.000000pt}%
\definecolor{currentstroke}{rgb}{0.000000,0.000000,0.000000}%
\pgfsetstrokecolor{currentstroke}%
\pgfsetstrokeopacity{0.700000}%
\pgfsetdash{}{0pt}%
\pgfpathmoveto{\pgfqpoint{8.365658in}{1.286659in}}%
\pgfpathcurveto{\pgfqpoint{8.370701in}{1.286659in}}{\pgfqpoint{8.375539in}{1.288663in}}{\pgfqpoint{8.379106in}{1.292229in}}%
\pgfpathcurveto{\pgfqpoint{8.382672in}{1.295796in}}{\pgfqpoint{8.384676in}{1.300633in}}{\pgfqpoint{8.384676in}{1.305677in}}%
\pgfpathcurveto{\pgfqpoint{8.384676in}{1.310721in}}{\pgfqpoint{8.382672in}{1.315558in}}{\pgfqpoint{8.379106in}{1.319125in}}%
\pgfpathcurveto{\pgfqpoint{8.375539in}{1.322691in}}{\pgfqpoint{8.370701in}{1.324695in}}{\pgfqpoint{8.365658in}{1.324695in}}%
\pgfpathcurveto{\pgfqpoint{8.360614in}{1.324695in}}{\pgfqpoint{8.355776in}{1.322691in}}{\pgfqpoint{8.352210in}{1.319125in}}%
\pgfpathcurveto{\pgfqpoint{8.348643in}{1.315558in}}{\pgfqpoint{8.346640in}{1.310721in}}{\pgfqpoint{8.346640in}{1.305677in}}%
\pgfpathcurveto{\pgfqpoint{8.346640in}{1.300633in}}{\pgfqpoint{8.348643in}{1.295796in}}{\pgfqpoint{8.352210in}{1.292229in}}%
\pgfpathcurveto{\pgfqpoint{8.355776in}{1.288663in}}{\pgfqpoint{8.360614in}{1.286659in}}{\pgfqpoint{8.365658in}{1.286659in}}%
\pgfpathclose%
\pgfusepath{fill}%
\end{pgfscope}%
\begin{pgfscope}%
\pgfpathrectangle{\pgfqpoint{6.572727in}{0.473000in}}{\pgfqpoint{4.227273in}{3.311000in}}%
\pgfusepath{clip}%
\pgfsetbuttcap%
\pgfsetroundjoin%
\definecolor{currentfill}{rgb}{0.127568,0.566949,0.550556}%
\pgfsetfillcolor{currentfill}%
\pgfsetfillopacity{0.700000}%
\pgfsetlinewidth{0.000000pt}%
\definecolor{currentstroke}{rgb}{0.000000,0.000000,0.000000}%
\pgfsetstrokecolor{currentstroke}%
\pgfsetstrokeopacity{0.700000}%
\pgfsetdash{}{0pt}%
\pgfpathmoveto{\pgfqpoint{7.651325in}{0.953499in}}%
\pgfpathcurveto{\pgfqpoint{7.656369in}{0.953499in}}{\pgfqpoint{7.661207in}{0.955503in}}{\pgfqpoint{7.664773in}{0.959070in}}%
\pgfpathcurveto{\pgfqpoint{7.668340in}{0.962636in}}{\pgfqpoint{7.670343in}{0.967474in}}{\pgfqpoint{7.670343in}{0.972517in}}%
\pgfpathcurveto{\pgfqpoint{7.670343in}{0.977561in}}{\pgfqpoint{7.668340in}{0.982399in}}{\pgfqpoint{7.664773in}{0.985965in}}%
\pgfpathcurveto{\pgfqpoint{7.661207in}{0.989532in}}{\pgfqpoint{7.656369in}{0.991536in}}{\pgfqpoint{7.651325in}{0.991536in}}%
\pgfpathcurveto{\pgfqpoint{7.646282in}{0.991536in}}{\pgfqpoint{7.641444in}{0.989532in}}{\pgfqpoint{7.637877in}{0.985965in}}%
\pgfpathcurveto{\pgfqpoint{7.634311in}{0.982399in}}{\pgfqpoint{7.632307in}{0.977561in}}{\pgfqpoint{7.632307in}{0.972517in}}%
\pgfpathcurveto{\pgfqpoint{7.632307in}{0.967474in}}{\pgfqpoint{7.634311in}{0.962636in}}{\pgfqpoint{7.637877in}{0.959070in}}%
\pgfpathcurveto{\pgfqpoint{7.641444in}{0.955503in}}{\pgfqpoint{7.646282in}{0.953499in}}{\pgfqpoint{7.651325in}{0.953499in}}%
\pgfpathclose%
\pgfusepath{fill}%
\end{pgfscope}%
\begin{pgfscope}%
\pgfpathrectangle{\pgfqpoint{6.572727in}{0.473000in}}{\pgfqpoint{4.227273in}{3.311000in}}%
\pgfusepath{clip}%
\pgfsetbuttcap%
\pgfsetroundjoin%
\definecolor{currentfill}{rgb}{0.993248,0.906157,0.143936}%
\pgfsetfillcolor{currentfill}%
\pgfsetfillopacity{0.700000}%
\pgfsetlinewidth{0.000000pt}%
\definecolor{currentstroke}{rgb}{0.000000,0.000000,0.000000}%
\pgfsetstrokecolor{currentstroke}%
\pgfsetstrokeopacity{0.700000}%
\pgfsetdash{}{0pt}%
\pgfpathmoveto{\pgfqpoint{9.634312in}{1.521981in}}%
\pgfpathcurveto{\pgfqpoint{9.639356in}{1.521981in}}{\pgfqpoint{9.644193in}{1.523985in}}{\pgfqpoint{9.647760in}{1.527552in}}%
\pgfpathcurveto{\pgfqpoint{9.651326in}{1.531118in}}{\pgfqpoint{9.653330in}{1.535956in}}{\pgfqpoint{9.653330in}{1.541000in}}%
\pgfpathcurveto{\pgfqpoint{9.653330in}{1.546043in}}{\pgfqpoint{9.651326in}{1.550881in}}{\pgfqpoint{9.647760in}{1.554447in}}%
\pgfpathcurveto{\pgfqpoint{9.644193in}{1.558014in}}{\pgfqpoint{9.639356in}{1.560018in}}{\pgfqpoint{9.634312in}{1.560018in}}%
\pgfpathcurveto{\pgfqpoint{9.629268in}{1.560018in}}{\pgfqpoint{9.624430in}{1.558014in}}{\pgfqpoint{9.620864in}{1.554447in}}%
\pgfpathcurveto{\pgfqpoint{9.617298in}{1.550881in}}{\pgfqpoint{9.615294in}{1.546043in}}{\pgfqpoint{9.615294in}{1.541000in}}%
\pgfpathcurveto{\pgfqpoint{9.615294in}{1.535956in}}{\pgfqpoint{9.617298in}{1.531118in}}{\pgfqpoint{9.620864in}{1.527552in}}%
\pgfpathcurveto{\pgfqpoint{9.624430in}{1.523985in}}{\pgfqpoint{9.629268in}{1.521981in}}{\pgfqpoint{9.634312in}{1.521981in}}%
\pgfpathclose%
\pgfusepath{fill}%
\end{pgfscope}%
\begin{pgfscope}%
\pgfpathrectangle{\pgfqpoint{6.572727in}{0.473000in}}{\pgfqpoint{4.227273in}{3.311000in}}%
\pgfusepath{clip}%
\pgfsetbuttcap%
\pgfsetroundjoin%
\definecolor{currentfill}{rgb}{0.127568,0.566949,0.550556}%
\pgfsetfillcolor{currentfill}%
\pgfsetfillopacity{0.700000}%
\pgfsetlinewidth{0.000000pt}%
\definecolor{currentstroke}{rgb}{0.000000,0.000000,0.000000}%
\pgfsetstrokecolor{currentstroke}%
\pgfsetstrokeopacity{0.700000}%
\pgfsetdash{}{0pt}%
\pgfpathmoveto{\pgfqpoint{7.958342in}{1.656994in}}%
\pgfpathcurveto{\pgfqpoint{7.963386in}{1.656994in}}{\pgfqpoint{7.968224in}{1.658998in}}{\pgfqpoint{7.971790in}{1.662565in}}%
\pgfpathcurveto{\pgfqpoint{7.975356in}{1.666131in}}{\pgfqpoint{7.977360in}{1.670969in}}{\pgfqpoint{7.977360in}{1.676013in}}%
\pgfpathcurveto{\pgfqpoint{7.977360in}{1.681056in}}{\pgfqpoint{7.975356in}{1.685894in}}{\pgfqpoint{7.971790in}{1.689460in}}%
\pgfpathcurveto{\pgfqpoint{7.968224in}{1.693027in}}{\pgfqpoint{7.963386in}{1.695031in}}{\pgfqpoint{7.958342in}{1.695031in}}%
\pgfpathcurveto{\pgfqpoint{7.953298in}{1.695031in}}{\pgfqpoint{7.948461in}{1.693027in}}{\pgfqpoint{7.944894in}{1.689460in}}%
\pgfpathcurveto{\pgfqpoint{7.941328in}{1.685894in}}{\pgfqpoint{7.939324in}{1.681056in}}{\pgfqpoint{7.939324in}{1.676013in}}%
\pgfpathcurveto{\pgfqpoint{7.939324in}{1.670969in}}{\pgfqpoint{7.941328in}{1.666131in}}{\pgfqpoint{7.944894in}{1.662565in}}%
\pgfpathcurveto{\pgfqpoint{7.948461in}{1.658998in}}{\pgfqpoint{7.953298in}{1.656994in}}{\pgfqpoint{7.958342in}{1.656994in}}%
\pgfpathclose%
\pgfusepath{fill}%
\end{pgfscope}%
\begin{pgfscope}%
\pgfpathrectangle{\pgfqpoint{6.572727in}{0.473000in}}{\pgfqpoint{4.227273in}{3.311000in}}%
\pgfusepath{clip}%
\pgfsetbuttcap%
\pgfsetroundjoin%
\definecolor{currentfill}{rgb}{0.993248,0.906157,0.143936}%
\pgfsetfillcolor{currentfill}%
\pgfsetfillopacity{0.700000}%
\pgfsetlinewidth{0.000000pt}%
\definecolor{currentstroke}{rgb}{0.000000,0.000000,0.000000}%
\pgfsetstrokecolor{currentstroke}%
\pgfsetstrokeopacity{0.700000}%
\pgfsetdash{}{0pt}%
\pgfpathmoveto{\pgfqpoint{9.283712in}{1.067245in}}%
\pgfpathcurveto{\pgfqpoint{9.288756in}{1.067245in}}{\pgfqpoint{9.293594in}{1.069248in}}{\pgfqpoint{9.297160in}{1.072815in}}%
\pgfpathcurveto{\pgfqpoint{9.300726in}{1.076381in}}{\pgfqpoint{9.302730in}{1.081219in}}{\pgfqpoint{9.302730in}{1.086263in}}%
\pgfpathcurveto{\pgfqpoint{9.302730in}{1.091306in}}{\pgfqpoint{9.300726in}{1.096144in}}{\pgfqpoint{9.297160in}{1.099711in}}%
\pgfpathcurveto{\pgfqpoint{9.293594in}{1.103277in}}{\pgfqpoint{9.288756in}{1.105281in}}{\pgfqpoint{9.283712in}{1.105281in}}%
\pgfpathcurveto{\pgfqpoint{9.278668in}{1.105281in}}{\pgfqpoint{9.273831in}{1.103277in}}{\pgfqpoint{9.270264in}{1.099711in}}%
\pgfpathcurveto{\pgfqpoint{9.266698in}{1.096144in}}{\pgfqpoint{9.264694in}{1.091306in}}{\pgfqpoint{9.264694in}{1.086263in}}%
\pgfpathcurveto{\pgfqpoint{9.264694in}{1.081219in}}{\pgfqpoint{9.266698in}{1.076381in}}{\pgfqpoint{9.270264in}{1.072815in}}%
\pgfpathcurveto{\pgfqpoint{9.273831in}{1.069248in}}{\pgfqpoint{9.278668in}{1.067245in}}{\pgfqpoint{9.283712in}{1.067245in}}%
\pgfpathclose%
\pgfusepath{fill}%
\end{pgfscope}%
\begin{pgfscope}%
\pgfpathrectangle{\pgfqpoint{6.572727in}{0.473000in}}{\pgfqpoint{4.227273in}{3.311000in}}%
\pgfusepath{clip}%
\pgfsetbuttcap%
\pgfsetroundjoin%
\definecolor{currentfill}{rgb}{0.993248,0.906157,0.143936}%
\pgfsetfillcolor{currentfill}%
\pgfsetfillopacity{0.700000}%
\pgfsetlinewidth{0.000000pt}%
\definecolor{currentstroke}{rgb}{0.000000,0.000000,0.000000}%
\pgfsetstrokecolor{currentstroke}%
\pgfsetstrokeopacity{0.700000}%
\pgfsetdash{}{0pt}%
\pgfpathmoveto{\pgfqpoint{9.519123in}{1.165315in}}%
\pgfpathcurveto{\pgfqpoint{9.524167in}{1.165315in}}{\pgfqpoint{9.529005in}{1.167319in}}{\pgfqpoint{9.532571in}{1.170885in}}%
\pgfpathcurveto{\pgfqpoint{9.536138in}{1.174451in}}{\pgfqpoint{9.538142in}{1.179289in}}{\pgfqpoint{9.538142in}{1.184333in}}%
\pgfpathcurveto{\pgfqpoint{9.538142in}{1.189376in}}{\pgfqpoint{9.536138in}{1.194214in}}{\pgfqpoint{9.532571in}{1.197781in}}%
\pgfpathcurveto{\pgfqpoint{9.529005in}{1.201347in}}{\pgfqpoint{9.524167in}{1.203351in}}{\pgfqpoint{9.519123in}{1.203351in}}%
\pgfpathcurveto{\pgfqpoint{9.514080in}{1.203351in}}{\pgfqpoint{9.509242in}{1.201347in}}{\pgfqpoint{9.505676in}{1.197781in}}%
\pgfpathcurveto{\pgfqpoint{9.502109in}{1.194214in}}{\pgfqpoint{9.500105in}{1.189376in}}{\pgfqpoint{9.500105in}{1.184333in}}%
\pgfpathcurveto{\pgfqpoint{9.500105in}{1.179289in}}{\pgfqpoint{9.502109in}{1.174451in}}{\pgfqpoint{9.505676in}{1.170885in}}%
\pgfpathcurveto{\pgfqpoint{9.509242in}{1.167319in}}{\pgfqpoint{9.514080in}{1.165315in}}{\pgfqpoint{9.519123in}{1.165315in}}%
\pgfpathclose%
\pgfusepath{fill}%
\end{pgfscope}%
\begin{pgfscope}%
\pgfpathrectangle{\pgfqpoint{6.572727in}{0.473000in}}{\pgfqpoint{4.227273in}{3.311000in}}%
\pgfusepath{clip}%
\pgfsetbuttcap%
\pgfsetroundjoin%
\definecolor{currentfill}{rgb}{0.993248,0.906157,0.143936}%
\pgfsetfillcolor{currentfill}%
\pgfsetfillopacity{0.700000}%
\pgfsetlinewidth{0.000000pt}%
\definecolor{currentstroke}{rgb}{0.000000,0.000000,0.000000}%
\pgfsetstrokecolor{currentstroke}%
\pgfsetstrokeopacity{0.700000}%
\pgfsetdash{}{0pt}%
\pgfpathmoveto{\pgfqpoint{9.659079in}{1.618766in}}%
\pgfpathcurveto{\pgfqpoint{9.664123in}{1.618766in}}{\pgfqpoint{9.668961in}{1.620770in}}{\pgfqpoint{9.672527in}{1.624336in}}%
\pgfpathcurveto{\pgfqpoint{9.676094in}{1.627903in}}{\pgfqpoint{9.678097in}{1.632741in}}{\pgfqpoint{9.678097in}{1.637784in}}%
\pgfpathcurveto{\pgfqpoint{9.678097in}{1.642828in}}{\pgfqpoint{9.676094in}{1.647666in}}{\pgfqpoint{9.672527in}{1.651232in}}%
\pgfpathcurveto{\pgfqpoint{9.668961in}{1.654799in}}{\pgfqpoint{9.664123in}{1.656802in}}{\pgfqpoint{9.659079in}{1.656802in}}%
\pgfpathcurveto{\pgfqpoint{9.654036in}{1.656802in}}{\pgfqpoint{9.649198in}{1.654799in}}{\pgfqpoint{9.645631in}{1.651232in}}%
\pgfpathcurveto{\pgfqpoint{9.642065in}{1.647666in}}{\pgfqpoint{9.640061in}{1.642828in}}{\pgfqpoint{9.640061in}{1.637784in}}%
\pgfpathcurveto{\pgfqpoint{9.640061in}{1.632741in}}{\pgfqpoint{9.642065in}{1.627903in}}{\pgfqpoint{9.645631in}{1.624336in}}%
\pgfpathcurveto{\pgfqpoint{9.649198in}{1.620770in}}{\pgfqpoint{9.654036in}{1.618766in}}{\pgfqpoint{9.659079in}{1.618766in}}%
\pgfpathclose%
\pgfusepath{fill}%
\end{pgfscope}%
\begin{pgfscope}%
\pgfpathrectangle{\pgfqpoint{6.572727in}{0.473000in}}{\pgfqpoint{4.227273in}{3.311000in}}%
\pgfusepath{clip}%
\pgfsetbuttcap%
\pgfsetroundjoin%
\definecolor{currentfill}{rgb}{0.993248,0.906157,0.143936}%
\pgfsetfillcolor{currentfill}%
\pgfsetfillopacity{0.700000}%
\pgfsetlinewidth{0.000000pt}%
\definecolor{currentstroke}{rgb}{0.000000,0.000000,0.000000}%
\pgfsetstrokecolor{currentstroke}%
\pgfsetstrokeopacity{0.700000}%
\pgfsetdash{}{0pt}%
\pgfpathmoveto{\pgfqpoint{9.654489in}{2.285775in}}%
\pgfpathcurveto{\pgfqpoint{9.659533in}{2.285775in}}{\pgfqpoint{9.664371in}{2.287779in}}{\pgfqpoint{9.667937in}{2.291346in}}%
\pgfpathcurveto{\pgfqpoint{9.671504in}{2.294912in}}{\pgfqpoint{9.673508in}{2.299750in}}{\pgfqpoint{9.673508in}{2.304794in}}%
\pgfpathcurveto{\pgfqpoint{9.673508in}{2.309837in}}{\pgfqpoint{9.671504in}{2.314675in}}{\pgfqpoint{9.667937in}{2.318241in}}%
\pgfpathcurveto{\pgfqpoint{9.664371in}{2.321808in}}{\pgfqpoint{9.659533in}{2.323812in}}{\pgfqpoint{9.654489in}{2.323812in}}%
\pgfpathcurveto{\pgfqpoint{9.649446in}{2.323812in}}{\pgfqpoint{9.644608in}{2.321808in}}{\pgfqpoint{9.641041in}{2.318241in}}%
\pgfpathcurveto{\pgfqpoint{9.637475in}{2.314675in}}{\pgfqpoint{9.635471in}{2.309837in}}{\pgfqpoint{9.635471in}{2.304794in}}%
\pgfpathcurveto{\pgfqpoint{9.635471in}{2.299750in}}{\pgfqpoint{9.637475in}{2.294912in}}{\pgfqpoint{9.641041in}{2.291346in}}%
\pgfpathcurveto{\pgfqpoint{9.644608in}{2.287779in}}{\pgfqpoint{9.649446in}{2.285775in}}{\pgfqpoint{9.654489in}{2.285775in}}%
\pgfpathclose%
\pgfusepath{fill}%
\end{pgfscope}%
\begin{pgfscope}%
\pgfpathrectangle{\pgfqpoint{6.572727in}{0.473000in}}{\pgfqpoint{4.227273in}{3.311000in}}%
\pgfusepath{clip}%
\pgfsetbuttcap%
\pgfsetroundjoin%
\definecolor{currentfill}{rgb}{0.993248,0.906157,0.143936}%
\pgfsetfillcolor{currentfill}%
\pgfsetfillopacity{0.700000}%
\pgfsetlinewidth{0.000000pt}%
\definecolor{currentstroke}{rgb}{0.000000,0.000000,0.000000}%
\pgfsetstrokecolor{currentstroke}%
\pgfsetstrokeopacity{0.700000}%
\pgfsetdash{}{0pt}%
\pgfpathmoveto{\pgfqpoint{9.831714in}{1.441454in}}%
\pgfpathcurveto{\pgfqpoint{9.836757in}{1.441454in}}{\pgfqpoint{9.841595in}{1.443458in}}{\pgfqpoint{9.845161in}{1.447024in}}%
\pgfpathcurveto{\pgfqpoint{9.848728in}{1.450591in}}{\pgfqpoint{9.850732in}{1.455429in}}{\pgfqpoint{9.850732in}{1.460472in}}%
\pgfpathcurveto{\pgfqpoint{9.850732in}{1.465516in}}{\pgfqpoint{9.848728in}{1.470354in}}{\pgfqpoint{9.845161in}{1.473920in}}%
\pgfpathcurveto{\pgfqpoint{9.841595in}{1.477486in}}{\pgfqpoint{9.836757in}{1.479490in}}{\pgfqpoint{9.831714in}{1.479490in}}%
\pgfpathcurveto{\pgfqpoint{9.826670in}{1.479490in}}{\pgfqpoint{9.821832in}{1.477486in}}{\pgfqpoint{9.818266in}{1.473920in}}%
\pgfpathcurveto{\pgfqpoint{9.814699in}{1.470354in}}{\pgfqpoint{9.812695in}{1.465516in}}{\pgfqpoint{9.812695in}{1.460472in}}%
\pgfpathcurveto{\pgfqpoint{9.812695in}{1.455429in}}{\pgfqpoint{9.814699in}{1.450591in}}{\pgfqpoint{9.818266in}{1.447024in}}%
\pgfpathcurveto{\pgfqpoint{9.821832in}{1.443458in}}{\pgfqpoint{9.826670in}{1.441454in}}{\pgfqpoint{9.831714in}{1.441454in}}%
\pgfpathclose%
\pgfusepath{fill}%
\end{pgfscope}%
\begin{pgfscope}%
\pgfpathrectangle{\pgfqpoint{6.572727in}{0.473000in}}{\pgfqpoint{4.227273in}{3.311000in}}%
\pgfusepath{clip}%
\pgfsetbuttcap%
\pgfsetroundjoin%
\definecolor{currentfill}{rgb}{0.993248,0.906157,0.143936}%
\pgfsetfillcolor{currentfill}%
\pgfsetfillopacity{0.700000}%
\pgfsetlinewidth{0.000000pt}%
\definecolor{currentstroke}{rgb}{0.000000,0.000000,0.000000}%
\pgfsetstrokecolor{currentstroke}%
\pgfsetstrokeopacity{0.700000}%
\pgfsetdash{}{0pt}%
\pgfpathmoveto{\pgfqpoint{9.821301in}{2.319087in}}%
\pgfpathcurveto{\pgfqpoint{9.826345in}{2.319087in}}{\pgfqpoint{9.831182in}{2.321091in}}{\pgfqpoint{9.834749in}{2.324657in}}%
\pgfpathcurveto{\pgfqpoint{9.838315in}{2.328224in}}{\pgfqpoint{9.840319in}{2.333061in}}{\pgfqpoint{9.840319in}{2.338105in}}%
\pgfpathcurveto{\pgfqpoint{9.840319in}{2.343149in}}{\pgfqpoint{9.838315in}{2.347986in}}{\pgfqpoint{9.834749in}{2.351553in}}%
\pgfpathcurveto{\pgfqpoint{9.831182in}{2.355119in}}{\pgfqpoint{9.826345in}{2.357123in}}{\pgfqpoint{9.821301in}{2.357123in}}%
\pgfpathcurveto{\pgfqpoint{9.816257in}{2.357123in}}{\pgfqpoint{9.811420in}{2.355119in}}{\pgfqpoint{9.807853in}{2.351553in}}%
\pgfpathcurveto{\pgfqpoint{9.804287in}{2.347986in}}{\pgfqpoint{9.802283in}{2.343149in}}{\pgfqpoint{9.802283in}{2.338105in}}%
\pgfpathcurveto{\pgfqpoint{9.802283in}{2.333061in}}{\pgfqpoint{9.804287in}{2.328224in}}{\pgfqpoint{9.807853in}{2.324657in}}%
\pgfpathcurveto{\pgfqpoint{9.811420in}{2.321091in}}{\pgfqpoint{9.816257in}{2.319087in}}{\pgfqpoint{9.821301in}{2.319087in}}%
\pgfpathclose%
\pgfusepath{fill}%
\end{pgfscope}%
\begin{pgfscope}%
\pgfpathrectangle{\pgfqpoint{6.572727in}{0.473000in}}{\pgfqpoint{4.227273in}{3.311000in}}%
\pgfusepath{clip}%
\pgfsetbuttcap%
\pgfsetroundjoin%
\definecolor{currentfill}{rgb}{0.127568,0.566949,0.550556}%
\pgfsetfillcolor{currentfill}%
\pgfsetfillopacity{0.700000}%
\pgfsetlinewidth{0.000000pt}%
\definecolor{currentstroke}{rgb}{0.000000,0.000000,0.000000}%
\pgfsetstrokecolor{currentstroke}%
\pgfsetstrokeopacity{0.700000}%
\pgfsetdash{}{0pt}%
\pgfpathmoveto{\pgfqpoint{8.129570in}{1.376942in}}%
\pgfpathcurveto{\pgfqpoint{8.134614in}{1.376942in}}{\pgfqpoint{8.139452in}{1.378946in}}{\pgfqpoint{8.143018in}{1.382512in}}%
\pgfpathcurveto{\pgfqpoint{8.146585in}{1.386079in}}{\pgfqpoint{8.148589in}{1.390917in}}{\pgfqpoint{8.148589in}{1.395960in}}%
\pgfpathcurveto{\pgfqpoint{8.148589in}{1.401004in}}{\pgfqpoint{8.146585in}{1.405842in}}{\pgfqpoint{8.143018in}{1.409408in}}%
\pgfpathcurveto{\pgfqpoint{8.139452in}{1.412974in}}{\pgfqpoint{8.134614in}{1.414978in}}{\pgfqpoint{8.129570in}{1.414978in}}%
\pgfpathcurveto{\pgfqpoint{8.124527in}{1.414978in}}{\pgfqpoint{8.119689in}{1.412974in}}{\pgfqpoint{8.116123in}{1.409408in}}%
\pgfpathcurveto{\pgfqpoint{8.112556in}{1.405842in}}{\pgfqpoint{8.110552in}{1.401004in}}{\pgfqpoint{8.110552in}{1.395960in}}%
\pgfpathcurveto{\pgfqpoint{8.110552in}{1.390917in}}{\pgfqpoint{8.112556in}{1.386079in}}{\pgfqpoint{8.116123in}{1.382512in}}%
\pgfpathcurveto{\pgfqpoint{8.119689in}{1.378946in}}{\pgfqpoint{8.124527in}{1.376942in}}{\pgfqpoint{8.129570in}{1.376942in}}%
\pgfpathclose%
\pgfusepath{fill}%
\end{pgfscope}%
\begin{pgfscope}%
\pgfpathrectangle{\pgfqpoint{6.572727in}{0.473000in}}{\pgfqpoint{4.227273in}{3.311000in}}%
\pgfusepath{clip}%
\pgfsetbuttcap%
\pgfsetroundjoin%
\definecolor{currentfill}{rgb}{0.993248,0.906157,0.143936}%
\pgfsetfillcolor{currentfill}%
\pgfsetfillopacity{0.700000}%
\pgfsetlinewidth{0.000000pt}%
\definecolor{currentstroke}{rgb}{0.000000,0.000000,0.000000}%
\pgfsetstrokecolor{currentstroke}%
\pgfsetstrokeopacity{0.700000}%
\pgfsetdash{}{0pt}%
\pgfpathmoveto{\pgfqpoint{9.330031in}{2.070713in}}%
\pgfpathcurveto{\pgfqpoint{9.335075in}{2.070713in}}{\pgfqpoint{9.339912in}{2.072716in}}{\pgfqpoint{9.343479in}{2.076283in}}%
\pgfpathcurveto{\pgfqpoint{9.347045in}{2.079849in}}{\pgfqpoint{9.349049in}{2.084687in}}{\pgfqpoint{9.349049in}{2.089731in}}%
\pgfpathcurveto{\pgfqpoint{9.349049in}{2.094774in}}{\pgfqpoint{9.347045in}{2.099612in}}{\pgfqpoint{9.343479in}{2.103179in}}%
\pgfpathcurveto{\pgfqpoint{9.339912in}{2.106745in}}{\pgfqpoint{9.335075in}{2.108749in}}{\pgfqpoint{9.330031in}{2.108749in}}%
\pgfpathcurveto{\pgfqpoint{9.324987in}{2.108749in}}{\pgfqpoint{9.320150in}{2.106745in}}{\pgfqpoint{9.316583in}{2.103179in}}%
\pgfpathcurveto{\pgfqpoint{9.313017in}{2.099612in}}{\pgfqpoint{9.311013in}{2.094774in}}{\pgfqpoint{9.311013in}{2.089731in}}%
\pgfpathcurveto{\pgfqpoint{9.311013in}{2.084687in}}{\pgfqpoint{9.313017in}{2.079849in}}{\pgfqpoint{9.316583in}{2.076283in}}%
\pgfpathcurveto{\pgfqpoint{9.320150in}{2.072716in}}{\pgfqpoint{9.324987in}{2.070713in}}{\pgfqpoint{9.330031in}{2.070713in}}%
\pgfpathclose%
\pgfusepath{fill}%
\end{pgfscope}%
\begin{pgfscope}%
\pgfpathrectangle{\pgfqpoint{6.572727in}{0.473000in}}{\pgfqpoint{4.227273in}{3.311000in}}%
\pgfusepath{clip}%
\pgfsetbuttcap%
\pgfsetroundjoin%
\definecolor{currentfill}{rgb}{0.127568,0.566949,0.550556}%
\pgfsetfillcolor{currentfill}%
\pgfsetfillopacity{0.700000}%
\pgfsetlinewidth{0.000000pt}%
\definecolor{currentstroke}{rgb}{0.000000,0.000000,0.000000}%
\pgfsetstrokecolor{currentstroke}%
\pgfsetstrokeopacity{0.700000}%
\pgfsetdash{}{0pt}%
\pgfpathmoveto{\pgfqpoint{8.165087in}{2.956391in}}%
\pgfpathcurveto{\pgfqpoint{8.170131in}{2.956391in}}{\pgfqpoint{8.174969in}{2.958395in}}{\pgfqpoint{8.178535in}{2.961961in}}%
\pgfpathcurveto{\pgfqpoint{8.182101in}{2.965527in}}{\pgfqpoint{8.184105in}{2.970365in}}{\pgfqpoint{8.184105in}{2.975409in}}%
\pgfpathcurveto{\pgfqpoint{8.184105in}{2.980452in}}{\pgfqpoint{8.182101in}{2.985290in}}{\pgfqpoint{8.178535in}{2.988857in}}%
\pgfpathcurveto{\pgfqpoint{8.174969in}{2.992423in}}{\pgfqpoint{8.170131in}{2.994427in}}{\pgfqpoint{8.165087in}{2.994427in}}%
\pgfpathcurveto{\pgfqpoint{8.160043in}{2.994427in}}{\pgfqpoint{8.155206in}{2.992423in}}{\pgfqpoint{8.151639in}{2.988857in}}%
\pgfpathcurveto{\pgfqpoint{8.148073in}{2.985290in}}{\pgfqpoint{8.146069in}{2.980452in}}{\pgfqpoint{8.146069in}{2.975409in}}%
\pgfpathcurveto{\pgfqpoint{8.146069in}{2.970365in}}{\pgfqpoint{8.148073in}{2.965527in}}{\pgfqpoint{8.151639in}{2.961961in}}%
\pgfpathcurveto{\pgfqpoint{8.155206in}{2.958395in}}{\pgfqpoint{8.160043in}{2.956391in}}{\pgfqpoint{8.165087in}{2.956391in}}%
\pgfpathclose%
\pgfusepath{fill}%
\end{pgfscope}%
\begin{pgfscope}%
\pgfpathrectangle{\pgfqpoint{6.572727in}{0.473000in}}{\pgfqpoint{4.227273in}{3.311000in}}%
\pgfusepath{clip}%
\pgfsetbuttcap%
\pgfsetroundjoin%
\definecolor{currentfill}{rgb}{0.127568,0.566949,0.550556}%
\pgfsetfillcolor{currentfill}%
\pgfsetfillopacity{0.700000}%
\pgfsetlinewidth{0.000000pt}%
\definecolor{currentstroke}{rgb}{0.000000,0.000000,0.000000}%
\pgfsetstrokecolor{currentstroke}%
\pgfsetstrokeopacity{0.700000}%
\pgfsetdash{}{0pt}%
\pgfpathmoveto{\pgfqpoint{7.494331in}{1.374946in}}%
\pgfpathcurveto{\pgfqpoint{7.499374in}{1.374946in}}{\pgfqpoint{7.504212in}{1.376950in}}{\pgfqpoint{7.507778in}{1.380517in}}%
\pgfpathcurveto{\pgfqpoint{7.511345in}{1.384083in}}{\pgfqpoint{7.513349in}{1.388921in}}{\pgfqpoint{7.513349in}{1.393964in}}%
\pgfpathcurveto{\pgfqpoint{7.513349in}{1.399008in}}{\pgfqpoint{7.511345in}{1.403846in}}{\pgfqpoint{7.507778in}{1.407412in}}%
\pgfpathcurveto{\pgfqpoint{7.504212in}{1.410979in}}{\pgfqpoint{7.499374in}{1.412983in}}{\pgfqpoint{7.494331in}{1.412983in}}%
\pgfpathcurveto{\pgfqpoint{7.489287in}{1.412983in}}{\pgfqpoint{7.484449in}{1.410979in}}{\pgfqpoint{7.480883in}{1.407412in}}%
\pgfpathcurveto{\pgfqpoint{7.477316in}{1.403846in}}{\pgfqpoint{7.475312in}{1.399008in}}{\pgfqpoint{7.475312in}{1.393964in}}%
\pgfpathcurveto{\pgfqpoint{7.475312in}{1.388921in}}{\pgfqpoint{7.477316in}{1.384083in}}{\pgfqpoint{7.480883in}{1.380517in}}%
\pgfpathcurveto{\pgfqpoint{7.484449in}{1.376950in}}{\pgfqpoint{7.489287in}{1.374946in}}{\pgfqpoint{7.494331in}{1.374946in}}%
\pgfpathclose%
\pgfusepath{fill}%
\end{pgfscope}%
\begin{pgfscope}%
\pgfpathrectangle{\pgfqpoint{6.572727in}{0.473000in}}{\pgfqpoint{4.227273in}{3.311000in}}%
\pgfusepath{clip}%
\pgfsetbuttcap%
\pgfsetroundjoin%
\definecolor{currentfill}{rgb}{0.127568,0.566949,0.550556}%
\pgfsetfillcolor{currentfill}%
\pgfsetfillopacity{0.700000}%
\pgfsetlinewidth{0.000000pt}%
\definecolor{currentstroke}{rgb}{0.000000,0.000000,0.000000}%
\pgfsetstrokecolor{currentstroke}%
\pgfsetstrokeopacity{0.700000}%
\pgfsetdash{}{0pt}%
\pgfpathmoveto{\pgfqpoint{7.507483in}{1.653671in}}%
\pgfpathcurveto{\pgfqpoint{7.512527in}{1.653671in}}{\pgfqpoint{7.517365in}{1.655675in}}{\pgfqpoint{7.520931in}{1.659242in}}%
\pgfpathcurveto{\pgfqpoint{7.524498in}{1.662808in}}{\pgfqpoint{7.526502in}{1.667646in}}{\pgfqpoint{7.526502in}{1.672689in}}%
\pgfpathcurveto{\pgfqpoint{7.526502in}{1.677733in}}{\pgfqpoint{7.524498in}{1.682571in}}{\pgfqpoint{7.520931in}{1.686137in}}%
\pgfpathcurveto{\pgfqpoint{7.517365in}{1.689704in}}{\pgfqpoint{7.512527in}{1.691708in}}{\pgfqpoint{7.507483in}{1.691708in}}%
\pgfpathcurveto{\pgfqpoint{7.502440in}{1.691708in}}{\pgfqpoint{7.497602in}{1.689704in}}{\pgfqpoint{7.494036in}{1.686137in}}%
\pgfpathcurveto{\pgfqpoint{7.490469in}{1.682571in}}{\pgfqpoint{7.488465in}{1.677733in}}{\pgfqpoint{7.488465in}{1.672689in}}%
\pgfpathcurveto{\pgfqpoint{7.488465in}{1.667646in}}{\pgfqpoint{7.490469in}{1.662808in}}{\pgfqpoint{7.494036in}{1.659242in}}%
\pgfpathcurveto{\pgfqpoint{7.497602in}{1.655675in}}{\pgfqpoint{7.502440in}{1.653671in}}{\pgfqpoint{7.507483in}{1.653671in}}%
\pgfpathclose%
\pgfusepath{fill}%
\end{pgfscope}%
\begin{pgfscope}%
\pgfpathrectangle{\pgfqpoint{6.572727in}{0.473000in}}{\pgfqpoint{4.227273in}{3.311000in}}%
\pgfusepath{clip}%
\pgfsetbuttcap%
\pgfsetroundjoin%
\definecolor{currentfill}{rgb}{0.993248,0.906157,0.143936}%
\pgfsetfillcolor{currentfill}%
\pgfsetfillopacity{0.700000}%
\pgfsetlinewidth{0.000000pt}%
\definecolor{currentstroke}{rgb}{0.000000,0.000000,0.000000}%
\pgfsetstrokecolor{currentstroke}%
\pgfsetstrokeopacity{0.700000}%
\pgfsetdash{}{0pt}%
\pgfpathmoveto{\pgfqpoint{10.420660in}{1.774299in}}%
\pgfpathcurveto{\pgfqpoint{10.425704in}{1.774299in}}{\pgfqpoint{10.430541in}{1.776303in}}{\pgfqpoint{10.434108in}{1.779869in}}%
\pgfpathcurveto{\pgfqpoint{10.437674in}{1.783436in}}{\pgfqpoint{10.439678in}{1.788273in}}{\pgfqpoint{10.439678in}{1.793317in}}%
\pgfpathcurveto{\pgfqpoint{10.439678in}{1.798361in}}{\pgfqpoint{10.437674in}{1.803198in}}{\pgfqpoint{10.434108in}{1.806765in}}%
\pgfpathcurveto{\pgfqpoint{10.430541in}{1.810331in}}{\pgfqpoint{10.425704in}{1.812335in}}{\pgfqpoint{10.420660in}{1.812335in}}%
\pgfpathcurveto{\pgfqpoint{10.415616in}{1.812335in}}{\pgfqpoint{10.410778in}{1.810331in}}{\pgfqpoint{10.407212in}{1.806765in}}%
\pgfpathcurveto{\pgfqpoint{10.403646in}{1.803198in}}{\pgfqpoint{10.401642in}{1.798361in}}{\pgfqpoint{10.401642in}{1.793317in}}%
\pgfpathcurveto{\pgfqpoint{10.401642in}{1.788273in}}{\pgfqpoint{10.403646in}{1.783436in}}{\pgfqpoint{10.407212in}{1.779869in}}%
\pgfpathcurveto{\pgfqpoint{10.410778in}{1.776303in}}{\pgfqpoint{10.415616in}{1.774299in}}{\pgfqpoint{10.420660in}{1.774299in}}%
\pgfpathclose%
\pgfusepath{fill}%
\end{pgfscope}%
\begin{pgfscope}%
\pgfpathrectangle{\pgfqpoint{6.572727in}{0.473000in}}{\pgfqpoint{4.227273in}{3.311000in}}%
\pgfusepath{clip}%
\pgfsetbuttcap%
\pgfsetroundjoin%
\definecolor{currentfill}{rgb}{0.993248,0.906157,0.143936}%
\pgfsetfillcolor{currentfill}%
\pgfsetfillopacity{0.700000}%
\pgfsetlinewidth{0.000000pt}%
\definecolor{currentstroke}{rgb}{0.000000,0.000000,0.000000}%
\pgfsetstrokecolor{currentstroke}%
\pgfsetstrokeopacity{0.700000}%
\pgfsetdash{}{0pt}%
\pgfpathmoveto{\pgfqpoint{9.917552in}{0.935030in}}%
\pgfpathcurveto{\pgfqpoint{9.922596in}{0.935030in}}{\pgfqpoint{9.927433in}{0.937034in}}{\pgfqpoint{9.931000in}{0.940601in}}%
\pgfpathcurveto{\pgfqpoint{9.934566in}{0.944167in}}{\pgfqpoint{9.936570in}{0.949005in}}{\pgfqpoint{9.936570in}{0.954049in}}%
\pgfpathcurveto{\pgfqpoint{9.936570in}{0.959092in}}{\pgfqpoint{9.934566in}{0.963930in}}{\pgfqpoint{9.931000in}{0.967496in}}%
\pgfpathcurveto{\pgfqpoint{9.927433in}{0.971063in}}{\pgfqpoint{9.922596in}{0.973067in}}{\pgfqpoint{9.917552in}{0.973067in}}%
\pgfpathcurveto{\pgfqpoint{9.912508in}{0.973067in}}{\pgfqpoint{9.907670in}{0.971063in}}{\pgfqpoint{9.904104in}{0.967496in}}%
\pgfpathcurveto{\pgfqpoint{9.900538in}{0.963930in}}{\pgfqpoint{9.898534in}{0.959092in}}{\pgfqpoint{9.898534in}{0.954049in}}%
\pgfpathcurveto{\pgfqpoint{9.898534in}{0.949005in}}{\pgfqpoint{9.900538in}{0.944167in}}{\pgfqpoint{9.904104in}{0.940601in}}%
\pgfpathcurveto{\pgfqpoint{9.907670in}{0.937034in}}{\pgfqpoint{9.912508in}{0.935030in}}{\pgfqpoint{9.917552in}{0.935030in}}%
\pgfpathclose%
\pgfusepath{fill}%
\end{pgfscope}%
\begin{pgfscope}%
\pgfpathrectangle{\pgfqpoint{6.572727in}{0.473000in}}{\pgfqpoint{4.227273in}{3.311000in}}%
\pgfusepath{clip}%
\pgfsetbuttcap%
\pgfsetroundjoin%
\definecolor{currentfill}{rgb}{0.993248,0.906157,0.143936}%
\pgfsetfillcolor{currentfill}%
\pgfsetfillopacity{0.700000}%
\pgfsetlinewidth{0.000000pt}%
\definecolor{currentstroke}{rgb}{0.000000,0.000000,0.000000}%
\pgfsetstrokecolor{currentstroke}%
\pgfsetstrokeopacity{0.700000}%
\pgfsetdash{}{0pt}%
\pgfpathmoveto{\pgfqpoint{9.715546in}{1.422054in}}%
\pgfpathcurveto{\pgfqpoint{9.720590in}{1.422054in}}{\pgfqpoint{9.725428in}{1.424058in}}{\pgfqpoint{9.728994in}{1.427625in}}%
\pgfpathcurveto{\pgfqpoint{9.732560in}{1.431191in}}{\pgfqpoint{9.734564in}{1.436029in}}{\pgfqpoint{9.734564in}{1.441073in}}%
\pgfpathcurveto{\pgfqpoint{9.734564in}{1.446116in}}{\pgfqpoint{9.732560in}{1.450954in}}{\pgfqpoint{9.728994in}{1.454520in}}%
\pgfpathcurveto{\pgfqpoint{9.725428in}{1.458087in}}{\pgfqpoint{9.720590in}{1.460091in}}{\pgfqpoint{9.715546in}{1.460091in}}%
\pgfpathcurveto{\pgfqpoint{9.710503in}{1.460091in}}{\pgfqpoint{9.705665in}{1.458087in}}{\pgfqpoint{9.702098in}{1.454520in}}%
\pgfpathcurveto{\pgfqpoint{9.698532in}{1.450954in}}{\pgfqpoint{9.696528in}{1.446116in}}{\pgfqpoint{9.696528in}{1.441073in}}%
\pgfpathcurveto{\pgfqpoint{9.696528in}{1.436029in}}{\pgfqpoint{9.698532in}{1.431191in}}{\pgfqpoint{9.702098in}{1.427625in}}%
\pgfpathcurveto{\pgfqpoint{9.705665in}{1.424058in}}{\pgfqpoint{9.710503in}{1.422054in}}{\pgfqpoint{9.715546in}{1.422054in}}%
\pgfpathclose%
\pgfusepath{fill}%
\end{pgfscope}%
\begin{pgfscope}%
\pgfpathrectangle{\pgfqpoint{6.572727in}{0.473000in}}{\pgfqpoint{4.227273in}{3.311000in}}%
\pgfusepath{clip}%
\pgfsetbuttcap%
\pgfsetroundjoin%
\definecolor{currentfill}{rgb}{0.993248,0.906157,0.143936}%
\pgfsetfillcolor{currentfill}%
\pgfsetfillopacity{0.700000}%
\pgfsetlinewidth{0.000000pt}%
\definecolor{currentstroke}{rgb}{0.000000,0.000000,0.000000}%
\pgfsetstrokecolor{currentstroke}%
\pgfsetstrokeopacity{0.700000}%
\pgfsetdash{}{0pt}%
\pgfpathmoveto{\pgfqpoint{9.740088in}{1.655595in}}%
\pgfpathcurveto{\pgfqpoint{9.745132in}{1.655595in}}{\pgfqpoint{9.749970in}{1.657599in}}{\pgfqpoint{9.753536in}{1.661166in}}%
\pgfpathcurveto{\pgfqpoint{9.757103in}{1.664732in}}{\pgfqpoint{9.759107in}{1.669570in}}{\pgfqpoint{9.759107in}{1.674613in}}%
\pgfpathcurveto{\pgfqpoint{9.759107in}{1.679657in}}{\pgfqpoint{9.757103in}{1.684495in}}{\pgfqpoint{9.753536in}{1.688061in}}%
\pgfpathcurveto{\pgfqpoint{9.749970in}{1.691628in}}{\pgfqpoint{9.745132in}{1.693632in}}{\pgfqpoint{9.740088in}{1.693632in}}%
\pgfpathcurveto{\pgfqpoint{9.735045in}{1.693632in}}{\pgfqpoint{9.730207in}{1.691628in}}{\pgfqpoint{9.726641in}{1.688061in}}%
\pgfpathcurveto{\pgfqpoint{9.723074in}{1.684495in}}{\pgfqpoint{9.721070in}{1.679657in}}{\pgfqpoint{9.721070in}{1.674613in}}%
\pgfpathcurveto{\pgfqpoint{9.721070in}{1.669570in}}{\pgfqpoint{9.723074in}{1.664732in}}{\pgfqpoint{9.726641in}{1.661166in}}%
\pgfpathcurveto{\pgfqpoint{9.730207in}{1.657599in}}{\pgfqpoint{9.735045in}{1.655595in}}{\pgfqpoint{9.740088in}{1.655595in}}%
\pgfpathclose%
\pgfusepath{fill}%
\end{pgfscope}%
\begin{pgfscope}%
\pgfpathrectangle{\pgfqpoint{6.572727in}{0.473000in}}{\pgfqpoint{4.227273in}{3.311000in}}%
\pgfusepath{clip}%
\pgfsetbuttcap%
\pgfsetroundjoin%
\definecolor{currentfill}{rgb}{0.127568,0.566949,0.550556}%
\pgfsetfillcolor{currentfill}%
\pgfsetfillopacity{0.700000}%
\pgfsetlinewidth{0.000000pt}%
\definecolor{currentstroke}{rgb}{0.000000,0.000000,0.000000}%
\pgfsetstrokecolor{currentstroke}%
\pgfsetstrokeopacity{0.700000}%
\pgfsetdash{}{0pt}%
\pgfpathmoveto{\pgfqpoint{7.619632in}{1.641686in}}%
\pgfpathcurveto{\pgfqpoint{7.624675in}{1.641686in}}{\pgfqpoint{7.629513in}{1.643689in}}{\pgfqpoint{7.633080in}{1.647256in}}%
\pgfpathcurveto{\pgfqpoint{7.636646in}{1.650822in}}{\pgfqpoint{7.638650in}{1.655660in}}{\pgfqpoint{7.638650in}{1.660704in}}%
\pgfpathcurveto{\pgfqpoint{7.638650in}{1.665747in}}{\pgfqpoint{7.636646in}{1.670585in}}{\pgfqpoint{7.633080in}{1.674152in}}%
\pgfpathcurveto{\pgfqpoint{7.629513in}{1.677718in}}{\pgfqpoint{7.624675in}{1.679722in}}{\pgfqpoint{7.619632in}{1.679722in}}%
\pgfpathcurveto{\pgfqpoint{7.614588in}{1.679722in}}{\pgfqpoint{7.609750in}{1.677718in}}{\pgfqpoint{7.606184in}{1.674152in}}%
\pgfpathcurveto{\pgfqpoint{7.602618in}{1.670585in}}{\pgfqpoint{7.600614in}{1.665747in}}{\pgfqpoint{7.600614in}{1.660704in}}%
\pgfpathcurveto{\pgfqpoint{7.600614in}{1.655660in}}{\pgfqpoint{7.602618in}{1.650822in}}{\pgfqpoint{7.606184in}{1.647256in}}%
\pgfpathcurveto{\pgfqpoint{7.609750in}{1.643689in}}{\pgfqpoint{7.614588in}{1.641686in}}{\pgfqpoint{7.619632in}{1.641686in}}%
\pgfpathclose%
\pgfusepath{fill}%
\end{pgfscope}%
\begin{pgfscope}%
\pgfpathrectangle{\pgfqpoint{6.572727in}{0.473000in}}{\pgfqpoint{4.227273in}{3.311000in}}%
\pgfusepath{clip}%
\pgfsetbuttcap%
\pgfsetroundjoin%
\definecolor{currentfill}{rgb}{0.993248,0.906157,0.143936}%
\pgfsetfillcolor{currentfill}%
\pgfsetfillopacity{0.700000}%
\pgfsetlinewidth{0.000000pt}%
\definecolor{currentstroke}{rgb}{0.000000,0.000000,0.000000}%
\pgfsetstrokecolor{currentstroke}%
\pgfsetstrokeopacity{0.700000}%
\pgfsetdash{}{0pt}%
\pgfpathmoveto{\pgfqpoint{9.385049in}{1.610089in}}%
\pgfpathcurveto{\pgfqpoint{9.390093in}{1.610089in}}{\pgfqpoint{9.394931in}{1.612092in}}{\pgfqpoint{9.398497in}{1.615659in}}%
\pgfpathcurveto{\pgfqpoint{9.402063in}{1.619225in}}{\pgfqpoint{9.404067in}{1.624063in}}{\pgfqpoint{9.404067in}{1.629107in}}%
\pgfpathcurveto{\pgfqpoint{9.404067in}{1.634150in}}{\pgfqpoint{9.402063in}{1.638988in}}{\pgfqpoint{9.398497in}{1.642555in}}%
\pgfpathcurveto{\pgfqpoint{9.394931in}{1.646121in}}{\pgfqpoint{9.390093in}{1.648125in}}{\pgfqpoint{9.385049in}{1.648125in}}%
\pgfpathcurveto{\pgfqpoint{9.380006in}{1.648125in}}{\pgfqpoint{9.375168in}{1.646121in}}{\pgfqpoint{9.371601in}{1.642555in}}%
\pgfpathcurveto{\pgfqpoint{9.368035in}{1.638988in}}{\pgfqpoint{9.366031in}{1.634150in}}{\pgfqpoint{9.366031in}{1.629107in}}%
\pgfpathcurveto{\pgfqpoint{9.366031in}{1.624063in}}{\pgfqpoint{9.368035in}{1.619225in}}{\pgfqpoint{9.371601in}{1.615659in}}%
\pgfpathcurveto{\pgfqpoint{9.375168in}{1.612092in}}{\pgfqpoint{9.380006in}{1.610089in}}{\pgfqpoint{9.385049in}{1.610089in}}%
\pgfpathclose%
\pgfusepath{fill}%
\end{pgfscope}%
\begin{pgfscope}%
\pgfpathrectangle{\pgfqpoint{6.572727in}{0.473000in}}{\pgfqpoint{4.227273in}{3.311000in}}%
\pgfusepath{clip}%
\pgfsetbuttcap%
\pgfsetroundjoin%
\definecolor{currentfill}{rgb}{0.267004,0.004874,0.329415}%
\pgfsetfillcolor{currentfill}%
\pgfsetfillopacity{0.700000}%
\pgfsetlinewidth{0.000000pt}%
\definecolor{currentstroke}{rgb}{0.000000,0.000000,0.000000}%
\pgfsetstrokecolor{currentstroke}%
\pgfsetstrokeopacity{0.700000}%
\pgfsetdash{}{0pt}%
\pgfpathmoveto{\pgfqpoint{6.836372in}{1.166737in}}%
\pgfpathcurveto{\pgfqpoint{6.841416in}{1.166737in}}{\pgfqpoint{6.846253in}{1.168741in}}{\pgfqpoint{6.849820in}{1.172308in}}%
\pgfpathcurveto{\pgfqpoint{6.853386in}{1.175874in}}{\pgfqpoint{6.855390in}{1.180712in}}{\pgfqpoint{6.855390in}{1.185756in}}%
\pgfpathcurveto{\pgfqpoint{6.855390in}{1.190799in}}{\pgfqpoint{6.853386in}{1.195637in}}{\pgfqpoint{6.849820in}{1.199203in}}%
\pgfpathcurveto{\pgfqpoint{6.846253in}{1.202770in}}{\pgfqpoint{6.841416in}{1.204774in}}{\pgfqpoint{6.836372in}{1.204774in}}%
\pgfpathcurveto{\pgfqpoint{6.831328in}{1.204774in}}{\pgfqpoint{6.826490in}{1.202770in}}{\pgfqpoint{6.822924in}{1.199203in}}%
\pgfpathcurveto{\pgfqpoint{6.819358in}{1.195637in}}{\pgfqpoint{6.817354in}{1.190799in}}{\pgfqpoint{6.817354in}{1.185756in}}%
\pgfpathcurveto{\pgfqpoint{6.817354in}{1.180712in}}{\pgfqpoint{6.819358in}{1.175874in}}{\pgfqpoint{6.822924in}{1.172308in}}%
\pgfpathcurveto{\pgfqpoint{6.826490in}{1.168741in}}{\pgfqpoint{6.831328in}{1.166737in}}{\pgfqpoint{6.836372in}{1.166737in}}%
\pgfpathclose%
\pgfusepath{fill}%
\end{pgfscope}%
\begin{pgfscope}%
\pgfpathrectangle{\pgfqpoint{6.572727in}{0.473000in}}{\pgfqpoint{4.227273in}{3.311000in}}%
\pgfusepath{clip}%
\pgfsetbuttcap%
\pgfsetroundjoin%
\definecolor{currentfill}{rgb}{0.993248,0.906157,0.143936}%
\pgfsetfillcolor{currentfill}%
\pgfsetfillopacity{0.700000}%
\pgfsetlinewidth{0.000000pt}%
\definecolor{currentstroke}{rgb}{0.000000,0.000000,0.000000}%
\pgfsetstrokecolor{currentstroke}%
\pgfsetstrokeopacity{0.700000}%
\pgfsetdash{}{0pt}%
\pgfpathmoveto{\pgfqpoint{9.283312in}{2.025241in}}%
\pgfpathcurveto{\pgfqpoint{9.288356in}{2.025241in}}{\pgfqpoint{9.293193in}{2.027245in}}{\pgfqpoint{9.296760in}{2.030812in}}%
\pgfpathcurveto{\pgfqpoint{9.300326in}{2.034378in}}{\pgfqpoint{9.302330in}{2.039216in}}{\pgfqpoint{9.302330in}{2.044260in}}%
\pgfpathcurveto{\pgfqpoint{9.302330in}{2.049303in}}{\pgfqpoint{9.300326in}{2.054141in}}{\pgfqpoint{9.296760in}{2.057707in}}%
\pgfpathcurveto{\pgfqpoint{9.293193in}{2.061274in}}{\pgfqpoint{9.288356in}{2.063278in}}{\pgfqpoint{9.283312in}{2.063278in}}%
\pgfpathcurveto{\pgfqpoint{9.278268in}{2.063278in}}{\pgfqpoint{9.273431in}{2.061274in}}{\pgfqpoint{9.269864in}{2.057707in}}%
\pgfpathcurveto{\pgfqpoint{9.266298in}{2.054141in}}{\pgfqpoint{9.264294in}{2.049303in}}{\pgfqpoint{9.264294in}{2.044260in}}%
\pgfpathcurveto{\pgfqpoint{9.264294in}{2.039216in}}{\pgfqpoint{9.266298in}{2.034378in}}{\pgfqpoint{9.269864in}{2.030812in}}%
\pgfpathcurveto{\pgfqpoint{9.273431in}{2.027245in}}{\pgfqpoint{9.278268in}{2.025241in}}{\pgfqpoint{9.283312in}{2.025241in}}%
\pgfpathclose%
\pgfusepath{fill}%
\end{pgfscope}%
\begin{pgfscope}%
\pgfpathrectangle{\pgfqpoint{6.572727in}{0.473000in}}{\pgfqpoint{4.227273in}{3.311000in}}%
\pgfusepath{clip}%
\pgfsetbuttcap%
\pgfsetroundjoin%
\definecolor{currentfill}{rgb}{0.127568,0.566949,0.550556}%
\pgfsetfillcolor{currentfill}%
\pgfsetfillopacity{0.700000}%
\pgfsetlinewidth{0.000000pt}%
\definecolor{currentstroke}{rgb}{0.000000,0.000000,0.000000}%
\pgfsetstrokecolor{currentstroke}%
\pgfsetstrokeopacity{0.700000}%
\pgfsetdash{}{0pt}%
\pgfpathmoveto{\pgfqpoint{7.758034in}{1.639353in}}%
\pgfpathcurveto{\pgfqpoint{7.763078in}{1.639353in}}{\pgfqpoint{7.767915in}{1.641357in}}{\pgfqpoint{7.771482in}{1.644923in}}%
\pgfpathcurveto{\pgfqpoint{7.775048in}{1.648490in}}{\pgfqpoint{7.777052in}{1.653328in}}{\pgfqpoint{7.777052in}{1.658371in}}%
\pgfpathcurveto{\pgfqpoint{7.777052in}{1.663415in}}{\pgfqpoint{7.775048in}{1.668253in}}{\pgfqpoint{7.771482in}{1.671819in}}%
\pgfpathcurveto{\pgfqpoint{7.767915in}{1.675386in}}{\pgfqpoint{7.763078in}{1.677389in}}{\pgfqpoint{7.758034in}{1.677389in}}%
\pgfpathcurveto{\pgfqpoint{7.752990in}{1.677389in}}{\pgfqpoint{7.748152in}{1.675386in}}{\pgfqpoint{7.744586in}{1.671819in}}%
\pgfpathcurveto{\pgfqpoint{7.741020in}{1.668253in}}{\pgfqpoint{7.739016in}{1.663415in}}{\pgfqpoint{7.739016in}{1.658371in}}%
\pgfpathcurveto{\pgfqpoint{7.739016in}{1.653328in}}{\pgfqpoint{7.741020in}{1.648490in}}{\pgfqpoint{7.744586in}{1.644923in}}%
\pgfpathcurveto{\pgfqpoint{7.748152in}{1.641357in}}{\pgfqpoint{7.752990in}{1.639353in}}{\pgfqpoint{7.758034in}{1.639353in}}%
\pgfpathclose%
\pgfusepath{fill}%
\end{pgfscope}%
\begin{pgfscope}%
\pgfpathrectangle{\pgfqpoint{6.572727in}{0.473000in}}{\pgfqpoint{4.227273in}{3.311000in}}%
\pgfusepath{clip}%
\pgfsetbuttcap%
\pgfsetroundjoin%
\definecolor{currentfill}{rgb}{0.993248,0.906157,0.143936}%
\pgfsetfillcolor{currentfill}%
\pgfsetfillopacity{0.700000}%
\pgfsetlinewidth{0.000000pt}%
\definecolor{currentstroke}{rgb}{0.000000,0.000000,0.000000}%
\pgfsetstrokecolor{currentstroke}%
\pgfsetstrokeopacity{0.700000}%
\pgfsetdash{}{0pt}%
\pgfpathmoveto{\pgfqpoint{9.315670in}{1.237615in}}%
\pgfpathcurveto{\pgfqpoint{9.320714in}{1.237615in}}{\pgfqpoint{9.325551in}{1.239619in}}{\pgfqpoint{9.329118in}{1.243186in}}%
\pgfpathcurveto{\pgfqpoint{9.332684in}{1.246752in}}{\pgfqpoint{9.334688in}{1.251590in}}{\pgfqpoint{9.334688in}{1.256634in}}%
\pgfpathcurveto{\pgfqpoint{9.334688in}{1.261677in}}{\pgfqpoint{9.332684in}{1.266515in}}{\pgfqpoint{9.329118in}{1.270081in}}%
\pgfpathcurveto{\pgfqpoint{9.325551in}{1.273648in}}{\pgfqpoint{9.320714in}{1.275652in}}{\pgfqpoint{9.315670in}{1.275652in}}%
\pgfpathcurveto{\pgfqpoint{9.310626in}{1.275652in}}{\pgfqpoint{9.305789in}{1.273648in}}{\pgfqpoint{9.302222in}{1.270081in}}%
\pgfpathcurveto{\pgfqpoint{9.298656in}{1.266515in}}{\pgfqpoint{9.296652in}{1.261677in}}{\pgfqpoint{9.296652in}{1.256634in}}%
\pgfpathcurveto{\pgfqpoint{9.296652in}{1.251590in}}{\pgfqpoint{9.298656in}{1.246752in}}{\pgfqpoint{9.302222in}{1.243186in}}%
\pgfpathcurveto{\pgfqpoint{9.305789in}{1.239619in}}{\pgfqpoint{9.310626in}{1.237615in}}{\pgfqpoint{9.315670in}{1.237615in}}%
\pgfpathclose%
\pgfusepath{fill}%
\end{pgfscope}%
\begin{pgfscope}%
\pgfpathrectangle{\pgfqpoint{6.572727in}{0.473000in}}{\pgfqpoint{4.227273in}{3.311000in}}%
\pgfusepath{clip}%
\pgfsetbuttcap%
\pgfsetroundjoin%
\definecolor{currentfill}{rgb}{0.127568,0.566949,0.550556}%
\pgfsetfillcolor{currentfill}%
\pgfsetfillopacity{0.700000}%
\pgfsetlinewidth{0.000000pt}%
\definecolor{currentstroke}{rgb}{0.000000,0.000000,0.000000}%
\pgfsetstrokecolor{currentstroke}%
\pgfsetstrokeopacity{0.700000}%
\pgfsetdash{}{0pt}%
\pgfpathmoveto{\pgfqpoint{7.663127in}{3.011588in}}%
\pgfpathcurveto{\pgfqpoint{7.668171in}{3.011588in}}{\pgfqpoint{7.673009in}{3.013592in}}{\pgfqpoint{7.676575in}{3.017158in}}%
\pgfpathcurveto{\pgfqpoint{7.680142in}{3.020724in}}{\pgfqpoint{7.682145in}{3.025562in}}{\pgfqpoint{7.682145in}{3.030606in}}%
\pgfpathcurveto{\pgfqpoint{7.682145in}{3.035650in}}{\pgfqpoint{7.680142in}{3.040487in}}{\pgfqpoint{7.676575in}{3.044054in}}%
\pgfpathcurveto{\pgfqpoint{7.673009in}{3.047620in}}{\pgfqpoint{7.668171in}{3.049624in}}{\pgfqpoint{7.663127in}{3.049624in}}%
\pgfpathcurveto{\pgfqpoint{7.658084in}{3.049624in}}{\pgfqpoint{7.653246in}{3.047620in}}{\pgfqpoint{7.649679in}{3.044054in}}%
\pgfpathcurveto{\pgfqpoint{7.646113in}{3.040487in}}{\pgfqpoint{7.644109in}{3.035650in}}{\pgfqpoint{7.644109in}{3.030606in}}%
\pgfpathcurveto{\pgfqpoint{7.644109in}{3.025562in}}{\pgfqpoint{7.646113in}{3.020724in}}{\pgfqpoint{7.649679in}{3.017158in}}%
\pgfpathcurveto{\pgfqpoint{7.653246in}{3.013592in}}{\pgfqpoint{7.658084in}{3.011588in}}{\pgfqpoint{7.663127in}{3.011588in}}%
\pgfpathclose%
\pgfusepath{fill}%
\end{pgfscope}%
\begin{pgfscope}%
\pgfpathrectangle{\pgfqpoint{6.572727in}{0.473000in}}{\pgfqpoint{4.227273in}{3.311000in}}%
\pgfusepath{clip}%
\pgfsetbuttcap%
\pgfsetroundjoin%
\definecolor{currentfill}{rgb}{0.127568,0.566949,0.550556}%
\pgfsetfillcolor{currentfill}%
\pgfsetfillopacity{0.700000}%
\pgfsetlinewidth{0.000000pt}%
\definecolor{currentstroke}{rgb}{0.000000,0.000000,0.000000}%
\pgfsetstrokecolor{currentstroke}%
\pgfsetstrokeopacity{0.700000}%
\pgfsetdash{}{0pt}%
\pgfpathmoveto{\pgfqpoint{7.816652in}{3.108758in}}%
\pgfpathcurveto{\pgfqpoint{7.821696in}{3.108758in}}{\pgfqpoint{7.826533in}{3.110762in}}{\pgfqpoint{7.830100in}{3.114329in}}%
\pgfpathcurveto{\pgfqpoint{7.833666in}{3.117895in}}{\pgfqpoint{7.835670in}{3.122733in}}{\pgfqpoint{7.835670in}{3.127777in}}%
\pgfpathcurveto{\pgfqpoint{7.835670in}{3.132820in}}{\pgfqpoint{7.833666in}{3.137658in}}{\pgfqpoint{7.830100in}{3.141224in}}%
\pgfpathcurveto{\pgfqpoint{7.826533in}{3.144791in}}{\pgfqpoint{7.821696in}{3.146795in}}{\pgfqpoint{7.816652in}{3.146795in}}%
\pgfpathcurveto{\pgfqpoint{7.811608in}{3.146795in}}{\pgfqpoint{7.806771in}{3.144791in}}{\pgfqpoint{7.803204in}{3.141224in}}%
\pgfpathcurveto{\pgfqpoint{7.799638in}{3.137658in}}{\pgfqpoint{7.797634in}{3.132820in}}{\pgfqpoint{7.797634in}{3.127777in}}%
\pgfpathcurveto{\pgfqpoint{7.797634in}{3.122733in}}{\pgfqpoint{7.799638in}{3.117895in}}{\pgfqpoint{7.803204in}{3.114329in}}%
\pgfpathcurveto{\pgfqpoint{7.806771in}{3.110762in}}{\pgfqpoint{7.811608in}{3.108758in}}{\pgfqpoint{7.816652in}{3.108758in}}%
\pgfpathclose%
\pgfusepath{fill}%
\end{pgfscope}%
\begin{pgfscope}%
\pgfpathrectangle{\pgfqpoint{6.572727in}{0.473000in}}{\pgfqpoint{4.227273in}{3.311000in}}%
\pgfusepath{clip}%
\pgfsetbuttcap%
\pgfsetroundjoin%
\definecolor{currentfill}{rgb}{0.993248,0.906157,0.143936}%
\pgfsetfillcolor{currentfill}%
\pgfsetfillopacity{0.700000}%
\pgfsetlinewidth{0.000000pt}%
\definecolor{currentstroke}{rgb}{0.000000,0.000000,0.000000}%
\pgfsetstrokecolor{currentstroke}%
\pgfsetstrokeopacity{0.700000}%
\pgfsetdash{}{0pt}%
\pgfpathmoveto{\pgfqpoint{9.033129in}{2.040819in}}%
\pgfpathcurveto{\pgfqpoint{9.038172in}{2.040819in}}{\pgfqpoint{9.043010in}{2.042823in}}{\pgfqpoint{9.046577in}{2.046390in}}%
\pgfpathcurveto{\pgfqpoint{9.050143in}{2.049956in}}{\pgfqpoint{9.052147in}{2.054794in}}{\pgfqpoint{9.052147in}{2.059838in}}%
\pgfpathcurveto{\pgfqpoint{9.052147in}{2.064881in}}{\pgfqpoint{9.050143in}{2.069719in}}{\pgfqpoint{9.046577in}{2.073285in}}%
\pgfpathcurveto{\pgfqpoint{9.043010in}{2.076852in}}{\pgfqpoint{9.038172in}{2.078856in}}{\pgfqpoint{9.033129in}{2.078856in}}%
\pgfpathcurveto{\pgfqpoint{9.028085in}{2.078856in}}{\pgfqpoint{9.023247in}{2.076852in}}{\pgfqpoint{9.019681in}{2.073285in}}%
\pgfpathcurveto{\pgfqpoint{9.016114in}{2.069719in}}{\pgfqpoint{9.014111in}{2.064881in}}{\pgfqpoint{9.014111in}{2.059838in}}%
\pgfpathcurveto{\pgfqpoint{9.014111in}{2.054794in}}{\pgfqpoint{9.016114in}{2.049956in}}{\pgfqpoint{9.019681in}{2.046390in}}%
\pgfpathcurveto{\pgfqpoint{9.023247in}{2.042823in}}{\pgfqpoint{9.028085in}{2.040819in}}{\pgfqpoint{9.033129in}{2.040819in}}%
\pgfpathclose%
\pgfusepath{fill}%
\end{pgfscope}%
\begin{pgfscope}%
\pgfpathrectangle{\pgfqpoint{6.572727in}{0.473000in}}{\pgfqpoint{4.227273in}{3.311000in}}%
\pgfusepath{clip}%
\pgfsetbuttcap%
\pgfsetroundjoin%
\definecolor{currentfill}{rgb}{0.127568,0.566949,0.550556}%
\pgfsetfillcolor{currentfill}%
\pgfsetfillopacity{0.700000}%
\pgfsetlinewidth{0.000000pt}%
\definecolor{currentstroke}{rgb}{0.000000,0.000000,0.000000}%
\pgfsetstrokecolor{currentstroke}%
\pgfsetstrokeopacity{0.700000}%
\pgfsetdash{}{0pt}%
\pgfpathmoveto{\pgfqpoint{7.749487in}{2.552593in}}%
\pgfpathcurveto{\pgfqpoint{7.754531in}{2.552593in}}{\pgfqpoint{7.759369in}{2.554597in}}{\pgfqpoint{7.762935in}{2.558163in}}%
\pgfpathcurveto{\pgfqpoint{7.766502in}{2.561730in}}{\pgfqpoint{7.768506in}{2.566568in}}{\pgfqpoint{7.768506in}{2.571611in}}%
\pgfpathcurveto{\pgfqpoint{7.768506in}{2.576655in}}{\pgfqpoint{7.766502in}{2.581493in}}{\pgfqpoint{7.762935in}{2.585059in}}%
\pgfpathcurveto{\pgfqpoint{7.759369in}{2.588626in}}{\pgfqpoint{7.754531in}{2.590629in}}{\pgfqpoint{7.749487in}{2.590629in}}%
\pgfpathcurveto{\pgfqpoint{7.744444in}{2.590629in}}{\pgfqpoint{7.739606in}{2.588626in}}{\pgfqpoint{7.736040in}{2.585059in}}%
\pgfpathcurveto{\pgfqpoint{7.732473in}{2.581493in}}{\pgfqpoint{7.730469in}{2.576655in}}{\pgfqpoint{7.730469in}{2.571611in}}%
\pgfpathcurveto{\pgfqpoint{7.730469in}{2.566568in}}{\pgfqpoint{7.732473in}{2.561730in}}{\pgfqpoint{7.736040in}{2.558163in}}%
\pgfpathcurveto{\pgfqpoint{7.739606in}{2.554597in}}{\pgfqpoint{7.744444in}{2.552593in}}{\pgfqpoint{7.749487in}{2.552593in}}%
\pgfpathclose%
\pgfusepath{fill}%
\end{pgfscope}%
\begin{pgfscope}%
\pgfpathrectangle{\pgfqpoint{6.572727in}{0.473000in}}{\pgfqpoint{4.227273in}{3.311000in}}%
\pgfusepath{clip}%
\pgfsetbuttcap%
\pgfsetroundjoin%
\definecolor{currentfill}{rgb}{0.993248,0.906157,0.143936}%
\pgfsetfillcolor{currentfill}%
\pgfsetfillopacity{0.700000}%
\pgfsetlinewidth{0.000000pt}%
\definecolor{currentstroke}{rgb}{0.000000,0.000000,0.000000}%
\pgfsetstrokecolor{currentstroke}%
\pgfsetstrokeopacity{0.700000}%
\pgfsetdash{}{0pt}%
\pgfpathmoveto{\pgfqpoint{9.647676in}{1.397892in}}%
\pgfpathcurveto{\pgfqpoint{9.652719in}{1.397892in}}{\pgfqpoint{9.657557in}{1.399895in}}{\pgfqpoint{9.661123in}{1.403462in}}%
\pgfpathcurveto{\pgfqpoint{9.664690in}{1.407028in}}{\pgfqpoint{9.666694in}{1.411866in}}{\pgfqpoint{9.666694in}{1.416910in}}%
\pgfpathcurveto{\pgfqpoint{9.666694in}{1.421953in}}{\pgfqpoint{9.664690in}{1.426791in}}{\pgfqpoint{9.661123in}{1.430358in}}%
\pgfpathcurveto{\pgfqpoint{9.657557in}{1.433924in}}{\pgfqpoint{9.652719in}{1.435928in}}{\pgfqpoint{9.647676in}{1.435928in}}%
\pgfpathcurveto{\pgfqpoint{9.642632in}{1.435928in}}{\pgfqpoint{9.637794in}{1.433924in}}{\pgfqpoint{9.634228in}{1.430358in}}%
\pgfpathcurveto{\pgfqpoint{9.630661in}{1.426791in}}{\pgfqpoint{9.628657in}{1.421953in}}{\pgfqpoint{9.628657in}{1.416910in}}%
\pgfpathcurveto{\pgfqpoint{9.628657in}{1.411866in}}{\pgfqpoint{9.630661in}{1.407028in}}{\pgfqpoint{9.634228in}{1.403462in}}%
\pgfpathcurveto{\pgfqpoint{9.637794in}{1.399895in}}{\pgfqpoint{9.642632in}{1.397892in}}{\pgfqpoint{9.647676in}{1.397892in}}%
\pgfpathclose%
\pgfusepath{fill}%
\end{pgfscope}%
\begin{pgfscope}%
\pgfpathrectangle{\pgfqpoint{6.572727in}{0.473000in}}{\pgfqpoint{4.227273in}{3.311000in}}%
\pgfusepath{clip}%
\pgfsetbuttcap%
\pgfsetroundjoin%
\definecolor{currentfill}{rgb}{0.127568,0.566949,0.550556}%
\pgfsetfillcolor{currentfill}%
\pgfsetfillopacity{0.700000}%
\pgfsetlinewidth{0.000000pt}%
\definecolor{currentstroke}{rgb}{0.000000,0.000000,0.000000}%
\pgfsetstrokecolor{currentstroke}%
\pgfsetstrokeopacity{0.700000}%
\pgfsetdash{}{0pt}%
\pgfpathmoveto{\pgfqpoint{8.130323in}{2.826172in}}%
\pgfpathcurveto{\pgfqpoint{8.135367in}{2.826172in}}{\pgfqpoint{8.140205in}{2.828176in}}{\pgfqpoint{8.143771in}{2.831743in}}%
\pgfpathcurveto{\pgfqpoint{8.147338in}{2.835309in}}{\pgfqpoint{8.149342in}{2.840147in}}{\pgfqpoint{8.149342in}{2.845190in}}%
\pgfpathcurveto{\pgfqpoint{8.149342in}{2.850234in}}{\pgfqpoint{8.147338in}{2.855072in}}{\pgfqpoint{8.143771in}{2.858638in}}%
\pgfpathcurveto{\pgfqpoint{8.140205in}{2.862205in}}{\pgfqpoint{8.135367in}{2.864209in}}{\pgfqpoint{8.130323in}{2.864209in}}%
\pgfpathcurveto{\pgfqpoint{8.125280in}{2.864209in}}{\pgfqpoint{8.120442in}{2.862205in}}{\pgfqpoint{8.116876in}{2.858638in}}%
\pgfpathcurveto{\pgfqpoint{8.113309in}{2.855072in}}{\pgfqpoint{8.111305in}{2.850234in}}{\pgfqpoint{8.111305in}{2.845190in}}%
\pgfpathcurveto{\pgfqpoint{8.111305in}{2.840147in}}{\pgfqpoint{8.113309in}{2.835309in}}{\pgfqpoint{8.116876in}{2.831743in}}%
\pgfpathcurveto{\pgfqpoint{8.120442in}{2.828176in}}{\pgfqpoint{8.125280in}{2.826172in}}{\pgfqpoint{8.130323in}{2.826172in}}%
\pgfpathclose%
\pgfusepath{fill}%
\end{pgfscope}%
\begin{pgfscope}%
\pgfpathrectangle{\pgfqpoint{6.572727in}{0.473000in}}{\pgfqpoint{4.227273in}{3.311000in}}%
\pgfusepath{clip}%
\pgfsetbuttcap%
\pgfsetroundjoin%
\definecolor{currentfill}{rgb}{0.127568,0.566949,0.550556}%
\pgfsetfillcolor{currentfill}%
\pgfsetfillopacity{0.700000}%
\pgfsetlinewidth{0.000000pt}%
\definecolor{currentstroke}{rgb}{0.000000,0.000000,0.000000}%
\pgfsetstrokecolor{currentstroke}%
\pgfsetstrokeopacity{0.700000}%
\pgfsetdash{}{0pt}%
\pgfpathmoveto{\pgfqpoint{8.963549in}{2.937890in}}%
\pgfpathcurveto{\pgfqpoint{8.968592in}{2.937890in}}{\pgfqpoint{8.973430in}{2.939894in}}{\pgfqpoint{8.976996in}{2.943460in}}%
\pgfpathcurveto{\pgfqpoint{8.980563in}{2.947026in}}{\pgfqpoint{8.982567in}{2.951864in}}{\pgfqpoint{8.982567in}{2.956908in}}%
\pgfpathcurveto{\pgfqpoint{8.982567in}{2.961952in}}{\pgfqpoint{8.980563in}{2.966789in}}{\pgfqpoint{8.976996in}{2.970356in}}%
\pgfpathcurveto{\pgfqpoint{8.973430in}{2.973922in}}{\pgfqpoint{8.968592in}{2.975926in}}{\pgfqpoint{8.963549in}{2.975926in}}%
\pgfpathcurveto{\pgfqpoint{8.958505in}{2.975926in}}{\pgfqpoint{8.953667in}{2.973922in}}{\pgfqpoint{8.950101in}{2.970356in}}%
\pgfpathcurveto{\pgfqpoint{8.946534in}{2.966789in}}{\pgfqpoint{8.944530in}{2.961952in}}{\pgfqpoint{8.944530in}{2.956908in}}%
\pgfpathcurveto{\pgfqpoint{8.944530in}{2.951864in}}{\pgfqpoint{8.946534in}{2.947026in}}{\pgfqpoint{8.950101in}{2.943460in}}%
\pgfpathcurveto{\pgfqpoint{8.953667in}{2.939894in}}{\pgfqpoint{8.958505in}{2.937890in}}{\pgfqpoint{8.963549in}{2.937890in}}%
\pgfpathclose%
\pgfusepath{fill}%
\end{pgfscope}%
\begin{pgfscope}%
\pgfpathrectangle{\pgfqpoint{6.572727in}{0.473000in}}{\pgfqpoint{4.227273in}{3.311000in}}%
\pgfusepath{clip}%
\pgfsetbuttcap%
\pgfsetroundjoin%
\definecolor{currentfill}{rgb}{0.127568,0.566949,0.550556}%
\pgfsetfillcolor{currentfill}%
\pgfsetfillopacity{0.700000}%
\pgfsetlinewidth{0.000000pt}%
\definecolor{currentstroke}{rgb}{0.000000,0.000000,0.000000}%
\pgfsetstrokecolor{currentstroke}%
\pgfsetstrokeopacity{0.700000}%
\pgfsetdash{}{0pt}%
\pgfpathmoveto{\pgfqpoint{8.296973in}{2.700549in}}%
\pgfpathcurveto{\pgfqpoint{8.302017in}{2.700549in}}{\pgfqpoint{8.306855in}{2.702553in}}{\pgfqpoint{8.310421in}{2.706119in}}%
\pgfpathcurveto{\pgfqpoint{8.313987in}{2.709685in}}{\pgfqpoint{8.315991in}{2.714523in}}{\pgfqpoint{8.315991in}{2.719567in}}%
\pgfpathcurveto{\pgfqpoint{8.315991in}{2.724611in}}{\pgfqpoint{8.313987in}{2.729448in}}{\pgfqpoint{8.310421in}{2.733015in}}%
\pgfpathcurveto{\pgfqpoint{8.306855in}{2.736581in}}{\pgfqpoint{8.302017in}{2.738585in}}{\pgfqpoint{8.296973in}{2.738585in}}%
\pgfpathcurveto{\pgfqpoint{8.291929in}{2.738585in}}{\pgfqpoint{8.287092in}{2.736581in}}{\pgfqpoint{8.283525in}{2.733015in}}%
\pgfpathcurveto{\pgfqpoint{8.279959in}{2.729448in}}{\pgfqpoint{8.277955in}{2.724611in}}{\pgfqpoint{8.277955in}{2.719567in}}%
\pgfpathcurveto{\pgfqpoint{8.277955in}{2.714523in}}{\pgfqpoint{8.279959in}{2.709685in}}{\pgfqpoint{8.283525in}{2.706119in}}%
\pgfpathcurveto{\pgfqpoint{8.287092in}{2.702553in}}{\pgfqpoint{8.291929in}{2.700549in}}{\pgfqpoint{8.296973in}{2.700549in}}%
\pgfpathclose%
\pgfusepath{fill}%
\end{pgfscope}%
\begin{pgfscope}%
\pgfpathrectangle{\pgfqpoint{6.572727in}{0.473000in}}{\pgfqpoint{4.227273in}{3.311000in}}%
\pgfusepath{clip}%
\pgfsetbuttcap%
\pgfsetroundjoin%
\definecolor{currentfill}{rgb}{0.993248,0.906157,0.143936}%
\pgfsetfillcolor{currentfill}%
\pgfsetfillopacity{0.700000}%
\pgfsetlinewidth{0.000000pt}%
\definecolor{currentstroke}{rgb}{0.000000,0.000000,0.000000}%
\pgfsetstrokecolor{currentstroke}%
\pgfsetstrokeopacity{0.700000}%
\pgfsetdash{}{0pt}%
\pgfpathmoveto{\pgfqpoint{10.308772in}{1.806213in}}%
\pgfpathcurveto{\pgfqpoint{10.313816in}{1.806213in}}{\pgfqpoint{10.318653in}{1.808217in}}{\pgfqpoint{10.322220in}{1.811784in}}%
\pgfpathcurveto{\pgfqpoint{10.325786in}{1.815350in}}{\pgfqpoint{10.327790in}{1.820188in}}{\pgfqpoint{10.327790in}{1.825232in}}%
\pgfpathcurveto{\pgfqpoint{10.327790in}{1.830275in}}{\pgfqpoint{10.325786in}{1.835113in}}{\pgfqpoint{10.322220in}{1.838679in}}%
\pgfpathcurveto{\pgfqpoint{10.318653in}{1.842246in}}{\pgfqpoint{10.313816in}{1.844250in}}{\pgfqpoint{10.308772in}{1.844250in}}%
\pgfpathcurveto{\pgfqpoint{10.303728in}{1.844250in}}{\pgfqpoint{10.298891in}{1.842246in}}{\pgfqpoint{10.295324in}{1.838679in}}%
\pgfpathcurveto{\pgfqpoint{10.291758in}{1.835113in}}{\pgfqpoint{10.289754in}{1.830275in}}{\pgfqpoint{10.289754in}{1.825232in}}%
\pgfpathcurveto{\pgfqpoint{10.289754in}{1.820188in}}{\pgfqpoint{10.291758in}{1.815350in}}{\pgfqpoint{10.295324in}{1.811784in}}%
\pgfpathcurveto{\pgfqpoint{10.298891in}{1.808217in}}{\pgfqpoint{10.303728in}{1.806213in}}{\pgfqpoint{10.308772in}{1.806213in}}%
\pgfpathclose%
\pgfusepath{fill}%
\end{pgfscope}%
\begin{pgfscope}%
\pgfpathrectangle{\pgfqpoint{6.572727in}{0.473000in}}{\pgfqpoint{4.227273in}{3.311000in}}%
\pgfusepath{clip}%
\pgfsetbuttcap%
\pgfsetroundjoin%
\definecolor{currentfill}{rgb}{0.993248,0.906157,0.143936}%
\pgfsetfillcolor{currentfill}%
\pgfsetfillopacity{0.700000}%
\pgfsetlinewidth{0.000000pt}%
\definecolor{currentstroke}{rgb}{0.000000,0.000000,0.000000}%
\pgfsetstrokecolor{currentstroke}%
\pgfsetstrokeopacity{0.700000}%
\pgfsetdash{}{0pt}%
\pgfpathmoveto{\pgfqpoint{9.307642in}{1.598115in}}%
\pgfpathcurveto{\pgfqpoint{9.312686in}{1.598115in}}{\pgfqpoint{9.317523in}{1.600119in}}{\pgfqpoint{9.321090in}{1.603686in}}%
\pgfpathcurveto{\pgfqpoint{9.324656in}{1.607252in}}{\pgfqpoint{9.326660in}{1.612090in}}{\pgfqpoint{9.326660in}{1.617133in}}%
\pgfpathcurveto{\pgfqpoint{9.326660in}{1.622177in}}{\pgfqpoint{9.324656in}{1.627015in}}{\pgfqpoint{9.321090in}{1.630581in}}%
\pgfpathcurveto{\pgfqpoint{9.317523in}{1.634148in}}{\pgfqpoint{9.312686in}{1.636152in}}{\pgfqpoint{9.307642in}{1.636152in}}%
\pgfpathcurveto{\pgfqpoint{9.302598in}{1.636152in}}{\pgfqpoint{9.297761in}{1.634148in}}{\pgfqpoint{9.294194in}{1.630581in}}%
\pgfpathcurveto{\pgfqpoint{9.290628in}{1.627015in}}{\pgfqpoint{9.288624in}{1.622177in}}{\pgfqpoint{9.288624in}{1.617133in}}%
\pgfpathcurveto{\pgfqpoint{9.288624in}{1.612090in}}{\pgfqpoint{9.290628in}{1.607252in}}{\pgfqpoint{9.294194in}{1.603686in}}%
\pgfpathcurveto{\pgfqpoint{9.297761in}{1.600119in}}{\pgfqpoint{9.302598in}{1.598115in}}{\pgfqpoint{9.307642in}{1.598115in}}%
\pgfpathclose%
\pgfusepath{fill}%
\end{pgfscope}%
\begin{pgfscope}%
\pgfpathrectangle{\pgfqpoint{6.572727in}{0.473000in}}{\pgfqpoint{4.227273in}{3.311000in}}%
\pgfusepath{clip}%
\pgfsetbuttcap%
\pgfsetroundjoin%
\definecolor{currentfill}{rgb}{0.127568,0.566949,0.550556}%
\pgfsetfillcolor{currentfill}%
\pgfsetfillopacity{0.700000}%
\pgfsetlinewidth{0.000000pt}%
\definecolor{currentstroke}{rgb}{0.000000,0.000000,0.000000}%
\pgfsetstrokecolor{currentstroke}%
\pgfsetstrokeopacity{0.700000}%
\pgfsetdash{}{0pt}%
\pgfpathmoveto{\pgfqpoint{7.812880in}{1.192449in}}%
\pgfpathcurveto{\pgfqpoint{7.817924in}{1.192449in}}{\pgfqpoint{7.822762in}{1.194452in}}{\pgfqpoint{7.826328in}{1.198019in}}%
\pgfpathcurveto{\pgfqpoint{7.829895in}{1.201585in}}{\pgfqpoint{7.831899in}{1.206423in}}{\pgfqpoint{7.831899in}{1.211467in}}%
\pgfpathcurveto{\pgfqpoint{7.831899in}{1.216510in}}{\pgfqpoint{7.829895in}{1.221348in}}{\pgfqpoint{7.826328in}{1.224915in}}%
\pgfpathcurveto{\pgfqpoint{7.822762in}{1.228481in}}{\pgfqpoint{7.817924in}{1.230485in}}{\pgfqpoint{7.812880in}{1.230485in}}%
\pgfpathcurveto{\pgfqpoint{7.807837in}{1.230485in}}{\pgfqpoint{7.802999in}{1.228481in}}{\pgfqpoint{7.799433in}{1.224915in}}%
\pgfpathcurveto{\pgfqpoint{7.795866in}{1.221348in}}{\pgfqpoint{7.793862in}{1.216510in}}{\pgfqpoint{7.793862in}{1.211467in}}%
\pgfpathcurveto{\pgfqpoint{7.793862in}{1.206423in}}{\pgfqpoint{7.795866in}{1.201585in}}{\pgfqpoint{7.799433in}{1.198019in}}%
\pgfpathcurveto{\pgfqpoint{7.802999in}{1.194452in}}{\pgfqpoint{7.807837in}{1.192449in}}{\pgfqpoint{7.812880in}{1.192449in}}%
\pgfpathclose%
\pgfusepath{fill}%
\end{pgfscope}%
\begin{pgfscope}%
\pgfpathrectangle{\pgfqpoint{6.572727in}{0.473000in}}{\pgfqpoint{4.227273in}{3.311000in}}%
\pgfusepath{clip}%
\pgfsetbuttcap%
\pgfsetroundjoin%
\definecolor{currentfill}{rgb}{0.267004,0.004874,0.329415}%
\pgfsetfillcolor{currentfill}%
\pgfsetfillopacity{0.700000}%
\pgfsetlinewidth{0.000000pt}%
\definecolor{currentstroke}{rgb}{0.000000,0.000000,0.000000}%
\pgfsetstrokecolor{currentstroke}%
\pgfsetstrokeopacity{0.700000}%
\pgfsetdash{}{0pt}%
\pgfpathmoveto{\pgfqpoint{8.647794in}{1.124848in}}%
\pgfpathcurveto{\pgfqpoint{8.652838in}{1.124848in}}{\pgfqpoint{8.657676in}{1.126851in}}{\pgfqpoint{8.661242in}{1.130418in}}%
\pgfpathcurveto{\pgfqpoint{8.664809in}{1.133984in}}{\pgfqpoint{8.666813in}{1.138822in}}{\pgfqpoint{8.666813in}{1.143866in}}%
\pgfpathcurveto{\pgfqpoint{8.666813in}{1.148909in}}{\pgfqpoint{8.664809in}{1.153747in}}{\pgfqpoint{8.661242in}{1.157314in}}%
\pgfpathcurveto{\pgfqpoint{8.657676in}{1.160880in}}{\pgfqpoint{8.652838in}{1.162884in}}{\pgfqpoint{8.647794in}{1.162884in}}%
\pgfpathcurveto{\pgfqpoint{8.642751in}{1.162884in}}{\pgfqpoint{8.637913in}{1.160880in}}{\pgfqpoint{8.634347in}{1.157314in}}%
\pgfpathcurveto{\pgfqpoint{8.630780in}{1.153747in}}{\pgfqpoint{8.628776in}{1.148909in}}{\pgfqpoint{8.628776in}{1.143866in}}%
\pgfpathcurveto{\pgfqpoint{8.628776in}{1.138822in}}{\pgfqpoint{8.630780in}{1.133984in}}{\pgfqpoint{8.634347in}{1.130418in}}%
\pgfpathcurveto{\pgfqpoint{8.637913in}{1.126851in}}{\pgfqpoint{8.642751in}{1.124848in}}{\pgfqpoint{8.647794in}{1.124848in}}%
\pgfpathclose%
\pgfusepath{fill}%
\end{pgfscope}%
\begin{pgfscope}%
\pgfpathrectangle{\pgfqpoint{6.572727in}{0.473000in}}{\pgfqpoint{4.227273in}{3.311000in}}%
\pgfusepath{clip}%
\pgfsetbuttcap%
\pgfsetroundjoin%
\definecolor{currentfill}{rgb}{0.127568,0.566949,0.550556}%
\pgfsetfillcolor{currentfill}%
\pgfsetfillopacity{0.700000}%
\pgfsetlinewidth{0.000000pt}%
\definecolor{currentstroke}{rgb}{0.000000,0.000000,0.000000}%
\pgfsetstrokecolor{currentstroke}%
\pgfsetstrokeopacity{0.700000}%
\pgfsetdash{}{0pt}%
\pgfpathmoveto{\pgfqpoint{8.008567in}{1.202127in}}%
\pgfpathcurveto{\pgfqpoint{8.013611in}{1.202127in}}{\pgfqpoint{8.018449in}{1.204131in}}{\pgfqpoint{8.022015in}{1.207697in}}%
\pgfpathcurveto{\pgfqpoint{8.025582in}{1.211263in}}{\pgfqpoint{8.027586in}{1.216101in}}{\pgfqpoint{8.027586in}{1.221145in}}%
\pgfpathcurveto{\pgfqpoint{8.027586in}{1.226189in}}{\pgfqpoint{8.025582in}{1.231026in}}{\pgfqpoint{8.022015in}{1.234593in}}%
\pgfpathcurveto{\pgfqpoint{8.018449in}{1.238159in}}{\pgfqpoint{8.013611in}{1.240163in}}{\pgfqpoint{8.008567in}{1.240163in}}%
\pgfpathcurveto{\pgfqpoint{8.003524in}{1.240163in}}{\pgfqpoint{7.998686in}{1.238159in}}{\pgfqpoint{7.995120in}{1.234593in}}%
\pgfpathcurveto{\pgfqpoint{7.991553in}{1.231026in}}{\pgfqpoint{7.989549in}{1.226189in}}{\pgfqpoint{7.989549in}{1.221145in}}%
\pgfpathcurveto{\pgfqpoint{7.989549in}{1.216101in}}{\pgfqpoint{7.991553in}{1.211263in}}{\pgfqpoint{7.995120in}{1.207697in}}%
\pgfpathcurveto{\pgfqpoint{7.998686in}{1.204131in}}{\pgfqpoint{8.003524in}{1.202127in}}{\pgfqpoint{8.008567in}{1.202127in}}%
\pgfpathclose%
\pgfusepath{fill}%
\end{pgfscope}%
\begin{pgfscope}%
\pgfpathrectangle{\pgfqpoint{6.572727in}{0.473000in}}{\pgfqpoint{4.227273in}{3.311000in}}%
\pgfusepath{clip}%
\pgfsetbuttcap%
\pgfsetroundjoin%
\definecolor{currentfill}{rgb}{0.127568,0.566949,0.550556}%
\pgfsetfillcolor{currentfill}%
\pgfsetfillopacity{0.700000}%
\pgfsetlinewidth{0.000000pt}%
\definecolor{currentstroke}{rgb}{0.000000,0.000000,0.000000}%
\pgfsetstrokecolor{currentstroke}%
\pgfsetstrokeopacity{0.700000}%
\pgfsetdash{}{0pt}%
\pgfpathmoveto{\pgfqpoint{8.385035in}{1.679180in}}%
\pgfpathcurveto{\pgfqpoint{8.390079in}{1.679180in}}{\pgfqpoint{8.394917in}{1.681183in}}{\pgfqpoint{8.398483in}{1.684750in}}%
\pgfpathcurveto{\pgfqpoint{8.402049in}{1.688316in}}{\pgfqpoint{8.404053in}{1.693154in}}{\pgfqpoint{8.404053in}{1.698198in}}%
\pgfpathcurveto{\pgfqpoint{8.404053in}{1.703241in}}{\pgfqpoint{8.402049in}{1.708079in}}{\pgfqpoint{8.398483in}{1.711646in}}%
\pgfpathcurveto{\pgfqpoint{8.394917in}{1.715212in}}{\pgfqpoint{8.390079in}{1.717216in}}{\pgfqpoint{8.385035in}{1.717216in}}%
\pgfpathcurveto{\pgfqpoint{8.379991in}{1.717216in}}{\pgfqpoint{8.375154in}{1.715212in}}{\pgfqpoint{8.371587in}{1.711646in}}%
\pgfpathcurveto{\pgfqpoint{8.368021in}{1.708079in}}{\pgfqpoint{8.366017in}{1.703241in}}{\pgfqpoint{8.366017in}{1.698198in}}%
\pgfpathcurveto{\pgfqpoint{8.366017in}{1.693154in}}{\pgfqpoint{8.368021in}{1.688316in}}{\pgfqpoint{8.371587in}{1.684750in}}%
\pgfpathcurveto{\pgfqpoint{8.375154in}{1.681183in}}{\pgfqpoint{8.379991in}{1.679180in}}{\pgfqpoint{8.385035in}{1.679180in}}%
\pgfpathclose%
\pgfusepath{fill}%
\end{pgfscope}%
\begin{pgfscope}%
\pgfpathrectangle{\pgfqpoint{6.572727in}{0.473000in}}{\pgfqpoint{4.227273in}{3.311000in}}%
\pgfusepath{clip}%
\pgfsetbuttcap%
\pgfsetroundjoin%
\definecolor{currentfill}{rgb}{0.993248,0.906157,0.143936}%
\pgfsetfillcolor{currentfill}%
\pgfsetfillopacity{0.700000}%
\pgfsetlinewidth{0.000000pt}%
\definecolor{currentstroke}{rgb}{0.000000,0.000000,0.000000}%
\pgfsetstrokecolor{currentstroke}%
\pgfsetstrokeopacity{0.700000}%
\pgfsetdash{}{0pt}%
\pgfpathmoveto{\pgfqpoint{9.265493in}{2.003184in}}%
\pgfpathcurveto{\pgfqpoint{9.270537in}{2.003184in}}{\pgfqpoint{9.275375in}{2.005188in}}{\pgfqpoint{9.278941in}{2.008754in}}%
\pgfpathcurveto{\pgfqpoint{9.282507in}{2.012321in}}{\pgfqpoint{9.284511in}{2.017158in}}{\pgfqpoint{9.284511in}{2.022202in}}%
\pgfpathcurveto{\pgfqpoint{9.284511in}{2.027246in}}{\pgfqpoint{9.282507in}{2.032084in}}{\pgfqpoint{9.278941in}{2.035650in}}%
\pgfpathcurveto{\pgfqpoint{9.275375in}{2.039216in}}{\pgfqpoint{9.270537in}{2.041220in}}{\pgfqpoint{9.265493in}{2.041220in}}%
\pgfpathcurveto{\pgfqpoint{9.260449in}{2.041220in}}{\pgfqpoint{9.255612in}{2.039216in}}{\pgfqpoint{9.252045in}{2.035650in}}%
\pgfpathcurveto{\pgfqpoint{9.248479in}{2.032084in}}{\pgfqpoint{9.246475in}{2.027246in}}{\pgfqpoint{9.246475in}{2.022202in}}%
\pgfpathcurveto{\pgfqpoint{9.246475in}{2.017158in}}{\pgfqpoint{9.248479in}{2.012321in}}{\pgfqpoint{9.252045in}{2.008754in}}%
\pgfpathcurveto{\pgfqpoint{9.255612in}{2.005188in}}{\pgfqpoint{9.260449in}{2.003184in}}{\pgfqpoint{9.265493in}{2.003184in}}%
\pgfpathclose%
\pgfusepath{fill}%
\end{pgfscope}%
\begin{pgfscope}%
\pgfpathrectangle{\pgfqpoint{6.572727in}{0.473000in}}{\pgfqpoint{4.227273in}{3.311000in}}%
\pgfusepath{clip}%
\pgfsetbuttcap%
\pgfsetroundjoin%
\definecolor{currentfill}{rgb}{1.000000,1.000000,1.000000}%
\pgfsetfillcolor{currentfill}%
\pgfsetlinewidth{1.003750pt}%
\definecolor{currentstroke}{rgb}{0.000000,0.000000,0.000000}%
\pgfsetstrokecolor{currentstroke}%
\pgfsetdash{}{0pt}%
\pgfsys@defobject{currentmarker}{\pgfqpoint{-0.098209in}{-0.098209in}}{\pgfqpoint{0.098209in}{0.098209in}}{%
\pgfpathmoveto{\pgfqpoint{0.000000in}{-0.098209in}}%
\pgfpathcurveto{\pgfqpoint{0.026045in}{-0.098209in}}{\pgfqpoint{0.051028in}{-0.087861in}}{\pgfqpoint{0.069444in}{-0.069444in}}%
\pgfpathcurveto{\pgfqpoint{0.087861in}{-0.051028in}}{\pgfqpoint{0.098209in}{-0.026045in}}{\pgfqpoint{0.098209in}{0.000000in}}%
\pgfpathcurveto{\pgfqpoint{0.098209in}{0.026045in}}{\pgfqpoint{0.087861in}{0.051028in}}{\pgfqpoint{0.069444in}{0.069444in}}%
\pgfpathcurveto{\pgfqpoint{0.051028in}{0.087861in}}{\pgfqpoint{0.026045in}{0.098209in}}{\pgfqpoint{0.000000in}{0.098209in}}%
\pgfpathcurveto{\pgfqpoint{-0.026045in}{0.098209in}}{\pgfqpoint{-0.051028in}{0.087861in}}{\pgfqpoint{-0.069444in}{0.069444in}}%
\pgfpathcurveto{\pgfqpoint{-0.087861in}{0.051028in}}{\pgfqpoint{-0.098209in}{0.026045in}}{\pgfqpoint{-0.098209in}{0.000000in}}%
\pgfpathcurveto{\pgfqpoint{-0.098209in}{-0.026045in}}{\pgfqpoint{-0.087861in}{-0.051028in}}{\pgfqpoint{-0.069444in}{-0.069444in}}%
\pgfpathcurveto{\pgfqpoint{-0.051028in}{-0.087861in}}{\pgfqpoint{-0.026045in}{-0.098209in}}{\pgfqpoint{0.000000in}{-0.098209in}}%
\pgfpathclose%
\pgfusepath{stroke,fill}%
}%
\begin{pgfscope}%
\pgfsys@transformshift{8.012582in}{2.191332in}%
\pgfsys@useobject{currentmarker}{}%
\end{pgfscope}%
\begin{pgfscope}%
\pgfsys@transformshift{9.606963in}{1.596483in}%
\pgfsys@useobject{currentmarker}{}%
\end{pgfscope}%
\end{pgfscope}%
\begin{pgfscope}%
\pgfpathrectangle{\pgfqpoint{6.572727in}{0.473000in}}{\pgfqpoint{4.227273in}{3.311000in}}%
\pgfusepath{clip}%
\pgfsetbuttcap%
\pgfsetroundjoin%
\definecolor{currentfill}{rgb}{0.121569,0.466667,0.705882}%
\pgfsetfillcolor{currentfill}%
\pgfsetlinewidth{1.003750pt}%
\definecolor{currentstroke}{rgb}{0.000000,0.000000,0.000000}%
\pgfsetstrokecolor{currentstroke}%
\pgfsetdash{}{0pt}%
\pgfsys@defobject{currentmarker}{\pgfqpoint{-0.028432in}{-0.049105in}}{\pgfqpoint{0.036993in}{0.049105in}}{%
\pgfpathmoveto{\pgfqpoint{0.004270in}{0.038961in}}%
\pgfpathquadraticcurveto{\pgfqpoint{-0.005609in}{0.038961in}}{\pgfqpoint{-0.010600in}{0.029223in}}%
\pgfpathquadraticcurveto{\pgfqpoint{-0.015570in}{0.019506in}}{\pgfqpoint{-0.015570in}{-0.000030in}}%
\pgfpathquadraticcurveto{\pgfqpoint{-0.015570in}{-0.019486in}}{\pgfqpoint{-0.010600in}{-0.029223in}}%
\pgfpathquadraticcurveto{\pgfqpoint{-0.005609in}{-0.038961in}}{\pgfqpoint{0.004270in}{-0.038961in}}%
\pgfpathquadraticcurveto{\pgfqpoint{0.014231in}{-0.038961in}}{\pgfqpoint{0.019202in}{-0.029223in}}%
\pgfpathquadraticcurveto{\pgfqpoint{0.024192in}{-0.019486in}}{\pgfqpoint{0.024192in}{-0.000030in}}%
\pgfpathquadraticcurveto{\pgfqpoint{0.024192in}{0.019506in}}{\pgfqpoint{0.019202in}{0.029223in}}%
\pgfpathquadraticcurveto{\pgfqpoint{0.014231in}{0.038961in}}{\pgfqpoint{0.004270in}{0.038961in}}%
\pgfpathclose%
\pgfpathmoveto{\pgfqpoint{0.004270in}{0.049105in}}%
\pgfpathquadraticcurveto{\pgfqpoint{0.020196in}{0.049105in}}{\pgfqpoint{0.028594in}{0.036506in}}%
\pgfpathquadraticcurveto{\pgfqpoint{0.036993in}{0.023928in}}{\pgfqpoint{0.036993in}{-0.000030in}}%
\pgfpathquadraticcurveto{\pgfqpoint{0.036993in}{-0.023928in}}{\pgfqpoint{0.028594in}{-0.036527in}}%
\pgfpathquadraticcurveto{\pgfqpoint{0.020196in}{-0.049105in}}{\pgfqpoint{0.004270in}{-0.049105in}}%
\pgfpathquadraticcurveto{\pgfqpoint{-0.011635in}{-0.049105in}}{\pgfqpoint{-0.020033in}{-0.036527in}}%
\pgfpathquadraticcurveto{\pgfqpoint{-0.028432in}{-0.023928in}}{\pgfqpoint{-0.028432in}{-0.000030in}}%
\pgfpathquadraticcurveto{\pgfqpoint{-0.028432in}{0.023928in}}{\pgfqpoint{-0.020033in}{0.036506in}}%
\pgfpathquadraticcurveto{\pgfqpoint{-0.011635in}{0.049105in}}{\pgfqpoint{0.004270in}{0.049105in}}%
\pgfpathclose%
\pgfusepath{stroke,fill}%
}%
\begin{pgfscope}%
\pgfsys@transformshift{8.012582in}{2.191332in}%
\pgfsys@useobject{currentmarker}{}%
\end{pgfscope}%
\end{pgfscope}%
\begin{pgfscope}%
\pgfpathrectangle{\pgfqpoint{6.572727in}{0.473000in}}{\pgfqpoint{4.227273in}{3.311000in}}%
\pgfusepath{clip}%
\pgfsetbuttcap%
\pgfsetroundjoin%
\definecolor{currentfill}{rgb}{1.000000,0.498039,0.054902}%
\pgfsetfillcolor{currentfill}%
\pgfsetlinewidth{1.003750pt}%
\definecolor{currentstroke}{rgb}{0.000000,0.000000,0.000000}%
\pgfsetstrokecolor{currentstroke}%
\pgfsetdash{}{0pt}%
\pgfsys@defobject{currentmarker}{\pgfqpoint{-0.021837in}{-0.049105in}}{\pgfqpoint{0.036634in}{0.049105in}}{%
\pgfpathmoveto{\pgfqpoint{-0.019922in}{-0.037928in}}%
\pgfpathlineto{\pgfqpoint{0.001779in}{-0.037928in}}%
\pgfpathlineto{\pgfqpoint{0.001779in}{0.037002in}}%
\pgfpathlineto{\pgfqpoint{-0.021837in}{0.032266in}}%
\pgfpathlineto{\pgfqpoint{-0.021837in}{0.044369in}}%
\pgfpathlineto{\pgfqpoint{0.001652in}{0.049105in}}%
\pgfpathlineto{\pgfqpoint{0.014933in}{0.049105in}}%
\pgfpathlineto{\pgfqpoint{0.014933in}{-0.037928in}}%
\pgfpathlineto{\pgfqpoint{0.036634in}{-0.037928in}}%
\pgfpathlineto{\pgfqpoint{0.036634in}{-0.049105in}}%
\pgfpathlineto{\pgfqpoint{-0.019922in}{-0.049105in}}%
\pgfpathlineto{\pgfqpoint{-0.019922in}{-0.037928in}}%
\pgfpathclose%
\pgfusepath{stroke,fill}%
}%
\begin{pgfscope}%
\pgfsys@transformshift{9.606963in}{1.596483in}%
\pgfsys@useobject{currentmarker}{}%
\end{pgfscope}%
\end{pgfscope}%
\begin{pgfscope}%
\pgfsetbuttcap%
\pgfsetroundjoin%
\definecolor{currentfill}{rgb}{0.000000,0.000000,0.000000}%
\pgfsetfillcolor{currentfill}%
\pgfsetlinewidth{0.803000pt}%
\definecolor{currentstroke}{rgb}{0.000000,0.000000,0.000000}%
\pgfsetstrokecolor{currentstroke}%
\pgfsetdash{}{0pt}%
\pgfsys@defobject{currentmarker}{\pgfqpoint{0.000000in}{-0.048611in}}{\pgfqpoint{0.000000in}{0.000000in}}{%
\pgfpathmoveto{\pgfqpoint{0.000000in}{0.000000in}}%
\pgfpathlineto{\pgfqpoint{0.000000in}{-0.048611in}}%
\pgfusepath{stroke,fill}%
}%
\begin{pgfscope}%
\pgfsys@transformshift{8.796015in}{0.473000in}%
\pgfsys@useobject{currentmarker}{}%
\end{pgfscope}%
\end{pgfscope}%
\begin{pgfscope}%
\definecolor{textcolor}{rgb}{0.000000,0.000000,0.000000}%
\pgfsetstrokecolor{textcolor}%
\pgfsetfillcolor{textcolor}%
\pgftext[x=8.796015in,y=0.375778in,,top]{\color{textcolor}\sffamily\fontsize{10.000000}{12.000000}\selectfont 0.10}%
\end{pgfscope}%
\begin{pgfscope}%
\pgfsetbuttcap%
\pgfsetroundjoin%
\definecolor{currentfill}{rgb}{0.000000,0.000000,0.000000}%
\pgfsetfillcolor{currentfill}%
\pgfsetlinewidth{0.803000pt}%
\definecolor{currentstroke}{rgb}{0.000000,0.000000,0.000000}%
\pgfsetstrokecolor{currentstroke}%
\pgfsetdash{}{0pt}%
\pgfsys@defobject{currentmarker}{\pgfqpoint{0.000000in}{-0.048611in}}{\pgfqpoint{0.000000in}{0.000000in}}{%
\pgfpathmoveto{\pgfqpoint{0.000000in}{0.000000in}}%
\pgfpathlineto{\pgfqpoint{0.000000in}{-0.048611in}}%
\pgfusepath{stroke,fill}%
}%
\begin{pgfscope}%
\pgfsys@transformshift{8.805201in}{0.473000in}%
\pgfsys@useobject{currentmarker}{}%
\end{pgfscope}%
\end{pgfscope}%
\begin{pgfscope}%
\definecolor{textcolor}{rgb}{0.000000,0.000000,0.000000}%
\pgfsetstrokecolor{textcolor}%
\pgfsetfillcolor{textcolor}%
\pgftext[x=8.805201in,y=0.375778in,,top]{\color{textcolor}\sffamily\fontsize{10.000000}{12.000000}\selectfont 0.11}%
\end{pgfscope}%
\begin{pgfscope}%
\pgfsetbuttcap%
\pgfsetroundjoin%
\definecolor{currentfill}{rgb}{0.000000,0.000000,0.000000}%
\pgfsetfillcolor{currentfill}%
\pgfsetlinewidth{0.803000pt}%
\definecolor{currentstroke}{rgb}{0.000000,0.000000,0.000000}%
\pgfsetstrokecolor{currentstroke}%
\pgfsetdash{}{0pt}%
\pgfsys@defobject{currentmarker}{\pgfqpoint{0.000000in}{-0.048611in}}{\pgfqpoint{0.000000in}{0.000000in}}{%
\pgfpathmoveto{\pgfqpoint{0.000000in}{0.000000in}}%
\pgfpathlineto{\pgfqpoint{0.000000in}{-0.048611in}}%
\pgfusepath{stroke,fill}%
}%
\begin{pgfscope}%
\pgfsys@transformshift{8.814388in}{0.473000in}%
\pgfsys@useobject{currentmarker}{}%
\end{pgfscope}%
\end{pgfscope}%
\begin{pgfscope}%
\definecolor{textcolor}{rgb}{0.000000,0.000000,0.000000}%
\pgfsetstrokecolor{textcolor}%
\pgfsetfillcolor{textcolor}%
\pgftext[x=8.814388in,y=0.375778in,,top]{\color{textcolor}\sffamily\fontsize{10.000000}{12.000000}\selectfont 0.12}%
\end{pgfscope}%
\begin{pgfscope}%
\pgfsetbuttcap%
\pgfsetroundjoin%
\definecolor{currentfill}{rgb}{0.000000,0.000000,0.000000}%
\pgfsetfillcolor{currentfill}%
\pgfsetlinewidth{0.803000pt}%
\definecolor{currentstroke}{rgb}{0.000000,0.000000,0.000000}%
\pgfsetstrokecolor{currentstroke}%
\pgfsetdash{}{0pt}%
\pgfsys@defobject{currentmarker}{\pgfqpoint{0.000000in}{-0.048611in}}{\pgfqpoint{0.000000in}{0.000000in}}{%
\pgfpathmoveto{\pgfqpoint{0.000000in}{0.000000in}}%
\pgfpathlineto{\pgfqpoint{0.000000in}{-0.048611in}}%
\pgfusepath{stroke,fill}%
}%
\begin{pgfscope}%
\pgfsys@transformshift{8.823575in}{0.473000in}%
\pgfsys@useobject{currentmarker}{}%
\end{pgfscope}%
\end{pgfscope}%
\begin{pgfscope}%
\definecolor{textcolor}{rgb}{0.000000,0.000000,0.000000}%
\pgfsetstrokecolor{textcolor}%
\pgfsetfillcolor{textcolor}%
\pgftext[x=8.823575in,y=0.375778in,,top]{\color{textcolor}\sffamily\fontsize{10.000000}{12.000000}\selectfont 0.13}%
\end{pgfscope}%
\begin{pgfscope}%
\pgfsetbuttcap%
\pgfsetroundjoin%
\definecolor{currentfill}{rgb}{0.000000,0.000000,0.000000}%
\pgfsetfillcolor{currentfill}%
\pgfsetlinewidth{0.803000pt}%
\definecolor{currentstroke}{rgb}{0.000000,0.000000,0.000000}%
\pgfsetstrokecolor{currentstroke}%
\pgfsetdash{}{0pt}%
\pgfsys@defobject{currentmarker}{\pgfqpoint{0.000000in}{-0.048611in}}{\pgfqpoint{0.000000in}{0.000000in}}{%
\pgfpathmoveto{\pgfqpoint{0.000000in}{0.000000in}}%
\pgfpathlineto{\pgfqpoint{0.000000in}{-0.048611in}}%
\pgfusepath{stroke,fill}%
}%
\begin{pgfscope}%
\pgfsys@transformshift{8.832762in}{0.473000in}%
\pgfsys@useobject{currentmarker}{}%
\end{pgfscope}%
\end{pgfscope}%
\begin{pgfscope}%
\definecolor{textcolor}{rgb}{0.000000,0.000000,0.000000}%
\pgfsetstrokecolor{textcolor}%
\pgfsetfillcolor{textcolor}%
\pgftext[x=8.832762in,y=0.375778in,,top]{\color{textcolor}\sffamily\fontsize{10.000000}{12.000000}\selectfont 0.14}%
\end{pgfscope}%
\begin{pgfscope}%
\pgfsetbuttcap%
\pgfsetroundjoin%
\definecolor{currentfill}{rgb}{0.000000,0.000000,0.000000}%
\pgfsetfillcolor{currentfill}%
\pgfsetlinewidth{0.803000pt}%
\definecolor{currentstroke}{rgb}{0.000000,0.000000,0.000000}%
\pgfsetstrokecolor{currentstroke}%
\pgfsetdash{}{0pt}%
\pgfsys@defobject{currentmarker}{\pgfqpoint{0.000000in}{-0.048611in}}{\pgfqpoint{0.000000in}{0.000000in}}{%
\pgfpathmoveto{\pgfqpoint{0.000000in}{0.000000in}}%
\pgfpathlineto{\pgfqpoint{0.000000in}{-0.048611in}}%
\pgfusepath{stroke,fill}%
}%
\begin{pgfscope}%
\pgfsys@transformshift{8.841948in}{0.473000in}%
\pgfsys@useobject{currentmarker}{}%
\end{pgfscope}%
\end{pgfscope}%
\begin{pgfscope}%
\definecolor{textcolor}{rgb}{0.000000,0.000000,0.000000}%
\pgfsetstrokecolor{textcolor}%
\pgfsetfillcolor{textcolor}%
\pgftext[x=8.841948in,y=0.375778in,,top]{\color{textcolor}\sffamily\fontsize{10.000000}{12.000000}\selectfont 0.15}%
\end{pgfscope}%
\begin{pgfscope}%
\pgfsetbuttcap%
\pgfsetroundjoin%
\definecolor{currentfill}{rgb}{0.000000,0.000000,0.000000}%
\pgfsetfillcolor{currentfill}%
\pgfsetlinewidth{0.803000pt}%
\definecolor{currentstroke}{rgb}{0.000000,0.000000,0.000000}%
\pgfsetstrokecolor{currentstroke}%
\pgfsetdash{}{0pt}%
\pgfsys@defobject{currentmarker}{\pgfqpoint{0.000000in}{-0.048611in}}{\pgfqpoint{0.000000in}{0.000000in}}{%
\pgfpathmoveto{\pgfqpoint{0.000000in}{0.000000in}}%
\pgfpathlineto{\pgfqpoint{0.000000in}{-0.048611in}}%
\pgfusepath{stroke,fill}%
}%
\begin{pgfscope}%
\pgfsys@transformshift{8.851135in}{0.473000in}%
\pgfsys@useobject{currentmarker}{}%
\end{pgfscope}%
\end{pgfscope}%
\begin{pgfscope}%
\definecolor{textcolor}{rgb}{0.000000,0.000000,0.000000}%
\pgfsetstrokecolor{textcolor}%
\pgfsetfillcolor{textcolor}%
\pgftext[x=8.851135in,y=0.375778in,,top]{\color{textcolor}\sffamily\fontsize{10.000000}{12.000000}\selectfont 0.16}%
\end{pgfscope}%
\begin{pgfscope}%
\pgfsetbuttcap%
\pgfsetroundjoin%
\definecolor{currentfill}{rgb}{0.000000,0.000000,0.000000}%
\pgfsetfillcolor{currentfill}%
\pgfsetlinewidth{0.803000pt}%
\definecolor{currentstroke}{rgb}{0.000000,0.000000,0.000000}%
\pgfsetstrokecolor{currentstroke}%
\pgfsetdash{}{0pt}%
\pgfsys@defobject{currentmarker}{\pgfqpoint{0.000000in}{-0.048611in}}{\pgfqpoint{0.000000in}{0.000000in}}{%
\pgfpathmoveto{\pgfqpoint{0.000000in}{0.000000in}}%
\pgfpathlineto{\pgfqpoint{0.000000in}{-0.048611in}}%
\pgfusepath{stroke,fill}%
}%
\begin{pgfscope}%
\pgfsys@transformshift{8.860322in}{0.473000in}%
\pgfsys@useobject{currentmarker}{}%
\end{pgfscope}%
\end{pgfscope}%
\begin{pgfscope}%
\definecolor{textcolor}{rgb}{0.000000,0.000000,0.000000}%
\pgfsetstrokecolor{textcolor}%
\pgfsetfillcolor{textcolor}%
\pgftext[x=8.860322in,y=0.375778in,,top]{\color{textcolor}\sffamily\fontsize{10.000000}{12.000000}\selectfont 0.17}%
\end{pgfscope}%
\begin{pgfscope}%
\pgfsetbuttcap%
\pgfsetroundjoin%
\definecolor{currentfill}{rgb}{0.000000,0.000000,0.000000}%
\pgfsetfillcolor{currentfill}%
\pgfsetlinewidth{0.803000pt}%
\definecolor{currentstroke}{rgb}{0.000000,0.000000,0.000000}%
\pgfsetstrokecolor{currentstroke}%
\pgfsetdash{}{0pt}%
\pgfsys@defobject{currentmarker}{\pgfqpoint{0.000000in}{-0.048611in}}{\pgfqpoint{0.000000in}{0.000000in}}{%
\pgfpathmoveto{\pgfqpoint{0.000000in}{0.000000in}}%
\pgfpathlineto{\pgfqpoint{0.000000in}{-0.048611in}}%
\pgfusepath{stroke,fill}%
}%
\begin{pgfscope}%
\pgfsys@transformshift{8.869509in}{0.473000in}%
\pgfsys@useobject{currentmarker}{}%
\end{pgfscope}%
\end{pgfscope}%
\begin{pgfscope}%
\definecolor{textcolor}{rgb}{0.000000,0.000000,0.000000}%
\pgfsetstrokecolor{textcolor}%
\pgfsetfillcolor{textcolor}%
\pgftext[x=8.869509in,y=0.375778in,,top]{\color{textcolor}\sffamily\fontsize{10.000000}{12.000000}\selectfont 0.18}%
\end{pgfscope}%
\begin{pgfscope}%
\pgfsetbuttcap%
\pgfsetroundjoin%
\definecolor{currentfill}{rgb}{0.000000,0.000000,0.000000}%
\pgfsetfillcolor{currentfill}%
\pgfsetlinewidth{0.803000pt}%
\definecolor{currentstroke}{rgb}{0.000000,0.000000,0.000000}%
\pgfsetstrokecolor{currentstroke}%
\pgfsetdash{}{0pt}%
\pgfsys@defobject{currentmarker}{\pgfqpoint{0.000000in}{-0.048611in}}{\pgfqpoint{0.000000in}{0.000000in}}{%
\pgfpathmoveto{\pgfqpoint{0.000000in}{0.000000in}}%
\pgfpathlineto{\pgfqpoint{0.000000in}{-0.048611in}}%
\pgfusepath{stroke,fill}%
}%
\begin{pgfscope}%
\pgfsys@transformshift{8.878695in}{0.473000in}%
\pgfsys@useobject{currentmarker}{}%
\end{pgfscope}%
\end{pgfscope}%
\begin{pgfscope}%
\definecolor{textcolor}{rgb}{0.000000,0.000000,0.000000}%
\pgfsetstrokecolor{textcolor}%
\pgfsetfillcolor{textcolor}%
\pgftext[x=8.878695in,y=0.375778in,,top]{\color{textcolor}\sffamily\fontsize{10.000000}{12.000000}\selectfont 0.19}%
\end{pgfscope}%
\begin{pgfscope}%
\pgfsetbuttcap%
\pgfsetroundjoin%
\definecolor{currentfill}{rgb}{0.000000,0.000000,0.000000}%
\pgfsetfillcolor{currentfill}%
\pgfsetlinewidth{0.803000pt}%
\definecolor{currentstroke}{rgb}{0.000000,0.000000,0.000000}%
\pgfsetstrokecolor{currentstroke}%
\pgfsetdash{}{0pt}%
\pgfsys@defobject{currentmarker}{\pgfqpoint{0.000000in}{-0.048611in}}{\pgfqpoint{0.000000in}{0.000000in}}{%
\pgfpathmoveto{\pgfqpoint{0.000000in}{0.000000in}}%
\pgfpathlineto{\pgfqpoint{0.000000in}{-0.048611in}}%
\pgfusepath{stroke,fill}%
}%
\begin{pgfscope}%
\pgfsys@transformshift{8.887882in}{0.473000in}%
\pgfsys@useobject{currentmarker}{}%
\end{pgfscope}%
\end{pgfscope}%
\begin{pgfscope}%
\definecolor{textcolor}{rgb}{0.000000,0.000000,0.000000}%
\pgfsetstrokecolor{textcolor}%
\pgfsetfillcolor{textcolor}%
\pgftext[x=8.887882in,y=0.375778in,,top]{\color{textcolor}\sffamily\fontsize{10.000000}{12.000000}\selectfont 0.20}%
\end{pgfscope}%
\begin{pgfscope}%
\pgfsetbuttcap%
\pgfsetroundjoin%
\definecolor{currentfill}{rgb}{0.000000,0.000000,0.000000}%
\pgfsetfillcolor{currentfill}%
\pgfsetlinewidth{0.803000pt}%
\definecolor{currentstroke}{rgb}{0.000000,0.000000,0.000000}%
\pgfsetstrokecolor{currentstroke}%
\pgfsetdash{}{0pt}%
\pgfsys@defobject{currentmarker}{\pgfqpoint{0.000000in}{-0.048611in}}{\pgfqpoint{0.000000in}{0.000000in}}{%
\pgfpathmoveto{\pgfqpoint{0.000000in}{0.000000in}}%
\pgfpathlineto{\pgfqpoint{0.000000in}{-0.048611in}}%
\pgfusepath{stroke,fill}%
}%
\begin{pgfscope}%
\pgfsys@transformshift{8.897069in}{0.473000in}%
\pgfsys@useobject{currentmarker}{}%
\end{pgfscope}%
\end{pgfscope}%
\begin{pgfscope}%
\definecolor{textcolor}{rgb}{0.000000,0.000000,0.000000}%
\pgfsetstrokecolor{textcolor}%
\pgfsetfillcolor{textcolor}%
\pgftext[x=8.897069in,y=0.375778in,,top]{\color{textcolor}\sffamily\fontsize{10.000000}{12.000000}\selectfont 0.21}%
\end{pgfscope}%
\begin{pgfscope}%
\pgfsetbuttcap%
\pgfsetroundjoin%
\definecolor{currentfill}{rgb}{0.000000,0.000000,0.000000}%
\pgfsetfillcolor{currentfill}%
\pgfsetlinewidth{0.803000pt}%
\definecolor{currentstroke}{rgb}{0.000000,0.000000,0.000000}%
\pgfsetstrokecolor{currentstroke}%
\pgfsetdash{}{0pt}%
\pgfsys@defobject{currentmarker}{\pgfqpoint{0.000000in}{-0.048611in}}{\pgfqpoint{0.000000in}{0.000000in}}{%
\pgfpathmoveto{\pgfqpoint{0.000000in}{0.000000in}}%
\pgfpathlineto{\pgfqpoint{0.000000in}{-0.048611in}}%
\pgfusepath{stroke,fill}%
}%
\begin{pgfscope}%
\pgfsys@transformshift{8.906256in}{0.473000in}%
\pgfsys@useobject{currentmarker}{}%
\end{pgfscope}%
\end{pgfscope}%
\begin{pgfscope}%
\definecolor{textcolor}{rgb}{0.000000,0.000000,0.000000}%
\pgfsetstrokecolor{textcolor}%
\pgfsetfillcolor{textcolor}%
\pgftext[x=8.906256in,y=0.375778in,,top]{\color{textcolor}\sffamily\fontsize{10.000000}{12.000000}\selectfont 0.22}%
\end{pgfscope}%
\begin{pgfscope}%
\pgfsetbuttcap%
\pgfsetroundjoin%
\definecolor{currentfill}{rgb}{0.000000,0.000000,0.000000}%
\pgfsetfillcolor{currentfill}%
\pgfsetlinewidth{0.803000pt}%
\definecolor{currentstroke}{rgb}{0.000000,0.000000,0.000000}%
\pgfsetstrokecolor{currentstroke}%
\pgfsetdash{}{0pt}%
\pgfsys@defobject{currentmarker}{\pgfqpoint{0.000000in}{-0.048611in}}{\pgfqpoint{0.000000in}{0.000000in}}{%
\pgfpathmoveto{\pgfqpoint{0.000000in}{0.000000in}}%
\pgfpathlineto{\pgfqpoint{0.000000in}{-0.048611in}}%
\pgfusepath{stroke,fill}%
}%
\begin{pgfscope}%
\pgfsys@transformshift{8.915442in}{0.473000in}%
\pgfsys@useobject{currentmarker}{}%
\end{pgfscope}%
\end{pgfscope}%
\begin{pgfscope}%
\definecolor{textcolor}{rgb}{0.000000,0.000000,0.000000}%
\pgfsetstrokecolor{textcolor}%
\pgfsetfillcolor{textcolor}%
\pgftext[x=8.915442in,y=0.375778in,,top]{\color{textcolor}\sffamily\fontsize{10.000000}{12.000000}\selectfont 0.23}%
\end{pgfscope}%
\begin{pgfscope}%
\pgfsetbuttcap%
\pgfsetroundjoin%
\definecolor{currentfill}{rgb}{0.000000,0.000000,0.000000}%
\pgfsetfillcolor{currentfill}%
\pgfsetlinewidth{0.803000pt}%
\definecolor{currentstroke}{rgb}{0.000000,0.000000,0.000000}%
\pgfsetstrokecolor{currentstroke}%
\pgfsetdash{}{0pt}%
\pgfsys@defobject{currentmarker}{\pgfqpoint{0.000000in}{-0.048611in}}{\pgfqpoint{0.000000in}{0.000000in}}{%
\pgfpathmoveto{\pgfqpoint{0.000000in}{0.000000in}}%
\pgfpathlineto{\pgfqpoint{0.000000in}{-0.048611in}}%
\pgfusepath{stroke,fill}%
}%
\begin{pgfscope}%
\pgfsys@transformshift{8.924629in}{0.473000in}%
\pgfsys@useobject{currentmarker}{}%
\end{pgfscope}%
\end{pgfscope}%
\begin{pgfscope}%
\definecolor{textcolor}{rgb}{0.000000,0.000000,0.000000}%
\pgfsetstrokecolor{textcolor}%
\pgfsetfillcolor{textcolor}%
\pgftext[x=8.924629in,y=0.375778in,,top]{\color{textcolor}\sffamily\fontsize{10.000000}{12.000000}\selectfont 0.24}%
\end{pgfscope}%
\begin{pgfscope}%
\pgfsetbuttcap%
\pgfsetroundjoin%
\definecolor{currentfill}{rgb}{0.000000,0.000000,0.000000}%
\pgfsetfillcolor{currentfill}%
\pgfsetlinewidth{0.803000pt}%
\definecolor{currentstroke}{rgb}{0.000000,0.000000,0.000000}%
\pgfsetstrokecolor{currentstroke}%
\pgfsetdash{}{0pt}%
\pgfsys@defobject{currentmarker}{\pgfqpoint{0.000000in}{-0.048611in}}{\pgfqpoint{0.000000in}{0.000000in}}{%
\pgfpathmoveto{\pgfqpoint{0.000000in}{0.000000in}}%
\pgfpathlineto{\pgfqpoint{0.000000in}{-0.048611in}}%
\pgfusepath{stroke,fill}%
}%
\begin{pgfscope}%
\pgfsys@transformshift{8.933816in}{0.473000in}%
\pgfsys@useobject{currentmarker}{}%
\end{pgfscope}%
\end{pgfscope}%
\begin{pgfscope}%
\definecolor{textcolor}{rgb}{0.000000,0.000000,0.000000}%
\pgfsetstrokecolor{textcolor}%
\pgfsetfillcolor{textcolor}%
\pgftext[x=8.933816in,y=0.375778in,,top]{\color{textcolor}\sffamily\fontsize{10.000000}{12.000000}\selectfont 0.25}%
\end{pgfscope}%
\begin{pgfscope}%
\pgfsetbuttcap%
\pgfsetroundjoin%
\definecolor{currentfill}{rgb}{0.000000,0.000000,0.000000}%
\pgfsetfillcolor{currentfill}%
\pgfsetlinewidth{0.803000pt}%
\definecolor{currentstroke}{rgb}{0.000000,0.000000,0.000000}%
\pgfsetstrokecolor{currentstroke}%
\pgfsetdash{}{0pt}%
\pgfsys@defobject{currentmarker}{\pgfqpoint{0.000000in}{-0.048611in}}{\pgfqpoint{0.000000in}{0.000000in}}{%
\pgfpathmoveto{\pgfqpoint{0.000000in}{0.000000in}}%
\pgfpathlineto{\pgfqpoint{0.000000in}{-0.048611in}}%
\pgfusepath{stroke,fill}%
}%
\begin{pgfscope}%
\pgfsys@transformshift{8.943003in}{0.473000in}%
\pgfsys@useobject{currentmarker}{}%
\end{pgfscope}%
\end{pgfscope}%
\begin{pgfscope}%
\definecolor{textcolor}{rgb}{0.000000,0.000000,0.000000}%
\pgfsetstrokecolor{textcolor}%
\pgfsetfillcolor{textcolor}%
\pgftext[x=8.943003in,y=0.375778in,,top]{\color{textcolor}\sffamily\fontsize{10.000000}{12.000000}\selectfont 0.26}%
\end{pgfscope}%
\begin{pgfscope}%
\pgfsetbuttcap%
\pgfsetroundjoin%
\definecolor{currentfill}{rgb}{0.000000,0.000000,0.000000}%
\pgfsetfillcolor{currentfill}%
\pgfsetlinewidth{0.803000pt}%
\definecolor{currentstroke}{rgb}{0.000000,0.000000,0.000000}%
\pgfsetstrokecolor{currentstroke}%
\pgfsetdash{}{0pt}%
\pgfsys@defobject{currentmarker}{\pgfqpoint{0.000000in}{-0.048611in}}{\pgfqpoint{0.000000in}{0.000000in}}{%
\pgfpathmoveto{\pgfqpoint{0.000000in}{0.000000in}}%
\pgfpathlineto{\pgfqpoint{0.000000in}{-0.048611in}}%
\pgfusepath{stroke,fill}%
}%
\begin{pgfscope}%
\pgfsys@transformshift{8.952189in}{0.473000in}%
\pgfsys@useobject{currentmarker}{}%
\end{pgfscope}%
\end{pgfscope}%
\begin{pgfscope}%
\definecolor{textcolor}{rgb}{0.000000,0.000000,0.000000}%
\pgfsetstrokecolor{textcolor}%
\pgfsetfillcolor{textcolor}%
\pgftext[x=8.952189in,y=0.375778in,,top]{\color{textcolor}\sffamily\fontsize{10.000000}{12.000000}\selectfont 0.27}%
\end{pgfscope}%
\begin{pgfscope}%
\pgfsetbuttcap%
\pgfsetroundjoin%
\definecolor{currentfill}{rgb}{0.000000,0.000000,0.000000}%
\pgfsetfillcolor{currentfill}%
\pgfsetlinewidth{0.803000pt}%
\definecolor{currentstroke}{rgb}{0.000000,0.000000,0.000000}%
\pgfsetstrokecolor{currentstroke}%
\pgfsetdash{}{0pt}%
\pgfsys@defobject{currentmarker}{\pgfqpoint{0.000000in}{-0.048611in}}{\pgfqpoint{0.000000in}{0.000000in}}{%
\pgfpathmoveto{\pgfqpoint{0.000000in}{0.000000in}}%
\pgfpathlineto{\pgfqpoint{0.000000in}{-0.048611in}}%
\pgfusepath{stroke,fill}%
}%
\begin{pgfscope}%
\pgfsys@transformshift{8.961376in}{0.473000in}%
\pgfsys@useobject{currentmarker}{}%
\end{pgfscope}%
\end{pgfscope}%
\begin{pgfscope}%
\definecolor{textcolor}{rgb}{0.000000,0.000000,0.000000}%
\pgfsetstrokecolor{textcolor}%
\pgfsetfillcolor{textcolor}%
\pgftext[x=8.961376in,y=0.375778in,,top]{\color{textcolor}\sffamily\fontsize{10.000000}{12.000000}\selectfont 0.28}%
\end{pgfscope}%
\begin{pgfscope}%
\pgfsetbuttcap%
\pgfsetroundjoin%
\definecolor{currentfill}{rgb}{0.000000,0.000000,0.000000}%
\pgfsetfillcolor{currentfill}%
\pgfsetlinewidth{0.803000pt}%
\definecolor{currentstroke}{rgb}{0.000000,0.000000,0.000000}%
\pgfsetstrokecolor{currentstroke}%
\pgfsetdash{}{0pt}%
\pgfsys@defobject{currentmarker}{\pgfqpoint{0.000000in}{-0.048611in}}{\pgfqpoint{0.000000in}{0.000000in}}{%
\pgfpathmoveto{\pgfqpoint{0.000000in}{0.000000in}}%
\pgfpathlineto{\pgfqpoint{0.000000in}{-0.048611in}}%
\pgfusepath{stroke,fill}%
}%
\begin{pgfscope}%
\pgfsys@transformshift{8.970563in}{0.473000in}%
\pgfsys@useobject{currentmarker}{}%
\end{pgfscope}%
\end{pgfscope}%
\begin{pgfscope}%
\definecolor{textcolor}{rgb}{0.000000,0.000000,0.000000}%
\pgfsetstrokecolor{textcolor}%
\pgfsetfillcolor{textcolor}%
\pgftext[x=8.970563in,y=0.375778in,,top]{\color{textcolor}\sffamily\fontsize{10.000000}{12.000000}\selectfont 0.29}%
\end{pgfscope}%
\begin{pgfscope}%
\pgfsetbuttcap%
\pgfsetroundjoin%
\definecolor{currentfill}{rgb}{0.000000,0.000000,0.000000}%
\pgfsetfillcolor{currentfill}%
\pgfsetlinewidth{0.803000pt}%
\definecolor{currentstroke}{rgb}{0.000000,0.000000,0.000000}%
\pgfsetstrokecolor{currentstroke}%
\pgfsetdash{}{0pt}%
\pgfsys@defobject{currentmarker}{\pgfqpoint{0.000000in}{-0.048611in}}{\pgfqpoint{0.000000in}{0.000000in}}{%
\pgfpathmoveto{\pgfqpoint{0.000000in}{0.000000in}}%
\pgfpathlineto{\pgfqpoint{0.000000in}{-0.048611in}}%
\pgfusepath{stroke,fill}%
}%
\begin{pgfscope}%
\pgfsys@transformshift{8.979750in}{0.473000in}%
\pgfsys@useobject{currentmarker}{}%
\end{pgfscope}%
\end{pgfscope}%
\begin{pgfscope}%
\definecolor{textcolor}{rgb}{0.000000,0.000000,0.000000}%
\pgfsetstrokecolor{textcolor}%
\pgfsetfillcolor{textcolor}%
\pgftext[x=8.979750in,y=0.375778in,,top]{\color{textcolor}\sffamily\fontsize{10.000000}{12.000000}\selectfont 0.30}%
\end{pgfscope}%
\begin{pgfscope}%
\pgfsetbuttcap%
\pgfsetroundjoin%
\definecolor{currentfill}{rgb}{0.000000,0.000000,0.000000}%
\pgfsetfillcolor{currentfill}%
\pgfsetlinewidth{0.803000pt}%
\definecolor{currentstroke}{rgb}{0.000000,0.000000,0.000000}%
\pgfsetstrokecolor{currentstroke}%
\pgfsetdash{}{0pt}%
\pgfsys@defobject{currentmarker}{\pgfqpoint{0.000000in}{-0.048611in}}{\pgfqpoint{0.000000in}{0.000000in}}{%
\pgfpathmoveto{\pgfqpoint{0.000000in}{0.000000in}}%
\pgfpathlineto{\pgfqpoint{0.000000in}{-0.048611in}}%
\pgfusepath{stroke,fill}%
}%
\begin{pgfscope}%
\pgfsys@transformshift{8.988936in}{0.473000in}%
\pgfsys@useobject{currentmarker}{}%
\end{pgfscope}%
\end{pgfscope}%
\begin{pgfscope}%
\definecolor{textcolor}{rgb}{0.000000,0.000000,0.000000}%
\pgfsetstrokecolor{textcolor}%
\pgfsetfillcolor{textcolor}%
\pgftext[x=8.988936in,y=0.375778in,,top]{\color{textcolor}\sffamily\fontsize{10.000000}{12.000000}\selectfont 0.31}%
\end{pgfscope}%
\begin{pgfscope}%
\pgfsetbuttcap%
\pgfsetroundjoin%
\definecolor{currentfill}{rgb}{0.000000,0.000000,0.000000}%
\pgfsetfillcolor{currentfill}%
\pgfsetlinewidth{0.803000pt}%
\definecolor{currentstroke}{rgb}{0.000000,0.000000,0.000000}%
\pgfsetstrokecolor{currentstroke}%
\pgfsetdash{}{0pt}%
\pgfsys@defobject{currentmarker}{\pgfqpoint{0.000000in}{-0.048611in}}{\pgfqpoint{0.000000in}{0.000000in}}{%
\pgfpathmoveto{\pgfqpoint{0.000000in}{0.000000in}}%
\pgfpathlineto{\pgfqpoint{0.000000in}{-0.048611in}}%
\pgfusepath{stroke,fill}%
}%
\begin{pgfscope}%
\pgfsys@transformshift{8.998123in}{0.473000in}%
\pgfsys@useobject{currentmarker}{}%
\end{pgfscope}%
\end{pgfscope}%
\begin{pgfscope}%
\definecolor{textcolor}{rgb}{0.000000,0.000000,0.000000}%
\pgfsetstrokecolor{textcolor}%
\pgfsetfillcolor{textcolor}%
\pgftext[x=8.998123in,y=0.375778in,,top]{\color{textcolor}\sffamily\fontsize{10.000000}{12.000000}\selectfont 0.32}%
\end{pgfscope}%
\begin{pgfscope}%
\pgfsetbuttcap%
\pgfsetroundjoin%
\definecolor{currentfill}{rgb}{0.000000,0.000000,0.000000}%
\pgfsetfillcolor{currentfill}%
\pgfsetlinewidth{0.803000pt}%
\definecolor{currentstroke}{rgb}{0.000000,0.000000,0.000000}%
\pgfsetstrokecolor{currentstroke}%
\pgfsetdash{}{0pt}%
\pgfsys@defobject{currentmarker}{\pgfqpoint{0.000000in}{-0.048611in}}{\pgfqpoint{0.000000in}{0.000000in}}{%
\pgfpathmoveto{\pgfqpoint{0.000000in}{0.000000in}}%
\pgfpathlineto{\pgfqpoint{0.000000in}{-0.048611in}}%
\pgfusepath{stroke,fill}%
}%
\begin{pgfscope}%
\pgfsys@transformshift{9.007310in}{0.473000in}%
\pgfsys@useobject{currentmarker}{}%
\end{pgfscope}%
\end{pgfscope}%
\begin{pgfscope}%
\definecolor{textcolor}{rgb}{0.000000,0.000000,0.000000}%
\pgfsetstrokecolor{textcolor}%
\pgfsetfillcolor{textcolor}%
\pgftext[x=9.007310in,y=0.375778in,,top]{\color{textcolor}\sffamily\fontsize{10.000000}{12.000000}\selectfont 0.33}%
\end{pgfscope}%
\begin{pgfscope}%
\pgfsetbuttcap%
\pgfsetroundjoin%
\definecolor{currentfill}{rgb}{0.000000,0.000000,0.000000}%
\pgfsetfillcolor{currentfill}%
\pgfsetlinewidth{0.803000pt}%
\definecolor{currentstroke}{rgb}{0.000000,0.000000,0.000000}%
\pgfsetstrokecolor{currentstroke}%
\pgfsetdash{}{0pt}%
\pgfsys@defobject{currentmarker}{\pgfqpoint{0.000000in}{-0.048611in}}{\pgfqpoint{0.000000in}{0.000000in}}{%
\pgfpathmoveto{\pgfqpoint{0.000000in}{0.000000in}}%
\pgfpathlineto{\pgfqpoint{0.000000in}{-0.048611in}}%
\pgfusepath{stroke,fill}%
}%
\begin{pgfscope}%
\pgfsys@transformshift{9.016497in}{0.473000in}%
\pgfsys@useobject{currentmarker}{}%
\end{pgfscope}%
\end{pgfscope}%
\begin{pgfscope}%
\definecolor{textcolor}{rgb}{0.000000,0.000000,0.000000}%
\pgfsetstrokecolor{textcolor}%
\pgfsetfillcolor{textcolor}%
\pgftext[x=9.016497in,y=0.375778in,,top]{\color{textcolor}\sffamily\fontsize{10.000000}{12.000000}\selectfont 0.34}%
\end{pgfscope}%
\begin{pgfscope}%
\pgfsetbuttcap%
\pgfsetroundjoin%
\definecolor{currentfill}{rgb}{0.000000,0.000000,0.000000}%
\pgfsetfillcolor{currentfill}%
\pgfsetlinewidth{0.803000pt}%
\definecolor{currentstroke}{rgb}{0.000000,0.000000,0.000000}%
\pgfsetstrokecolor{currentstroke}%
\pgfsetdash{}{0pt}%
\pgfsys@defobject{currentmarker}{\pgfqpoint{0.000000in}{-0.048611in}}{\pgfqpoint{0.000000in}{0.000000in}}{%
\pgfpathmoveto{\pgfqpoint{0.000000in}{0.000000in}}%
\pgfpathlineto{\pgfqpoint{0.000000in}{-0.048611in}}%
\pgfusepath{stroke,fill}%
}%
\begin{pgfscope}%
\pgfsys@transformshift{9.025683in}{0.473000in}%
\pgfsys@useobject{currentmarker}{}%
\end{pgfscope}%
\end{pgfscope}%
\begin{pgfscope}%
\definecolor{textcolor}{rgb}{0.000000,0.000000,0.000000}%
\pgfsetstrokecolor{textcolor}%
\pgfsetfillcolor{textcolor}%
\pgftext[x=9.025683in,y=0.375778in,,top]{\color{textcolor}\sffamily\fontsize{10.000000}{12.000000}\selectfont 0.35}%
\end{pgfscope}%
\begin{pgfscope}%
\pgfsetbuttcap%
\pgfsetroundjoin%
\definecolor{currentfill}{rgb}{0.000000,0.000000,0.000000}%
\pgfsetfillcolor{currentfill}%
\pgfsetlinewidth{0.803000pt}%
\definecolor{currentstroke}{rgb}{0.000000,0.000000,0.000000}%
\pgfsetstrokecolor{currentstroke}%
\pgfsetdash{}{0pt}%
\pgfsys@defobject{currentmarker}{\pgfqpoint{0.000000in}{-0.048611in}}{\pgfqpoint{0.000000in}{0.000000in}}{%
\pgfpathmoveto{\pgfqpoint{0.000000in}{0.000000in}}%
\pgfpathlineto{\pgfqpoint{0.000000in}{-0.048611in}}%
\pgfusepath{stroke,fill}%
}%
\begin{pgfscope}%
\pgfsys@transformshift{9.034870in}{0.473000in}%
\pgfsys@useobject{currentmarker}{}%
\end{pgfscope}%
\end{pgfscope}%
\begin{pgfscope}%
\definecolor{textcolor}{rgb}{0.000000,0.000000,0.000000}%
\pgfsetstrokecolor{textcolor}%
\pgfsetfillcolor{textcolor}%
\pgftext[x=9.034870in,y=0.375778in,,top]{\color{textcolor}\sffamily\fontsize{10.000000}{12.000000}\selectfont 0.36}%
\end{pgfscope}%
\begin{pgfscope}%
\pgfsetbuttcap%
\pgfsetroundjoin%
\definecolor{currentfill}{rgb}{0.000000,0.000000,0.000000}%
\pgfsetfillcolor{currentfill}%
\pgfsetlinewidth{0.803000pt}%
\definecolor{currentstroke}{rgb}{0.000000,0.000000,0.000000}%
\pgfsetstrokecolor{currentstroke}%
\pgfsetdash{}{0pt}%
\pgfsys@defobject{currentmarker}{\pgfqpoint{0.000000in}{-0.048611in}}{\pgfqpoint{0.000000in}{0.000000in}}{%
\pgfpathmoveto{\pgfqpoint{0.000000in}{0.000000in}}%
\pgfpathlineto{\pgfqpoint{0.000000in}{-0.048611in}}%
\pgfusepath{stroke,fill}%
}%
\begin{pgfscope}%
\pgfsys@transformshift{9.044057in}{0.473000in}%
\pgfsys@useobject{currentmarker}{}%
\end{pgfscope}%
\end{pgfscope}%
\begin{pgfscope}%
\definecolor{textcolor}{rgb}{0.000000,0.000000,0.000000}%
\pgfsetstrokecolor{textcolor}%
\pgfsetfillcolor{textcolor}%
\pgftext[x=9.044057in,y=0.375778in,,top]{\color{textcolor}\sffamily\fontsize{10.000000}{12.000000}\selectfont 0.37}%
\end{pgfscope}%
\begin{pgfscope}%
\pgfsetbuttcap%
\pgfsetroundjoin%
\definecolor{currentfill}{rgb}{0.000000,0.000000,0.000000}%
\pgfsetfillcolor{currentfill}%
\pgfsetlinewidth{0.803000pt}%
\definecolor{currentstroke}{rgb}{0.000000,0.000000,0.000000}%
\pgfsetstrokecolor{currentstroke}%
\pgfsetdash{}{0pt}%
\pgfsys@defobject{currentmarker}{\pgfqpoint{0.000000in}{-0.048611in}}{\pgfqpoint{0.000000in}{0.000000in}}{%
\pgfpathmoveto{\pgfqpoint{0.000000in}{0.000000in}}%
\pgfpathlineto{\pgfqpoint{0.000000in}{-0.048611in}}%
\pgfusepath{stroke,fill}%
}%
\begin{pgfscope}%
\pgfsys@transformshift{9.053244in}{0.473000in}%
\pgfsys@useobject{currentmarker}{}%
\end{pgfscope}%
\end{pgfscope}%
\begin{pgfscope}%
\definecolor{textcolor}{rgb}{0.000000,0.000000,0.000000}%
\pgfsetstrokecolor{textcolor}%
\pgfsetfillcolor{textcolor}%
\pgftext[x=9.053244in,y=0.375778in,,top]{\color{textcolor}\sffamily\fontsize{10.000000}{12.000000}\selectfont 0.38}%
\end{pgfscope}%
\begin{pgfscope}%
\pgfsetbuttcap%
\pgfsetroundjoin%
\definecolor{currentfill}{rgb}{0.000000,0.000000,0.000000}%
\pgfsetfillcolor{currentfill}%
\pgfsetlinewidth{0.803000pt}%
\definecolor{currentstroke}{rgb}{0.000000,0.000000,0.000000}%
\pgfsetstrokecolor{currentstroke}%
\pgfsetdash{}{0pt}%
\pgfsys@defobject{currentmarker}{\pgfqpoint{0.000000in}{-0.048611in}}{\pgfqpoint{0.000000in}{0.000000in}}{%
\pgfpathmoveto{\pgfqpoint{0.000000in}{0.000000in}}%
\pgfpathlineto{\pgfqpoint{0.000000in}{-0.048611in}}%
\pgfusepath{stroke,fill}%
}%
\begin{pgfscope}%
\pgfsys@transformshift{9.062430in}{0.473000in}%
\pgfsys@useobject{currentmarker}{}%
\end{pgfscope}%
\end{pgfscope}%
\begin{pgfscope}%
\definecolor{textcolor}{rgb}{0.000000,0.000000,0.000000}%
\pgfsetstrokecolor{textcolor}%
\pgfsetfillcolor{textcolor}%
\pgftext[x=9.062430in,y=0.375778in,,top]{\color{textcolor}\sffamily\fontsize{10.000000}{12.000000}\selectfont 0.39}%
\end{pgfscope}%
\begin{pgfscope}%
\pgfsetbuttcap%
\pgfsetroundjoin%
\definecolor{currentfill}{rgb}{0.000000,0.000000,0.000000}%
\pgfsetfillcolor{currentfill}%
\pgfsetlinewidth{0.803000pt}%
\definecolor{currentstroke}{rgb}{0.000000,0.000000,0.000000}%
\pgfsetstrokecolor{currentstroke}%
\pgfsetdash{}{0pt}%
\pgfsys@defobject{currentmarker}{\pgfqpoint{0.000000in}{-0.048611in}}{\pgfqpoint{0.000000in}{0.000000in}}{%
\pgfpathmoveto{\pgfqpoint{0.000000in}{0.000000in}}%
\pgfpathlineto{\pgfqpoint{0.000000in}{-0.048611in}}%
\pgfusepath{stroke,fill}%
}%
\begin{pgfscope}%
\pgfsys@transformshift{9.071617in}{0.473000in}%
\pgfsys@useobject{currentmarker}{}%
\end{pgfscope}%
\end{pgfscope}%
\begin{pgfscope}%
\definecolor{textcolor}{rgb}{0.000000,0.000000,0.000000}%
\pgfsetstrokecolor{textcolor}%
\pgfsetfillcolor{textcolor}%
\pgftext[x=9.071617in,y=0.375778in,,top]{\color{textcolor}\sffamily\fontsize{10.000000}{12.000000}\selectfont 0.40}%
\end{pgfscope}%
\begin{pgfscope}%
\pgfsetbuttcap%
\pgfsetroundjoin%
\definecolor{currentfill}{rgb}{0.000000,0.000000,0.000000}%
\pgfsetfillcolor{currentfill}%
\pgfsetlinewidth{0.803000pt}%
\definecolor{currentstroke}{rgb}{0.000000,0.000000,0.000000}%
\pgfsetstrokecolor{currentstroke}%
\pgfsetdash{}{0pt}%
\pgfsys@defobject{currentmarker}{\pgfqpoint{-0.048611in}{0.000000in}}{\pgfqpoint{-0.000000in}{0.000000in}}{%
\pgfpathmoveto{\pgfqpoint{-0.000000in}{0.000000in}}%
\pgfpathlineto{\pgfqpoint{-0.048611in}{0.000000in}}%
\pgfusepath{stroke,fill}%
}%
\begin{pgfscope}%
\pgfsys@transformshift{6.572727in}{0.735427in}%
\pgfsys@useobject{currentmarker}{}%
\end{pgfscope}%
\end{pgfscope}%
\begin{pgfscope}%
\definecolor{textcolor}{rgb}{0.000000,0.000000,0.000000}%
\pgfsetstrokecolor{textcolor}%
\pgfsetfillcolor{textcolor}%
\pgftext[x=6.146601in, y=0.682666in, left, base]{\color{textcolor}\sffamily\fontsize{10.000000}{12.000000}\selectfont \ensuremath{-}2.0}%
\end{pgfscope}%
\begin{pgfscope}%
\pgfsetbuttcap%
\pgfsetroundjoin%
\definecolor{currentfill}{rgb}{0.000000,0.000000,0.000000}%
\pgfsetfillcolor{currentfill}%
\pgfsetlinewidth{0.803000pt}%
\definecolor{currentstroke}{rgb}{0.000000,0.000000,0.000000}%
\pgfsetstrokecolor{currentstroke}%
\pgfsetdash{}{0pt}%
\pgfsys@defobject{currentmarker}{\pgfqpoint{-0.048611in}{0.000000in}}{\pgfqpoint{-0.000000in}{0.000000in}}{%
\pgfpathmoveto{\pgfqpoint{-0.000000in}{0.000000in}}%
\pgfpathlineto{\pgfqpoint{-0.048611in}{0.000000in}}%
\pgfusepath{stroke,fill}%
}%
\begin{pgfscope}%
\pgfsys@transformshift{6.572727in}{1.153622in}%
\pgfsys@useobject{currentmarker}{}%
\end{pgfscope}%
\end{pgfscope}%
\begin{pgfscope}%
\definecolor{textcolor}{rgb}{0.000000,0.000000,0.000000}%
\pgfsetstrokecolor{textcolor}%
\pgfsetfillcolor{textcolor}%
\pgftext[x=6.146601in, y=1.100860in, left, base]{\color{textcolor}\sffamily\fontsize{10.000000}{12.000000}\selectfont \ensuremath{-}1.5}%
\end{pgfscope}%
\begin{pgfscope}%
\pgfsetbuttcap%
\pgfsetroundjoin%
\definecolor{currentfill}{rgb}{0.000000,0.000000,0.000000}%
\pgfsetfillcolor{currentfill}%
\pgfsetlinewidth{0.803000pt}%
\definecolor{currentstroke}{rgb}{0.000000,0.000000,0.000000}%
\pgfsetstrokecolor{currentstroke}%
\pgfsetdash{}{0pt}%
\pgfsys@defobject{currentmarker}{\pgfqpoint{-0.048611in}{0.000000in}}{\pgfqpoint{-0.000000in}{0.000000in}}{%
\pgfpathmoveto{\pgfqpoint{-0.000000in}{0.000000in}}%
\pgfpathlineto{\pgfqpoint{-0.048611in}{0.000000in}}%
\pgfusepath{stroke,fill}%
}%
\begin{pgfscope}%
\pgfsys@transformshift{6.572727in}{1.571816in}%
\pgfsys@useobject{currentmarker}{}%
\end{pgfscope}%
\end{pgfscope}%
\begin{pgfscope}%
\definecolor{textcolor}{rgb}{0.000000,0.000000,0.000000}%
\pgfsetstrokecolor{textcolor}%
\pgfsetfillcolor{textcolor}%
\pgftext[x=6.146601in, y=1.519055in, left, base]{\color{textcolor}\sffamily\fontsize{10.000000}{12.000000}\selectfont \ensuremath{-}1.0}%
\end{pgfscope}%
\begin{pgfscope}%
\pgfsetbuttcap%
\pgfsetroundjoin%
\definecolor{currentfill}{rgb}{0.000000,0.000000,0.000000}%
\pgfsetfillcolor{currentfill}%
\pgfsetlinewidth{0.803000pt}%
\definecolor{currentstroke}{rgb}{0.000000,0.000000,0.000000}%
\pgfsetstrokecolor{currentstroke}%
\pgfsetdash{}{0pt}%
\pgfsys@defobject{currentmarker}{\pgfqpoint{-0.048611in}{0.000000in}}{\pgfqpoint{-0.000000in}{0.000000in}}{%
\pgfpathmoveto{\pgfqpoint{-0.000000in}{0.000000in}}%
\pgfpathlineto{\pgfqpoint{-0.048611in}{0.000000in}}%
\pgfusepath{stroke,fill}%
}%
\begin{pgfscope}%
\pgfsys@transformshift{6.572727in}{1.990011in}%
\pgfsys@useobject{currentmarker}{}%
\end{pgfscope}%
\end{pgfscope}%
\begin{pgfscope}%
\definecolor{textcolor}{rgb}{0.000000,0.000000,0.000000}%
\pgfsetstrokecolor{textcolor}%
\pgfsetfillcolor{textcolor}%
\pgftext[x=6.146601in, y=1.937249in, left, base]{\color{textcolor}\sffamily\fontsize{10.000000}{12.000000}\selectfont \ensuremath{-}0.5}%
\end{pgfscope}%
\begin{pgfscope}%
\pgfsetbuttcap%
\pgfsetroundjoin%
\definecolor{currentfill}{rgb}{0.000000,0.000000,0.000000}%
\pgfsetfillcolor{currentfill}%
\pgfsetlinewidth{0.803000pt}%
\definecolor{currentstroke}{rgb}{0.000000,0.000000,0.000000}%
\pgfsetstrokecolor{currentstroke}%
\pgfsetdash{}{0pt}%
\pgfsys@defobject{currentmarker}{\pgfqpoint{-0.048611in}{0.000000in}}{\pgfqpoint{-0.000000in}{0.000000in}}{%
\pgfpathmoveto{\pgfqpoint{-0.000000in}{0.000000in}}%
\pgfpathlineto{\pgfqpoint{-0.048611in}{0.000000in}}%
\pgfusepath{stroke,fill}%
}%
\begin{pgfscope}%
\pgfsys@transformshift{6.572727in}{2.408205in}%
\pgfsys@useobject{currentmarker}{}%
\end{pgfscope}%
\end{pgfscope}%
\begin{pgfscope}%
\definecolor{textcolor}{rgb}{0.000000,0.000000,0.000000}%
\pgfsetstrokecolor{textcolor}%
\pgfsetfillcolor{textcolor}%
\pgftext[x=6.254626in, y=2.355444in, left, base]{\color{textcolor}\sffamily\fontsize{10.000000}{12.000000}\selectfont 0.0}%
\end{pgfscope}%
\begin{pgfscope}%
\pgfsetbuttcap%
\pgfsetroundjoin%
\definecolor{currentfill}{rgb}{0.000000,0.000000,0.000000}%
\pgfsetfillcolor{currentfill}%
\pgfsetlinewidth{0.803000pt}%
\definecolor{currentstroke}{rgb}{0.000000,0.000000,0.000000}%
\pgfsetstrokecolor{currentstroke}%
\pgfsetdash{}{0pt}%
\pgfsys@defobject{currentmarker}{\pgfqpoint{-0.048611in}{0.000000in}}{\pgfqpoint{-0.000000in}{0.000000in}}{%
\pgfpathmoveto{\pgfqpoint{-0.000000in}{0.000000in}}%
\pgfpathlineto{\pgfqpoint{-0.048611in}{0.000000in}}%
\pgfusepath{stroke,fill}%
}%
\begin{pgfscope}%
\pgfsys@transformshift{6.572727in}{2.826400in}%
\pgfsys@useobject{currentmarker}{}%
\end{pgfscope}%
\end{pgfscope}%
\begin{pgfscope}%
\definecolor{textcolor}{rgb}{0.000000,0.000000,0.000000}%
\pgfsetstrokecolor{textcolor}%
\pgfsetfillcolor{textcolor}%
\pgftext[x=6.254626in, y=2.773638in, left, base]{\color{textcolor}\sffamily\fontsize{10.000000}{12.000000}\selectfont 0.5}%
\end{pgfscope}%
\begin{pgfscope}%
\pgfsetbuttcap%
\pgfsetroundjoin%
\definecolor{currentfill}{rgb}{0.000000,0.000000,0.000000}%
\pgfsetfillcolor{currentfill}%
\pgfsetlinewidth{0.803000pt}%
\definecolor{currentstroke}{rgb}{0.000000,0.000000,0.000000}%
\pgfsetstrokecolor{currentstroke}%
\pgfsetdash{}{0pt}%
\pgfsys@defobject{currentmarker}{\pgfqpoint{-0.048611in}{0.000000in}}{\pgfqpoint{-0.000000in}{0.000000in}}{%
\pgfpathmoveto{\pgfqpoint{-0.000000in}{0.000000in}}%
\pgfpathlineto{\pgfqpoint{-0.048611in}{0.000000in}}%
\pgfusepath{stroke,fill}%
}%
\begin{pgfscope}%
\pgfsys@transformshift{6.572727in}{3.244595in}%
\pgfsys@useobject{currentmarker}{}%
\end{pgfscope}%
\end{pgfscope}%
\begin{pgfscope}%
\definecolor{textcolor}{rgb}{0.000000,0.000000,0.000000}%
\pgfsetstrokecolor{textcolor}%
\pgfsetfillcolor{textcolor}%
\pgftext[x=6.254626in, y=3.191833in, left, base]{\color{textcolor}\sffamily\fontsize{10.000000}{12.000000}\selectfont 1.0}%
\end{pgfscope}%
\begin{pgfscope}%
\pgfsetbuttcap%
\pgfsetroundjoin%
\definecolor{currentfill}{rgb}{0.000000,0.000000,0.000000}%
\pgfsetfillcolor{currentfill}%
\pgfsetlinewidth{0.803000pt}%
\definecolor{currentstroke}{rgb}{0.000000,0.000000,0.000000}%
\pgfsetstrokecolor{currentstroke}%
\pgfsetdash{}{0pt}%
\pgfsys@defobject{currentmarker}{\pgfqpoint{-0.048611in}{0.000000in}}{\pgfqpoint{-0.000000in}{0.000000in}}{%
\pgfpathmoveto{\pgfqpoint{-0.000000in}{0.000000in}}%
\pgfpathlineto{\pgfqpoint{-0.048611in}{0.000000in}}%
\pgfusepath{stroke,fill}%
}%
\begin{pgfscope}%
\pgfsys@transformshift{6.572727in}{3.662789in}%
\pgfsys@useobject{currentmarker}{}%
\end{pgfscope}%
\end{pgfscope}%
\begin{pgfscope}%
\definecolor{textcolor}{rgb}{0.000000,0.000000,0.000000}%
\pgfsetstrokecolor{textcolor}%
\pgfsetfillcolor{textcolor}%
\pgftext[x=6.254626in, y=3.610028in, left, base]{\color{textcolor}\sffamily\fontsize{10.000000}{12.000000}\selectfont 1.5}%
\end{pgfscope}%
\begin{pgfscope}%
\pgfsetrectcap%
\pgfsetmiterjoin%
\pgfsetlinewidth{0.803000pt}%
\definecolor{currentstroke}{rgb}{0.000000,0.000000,0.000000}%
\pgfsetstrokecolor{currentstroke}%
\pgfsetdash{}{0pt}%
\pgfpathmoveto{\pgfqpoint{6.572727in}{0.473000in}}%
\pgfpathlineto{\pgfqpoint{6.572727in}{3.784000in}}%
\pgfusepath{stroke}%
\end{pgfscope}%
\begin{pgfscope}%
\pgfsetrectcap%
\pgfsetmiterjoin%
\pgfsetlinewidth{0.803000pt}%
\definecolor{currentstroke}{rgb}{0.000000,0.000000,0.000000}%
\pgfsetstrokecolor{currentstroke}%
\pgfsetdash{}{0pt}%
\pgfpathmoveto{\pgfqpoint{10.800000in}{0.473000in}}%
\pgfpathlineto{\pgfqpoint{10.800000in}{3.784000in}}%
\pgfusepath{stroke}%
\end{pgfscope}%
\begin{pgfscope}%
\pgfsetrectcap%
\pgfsetmiterjoin%
\pgfsetlinewidth{0.803000pt}%
\definecolor{currentstroke}{rgb}{0.000000,0.000000,0.000000}%
\pgfsetstrokecolor{currentstroke}%
\pgfsetdash{}{0pt}%
\pgfpathmoveto{\pgfqpoint{6.572727in}{0.473000in}}%
\pgfpathlineto{\pgfqpoint{10.800000in}{0.473000in}}%
\pgfusepath{stroke}%
\end{pgfscope}%
\begin{pgfscope}%
\pgfsetrectcap%
\pgfsetmiterjoin%
\pgfsetlinewidth{0.803000pt}%
\definecolor{currentstroke}{rgb}{0.000000,0.000000,0.000000}%
\pgfsetstrokecolor{currentstroke}%
\pgfsetdash{}{0pt}%
\pgfpathmoveto{\pgfqpoint{6.572727in}{3.784000in}}%
\pgfpathlineto{\pgfqpoint{10.800000in}{3.784000in}}%
\pgfusepath{stroke}%
\end{pgfscope}%
\begin{pgfscope}%
\definecolor{textcolor}{rgb}{0.000000,0.000000,0.000000}%
\pgfsetstrokecolor{textcolor}%
\pgfsetfillcolor{textcolor}%
\pgftext[x=8.686364in,y=3.867333in,,base]{\color{textcolor}\sffamily\fontsize{12.000000}{14.400000}\selectfont ε = 0.36}%
\end{pgfscope}%
\begin{pgfscope}%
\definecolor{textcolor}{rgb}{0.000000,0.000000,0.000000}%
\pgfsetstrokecolor{textcolor}%
\pgfsetfillcolor{textcolor}%
%\pgftext[x=6.000000in,y=4.214000in,,top]{\color{textcolor}\sffamily\fontsize{12.000000}{14.400000}\bfseries\selectfont DBSCAN (métrica manhattan)}%
\end{pgfscope}%
\end{pgfpicture}%
\makeatother%
\endgroup%
}}
\end{center}

\section{Conclusión}

Mi conclusión es que el algoritmo \textit{KMeans} tiene una mayor capacidad para
distinguir vecindades de estados que están muy cercanas, pero que se acumulan
entorno a puntos concretos, mientras que \textit{DBSCAN} no consigue hacer estas
distinciones debido a que está considerando los valores de la bola de radio
igual al umbral de distancia.

Además no he notado grandes diferencias entre usar una métrica u otra en el
algoritmo \textit{DBSCAN}.

\section{Código}

El siguiente código con la implementación también está adjunto en la entrega
y disponible, junto con esta memoria, en un repositorio git en el siguiente
enlace:
\href{https://www.github.com/haztecaso/gcomp22}{github.com/haztecaso/gcomp22}.
\vspace{1em}

\lstinputlisting[linerange={8}, firstnumber=8]{practica3.py}

\end{document}
