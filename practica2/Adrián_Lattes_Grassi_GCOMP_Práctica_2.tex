\documentclass[10pt, spanish]{article}

\usepackage[none]{hyphenat}
\usepackage[utf8]{inputenc}
\usepackage[T1]{fontenc}
\usepackage[spanish,es-nodecimaldot]{babel}
\usepackage{csquotes}
\usepackage{multicol}
\usepackage{pgf}

\usepackage{geometry}
\def\margin{20mm}
\geometry{
    a4paper,
    left=\margin,
    right=\margin,
    top=\margin,
    bottom=\margin
}

% \usepackage[sorting=none]{biblatex}
\usepackage{hyperref}

\usepackage{lmodern}
\usepackage{enumitem}
%\setenumerate{label=(\alph*),leftmargin=0.6cm}
% \setitemize{label=---,leftmargin=0.6cm}
\usepackage{amssymb}
\usepackage{mathtools}

\renewcommand*{\thefootnote}{\fnsymbol{footnote}}

% For diagrams
\usepackage{tikz}
\usetikzlibrary{arrows}

% Theorems
\usepackage{amsthm}
\usepackage{thmtools}
\newtheorem*{lema}{Lema}
\newtheorem*{obs}{Observación}
\newtheorem*{nota}{Nota}

\addto\captionsspanish{\renewcommand\proofname{Solución}}

\theoremstyle{definition}
\newtheorem*{defin}{Definición}
\newtheorem*{prop}{Proposición}
\newtheoremstyle{break}{}{}{}{}{\bfseries}{.}{.5em}{Ejercicio #2}
\theoremstyle{break}
\newtheorem{ej}{Ejercicio}

% No indentation
\setlength\parindent{0pt}
\setlength{\parskip}{0.5em}
\let\emptyset\varnothing

% Custom math commands
\newcommand{\N}{\mathbb{N}}
\newcommand{\Z}{\mathbb{Z}}
\newcommand{\Q}{\mathbb{Q}}
\newcommand{\R}{\mathbb{R}}

\renewcommand{\geq}{\geqslant}
\renewcommand{\leq}{\leqslant}

\DeclareMathSymbol{*}{\mathbin}{symbols}{"01}

\DeclareMathOperator{\im}{Im}
%\DeclareMathOperator{\ker}{Ker}
\DeclareMathOperator{\mcm}{mcm}
\DeclareMathOperator{\mcd}{mcd}

\DeclareMathOperator{\bijmap}{\ \rlap{\ensuremath{\rightarrowtail}}%
    {\ensuremath{\mkern2mu\twoheadrightarrow}}}

\newcommand{\Zp}{\mathbb{Z}_{(p)}}

\usepackage{graphicx}
% \graphicspath{ {./plots/} }


\usepackage{color}
\definecolor{gray75}{gray}{0.75}
\definecolor{pycommentcol}{rgb}{0.3,0.3,0.3}     % gray
\definecolor{pystatecol}{rgb}{0,0,0.7}           % blue
\definecolor{pystringcol}{rgb}{0,0.6,0}          % green
\definecolor{pyinbuiltscol}{rgb}{0.55,0.15,0.55} % plum
\definecolor{pyspecialcol}{rgb}{0.8,0.45,0.12}   % orange
\definecolor{mygray}{gray}{0.3}
\newcommand*{\pyfontfamily}{\fontfamily{DejaVuSansMono-TLF}\selectfont}

\usepackage{listings}
\lstset{inputpath=./}
\usepackage{xcolor}


\usepackage{textgreek}
\newcommand\pythonstyle{\lstset{
        language=Python,
        literate=%esto es para que acepte acentos
        {á}{{\'a}}1
        {í}{{\'i}}1
        {é}{{\'e}}1
        {ý}{{\'y}}1
        {ú}{{\'u}}1
        {ó}{{\'o}}1
        {ě}{{\v{e}}}1
        {š}{{\v{s}}}1
        {č}{{\v{c}}}1
        {ř}{{\v{r}}}1
        {ž}{{\v{z}}}1
        {ď}{{\v{d}}}1
        {ť}{{\v{t}}}1
        {ñ}{{\~n}}1
        {ň}{{\v{n}}}1                
        {ů}{{\r{u}}}1
        {Á}{{\'A}}1
        {Í}{{\'I}}1
        {É}{{\'E}}1
        {Ý}{{\'Y}}1
        {Ú}{{\'U}}1
        {Ó}{{\'O}}1
        {Ě}{{\v{E}}}1
        {Š}{{\v{S}}}1
        {Č}{{\v{C}}}1
        {Ř}{{\v{R}}}1
        {Ž}{{\v{Z}}}1
        {Ď}{{\v{D}}}1
        {Ť}{{\v{T}}}1
        {Ň}{{\v{N}}}1                
        {ε}{{\textepsilon}}1                
        {±}{{$\pm$}}1                
        {Ů}{{\r{U}}}1,
        basicstyle=\pyfontfamily\scriptsize,
        commentstyle=\color{pycommentcol}\itshape,
        emph={self,cls,@classmethod,@property}, % Custom highlighting
        emphstyle=\color{pyspecialcol}\itshape, % Custom highlighting style
        morestring=[b]{"""},
        stringstyle=\color{pystringcol},
        keywordstyle=\color{pystatecol},        % statements
        % remove any inbuilt functions from keywords
        deletekeywords={print},
        % Switch to predefined class that contain many, but not all,
        % inbuilt functions and classes
        classoffset=1,
        % add any inbuilts, not statements
        morekeywords={print,None,TypeError},
        keywordstyle=\color{pyinbuiltscol},
        frame=leftline,
        numberstyle=\sffamily\tiny\color{mygray},
        stepnumber=1,
        numbers=left,
        numbersep=10pt,                      
        showstringspaces=false            
}}

\usepackage[labelformat=empty, labelfont={bf,it}, textfont=bf]{caption}%ponga solo el nombre en los codigos

\pythonstyle
\begin{document}
\selectfont{\Large\textbf{Geometría computacional: Práctica 2}\hfill Adrián Lattes  Grassi} \noindent\rule{17cm}{1pt}

\section{Introducción}

En esta práctica he implementado en python el algoritmo de codificación de
Huffman, uno de decodificación y funciones auxiliares para comprobar el Primer
Teorema de Shannon.

\section{Método}

He implementado los algoritmos mediante programación orientada a objetos para
aprovechar ciertas comodidades de python como la sobrecarga de operadores.
Gracias a esto tambien se obtiene un código bastante modular y reutilizable.


Además he implementado un método \texttt{graph} para visualizar los árboles de
huffman (he incluido ejemplos de visualizaciones en los anexos).

Para comprobar el Primer Teorema de Shannon he utilizado la siguiente definición
de entropía, para un sistema de estados no equiprobables:

\[ H(X) =-\sum_{i=1}^{n}P(x_i) \log_2{P(x_i)} \] 

donde $X$ es una variable aleatoria discreta con $x_1,\ldots,x_n$ posibles
estados que ocurren con probabilidades $P(x_1),\ldots,P(x_n)$. En nuestro caso
$C=X$ es el conjunto de caracteres de un idioma y las probabilidades son las
obtenidas a partir de las frecuencias de las muestras (los textos anexos).

\section{Resultados}
En la siguiente gráfica (includa como vectorial para poder visualizarse
ampliando el documento) se puede ver el árbol de Huffman correspodiente al texto
de referencia en Castellano.

\begin{center}
    \includegraphics[scale=0.1]{./arbol_es.pdf}
\end{center}

\subsection{Comprobación del Primer Teorema de Shannon}

El Primer Teorema de Shannon nos da la siguiente cota para la longitud esperada
de la codificación de Huffman: \[ H(C) \leq L(C) < H(C) + 1 \] donde $H(C)$
es la entropía del sistema y $L(C)$ la longitud esperada.

Los valores obtenidos para las muestras de textos en Español e Inglés son:

\begin{itemize}
    \item Español: $ H(C) = 4.4074$ y $ L(C) = 4.4439$ con lo que
        \[H(C) = 4.4074 \leq L(C) = 4.4439 < H(C) + 1 = 5.4074.\] 
    \item Inglés:  $H(C) = 4.1309$ y $L(C) = 4.1726$ con lo que
        \[H(C) = 4.1309 \leq L(C) = 4.1726 < H(C) + 1 = 5.1309.\]
\end{itemize} 

\subsection{Codificación de la palabra 'medieval'}

Una posible codificación binaria no comprimida para la palabra 'medieval' es la
dada por la codificación \textit{ASCII}, que codifica cada caracter mediante un
byte (ocho bits). Por tanto la longitud de la codificación 'medieval' es
$8*\text{long}(\text{'medieval'})=8*8=64$.

Utilizando una codificación de Huffman se consigue utilizar códigos más cortos
para los caracteres de mayor frecuencia, con la finalidad de reducir la longitud
final de la codificación. Estas son las codificaciones que he obtenido para la
palabra medieval, con sus longitudes correspodientes:
\begin{itemize}
    \item Español: El código obtenido es
        $11000101000000110010100111001000110101$, de longitud $38$.  
    \item Inglés: El código obtenido es
        $010011101110111110011000111010100000111001110101$, de longitud $48$.  
\end{itemize}

Con lo que se comprueba que en cierto sentido se ha conseguido el objetivo de
reducir las codificaciones. El punto delicado es que las frecuencias esperadas de
caracteres con las que se construyen los árboles de Huffman se correspondan
realmente con las frecuencias de los textos que queremos codificar. De no ser
así podríamos generar codificaciones incluso menos óptimas que la
\textit{ASCII}. 

\subsection{Decodificación de un código dado}

He hecho una implementación del algoritmo de Huffman inspirada en la
implementación clásica, utilizando montículos. Debido a esto el árbol resultante
es distinto al de la plantilla, puesto que el algoritmo de ordenación de
montículos no tiene la misma estabilidad que el sort de numpy.

Por tanto con mi implementación no es posible decodificar el código
proporcionado en el enunciado de la entrega. En lugar de esto he incluido en el
código comprobaciones de que los códigos de la palabra 'medieval' obtenidos en
el apartado anterior se decodifican correctamente en la palabra 'medieval'.

\section{Código}

El siguiente código con la implementación también está adjunto en la entrega
y disponible, junto con esta memoria, en un repositorio git en el siguiente
enlace:
\href{https://www.github.com/haztecaso/gcomp22}{github.com/haztecaso/gcomp22}.
\vspace{1em}

\lstinputlisting[linerange={4}, firstnumber=4]{practica2.py}

\end{document}
